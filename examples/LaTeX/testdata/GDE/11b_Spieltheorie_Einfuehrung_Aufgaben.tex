\subsection{Aufgaben}

\begin{enumerate}
  \item Gibt es im Hirschjagdspiel (Seite \pageref{Hirschjagdspiel}) eine
  stark oder schwach dominante bzw. dominierte Strategie? (Begründe!)

  \item Bestimme die Nash-Gleichgewichte im Hirschjagdspiel.

  \item Im Gefangenendilemma (Seite \pageref{Gefangenendilemma}) ist das
  Nash-Gleichgewicht Pareto-Ineffizient. Erklären Sie, wie es dazu kommt.
  (i.e. Worin unterscheiden sich die Überlegungen, die man zu Bestimmung des
  Nash-Gleichgewichts und zur bestimmung der Pareto-Effizienten Zustände
  anstellt?)

  \item Würde es im Gefangenendilemma den Gefangenen helfen, wenn sie
  miteinander kommunizieren können? (Begründe!)
  
  \item Würde es im Hirschjagdspiel helfen, wenn die Spieler miteinander
  kommunizieren können? (Begründe!)

  \item Löse durch sukzessive Dominanz:

\begin{center}
\begin{tabular}{c|c|c|c|c|}
\multicolumn{1}{c}{} & 
\multicolumn{1}{c}{$S_1$} &
\multicolumn{1}{c}{$S_2$} &
\multicolumn{1}{c}{$S_3$} &
\multicolumn{1}{c}{$S_4$} \\ \cline{2-5}

$Z_1$ & 0 & 1 & 7 & 7  \\ \cline{2-5}
$Z_2$ & 4 & 1 & 2 & 10 \\ \cline{2-5}
$Z_3$ & 3 & 1 & 0 & 25 \\ \cline{2-5}
$Z_4$ & 0 & 0 & 7 & 10 \\ \cline{2-5}

\end{tabular}

{\footnotesize Quelle: Resnik, Choices, S.128 \cite[]{resnik:1987}}
\end{center}
 
  \item Löse durch sukzessive Dominanz:

\begin{center}
\setlength{\parskip}{0.5cm}
\begin{tabular}{c|c|c|c|c|}
\multicolumn{1}{c}{} & 
\multicolumn{1}{c}{$S_1$} &
\multicolumn{1}{c}{$S_2$} &
\multicolumn{1}{c}{$S_3$} &
\multicolumn{1}{c}{$S_4$} \\ \cline{2-5}

$Z_1$ & 2 & 2 & 4 & 5  \\ \cline{2-5}
$Z_2$ & 7 & 1 & 5 & 3 \\ \cline{2-5}
$Z_3$ & 4 & 2 & 3 & 1 \\ \cline{2-5}
$Z_4$ & 2 & 1 & 0 & 1 \\ \cline{2-5}

\end{tabular}

{\footnotesize Quelle: Resnik, Choices, S.129 \cite[]{resnik:1987} (mit einer
kleinen Abwandlung)}
\end{center}

\item Bei einer Spielshow soll ein Kandidat vorhersagen, ob eine rote oder eine
grüne Lampe aufleuchten wird. Sagt er richtig vorher gewinnt er €100 Euro. Der
Kandidat, weiß, dass die rote Lampe mit 60\% Wahrscheinlichkeit aufleuchten
wird, die Grüne mit 40\% Wahrscheinlichkeit. Das Spiel wird für 10 Runden
wiederholt. Wie oft sollte der Kandidat "`rot"' und wie oft "`grün"' vorher
sagen?

\item Finde ein genmischtes Gleichgewicht für das Knobelspiel (Zeige, dass es
sich um ein gemischtes Gleichgewicht handelt.):

\begin{center}
\begin{tabular}{cc|c|c|c|}
& \multicolumn{1}{c}{} & \multicolumn{3}{c}{{\bf Spaltenspieler}} \\
& \multicolumn{1}{c}{} & \multicolumn{1}{c}{Stein} 
& \multicolumn{1}{c}{Schere} &  \multicolumn{1}{c}{Papier}  \\
\cline{3-5} 
& Stein              & 0,0     & 1,-1   &  -1,1 \\
\cline{3-5} {\bf Zeilenspieler}  
& Schere             & -1,1    & 0,0    & 1,-1 \\ \cline{3-5}
& Papier             & 1,-1    & -1,1   & 0,0 \\ \cline{3-5}
\end{tabular}
\end{center}

\item Angenommen im "`Passende Münzen"' Spiel spielt Spieler 2 die Strategie
(70\% Kopf, 30\% Zahl). Welche reine oder gemischte Strateige ist die beste
Antwort von Spieler 1 auf die gemischte Strategie von Spieler 2? Wie hoch ist
dann der Wert des Spiels für jeden Spieler?

\begin{center}
\begin{tabular}{cc|c|c|}
& \multicolumn{1}{c}{} & \multicolumn{2}{c}{\bf Spieler 2} \\
& \multicolumn{1}{c}{} & \multicolumn{1}{c}{Kopf} & \multicolumn{1}{c}{Zahl}
\\ \cline{3-4} 
& Kopf                 & 1,-1                      & -1,1  \\ \cline{3-4}
\raisebox{1.5ex}[-1.5ex]{{\bf Spieler 1}} 
& Zahl                 & -1,1                      & 1,-1 \\ \cline{3-4}
\end{tabular}
\end{center}

\item Bestimme das gemischte Gleichgewicht des folgenden asymmetrischen
"`Passende Münzen"'-Spiels:

\begin{center}
\begin{tabular}{cc|c|c|}
& \multicolumn{1}{c}{} & \multicolumn{2}{c}{\bf Spieler 2} \\
& \multicolumn{1}{c}{} & \multicolumn{1}{c}{Kopf} & \multicolumn{1}{c}{Zahl}
\\ \cline{3-4} 
& Kopf                 & -5                       & 10  \\ \cline{3-4}
\raisebox{1.5ex}[-1.5ex]{{\bf Spieler 1}} 
& Zahl                 & 20                       & -10 \\ \cline{3-4}
\end{tabular}
\end{center}

\end{enumerate}
