\subsection{Aufgaben}

\begin{enumerate}
  \item Betrachte folgende beiden Entscheidungssituationen: 
\begin{quote}
{\em Situation A:}
\begin{enumerate}
  \item Alternative: 12 Mio € mit 10\% Chance und 0 € mit 90\%
  \item Alternative: 1 Mio € mit 11\% Chance und 0 € mit 89\%
\end{enumerate}

{\em Situation B:}
\begin{enumerate}
  \item Alternative: 1 Mio € sicher
  \item Alternative: 12 Mio € mit 10\%, 1 Mio € mit 89\% und 0 € mit 1\% 
\end{enumerate}
\end{quote}
  \begin{enumerate}
    \item {\em Berechne für beide Situationen den montären Erwartungswert jeder
    Alternative}
    \item {\em Zeige: Auch wenn man den Nutzen nicht mit dem Geldwert
    gleichsetzt, sondern beispielsweise einen abnehmenden Grenznutzen des
    Geldes annimmt, ist die Nutzendifferenz von Alternative 1 und 2 in
    Situation A dieselbe wie die von Alternative 2 und 1 in Situation B.}
  \end{enumerate}

  \item Ein Spieler wird vor die Wahl gestellt, entweder auf einen Münzwurf mit
  einer gleichmäßigen Münze zu wetten (A), oder auf einen Münzwurf zu wetten, bei
  dem die Münze manipuliert ist, so dass sie häufiger auf einer der beiden Seiten
  landet, ohne dass aber bekannt ist, auf welcher (B). \cite[S. 109]{resnik:1987}
  
  {\em Zeige}: Falls der Spieler lieber an Spiel A teilnimmt als an Spiel B, dann
  impliziert das, dass er bei Spiel B nicht indifferent zwischen Kopf oder Zahl
  sein kann, wie es das Indifferenzprinzip fordern würde.
  
  Ansatz: 1. Zeige: Wenn der Spieler in Spiel A auf Kopf setzt und Spiel A Spiel
  B vorzieht, dann nimmt er implizizt an, dass die Wahrscheinlichkeit von
  "`Kopf"' in Spiel B kleiner als 1/2 ist.
  
  2. Zeige: Wenn der Spieler in Spiel B indifferent zwischen Kopf und Zahl ist,
  dann impliziert dies die Annahme, dass er beiden Ergebnissen eine subjektive
  Wahrscheinlichkeit von 50\% zuweist.
  
  Hilfe: Nimm an, dass der Spieler 1 € gewinnen kann, wenn er richtig wettet und
  0 € wenn er falsch wettet. Bezeichne mit $x$ die Entscheidung bei Spiel A auf
  Kopf zu setzen, mit $y$ die Entscheidung, bei Spiel B auf Kopf zu setzen und
  mit $z$ die Entscheidung bei Spiel B auf Zahl zu setzen. Wie sieht der
  Erwartungsnutzen (bzw. -wert) $EU(x), EU(y)$ und $EU(z)$ aus?
  
  (Die subjektive Wahrscheinlichkeitstheorie und die Nutzentheorie dürfen dabei
  vorausgesetzt werden!)
  
  (Zusatzfrage: Was besagt dieses Resultat?)
    

 
%  \item Wiederholungsaufgabe: Betrachte folgende beiden Entscheidungstabellen:
%  
% \begin{center}
% \begin{tabular}{c|c|c|c|c|cc|c|c|c|c|}
% \multicolumn{1}{c}{} & \multicolumn{4}{c}{Tabelle 1:} &
% \multicolumn{2}{c}{} & \multicolumn{4}{c}{Tabelle 2:}
% \\
% \cline{2-5} \cline{8-11}
% $A_1$ & 0 &  7 & -2  &  3 & & $A_1$ & 7 & 2 & 1 & 0 \\ 
% \cline{2-5} \cline{8-11} 
% $A_2$ & 3 &  4 & 4   & 19 & & $A_2$ & 3 & 2 & 1 & 5 \\ 
% \cline{2-5} \cline{8-11}
% $A_3$ & 2 & 12 & 7   &  3 & & $A_3$ & 1 & -1& 6 & 3 \\ 
% \cline{2-5} \cline{8-11}
% \end{tabular}
% \end{center}
% 
% Welche Entscheidungen sollten a) nach der Maximin-Regel getroffen werden? und b)
% nach der Minimax-Bedauerns-Regel?
 
  \item Quizfrage: Wie groß ist die Wahrscheinlichkeit 6 Richtige im Lotto zu
  bekommen?
\end{enumerate}
