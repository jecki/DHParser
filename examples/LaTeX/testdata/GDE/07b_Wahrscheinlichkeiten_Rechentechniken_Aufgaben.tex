\subsection{Aufgaben}
\begin{enumerate}
 
\item\label{beweis1} Ca. 3\% aller 70-jährigen haben Alzheimer. Auch
wenn Alzheimer bisher nicht geheilt werden kann, ist die
Früherkennung eine wichtige Voraussetzung für vorbeugende, den
Krankheitsverlauf evtl. mildernde Maßnahmen. Leider lässt sich
Alzheimer nur schwer präzise diagnostizieren. (Erst durch
Gewebeuntersuchungen am verstorbenen Patienten lässt sich mit
Sicherheit feststellen, ob eine Alzheimererkrankung vorlag.)
Angenommen einmal, die Forschung hätte einen Gedächtnistest
entwickelt, durch den eine vorliegende Alzheimererkrankung
mit 95\%-iger Sicherheit diagnostizierbar ist, an dem im
Durchschnitt aber auch 2\% der älteren Menschen scheitern, selbst wenn sie
nicht an Alzheimer erkrankt sind.

Durch einen zweiten Test, der etwas weniger zuverlässig ist als der erste, wird
eine Vorliegende Krankheit in 90\% aller Fälle richtig erkannt und mit 10\%
Wahrscheinlichkeit wird Fehlalarm gegeben, obwohl gar keine Erkrankung vorliegt.
 
{\em Aufgabe}: Angenommen beide Tests fallen positiv aus. Zeigen Sie durch
Rechnung: Es spielt keine Rolle, in welcher Reihenfolge die Tests durchgeführt werden.

\item Angenommen bei der vorhergehenden Aufgabe würde mindestens einer der
Tests negativ ausfallen, können Sie dann auch eine präzise Aussage über die
Wahrscheinlichkeit einer Erkrankung machen? 

\item\label{beweis2} Zeigen Sie, dass die Bayes'sche Formel für das inverse
Ereignis, dass der Patient nicht krank ist, obwohl der Test positiv ausgefallen ist (nicht zu
verwechseln mit dem Ereignis, dass er nicht krank ist, wenn der Test negativ
ausgefallen ist!) wie wir es erwarten würden gleich 1 minus dem ursprünglichen
Ereignis ist, also $P(\neg p|q) = 1 - P(p|q)$. Zeigen Sie dies durch
eine Rechnung für das gegebene Beispiel.

\item Eine Menge von Ereignissen $\{p_1, p_2, \ldots, p_n\}$ heisst paarweise
unvereinbar, wenn für jedes Paar $p_i,p_k$ gilt, dass $p_i$ und $p_k$
miteinander unvereinbar sind. Dagegen nennt man eine Menge $\{p_1, p_2, \ldots,
p_n\}$ von Ereignissen vollständig unvereinbar, wenn niemals alle Ereignisse aus
der Menge eintreten können. Zeigen Sie: Paarweise Unvereinbarkeit ist {\em stärker} als
vollständige Unvereinbarkeit, indem eine Menge paarweise unvereinbarer
Ereignisse immer auch vollständig unvereinbar ist, aber nicht umgekehrt.

\item a) Zeigen Sie, aus dem 3. kolmogorowschen Axiom (wenn p und q unvereinbar,
dann $P(p \vee q) = P(p) + P(q)$) folgt: Für jede endliche Menge von paarweise
unvereinbaren Ereignissen $p_i$ mit $0 \leq i < n, n \in \mathbb{N}$ gilt:
\[P(\bigvee_{0 \leq i < n} p_i) = 
P(p_1 \vee p_2 \vee \ldots) = 
P(p_1) + P(p_2) + \ldots = 
\sum_{0 \leq i < n} P(p_i)\]
Warum kann man nicht in gleicher Weise das Axiom 3' ($P(\sum_{i=0}^{\infty} p_i)
= \sum_{i=0}^{\infty} P(p_i)$) aus Axiom 3 ableiten? (Bemerkung: Wäre eine
solche Ableitung möglich, dann müsste man Axiom 3' auch nicht als Axiom einführen.)

\item\label{beweis3} Leiten Sie eine Formel für $P(q_1 \vee q_2 \vee q_3)$
analog zum Corollar 5 aus der Vorlesung her.

\item Bonferroni's Ungleichung besagt:
\[P(p \wedge q) \geq P(p) + P(q) - 1\]
Beweisen Sie Bonferroni's Ungleichung.

\item In der Vorlesung wurde für die Berechnung von
Wahrscheinlichkeiten und-verknüpfter Ereignisse das Beispiel eines
Aktienunternehmens U angeführt, das eine Gewinnwarnung ausgibt. Dabei war:
\begin{itemize}
  \item q die Aussage "`U gibt eine Gewinnwarnung aus"'
  \item p die Aussage "`Der Aktienkurs von U"' steigt  
\end{itemize}
Die Wahrscheinlichkeit, dass U eine Gewinnwarnung ausgibt {\em und} der
Aktienkurs von U steigt, wurde berechnet nach:
\[ P(p \wedge q) = P(q) \cdot P(p|q) \]
Wegen der Kommutativität des und-Operators $\wedge$ hätte man, rein
mathematisch betrachtet, aber auch
\[ P(p \wedge q) = p(p) \cdot P(q|p) \]
rechnen dürfen. Wie müsste man die zweite Formel in Worten wiedergeben? Führt
dies zu einer sinnvollen Interpretation? Wonach richtet sich, welche der
beiden Formeln man verwenden wird?

\item Zeigen Sie: a) Die Wahrscheinlichkeit, dass mindestens eines von
einer Menge von paarweise unvereinbaren Ereignissen eintritt, ist gleich der
Summe der Wahrscheinlichkeiten der einzelnen Ereignisse. 

b) Die Wahrscheinlichkeit, dass alle Ereignisse einer Menge von paarweise
unabhängigen Ereignissen eintreten, ist gleich dem Produkt der
Wahrscheinlichkeiten der einzelnen Ereignisse.

c) Wenn die Ereignisse nicht paarweise unvereinbar bzw. unabhängig sind, wird
die entsprechende Wahrscheinlichkeit dann größer oder kleiner?

\item Sei $p_1, p_2, \ldots, p_n$ eine Menge von Ereignissen, die paarweise
unvereinbar sind, von denen aber ein Ereignis auf jeden Fall eintreten muss.
Sei q weiterhin ein Ereignis dessen Wahrscheinlichkeit nicht 0 ist. a) Zeige,
dass dann folgende erweiterte Form des Bayes'schen Lehrsatzes gilt:

\[ P(p_i|q) = \frac{P(q|p_i)P(p_i)}{\sum_{i=1}^{n}P(q|p_i)P(p_i)}
\]
 
({\em Hinweis:} Der Beweis kann ganz analog zu dem Beweis des Bayes'schen
Lehrsatzes aus der Vorlesung geführt werden.)

 
\item Welche Bedingung muss für $P(q|p)$ (positiv-positiv Rate) und $P(q|\neg
p)$ gelten, damit: \begin{enumerate}
                     \item $P(p|q) > P(p)$
                     \item $P(p|q) < P(p)$
                     \item $P(p|q) = P(p)$
\end{enumerate}
Mit anderen Worten: Unter welcher Bedingung unterstützt ein positiver
Testausgang $q$ die Wahrscheinlichkeit dafür, das $p$ stimmt, und unter welcher
Bedingung verringert er sie?

Ansatz: Zeige, unter welcher Bedingung diese Ungleichung gilt (bzw. die
umgekehrte Ungleichung bzw. die entsprechende Gleichung): 
\[ \frac{P(q|p)\cdot P(p)}{P(q|p)P(p) + P(q|\neg p)P(\neg p)} > P(p) \]

Zusatz: Zeige, dass im letzten Fall, d.h. wenn $P(p|q) = P(p)$, auch gilt, dass
$p$ und $q$ statistisch unabhängig sind. 


~\\{\bf schwierigere Aufgaben}\\

\item Beweisen Sie den Zusammenhang, der in
Aufgabe \ref{beweis1} durch eine Beispiel-Rechnung illustriert wurde,
mathematisch.

\item Führen Sie den vollständigen mathematischen Beweis für den in Aufgabe
\ref{beweis2} behaupteten Zusammenhang.

\item Ergänzung zu Aufgabe \ref{beweis3}: Können Sie auch eine entsprechende
Vor\-schrift für $P(\bigvee\limits_{i = 1}^{n} q_i)$ formulieren?
  
\end{enumerate}
