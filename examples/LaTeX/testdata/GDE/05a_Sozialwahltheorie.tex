\section{Sozialwahltheorie}
\label{Sozialwahltheorie}

Bisher haben wir uns nur mit individuellen Entscheidungen beschäftigt. Für die
Anwendung der Theorie ist es dabei weniger wichtig, ob die Akteure bzw.
"`Agenten"' tatsächlich einzelne Individuen sind, oder ob sie etwa Gruppen oder
Körperschaften sind. Entscheidend ist, dass sie über eine ganz bestimmte
Präferenzrelation verfügen, die die Bedingungen für Präferenzrelationen erfüllt,
also Ordnung, Transitivität etc. (siehe Kapitel \ref{Praeferenzen}, ab Seite
\pageref{Praeferenzen}). \marginline{kollektive Entscheidungen} Die
Sozialwahltheorie beschäftigt sich nun genau mit der Frage, wie eine Gruppe von
Individuen kollektive Entscheidungen treffen kann, wenn man noch nicht von
vornherein eine kollektive Präferenzrelation als gegeben betrachtet. Man könnte
auch sagen, dass das Problem bzw. eines der Hauptprobleme der Sozialwahltheorie
darin besteht, wie man individuelle Präferenzen auf kollektive Präferenzen
abbilden kann. Um ein Problem handelt es sich insofern, als die individuellen
Präferenzen einer Gruppe von Menschen höchst unterschiedlich beschaffen sein
können, selbst wenn man einmal annimmt, dass jedes Mitglied der Gruppe über eine
im Sinne der Theorie gültige Präferenzrelation verfügt. Wie wir sehen werden,
kann es zu Schwierigkeiten kommen, wenn man daraus eine kollektive
Präferenzrelation ableiten will, die immer noch die Bedingungen einer
wohlgeordneten Präferenzrelation erfüllt.

Die individuellen Präferenzen sämtlicher Individuen zusammengenommen, bezeichnet
man auch als "`{\em Präferenzprofil} "'. Ein Präferenzprofil ist also eine Menge
von individuellen Präferenzrelationen.\marginline{soziale Wohlfahrtsfunktion} Die
Abbildung des Profils von individuellen Präferenzrelationen auf eine einzelne
kollektive Präferenzrelation nennt man eine "`{\em soziale Wohlfahrtsfunktion}"'
oder, im Zusammenhang der Entscheidungstheorie, auch ein "`{\em
Kollektiventscheidungsverfahren}"'. Mathematisch betrachtet haben wir es dabei mit
folgenden Gegenständen zu tun:

\begin{enumerate}
  \item Mit einer Menge ${\cal X} = \{x, y, z, \ldots\}$ von Alternativen oder
  Güterbündeln, die jeweils mit kleinen Buchstaben bezeichnet werden. Die Menge
  aller auf ${\cal X}$ möglichen Präferenzrelationen soll mit ${\cal R}$
  bezeichnet werden. (Für die definierenden Eigenschaften einer gültigen
  Präferenzrelation siehe \pageref{Ordnungsaxiome}.)
 
  \item Mit einer bestimmten Anzahl von Individuen $A, B, C, \ldots$, die mit
  Großbuchstaben vom Anfang des Alphabets bezeichnet werden. Die Individuen kann
  man sich durchnummeriert denken, so dass man sinnvollerweise statt von $A$, $B$ oder
  $C$ auch vom ersten, zweiten oder dritten Individuuem oder ganz allgemein vom
   "`$i$-ten Individuuem"' sprechen kann.
  
  \item Mit individuellen Prä\-fer\-enz\-re\-la\-tion\-en, wobei jedes
  Individuum na\-tür\-lich eigene Prä\-fer\-enz\-en hat. Um anzuzeigen, wessen
  Präferenzen gemeint sind, kann man einen Index an das Präferenzzeichen
  anhängen, d.h. $x \succ_i y$, bedeutet, dass das $i$-te Individuum $x$
  gegenüber $y$ vorzieht. Die gesammte Präferenzrelation eines Individuums kann
  man mit $R_i$ bezeichnen.
 
  \item Mit einer kollektiven Prä\-fer\-enz\-re\-la\-tion, d.i. diejenige 
  Prä\-fer\-enz\-re\-la\-tion, die später für das Kollektiv gelten soll, und
  die, solange nichts Näheres darüber bestimmt ist, völlig unabhängig von den
  individuellen Präferenzen ist. Um zu kennzeichnen, dass kollektive Präferenzen gemeint
  sind, wird der Index $K$ an das Präferenzzeichen angehängt, also etwa
  $x \succ_K y$. Die gesamte kollektive Präferenzrelation wird wiederum mit
  $R_K$ bezeichnet.
   
  \item Mit Profilen von individuellen Präferenzen. Ein Profil ist dabei ein
  Tupel von individuellen Präferenzrelationen, in der für jedes Individuum
  genau eine Präferenzrelation $R_i$ festgelegt ist. Wenn wir ein
  beliebiges Präferenzprofil mit $P$ bezeichnen, dann gilt $P =
  (R_1,\ldots, R_n)$. Zwei Präferenzprofile $P_1, P_2$
  unterscheiden sich dann, wenn mindestens ein Individuum in $P_1$ andere 
  Präferenzen hat als in $P_2$.
  (Und es hat andere Präferenzen, wenn es wenigstens bezüglich eines Paars von 
  Alternativen eine andere Ordnung vornimmt.)
  
  \item Mit der Menge aller möglichen Präferenzprofile ${\cal P}$, die, wie der
  Name schon sagt, jedes nur denkbare Profil von wohlgeordneten
  individuellen Präferenzen enthält.
\end{enumerate}

Eine {\em Kollektiventscheidungsverfahren}\marginline{mathematische Definition}
 (auch "`soziale"' bzw.
"`ge\-sell\-schaft\-liche Wohl\-fahrts\-funk\-tion"' oder einfach
"`Sozialwahlfunktion"') ist nun eine Funktion $f: {\cal P} \mapsto {\cal R}$,
die jedem Präferenzprofil $P \in {\cal P}$ eine "`kollektive"' Präferenzrelation
$R_K \in {\cal R}$ zuordnet. Man kann auch schreiben: $f(P_1,\ldots, P_n)
= R_K$, wobei $(P_1,\ldots, P_n)$ ein bestimmtes Präferenzprofil ist, und
$R_K$ diejenige Präferenzrelation, die diesem Profil durch die
Sozialwahlfunktion $f$ zugeorndet wird.

Mit Hilfe dieses technischen Apparats kann die Frage untersucht werden, welche
Entscheidungs- bzw. Abstimmungsprozeduren zum Treffen von Kollektiventscheidungen
geeignet sind. Z.B. kann man damit die Frage untersuchen, ob die Entscheidung
nach dem demokratischen Mehrheitsprinzip zu effizienten, gerechten und
konsequenten Kollektiventscheidungen führt. Dazu müssen die entsprechenden
Anforderungen an eine Sozialwahlfunktion (Effizienz, Gerechtigkeit etc.)
natürlich zunächst mathematisch umschrieben werden. In diesem Zusammenhang ist es
wichtig darauf hinzuweisen, dass die Sozialwahltheorie keineswegs die einzige
Theorie ist, die sich mit diesen Fragen beschäftigt. Vielmehr werden die
entsprechenden Fragen in der politischen Philosophie schon seit der Antike
thematisiert, und schon längst bevor es die Sozialwahltheorie als eigenes
Fachgebiet gab, sind auf viele der von ihr untersuchten Probleme praxistaugliche
Lösungen gefunden worden. Was die Sozialwahltheorie von früheren Ansätzen
unterscheidet ist der formale mathematische Rahmen, in dem sie diese Probleme
untersucht.\marginline{Grenzen der Sozialwahltheorie} Leider erweist sich dieser
formale Rahmen nicht immer als ein Vorteil, indem viele wichtige Probleme und
Fragestellungen, die im Zusammenhang mit kollektiven Entscheidungsprozessen
stehen, sich innerhalb dieses Rahmens entweder überhaupt nicht oder nicht adäquat
artikulieren lassen. Die Sozialwahltheorie gibt nur einen ganz bestimmten
Blickwinkel auf solche Phänomene wie das der demokratischen Mehrheitsentscheidung
frei. Was z.B. weitgehend ausgespart bleibt, sind sogenannte "`deliberative"'
Prozesse, also diejenigen Vorgänge, in denen sich -- in der ökonomistischen
Sprache formuliert -- die Präferenzen der Individuen in Folge von öffentlichen
Diskussionen veränderen, aneinander anpassen oder sich dissozieren und in Lager
aufteilen. Und in einer nicht ökonomistischen Sprache formuliert, sind
deliberative Prozesse all diejenigen Diskussions- und Meinungsbildungsprozesse,
die, besonders in Demokratien, politischen Entscheidungen oder Abstimmungen
voraus zu gehen pflegen. Will man ein richtiges und vollständiges Bild von der
Natur demokratischer politischer Entscheidungsprozesse gewinnen, so ist die
Sozialwahltheorie allein dafür völlig unzureichend und sollte unbedingt durch
andere Theorien, z.B. solche, die deliberative Prozesse zum Gegenstand haben,
ergänzt werden. Zur klassischen politischen Philosophie steht die
Sozialwahltheorie also bestenfalls im Verhältnis einer Ergänzung. Keineswegs
handelt es sich dabei um eine "`streng wissentschaftliche"' Alternative, die die
traditionelle politische Philosophie ablösen oder ersetzen könnte.

\subsection{Zum Einstieg: Das Condorcet-Paradox}
\label{condorcetParadox}
Der grundlegende Widerspruch, auf dem in der ein- oder anderen Form viele der
Unmöglichkeitsbeweise der Sozialwahltheorie aufbauen, lässt sich beispielhaft
am sogenannten Condorcet-Paradox erläutern. Angenommen, wir haben drei
Individuen $A$,$B$, $C$, die über drei Alternativen
$x$,$y$,$z$ abstimmen wollen. Alle Individuen sind dabei gleichberechtigt. Ihre
Präferenzen sind folgendermaßen verteilt:

\begin{center}
\begin{tabular}{ccc}
\label{condorcetParadoxTabelle}
$A$ & $B$ & $C$ \\
\cline{1-3}
$z$ & $x$ & $y$ \\
$x$ & $y$ & $z$ \\
$y$ & $z$ & $x$ \\
\end{tabular}
\end{center}

Welche Alternative sollte gewählt werden? Jede Alternative steht einmal an
erster, einmal an zweiter und einmal an dritter Stelle. Man kann also keine
Alternative ohne Weiteres als die kollektiv beste auszeichnen, wenn man nicht
eines der Individuen in ungerechter Weise bevorzugen will. Das Problem lässt sich
auch nicht einfach verfahrenstechnisch lösen. Denn wollte man zum Beispiel
Stichwahlen durchführen, so würde im ersten Wahlgang jede Alternative die gleiche
Stimmenzahl erhalten, so dass man keine Alternative für den zweiten Wahlgang
ausschließen könnte. Wollte man paarweise Stichwahlen durchführen, so ergibt sich
jeweils, dass $x \succ_K y$, $y \succ_K z$, aber ebenso auch $z \succ_K x$. Bei
jedem dieser Paare wird ja das vordere Glied von jeweils zwei Individuen
bevorzugt.\marginline{Condorcet-Kriterium} Man nennt den Mechanismus von
paarweisen Stichwahlen zur Bestimmung der bevorzugten Alternative aus einer Menge
von Alternativen über die mehrere Individuen (möglicherweise) unterschiedliche
Präferenzen haben auch {\em Condorcet-Kriterium} (nach dem Marquis des Condorcet,
einem französischen Philosphen und Mathematiker des 18. Jahrhunderts, der dieses
Kriterium vorgeschlagen hat). Das Condorcet-Kriterium zur Bestimmung der
kollektiven Präferenzen würde also zu {\em zyklischen Präferenzen} führen, weil
$x \succ_K y \succ_K z \succ_K x$ gilt. \marginline{zyklische Präferenzen} Damit
wäre aber die Transitivität der kollektiven Präferenzrelation verletzt. Nun haben
wir zwar gesehen, dass intransitive Präferenzen keineswegs "`unnatürlich"' sein
müssen (siehe Seite \pageref{intransitivePraeferenzen}). Das vorliegende Beispiel
zeigt ja gerade, dass sie auf eine ganz natürliche und naheliegende Weise
(paarweise Stichwahlen) zustande kommen können. Aber intransitive Präferenzen
werfen trotzdem sowohl theoretische ("`Geldpumpenargument"', siehe Seite
\pageref{Geldpumpenargument}) als auch praktische Probleme auf. Denn welche
Alternative soll man im Fall zyklischer kollektiver Präferenzen wählen, wenn man
vermeiden will, irgendjemanden zu bevorzugen. Eine der naheliegendsten Lösungen
um mit "`Pattsituationen"' dieser Art umzugehen, besteht darin das Los
entscheiden zu lassen, denn beim Losverfahren bleibt die demokratische Gleichheit dadurch
gewahrt, dass jeder die gleichen Chancen hat.\marginline{Losentscheid zur
Auflösung von Zyklen} Es ist daher auch nicht verwunderlich, dass wir dieses
Mittel seit der Antike in zahlreichen Satzungen und Verfassungen für
u.a. diejenigen Fälle vorgesehen finden, in denen eine Abstimmung nicht zu einem
eindeutigen Ergebnis führt \cite[]{delong:1991}. (Ein anderer wichtiger Grund
für den Einsatz des Losverfahren ist, dass es sich nicht wie Abstimmungen durch
Stimmenkauf oder Erpessung manipulieren lässt. Bei historischen Beispielen der
Verlosung von Ämtern (z.B. im antiken Athen oder in den italienischen Republiken
in der Zeit der Renaissance) kommt hinzu,\footnote{Darauf hat mich Rudolf
Schüssler aufmerksam gemacht.} dass man auf diese Weise verhindern wollte, dass
dieselben Ämter immer in der Hand derselben Familien bleiben.)

Die mögliche Entstehung {\em zyklischer kollektiver Präferenzen} ist nur eins von
mehreren Problemen, an denen Abstimmungsverfahren leiden können. Ein weiteres
mögliches Problem bestimmter Abstimmungsverfahren, das bei "`ungünstig"'
verteilten individuellen Präferenzen auftreten kann, ist das der {\em
Pfadabhängigkeit}. Angenommen, wir hätten uns entschlossen, statt, wie eben, über
alle Paare abzustimmen, zunächst zwischen einem beliebig herausgegriffenen Paar
von Alternativen abszustimmen und dann zwischen dem Gewinner dieser Abstimmung
und der verbleibenden Alternative. (Sollte es mehr als drei Alternativen geben,
kann man das Verfahren einfach noch einmal durchführen, solange bis am Ende eine
Alternative gewonnen hat.) Die Teilnehmer $A,B$ und $c$ aus der Tabelle auf Seite
\pageref{condorcetParadoxTabelle} würden also z.B. zuerst über $x$ und $y$
abstimmen, wobei $x$ mit 2 Stimmen zu einer Stimme gewinnt. Dann stimmen sie über die
verbleibende Alternative $x$ oder $z$ ab. Diesmal gewinnt $z$ mit 2:1
Stimmen.\marginline{Pfadab\-hängigkeit} Das Problem
besteht nun darin, dass eine ganz andere Alternative gewonnen hätte, wenn nicht
mit der Abstimmung über $x$ und $y$ begonnen worden wäre, sondern z.B. mit der
Abstimmung über $x$ und $z$ begonnen, dann hätte sich zunächst $z$ gegen $x$
behauptet, aber bei der anschließenden Stichwahl zwischen $z$ und $y$ hätte $y$
gewonnen. Das Abstimmungsergebnis hängt also (bei entsprechend ungünstig
verteilten Präferen) in kontingenter Weise von der Reihenfolge der Abstimmung
(bzw. dem gefählten "`Pfad"') ab. Man könnte auch sagen, der Sieg von $z$ im
ersten Fall bzw. von $y$ im zweiten Fall ist bloß ein "`Artefakt des
Abstimmungsmechanismus"'. (Eine präzise Definition des Begriffs des "`Artefakts
eines Abstimmungsmechanismus"' könnte lauten: Eine Artefakt eines
Abstimmungmechanismus ist ein Abstimmungsergebnis, das nur durch die Verletzung
unserer Erwartungen an einen fairen und vernünftigen Abstimmungsmechanismus
zustande gekommen ist. In dem Beispiel eben wäre dan die Erwartung verletzt, dass
ein Abstimmungsmechanismus pfadunabhängig sein sollte.) Unter Umständen könnte
dieses Problem sogar Manipulationsmöglichkeiten für einen geschickten
Wahlleiter eröffnen, der die Reihenfolge der Stichwahlen festlegen darf (siehe
Übungsaufgabe \ref{AufgPL0} auf Seite \pageref{AufgPL0}). 

Dasselbe Beispiel verdeutlicht zugleich ein weiteres Problem -- wenn man es für
ein Problem hält --, nämlich das des {\em strategischen Wählens}.
\marginline{strategisches Wählen} Nehmen wir an, die Reihenfolge der Abstimmungen
sei bereits dahingehend festgelegt, dass zunächst zwischen $x$ und $y$ und dann
zwischen der Siegeralternative und $z$ abgestimmt wird. Angenommen nun,
Individuum $B$ würde in der ersten Runde nicht für $x$, sondern "`strategisch"',
d.h. entgegen den eigenen Präferenzen, für $y$ stimmen, dann würde sich $y$ in
der zweiten Runde durchsetzen und $B$ hätte vermieden, dass die aus $B$s Sicht
schlechteste Alternative $C$ gewinnt. "`Strategisches Wählen"' kann man insofern
als ein Problem ansehen, als die Transparenz eines Abstimmungsvorgangs darunter
leidet, erst recht dann, wenn sich alle Beteiligten solcher Ticks bedienen.
  
Nun wäre es sehr naheliegend, um solche Probleme zu vermeiden, die Forderung zu
erheben, nur solche Abstimmungsverfahren zu verwenden, bei denen keine
"`Artefakte"' auftreten können. Leider gibt es, wie u.a. der weiter unten
(Kapitel \ref{SatzVonArrow}) zu besprechende Satz von Arrow zeigt, kein
Verfahren, das in dieser Hinsicht alle Wünsche erfüllen könnte. Irgendwelche
(möglichen) Artefakte muss man bei jedem Abstimmungsmechanismus in Kauf nehmen.
Und welches Abstimmungsverfahren man unter dieser Bedingung für das
"`bestmögliche"' hält, hängt wiederum davon ab, welche Einschränkungen man bereit
ist in Kauf zu nehmen. Darüber und auch über die Frage, wie gravierend diese
Schwierigkeiten insgesamt sind, werden wir uns ausführlich im nächsten Kapitel
(Kapitel \ref{SozialwahltheorieDiskussion}) unterhalten.\footnote{Alle hier
aufgezählten "`Probleme"' und noch einige mehr werden nicht ohne einen gewissen
Hang zur Dramatisierung bei William Riker breit getreten \cite[]{riker:1982}.
Eine knappe und sehr verständliche Zusammenfassung der beschriebenen Phänomene
findet man bei Gerry Mackie \cite[S. 5-9]{mackie:2003}, der Riker's skeptischen
Schlussfolgerungen bezüglich demokratischer Entscheidungsverfahren ansonsten aber
entschieden wirderspricht.}
 
Schließlich, und als wären die aufgezählten Probleme: zyklische kollektive
Präferenzen, Pfadabhängikeit, Manipulation durch Festlegung der
Abstimmungsorgnung bzw. -reihenfolge, strategisches Wählen nicht schon genug,
kann man auch das der Tatsache, dass es keinen einzigen Abstimmungsmechanismus
gibt, der alle Probleme vermeidet, sondern eine Vielzahl von alternativen
Abstimmungsverfahren mit jeweils unterschiedlichen Schwierigkeiten, ein Problem
machen. Denn da unterschiedliche Abstimmungsmechanismen unter Umständen zu
unterschiedlichen Ergebnissen führen, so entsteht auch auf dieser ebene ein
Kontingenzproblem: Wie kann man noch von einem Abstimmungsverfahren sagen, dass
es die individuellen Präferenzen in angemessener Form berücksichtigt und zu einer
kollektiven Präferenz bündelt, wenn es mehrere mehr oder weniger gleich guter und
gleich schlechter Verfahren gibt, die möglicherweise zu unterschiedlichen
Ergebnissen führen.

Zum Schluss sein noch darauf hingewiesen, dass es sich bei den hier beschriebenen
Phänomenen nicht ausschließlich um ein Problem von Abstimmungen und
Kollektiventscheidungen (auch wenn es dabei vielleicht häufiger auftritt), denn
nach dem gleichen Muster kann man -- wie zuvor (S.
\pageref{intransitivePraeferenzen}) schon einmal angedeutet -- auch zyklische
individuelle Präferenzen konstruieren. Insofern ist es ein Problem, dass den Kern
der Theorie betrifft. Dazu ein Beispiel: Eine Person steht vor der Wahl mit
welchem ihrer drei Kollegen und Kolleginnen Peter, Lisa und Klaus sie gemeinsam
an einem Projekt arbeiten möchte. Die drei Kollegen und Kolleginnen unterscheiden
sich dabei hinsichtlich der drei Eigenschaften nett, fleißig und pünktlich. In
der folgenden Tabelle ist die Rangfolge der Kollegen und Kolleginnen für jede
dieser Eigenschaften angegeben:

\begin{center}
\begin{tabular}{c|ccc}
   & nett & fleißig & pünktlich \\
\cline{1-4}
1. & Peter & Lisa  & Klaus \\
2. & Lisa  & Klaus & Peter \\
3. & Klaus & Peter & Lisa \\
\end{tabular}
\end{center}

Geht man danach, welcher Kollege bei mehr guten Eigenschaften besser ist als ein
anderer (paarweiser Vergleich nach dem Condorcet-Verfahren), so ergibt sich auf
ganz natürliche Weise die "`zyklische"' Präferenzstruktur: $Peter \succ Lisa
\succ Klaus \succ Peter$.

Das Muster der Verteilung individueller Präferenzen, das sich in beiden Tabellen
wiederfindet, tritt in der Sozialwahltheorie ebenso wie in der Wahl- und
Abstimmungstheorie sehr häufig auf. Viele "`paradoxe"' Ergebnisse in diesen
Theorien beruhen in der ein- oder anderen Weise auf diesem Muster, so auch das
weiter unten folgende "`Paradox des Liberalismus"'.



\subsection{Das sogenannte "`Paradox des Liberalismus"'} 
\label{LiberalismusParadox}
Nach diesem Einstieg gehen wir nun zunächst zu einem der einfacheren Beispiele
der Sozialwahltheorie über, dem sogennanten "`Paradox des Lieberalismus"' von
Amartya Sen \cite[]{kliemt-lahno:2005}. Die Bezeichnung erscheint -- zumindest im
Deutschen -- ein wenig unglücklich, denn es handelt sich dabei
eher um ein Paradox der Demokratie als des Liberalismus im engeren Sinne. Hinter
dem Namen verbirgt sich jedenfalls Folgendes: Um faire Kollektiventscheidungen über eine
Menge von Alternativen zu treffen, soll eine "`Verfassung"' verabschiedet werden,
die ein entsprechendes Entscheidungsverfahren vorgibt, das folgenden Bedingungen
genügt:

\marginline{Vor\-aus\-setzungen}
\begin{enumerate} 
  \item {\em Minimale Fairness}\footnote{Zuweilen wird diese Bedingung auch als
  "`Bedingung des minimalen Liberalismus"' bezeichnet
  \cite[]{kliemt-lahno:2005}. Aber die Bezeichnung ist schon deshalb irreführend, 
  weil "`Liberalismus"' eigentlich meint, dass es 
  bestimmte Dinge gibt, die überhaupt nicht kollektiv entschieden werden
  müssen, nicht aber, dass bei einem Kollektiventscheidungsverfahren jeder
  einmal zum Zuge kommen müsse.} (Prärogativrecht): Jeder soll das Recht haben,
  die Kollektiventscheidung für mindestens ein Paar von Alternativen festzulegen. 
  Wer über welches Paar von Alternativen 
  entscheiden darf, wird in der Verfassung festgelegt. Die Bedingung der
  "`minimalen Fairness"' garantiert jedem, nicht vollständig übergangen zu
  werden.
  
  \item {\em Unbeschränkter Bereich}: Jedes beliebige individuelle
  Präferenzprofil ist zugelassen (sofern es die Bedingungen einer wohlgeformten
  Präferenzrelation erfüllt). Diese Bedingung besagt einerseits, dass die
  Individuen völlig frei sind, ihre persönlichen Präferenzen zu wählen, und
  andererseits, dass die gesuchte Entscheidungsprozedur der Möglichkeit
  beliebig verteilter individueller Präferenzen Rechnung tragen muss.
  
  \item {\em Einstimmigkeit} oder auch "`Pareto-Effizienz"': Wenn alle
  Individuen eine bestimmte Alternative einer anderen vorziehen, dann sollte
  auch nach dem Kollektiventscheidungsverfahren diese Alternative vor der anderen
  rangieren.\footnote{Da sie etwas leichter zu verstehen ist, wurde hier als
  Voraussetzung die {\em schwache} Pareto-Bedingung anstatt der sonst üblichen
  {\em starken} Paretobedingung gewählt. Der Beweis lässt sich aber genauso
  mit der starken Pareto-Bedingung führen (siehe Aufgabe \ref{AufgPareto}).}
\end{enumerate}

\marginline{Beweis}
Allen drei Bedingungen kommt ein gewisser Grad von Selbst\-ver\-ständ\-lich\-keit
zu, d.h. man ist leicht geneigt zu verlangen, dass jede einigermaßen faire und
sinnvolle Entscheidungsprozedur mindestens diese drei Bedingungen
erfüllt. Es lässt sich nun jedoch zeigen, dass es unmöglich ist, alle drei
Bedingungen auf einmal zu erfüllen. Um das zu zeigen, gehen wir von dem
einfachsten Fall aus, in dem wir es mit zwei Individuen und drei Alternativen zu
tun haben. Die Individuen bezeichnen wir mit $A$ und $B$, die Alternativen mit
$x,y,z$. Nun soll in der "`Verfassung"' festgeschrieben werden, wer über welches
Paar von Alternativen entscheiden darf. Wir nehmen an, dass das Individuum $A$
über $y$ und $z$ und Individuum $B$ über $x$ und $z$ entscheiden darf, d.h. wenn
$P$ die Menge der Alternativen bezeichnet, über die ein Individuum die
"`Prärogative"' ausübt, dann gilt:

\[P_A = \{x,z\} \]
\[P_B = \{y,z\} \]

Die Unmöglichkeit eines Entscheidungsverfahrens, das alle drei Bedingungen
erfüllt, ist dann bewiesen, wenn wir Präferenzen für $A$ und $B$ finden, mit
denen keine eindeutige Kollektiventscheidung mehr getroffen werden kann.
Dies ist aber für folgende Präferenzen der Fall:

\[A:\qquad y \succ x \succ z \]
\[B:\qquad z \succ y \succ x \]

Mit diesen Präferenzen kann keine der drei Alternativen als die beste
gewählt werden, denn: 
\begin{enumerate}
  \item Aufgrund der Präferenzen von $A$, und da $A$ die
  Prärogative über $x$ und $z$ ausübt, kann $z$ nicht gewählt werden.
  \item Aufgrund der Präferenzen von $B$, und da $B$ die Prärogative
  über $y$ und $z$ ausübt, kann $y$ nicht gewählt werden.
  \item Aufgrund der Einstimmigkeitsbedingung und der Präferenzen beider, kann
  aber auch nicht $x$ gewählt werden.  
\end{enumerate}
Damit ist gezeigt, dass es unmöglich ist, ein Entscheidungsverfahren zu finden,
dass die Präferenzen von $A$ und $B$ unter Berücksichtigung der Fairness-,
Unbeschränktheits- und Einstimmigkeitsbedingung auf kollektive Präferenzen
abbilden kann, da keine der möglichen Alternativen in der kollektiven
Präferenzordnung an erster Stelle auftauchen dürfte.

Die Gültigkeit des Beweises hängt nicht davon ab, welche Prärogativen man wählt
(Übungsaufgabe \ref{AufgPL1}). Es ist aber sehr wohl entscheidend für den
Beweis, dass die Prärogativen im vorhinein festgelegt werden, d.h. bevor etwas über die
Präferenzen der Individuen bekannt ist (Übungsaufgabe \ref{AufgPL2}).

\marginline{Beweistechnik}
An dieser Stelle sei ein kleiner Einschub gestattet zu der
Frage: Wie kommt man auf diese Lösung? Die Beweisführung gelingt nämlich nur, wenn man
zuvor die Präferenzen der Individuen geschickt festlegt. Wie findet man aber
heraus, welches die Präferenzen sind, mit denen sich der Beweis nachher richtig
führen lässt? Nun, in diesem Fall sollte man versuchen, die Präferenzen
ausgehend von den drei Bedingungen zu wählen (wobei die Bedingung des
unbestimmten Bereiches schon dadurch abgegolten ist, dass wir die Präferenzen
frei wählen dürfen, und hier also nicht noch einmal in Betracht kommt). Dabei
ist es hilfreich, wenn man mit der Einstimmigkeitsbedingung anfängt. Damit man 
aufgrund der Einstimmigkeitsbedingung eine Alternative ausschließen kann, müssen
die Präferenzen beider Individuen auf jeden Fall bei einem Paar von Alternativen
(hier $x$ und $y$) gleichgeordnet sein. So scheidet aufgrund der
Einstimmigkeitsbedingung schon einmal eine Alternative aus. 
Die verbleibende Alternative ($z$) muss nun so in die Präferenzen eingeordnet werden, 
dass mit Hilfe der Prärogative des einen Individuums, die bevorzugte der beiden
anderen Alternativen ($y$) ausfällt, und dass zugleich die verbleibende
Alternative ($z$) ausgeschlossen wird.

\marginline{Geringe inhaltliche Bedeutung}
Kann man aus diesem Beweis inhaltliche Schluss\-fol\-ger\-ung\-en be\-züg\-lich
der Demokratie bzw. der Möglichkeit und Fairness demokratischer
Entscheidungsverfahren ziehen? Mit einiger Vorsicht kann wohl folgende
Schlussfolgerung gezogen werden: Eine Idealvorstellung dergestalt, dass in der
Demokratie den Interessen jedes Bürgers (ausgedrückt durch die Präferenzen)
wenigstens eine gewisse Berücksichtigung (ausgedrückt durch die Prärogative)
garantiert (unbeschränkter Bereich) werden könnte, lässt sich nicht unter allen
Umständen (Effizienz- bzw. Einstimmigkeitsgebot) halten.

Wie man sieht -- aber das ist ein Grundproblem des Ansatzes -- sind inhaltlich
nur relative schwache, d.h. nahe an der Grenze zur reinen Binsenweisheit
liegende Schlussfolgerungen möglich. Denn, dass in der Demokratie nicht alle
Interessen berücksichtigt werden (können), ist schon aus anderen, pragmatischen
Gründen relativ offensichtlich. Zugleich ist aber jedem die Möglichkeit und damit
auch die Chance gegeben, für die eigenen Interessen zu kämpfen. Dass diese
Chancen höchst ungleich verteilt sind, stimmt leider ebenso, hängt aber weniger
mit logisch-mathematischen Abbildungsproblemen als mit der
innergesellschaftlichen Reichtums-, Macht- und Einkommensverteilung etc.
zusammen.\footnote{Aufschlussreich hinsichtlich der Machtressourcenverteilung
als Funktionsvoraussetzung der Demokratie ist die Zusammenfassung bei Schmidt
\cite[S. 438ff.]{schmidt:2000}.}

Aber auch wenn keine
unmittelbaren starken demokratietheoretischen Schlussfolgerungen aus dem
"`Paradox des Liberalismus"' gezogen werden können, ist ein Verständnis der
logischen Eigenschaften von Abstimmungs- bzw. Kollektiventscheidungsverfahren
-- neben den nicht minder wichtigen psychologischen Rahmenbedingungen --
wichtig, wenn es um die Frage geht, welche Abstimmungsverfahren man für welchen
Zweck heranziehen bzw. wie man sie gestalten sollte.

\subsection{Der "`Klassiker"' der Sozialwahltheorie: Der Satz von Arrow}
\label{SatzVonArrow}
Ein historischer Vorläufer des sogennanten "`Paradox des Liberalismus"' und
recht eigentlich der Klassiker der Sozialwahltheorie ist allerdings der "`Satz
von Arrow"'. Der Beweis des "`Satzes von Arrow"' ist einiges komplizierter als
das "`Paradox"' des Liberalismus, sollte aber, da er im Grunde nur relativ
elementare mathematische Mittel voraussetzt, dennoch verständlich sein. Um es
so einfach wie möglich zu machen, wird der Beweis in drei Teilbeweise zerlegt,
die wir Schritt für Schritt durchgehen werden. 

{\em Interessierte können sich
gerne auch den zweiten und dritten Beweis in diesem Skript durchlesen.
Besonders der dritte Beweis sollte, da er recht ähnlich ist, nicht mehr allzu
schwer verständlich sein, wenn man den ersten Beweis erst einmal begriffen hat!}

\subsubsection{Das Theorem}

Der Satz von Arrow zeigt -- ähnlich wie Sens sog. "`Paradox des Liberalismus"' --
dass eine Abbildung individueller Präferenzen auf eine kollektive
Präferenzordnung nicht mehr möglich ist, wenn man nur ein par
"`selbstverständliche"' Anforderungen an diese Abbildung stellt. Wenn wir dieses
zunächst einmal mathematisch abstrakte Resultat auf demokratische
Entscheidungsfindungsprozesse übertragen, dann besagt es, dass bestimmte
normative Kriterien wie etwa 1) dass jeder eine faire Chance bekommen soll, 2)
dass die Entscheidungsfindung effizient sein soll, 3) dass die
Entscheidungsprozedur auch bei höchst unterschiedlichen Meinungen noch
funktioniert, miteinander unvereinbar sein können. Da man dies den entsprechenden
normativen Kriterien nicht unmittelbar ansieht, hat das Resultat schon einige
Bedeutung, indem es uns auf einen möglichen Zielkonflikt aufmerksam macht. Wie
bei beinahe allen Resultaten der Sozialwahltheorie muss man allerdings auch hier
die Frage stellen, inwieweit die abstrakt-mathematische Formulierung die
entsprechenden konkret-empirischen Zusammenhänge richtig erfasst.

Zum Anforderungskatalog, auf den sich der Satz von Arrow bezieht, 
gehören nun folgende Bedingungen:

\begin{enumerate}\label{ArrowVoraussetzungen}\marginline{Arrows Vor\-aus\-setzungen}
  \item {\em Diktaturfreiheit}: Es dürfen sich nicht in jedem Fall (d.h.
  bei jedem möglichen Profil von individuellen Präferenzen)
  die Präferenzen von ein- und demselben Individuum durchsetzen.
  
  {\footnotesize Diese 
  Bedingung ist vergleichweise schwächer als die Bedingung der 
  "`minimalen Fairness"' im Falle des Paradoxes des Liberalismus, 
  indem sie immer noch zulässt, dass einzelne Individuen völlig übergangen
  werden, solange nicht alle bis auf ein Individuum übergangen werden.}
  
  \item {\em Unbeschränkter Bereich}: Jedes beliebige individuelle
  Präferenzprofil, das die Bedingungen einer wohlgeformten
  Präferenzrelation erfüllt, ist zugelassen. 

  \item {\em Einstimmigkeit} bzw. {\em Pareto-Effizienz}: Wenn alle Individuen
  eine bestimmte Alternative einer anderen vorziehen, dann sollte auch nach dem
  Kollektiventscheidungsverfahren diese Alternative der anderen vorgezeogen
  werden.\footnote{Statt der schwachen Pareto-Bedingung kann man hier ebenso gut
  die starke Paretobedingung einsetzen (siehe Aufgabe \ref{AufgPareto}).}

  \item {\em Unabhängigkeit von dritten\footnote{Häufig wird diese Bedingung auch
  "`Unabhängigkeit von {\em irrelevanten} Alternativen"' genannt. Wie bereits
  zuvor (Seite \pageref{dritteAlternativen}) an einigen Beispielen dargelegt, ist
  diese Bezeichnung irreführend, da dritte Alternativen in manchen Fällen sehr
  wohl und zu Recht einen Einfluss auf die Rangordnung eines Paars von
  Alternativen ausüben.} Alternativen} bzw. {\em Paar\-wei\-se
  Un\-ab\-häng\-ig\-keit}: Die Anordnung, die das Kollektiventscheidungsverfahren
  zwei Alternativen zuweist, sollte allein von der Ordnung {\em dieser} beiden
  Alternativen in den Präferenzen der Individuen abhängen und nicht davon, wie
  andere Alternativen in den Präferenzen der Individuen eingeordnet sind.

  {\footnotesize Anders als bei der Paretobedingung legt die Bedingung der
  Unabhängigkeit von dritten Alternativen nicht fest, welche kollektive
  Wahl getroffen werden soll, wenn unterschiedliche Individuuen bezüglich
  bestimmter Alternativen übereinstimmen, sondern vielmehr, welche Wahl
  getroffen werden soll, wenn unterschiedliche Präferenzprofile bezüglich der
  Anordnung bestimmter Alternativen übereinstimmen. Dabei können die
  Individuuen innerhalb der Präferenzordnungen diese Alternativen sehr wohl
  unterschiedlich anordnen (siehe dazu die Aufgaben \ref{AufgArrow1} und
  \ref{AufgArrow2}).}
 
% Wenn sich die Anordnung bestimmter Güter ($x$, $y$) durch die Bürger in einem
% gegebenen Präferenzprofil ($P_1$) nicht von der Anordnung in einem anderen
% Präferenzprofil ($P_2$) unterscheidet, dann sollte sich die Anordnung dieser
% Güter in der durch die soziale Wohlfahrtfunktion aus dem einen Profil
% gewonnenen Präferenzordnung nicht von der aus dem anderen Profil gewonnenen
% unterscheiden.
  
% {\footnotesize Diese Bedingung, die beim sog. "`Paradox des Liberalismus"'
% nicht vorkommt, darf nicht mit der Pareto-Bedingung verwechselt werden. Bei
% der
% Bedingung der Pareto-Effizienz geht es darum, dass alle Individuen in ein- und
% demselben Präferenzprofil ein- und dieselbe Präferenz bezüglich zweier Güter
% haben. Hier geht es aber darum, dass die Präferenzen einzelner Individuen
% bezüglich bestimmter Güter, die sich von Individuum zu Individuum sehr wohl
% unterscheiden können, in {\em unterschiedlichen} Profilen genauso
% wiederkehren.}
  
% {\footnotesize Mann könnte das Prinzip der Unabhängigkeit von irrelevanten
% Alternativen auch so formulieren: Wenn dieselben Präferenzen in
% unterschiedlichen Präferenzprofilen eingebettet sind, darf dies keinen
% Unterschied in der Abbildung dieser Präferenzen durch die soziale
% Wohlfahrtsfunktion bewirken.}
 
\end{enumerate}


% \setlength{\parindent}{0em}
{\bf Theorem (Satz von Arrow)}:\marginline{Satz von Arrow} {\em Es gibt (bei zwei
oder mehr Individuen und drei oder mehr zur Wahl stehenden Alternativen) kein
Kollektiventscheidungsverfahren, das individuelle Präferenzordnungen so auf eine
kollektive Präferenzordnung abbildet, dass die Bedingungen der Diktaturfreiheit,
der Ein\-stim\-migkeit und der Unabhängigkeit von dritten Alternativen für alle
denk\-bar\-en indvididuellen Präferenzordnungen erfüllt sind.}
~\\

Um den Beweis des Theorems vorzubereiten, führen wir zunächst zwei weitere
Definitionen ein:
\begin{enumerate}\marginline{entscheidende Mengen}
  \item Eine Menge von Individuen ist {\em vollständig entscheidend} für $x$
  über $y$, wenn das Kollektiventscheidungsverfahren $x \succ_K y$ liefert,
  sobald jedes Individuum aus dieser Menge $x$ gegenüber $y$ vorzieht.

  \item Eine Menge von Individuen ist {\em beinahe entscheidend} für $x$ über
  $y$, wenn das Kollektiventscheidungsverfahren $x \succ_K y$ liefert, sobald
  alle Individuen aus dieser Menge $x$ gegenüber $y$ vorziehen {\em und} alle
  Individuen außerhalb dieser Menge $y$ gegenüber $x$ vorziehen.
  
  {\footnotesize Umgangssprachlich besagt die Definition also, dass eine Menge
  von Individuen "`beinahe entscheidend"' ist, wenn sie nur in dem Extremfall
  maximaler Opposition von außerhalb entscheidend ist, aber nicht in anderen
  Fällen. Es gilt daher, dass eine Menge von Individuen, die "`entscheidend"'
  ist, immer auch "`beinahe entscheidend"' ist, aber nicht umgekehrt.}
  
  Anmerkungen:
  \begin{enumerate}
    \item Wenn eine Menge von Individuen beinahe (bzw. vollständig) entscheidend
    für $x$ über $y$ ist, so muss noch lange nicht gelten, dass sie auch
    beinahe (bzw. vollständig) entscheidend für $y$ über $x$ ist.
    \item \label{Anmerkung2} \marginline{Existenz mindestens einer entscheidenden
    Menge} Für jede Menge von Individuen und jedes Paar von
    Alternativen gibt es wenigstens eine beinahe (bzw. eine vollständig) 
    entscheidende Menge. Aufgrund der Einstimmigkeitsbedingung ist für jedes
    Paar von Alternativen nämlich die
    Menge aller Individuen eine zugleich beinahe als auch
    vollständig entscheidende Menge, denn, sobald alle Individuen $x$ der
    Alternative $y$ vorziehen, fordert die Einstimmigkeitsbedingung, 
    dass auch kollektiv $x \succ_K y$ gilt.
    \item Wenn eine Menge, die nur ein Individuum enthält, vollständig
    entscheidend sowohl für $x$ über $y$ als auch für $y$ über $x$ ist, dann
    soll das Individuum "`Diktator"' für die Alternative $x$ oder $y$ heißen.
  \end{enumerate} 

%     
%   \item Ein Individuum ist eine Diktatorin oder ein {\em Diktator für $x$ über
%   $y$} in dem Fall, dass die Menge, die nur aus ihm allein besteht, entscheidend
%   für $x$ über $y$ ist.
% 
%   \item Ein Individuum ist beinahe Diktator oder {\em beinahe Diktatorin für
%   $x$ über $y$}, wenn die Menge, die nur aus diesem Individuum besteht, beinahe
%   entscheidend für $x$ über $y$ ist.
% 
%   \item Ein Individuum ist {\em uneingeschränkt Diktator} oder Diktatorin, wenn
%   es für jedes Paar von Alternativen Diktator oder Diktatorin ist.
% 
%   \item Eine soziale Wohlfahrtsfunktion ist eine {\em diktatorische
%   Wohlfahrtsfunktion}, wenn es ein Individuum gibt, dass bei dieser
%   Wohlfahrtsfunktion Diktator oder Diktatorin ist. (In diesem Fall 
%   verletzt die Wohlfahrtsfunktion das Prinzip der Diktaturfreiheit.)
\end{enumerate}

\subsubsection{Der Beweis des Theorems}
\label{BeweisArrow}
\marginline{Beweis nach Vickrey}
Der wahrscheinlich einfachste Beweis, der sich für den Satz von Arrow finden
lässt, folgt weitgehend Dennis Mueller \cite[S. 583f.]{mueller:2003}, der
sich für seine Skizze wiederum auf William Vickrey stützt. Der Satz von Arrow wird
dabei über drei Zwischenschritte (Lemmata) bewiesen:
\begin{enumerate}\marginline{Grobstruktur des Beweises}
  \item Lemma: Sei $D$ eine Teilmenge von Individuen, die
  beinahe entscheidend für $x$ über $y$ ist, dann ist $D$
  beinahe entscheidend für alle Alternativen.
  
  \item Lemma: Sei $D$ beinahe entscheidend für alle
  Alternativen, dann enthält $D$ ein Individuum $J$, das
  (bereits allein) beinahe entscheidend für alle Alternativen ist.
 
  \item Lemma: Ist ein Individuum $J$ beinahe entscheidend für
  alle Alternativen, dann ist $J$ auch {\em vollständig} entscheidend
  für alle Alternativen (und damit Diktator für alle Alternativen).
\end{enumerate}


\paragraph{Beweis von Lemma 1} Sei $D$ eine Teilmenge von Individuen, die
  beinahe entscheidend für $x$ über $y$ ist, dann ist $D$
  beinahe entscheidend für alle Alternativen. 

\begin{enumerate}
  \item Sei $D$ eine Menge von Individuen, die beinahe
    entscheidend für $x$ über $y$ ist, wobei $x$ und $y$ irgendein Paar von
    Alternativen ist. ({\em Anmerkung \ref{Anmerkung2} auf Seite
    \pageref{Anmerkung2}})
 
  ~\\{\em 1. Teil (Ersetzbarkeit von rechts) }\label{Lemma1ErsetzbarkeitVonRechts} 
 
  \item Annahme: Für alle Individuen in $D$ und eine beliebige dritte
  Alternative $u$ gelte $x \succ y \succ u$ und für alle anderen Individuen $y
  \succ u \succ x$. ({\em Unbeschränkter Bereich})
  
  \item Dann gilt für das Kollektiv: $x \succ_K y$. ({\em $D$ ist nach
  1. beinahe entscheidend})
  
  \item Und es gilt für das Kollektiv: $y \succ_K u$. ({\em
  Einstimmigkeit iVm 2.})
  
  \item Und es gilt für das Kollektiv: $x \succ_K u$. ({\em Transitivität iVm 3.
  und 4.})

  \item Für das Kollektiv muss $x \succ_K u$ unabhängig davon gelten, wie die
  anderen Alternativen, einschließlich $y$, von den Individuen eingeordnet
  werden. ({\em Unabhängigkeit von dritten Alternativen})

  \item Also ist $D$ beinahe entscheidend für $x$ über $u$ (für jedes beliebige
  $u$, das nicht identisch mit $x$ oder $y$ ist). ({\em Definition beinahe
  entscheidender Mengen iVm 2., 5. und 6.})

    ~\\{\em Damit ist der erste Teil des Beweises von Lemma 1 abgeschlossen. Was
    bis hierher bewiesen wurde ist: Wenn eine Menge $D$ für $x \succ_K y$
    entscheidend ist, dann dürfen wir in dieser Formel den {\em rechten} Term
    (also das $y$) durch jede beliebige dritte Alternative ($u$) ersetzen, und
    die Aussage stimmt immer noch. Nun wird noch gezeigt, dass das für den
    linken Term (also das $x$) ganz genauso gilt.}
   
    ~\\{\em 2. Teil (Ersetzbarkeit von links)}

  \item Nun nehme man anstelle der unter Punkt 2 getroffenen Annahme für $D$ die
  Präferenzen $u' \succ x \succ y$ an, und für alle anderen Individuen $y \succ
  u' \succ x$. Dabei kann $u'$ jede beliebige Alternative außer $x$ und $y$
  sein. ({\em Unbeschränkter Bereich})
  
  \item Dann gilt für das Kollektiv: $x \succ_K y$. ({\em $D$ ist nach
  1. beinahe entscheidend})
  
  \item Und es gilt für das Kollektiv: $u' \succ_K x$. ({\em
  Einstimmigkeit iVm 8.})
  
  \item Und es gilt für das Kollektiv: $u' \succ_K y$. ({\em Transitivität iVm
  9. und 10.})

  \item Für das Kollektiv muss $u' \succ_K y$ unabhängig davon gelten, wie die
  anderen Alternativen, einschließlich $x$, von den Individuen eingeordnet
  werden. ({\em Unabhängigkeit von dritten Alternativen})  

  \item \label{KleineBeweisAufgabe} Dann gilt auch: $D$ ist beinahe entscheidend
  für $u' \succ_K y$ (wobei $u'$ eine beliebige Alternative außer $x$ und $y$
  ist). ({\em Definition beinahe entscheidender Mengen iVm 2., 11. und 12.})  

    ~\\{\em Damit ist gezeigt, dass wir auch den {\em linken} Term (das $x$)
    in der Aussage, dass $D$ eine entscheidende Menge für $x \succ_K y$ ist,
    durch eine beliebige dritte Alternative ($u'$) ersetzen dürfen, ohne dass die
    Aussage falsch wird. Zusammen mit dem Resultat vom ersten Teil des
    Beweises bedeutet das, dass wir in der Formel $x$ und $y$ beliebig
    durch andere Alternativen ersetzen dürfen (siehe Übungsaufgabe
    \ref{AufgArrow4}).}
   
    ~\\{\em Schluss}

  \item Aber dann ist $D$ beinahe entscheidend für alle Paare von Alternativen.
  ({\em Sukzessives Ersetzen von {\em u} im 1.Teil und von {\em u'} im 
  2.Teil des Beweises})
\end{enumerate}


\paragraph{Beweis von Lemma 2} Sei $D$ beinahe entscheidend für alle
  Alternativen, dann enthält $D$ ein Individuum, das bereits
  allein beinahe entscheidend für alle Alternativen ist.

\begin{enumerate}
  \item Sei $D$ eine Menge von Individuen, die beinahe entscheidend für alle
  Alternativen ist. ({\em Anmerkung \ref{Anmerkung2} auf Seite
    \pageref{Anmerkung2} iVm Lemma 1})
  
  \item Wenn $D$ aus nur einem Individuum besteht, dann gilt die Folgerung von
  Lemma 2 bereits. ({\em offensichtlich})
  
  \item Besteht $D$ aus zwei oder mehr Individuen, dann kann $D$ in zwei
  nichtleere, disjunkte Teilmengen $A$ und $B$ aufgeteilt werden. ({\em
  elementare Mengentheorie})
  
  \item Angenommen, für alle Individuen aus $A$ gelte $x \succ y \succ u$, für
  Individuen aus $B$ gelte $y \succ u \succ x$ und für alle anderen Individuen
  gelte $u \succ x \succ y$. ({\em Unbeschränkter Bereich})
  
  \item Für das Kollektiv gilt $y \succ_K u$. ({\em $A \cup B = D$ (3.) und $D$
  ist beinahe entscheidend (1.) iVm mit den angenommenen Präferenzen (4.)})
  
  ~\\{\em Fallunterscheidung: 1. Fall}
  
  \item Falls für das Kollektiv $y \succ_K x$ gilt, dann ist $B$
  beinahe entscheidend für $y$ über $x$. ({\em Definition von "`beinahe
  entscheidend"' iVm mit den Präferenzen (4.) und der Unabhängigkeit von
  dritten Alternativen})
  
  \item Aber dann ist $B$ auch beinahe entscheidend für alle Alternativen.
  ({\em Lemma 1})
  
  ~\\{\em Fallunterscheidung: 2. Fall}
   
  \item Falls für das Kollektiv $x \succ_K y$ gilt, dann gilt für das Kollektiv
  auch $x \succ_K u$. ({\em Transitivität iVm 5.})
  
  \item Aber dann ist $A$ beinahe entscheidend für $x$ über $u$. ({\em
  Definition von "`beinahe entscheidend"' iVm mit den Präferenzen (4.) und der
  Unabhängigkeit von dritten Alternativen})

  \item Und $A$ ist auch beinahe entscheidend für jede andere Alternative.
  ({\em Lemma 1})

  ~\\{\em Ende der Fallunterscheidung}

  \item Eine echte Teilmenge von $D$ (nämlich entweder $A$ oder $B$) ist beinahe
  entscheidend für alle Alternativen. ({\em Zusammenführung der Konsequenzen
  beider Fälle der Fallunterscheidung})
   
  \item Es gibt eine Teilmenge von $D$, die nur ein Individuum enthält, das für
  alle Alternativen beinahe entscheidend ist. ({\em Wiederholung der Schritte
  2.-11. für diejenige echte Teilmenge von $D$, die beinahe entscheidend für alle
  Alternativen ist, solange, bis sie nur noch ein Individuum enthält.})
\end{enumerate}


\paragraph{Beweis von Lemma 3} Ist ein Individuum beinahe entscheidend für
  alle Alternativen, dann ist dasselbe Individuum auch {\em vollständig} entscheidend
  für alle Alternativen.
  
\begin{enumerate}
  \item Sei $J$ das Individuum, das beinahe entscheidend für alle
  Alternativen ist. ({\em Anmerkung \ref{Anmerkung2} auf Seite
  \pageref{Anmerkung2} iVm Lemma 1 und Lemma 2})
 
  \item Angenommen, für $J$ gelten die Präferenzen $x \succ y \succ u$ und für
  alle anderen Individuen gelte sowohl $y \succ x$ als auch $y \succ u$,
  wobei für die Ordnung von $x$ und $u$ bei den anderen Individuen beliebiges
  gelten kann. ({\em Unbeschränkter Bereich})
  
  \item Dann gilt für das Kollektiv $x \succ_K y$. ({\em $J$ ist beinahe
  entscheidend für alle Alternativen, also auch insbesondere für $x \succ_K y$
  iVm 2.})
  
  \item Und es gilt für das Kollektiv $y \succ_K u$. ({\em Einstimmigkeit iVm
  2.})
  
  \item Dann gilt für das Kollektiv aber auch $x \succ_K u$. ({\em
  Transitivität})
  
  \item Für das Kollektiv muss $x \succ_K u$ unabhängig davon gelten, wie die
  anderen Alternativen, einschließlich $y$, von den Individuen eingeordnet
  werden. ({\em Unabhängigkeit von dritten Alternativen})
  
  \item Dann ist $J$ vollständig entscheidend für $x \succ_K u$. ({\em
  Definition von "`vollständig entscheidend"' iVm 2., insbesondere da unter
  2. die Ordnung von $x$ und $u$ für alle Individuen außer $x$ offen gelassen
  wurde.})

  \item \label{Lemma3Schritt8} In den Schritten 2. bis 7. wurde gezeigt, dass
  $J$ vollständig entscheidend für $x \succ_K u$ ist -- bei beliebig gewählten, aber bestimmten
  $x,y,u$. Um nun von irgendeinem Paar $v,w$ zu zeigen, dass $J$ vollständig
  entscheidend für $v \succ_K w$ ist, ersetze man im 2. Beweisschritt des Lemmas
  $x$ durch $v$, $u$ durch $w$ und $y$ durch eine beliebige Alternative außer $v$
  und $w$ und gehe dann die Schritte 2. bis 7. für die eingesetzten Alternativen
  durch.

  \item $J$ ist {\em vollständig} entscheidend für alle Alternativen. ({\em
  Anwendung des letzten Schrittes auf jedes mögliche Paar von Alternativen.})
  
\end{enumerate}

Aus der Voraussetzung, dass es immer eine Teilmenge $D$ und ein Paar von
Alternativen $x$ und $y$ gibt, für die $D$ beinahe entscheidend ist (siehe
Anmerkung \ref{Anmerkung2} auf Seite \pageref{Anmerkung2}) ergibt sich in
Verbindung mit Lemma 1, 2 und 3, dass es ein Individuum $J$ gibt, dass 
{\em vollständig entscheidend} für alle Alternativen ist. Da dies dem Prinzip
der {\em Diktaturfreiheit} widerspricht, ist es nicht möglich die
Voraussetzungen des unbeschränkten Bereichs, der Einstimmigkeit, der
Unabhängigkeit von dritten Alternativen und der Diktaturfreiheit
gleichzeitig zu erfüllen. Damit ist der Satz von Arrow bewiesen.

\subsubsection{Ein alternativer Beweis}
\label{AlternativerBeweis}

\marginline{Beweis nach Geanakoplos}
Dasselbe Theorem kann auch auf andere Weise bewiesen werden. Zum tieferen
Verständnis und weil dieser zweite Beweis etwas andere Beweistechniken
einsetzt, sei er hier auch aufgeführt. Der Beweis stammt von John Geanakoplos
\cite[]{geanakoplos:1996} und läuft folgendermaßen:

\paragraph{Teil 1}

\begin{enumerate}
  \item Gegeben sei eine Menge von mindestens drei Alternativen, die mit
  Kleinbuchstaben $x$, $y$, $z$,\ldots bezeichnet werden.
  
  \item Wenn alle Individuen $y$ am wenigstens schätzen, dann muss
  $y$ auf Grund des {\em Einstimmigkeitsprinzips} auch die schlechteste 
  kollektive Wahl sein.
  
  Ein Präferenzprofil, bei dem alle Individuen $y$ als die schlechteste
  Alternative bewerten, nennen wir ein "`Profil vom Typ 1"' oder kürzer: {\em
  Profil 1}.\footnote{Die abgekürzte Benennung, die suggeriert, es handele sich
  dabei nur um ein einzelnes Präferenzprofil und nicht vielmehr um eine ganze
  Gruppe von Präferenzprofilen, ist -- wie sich aus dem Folgenden ergibt --
  durch die Bedingung der Unabhängigkeit von dritten Alternativen
  gerechtfertigt.}

  \item Wenn andererseits alle Individuen $y$ am meisten schätzen, dann muss $y$
  ebenfalls auf Grund des Einstimmigkeitsprinzips die beste kollektive Wahl sein.
  
  Ein Präferenzprofil, bei dem alle Individuen $y$ als die beste Alternative
  bewerten, nennen wir ein "`Profil vom Typ 2"' oder kürzer: {\em Profil 2}.
  
  \item Wir betrachten nun einen Übergang von {\em Profil 1} zu {\em Profil 2},
  bei dem die Individuen ausgehend von einem Präferenzprofil vom Typ 1
  nacheinander die Alternative $y$ vom letzten auf den ersten Platz rücken. Die
  Reihenfolge, in der die Individuen diese Änderung vornehmen, ist beliebig
  wählbar, stehe danach aber für den Rest des Beweises fest. Der "`Übergang"'
  besteht also aus einer Anzahl von Schritten, die der Anzahl der Individuen
  entspricht, auf einem Pfad von Präferenzprofilen. Der Pfad ist nicht
  eindeutig,\footnote{Eindeutig ist wohl aber die Reihenfolge der Individuen beim
  "`Übergang"' (zumindest an dieser Stelle des Beweises). Der Pfad ist also nicht
  zu verwechseln mit der Folge der Individuen sondern stellt, gegeben eine
  bestimte festgelegte Folge von Individuen, die Folge der Präferenzprofile dar,
  die entsteht, wenn die Individuen nach einander ihre Präferenzen auf die
  beschriebene Weise ändern.} da in jedem Schritt nur die Position von $y$ für
  alle Individuen festgelegt ist, nicht aber die der anderen Alternativen.
 
  Für diesen Übergang gilt:
 
  \item Bei dem Individuum, bei dem $y$ den letzten Platz in der kollektiven
  Präferenzordnung verlässt (es sei das "`{\em zentrale Individuum}"' oder auch
  das $n$-te Individuum genannt\footnote{Dabei steht $n$ für die Anzahl der
  Schritte bei vorgegebenem Übergang (Punkt 4.), bis das "`zentrale Individuum"'
  erreicht ist.}), rückt $y$ in der kollektiven Präferenzordnung vom letzten
  Platz sogleich auf den ersten Platz. Es gibt keine Zwischenstufen, denn sonst
  gäbe es ein Profil, bei dem alle Individuen bis zum $n$-ten Individuum $y$ an
  die Spitze stellen, alle Individuen ab dem $n$-ten $y$ aber (noch) ans Ende
  stellen, während $y$ in der kollektiven Präferenzordnung zwischen zwei
  Alternativen steht, die $x$ und $z$ genannt seien, so dass $x \succ_K y \succ_K
  z$. Nun könnten aber alle Individuen ihre Präferenzen so abändern, dass $z$ vor
  $x$ eingeordnet wird, ohne dass dadurch das relative Verhältnis von $z$ zu $y$
  bzw. von $x$ zu $y$ in den individuellen Präferenzen geändert wird, da in den
  individuellen Präferenzen $y$ entweder ganz am Anfang oder ganz am Ende, d.h.
  entweder vor $x$ und $z$ oder nach $x$ und $z$ kommt. Aufgrund des
  Einstimmigkeitsprinzips müsste dann aber gelten $z \succ_K x$, und da zuvor
  angenommen wurde $x \succ_K y$, wegen der Transitivität auch $z \succ_K y$, was
  im Widerspruch zur Unabhängigkeit von dritten Alternativen steht.
  
  \item Welches Individuum das {\em zentrale Individuum} ist, ist unabhängig vom
  gewählten Pfad. Denn die Präferenzen der Individuen stimmen hinsichtlich der
  relativen Ordnung von $y$ zu allen anderen Alternativen beim gleichen Schritt
  zwischen sämtlichen möglichen Pfaden überein. Wegen der Unabhängikeit von
  dritten Alternativen, muss $y$ dann aber auch innerhalb der kollektiven
  Präferenzen beim gleichen Schritt an derselben Position (Anfang oder Ende)
  stehen.
  
  \item Das $n$-te Individuum ist auch das {\em zentrale Individuum} bezüglich
  jeder Teilmenge von Alternativen, die $y$ enthält, denn jeder mögliche Pfad bei
  allen Alternativen ist auch ein möglicher Pfad, wenn die Betrachtung auf eine
  Teilmenge von Alternativen beschränkt wird. Da die Eigenschaft, zentrales
  Individuum zu sein, pfadunabhängig ist (siehe den vorhergehenden Punkt), muss
  das zentrale Individuum für die Teilmenge dasselbe sein.
\end{enumerate}

\paragraph{Teil 2}

\begin{enumerate}
  \item Man betrachte nun die Folge von Individuen, in der alle bis zum $n$-ten
  Individuum $y$ an die erste Position setzen, alle ab dem $n$-ten Individuum
  $y$ an die letzte Position setzen, während das $n$-te Individuum $y$ nach
  einer Alternative $x$ und vor einer Alternative $z$ einordnet, 
  also $x \succ_n y \succ_n
  z$.\footnote{Das Suffix "`n"' bei $\succ_n$ deutet an, dass es sich hier um
  die Präferenzen des $n$-ten Individuums handelt.} (Ein Präferenzprofil, dass
  damit übereinstimmt nennen wir Profil vom Typ 3 oder einfach {\em Profil 3}).

  \item\label{Teil2Punkt2} Beschränkt man die Betrachtung auf alle Alternativen
  $\succeq_n y$, so zeigt sich, da das $n$-te Individuum zentrales Individuum
  ist, dass für die kollektiven Präferenzen $x \succ_K y$ gelten muss.
  
  \item Beschränkt man umgekehrt die Betrachtung auf alle Alternativen $\preceq_n
  z$, so zeigt sich aus demselben Grund, dass die kollektive Präferenz $y \succ_K
  z$ gelten muss.
  
  \item Aufgrund der Transititvität folgt aus $x \succ_K y$ und $y \succ_K z$,
  dass $x \succ_K z$ gilt, und zwar für alle Profile vom Typ 3.
  
  \item Wegen der Unabhängigkeit von dritten Alternativen muss $x \succ_K z$
  unabhängig davon gelten, wie die Individuen $y$ zu $x$ und $z$ einordnen. Damit
  gilt $x \succ_K z$ aber genau dann, wenn das zentrale Individuum $x \succ_n z$
  festlegt.\footnote{Anmerkung: Bis zu dieser Stelle spielte die Reihenfolge der
  Individuum beim "`Übergang"' (siehe Teil 1 des Beweises) noch eine Rolle.
  Dieses Resultat ist aber unabhängig von der beim Übergang gewählten
  Reihenfolge.} M.a.W.: Das "`zentrale Individuum"' ist entscheidend für $x$ über
  $z$.\footnote{Zur Erinnerung: Damit, dass das "`zentrale Individuum"'
  entscheidend für $x$ über $z$ (in dieser Reihenfolge!) ist, ist noch nicht
  gesagt, dass das "`zentrale Individuum"' auch entscheidend für $z$ über $x$
  (umgekehrte Reihenfolge!) ist. Das wird erst im folgenden Schritt gezeigt. Und
  erst dann kann man auch sagen, dass das zentrale Individuum insgesamt Diktator
  für das Alternativenpaar $x$,$z$ ist.}
  
  \item Durch Vertauschen von $z$ und $x$ in den Schritten 1.-5. erhält
  weiterhin, dass auch umgekehrt $z \succ_K x \Leftrightarrow z \succ_n x$.
  M.a.W.: Das "`zentrale Individuum"' ist Diktator für das Paar von Alternativen
  $x$, $z$.
\end{enumerate}

\paragraph{Teil 3}

\begin{enumerate}
  \item Entsprechend des bisherigen Beweisgangs können wir zeigen, dass es
  nicht nur für $x$ und $z$, sondern für jedes Paar von Alternativen {\em einen}
  Diktator gibt. Zu zeigen ist noch, dass es sich dabei jedesmal um ein- und
  denselben Diktator handelt.
  
  \item Es gibt also für die Alternativen $x$ und $y$,
  $y$ und $z$ jeweils\footnote{An dieser Stelle ist noch nicht klar, dass es
  ein- und derselbe ist.} einen Diktator.

  \item Dann kann aber kein dritter Diktator allein über $x$ und $z$ entscheiden,
  denn wenn der erste Diktator $x \succ_K y$ festsetzt und der zweite $y \succ_K
  z$, dann ist der dritte wegen der Transitivität nicht mehr frei $x \prec_K z$
  festzulegen. (Dasselbe gilt, wenn man die Zeichen $\succ_K$ und $\prec_K$ im
  vorhergehenden Satz jeweils vertauscht.) Also muss der dritte Diktator
  identisch mit einem der ersten beiden Diktatoren sein.
 
  \item Ist der dritte Diktator aber identisch mit dem ersten, dann kann der
  erste Diktator über $x$ und $y$ und über $x$ und $z$ entscheiden. Wenn der
  erste Diktator nun aber $y \succ_K x \succ_K z$ und damit auf Grund der
  Transitivität $y \succ_K z$ bestimmt, dann ist der zweite nicht mehr frei, $z
  \succ_K x$ fest zu setzen. Also muss der erste Diktator auch identisch mit dem
  zweiten sein.
  
  \item Da $x$, $y$ und $z$ beliebig gewählt wurden, gibt es für jedes Tripel von
  Alternativen genau einen Diktator. Dann gibt es aber überhaupt nur einen
  Diktator, denn jeder Diktator, der über ein Tripel entscheidet, in dem zwei der
  Alternativen $x$, $y$ und $z$ vorkommen, muss mit dem Diktator über $x$, $y$
  und $z$ identisch sein (nur einer von beiden kann ja über dieses Paar
  entscheiden). Für jede beliebige Alternative $u$ außer $x$, $y$, $z$, muss aber
  der Diktator über $x$, $y$, $u$ dann auch identisch mit dem von $x$, $y$, $z$
  sein. Also ist der Diktator von $x$, $y$, $z$, Diktator für alle Alternativen.
\end{enumerate}

Damit ist bewiesen, dass es unter den Bedingungen der Unabhängigkeit von
dritten Alternativen, des unbeschränkten Bereichs und der Einstimmigkeit
(Pareto-Effizienz) bei drei oder mehr Alternativen immer einen Diktator gibt.
Die Bedingung der Diktatorfreiheit ist also nicht mehr erfüllbar, wenn die drei
anderen Bedingungen erfüllt sind.


\subsubsection{Ein dritter Beweis}

Der folgende Beweis stammt aus dem Buch von Resnik \cite[S. 186ff.]{resnik:1987}.
Der Beweis ähnelt sehr stark dem ersten hier vorgestellten Beweis.
Nur wird diesmal nicht zuerst gezeigt, dass es
eine Teilmenge von Individuen gibt, die beinahe entscheidend für alle
Alternativen ist und dann, dass sie tatsächlich nur aus einem Individuum besteht.
Sondern es wird zuerst gezeigt, dass es ein Individuum gibt, dass für eine
Alternative beinahe entscheidend ist, und dann, dass daraus folgt, dass dieses
Individuum für alle Alternativen nicht nur beinahe sondern vollständig
entscheidend ist. Die einzelnen Beweisschritte sind aber zum Teil ähnlich wie
beim ersten Beweis, so dass die Lektüre des zweiten Beweises gut zur Übung und
zum besseren Verständnis dienen kann.

Zunächst wird folgendes Lemma bewiesen: 

\begin{quote}
{\bf Lemma 1}: {\em Es existiert immer ein Individuum, das für irgendein Paar
von Alternativen beinahe entscheidend ist.}
\end{quote} 

{\em Beweis}: Wie oben angemerkt existieren "`entscheidende"' Mengen für jedes
Paar von Alternativen. Da jede "`entscheidende"' Menge immer auch "`beinahe
entscheidend"' ist, existieren für jedes Paar von Alternativen auch beinahe
"`entscheidende"' Mengen. 

Wir setzten voraus, dass die Menge der Individuen und Alternativen endlich ist.
Dann existiert wenigstens eine "`beinahe entscheidende"' Menge, die keine echte
Teilmenge enthält, die "`beinahe entscheidende"' Menge wäre, denn: Man beginne
mit irgend einer beliegigen "`beinahe entscheidenden"' Menge. Hat diese Menge
noch (nicht-leere) Teilmengen, die "`beinahe entscheidende"' Mengen sind, dann
wähle man irgend eine dieser "`beinahe entscheidenden"' Teilmengen und stelle für
diese Teilmenge dieselbe Untersuchung an, solange bis man bei einer Menge
angekommen ist, die keine echten Teilmengen mehr enthält, die ihrerseits
"`beinahe entscheidende"' Mengen irgendeines Paares von Alternativen sind.

Wir verfügen damit über eine "`minimale Menge"', die "`beinahe entscheidend"'
bezüglich eines bestimmten Paares von Alternativen ist. Wenn wir zeigen können,
dass diese "`minimale Menge"' nur noch ein einziges Individuum enthält,
dann haben wir das Lemma bewiesen. Dazu kann ein Widerspruchsbeweis geführt werden.
Wir nehmen also an, es gäbe eine entsprechende "`minimale beinahe entscheidende
Menge"', die mehrere Individuen enthält und zeigen, dass diese Annahme zu einem
Widerspruch führt.

Angenommen also, $M$ sei eine "`minimale beinahe entscheidende Menge"' für die
Alternative $x$ über $y$, die mehrere Individuen enthält. Man betrachte ein
beliebiges Individuum $J$ aus der Menge $M$. Da die Menge $M$ mehr Individuen als
nur $J$ enthält, und da möglicherweise noch ein "`Rest"' von
Individuen existiert, die nicht zu $M$ gehören, kann man folgende drei
unterschiedlichen Gruppierungen betrachten: 1) Die Menge, die nur aus dem 
Individuum $J$ besteht.
2) Die Menge, die aus den Individuen von $M$ ohne $J$ besteht, kurz: $M-J$. 3)
Der "`Rest"', d.h. alle Individuen, die nicht zu $M$ gehören.

Da jedes beliebige Präferenzprofil zugelassen ist ("`unbeschränkter Bereich"')
und sich die Eigenschaft eine (minimale) "`beinahe entscheidende"' Menge zu
sein auf alle Präferenzprofile bezieht, muss sie sich auch bei jedem beliebigen
einzelnen Präferenzprofil bewähren. Man nehme an, dass es mindestens
drei Güter gibt und betrachte nun folgendes Präferenzprofil:

\begin{center}
\begin{tabular}{ccc}
$J$ & $M-J$ & Rest \\
\cline{1-3}
$z$ & $x$ & $y$ \\
$x$ & $y$ & $z$ \\
$y$ & $z$ & $x$ \\
\end{tabular}

\vspace{0.5cm}
{\small Quelle: \cite[S. 188]{resnik:1987}}
\end{center}

Da $M$ eine "`beinahe entscheidende"' Menge für $x$ über $y$ ist und in diesem
Präferenzprofil für alle Mitglieder von $M$ gilt: $x \succ y$, und alle
Nicht-Mitglieder gilt: $y \succ x$, so muss die Wohlfahrtsfunktion diesem
Präferenzprofil kollektive Präferenzen zuordnen, bei denen $x \succ y$ gilt.
Darüber hinaus muss die Wohlfahrtsfunktion natürlich auch festlegen, welche
Beziehung ($\succ$, $\prec$ oder $\sim$) zwischen $x$ und $z$ zu gelten hat.
Wir betrachten die drei Möglichkeiten im Einzelnen, und zeigen, dass jede davon
zu einem Widerspruch führt. Dabei ist zu beachten, dass wir nicht
ausgeschlossen haben, dass die Menge "`Rest"' leer sein kann. Die folgenden
Argumente funktionieren aber (wovon man sich leicht überzeugen kann) auch
in dem Fall, dass die "`Rest"'-Gruppe leer ist.

\begin{enumerate}
  \item Angenommen nach der Wohlfahrtsfunktion gilt für dieses Präferenzprofil
  $x \succ z$. Dann muss die Wohlfahrtsfunktion nach der Bedingung der
  Unabhängigkeit von dritten Alternativen $x \succ z$ auch für alle
  anderen Präferenzprofile liefern, nach denen $x$ und $z$ für jedes Individuum
  in derselben Weise relativ zueinander geordnet sind wie in dem gegebenen
  Präferenzprofil. Damit liefert die Wohlfahrtsfunktion aber immer $x \succ z$,
  wenn für alle Individuen in $M-J$ gilt $x \succ z$ und für alle Individuen,
  die nicht in $M-J$ enthalten sind $z \succ x$. Damit ist $M-J$ aber "`beinahe
  entscheidende"' Menge für $x$ über $z$. Nach der Konstruktion von $M$ hätte
  $M$ als "`minimale beinahe entscheidende Menge"' (für $x$ über $y$) aber keine
  Teilmenge mehr enthalten dürfen, die noch "`beinahe entscheidende"' Menge
  {\em irgendeines} Paars von Alternativen ist. Also liegt hier ein Widerspruch
  vor, so dass die Möglichkeit, dass die Wohlfahrtfunktion dem oben stehenden
  Präferenzprofil kollektive Präferenzen zuordnet, die $x \succ z$ enthalten,
  ausgeschlossen ist.
  
  \item Angenommen, die Wohlfahrtsfunktion legt für dieses Präferenzprofil $x
  \sim z$ fest. Dann ergibt sich, da bereits $x \succ y$ gilt, dass auch $z
  \succ y$. Da $J$ aber $z$ gegenüber $y$ vorzieht, während alle anderen
  Individuen $y$ gegenüber $z$ vorziehen, wäre nach dem gleichen Argument wie
  im 1.Fall $J$ beinahe entscheidend für die Alternative $z$ über $y$, was
  ebenfalls der Minimalität von $M$ widerspricht. 
  Damit scheidet die zweite Möglichkeit auch aus.

  \item Angenommen, die Wohlfahrtsfunktion liefert $z \succ x$. Dann gilt wegen
  $x \succ y$ und der Transitivität der Präferenzrelation auch $z \succ y$.
  Dann liegt aber wiederum der Fall vor, dass bei dem oben angegebenen
  Präferenzprofil für $J$ gilt: $z \succ y$, aber für alle anderen Individuen:
  $y \succ z$, woraus sich mit Hilfe der Bedingung der Unabhängigkeit von
  dritten Alternativen wiederum ergibt, dass $J$ "`beinahe entscheidend"'
  für $z \succ y$ ist, im Widerspruch zur Minimalität von $M$. Auch diese
  Möglichkeit scheidet aus.
\end{enumerate}

Da alle Möglichkeiten zum Widerspruch führen, kann die Wohlfahrtsfunktion die
individuellen Präferenzen nicht auf kollektive Präferenzen abbilden, sofern die
minimale "`beinahe entscheidende"' Menge $M$ noch mehr als ein Individuum
enthält. 

Das erste Lemma scheint alleine noch nicht viel zu besagen, denn von dem
Individuum, aus dem die Menge $M$ am Ende besteht, ist zunächst nur bewiesen, dass es
lediglich beinahe entscheidend ist, und auch das nur für ein Paar von
Alternativen. Ein zweites Lemma zeigt aber, dass weit mehr dahinter steckt: 

\begin{quote}
{\bf Lemma 2:} {\em Ein Individuum,
das für irgendein Paar von Alternativen beinahe entscheidend ist, ist
entscheidend für jedes Paar von Alternativen.}
\end{quote}

{\em Beweis}: Wir nehmen an, dass das Individuum $J$ beinahe entscheidend für
$x$ über $y$ ist. Es muss nun gezeigt werden, dass es dann auch entscheidend
(und zwar nicht bloß {\em beinahe} entscheidend!) für alle Paare von
Alternativen ist. Dies ist dann bewiesen, wenn wir zwei weitere
Alternativen $a$ und $b$ in die Betrachtung einbeziehen und beweisen können,
dass $J$ in folgenden sieben Fällen entscheidend ist: 1) $x$ über $y$; 
2) $y$ über $x$; 3) $x$ über $a$; 4) $a$ über $x$; 5) $y$ über $a$; 6) $a$ über
$y$; 7) $a$ über $b$. 

Da $a$ und $b$ beliebig wählbar sind, schließt der Beweis automatisch ("`ohne
Beschränkung der Allgemeinheit"') alle weiteren Alternativen mit ein, die es
außer $x,y,a$ und $b$ noch geben könnte. Gibt es außer $x$ und $y$ nur noch eine
oder gar keine weiteren Alternativen, dann fallen nur einige der betrachteten
Fälle weg, und der Beweis gilt trotzdem. Aus Gründen der Konvenienz werden in dem
folgenden Beweis die Fälle in einer anderen Reihenfolge behandelt (vgl.
\cite[S.190/191]{resnik:1987}). Nun zu den Fällen im Einzelnen:

\begin{enumerate}
  \item Fall {\em $x$ über $a$}: Wir betrachten das Präferenzprofil, in dem
  $J$ die Alternativen $x,y$ und $a$ in der Reihenfolge $x \succ y \succ a$
  ordnet, und in denen die anderen Individuen die Alternative $y$ sowohl $x$
  als auch $a$ vorziehen, wobei zwischen $x$ und $a$ jede mögliche Reihenfolge
  zugelassen sei. 
  
  Da $J$ nach Voraussetzung beinahe entscheidend für $x$ über $y$ ist, muss die
  Wohlfahrtsfunktion bei einem solchen Profil $x \succ y$ liefern. Da aber
  ebenfalls für alle Individuen $y \succ a$ gilt, muss auf Grund der Bedingung
  der Pareto-Effizienz auch die Wohlfahrtsfunktion $y \succ a$ für ein
  derartiges Präferenzprofil liefern. Da aber schon $x \succ y$ gilt, liefert
  die Sozialwahlfunktion aufgrund der Transitivität von Präferenzen auch $x
  \succ a$. Auf Grund der Bedingung der Unabhängigkeit von dritten Alternativen
  gilt aber, dass die Wohlfahrtsfunktion $x \succ a$ für alle Präferenzprofile liefern
  muss, in denen $x$ und $a$ in derselben Weise relativ zueinander geordnet
  sind, wie in dem betrachteten Beispiel. In dem Beispiel hat $J$ aber $x$ vor
  $a$ eingeordnet, während bei allen anderen Individuen die Ordnung beliebig
  war. Das bedeutet aber, dass die Wohlfahrtsfunktion $x \succ a$ liefert,
  sobald $J$ die Ordnung $x \succ a$ festlegt. Damit ist $J$ entscheidend
  (nicht bloß nahezu entscheidend!) für $x$ über $a$.
  
  \item Fall {\em $a$ über $y$}: Wir betrachten das Präferenzprofil, in dem
  für $J$ die Präferenz $a \succ x \succ y$ gilt, und in dem für alle anderen
  Individuen $a \succ x$ und $y \succ x$ gilt, d.h. in dem $a$ und $y$ der
  Alternative $x$ vorgezogen werden, während die Reihenfolge zwischen $a$ und
  $y$ nicht festgelegt sein soll. 
  
  Weil $J$ beinahe entscheidend für $x$ über $y$ ist gilt, dass die
  Wohlfahrtsfunktion bei den angenommenen Präferenzen $x \succ y$ liefert.
  Aufgrund der Einstimmigkeit (Pareto-Effizienz) muss die Wohlfahrtsfunktion
  aber auch $x \succ a$ festlegen. Aufgrund der Unabhängigkeit von dritten
  Alternativen gilt das letztere wann immer $J$ die Präferenz $a \succ y$
  enthält. Damit ist $J$ aber
  entscheidend für $a$ über $y$.
  
  \item Fall {\em $y$ über $a$}: Betrachtet sei folgendes Präferenzprofil: Für
  $J$ gilt $y \succ x \succ a$; für alle anderen gilt $a,y \succ x$. 
  
  Gemäß der Bedingung der Pareto-Effizienz liefert die Wohlfahrtsfunktion für
  dieses Profil $y \succ x$. Da $J$ entscheidend ist für $x$ über $a$, liefert
  sie auch $x \succ a$ und, wegen der Transitivität der Präferenzrelation
  schließlich auch $y \succ a$. 
 
  Wiederum muss, wenn die Wohlfahrtsfunktion $y \succ a$ für ein Profil liefert,
  in dem $J$ die Alternative $y$ vor $a$ stellt, während die Ordnung von $y$ und $a$
  für die anderen Individuen nicht festgelegt ist, auf Grund der
  Bedingung der Unabhängigkeit von dritten Alternativen die
  Wohlfahrtsfunktion $y \succ a$ bei allen Profilen liefern, die $y$ und $a$ in
  derselben Weise ordnen, d.h. bei allen Profilen, in denen für $J$ gilt: $y
  \succ a$. Damit ist $J$ aber entscheidend für $y$ über $a$.

  \item Fall {\em $a$ über $x$}: Man betrachte zunächst das Profil, in dem für
  $J$ gilt: $a \succ y \succ x$, während $y \succ x,a$ für die anderen
  Individuen gilt. 
  
  Wir wissen bereits, dass $J$ entscheidend für $a \succ y$ ist. Aufgrund der
  Bedingung der Pareto-Effizienz liefert die Wohlfahrtsfunktion aber auch $y
  \succ x$. Analog zu den vorhergehenden Fällen können wir daraus mit Hilfe
  der Bedingung der Unabhängigkeit von dritten Alternativen ableitent,
  dass $J$ entscheidend für $a$ über $x$ ist.
  
  \item Fall {\em $x$ über $y$}: Wir betrachten das Profil, in dem für $J$ gilt:
  $x \succ a \succ y$. Wir wissen bereits, dass $J$ entscheidend für $x \succ a$
  und ebenso für $a \succ y$ ist. Also muss die Wohlfahrtsfunktion für dieses
  Profil $x \succ y$ liefern. Analog zu den vorhergehenden Fällen lässt sich
  dann mit Hilfe der Bedingung der Unabhängigkeit von dritten Alternativen
  schließen, dass $J$ entscheidend für $x \succ y$ ist.

  \item Fall {\em $y$ über $x$}: Wie im vorhergehenden Fall, nur dass
  diesmal $x$ und $y$ vertauscht sind.
  
  \item Fall {\em $a$ über $b$}: Wir betrachten ein Profil, in dem für $J$ gilt
  $a \succ x \succ b$. Analog zu dem vorhergehenden Fall, können wir dann
  zeigen, dass $J$ entscheidend für $a$ über $b$ ist.

\end{enumerate}

In jedem der Fälle ist $J$ also "`entscheidend"', womit das zweite Lemma
bewiesen ist. Aus dem ersten und dem zweiten Lemma ergibt sich zusammengenommen
der Satz von Arrow, der damit ebenfalls bewiesen ist.

\subsubsection{Resumé}

Nachdem der "`Satz von Arrow"' mathematisch bewiesen ist, stellt sich nun
erst die eigentliche Frage, wie er inhaltlich beurteilt werden muss. Der Satz
von Arrow scheint zu zeigen, dass es nicht möglich ist, aus individuellen Präferenzen
kollektive Entscheidungen abzuleiten, die gleichermaßen effizient, vernünftig und
(hinsichtlich der Berücksichtigung der unterschiedlichen individuellen
Präferenzen) gerecht sind. Aber wie weit reicht diese Erkenntnis? Dass es bei der
kollektiven Entscheidungsfindung Zielkonflikte zwischen Gerechtigkeitsansprüchen
und Effizienzforderungen (hier repräsentiert durch die Einstimmigkeitsbedingung)
geben kann, wissen wir schon aus der politischen Lebenserfahrung. Dass sie -- wie
der Satz von Arrow nahelegt -- unvermeidlich sind, ist eine wichtige Einsicht.
Dennoch stellt sich die Frage wie relevant derartige logische Beweisführungen in
der Praxis sein können. Immerhin mag in der politischen Praxis die Vereinbarung
von Gerechtigkeits- und Effizienzansprüchen noch an vielen weiteren Hindernissen
scheitern als bloß dem im Satz von Arrow erfassten logischen Abbildungsproblem.
Und die Zielsetzung, Gerechtigkeits- und Effizienzansprüche {\em möglichst
weitgehend} miteinander zu vereinbaren, wird durch den Satz von Arrow keineswegs
sinnlos.

% Wie hilfreich und relevant der Satz von Arrow und die zahlreichen sich an ihn
% anschließenden Untersuchungen dabei sind, kann durchaus unterschiedlich
% bewertet werden. Donald Green und Ian Shapiro kommen in ihrer ebenso
% umfassenden wie kritischen Studie zu dem Ergebnis, dass die von Arrow
% inspirierte wissenschaftliche Literatur kaum relevante Ergebnisse zu Tage
% gefördert hat \cite{green-shapiro:1994}. Dem widerspricht Müller
% \cite[]{mueller:2003}, unter anderem mit dem Hinweis, dass die Erkenntnisse des
% entsprechenden Forschungszweiges zumindest für das Design und Verständnis von
% Wahl- und Abstimmungsmechanismen eine gewisse Relevanz haben müssten. Denn wenn
% wir ein bestimmtes Wahl- oder Abstimmungsverfahren vereinbaren, sollten wir
% dessen logische Eigenschaften besser gut verstehen.
% 
% Wir können diese Debatte im Rahmen dieser Vorlesung leider nicht mehr
% ausführlich behandeln. Soviel sollte jedoch deutlich sein: Mit dem Beweis
% bestimmter logischer oder mathematischer Theoreme allein ist es noch nicht
% getan. Danach fängt erst die -- in einer gewissen Hinsicht viel kompliziertere
% -- Interpretationsarbeit an.

