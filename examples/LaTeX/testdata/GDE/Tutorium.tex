\newpage
\subsection{Übungen 1 (15. April)}

\begin{itemize}
 \item Eine Ölfirma erwägt an einer bestimmten Stelle in der Nordsee nach Öl zu
 bohren. Es ist nicht absolut sicher, ob sich an dem entsprechenden Ort
 tatsächlich Öl befindet. Um dies mit Sicherheit festzustellen, kann die Firma
 eine Expertise durchführen lassen. Der Bau einer Bohrinsel
 kostet € 1.000.000. Liefert die Bohrinsel tatsächlich Öl, so gewinnt die
 Ölfirma abzüglich der Betriebskosten € 10.000.000. Die Durchführung einer
 Expertise kostet € 250.000. {\em Zeichne den Entscheidungsbaum und die
 Entscheidungstabelle} Diskussion: Sollte die Firma in jedem Fall eine Expertise
 durchführen?

 \item a) Wenn man mit zwei Würfeln würfelt, 
 ist die Wahrscheinlichkeit eine 6 zu würfeln dann genauso groß wie die 
 Wahrscheinlichkeit eine 12 zu würfeln? (Wie groß ist die Wahrscheinlichkeit mit zwei
 Würfeln eine 6 zu würfeln? Wie groß ist die Wahrscheinlichkeit mit zwei
 Würfeln eine 12 zu würfeln?)

\item a) Wie groß ist die Wahrscheinlichkeit
mit zwei Würfeln ein Pasch zu würfeln? b) Wie groß ist die Wahrscheinlichkeit,
dass ein Pasch gewürfelt worden ist, wenn die Summe der Augen nicht größer als
3 ist?

\item Ein Gremium von 5 Personen wird zufällig aus einer Gruppe von 5 Männern
und 10 Frauen besetzt. Wie groß ist die Wahrscheinlichkeit, dass das Gremium aus 2
Männern und 3 Frauen besteht? Wie groß ist die Wahrscheinlichkeit, dass das
Gremium nur aus Frauen besteht?

\item Was ist der Unterschied zwischen "`endlich"', "`abzählbar unendlich"' und
"`überabzählbar unendlich"'?

\end{itemize}


\subsection{Übungen 2 (22. April)}

\begin{itemize}

\item Zur Diskussion: Eine Schönheitschirugin steht vor der Entscheidung, ob sie
einem Patienten ein Gesichtslifting empfehlen soll oder nicht? Wird sie bei
ihrer Entscheidung im Hinblick auf die Risiken einer Operation die Maximin-Regel
anwenden? (Wie wird sie sich entscheiden, wenn a) ärztliche Kunstfehler in der
Regel mit der Zahlung von Schmerzensgeld bestraft werden oder b) mit dem
Verlust der Zulassung?)

\item Konstruiere die "`Minimax-Bedauern"'-Tabellen und finde die Handlungen,
bei denen das "`Bedauern"' minimiert wird:

\begin{center}
\begin{tabular}{c|c|c|c|cc|c|c|c|}
\multicolumn{1}{c}{} & \multicolumn{3}{c}{Tabelle 1:} &
\multicolumn{2}{c}{} & \multicolumn{3}{c}{Tabelle 2:}
\\
\cline{2-4} \cline{7-9}
$A_1$ &  7 &  0 &  4  & & $A_1$ &  5 & 20 &  6  \\ 
\cline{2-4} \cline{7-9} 
$A_2$ &  5 & 21 & 11  & & $A_2$ & -3 &  8 & 10  \\ 
\cline{2-4} \cline{7-9}
$A_3$ & 10 & -5 & -1  & & $A_3$ &  4 &  5 &  9  \\ 
\cline{2-4} \cline{7-9}
\end{tabular}

{\tiny Quelle: Michael D. Resnik: Choices. An Introduction to Decision Theory,
Minnesota 2000, S. 32.}
\end{center} 

\item Zeige: Die "`Minimax-Bedauern"'-Regel schließt nicht zwangläufig alle
dominierten Handlungen aus. (Eine triviale Konsequenz aus der Lösung zu Aufgabe
4! Warum?)

\item Zeige: Wenn man eine Entscheidungstabelle positiv linear in eine andere
überführt, dann ist auch die zugehörige Bedauernstabelle eine positiv linear
transformierte (genaugenommen sogar ein positives Vielfaches, warum?) der
ursprünglichen Bedauernstabelle. Was müsste man von der Minimax-Bedauernsregel
halten, wenn das nicht der Fall wäre?

\item Zeige: Positiv lineare Transformationen sind transitiv, d.h. wenn die
Skala u' durch positiv lineare Transformation aus der Skala u hervorgeht und
Skala u' durch eine (nicht notwendigerweise dieselbe) positiv lineare
Transformation in u'' überführt werden kann, dann kann gibt es auch eine
positiv lineare Transformation, die u unmittelbar in u'' überführt. Warum ist
diese Eigenschaft wichtig?

\item Zeige: Jede Entscheidungstabelle kann durch eine positiv lineare
Transformation in eine Entscheidungstabelle überführt werden, deren maximaler
Eintrag 1 und deren minimaler Eintrag 0 ist.

\end{itemize}
