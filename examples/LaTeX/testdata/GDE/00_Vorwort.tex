\chapter*{Vorwort}

Dieses Skript gehört zur Vorlesung "`Grundlagen des Entscheidens I"', die ich im
Sommersemester 2008 in Bayreuth gehalten habe. Inhaltlich habe ich mich dabei
weitgehend an die bewährte Einführung von {\em Micheal D. Resnik: Choices. An
Introduction to Decision Theory, University of Minnesota Press, 5th ed. 2000
\cite{resnik:1987}} gehalten, die eine ansprechende Stoffauswahl mit nicht
übermäßig schwierigen mathematischen Beweisen verbindet. An vielen Stellen bin
ich aber auch von Resnik abgewichen. So habe ich besonders für die
Wahrscheinlichkeitsrechnung außer gängigen mathematischen Lehrbüchern vor allem
die sehr gelungene Darstellung von Donald Gillies \cite{gillies:2000}
herangezogen. Auch die Darstellung der Spieltheorie stützt sich überweigend auf
andere Quellen. Da ich die Vorlesung zum erstenmal gehalten habe, enthält das
Skript zweifellos noch zahlreiche Flüchtigkeits- und Tippfehler, die ich bei
Gelegenheit noch zu korriegieren hoffe. (Wer Lust hat ein wenig Korrektur zu
lesen, oder wer Fehler, besonders inhaltlicher Art(!) im Skript entdeckt, teile
es mir bitte mit: eckhart\_arnold@hotmail.com) Auch bleibt es nicht aus, dass
ich im Nachhinein viele Dinge anders machen würde. Im einzelnen sehe ich folgende Punkte, an denen sich eine Überarbeitung der Vorlesung bzw. des Konzepts der Vorlesung lohnen würde:

\begin{itemize}
  \item Über der Darstellung des dogmatischen Lehrstoffes ist leider die Kritik
  und die Erörterung von (besseren) Alternativen zu kurz gekommen. Besonders in
  den letzten Abschnitten der Vorlesung, also der Spieltheorie und der
  Sozialwahltheorie, wäre es wichtig noch ausführlicher zu erörtern, warum die
  entsprechenden Ansätze nur eine äußerst begrenzte Sichtweise auf menschliches
  Handeln (Spieltheorie) bzw. politische Ordnung und politische
  Entscheidungsfindung (Sozialwahltheorie) ermöglichen. Hinsichtlich der
  Entscheidungs- und Spieltheorie wäre es sicherlich empfehlenswert auch
  Ansätze aus der Psychologie und der experimentellen Spieltheorie zum
  Verständnis menschlichen Handelns und Entscheidens stärker einzubeziehen. 
  Bei der Sozialwahltheorie, die in dieser
  Vorlesung allerdings nur sehr kurz angerissen wird, würde es lohnend sein, 
  auch alternative Ansätze der Demoktratietheorie anzusprechen, 
  um zu vermeiden, dass ein falsches Bild vom Gegenstandsbereich dieser
  Theorien entsteht. Unweigerlich formen nämlich die Theorien, mit denen wir
  uns beschäftigen, das Gesamtbild des Gegenstandes, auf den sie sich beziehen.
  Ich könnte es mir leicht machen, und die Stoffauswahl durch den Gesichtspunkt
  thematischer Beschränkung auf die formale Entscheidungstheorie verteidigen.
  Aber dagegen rebelliert mein intellektuelles Gewissen. Denn
  wenn die entsprechenden Theorien nur Teilaspekte des Gegenstandes
  abdecken können, dann entsteht beinahe unvermeidlich ein verzerrtes 
  Gesamtbild. Im Extremfall wäre man klüger geblieben, 
  hätte man sich
  gar nicht mit der wissenschaftlichen Theorie abgegeben, sondern sich bloß auf
  den eigenen gesunden Menschenverstand bei der Beurteilung der Sache verlassen. 
  Gerade der Philosophie, die doch immer die übergreifenden Zusammenhänge im
  Auge behalten sollte, steht es nicht an, sich mit thematischer
  Selbstbeschränkung herauszureden.

  \item Was nun die Auswahl der Themen angeht, so scheint mir, dass vor allem
  die Aufnahme der an sich sehr interessanten philosophischen
  Wahrscheinlichkeitstheorien (v. Mises und Ramsey-De Finetti, siehe Kapitel
  \ref{philosophischeWahrscheinlichkeitstheorien}) zu überdenken ist. Nicht so
  sehr wegen der mathematischen und gedanklichen Anspruchshöhe als deshalb,
  weil der Stoff einerseits zwar wohl zum geistigen Hintergrund der
  Entscheidungstheorie gehört aber für die folgenden Themen nicht unbedingt
  vorausgesetzt werden muss und zudem eine eigene, ausführlichere Behandlung
  verdienen würde. 
  
  Ebenfalls zu überdenken scheint mir in diesem Fall die Aufnahme der
  Neumann-Morgensternschen Nutzentheorie (Kapitel \ref{NeumannMorgenstern}).
  Meine Motivation dafür sie aufzunehmen bestand darin, dass sie auch in den
  Lehrbüchern etwa zur Spieltheorie \cite{myerson:1991} auftritt, wobei die
  Motivation zu der doch seltsamen Konstruktion der Lotterien oft etwas im
  Dunkel bleibt. Mir scheint, dass die Neumann-Morgensternsche Nutzentheorie im
  wesentlichen auf einer Illusion beruht, der Illusion nämlich man würde 
  kardinale Nutzenwerte eines Tages so präzise messen können wie die
  Temperatur. Diesen Vergleich zur Physik führen Neumann und
  Morgenstern selbst an, wie irreführende Vergleiche mit der Physik ja immer zu
  den Requisiten mathematikbegeisterter Sozialwissenschaftler gehören. Aber
  nach 60 Jahren -- das Buch von Neumann und Morgenstern erschien 1947 -- sind
  wir von einer präzisen Messung von kardinalen Nutzenwerten immer noch genauso
  weit entfernt wie damals. Wozu soll die gewaltsame mathematische Konstruktion
  kardinaler Nutzenwerte gut sein, wenn man sie doch nicht präziser messen kann als 
  durch die Frage "`Wieviel Geld gibst Du mir dafür?"' Mag sein, dass die
  Neumann-Morgensternsche Nutzentheorie zu den unveräußerlichen Grundlagen der
  Spieltheorie und der Volkswirtschaftslehre gehört. Für sich betrachtet wirkt
  sie eher wie eine müßige mathematische Spielerei.
  
  \item Es hat sich gezeigt, dass besonders die mathematischen Beweise viele
  Leute vor schwer überwindliche Hindernisse stellen. Die didaktisch
  wohlverständliche Aufbereitung mathematischer Beweise stellt dabei eine nicht
  zu unterschätzende Herausforderung dar, die viel Zeit und Mühe erfordert.
  Resnik hat sich in seinem Lehrbuch dankenswerter Weise möglichst einfacher
  Beweisführungen bedient. Ich habe soweit als möglich versucht, die
  Beweisführungen nochmals einfacher und verständlicher darzustellen, aber ich
  möchte nicht behaupten, dass in dieser Hinsicht nicht noch ein Übriges getan
  werden könnte.
  
  Auch wenn es billig klingt, so kann ich in diesem Punkt doch den
  Mathematikunterricht in der Schule nicht ganz von Tadel freihalten, weil man
  dort zwar tüchtig rechnen lernt aber keine richtige Mathematik, d.h. keine
  Beweisführungen.
  
  \item In diesem Zusammenhang ist einzuräumen, dass die Nomenklatur in meinem
  Skript zum Teil uneinheitlich und manchmal ungünstig gewählt ist, besonders
  bei der Wahrscheinlichkeitsrechnung. In der Fachliteratur gibt es
  unterschiedliche Arten die Wahrscheinlichkeitstheorie darzustellen. In
  mathematischen Lehrbüchern ist die Mengenschreibweise üblich, d.h. man
  bezieht die Wahrscheinlichkeiten auf Ereignismengen. In der philosophisch
  orientierten Literatur greift man lieber auf eine aussagenlogische
  Schreibweise zurück. Meist habe ich das letztere gewählt. Für eine zukünftige
  Überarbeitung wäre aber eine einheitliche Schreibweise und dann
  höchstwahrscheinlich die Mengenschreibweise wünschenswert. (In diesem
  Zusammenhang scheint mir, dass die logische "`und"'-Verknüpfung bzw. die
  Schnittmenge mit einigem Gewinn für die Lesbarkeit durch das
  Zeichen "`\&"' statt durch das Zeichen "`$\wedge$"' dargestellt werden kann.)
  Vielleicht wäre darüber hinaus ein kurzer Anhang zur formalen
  Logikschreibweise, deren Kenntnis hier vorausgesetzt wird, empfehlenswert.
\end{itemize}

\begin{flushright}
Eckhart Arnold, Bayreuth, den 25. Juli 2008
\end{flushright}

