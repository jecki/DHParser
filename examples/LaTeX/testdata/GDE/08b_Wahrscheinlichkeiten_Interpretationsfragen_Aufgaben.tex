\subsection{Aufgaben}

\begin{enumerate}
  \item Das "`Gesetz der großen Zahlen"' besagt, dass in allen Zufallsfolgen
  der Häufigkeitsgrenzwert eines Merkmals $A$ mit einer Wahrscheinlichkeit von
  1 gleich dem Wahrscheinlichkeitswert $r$ von $A$ ist. Um einzusehen, dass dies
  nicht ein- und dasselbe ist, wie zu sagen, der Häufigkeitsgrenzwert beträgt
  $r$, muss man sich klar machen, dass eine Wahrscheinlichkeit von 1 noch nicht
  bedeutet, dass irgendein Ereignis mit Sicherheit eintritt. (Zur Erinnerung:
  Die Kolmogorowschen Axiome fordern lediglich, dass ein sicheres Ereignis die
  Wahrscheinlichkeit 1 hat, aber nicht umgekehrt.) Finden Sie Beispiele für:
  \begin{enumerate}
    \item Eine Ereignis, dessen Wahrscheinlichkeit 0 ist, das aber trotzdem
    möglich ist.
    \item Eine Ereignisfolge, innerhalb derer ein Merkmal unendlich oft
    auftritt, aber trotzdem die Wahrscheinlichkeit 0 hat.
  \end{enumerate}
  (Übrigens, die Lösung zu dieser Aufgabe ist bereits an anderer Stelle in
  diesem Skript versteckt. Aber Nachdenken lohnt mehr als suchen\ldots)
  
  \item Das dritte kolmogorowsche Axiom besagt, dass für Ereignisse, die sich
  ausschließen gilt: \[P(p \vee q) = P(p) + P(q)\]
  Zeige, dass das dritte Axiom {\em äquivalent} ist zu dem Axiom 3*: 
  Seien $q_1,\ldots q_n$ Ereignisse, die sich paarweise 
  ausschließen (Exklusivität), von denen aber eins eintreten muss
  (Vollständigkeit), dann gilt:
  \[ P(q_1) + \ldots + P(q_n) = 1 \]
  
  \item Zeige, dass man durch aufsummieren der Gleichungen:
  \[q_iG_i = q_i(q_iS_i + \ldots + q_nS_n) - q_iS_i 
    \qquad \mbox{mit} \qquad 1 \leq i \leq n \] 
  über den index $i$ das Ergebnis: \[ q_1G_1 + q_2G_2 + \ldots + q_nG_n = 0 \]
  erhält, sofern $\sum_{i=1}^n q_i = 1$
  
  \item Zeige durch Ausrechnen und unter Verwendung von $a=b\cdot c, 0
  \leq a,b,c \leq 1$, dass in den folgenden drei Gleichungen sowohl $\alpha$ als
  auch $\beta$ und $\gamma$ Null sind werden. 
    \[ \alpha = bc(a-1) + (1-b)ca + (1-c)a \]
    \[ \beta  = bc(b-1) + (1-b)cb  \]
    \[ \gamma = bc(c-1) + (1-b)c(c-1) + (1-c)c \]
  
  \item Wenn ein Wettender über eine Informationen $I$ verfügt, die für die
  Ereignisse, auf die gewettet werden kann, relevant ist, dann muss er seine
  Wahrscheinlichkeiten entsprechend $P_{neu}(a) = P_{alt}(a|I)$ anpassen.
  Zeige: Wenn der Wettende für irgendeine Aussage $a$ die Wahrscheinlichkeit 
  $P_{neu}(a) \neq P_{alt}(a|I)$ wählt, dann ist es für einen geschickten
  Buchmacher möglich eine "`todsichere Wette"' abzuschließen.
  {\em Hinweis}: Der Buchmacher muss dazu sowohl auf $a$ als auch auf $I$ eine
  Wette abschließen und die Wettbeträge entsprechend aufeinander abstimmen.
  Dabei weiß er, ob $P_{neu}(a) < P_{alt}(a|I)$ oder $P_{neu}(a) >
  P_{alt}(a|I)$. 
  %Zur Vereinfachung kann man sich die Situation am sog.
  %"`Ziegenproblem"' (siehe Wikipedia) klar machen.
\end{enumerate}
