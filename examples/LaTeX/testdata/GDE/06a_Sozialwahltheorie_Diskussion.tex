\section{Zur Diskussion der Sozialwahltheorie}
\label{SozialwahltheorieDiskussion}

Nachdem im letzten Kapitel der mathematische Beweis des Satzes von Arrow
ausführlich besprochen wurde, soll nun die Frage erörtert werden, was das Theorem
von Arrow, das "`Paradox des Liberalismus"' und verwandte mathematische Sätze
inhaltlich aussagen. Solche Benennungen wie "`Paradox des Liberalismus"'
suggerieren ja bereits, dass sie bestimmte Schlussfolgerungen über die Natur
politischer Entscheidungsprozesse implizieren. Wie verhält es sich damit?

Da es sich bei der Sozialwahltheorie zunächst einmal um eine abstrakte
mathematische Theorie handelt steht der Anwendungsbereich nicht von vorn herein
genau fest (etwa so wie ja auch die Differentialrechnung in der Physik genauso
wie in der Volkswirtschaftslehre ihre Anwendung findet). Man kann sie auf die
Entscheidungsprozesse in der großen Politik und die Demokratie im Ganzen
beziehen, aber ebenso könnte man sie auch auf alle möglichen kollektiven
Entscheidungsprozesse im kleinen Rahmen bei Unternehmen, Vereinen etc. beziehen.
Wollte man die Frage streng systematisch angehen, so müsste man zunächst
untersuchen 1) auf welche Arten kollektiver Entscheidungsprozesse sich die
Theorie überhaupt anwenden lässt, 2) welche Aspekte dieser Entscheidungsprozesse
sie erfasst und -- mindestens ebenso wichtig! -- 3) welche Aspekte sie nicht
erfasst, 4) zu welchen Befunden sie bezüglich der von ihr erfassten Aspekte
gelangt und 5) ob diese Befunde richtig und stimmig sind.

Im Rahmen dieser Vorlesung würde es allerdings zu weit führen, alle diese Aspekte
erschöpfend zu behandeln, zumal wir mit dem Condorcet-Paradox, dem sogennanten
"`Paradox des Liberalismus"' und dem Satz von Arrow nur einen sehr kleinen
Ausschnitt aus der Sozialwahltheorie kennen gelernt haben. Wir werden uns auf die
Erörterung der Frage beschränken, inwieweit der Satz von Arrow Grenzen
demokratischer Wahl- und Entscheidungsprozesse aufzeigt, und welche Auswirkungen
er auf unser Demokratieverständnis hat bzw. haben sollte.

Die These, dass der Satz von Arrow bedeutsame Konsequenzen für unser
Demokratieverständnis hat, ist recht häufig vertreten worden, u.a. von
Nida-Rümelin, dessen Standpunkt wir als erstes behandeln werden. Sehr viel
gründlicher wurde eine ähnliche These von dem Politikwissenschaftler William
Riker und seiner Schule wissenschafttlich ausgebaut \cite[]{riker:1982}. Für
Riker zeigt der Satz von Arrow, dass demokratische Entscheidungsprozesse
\marginline{Rikers These der Fragilität der Demokratie}
grundsätzlich fragil und nur sehr begrenzt dazu in der Lage sind, den "`Willen"'
eines Kollektivs (etwa des Staatsvolks) zum Ausdruck zu bringen. Er zieht daraus
tendenziell libertäre Konsequenzen, d.h. angesichts des fragilen Charakters
demokratischer Entscheidungsprozesse sollten von vornherein möglichst wenig
Gegenstände überhaupt zur Disposition kollektiver Entscheidungen gestellt werden.
Weiterhin sei der Sinn demokratischer Wahlen nicht in erster Linie darin zu
sehen, die Politik im Sinne der mehrheitlich vom Volk gewählten Richtung
festzulegen, sondern lediglich darin, dass sie -- neben Gewaltenteilung,
Verfassungsgerichtsbarkeit etc. -- ein weiteres Mittel der Machtkontrolle sind,
indem sie es ermöglichen, einer Regierung die Macht durch Abwahl wieder zu
entziehen. Diese Sichtweise ist sehr gründlich von Gerry Mackie kritisiert
worden, der den theoretischen Befund Rikers für äußerst schwach begründet und
dessen empirische Belege sämtlich für verkehrt hält. Wir werden in diesem Kapitel
einige der wichtigsten Punkte aus dieser (recht komplexen) Diskussion
herausgreifen und erörtern.

\subsection{Der Satz von Arrow als Widerlegung der "`identären"' Demokratie}

Nach Nida-Rümelins Ansicht sind der Satz von Arrow und verwandte Ergebnisse der
Sozialwahltheorie für "`die Entwicklung eines angemessenen
Demokratieverständnisses -- ex negativo -- bedeutsam"', indem sie "`den Bereich
zulässiger Demokratiekonzeptionen"' durch apriorische Argumente, die "`die
logische Konsistenz von Normen- und Regelsystemen"' betreffen, einschränken. Man
kann ihre Ergebnisse als Argumente gegen die "`Identitätstheorie"' der Demokratie
auffassen. Unter der "`Identitätstheorie der Demokratie"' versteht Nida-Rümelin
"`die Vorstellung, Demokratie verlange die Konstituierung eines kollektiven
Akteurs, dessen Entscheidungen als Aggregation der individuellen Bürgerinteressen
verstanden werden können"' \cite[S. 185]{nida-ruemelin:1991}. Er glaubt, dass der
Satz von Arrow vor dem Hintergrund dieses Demokratieverständnisses "`eine
ernsthafte Herausforderung für die Demokratietheorie"' \cite[S.
186]{nida-ruemelin:1991} darstellt, zeigt er doch seiner Ansicht nach, dass
"`wesentliche Elemente unserer vortheoretischen Demokratievorstellung nicht
tragfähig sind"' \cite[S. 187]{nida-ruemelin:1991}. Als Alternative zu dieser
vermeintlich defizitären "`vortheoretischen Demokratievorstellung"' empfiehlt
sich für Nida-Rümelin eine Demokratievorstellung, die sich "`auf strukturelle,
auf einem praktischen Konsens über sekundäre Regeln beruhende Normen"' \cite[S.
186]{nida-ruemelin:1991} stützt. Den Begriff der sekundären Regeln übernimmt
Nida-Rümelin dabei von dem Rechtsphilosophen H.L.A. Hart, der damit diejenigen
(institutionellen) Regeln bezeichnet, nach denen wir in der Gesellschaft regelen
festlegen, also z.B. die Geschäftsordnung des Parlaments, die regelt auf welchem
Weg Gesetze erlassen werden, im Gegensatz zu den "`primären Normen", also etwa
Gesetzen, die regeln, welches Verhalten verboten oder erlaubt ist.

Um Nida-Rümelins Deutung zu untersuchen, ist Folgendes zu untersuchen:

\begin{enumerate}
  \item Inwiefern betrifft sein Begriff der "`Identitätstheorie der
  Demokratie"' einschlägige Demokratiekonzeptionen, insbesondere: Inwieweit
  gibt er das vortheoretische Demokratieverständnis richtig wieder?
  \item Greift seine auf den Satz von Arrow gestützte Kritik an der
  "`Identitätstheorie der Demokratie"', d.h. leidet diese Demokratiekonzeption
  tatsächlich an einem Mangel an logischer Konsistenz, den der Satz von Arrow
  nachweist?
  \item Kann die von Nida-Rümelin skizzierte Alternative das Problem lösen?
\end{enumerate}

\subsubsection{Die "`Identitätstheorie der Demokratie"'} 
    
Es ist immer ein wenig
schwierig einzuschätzen, worin das vortheoretische Verständnis von etwas, also
z.B. das vortheoretische Verständnis von Demokratie besteht. Nach einem sehr
naiven Verständnis, das unmittelbar an die Wortbedeutung
anknüpft,\footnote{Govanni Sartori nennt dieses sehr naive Verständnis
von Demokratie deshalb auch "`Etymologische Demokratie"' \cite[S.
29ff.]{sartori:1987}.} ist Demokratie schlicht die Herrschaft des Volkes, wobei
mehr oder weniger offen bleibt, wie diese Herrschaft des Volkes auszusehen hat.
Es ist naheliegend, aber keineswegs selbstverständlich, anzunehmen, dass die
"`Herrschaft des Volkes"' durch irgendeine Form von Mehrheitsentscheid
ausgedrückt wird. Nimmt man das aber an, so könnten der Satz von Arrow und
verwandte Theoreme möglicherweise Grenzen der "`Identitätstheorie der
Demokratie"' aufzeigen, sofern die durch den Satz von Arrow gezogenen Grenzen
für die Aggregation individueller zu kollektiven Präferenzen sich als einschneidend
genug erweist, um eine durch Mehrheitsentscheid zum Ausdruck gebrachte 
"`Herrschaft des Volkes"' sinnlos werden zu lassen. Ob das der Fall ist, wird
im Laufe des Kapitels noch zu erörtern sein. Dass es, wenn es der Fall ist,
unabhängig vom Satz von Arrow auch noch andere und möglicherweise sehr viel
wichtigere Gründe gibt, diese sehr naive Vorstellung von Demokratie abzulehnen
\cite[S. 29ff.]{sartori:1987}, wird von Nida-Rümelin dabei zugestanden
\cite[S. 185]{nida-ruemelin:1991} und braucht hier nicht thematisiert zu werden.

Fraglich ist allerdings, ob eine "`Identitätstheorie der Demokratie"' nicht
auch anders verstanden werden kann. Nida-Rümelin zufolge "`bildet die
Vorstellung einer Zusammenfassung individueller Interessen zu einem
Gemeininteresse qua Abstimmungsverfahren den Kern der durch die französische
Revolution geprägten Demokratiekonzeption"' \cite[S. 191]{nida-ruemelin:1991}. 
Richtig ist sicherlich, dass die durch die französische Revolution geprägte
Demokratietheorie das Element der Volkssouveränität vergleichsweise stärker
gegenüber anderen Elementen betont wie etwa dem der Machtkontrolle als etwa die
angelsächsische Tradition. Zugleich beruht diese sich sehr stark auf
Jean-Jacques Rousseau als ihren Vordenker stützende Demokratiekonzeption auf
einem Verständnis von Volkssouveränität, dem gerade nicht die "`Zusammenfassung
individueller Interessen zu einem Gemeininteresse"' zu Grunde liegt. Rousseau
\marginline{Rousseaus Demokratietheorie}
unterschied sehr genau zwischen der "`volunté de tous"', dem Willen aller, der
in etwa der aus den individuellen Präferenzen aggregierten kollektiven
Präferenzrelation im theoretischen Rahmen der Sozialwahltheorie entsprechen
würde, und der "`volonté générale"', dem allgemeinen Willen, der das Gemeinwohl
repräsentiert, und der bei Rousseau gerade nicht durch Aggregation von
Einzelinteressen ("`volonté particulière"') entsteht, sondern so etwas wie das
bessere Gewissen und den höheren Willen der Bürger verkörpert, soweit sie
sich dem Gemeinwohl verpflichtet fühlen. Gegen Rousseaus Demokratietheorie gibt
es viele Einwände \cite[S. 103ff.]{schmidt:2000} -- unter anderem wird ihr
vorgeworfen, dass sie kollektivistisch sei -- aber durch Argumente die sich auf
den Satz von Arrow und verwandte Befunde der Sozialwahltheorie stützen könnten,
ist die Rousseausche Variante einer Identitätstheorie -- ebenso wie die
meisten anderen kollektivistischen Gesellschaftstheorien -- von vornherein nicht
angreifbar. Sofern die auf den Satz von Arrow gestützte Kritik an
demokratischen Abstimmungsverfahren überhaupt Stich hält, wäre -- stark
vereinfacht gesprochen -- die angelsächsische Tradition der Demokratietheorie
also stärker davon betroffen als die französische.

\subsubsection{Die Frage der Durchschlagskraft der auf den Satz von Arrow
gestützten Kritik an der Identitätstheorie} Wenn wir uns aber schon einmal auf
eine solche Identitätstheorie der Demokratie verständigen, bei der die Identität
von Herrschern und Beherrschten durch die "`Zusammenfassung individueller
Interessen zu einem Gemeininteresse qua Abstimmungsverfahren"' \cite[S.
191]{nida-ruemelin:1991} zustande kommt, dann ist die Frage zu untersuchen, ob
der Satz von Arrow tatsächlich die Unmöglichkeit einer derartigen
Identitätstheorie erweist. Mehrere Aspekte sind hier zu unterscheiden:

\paragraph{a) Relevanz der auf Arrow gestützten Kritik der "`Identitätstheorie"'}
Weitgehend ausgespart bleiben soll hier wiederum die Frage der Relevanz der auf
den Satz von Arrow gestützten Einwände. Wie schon gegen die Rousseau'sche
Demokratietheorie gibt es auch gegen diese Art von Identitätstheorie unabhängig
von Arrow weitere Einwände, die möglicherweise sehr viel einschlägiger sind. Der
historisch wirksamste Einwand gegen diese Art von Identitätstheorie dürfte
\marginline{Demokratie als "`Mehrheitstyrannei"'}
derjenige sein, dass reine Demokratie dieser Art auf eine "`Mehrheitstyrannei"'
hinausläuft. Die Kritiker der "`Mehrheitstyrannei"' bestritten dabei nicht, dass
es in der Demokratie die "`Mehrheit"' ist, die (schlimmstensfalls) tyrannisch
herrscht, nur bezweifelten sie, dass die Mehrheit immer im Einklang mit dem
Gemeinwohl und unter der Achtung der Rechte der Minderheit herrschen würde.
Diejenigen der heutigen Demokratiekritiker, die sich auf die Sozialwahltheorie
stützen, bestreiten bereits, dass die durch eine Wahl herbeigeführte Entscheidung
in jedem Fall Ausdruck des Willens der Mehrheit ist. Unabhängig von den
technischen Beschränkungen der Abbildung individueller auf kollektive
Präferenzen, wie sie uns der Satz von Arrow vor Augen führt, ist schon die
Tatsache, dass es bei so gut wie allen Mehrheitsentscheidungen eine Minderheit
gibt, die die Überzeugung der Mehrheit nicht teilt, Grund genug dafür, die
Vorstellung, dass demokratische Mehrheitsentscheidungen eine Identität von
Herrschern und Beherrschten herbeiführen in einem anderen als bloß sehr schwachen
symbolischen Sinne zurückzuweisen.\footnote{Vgl. dazu auch \cite{sartori:1987},
besonders das 2. Kapitel.} Es bleibt hinsichtlich der "`Identitätstheorie"' also
nur noch die Frage, ob der Satz von Arrow dem noch ein weiteres hinzufügt. Man
muss aber gar nicht unbedingt nur wie Nida-Rümelin auf die Identitätstheorie
abstellen. Denn auch unabhängig von der "`Identitätstheorie"' stellt sich die
Frage, inwiefern demokratische Mehrheitsentscheidungsverfahren Legitimität
erzeugen können und zur effizienten Lösung politischer Probleme taugen. Der Satz
von Arrow spricht in dieser Hinsicht eine bestimmte Art von möglichen Problemen
an.
  
\paragraph{b) Die Gültigkeit der Voraussetzungen des Satzes von Arrow}

Bevor wir sagen können, dass der Satz von Arrow mögliche Probleme
demokratischer Entscheidungsprozesse beschreibt, müssen wir uns erstens
überlegen, ob demokratische Entscheidungsprozesse mit einem theoretischen
Modell der Abbildung individueller auf kollektive Präferenzen richtig
beschrieben werden und zweitens, wenn dies der Fall ist, ob die Voraussetzungen
des Satzes von Arrow tatsächlich notwendige\footnote{Dass es keine
hinreichenden Bedingungen sind, dürfte offensichtlich sein. Die auf Arrow
gestützten demokratieskeptischen Argumente laufen denn auch normalerweise so:
Wenn sich diese "`harmlosen"' Bedingungen (d.i. die Voraussetzungen für den Satz
von Arrow) schon nicht erfüllen lassen, dann muss man nach anspruchsvolleren
Bedingungen gar nicht erst fragen.} Mindestbedingungen demokratischer
Entscheidungsprozesse repräsentieren. 

Hinsichtlich des ersten Punktes, dass das Modell der Abbildung individueller auf
kollektive Präferenzen demokratische Entscheidungsprozesse richtig erfasst, liegt
zunächst der Einwand nahe, dass demokratische Entscheidungsprozesse in erster
Linie deliberative Prozesse sind, bei denen die Gegenstände der politischen
Entscheidungen und die sich bietenden Alternativen erst in öffentlichen
Diskussionsprozessen bestimmt werden. Die individuellen Präferenzen sind nach
dieser Sichtweise nicht einfach ein Eingangsparameter des politischen Prozesses,
sondern zumindest teilweise bilden sie sich erst im Laufe des Prozesses, wandeln
sich, gleichen sich aneinander an, oder dissoziieren sich voneinander, ordnen
sich nach politischen Lagern etc. All diese Vorgänge und wohlbekannten Phänomene
werden von der Sozialwahltheorie bisher noch wenig erfasst.\footnote{Das Thema
"`Wandel von Präferenzen"' findet innerhlab dieser Schule erst neuerlich größere
Beachtung. Bei den im letzten Kapitel besprochenen Ansätzen (Satz von Arrow,
"`Paradox des Liberalismus"') werden die individuellen Präferenzen noch als
gegeben vorausgesetzt und von ihrem möglichen Wandel mit der Zeit oder infolge
von Diskussionprozessen, die Abstimmungen in der Demokratie typischerweise voraus
gehen, wird zunächst abstrahiert.} Dennoch wird die Sozialwahltheorie soweit ihr
das Modell der Aggregation von Präferenzen zu Grunde liegt durch die Allgegenwart
deliberativer Prozesse in der Politik nicht überflüssig gemacht. Denn auch
deliberative Prozesse führen nicht dazu, dass sämtliche Unterschiede zwischen den
Präferenzen von Individuen und Gruppen eingeebnet werden. Am Ende wird auch in
der Demokratie zwischen verschiedenen Alternativen abgestimmt, die von
unterschiedlichen Lagern präferiert werden. Spätestens dann sind wir wieder bei
der Aggregation von individuellen zu Kollektivpräferenzen. Das Vorhandensein
deliberativer Prozesse macht die Präferenzaggregation also nicht überflüssig.
Bestenfalls bewirken deliberative Prozesse, dass Arrows Bedingung des
unbeschränkgen Bereichs (von möglichen individuellen Präferenzprofilen) in der
Praxis nur stark entschärft auftritt. 

Um den zweiten Punkt zu klären, ist es notwendig, die unterschiedlichen
Voraussetzungen von Arrow durchzugehen und darauf hin zu untersuchen, ob sie
tatsächlich unerlässlich sind. Darüber gibt es, wie man sich denken kann, eine
breite Diskussion. Im folgenden sollen nur kurz die wichtigsten theoretischen
(zu den empirischen, siehe unten) Einwände angesprochen werden:

\subparagraph{Transitivität der kollektiven Präferenzen}

Eine der Voraussetzungen von Arrow bestand darin, dass die kollektiven
Präferenzen transitiv sein müssten. Diese Forderung lässt sich dadurch motivieren, dass
intransitive Präferenzen zu bestimmten Problemen führen kann, wie sie durch das
Geldpumpenargumentversinnbildlicht werden. Analog zum Geldpumpenargument kann man
sich im politischen Kontext theoretisch einen Manipulator vorstellen, der einen
Zyklus innerhalb der kollektiven Präferenzen dazu nutzt, um eine bestimmte
politische Agenda durchzusetzen. Aber ebenso wie beim Geldpumpenargument wäre
auch im politischen Kontext der Einwand angebracht, dass sich eine solche
Ausbeutungstechnik praktisch kaum verwirklichen lassen dürfte. Und, wie schon
zuvor erläutert (Seite \pageref{Geldpumpenargument}), zeigt das Argument nicht
dass intransitive oder zyklische Präferenzen schlechthin absurd sind. 
\marginline{Transitivität ist nicht unterlässlich}
Insofern als es denkbar ist, für die aus zyklischen Präferenzen möglicherweise
resultierenden Probleme, praktische Lösungen zu finden, kann man nicht sagen,
dass die Erzeugung transitiver kollektiver Präferenzen zu den unerlässlichen
Bedingungen eines akzeptablen Abstimmungsverfahrens gehört, auch wenn es
natürlich wünschenswert wäre.

\subparagraph{Unbeschränkter Bereich der individuellen Präferenzen}
\label{DiskussionUnbeschraenkterBereich}
Der Bedingung des "`unbeschränkten Bereichs"' kann man unterschiedliche
Interpretationen geben:

\begin{enumerate}
  \item Unbeschränkter Bereich heisst, dass weder die Menge der Güter, über die
  die Präferenzen gebildet werden sollen, in irgendeiner Weise beschränkt ist,
  noch die Art und Weise wie diese Güter durch die
  individuellen Präferenzrelationen angeordnet werden (solange die üblichen
  Bedingungen wohlgeformter Präferenzrelationen wie Zusammenhang, Transitivität
  etc., erfüllt sind).
\marginline{Beliebige Güter oder nur beliebige Anordnung der Güter?} 
  \item Unbeschränkter Bereich bedeutet, dass zwar die Menge der Güter, über
  die die Präferenzen gebildet werden sollten, eingeschränkt sein kann, nicht
  aber die Ordnung der Güter innerhalb der individuellen Präferenzen.
\end{enumerate}

Gegen die erste Interpretation spricht ein logischer und ein
normativ-politischer Einwand. Der logische Einwand ist der Folgende:
\marginline{Logischer Einwand}
Angenommen, die Menge der möglichen Güter, auf die sich die individuelle
Präferenzrelation beziehen darf, wäre in jeder Hinsicht unbeschränkt, dann
können die Individuen auch Präferenzen darüber bilden, ob sie z.B. die
Gültigkeit der Bedingung der paarweisen Unabhängigkeit bei Abstimmungsverfahren
gegenüber der Ungültigkeit dieser Bedingung bevorzugen oder nicht. Angenommen
nun, die individuellen Präferenzen sind so verteilt, dass alle Individuen
einhellig dagegen sind, die paarweise Unabhängigkeit zur Voraussetzung
eines Abstimmungsverfahrens zu machen, dann kann man diese Bedingung nur unter
Bruch der Effizienz-Bedingung ("`Pareto-Kriterium"') aufrecht erhalten. Mit
anderen Worten: Bei einer weiten Auslegung des "`unbeschränkten Bereichs"'
geraten die Bedingungen Arrows also untereinander in einen Widerspruch.

Der normativ-politische Einwand, dass wir schon aus moralischen
Gründen bestimmte Güter und Präferenzen, z.B. solche
\marginline{Normativer Einwand} 
die Menschenrechtsverletzungen beinhalten oder
die auf die Abschaffung der Demokratie oder die Wiedereinführung der Sklaverei
zielen oder dergleichen, von vornherein ausschließen. Es ist nicht ganz klar, ob
man moralische Restriktionen stets so modellieren kann, dass sie sich nur im
Sinne einer Beschränkung der Menge der zur Disposition stehenden Güter auswirken,
oder ob sie in manchen Fällen nur so modelliert werden können, dass die Menge der
möglichen Präferenzrelationen über einer Gütermenge eingeschränkt wird. Im
ersteren Fall würde man lediglich von der ersten zur zweiten Interpretation der
Bedingung des "`unbeschränkten Bereichs"' übergehen müssen. Im zweiten Fall wäre
dann immer noch die Frage, ob durch die Beschränkung der zugelassenen
Präferenzordnungen infolge moralischer Restriktionen alle problemerzeugenden
Präferenzprofile (im Sinne des Satzes von Arrow) wegfallen. Da man dies nicht
annehmen kann, sollte man vorsichtshalber davon ausgehen, dass sich moralische
Restriktionen (wie schon zuvor die deliberativen Prozesse) höchstens dahingehend
auswirken, dass die Bedingung des unbeschränkten Bereichs möglicherweise
entschärfen.\footnote{"`Entschärfen"' in dem Sinne, dass problemative
Präferenzprofile seltener oder unwahrscheinlicher werden.} Ein Einwand, der zur
gänzlichen Zurückweisung der Bedingung des unbeschränkten Bereichs führt, ergibt
sich aus der Berücksichtigung moralischer Restriktionen also nicht. 

Die Diskussion zeigt aber, dass man die Bedingung des unbeschränkten Bereichs
nicht schon dadurch verteidigen, dass jede Einschränkung des Bereichs
zugelassener Präferenzen notwendigerweise autoritär oder paternalistisch 
und mit elementaren Prinzipien des Liberalismus und der Demokratie unvereinbar
wäre (VERWEIS MUELLER).

\subparagraph{Pareto-Effizienz}

Das Kriterium der Pareto-Effizienz scheint zunächst hochgradig selbstevident zu
sein. Warum sollte man eine bestimmte Entscheidung treffen, wenn es eine andere
gibt, bei der es einigen besser aber niemandem schlechter ergehen würde? Aber
man kann die Sache auch von einem anderen Gesichtspunkt betrachten: Wenn man
die Wahl hat zwischen einer pareto-effizienten Diktatur und einer
pareto-ineffizienten Demokratie, sollte man dann nicht lieber die
pareto-ineffiziente Demokratie vorziehen. Natürlich käme es wohl auch darauf
an, wie "`ineffizient"' die Demokratie wäre. 
\marginline{Grenzen der Pareto-Effizienz} Aber dass maximale
Pareto-Effizienz zu einem notwendigen Kriterium eines
Kollektiventscheidungsverfahrens erklärt wird, und damit gegenüber anderen
Werten und Zielsetzungen nicht mehr abwägugngsfähig ist, ist alles andere als
apriori selbstverständlich. 

\subparagraph{Unabhängigkeit von dritten Alternativen}

Das Prinzip der Unabhängigkeit von dritten Alternativen wirft ähnliche Fragen
auf wie die Transitivität, ist dabei aber noch um einiges umstrittener.
Motivieren lässt sich dieses Prinzip zunächst dadurch, dass ohne dieses Prinzip
durch hinzufügen von weiteren "`irrelevanten"' Alternativen das
Abstimmungsergebnis theoretisch manipuliert werden kann. Diese Motivation ist
ähnlich wie das Geldpumpenargument pragmatischer und nicht logischer Natur.
Insofern müssen die Probleme, die durch den Wegfall der paarweisen
Unabhängigkeit entstehen können, nicht von vornherein als unüberwindlich
angesehen werden. Umgekehrt wirft das Prinzip, nur paarweise Vergleiche zwischen
den vorhandenen Alternativen zuzulassen, seinerseits Probleme auf, denn es führt
dazu, dass ein möglicherweise sehr relevanter Teil der Informationen über die
Rangfolge der Präferenzen zwangläufig vernachlässigt werden muss.

\marginline{Beispiel für die Relevanz dritter Alternativen}
Dazu ein Beispiel: Angenommen der Kleingärtnerverein entscheidet darüber, welche
Getränke zur Jahreshauptversammlung gereicht werden sollen. Der Getränkehändler
bietet einen fetten Preisnachlass an, wenn nur er nur eine Sorte Getränke liefern
muss. Den Preisnachlass wollen unsere Kleingärtner natürlich unbedingt in
Anspruch nehmen. Es bleibt also nur noch die Frage des Auswahl des Getränks. Die
eine Hälfte der Kleingärtner habe die Präferenz: $Bier \succ Cola \succ Limo
\succ Wasser \succ Saft$ Die andere Hälfte der Kleingärtner -- womöglich
Antialkoholiker -- habe die Präferenz $Cola \succ Limo \succ Wasser \succ Saft
\succ Bier$. Nun teilt der Getränkehändler weiterhin mit, dass abgesehen von
Bier und Cola wegen eines Streiks keine anderen Getränke mehr lieferbar sind.
Die Kleingärtner sind also gezwungen zwischen Bier und Cola zu entscheiden, da
alle anderen Alternativen jetzt "`irrelevant"' geworden sind. Frage: Sollte bei
dieser Entscheidung die Information, dass Bier bei der einen Gruppe an der
allerletzten Stelle steht, während beide Gruppen Cola an die erste oder zweite
Stelle setzten, wirklich nicht in die Entscheidung einbezogen werden dürfen?
Das Prinzip der paarweisen Unabhängigkeit würde es verbieten, solche
Informationen zu verwenden. Das Beispiel legt jedoch eher nahe, dass das
Prinzip der paarweisen Unabhängigkeit zu eng gefasst ist, indem es nicht nur
die Unabhängigkeit von "`irrelevanten"' Alternativen sicherstellt, sondern --
je nach Umständen -- auch relevante Alternativen aus der Betrachtung
ausschließt.

\subparagraph{Diktaturfreiheit}

Das Prinzip der Diktaturfreiheit ist wohl das einzige, an dem man ohne wenn und
aber festhalten wird, wenn man die Theorie auf die Politik übertragen will. Denn
wenn man schon die Diktatur zulässt, benötigt man auch keine Theorie der
Abstimmung mehr. Es ist ja zuallererst das Problem, ein Abstimmungs- bzw.
Kollektiventscheidungsverfahren zu finden, dass möglichst Vielen möglichst
gerecht wird, welches die Entwicklung dieser Theorie motiviert. Wenn man nun das
Prinzip der Diktaturfreiheit aufgibt, bräuchte man auch die Theorie nicht
mehr.\footnote{Unabhängig von der Sozialwahltheorie kann man aber immer noch die
Frage diskutieren, ob in bestimmten Situationen diktatorische
Entscheidungsverfahren nicht empfehlenswert sein können. Bekanntlich kannte die
römische Republik die Institution einer Diktatur auf Zeit (1 Jahr), um bei
besonderen Bedrohungen der politischen Ordnung die Fähigkeit zu raschen
Entscheidungen sicher zu stellen.} Zudem ist das Prinzip der Diktaturfreiheit so
defensiv gefasst, dass man sich anders als bei den anderen Prinzipien kaum noch
Abschwächungen vorstellen kann.

Wie man sieht, hält abgesehen von der Diktaturfreiheit nur das Prinzip des
unbeschränkten Bereichts, wenn man es nicht zu weit auslegt, den möglichen
Einwänden stand. Bei allen anderen Prinzipen kann man zwar zugestehen, dass sie
berechtigte Anliegen artikulieren, dabei aber oft enger sind, als dies
notwendig erscheint, und zugleich andere, ebenso berechtigte Anliegen,
ausschließen (wie z.B. die Berücksichtung der gesammten Rangordnung und nicht
nur des paarweisen Verhältnisses von Gütern nach den gegebenen Präferenzen). 
Anstatt sich die Bedingungen Arrows also als notwendige Voraussetzungen
vorzustellen, die jedes akzeptable Kollektiventscheidungsverfahren mindestens
erfüllen muss, sollte man sie lieber als wünschenswerte Voraussetzungen
betrachten, von denen der Satz von Arrow zeigt, dass sie nicht alle
gleichzeitig erfüllbar sind, so dass man Abwägungen treffen und eventuell
Abstriche machen muss. Es kann aber nicht ernsthaft die Rede davon sein, dass
die Ergebnisse der Sozialwahltheorie -- von denen nicht wenige übrigens
zeigen, dass sich Arrows negatives Resultat schon bei geringfügiger Aufweichung
seiner Voraussetzungen in ein positives Resultat verwandelt
\cite[S. 585ff.]{mueller:2003}, wenn auch jeweils mit mehr oder weniger
erwünschten Nebeneffekten wie z.B. zyklische kollektive Präferenzen -- "`den
Bereich zulässiger Demokratiekonzeptionen ein[schränken]"' und "`Diese
Einschränkung .. apriorisch [ist]"' \cite[S. 185]{nida-ruemelin:1991}.

\paragraph{c) Die Frage der empirischen Möglichkeit und Häufigkeit von
"`Problemfällen"' bei der Aggregation von individuellen Präferenzen}

Das Paradox des Liberalismus und der Satz von Arrow zeigen, dass eine bestimmte
Menge von wünschenswerten Bedingungen nicht miteinander vereinbar sind. Die
Beweise beruhten unter anderem darauf, dass die Bedingung des unbeschränkten
Bereichs in der Weise ausgenutzt wurde, dass gezielt solche Profile
individueller Präferenzen konstruiert wurden, für die die Erfüllung der anderen
Bedingungen nicht mehr möglich ist. Diesen Sachverhalt kann man aber auch so
zu interpretieren versuchen, dass das Condorcet Paradox, der Satz von Arrow
oder das Paradox des Liberalismus und andere verwandte Theorme in der Praxis
nur ganz bestimmte Problemfälle betreffen. So entsteht die zyklische
Präferenzverteilung beim Condorcet-Paradox nur bei ganz bestimmten, "`unglücklich"'
verteilten individuellen Präferenzen. Insofern muss das Condorcet-Paradox nicht
bedeuten, dass demokratische Abstimmungsverfahren grundsächtlich nicht robust
wären (in dem Sinne, dass sie keine wohlgeordneten kollektiven Präferenzen
liefern). Es bedeutet zunächst nur, dass sie in besonderen Fällen nicht robust
sind. Die Frage, die sich dann stellt, ist diejenige, wie häufig derartige
Fälle vorkommen, d.h. ob es sich dabei um seltene Einzelfälle oder um einen
häufig auftretenden Regelfall handelt. Diese Frage kann man auf
unterschiedliche Art und Weise untersuchen: 1) Durch analytische Überlegungen
betreffend die Häufigkeit bzw. Wahrscheinlichkeit von Präferenzprofilen, die
z.B. zu zyklischen kollektiven Präferenzen führen, 2) durch Computersimulationen
und 3) empirisch, indem man nach Beispielen sucht, wo entsprechende
Präferenzprofile aufgetreten sind. Eine ausführliche Übersicht über derartige
Studien (weiter unten mehr dazu) liefert Gerry Mackie \cite[S.
46ff.]{mackie:2003}, der zu dem Ergebnis kommt, dass die Problemfälle logisch
möglich aber empirisch eher unwahrscheinlich sind.\footnote{Mackie reagiert
damit auf die gegenteilige These William Rikers \cite[S. 119ff.]{riker:1982}.}

Man könnte an dieser Stelle immer noch den Einwand vorbringen, dass
demokratische Abstimmungs- und Entscheidungsprozesse sich gerade in solchen
(wenn auch seltenen) kritischen Ausnahmefällen bewähren sollten. Dazu ist
zweierlei zu sagen:

\begin{enumerate}
  \item Wenn die "`Problemfälle"' wirklich nur selten sind, dann genügt dies
  bereits um die These der "`analytischen"' Widerlegung der identären
  Demokratie durch Arrow \cite[]{nida-ruemelin:1991} bzw. der mit
  Arrow begründeten Unfähigkeit demokratischer Entscheidungsverfahren, den
  "`Volkswillen"' zum Ausdruck zu bringen \cite[]{riker:1982} zu erschüttern.
  
  \item Treten die "`Problemfälle"' nur selten auf, dann erscheint -- rein
  technisch betrachtet -- folgende Abhilfe denkbar. Man verwende irgendein
  einigermaßen brauchbares Abstimmungsverfahren, z.b. Condorcet (parweise
  Abstimmung zwischen allen Paaren von Alternativen). Treten zyklische
  Präferenzen auf, schalte man auf einer anderes Abstimmungsverfahren, z.B.
  Borda-Zählung (siehe Übungsaufgabe \ref{BordaAufgabe}, Seite
  \pageref{BordaAufgabe}) um. Durch den Satz von Arrow ist zwar klar, dass auch
  ein solches kombiniertes Verfahren nicht alle Bedingungen erfüllen kann. So
  verletzt z.B. das Borda-Verfahren die Bedingung der paarweisen
  Unabhängigkeit. Aber da es als Teil eines kombinierten Verfahrens auftritt,
  muss diese Verletzung nur noch in den (vermutlich) wenigen Fällen in Kauf
  genommen werden, in denen das Condorcet-Verfahren intransitive Präferenzen
  liefert.
\end{enumerate}


\subsubsection{Eine "`strukturelle Konzeption kollektiver Rationalität"' als
Alternative?}

Bleibt, was den Entwurf Nida-Rühmelins betrifft, schließlich die Frage, ob er
eine gangbare Alternative anbieten kann. Sein Vorschlag, der ganz dem Kanon der
liberalen Demokratietheorie entspricht, sieht ein zweistufiges Verfahren vor, bei
dem individuelle Rechte den kollektiven Entscheidungen vorgeordnet werden
\cite[S. 196ff.]{nida-ruemelin:1991}. Mit anderen Worten: Kollektive
Entscheidungen dürfen sich von vornherein nur auf einen bestimmten Bereich von
Entscheidungsgegenständen beziehen, während andere Gegenstände, weil sie
individuelle Rechte berühren von vornherein nicht zur Disposition kollektiver
Entscheidungen führen.  Da damit aber nur die Menge der zur kollektiven
Entscheidung zugelassenen Güter nicht aber die Ordnung der individuellen
Präferenzen über diese Güter beschränkt ist (siehe dazu auch die Diskussion der
Bedingung des "`unbeschränkten Bereichs"' weiter oben auf Seite
\pageref{DiskussionUnbeschraenkterBereich}), bleibt vollkommen unersichtlich,
inwiefern sich auf diese Weise die durch den Satz von Arrow aufgeworfenen
Probleme vermeiden lassen sollen. Möglicherweise fallen die Probleme weniger
gravierend aus, weil derartige strukturelle Beschränkungen z.B. die Menge der zur
Wahl stehenden Güter verringern könnten, aber Nida-Rümelin erläutert dies nicht.
Insofern löst die "`strukturelle Rationalität"' Nida-Rümelins weder das Problem
noch kann man sie umgekehrt in sinnvoller Weise durch die von Arrow, Sen und
anderen aufgeworfenen Schwierigkeiten kollektiver Entscheidungsfindung
motivieren.


\section{Die These des "`demokratischen Irrationalismus"'}

Bereits einige Jahre zuvor und sehr viel wirkungsmächtiger hat William Riker
gestützt auf den Satz von Arrow die These vertreten, dass jedes kollektive
Entscheidungsverfahren (und damit insbesondere auch alle demokratischen
Entscheidungsverfahren) in vielfach chaotisch, von Zufällen bestimmt, kurz, in
hohem Grade sinnlos sind:

\begin{small}
\begin{quotation}
The main thrust of Arrow's theorem and all the associated literature is that
there is an unresolvable tension between logicality and fairness. To guarantee
an ordering or a consistent path, independent choice requires that there be
some sort of concentration of power (dictators, oligarchies or collegia of
vetoers) in sharp conflict with democratic ideals. \ldots

These conflicts have been investigated in great detail, especially in the last
decade; but no adequate resolution of the tension has been discovered, and it
appears quite unlikely that any will be. The unavoidable inference is,
therefore, that, so long as a society preserves democratic insitutions, its
members can expect that some of their social choices will be unordered or
inconsistent. And when this is true, no meaningful choice can be made. If $y$
is in fact chosen -- given the mechanism of choice and the profile of
individual valuations -- then to say that $x$ is best or right or more desired
is probably false. But it would also be equally false to say that $y$ is best
or right or most desired. And in that sense, the choice lacks meaning.
\cite[S. 136]{riker:1982}
\end{quotation}
\end{small}

William Riker steht mit dieser Auffassung keineswegs allein. Auch wenn er sie in
einer vergleichsweise scharfen Form vertritt, so handelt es sich dabei um eine
Konsquenz, die von zahlreichen Autoren aus dem Satz von Arrow gezogen wird
\cite[S. 10-15]{mackie:2003}\footnote{Dies könnte vielleicht auch damit zusammen
hängen, dass die meisten dieser Autoren aus dem ökonomischen Spektrum stammen und
in der Staatsphilosophie die liberalen Werte höher als die demokratischen
schätzen.} Diese Sichtweise ist von Rikers Kritiker Gerry Mackie als die These
des "`demokratischen Irrationalismus"' bezeichnet worden. Riker selbst, der sich
-- durchaus glaubwürdig -- als liberaler Demokrat verstand, hat diese Bezeichnung
nicht gebraucht, sie trifft seine These aber sehr gut. Riker geht nicht soweit,
demokratische Entscheidungsverfahren grundsätzlich abzulehnen, aber seiner
Ansicht nach müssen wir ihren Sinn und ihre Funktion anders verstehen. Der Sinn
demokratischer Wahlen liegt für ihn nicht darin, dem Willen der Mehrheit
politisch Geltung zu verschaffen, sondern er allein darin, dass durch das
Instrument der Wahl die Führung abgewählt und von Zeit zu Zeit ausgewechselt
werden kann. Der Sinn demokratischer Wahlen erschöpft sich für ihn also allein in
der Funktion der Machtkontrolle. Diese sehr reduzierte Deutung demokratischer
Wahlen hat zugleich die Nebenwirkung, politischen Entscheidungen in der
Demokratie ihre Legitimität zu entziehen, da ja nicht mehr gut behauptet werden
kann, dass sie durch den Mehrheitswillen legitimiert sind.

Ähnlich wie Nida-Rühmelin glaubt Riker, dass seine These wesentelich
anylitscher Natur ist, und sich im Wesentlichen durch die mathematische Analyse
von Wahlverfahren begründen lässt. Dennoch liefert er auch eine Reihe
historischer Beispiele, die seine These stützen sollen. 

Im einzelnen beruht Rikers These auf folgenden Punkten:

\begin{enumerate}
  \item {\em Nicht-Existenz einer wahren sozialen Wahl}: Es gibt kein
  Wahlverfahren, dass alle Bedingungen der Fairness und Konsistenz erfüllt. 
  Unter denen, die sie nicht erfüllen, gibt es mehrere
  gleich gute bzw. gleich schlechte Verfahren, die aber in bestimmten
  Fällen, von denen Riker glaubt, dass sie recht häufig vorkommen, jeweils
  andere Ergebnisse liefern, so dass man von keiner Methode sagen 
  kann sie liefere die "`wahre"'
  soziale Wahl. \cite[S. 111ff.]{riker:1982}

  \item {\em Sinnlosigkeit der sozialen Wahl}: Bei allen demokratischen
  Entscheidungsverfahren, werden einige Entscheidungen ungeordnet oder
  inkonsistent sein (intransitive kollektive Präferenzen!). In diesem Fall ist
  die soziale Wahl sinnlos. \cite[S. 136ff.]{riker:1982}

  \item {\em Verdeckung der wahren Präferenzen durch strategisches
  Wahlverhalten}: Durch "`startegisches Wählen"' verdecken die Akteure ihre
  wircklichen Präferenzen, so dass am Ende nicht mehr deutlich ist, inwiefern
  eine getroffene soziale Entscheidung Ausdruck der wirklichen Präferenzen der
  Individuen ist. \cite[S. 167ff.]{riker:1982}
  
  \item {\em Manipulationsanfälligkeit der sozialen Wahl}: Viele demokratische
  Wahlverfahren erweisen sich als manipulationsanfällig (z.B. durch die
  Einführung "`irrelevanter"' Alternativen, sofern es sich um Verfahren
  handelt, die die Bedingung der paarweisen Unabhängigkeit verletzen). Auch dies
  erschüttert die Glaubwürdigkeit der sozialen Wahl. \cite[S.
  192ff.]{riker:1982}
\end{enumerate}

Im folgenden sollen -- im Wesentlichen anhand der Kritik Mackies
\cite[]{mackie:2003} -- die ersten beiden Punkte einer (vorwiegend)
theoretischen Kritik unterzogen werden und die letzten beiden Punkte 
anhand historischer Beispiele untersucht werden.

HIER FEHLT NOCH EIN TEIL DES KAPITELS !!!

\subsection{Historische Beispiele}

Diejenigen der historischen Beispiele Rikers, die hier diskutiert werden
sollen, führen uns in die Zeit unmittelbar vor dem amerikanischen Bürgerkrieg.
Daher ist zunächst etwas zum historischen Hintergrund zu sagen. 

Im Laufe der ersten Hälfte des 19. Jahrhunderts hatte sich in den Vereinigten
Staaten ein Zwei-Parteien System herausgebildet, mit den ``Whigs'' auf der einen
und den ``Demokraten'' auf der anderen Seite. Diese Parteien waren zunächst
Sammlungsbewegungen ohne scharfes ideologisches Profil. In beide Parteien
strömten die früheren Föderalisten (was zeigt, dass die Spaltung zwischen
Föderalisten und Antiföderalisten aus der Gründungszeit überwunden war) und beide
Parteien waren sektionsübergreifend in dem Sinne, dass die Parteigrenzen auch
nicht strikt entlang geographischer Regionen (etwa Nordstaaten-Südstaaten oder
Neu England-Westen) verliefen. Dies änderte sich jedoch in der
Mitte der 19. Jahrhunderts und einer der wesentlichen Auslöser war die am 8.
August 1846 ins Repräsentanten-Haus eingebrachte Wilmot-Klausel, die ein Verbot
der Sklaverei in den neuerworbenen (bzw. neu zu erwerbenden) Gebieten Texas und
New Mexico forderte. Der Vorstoß scheiterte zwar, führte aber die Frage der
Sklaverei in den neuen Gebieten als bestimmendes Thema der amerikanischen Politik
der folgenden Jahrzehnte ein. Das Thema Sklaverei bewirkte eine zunehmende
Polarisierung der politischen Lager, wobei die Frontlinien mehr und mehr entlang
der Sektionsgrenzen verliefen. Diese Veränderung der politischen Landschaft
spiegelte sich in der Umformung des Parteiensystems wieder. Die Partei der Whigs
zerfiel und ging schließlich größtenteils in der 1954 von
Anti-Sklaverei-Aktivisten neu geründeten Partei der Republikaner auf. Die
Demokraten wandelten sich mehr und mehr zu einer Südstaaten-Partei, eine Prägung,
die sie bis weit ins 20. Jahrhundert beibehalten sollten. Daneben entstanden als
Übergangserscheinung in der Mitte des 19. Jahrhunderts eine Reihe kurzlebiger
Parteien, von für uns aber nur die gegen die Sklaverei gerichtete ``Free Soil
Party'' im Zusammenhang mit der Wilmot-Klausel eine Rolle spielt.

Die Präsidentschaftswahl von 1860, neben der Wilmot-Klausel das zweite Beispiel
Rikers, das hier besprochen werden soll, fand in einer aufgeheizten Atmosphäre
statt. Von den Vier Kandidaten vertrat der schließlich zum Präsidenten gewählte
Abraham Lincoln die vergleichsweise ``radikalste'' Anti-Sklaverei Position. Noch
bevor er sein Amt am 4. März 1861 antrat hatten die Südstaaten mit der Sezession
begonnen. Riker zufolge lagen sowohl bei den Abstimmungen im
Repräsentantenhaus über die Wilmot-Klausel als auch bei den 
Präsidentschaftswahlen von 1960 zyklische Präferenzen vor. Der Ausgang der Wahl
und damit der folgenschweren Ereignisse, die zum amerikanischen Bürgerkreig
führten, waren Rikers Deutung zufolge, also ein eher zufälliges Artefakt des
Wahlsystems. Auch wenn die Abschaffung der Sklaverei, die sich dadurch
ergab, natürlich befürwortenswert ist: ``A fortunate {\em by-product} of
that process was the abolition of slavery'' \cite[S. 232,
hervorhebung von mir, E.A.]{riker:1982}

Wie Riker seine Deutung(en) belegt soll nun im Einzelnen untersucht werden.

\subsubsection{Die Wilmot-Klausel}

Bei der Wilmot-Klausel handelt es sich um eine vom Abgeordneten David Wilmot
vorgeschlagene Ergänzung zu einem vom damals regierenden Präsidenten James K.
Polk eingebrachtem Budget-Gesetz. Das Budgetgesetz (``appropriations bill'') von
Polk sah vor einer größere Summe an Haushaltsmitteln zur Bestechung der
mexikanischen Armee einzustzen, um den Krieg mit Mexiko, der über die Annexion
von Texas entbrannt war, frühzeitig und mit vorteilhaftem Friedensschluss zu
beenden. Wilmot brachte nun erstmals den Ergänzungsvorschlag ein, dass die Mittel
dafür nur bewilligt werden sollten, wenn die neu erworbenen Territorien den
``freien'' Staaten bzw. Territorien zugeschlagen würden, in denen das Verbot der
Sklaverei galt. Der Antrag wurde in den folgenden Jahren mehrmals eingebracht.
Wie bei derartigen Anträgen üblich, fanden mehrere Lesungen und Abstimmungen
darüber statt. Das um den Antrag erweiterte Gesetz wurde schließlich vom
Repräsentatenhaus verabschiedet, scheiterte aber im Senat durch einen
filibuster.\footnote{Mangels einer Redezeitbegrenzung ist es im amerikanischen
Senat möglich, durch beliebig lange Reden einen Gestzesentwurf zu blockieren,
auch wenn (zunächst) keine Chance besteht, ihn durch eine Abstimmungsmehrheit zu
Fall zu bringen. Dieses Vorgehen wird als ``filibuster'' bezeichnet.}

Im Ergebnis wurde also weder das Budgetgesetz noch das um die Wilmot-Klausel
erweiterte Budgetgesetz verabschiedet, sondern der Krieg mit Mexiko noch einige
Jahre weiter geführt. Auch wenn der Krieg schließlich siegreich beendet wurde,
so war dies -- nach Rikers in diesem Punkt glaubwürdiger Deutung -- die von
den meisten am wenigsten präferierte Alternative. Wie kam es dann aber, dass
gerade diese Alternative gewählt wurde. Riker zufolge ist das auf einen Zyklus
in den Präferenzen zurückzuführen. Bezüglich der Mitglieder des
Repräsentantenhauses führt zunächst er folgende Schätzung ihrer Präferenzen an
(``There is not enough votes to ascertain preference orders, but it is easy to
guess what they were.'' \cite[S. 227]{riker:1982}), wobei $a$ für das
ursprüngliche Budgetgesetz steht, $b$ für das Budgetgesetz mit Wilmot-Klauses 
und $c$ für den status quo:

\begin{center}
\begin{tabular}{rll}
Abgeordnete & Faktion & Präferenzen \\ \cline{1-3}
7  & Northern administration Democrats     & $abc$ \\
51 & Northern Free Soil Democrats (Wilmot) & $bac$ \\
8  & Border Democrats                      & $abc$ or $acb$ \\
46 & Southern Democrats                    & $acb$ \\
2  & Nothern Prowar Whigs                  & $cab$ \\
39 & Nothern Antiwar Whigs                 & $cba$ \\
3  & Border Whigs                          & $bac$ or $bca$ \\
16 & Southern and Border Whigs             & $acb$ \\ \cline{1-3}
\end{tabular}
Quelle: \cite{riker:1982}, S. 227.
\end{center}

Aus dieser Schätzung lassen sich die kollektiven Präferenzen ableiten:
\begin{enumerate}
  \item $b \succ_K a$, was wie Riker (leider irrtümlich) glaubt auch das
  Ergebnis einer der Abstimmungen war, die am 8. August 1846 stattfanden. 
  \item $a \succ_K c$, weil zu erwarten ist, dass die Demokraten ihren
  Präsidenten unterstützen.
  \item $c \succ_K b$, auf Grund einer Mehrheit von Südstaatlern, die die
  Wilmot-Klausel ablehnen und Nordstaaten-Whigs, die den Krieg ablehnen, und
  dementsprechend, wie Riker glaubt, jede Art von Kriegspolitik der
  Administration obstruieren.
\end{enumerate}

Es liegt, wenn man dieser Schätzung folgt, also ein Zyklus vor, oder mit Rikers
Worten: ``So there is a clearcut cyclical majority, which is of course complete
disequilibrium.''\cite[S. 227]{riker:1982} Seiner Ansicht nach handelt es sich
dabei um den letzten und schließlich erfolgreichen Versuch der Whigs, ein
politisches Thema zu konstruieren, mit dem es ihnen gelingen würde die
demokratische Partei zu spalten: ``the Wilmot Proviso \ldots may thus be
regarded as the final act in the construction of the slavery issue.''\cite[S.
227]{riker:1982}. Seine Ansicht, dass das Aufkommen des Sklaverei-Themas 
vorwiegend strategischen Überlegungen und politischem Opportunismus zu
verdanken ist, stützt sich dabei (lediglich) auf einige Tagebuch-Äußerungen des
Präsidenten Polk, der den Leuten, die ihm das Regieren schwer machten, allein
solch oberflächliche Motive zugestehen mochte.

Was ist von Rikers Deutung zu halten? Folgt man Mackies detaillierter Kritik,
dann beruht sie zunächst auf einigen sachlichen Fehlern, deren gröbster der
ist, dass Riker eine Abstimmung über das um die Wilmot-Klausel erweiterte
Budget-Gesetz mit einer Abstimmung über die Erweiterung (also nur die
Wilmot-Klausel) verwechselt. Dementsprechend deutet Riker ein
Abstimmungsergebnis als Ausdruck von $b \succ_K a$, welches in Wirklichkeit $b
\succ_K c$ ausdrückt. Die Präferenz $b \succ_K c$ wird nicht nur durch eine
sondern gleich durch mehrere Abstimmungen im Reprästentantenhaus bestästigt
\cite[S. 243ff.]{mackie:2003}. Damit ist aber nicht nur Rikers Annahme, dass $c
\succ_K b$, hinfällig, sondern auch die, dass überhaupt in dieser Frage
zyklische Präferenzen vorlagen. Seine Schätzung der Präferenzen der einzelnen
Faktionen im Repräsentantenhaus ignoriert vorliegende Abstimmungsergebnisse,
durch die sich z.B. die Annahme, die Nordstaaten Whigs hätten als Gegner des
Mexiko-Krieges Obstruktionspolitik betrieben, eindeutig wiederlegen lässt.
Mackie vermutet, dass ihre Haltung eher die gewesen ist, den Krieg abzulehnen,
aber den Truppen im Feld dennoch volle Unterstützung zuzusichern, eine Haltung
für die man auch in anderen Kriegen beispiele findet \cite[S. 248]{mackie:2003}.
(sinngemäss kann man die Haltung so ausdrücken: ``Wir sind gegen den Krieg, aber
wir lassen unsere Jungs trotzdem nicht im Stich!'') Rikers Fehler ist umso
peinlicher, als er als Politikwissenschaftler hätte wissen müssen, dass über
einen Gesetzesentwurf, bevor er vom Repräsentantenhaus an den Senat
weitergeleitet wird, erst noch einmal im Ganzen abgestimmt wird. Peinlich ist
der Fehler nicht nur für Riker, sondern auch für diejenigen (vorwiegend
Vertreter des Public Choice Ansatzes!), die ihn nicht bemerkt haben. Von Mackie
wird dies dementsprechend bissig kommentiert:

\begin{quotation}
Theoretically, any reader should be able to detect the nonsensical error
emodied in Riker's claim that SQ > WP [$c \succ_K b$, E.A.] even without going
back to check the references to the records of Congress, yet for almost twenty
years many intelligent people have repeated this story without reporting the
error. I feel that it is my reluctant duty to report a problem with
public-choice style of explanation. This style of explanation is often not
immediately intuitive yet is gilded with an abstract formalism that suggests
that something important and believable is being said. I am not the first to
suggest that there is no necessary relationship between formalism and
profundity, and that it is just as possible that such models obscure as that
they reveal. \cite[S. 246]{mackie:2003}\footnote{Man müsste auf diesem Fehler
nicht herumreiten, wenn es ein Einzellfall wäre. Aber leider sind derartige
Schwächen in der empirischen Anwendung des Public Choice-Ansatzes ein häufig
anzutreffendes Problem.}
\end{quotation}

Auch Rikers These, dass sich die ``Konstruktion'' des Sklaverei-Themas vor allem
politisch-taktischem Opportunismus verdankte, und sich dieses Thema in einer Art
von natürlichem Selektionsprozess als dasjenige durchgsetzt hat, mit dem es den
Whigs gelang ihre Gegener zu spalten, erscheint fragwürdig. Noch Anfang des 19.
Jahrhunderts ein von der Politik eher unbeachtetes Thema teilte es in der Mitte
des 19. Jahrhunderts das ganze Land in zwei Lager. Sowohl die Presbyterianer als
auch Methodisten spalteten sich darüber. (Warum hätten Sie das wegen eines
Themas, das aus bloß taktischen Gründen von der Teilen der politischen Klasse
``konstruiert'' worden ist, tun sollen?) Dass sich das Thema Sklaverei in der
Politik auch mit ``opportunistischen'' Erwägungenen verband -- so lehnte die Free
Soil Party die Sklaverei auch wegen der geführchteten Konkurrenz durch billige
Sklavenarbeit ab -- schliesst nicht aus, dass es aus Sicht vieler Politiker und
anderer Bürger zugleich ein genuin moralisches Anliegen war. Angesichts der
Leidenschafttlichkeit, mit der über die Frage der Sklaverei im Vorfeld des
Bürgerkriegs gestritten wurde, wirkt Rikers Deutung eher etwas gezwungen.

\subsubsection{Die Präsidentschaftswahl von 1860}

In der Präsidentschaftswahl von 1860 erblickt Riker geradezu eine Wiederholung
des -- wie wir gesehen haben in Wirklichkeit gar nicht vorhandenen --
Ungleichgewichts bei der Entscheidung über die Wilmot-Klausel. Bei der
Präsidentschaftswahl von 1860 traten vier Kandidaten an, Abraham Lincoln
(Republican Party), Stephen Douglas (Northern Democrats), John Breckinridge
(Southern Democrats), John Bell (Constitutional Union Party). Abraham Lincoln
gewann die Wahl obwohl auf Douglas die meisten Stimmen entfielen. Dieses Phänomen
ist leicht durch das amerikanische Wahlsystem zu erklären, bei dem zunächst
innerhalb der einzelnen Bundesstaaten über den Präsidenten abgestimmt
wird,\footnote{Zu Lincolns Zeiten galt das noch nicht für alle Bundesstaaten. In
South Carolina etwa entschied die politische Elite statt der Bürger für welchen
Präsidenten die Wahlmänner votieren sollten.} und die dann Wahlmänner in das
bundesweite Wahlmännerkollegium (``electoral college'') entsenden, das dann den
Präsidenten wählt. In der Regel stimmen {\em alle} Wahlmänner desselben
Bundesstates für den Kandidaten, auf den im Bundesstaat die meisten Stimmen
entfallen sind, was zu erheblichen Verzerrungen des Ergebnisses führen kann und
in diesem Fall auch geführt hat.

Riker glaubt, dass darüber hinaus die kollektiven Präferenzen der Amerikaner
bezüglich drei der vier Präsidentschaftskandidaten in einem Zyklus gefangen
waren, dass also $Douglas \succ_K Lincoln \succ_K Bell \succ_K Douglas \succ_K
Breckinridge$ galt. In Ermangelung von zuverlässugen Daten über die Präferenzen
über alle vier Kandidaten\footnote{Bekannt ist nur, welcher Kanditat in den
einzelnen Regionen an die Spitze kam} rechtfertigt Riker seine These wiederum mit
einer Schätzung der vollen Präferenzen, die er nach Regionen
aufschlüsselt.\cite[S. 230/231]{riker:1982} Wie er zu seiner Schätzung kommt,
bleibt im Dunkeln. Mit seiner Schätzung ergibt sich aber der von ihm behauptete
Zyklus. In den Fällen, in denen das Condorcet-Verfahren (paarweise Stichwahl
über alle Paare von Alternativen) einen Zyklus zu Tage fördert, hat das die
Folge, das unterschiedliche, wenn auch jeweils mit gutem Recht als demokratisch
angesehene Wahlverfahren zu unterschiedlichen Ergebnissen führen. Riker führt
mehrere solcher Verfahren und die sich aus ihnen ergebenden kollektiven
Präferenzen an:

\begin{enumerate}
  \item Mehrheitswahlrecht: $Lincoln \succ Douglas \succ Breckinridge \succ
  Bell$
  \item Paarweiser Vergleich (Condorcet): $Douglas \succ Lincoln \succ Bell 
  \succ Douglas \succ Breckinridge$
  \item Borda Zählung (siehe Aufgabe \ref{BordaAufgabe}): $Douglas \succ Bell
  \succ Lincoln \succ Breckinridge$
  \item Wahl durch Zustimmung (zwei Stimmen): $Bell \succ Lincoln \succ Douglas
  \succ Breckinridge$
  \item Wahl durch Zustimmung (drei Stimmen): $Douglas \succ Bell \succ Lincoln
  \succ Breckinridge$
\end{enumerate}

Bei fünf unterschiedlichen Wahlsystemen gewinnt Douglas zweimal, sonst jedesmal
ein anderer. Damit kann Riker seine These der Sinnlosigkeit der
sozialen Wahl stützen (im Falle eines Ungleichgewichts und zugleich im
Allgemeinen, wenn man mit Riker annimmt, dass solche Ungleichgewichte häufig
vorkommen). Selbst wenn man nämlich die Lincoln-Wahl mit Hinweis auf das
amerikanische Mehrheitswahlsystem, das bekanntermaßen zu starken Verzerrungen
führen kann, kritisiert, so zeigt sich, wenn man Riker folgt, dass ein ``besseres''
Wahlsystem hier auch keine Abhilfe schafft, da unterschiedliche ``bessere''
Wahlsysteme zu unterschiedlichen Ergebnissen führen, womit jedes Ergebnis als
ein zufälliges Artefakt des jeweiligen Wahlsystems erscheint. 

Mackies Kritik an Rikers Analyse fällt ziemlich elaboriert aus. Das hängt damit
zusammen, dass auch Mackie nicht um das Problem herum kommt, dass wir über keine
zuverlässigen Daten über die vollen Präferenzen der Bürger bezüglich ihrer vier
Kandidaten verfügen, die er aber ebenso benötigen würde, um Riker widerlegen zu
können, wie Riker sie bräuchte, um seine These aufzustellen. Eine der wichtigsten
Annahmen von Riker ist dabei die, dass die meisten Lincoln-Wähler Bell und nicht
Douglas an die zweite Stelle setzten. Mackie zieht für seine, von Riker
abweichende Schätzung, drei unterschiedliche Informationsquellen heran: 1. Das
historische Wissen über die damals verbreiteten politischen Standpunkte. 2.
Aggregierte Daten auf Landkreis, Staats- und Sektionsebene. 3. Die aus einer
anderen Studie übernommenen Ergebnisse einer Umfrage unter Fachhistorikern
dieser Epoche bezüglich der vermuteten Präferenzordnung der damaligen Wähler.
\cite[S. 277]{mackie:2003} Im Ergebnis kommt Mackie dabei zu einer anderen 
Präferenzordnung aus der sich kein Zyklus der nach dem
Condorcet-Verfahren abgeleiteten kollektiven Präferenzen mehr ergibt. Bis auf
das Mehrheitswahlrecht, dessen Schwächen hinlänglich bekannt sind, liefern alle
von Riker zum Vergleich herangezogenen Verfahren dasselbe Ergebnis: Douglas
hätte die Wahl gewinnen müssen. Von einem Ungleichgewicht keine Spur.

Um dieses Thema abzuschließen, könnte man angesichts der Tatsache, das Lincoln
eine Wahl gewonnen hat, in der ein anderer Kandidat, nämlich Douglas, die
meisten Stimmen erhielt, könnte man immer noch die Frage an Rikers
These der ``Zufälligkeit'' demokratischer Entscheidungen angelehnte
Frage aufwerfen, ob nicht der Bürgerkrieg auch ein Artefakt der amerikanischen
Mehrheitswahlsystems gewesen ist. Oder anders gefragt: Was wäre geschehen, wenn
Douglas die Wahl gewonnen hätte? Kontrafaktische historische Überlegungen sind
immer eine heikle Sache, denn wir verfügen ebenso wenig über das Wissen, das
uns ermöglichen würde, die Möglichkeit oder Wahrscheinlichkeit alternativer
Geschichtsverläufe zuverlässig einzuschätzen, wie wir die Geschichte vorher
sagen können. Trotzdem sollen zu dieser Frage einige Überlegungen angeführt
werden: 1. Die Polarisierung des Landes durch die Sklavereifrage, war nicht die
Folge einer oder weniger einzelner, möglicherweise zufälliger politischer
Entscheidungen, sondern einer ganzen Reihe von sozialen, wirtschaftlichen und
politischen Vorgängen. Insofern war sie für eine gewisse Zeit eine relativ
stabile Konstante der amerikanischen Politik. 2. Douglas vertrat in der
Sklavereifrage die Doktrin der ``popular souvereignity'', der zufolge die neu
hinzugekommenen Territorien darüber auf lokaler Ebene selbst entscheiden
sollten. Diese Politik hatten auch die meisten Präsidenten vor Lincoln
verfolgt, in der Hoffnung durch diese sehr politische Haltung die Wogen glätten
und die Streitfrage auf Bundesebene entschärfen zu können. Diese Hoffnung hatte
sich schon vorher als trügerisch erwiesen. Am Vorabend der Wahl waren die
Südstaatler kaum noch bereit, sich mit dem vermittelnden Standpunkt Douglas'
zufrieden zu geben, was auch daran deutlich wird, dass die Southern Democrats
mit Breckinridge einen Kandidaten aufstellten, der einen viel entschiedeneren
Pro-Sklaverei Standpunkt vertrat. Insofern ist es fragwürdig ob Douglas als
Gewinner der Wahl die Sezession der Südstaaten hätte verhindern können. 3.
Letzteres gilt umso mehr als Lincoln die Bereitschaft signalisierte, den
Südstaaten weitgehend entgegen zu kommen, auch in der Sklavereifrage. Nur an
einer Forderung hielt er unverbrüchlich fest: Die Sezessionisten müssten sich
wieder in die Union eingliedern. Wenn diese -- natürlich sehr spekulativen --
Überlegungen stimmen, dann hätte ein anderes Wahlsystem (und damit ein anderer
Präsident) an der Sezession und dem darauf folgenden Bürgerkrieg nichts
geändert. Der weitere Verlauf der amerikanischen Geschichte wäre dann in jedem
Fall kein ``zufälliges'' Artefakt des Wahlsystems mehr gewesen.

\section{Fazit}

Wie wir gesehen haben ist weder die Kritik der identären Demokratie noch die
These des ``demokratischen Irrationalismus'', soweit sich beide auf Ergebnisse
des Public Choice Ansatzes wie etwa den Satz von Arrow stützen, besonders
überzeugend. Insbesondere bei der empirischen Anwednung seiner Ergebnisse zeigt
der Public Choice Ansatz bisher noch erhebliche Schwächen. Insofern dies, wie
bei Riker, sehr häufig auch mit handwerklicher Schlamperei zu tun hat, besteht
natürlich Hoffnung, dass sich dies bei einer umsichtigeren Interpretation und
Anwendnung der Ergebnisse noch ändern könnte. Gerry Mackie, auf dessen Kritik
an Riker ich hier zurückgegriffen habe, versteht sich selbst deshalb auch nicht
als Kritiker des Public Choice Ansatzes, sondern möchte der politik- und
demokratieskeptischen Sichtweise einen ``konstruktiven'' Public-Choice Ansatz
entgegenstellen, bei dem nicht unter Berufung auf Arrow die Demokratie
grundsätzlich in Frage gestellt wird, sondern die Social Choice bzw. Public
Choice Theorie genutzt wird, um für unterschiedliche Situationen möglichst
optimale demokratische Abstimmungs- und Verfahrensweisen zu entwerfen. Der Satz
von Arrow zeigt, dass dies ganz ohne Kompromisse nicht möglich ist, aber gerade
das macht die Aufgabe spannend.

Meine eigene Meinung über Public Choice ist etwas skeptischer: Der Ansatz mag
für Spezialthemen, wie die Analyse von Wahlsystemen geeignet sein. Abgesehen
davon ist er weder für die Politikwissenschaft noch für die politische
Philosophie besonders relevant, ganz einfach, weil sich die meisten Vorgänge in
der Politik mit dem Begriffsrepertoire von Public Choice überhaupt nicht
angemessen erfassen und artikulieren lassen. Wer etwas über politische
Philosophie oder darüber, nach welchen Gesetzen Politik abläuft, sollte nach
den Werken aus dem Bereich ``Public Choice'' deshalb höchstens als allerletztes
greifen. 

Aber natürlich kann man auch eine andere Meinung dazu vertreten. Und wer mehr
über die gegenteilige Meinung erfahren möchte, der kann zum Beispiel zu den
Werken von Riker \cite[]{riker:1982} oder dem Public-Choice-Kompendium von Dennis
Mueller\cite[]{mueller:2003} greifen. Um sich über den methodischen Wert von
Public Choice ein Bild zu machen empfehle ich als Vergleich, besonders zu Riker,
die Lektüre eines Werkes wie Govanni Sartories ``Demokratietheorie''
\cite[]{sartori:1987}. Es bietet sich zum Vergleich deshalb besonders an, weil
der Autor politisch eine ähnliche liberale Richtung vertritt wie Riker, weil das
Werk in derselben Zeit wie Rikers wichtigste Bücher entstanden ist, und weil es
andererseits aber auf die Formalismen des Public Choice Ansatzes völlig
verzichtet und statt dessen einen rein verbalen Diskurs über die Themen
Demokratie und Liberalismus führt. Unnötig zu sagen, dass ich das Buch Sartoris
für viel gehaltvoller und dessen politikphilosophischen statt des
mathematischen-formalen Ansatzes inhaltlich für sehr viel fruchtbarer halte. Aber
darüber ist jeder aufgerufen, sich eine eigene Meinung bilden.
