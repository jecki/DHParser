\documentclass[10pt, a4paper, german]{article}
\usepackage[utf8x]{inputenc}
\usepackage{ucs} % unicode
%\usepackage[T1]{fontenc}
%\usepackage{t1enc}
%\usepackage{type1cm}
\usepackage[german]{babel}
 
\usepackage{eurosym} 
\usepackage{amsmath, amssymb}
\usepackage{graphicx}
\usepackage{natbib}
\usepackage{rotating}

\numberwithin {equation}{section}

\sloppy

\begin{document}

\begin{center}
{\large Universität Bayreuth: Philosophy \& Economics, SoSe 2009}
\end{center}
\vspace{0.4em}
\begin{center}
{\huge {\bf Lösung} zur Nachklausur: Grundlagen des Entscheidens I}
\end{center}
\vspace{0.0em}
\begin{center}
Datum: 8. Dezember 2009\\
Dozent: Eckhart Arnold
\end{center}


\section{Aufgabe: Bayes und Entscheidungsbäume}

{\em Lösung zu Aufgabe 1.1 a):} Wenn mit $g$ das Ereignis bezeichnet wird, dass
Gold vorhanden ist und mit $t$ das Ereignis, dass die Prognose positiv ausfällt,
dann berechnet sich die Wahrscheinlichkeit, dass Gold vorhanden ist, sofern die
Prognose positiv ausgefallen ist, $P(g|t)$ nach dem Bayes'schen Lehrsatz wie
folgt:
\begin{eqnarray}
P(g|t) & = &
\frac{P(t|g)\cdot P(g)}{P(t|g)\cdot P(g) + P(t|(\neg g)\cdot P(\neg g)} \\
 & = & \frac{0,88\cdot 0,45}{0.88\cdot 0,45 + 0,03\cdot 0,55} = 0,96 \\
\end{eqnarray}

{\em Lösung zu Aufgabe 1.1 b):} Analog dazu gilt für die Wahrscheinlichkeit,
mit der Gold vorhanden ist, selbst wenn die Expertise negativ ausfällt,
$P(g|\neg t)$:

\begin{eqnarray}
P(g|\neg t) & = &
\frac{P(\neg t|g)\cdot P(g)}{P(\neg t|g)\cdot P(g) + P(\neg t|(\neg g)\cdot
P(\neg g)}
\\
 & = & \frac{0,12\cdot 0,45}{0.12\cdot 0,45 + 0,97\cdot 0,55} = 0,092 \\
\end{eqnarray}

\vspace{0.25cm}

{\em Zur Bewertung: 2 Punkte möglich; \\
Abzüge für fehlerhafte Wahrscheinlichkeiten, Rechnungen etc.}

\vspace{0.5cm}

{\em Lösungshinweis zu Aufgabe 1.2:} Der erste Knoten (von links) des
Entscheidungsbaums sollte die Entscheidung darüber repräsentieren, ob ein Test
durchgeführt wird oder nicht. Sofern der Test durchgeführt wird, sollte in dem
entsprechenden Teilbaum als nächstes ein Ereignisknoten für das positive oder
negative Ergebnis des Tests folgen. Der Ereignisknoten für das Ereignis, dass
Gold vorhanden oder nicht vorhanden ist, darf erst ganz am Ende folgen, also
insbesondere nach der Entscheidung, ob investiert wird oder nicht. Dies
entspricht nicht nur der zeitlichen Reihenfolge (da erst nach der Entscheidung
über die Investition bekannt wird, ob Gold vorhanden ist), sondern darüber
hinaus ist es auch nur so möglich, den Entscheidungsbaum aufzulösen, da man
sonst den Entscheidungsknoten bezüglich der Investition nicht durch das
Ergebnis für die günstigere Entscheidung ersetzen kann.

\vspace{0.25cm}

{\em Zur Bewertung:2 Punkte möglich; \\
1 Punkt jeweils für den richtigen oberen bzw. unteren Teilbaum.}


\vspace{0.5cm}

{\em Lösung zu Aufgabe 1.3:} Der Manager würde zwar die richtige Entscheidung
treffen, aber seine Begründung ist falsch: Nicht schon deshalb, weil der
Erwartungsgewinn (auch) ohne Expertise höher ist als die Investition, erübrigt
sich die Expertise. Sie erübrigt sich vielmehr nur dadurch, dass in diesem
speziellen Fall der Erwartungsnutzen für die Investition selbst in dem Fall
noch positiv ist, dass die Expertise negativ ausfällt. Erst dadurch hat die
Expertise keinen Einfluss auf die Entscheidung mehr.

Rechnerisch lässt sich dies ebensogut dadurch zeigen, dass der Erwartungsnutzen
ohne Expertise höher ist als mit Expertise.

Um sich klar zu machen, dass das Argument des Managers falsch ist, kann man die
Aufgabe so abändern, dass die Einnahmen nur 300 Mio Euro statt 600 Mio Euro
betragen. In diesem Fall würde dieselbe Argumentation des Managers zu einer
falschen Handlungsempfehlung führen.

\vspace{0.25cm}

{\em Zur Bewertung: 2 Punkte möglich;\\
Wurde nur der Erwartungsnutzen falsch berechnet, aber so, dass noch die
richtige Empfehlung herauskam gab es (sofern die Argumentation sonst stimmte)
1,5 Punkte;\\
Wurde der Erwartungsnutzen derart falsch berechnet, dass dies zu einer falschen
Empfehlung führte, dann gab es noch 1 Punkt, sofern erkannt wurde, dass
das Argument des Managers in der Form falsch ist. }

\section{Aufgabe: Sozialwahltheorie}

{\em Lösung zu Aufgabe 2.1:} Die Unmöglichkeit eines Entscheidungsverfahrens,
das alle drei Bedingungen erfüllt, ist dann bewiesen, wenn wir Präferenzen für $A$ und $B$ finden, mit
denen keine Alternative als die kollektiv beste Alternative ausgewiesen werden
kann. Dies ist aber für folgende Präferenzen der
Fall:

\[A:\qquad y \succ x \succ z \]
\[B:\qquad z \succ y \succ x \]

Mit diesen Präferenzen kann keine der drei Alternativen als die beste
gewählt werden, denn:
\begin{enumerate}
  \item Aufgrund der Präferenzen von $A$, und da $A$ die
  Prärogative über $x$ und $z$ ausübt, kann $z$ nicht gewählt werden.
  \item Aufgrund der Präferenzen von $B$, und da $B$ die Prärogative
  über $y$ und $z$ ausübt, kann $y$ nicht gewählt werden.
  \item Aufgrund der Einstimmigkeitsbedingung und der Präferenzen beider, kann
  aber auch nicht $x$ gewählt werden.
\end{enumerate}

\vspace{0.25cm}

{\em Zur Bewertung: 3 Punkte für die richtige Lösung}

\vspace{0.5cm}

{\em Lösung zu Aufgabe 2.2:} Hier sind 3 Fälle zu unterscheiden:
\begin{enumerate}
  \item Beide Individuen haben genau dieselben Präferenzen. Dieser Fall
  ist trivial: Die Festlegung einer kollektiven Präferenz die allen
  Bedingungen genügt, ist durch die Übernahme der (gemeinsamen)
  individuellen Präferenzen möglich.

  \item Beide Individuen stimmen an einer Stelle überein, also z.B.
\[A:\qquad y \succ x \succ z \]
\[B:\qquad z \succ x \succ y \]
  wo $A$ und $B$ auf der 2.Stelle übereinstimmen.

  Ohne Beschränkung der Allgemeinheit kann angenommen werden, dass dasjenige Gut
  auf der Stelle, auf der die Präferenze von $A$ und $B$ übereinstimmen das Gut
  $x$ ist. Wähle nun die kollektiven Präferenzen so, dass $x$ an der selben
  Stelle steht wie in den individuellen Präferenzen. Gebe nun $A$ die
  Prärogative über $x$ und $y$ und $B$ die Prärogative über $x$ und $z$ und
  ordne dann $y$ und $z$ entsprechend den Prärogativen in die kollektiven
  Präferenzen ein.

  \item Beide Individuen stimmen auf keiner Stelle überein. Ohne Beschränkung
  der Allgemeinheit kann folgendes Präferenzprofil angenommen werden (soll
  heißen: Jedes andere Präferenzprofil, bei dem die Individuen auf keiner
  Stelle übereinstimmen lässt sich durch simples Umbenennen der Güter auf
  dieses Präferenzprofil zurückführen):

\[A:\qquad y \succ x \succ z \]
\[B:\qquad z \succ y \succ x \]

  Gebe $A$ die Prärogative über $x$ und $z$ und $B$ über $x$ und $y$. Dann ist
  folgende kollektive Präferenzordnung immer noch möglich:
\[ y \succ_K x \succ_K z \]
\end{enumerate}
In jedem Fall ist es also möglich eine kollektive Präferenzordnung zu finden,
die im Einklang mit den gegebenen Bedingungen steht. q.e.d.

\vspace{0.25cm}

{\em Zur Bewertung: 3 Punkte möglich; \\
Für die richtige Behandlung des kompliziertesten Falles (3. Fall oben) gab es 2
Punkte; \\
Dafür, dass erkannt worden ist, dass eine Fallunterscheidung notwendig ist, gab
es noch einmal 1/2 Punkte.\\
Is mindestens der 2. Fall auch noch richtig erörtert worden (der 1. Fall ist
trivial), so gab es noch einen weitern 1/2 Punkt.}

\section{Aufgabe: Lotterien}

{\em Lösung zu Aufgabe 3.1:} Durch Ausmultiplizieren erhält man:
\[
L[(0.4\cdot 0.1, 0.4 \cdot 0.9, 0.6\cdot 0.4, 0.6\cdot 0.6), (x_1, x_2,
x_3, x_4)]
\]
\[
= L[(0.04, 0.36, 0.24, 0.36), (x_1, x_2, x_3, x_4)]
\]

\vspace{0.25cm}

{\em Zur Bewertung: 2 Punkte für die richtige Lösung.}


\vspace{0.5cm}

{\em Lösung zu Aufgabe 3.2:} Es gibt mehrere Lösungen, je nachdem welche der
drei Variablen $x_1, x_2, x_3$ man in der inneren 2-Güter Lotterie zusammenfasst.
Eine mögliche Lösung lautet:
\[
L[(0.5, 0.5), (x_2; L[(0.2, 0.8),(x_1; x_3)])]
\]
Alternative Lösungen sind:
\[
L[0.5; L[0.2; (x_1, x_2)], L[0.2,(x_2, x_3)])]
\]
oder:
\[
L[(0.1, 0.9), (x_1; L[(5/9, 4/9),(x_2; x_3)])]
\]

\vspace{0.25cm}

{\em Zur Bewertung: 2 Punkte möglich; \\
Teilpunkte für die richtige Form der Lösung, auch wenn die Wahrscheinlichkeiten
nicht stimmten, z.B. wenn sich die Wahrscheinlichkeiten der inneren Lotterie
nicht zu 1 aufaddieren!\\
Falsch (0 Punkte), wenn entweder bei der Lösung einzelne der Güter ``verloren''
gegangen sind, oder am Ende immer noch irgendwo eine 3-Güter Lotterie stand.}


\vspace{0.5cm}

{\em Mögliche Lösung zu Aufgabe 3.3:} Das Beispiel aus Aufgabe 1 lässt sich auf
einfach verschachtelte 2-Güter Lotterien mit beliebigen Zahlenwerten
verallgemeinern:
\[
L[(a_1, a_2), (L[(b_1, b_2),(x_1, x_2)], L[(c_1, c_2),(x_3, x_4)])]
\]
\[
= L[(a_1\cdot b_1, a_1\cdot b_2, a_2\cdot c_1, a_2\cdot c_2),(x_1, x_2, x_3,
x_4)]
\]

Das selbe Verfahren des Ausmultiplizierens lässt sich leicht auf einfach
verschachtelte $n$-Güter Lotterien übertragen, indem man jede Wahrscheinlichkeit
der äußeren Lotterie mit jeder Wahrscheinlichkeit jeder inneren Lotterie
mutlipliziert. Mehrfach vorkommende Güter können (müssen aber nicht einmal)
danach durch Aufaddieren der Wahrscheinlichkeiten zusammengefasst werden.

Nun kann man aber eine beliebig tief verschachtelte $n$-Güter Lotterie
schrittweise von innen nach außen in eine unverschachtelte Lotterie umformen.

\vspace{0.25cm}

{\em Zur Bewertung: 2 Punkte möglich; \\
richtige Idee (z.B. dass man das Problem in Analogie(!) zur Reduzierbarkeit
lösen kann): 0.5 Punkte\\
richtige Lösung (durch Ausmultiplizieren): 1-2 Punkte, je nachdem wie genau und
deutlich die Vorgehensweise beschrieben wurde.}

\vspace{0.5cm}

{\em Mögliche Lösung zu Aufgabe 3.4:} Eine unverschachtelte $n$-Güter Lotterie
hat die Form:

\[
L[(p_1,\ldots, p_n),(x_1,\ldots, x_n)]
\]
Als Erstes denkt man sich die Güter und Wahrscheinlichkeiten
paarweise gruppiert:

\[
[((p_1, p_2), \ldots, (p_{n-1}, p_n)),((x_1, x_2), \ldots, (x_{n-1}, x_n))]
\]

(Ggf., d.h. bei einer ungeraden Anzahl von Gütern, bleibt am Ende ein einzelnes
Gut stehen, was nicht weiter schadet.)

Nun fasst man die Güter und Wahrscheinlichkeitspaare zu Lotterien zusammen, um
die einfach verachtelte Lotterie $L^1$ zu erhalten. Dazu werden die
Wahrscheinlichkeiten eines jeden Güterpaares in der äußeren Lotterie addiert (was
die Wahrscheinlichkeit ergibt, mit der man eines der beiden Güter des Paares
erhalten kann). In der entsprechenden inneren Lotterie über die beiden Güter
müssen die Wahrscheinlichkeiten für jedes einzelne Gut nun noch auf 1 normiert
werden. Als Formel geschrieben, sieht das so aus:

\[
L^1[(p_1+p_2, \ldots, p_{n-1}+p_n),
\]
\[
(L[(\frac{p_1}{p_1+p_2},
\frac{p_2}{p_1+p_2}),(x_1, x_2)], L[(\frac{p_{n-1}}{p_{n-1}+p_n},
\frac{p_n}{p_{n-1}+p_n}),(x_{n-1}, x_n)])]
\]

So wie $L^1$ konstruiert wurde, bilden die inneren Lotterien maximal 2-Güter
Lotterien. Wie man ggf. durch ausmultiplizieren leicht nachweisen kann,
entspricht $L^1$ der ursprünglichen unverschachtelten $n$-Güter Lotterie $L$.
Die Lotterie $L^1$ enthält ihrerseits maximal $n/2+1$ Güter.

Nun wendet man dasselbe Verfahren auf $L^1$ an und erhält eine zweifach
verschachtelte Lotterie $L^2$ (deren sämtliche innere Lotterien
maximal 2-Güter Lotterien sind). Das Verfahren wird so lange fort geführt, bis
die äußere Lotterie nur noch 2 Güter (bzw. Lotterien) enthält.

\vspace{0.5cm}

{\em Alternative Lösung zu Aufgabe 3.4:}
Unverschachtelte Lotterien kann man auch in verschachtelte 2-Güter Lotterien
umwandeln, indem man schrittweise das jeweils erste Gut und die erste
Wahrscheinlichkeit der innersten Lotterie ``abtrennt''. Dabei kann man
praktischerweise auf die Kurzschreibweise für 2-Güter Lotterien zurückgreifen,
d.h. man schreibt $L[p1; x_1, x_2]$ statt $L[(p_1, p_2), (x_1, x_2)]$, da man
ja weiss, dass $p_2 = 1-p_1$ gelten muss (weil die Definition von Lotterien
fordert, dass eines der Güter auf jeden Fall ``gewonnen'' wird).

Am Anfang steht zunächst die unverschachtelte Lotterie:
\[
L[(p_1,\ldots, p_n),(x_1,\ldots, x_n)]
\]
Die verschachtelte Lotterie hat dann die Form:
\[
L[p_1; x1, L[p^*_2, x_2, L[p^*_3, x_3, \ldots]]]
\]
Die Frage ist nun: Wie berechnet man $p^*_2$,$p^*_3$,\ldots ? 
Dazu muss man sich klar machen, dass $p^*_2$ die Wahrscheinlichkeit für das Gut
$x_2$ bezogen auf die Wahrscheinlichkeit derjenigen Lotterie, in der
$p^*_2$ und $x_2$ vorkommen, selbst ist. (Letztere ist aber genau die inverse
Wahrscheinlichkeit, der in der nächst äußeren Lotterie angegebenen
Wahrscheinlichkeit, denn angegeben ist wird ja immer die Wahrscheinlichkeit für
das jeweils ``abgetrennte'' Gut, so dass auf die Lotterie, die (direkt und
indirekt) die restlichen Güter enthält, die entsprechende inverse
Wahrscheinlichkeit entfällt.)

Daraus ergibt sich die Lösung:
\[
L[(p_1; x1, L[p_2/(1-{\bf p_1}); x_2, L[p_3/(1-{\bf p2/(1-p_1))}, \ldots]]]
\]

\vspace{0.25cm}

{\em Zur Bewertung: 2 Punkte möglich;\\ richtige Idee erkennbar: 0,5 Punkte;
Form der verschachtelten Lotterie richtig angegeben: 1 Punkt;\\ Komplett
richtige Lösung, einschließlich der Angabe, wie die inneren
Wahrscheinlichkeiten zu bestimmen sind: 2 Punkte;\\ Abzüge für Schönheitsfehler
etc.}

\end{document}

