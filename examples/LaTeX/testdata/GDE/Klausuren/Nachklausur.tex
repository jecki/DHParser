
\subsection{Die Nachklausur}

\subsubsection{Aufgabe: Entscheidungsbäume}

Ein Automobilunternehmen möchte ein neu zu entwickelndes Elektroauto auf den
Markt bringen, das zwar weniger schnell fährt aber dafür unglaublich sparsam
ist. Es besteht kein Zweifel daran, dass die Entwicklung eines solchen Wagens
technisch möglich ist. Allerdings würden bis zur Marktreife immer noch
Forschungs- und Entwicklungskosten von 10 Mio Euro anfallen. Wird der neue
Wagen vom Markt akzeptiert, so rechnet die Firma mit einem Ertrag von 15
Mio Euro (worin die Forschungs- und Entwicklungskosten {\em nicht} eingerechnet sind).

Aus Konsumentenbefragungen schließt das Management der Firma, dass die Chance
auf einen Markterfolg bei 60\% liegt. Nun erwägt die Firma, die Markteinführung
des neuen Wagens durch eine breit angelegte Werbekampagne abzustützen. Die
Werbekampagne würde noch einmal mit 1 Mio Euro zu Buche schlagen, aber die
Chance auf einen Markterfolg auf 80\% erhöhen.

\vspace{0.5cm}

{\bf Aufgaben}
\begin{enumerate}
  \item Stellen Sie den Entscheidungsbaum auf.
  \item Wie hoch ist der erwartete Ertrag, wenn die Firma eine Werbekampagne
  durchführt? 
  \item Sollte die Firma demnach eine Werbekampagne durchführen?
\end{enumerate}

\vspace{1cm}

\subsubsection{Aufgabe: Bayes'scher Lehrsatz}

Frau Schmitz ist Einkäuferin für die Gemüseabteilung eines großen
Supermarktes. Ihr ist bekannt, dass ca. 3\% des von Zwischenhändlern
angebotenen Gemüses übermäßig stark durch Pestizide belastet ist. Daher führt
sie vor der Abnahme der Ware immer einen standardisierten Schnelltest
auf Pestizidbelastung durch. Dieser Schnelltest hat eine positiv-positiv Rate
von 90\% und eine positiv-negativ Rate von 5\%. 

\vspace{0.5cm}

{\bf Aufgaben}:
\begin{enumerate}
  \item Aufgabe Ein Zwischenhändler bietet Ihr eine Ladung Gurken an, die von
  ihr {\em positiv} gestestet wird. Mit welcher Wahrscheinlichkeit ist damit zu
  rechnen, dass die Gurken tatsächlich pestizid-belastet sind?
  \item Der Zwischenhändler protestiert und verlangt einen zweiten Test nach
  einem aufwändigeren aber genaueren Verfahren. Die Kenndaten dieses Verfahrens 
  sind eine positiv-positiv Rate von 98 \% und eine positiv-negativ Rate von
  1\%. Angenommen der zweite Test nach dem aufwändigeren Verfahren fällt
  negativ aus: Mit welcher Wahrscheinlichkeit ist dann dennoch mit einer
  Pestizidbelastung zu Rechnen?
\end{enumerate}

\vspace{1cm}


\subsubsection{Aufgabe: Einfache Spiele}

Gegeben sei folgende Spielmatrix ("`Chicken Game"'):

\begin{center}
\begin{tabular}{c|c|c|}
\multicolumn{1}{c}{} & \multicolumn{1}{c}{K} &
                               \multicolumn{1}{c}{D} \\ \cline{2-3} 
K               & 0, 0           & -1,1      
\\ \cline{2-3} 
D               & 1,-1           & -10,-10
\\ \cline{2-3}
\end{tabular}
\end{center}  


\vspace{0.5cm}

{\bf Aufgabe}: Bestimme alle Gleichgewichte des Spiels.

\vspace{0.75cm}

\subsubsection{Aufgabe: Wiederholte Spiele}

Im einem paarweisen, unbestimmt oft {\em wiederholten Gefangendilemma} mit
folgender Auszahlungsmatrix

\begin{center}
\begin{tabular}{c|c|c|}
\multicolumn{1}{c}{} & \multicolumn{1}{c}{Kooperiere} &
                               \multicolumn{1}{c}{Defektiere} \\ \cline{2-3} 
Kooperiere      & 3, 3           & 0, 5     
\\ \cline{2-3} 
Defektiere      & 5, 0           & 1, 1
\\ \cline{2-3}
\end{tabular}
\end{center}  

seien folgende vier Strategien vertreten:
\begin{enumerate}
  \item {\em Tit for Tat}: Kooperiert in der ersten Runde und kooperiert in den
  folgenden Runden immer genau dann, wenn die Gegnerstrategie in der vorhergehenden
  Runde auch kooperiert hat.
  \item {\em Random}: Kooperiert oder defektiert vollkommen zufällig.
  \item {\em Dove}: Kooperiert immer.
  \item {\em Hawk}: Defektiert immer.
\end{enumerate} 

\vspace{0.5cm}

{\bf Aufgaben}: Besimme die Durchschnittspunktzahl, die
\begin{enumerate}
  \item {\em Dove} gegen {\em Random} erhält.
  \item {\em Hawk} gegen {\em Random} erhält.
  \item {\em Tit for Tat} gegen {\em Random} erhält.
\end{enumerate}


\vspace{1cm}

\subsubsection{Aufgabe: Beweisaufgabe}

Das sogennante "`Paradox des Liberalismus"' besagt, dass es {\em kein}
Verfahren zum Treffen kollektiver Entscheidungen gibt, welches den weiter
unten angegebenen Bedingungen genügt. Dabei seien mit Kleinbuchstaben
$x,y,z$ die Güter bezeichnet, über deren Anordnung in einer
kollektiven Präferenzrelation entschieden werden muss. Mit
Großbuchstaben $A,B$ seien die Individuen bezeichnet, die dem
Kollektiv angehören. Die Präferenzen eines Individuums $I$ seien mit
$\succ_I, \prec_I, \sim_I $ symbolisiert. Die kollektiven
Präferenzen seien dagegen mit $\succ_K, \prec_K, \sim_K $
bezeichnet. Als Bedingungen gelten:

\begin{enumerate}
  \item {\em minimale Fairness}: Für jedes beteiligte Individuum gilt: Seine
  Präferenzen setzten sich mindestens bei einem Paar von Alternativen durch,
  d.h. \[ \forall_I \exists_{x,y} \quad x \succ_I y \Rightarrow x \succ_K y \]
  \item {\em unbeschränkter Bereich}: Jedes beliebige individuelle
  Präferenzprofil ist zugelassen (solange die Präferenzen wohlgeformt sind).
  \item {\em Pareto-Effizienz}: Wenn {\em alle} Individuen eine bestimmte
  Alternative einer anderen vorziehen, dann sollte die Alternative auch nach der
  kollektiven Präferenzordnung vorgezogen werden, d.h.
  \[ (\forall_I \quad x \succ_I y) \Rightarrow x \succ_K y \]
\end{enumerate}

Angenommen nun, es existiere ein Kollektiv $K$, dem zwei Individuen $A$
und $B$ angehören und es stünden drei Alternativen $x,y,z$ zur Auswahl.

Weiterhin sei gemäß der Bedingung 1 ("`minimale Fairness"') festgelegt, dass
sich bezüglich der Alternative $x$ oder $z$ die Präferenzen des Individuums $A$
durchsetzen und bezüglich der Altenative $y$ oder $z$ die Präferenzen des
Individuums $B$. 


\vspace{0.5cm}

{\bf Aufgabe}: Zeige: Bei "`ungünstig"' verteilten Präferenzen der
Individuen $A$ und $B$ ist es unmöglich unter Erfüllung aller drei
Bedingungen eine der Alternativen $x,y,z$ als die kollektiv am meisten
bevorzugte auszuzeichnen.

\vspace{1cm}

{\em Tipp}: Gehe in folgenden Schritten vor:
\begin{enumerate}
  \item Wähle zuerst möglichst "`ungünstig"' verteilte Präferenzen für $A$ und
  $B$.
 
  {\footnotesize Nur dann kann das Problem überhaupt entstehen. Wenn $A$ und
  $B$ dieselben oder sehr ähnlich Präferenzen hätten, würde die Abbildung ihrer
  individuellen Präferenzen auf eine kollektive Präferenzordnung keinerlei
  Schwierigkeiten aufwerfen.}

  \item Zeige für jedes des Güter $x,y,z$ einzeln, dass es aufgrund einer oder
  mehrerer der drei Bedingungen bei den für $A$ und $B$ festgelegten
  Präferenzen nicht die kollektiv bevorzugte Alternative sein kann.
  
  {\footnotesize Kann man dies für jede der Alternativen zeigen, dann ist damit
  bewiesen, dass die Entscheidung über eine kollektive Präferenzordnung
  unmöglich ist, denn bei einer solche kollektive Präferenzordnung müsste ja
  irgendeine eine Alternative an der Spitze stehen.}
\end{enumerate}

