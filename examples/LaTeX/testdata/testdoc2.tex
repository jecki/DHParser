
\documentclass[12pt, english, a4paper]{article}
\begin{document}

\title{What's wrong with social simulations?}

\date{September 2013; last revision: March 2016}
 
\maketitle

\begin{abstract}
This paper tries to answer the question why the epistemic value of so
many social simulations is questionable. I consider the epistemic
value of a social simulation as questionable if it contributes neither
directly nor indirectly to the understanding of empirical reality. In
order to justify this allegation I rely mostly but not entirely on the
survey by \citet{heath-et-al:2009} according to which 2/3 of all
agent-based-simulations are not properly empirically validated.  In
order to understand the reasons why so many social simulations are of
questionable epistemic value, two classical social simulations are
analyzed with respect to their possible epistemic justification:
Schelling’s neighborhood segregation model \citep{schelling:1971} and
Axelrod’s reiterated Prisoner’s Dilemma simulations of the evolution
of cooperation \citep{axelrod:1984}. It is argued that Schelling’s
simulation is useful, because it can be related to empirical reality,
while Axelrod’s simulations and those of his followers cannot be
related to empirical reality and therefore their scientific value
remains doubtful. Finally, I critically discuss some of the typical
epistemological background beliefs of modelers as expressed in Joshua
Epsteins’s keynote address ``Why model?''
\citep{epstein:2008}. Underestimating the importance of empirical
validation is identified as one major cause of failure for social
simulations.
\end{abstract}

\newpage

\tableofcontents

\section{Introduction}

In this paper I will try to answer the question: Why is the epistemic
value of so many social simulations questionable? Under social
simulations I understand computer simulations of human interaction as
it is studied in the social sciences. The reason why I consider the
epistemic value of many social simulations as questionable is that
many simulation studies cannot give an answer to the most salient
question that any scientific study should be ready to answer: “How do
we know it’s true?” or, if specifically directed to simulation
studies: “How do we know that the simulation simulates the phenomenon
correctly that it simulates?” Answering this question requires some
kind of empirical validation of the simulation. The requirement of
empirical validation is in line with the widely accepted notion that
science is demarcated from non-science by its empirical testability or
falsifiability. Many simulation studies, however, do not offer any
suggestion how they could possibly be validated empirically.

A frequent reply by simulation scientists is that no simulation of
empirical phenomena was intended, but that the simulation only serves
a “theoretical” purpose. Then, however, another equally salient
question should be answered: “Why should we care about the results?”
It is my strong impression that many social simulation studies cannot
answer either this or the first question. This is not to say that the
use of computer programs for answering purely theoretical questions is
generally or necessarily devoid of value. The computer assisted proofs
of the four color theorem \citep{wilson:2002} are an important
counterexample. But in the social sciences it is hard to find
similarly useful examples of the use of computers for purely
theoretical purposes. In any case, the social sciences are empirical
sciences. Therefore, social simulations should contribute either
directly or indirectly to our understanding of social phenomena in the
empirical world.

There exist many different types of simulations but I will restrict
myself to agent-based and game theoretical simulations.  I do not make
a sharp difference between models and simulations. For the purpose of
this paper I identify computer simulations just with programmed
models. Most of my criticism of the practice of these simulation types
can probably be generalized to other types of simulations or models in
the social sciences and maybe also to some instances of the simulation
practice in the natural sciences. It would lead too far afield to
examine these connections here, but it should be easy to determine in
other cases whether the particulars of bad simulation practice against
which my criticism is directed are present or not.

In order to bring my point home, I rely on the survey by
\citet{heath-et-al:2009} on agent-based modeling practice for a
general overview and on two example cases that I examine in detail. I
start by discussing the survey which reveals that in an important
sub-field of social simulations, namely, agent based simulations,
empirical validation is commonly lacking. After that I first discuss
Thomas Schelling’s well-known neighborhood segregation model. This is
a model that I do not consider as being devoid of epistemic
value. For, unlike most social simulations, it can be empirically
falsified. The discussion of the particular features that make this
model scientifically valuable will help us to understand why the
simulation models discussed in the following fail to be so.

The simulation models that I discuss in the following are simulations
in the tradition of Robert Axelrod’s “Evolution of Cooperation”
\citep{axelrod:1984}. Although the modeling tradition initiated by
Axelrod has delivered hardly any tenable and empirically applicable
results, it still continues to thrive today. By some, Axelrod’s
approach is still taken as a role model
\citep[208-209]{rendell-et-al:2010a}, although there has been severe
criticism by others \citep{arnold:2008, binmore:1994, binmore:1998}.

Finally, the question remains why scientists continue to produce such
an abundance of simulation studies that fail to be empirically
applicable. Leaving possible sociological explanations like the
momentum of scientific traditions, the cohesion of peer groups, the
necessity of justifying the investment in acquiring particular skills
(e.g. math and programming) aside, I confine myself to the ideological
background of simulation scientists. In my opinion the failure to
produce useful results has a lot to do with the positivist attitude
prevailing in this field of the social sciences. This attitude
includes the dogmatic belief in the superiority of the methods of
natural sciences like physics in any area of science. Therefore,
despite frequent failure, many scientists continue to believe that
formal modeling is just the right method for the social sciences. The
attitude is well described in \citet{shapiro:2005}. Such attitudes are
less often expressed explicitly in the scientific papers. Rather they
form a background of shared convictions that, if not simply taken for
granted as “unspoken assumptions”, find their expression in informal
texts, conversations, blogs, keynote speeches. I discuss Joshua
Epstein’s keynote lecture “Why Model?” \citep{epstein:2008} as an
example.


\section{Simulation without validation in agent-based models}

In this section I give my interpretation of a survey by
\citet{heath-et-al:2009} on agent-based-simulations. I do so with the
intention of substantiating my claim that many social simulations are
indeed useless. This is neither the aim nor the precise conclusion
that \citet{heath-et-al:2009} draw, but their study does reveal that
two thirds of the surveyed simulation studies are not completely
validated and the authors of the study consider this state of affairs
as ``not acceptable'' \citep[4.11]{heath-et-al:2009}. Thus my reading
does not run counter the results of the survey. And it follows as a
natural conclusion, if one accepts that a) an unvalidated simulation
is - in most of the cases - a useless one and b) agent-based
simulations make up a substantial part of social simulations.

The survey by \citet{heath-et-al:2009} examines agent-based mode- ling
practices between 1998 and 2008. It encompasses “279 articles from 92
unique publication outlets in which the authors had constructed and
analyzed an agent-based model” (Heath, Hill and Ciarallo, 2009,
abstract). The articles stem from different fields of the social
sciences including, business, economics, public policy, social
science, traffic, military and also biology. The authors are not only
interested in verification and validation practices, but the results
concerning these are the results that I am interested in
here. Verification and validation concern two separate aspects of
securing the correctness of a simulation model.  Verification,
as the term is used in the social simualtions community, roughly
concerns the question whether the simulation software is bug-free and
correctly implements the intended simulation model. Validation
concerns the question whether the simulation model represents the
simulated empirical target system adequately (for the intended
purpose).

Regarding verification, Heath, Hill and Ciarallo notice that ``Only 44
(15.8\%) of the articles surveyed gave a reference for the reader to
access or replicate the model. This indicates that the majority of the
authors, publication outlets and reviewers did not deem it necessary
to allow independent access to the models.  This trend appears
consistently over the last 10 years''
\citep[3.6]{heath-et-al:2009}. This astonishingly low figure can in
part be explained by the fact that as long as the model is described
with sufficient detail in the paper, it can also be replicated by
re-programming it from the model description. It must not be forgotten
that the replication of computer simulation results does not have the
same epistemological importance as the replication of experimental
results. While the replication of experiments adds additional
inductive support to the experimental results, the replication of
simulation results is merely a means for checking the simulation
software for programming errors (“bugs”). Hence the possibility of
precise replication is not an advantage that simulations enjoy over
material experiments, as for example \citet[248]{reiss:2011}
argues. Obviously, if the same simulation software is run in the same
system environment the same results will be produced, no matter
whether this is done by a different team of researchers at a different
time and place with different computers. Even if the model is
re-implemented the results must necessarily be the same provided that
both the model and the system environment are fully specified and no
programming errors have been made in the original implementation or
the re-implementation.\footnote{A possible exception concerns the
  frequent use of random numbers. As long as only pseudo random
  numbers with the same random number generator and the same “seed”
  are used, the simulation is still completely deterministic. This not
  to say that sticking to the same “seeds” is good practice other than
  for debugging.} Replication or reimplementation can, however, help
to reveal such errors.\footnote{I am indebted to Paul Humphreys for
  pointing this out to me.} It can therefore be considered as one of
several possible means for the verification (but not validation) of a
computer simulation. Error detection becomes much more laborious if no
reference to the source code is provided. And it does happen that
simulation models are not specified with sufficient detail to
replicate them \citep{will-hegselmann:2008}. Therefore, the rather low
proportion of articles that provide a reference to access or replicate
the simulation is worrisome.

More important than the results concerning verification is what Heath,
Hill and Ciarallo find out about validation or, rather, the lack of
validation:

\begin{quote}
  Without validation a model cannot be said to be representative of
  anything real. However, 65\% of the surveyed articles were not
  completely validated.  This is a practice that is not acceptable in
  other sciences and should no longer be acceptable in ABM practice
  and in publications associated with
  ABM. \citep[4.11]{heath-et-al:2009}
\end{quote}

This conclusion needs a little further commentary. The figure of 65\%
of not completely validated simulations is an average value over the
whole period of study. In the earlier years that are covered by the
survey hardly any simulation was completely validated. Later this
figure decreases, but a ratio of less than 45\% of completely
validated simulation studies remains constant during the last 4 yours
of the period covered \citep[3.10]{heath-et-al:2009}.

Furthermore it needs to be qualified what Heath, Hill and Ciarallo
mean when they speak of complete validation.  The authors make a
distinction between conceptual validation and operational validation.
Conceptual validation concerns the question whether the mechanisms
built into the model represent the mechanisms that drive the modeled
real system. An “invalid conceptual model indicates the model may not
be an appropriate representation of reality.” Operational validation
then “validates results of the simulation against results from the
real system.” \citep[2.13]{heath-et-al:2009}. The demand for complete
validation is well motivated: “If a model is only conceptually
validated, then it [is] unknown if that model will produce correct
output results.” \citep[4.12]{heath-et-al:2009}. For even if the
driving mechanisms of the real system are represented in the model, it
remains – without operational validation – unclear whether the
representation is good enough to produce correct output results. On
the other hand, a model that has been operationally validated only,
may be based on a false or unrealistic mechanism and thus fail to
explain the simulated phenomenon, even if the data matches. Heath,
Hill and Ciarallo do not go into much detail concerning how exactly
conceptual and operational validation are done in practice and under
what conditions a validation attempt is to be considered as successful
or as a failure.

But do really all simulations need to be validated both conceptually
and operationally as Heath, Hill and Ciarallo demand? After all, some
simulations may – just like thought experiments – have been intended
to merely prove conceptual possibilities.  One would usually not
demand an empirical (i.e. operational) validation from a thought
experiment. Heath, Hill and Ciarallo themselves make a distinction
between the generator, mediator and predictor role of a simulation
\citep[2.16]{heath-et-al:2009}. In the generator role simulations are
merely meant to generate hypotheses. Simulations in the mediator role
“capture certain behaviors of the system and [..] characterize how the
system may behave under certain scenarios” (3.4) and only simulations
in the predictor role are actually calculating a real system. All of
the surveyed studies fall into the first two categories. Obviously,
the authors require complete validation even from these types of
simulations.

This can be disputed. As stated in the introduction, in order to be
useful, a simulation study should make a contribution to answering
some relevant question of empirical science. This contribution can be
direct or indirect. The contribution is direct if the model can be
applied to some empirical process and if it can be tested empirically
whether the model is correct. The model’s contribution is indirect, if
the model cannot be applied empirically, but if we can learn something
from the model which helps us to answer an empirical question, the
answer to which we would not have known otherwise. The latter kind of
simulations can be said to function as thought experiments. It would
be asking too much to demand complete empirical validation from a
thought experiment.

But does this mean that the figures from Heath, Hill and Ciarallo
concerning the validation of simulations need to be interpreted
differently by taking into account that some simulations may not
require complete validation in the first place? This objection would
miss the point, because the scenario just discussed is the exception
rather than the rule. Classical thought experiments like Schrödinger’s
cat usually touch upon important theoretical disputes. However, as
will become apparent from the discussion of simulations of the
evolution of cooperation, below, computer simulation studies all too
easily lose the contact to relevant scientific questions. We just do
not need all those digital thought experiments on conceivable variants
of one and the same game theoretical model of cooperation. And the
same surely applies to many other traditions of social modeling as
well. But if this is true, then the figure of 65\% of not completely
validated simulation studies in the field of agent-based simulations
is alarming indeed.\footnote{For a detailed discussion of the cases in
  which even unvalidated simulations can be considered as useful, see
  \citet{arnold:2013}. There are such cases, but the conditions under
  which this is possible appear to be quite restrictive.}

Given how important empirical validation is, “because it is the only
means that provides some evidence that a model can be used for a
particular purpose.” \citep[4.11]{heath-et-al:2009}, it is surprising
how little discussion this important topic finds in the textbook
literature on social simulations. \citet{gilbert-troitzsch:2005}
mention validation as an important part of the activity of conducting
computer simulations in the social sciences, but then they dedicate
only a few pages to it (22-25). \citet[98]{salamon:2011} also mentions
it as an important question without giving any satisfactory answer to
this question and without providing readers with so much as a hint
concerning how simulations must be constructed so that their validity
can be empirically tested. \citet{railsback-grimm:2011} dedicate many
pages to describing the ODD-protocol, a protocol that is meant to
standardize agent-based simulations and thus to facilitate the
construction, comparison and evaluation of agent-based simulations.
Arguably the most important topic, empirical validation of agent-based
simulations, is not an explicit part of this protocol.  One could
argue that this is simply a different matter, but then, given the
importance of this topic it is slightly disappointing that Railsback
and Grimm do not treat it more explicitly in their book.

Summing it up, the survey by Heath, Hill and Ciarallo shows that an
increasingly important sub-discipline of social simulations, namely
the field of agent-based simulations faces the serious problem that a
large part of its scientific literature consists of unvalidated and
therefore most probably useless computer simulations. Moreover,
considering the textbook literature on agent-based simulations one can
get the impression that the scientific community is not at all
sufficiently aware of this problem.


\section{How a model works that works: Schelling’s neighborhood segregation model}

Moving from the general finding to particular examples, I now turn to
the discussion of Thomas Schelling’s neighborhood segregation
model. Schelling’s neighborhood segregation model
\citep{schelling:1971} is widely known and has been amply discussed
not only among economists but also among philosophers of science as a
role model for linking micro-motifs with macro-outcomes. I will
therefore say little about the model itself, but concentrate on the
questions if and, if so, how it fulfills my criteria for epistemically
valuable simulations.

Schelling’s model was meant to investigate the role of individual
choice in bringing about the segregation of neighborhoods that are
either predominantly inhabited by blacks or by whites. Schelling
considered the role of preference based individual choice as one of
many possible causes of this phenomenon – and probably not even the
most important, at least not in comparison to organized action and
economic factors as two other possible causes
\citep[144]{schelling:1971}.

In order to investigate the phenomenon, Schelling used a checkerboard
model where the fields of the checkerboard would represent houses. The
skin color of the inhabitants can be represented for example by
pennies that are turned either heads or tails.\footnote{Schelling’s
  article was published before personal computers existed. Today one
  would of course use a computer.  A simple version of Schelling’s
  model can be found in the netlogo models library
  \citep{Wilensky1999}.} Schelling assumed a certain tolerance
threshold concerning the number of differently colored inhabitants in
the neighborhood, before a household would move to another place. A
result that was relatively stable among the different variants of the
model he examined was that segregated neighborhoods would emerge –
even if the threshold preference for equally colored neighbors was far
below 50\%, which means that segregation emerged even if the
inhabitants would have been perfectly happy to live in an integrated
environment with a mixed population. As \citet{aydinonat:2007}
reports, the robustness of this result has been confirmed by many
subsequent studies that employed variants of Schelling’s model. At the
end of his paper Schelling discusses “tipping” that occurs when the
entrance of a new minority starts to cause the evacuation of an area
by its former inhabitants. In this connection Schelling also mentions
an alternative hypothesis according to which inhabitants do not react
to the frequency of similar or differently colored neighbors but on
their on expectation about the future ratio of differently colored
inhabitants. He assumes that this would aggravate the segregation
process, but he does not investigate this hypothesis further
\citep[185-186]{schelling:1971} and his model is built on the
assumption that individuals react to the actual and not the future
ratio of skin colors.

Is this model scientifically valuable? Can we draw conclusions from
this model with respect to empirical reality and can we check whether
these conclusions are true? Concerning these questions the following
features of this model are important:

\begin{enumerate}

\item The assumptions on which the model rests can be tested
  empirically. The most important assumption is that individuals have
  a threshold for how many neighbors of a different color they
  tolerate and that they move to another neighborhood if this
  threshold is passed. This assumption can be tested empirically with
  the usual methods of empirical social research (and, of course,
  within the confinements of these methods). Also, the question
  whether people base their decision to move on the frequency of
  differently colored neighbors or on their on expectation concerning
  future changes of the neighborhood can be tested empirically.

\item The model is highly robust. Changes of the basic setting and
  even fairly large variations of its input parameters, e.g. tolerance
  threshold, population size, do not lead to a significantly different
  outcome. Therefore even if the empirical measurement of, say, the
  tolerance threshold, is inaccurate, the model can still be applied.
  Robustness in this sense is directly linked to empirical
  testability. It should best be understood as a relational property
  between the measurement (in-)accuracy of the input parameters and
  the stability of the output values of a simulation.\footnote{There
    are of course different concepts of robustness. I consider this
    relational concept of robustness as the most important concept. An
    important non-relational concept of robustness is that of
    derivational robustness analysis
    \citep{kuorikoski-lehtinen:2009}. See below.}

\item The model captures only one of many possible causes of
  neighborhood segregation. Before one can claim that the model
  explains or, rather, contributes to an explanation of neighborhood
  segregation, it is necessary to identify the modeled mechanism
  empirically and to estimate its relative weight in comparison with
  other actual causes. While the model shows that even a preference
  for integrated neighborhoods (if still combined with a tolerance
  limit) can lead to segregation, it may in reality still be the case
  that latent or manifest racism causes segregation. The model alone
  is not an explanation. (Schelling was aware of this.)

\item Besides empirical explanation another possible use of the model
  would be policy advice. In this respect the model could be useful
  even if it does not capture an actual cause. For public policy must
  also be concerned about possible future causes.

  Assume for example, that manifest racism was a cause of neighborhood
  segregation, but that due to increasing public awareness racism is
  on the decline. Then the model can demonstrate that even if all
  further possible causes, e.g. economic causes, be removed as well,
  this might still not result in desegregated
  neighborhoods\footnote{But then, would we really worry about
    segregated neighborhoods, if the issue wasn't tied to racial
    discrimination and social injustice? After all, ethnic or
    religious groups in Canada also often live in segregated areas
    (``Canadian mosaic''). But other than in the U.S. this is hardly
    an issue. Therefore, Schelling's model -- for all its
    epistemological merits that are discussed here -- really seems to
    miss the point in terms of scientific relevance. Discrimination is
    the important point here, not segregation. But Schelling's model
    induces us to frame the question in a way that makes us miss the
    point. ({\em This comment has been added later as the result of
      some discussions I had on this point. E.A., March 25th 2016.})} -
    provided, of course, that the basic assumption about a tolerance
    threshold is true.

  Thus, for the purpose of policy advice a model does not need to
  capture actual causes. It can be counter-factual, but it must still
  be realistic in the sense that its basic assumptions can be
  empirically validated. Therefore, while the purpose of policy advice
  justifies certain counter-factual assumptions in a model, it cannot
  justify unrealistic and unvalidated models. This generally holds for
  models that are meant to describe possible instead of actual
  scenarios.

\end{enumerate}

Schelling did not validate his model empirically. But for classifying
the model as useful it is sufficient that it can be validated.  Now,
the interesting question is: Can the model be validated and is it
valid? Recent empirical research on the topic of neighborhood
segregation suggests that inhabitants react to anticipated future
changes in the frequency of differently colored neighbors rather than
the frequency itself \citep[124-125]{ellen:2000}. An important role is
played by the fear of whites that they might end up in an all-black
neighborhood. Thus, the basic assumption of the model that individuals
react upon the ratio of differently colored inhabitants in their
neighborhood is wrong and one can say that the model is in this sense
falsified.\footnote{There are two senses in which a model (or more
  precisely: a model-based explanation) can be falsified: a) if the
  model’s assumptions are empirically not valid as in this case and b)
  if the causes the model captures are (i) either blocked by factors
  not taken into account in the model or (ii) cannot be disentangled
  from other possible causes or (iii) turn out to be irrelevant in
  comparison with other, stronger or otherwise more important causes
  for the same phenomenon. The connection between the model’s
  assumptions and its output, being a logical one, can, of course, not
  be empirically falsified.}

The strong emphasis that is placed on empirical validation here stands
in contrast to some of the epistemological literature on simulations
and models. Robert Sugden, noticing that “authors typically say very
little about how their models relate to the real world”, treats models
like that of Schelling (which is one of his examples
\citep[6-8]{sugden:2000}) as “credible counterfactual worlds”
\citep[3]{sugden:2009} which are not intended to raise any particular
empirical claims. Even though the particular relation to the real
world is not clear, Sugden believes that such models can inform us
about the real world. His account suffers from the fact that he
remains unclear about how we can tell a counter-factual world that is
credible from one that is incredible, if there is no empirical
validation.

A possible candidate for stepping in this gap of Sugden’s account is
Kuorikoski’s and Lehtinen’s concept of “derivational robustness
analysis” \citep{kuorikoski-lehtinen:2009}. According to this concept
conclusions from unrealistic models to reality might be vindicated if
the model remains robust under variations of its unrealistic
assumptions. For example, in Schelling’s model the checkerboard
topography could be replaced by other different topographies
\citep[441]{aydinonat:2007}. If the model still yields the same
results about segregation, we are – if we follow the idea of
“derivational robustness analysis” – entitled to draw the inductive
conclusion that the model’s results would still be the same if the
unrealistic topographies were exchanged by the topography of some real
city, even though we have not tested it with a real topography.  A
problem with this account is that it requires an inductive leap of a
potentially dangerous kind: How can we be sure that the inductive
conclusion derived from varying unrealistic assumptions holds for the
conditions in reality which differ from any of these assumptions?

Some philosophers also dwell on the analogy between simulations and
experiments and consider simulations as “isolating devices” similar to
experiments \citep{maeki:2009}. But the analogy between simulations
and experiments is rather fragile, because other than experiments
simulations are not empirical and do not allow us to learn anything
about the world apart from what is implied in the premises of the
simulation. In particular, we can – without some kind of empirical
validation – never be sure whether the causal mechanism modeled in the
simulation represents a real cause isolated in the model or does not
exist in reality at all.

Summing it up, it is difficult, if not impossible, to claim that
models can inform us about reality without any kind of empirical
validation. Schelling’s model, however, appears to be a scientifically
useful model, at least in the sense that it can be validated (or
falsified for that matter). The most decisive features of the model in
this respect are its robustness and the practical feasibility of
identifying the modeled cause in empirical reality. Next we will see
how models fare when these features are not present.


\section{How models fail: The Reiterated Prisoner’s  Dilemma model}

Robert Axelrod’s computer simulations of the Reiterated Prisoner’s
Dilemma (RPD) \citep{axelrod:1984} are well known and still considered
by some as a role model for successful simulation research
\citep[408-409]{rendell-et-al:2010a}. What is not so widely known is
that the simulation research tradition initiated by Axelrod has
remained entirely unsuccessful in terms of generating explanations for
empirical instances of cooperation. What are the reasons for this lack
of explanatory success? And how come that Axelrod’s research design is
none the less considered as a role model today?

Axelrod had the ingenious idea to advertise a public computer
tournament where participation was open to everybody. Participants
were asked to hand in their guess at a best strategy in the reiterated
two person Prisoner’s Dilemma in the form of an algorithmic
description or computer program. This provided Axelrod with a rich,
though naturally very contingent set of diverse strategies and it had
the, surely welcome, side-effect of generating attention for Axelrod’s
research project. Axelrod ran a sequence of two tournaments. As is
well known the rather simplistic strategy {\em Tit For Tat} won both
tournaments.

In the Prisoner’s Dilemma Game the players can decide whether to
cooperate or not to cooperate. Mutual cooperation yields a higher
payoff than mutual non-cooperation, but it is best to cheat by letting
the other player cooperate while not cooperating oneself. And it is
worst to be cheated, i.e. to cooperate while the other player does
not. {\em Tit For Tat} cooperates in the first round of the Repeated
Prisoner’s Dilemma, but if the other player cheats, then {\em Tit For
  Tat} will punish the other player by not cooperating in the
following round.\footnote{For a detailed description RPD-model and the
  tournament see Axelrod (1984). An open-source implementation is
  available from: \url{www.eckhartarnold.de/apppages/coopsim}.}
Axelrod analyzed the course of the tournament in order to understand
just why {\em Tit For Tat} was such a successful strategy. He
concluded that it is a number of characteristics that determine the
success of a strategy in the Reiterated Prisoner’s Dilemma
\citep[chapter 6]{axelrod:1984}: Successful strategies are (1)
“friendly”, i.e. they start with cooperative moves, (2) envy-free, (3)
punishing, but also (4) forgiving. Axelrod furthermore believed that
repeated interaction is a necessary requirement for cooperation to
evolve and that, of course, {\em Tit For Tat} is generally quite a
good strategy in Reiterated Prisoner’s Dilemma situations.

Unfortunately for Axelrod, the Reiterated Prisoner’s Dilemma model is
anything but robust. For each of his conclusions, variations of the
RPD-model can be constructed where the conclusion becomes invalid
\citep[107]{arnold:2013}. It is even possible to construct a variant
that allows strategies to break off the repeated interaction at will
and that does not lead to the breakdown of cooperation
\citep{schuessler:1990}. The failure to derive any robust results
highlights the danger of drawing generalizing conclusions from models
and of relying on models as a tool of theoretical investigation. This
point has most strongly been emphasized by Ken Binmore, who describes
the popularity that Axelrod’s model enjoyed derogatorily as the “The
Tit-For-Tat Bubble” \citep[194]{binmore:1994}. Because the folk
theorem from game theory implies that there are infinitely many
equilibria in the Reiterated Prisoner’s Dilemma, there is not much
reason to assign of all things the {\em Tit For Tat}-equilibrium a
special place \citep[313-317]{binmore:1994}. If one follows Binmore’s
criticism then it is not the reiterated Prisoner’s Dilemma that
explains why {\em Tit For Tat} is such a good strategy, but rather the
fact that {\em Tit For Tat} is a very salient and easily understood
mode of behavior in many areas of life that explains why people so
easily believed in the superiority of the {\em Tit For Tat} strategy
in the RPD
game. %(See \citet[198]{binmore:1994} and \citet[317-319]{binmore:1998}.)

It is not only its lack of robustness that troubles Axelrod’s
model. It is also the difficulty of relating it to any concrete
empirical subject matter – a problem that Axelrod shares with many
game theoretical explanations.\footnote{This is very frankly admitted
  by the leading game theorist \citet{rubinstein:2013} in a newspaper
  article. Rubinstein resorts to an aesthetic vindication of game
  theory (“flowers in the garden of God”).}  Axelrod himself had
offered a very impressive example of empirical application by relating
the RPD model to the silent “Live and Let Live” agreement that emerged
between enemy soldiers on some of the quieter stretches of the western
front in the First World War. However, as critics were quick to point
out \citep{battermann-et-al:1998, schuessler:1990}, it is not at all
clear whether this situation really is a Prisoner’s Dilemma situation,
let alone how the numerical values of the payoff parameters could be
assessed. But precise numerical payoff values would be necessary since
Axelrod’s model is not robust against changes of the numerical values
of the payoff parameters within the boundaries that the Prisoner’s
Dilemma game allows \citep[80]{arnold:2008}. Also, Axelrod’s model
could not explain why “Live and Let Live” occurred only on some
stretches of the front line \citep[180]{arnold:2008}. Therefore,
Axelrod’s theory of the evolution of cooperation could not really add
anything substantial to the historical explanation of the “Live and
Let Live” by Tony \citet{ashworth:1980}.

The chapter from Axelrod’s book on the “Live and Let Live”-system
shows that he did not understand his model only as a normative model,
but at least also as an explanatory model. And the model was certainly
understood as potentially explanatory by the biologists who were
trying to apply it to cooperative behavior among animals (see
below). The distinction is important, because the validation
requirements for normative models are somewhat relaxed in comparison
to explanatory models. After all, we would not expect from a model
that is meant to generate advice for rationally adequate behavior to
correctly predict the behavior of unadvised and potentially irrational
agents. Still, even normative models must capture the essentials of
the empirical situations to which they are meant to be applied well
enough to generate credible advice. Here, too, robustness is an
important issue. For similar reasons as in the descriptive case it
would be dangerous to trust the advice given on the basis of a
non-robust model.

Thus, in contrast to Schelling’s model Axelrod’s model is neither
robust nor can the postulated driving factors of the emergent
phenomenon (stable cooperation) easily be identified empirically. In
Schelling’s case the driving factor was the assumed tolerance
threshold, in Axelrod’s case it is the payoff parameters of the
Prisoner’s Dilemma.  Therefore, two important prerequisites
(robustness and empirical identifiability) for the application of a
formal model to a social process appear to be absent in Axelrod’s
case.

The popularity of Axelrod’s computer tournaments had the consequence
that it became a role model for much of the subsequent simulation
research on the evolution of cooperation. It spawned myriads of
similar simulation studies on the evolution of cooperation
\citep{dugatkin:1997, hoffmann:2000}. Unfortunately, most of these
simulation studies remained unconnected to empirical research. Axelrod
had – most probably without intending it – initiated a self-sustaining
modeling tradition where modelers would orientate their next research
project on the models that they or others had published before without
paying much attention to what kind of models might be useful from an
empirical perspective. Instead it was more or less silently assumed
that because of the generality of the model investigations of the
reiterated Prisoner’s Dilemma model would surely be useful.

How little contact the modeling tradition initiated by Axelrod had to
empirical research becomes very obvious in a survey of empirical
research on the evolution of cooperation in biology by
\citet{dugatkin:1997}. In the beginning, Dugatkin lists several dozens
of game theoretical simulation models of the evolution of cooperation,
an approach to which Dugatkin himself is very favorable. However, none
of the models can be related to particular instances of cooperation in
animal wildlife. A seemingly insurmountable obstacle in this respect
is that payoff parameters usually cannot be measured. It is just very
difficult to measure precisely the increased reproductive success,
say, that apes that reciprocate grooming enjoy over apes that don’t.

The most serious attempt to apply Axelrod’s model was undertaken by
\citet{milinski:1987} in a study on predator inspection behavior in
shoal fishes like sticklebacks. When a predator approaches, it happens
that one or two sticklebacks leave the shoal and carefully swim closer
to the predator. The hypothesis was that if two sticklebacks approach
the predator they play a Reiterated Prisoner’s Dilemma and make the
decision to turn back based on a {\em Tit For Tat} strategy taking
into account whether the partner fish stays back or not. This was
tested experimentally by \citet{milinski:1987} as well as others
\citep[59-69]{dugatkin:1998}. While in his 1987-paper Milinski himself
believed that the hypothesis could be confirmed, it was after a long
controversy ultimately abandoned. In a joint paper on the same topic
that appeared ten years later \citet{milinski-parker:1997} do not draw
on the RPD model any more. In fact they treat it as an unresolved
question whether the observed behavior is cooperative at all.

In a later discussion, Dugatkin explained the problem when linking the
model research about cooperation to the empirical research in biology
by the difficulty of establishing a feedback-loop between model
research and empirical research \citep[57-58]{dugatkin:1998a}. The
empirical results were never fed back into the model building process
and the obstacles when trying to apply the models were never
considered by the modelers. Without a feedback-loop between
theoretical and empirical research, however, the model-building
process soon reaches a stalemate where models remain detached from
reality.

The frustration about this kind of pure model research is well
expressed in a polemical article by Peter
\citet{hammerstein:2003}. “Why is there such a discrepancy between
theory and facts?” asks \citet[83]{hammerstein:2003} and continues: “A
look at the best known examples of reciprocity shows that simple
models of repeated games do not properly reflect the natural
circumstances under which evolution takes place. Most repeated animal
interactions do not even correspond to repeated games.” In saying so,
Hammerstein is by no means opposed to employing game theory in
biology. It’s just that in the aftermath of Axelrod most simulation
studies on the evolution of cooperation focused on the Reiterated
Prisoner’s Dilemma or similar repeated games. This shows that the
demand for empirical validation has an important side effect besides
allowing to judge the truth and falsehood of the models themselves: It
forces the modelers to concern themselves seriously with the empirical
literature and the empirical phenomena that their models address. If
they do so, there is hope that this will lead quite naturally to the
choice of simulation models that address relevant questions of
empirical research. Or, as \citet[92]{hammerstein:2003} nicely puts
it: “Most certainly, if we invested the same amount of energy in the
resolution of all problems raised in this discourse, as we do in the
publishing of toy models with limited applicability, we would be
further along in our understanding of cooperation.”

Just how little model researchers care for the empirical content of
their research is inadvertently demonstrated by a research report on
the evolution of cooperation that appeared roughly 20 years after
the publication of Axelrod’s first paper about his computer tournament
\citep{hoffmann:2000}. There is only one brief passage where the
author of this research report talks about empirical applications of
the theory of the evolution of cooperation. And in this passage there
is but one piece of empirical literature that the author quotes, the
study on predator inspection in sticklebacks by \citet{milinski:1987}!
Nevertheless, Hoffmann believes that the “general framework is
applicable to a host of realistic scenarios both in the social and
natural worlds” \citep[4.3]{hoffmann:2000}. Much more believable is
Dugatkin’s summary of the situation: “Despite the fact that game
theory has a long standing tradition in the social sciences, and was
incorporated in behavioral ecology 20 years ago, controlled tests of
game theory models of cooperation are still relatively rare. It might
be argued that this is not the fault of the empiricists, but rather
due to the fact that much of the theory developed is unconnected to
natural systems and thus may be mathematically intriguing but
biologically meaningless” \citep[57]{dugatkin:1998a}. That this fact
could escape the attention of the modelers tells a lot about the
prevailing attitude of modelers towards empirical research.


\section{An ideology of modeling}

The examples discussed previously indicate that simulation models can
be a valuable tool to study some of the possible causes of some social
phenomena. However, the examples also content that a) modeling approaches
in the social sciences can easily fail to deliver resilient results,
that b) social simulations are not yet generally embedded in a
research culture where the critical assessment of the (empirical)
validity of the simulation models is a salient part of the research
process and that c) the significance of pure simulation results is
likely to be overrated.

Unsurprisingly, simulation models in the social sciences excel when
studying those causes that can be represented by a mathematical model
as in the case of Schelling’s neighborhood segregation model. Part of
the secret of Schelling’s success is surely that he had a good
intuition for picking those example cases where mathematical models
really work. But many of the causal connections that are of interest
in the social science cannot be described mathematically.  For
example, the question how the proliferation and easy accessibility of
adult content in the internet shapes the attitude of youngsters
towards love, sex and relationships, is hardly a question that could
be answered with mathematical models. Or, if we want to understand
what makes people follow orders to slaughter other people even in
contradiction to their acquired moral codes
\citep{Browning:1992}, then any reasonable answer to this
question will hardly have the form of a mathematical model.\footnote{A
  good discussion of the respective merits and limitations of
  different research paradigms in the social sciences can be found in
  \citet{moses-knutsen:2012}.}

Unfortunately, the field of social simulations has by now become so
much of a specialized field that modelers are hardly aware of the
strong limitations of their approach in comparison with conventional,
model-free methods in the social sciences. There is a widespread,
though not necessarily always outspoken belief that more or less
everything can -- somehow -- be cast into a simulation model. Part of
the reason for this belief may be the fact that with computers the
power of modeling techniques has indeed greatly increased. This belief
has found explicit expression in Joshua Epstein’s keynote address to
the Second World Congress of Social Simulation under the title “Why
model?” \citep{epstein:2008}.

In the following I am going to discuss Epstein’s arguments and point
out the misconceptions underlying this belief. In my opinion these
misconceptions are to no small degree responsible for the misguided
practices in the field of social simulations.  Epstein sets out by
arguing that it is never wrong to model, because – as he believes –
there exists only the choice between explicit and implicit models,
anyway:

\begin{quote}

  The first question that arises frequently -- sometimes innocently
  and sometimes not -- is simply, "Why model?"Imagining a rhetorical
  (non-innocent) inquisitor, my favorite retort is, "You are a
  modeler."Anyone who ventures a projection, or imagines how a social
  dynamic -- an epidemic, war, or migration -- would unfold is running
  some model. But typically, it is an implicit model in which the
  assumptions are hidden, their internal consistency is untested,
  their logical con- sequences are unknown, and their relation to data
  is unknown. But, when you close your eyes and imagine an epidemic
  spreading, or any other social dynamic, you are running some model
  or other. It is just an implicit model that you haven’t written down
  (see Epstein 2007).

  ...

  The choice, then, is not whether to build models; it’s whether to
  build explicit ones. In explicit models, assumptions are laid out in
  detail, so we can study exactly what they entail. On these
  assumptions, this sort of thing happens. When you alter the
  assumptions that is what happens. By writing explicit models, you let
  others replicate your results. \citep[1.2-1.5]{epstein:2008}
\end{quote}

It is not entirely clear whether Epstein restricts his arguments to
projections, but even in this case it is most likely false. It is
simply not possible to cast anything that can be described in natural
language into the form of a mathematical or computer model. But then
we also cannot assume that this must be possible, if projections to
the future are concerned. It is of course always commendable to make
one’s own assumptions explicit. But this does not require modeling.

In addition, there are certain dangers associated with mathematical
and computational modeling:

\begin{enumerate}
\item the danger of underrating or ignoring those causal connections
  that do not lend themselves to formal descriptions.

\item the danger of arbitrary ad hoc decisions when modeling causes of
  which we only have a vague empirical understanding.  The necessity
  to specify everything precisely easily leads to the sin of false
  precision, which consists in assuming detailed knowledge where in
  fact there is none.

\item the danger of conferring a deceptive impression of understanding
  even if the model is not validated.

\item the shaping and selection of scientific questions by the
  requirements of modeling, rather than by other, arguably more
  important, criteria of relevance as, for example, the social impact
  or relevance for public policy.

\end{enumerate}

That Epstein mentions replicability as another advantage of explicit
modeling is ironic given that it is still quite uncommon in published
simulation studies to give a reference for the reader to access and
replicate the model (as described further above). More worrisome,
however, is Epstein’s attitude towards validation:

\begin{quote}

  ... I am always amused when these same people challenge me with the
  question, "Can you validate your model?" The appropriate retort, of
  course, is, "Can you validate yours?"At least I can write mine down
  so that it can, in principle, be calibrated to data, if that is what
  you mean by "validate,"a term I assiduously avoid (good Popperian
  that I am). \citep[1.4]{epstein:2008}

\end{quote}

Calibration (i.e. fitting a model to data) is of course neither the
same nor a proper substitute for validation (testing a model against
data), as Epstein knows. Validation in the sense of empirical testing
of a model, hypothesis or theory is a common standard in almost all
sciences, including those sciences mentioned earlier that usually do
not rely on formal models like history, ethnology, sociology,
political science. It is obviously not the case that validation
presupposes explicit modeling, for otherwise history as an empirical
science would be impossible.

Epstein furthermore advances 16 reasons for building models other than
prediction \citep[1.9-1.17]{epstein:2008}. None of these reasons is
exclusively a reason for employing models, though. The functions, for
example, of guiding data collection or discovering new questions can
be fulfilled by models and also by any other kind of theoretical
reasoning.  Nor is it an exclusive virtue of the modeling approach
“that it enforces a scientific habit of mind”
\citep[1.6]{epstein:2008}. Here Epstein is merely articulating the
positivistic stock prejudice of the superiority, if only of a didactic
kind, of formal methods. Given what \citet{heath-et-al:2009} have
found out about the lack of proper validation of many agent-based
simulations one might even be inclined to believe the opposite about
the simulation method’s aptitude to encourage a scientific habit of
mind.

It fits into the picture of a somewhat dogmatic belief in the power of
modeling approaches that modelers consider the lack of acceptance of
their method often as more of a psychological problem on the side of
the recipients to be addressed by better propaganda \citep[2.11-2.12,
3.22-3.26]{barth-et-al:2012}, rather than a consequence of the still
immature methodological basis of many agent-based simulation
studies. This attitude runs the risk of self-deception, because one of
the major reasons why non-modelers tend to be skeptical of agent-based
simulations is that they perceive such simulations as highly
speculative. As we have seen, the skeptics have good reason to do so.

\section{Conclusions}

It is in my opinion not least because of the abundance of simulations
with low empirical impact that “social simulation is not yet
recognized in the social science mainstream”
\citep[abstract]{squazzoni-casnici:2013}. Why should a mainstream
social scientist take simulation studies seriously, if he or she
cannot be sure about the reliability of the results, because the
simulations have never been validated? If modelers started to take the
requirement of empirical validation more seriously, I expect two
changes to occur – both of them beneficial: 1) Social simulations will
become more focused in scope. Scientists will not attempt to cast
anything into the form of a computer simulation from classical social
contract philosophy \citep{skyrms:1996, skyrms:2004} to, well, the
whole world \citep{futureict:2013, livingearth:2013}, but they will
develop a better feeling for when simulations can be empirically
validated and when not, and they will mostly leave out those problems
where computer simulations cannot be applied with some hope of
producing empirically applicable results. 2) Yet, while the simulation
method will become more focused in scope, it will at the same time
become much more useful in practice, because simulations will more
frequently yield results that other scientists can rely on without
needing to worry about their speculative character and potential lack
of reliability.


\singlespacing
%\bibliographystyle{plainnat}
\bibliographystyle{apsr}
\bibliography{bibliography}

\end{document}
