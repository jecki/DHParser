\documentclass[12pt,a4paper,ngerman]{article}
\usepackage{ae}
\usepackage[german, ngerman]{babel}
\usepackage[utf8x]{inputenc}
\usepackage{ucs}
\usepackage[T1]{fontenc}
\usepackage{t1enc}
\usepackage{type1cm}
\usepackage{graphicx}

\usepackage{ifpdf}
\ifpdf
\usepackage{xmpincl}
\usepackage[pdftex]{hyperref}
\hypersetup{
    colorlinks,
    citecolor=black,
    filecolor=black,
    linkcolor=black,
    urlcolor=black,
    bookmarksopen=true,     % Gliederung öffnen im AR
    bookmarksnumbered=true, % Kapitel-Nummerierung im Inhaltsverzeichniss anzeigen
    bookmarksopenlevel=1,   % Tiefe der geöffneten Gliederung für den AR
    pdfstartview=FitV,       % Fit, FitH=breite, FitV=hoehe, FitBH
    pdfpagemode=UseOutlines, % FullScreen, UseNone, UseOutlines, UseThumbs 
}
\includexmp{Evolutionstheorien}
\pdfinfo{
  /Author (Eckhart Arnold)
  /Title (Der Einsatz evolutionärer Computermodelle bei der Untersuchung historischer und politischer Fragestellungen)
  /Subject (Evolutionäre Computermodelle in den Sozialwissenschaften)
  /Keywords (Evolution der Kooperation, Robert Axelrod, Soziale Evolution, Soziale Simulationen)
}
\fi

%\renewcommand{\baselinestretch}{1.5}

\newcommand{\high}[1]{\raisebox{0.7ex}{#1}}
\newcommand{\low}[1]{\raisebox{-0.7ex}{#1}}

\sloppy

\begin{document}

\begin{titlepage}

\title{Der Einsatz evolutionärer Computermodelle bei der Untersuchung
  historischer und politischer Fragestellungen}
\author{Eckhart Arnold}
\date{1.Dezember 2002}

\setlength{\parindent}{0em}
%\begin{center} {\tt\large ENTWURF} \end{center}
%\setlength{\parskip}{5cm}

\begin{center} {\large Der Einsatz evolutionärer Computermodelle bei der Untersuchung
  historischer und politischer Fragestellungen} \end{center}

\setlength{\parskip}{7cm}

von:\\[0.5cm]Eckhart Arnold\\
%Treppenstr. 6\\
%99089 Erfurt\\
%Tel.: 0361 / 65 36 741\\
%E-Mail: eckhart\b{ }arnold@yahoo.de

\setlength{\parskip}{1cm}

Erfurt, 1. Dezember 2002

\end{titlepage}

\pagenumbering{roman} \tableofcontents

\newpage

\pagenumbering{arabic}

\setcounter{page}{1}


\section{Einleitung}

Wie viele andere bedeutende wissenschaftliche Entdeckungen auch, hat
die Evolutionstheorie seit ihrer Erfindung Wirkungen entfaltet, die
weit über den Kreis ihrer ursprünglichen Anwendung hinaus reichen. Daher
verwundert es nicht, dass schon bald nach Darwins Entdeckung die
wichtigsten Stichworte seiner Theorie (bzw. der durch Herbert Spencer
und andere popularisierten Form von Darwins Theorie (Vgl. Koch 1973, S.38ff.)) in den
Gesellschaftswissenschaften auftauchten - nicht selten in Form kruder
Analogien und Übertragungen von halbverstandenen biologischen
Erkenntnissen. So bildete bis etwa zur Mitte des
20. Jahrhunderts der "`Sozialdarwinismus"', d.i. die Ansicht,
dass auch das Zusammenleben der Menschen in der Gesellschaft sowie der
Wettbewerb zwischen den Gesellschaften durch einen "`Kampf ums Dasein"'
und das "`Überleben des Stärksten"' bestimmt sei, die wohl dominanteste
Form der Übertragung evolutionstheoretischer Vorstellungen auf den
gesellschaftlichen Bereich. Nicht zuletzt wegen seiner verheerenden
normativen Konsequenzen gilt der Sozialdarwinismus heutzutage zu Recht
als diskreditiert. 

Sieht man von solchen Missverständnissen jedoch einmal ab, 
so erscheint die grundlegende Frage, ob nicht die Entwicklung
kultureller und sozialer Systeme oder auch politischer
Ordnungen durch evolutionäre Mechanismen erklärt
werden kann, nach wie vor interessant. Es ist 
durchaus denkbar, dass auch im Wettbewerb der Gesellschaftsformen oder
der Produktionstechniken die Mechanismen der Mutation und Selektion
wirksam sind, die - ähnlich wie in der natürlichen Evolution - unter
gegebenen Umweltbedingungen einen evolutionären Entwicklungsprozess
herbeiführen.

In diesem Aufsatz soll deshalb einmal der Frage nachgegangen werden,
wie evolutionäre Erklärungen historischer, gesellschaftlicher und
politischer Vorgänge beschaffen sein könnten, und bei welchen
Vorgängen dieser Art ein evolutionärer Erklärungsansatz
erfolgversprechend erscheint. Um ein Ergebnis dieses Aufsatzes vorweg
zu nehmen: Es gibt eine Reihe von Phänomenen aus dem Bereich der
historischen und politischen Wissenschaften, bei denen ein
Erklärungsansatz mit Hilfe evolutionärer Modelle sich als fruchtbar
erweisen könnte. Dazu gehören z.B. historische Vorgänge, die sich über
große Zeiträume hinweg abspielen, wie etwa die Entstehung sozialer und
politischer Normen (sowohl solcher, die ethischer Natur sind, als auch
solcher, die man eher als bloße Klugheitsregeln einstufen würde). Da
die Evolutionstheorie eine kausale Erklärung teleologischer Strukturen
darstellt, könnte ein evolutionärer Ansatz besonders dann hiflreich
sein, wenn es darum geht die Entstehung leistungsfähiger sozialer
Institutionen und wirklichkeitstauglicher Normensysteme zu erklären,
die in der Form, in der sie sich schließlich durchsetzen, nicht zuvor
geplant gewesen sein konnten. Was die Untersuchung von Normen betrifft
kann dabei interessanterweise nicht nur die empirische Frage, warum
diese oder jene Norm sich durchgesetzt hat, sondern auch die ethische
Frage, ob eine bestimmte Norm als moralisch gültig anerkannt werden
sollte, unter einem evolutionären Gesichtspunkt betrachtet werden.

Im folgenden werde ich zunächst in abstracto beschreiben, wie
evolutionäre Erklärungen kultureller Phänomene konstruiert werden
können, und welche Bedingungen erfüllt sein müssen, damit eine solche
evolutionäre Erklärung als vollständig gelten kann. Dazu werde ich
zunächst eine allgemeine Evolutionstheorie vorstellen, die den Rahmen
für Erklärungen historischer Vorgänge abgeben soll. Innerhalb dieses
theoretischen Rahmens soll dann besonderes Augenmerk auf den Einsatz von
Computersimulationen gerichtet werden. Hierfür werde ich in Anlehnung
an Robert Axelrods bahnbrechende Untersuchungen (Axelrod 1984) eine
Computersimulation vorstellen, die es erlaubt, evolutionäre - in
diesem Fall speziell populationsdynamische Vorgänge - am Modell zu
studieren. 

Schließlich soll an einigen Beispielen wenigstens andeutungsweise
gezeigt werden, wie evolutionäre Erklärungen in der Philosophie und
den Gesellschaftswissenschaften eingesetzt werden können.


\section{Evolutionäre Erklärungen}

\subsection{Eine allgemeine (axiomatische) Evolutionstheorie als Grundlage}

Bevor der Versuch unternommen wird, gesellschaftliche Vorgänge
evolutionär zu erklären, ist es notwendig, sich einen möglichst klaren
Begriff davon zu bilden, was ein evolutionärer Prozess ist. Nur so
wird es später möglich sein zu unterscheiden, ob ein evolutionärer
Erklärungsversuch tatsächlich eine Erklärung liefert, oder ob es sich
nur um die erzählerische Beschreibung eines historischen Vorgangs in
einem evolutionswissenschaftlichen Jargon handelt. Gefordert ist also
zunächst eine allgemeine Theorie der Evolution.

% in einer Fussnote auf die Theoriediskussion in der
% Geschichtswissenschaft hinweisen (Popper, Hempel, etc.)

Für eine solche Theorie wird hier zunächst auf die bei Schurz (Schurz 2001)
beschriebene axiomatische Evolutionstheorie
zurückgegriffen. Anschließend soll auf die Frage eingegangen werden, ob
eine Evolutionstheorie in dieser Form als Erklärungsgrundlage bereits
gehaltvoll genug ist, und wie sie für die Erklärung kultureller
Phänomene fruchtbar gemacht werden kann.

Die axiomatische Evolutionstheorie besagt folgendes: Evolution, d.h. die
Entwicklung von "`Organismen"' zu immer angepassteren ("`fitteren"') Formen,
findet mit einer bestimmten Wahrscheinlichkeit statt, wenn vier grundlegende
Bedingungen erfüllt sind, die Schurz mit folgenden Worten beschreibt.:

\begin{itemize}
  
\item "`{\em (V)ariation}: Das Erbmaterial eines Organismus variiert durch
  verschiedene Prozesse und diese Variationen haben Einfluss auf die Fitness
  (=Reproduktionsrate) des Organismus bzw. der Spezies.
  
\item {\em (R)eproduktion}: [Das Erbmaterial] eines Organismus wird von
  Generation zu Generation reproduziert.

\item {\em (S)elektion}: Die Umgebungsbedingungen üben auf den
Organismus bzw. die Spezies einen Selektionsdruck aus, der die
Reproduktionsrate begrenzt. Dieser Selektionsdruck führt zu
unterschiedlichen Reproduktionsraten unterschiedlicher Spezien
bzw. unterschiedlicher Varianten derselben Spezies, mit der Folge,
dass sich die Varianten bzw. Spezien mit der höheren Reproduktionsrate
langfristig durchsetzen.

\item {\em UST}: Die zeitliche Änderungsrate der selektiv wirksamen
Umgebungsparameter ist entweder gering, oder aber
regulär-periodisch."'

\end{itemize} 

(Schurz 2000, 329f., 335; vgl. auch Wieser 1994b, 16)\vspace{12pt}

Natürlich sind es im Bereich der Kulturwissenschaften nicht
Organismen, die sich entwickeln, sondern soziale Normen und
Institutionen. Aber die Gesetzmäßigkeiten sind genau dieselben.
Reichen aber diese vier Bedingungen bereits hin, um ein bestimmtes
Phänomen evolutionär zu erklären?

Man könnte einwenden, dass ein wesentlicher Aspekt bei
dieser axiomatischen Evolutionstheorie eher verschleiert wird, nämlich
die Tatsache, dass die Existenzmöglichkeit (und damit auch die
Möglichkeit der Evolution) bestimmter Organismen auch von
ontologischen Voraussetzungen abhängig ist. Dass sich
z.B. ein Vogelflügel in der Evolution hat entwickeln können, war nur
möglich, weil das System Luft-Vogelflügel aufgrund der Gesetze der
Aerodynamik tatsächlich funktionert. Nun sind derartige ontologische
Voraussetzungen zwar implizit in der {\em Bedingung S} enthalten. Aber
es ist nicht unwichtig, sich vor Augen zu halten, dass in die
{\em Bedingung S} zum Teil sehr anspruchsvolle ontologische
Voraussetzungen eingehen können. Dies bedeutet, dass das Erfülltsein von S
im Einzelfall sehr schwer nachzuweisen sein dürfte, da dies die Klärung
der ontologischen Möglichkeitsbedingungen erfordert. 

\subsection{Die Anwendung der allgemeinen Evolutionstheorie auf kulturwissenschaftliche und historische Fragestellungen}

Bei der Erklärung historischer Vorgänge fällt die eben erwähnte Schwierigkeit zum
Glück weniger gravierend aus, da es sich hierbei um ex-post
Erklärungen handelt, womit die Existenzmöglichkeit des zu Erklärenden
zwangsläufig gegeben ist. Dennoch besteht ein wichtiger Unterschied
darin, ob erklärt werden soll, warum eine soziale Insitution
funktioniert, oder ob erklärt werden soll, wie sie entstehen und sich
durchsetzen konnte. Das Funktionieren einer sozialen Insitution kann
(ohne Zirkelschluss) nicht allein durch ihre evolutionäre Entstehung
erklärt werden. Umgekehrt ist mit dem Nachweis der
Funktionstüchtigkeit oder der Zweckmäßigkeit bestimmter sozialer
Insitutionen noch nicht erklärt, wie und warum sie entstehen
konnten. Denkbar ist aber, dass das die Entstehung generell
leistungsfähiger sozialer Institutionen dadurch erklärt werden kann,
dass die Bedingungen für evolutionäre Entwicklungsprozesse im Sine der
axiomatischen Evolutionstheorie gegeben sind, da derartige Prozesse
dort, wo dies möglich ist, vergleichsweise rasch zur Herausbildung
"`quasi-teleologischer"' Formen führen. Eine solche Erklärung wird
weiter unten am Beispiel der von Eric L. Jones (Jones 1981) gebenen
Begründung für den historischen "`Erfolg"' des europäischen Kontinents
erörtert werden.

Zur Beantwortung der Frage, ob eine axiomatische Evolutionstheorie wie
die oben skizzierte bereits hinreichend gehaltvoll ist, um die
Herausbildung leistungsfähiger sozialer Institutionen wenigstens im
Prinzip zu erklären, müsste ausserdem noch genauer untersucht werden,
ob evolutionäre Prozesse, die diesen vier Axiomen genügen, bereits mit
hinreichender Wahrscheinlichkeit zur Herausbildung
"`quasi-teleologischer"' Formen führen. Man könnte dazu den durch
diese Axiome beschriebenen evolutionären Prozess als einen
Optimierungsalgorithmus auffassen und Überlegungen
zu dessen Effizienz anstellen (vgl. Axelrod 1997, 10ff., vgl. Wagner 1994). 
Darüber hinaus könnte man die Frage aufwerfen, ob
die oben beschriebene Evolutionstheorie nicht speziell für den Bereich
der Kulturwissenschaften noch weiter verfeinert könnte.

Für den Bereich der biologischen Evolution ist dies möglich, indem man
die Merkmale genetischer Vererbung und Entwicklung durch weitere
Axiome näher eingrenzt (vgl. Schurz 1998, 329ff.). Für die
Beschreibung kultureller Evolutionsprozesse wäre ein ähnliches
Vorgehen denkbar, etwa indem man versucht, die Reproduktions- und
Selektionsbedinungen von Kulturtechniken (Normen, Institutionen,
Technologien) durch Gesetzmäßigkeiten genauer zu erfassen. Ein erster
Ansatz müsste in der Untersuchung der Frage bestehen, wie sich
Kulturtechniken durchsetzen. Grundsätzlich kann dies auf zweierlei
Weise geschehen: 1. indem die Gesellschaften, die sie angenommen
haben, andere Gesellschaften, die nicht darüber verfügen, verdrängen
(durch Krieg, Zerstörung ihrer Lebensgrundlagen etc.)  2. indem
Gesellschaften, die noch nicht über eine bestimmte Kulturtechnik
verfügen, sie von einer anderen Gesellschaft, die sie bereits besitzt,
erlernen (was wiederum auf zweierlei Weise geschehen kann: dadurch
dass die Kulturtechnik selbst oder dadurch dass Idee ihrer Möglichkeit
weitergegeben wird (vgl. Diamond 1998, S.215ff.)). Wenn es nun
gelänge, über die Geschwindigkeit der Ausbreitung von Kulturtechniken
auf diesen Wegen genauere Aussagen zu treffen, so erscheint es
zumindest im Prinzip denkbar - analog etwa zu den
populationsdynamischen Gesetzen in der biologischen Evolution -
Gesetzmäßigkeiten für die Verbreitung von Kulturtechniken
aufzustellen, wenn auch in den Kulturwissenschaften nicht dieselbe
Genauigkeit der Vorhersagen erwartet werden darf.

Um einen entsprechenden historischen Entwicklungsvorgang nun mit Hilfe der allgemeinen
Evolutionstheorie erklären zu können, müssen im wesentlichen drei
Bedingungen erfüllt sein:

\begin{enumerate}

\item Die Voraussetzungen für einen evolutionären Prozess müssen gegeben
  sein, d.h. es muss gezeigt werden, dass für den beschriebenen
  Vorgang mindestens die vier Grundaxiome der allgemeinen
  Evolutionstheorie (V, R, S und UST) erfüllt sind und gegebenenfalls
  noch weitere Bedingungen, die speziell für gesellschaftliche und
  kulturelle Entwicklungsprozesse gelten.

\item Das Ergebnis des historischen Vorgangs muss nach Maßgabe dieser Axiome,
  insbesondere der Selektionsbedingungen, auch wahrscheinlich gewesen
  sein. Sonst wäre die Hypothese, dass bei dem untersuchten Vorgang ein
  evolutionärer Prozess im Sinne der axiomatischen Evolutionstheorie
  stattgefunden hat, bereits widerlegt.

\item Der evolutionäre Prozess muss nachgewiesen werden, indem die
  Zwischenstadien dieses Prozesses identifiziert werden. Falls die
  historischen Quellen lückenhaft sind, muss zumindest die Möglichkeit solcher
  Zwischenstadien dargelegt werden können.

\end{enumerate}

Nur wenn alle drei Bedinungen erfüllt sind, kann die evolutionäre
Erklärung eines historischen Vorganges als vollständig gelten. Wird
eine dieser Bedingungen nicht erfüllt, so ist die evolutionäre
Erklärung entweder falsch (wenn Bedingung 1 oder 2 verletzt wird) oder
sie bleibt eine bloße Behauptung (sofern Bedingung 3 nicht erfüllt
ist).

%Aber
%auch wenn dies der Fall ist, könnte eine erfolgreiche evolutionäre
%Erklärung eines historischen Vorganges immer noch "`gestürzt"' werden,
%wenn es für denselben historischen Vorgang eine andere Erklärung gibt,
%die im ganzen plausibler erscheint. Denn ebenso, wie eine bestehende
%Theorie meist nicht schon mit ihrer Falsifikation fallen gelassen
%wird, sondern erst wenn durch eine andere Theorie verdrängt wird, kann
%eine wissenschaftlichte Theorie umgekehrt verdrängt werden, ohne
%widerlegt worden zu sein, wenn eine überzeugendere Alternativtheorie
%aufgestellt wird. Dies mag vom logischen Standpunkt aus nicht ganz
%schlüssig erscheinen, dennoch handelt es dabei um ein in den
%Gesellschaftswissenschaften häufig auftretendes
%erkenntnistheoretisches Phänomen.\footnote{Dies hängt mit der
%narrativen Form wissenschaftlicher Arbeiten in den
%Gesellschaftswissenschaften zusammen, die es erlaubt eine inhaltliche
%Geschlossenheit auch dann noch zu suggerieren, wenn die Wirklichkeit
%nur sehr lückenhaft erkannt ist. Man denk etwa daran, wie
%vergleichsweise dürftig sich rein diplomatiegeschichtliche Erklärungen
%des ersten Weltkrieges im Lichte eines umfassenderen
%gesellschaftsgeschichtlichen Ansatzes ausnehmen (vgl. Wehler
%1995, ???). Ein besonders frappierendes Beispiel für dieses Phänomen zeigt
%Allison in seiner Abhandlung über die Kuba-Krise auf (Allison / Zelikow
%1999). Allison spielt darin drei unterschiedlich feinkörnige
%Erklärungsmodelle durch, die jedes für sich genommen schlüssig und
%vollständig erscheinen, aber zugleich durch das jeweils nachfolgende
%Modell praktisch vollkommen entwertet werden.}

\section{Computermodelle zur Simulation evolutionärer Vorgänge}

Soeben wurde umrisshaft gezeigt, wie eine allgemeine Evolutionstheorie durch Axiome
beschrieben werden kann, und wie diese Theorie auf historische
Vorgänge angewendet werden könnte. Doch gerade die Anwendung dieser
Theorie auf bestimmte wissenschaftliche Probleme wirft noch weitere
Fragen auf.  Denn selbst wenn in einer gegebenen Situation alle
wirksamen Selektionsbedingungen fesgestellt worden sind, bleibt noch
offen, welche Organismen (bzw. welche sozialen Normen, Insitutionen,
Kulturtechniken) nach Maßgabe dieser Selektionsbedingungen die
"`Fitteren"' sind, und nach welchem Muster und wie rasch
dementsprechend der evolutionäre Prozess ablaufen wird. In einigen
Fällen mag die Antwort zwar offensichtlich sein, da manche 
evolutionäre Entwicklungen nur unter ganz bestimmten Umweltbedingungen
möglich sind. So konnte sich beispielsweise die Viehzucht nur in
solchen geographischen Regionen entwickeln, wo Tiere leben, die
domestizierbar sind (Diamond 1998, S.85ff.). In anderen Fällen liegt die
Antwort jedoch nicht unmittelbar auf der Hand. Es empfiehlt sich daher
evolutionäre Vorgänge zunächst am Modell zu studieren, um so 
die Bedingungen evolutionärer Vorgänge and typische Entwicklungsmuster
festzustellen, nach denen in der Wirklichkeit Ausschau gehalten
werden muss.

Während die Untersuchung evolutionärer Prozesse in der Biologie
naturgemäß sehr weit fortgeschritten ist, trifft man in den
Gesellschaftwissenschaften (von Ausnahmen abgesehen) allenfalls in der
Ökonomie auf entsprechend ausgearbeitete Modelle. Letztere werden
zumeist auf spieltheoretischer Grundlage und auf mathematisch zum Teil
sehr anspruchsvollen Niveau gebildet (z.B. Binmore/Samuelson 1992). Zu
der mathematematischen Modellbildung tritt als eine vergleichsweise
junge Technik der Einsatz von Computersimulationen
hinzu. Computersimulationen sind dabei als eine Ergänzung und
Erweiterung der mathematischen Modellbildung zu verstehen. Im
Gegensatz zu diesen zeichnen sie sich allerdings häufig durch eine
größere Einfachheit und Anschaulichkeit aus. Zudem gibt es bstimmte
Probleme, die sich mathematisch nur sehr schwer in den Griff bekommen
lassen, die aber mühelos auf einem Computer programmiert werden
können.

Eine Pionierleistung auf diesem Gebiet stellt Robert Axelrods
"`Evolution der Kooperation"' (Axelrod 1984) dar. Axelrod untersuchte in
diesem Werk mit Hilfe von Computerturnieren, in denen er eine Anzahl
von Spielern mit unterschiedlichen Strategien parweise gegeneinander
antreten ließ, welche Strategien im wiederholten
Gefangenendilemmaspiel besonders erfolgreich abschneiden, und ob eher
kooperative Strategien oder eher destruktive Strategien eine Chance
haben, sich auf lange Sicht evolutionär durchzusetzen.
 
Um das Prinzip derartiger Computersimulationen besser zu verdeutlichen,
werde ich im folgenden eine einfache Simulation des iterierten
Gefangenendilemmas nach dem Vorbild des Computerturniers in Robert
Axelrods "`Evolution der Kooperation"' (Axelrod 1984)
vorstellen. Dabei werde ich auch einige Varianten von Axelrods
Simulation durchspielen, bei denen es um den Einfluss von
Missverständnissen bzw. Fehlleistungen und um die evolutionäre
Stabilität erfolgreicher Strategien unter dem Druck degenerativer
Mutationen geht.

Es zeigt sich dabei übrigens, dass Axelrods Ergebnisse sehr stark von
der Wahl bestimmter Vorgaben wie der Strategiemenge und der
Auszahlungsparameter abhängig sind, und dass bei geänderten
Bedingungen viele von Axelrods Behauptungen nicht mehr ohne weiteres
aufrecht erhalten werden können. Dies bedeutet, dass man die
Genauigkeit der Ergebnisse Axelrods und ähnlicher Computersimulationen
für die Beurteilung von Vorgängen aus dem gesellschaftlichen oder
politischen Bereich nicht überschätzen darf, da in diesen Bereichen
die ermittelten Parameter oft nur ungenaue Schätzwerte wiedergeben, so
dass die Instabilität des Modells gegenüber der Veränderung der
Paramter sich auf die Schlussfolgerungen vom Modell auf die
Wirklichkeit überträgt.

Beabsichtigt ist aber weniger die Widerlegung oder Relativierung von
Axelrods Ergebnissen - die Schwächen seiner Theorie sind in der
einschlägigen Literatur eingehend diskutiert worden (vgl. Binmore 1994,
194ff., vgl. Schüßler 1990, 33ff.) - als vielmehr die Darstellung der Einsatzmöglichkeiten von
Computersimulationen an einem hinreichend einfachen Beispiel.


\subsection{Ein Beispiel: Die Simulation des iterierten Gefangenendilemmas}

Bei dem von Axelrod durchgeführten Computerturnier handelt es sich um
eine Simulation des iterierten Gefangenendilemmas. Ein einfaches (nicht iteriertes)
Gefangenendilemma kann als ein Spiel beschrieben werden, in dem zwei
Spieler die Wahl haben zu kooperieren oder nicht zu kooperieren (also
zu "`defektieren"'). Der Gewinn, den jeder Spieler erhält, hängt vom
Verhalten beider Spieler ab. Ein Spieler, der kooperiert, während der
andere defektiert, erhält überhaupt nichts, während sein Gegenspieler
den höchst möglichen Gewinn einstreicht. Defektieren beide Spieler, so
bekommen sie zwar einen Gewinn, doch ihr Gewinn fällt nur sehr gering
aus. Kooperieren beide Spieler, so bekommen sie einen recht
ansehnlichen Gewinn, der aber nicht dem Höchstgewinn
entspricht. Eigentlich wäre es für beide Spieler am besten zu
kooperieren, wenn sie nur sicher gehen könnten, dass der andere sich
ebenso verhält. Gerade dies ist im einfachen Gefangenendilemma aber
nicht der Fall.

Anders verhält es sich beim wiederholten ("`iterierten"')
Gefangenendilemma. Hier spielen die beiden Spieler eine Folge von
Gefangenendilemmasituationen durch. Zwar wissen sie wiederum nicht, ob
der andere Spieler als nächstes kooperieren oder defektieren
wird. Aber sie verfügen über die gesamte Folge der vergangenen Züge
als Anhaltspunkt und zudem können sie in der nächsten Runde auf das
Verhalten des Gegners in der jetzigen Runde reagieren.

Dass das iterierte Gefangenendilemma (ebenso wie das einfache
Gefangenendilemma) ein plausibles Modell vieler typischer sozialer
Situationen ist, bedarf kaum einer weiteren Erörterung. Ein gutes
Beispiel sind etwa die Verhandlungen in Parlamentsausschüssen, bei
denen dieselben Verhandlungspartner immer wieder aufeinander treffen
und jeder bestimmte Ziele verfolgt, bei denen er von der Zustimmung
der anderen abhängig ist.\footnote{Dies gilt natürlich vor allem in
Politischen Systemen mit schwach ausgeprägter Fraktionsdisziplin wie
den USA.}

Eine wichtige Eigenschaft des iterierten Gefangenendilemmas besteht
darin, dass es keine eindeutig beste Strategie gibt.  Wie erfolgreich
eine bestimmte Strategie ist, hängt immer auch von der Strategie des
Gegenspielers ab. Welche Strategie ist dann aber insgesammt, d.h. wenn
man sie gegen eine Vielzahl sehr unterschiedlicher Gegenspieler
antreten lässt, am erfolgversprechendsten? Um diese Frage zu klären,
führte Axelrod sein Computerturnier durch. Eine Implementierung dieses
Computerturnieres soll im folgenden beschrieben werden. Sie
unterscheidet sich von Axelrods Computerturnier durch die
vergleichsweise geringe Anzahl teilnehmender Strategien. Doch geht es
hier im wesentlichen um die Verdeutlichung des Prinzips. Abgesehen
davon stellt sich das - weiter unten noch anzusprechende - Problem der
kontingenten Strategiemenge genauso für die größere Anzahl von
Strategien in Axelrods Computerturnier.


\subsubsection{Die Implementation des Computerturniers}

Insgesamt nahmen 12 unterschiedliche Strategien an dem hier
beschriebenen Computerturnier teil. Es lohnt sich nicht, alle
Strategien einzeln zu beschreiben, zumal ihre Namen meist
selbsterklärend sind ({\em Random, Tit for Tat, Always friendly}). Die
drei Strategien {\em GraciousTFT, Tester} und {\em Analyst} sollen
jedoch mit Hilfe von Quellcodebeispielen etwas näher erläutert
werden. Die Beispiele aus dem Programmcode sind in der
Programmiersprache Python geschrieben, einer Interpretersprache, die
sich wegen ihrer Einfachheit und der guten Lesbarkeit des
Programmcodes für solche Aufgaben empfiehlt.

Die Strategie {\em GraciousTFT} ist eine Variante von {\em Tit for
  Tat}, die jedoch längere Folgen gegenseitiger Bestrafungen erkennt
und durch ein Friedensangebot zu beendigen versucht. Damit beseitigt
{\em GraciousTFT} eine Schwäche, die {\em Tit for Tat} im
Zusammentreffen mit böswilligen Varianten des eigenen Typs
aufweist. Im folgenden Programmausschnitt steht ein Rückgabewert von 1
für Kooperation und ein Rückgabewert von 0 für eine Defektion.

\begin{scriptsize}
\begin{samepage}
\begin{center}GraciousTFT:\end{center}
\begin{verbatim}
    def nextMove(self, round, myMoves, opMoves):
        if round == 1:
            return 1                                      # start friendly
        elif round > 6 and ((opMoves[-5:] == [0,0,0,0,0] and \
                             myMoves[-5:] == [0,0,0,0,0]) or \
                            (opMoves[-5:] == [0,1,0,1,0] and \
                             myMoves[-5:] == [1,0,1,0,1])):
            return 1                                      # peace offer
        else:
            if opMoves[-1] == 1: return 1                 # play tit for tat
            else:                return 0
\end{verbatim}
\end{samepage}
\end{scriptsize}

Die Strategie {\em Tester} ist dem Buch von Axelrod entnommen. Sie versucht
zunächst durch eine mutwillige Defektion festzustellen, ob sich der Gegner
ausnutzen lässt. Wenn ja, dann defektiert sie bei jedem zweiten Zug. Wenn nicht,
spielt sie Tit for Tat.

\begin{scriptsize}
\begin{center}Tester:\end{center}
\begin{verbatim}
    def nextMove(self, round, myMoves, opMoves):
        if round <= 2:
            return 0                                      
        elif round == 3:
            if opMoves[-1] == 0:  self.state = "TFT"
            else:                 self.state = "Deceiver"
            return 1
        elif round == 4:
            return 1
        else:
            if self.state == "TFT":
                if opMoves[-1] == 1:  return 1
                else:                 return 0
            else:
                if round % 2 == 1:    return 0
                else:                 return 1

\end{verbatim}
\end{scriptsize}

{\em Analyst} schließlich analysiert jeweils die zehn letzten Spielzüge, um
festzustellen, ob die gegnerische Strategie ausnutzbar oder böswillig
ist. Erweist sie sich als ausnutzbar, so defektiert {\em Analyst}. Ist der
Gegner böswillig, dann wehrt sich {\em Analyst} ebenfalls durch
Defektion. Wenn beides nicht eindeutig feststellbar ist, dann spielt Analyst
Tit for Tat. Die ersten zehn Züge werden zufällig gewählt.

\begin{scriptsize}
\begin{center}Analyst:\end{center}
\begin{verbatim}
    def nextMove(self, round, myMoves, opMoves):
        if round <= 10:
            return whrandom.randint(0, 1)   # play random at the beginning
        else:

            # analyse
            
            ex_attempt, ex_success = 0,0
            opex_opportunity, opex_attempt = 0, 0
            
            i = -9
            while i <= -1:
                if myMoves[i-1] == 0:
                    ex_attempt += 1
                    if opMoves[i] != 0: ex_success += 1 # opponent did
                                                        # not punish exploit!
                else:
                    opex_opportunity += 1
                    if opMoves[i] == 0: opex_attempt += 1  # opponnent played 
                                                           # defective without
                                                           # reason
                i += 1

            # and react accordingly

            ret = -1
            if (ex_attempt > 0):
                if (float(ex_success) / float(ex_attempt)) >= 0.6:
                    return 0                    # keep exploiting
                else: ret = 1                   # try to be friendly again

            if opex_opportunity > 0:
                if (float(opex_attempt) / float(opex_opportunity)) <= 0.4:
                    return 1            # opponent isn't really bad
                else:
                    return 0            # opponent tried to deceive to often
            else:
                if ret != -1: return ret      # fallback
                else:
                    if opMoves[-1] == 1: return 1   # play TFT if clueless
                    else:                return 0
\end{verbatim}
\end{scriptsize}


\subsubsection{Die Ergebnisse des Computerturniers}

Das Turnier wurde zunächst mit denselben Parametern wie bei Axelrod
(Auszahlungsparameter T=5, R=3, P=1, S=0; Rundenzahl w=200) fünfmal
mit allen Spielern durchgeführt. In vier Fällen gewann {\em Tester}
das Turnier, einmal jedoch gewann {\em Analyst}. Der unterschiedliche
Ausgang des Turniers erklärt sich dadurch, dass einzelne Strategien
(Pseudo-)Zufallszahlen verwenden. Besonders bei {\em Analyst} führt
dies zu starken Schwankungen zwischen den einzelnen Durchläufen. Bei
zwei Durchläufen gelang es {\em Analyst} daher die Strategie {\em
MassiveResponse} (zwei Defektionen zur Strafe für eine Defektion des
Gegners) richtig einzuschätzen und auf ein überwiegend kooperatives
Spielverhalten einzuschwenken, während {\em Tester} gegen {\em
MassiveResponse} regelmäßig schlecht spielte. Dadurch konnte sich {\em
Analyst} in einem Fall den entscheidenden Vorsprung sichern und
verfehlte den Sieg im anderen Fall nur knapp.

Die typische Rangfolge sieht jedoch folgendermaßen aus:

\begin{scriptsize}
\begin{samepage}
\begin{center}Turnier 2\end{center}
\begin{verbatim}
Tester:               6025
GraciousTFT:          5981
Tit for tat:          5912
Analyst:              5750
DelayedTFT:           5567
Tit for two tats:     5557
MaliciousTFT:         5461
Cheater:              5407
Random:               4893
Always friendly:      4764
MassiveResponse:      4748
Utterly destructive:  3748
\end{verbatim}
\end{samepage}
\end{scriptsize}

In allen Durchläufen lag {\em GraciousTFT} vor {\em Tit for Tat}, was mit dem
oben beschriebenen Problem von {\em Tit for Tat} zusammen hängen mag. Ein
Blick auf den Spielablauf verdeutlicht dies:

\begin{scriptsize}
\begin{center}Tit for Tat : MaliciousTFT\end{center}
\begin{verbatim}
1:0 0:1 1:0 0:1 1:0 0:1 1:0 0:1 1:0 0:1 1:0 0:1 1:0 0:1 
1:0 0:1 1:0 0:1 1:0 0:1 1:0 0:1 1:0 0:1 1:0 0:1 1:0 0:1 
1:0 0:1 1:0 0:1 1:0 0:1 1:0 0:1 1:0 0:1 1:0 0:1 1:0 0:1 
1:0 0:1...
\end{verbatim}
\begin{center}(1=Kooperation, 0=Defektion)\end{center}
\end{scriptsize}

Im Vergleich dazu der Beginn der Partie {\em GraciousTFT} gegen {\em
  MaliciousTFT}:

\begin{scriptsize}
\begin{center}GraciousTFT : MaliciousTFT\end{center}
\begin{verbatim}
1:0 0:1 1:0 0:1 1:0 0:1 1:0 1:1 1:1 1:1 1:1 1:1 1:1 1:1 
1:1 1:1 1:1 1:1 1:1 1:1 1:1 1:1 1:1 1:1 1:1 1:1 1:1 1:1 
1:1 1:1 1:1 1:1 1:1 1:1 1:1 1:1 1:1 1:1 1:1 1:1 1:1 1:1 
1:1 1:1...
\end{verbatim}
\end{scriptsize}

Dass anders als in Axelrods Turnieren {\em Tester} und nicht {\em Tit for Tat}
Sieger wurde, ist dadurch zu erklären, dass relativ viele Strategien am
Turnier teilnahmen (immerhin ein sechstel), die sich sehr leicht ausbeuten
lassen, nämlich {\em Always friendly} und {\em Tit for two tats}, was von {\em
  Tester} weidlich ausgenutzt werden konnte. Streicht man eine dieser beiden
Strategien aus dem Turnier, so gerät {\em Tester} sofort ins Hintertreffen.

Ändert man nun die Paramter für das Turnier ein wenig ab, indem man
als Belohnung für wechselseitige Kooperation R=4 Punkte auszahlt, so
ändert sich das Bild: {\em Tester} fällt zurück, während - wie zu
erwarten - {\em GraciousTFT} den ersten Rang einnimmt. {\em Tit for
Tat} belegt immer noch Platz zwei, was immerhin für eine gewisse
Stabilität dieser Strategie gegenüber einer Veränderung der
Turnierbedinungen spricht:

\begin{scriptsize}
\begin{center}Turnier 6 (T=5, R=4, P=1, S=0) \end{center}
\begin{verbatim}
GraciousTFT:          7786
Tit for tat:          7524
Tit for two tats:     7284
Tester:               7077
...
\end{verbatim}
\end{scriptsize}

Wird statt dessen die Bösartigkeit wieder höher belohnt, so schnellt {\em
  Tester} sogleich wieder an die Spitze. Erstaunlich ist, dass auch diesmal
{\em GraciousTFT} vor {\em Tit for Tat} an Platz zwei steht, woraus
vermutungsweise die Schlussfolgerung abgeleitet werden kann, dass {\em
  GraciousTFT} eine echte Verbesserung von {\em Tit for Tat} darstellt:

\begin{scriptsize}
\begin{center}Turnier 9 (T=7, R=4, P=1, S=0) \end{center}
\begin{verbatim}
Tester:               8151
GraciousTFT:          7911
Tit for tat:          7834
Tit for two tats:     7368
...
\end{verbatim}
\end{scriptsize}

Alles in allem zeigt sich jedoch, dass Varianten von {\em Tit for Tat}
durchaus keine schlechte Strategie darstellen, auch wenn Axelrods
Bevorzugung von {\em Tit for Tat} anhand dieser Resultate nicht
nachvollzogen werden kann. Natürlich ist dieses Ergebnis mit Vorsicht
zu genießen, da - wie bereits erwähnt - die Simulation bei weitem zu
wenig Strategien enthält, um aussagekräftige Untersuchungen zu
erlauben.

Die Tatsache, dass {\em GraciousTFT} besser abschneidet als {\em Tit
for Tat} ist intuitiv sehr einleuchtend. Übertragen auf wirkliche
Situationen würde {\em Tit for Tat}, sobald nur ein Missverständnis
auftritt, ein Verhalten herbeiführen, dass einer "`Blutrache"'
verglichen werden könnte, die kein Ende mehr nimmt. Ein Verhalten also, 
dass sich auf Dauer für alle Beteiligten als sehr unvorteilhaft erweist.

%Nach der Darstellung dieser (sehr einfachen) Computersimulation des iterierten
%Gefangenendilemmas und einiger Varianten, die sich aus der Veränderung der
%Auszahlungsmatrix ergeben, sollen nun weitere denkbare Abwandlungen der
%Simulation diskutiert werden, die eine spätere Implementation lohnen
%könnten. Als zwei naheliegende Varianten werden im folgenden das iterierte
%Gefangenendilemma mit asymetrischen Auszahlungen und das iterierte
%Gefangenendilemma mit variablen Auszahlungen näher betrachtet. Beide Varianten
%sind naheliegend, weil sie der Wirklichkeit in den meisten Fällen näher kommen
%als das symmetrische Gefangenendilemma mit gleichbleibenden Auszahlungen.

\subsubsection{Iteriertes Gefangenendilemma mit variablen Auszahlungen: Lohnt sich die
  {\em Heiratsschwindler}-Strategie?}

Neben der Abwandlung der konstanten Auszahlungsparameter, wäre noch
einer weitere Variante denkbar, bei der die Auszahlungsparamter im
Verlauf des Turnieres variabel sind. Zum Beispiel könnten hin- und
wieder Spielrunden eingestreut werden, bei denen die Auszahlung
gegenüber den gewöhnlichen Runden verdoppelt oder verdreifacht wird.

Bei dieser Abwandlung des Simulationsszenarios - für das sich in
der Wirklichkeit ebenfalls leicht Beispiele finden lassen - wäre sogar
die Überlegung anzustellen, ob durch diese Variante die Ergebnisse der
ursprünglichen Simulation nicht wesentlich in Frage gestellt
werden. Ein wichtiges Ergebnis der ursprünglichen
Simulation besteht darin, dass im iterierten Gefangenendilemma
kooperative Strategien leistungsfähiger sind als destruktive Strategien. Man
könnte auch davon sprechen, das im iterierten Gefangenendilemma das Dilemma
aufgebrochen wird, indem die Spieler durch die Drohung von Sanktionen in den
nachfolgenden Spielzügen zum beiderseitgen Vorteil zur Kooperation angehalten
werden. Dies führt zu der (erfreulichen) Schlussfolgerung, dass kooperatives
Verhalten zwischen Menschen nicht nur in der Moral sondern bereits im
Eigennutz eine Stütze findet, was Hoffnungen hinstsichtlich der
Durchsetzungsfähigkeit von kooperativem Verhalten weckt. 

Aber hat dieses Ergebnis auch dann noch Bestand, wenn die Höhe der
Auszahlungen von Fall zu Fall variiert? Sollte man nicht befürchten, dass die
Kooperation gerade dann einbricht, wenn die Auszahlungen besonders hoch und
damit der Betrug besonders lohnend ist (Schurz 2001, S.356f.)? Die Strategie,
die so vorgeht, dass sie bei niedrigen Auszahlungen kooperativ spielt (sofern
die Kooperation gegenseitig ist), bei hohen Auszahlungssummen aber
überraschend defektiert, soll im folgenden als {\em Heiratsschwindler}
bezeichnet werden. Die zu untersuchende Frage lautet also: Kann die Strategie
{\em Heiratsschwindler} im iterierten Gefangenendilemma mit variablen
Auszahlungen kooperative Strategien wie z.B. {\em Tit for Tat} verdrängen?

Um die Erfolgsaussichten einer Strategie wie {\em Heiratsschwindler}
richtig einzuschätzen, müsste eine neue Simualtion unter
entsprechenden Bedingungen durchgeführt werden. Aber auch ohne eine
weitere Simulation lassen sich einige grundsätzliche Überlegungen zu
der Frage anstellen, ob kooperative Strategien dem Eindringen von {\em
  Heiratsschwindler} hilflos ausgeliefert sind. Dafür kann auf den
Begriff der kollektiven Stabilität von Strategien zurückgegriffen
werden (vgl. Axelrod 1984, Anhang B).\footnote{Der Begriff der {\em
    kollektiven Stabilität} ist einer der schwächsten von einer Reihe
  möglicher Begriffe der {\em evolutionären Stabilit}. Ein
  vergleichsweise stärkerer Begriff ist der Begriff der {\em
    evolutionären Stabilität} nach Maynard-Smith, der fordert, dass
  $V(B/A) < V(A/A)$. Auf die Problematik der Begriffsbildung kann an
  dieser Stelle leider nicht näher eingegangen werden (vgl. dazu
  Binmore 1998, 319ff., Binmore/Samuelson 1992, 277ff.).}

Eine Strategie A ist kollektiv stabil, wenn jede beliebige andere Strategie B
im Wettkampf mit ihr nicht mehr Punkte erhält als Strategie A im Wettkampf mit
sich selbst gewinnt. Wenn wir die Punkte, die eine Strategie X in einem
Wettkampf mit der Strategie Y erhält, mit V(X/Y) bezeichnen, dann ist
Strategie A kollektiv stabil, wenn gilt: $V(B/A) \le V(A/A)$ für jede beliebige
Strategie B.

Da diese Definition der kollektiven Stabilität unabhängig von der Höhe der
Auszahlungen ist, lässt sie sich auch auf das iterierte Gefangenendilemma mit
variierenden Auszahlungen anwenden. Unsere Frage lautet dann: Gibt es auch bei
variierenden Auszahlungen kooperative Strategien, die kollektiv stabil sind?
Unter einer kooperativen Strategie verstehe ich dabei eine Strategie, die
nicht unmotiviert defektiert, so dass jede hinreichend freundlich gesonnene
Gegnerstrategie jederzeit mit ihr kooperieren kann. {\em Tit for Tat} ist in
diesem Sinne eine kooperative Strategie. Wie Axelrod bewiesen hat
(Axelrod 1984, Anhang B), ist {\em Tit for Tat} im iterierten
Gefangenendilemma mit gleichbleibenden Auszahlungen kollektiv stabil.

Zu untersuchen ist also die Behauptung, dass sich auch bei
variierenden Auszahlungen immer eine kooperative Strategie
konstruieren lässt, die kollektiv stabil ist. Um die Wahrheit dieser
Behauptung zu zeigen, soll der Einfachheit halber zunächst angenommen
werden, dass in jeder fünften Spielrunde die Auszahlung deutlich höher
ausfällt als in jeder anderen Spielrunde. Ansonsten sollen die
Auszahlungen allerdings immer von gleicher Höhe sein. Nun betrachte
man folgende aus zwei Teilstrategien zusammengesetzte Strategie:

\begin{enumerate}

\item Für jede Runde mit gewöhnlicher Auszahlung spiele {\em Tit for
    Tat}. Nimm dabei Bezug auf die jeweils letzte Runde mit gewöhnlicher
  Auszahlung (d.h. in der dritten Runde bezieht sich die Reaktion auf das
  Gegnerverhalten der zweiten Runde, aber in der sechsten Runde auf das der vierten Runde).

\item Für jede durch fünf teilbare Runde spiele ebenfalls {\em Tit for Tat},
  aber beziehe Dich dabei immer auf die jeweils letzte Runde mit durch fünf
  teilbarer Rundenzahl. (Hat der Gegner also in der fünften Runde defektiert,
  so folgt die Bestrafung dafür erst in der zehnten Runde, unabhängig davon, wie
  der Gegner in der neunten Runde gespielt hat.)

\end{enumerate}

Nennen wir nun diese Strategie A und die beiden Teilstrategien der
Einfachheit halber $A_1$ und $A_2$. Tritt Strategie A im Wettkampf
gegen sich selbst an, so gilt für die Gesmthöhe aller Auszahlungen
offensichtlich: $V(A/A) = V_1(A/A) + V_2(A/A)$. Dabei bezeichnet $V_1$
die Summe der Auszahlungen in den Runden 1..4, 5..9, 11..14 usw. und
$V_2$ die Summe der Auszahlungen in den Runden 5, 10, 15, 20, 25
usw. Da $A_1$ und $A_2$ unabhängig voneinander sind, gilt ebenso
$V(A/A) = V_1(A_1/A_1) + V_2(A_2/A_2)$.

Lassen wir Strategie A nun gegen eine beliebige Strategie B antreten,
dann lässt sich die Auszahlung $V(B/A)$ auf folgende Weise zerlegen:
$V(B/A) = V_1(B/A) + V_2(B/A)$. Da aber $A_1$ und $A_2$ jeweils
kollektiv stabil sind (denn es handelt sich ja in beiden Fällen um die
Strategie {\em Tit for Tat} bei gleichbleibenden Auszahlungen) so
gilt: $V_1(B/A) \le V_1(A_1/A_1)$ und $V_2(B/A) \le
V_2(A_2/A_2)$.
Dann gilt insbesondere aber auch:
$V(B/A) = V_1(B/A) + V_2(B/A) \le V_1(A_1/A_1) + V_2(A_2/A_2) =
V(A/A)$.
Nach der Definition ist Strategie A damit kollektiv
stabil. Insbesondere kann dann auch die Strategie {\em
  Heiratsschwindler} nicht in Strategie A eindringen. Entscheidend für
die Konstruktion der kollektiv stabilen kooperativen Strategie A ist
lediglich, dass vorher bekannt ist, wann eine besonders hohe
Auszahlung erfolgt. Dies müssen wir allerdings voraussetzen, da sonst
auch die {\em Heiratsschwindler}-Strategie nicht konstruiert werden
könnte. Weiterhin dürfen die besonders hohen Auszahlungen nicht zu
selten erfolgen, da es sonst zu demselben Phänomen kommt, das auch bei
einer nur sehr geringen Anzahl von Spielrunden auftritt, und das darin
besteht, dass Defektionen nicht mehr wirksam bestraft werden können.

Aufgrund der vereinfachenden Annahmen, dass es nur zwei
unterschiedliche Auzsahlungsniveaus gibt, wobei die höhrere Auszahlung
periodisch in jedem fünften Zug auftritt, behandelt der eben geführte
Beweis bisher lediglich einen Spezialfall. Eine Verallgemeinerung des
Beweises ist folgendermaßen möglich: Wir nehmen statt zwei
unterschiedlicher Auszahlungshöhen eine Menge von {\em k}
unterschiedlichen Auszahlungshöhen
$\left\lbrace {\cal T}_1,{\cal T}_2,\ldots,{\cal T}_k \right\rbrace$
an. Dabei steht jedes ${\cal T}_i$ mit $\left( 1 \le i \le k\right) $
für ein Tupel $\left( T_i,R_i,P_i,S_i\right) $ von
Auszahlungsparametern, die den beiden Bedingungen
$ T_i > R_i > P_i > S_i $ und $ T_i + S_i < 2R_i $
genügen.\footnote{Die erste Bedingung ist die Voraussetzung dafür,
  dass es sich um ein Gefangenendilemma habndelt. Die zweite Bedingung
  stellt sicher, dass eine dauerhafte wechselseitige Kooperation
  gewinnbringender für die Spieler ist als eine Folge von Spielzügen,
  bei der immer abwechselnd ein Spieler kooperiert, während der andere
  defektiert. Letzteres würde das Modell grundlegend verändern und
  Rahmenbedingungen schaffen, in der sich die abwechselnde
  gegenseitige Beschädigung der Spieler als optimale kooperative
  Struktur herausbilden würde.}

Weiterhin nehmen wir an, dass die Folge der Spielrunden
${\cal R}=r_1, r_2,\ldots, r_n$ aus Teilfolgen ${\cal R}_i$ mit
$\left( 1 \le i \le k\right) $ von hinreichender Länge (mindestens
zwei Züge) zusammengesetzt ist, in denen die Auszahlung jeweils
entsprechend den Auszahlungsparametern ${\cal T}_i$ erfolgt. Es wird
voraussgesetzt, dass die Strategien in jeder Runde wissen, nach
welchem Tupel ${\cal T}_i$ die Auszahlungen erfolgen, d.h. welcher
Teilfolge diese Runde zugeordnet ist. Die Strategien verfügen jedoch
nicht über die Information, wann die letzte Runde einer Teilfolge
erreicht ist.

Nun lässt sich ganz analog zu dem oben gegebenen Beweis für den
vereinfachten Fall eine Strategie $A$ konstruieren, die über die
Teilfolgen ${\cal R}_i$ jeweils {\em TitForTat} spielt. Für die
Auszahlung, welche die Strategie $V(A/A)$ im Spiel gegen sich selbst
erhält, gilt $V(A/A) = \sum_{i=1}^k V_i(A_i/A_i)$, wobei $V_i$ für die
Gesamtauszahlung über die Teilfolge ${\cal R}_i$ steht und die
Teilstrategie $A_i$ als {\em TitForTat} über die Teilfolge
${\cal R}_i$ der Spielrunden definiert ist. Wie im vereinfachten Fall
lässt sich für jede denkbare Angreiferstrategie $B$ die Ungleichung
aufstellen:

\begin{displaymath}
V(B/A) = \sum_{i=1}^k V_i(A/B) \le \sum_{i=1}^k V_i(A_i/A_i) = V(A/A)
\end{displaymath}

womit $A$ kollektiv stabil ist. Folglich existiert auch bei nicht
periodisch und in unterschiedlicher Höhe variierenden Auszahlungen
keine {\em Heiratsschwindler}-Strategie, die in $A$ eindringen kann.


\subsection{Erweiterung zur populationsdynamischen Simulation}

Der Turniererfolg einer bestimmten Strategie sagt noch nicht zwingend
etwas über den langfristigen Erfolg dieser Strategie im evolutionären
Wettbewerb aus. Um diese Frage zu untersuchen, soll nun das
ursprüngliche Computerturnier in einem zweiten Schritt zu einer
populationsdynamischen Simulation ausgebaut werden.

Dazu wird mit einer Population von zehntausend Spielern, die auf die
gegebene Menge von Strategien gleichmäßig verteilt werden, eine Serie
von Turnieren durchgespielt, wobei diejenigen Strategien, die sich als
besonders erfolgreich erweisen, in der jeweils folgenden Runde von
einer größeren Anzahl von Spielern angewendet werden, während die
weniger erfolgreichen Strategien eine geringere Anzahl von Spielern
zugewiesen bekommen, bis sie möglicherweise irgendwann ganz
"`aussterben"'.\footnote{ Die Neuverteilung der Spielerpopulation auf
  die Strategien wird dabei nach folgender Vorschrift bestimmt:
  Zunächst wird der Mittelwert der Resultate aller Strategien
  berechnet. Dann wird für jede Strategie der Quotient aus ihrer
  Punktzahl und dem Mittelwert der Punktzahlen aller Strategien
  gebildet. Der ermittelte Wert dient als Faktor für die
  Größenänderung der Population der Strategie. In einem letzten
  Schritt werden die Populationen aller Strategien soweit skaliert,
  dass ihre Summe wieder einer vorgegebenen Gesamtpopulation von
  10.000 Individuen entspricht, wobei kleine Abweichungen zugelassen
  werden, um Rundungsfehler möglichst zu vermeiden. Strategien, von
  denen nur noch weniger als zwei Individuen übrig geblieben sind,
  werden aus dem Rennen gezogen.

  Dieses Verfahren zur Ermittlung der Neuverteilung der Individuen ist
  mehr oder weniger willkürlich gewählt, genügt aber der Bedingung,
  dass eine Strategie sich umso erfolgreicher vermehrt, je größer ihre
  Punktzahl ist. Natürlich könnte man ebenso gut auch jede andere
  streng monoton steigende Abbildung der erreichten Punktzahlen auf
  die Größenänderungen der Populationen wählen. In den meisten Fällen
  dürfte dies nur zu einer Beschleunigung oder Verlangsamung der zu
  beobachtenden Phänomene führen.} Es würde an dieser Stelle zu weit
führen, auf alle technischen Einzelheiten dieser Simulation
detailliert einzugehen.

Lediglich eine der (gegenüber der im letzten Abschnitt beschriebenen
Simulation geringfügig erweiterten) Strategiemenge neu hinzugefügte
Strategie soll kurz vorgestellt werden, da sie in der
populationsdynamischen Simulation unter bestimmten Bedingungen sehr
erfolgreich ist.

Die Strategie {\em Pawlow} beginnt mit zwei Defektionen und ändert nur
dann ihr Verhalten, wenn sie vom Gegner bestraft wird (d.h. wenn der
Gegner mit einer Defektion antwortet), wobei es keine Rolle spielt,
ob sie für kooperatives oder unkooperatives Verhalten "`bestraft"'
worden ist. Als Sicherheitsmechanismus (um "`Missverständnisse"' zu
vermeiden) ändert sie ihre Strategie aber nur höchstens jede zweite
Runde.

\begin{scriptsize}
\begin{center}Pawlow:\end{center}
\begin{verbatim}
    def nextMove(self, round, myMoves, opMoves):
        if round <= 2:
            return 0                                    # be naughty
        else:
            if opMoves[-1] == 0 and myMoves[-1] == myMoves[-2]:
                if myMoves[-1] == 0:    return 1        # learned something
                else:                   return 0
            else:
                return myMoves[-1]
\end{verbatim}
\end{scriptsize}


\subsubsection{Die Ergebnisse der populationsdynamischen Simulation}

\begin{figure}
\begin{center}
\includegraphics[width=12cm]{SimB1.eps}
\caption{\label{SimB1} Populationsdynamische Simulation ohne
Mutationen oder Rauschen.}
\end{center}
\end{figure}

Das Ergebnis der populationsdynamischen Simulation entspricht den
Erwartungen (Abbildung \ref{SimB1}). Die Strategie {\em GraciousTFT}
setzt sich bald an die Spitze und dominiert das Feld, dicht gefolgt
von {\em Tit for Tat}. Dass zwischen beiden Strategien ein Abstand
bleibt, obwohl sie nach dem Verschwinden der meisten anderen
Strategien stets die gleiche Durchschnittspunktzahl erziehlen, hängt
mit der Verteilungsfunktion zusammen, die Verschiebungen zwischen den
Strategien nur bei unterschiedlicher Punktzahl erlaubt und damit einen
einmal errungenen Vorteil einer gleich starken Strategie konserviert.

Auffällig ist das kontinuierliche Absinken der anfangs sehr starken
Strategie {\em Tester}. Dies ist dadurch zu erklären, dass nach dem
Verschwinden der meisten Strategien keine gegnerische Strategie mehr
vorhanden ist, die von {\em Tester} ausgebeutet werden kann. Durch
seinen Ausbeutungsversuch in den ersten beiden Runden schneidet {\em
  Tester} dann immer etwas schlechter ab als {\em Tit for Tat} und
{\em GraciousTFT}.

Ein ähnliches Problem betrifft auch {\em Pawlow}, wobei es allerdings
erstaunlich ist, das die im Turnier eher mittelmäßige Strategie {\em
  Pawlow} überhaupt so lange durchhalten kann. Letzteres dürfte darauf
zurückzuführen sein, dass die Strategie {\em Pawlow} über die
Fähigkeit verfügt, sich relativ gut an ihr Millieu anzupassen.

Diese Fähigkeit bewährt sich besonders dann wenn die
populationsdynamische Simulation unter erschwerenden Bedingungen
durchgeführt wird, wie die folgenden Beispiele zeigen.

\subsubsection{Der Einfluss von Rauschen auf die Populationsdynamik}

Es ist unrealistisch anzunehmen, dass bei einer so großen Anzahl von
Interaktionen zwischen den Spielern, wie sie bei der
populationsdynamischen Simulation angenommen wird, nicht gelegentlich
auch Fehlleistungen und Missverständnise auftreten sollten. Um diesen
Aspekt zu simulieren wird ein Rauschparameter in die Simulation
eingefügt. Dabei werden mit einer bestimmten Wahrscheinlichkeit die
Züge der Spieler ins Gegenteil verkehrt, d.h. einem Spieler, der
kooperativ spielen wollte, wird eine Defektion untergeschoben und
umgekehrt. Alles kommt nun darauf an, ob die Strategien in der Lage
sind solche Fehler wieder auszugleichen.

\begin{figure}
\begin{center}
\includegraphics[width=12cm]{SimA2.eps}
\caption{\label{SimA2} {\small Populationsdynamische Simulation mit Rauschen (1\%).}}
\end{center}
\end{figure}

Bei einem 1\%-igen Rauschen (Abbildung \ref{SimA2}) können sich nur
{\em GraciousTFT} und {\em Pawlow} längerfristig erfolgreich behaupten
können. Beide Strategien verfügen über Mechanismen (siehe dazu die
Beschreibungen der Strategien weiter oben), die ihnen eine gewisse
Fehlertoleranz verleihen. Das Ergebnis bestätigt auch die eben
ausgesprochene Vermutung über die Anpassungsfähigkeit von {\em
  Pawlow}. Zu Erwähnen ist allerdings, dass bei einer Erhöung des
Rauschparameters auf 5\% der Vorteil für {\em Pawlow} verschwindet, so
dass diese Strategie kaum besser abschneidet als die anderen
Strategien mit Ausnahme von {\em GraciousTFT}.

\subsubsection{Der Einfluss von Mutationen}

Obwohl die populationsdynamische Simulation einer evolutionären
Entwicklung schon nahe kommt, fehlt in diesem Modell noch ein
wichtiger Aspekt. Zwar ändert sich die Strategiemenge im Laufe der
Simulation durch das Wegfallen erfolgloser Strategien, aber es kommen
keine neuen Strategien hinzu. Ein erster Ansatz, um diesen
wesentlichen Aspekt evolutionärer Entwicklungen in das Modell zu
integrieren, besteht darin, einen Mutationsparameter einzuführen, der
nach jedem Durchgang die Mutation eines bestimmten Prozentsatzes der
Strategien bewirkt. Dabei wird der Einfachheit halber zunächst
angenommen, dass durch Mutationen nur vereinfachte Strategien
entstehen können, nämlich {\em AlwaysD} (defektiere immer) und {\em
  AlwaysC} (kooperiere immer), die beide jeweils die Hälfte der
mutierten Strategien stellen.

Im Vergleich zur ursprünglichen populationsdynamischen Simulation ohne
Mutationen ergibt sich mit Mutationen ein dramatisch verändertes Bild
(Abbildung \ref{SimB3}). Diesmal wird die Szene von den beiden
Strategien {\em Tester} und {\em Pawlow} beherrscht, die von dem
kontinuierlichen Zustrom des ohne weiteres ausbeutbaren Mutanten {\em
  AlwaysC} profitieren. Demgegenüber geraten {\em Tit for Tat} und
{\em GraciousTFT}, die keine unmotivierten Defektionen unternehmen und
daher {\em AlwaysC} auch nicht auszubeuten versuchen, deutlich ins
Hintertreffen. Wie man sieht, ist der evolutionäre Erfolg von
"`gutwilligen"' Strategien wie {\em Tit for Tat} oder {\em
  GraciousTFT} daran geknüpft, dass sie in einer Umwelt agieren, in
der keine "`degenerativen Mutationen"' auftreten, die die Population
unterwandern. Wie wahrscheinlich eine solche Annahme ist, hängt
letztlich von dem Sachbereich ab, der mit einem evolutionären Modell
beschrieben werden soll. Bereits so lässt sich aber festhalten, dass
einige der von Axelrod getroffenen generellen Feststellungen wie
diejenige, dass die Strategie {\em Tit for Tat} immer eine sichere
Wahl sei, oder dass erfolgreiche Strategien in aller Regel
"`gutwillig"' sein müssen, sich in dieser Allgemeingültigkeit nicht
aufrecht erhalten lassen.

\begin{figure}
\begin{center}
\includegraphics[width=12cm]{SimB3.eps}
\caption{\label{SimB3} {\small Populationsdynamische Simulation mit
Mutationen (1\%).}}
\end{center}
\end{figure}

\subsection{Möglichkeiten und Grenzen von Computermodellen bei der Untersuchung evolutionärer Prozesse}

Um die Erklärungskraft von Computermodellen bei der Untersuchung
evolutionärer Vorgänge im Bereich der Gesellschaftswissenschaften
richtig einzuschätzen, muss man berücksichtigen, dass in diesem
Bereich die relevanten Einflüsse selten mit hinreichender Genauigkeit
bestimmt oder auch nur vollständig benannt werden können.\footnote{Wie
  wollte man beispielsweise die Agressivität eines Staates beziffern?
  Agressivität kann sich in militärischer Aufrüstung ebenso ausdrücken
  wie in rhetorischem "`Säbelrasseln"' der Regierung. Wie das Beispiel
  zeigt, ist es oft noch nicht einmal möglich, eine auch nur halbwegs
  genaue Ordnungsrelation ("`agressiver als"') aufzustellen.} Von
Ausnahmefällen abgesehen dürfte es daher kaum möglich sein,
Computermodelle zu erstellen, die genaue Prognosen ermöglichen.

Dennoch kann der Einsatz von Computermodellen sinnvoll sein, um die
Muster zu studieren, nach denen evolutionäre Prozesse ablaufen. Eine
gewisse Vorsicht ist jedoch geboten, wenn aus einer Computersimulation
verallgemeinernde Schlussfolgerungen gezogen werden sollen. Sonst
besteht die Gefahr, dass zufällige, d.h. von der Wahl bestimmter
Paramter abhängige Simulationsergebnisse voreilig zu allgemeinen
Regeln hypostasiert werden.

Aus heutiger Sicht erscheint der Ansatz von Axelrods "`Evolution der
Kooperation"', der hier einmal als Beispiel nachvollzogen wurde,
gerade in dieser Hinsicht noch als zu unvorsichtig, um nicht zu sagen
naiv. So beruht die Tatsache, dass in den beiden aufeinanderfolgenden
Turnieren, die Axelrod in seinem Buch beschreibt (Axelrod 1984,
Kap. 2), jedesmal die Strategie {\em Tit for Tat} als Sieger
hervorging, wahrscheinlich nur auf Zufall. Denn der Erfolg von {\em
  Tit for Tat} hängt nicht zuletzt von der Auswahl der Strategiemenge
ab, die bei Axelrod hochgradig kontingent ist, indem sie durch eine
Art von Preisausschreiben bestimmt wurde. Versucht man systematischer
vorzugehen und als Strategiemenge alle Strategien mit einer gewissen
Komplexität zu Grunde zu legen, also beispielsweise alle Strategien,
die sich als endliche Automaten darstellen lassen, die bis zu zwei
Zustände speichern können, dann gewinnt mitnichten {\em Tit For
  Tat}. Wie die Abbildung \ref{Pure} zeigt, geht in diesem Falle
vielmehr die Strategie {\em Grim} (Ewige Vergeltung) als Sieger aus
dem Wettkampf hervor (vgl. auch Binmore 1998, 322). Allerdings ist
auch damit noch nicht das letzte Wort gesprochen, denn es könnte nun
wiederum eingewandt werden, dass die Menge aller ein- und zweistufigen
Automaten keine sinnvolle Ausgangsbasis ist, da sie eine ungewöhnlich
große Anzahl ausgesprochen "`dummer"' Strategien enthält, die auch auf
ständige Defektion mit fortgesetzter Kooperation reagieren, was kaum
realistisch erscheint.

\begin{figure}
\begin{center}
\includegraphics[width=12cm]{Pure.eps}
\caption{\label{Pure} {\small Populationsdynamische Simulation mit endlichen Automaten.}}
\end{center}
\end{figure}

Diese Einwände verdeutlichen, dass die Interpretation der Ergebnisse
von Computersimulation keineswegs trivial ist und durch theoretische
Überlegungen abgestützt werden muss.

\section{Beispiele für evolutionäre Erklärungsansätze im Bereich der Kulturwissenschaften}

\subsection{Die evolutionäre Erklärung historischer Prozesse}

Auch wenn, wie in der Einleitung bereits angemerkt, Metaphern aus der
Biologie und besonders der Evolutionstheorie Eingang in die
Geschichtsschreibung gefunden haben, stehen die langsam ablaufenden
Prozesse evolutionärer Entwicklungen für gewöhnlich nicht im
Mittelpunkt des Interesses von Fachhistorikern. Die Gründe dafür sind
zweierlei: Zum einen befasst sich die Historie tradionellerweise
vorwiegend mit der politischen Geschichte und teilt die Epochen daher
nach politischen Großereignissen wie Kriegen, Revolutionen, der
Gründung und dem Untergang von Staaten und Reichen ein. Solche
Ereignisse lassen sich genauer datieren und sind leichter fassbar als
die schleichend verlaufenden evolutionären
Entwicklungsprozesse. Historiker betrachten die Ergebnisse solcher
Prozesse wie z.B. das technische Niveau in einer bestimmten Epoche
daher eher als historische Rahmenbedinungen, die als gegeben
vorausgesetzt werden müssen. Zum anderen scheint es häufig an einer
geeigneten konzeptionellen Grundlage zu fehlen, um evolutionäre
Prozesse erfassen und sie als solche verstehen zu können. Dieser
Mangel äußert sich manchmal darin, dass evolutionäre Prozesse
gewaltsam zu zeitlich leicht datierbaren historischen
Schlüsselereignissen umgebogen werden, wie Eric Jones dies am Beispiel
der sogenannten "`industriellen Revolution"' kritisiert (Jones 1988).

Im folgenden werde ich kurz zwei erfolgreiche Beispiele für evolutionäre
Erklärungen in der Geschichtswissenschaft anführen. Eines davon stammt
charakteristischerweise nicht von einem Historiker sondern von einem Biologen.

% Warum evolutionäre Prozesse in der Historie so oft vernachlässigt werden:
% 1. Grund: Periodisierung der Geschichte nach großen Ereignissen
% 2. Grund: Mangel an einer brauchbaren theoretischen Grundlage

\subsubsection{Die neolithische "`Evolution"' (J. Diamond)}

In seinem Buch "`Guns, Germs and Steel"' (Diamond 1998) unternimmt Jared
Diamond den anspruchsvollen Versuch, die zivilisatorische Entwicklung
menschlicher Gesellschaften (und insbesondere den Entwicklungsvorsprung der
eurasischen Gesellschaften) aus natürlichen Umweltbedingungen ihrer
geographischen Region zu erklären. So erklärt Diamond beispielsweise die
Geschwindigkeit, mit der Ackerbau und Viehzucht entwickelt wurden und sich
ausbreiten konnten, aus dem Vorhandensein (oder eben Nicht-Vorhandensein) von
Pflanzen und Tieren die zur Domestikation geeignet waren, und damit ob sich
diese Kulturtechniken entlang einer Ost-West-Achse (Eurasien) oder entlang
einer Nord-Süd-Achse (Amerika), die sich über unterschiedliche Klimazonen
ersteckt, ausbreiten mussten.  Diamonds Erklärung erscheint deshalb so
zwingend, weil sie zeigt, dass die Entwicklung grundlegender zivilisatorischer
Errungenschaften von Umweltbedingungen abhängig ist, deren Fehlen auch der
größte menschliche Erfindungsgeist nicht ausgleichen kann.

Inwiefern handelt es sich dabei um einen evolutionären
Erklärungsansatz? Die Wege, auf denen sich Kulturtechniken ausbreiten
und selektiv gegen andere Kulturtechniken durchsetzen (kriegerische
Verdrängung, kulturelles Lernen), wurden bereits
angesprochen. Entscheidend ist außerdem, dass es ein hinreichend
großes Versuchfeld gab, indem unterschiedliche Kulturtechniken erprobt
und verfeinert werden konnten. Doch auch dies ist unproblematisch, da
Diamond relativ lange Zeitspannen betrachtet, und zudem geographische
Räume behandelt, in denen oft eine Vielzahl untereinander mehr oder
weiniger lose verbundener Gesellschaften existierten. Insbesondere
gelingt es Diamond auch die Zwischenstadien der evolutionären
Entwicklung zu identifizieren, indem er z.B. nachweist, dass auch
Völker, die noch keinen Ackerbau betreiben, in manchen Fällen als
mögliche Vorstufe des Ackerbaus bereits Nutzgärten pflegen.

Von besonderem Interesse ist Diamonds Darstellung der Evolution der
politischen Verbände vom Stammesverband bis zum großen Flächenstaat.
Großstaaten stützen sich auf eine Reihe von politischen, religiösen,
wirtschaftlichen, rechtlichen und militärischen Insitutionen, die in
ihrem Zusammenspiel so komplex sind, dass sie in ihrer Gesamtheit kaum
von genialen Reichsgründern oder Zivilisationsstiftern erfunden worden
sein können. Obwohl Diamond für die schrittweise Entwicklung der
großen Staatswesen eine relativ schlüssige Typologie anbietet, bleibt
dieses Gebiet innerhalb seiner Theorie am ausbaufähigsten. Die
Nachzeichnung der evolutionären Prozesse die zur Entwicklung der
komplizierten Institutionengefüge von Staaten geführt haben, ist so
gesehen noch eine Herausforderung. (Auch wenn in dieser Hinsicht z.B.
von Max Weber und dessen Nachfolgern einiges geleistet worden
ist. Aber gerade diese Vorleistungen ließen sich mit Hilfe
evolutionärer Erklärungsansätze womöglich noch verfeinern.)

\subsubsection{"`Das Wunder Europas"' (Eric Lionel Jones)}

Ähnlich wie Diamond versucht auch Eric Jones den historischen "`Erfolg"'
Europas zu erklären (Jones 1991). Anders als dieser schreibt er jedoch nicht
die Geschichte der Menschheit seit den letzten 13.000 Jahren, sondern er
beschränkt sich auf die jüngste Zeit vom Mittelalter bis zur Gegenwart. Bei
seiner Erklärung misst er dem schleichenden Prozess stetiger wirtschaftlicher
und technischer Entwicklung wesentlich größere Bedeutung bei als
vermeintlichen historischen Durchbrüchen wie der Entdeckung der
Gravitationskraft oder der industriellen Revolution. Ohne dies immer explizit
zu erwähnen, legt Jones dabei ein evolutionäres Erklärungsschema zu Grunde.
Demnach kam dem europäischen Kontinent gerade seine politische Kleinräumigkeit
zu Gute. Diese ermöglichte es, neue Errungenschaften gleich welcher Art
gewissermaßen evolutionär auszuprobieren. Wenn ein Erfinder oder ein Entdecker
bei einem bestimmten Herrscher kein Gehör fand, so brauchte er nicht weit zu
reisen, um einen anderen zu finden, der vielleicht größeres Interesse zeigte,
und dafür im günstigsten Fall mit einer Erfindung belohnt wurde, die ihm
einen wichtigen Vorteil gegenüber seinen Rivalen sichern konnte. Aus demselben
Grund konnten es sich die Herrscher in Europa auch nicht erlauben den
wirtschaftlich potenten Schichten, deren Angehörige nur eine Grenze
überschreiten mussten um sich der Willkür eines bestimmten Herrschers zu
entziehen, bis auf den letzten Blutstropfen Steuern abzupressen. In
zentralisierten Staaten wie dem chinesischen Kaiserreich verhielt sich dies
anders: Hier konnte der Herrscher eine neue Errungenschaft ohne Mühe per
Dekret verbieten, wenn sie ihm nicht zusagte, und Jones führt zahlreiche
Beispiele auf, wo dies auch der Fall gewesen ist (Jones 1991, 77, 231ff.). 

Natürlich bildet dieser Gesichtspunkt nur einen Teilaspekt von Jones
Erklärung. Die Nationalstaatsbildung und der Aufstieg der Handel
treibenden Klassen waren für das "`Wunder Europa"' von kaum gringerer
Bedeutung. Dennoch läßt sich aus Jones Darstellung etwas vereinfachend
die Quintessenz ziehen, dass Europa gerade deshalb so erfolgreich war,
weil hier evolutionäre Trial and Error-Prozesse wirksam werden
konnten.\footnote{In ähnlicher Weise wird übrigens oft auch der Erfolg
  des US-amerikanischen Föderalismus gegenüber dem europäischen (und
  erst recht dem bundesdeutschen) Föderalismus gedeutet. Im
  amerikanischen Föderalismus herrscht gerade das richtige Maß an
  Unabhängigkeit der einzelnen Staaten, welches das Ausprobieren
  unterschiedlicher Wege ermöglicht. Zugleich gibt es kaum Barrieren
  (wie etwa die Sprachbarrieren in Europa), die der raschen
  Verbreitung erfolgreicher Vorbilder im Wege stehen.}

\subsection{Evolutionäre Stabilität ethischer Normen}

Zu guter Letzt soll noch eine Anwendungsmöglichkeit evolutionärer
Theorien im Bereich der Ethik diskutiert werden. Dass evolutionäre
Theorien herangezogen werden können, um die Entstehung von moralischen
und rechtlichen Normen empirisch zu erklären, bedarf nach dem bisher
gesagten keiner weiteren Begründung mehr. Dass ein evolutionärer
Ansatz aber auch bei der Diskussion der philosophischen Frage weiter
helfen kann, welche Normen denn nun gelten {\em sollen}, erscheint auf
den ersten Blick alles andere als einleuchtend. Schließlich gilt es
seit David Hume als ein ehernes Gesetz, dass man nicht von Tatsachen
auf Normen schließen kann, so daß insbesondere die Tatsache, dass sich
irgendwelche Normen evolutionär durchgesetzt haben, keineswegs besagt,
dass diese Normen auch gut und richtig sind.

Dennoch kann der Gegensatz zwischen Sein und Sollen überbrückt werden,
wenn man ein (zwangsläufig) normatives Prinzip einführt, das einen
Zusammenhang zwischen diesen beiden Ebenen ausdrücklich herstellt. In
der Tat enthalten fast alle vorkommenden ethischen Systeme mit
Ausnahme solcher, die man im Sinne Max Webers als rein
"`gesinnungsethisch"' charakterisieren müsste, explizit oder implizit
derartige Prinzipien. Gerade im Bereich der politischen Moral wird
gemeinhin ein Prinzip angenommen, das besagt, dass keine
politisch-ethische Norm Gültigkeit beanspruchen kann, deren Befolgung
der Selbstaufgabe des eigenen Landes gleichkommt.\footnote{Das soll
  nicht bedeuteten, dass es nicht auch Extremfälle gibt, in denen die
  Selbstaufgabe des eigenen Landes das kleinere von zwei Übeln ist.}
Man könnte dieses Prinzip das {\em Prinzip der Realitätsadäquatheit}
nennen.

Der Begriff der evolutionären (bzw. kollektiven) Stabilität könnte nun zur
näheren Konkretisierung dieses Prinzips verwendet werden. Dadurch wäre
zweierlei gewonnen: Zum einen würde der Fehlschluss vermieden werden, dass dem
Realitätsadäquatheitsprinzip nur durch ein besonders rücksichtsloses und
nationalegoistisches Verhalten genüge getan werden könnte (ein Fehlschluss zu
dem die Schule des "`Politischen Realismus"' manchmal neigt), da es eine
gewisse Bandbreite evolutionär stabiler Verhaltensweisen gibt, darunter - wie
zuvor gezeigt - auch solche die grundsätzlich eher kooperativ ausgerichtet
sind.\footnote{Durch das Kriterium der evolutionären Stabilität wird also die
politsch-ethische Entscheidung nicht schon erübrigt, womit den
Entscheidungsträgern freilich auch die Übernahme von Verantwortung (für ihre
normative Wahl) nicht erspart bleibt.} Zum anderen bleibt es möglich, allzu
illusorische oder idealistische Normforderungen wie z.B. einen kategorischen
Pazifismus wohlbegründet abzuweisen.

\newpage

\section{Zitierte Literatur}

\setlength{\parindent}{0ex}
\setlength{\parskip}{2ex}


{\bf Axelrod, Robert (1984)}: Die Evolution der Kooperation,
Oldenbourg, München (5 Aufl. 2000; engl. Original 1984).

{\bf Axelrod, Robert (1997)}: The Complexity of Cooperation. Agent-Based Models of
Competition and Collaboration, Princeton University Press, Princeton.

{\bf Binmore, Ken / Samuelson, Larry (1992)}: Evolutionary Stability in Repeated Games Played by finite Automata, in: Journal of Economic Theory 57 (2/1992), 278-305.

{\bf Binmore, Ken (1994)}: Game Theory and the Social Contract I. Playing Fair, MIT Press, Cambridge (Massachusetts), London (England) (4. Nachdruck 2000). 

{\bf Binmore, Ken (1998)}: Game Theory and the Social Contract II. Just Playing, MIT Press, Cambridge (Massachusetts), London (England).

{\bf Diamond, Jared (1998)}: Guns, germs and steel: a short history of everybody for
the last 13000 years, Vintage Random House, London.

{\bf Jones, Eric L. (1981)}: Das Wunder Europa : Umwelt, Wirtschaft und Geopolitik in
der Geschichte Europas und Asiens, J.C.B. Mohr, Tübingen (deutsch 1991, engl. Original 1981).

{\bf Jones, Eric L. (1988)}: Growth Reccurring. Economic Change in World History,
Oxford.

{\bf Koch, Hannsjoachim (1973)}: Der Sozialdarwinismus. Seine Genese und sein Einfluß auf das imperialistische Denken, C.H. Beck, München.

{\bf Maynard-Smith, John (1982)}: Evolution and the Theory of Games, Cambridge Univ. Press, Cambridge (8. Aufl. 2000).


{\bf Schurz, Gerhard (2001)}: Natürliche und kulturelle Evolution: Skizze einer
verallgemeinerten Evolutionstheorie, in: Wickler, Wolfgang / Salwiczek, Lucie:
Wie wir die Welt erkennen. Erkenntnisweisen im interdisziplinären Diskurs,
München.

{\bf Schüssler, Rudolf (1990)}: Kooperation unter Egoisten: vier Dilemmata, R.Oldenbourg Verlag, München (2.Aufl. 1997) 

{\bf Wagner, Günter P. (1994)}: Der Dialog zwischen Evolutionsforschung und Computerwissenschaft, in: Wieser (1994a, Hg.), 221-233.

{\bf Wieser, Wolfgang (1994a, Hg.)}: Die Evolution der
Evolutionstheorie. Von Darwin zur DNA, Wissenschafttliche
Buchgesellschaft, Darmstadt.

{\bf Wieser, Wolfgang (1994b)}: Gentheorien und Systemtheorien. Wege
und Wandlungen der Evolutionstheorie im 20. Jahrhundert, in: Wieser
(1994a, Hg.), 15-48.

\newpage

\section{Anhang: Programmcode des Computerturniers}

Das Programm für das Computerturnier (Kapitel 3.1) ist in zwei Module ({\em Tournament.py} und {\em Strategy.py}) aufgeteilt. Das Modul {\em Tournament.py}
enthält die beiden Klassen {\em Match} und  {\em Tournament}, wobei {\em Match} ein Spiel zwischen zwei Spielern steuert, während {\em Tournament} den Ablauf des gesamten Turniers festlegt und dazu für jede mögliche Spielerpaarung ein Objekt der Klasse {\em Match} erzeugt (siehe die Methode {\em runTournament} der Klasse {\em Tournament}). 

Das Modul {\em Strategy.py} enthält die Strategien, die am Turnier teilnehmen können. Alle Strategien sind als abgeleitete Klassen der abstrakten Basisklasse {\em Strategy} definiert. Sie enthalten als einzige Methode die Methode {\em nextMove}, der die Nummer des gegenwärtigen Zuges sowie die Listen aller bisherigen eigenen Züge und aller bisherigen gegenerischen Züge übergeben werde müssen. Der Rückgabewert ist entweder 0 oder 1, wobei 0 einer Defektion entspricht, während 1 Kooperation bedeutet.

Aus Platzgründen wird das wesentlich umfangreichere Programm für die populationsdynamische Simulation (ebenso wie der Programmcode für die Benutzschnittstelle) hier nicht mehr mit abgedruckt.


\subsection{Tournament.py}

\begin{scriptsize}
\begin{verbatim}
"""Classes for matches and tournaments.

The the instances of the classes Match and Tournament can be
connected to visualizing objects by appending a signal function to
the signalXXXXXX lists of the class instance. Signal functions have
the general form:

    function(originating_object, *argument_list)

See the source code to find out which signals are defined and which
arguments are passed to the signals.

"""



###########################################################
#
#  class Match
#
###########################################################


class Match:
    """Let two strategies play one series of iterated prisoners
    dilemmas and record the results.

    signals:
        signalROUND_FINISHED (match)   - sent when one round of the match
                                         has been finished

    flags (read the "state" field, e.g.  m = Match(p1, p2); flag=m.state):
        MATCH_READY             - match has not yet started
        MATCH_RUNNING           - match is currently running
        MATCH_FINISHED          - match has been finished
    """


    def __init__(self, player1, player2, T=5, R=3, P=1, S=0, w=200):
        """Initialize class with player strategies."""

        self.signalROUND_FINISHED = []  # dummy function with
                                        # argument list

        self.player1 = player1          # strategy of player 1
        self.player2 = player2          # strategy of player 2
        self.moves1 = []                # sequence of moves from player 1
        self.moves2 = []                # sequence of moves from player 2   
        self.score1 = 0                 # absolute score of player 1
        self.score2 = 0                 # absolute score of player 2
        self.round = 0                  # number of round
        self.state = "MATCH_READY"

        self.T = T  # reward for defecting if the other player cooperated
        self.R = R  # reward for mutual cooperation
        self.P = P  # "reward" for mutual defection
        self.S = S  # "reward" for cooperation if the other player defected
        self.w = w  # number of rounds


    def score(self, m1, m2):
        """Return the score of player one for the moves m1 and m2 of
        player one and player two respectively. In order to determine the
        score of player two just swap the parameters m1 and m2."""

        if (m1, m2) == (0, 1):      return self.T
        elif (m1, m2) == (1, 1):    return self.R
        elif (m1, m2) == (0, 0):    return self.P
        else:                       return self.S


    def nextRound(self):
        """Let the players meet and update the respective variables with
        the results. Then, determine if iteration stops after this round
        (random > w). Return 0 if it does, 1 otherwise."""

        self.round += 1
        
        m1 = self.player1.nextMove(self.round, self.moves1, self.moves2)
        m2 = self.player2.nextMove(self.round, self.moves2, self.moves1)

        self.moves1.append(m1)
        self.moves2.append(m2)

        self.score1 += self.score(m1, m2)
        self.score2 += self.score(m2, m1)

        if self.round >= self.w:    return 0
        else:                       return 1


    def match(self):
        """Play one match. Call notifier function after each round."""

        self.state = "MATCH_RUNNING"

        cont = 1
        while cont:
            cont = self.nextRound()
            for s in self.signalROUND_FINISHED: s(self) # callback

        self.state = "MATCH_FINISHED"




###########################################################
#
#  class Tournament
#
###########################################################


class Tournament:
    """Set up and play a tournament.

    signals:
        signalENLIST_PLAYER (tournament, player)
            a new player has been enlisted
        signalNEW_MATCH (tournament, match)
            a new match has just begun
        signalMATCH_COMPLETED (tournament, match)
            match "match" has been completed
        signalTOURNAMENT_FINISHED (tournament)          
            the tournament has been finished

    flags (read the "state" field, e.g.  t = Tournament(); flag=t.state):
        TOURNAMENT_READY        - match has not yet started
        TOURNAMENT_RUNNING      - match is currently running
        TOURNAMENT_FINISHED     - match is finished
    """


    def __init__(self, name = "Tournament", T=5, R=3, P=1, S=0, w=200):
        """Initialize class with match notifier and 
        identification string."""

        self.signalENLIST_PLAYER       = []
        self.signalNEW_MATCH           = []
        self.signalMATCH_COMPLETED     = []
        self.signalTOURNAMENT_FINISHED = []
        
        self.name = name    # identification string
        self.player = []    # list of players
        self.match  = []    # list of matches
        self.score = {} # dictionary of player scores (indexed by players)
        self.state = "TOURNAMENT_READY"

        self.T = T  # Parameters defining the reward (or punishment resp.)
        self.R = R  # for cooperation or defection.
        self.P = P  # These parameters are directly passed to Match.__init__()
        self.S = S      
        self.w = w  # number of rounds



    def __cmpfunc(self, p1, p2):
        if self.score[p1] > self.score[p2]:     return -1
        elif self.score[p1] == self.score[p2]:  return 0
        else:                                   return 1


    def enlistPlayer(self, newPlayer):
        """Adds a new Player to the tournament."""

        self.player.append(newPlayer)
        for s in self.signalENLIST_PLAYER: s(self, newPlayer)


    def runTournament(self):
        """Run the tournament.

        Let each player play once against every other player.
        Invoke signalNEW_MATCH before a new new match is started and
        signalMATCH_COMPLETED after it is finished. When all matches are
        completed, calculate the scores of the players and sort
        self.player according to the players rank in the tournament."""

        self.state = "TOURNAMENT_RUNNING"

        self.match = []
        for p in self.player: self.score[p] = 0

        for i in range(len(self.player)):
            for k in range(i+1, len(self.player)):
                p1 = self.player[i]
                p2 = self.player[k]

                m = Match(p1, p2, self.T, self.R, self.P, self.S, self.w)

                self.match.append(m)
                for s in self.signalNEW_MATCH: s(self, m)
                
                m.match()
                self.score[p1] += m.score1
                self.score[p2] += m.score2
                for s in self.signalMATCH_COMPLETED: s(self, m)

        self.player.sort(self.__cmpfunc)

        self.state = "TOURNAMENT_FINISHED"
        for s in self.signalTOURNAMENT_FINISHED: s(self)

\end{verbatim}
\end{scriptsize}

\newpage

\subsection{Strategy.py}

\begin{scriptsize}
\begin{verbatim}

"""Some basic strategies."""


import whrandom


class Strategy:
    """An abstract class for a player strategy.

    Any concrete player strategy in the tournament is a child
    class of this one."""

    name = "no strategy"


    def nextMove(self, round, myMoves, opMoves):
        """Determine the next move (either 1 to coperate or 0 to defect)
        based on the sequences of all previous moves.

        Parameters:
        round     - Number of the current round starting with 1
        myMoves   - List of all previous moves of this strategy in this match
        opMoves   - List of all previous moves of the opponent strategy
        """
        pass



###########################################################
#
#   Trivial strategies
#
###########################################################


class AlwaysFriendly(Strategy):
    """This strategy never defects."""

    name = "Always friendly"

    def nextMove(self, round, myMoves, opMoves):
        return 1                                        # never defect



class UtterlyDestructive(Strategy):
    """This strategy always defects."""

    name = "Utterly destructive"

    def nextMove(self, round, myMoves, opMoves):
        return 0                                        # always defect



class Random(Strategy):
    """This strategy chooses its moves at random."""

    name = "Random"

    def nextMove(self, round, myMoves, opMoves):
        return whrandom.randint(0, 1)                   # play randomly




############################################################
#
#   simple strategies
#
############################################################


class TitForTat(Strategy):
    """Play cooperatively only if the other player did 
    so in the previous round. Start friendly."""

    name = "Tit for tat"

    def nextMove(self, round, myMoves, opMoves):
        if round == 1:
            return 1                                    # start friendly
        else:
            if opMoves[-1] == 1: return 1
            else:                return 0



class TitForTwoTats(Strategy):
    """Play friendly if the oponent did not defect in the
    two previous rounds. Start friendly."""

    name = "Tit for two tats"

    def nextMove(self, round, myMoves, opMoves):
        if round <=  2:
            return 1                                    # start friendly
        else:
            if (opMoves[-2:] == [0,0]): return 0
            else:                       return 1



class MassiveResponse(Strategy):
    """Play 'Tit for Tat' but punish twice for every single defection of
    the opponent. Start friendly though."""

    name = "MassiveResponse"

    def nextMove(self, round, myMoves, opMoves):
        if round == 1:
            return 1                                    # start friendly
        elif round == 2:
            if opMoves[-1] == 0:      return 0          # Tit for Tat
            else:                     return 1
        else:
            if opMoves[-2:] != [1,1]: return 0          # Massive response
            else:                     return 1



class Cheater(Strategy):
    """Play friendly if the oponent did not defect in the two
    previous rounds (like 'Tit for two Tats'). But try to cheat
    (play destrcutive) every 7th round."""

    name = "Cheater"

    def nextMove(self, round, myMoves, opMoves):
        if round <= 2:
            return 1                                    # start friendly
        else:
            if (round % 7) != 0:                        # play Tf2T usually
                if (opMoves[-2:] == [0,0]): return 0
                else:                       return 1
            else:                                       # but cheat sometimes
                return 0



class GraciousTfT(Strategy):
    """Play 'Tit for Tat', but play cooperatively (as an offer of peace),
    if there have already been five rounds of mutual defection (or
    alternating defection and cooperation) in sequence."""

    name = "GraciousTfT"

    def nextMove(self, round, myMoves, opMoves):
        if round == 1:
            return 1                                      # start friendly
        elif round > 6 and ((opMoves[-5:] == [0,0,0,0,0] and \
                             myMoves[-5:] == [0,0,0,0,0]) or \
                            (opMoves[-5:] == [0,1,0,1,0] and \
                             myMoves[-5:] == [1,0,1,0,1])):
            return 1                                      # peace offer
        else:
            if opMoves[-1] == 1: return 1                 # play tit for tat
            else:                return 0





class MaliciousTfT(Strategy):
    """Play 'Tit for Tat' but start unfriendly."""

    name = "MaliciousTfT"

    def nextMove(self, round, myMoves, opMoves):
        if round == 1:
            return 0                                    # start unfriendly
        else:
            if opMoves[-1] == 1: return 1
            else:                return 0



class DelayedTfT(Strategy):
    """Play friendly if the opponent did so three moves before, otherwise
    do not cooperate. Play friendly at the beginning."""

    name = "DelayedTfT"

    def nextMove(self, round, myMoves, opMoves):
        if round <= 3:
            return 1
        else:
            if opMoves[-3] == 1: return 1
            else:                return 0




############################################################
#
#   slightly more comples strategies
#
############################################################


class Tester(Strategy):
    """Defect in the first round in order to test
    the opponents reaction.  Based on the opponents reaction, play either
    Tit for Tat (starting friendly) or try to deceive opponent by playing
    cooperatively in the second and third round and then defecting every
    second round. (See Axelrod, ch. 2, p.40)"""

    name = "Tester"

    def nextMove(self, round, myMoves, opMoves):
        if round <= 2:
            return 0
        elif round == 3:
            if opMoves[-1] == 0:  self.state = "TFT"
            else:                 self.state = "Deceiver"
            return 1
        elif round == 4:
            return 1
        else:
            if self.state == "TFT":
                if opMoves[-1] == 1:  return 1
                else:                 return 0
            else:
                if round % 2 == 1:    return 0
                else:                 return 1


        
class Analyst(Strategy):
    """Play random for the first ten rounds. Then try to analyse 
    the opponents strategy based on the opponents reactions. If either
    the opponent could be exploited very  well or if the opponent 
    did attempt to exploit this strategy to often, play non-cooperatively.
    Play cooperativly, if it wasn't possible to exploit the opponent
    and if the opponent played fair as well."""

    name = "Analyst"

    def nextMove(self, round, myMoves, opMoves):
        if round <= 10:
            return whrandom.randint(0, 1)   # play random at the beginning
        else:

            # analyse

            ex_attempt, ex_success = 0,0
            opex_opportunity, opex_attempt = 0, 0

            i = -9
            while i <= -1:
                if myMoves[i-1] == 0:
                    ex_attempt += 1
                    if opMoves[i] != 0: ex_success += 1 # opponent did
                                                        # not punish exploit!
                else:
                    opex_opportunity += 1
                    if opMoves[i] == 0: opex_attempt += 1  # opponnent played
                                                           # defective without
                                                           # reason
                i += 1

            # and react accordingly

            ret = -1
            if (ex_attempt > 0):
                if (float(ex_success) / float(ex_attempt)) >= 0.6:
                    return 0                    # keep exploiting
                else: ret = 1                   # try to be friendly again

            if opex_opportunity > 0:
                if (float(opex_attempt) / float(opex_opportunity)) <= 0.4:
                    return 1            # opponent isn't really bad
                else:
                    return 0            # opponent tried to deceive to often
            else:
                if ret != -1: return ret      # fallback
                else:
                    if opMoves[-1] == 1: return 1   # play TfT if clueless
                    else:                return 0

\end{verbatim}
\end{scriptsize}

\end{document}
