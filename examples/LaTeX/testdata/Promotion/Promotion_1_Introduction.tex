

\chapter{Introduction}

\pagenumbering{arabic}
\setcounter{page}{7}

In this book I examine evolutionary explanations of altruism that are based
on computer simulations. When speaking of {\em explanations} of
altruism, this means that this book is not primarily a study that tries to explain
altruism itself, but a critical examination of how these explanations work.
Its aim is twofold: On the one hand, it will expound this type of
explanations of altruism, describe its working mechanisms and the results that
can be obtained. In this respect this book strongly draws on the simulation
based approach to the evolution of altruism that was pioneered by Robert
Axelrod and William D. Hamilton \cite[]{axelrod-hamilton:1981} and made
popular through Axelrod's book on ``The Evolution of Cooperation''
\cite[]{axelrod:1984}. However, after the more than twenty years that have
passed since the publication of this book, the fact can hardly be ignored that
the simulation-based approach to the explanation of altruism did not quite
live up to the very expectations and aspirations that it once gave rise to and
to the ``simulation hype'' it caused. Therefore, this book will on the other
hand broadly discuss the limits of this approach. My aim is to give a
clear diagnosis of this failure, to explain why this approach remained largely
unsuccessful and also to point out what lessons regarding the research design of
computer simulations can be learned in order to allow a more purposeful employment of computer simulations for scientific explanations in the future.

In this introduction, I first say a few words about the topic and
theoretical background that is, about why the evolution of altruism is a topic
that interests us, why an evolutionary approach may be suitable to tackle the
question of altruism and, finally, how computer simulations come into play
here. Then, I briefly explain my method for examining the simulation-based evolutionary explanations of altruism and its alleged failure.
Basically, my method consists in conducting some simulations in the common
fashion of this approach myself and looking at the corresponding empirical
research both in biology and in the social sciences. I also give in this
introduction a very brief overview of the main results of my inquiry.
Finally, I present the structuring of this book and, in this context,
further describe the methodological decisions I have taken.

\section{The explanation of altruism as a scientific problem}

The explanation of altruism poses an intriguing riddle both in biology and in
the social sciences. In biology the question is how, if ``survival of the
fittest'' is the rule, altruistic behavioral traits can evolve when altruism
means by definition the giving-up of some of an organism's own fitness in
order to increase the fitness of another organism. Yet, as ants, honeybees or
the behavior of brood care in almost any species testify, altruism does exist
in nature. How then, did it arise?

Similarly, while we all believe that humans are moral creatures that can by
proper education and appropriate incentives learn to behave as altruists, the
question still remains why, if -- as we observe in many areas of life --
egoism is the road to success, altruistic norms should continue to enjoy a
high and general esteem. Should not a lack of secular success of the adherents
of altruistic norms mark such norms as unrealistic if not foolish?

Moreover, altruism raises not only important questions in the empirical
sciences, but also for moral philosophy and metaphysics.  For, when we
postulate altruistic moral norms we surely want to know (if we are not pure
{\em Gesinnungsethiker}) whether and to what degree we can realistically
expect obedience to these norms. From a metaphysical perspective the question
of the viability of altruism links to the old question of whether the world as
a whole is good or bad and, if bad, whether it can be made any better or if we
will have to cope with the fact that ``the realm of virtue is not from this
earth'' \cite[]{schopenhauer:1831}.

Thus, the existence of altruism demands an explanation and the desirability of
altruism calls for an understanding of the circumstances under which altruism
can flourish. In this book an examination will be made as to what an
evolutionary simulation-based approach can contribute to the understanding of
altruism.

\section{Method and central theses}

Why use an evolutionary approach for the explanation of altruism? In biology
the answer to this question is obvious: Any phenotypic trait of any organism
must -- according to Darwin's theory of evolution -- have evolved through
natural selection. If a certain organism or species exposes an altruistic
behavioral trait then there must be an evolutionary explanation for it. The
situation is different in the social sciences.  As is usual in the social
sciences there exist many competing paradigms upon which a scientist could
draw in order to explain the genesis of social norms, including norms that
prescribe altruistic conduct. The evolutionary theory of culture which seeks
to apply the principles of the Darwinian theory of evolution (reproduction,
variation, selection) to the evolution of cultural traits is a comparatively
young contender. Its practical value for the social sciences is still
disputed\footnote{See Bryant \cite[]{bryant:2004} for a fundamental criticism
  of the evolutionary theory of culture.}  and, due to the fact that there exist
many good alternative explanations for cultural developments, it would
be too much to expect that the evolutionary theory of culture could repeat in
the social sciences the very success that Darwinism had in biology. Yet, there are
some good points in favor of it. First of all, the evolutionary theory of
culture may prove able to explain things that other theories of cultural
developments cannot explain.\footnote{See Arnold \cite[]{arnold:2002} for some
  speculations on this topic.} Then, where it proves able to explain cultural
developments, it most probably can provide general patterns of explanation
that can be applied both in biology and in social sciences.
% \footnote{I am
%   speaking here of ``patterns of explanation'' instead of just
%   ``explanations'' because the generality of these patterns lies on a purely
%   epistemic level.  Ontologically, genetic evolution in biology and cultural
%   evolution in the social sciences relate to a different set of causes. Only,
%   the same or very similar abstract models of evolutionary processes may be
%   applied in both areas.} 
If the evolutionary theory of culture should prove
to be successful then it could be regarded as a great advance in terms of the
economy of knowledge. Finally, explanatory patterns that cover different areas
of research may profit from synergistic effects, which means that an
advancement of modeling or empirical research in one of the fields may carry
over to the other fields.

However, there are also downsides to such a generalizing approach.  Most
notably there is the danger of overlooking peculiarities of the respective
areas of research and, as always with generalizing, there is the danger of
oversimplification. Ultimately, the choice to use an evolutionary approach to
study altruism is -- as far as the social sciences are concerned -- to some
degree a matter of preference and motivated by the desire to find an
explanation for altruism as broad as possible.

Given that it has been decided to use an evolutionary approach to study
altruism the next question would be why computer simulations should be
employed to furnish the evolutionary research on altruism.  In principle,
there would be four different alternatives: 1) One could rely on purely verbal
reasoning to explain the evolution of altruism.  But then, verbal evolutionary
explanations tend to be notoriously weak. It is almost always possible to
construct some sort of evolutionary story of why some certain trait had to
evolve and often it is just as easy to explain on the same level why its
opposite should have evolved (even if in fact it did not) if only because it
is usually easy to feign some plausible selective conditions under which the
trait in question would be advantageous. 2) Another alternative is
mathematical modeling. It allows -- as one should presume -- for a very
precise expression of the concepts in question, but it can easily become
extremely complicated and tedious, once it rises above the mere expression of
the concept of, say, reciprocal altruism to models that can halfway
realistically depict a situation in the real world where altruism
evolved.\footnote{See Boorman and Levitt \cite[]{boorman-levitt:1980} for a
  comprehensive treatment of the mathematical modeling on the genetics of
  altruism. It seems that Boorman and Levitt received comparatively little
  attention in the philosophical literature on the evolution of altruism. 
  This may be due the difficulties for most readers to understand
  the mathematical presentation or to the fact that computer simulations of
  altruism have become so popular in the meantime.} 3) The latter problem can
potentially be addressed by numerical models, which class includes also the
computer simulations of altruism. Computer simulations are an extremely
flexible, easy to use and powerful tool. Of course all computer simulations
rely on mathematical background theories such as, for example, evolutionary
game theory. In this sense there does not really exist an opposition between
computer simulations and mathematics but rather a dependency. But with
computer simulations it is easily possible to go beyond what can be modeled
in purely mathematical terms.
% Yet, there can be a tension between these two approaches in case that
% computer simulations are used to produce results that could easily be
% derived mathematically\footnote{This is one of the (implicit) charges in Ken
%   Binmore's criticism of Axelrod \cite[]{binmore}.} or in the other case
% that laborious mathematical proves are given where a simple calculation
% would fully suffice.
Because of their ease and power, computer simulations seem to have been
regarded by many as the tool of choice for the study of the evolution of altruism.
4) Last but not least, there is the empirical approach to altruism, which roughly means
looking at empirical instances of potentially altruistic behavior and drawing
inferences about these by means of common reasoning.

In principle, the empirical approach should not be regarded as an
alternative to the theoretical approaches described above.  For, any
systematic empirical research must be guided by theories or at least
theoretical preconceptions about the subject matter. In turn, the models and
theories should of course be tested against empirical data.  However, in
practice there really exist two approaches with quite a different style and
flavor to each of them.  The empirical approach is a ``bottom up'' approach,
where scientists start with empirical observations and gradually develop more
and more complex models to account for them.  The theoretical approach (as
opposed to the empirical approach) is what could be called a ``top down''
approach, where scientists start with theoretical considerations and models
and then (hopefully) adjust them to the empirical instances that these are to
be applied to.  Unfortunately, in the case of the research on altruism there
exists a wide gap between the theoretical and the empirical
research\footnote{See Dugatkin \cite[]{dugatkin:1998} for a discussion of this
  problem.}. From the vast amount of computer simulations on altruism
produced, hardly any has been successfully applied in empirical research. 
Partly, this gap is due to the division of labor in science, where one group
of scientists develops the models and another group does the empirical
research. But this alone cannot explain why there is such a lasting
discrepancy between the computer simulation based theories and the empirical
research.\footnote{See Hammerstein \cite[]{hammerstein:2003} for a vivid
depiction of this discrepancy.} The discussion of this problem, the understanding of its causes and the consequences that should be drawn form the central topic of this book.

In the course of this book, I look at both computer simulations
and empirical research in order to examine this question. Purely
mathematical models of altruism will not be discussed. The reasons for
leaving them out are primarily of pragmatic nature. The
epistemological questions concerning mathematical models are not exactly the
same as those concerning computer simulations, although presumably many of the
results about the epistemology of computer simulations arrived at in this
book will also hold true for purely mathematical models. Also, the just
mentioned problem of a strong discrepancy between theoretical
modeling and empirical research in the study of the evolution of
altruism seems to be even more glaring in the case of computer
simulations if only because the use of computer simulations makes
modeling much easier and more powerful so that the mere popularity of
this tool has exposed dangers that are already imminent in
purely mathematical modeling.

In order to better understand how computer simulations of the evolution of
altruism work, several simulations and simulation series in the Axelrod-fashion
will be carried through to simulate different kinds of altruism. There are
basically three different kinds of altruism: Reciprocal altruism, kin
selection and group selection. Most simulations will be done on reciprocal
altruism and some on group selection. For the sake of completeness, kin
selection will also briefly be discussed but not simulated. Although they are intended to illustrate the use of a certain method rather than to be particularly original,
the simulations presented here are new in the sense that they are not merely repetitions of
computer simulations that have already been carried out and described in the
scientific literature on the subject. It is, however, one 
of the main points to be established in this book that {\em the results of such purely theoretical simulations (be they as new or unique as they may) are typically not of
  great scientific relevance}.\footnote{The reason why I do not think they are
  is explained in chapter \ref{summaryReciprocalAltruism}.}

Just how irrelevant very many of the models of reciprocal altruism are becomes
obvious when they are held against the empirical research on altruism. No
empirical research has been done specifically for this book. Instead I review
some of the empirical research that has been done in biology and in the social
sciences, especially in behavioral economics. Not being a specialist in
either biology or economics I am quite aware of the dangers involved with
reviewing the results of branches of science that one can at best claim to
have a laymans knowledge of.\footnote{My field of specialization is political
science. Regarding political science, however, I seriously doubt that computer
simulations of the evolution of cooperation can provide us with
any important insights beyond mere trivialities. See Arnold
\cite[]{arnold:2005b} for an extensive criticism of this approach, which also
contains {\em in nuce} some of the arguments that have been expounded in
greater detail in this book. In this scepticism regarding the value of mathematical models for political science I feel strongly confirmed by the criticism of the rational choice approach as applied to the political sciences by Ian Shapiro and Donald Green \cite[]{green-shapiro:1994, shapiro:2005}, which unfortunately I had not been acquainted with at the time of writing this book.} The dangers include misunderstanding, misrepresenting, mistaking the
inessential for the essential etc. But these are problems that any kind of
interdisciplinary research faces. The only secure way to avoid these dangers
would be to refrain from interdisciplinary research altogether or to ignore
scientific results in philosophy, neither of which can seriously be considered
an option. To the extent to which the more recent scientific research in the
two above mentioned fields has found its way into textbooks it is still fairly
easy to access. Therefore, I have tried as far as possible to rely on this
kind of scientific literature. However, the latest research can only be found
in articles in scientific journals. As far as these are concerned, I can only
say that I have tried to report the content of the articles that I have quoted
as faithfully and accurately as I could as a layman.
% The rather
% critical conclusion that I draw about the scientific value of the majority of
% the simulations on the evolution of cooperation from looking into the
% empirical research may not be uncontested, but it is (with different emphasis)
% also shared by some of the authorities in the field of evolutionary game
% theory \cite[]{hammerstein:2003} \cite[]{dugatkin:1998} so that I feel that I
% am probably not completely mistaken about it after all.

Having shown by examining the empirical research that computer models of the
evolution of cooperation or altruism can tell us only very little about how
altruism evolves, this naturally raises the question why they failed to do so.
My answer to this question, which is at the same time my central thesis,
generalizes from the simulations of the evolution of altruism and states that
{\em the main reason why computer simulations often fail to fulfill their
  expectations in science is that the epistemological conditions under which
  they can possibly explain or prove something are not yet well understood}.
Computer simulations are still a relatively new tool in science so that ``best
practices'' for their design or employment are only beginning to emerge. There
still seem to exist quite a few insecurities as to how computer simulations
can be used properly in the context of scientific explanations. At any rate,
the ``tradition'' of Axelrod-style simulations of the evolution of cooperation
seems to have gone astray if the aim really was to explain how cooperation or
altruism evolves. That a whole school or ``tradition'', if I may call it so,
of science is going amiss may be due to the fact that the very business of
science sometimes proceeds in an astonishingly naive if not narrow-minded way.
In this case, Axelrod had set with his computer simulations a seemingly
successful new role model for the study of the evolution of cooperation. What
could have been more advisory for aspiring scientists in this field than to
pick up Axelrod's model, change it here and there a little bit or even
challenge it by designing a similar model that would lead to divergent
conclusions and thus produce fascinating new results about the evolution of
altruism?  And it was so easy: One only needed to know a little bit about
computer programming and one could do research on ``the evolution of
cooperation''. (Even philosophers could do that!)  Now, the naivety with which
science sometimes proceeds -- and it certainly proceeded too naively in this
case -- is to some degree to be excused because if one wants to examine some
subject matter one cannot for (economical reasons) at the same time occupy
oneself too much with the examination of the method of the examination of the
same subject matter. But if this is true then it surely is a philosopher's job
to make up leeway and to reflect on what science does and whether it does right what it does. Therefore, the final and most important part of this book is
dedicated to the discussion of the epistemological conditions under which
computer simulations can be used in the context of scientific explanations.
Just as we demand from ordinary scientific theories that they be empirically
testable before we grant them the honorable status of a ``scientific'' theory
(that is a theory that can potentially explain certain empirical phenomena),
{\em we need criteria for computer simulations that allow us to classify
  computer simulations into those for which it can (empirically) be decided if
  they simulate correctly and those for which this cannot be done.}  The
criteria I am going to propose in this book are those of {\em empirical
  adequacy}, {\em robustness} and {\em non triviality}.  ``Empirical
adequacy'' means that all causal factors that have a significant impact on the
outcome of the simulated process are somewhere represented in the simulation.
``Robustness'' requires that the output of the simulation is stable within the
range of measurement inaccuracy of the input parameters.  And ``non
triviality'' simply requires that the output of the simulation gives us some
important information about the outcome of the simulated empirical process.
(The last criteria may seem trivial or self-evident itself, but unfortunately
experience has shown that this is not the case.\footnote{To me it seems that
  the sort of computer simulations that Brian Skyrms devised for the study of
  the ``social contract'' \cite[]{skyrms:1996} or ``social structure''
  \cite{skyrms:2004} are trivial to a point where they must be regarded as
  mere toys. It would be very difficult to draw from his simulations any
  tenable conclusions with regard to the subject matter of political order
  (social contract) or social structure that they are allegedly related
  to. (For a criticism of Skyrms see Arnold \cite[]{arnold:2005b}.) A similar
  objection holds for Schüßler's simulations of cooperation on ``anonymous
  markets'', only that Schüßler is at least aware of the problem and honest
  enough to discuss it \cite[p.\  91f.]{schuessler:1997}.}) These criteria
raise the bar for ``explanatory simulations'' quite high and it will be
discussed at some length if such strict criteria are really necessary.  But if
they are more or less accepted then it follows that the sort of example
simulations that have been presented in this book to demonstrate the
principle of Axelrod-style simulations, and with them very many of the
simulations published in the literature on the evolution of altruism must be
rated as insufficient if any explanatory claim would be based on them.  This
is quite in accordance with the lack of empirical success of the simulation-based approach to altruism mentioned earlier. But with the above mentioned
criteria at hand we can better understand just why most of the computer models
of the evolution of altruism had to fail.

Once the epistemological conditions for the proper application of
computer simulations in an explanatory context are well understood, it
is not only possible to soundly criticize the misguided use of
computer simulations. It is just as well possible to derive guidelines
of how to design and use computer simulations properly. In order to supplement
the critical discussion of what I consider to be a failure of computer
simulations with a positive outlook for the
future, I offer my own proposal for such guidelines in form of a few
simple recipes that scientists can follow if they want to be assured that
their simulations are epistemically valid.

\section{On the structure of this book}

The book is organized into four parts. In the first part (chapter
\ref{altruismRiddle} and chapter \ref{generalizedEvolution}) I explain why
the existence of altruism, which is a fact of the natural as well as the social
world, poses a scientific and philosophical problem. Furthermore, I give a
definition of altruism that is broad enough for both biology and the social
sciences and I justify this definition at some length. The first part
closes with an exposition of the ``generalized theory of evolution''
\cite[]{schurz:2001}, which constitutes the greater theoretical context into
which the following models of the evolution of altruism can be
integrated.\footnote{Of course the models of the evolution altruism do not
necessarily need to be understood in the context of a {\em generalized} theory
of evolution. For example, as long as we only talk about altruism among animals
it would suffice to interpret them against the background of the theory of
evolution in biology. But as evolutionary explanations of altruism can be given
both in biology and in the social sciences a generalized theory of evolution that
does not confine itself to genetic evolution alone provides a very suitable
paradigmatic background.} Because the application of evolutionary theory
outside the field of biology is a controversial issue, the different
flavors of theories of cultural evolution will be discussed at some length.

In the second part (chapter \ref{modeling}) the three basic evolutionary
explanations of altruism will be explained and the modeling on the evolution
of altruism will be discussed. The presentation of a whole field or branch of
science always raises a certain methodological question: Should one rather
give an extensive but in its details necessarily sketchy overview over the
whole field or should one present and discuss a few select examples ``pars pro
toto'' in all detail.  I have taken the second approach and will present a few
self-made computer simulations in order to demonstrate how this type of
modeling works in detail.  Of course, I could also have taken models that were
described in articles in scientific journals. But usually the description in
journal articles does not present all the details of a simulation, hardly ever
is the source code of the simulation software given and often the information
is too sketchy to reconstruct the simulation in an unambiguous way. Also,
programming simulations on one's own is quite an instructive exercise. It
allows one to notice how many ad hoc decisions enter into the construction of
a simulation.  By presenting the computer simulations and their results in
detail it will be possible to point out both the usual working mechanisms of
such simulations as well as the common traps and pitfalls of simulations. The
description of these (as I hope) paradigmatic example simulations will be
supplemented by a review of a selection of the simulations of the evolution of
altruism published in the respective literature. The discussion will cover all forms of evolutionary altruism that is, reciprocal
altruism, kin selection and group selection.  The greatest emphasis is laid on
reciprocal altruism as this is the type of altruism for which the method of
computer simulations has been used the most excessively. As will become
apparent from the discussion of the simulations conducted by myself as well as
those published in the literature on the subject, there is an arbitrary large
space of logical possibilities that could be explored by simulations while at
the same time hardly any generalizable results can be derived from simulations
alone. The reason why all three forms of altruism are covered even though
reciprocal altruism would arguably have sufficed to prove the point against
the method computer simulations is that these different forms of altruism do
often not appear strictly separated in the empirical literature on the subject
(if only because it is often very difficult to tell apart the different forms
of altruism in an empirical context) and it would otherwise be difficult to
compare the simulation studies with the empirical research.

% Driter Teil: Empirische Forschung

In the third part (chapter \ref{empiricalResearch}) of this book the results
of the computer simulations will be contrasted with the empirical research on
the evolution of altruism. It is here where it becomes most obvious that a
wide gap exists between the simulation research and the assumptions about the
evolution of altruism based on it on the one hand and the empirical research
on the other hand. Again, when presenting the results of the empirical
research on the evolution of altruism, a similar methodological issue as in
the case of the presentation of the simulation research arises. Should one
rather give a broad overview of the research or should one discuss only a few
exemplary studies in detail. I have tried to combine both approaches and
therefore give a broad -- though for the sheer size of the topic necessarily
incomplete -- overview of the empirical research (in biology) first.  
This way the fact can be assessed that cases where empirical researchers could make good use of the results of simulation studies on the evolution of altruism are extremely rare.
In order to understand just why they are so rare, I pick out some
examples (both from biology and from social sciences) and discuss them in
detail. Since I am going to make a case against the simulation based approach,
I was careful to pick out examples that could (at the time of their
publication) be considered as showcases for the application of the results of
simulation based research to empirical problems of the evolution of
altruism. If these fail then the simulation based approach in its present form
is confronted with a serious problem. And they do fail, as I hope to be able to
demonstrate.

Turning from the diagnosis of failure in the third (and partly already the
second part) of this book to the explanation of the failure in the fourth part
(chapter \ref{limitsOfModeling}), I propose and discuss the above
mentioned criteria for ``explanatory simulations''. It can easily be seen that
hardly any of the simulations on the ``evolution of cooperation'' meets these
criteria. It is more difficult to show that the fulfillment of these criteria
is both necessary and sufficient for a computer simulation to claim
explanatory power in a scientific context. Since the epistemology of computer
simulations is a relatively young field in the philosophy of science with many
open questions, I can hardly maintain to have found the definite answer to the
question of potentially explanatory qualities of computer simulations. The
fourth part therefore has more or less the character of a philosophical
discussion that is, I try to defend these criteria as good as possible
against conceivable objections. Given that the proposed criteria provide at
least a reasonable guidance, I finally turn to practical considerations and try
to devise some ``recipes'' for the proper use of computer simulations in a
scientific context.

In a short concluding chapter the results of this book will be summed up.
The main results are that the simulation based approach to the study of the
evolution of altruism was largely a failure. This failure resulted from a lack
of understanding of the epistemological conditions and requirements of the
employment of computer simulations in the context of scientific explanations.
Yet, if carefully applied, computer simulations can be a very valuable tool of
scientific research. Regarding the requirements of ``good'' computer
simulations, I have made a few proposals in the last part of my book. These
may or may not prove sufficient and practical in the future, but if I was able
to convey a sense for the necessity to take epistemological considerations
into account for a proper research design of simulation based research, then
my attempts have not been wholly futile.




