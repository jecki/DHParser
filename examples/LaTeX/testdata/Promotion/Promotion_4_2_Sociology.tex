\newpage

\section{Empirical findings in the social sciences}
\label{sociology}

Empirical research on cooperation and altruism in the social sciences falls
roughly into two different categories. One part of the empirical research
consists of laboratory experiments, where the predictions of game theory are
tested by letting subjects play different types of cooperation games. For this
type of research subjects are placed in highly stylized and artificial
laboratory situations. This allows creating situations which are somewhat
similar to the highly abstract settings presupposed by mathematical models or
computer simulations of cooperation. Although the laboratory experiments are
usually not designed to match a particular model\footnote{They are rather
  designed with certain research questions in mind, taking into account the
  pragmatic restrictions of the laboratory and not always strictly relating to
  theoretical results.}, they allow for some degree of comparison between
theoretical results and empirical reality. Following as before the pars pro
toto approach, I discuss two selected examples of this type of research
and highlight the epistemological issues involved.
% In many cases the experimental results deviate
% significantly from what game theory based on a strict rational actor
% model predicts. Different hypotheses have been invented to explain
% the often ``more than rational'' level of cooperation among
% humans. One such hypothesis is the maladaption hypothesis which states
% that the relatively high level of cooperation among humans evolved in
% the hunter gatherer stage of human history, where it was indeed
% rational for individuals living in small groups of relatively closely
% related individuals to cooperate, but has to be counted as a
% maladaption in the mass societies of today. Apart from the obvious
% uninspiredness of this argument, the question would be, if not the
% seemingly irrational behavior of humans reflects some deeper
% evolutionary meaning, not yet covered by our game theoretical
% models.

The other and more important part of empirical research on cooperation would
be real world examples that potentially expose the patterns of cooperation
predicted by the theory. In spite of the extreme popularity of Axelrod's book
on the ``Evolution of Cooperation'' \cite[]{axelrod:1984}, there exist only
relatively few empirical field studies that make use of the theory of the
evolution of cooperation which is based on the repeated Prisoner's
Dilemma.\footnote{For an overview of the literature that relates to Axelrod's
  theory see \cite[]{axelrod-dambrosio:1994}, \cite[]{gotts-polhill-law:2003} or
\cite[]{hoffmann:2000}. It is characteric that the the only empirical 
application scenario that the latter quotes is the ultimately failed attempt of 
Milinski to interpret the predator inspection behavior of sticklebacks in terms of 
the repeated Prisoner's Dilemma (see also chapter \ref{sticklebacks}). All three surveys 
strengthen the impression that the modeling business is mostly self-contained and quite 
detached from the empirical research.} Usually, such studies rather
draw a sort of general inspiration from the ideas related to the repeated
Prisoner's Dilemma model than relate to any simulation models in
particular. But then -- as has already become apparent in the biological case
-- the respective computer simulations are not really suitable for empirical
application. For, it is often close to impossible to measure the relevant
parameters, to exclude interferences of coefficients not captured by the
theory or just to ascertain which kind of game is played in a given situation.
In order to show what difficulties are involved when one tries to apply the
results of computer simulations to real world problems, I discuss the
application of the reiterated Prisoner's Dilemma model of the evolution of
altruism to the ``live and let live'' system that evolved in the First World
War among soldiers of opposing armies.  This example is particularly well
suited for demonstrating the epistemological issues involved in the use of
simulation models in the context of empirical research because it has
originally been advanced as a showcase to demonstrate the power of the
simulation based approach to the study of the evolution of altruism \cite[p.
67-79]{axelrod:1984}. The example will be discussed in depth and it will be
demonstrated that far from being a showcase for the use of simulation models
it exposes some severe limitations of this method. The criticism will be
elaborated thoroughly in the final chapter (chapter \ref{limitsOfModeling}).


\subsection{Laboratory experiments}
\label{laboratoryExperiments}

\subsubsection{The evolution of institutions}
\label{economicsInstitutions}

Laboratory experiments usually center around simple ``standard'' dilemma
situations like the Prisoner's Dilemma or a public goods problem. One
particular question that has been examined experimentally is that of how
punishing institutions can evolve. The evolution of punishing institutions is a
riddle because in those situations where a punishing institution would be
needed to solve a dilemma, a new dilemma arises that precludes the evolution
of punishing institutions. One such constellation has been examined
experimentally by Gürerk, Irlenbusch and Rockenbach
\cite[]{guererk-irlenbusch-rockenbach:2006}. They set up an experiment where
subjects interact anonymously in a public goods dilemma for 30 rounds. Each
subject can decide which amount of its income to donate for the provision of a
public good. The return value was such that each subject profited strongly
from the overall contributions, but still had an incentive to let the others
pay the public good and not to contribute himself or herself.  Typically, the
provision of public goods degrades in such a situation after only a few
rounds.\footnote{It should be remembered that the provision of a public good
  is an N-person dilemma. In an N-person dilemma the evolution of reciprocal
altruism faces much stronger barriers than in the 2-person dilemma, though it
has indeed been demonstrated that there exists a theoretical possibility for
conditional cooperation to be stable even in an N-person
game \cite[p.\ 82ff.]{taylor:1987}.}
To make matters more interesting, the subjects could choose to join either of
two groups, one group that was provided with a sanctioning institution and one
that was sanctioning free. After each round, subjects could choose to change
groups. The sanctioning institution worked in the following way: In the group
with sanctioning, each participant was allowed to punish or reward other
participants within the same group. Both punishment and rewards cost the
punishing or rewarding subject one monetary unit. Persons punished would lose
three monetary units, while persons rewarded would gain one monetary unit.
(Punishments and rewards were issued in the same round after the contributions
were made.) The decision to punish or to reward was left entirely at the
discretion of the participants. Since punishment was costly, the provision of
punishment therefore constituted a second level free rider problem.

Although the authors of the experiment were not primarily concerned with
studying evolutionary social processes, their experimental setup does in fact
resemble a kind of group selection scenario with two levels of selection. On
the within-group level selection takes place between a cooperative and a non
cooperative strategy, both of which -- as one could say -- compete for players
adopting them. At the same time a between group selection process takes place
between the sanctioning and the non sanctioning group, which compete for group
members. (See also chapter \ref{culturalGroupSelection}.) However, the
situation does not exactly reflect the group selection model presented
earlier, insofar as the groups do not merely differ in the composition but
also by the presence or absence of the sanctioning institution and because --
as will be seen -- the selection pressure within the sanctioning group does
not counteract the between-group selection pressure as it is assumed in the
theoretically most interesting case of group selection (see chapter
\ref{groupSelection}).

The result of the experiment was that after 30 rounds almost everybody had
joined the group with the sanctioning institution and almost everybody
cooperated almost entirely (i.e.\ donated almost all of the income to the
provision of the public good). About 3/4 of the subjects in the sanctioning
group did exert punishment. (Rewards proved to be far less effective, since
they even had a slightly negative influence on cooperation, supposedly, because
subjects thus rewarded conclude that they have given too much.) Interestingly,
subjects that changed the group quickly adopted the behavior common in their
new group, both with regard to cooperation and non cooperation and with regard to
punishment and refraining from punishment. The main question that this
experiment raises is why punishing behavior did not erode as did cooperation
in the non sanctioning group. Given that punishment is a second order public
good and that it thus raises a free rider problem that is structurally similar
to the original social dilemma situation simulated in the experiment, this
appears quite surprising. Several explanations are possible\footnote{Only some
  of these are discussed in Gürerk's, Irlenbusch's and Rockenbach's paper.}: 1)
Human beings are just not so rational as the theory of public goods assumes.
Therefore, in some instances (second level problems) they provide goods, even
though they would be better off cheating.  But then, why didn't a more
rational mode of behavior evolve if this would entail greater revenues?  2)
The subjects come from a society where certain modes of behavior including
punishment, revenge etc. have -- for whatever reason -- already evolved and
are now transferred by them to the game. 3) There exists a certain amount of
conformism. That is, people imitate other people's behavior and only deviate
if they have a strong incentive to do so. As the necessity of punishment
decreases over time because people that tried to cheat and were punished have
learned their lesson, conformism suffices to uphold punishing behavior before
it can deteriorate (in the end the payoff disadvantage of punishers is
about 2\% compared to non punishing cooperators). In other words, the
costs of punishment become small enough to fall under the conformism
threshold. 4) There is also a possibility that the second level public goods
problem falls into a category where it may even pay for a participant to
provide the good all on his or her own even though nobody else is willing to
share the cost. In order to find out whether the problem of providing costly
punishment falls into this category of public goods problems, it would be
necessary to measure the gain in the provision of the first level public good
that is effected by the successful betterment of reluctant cooperators.

Having thus briefly discussed the results of a typical study of experimental
economics, the question shall now be considered, how this can be related to
simulation models of the kind that have been presented previously (chapter
\ref{modeling}). There are a few things to say regarding this question: The
setup of the experiment does not precisely fit any of the simulation models
presented earlier, neither does it closely resemble any other particular
simulation study that has been published on the evolution of cooperation. It
follows the common pattern of public goods problems as they are also
expressed in the respective models that illustrate the theory of public goods.
Of course it would be easy to draw up a computer simulation that more or less
resembles the experimental setup. But what could the goal and possible benefit
of such an endeavor be? As experiments provide {\em prima facie} stronger
evidence than any simulation, why would anything need to be demonstrated by a
computer simulation that has already been shown experimentally? One might
reason that computer simulations could be helpful for deciding between the
four possible explanations given for the punishment cooperation above. But
this would only be the case if the decision between these alternative
explanations were one that did on one point or other rest on the question
of the mere theoretical possibility of any of these and this is not the case,
except perhaps for the last alternative for which, however, a simple
calculation should suffice. In order to decide between the other three
alternative explanations, further experiments or further measurements would be
required, but not more models.

Still, experiments of economic behavior provide a type of empirical
research where a close fit between model and empirical reality is
comparatively easy to achieve. If indulging in computer simulations of the
evolution of altruism appears so little rewarding because it is so far
removed from any empirical problem of cooperation or altruism -- an impression
that was very much strengthened by the earlier discussion of the empirical
literature on the evolution of altruism in biology -- , experimental economics
finally offers a basis where simulation models of the evolution of altruism
and empirical research can be linked together in a more than merely
metaphorical and story telling way. One might wonder why this should work in
economics but not in biology. The probable reason is that for the simulations
to be applied in biology it would be necessary to measure the reproduction
relevant fitness payoff of certain types of behavior, which obviously is a
task that is extremely difficult to accomplish in most cases. The one
exceptional example given in Dugatkin's comprehensive empirical meta-study
\cite[]{dugatkin:1997} where payoff parameters were actually measured, was an
experimental study about blue jays.  And even in this case the measured payoff
parameters did not resemble a payoff in terms of the reproduction rate (see
page \pageref{bluejays}).

Experimental studies such as the one outlined above can potentially be linked
to computer simulations because they take place in an artificial laboratory
setting that is streamlined and simple enough to reproduce it in a
mathematical model of a computer simulation. But at the same time experimental
laboratory studies raise certain epistemological concerns of their own, which
are similar to those of computer simulations. Regarding computer simulations
of the evolution of cooperation, there exists the problem of transferring the
results of the computer simulations to empirical situations. As has been
demonstrated in the case of biology (see chapter \ref{biology}) this can be a
very difficult problem to solve, especially if the simulations are not
designed to fit empirical problems but merely express more or less plausible
theoretical assumptions. Now, a similar transfer problem exists for
experimental research in economics. For, how are we to know if the behavior
of participants in a laboratory experiment is the same as the behavior of
people in ``real'' life? Typically, the laboratory situations are very much
simplified compared to the real life situations they are supposed to resemble.
Interfering factors such as the psychological factors that drive our behavior
in small group interactions are deliberately excluded by putting the
participants into small closed boxes, where they sit in front of a computer
screen and only receive information about the other participant's choices
without ever getting to see their faces or being able to talk to them.
Furthermore in many of the experimental studies the participants are
university students and not a representative sample of the population. These
few remarks should suffice to indicate that there exists a transfer problem in
the case of experimental economical research as well. It seems that when the
%%% !!!Quelle Rainer Schnell, Hill, E.Esser: Methoden, Laborexperimente
explanatory gap between models and reality is closed by designing experiments
which resemble the models, another gap is opened between experiments and the 
empirical world outside the experiments.

\subsubsection{Trust and cooperation in internet auctions}
\label{economicsTrust}

Can the just mentioned dilemma ever be solved? In fact the dilemma can be
solved in certain special cases. It can be narrowed or closed if 1) either, we
are lucky and find some empirical setting that is indeed simple enough to be
easily compared to laboratory setups, or 2) in cases where
economic institutions have deliberately been designed to match a previously
tried experimental setup. (For example, in order to exploit a certain
experimentally proven effect.) A very prominent example that fits these
conditions is provided by the economic research on the behavior of buyers
and sellers in internet auctions. Internet auctions provide by their very
nature a simple and streamlined setting that strongly resembles that of
laboratory experiments. Furthermore, some of the economists that have studied
the behavior in internet auctions also work as consultants for internet
auction companies like eBay. Therefore, we can also expect that the concrete
procedures of such auctions are to some degree designed according to precepts
learned from economic experiments.

In the following I describe one series of experiments on the behavior of
internet traders that was conducted by Gary E. Bolton, Elena Katok and Axel
Ockenfels \cite[]{bolton-katok-ockenfels:2004}. The problem that their series
of experiments is centered around is that of why internet traders trust each
other. Described in game theoretical terms an internet auction is an
asymmetric one-shot and non zero-sum game. It is asymmetric because it is the
rule that first the buyer sends the money and upon receiving the money the
seller sends the product to the buyer. This means that the seller can cheat,
but not the buyer. If the buyer enters upon the interaction, the buyer must
therefore trust the seller. The game is one-shot because
typically neither the buyer nor the seller have met before, nor will they be
trading partners after the trade has taken place. Finally it is a non
zero-sum game because both the buyer and the seller profit from the
interaction. If they did not, then either the buyer would not bother to enter
upon the interaction or the seller would not offer his product. Bolton, Katok
and Ockenfels model these conditions by assuming that both buyer and seller
retain a payoff of 35 if no transaction takes place. If the transaction takes
place, both buyer and seller receive a payoff of 50. And if the seller cheats
that is if the seller takes the money but does not send the product to the
buyer, then the seller receives a payoff of 70 while the buyer ends up with a
zero payoff. (See figure \ref{trustGame1}.) Except for the asymmetry the situation is thus the same as in
the Prisoner's Dilemma game. Theoretically, no interaction should take place.
For, if both trading partners were rational egoistic utility maximizers, then
the seller would be sure to cheat if an interaction did take place and the
buyer, anticipating the seller's cheating, would not even initiate the
interaction \cite[p.\ 188]{bolton-katok-ockenfels:2004}.

\begin{figure}
\begin{center}
\setlength{\unitlength}{1cm}
\begin{picture}(10,8)(-1,0)
\put(2,7){\makebox(6,1){Buyer}}
\put(5,7){\line(-1,-1){5}}
\put(5,7){\line(1,-1){2}}
\put(4,4){\makebox(6,1){Seller}}
\put(7,4){\line(-1,-1){2}}
\put(7,4){\line(1,-1){2}}

\put(2,5.5){\makebox(1,1){{\small don't buy}}}
\put(2.0,5.0){\makebox(1,1){{\small 73\% }}}
\put(6.5,5.5){\makebox(1,1){{\small buy}}}
\put(7.0,5.0){\makebox(1,1){{\small 27\% }}}

\put(4.5,2.5){\makebox(1,1){{\small ship}}}
\put(4.5,2.0){\makebox(1,1){{\small 37\% }}}
\put(9.0,2.5){\makebox(1,1){{\small don't ship}}}
\put(9.5,2.0){\makebox(1,1){{\small 63\% }}}

\put(-0.5,1){\makebox(1,1){35, 35}}
\put(4.5,1){\makebox(1,1){50, 50}}
\put(8.5,1){\makebox(1,1){0, 70}}
\end{picture}
\caption{\label{trustGame1} The original trust game used in the experiments by Bolton, Katok and Ockenfels.
Source: \cite[]{bolton-katok-ockenfels:2004}. The percentage values indicate how many subjects chose which
course of action in the experiment.} 
\end{center}
\end{figure}

Now, everyone knows that people in this world (luckily) are not totally
rational egoistic utility maximizers, as classical economic theory assumes,
but that they are also driven by normative concerns such as fairness
considerations. Bolton, Katok and Ockenfels distinguish three different types
of such concerns: Fairness in terms of reciprocity, fairness in terms of equal
distribution and, finally, collective efficiency concerns. Reciprocity as a
fairness concern\footnote{This should not be confused with {\em reciprocal
    altruism} in evolutionary models of the repeated Prisoner's Dilemma, which
  does not evolve because of any fairness concern but because it yields the
  highest payoff in the long run.} does in this context mean that the seller
might be induced to send the product to the buyer because he or she feels
obliged to do so since the buyer has sent the money. Fairness in terms of
equal distribution means that the seller cooperates because otherwise the
outcome would result in a very uneven distribution of goods (70 vs.\ 0 instead
of 50 vs.\ 50). And the seller is driven by efficiency concerns if his reason
is that the net
result for both players is higher than when cheating (100 vs.\ 70). The model as
it stands does not allow distinguishing between these motives. Therefore,
Bolton, Katok and Ockenfels draw up an additional model, where buyer and
seller retain 105 and 35 points if no interaction takes place, both end up
with an equal payoff of 70 if a trade is made and the seller cheats, and where
the buyer earns 120 and the seller 50 points if the seller does not cheat. 
(See figure \ref{trustGame2}.) From
the perspective of rational choice theory this second model is equivalent to
the first one: Both trading partners would be better off if the trade took
place and the seller did not cheat than if no trade took place at all. At the
same time, if the trade is initiated by the buyer, the seller gains most if he
cheats, wherefore -- anticipating rationality of the seller -- the buyer would be
best off not to initiate the trade at all. However, with regard to fairness
concerns, the buyer would initiate a trade and the seller would cheat if both
were driven by a ``fairness as equality'' ideal, while the seller would not cheat
if driven by reciprocity or efficiency concerns \cite[p.
191]{bolton-katok-ockenfels:2004}.

In an experiment participants were asked to play one of these two games in
either the role of the seller or the role of the buyer (that is no participant
played the game twice). While in the first game (where participants receive an
equal payoff if the seller cooperates) 37\% of the sellers did not cheat,
only 7\% of the sellers did not cheat in the second game.
Interestingly, even though the buyers should expect to be cheated in the
second game (just as or even more so than in the first game), they were much
more willing to buy in the second game (46\%) than in the first game (27\%).
These results strongly suggest that distributional fairness plays a
predominant role in this type of interaction, while efficiency and reciprocity
seem to be negligible motives \cite[p.\ 193ff.]{bolton-katok-ockenfels:2004}.

\begin{figure}
\begin{center}
\setlength{\unitlength}{1cm}
\begin{picture}(10,8)(-1,0)
\put(2,7){\makebox(6,1){Buyer}}
\put(5,7){\line(-1,-1){5}}
\put(5,7){\line(1,-1){2}}
\put(4,4){\makebox(6,1){Seller}}
\put(7,4){\line(-1,-1){2}}
\put(7,4){\line(1,-1){2}}

\put(2,5.5){\makebox(1,1){{\small don't buy}}}
\put(2.0,5.0){\makebox(1,1){{\small 54\% }}}
\put(6.5,5.5){\makebox(1,1){{\small buy}}}
\put(7.0,5.0){\makebox(1,1){{\small 46\% }}}

\put(4.5,2.5){\makebox(1,1){{\small ship}}}
\put(4.5,2.0){\makebox(1,1){{\small 7\% }}}
\put(9.0,2.5){\makebox(1,1){{\small don't ship}}}
\put(9.5,2.0){\makebox(1,1){{\small 93\% }}}

\put(-0.5,1){\makebox(1,1){105, 35}}
\put(4.5,1){\makebox(1,1){120, 50}}
\put(8.5,1){\makebox(1,1){70, 70}}
\end{picture}
\caption{\label{trustGame2} A slightly modified variant of the original trust game. Source:
\cite[]{bolton-katok-ockenfels:2004}.} 
\end{center}
\end{figure}

In both games the sellers thus proved to be more trustworthy than
their rational self interest would suggest. However, even in the original game
the degree of trustworthiness (37\%) would not be enough to make the game
profitable in monetary terms.\footnote{As can easily be verified, the expected
payoff of buying exceeds the payoff of not buying only when the probability
of meeting a trustworthy seller is greater than 70\%.} Taking the question one
step further, Bolton, Katok and Ockenfels proceed to examine how institutional
arrangements can influence the development of trust. In the case of online
auctions, the primary institution to allow the development of trust is the
rating mechanism. To examine the effects of such a mechanism, Bolton, Katok
and Ockenfels do, however, start with a setting without such a mechanism. In
contrast to the previous experiment the participants play the game repeatedly,
but with changing partners and without any information about the previous
interactions of the new partner. This setting is called by Bolton, Katok and
Ockenfels the {\em Strangers market} \cite[p.\ 196]{bolton-katok-ockenfels:2004}.
The results in the Strangers market are very similar to those in the original
experiment (on average 37\% of the buyers were willing to buy, while 39\% of
the sellers actually shipped the product). What the average values conceal is
that over time (the participants played the game 30 times) trust collapsed.
Obviously, the participants learned that their trust is not sufficiently
rewarded in this setting. This was to be expected.

To study the effects of institutional arrangements, Bolton, Katok and
Ockenfels contrasted the {\em Strangers market} with two further settings, the
{\em Reputation market} and the {\em Partners market}. In the Reputation
market, a feedback mechanism was introduced that informed the buyers about all
previous interactions of the seller. This is similar to the feedback mechanism
in internet auctions such as eBay. Only that in the real internet auctions the
feedback consists in a rating by the buyers in previous auctions,\footnote{As
  is well known, the ratings by disappointed buyers are not always fair, which
  in some cases also leads to lawsuits between buyer and seller.}  while in
the experiment the feedback accurately informed about the real behavior of
the seller in the experiment. In the Reputation market trust and cooperation
did not collapse as in the Strangers market. Instead, 56\% of the buyers were
willing to enter into a trade and 73\% of the sellers did not cheat.
Interestingly, the rate of cooperation of the sellers is very close to the
theoretical borderline of 70\% where trade becomes profitable in this game
\cite[p.\ 198]{bolton-katok-ockenfels:2004}. In the Partners market, which is
distinguished from the Reputation market by the fact that the same partners
interact throughout the whole repeated game, the rates of buyer's trust and
seller's cooperativeness were yet significantly higher than in the Reputation
market (83\% and 87\%). (Again, this result is unexplainable by normative
economic theory based on the rational actor model \cite[p.
199]{bolton-katok-ockenfels:2004}.)

The experimental setup that Bolton, Katok and Ockenfels used is still
in many respects simpler than the real world situation of internet auctions
with a rating system. In internet auctions the seller may not only cheat by
not shipping the paid product but also by shipping a product of lower
quality than advertised, the information propagated through the rating system
may not be completely accurate, both buyers and sellers can still take resort
to the legal system if they are unsatisfied, which means that cheaters do not
only bear the risk of a bad rating but also that of being sued. Still, the
%%% !!!Niklas: Wie gut funkioniert eBay in Nigeria?
experimental setup comes quite close to what happens in internet auctions.
Although it has not been done in this particular study, it is well imaginable
to compare the data gathered in this or similar experiments with that gathered
from real internet auctions. This would in principle allow checking whether
such experiments are realistic.

\subsubsection{Conclusions}

What can we learn from the experimental research in economics for the
explanatory validity of results obtained by computer simulations such as those
presented in chapter \ref{modeling}? It has already been noted (chapter
\ref{summaryReciprocalAltruism}) that computer simulations which are not tied
to specific empirical constellations can at best prove theoretical
possibilities, which as such are often not very informative. One way to link
computer simulations to empirical constellations would be to create
experimental setups which reflect the simplifying modeling assumptions.
(Neither of the previously discussed experiments was of course meant to verify
any computer simulations,\footnote{In fact, it seems that computer simulations
  do not play a very important role in this branch of research. In the very
  issue of ``Analyse \& Kritik'' (1/2004) from which Bolton, Katok and
  Ockenfels' paper \cite[]{bolton-katok-ockenfels:2004} was taken and which
  was as a whole dedicated to the topic of ``online cooperation'', not a
  single simulation study appeared among the 17 articles of the issue.} but
given the way these experiments work, one could use similar experiments that
match the setup of certain computer simulations.) Of course this requires
that the computer simulations use settings that can at least in principle be
reproduced experimentally. For population dynamical simulations of tournaments
of the 200 times reiterated Prisoner's Dilemma this might turn out to be a bit
impractical.

But when one of the restrictions of the method of employing computer
simulations is that in the first instance they only allow us to demonstrate
theoretical possibilities, then one of the restrictions of the experimental
method is that {\em prima facie} it only allows us to demonstrate {\em
  practical possibilities} and that we still do not know how much impact these
practical possibilities have outside the laboratory or -- to put it simply --
how realistic they are. The gap between the demonstration of theoretical or
practical possibilities and empirical reality (outside the lab) can under
favorable circumstances be closed, either because we are lucky enough to find
a constellation in the real world that is simple enough to match our models,
or because we examine social institutions that have been designed according to
precepts gained by model research and laboratory testing. (Again, these
considerations are somewhat tentative and the previously discussed examples of
economical experiments do not suffice to fully warrant such conclusions but
they should suffice to show their plausibility.)

The question remains, how many of the empirical questions that are of interest
to us in the social sciences are of such a kind that they can be tackled with
the help simulation models in the way hinted at above.

\subsection{A real world example: Altruism among enemies?}
\label{realWorldEvidence}

It has just been argued that there is some hope to link simulation models with
empirical reality via laboratory experiments. Usually, however, when it comes
to finding real world evidence for models of the evolution of altruism in the
social sciences, things start to get difficult. Of course it is easy to think
of many situations which more or less resemble a repeated Prisoner's Dilemma
(or some other game): the power game of politics for example, or negotiations
between opposing political parties when it comes to decisions that need the
full consent of all participants. But the problem is that this ``more or
less'' resemblance is simply not enough to explain the situations in question
with sparse models such as those described in chapter \ref{modeling}. Rather
than enumerating further examples where our models might apply (or might
not apply, as the case may be), I am now going to discuss one such example in
depth to highlight the (notorious) difficulties that formal modeling faces in
the social sciences outside the field of economics.

The example to be discussed is a sort of ``classic'' of the theory of the
evolution of cooperation. It is the ``live and let live''-system that
developed at certain stretches of the front line in the trench war of the
First World War. The ``live and let live'' system in the First World War is
already discussed in Robert Axelrod's ``Evolution of Cooperation'' as a prime
example for his theory of the ``evolution of cooperation'' (which is more or
less what was here discussed under the heading of ``reciprocal altruism'').
Because the phenomenon itself is so surprising, it is one of the most stunning
examples that have been given for the ``evolution of cooperation'' in a social
science context.  Axelrod's exposition of the ``live and let live'' system has
led to much subsequent discussion and criticism most of which centered around
the question of whether Axelrod's interpretation of the situation was correct
from a game theoretical point of view. Was the situation of the soldiers of
the opposing forces really a repeated Prisoner's Dilemma or some other game
or, rather, a collective action problem? Were the soldiers of the opposite
front lines the players of the reiterated Prisoner's Dilemma or were the
soldiers caught in a Prisoner's Dilemma against their own military
staff?\footnote{For a summary of the discussion of Axelrod's example in the
  more game theoretically orientated literature see Schüßler \cite[p.
  33ff.]{schuessler:1997}.} More important than the problem what kind of game
theoretical model can be applied to the ``live and let live'' system is the
question {\em if} Axelrod's interpretation of the ``live and let live'' system
in terms of evolutionary game theory yields any explanatory power, given that
it is by and large correct. Or, to put it more bluntly: Can an explanation in
terms of reciprocal altruism give us an explanation of the ``live and let
live'' system that goes beyond what can immediately be inferred from the
historical description of the phenomenon alone?

Axelrod's interpretation of the ``live and let live'' system rests on an
extensive historical study of the phenomenon by the sociologist Tony Ashworth
\cite[]{ashworth:1980}, a debt that Axelrod does, of course, fully acknowledge.
Tony Ashworth is neither a game theorist, nor does he try to explain the
emergence of the ``live and let live'' system evolutionarily. Yet, Ashworth
does not only describe what happens but also offers an explanation why the
``live and let live'' system could emerge on a certain front section, how it
could be sustained over a considerable period of time and why it eventually
broke down again. The crucial question that concerns us here is whether a
better explanation for this phenomenon can be given in terms of
reciprocal altruism or if at least new light is cast on some of the aspects
of the historical events in the First World War that Ashworth has described in
his book. In order to answer the question, the explanation that Ashworth
offers in his historical treatment must be reconstructed first. For, as it is
common in historical literature, description and explanation of the historical
events are interwoven in one and the same narrative in Ashworth's book.

Let's first look at the descriptive side and ask the question that all studies
in history begin with: What has happend? In our collective memory the First
World War is commonly remembered as an unusually brutal and destructive war.
It is associated with images of large scale battles, like the battle of Verdun
or the battle at the Somme, during which tens of thousands of soldiers died
within just a few weeks \cite[p.\  52]{james:2003}. It is much less known that
aside from the scenes of the great battles an astonishing calmness often
prevailed over long stretches of the front line. And this calmness prevailed
although the soldiers in the trenches virtually eyeballed their opponents on
the other side. Moreover, as Ashworth demonstrates in his study, these phases
of calmness were not merely the expression of comparatively less intensive
fighting but the result of a tacit mutual agreement following a kind of ``live
and let live'' principle. Of course this ``live and let live''-system was at
no time officially tolerated by the military doctrine and open fraternizing
was met with severe disciplinary measures.

But what did the ``live and let live'' system consist of if open
arrangements were impossible? Ashworth identifies several forms that the
``live and let live'' system could take: The exchange of shells and
bullets could be limited to certain times of the day. The shooting could
be directed to always the same targets, which the enemy soldiers only
needed to avoid getting close to if they wanted to stay alive.
Finally, it was possible to miss the opposing soldiers on purpose when
ordered to shoot at them. This way the soldiers in the trenches could at
the same time report the consumption of ammunition to headquarters and
signalize their opponents that they did not really intend to hurt
them. All this was of course based on mutuality and the conduct could
be changed any minute if the other side did not comply. Ashworth has
summarized these aspects of the ``live and let live'' system under the
short formula of the ``ritualisation of aggression''
\cite[p.\ 99ff.]{ashworth:1980}. The ritualization of aggression between
the opponents was completed by the emergence of a proper ethic among
the fellow comrades in arms, according to which ``disquieters'' or
``stirrers'' that did not honor the tacit agreement of ``live and let
live'' were hated and disdained \cite[p.\ 135ff.]{ashworth:1980}. 

This was just a very brief outline of the most important aspects of the
``live and let live'' system. In his book Ashworth discusses many more
factors, such as the role of different branches of the armed service and the
line of command. But it would lead too far to discuss all these details
here, although they are by no means unimportant and it is
furthermore by no means unimportant that in the game theoretic analysis
all of these subtleties must almost by necessity be left unconsidered.

Now that we have seen what the ``live and let live'' system consists of, how
does Ashworth {\em explain} it?  Because the ``live and let live'' system was
widespread one must expect that it has generic causes (in contradistinction to
singular historical causes). According to Ashworth's rough estimate it
occurred during one third of the front tours of an average division. This also
means that it occurred {\em only} during one third of the front tours.  If one
wants to explain why it occurred, one must also explain why in most
cases it did not occur. In Ashworth's treatment, the following preconditions
and causes for the ``live and let live'' system can be identified:

\begin{enumerate}

\label{liveAndLetLive}
\item The strategical deadlock. It was virtually impossible to move the
  front line for either side.

\item The natural desire of most soldiers to survive the war.

\item The impersonal, ``bureaucratic structure of aggression'' \cite[p.
76ff.]{ashworth:1980}.

\item Empathy with the soldiers on the other side of the front.

\item The ``esprit de corps'' that can, however, be both either conductive or
  (in the case of elite troops) impedimental to the emergence of the ``live
  and let live'' system.

\item Whether elite troops or non elite troops were fighting on either side.
  ``Live and let live'' was much less frequent where elite troops were
  involved.

\item The branch of service. Infantry soldiers had to face a much greater
  danger and consequently had a greater interest in ``live and let live'' than
  artillery soldiers.

\item The limited means of the military leadership to suppress ``live and let
  live''. (Only later did they find an effective way to do so by organizing
  raids on the enemy trenches.)

\item Initial causes such as Christmas truces, bad weather periods when
  fighting was impossible, coincidental temporary ceasefire due to similar
  daily routines on both sides (for example, same meal times).

\end{enumerate}

But why, then, did not the ``live and let live'' system occur everywhere and
all the time? One could of course think of many plausible answers to this
question. Because the ``live and let live'' system did not comply with the objectives
and the very purpose of military warfare it is natural to assume that it was
in many cases successfully suppressed by the military leadership. But as
Ashworth is able to demonstrate from the historical sources it was for a long
time almost impossible for the military leaders to efficiently suppress what
in their eyes must have been a great nuisance to their military mission. It
took them quite a while to find the right means to break the ``live and let
live'' system. (But when they finally succeeded in doing so, their success was
lasting.) Furthermore, one might assume that the ``live and let live'' system
was quite error prone as no explicit agreements with the other side could be
made. But the most decisive factor among the above listed causes for the
emergence or non emergence of the ``live and let live'' system was --
according to Ashworth's empirical study -- whether the troops involved were
elite troops or ``regular'' troops.\footnote{Among the British troops there
  was no formal division between elite and non elite, but, as Ashworth points
  out, military staff as well as the common soldier new fairly well which
  troop was elite and which was not.} Only when non elite troops were facing each
other was there a high chance for the ``live and let live'' system to emerge
and to be sustained.

The means by which the military leadership finally managed to break
the ``live and let live'' system was the ordering of raids into the enemy
trenches. Raids could not be faked nor could they be ritualized because
either the enemy had casualties or the soldiers of one's own side did
not come back. And by stirring up emotions of hatred and revenge the
raids deprived the ``live and let live'' system of its
emotional foundation in mutual empathy \cite[p. 176ff.]{ashworth:1980}.

So much for Ashworth's historical description of the ``live and let
live'' system and his explanation of these suprising historical events.
What can Axelrod's interpretation on the background of the theory of
the ``Evolution of Cooperation'' add to this explanation?

First and foremost Axelrod argues that the situation of the soldiers in
the trench warfare can be interpreted as a repeated Prisoner's Dilemma. In
order to do so, Axelrod needs to show that the options that were
available to the actors in the historical situation correspond to the
possible choices of the players in a repeated two person game and are
valued by the soldiers in such a way that the game is a Prisoner's
Dilemma. That this is indeed the case is demonstrated by Axelrod quite
persuasively: In the historical situation single sided defection would
mean to fight and meet so little resistance that victory is possible.
Clearly, this would be the preferred alternative on any side of the
front. Thus, even without assigning particular preference values, we
can safely assume that $T > R,P,S$. But if it was not possible to
break through the enemy front line then it was certainly better to
``keep quiet'' as long as the opponents were willing to ``keep quiet''
because such an arrangement drastically increased the prospects of
survival (in Axelrod's formal notation this means that $R > P,S$).
Furthermore, mutual abstinence from serious fighting was certainly to
be preferred to alternating single sided fighting if that should be
considered a viable option at all. Therefore $R > (T+S)/2$ can also be
granted. But if the opposing side was not willing to ``keep quiet'' by
ritualizing aggression in the previously described way then it was
still better to fight back then to let oneself be overrun ($P > S$). 
%%% !!! Wo steckst die Erklärungsarbeit

In order to apply the theory of the ``evolution of cooperation'' to the
situation of the soldiers in the trenches of World War I, some further points
need to be clarified such as whether the ``game'' played really was a {\em
  repeated} Prisoner's dilemma, which requires the identity of the players
over a longer period of time. Even though the soldiers at the front were
periodically exchanged by fresh troops, the predecessors had to familiarize
their successors with the situation at their section of the front. Therefore the
successors could pick up the ``game'' exactly at the point where their
predecessors had left it. It is a bit less obvious what the evolutionary
transmission mechanism that led to the spreading of the ``live and let live''
system consists of. Axelrod hints to the fact that the system spread over
neighboring sections of the front.  But, as has been indicated earlier, one
may also assume that the ``live and let live'' system started independently in
many different sections of the front. It does not seem to disturb Axelrod that
the way the ``live and let live'' system was initiated and transmitted bears
only very little resemblance to the population dynamical transmission
mechanism in his simulation model.

Save for this last point it can be granted that Axelrod's analysis is by
and large convincing.
But in how far does Axelrod's interpretation go beyond Ashworth's study as far
as its explanatory power is concerned? If we consider the whole bundle of
conditions that Ashworth discusses as causes of the ``live and let live''
system (see page \pageref{liveAndLetLive}), it becomes obvious that only one
of these conditions is captured by Axelrod's game theoretical interpretation.
This condition for the ``live and let live'' system is the strategic situation
of the soldiers in the trenches, which Axelrod describes as a repeated
Prisoner's Dilemma. It is important to realize that by doing so Axelrod
captures only one of many causes for the ``live and let live''-system.
Therefore, the evolutionary theory of Axelrod cannot reasonably be regarded as
an alternative explanation to the one which is offered by Tony Ashworth in his
historical narrative. At best, the theory of reciprocal altruism offers a more
precise treatment of one single component of Ashworth's
explanation.\footnote{This is
  a point that Axelrod seems to be aware of as he mentions that some of the
  insights of Ashworth's study, such as the emergence of an ethics of
  cooperation, might be used to extend his theory of the evolution of
  cooperation.} Whether this is really the
case, shall occupy us now.

Is Axelrod at least able to provide a more precise understanding of
at least this particular aspect with the help of evolutionary game
theory? In order to find out whether such a claim would be warrented it
must be examined whether the situation of the soldiers in
the trenches can really be described as a repeated
Prisoner's Dilemma. Against Axelrod's interpretation the objection has
been raised that the front soldiers may have been primarily interested
in their own survival after all and that, compared to their survival,
being victorious in the battle was much less important to them. Then the
soldiers would not really gain any advantage by single sided defection.
(The payoff parameter $T$ would be lower or equal the payoff
parameter $R$ in Axelrod's notation.) If this interpretation is followed
then the problem the soldiers had to solve was a mere coordination
problem and not a Prisoner's Dilemma. Independently of how the question
is to be answered the objection shows that the assessment of a
given situation in terms of game theory is by no means a trivial and
unambiguous task. The difficulties become even greater when it comes to
estimating concrete values for the different payoff parameters. Axelrod
confines himself to establishing the relative proportions of the payoff
parameters that are expressed in the two inequalities $T > R > P > S$ and
$2R > T+S$, although his model is in fact sensitive to changes in
numerical values of the parameters -- as has been demonstrated by the
simulations in section \ref{refinedModel}.

But there exists an even more serious objection to Axelrod's
interpretation: The described strategical stalemate was (save for the
great battles) more or less the same at all sections of the front line.
Nonetheless, the longitudinal analysis showed that the ``live and let
live'' system occurred on average only during roughly one third of the
front tours \cite[p.\ 171-175]{ashworth:1980}. This empirical fact poses
a real problem for Axelrod's theory because his theory postulates that
in the reiterated Prisoner's Dilemma cooperative strategies will
{\em usually} prevail. However, as the more extensive series
of simulations that has been presented earlier (see section
\ref{refinedModel}) has shown in accordance with earlier criticisms of
Axelrod's approach by mathematical game theorists
\cite[p.\ 313ff.]{binmore:1998}, the theoretical foundation for Axelrod's
generalizing claim that cooperative strategies like {\em Tit for Tat}
enjoy a high advantage in the repeated Prisoner's Dilemma was lacking.
As the results of the simulation series suggest, it is not generally true
that cooperative strategies are the best strategies in the reiterated
Prisoner's Dilemma.  Depending on the particular circumstances,
uncooperative strategies like {\em Hawk} may be much more successful.
It might seem tempting to draw the conclusion that Axelrod's computer
model was too crude after all and that our more refined simulation
series which suggests an only limited evolutionary success of
cooperative strategies is in better accordance with the empirical
findings of Ashworth. Thus, while Axelrod's theory in its original
form failed it only needed to be refined a little bit on its technical
side to make it succeed.

Unfortunately, the epistemological situation is not as simple as that.
According to Ashworth, the major factor which determined the occurrence of the
``live and let live'' was whether the troops involved were elite troops or
merely regular soldiers.  Whenever elite troops were involved, the ``live and
let live'' system was very unlikely to occur. How can this factor (elite
soldiers or non elite soldiers) be reflected in our model? It can be done by
assuming that for elite troops a different set of payoff parameters holds
because elite soldiers value the viable options (fight hard or ``live and let
live'') according to a set of preferences that differs from that of ordinary
soldiers. For example, it is not implausible to assume that elite soldiers
might consider it dishonorable to avoid fighting just to save one's own life.
But while such an assumption might save our theory it remains doubtful whether
much is gained in terms of explanatory power.  For, instead of reverting to
simple standard assumptions about the payoff parameters in a given strategical
situation, it would be necessary to conduct an extensive historical inquiry in
order find out how different groups of soldiers may valuate one and the same
situation. (In fact, without such an inquiry we might not even be aware that
there is such an important difference between elite soldiers and non elite
soldiers.) But with the historical inquiry at hand, we would not need a game
theoretical model any more to tell us what happend. Or, to put it in another
way, almost all of the explanatory work would be done by the theories and
historical inquiries needed to determine the payoff parameters, while the game
theoretical model making use of this work would be little more than a trivial
and illustrating addition. Also, once it is accepted as a fact that it
depended on the elite status of the troops whether they would fight or attempt
to engage into ``live and let live'' with their enemies, this fact can be
explained more simply than by any game theoretical model by the rather
obvious assumption that elite soldiers are more likely to follow orders
involving great danger than ordinary soldiers. An assumption that has the
additional advantage that it is -- other than assumptions about payoff values
-- empirically very easily testable in comparable circumstances.

The more general lesson to be learned from this is that game theoretical
models prove to be useful only in situations where we can either proceed from
standard assumptions about the relevant payoff parameters or where reliable
measurement procedures for the input parameters of the models exist. Apart
from the fact that it leaves out too many causally relevant factors, this is
the second reason why the theory of the ``evolution of cooperation'' fails to
explain the sort of cooperation that emerged between the opposing soldiers in
the trench warfare of World War I. (And with this second reason it is clear
that it does not even provide a partial explanation.)

Following an influential argument from Carl Gustav Hempel \cite[]{hempel:1965}
it might still be objected that even though the game theoretical model cannot
offer more than an {\em ex post} explanation, it is still of scientific value
because it affords a {\em general} explanation for a course of historical
events and thus increases our understanding of historical processes of a
particular kind by subsuming them under general laws or principles.
Unfortunately this is not the case here. For, as we have seen, the theory of
the ``evolution of cooperation'' provides hardly an explanation for the
emergence of the ``live and let live''-system in World War I at all. It is not
well possible to defend a wrong explanation or a theory that is not an
explanation at all with the argument that it affords a generalization. To say
this does not mean that historians and social scientists do not need to or
should not be interested in general theories.  But in the social sciences and
especially in history, generalizations that are meaningful and rich in content
are typically found on lower levels of abstraction. One of the standard
methods for generating and testing general theories in history is the
comparison of similar chains of events under different historical
circumstances. For example, it might be interesting to compare the situation
in the First World War with that in other wars and with the aim of deriving a
generalized theory of fraternization, which could then in turn be applied to
the ``live and let live''-system and other comparable events.  But it seems
rather hopeless to seek a general theory for the explanation of the ``live and
let live'' system that is still meaningful and rich enough in content on the
level of abstraction of the theory of the ``evolution of cooperation''.

Summing it up, computer simulations of the ``evolution of cooperation'' hardly
add anything to our understanding of the ``live and let live'' system in the
trench warfare of the First World War. The emergence (or the ``evolution'', if
this term is preferred) of ``live and let live'' is due to an intricate
network of interlocking causes that cannot accurately be explained by
reference to simulations of the repeated Prisoner's Dilemma game. At best
there exists a vague metaphorical resemblance between the situation of the
soldiers in the trenches and the repeated Prisoner's Dilemma, but this alone
is not sufficient for an explanation and it is hardly sufficient to justify
the technical effort of a computer simulation in this particular case.

% What is true of this example is of course also true of all similar
% cases of human behaviour that are studied in the social sciences where
% the strategical situation as it can be described by game theory is
% just one among a number of interlocking causes. One might object that
% maybe just the choice of the example has been unlucky (even though the
% ``live and let live'' has been put forward by Axelrod as one of the
% schowcases of his theory of the ``evolution of cooperation'' with its
% simulation based method). The repeated Prisoner's Dilemma may not be a
% suitable model for the ``live and let live'' in World War I and
% similar situations. Yet, just because the evolutionary explanation
% referring to the repeated Prisoner's Dilemma does not work in this
% kind of situation, this does not mean that it cannot work in other
% cases. A pluralism of coexisting paradigms is typical for the social
% sciences, which means that if a certain approach fails to provide
% explanations for one specific class of social phenomena, it may still
% prove to be useful in other cases.  But to justify a certain
% theroetical approach there must exist at least one empirical
% application case, for which this approach provides (a) an explanation
% that is reasonably good and (b) one that is better than any
% alternative.\footnote{Clause (a) is necessary because it is usually
%   not true that any explanation is better than none. Therefore it is
%   not enough to demand that a certain explanation must be better than
%   the alternatives. If all alternatives are very poor explanations
%   than it could still be so bad that it would be more honest to admit
%   that one does not really have an explanation.}  Regarding the
% examples from biology, we have already seen that there does not exist
% a single case where the simulation based approach to the explanation
% of reciprocal altruism could be applied in any strict sense.
% Unfortunately, the situation appears to be very similar in the social
% sciences. Empirical showcases where the simulation based explanation
% of reciprocal altruism really works are either missing or prove to be
% serverly flawed upon closer inspection, as it has been demonstrated
% for the ``live and let live'' in the trenches of World War One,
% supposedly a show case for the ``evolution of cooperation''.  But if
% there are no examples of the succesful empirical application then it
% becomes difficult to avoid the conclusion that the simulation based
% approach to reciprocal altruism has not yet achieved its goal. In its
% present form it cannot explain the emergence and sustainment of {\em
%   any} empirically observable occurence of altruism other than in the
% sense of a vague and unspecific just so story. This still leaves open
% the questions of whether and how the simulation based approach can be
% further developed so as to overcome its present explanatory limits in
% the future. Or, simply put: Until today, game theoretical computer
% simulations have failed to explain altruism, but do they have any
% potential? It is with questions like this that the following section
% will be concerned.

\section{Conclusions}

The previous survey of empirical studies on the evolution of altruism provided
some interesting insights in how and why altruism and cooperation can evolve
even under unfavorable conditions. Regarding the epistemological merits of
simulation models for the explanation of evolutionary altruism, however, the
insights gained from looking at the empirical research are extremely sobering:
First of all, it is an undeniable fact that computer simulations on the
evolution of altruism have remained largely useless for empirical
research. And this does of course also mean that computer simulations of the
evolution of altruism hardly provide us with any knowledge about how altruism
really evolves. This seems to be especially true for repeated Prisoner's
Dilemma simulations of reciprocal altruism because they rely on a setting
that plays only a very marginal role in nature (see page
\pageref{impalaGrooming} for one of the few examples where it does). Secondly,
the in-depth discussion of two selected examples where the application of
simulation models failed despite the serious attempts of its supporters
precisely showed why the simulation models failed. In the biological example
the model failed because it relies on payoff parameters that could not be
measured, while the model is at the same time sensitive to changes of these
parameters. That the fitness relevant payoff is very hard to measure is a
general difficulty that evolutionary game theory faces in biology, though it
does not always turn out to be as fatal as in this instance.\footnote{See
  \cite[p.\ 9ff.]{hammerstein:1998} for some reflections on how to remedy this
  difficulty by means of clever interpretation.} In the sociological example
the repeated Prisoner's Dilemma model failed because from the many
interlocking causes that brought about cooperation between the enemy front
soldiers in World War One, it captured at best one cause that could be
described as ``the strategical situation'' of the front soldier. But then it
cannot seriously be maintained that cooperation occurred in the trenches in
virtue of the very factors for which it evolves in repeated Prisoner's Dilemma
simulations. Apart from that, the very same measurement problems and model
stability issues that have already been encountered in the biological example
reappear in the sociological example as well.

It should not be considered too much of a surprise that the simulation model
fared so badly in the sociological example. After all, formal mathematical
models can be used in the social sciences only in a few select areas, most
notably economics. The reason is that for many explanations that we give in
the social sciences, we have to draw on connections for which no formal
description exists. One may regret this state of affairs, but it certainly
does not get any better by ignoring all factors that cannot be rendered
formally. Therefore, in many cases an ordinary historiographical approach may
serve the needs of the social scientist much better than a seemingly more
refined simulation based approach. Other than that, part of the art of
applying formal models in a sociological context certainly consists in
picking out the right empirical situations for which a model based approach
might indeed be appropriate. How this can possibly be achieved has been hinted
at when discussing the internet auction example in section
\ref{economicsTrust}.

All in all, a look into the empirical literature is apt to strengthen some of
the skeptical conclusions about computer simulations on the evolution of
altruism that have been drawn at the end of the previous chapter (see chapter
\ref{summarySimulations}), most notably the impression is strengthened that
pure model research conveys a distorted picture of how and why altruism
evolves. If one really wants to understand how and why altruism evolves then
designing models based on ``plausible'' assumptions alone and uninformed by
concrete empirical research is certainly not the way to go. If the simulation
based approach to the explanation of the evolution of altruism has thus been a
failure then what remains to be clarified is just why it had to fail and what
a possible remedy could look like. This is what will occupy us in the next
chapter.
