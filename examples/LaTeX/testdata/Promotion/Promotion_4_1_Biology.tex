\chapter{Empirical research on the evolution of altruism}
\label{empiricalResearch}

% The fact that very similar models give rise to quite different
% conclusions raises certain questions concerning the usefulness of an
% abstract modeling approach. One thing that is out of doubt is that if
% abstract modeling is to have any scientific value at all it must be
% done on par with empirical research examining the very phenomena that
% the models aim to explain. Therefore, subsections two and three give
% an overview of the empirical research done in this field. Subsection
% two deals with the empirical research in biology. Although the theory
% of reciprocal altruism has been immensely popular in biology for a
% long time, biologists are becoming doubtful about whether reciprocal
% altruism has a strong impact in nature. There seem to be only very few
% clearcut examples of the sort of reciprocal altruism the theory
% postulates \cite[]{hammerstein:2003a}. In a way, the theory of
% reciprocal altruism postulates too much in biology. The opposite
% impression prevails when one turns to the empirical findings in social
% sciences. Empirical research in these fields shows that there are more
% and stronger instances of reciprocal altruism over and above what
% models of the repeated Prisoner's Dilemma suggest. One such type of
% reciprocal altruism that is not (yet) well reflected in current models
% is ``strong reciprocity'' which includes reciprocal behavior in form
% of rewards and punishments even where it incurs an extra cost on the
% reciprocator \cite[]{fehr-henrich:2003}.


% Having discussed at some length the sort of theoretical modeling that can be
% used to understand the evolution of altruism, it is now high time to turn to
% the empirical research on the topic. Purely theoretical modeling of altruism
% is anything but satisfactory, if only because there are arbitrarily many ways
% to do it. If it reveals anything then that there exists a wide rangeing and
% dazzling variety of ways in which altruism can evolve in repeated games or
% other model situations. As far as models of repeated games are concerned,
% this
% alone is not much of a surprise but to some extend merely a trivial
% consequence of the folk theorem.  Even if we confine our focus to reciprocal
% altruism for a moment: The enormous variety as well as the fact that there
% exist hardly any non trivial general principles of the evolution of
% cooperation or reciprocal altruism that remain true across all the different
% models should warn us not to draw any premature conclusions about how
% reciprocal altruism eveloves and what types of altruism can evolve in dilemma
% situations. For example, it is true that a pure equilibrium of genuine
% altruists (``{\em Doves}'') will not be stable. But any statement below this
% triviality is not warrented any more, for we have seen that a mixed
% equlibrium
% where the majority of players are genuine altruists can evolve under
% the respective circumstances (see the discussion of ``slip stream'' altruism
% in chapter \ref{slipstreamAltruism}). Similar difficulties in drawing
% substantial conclusions about the evoltuion of altruism from pure model
% research alone arise for the other types of evolutionary altruism, kin
% selection and group selection. Only, in the case of reciprocal altruism this
% flaw of the modeling approach is the most apparent as more model research has
% been done on reciprocal altruism.

The last chapter closed with the conclusion that substantial scientific
results about the evolution of altruism cannot be obtained by looking at
computer simulations alone. The situation would be different if there were
only one right way to model altruism. But because there are so many plausible
ways to do it only a look at the empirical examples can tell which one is
the right one. In the following we will therefore examine some of the
empirical research on altruism. We will first look at biology and then at the
social sciences. When surveying the research in these fields, there are two
questions that are important for us: First of all, we do of course want to
find out whether, how and why altruism evolves in nature and among humans.
Theoretical models and computer simulations demonstrate how it {\em could}
evolve.  Empirical research, hopefully, can tell us something about how, why
and where it {\em does} evolve. The second question concerns the method and
research strategy. Already in the previous chapter there has been opportunity
to raise some doubts concerning the usefulness of the tool of computer
simulations for the understanding of reciprocal altruism. Now we want to know
how these simulation models live up to the empirical research, that is whether
they are helpful for conducting such research and whether they prove valuable
for the explanation of the results of the empirical research.

A survey of empirical research on the evolution of altruism raises certain
methodological issues by itself, which shall briefly be discussed, before
entering into the discussion of the empirical material. First of all, there is
the question of the selection of the material. As the research on altruistic
behavior is a wide and varied field both in biology and in the social sciences
and as the focus of empirical scientists and the categories they employ are
often not the same as those the theoreticians develop, a selection of materials
is unavoidable. In the following, I have tried to choose examples that are most
closely linked to the theoretical models and to the concepts of reciprocal
altruism, kin selection and group selection described earlier. This criterion
of selection also has advantages for addressing our second question, the
question of the usefulness of simulations as a method. For, if this method
fails in those cases that we would assume it is best suited to deal with, then
we have good reason to assume that it is a bad method (at least in the way it
is applied today) without worrying that we might have been unfair. Still, it
must be admitted that the following selection of empirical example cases is
quite eclectic. This is unavoidable given the sheer extent of this field of
research, but -- as should frankly be admitted -- it is also partly due to the
fact that I am neither an expert in biology nor in experimental game theory.

Another methodological issue when surveying research, concerns the question as
to whether one should give a broad overview covering as much of the
research as possible or whether one should rather pick out a few examples and
discuss them in depth in order to demonstrate how the respective kind of
research works and what degree of credibility can be attributed to it.
Regarding the biological examples, I have tried to combine both approaches.
First, an overview of a larger number of empirical studies on reciprocal
altruism will be given to convey an idea of where this research stands. Then, 
one example will be picked out and discussed in depth to see how reliable the
results of this research are and especially how well the theoretical models do
when submitted to the ``on-road test''. For the social sciences I confine
myself to the discussion of a few select examples. The reason for this is that
while there exists a lot of empirical research on cooperation dilemmas of one
kind or other, there are hardly any empirical studies that are closely attuned
to the kind of models that have been discussed before.\footnote{This is even
  true for Axelrod's popular model of reciprocal altruism, which has spurred
  myriads of further model studies \cite[p.\ 24ff.]{dugatkin:1997}, but
  remained quite infertile for the empirical research.} It would be spurious
to present a summary of research on behavioral economics that mostly falls
outside the narrower topic of this book.\footnote{A fairly recent overview of the 
research on altruism in experimental economics can be found in \cite[]{fehr-fischbacher:2003}.
The bulk of this research is concerned with the question how altruism works among humans. While this
has some bearing on which kind of evolutionary explanations are more plausible than others, only few evolutionary models seem to be have been put to the empirical test directly.} But just as in the case of biology, one of the examples from the social sciences will be discussed in depth. For
the in depth discussion I have in both cases picked examples that were by
their authors intended as show cases for the application of reiterated
Prisoner's Dilemma models.  Therefore, these examples should be best suited to
assess the possible merits and defects of this type of modeling.

\section{The empirical discussion in biology}
\label{biology}

\subsection{Altruism among animals}

As in any other field of science the specialist literature on altruism in
biology comes in two different brands. First of all, there are articles in
different biological journals. Then, there are books on the topic written by
specialists that usually present the results of the research published in
articles in a condensed and simplified form. For a non-specialist it is
advisable to stick to the latter kind of literature, for otherwise there
exists a considerable danger of misunderstanding and of giving too much weight
to unimportant details and too little weight to important ones. Luckily, there
exists a treatment of the subject in book-form by an author who is strongly
committed to a game theoretical approach to the study of altruism. This
treatment is Lee Allan Dugatkin's already afore mentioned ``Cooperation Among
Animals'' \cite[]{dugatkin:1997}. In what follows I therefore present 
mostly examples from Dugatkin's book. Unfortunately, the book was issued
in 1997 and therefore does not cover the latest research. For this reason,
later on I also discuss an example of a study that has been published on
the topic since.

The empirical research which Dugatkin reviews, cannot always be sorted neatly
into different categories of altruism like reciprocal altruism, kin selection
or group selection. The reason for this is that when scientists set out to
research altruistic behavior in certain animal species they usually are not
sure beforehand what kind of altruism is concerned. And quite often the data
they are able to obtain does not allow making the distinction afterwards.
Often it is not even clear whether the behavioral trait in question is
altruistic at all or merely some kind of byproduct mutualism.\footnote{The
  difference between altruism and byproduct mutualism is that while both
  entail benefits for some other individual, it must in the case of altruism
  at least be possible to cheat, while in the case of byproduct mutualism
  cheating is impossible in principle that is, an exchange of benefits still
  may or may not take place, but if it takes place cheating is not an option.
  An example to illustrate this might be two people warming each other in
  winter by moving closer together. None can enjoy the warmth of the other
  without giving warmth him- or herself, which means that there is no way to
  cheat.}  In the following, different examples of cooperative and potentially
altruistic animal behavior that are described in Dugatkin's book will be
presented. The main aim is to clarify whether the theoretical categories for
altruistic behavior (reciprocal altruism, kin selection and group selection)
can be identified empirically and to what degree assumptions about the type of
altruism can be ascertained. Also, it will be asked in how far models such as
those presented in the previous chapter can be validated empirically and
whether and in how far these types of models have been useful to empirical
research.

\subsubsection{Cooperative behavior as it occurs in nature}

\paragraph{Egg Trading}

An often quoted example of reciprocal altruism in particular is that of egg
trading among hermaphroditic fish. According to Dugatkin it is best
documented for sea bass (Wolfsbarsch) \cite[p.\ 46]{dugatkin:1997}. Sea bass
(as well as many other egg trading fish species) parcel their eggs into small
packages. When mating, one fish starts by releasing a parcel of its eggs,
which typically consists of only a small fraction of the eggs it has. At
the same time the partner releases sperm. Then they switch roles and regularly
alternate the release of eggs subsequently. These cycles of alternating egg
spawning suggest an interpretation of this process as a repeated game. But is
the game a Prisoner's Dilemma and do the sea basses use a reciprocal strategy,
i.e.\ would they retaliate if being cheated? Dugatkin's answer is that it can
loosely be interpreted as a repeated Prisoner's Dilemma if the release of one
parcel of eggs by one partner and the following release or failure of release
by the other partner is interpreted as one round of the repeated game and if
it is assumed that producing eggs is more expensive than producing sperm.
Although it is difficult to quantify the costs, the latter assumption is
almost certain to be true \cite[p.\ 48]{dugatkin:1997}.  A problem is that due
to the lack of quantitative data (and -- as of now -- the lack of measurement
techniques to obtain such data), it is impossible to fill in the payoff matrix
of the game other than by rough estimates. But then it is not even sure
whether {\em Tit for Tat} is a suitable equilibrium strategy. Regarding the
question whether fish engaged in egg trading do in fact play {\em Tit for
  Tat}, there exists, according to Dugatkin, some anecdotal evidence (i.e.\  
non-systematic evidence from incidental observations) for certain types of fish
that they do in fact play some deviant version of {\em Tit for Tat}. It is reported
that black hamlets and chalk basses retaliate by waiting much longer to parcel
out eggs if a partner failed to reciprocate before. But sometimes they omit
retaliation, which suggests that they are really using a {\em Generous Tit for
  Tat} strategy \cite[p.\  48]{dugatkin:1997}.

The repeated Prisoner's Dilemma model of Axelrod and Hamilton
\cite[]{axelrod:1984} which assumes a fixed number of rounds or at least a
fixed termination probability is not the only model that can potentially be
applied to the egg trading behavior among fish. Dugatkin also describes
another interpretation of the egg trading behavior by R.C. Conner that is
related to a species of plycheate worms and according to which there is no
fixed termination probability but each partner decides continuously whether to
continue or to break off the interaction. For Connor this is simply a matter
of whether the benefit of staying\footnote{Although Dugatkin does not say
  anything about this in his report of Connor, one should assume here that
  what is meant is the {\em expected} benefit of staying, as the possible
  future benefit also varies according to when the other partner decides to
  break up the interaction.} exceeds the benefit of leaving and, given his
interpretation is right, he justly speaks of ``pseudo-reciprocity'' instead of
reciprocity \cite[p.\ 49]{dugatkin:1997}. However, without more precise
quantitative data it is not possible to decide this question.

\paragraph{Alloparenting}

Another type of potentially altruistic behavior is that of {\em alloparenting},
which according to Dugatkin means ``the dispensing of `parental' behavior to
young that are not one's own'' \cite[p.\ 101]{dugatkin:1997}.
``Alloparenting'' concerns sexually mature individuals that {\em could} also
produce offspring of their own. From an evolutionary point of view such a
behavior demands explanation because animals that want to spread their genes
should primarily be interested in raising their own children not those of
others. Nonetheless {\em alloparenting} is quite widespread and found among
various kinds of mammals, birds and fish. {\em Alloparenting} among fish
has been studied for {\em Lamprologus brichardi}, a type of perch (Barsch)
found in the Lake Tanganyika in East Africa. For this species it is typical
that the young stay at the nest for a while even after they have grown
sexually mature and help cleaning eggs and maintaining and defending the
territory. That this kind of helping activity is costly is illustrated by the
fact that the young that stay at the nest have a slower growth in comparison
with young that do not stay at the nest.  The benefits that mature young
derive from staying and helping at the nest include relative safety from
predators and rearing kin that is at least closely related even if it is not
their own.  (Other suggested benefits were not confirmed or at least not
measurable by experimental research.) This suggests that both byproduct
mutualism (safety from predators) and kin selection are involved in the {\em
  alloparenting} behavior of {\em Lamprologus brichardi}. But according to
Dugatkin there is also a reciprocal element present because when the mature
young start to reproduce themselves they are expelled from the nest by their
parents.\cite[p.\  50]{dugatkin:1997} The only factor promoting altruism that
could strictly be measured was that of kin selection, which of course is
relatively easy to measure. The assumption that byproduct mutualism and
reciprocal altruism are involved as well can, according to Dugatkin, be
confirmed by observation but it is not possible to actually measure the payoff
parameters of the game matrix and apply any of the game theoretic models, let
alone computer simulations in any strict sense.

In other species the {\em alloparenting} behavior naturally takes a different
form. A type of {\em alloparenting} common among many mammals is {\em
  allonursing} by giving milk to unrelated conspecifics. It has been
researched in some detail for the evening bat {\em Nycticeius humeralis},
where ``approximately 20\% of nursing bouts involved females feeding unrelated
pups'' \cite[p.\  109]{dugatkin:1997}. Among the discussed benefits are the
decrease of weight during foraging bouts following the nursing and the
decrease of chances of infection as a consequence of not storing surplus milk
in the mammary glands. Both of these advantages would fall under the category
of byproduct mutualism (which is according to our definition of altruism in
chapter \ref{altruismDefinition} not altruistic). But there could be more to
it. According to Dugatkin, who relates to a study by G.S. Wilkinson, females
are more likely to nurse unrelated female pups than unrelated male pups
\cite[p.\ 109]{dugatkin:1997}, which may be due to the fact that the males
disperse. If
this is true then this means that some degree of reciprocity is also involved.
Another variant of {\em alloparenting} which has been described for Rodriques
fruit bats consists in the provision of assistance in the birth process by
unrelated females (``midwives'') \cite[p.\  109]{dugatkin:1997}. Though it has
not been determined how the altruistic behavior has evolved in this case, it
is reasonable to assume that it is somehow connected with the extremely social
nature of the long-lived individuals of this bat species. Again, if this is
true, bat-``midwives'' would at best be described as reciprocal altruists
\cite[p.\ 109]{dugatkin:1997}. Given the social nature of this species, one
might -- by drawing a somewhat risky comparison -- speculate if these
altruistic acts may not somehow resemble the sort of friendship altruism among
humans that goes beyond the ``bookkeeping kind of altruism'' that reciprocal
altruism is often assumed to be \cite[]{silk:2003}. But this is of course just
a speculation.

Staying with the bats, one of the classical examples of animal altruism is
that of blood sharing among vampire bats \cite[p.\ 113/114]{dugatkin:1997}.
Empirical research indicates that it is a mixture of both kin selection and
reciprocal altruism.  Again, the precise conditions (i.e.\ payoffs) cannot be
measured, but several indications make the assumption highly plausible that
reciprocal altruism is involved: 1) A high probability of future interaction,
2) the relatively cheap cost of providing a meal in comparison to the benefit
of receiving one (the latter can be a question of life and death), which means
that the threshold to offering an altruistic benefit is low, and 3) the ability
of the vampire bats to recognize one another \cite[p.\ 
114]{dugatkin:1997}.  {\em Alloparenting} behavior is also documented for many
primate species, though here it typically does not include the provision of
food by the allomothers and usually the allomothers are immature animals
\cite[p.\  138]{dugatkin:1997} so that they do not fall under the strict
definition of {\em alloparenting} any more.

\paragraph{Alarm Signals}

Yet another type of potentially altruistic behavior that has attracted the
interest of researchers is that of giving alarm calls or alarm signals. As in
many of the other instances of possibly altruistic behavior the empirical
data is often too scarce to decide in any specific case whether giving an alarm
call really constitutes an instance of altruistic behavior or not. In willow
tits the giving of alarm calls seems to be related to the place in the
dominance hierarchy and thus probably falls into the category of byproduct
mutualism as the benefits derived by the survival of group members as a
consequence of giving a call depend on the position of the group member.
However, reciprocity has also been suggested in this context \cite[p.\ 
86]{dugatkin:1997}. In other bird species, downy woodpeckers and black-capped
chickadees, alarm calls mainly serve the purpose of mate protection, which is
demonstrated by the fact that alarm calls are not given in same sexed flocks.
Then alarm calls do not provide an example of altruism but of byproduct
mutualism. Still, byproduct mutualism sometimes is the first step in an
evolutionary history that may eventually lead to altruism. As Dugatkin
imparts, byproduct mutualism typically evolves in
harsh environments. In this case the ``harshness'' consists in ``the decreased
probability of acquiring new mates'' \cite[p.\ 86]{dugatkin:1997}. In terms of
chances of reproduction it may pay off to risk one's own survival (by giving
an alarm call) in order to increase the probability of survival of a mate.
Regarding the different explanations for the same type of behavior in willow
tits, chickadees and woodpeckers, it should be borne in mind that it is not
necessarily the case that the same type of behavior has the same evolutionary
causes if it occurs in different species.

Another species for which alarm calls have been studied quite extensively are
Belding's ground squirrels. Here it is quite well assessed that kinship based
altruism is the decisive factor for giving alarm calls.  For, typically alarm
calls are given by females, and in this species females are sedentary and breed
near their natal sites, while males leave their natal sites \cite[p.\ 
97/98]{dugatkin:1997}. The hypothesis is further strengthened by the
observation ``that `invading' (non-native) females gave alarm calls less
frequently than native females.'' \cite[p.\  98]{dugatkin:1997}. A fairly well
known example of alarm calls is that of alarm calls in vervets provided by
Cheney and Seyfarth in their book ``How monkeys see the world''. Among other
things Cheney and Seyfarth found out that the vervets' alarm calls vary
depending on whether the approaching predator is a leopard or an eagle or a
snake, with a different reaction elicited by the
respective alarm call in each case. With respect to altruism the important question is
whether the alarm call is really given with the intention to warn other
conspecifics as opposed to the possible intention to signal to the predating
animal that it does not need to bother because it has been detected \cite[p.\ 
136/137.]{dugatkin:1997}. But the former is obviously the case as different
alarm calls elicit different escape reactions. As alarm calls are given with a
higher probability either if offspring is present or if mates are present (in
the latter case there exists again a further dependency on the dominance
hierarchy), kinship and byproduct mutualism provide the most plausible
explanations.

That giving alarm signals does not necessarily need to be an instance of
altruistic behavior and not even a form of byproduct mutualism is illustrated
by the stotting behavior that occurs in Thomson's gazelles (and also in some
other less well studied species), a curious kind of behavior ``wherein
individuals take all four legs off the ground simultaneously and hold them
straight and stiff in the air'' \cite[p.\ 94]{dugatkin:1997}. From numerous
hypotheses that have been put forth to explain stotting only two could be
confirmed according to Dugatkin, namely that stotting is meant to inform the
predator of the health of the stotting animal (which means that the predator
will know that the stotting animal will be difficult to catch and will rather
``lock on'' some other individual) and that young animals stott to attract the
attention of their mother in dangerous situations \cite[p.\ 
95]{dugatkin:1997}. In both cases altruism or cooperation is not involved.

\paragraph{Grooming}
\label{grooming}

Most of the examples of cooperative or altruistic behavior among animals so
far have been examples of kin selection or byproduct mutualism, but in spite
of the fact that there is a strong ``skew towards reciprocity in the
theoretical literature'' \cite[p.\ 167]{dugatkin:1997} there have been very few
clearcut cases of reciprocal altruism, let alone of group selection. One kind
of behavior that from its very appearance seems to fit the conception of
reciprocal altruism quite well and is often mentioned as a kind of role model
in this context is that of grooming. Dugatkin relates several studies about
grooming in primates as well as other mammal species. \label{impalaGrooming}
One non-primate species where grooming has been studied are impala, an
antilope species. It is at the same time one of the rare examples that really
fits the model of a repeated game -- at least on a qualitative level.
According to Dugatkin who refers to two studies from Hart and Hart and
Mooring and Hart, impala exchange bouts of grooming, each bout consisting of a
repeated ``upward sweep of the tongue or the lower incisors along the neck of
the partner'' \cite[p.\  91]{dugatkin:1997}. These exchanges of grooming bouts
expose several striking features which strongly suggest that grooming in
impala is an instance of pure reciprocal altruism: 1) There is an almost
perfect match between bouts of grooming received and bouts delivered; 2) the
exchange of bouts ends after one partner stops allogrooming. This rules out
the possibility of byproduct mutualism, which could otherwise offer an
explanation if it is assumed that ticks provide some extra nutrition for the
impala; 3) there is no correlation with the rank in the dominance hierarchy
\cite[p.\ 91-94]{dugatkin:1997}. All in all, this finally seems to be a
clearcut example for the kind of reciprocal altruism that is described by the
repeated Prisoner's Dilemma model. However, even in this case the match
between model and empirical reality can be ascertained only on the basis of
qualitative similarity because a quantitative measurement of the payoff
parameters has not been done.

Grooming is also one of the most salient behavioral features of our closest
relatives in the animal world, the primates, and therefore has caught a lot of
attention by researchers.\label{primateGrooming} The patterns of grooming
exchanges among primates are much more complex than among the impala just
described. In primates, grooming can serve many different functions next to the
purpose of removing ectoparasites. Among these are the reduction of tension
(which could otherwise result in conflicts), coalition formation, where
grooming serves as a means to ``bribe'' others to become allies, and, more
general, grooming as an ``exchange currency'' to gain other favors in return.
While all these describe possible benefits of grooming, Dugatkin notices that
in most studies very little is said about the costs of grooming \cite[p.\ 
117]{dugatkin:1997}. But certainly there are costs. Apart from the time and
energy spent, it has been recorded that the lowered attention of mothers
engaged in grooming activities results in their unattended offspring being
significantly more often being harassed by other animals \cite[p.\ 
117/118]{dugatkin:1997}. There is good evidence that grooming is to a certain
degree reciprocal in chimpanzees, though the reciprocal nature of grooming is
not as clear cut as in the case of impala. In vervets (Meerkatzen) the
relation of grooming and coalition forming has been studied. Here grooming
does increase the probability of responding to solicitation calls for
unrelated animals but not for related animals (where the probability of
responding is high, anyway).  These results are not completely undisputed
\cite[p.\ 120]{dugatkin:1997}, but if they are true, then it appears to be a
case of reciprocal altruism because kinship can be ruled out and, as there
exists an opportunity for cheating (groomed animals could fail to respond to
solicitation calls), byproduct mutualism can be ruled out as well. Further
kinds of grooming in exchange for ``goods and services'' have been documented
in chimpanzees and macaques. In chimpanzees grooming sometimes is related to
food exchange \cite[p.\  123]{dugatkin:1997}. In an experiment conducted by
Stammbach, a single subordinate member of a group of macaques was trained to
operate a complex lever mechanism for food release (from which all group
members could eat).  While the subordinate ``specialist'' did not rise in
rank, it received significantly more grooming than before by other group
members. The acts of grooming did, however, not take place in strict
connection with acts of operating the mechanism \cite[p.\ 124]{dugatkin:1997}.
So, if any kind of reciprocity is involved here, it is not the strict type of
``bookkeeping reciprocity'' that the repeated Prisoner's Dilemma model
suggests. Quite a lot of studies on primates emphasize the factor of kinship
in grooming \cite[p.\ 124]{dugatkin:1997}.

\paragraph{Eusociality}
\label{eusociality}
The most astonishing example of cooperation in the animal kingdom is that
which is found in bee hives or ant hills, where a large state of insects
operates in what appears to be an extremely cooperative and coordinated
manner. Biologists call these kinds of insects {\em eusocial insects}, where
{\em eusociality} is defined by three criteria: 1) Reproductive division of
labour, 2) communal care for the young and 3) overlapping generations of
workers in the colony. Eusociality is not only found in insect species like
bees, wasps, ants, termites but also in certain vertebrates like naked mole
rats and Darmland mole rats. When one compares the forms of cooperation that
take place in eusocial animals with the other instances of cooperative
behavior that have been described in this chapter one cannot help but notice
the extraordinary qualitative difference that eusociality makes for
cooperation and altruism. Eusocial animals do not just cooperate with respect
to a single function (like grooming in mammals) but they seem to cooperate in
any possible form and manner. Of the many possible examples of cooperative
behavior among eusocial insects, Dugatkin describes in more detail the
cooperative behavior of honey bees in foraging, hive thermo-regulation and
anti-predator behavior.  When foraging, honey bees cooperate in different
ways. They inform each other about the location of food resources via the
famous ``waggle dance'' and they coordinate their foraging activity with
regard to the level of food supply in the hive in a complex manner \cite[p.\ 
152/153]{dugatkin:1997}. Hive thermo-regulation is achieved by the bees
behaving in such a way as to keep the temperature inside the bee hive at an
ideal 35 degrees Celsius. As the temperature of the whole hive only marginally
depends on the activity of a single bee, this raises a typical collective
goods problem, where one would expect that the individual bees are encouraged
to cheat. But in fact they do not \cite[p.\ 154/155]{dugatkin:1997}. Even more
admirable is the self sacrificial behavior of honey bees for the defense of
their colony. Because honey bees die when stinging, this behavior appears to
be an extreme case of altruism to the advantage of the colony.

How is the astonishing variety of forms of cooperative behavior as well as
the intensity that altruistic behavior reaches in eusocial animals to be
explained? The best known explanation is that by inclusive fitness. It has
been found out that eusocial insects are haplodiploid species, where the
males carry only a single (haploid) set of chromosomes while the females have
a double (diploid) set of chromosomes. The female descendants of the queen all
share the same genes from their father and on average 50\% of their mother's
genes. In consequence, the worker sisters are on average 75\% related to each
other. Thus cooperation in eusocial insects is easily explained by kinship,
one should think. But there are problems with applying the
inclusive-fitness-theory to eusocial animals. One problem is that there exist
eusocial species where the queen has multiple matings and others where there
are several queens in one colony \cite[p.\ 144]{dugatkin:1997}. Therefore,
kinship cannot be the only explanation for eusociality. Dugatkin
discusses in this context a number of alternative hypotheses on eusociality
\cite[p.\  144-149]{dugatkin:1997}. But rather than entering into the complex
debate about these hypotheses, which for a layman would be difficult to
present accurately anyway, I confine myself to a few general reflections
on eusociality as an example for the evolution of cooperation.

In order to do so, I distinguish between two different questions: 1) Why
do the workers in the colonies not reproduce? Or in other words,
why did centralized reproduction evolve and how is it maintained? 2) Given
that the workers cannot reproduce, why do they cooperate? I am
going to answer the second question first because it seems to be an almost
trivial question. If, for whatever concrete reason, the workers really cannot
reproduce individually, then it follows that the best thing they can
do to spread their genes is to cooperate as well and as completely as possible
with the rest of the colony. For, imagine that due to a mutation some of the
worker ants hatching in an anthill were lazy ants that did nothing to
contribute to the colony. Then although the lazy ants would greatly profit
from letting the others do all the work, they would not be able transform this
advantage into greater reproductive success within the hive simply because
they cannot reproduce themselves. At the same time the anthill as a whole
would suffer increased selection pressure from other anthills without lazy
ants. One could say that the scenario that explains the cooperation within
eusocial species is that of group selection, only that the within-group
selection that counteracts the evolution of altruism in group selection models
is inhibited.  Therefore, in order to produce altruism, evolution only has to
solve the technical problem of coordinating the behavior of the eusocial
insects as well as possible but evolution does not have to resolve a conflict
of reproductive interests any more, which in non-eusocial species acts against
the emergence of altruism. This explains both the extraordinary intensity of
altruistic behavior (up to self-sacrifice!)  as well as the great variety of
cooperative behavior in eusocial species.  Strictly speaking, however, our
definition of altruism in chapter \ref{altruismDefinition} would preclude
calling the cooperative behavior of eusocial insects altruistic if the
``benefits'' in the definition are understood in terms of reproductive
fitness. Because the workers in a colony do not reproduce, no
fitness costs are incurred by them by acting altruistically.

Given that the altruistic behavior of eusocial animals is easily explained by
(uninhibited) group selection, the remaining question is, how did the workers
ever become so altruistic as to stop reproducing individually and
why do they remain so? It is in answer to this question that other mechanisms
like inclusive fitness or byproduct mutualism come into play. In mole rats,
Dugatkin maintains, it was byproduct mutualism forwarded by harsh
environmental conditions such as successive prolonged droughts in the
evolutionary history of certain mole rat species that caused the evolution of
eusociality:

\begin{quotation}
  ... at the evolutionary onset of cooperation in naked mole rats, when
  reproductive division of labor was likely minimal, a ``harsh environment''
  central to byproduct mutualism, rather than kinship per se, may have been
  the predominant selective agent. \cite[p.\ 106]{dugatkin:1997}
\end{quotation}

Differently from typical eusocial insect species, mole rats have a diploid set of
chromosomes, which once more shows that eusociality does not by necessity
depend on the genetics of a haplodiploid set of chromosomes. Still, it is
plausible to assume that the close kinship ties in haplodiploid species
facilitate the evolutionary transition to a reproductive division of labor
because the fitness cost of giving up individual reproduction in favor of
centralized reproduction in a colony is much lower if the relatedness is
close. The mechanisms by which the reproductive division of labor is
maintained do -- as one should expect -- also vary from species to species.
For honeybees, for example, a mechanism called ``worker 'policing' '' has been
described, where the males that hatch from worker laid eggs\footnote{In
  honeybees workers lay eggs, but these are unfertilized and only develop into
  males, whereas the queen can control which of her eggs are fertilized and
  thus develop into females and which are not fertilized and develop into
  males.} are killed by other workers. The behavior is probably best
explained by kinship. (If the queen has multiple matings, workers are more
related to their brothers than to their nephews \cite[p.\ 
150]{dugatkin:1997}.) But Dugatkin also suggests that group selection may play
a role ``in that without policing a much greater degree of within-colony
aggression would exist, and this, in turn, could decrease group productivity''
\cite[p.\ 151]{dugatkin:1997}. Another obvious way to ensure the monopoly of
reproduction is aggression on part of the queen by which the workers are
coerced into their role. This has been reported for the previously mentioned
mole rats \cite[p.\ 106]{dugatkin:1997}.

If the alleged altruism of eusocial species is easily explained by the
reproductive division of labor, then the cooperation of several queens in one
colony must still be explained by the other mechanisms of the evolution of
altruism. And indeed, here we can find some striking cases of reciprocal
altruism and even group selection. One such case is the ``social contract''
that is found in paper wasps ({\em polistes fuscatus}) \cite[p.\ 
157/158]{dugatkin:1997}. In paper wasps dominant queens tolerate other,
subordinate queens in their nest. Both dominant and subordinate queens lay
queen-destined as well as worker-destined eggs. But subordinate queens
disappear by the time the workers emerge. Cooperation between dominant and
subordinate queens requires that they leave each other's eggs unharmed.
Experimental research has shown that subordinate queens reacted aggressively
to simulated oophagy on queen destined eggs, but not on worker destined eggs,
while the dominant queen did not show such a reaction. This strongly hints to
reciprocal altruism on part of the subordinate queens. The suggested reason
why dominant queens do not react to simulated oophagy at all is that they can
still produce queen-destined eggs after the subordinates are gone, while the
subordinates themselves do not get a second chance. For the dominant queen it
is a different deal, so to speak.

An example of cooperation between colony founding queens that is probably due
to group selection can be found in desert seed harvester ants ({\em Messor
  pergandei}) \cite[p.\ 159]{dugatkin:1997}. For some populations of this
species it has been observed that the queens jointly produce workers when
founding a colony. Once the workers have emerged, the queens fight to the death
until only one queen is left. Another feature of the desert seed harvester ant
is that different colonies are engaged in brood raiding against each other.
According to Dugatkin's account, the following holds:

\begin{quotation}
  In the case of {\em M. pergandei}, the trait of interest is the production
  of workers, which, although selected against within groups (via the cheater
  problem), may be selected for as groups with many cooperators survive brood
  raiding (i.e.\ differential productivity of groups). \cite[p.\ 
  160]{dugatkin:1997}
\end{quotation}

As the relative isolation of groups is a vital requirement for group selection
to operate towards the evolution of cooperation, it is no surprise that the
cooperative behavior only occurs in populations of {\em M. pergandei} ``where
environmental factors aggregate starting colonies, which occur only in the
sandy ravine bottoms where soil moisture is available'' \cite[p.\ 
160]{dugatkin:1997}. Other populations of the same species that live in
different habitats do not display cooperative behavior in the founding phase
of a colony, but here queens react aggressively to any rival right from the
beginning \cite[p.\  160/161]{dugatkin:1997}. The conclusion that cooperation
in {\em M. perganei} is a result of group selection has not gone completely
undisputed however. As in this case -- just as in any other of the empirical
instances of the evolution of cooperation in biology described so far -- no
quantitative measurement of payoffs could be made, it is of course difficult
to assess these findings beyond what can be deduced from the mere
phenomenology of this instance of cooperation. Still, similar results have
also been obtained for another ant species, {\em Acromymes versicolor}
\cite[p.\ 161]{dugatkin:1997}, which bestows the explanation by group selection
in this case with some additional credibility.

\subsubsection{Discussion: Do the computer models of altruism live up to the
empirical research in biology?}

The list of examples of cooperative and altruistic behavior among animals
that has just been given is, of course, far from being complete. Still, it
shows how far reaching and varied the forms of cooperative behavior that
exist in nature are. But apart from this scientific fact, which is
certainly interesting in its own right, our main concern here is to find out
in how far the kind of modeling of altruism that has been demonstrated in the
previous chapter proves to be helpful for the understanding of the empirical
instances of altruism and, if not, what are the causes for this failure. In
order to tackle these questions we must distinguish different levels of the
application of formal models and in particular of computer simulations to the
empirical problem:\label{twoLevelDistinction}

\begin{enumerate}

\item {\em Conceptual Level}: On this level the model is merely meant to
  demonstrate how a certain mechanism works in principle. For this purpose it
  is not necessary that the model is empirically very adequate or that the
  parameter values used in the model are based on more than plausible
  assumptions. Still, the model cannot be arbitrary. It must at least give us
  some indication of how the empirical phenomenon can be identified as one
  that falls within the class of phenomena which the model describes. For
example, repeated Prisoner's Dilemma models of reciprocal altruism indicate
that there must be repeated interaction and that the situation should be a
  (repeated) dilemma situation, not just one where the participants profit
  from their interaction anyway, as in byproduct mutualism. This alone -- as
  the previous brief survey of empirical examples has shown -- can already
  be difficult to determine.

\item {\em Application Level}: At this level we require that there is a close
  concordance between the model itself and the empirical phenomenon or class
  of phenomena that the model describes (or ``models''). The concordance must
  be close enough so that we can empirically determine 1) whether the model
  applies to the empirical phenomena in question and 2) whether it describes
  them correctly.  If the model contains quantitative magnitudes as input or
  output values then this implies that we must be able to measure these
  magnitudes in some way or other.

\end{enumerate}

We will elaborate on these two categories of models a little more in chapter
\ref{limitsOfModeling}. Here the distinction is made mainly to preclude a
certain defense strategy that is often used to excuse spurious modeling. This
defense strategy consists in replying, whenever somebody calls into question
that the model fits empirical reality, that it is just a model and that from a
model, being by definition a strongly simplified representation of reality, one
cannot expect a representation of the modeled empirical situation that is
accurate in every possible respect.  However, as not every model can be a
model for anything, there must be a limit up to which this excuse is
acceptable. And this limit certainly depends on what claim is connected with
the model. If the claim is that the model can actually be applied, the
requirements are certainly higher than when it is just meant to give
expression to a certain idea or concept.

Regarding the empirical examples from biology that have been presented so far,
it can safely be concluded that {\em not a single one} of the simulation
models of the kind that have been presented in chapter \ref{modeling} proved
to be applicable in a strict sense. In the beginning of his book on
``Cooperation among Animals'' \cite[]{dugatkin:1997} Dugatkin lists a whole
array of such models. But even though he is extremely sympathetic towards this
approach, he almost nowhere in his book refers to any of these models.
There is no instance -- except one which ultimately turned out to be a failure
(see chapter \ref{sticklebacks}) -- where the empirical research he presents
is related or can be related to any of the theoretical simulation models. The
reasons for this are hinted at by Dugatkin himself in the last chapter of his
book: Save for one exception, Dugatkin was not able to present a single
empirical study where the payoff parameters, which are crucial for the
application of any game theoretical model, have been or could be measured.
\label{bluejays} The one exception concerns an experimental study on blue jays,
where blue jays could trigger a ``cooperate'' or a ``defect'' button \cite[p.\ 
80/81]{dugatkin:1997} and thereby release food according to a Prisoner's Dilemma
game matrix or -- in a second experiment -- according to a stag hunt game
matrix (which is one way to circumscribe byproduct mutualism in game
theoretical terms). The result was that blue jays never cooperated in the
Prisoner's Dilemma, even though it was repeated, and always cooperated in the
stag hunt game. The authors of the experiment concluded that no strategies for
interaction in the repeated Prisoner's Dilemma have evolved in blue jays,
which leads them to doubt the ``general significance of the Prisoner's Dilemma
as a model of non-kin cooperation.'' \cite[quoted p.\ 80]{dugatkin:1997}.
Notwithstanding this skeptical conclusion about the Prisoner's Dilemma as a
proper model for non-kin cooperation, Dugatkin regards it at least as a
serious attempt to address the issue of quantifying the payoff matrix \cite[p.\ 
165]{dugatkin:1997}.  This can surely be granted, but it is still a long way
until a satisfatory mode of quantification will be reached. For, in
order to quantify the payoff matrix we would need to know the payoff values in
terms of reproductive fitness and not merely in terms of food release, which
does most probably not transform proportionally into relative numbers of
offspring.

If this was the only example where the empirical research was approaching the
measurement of payoff paramaters and if -- as we have seen in chapter
\ref{modeling} -- the computer models of altruism crucially depend on the
values of the payoff parameters then this means that the level of empirical
applicability of these models has not yet been reached -- at least not at the
time when Dugatkin compiled his surveying study on ``Cooperation among
Animals'' (1997).\footnote{This still seems to be true today (see the
  following section).}

But what about the conceptual level? If the computer models are not (yet)
really applicable, do they perhaps help us to form sound concepts and provide
us with categories of analysis? Even on the conceptual level, it has in many
cases been difficult to decide which type of altruism is at work in a specific
case and whether it is altruism at all and not merely byproduct mutualism. At
the same time, game theoretical models (though not game theoretical models
alone) allow for a relatively sharp conceptualization of different types of
altruism, which is helpful even if these types do in many instances not appear
in a pure form in nature (grooming among impala being one of the few
exceptions). One could say that on this level they serve a similar function as
the ``ideal types'' do in the social sciences according to Max Weber: Even
though they contain very strong abstractions they can help to get a better
grip on empirical reality. The heuristic benefits of game theoretical thinking
for the understanding of altruism become apparent in the case of grooming
among primates. Here, as Dugatkin notices \cite[p.\ 117]{dugatkin:1997},
behavioral ecologists have mostly focused on the benefits of grooming but not
often asked the question of the costs of this type of behavior. This is quite
understandable from the point of view of behavioral ecologists because from
its very appearance the grooming behavior does more strongly suggest to ask
the question of what it is good for than the question of its costs (which even
might seem quite negligible at first sight). But from the theoretical
perspective it is clear that the question of why this kind of potentially
altruistic behavior evolved is a question of benefits {\em and} costs. Thus,
theoretical reflection on models of altruism, even if they are toy models, may
help to direct the empirical research in a useful manner.

This said, there is of course an important caveat that has to be mentioned
right away. The benefits just described of modeling on the conceptual level
(clarifying and sharpening our concepts, directing empirical research) only
hold for the most elementary and simple models, but not for complicated
models, massive simulations and in general the whole baroque richesse of
theoretical models and simulations that can be derived from any simple model
by changing parameters, adding further ``plausible'' conditions etc. Judged
against the background of the empirical findings that are summarized in
Dugatkin's book (which, after all, is the book of an author who is very
sympathetic towards the modeling approach), simulations in the fashion of
those of which a small sample has been discussed in chapter
\ref{simulationsOverview} and of which a role model has been presented in
detail in chapter \ref{refinedModel} have turned out to be as good as
completely useless. Neither did they provide us with important insights on the
conceptual level that went beyond what can already be demonstrated by much
simpler toy models, nor was any simulation of this type empirically applicable
in the sense described above.

Given that the simulation models turned out to be largely useless for the
explanation of the evolution of altruism in nature, the question is, of course,
what are the reasons for this deplorable state of affairs. One possible
explanation could be that most of the empirical research surveyed in
Dugatkin's book was not designed to put any particular models of altruism or
cooperation to the test, but that the behavioral ecologists conducting such
research had other research interests. This might be especially true
for the field research on cooperative behavior as opposed to the
experimental research. Usually, there exists a time lag with which newly
invented concepts and methods pervade a whole science. If this was
true then maybe the only problem was that Dugatkin wrote his book too early,
at a time when only a small part of the empirical research was informed by the
latest models of the evolution of altruism? But then we should expect to find
more usage of simulation models in the empirical research on altruism that has
been published since. In order to check whether this is the case we will
briefly examine a more recent example of the empirical research on altruism in
the following section (section \ref{recentResearch}). It will turn out that
just as little use could be made of simulation models as in any of Dugatkin's
examples. In order to further pinpoint the difficulties that prevent the
application of simulation models, or, more precisely, the brand of simulation
models that has dominated the modeling of the ``evolution of altruism'' for a
long time, I finally discuss in depth one of the few examples where
biologists set out with Axelrod's and Hamilton's concept of reciprocal
altruism but soon became aware of the limits of this theoretical background
(see chapter \ref{sticklebacks}).

\subsection{A more recent example: Image scoring cleaner fish}
\label{recentResearch}

The discussion of Dugatkin's survey on ``Cooperation among Animals'' has shown
that there is a wide gap between the modeling of altruism and cooperation on the
one hand and the empirical research on cooperative behavior among animals on
the other hand. While the theoretical models did allow formulating certain
concepts of altruism, it was not possible to relate the simulation models of
altruism to the empirical instances of cooperative behavior in any more than
a metaphorical sense. But is this limitation due to systematic difficulties of
applying abstract simulation models or is it, maybe, just an interim problem
that can ultimately be overcome by more refined empirical research methods?
Since Dugatkin's survey dates from 1997 it is reasonable to ask whether the
situation has changed till then. Therefore, we will look at one recent
example of empirical research on altruism in biology. Again, the purpose of
the discussion of this example is primarily epistemological. No claim is
made that the examples discussed in the following, concern very important or
representative types of altruism in nature (although they fit well into the
overview of animal altruism given previously).  We want to find out, how much
use is made of theoretical models of altruism in typical empirical research
studies.

The study concerns ``Image scoring and cooperation in a cleaner fish
mutualism'' \cite[]{bshary-grutter:2006}. {\em Image scoring} is variant of
reciprocal altruism, where cooperation depends on whether the partner has been
seen to cooperate with others. Image scoring is thus a type of indirect
reciprocity because it is the altruistic act that has been bestowed unto
someone else that is being reciprocated. The rationale behind indirect
reciprocity is that someone who has behaved cooperatively towards someone else
may also behave cooperatively to oneself. Another type of indirect reciprocity
that does only occur among humans is reputation based cooperation, where one
gains reputation by cooperating with people that have a high
reputation. Differently from mere image scoring, reputation can be passed on by
telling about it. Image scoring only requires that the partner's behavior is
observed in a similar situation. In contrast to reputation based cooperation
the cognitive requirements for image scoring are therefore only comparatively
low. In fact they may be even lower than the cognitive requirements for the
evolution of altruism in repeated Prisoner's Dilemma situations because for
image scoring no bookkeeping or partner recognition is required so that it
does not come as a surprise that image scoring behavior can be found even
among relatively ``primitive'' animals.

In the cleaner fish {\em Labroides dimidiatus} (also known as Striped Cleaner
Wrasse, or in German: ``Putzerlippfisch'') that Bshary and Grutter
experimented with, the clients ``invite'' the cleaner fish for inspection.
The cleaner fish then usually feed upon the ectoparasites of the client. But
they could also feed on the mucus of the client and there is evidence that the
mucus is actually their preferred nourishment. Thus, the cleaner fish can
either cooperate by removing the ectoparasites or cheat by munching the
client's mucus. The client on the other hand cannot cheat the cleaners. Due to
the asymmetry of the situation, cooperation could not have been evolved via
direct reciprocity. That image scoring is a potential candidate for the
explanation of cleaner fish cooperation is suggested by field research on
cleaner fish according to which: ``Client fish almost always invite a
cleaner's inspection if they witnessed that the cleaner's last interaction ended
without conflict, invite less if they do not have such knowledge, and invite the
least if the last interaction ended with conflict.''
\cite[p.\ 975]{bshary-grutter:2006}.

In order to test the image scoring hypothesis Bshary and Grutter conducted two
experiments, one on the client behavior and one on the behavior of the
cleaner fish. In the first of these experiments a client was placed in the
middle of an aquarium divided by one-way mirrors into three basins. In one of
the side basins a group of cleaner fish fed on prawns attached to a model
client fish. In the other side basin a group of cleaners was placed without a
model. The result of this experiment was that the client spent significantly
more time near the group of cleaners that was engaged in cleaning activity.
This result suggests the conclusion that clients prefer cleaners that can be
observed to be cooperative over cleaners with an unknown cooperation level.

The second experiment was more complicated. This time the cleaners were placed
in either an image scoring or a non image scoring scenario. In both
scenarios the client fish was simulated by plates to each of which two different types
of food items, fish flakes and prawn items, were attached. {\em Labroides
  dimidiatus} prefers prawns to fish flakes just like it prefers mucus to
ectoparasites. The question that the experiment was intended to answer was
whether the cleaner fish would cooperate by feeding against their preferences
in the image scoring scenario. In both the image scoring and the non image
scoring scenario the cleaner fish could feed from two identical plates. In the image
scoring scenario both plates would be removed immediately after one prawn item
was eaten from one of the plates, while in the non image scoring scenario only
the plate from which the prawn item was eaten was removed. To make sure that
the cooperative or non cooperative behavior did not merely depend on the
sheer amount of nourishment available a third scenario was tested, where the
cleaner fish could feed only on one plate which was also removed immediately
after a prawn item was eaten. The result was that in the image-scoring
scenario the cleaner fish fed significantly more often against their
preference when feeding on the first plate than when feeding on the second
plate or a single plate or when feeding on the first plate in the
non-image-scoring scenario.

The experimental results thus strengthen the assumption that cooperation in
cleaner fish is due to image scoring. It is noteworthy that the cleaner
fish do not merely react to the presence of another client, a condition
which was fulfilled in the image scoring and the non image scoring
scenario, but to the reaction of the other client that is present. This means
that the cleaner fish do only cooperate if the clients actually engage in
image scoring.

Now the crucial question for our purpose, the assessment of the value of
theoretical models for the empirical research, is whether and to what level
Bshary and Grutter could draw upon theoretical models of the evolution of
cooperation. Bshary and Grutter do not make more than passing mention of the
mathematical models and computer simulations on image scoring \cite[p.\ 
975]{bshary-grutter:2006}. Not to enter upon a discussion of these models is
quite reasonable for them as the specific features of these models remain
completely irrelevant for their empirical research. It is only the basic
concept of indirect reciprocity that Bshary and Grutter draw upon for their
empirical research. The concept of simple indirect reciprocity requires
``image scoring by clients and an increased level of cooperation by cleaners
in the presence of image-scoring clients'' \cite[p.\ 796]{bshary-grutter:2006}.
Both these requirements have been tested experimentally by Bshary and Grutter.
Again, we find a concordance of theoretical modeling and empirical research
only on a basic conceptual level.


\subsection{An in-depth example: Do sticklebacks play the repeated
  Prisoner's Dilemma?}
\label{sticklebacks}

% In most of the examples that have been discussed before, not a
% particular model has been employed empirically (that is by measuring
% the input parameters in the field, determining the output in the model
% and then comparing the results of the model with observations in the
% field) but only the general ideas that the model simulations suggest
% have been used to describe the empirical situation. This procedure is
% hardly a textbook example for the empirical testing of a model. Yet,
% it is to some degree legitimate because in most cases discussed the
% explanation of the animal behaviour in terms of reciprocal altruism is
% the best available explanation from a limited number of possible
% explanations. In biology, we know that any explanation for a born in
% behavioural trait must be an evolutionary explanation. There exist
% basically only three explanations for altruistic traits within the
% evolutionary framework: reciprocal altruism, kin selection or group
% selection.\footnote{Byproduct mutualism, which some authors add as a
% fourth explanation \cite[]{dugatkin:1997}, has been omitted here
% because it does not fit the definition of altruism used here (see
% \ref{byproduct_mutualism}).} If we can rule out the other two then
% reciprocal altruism is left as the only possible explanation.  But,
% although it is possible to proceed in this way, it would still be very
% desirable if we could get down to actually testing our models.
% Unfortunately, as the following exmple shows, this is where the
% trouble starts.

In order to show what difficulties the attempt to apply the models of
reciprocal altruism meets in practice, I discuss in the following an
example where biologists tried to apply the theory of the ``evolution of
cooperation'' of Axelrod and Hamilton \cite[]{axelrod:1984} (which is based on
computer simulations that have been a role model to the ones presented above)
to a case of altruistic behavior in nature. The example concerns a
behavioral trait called ``predator inspection'' that is found in certain
types of shoal fish like sticklebacks.  The behavior of ``predator
inspection'' has among others been examined in two empirical studies by
Manfred Milinski and Milinski and Geoffrey Parker. The earlier of these two
studies \cite[]{milinski:1987} still draws heavily on Axelrod's and Hamilton's
model of the repeated Prisoner's Dilemma. The other study that has been
described in a paper that appeared ten years later
\cite[]{milinski-parker:1997} and employs a totally different theoretical
interpretation of the results. As I try to demonstrate in the following,
both studies taken together show that the choice of an appropriate formal
description of reciprocal altruism (or cooperation) raises very difficult
and often by no means unambiguous questions of interpretation and measurement.
Against this background any game theoretical model research that is not
closely linked to empirical questions must appear like a pure
``Glasperlenspiel''.\footnote{That this has nothing to do with the usual gap
  between theory and practice or between theoretical and empirical research
  but reflects a specific impasse of the modeling approaches in evolutionary
  game theory will have become clear at the end of this section and will be
  discussed again in chapter \ref{limitsOfModeling}.}

``Predator inspection'' is a behavior that is found (among other species) in
sticklebacks. Sticklebacks are small fish living in shoals. If a predator (a
pike for example) comes within a certain range of the shoal, it can be
observed that either a single stickleback or a pair of sticklebacks leaves the
shoal and carefully approaches the predator. The sticklebacks do so in order
to inspect the predator, presumably to gain information about the type, size,
location and movement of the predator. Typically, a pair of sticklebacks gets
much closer to the predator than a single stickleback. If the sticklebacks
approach as a pair, it can be observed that they advance with characteristic
jerky movements in such a way that one stickleback swims a short distance
ahead and then ``waits'' for the other, who follows in a similar jerky
movement \cite[p.\  433]{milinski:1987}.  This suggests interpreting the sequence of jerky
movements as a repeated Prisoner's Dilemma, where the sticklebacks play {\em
  Tit for Tat}. In his earlier paper Milinski tried to confirm this assumption
by simulating the partner stickleback with different types of mirrors so that
the mirrored fish either appeared at the same distance from the predator
(simulating a cooperative partner) or a little bit further behind (simulating
a non cooperative partner).  The result was that the sticklebacks advanced
much closer to the predator when they were accompanied by a cooperative
partner. Milinski interpreted this result as an empirical confirmation of
Axelrod's and Hamilton's theory of cooperation. By and large this seems
correct if we ignore for a moment the fact that the results of Axelrod's and
Hamilton's simulations were more contingent than was known at that time. But
there exists a problem in so far that Milinski confines himself to assessing
that the two inequalities $T>R>P>S$ and $2R > T+S$ hold. Now, as
the simulation results above show, the simulation is sensitive to changes in
the concrete values of the payoff parameters, and unfortunately these would be
very hard to measure in the case of the sticklebacks.

After much further experimental research on sticklebacks in the later paper, 
Milinski and Parker offer quite a different formal description of the same
behavioral trait of ``predator inspection''. There is not much talk about the
repeated Prisoner's Dilemma any more. While it is still true that the
situation of two sticklebacks approaching a predator can (at a certain
distance range) be interpreted as a Prisoner's Dilemma, this assertion alone
does not shed much light on the problem. Instead of meddling with the
Prisoner's Dilemma, Milinski and Parker therefore examined the possible
utility calculus that controls the behavior of the sticklebacks.\footnote{In
  the following Milinski's and Parker's construction will only be described in
  general terms. For the mathematical details see \cite{milinski-parker:1997}.
  A major problem of this construction, which is also the reason why Milinski
  and Parker only reach an ambiguous conclusion, is that the fitness benefits
  of inspection can only be guessed. While it is plausible to assume that the
  benefits decrease with decreasing distance from the predator, there exist
  no exact measurement procedures for the benefits. Therefore, both the type of
the function (Milinski and Parker present two alternatives, an exponentially
  decreasing and a linearly decreasing function) and its parameter values can
  only be guessed. -- In response to a criticism that appeared slightly
  earlier, Dugatkin, who worked theoretically and empirically on the same topic
  as Milinski, still defends the notion that predator inspection behavior is
  best understood as a {\em Tit for Tat} strategy \cite[]{dugatkin:1996}. But
  he misses out the problem that the respective Prisoner's Dilemma models are
  notoriously unstable and he seems to assume that there exist only the two
  alternatives to explain the behavior of predator inspection either as the
  outcome of a reiterated Prisoner's Dilemma or as byproduct mutualism. But
  as the later paper from Milinski and Parker \cite[]{milinski-parker:1997}
  suggests, these are not the only alternatives to conceive of predator
  inspection (see the main text below).} According to Milinski and Parker,
even a single stickleback will approach a predator up to the point where the
advantages (of gaining information about the predator) are balanced by the
risk of being eaten \cite[p.\  1241/1242]{milinski-parker:1997}. For the case
when two sticklebacks jointly approach the predator, Milinski and Parker offer
two alternative descriptions one that assumes cooperation \cite[p.\ 
1242]{milinski-parker:1997} and another one that does not necessarily
presuppose cooperation \cite[p.\  1242-1245]{milinski-parker:1997}. Milinski
and Parker do not ultimately reach a decision which of these descriptions is
the right one. For, even if one does not assume cooperation, two fish will
-- according to their model description -- move closer to the predator than a
single fish. The reason is this: The distance to the predator can be divided
into three zones, the ``far zone'', the ``match zone'' and the ``near zone''.
In the ``far zone'' that is, when the distance to the predator is still very
great, each of the two fish gets an advantage from moving closer to the
predator, even if the other fish stays back. In the ``match zone'' (medium
distance to the predator) a partner that has fallen behind will try to catch
up with its forerunner, although neither of the two partners gets an advantage
from taking the lead (from which it follows that both fish can only advance
synchronously if one does not assume at least a minimum of reciprocal
altruism). Finally, in the ``near zone'' the ``best reply'' of each fish is to
stay back behind the other one.

If there are two different theoretical descriptions of the behavior of a pair
of ``inspecting'' sticklebacks, one that assumes cooperation between the
sticklebacks and one that does not, then this raises the question which of
these is true or whether the sticklebacks in reality cooperate or do not
cooperate when jointly inspecting a predator. At the time of writing the
second paper Milinski and Parker come to the conclusion that the current state
of research does not allow to decide this question: ``However, it is not yet
possible to analyze quantitatively whether pairs are conforming to the
cooperative or non-cooperative ESS [{\bf E}volutionary {\bf S}table {\bf
S}trategy, E.A.].'' \cite[p.\ 1245]{milinski-parker:1997} How
can this result be reconciled with their earlier study that seemed to confirm
Axelrod's and Hamilton's theory of the ``evolution of cooperation''? The
answer is that obviously the earlier conclusions have been drawn too rashly,
probably due to a subtle misconception in the earlier experiment's setup: An
uncooperative fitness maximizing fish would never have behaved as the
uncooperative fish simulated by the mirror did. Therefore, the reaction of the
real fish that stopped at a specific distance from the predator does not
necessarily need to be interpreted as a ``punishment'' which is part of a {\em
  Tit for Tat} like strategy. The distance at which the real fish stopped may
just have been its optimal distance (from a purely ``egoistic'' point of view)
given the presence and distance of the simulated partner fish.

The result shows how difficult it is, even in a biological context, to apply
simulation models of reciprocal altruism such as those described above. 
The repeated Prisoner's Dilemma does not seem to be an
appropriate model for the sort of behavior Milinski examined. As has been
shown previously, other examples for reciprocal altruism from biology meet the
same difficulties.  The same conclusion is confirmed by other biologists that
work in the field of evolutionary game theory. An expert in this field, Peter
Hammerstein, writes: ``Why is there such a discrepancy between theory and
facts? A look at the best known examples of reciprocity shows that simple
models of repeated games do not properly reflect the natural circumstances
under which evolution takes place. Most repeated animal interactions do not
even correspond to repeated games.'' \cite[p.\ 83]{hammerstein:2003} In face of
the vast multitude of models of reciprocal altruism and the ``evolution of
cooperation'' this is a rather sobering conclusion.  Yet, it must be taken
seriously. And if it is taken seriously, it strongly confirms the skepticism
towards purely theoretical simulations that has already been expressed
earlier. As it appears, ``blind modeling'' (that is modeling that is not
informed by empirical research but relies only on plausible assumptions alone)
is not a proper research tool that allows us to find anything out about
reciprocal altruism beyond the merest truisms.

Is there really nothing that can be done about it? In a critical appraisal of
the game theoretical computer simulations in biology, Dugatkin described the
situation roughly as follows: In order for the models to contribute to
scientific progress, models and empirical research must be part of a feedback
loop that is, theoretical models may help to direct empirical research but
then the insights and results of the empirical research must be ``fed back''
into the construction and refinement of models \cite[p.\ 54ff.]{dugatkin:1998}.
Obviously, the feedback loop was not closed, insofar as the bulk of
simulations on the evolution of cooperation did never really take into account
the restrictions and conditions of the empirical research on the subject. The
question of the relation between empirical research and theoretical modeling
will be elaborated a little more in chapter \ref{limitsOfModeling}, where a
{\em build to order principle} of computer modeling will be proposed, according
to which models that aim to go beyond a merely conceptual level should always
be constructed around empirically measurable quantities. That the burden of
accommodation is thus laid on the theoreticians finds its justification in the
fact that much stronger restrictions apply when devising measurement
procedures (including the restriction that only certain quantities can be
measured at all) than for the design of models which has become comparatively
simple with the advent of computers.


