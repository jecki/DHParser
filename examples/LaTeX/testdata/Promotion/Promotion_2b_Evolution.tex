\chapter{The generalized theory of evolution as theoretical framework}
\label{generalizedEvolution}

Having defined altruism, it is now time to discuss the theoretical
framework in which the so defined concept of altruism is to be
applied.  In this book, I discuss altruism in a Darwinian evolutionary
framework with special emphasis on the method of computer simulations.
The application of Darwinian evolutionary concepts to the evolution of
altruism in a cultural context as it is intended by generalized
theories of evolution requires some explanation: While evolutionary
theory in biology is well established, the application of evolutionary
concepts in the social sciences is still the object of much
debate. There exist several different approaches to employing
evolutionary thinking in the social sciences. None of these attempts
goes uncontested and there is of course some dispute whether an
``evolutionary theory'' based on the concept of selection is of much
use in the social sciences at all.  Therefore, the questions
surrounding the application of evolutionary concepts to cultural
developments will be discussed at some length.

In the following, I first define the concepts of ``Darwinian
evolution'' and ``evolutionary theory in a Darwinian sense'' and I
describe how evolutionary explanations work. Then, I discuss in which
areas of science we can make use of evolutionary explanations and I
also briefly touch the question, how they relate to competing non
evolutionary theories.  The answer to this question is trivial only in
biology, where evolutionary theory remains uncontested and where the
explanation for the emergence of any (altruistic) trait must therefore
be found in the realm of evolutionary theory. But this is not the case
in the social sciences, where the employment of evolutionary
explanations requires some justification.  This becomes even more
important as there exist different brands of evolutionary theories in
the social sciences like sociobiology, which relies on genetic
evolution, and theories of cultural evolution, which seek to explain
the development of culture in analogy to biological evolution but not
by the evolution of genes itself.  And there exist mixed forms of both
theories. I confine myself to the discussion of evolutionary
psychology as an example of the genetic brand (section
\ref{geneticEvolution}) and the theory of cultural evolution (section
\ref{memeticalEvolution}), which assumes that the evolution of culture
is a process that proceeds largely independently from genetic
evolution. Alongside with the presentation of these approaches, I am
going to discuss some of the criticism that has been put forward
against these theories and point out possible limitations. Finally, I
discuss the place that the explanations for altruism based on
evolutionary computer simulations have within this theoretical
framework (section \ref{theoryAndModels}).

\section{The concept of Darwinian evolution}

The word ``evolution'', when taken in its most general meaning,
describes a process of change over time which has a determinable
direction. In the following, however, when speaking of ``evolution''
or ``evolutionary processes'' what is meant is always evolution in a
narrower, ``Darwinian'' sense. Evolution in a Darwinian sense is a
process the course of which is determined by the joint effect of three
factors (``Darwinian modules''): {\em reproduction}, {\em variation}
and, most prominently, {\em selection}. We assume here that there
exist some determinable evolving entities upon which these factors
act, resulting in a directed evolutionary process. In more detail
these three factors (or ``Darwinian modules'') can be characterized as
follows:\footnote{The characterization of Darwinian evolution follows
  \cite[p.\  329ff.]{schurz:2001}.}

\begin{enumerate}

\item {\em Reproduction}: There is a set of evolving entities (as for example
  genes in natural evolution or certain cultural traits\footnote{I am not, as
    it is often done, speaking of memes here because it is still doubtful
    whether memes exist as entities.} in cultural evolution) that is
  reproduced in generational cycles.

\item {\em Variation}: In every generation there is a
  certain amount of variation among the evolving entities,
  that is, they differ according to their respective
  properties. Even if some types of entities die out after a
  while, variation may still be kept up by the the
  spontaneous appearance of new types (``mutations'').

\item {\em Selection}: The reproduction rate of the evolving
  entities differs dependent on the interaction of the
  entities' properties with the environment.  Thus the
  environment selects for certain types of entities that can
  then be regarded as ``better adapted'' than other types of
  entities.

\end{enumerate}

These three factors are the defining characteristics of a ``Darwinian
evolutionary process''. It cannot be assumed that these three factors alone
lead to a directed evolutionary process without any further conditions being
fullfilled. For example, evolution can only take place in a more or less
stable environment \cite[p.\   336/336.]{schurz:2001}. Also, there must be a
large enough range of variable types of the evolving entity.  In the case of
the genome, this is certainly true, as there exists an enormous number of
possible combinations of the basic building blocks of the DNA. And one could
easily think of further conditions.  Still we will not make these conditions
part of our definition, but we will speak of ``Darwinian evolution'' if there
is a directed evolutionary process and if this process is determined by at
least the three factors {\em reproduction}, {\em variation} and {\em
  selection} plus potentially further conditions. We will not speak of
``Darwinian evolution'' or ``evolution in a strict sense'' if one or more of
these factors is absent. (To guard oneself against misunderstandings it is
important to keep in mind that especially in the social sciences the terms
``evolution'' and ``evolutionary'' are usually not used in the strict
Darwinian sense. Even where authors assume that they are describing some
social or cultural development process in Darwinian terms it may turn out upon
closer inspection that they are in fact not doing so, but that they are merely
applying some arbitrary selectionist paradigm.)

Having defined the concept of ``Darwinian evolution'' we can now define what a
``Darwinian evolutionary theory'' (or just ``evolutionary theory'') is.  An
``evolutionary theory'' is a theory of a development process that uses the
concept of Darwinian evolution for the explanation of this process. Now, this
raises the question to what empirical processes theories of Darwinian evolution can
reasonably be applied. If we think of evolution in the broadest sense there
are basically three strata of evolution which are of philosophical interest to
us: 1) The evolution of the universe, 2) the evolution of life (including the
evolution of humans), 3) the evolution of culture, i.e.\ human history. It
seems quite obvious that the evolution of the universe is not an instance of
Darwinian evolution. One could of course speak of the evolution of galaxies
and solar systems, but it is hard to envisage how reproduction and selection
come into play here. Therefore, when looking for possible instances of
Darwinian evolution, we must confine ourselves to the evolution of life and
the evolution of culture. The merits of the Darwinian theory of evolution in
the one field and its prospects in the other will be discussed in the
following.

\section{Biological evolution}

Only little needs to be said about biological evolution here. The
theory of evolution has by now for a long time been firmly established
in biology.\footnote{Most of the information about the biological
  theory of evolution that is given in the following is taken from the
  book of Ernst Mayr \cite[]{mayr:2001}.} As is well known it was
first put forward by Charles Darwin to account for the origin and the
diversity of the existing plant and animal species. Very simply put,
one could say it gives answers to three questions: Why is there such a
multitude of different species?  How did each single one of them come
into existence? (And in particular: How did the human race come into
existence?) How come that all of the species are so well adapted to
their respective environment? The answers that the theory of evolution
gives to these question are (last one first): 1) Living beings are so
well adapted to their environment because they have evolved through
natural selection. Those types that are not well adapted (or less well
adapted than other types sharing the same habitat) die out, leaving
just types that have a functional design that makes them well adapted
to their environment. 2) Each single species came into existence by
gradual evolution from (usually) more primitive species.  This also
explains how man came into being and thus gives an answer to one of
the most fundamental philosophical questions. 3) Finally, the variety
of species existing in this world is to be explained by the fact that
species may split due to spatial separation or other causes and then
evolve into different directions, occupying different ecological
niches.

But the theory of evolution is not only able to answer such general questions.
It also has a direct impact on the explanation of the characteristics and
phenotypic traits found in living organisms. In fact the theory of evolution
is the only theory that offers an ultimate explanation for why organisms have
certain traits. A phenotypic trait can be any characteristic feature of the
organism itself or of its behavior (some authors even include the nests and
buildings animals construct like spiders webs, rabbit burrows etc. into the
phenotype of the respective animal \cite[]{dawkins:1982}). Therefore,
behavioral characteristics such as altruistic or egoistic behavior must
also be regarded as part of the phenotype of an organism, and their existence
must be explained on the basis of the theory of evolution. This means that for
any specific trait of any organism it is either true that (1) it is a
functional adaptation to a certain aspect of its habitat and it has evolved by
natural selection or (2) it is a by-product of other adaptations of the
organism to its environment or (3) it is the heritage of an adaptation in the
evolutionary history of the organism or (4) it is the result of sexual
selection.

But why can we be so sure that the explanation of a certain trait must
fall into one of these categories or, more broadly speaking, why can
we be sure that the explanation for the existence of a phenotypic
trait of an organism must be an evolutionary one? The answer to this
question has two parts. One part of the answer is that the theory of
evolution is an extremely well confirmed theory. A broad range of
empirical evidence supports this theory
\cite[p.\  12ff.]{mayr:2001}. This evidence includes among other things
the fossil record, field observations as well as breeding experiments
and molecular genetics. The latter is of particular importance because
at the time when Darwin invented the theory of evolution, the specific
mechanism of inheritance had not yet been found. When the laws of
genetics were discovered this not only solved one of the major riddles
about evolution, namely whether acquired properties are passed on to
the descendants (which is not the case), but it also turned out that
genetics and the theory of evolution fit together like one puzzle
piece to another. If it is possible to link independently confirmed
theories together, as in this case the theory of evolution and the
theory of genetics, then this is always a major scientific
achievement. One could say that the successful linking of theories
strengthens both theories by providing them with additional, {\em
  holistic evidence}.

The other part of the answer to the question why the ultimate explanation for
any phenotypic trait of an organism can only be an evolutionary one consists
in the fact that the theory of evolution is without any competitors. It is the
only theory that can explain why organisms are functionally adapted to their
environment or, to put it in a more philosophical jargon, why organisms expose
a teleological structure. There is no other way to explain this teleological
structure of organisms than by the theory of evolution. This does not mean
that we can give a precise evolutionary explanation for each single instance
of a functional adaptation. But we can always be sure that there exists an
evolutionary explanation. And the reason why we can be sure it exists is that
(1) it is possible to give precise evolutionary explanations for functional
adaptations in many other cases (2) there is no case where the theory of
evolution has been falsified (for example by demonstrating that a certain
trait exists although it reduces fitness and is at the same time not an
artifact of sexual selection) (3) there is, as has just been mentioned, no
alternative theory that could possibly offer an explanation. We are therefore
entitled to assume that even in those cases where we cannot give a precise
account, the ultimate causes must still have been those that are described
by the theory of evolution.

I emphasize the point that there exists no rival to the theory of evolution in
biology so much because this is one of the major differences between
evolutionary theory in biology and evolutionary theory when it is applied to
cultural evolution. There exist many theories that account in one way or other
for cultural developments and evolutionary processes (in a broad sense) in
human history as well as for the functional adaptations or teleological
structures we may find in human cultures. We therefore cannot beforehand
assume that the explanation for any instance of functional adaptation or any
evolutionary process (in the broad sense of ``evolution'') in human culture
must be Darwinian in the above defined sense. Instead, if we want to make the
assertion that a certain feature of culture as for example a social norm or a
certain technology is the product of a Darwinian evolutionary process then we
have to demonstrate that our evolutionary explanation works precisely in this
case and we would have to defend it against possible alternative explanations.

Remaining in the field of biology, how does a ``precise'' evolutionary
explanation for a an evolved trait work? Basically, what evolutionary theory
asserts is the following: (1) There is a connection between the reproduction
rate and the adaptedness of an organism to its environment. (2) All traits that
are too complex to have evolved through a single genetic variation must
have evolved through a closed sequence of variation and selection cycles
with no gaps. These two assertions show, by the way, that the theory of
evolution is not, as it is sometimes charged with,
tautological\label{tautologyCharge} because both assertions can in principle
fail empirically.\footnote{Though the second assertion may be difficult to
falsify because one can always maintain that the intermediary steps filling
an alleged evolutionary gap have not been discovered yet. See also \cite[p.\ 
  335]{schurz:2001} for some remarks about the falsifiability of the theory of
  evolution.} (In fact the first assertion does fail in cases where a trait
has evolved through sexual selection in contrast to natural selection, but it
would lead too far to go into this topic here.) In order to demonstrate that a trait
has evolved in the sense of the Darwinian theory of evolution, what must be
shown is that organisms that have the trait are better adapted to their
environment and do therefore enjoy a higher reproduction rate than members of
the same species that do not possess this trait.  For more complex traits like
specialized organs the evolutionary history must be clarified. The task of
proving that a certain trait confers to its bearer an evolutionary advantage
is often not as easy as it might at first sight seem.  For, in order to give
such a proof the net result of all evolutionary forces acting upon the
organism because of this trait must be taken into account. The difficulties
involved in drawing up an evolutionary explanation for some trait can be
explained with the example of the long neck of a giraffe \cite[p.\ 
37-40]{dupre:2003}: It seems plausible to assume that the long neck of the
giraffe has evolved because it allows the giraffe to eat the leaves of trees
high above the ground. But then a long neck is also a very heavy neck and
should under this aspect probably be regarded as an evolutionary
disadvantage.  In order to precisely explain the fact that a giraffe has a
long neck on an evolutionary basis it would be necessary to give an accurate
account of the possible advantages and disadvantages such a long neck
might have. It is obvious that this is quite a
difficult thing to do, though not necessarily an impossible one.\footnote{As
  Dupré notices \cite[p.\   38]{dupre:2003}, this is much less of a problem in
  the case of specialized organs because here the evolutionary advantage
  (i.e.\ the specific purpose of the organ) is quite obvious.} Of course, we
can be sure that the long neck of the giraffe must have evolved for some such
reason as the advantage of picking leaves from trees. But then, the only
reason why we can assume this is because the theory of evolution has been so
well confirmed in other cases, not because we are able to track down the
selective forces in this particular case. We will later see that it is
precisely the problem of giving a quantitative account of the advantages and
disadvantages of certain types of animal behavior, which makes it so
difficult to test our theoretical assumptions about the evolution of altruism
empirically.\footnote{See chapter \ref{biology}.}  At the same time it is, of
course, all to easy to invent ``evolutionary stories'' about why some trait is
an adaption to the environment. This kind of evolutionary story telling is a
danger that is especially imminent in the application of evolutionary theory
to human culture, to which we will turn our attention now.

\section{Evolutionary theories of culture}
\label{culturalEvolution}

Darwinian evolutionary theories of culture come in many different flavors. In
a recent overview Kevin N. Laland and Gilian R. Brown discuss human
sociobiology, human behavioral ecology, evolutionary psychology, memetics,
gene-culture co-evolution \cite[]{laland-brown:2004}. The multitude of
different approaches alone shows that there is not one canonical way of
applying (Darwinian) evolutionary thinking to human culture. However, all of
these different approaches can be traced back to two basic types: Theories that
explain human behavior and human culture by the evolved genetic nature of man
and theories that assume an autonomous evolutionary process of culture that is
not determined by the human genes. The above mentioned approaches fall either
in the one or the other of these two categories or can be regarded as a mixed
form of both. To simplify matters, I discuss only the
two basic types in the following.

It is important not to forget that the Darwinian evolutionary theories of
culture constitute only a small fraction among the many theories of cultural
evolution or development that exist in the social sciences. For the
understanding and explanation of the process of civilization there are -- to
name just a few arbitrary examples -- the theory that civilization is a
process of rationalization (Max Weber), the theory that civilization is a
process of internalization of external compulsory forces (Norbert Elias),
modernization theories according to which progress in one realm, say,
technology, necessitates progress in other realms, say, governmental structure,
the theory of history as the history of class struggles (Karl Marx), and many
more. The social sciences did not wait for Darwinian theories of evolution
to arrive in order to explain functionalistic (or ``teleological'') structures
in the realm of human culture. Also, since Darwin's ``Origin of Species''
\cite[]{darwin:1859} there have been many attempts to apply Darwinian
approaches to human culture, none of which had a lasting success so far. This
alone does not exclude the possibility that one day one of these theories will
prevail, but it should make one suspicious about the bold claims sometimes
raised by evolutionary theorists. It is simply not very credible that any
Darwinian evolutionary theory of culture will supersede or integrate all the
existing Non-Darwinian theories of cultural evolution, many of which will
surely remain much better suited to their specific purposes. Besides, a
pluralism of paradigms is typical for the social sciences, and it would be very
suprising if this changed just now, although some of the proponents of the
newer Darwinian evolutionary theories of culture entertain such hopes
\cite[]{tooby-cosmides:1992, mesoudi-laland-whiten:2006}.

When, in the following, we confine our focus to Darwinian evolutionary
theories of culture, this should therefore be understood as a topical decision
and not as presuming that other approaches would not have anything important
to say about the evolution of altruism as far as human society is concerned.
On the other hand we will not expect the evolutionary theory of culture to
afford an overall explanation of altruism if human behavior is concerned. If
an evolutionary theory of culture can highlight some aspects of altruistic
behavior among humans then this should be considered as sufficient to give it
a right to existence among the many rivaling theories in the social sciences
that could possibly be consulted for the explanation of human altruism.

\subsection{Genetic theories of human behavior}
\label{geneticEvolution}
One important class of evolutionary theories of human culture is formed by the
genetic theories of human behavior. Sociobiology and evolutionary psychology
are the most recent and prominent representatives of this class. The genetic
approach to human behavior is motivated by the fact that human nature has
been formed by evolution just as the nature of any other animal. At the same
time the patterns of human behavior are much more flexible and variegated
than those of any other animal species and it is hard to deny that this
variety must be due to the cultural environment in which a human is raised. But
just to what extent our behavior is the result of genetically transmitted
properties and to what extent it is a cultural acquisition is subject to
debate. In this ``nature-nurture'' debate sociobiology and evolutionary
psychology clearly take the nature stance. As evolutionary psychology can in
many respects be regarded as the successor of sociobiology \cite[p.\ 
21]{dupre:2001}, only evolutionary psychology will (briefly) be discussed in
the following. It will be discussed under the following four aspects: (1) Its
motivation and scientific intention, (2) its basic conception of human nature,
in this case specifically of the human mind, (3) its research strategy and
major achievements and (4) a critical discussion of the approach with respect
to the question of how well it can possibly explain the evolution of altruism
in humans.

\subsubsection{(1) Motivation and scientific intention} 

The {\em locus classicus}
of evolutionary psychology is a programmatic manifest by John Tooby and Leda
Cosmides on ``The Psychological Foundations of Culture''
\cite[]{tooby-cosmides:1992}. In this more than a hundred pages long manifest
Tooby and Cosmides broadly describe their idea of a new science of culture
with evolutionary psychology in its center. In their opinion the existing
social sciences have come to a dead end, mainly, because they rest on a set of false
assumptions which is called the ``standard social science model'' by Tooby and
Cosmides and at the core of which Tooby and Cosmides suspect the belief that
the human mind is essentially a tabula rasa that gets its shape only by
childhood education and by the impact of the society an individual grows up in
\cite[p.\ 24ff.]{tooby-cosmides:1992}. According to Tooby and Cosmides, this
model has effectively prevented the social sciences from making rapid progress
and also made it difficult to connect them to adjoining human sciences like
evolutionary biology or neuro-science, although this would certainly be
desirable. Tooby and Cosmides are confident that once the ``standard social
science model'' is given up and replaced by their own more appropriate
model, this impasse will be resolved. It will then become possible to
integrate the social sciences into a unified field of human sciences and rapid
scientific progress will ensue \cite[p.\ 19ff.]{tooby-cosmides:1992}. The
appeal to the unity of sciences and the emphasis that is laid on the
connectivity to other scientific fields is fairly typical for the justification
of scientistic approaches in the social sciences. (We will see it recur in the
programmatic scriptures of the non-genetic theories of cultural evolution.) But
Tooby and Cosmides do not only base their claim on such assumed secondary
advantages. They also believe that their own model is simply more adequate when
it comes to explaining human psychology.

\subsubsection{(2) Basic conceptions} 

The model that Toby and Cosmides propose as the alternative to the ``standard
social science model'' is named by them ``integrated causal model'' \cite[p.\ 
23/24]{tooby-cosmides:1992}. It is primarily a model about the human mind and
can best be described by the popular toolbox metaphor. Rather than assuming --
as the ``standard social science model'' tacitly does according to Tooby's and
Cosmides' estimate -- that the mind is a tabula rasa or a sort of computer
that can be programmed in arbitrary ways to solve any sort of problem, they
assume that the human mind is a toolbox containing an intricate set of diverse
capabilities each of which is highly specialized in order to fulfill a certain
task. These context specific capabilities, they reason, must have evolved
through natural selection to address specific challenges in the environment of
the ancestral humans.  Why nature could not have provided humans with a
general problem solving brain rather than a toolbox-brain is a point on which
Tooby and Cosmides remain a bit vague. They suggest that it would have been
somehow uneconomical for evolution to do so. In evolutionary psychology, the
evolved context specific capabilities are commonly called ``modules''. One of
the prime example for such a module of the adapted mind is that of language
acquisition \cite[p.\ 70]{tooby-cosmides:1992}.  If, as Tooby and Cosmides do,
one follows Chomsky and assumes that there is a deep structure to language
which underlies all world languages and which must for various reasons be
connected to some inborn capability of language acquisition and language
generation, then this inborn capability is indeed an excellent example for a
highly specialized evolved module of the human brain.

\subsubsection{(3) Research strategy and achievements} 

In connection with their so called ``integrated causal model'' Tooby and
Cosmides propose a very concrete research design by which to prove the
existence of a ``module'' of the mind.  This design consists of five steps
\cite[p.\ 73/74]{tooby-cosmides:1992}: a) Identification of an {\em adaptive
  target}, which is a certain challenge or problem in the life world of our
ancestors to which the assumed module would pose a solution.  b) Considering
the {\em background conditions}, i.e.\ recurring structures of the ancestral
world of hunter gatherer societies, under which the module has evolved. In
relation to the ancestral world, evolutionary psychologists usually refer to
the late Pleistocene. They speak of the human life conditions of this period
as of the ``environment of evolutionary adaptedness'' \cite[p.\ 
69]{tooby-cosmides:1992} because they assume that major genetic adaptations
cannot have occurred in the relatively short period of time after the
invention of agriculture. c) Drawing up of a {\em design}: a description of
the module itself under the assumption that it is designed to meet the
requirements of the adaptive target. The last two steps would then be d) a
{\em performance examination} and e) a {\em performance evaluation} of the
design. Only if a design performs well (under ancestral conditions) can the
researcher assume that he or she has identified an adaptation.

A great number of research projects in evolutionary psychology have made use
of this research design scheme. One of the allegedly most impressive
achievements in this respect is Leda Cosmides' research on psychological
mechanisms for detecting cheaters \cite[p.\ 168/169]{laland-brown:2004}. Based
on previous works of Peter Wason, she could show by a series of experiments
that people have a highly developed ability for detecting violators of social
rules, but easily fail to solve analogous tasks when these are presented in a
different setting. The conclusion that the human brain is equipped with a
special module for cheater detection rather than with the general capacity of
solving logical puzzles that could then be directed to the task of cheater
detection appears quite compelling in this case.

\subsubsection{(4) Critical objections} 

However, in other areas such as mate
choice, homicide and rape\footnote{Evolutionary psychology seems to have
  inherited from sociobiology a certain liking for ``sex and crime'' themes
  \cite[p.\  44ff.]{dupre:1993}.} the results evolutionary psychology has
produced have been much more debated. But it would lead too far to enter into
the details of these controversies here, which are well described in \cite[p.\ 
48ff.]{dupre:2001} and \cite[p.\ 170ff.]{laland-brown:2004}. What is of
interest here is how well genetic evolution can possibly account for altruism
among humans.  If the basic assumptions of Tooby and Cosmides (and the
evolutionary psychologists following their approach) are correct, then
altruism among humans would in some way or other have to be explained by an
evolved altruism-module of the mind or by a reciprocity-module or by a
morality-module which includes altruism. In order to examine the question
whether an (evolutionary) explanation of human altruism should primarily be
sought on a genetic basis, we will first ask how credible the evolutionary
psychologist's approach by Tooby and Cosmides is in general when it comes to
understanding human culture and then how the case is to be decided for the
evolution of altruism in particular.

As far as the general case is concerned, Tooby and Cosmides have raised the
bar for themselves quite high by the bold claim that any kind of human
behavior could be explained in terms of the ``integrated causal model''.
After all, they were aiming at a new unified approach to the science of human
nature. But at the same time it is often very difficult to discern just when
and to what degree a certain regular pattern of human behavior that we find
in culture is due to an evolved module of the brain and when it is not. For
example, we could think of a group of people that enjoys singing folk songs
and dancing folk dances. Moreover, we know that dancing and singing are common
patterns of human behavior found across all or at least most cultures. Now,
are we to assume that there exists a module for folk dances or folk songs? Or,
is there a module for singing and dancing? Or, is there maybe just a module
for music and rhythm? If we decide for one of the latter two alternatives then
this means that there remain interesting and important questions about singing
and dancing that evolutionary psychology cannot explain on the basis of mental
modules, namely the questions of how and why certain types of folk dances and
folk songs evolved within a certain culture. But if in turn we are to decide
in favor of the first alternative and assume that there exists a specialized
module for folk dancing and folk songs then we are confronted with the problem
that we would have to assume many more specialized modules of the mind in
similar cases. We are then somehow left with the question where there is to be
an end -- if there ever is any -- to postulating highly specialized modules of
the mind.\footnote{For a more elaborate criticism of the use of ``modules'' in
  evolutionary psychology see \cite[p.\  40ff]{dupre:2001}. -- As a historical
  side note it may be mentioned that a very similar discussion had many years
  earlier already arisen in another context in connection with the
  philosophical anthropology of the 20th century. Arnold Gehlen, when
  justifying his assumption that human nature is highly flexible and is shaped
  not by inborn instincts but by the institutions society provides had to
  argue against the then so common drive-theories in psychology, which in some
  respect resemble the ``modules'' of evolutionary psychology
  \cite[p.\ 50ff.]{gehlen:1942}. Doing so he pointed to the simple fact that
  different authors postulated quite diverse numbers and kinds of drives, some
  authors needed more than 50 drives, others were content with only two or
  three. Gehlen concluded that other than for the organically represented
  drives (hunger and sex) there was no sure foundation for assuming the
  existence of drives and that therefore it was for pragmatic reasons
  advisable to circumvent the question of drives altogether and find some
  other key to the explanation of human behavior. Making this historical
  comment is not to say that the whole question would not have needed to be
  discussed in the context of evolutionary psychology if only the participants
  had known the history of philosophy a little better. Since the evolutionary
  psychologists had proposed a new research design, the question of innate
  capabilities (drives or modules) that direct human behavior certainly
  deserved reexamination, even if the result that the usefulness of the
  toolbox-metaphor of the human brain remains confined to only a limited array
  of questions touching human behavior has in the end turned out to be
  same as the one which philosophical anthropology had already arrived at
  half a century earlier.}

But it is not only the problem that there seems to be such a liberty of
postulating modules that has caught the attention of critics of evolutionary
psychology \cite[p.\ 40ff.]{dupre:2001}. The research design proposed by Tooby
and Cosmides is also flawed in another aspect, namely, regarding the
identification of an {\em adaptive target} and the {\em background conditions}. If
we are to believe the critics of this research design then the usual talk
about the life conditions of our ancestors that appears in studies of
evolutionary psychologists is often quite arbitrary and results in reiterating
the same clich\'es and stereotypes about the life of hunter gatherer societies
over and over again \cite[p.\ 23ff.]{dupre:2001}. One of the common clich\'es
about these societies is that they are highly egalitarian
\cite[p.\ 372ff.]{boyd-richerson-henrich:2003}. This may be empirically true,
but at the same time we find in modern societies a psychology that is very well
adjusted to the social hierarchies that pervade modern societies on
almost all levels of professional and private life. Therefore, it is doubtful
whether this assumption about the egalitarian character of ancestral societies
is helpful when we want to understand the behavior of humans in today's
societies. It is a feature not only of evolutionary psychology but also of
other strata of Darwinian evolutionary theories of human culture that at some
point or other they seem to revert to evolutionary just-so-story telling. In
the case of evolutionary psychology this point seems to be reached when it
comes to the question of the life conditions of our ancestors and the supposed
consequences these have for how we handle modern life \cite[p.\ 
21ff.]{dupre:2001}.

If thus the ``imperialist'' claim of evolutionary psychology to provide a
unified alternative to the ``standard social science model'' proves to be
largely unfounded and leaves open the possibility to seek explanations for
human behavior within other paradigms including that of a non-genetic theory
of cultural evolution, the question still remains if the foundations of human
altruism in particular are not, if only to some degree, genetically determined.
Generally, any trait that is constant across all human cultures is a good
candidate for a genetically determined feature of human nature. In the case of
altruistic behavior there are several indicators which render the assumption
plausible: Moral behavior and understanding of basic moral categories is
constant across different cultures. Even if the norms differ, one would expect
to find some norms commanding altruistic behavior in any culture. Another
pattern that is more or less universal is the markedly distinct behavior
between in-group (family, tribe or other association) behavior and out-group
behavior. Here again we could expect to find a kind altruistic behavior in
in-group relations that probably also has a genetic basis. Finally, specific
behavioral categories that can be found in any culture like that of
reciprocity may be an indication for the existence of certain types of
genetically programmed altruism.

The abstract models of altruism that will be discussed in chapter
\ref{modeling} can in principle be applied to both genetically and
culturally evolved altruism. Since, as has just been argued, there is
enough reason to assume that altruism among humans may also have a
genetic foundation, the possibility of interpreting these models
within a theoretical framework of genetic evolutionary theory of human
behavior, i.e.\ evolutionary psychology, should not be dismissed
altogether. On the other hand, there is no doubt that the scope,
strength and specific form of most altruistic norms is shaped by
culture. In the following we will therefore examine the theory of
cultural evolution as an alternative (or supplementary) framework for
understanding human altruism. A third possibility should at least be
mentioned here: It is quite plausible to assume -- as some researchers
do \cite[p.\ 241ff.]{laland-brown:2004} -- that, for a certain period
in the history of the human race, a co-evolution of genetic and
cultural altruism in humans has taken place, where both forms of
altruism evolved alongside each other mutually strengthening each
other.

\subsection{Cultural evolution as a Darwinian process}
\label{memeticalEvolution}
Once we reject the assumption that human behavior and, consequently, also the
development of human culture is to the larger degree determined by the genes,
a wide field of diverse theories that seek to explain the course of human
history or the evolution of human cultures opens up.  One of these theories,
but -- as has been stated earlier -- by no means the only such theory, is the
theory of cultural evolution that treats the evolution of human culture as a
Darwinian process, where reproduction, variation and selection of cultural
traits form the {\em agens} of human history. This theory comes in different
flavors either as a theory or ``science of cultural evolution''
\cite[]{mesoudi-laland-whiten:2006} or as ``memetics''
\cite[]{blackmore:1999}, i.e.\ a theory of cultural traits called ``memes'', or
as a ``generalized theory of evolution'' \cite[]{schurz:2001}. But I will in the
following only speak of the theory of ``cultural evolution'' and occasionally
point out the differences between its variants. Just as in the case of the
genetic evolutionary explanations of human culture, I discuss (1) the
intention and motivation of the theory of cultural evolution, (2) its basic
assumptions, (3) its resarch strategies and achievements and (4) critical
objections. Doing so, I rely mainly on the accounts given in
\cite[]{mesoudi-laland-whiten:2006}, \cite[]{schurz:2001} and
\cite[]{laland-brown:2004}. I do not so much take into account the
literature about ``memetics'' because the concept of a ``meme'' does not yet
seem ripe for serious scientific application.\footnote{See my objections on
  page \pageref{drawbacksOfMemetics}. -- The concept of a ``meme'' was
  originally invented by the biologist Richard Dawkins \cite[p.\ 
  304-322.]{dawkins:1976}. Later, however, Dawkins seems to have grown a bit
  suspicious of his own concept, for he writes ``My own feeling is that its
  [the meme concept's, E.A.] value may lie not so much in helping us to
  understand human culture as in sharpening our perception of genetic natural
  selection.'' \cite[p.\ 112]{dawkins:1982}. Dawkins preface to Susan
  Blackmore's manifesto ``The Meme Machine'' sounds equally sceptical \cite[p.\ 
  7-21.]{blackmore:1999}. Even Laland and Brown have to admit that the idea of
  ``memes'' has mainly been popular among ``computer geeks'' but not among
  serious social scientists \cite[p.\ 200]{laland-brown:2004}.} At the same
time ``memetics'' is not at all indispensable for a theory of cultural
evolution. For in any specific case of cultural evolution we can specify the
entity the evolution of which is in question, say a social norm or a social
institution, and study its evolution without assuming that this entity is an
instance of or composed of some such things as ``memes''. The question whether
memes exist or not is a question with respect to which one can remain
completely neutral as long as only a specific instance of cultural development
is to be explained on an evolutionary basis.

\subsubsection{(1) Motivation and scientific intention} 

There exist several levels of motivation and justification for the
Darwinian theory of cultural evolution, discerned by the ambition of
the respective scientific program. On the lowest level, the Darwinian
theory of cultural evolution tries to transfer the successful models
and methods from evolutionary biology to the study of cultural
development \cite[]{arnold:2005}. It is assumed that there exist
sufficient similarities between cultural development processes and
evolution in nature to warrant such a transfer. On a more ambitious
level, the theory of cultural evolution is motivated by the zeal to
provide a unified coherent framework for the whole body of sciences
dealing with human culture, just like the theory of evolution provides
the overarching conceptual framework in biology. At the same time it
is assumed that applying the same methods that are successful in
biology should allow the social sciences to yield much better results
than has hitherto been achieved in the fragmented landscape of social
science theories \cite[p.\ 329-332]{mesoudi-laland-whiten:2006}. This
is very much the same ``imperialist'' story as it has been told by
Tooby and Cosmides in their programmatic scripture on evolutionary
psychology: The social sciences supposedly find themselves in a
hopeless mess.  In no way are they are able to rival the success of
the natural sciences. The reason for this annoying state of affairs is
that they lack a unifying theoretical framework and proper exact
methodologies. If only the social scientists were willing to learn
from the exact sciences and adopt their methods and succumb to a
unified theory then a great leap forward in the scientific advancement
of the social sciences would be positively inescapable. (Or so the
story goes...)

Yet another, though similar, motivation for the theory of cultural evolution
is to fully exploit the potential of the Darwinian model of evolutionary
processes and to give it as broad a scope as possible.  This is the motivation
behind the ``generalized theory of evolution'' \cite[]{schurz:2001}. The sort
of generalization that is meant here, does not lie on the ontological level in
the sense that one type of evolutionary process is meant to explain as many
phenomena as possible, i.e.\ biological evolution as well as cultural
evolution, as it is done in human sociobiology and in evolutionary psychology,
both of which explain cultural developments largely by the same process of
genetic evolution that does also account for the evolution of species.
Rather, the generality is to be found on the level of theoretical abstraction.
It is assumed that the same core principles of Darwinianism can explain
evolutionary processes in different branches of science with different
evolving entities. In biology they describe the evolution of genes. In social
sciences they describe the evolution of diverse cultural traits or of
``memes'' (if we assume that such distinct entities as ``memes'' underlying
all cultural traits do exist).  Strictly speaking the generalized theory of
evolution is not a single theory but encompasses a family of evolutionary
theories. All of these have in common that they share the same three core
principles of Darwinian evolution described above. But each member of
the family is distinguished by additional specific principles or axioms which
further describe the evolutionary process in its realm. Thus a theory of
genetic evolution as one particular member of the family of
evolutionary theories can be further narrowed down by adding the laws which
describe genetic transmission and mutation. And a theory of cultural
evolution, another member of the family, could contain principles that
describe the transmission and change of cultural traits. In the following,
however, we will not be concerned with the generalized theory of evolution in
its broadest sense, but only with that part which concerns the evolution of
culture.

\subsubsection{(2) Basic assumptions} 

The central assumption of the evolutionary
theory of culture is that some cultural developments -- or, if we follow the
more ``imperialistic'' programs, {\em all} cultural developments -- can be
explained as cases of Darwinian evolution by reproduction, variation and
selection of certain cultural entities. What has to be clarified, when one
wants to construct an evolutionary theory of culture in analogy to the theory
of evolution in biology, is what the evolving entities are, and how
reproduction, variation and selection of these entities takes place.

\paragraph{The entities of cultural evolution} 

A cultural entity that evolves can be about anything: It can be
technology, it can be social norms and customs, it can be laws, it can
be institutions, it could possibly also be economic or political
systems and maybe, though this seems somewhat doubtful, it could be
even arts. Generally speaking, an evolving entity in cultural
evolution can be any discernible and identifiable cultural trait.  As
has been mentioned earlier, some authors apply the ubiquitous term
``meme'' for any of these entities. But there exist several drawbacks
to the ``meme''-terminology: \label{drawbacksOfMemetics} 1) The above
listed entities are of a very different kind and it must be expected
that the conditions of reproduction, variation and selection also
differ in each single case. But then it will not be of much use to try
to generalize over all of these different instances of cultural
evolution by inventing a ``meme'' theory. 2) Some authors hope that
the just mentioned limitation can be overcome by a neuronal definition
of the meme. A neuronal definition of the meme that defines the meme
in terms of its neuronal representation in the human brain would have
to be more atomic than the evolving cultural entities mentioned
before. However, the research in this direction is not very far
advanced to say the least. At present stage a neuronal definition of
the meme is science fiction.\footnote{While the search for the
  neuronal basis of cultural traits may be an interesting research
  program of its own, theories about the neuronal representation of
  cultural traits will begin to be useful for the explanation of
  cultural developments (which, after all, is the primary purpose of
  the theory of cultural evolution) only by the time when the neuronal
  representation ``can be clearly observed or measured'' and at the
  same time unambiguously be linked to the cultural phenomena the
  explanation of which is in question. At present stage neuro-science
  seems to be far from fulfilling this requirement.} 3) In fact, there
does not only exist no (generally accepted) neuronal definition of the
meme, but there exists no precise definition of the ``meme'' at
all. Usually what is offered is just examples and illustrations of a
rather trivial kind. Instead of providing a workable definition of the
meme, ``memetics'' is thus indulged in all kinds of dogmatic
discussions concerning the nature and essence of a meme, like the
question whether an evolving technology is itself to be considered a
meme or the blueprints for this technology or just the mental
representation of the blueprints in a person's mind.\footnote{See
  \cite[p.\ 164ff.]{salwiczek:2001} for a number of such discussion
  points, most of which are peculiar to the meme concept.} But these
dogmatic disputes are of little significance if one is primarily
interested in explaining the evolution of some kind of technology or
other cultural achievements.\footnote{A very extreme example for this
  purely dogmatic (if not almost ideological) and thus very
  uninspiring mode of discussion about evolutionary theory in the
  social sciences is delivered by Alex Rosenberg
  \cite[]{rosenberg:2005}.}

Fortunately, it is possible to circumvent the difficulties that surround the
meme concept. For, when we want to explain the evolution of certain cultural
traits like altruistic norms, we can simply study how these traits evolve in
terms of reproduction, variation and selection without even bothering whether
they are instances of some such thing as a meme or not.\footnote{One might
  object that any other concept in place of the ``meme'' would have to suffer
  the same drawbacks as the meme-concept. But this is not the case.
  Cavalli-Sforza and Feldman, for example, define the term ``cultural trait''
  (which is roughly their equivalent for ``meme'') as ``the result of any
  cultural action that can be clearly observed or measured on a discontinuous
  or continuous scale'' \cite[p.\ 73]{cavalli-sforza-feldman:1981}. The term
  ``cultural trait'' and the pertinent definition has several advantages
  over the ``meme''-terminology: 1) Right from the beginning it is clear that
  cultural traits can be many different things and that it does not denote a
  single type of entities as the ``meme''-terminology notoriously suggests.
  The largely meaningless question ``What is a meme?'' is thus evaded. 2) The
  requirement that the trait should be observable and measurable defies
  premature attempts of a neuronal definition of cultural traits (the other
  variant of the misleading attempt to find a fundamental, potentially hidden
  entity behind cultural evolutionary processes). 3) Finally, it is just a
  fact that memetics has spurred a good deal of poor quality literature on
  cultural evolution. It seems that despite Aristotle's saying that one should
  not argue about words, terminology does matter.}

\paragraph{Reproduction, variation and selection in cultural evolution}

\label{reproductionInCulturalEvolution} In cultural evolution the three
``Darwinian modules'' reproduction, variation and selection take a form that
is quite different from their counterparts in biology. More importantly, they
can differ depending on the evolving entity that is under consideration. Here,
only a very general overview can be given. {\em Reproduction} can take place
either through teaching and learning or through imitation or through both. It
seems obvious that some things can only be reproduced by teaching as, for the
example, the knowledge how to read and write, while other things can easily be
imitated. In contrast to the reproduction of genes in multicellular organisms
in biology, knowledge and cultural techniques can be transmitted horizontally
and not only vertically to descendants. Depending on the mode of reproduction
and the entity reproduced a different rate of reproduction or imitation errors
is to be expected. Such imitation errors already resemble one mode of {\em
  variation} in cultural evolution. Imitation errors occur unintentional, but
variation in cultural evolution frequently occurs as the result of intended
change. For example, many technologies we use are constantly being improved.
The evolution of technology is therefore one where variation occurs to a high
degree in the form of intentional change.

{\em Selection} is somewhat more difficult to specify for cultural evolution
because when dealing with cultural evolution, there are two different kinds of
selection that we can think of. First of all, there is the kind of selection
that occurs when people choose to keep or adopt a cultural technique (in the
broadest sense, encompassing technology as well as customs, norms, policies
etc.) or not to do so. This kind of selection is the selection of cultural
traits through the bearers of culture, that is, through
humans.\footnote{``Memeticists'' also like to speak in this connection of
  ``memes'' competing for brainspace. If a cultural technique has been adopted
  then -- described in ``meme-speak'' -- its ``memes'' have successfully
  competed for human brainspace.} A theoretical difficulty with this kind of
selection is that there can be all kinds of reasons why people adopt cultural
techniques and why not. They may do so because they believe that this will
make their life better or allow them to compete more successfully with other
people for power, wealth or prestige. But the reasons may also be of a wholly
idiosyncratic nature.  Therefore, the conditions of this kind of selection can
be difficult to specify. The other kind of selection occurs when people that
have adopted a certain cultural technology turn out to be very successful (or
vice versa very unsuccessful) and simply drive out other people who have not
adopted the respective techniques. This may be the case for intrasocietal
competition as well as for intersocietal competition. Both kinds of selection
are interrelated because, usually, if a certain cultural technique promotes
success (in inter- or intrasocietal competition) then people will want to
adopt it.

The basic assumptions of the theory of cultural evolution can thus be summed
up: Cultural evolution in a Darwinian sense occurs when some cultural entity
evolves through reproduction, variation and selection. The evolving entity can
be of arbitrary kind with the only restriction that it must be a discernible
and identifiable trait of culture. Similarly, the three ``Darwinian modules''
can take a somewhat different shape from case to case. Already at this point
it may be remarked that lacking an equivalent for the sure foundation that the
biological theory of evolution has in genetics, the theory of cultural
evolution turns out to be much more vague and less concise. But this does not
mean that it does not have its assets. Let us see what it can do:

\subsubsection{(3) Research strategy and achievements}
\label{culturalEvolutionAssets}
Given that it is reasonable to assume that at least some cultural development
processes follow a Darwinian pattern, it appears only natural to try to apply
some of the more specific methods used in evolutionary biology to problems of
the social sciences where appropriate. This is what Mesoudi, Laland and Whiten
suggest \cite[]{mesoudi-laland-whiten:2006} and they list a number of fields
where this is already being done or can be done with some hope of success. In
this respect they refer for example to linguistic studies of the development
of languages that apply cladistic methods similar to those used in biology
\cite[p.\ 333]{mesoudi-laland-whiten:2006}. The development of language seems
particularly well suited for an evolutionary explanation because it is a
gradual process and at the same time one that is largely unaffected by human
intentions, which could distort the selection processes that assumedly promote
the development of language. Another, even more striking instance of the
transfer of methods from evolutionary biology to the social sciences is the
use of evolutionary models borrowed from paleobiology by archaeologists in
order to trace the lineages in the development of human artifacts such as
coins or projectiles \cite[p.\ 334]{mesoudi-laland-whiten:2006}. Yet another
field where similarities between biological and cultural evolution can be
exploited is that of human behavioral ecology, which studies in how far
patterns of human behavior or, likewise, other cultural traits result from an
adaptation of human culture to the environment \cite[p.\ 
335]{mesoudi-laland-whiten:2006}.

While the examples just given all more or less concern the evolution
of cultures as wholes and are accordingly subsumed by Mesoudi, Laland
and Whiten under the heading of macroevolution, Mesoudi, Laland and
Whiten also find ample evidence for microevolution, that is, the
evolution and selection of cultural traits within cultures or single
societies. On a par with theoretical population genetics as a
subdiscipline in biology they enumerate a number of mathematical and
computer models of cultural evolution and gene-culture co-evolution as
they have been put forward by Boyd and Richerson
\cite[]{boyd-richerson:1985} and Cavalli-Sforza and Feldman
\cite[]{cavalli-sforza-feldman:1981} among others.  So far this
research has remained mostly theoretical and it therefore remains to
be seen whether the reservations against it will eventually cease as
they did in the corresponding case of the mathematical models of
population genetics in biology
\cite[p.\ 338]{mesoudi-laland-whiten:2006}.  More empirically
orientated research has been done on cultural transmission.  Mesoudi,
Laland and Whiten list several examples of experimental research.
Many of these start with a group of people that has to solve certain
tasks.  Then, the members of the group are replaced one by one in
order find out if and how traditions of solution strategies to the
tasks evolve and are transmitted. There are indeed some similarities
to research on the heritability of traits in biology.  The respective
field research on cultural transmission is criticized by Mesoudi,
Laland and Whiten for its not identifying the putative selection
pressure \cite[p.\  340]{mesoudi-laland-whiten:2006}. But this
criticism is somewhat question begging because the criticized
deficiency could equally well be interpreted as reflecting the fact
that the selectionist paradigm is simply not adequate for this type of
research.  Finally Mesoudi, Laland and Whiten mention some research on
``memetics'' as the cultural equivalent to genes in biology, but they
admit that the concept of the ``meme'' is still very debated in many
respects \cite[p.\ 342-344]{mesoudi-laland-whiten:2006}.

Summing it up, while it seems reasonable or at least worthwhile trying some of
the methods of evolutionary biology in the social sciences, there obviously
exist only few instances where this has already been done with success.
But this also means that some instances do indeed exist. And further instances
of the successful transfer of methods between evolutionary biology and the
social sciences could certainly be added to what Mesoudi, Laland and Whiten
mention.  From its type of modeling and its kind of thinking, economics seems
to be the branch of the social sciences which is the most akin to biology.  As
evolutionary game theory testifies, the transfer works in both
directions\footnote{See \cite[]{arnold:2005} for further examples of the
  transfer of models between biology and economics.} (which raises some further
doubts about the ``imperialist'' claim of Mesoudi, Laland and Whiten).  But it
is only on certain occasions that the transfer works and many fields of the
social sciences remain untouched by it.  Therefore, one should conclude that
rather than becoming the new overarching paradigm of the social sciences, the
theory of cultural evolution marks a border region between the social sciences and
biology where methods developed in biology can fruitfully be transferred to the
social sciences (and vice versa).


\subsubsection{(4) Critical objections}

\label{culturalEvolutionCriticism}
Surely, the weakest point of the theory of cultural evolution are the
``imperialist'' aspirations of some of its proponents. The naive expectation
that everything in social science must fit into an evolutionary framework
because this works so well in biology can of course hardly be taken
serious.\footnote{Similarly, the expectation that the propagation of knowledge
  must be explainable with reference to self replicating entities called
  ``memes'', just because the metaphor of the ``egoistic gene'' worked so well
  in genetics must be regarded as quite naive.} But then, the scientific value
of the evolutionary theory of culture does not depend on the fulfillment of
its far fetched ``imperialistic'' claims. If there exist but a few cases where
it proves to be useful then this would suffice to justify the approach. But
even as far as this goes the theory of cultural evolution has called forth
severe criticism. The following critical discussion of the theory of cultural
evolution is organized under the headings of three different questions each of
which highlights a particular problem of the theory of cultural evolution.
What will be discussed is 1) if the theory of cultural evolution is able to
explain any phenomena that could not be explained otherwise at all and what
these are (``What is the riddle?''), 2) where the theory of cultural evolution
does not explain anything new, if it does at least offer better explanation
(``Where are the advantages?'') and 3) if there exist any monographic studies
about cultural phenomena where the theory of cultural evolution has
successfully been employed (``Where are the showcases?''). The discussion
strongly relies on the criticism b Joseph Bryant \cite[]{bryant:2004}, which
is a very acute criticism of the theory of cultural evolution. While there are
certainly many scientists, especially in the humanities, who reject the
attempt to apply Darwinian thinking to the evolution of culture, only few --
like Joseph Bryant -- have taken the pains to deliver a detailed criticism.
Since in my oppinion such criticism is strongly needed and since on the other
hand there are enough books and papers heavily advertising the Darwinian
theory of cultural evolution \cite[]{mesoudi-laland-whiten:2006,
  laland-brown:2004, dennett:2006, dennett:1996}, the following critical
discussion is deliberately kept much more extensive than the previous
description of this theory.


\paragraph{1. What is the riddle?} 

A good scientific theory is one that gives us true answers to questions
arising from the empirical world. That is, it tells us something we wanted to
know or, to put it yet another way, it solves a riddle, right? Now, as has
been mentioned earlier, the theory of evolution in biology is certainly a
great theory because it solves some of the biggest riddles of our living
world, among others, the riddle why all living beings are so extremely well
designed to live in their respective habitats. But what is the riddle that the
theory of cultural evolution could possibly solve?  Design features of our
cultural world provide certainly much less of a riddle than those in the
natural world. Why is it, for example, that doors have handles?  It does not
take an evolutionary theory to answer this question because there exists a
much simpler and straightforward answer: Doors have handles because it is
very convenient to have a handle on a door and, therefore, people attach
handles to doors. Many functional adaptations in our cultural world can be
explained by the fact that they were intentionally designed to be that way.
So, when the evolution or development of culture is concerned there exists
right from the beginning much less of a riddle that a Darwinian theory of
evolution would be needed for to provide a solution.

There are two answers that can be given to this objection from the standpoint
of the evolutionary theory of culture. First of all, even though many features
of culture are intentionally designed and therefore do not require an
explanation by a selection process, many or even most of the long term
developments of culture can hardly be the result of conscious planning by
humans. But also these long term developments do often expose characteristics
of functional adaptation either of the whole culture or society to its living
environment or of the mutual sectors of the culture (religion, economics, law
etc.) to each other.  From the point of view of the evolutionary theory of
culture those features of cultural development that result from intentional
design merely represent single steps in the long term evolutionary process.
The fact that these single steps are in many cases consciously designed (in
contrast to the accidental mutations in genetic evolution) just means that the
evolutionary process will take place much faster \cite[p.\ 345]{schurz:2001}
\cite[p.\ 66]{cavalli-sforza-feldman:1981}.  Still, this means that the
evolutionary scheme will only be applicable in certain cases, while it will
not be of much use in many other cases: It will not help us along if the
cultural phenomena we want to explain represent just single steps in the
evolutionary scheme. But it may be a good candidate for an explanation if the
cultural phenomena we want to explain are unplanned adaptive long term
developments of culture. But even then it may be a mistake to assume that
``directed mutations'' (i.e.\ changes that are intentionally brought about by
humans with the aim of ameliorating some cultural technology) merely speed up
the processes of evolution.  The existence of directed mutations can lead to a
totally different adaptation process because adaptation can then occur as a
sequence of directed mutations on top of each other without any selection
being involved.\footnote{In this connection proponents of the evolutionary
  theory of culture sometimes casually refer to evolutionary ``trial and error
  processes'' just as if any trial and error process must by necessity be an
  evolutionary one (in a selectionist sense). But this is of course not true.
  (If a counter example should seriously be needed for proving this point:
  Backtracking algorithms rely on trial and error but are not evolutionary
  nonetheless.)} Also, there always remains the question whether it is really
possible to spell out such evolutionary explanations when it comes to
explaining any such long term developments of culture. This may be quite
difficult to accomplish, as we will see when we look at some monographic
studies which attempt to do so, later.

The other reply that could be given to the objection that there is not really
a riddle for the theory of cultural evolution that needs to be solved is that
even if this were true -- which is only partly the case, as has just been
argued -- the theory of cultural evolution may still prove to be valuable in
that it solves some of the problems of cultural developments that are already
addressed by other theories better than these theories do. It is three
advantages in particular that proponents of the theory of cultural evolution
claim for their approach apart from potentially solving riddles about
cultural developments yet unsolved: 1) {\em Generality}. The theory of
cultural evolution allows for greater generalization. This would especially be
the case if it is understood in the sense of the aforementioned generalized
theory of evolution. 2) {\em Unity}. Closely linked with the claim of greater
generality is the claim that it provides a unified scientific approach to the
problems of the social sciences with all the supposed benefits that come with
the unity of sciences.  3) {\em Greater scientific rigor.} Finally, it is often
claimed that the evolutionary approach is more scientific than many other
approaches in the social sciences. These three supposed advantages will be
discussed in detail below.

So far, we can summarize that there exist at least some questions (``riddles'')
about cultural developments, namely about unplanned long term adjustments or
adaptations within cultures, which it would be difficult to account for on the
basis of historical of sociological theories which are centered around
intentional action or intentional ``responses'' to ``challenges'' or the like
and for which an evolutionary theory {\em might} provide answers. The theory
of cultural evolution would then be worth while because it allows us to solve
new scientific riddles that have not been solved or not even been taken notice
of before. Other than that it is claimed that the theory of cultural evolution
can give better answers to existing riddles. Whether this latter claim is
warranted will be examined now.

\paragraph{2. Where are the advantages?}

A new scientific approach or theory can be justified either because it opens
up new fields of knowledge to us, formerly unknown or not paid attention to by
science or because it gives us better insights into existing subject matters.
As has been argued before, the theory of cultural evolution may be able to
solve some riddles that have never properly been considered before and in this
sense may allow us to gain new insights. But most of the time when proponents
of the theory of cultural evolution give examples for instances of cultural
evolution they refer to subject matter that is well known and covered by
existing scientific theories already.\footnote{This is particularly true for
one topic that seems to be a favorite among evolutionary theorists of culture,
namely, the history of religion. Examples of this brand are Wilson
\cite[]{wilson:2002} and Dennett \cite[]{dennett:2006}.} Where, then, are the
great advantages of giving an evolutionary account instead of staying with
conventional explanations? Supposedly, these advantages lie a) in the higher
level of generality of the evolutionary approach, b) in the unification or
linking of different fields of social sciences and even natural sciences
(biology) and c) in its being more scientific than other approaches. Let's
examine these claims one by one.

\subparagraph{a) Generalization.}

Generalization could be a possible benefit of an evolutionary
approach.  However, regarding generalizations we have to distinguish
between real scientific generalizations, where the relatively more
specialized laws and concepts can be deduced from the more general
laws and concepts, and a purely verbal generalization, where just the
same kind of jargon is applied in many different cases. An example for
the former, scientific type of generalization would be Newton's
mechanics in relation to the laws of Kepler and other more specific
laws like the law that states that the acceleration near the earth's
surface is $9,81 m/s^2$. A good example for the latter kind is Hegel's
dialectics in philosophy because Hegel believed that everything in
this world follows the principles of dialectics and in his
``Encyclopedia'' he cast as much as he knew about any science or
subject matter from physics to political philosophy and history into a
dialectical jargon. But it is extremely doubtful whether in this way
he achieved any informative kind of generalization beyond mere
jargonization.\footnote{The point I am making here against Hegel can
  most easily be explained using an example which Friedrich Engels, a
  most faithful adherent of Hegel's dialectics, has given
  \cite[p.\ 248]{engels:1878}. The example which was supposed to
  demonstrate that everything in this world follows the dialectical
  scheme of thesis, antithesis and synthesis runs as follows: Take a
  grain of barley and plant it in the earth. The grain of barley is
  the thesis so to say. In the earth the germ buds and a new barley
  plant develops, which is the negation of the corn (antithesis). But
  then, as according to dialectics the negation is to be followed by a
  negation of the negation (or a synthesis) the barley plant itself
  produces many new barley grains. Unfortunately, far from
  demonstrating how general dialectics is, the example merely shows
  how useless it is. For, in order to know that what grows from a
  barley grain is a barley plant and not, say, potatoes we need to
  know -- in addition to the dialectical principle -- the laws of
  botanics.  If, on the other hand, we already know the laws of
  botanics, we do not need to know dialectics any more to tell us what
  becomes of the barley grain.  Because the same problem occurs in as
  good as every other application case of dialectics, dialectics is
  quite useless if considered as a scientific method or theory. A
  similar case can, as I believe, easily be made against most examples
  of memetics.} If Joseph Bryant's criticism of the theory of cultural
evolution is right then the sort of generalization this theory
provides lies mainly on the verbal level of applying a vague and
ubiquitous jargon to all kinds of already well known phenomena. Bryant
speaks in this respect of a mere ``schematic 'repackaging' ''
\cite[p.\ 471]{bryant:2004} and demonstrates in two examples of
monographic evolutionary studies of cultural phenomena (Runciman's
evolutionary interpretation of the replacement of ``hoplite warrior
culture'' of Greek city states at the time of Alexander by larger
kingdoms that relied on standing armies and mercenaries
\cite[]{runciman:1990} and the interpretation of the development of
religion in evolutionary terms as it has been attempted among others
by Wilson \cite[]{wilson:2002}) that the evolutionary approach merely
results in a much less concise if not distortive presentation of well
known materials and at the same time falls short of the scientific
level of conventional, non evolutionary accounts.  Bryant's examples
will be discussed in detail below.\footnote{See page
  \pageref{Bryant_examples}.} In order to estimate in how far Bryant's
criticism is not only confined to the two examples he discusses but
resembles, as he believes, notorious weaknesses of the attempt to
apply Darwinian figures of explanation in social sciences, it would of
course be necessary to examine further examples of attempted
evolutionary explanations of social or cultural phenomena, which goes
beyond the scope of this book.  Still, a few further considerations
suggest that it may indeed prove difficult for the theory of cultural
evolution to achieve a kind of generalization that is truly
informative. As has been argued before, there exist different kinds of
cultural entities and the way these entities reproduce, change and are
selected is different according to the type of entity in
question. (This is different in biological evolution, where
reproduction and mutation occur in more or less the same way for most
organisms.) But then the hint that all cultural entities, the
evolution of which can be explained by a Darwinian theory of cultural
evolution, evolve through the three Darwinian modules reproduction,
variation and selection is just not very informative. In order to
explain how and why a specific cultural entity, say, computer
technology evolves, what we mainly need to know is the specific laws
and circumstances that govern the evolution of this particular
entity. But then we do not have much of a generalized theory any
more. We will see that this particular limitation of the evolutionary
approach will reappear when we examine the empirical applicability of
simulation models from evolutionary game theory to the social
sciences. Often when applying game theoretical models, the greatest
part of the explanatory work is done not by the very abstract and
general model but by the specific assumptions that enter into the
determination of the input parameters with which the model is
fed.\footnote{See chapter \ref{realWorldEvidence} for an example where
  this problem occurs.} To be fair, it has to be said that it is
always very difficult in the social sciences to find laws which are
very general and at the same time highly informative. It seems that
for the subject matter dealt with in the social sciences there usually
is a payoff between generality and being informative and that the most
useful laws and connections are to be found on an intermediate level
of generality. If this is true, however, then it also shows how naive
and misplaced some of the aspirations of the theory of cultural
evolution are.

Another consideration that suggests that it is advisable to be
suspicious of the generalizing claims of the theory of cultural
evolution is the fact that it is all too easy to cast any conventional
explanation of some cultural development into an evolutionary
jargon. Just take an arbitrary explanation of any cultural
development, relabel what would commonly be understood as ``causes''
into ``conditions of selection'' or some similar phrase from the
evolutionary dictionary and, voilà, ready is your evolutionary
explanation!  It is in particular the branch of ``memetics'' that
lends itself to this kind rephrasing. For example, assume social
scientists had found out that more and more reports about terrorism in
the news cause many people to live in fear of terrorism then here is
the evolutionary explanation: Fear of terrorism spreads because the
increasing number of reports about terrorism in the news exerts a
positive selection pressure on the ``be afraid of terrorists''-meme.
Of course, nothing substantial is gained by this kind of rephrasing
and certainly not a generalization of any valuable kind. The
memetics-literature is full of examples of this very trivial kind of
reframing common knowledge in an awkward evolutionary
jargon.\footnote{The following excerpt from a manifesto on memetics
  may serve to illustrate this charge. It is typical for the way in
  which examples are constructed in memetics-literature: ``A memetical
  example is the beginning of Christianity. Adherents of the new creed
  were prosecuted in order to conserve the previously dominant
  paganism and to destroy the upcoming religion. Kindreds of the faith
  in Jesus the Nazarean joined together, went into the underground and
  thus survived in the spiritual community. As the religious
  convictions, which one can also describe as memes, were adopted by
  those in power that is were copied, the memecomplex Christianity
  could prevail over the previous memepool (Constantinian turn!).''
  \cite[p.\ 129, my translation]{salwiczek:2001} This memetical account
  of a well known historical fact is distinguished from conventional
  descriptions by nothing but an awkward jargon. Technically speaking,
  the memetical account merely adds a few {\em irrelevant premises},
  namely the laws of memetics. The qualification of religious groups
  as ``memecomplexes'' does not at all help us to explain why one of
  them (Christianity) won over the other (Paganism). For, since both
  are memecomplexes just the like, the explanation must lie within
  specific properties of each of them and cannot be due to their being
  memecomplexes.  -- More examples of this kind of trivial rephrasing
  of common knowledge in an evolutionary jargon can easily be found in
  Susan Blackmore's book ``The meme machine''
  \cite[]{blackmore:1999}. Also quite notorious is the example of the
  chain letter, which is reiterated again and again in the memetical
  literature despite its utter
  triviality\cite[]{laland-brown:2004}. One might object that examples
  such as the one just quoted were merely meant to explain what the
  meme concept means and not to prove its value. But then, the
  literature on memetics often does not advance beyond such trivial
  demonstrations.} Of course trying to understand social developments
in evolutionary terms is not necessarily bound to end up in trivial
reframing of common knowledge. If done seriously it will be quite a
demanding task, where one has to consider carefully whether and how
the causes of the development in question can be interpreted either as
conditions of selection or variation or reproduction. And one cannot
be sure beforehand that the endeavor will be crowned by success. But
there also exists the cheap way of doing it that bears a similar risk
of intellectual self betrayel as Hegelian dialectics once did, which
by ardent Hegelians was believed to capture a general pattern
underlying any natural or historical process in this world whatsoever
\cite[]{hegel:1830, engels:1878}.

\subparagraph{b) Unification.}

Regarding the question whether the evolutionary theory of culture provides
the right framework for a {\em unified science} of cultural evolution, it can
only be repeated that at the present stage the Darwinian theory of cultural
evolution is far cry from offering anything that could fulfill this claim. It
may offer a few good models for a few special cases of cultural development,
but it simply covers too little of the vast and varied field of the evolution
(in a non Darwinian sense) of culture. At the same time it is by no means
clear that the Darwinian scheme is in all cases better or at least as good as
competing explanations for cultural development. Unified Science is a kind of
pertinacious myth. Some scientists and philosophers seem to believe that
whenever a science is being unified then this should give it a boost of
scientific discoveries \cite[p.\ 19ff.]{tooby-cosmides:1992} \cite[p.\ 
329ff.]{mesoudi-laland-whiten:2006}. But in fact it is the other way round:
Unification of sciences is the consequence rather than the prerequirement of
dramatic scientific advances. It arises more or less by itself whenever
different neighboring scientific fields have evolved far enough to merge. But
it is not much use trying to impose a unification. Some philosophers of
science even dispute that the sciences can be or should be unified at all
\cite[]{dupre:1993, cartwright:1999}. Notwithstanding the question whether the
skeptics of unified science are right or wrong, it seems fairly obvious that
unity or, what amounts to the same, connectivity to other, specifically the
natural sciences is a second rank criteria like generality, parsimony,
simplicity and others. First of all it matters whether a scientific theory can
explain anything and whether the explanation is true or false. And only if it
is true, we can start worrying about whether it is parsimonious, general or how
well it can be connected to other theories.

Besides, when talking about a unified science of culture, there is the
question of why it must necessarily be a unified evolutionary theory of
culture. One could equally well demand that it be a unified economic theory
of culture based on the utility calculus and the laws of the market.
Economics is, after all, no less rigorous than population genetics and it
certainly is much closer to the other social sciences than biology. In the end
it seems that the call for a unified science is just imperialist science badly
disguised.

\subparagraph{c) Scientific rigor.}

What about the evolutionary approach being {\em more scientific} than other
approaches in the social sciences? When Mesoudi, Laland and Whiten complain
that cultural or social anthropology has made so little progress in comparison
with evolutionary biology (and similarly when Tooby and Cosmides complain that
the social sciences in general have made very little progress lately) then
what they imply is that this is due to a lack of proper methodology and
scientific rigor. And they profess to offer a proper methodology by
transferring methods and research designs from evolutionary biology to
cultural anthropology. How successful this endeavor will be can ultimately
be judged only by looking at the very results and concrete examples of this
undertaking. But a few general considerations seem appropriate nonetheless: For
example, it cannot generally be assumed that transferring certain concepts,
methods or paradigms from one field of science where they have been employed
with greatest success to another field will retain their success or even just
their rigor.\footnote{A comparison of genetics in biology and the parallel
  approach of ``memetics'' in cultural evolution is quite instructive in this
  respect. The usual bad excuse that ``memetics'' is only at its very
  beginnings won't do here.  Apart from the fact that the ``meme'' concept is
  more than a quarter of a century old -- which is ages on the time scale of
  modern science -- and we still wait for great achievements, genetics in
  contrast was able to offer substantial new insights right from the beginning
  as is testified by the Mendelian laws. Daniel Dennet has a very simple
  explanation for the fact that memetics, despite its age and popularity, has
  hardly had any impact on cultural history. He believes that this omission is
  due to the wanton ignorance of `` 'humanist' minds'' \cite[p.\ 
  361]{dennett:1996}. But surely, if the only explanation for the lack of
  secular success of memetics that an ardent proponent of this concept can
  find is a kind of conspiracy theory then it is much more plausible to
  conclude that the meme concept found no followers among the experts in the
  cultural sciences because memetics is simply a bad theory.}  Moreover,
scientific rigor is not something that could be called in or that depends
merely on the willingness of the scientists to apply rigorous methods. It also
depends on the subject matter at hand. This is especially true for the
application of formal or mathematical methods. Many branches of science simply
do not lend themselves to mathematization. It would be laughable to complain
that, say, classical philology has not made quite the same progress during the
last two hundred years as astronomy or particle physics and to attribute this
supposed defect to the lack of rigorous scientific methodology. On the other
hand, it is of course always worthwhile trying. One just should not be overly
optimistic about the theory of cultural evolution allowing for a more rigorous
treatment of cultural developments than conventional theories of cultural
developments.  For the time being this seems to be a largely unfulfilled
claim, but the future may still prove the opposite.

Summing it up, in those areas where the theory of cultural evolution does not
solve any new riddles it does not have much to offer that would convince a
scientist to give preference to this particular theory over other more
conventional approaches. Its supposed generality seems to be accomplished
mainly at the expense of a loss of substantial content. To effect a
unification of cultural studies under its hood, it is just too sparsely
applicable, and whether the acclaimed scientific rigour of its models really
proves tenable remains to be seen. Of course, all of these considerations have
remained somewhat abstract and tentative. In the end the decisive step to
justify a scientific approach or paradigm is to employ it in scientific
practice. If anybody were able to draw up a really convincing explanation of
some cultural phenomenon based on the theory of cultural evolution then this
would certainly do more to the justification of this paradigm (and to the
abashment of its critics) than any abstract considerations. Therefore, let us
now have a look at some case studies.


\paragraph{3. Where are the showcases?}

The best proof of the fertility of a general scientific theory is when it
spawns many monographic studies where its laws and concepts find a useful and
appropriate application to specific subject matters. How does the theory of
cultural evolution fare in this respect? There certainly exist quite a few
monographic studies dedicated to this paradigm. But are they good enough to
convince us of the merits of this approach? For his criticism of the theory of
cultural evolution, Joseph Bryant has examined two such studies
\cite[]{bryant:2004}. Because the errors in reasoning he discovers seem to be
quite typical for the evolutionary approach, it is worthwhile to take a closer
look at his reasoning.

Bryant \label{Bryant_examples} sets out with some general
considerations about the question whether there really are any strong
analogies between biological evolution and evolutionary processes in culture
\cite[p.\ 459-469]{bryant:2004}. His criticism suffers a bit from the
fact that he assumes that the transfer of evolutionary constructs from
biology to the social sciences would require that we find some kind of
analogon to the laws of genetics in the cultural sphere, which is
obviously very difficult to find\cite[p.\ 461ff.]{bryant:2004}.  But in
fact the theory of cultural evolution does not at all rely on such
an analogy. Even those variants of the theory of cultural evolution
that employ the concept of the ``meme'' as a parallel concept to that
of the ``gene'' in biology do not assume that the proliferation of
``memes'' is guided by the same laws as govern the propagation of
``genes'' in biology.  Still, Bryant hits upon an important point
insofar as without any analogon to genetics the theory of cultural
evolution tends to be much less concise than its biological
counterpart. But more important than Bryant's general criticism are
his case studies of two attempts to employ the theory of cultural
evolution to the explanation of certain historical developments.
These attempts are: 1) Runciman's interpretation of the displacement
of the ancient Greek ``hoplite'' caste towards the end of the
classical period in ancient Greece \cite[p.\  470-481]{bryant:2004} and
2) Interpretations of the history of the Christian religion in
Darwinian evolutionary terms, a typical example of which is
D.S.Wilson's ``Darwin's Cathedral'' \cite[]{wilson:2002} \cite[p.\ 
481-488]{bryant:2004}. Both attempts fall within scientific fields
that are already well covered by a specialist literature on the
respective topics. It is therefore not new riddles that Runciman and
Wilson solve, but new solutions to old riddles they offer. The question
is: Are the evolutionary answers any better?

In Runciman's case, the vanishing of the ``hoplite''-caste and its
specific cultural codes in ancient Greece is interpreted as the result
of evolutionary forces acting against it \cite[]{runciman:1990}.
``Hoplites'' in ancient Greek were heavy armored soldiers that made up
the core of the Greek city state's armed forces. Because a full armour
was expensive it was the rich and nobles of the Greek cities that had
the honor to fight in the hoplite-phalanxes. An important aspect of
the hoplite warrior culture was that the Greek city states did not
entertain standing armies. Those that fought for their city state in
the army were citizens that took part in seasonal warfares and pursued
other obligations during the rest of the year. According to Runciman
the hoplite culture was ``doomed to extinction'' towards the end of
classical greek antiquity because it had evolved under different
circumstances and could not adapt quickly enough to a suddenly changed
environment \cite[p.\  355f.]{runciman:1990}. The hoplite armies with
their part-time warriors and with them the hoplite warrior and citizen
culture became replaced by standing armies supplemented with paid
mercenaries.  The paradigmatic case of these new and more successful
formations is the Macedonian army under Philipp and Alexander. This
change in the military sector was accompanied by changes in the social
formation and political organization. Although they still remained
important cultural centres for a long time afterwards, the Greek city
states were eventually replaced as major political players by the
rising new empires like the Macedonian and, later, the Roman
empire. One of the main driving forces behind the erosion of the
hoplite culture was the growing number of mercenaries as a result of
continuous warfare and political unrest among and within the Greek
{\em poleis}.  People who became expelled from their home cities often did
not have any other alternative than to let themselves be hired as
soldiers, and in a time of constant unrest there were always those in
need of their service. Runciman treats this process as an evolutionary
selection process, wherein the social model of the hoplite culture,
treated by him as a complex of interlocking social regulations on
different sectors, military, economic, social and religious
\cite[p.\ 351ff.]{runciman:1990}, was ultimately replaced by more
successful models of social organization.

Bryant finds fault with this evolutionary account of antique history
for two reasons: The first objection is that Runciman's account is
just another example of the mere rephrasing of terminology that, if we
follow Bryant's criticism, is one of the main effects of the
application of Darwinian evolutionary thinking to the social
sciences. According to Bryant what Runciman tells us does not got
beyond what we know from ordinary accounts of antique history. Nor is
Runciman able to give a better explanation. He merely casts well known
facts and connections into a peculiar evolutionary narrative
\cite[p.\ 470]{bryant:2004}. Without gaining any advantages by drawing
on evolutionary concepts Runciman's account turns thus out to be just
a less concise presentation of a well known subject matter.  But
Bryant finds an even greater flaw in Runciman's evolutionary
presentation. By employing evolutionary concepts which just do not fit
very well to the subject matter in question, Runciman slips into the
error of {\em retrospective determinism} \cite[p.\ 
478]{bryant:2004}.\footnote{Next to the fault of {\em retrospective
    determinism} Bryant identifies four other pitfalls that Runciman's
  evolutionary approach has fallen into: 1) Misidentification and
  misrepresentation of causal processes, 2) supplanting and effacing
  of the intentionality of real flesh-and-blood actors by ambiguous
  and implausible biological hypostazations such as memes, mutants and
  environmental pressures, 3) obfuscatory superimposition of
  internally most differentiated social processes and arrangements by
  screening abridgments such as fitness, adaptation and extinction, 4)
  Underplaying or bypassing of the ``ideational or symbolically
  constructed dimensions and characteristics of social life ... in a
  strained effort to reconfigure the field of action along the lines
  of an organism-environment duality'' \cite[p.\ 481]{bryant:2004}.}
Because evolutionary accounts of cultural processes typically downplay
the role of human intention, they underestimate the degree to which the
outcome of historical processes is liable to human action and
planning.  According to Bryant the social and cultural transitions
that took place at the end of the Hellenic age can be much better
understood with the figure of {\em challenge and response} than in
evolutionary terms.\footnote{To people unacquainted with the way
  explanations in social sciences usually work, the figure of
  ``challenge and response'' might as such appear much more vague,
  ambiguous and less concise than the concept of a Darwinian
  evolutionary process. But as spelled out by Bryant in the case of
  the ``hoplite culture'' it is in fact no less concise but at the
  same time much more appropriate to the subject matter than
  Runciman's evolutionary account \cite[p.\  470ff.]{bryant:2004}.} The
Greek city states faced a challenge in form of growing numbers of
soldier armies and the visible military advantages that could be
gained with professional combatants instead of part time combatants. But
in no way they were thereby ``doomed to extinction'' as Runciman
believed.  Rather, the question was if they were able to find an
adequate response to this challenge.  And indeed they did respond to
the challenge by employing professional militias themselves and by
forming alliances \cite[p.\ 480]{bryant:2004}. In the end they were not
successful, which may partly also be due to contingent factors such as the loss
of a small number of decisive battles. Had the response been
successful, the ``hoplite culture'' could have been retained with only
minor adjustments.

Of course one could ask at this point whether the fault really lies
with the theory of cultural evolution. Maybe Runciman just did not
make a very clever use of the evolutionary concepts. After all, there
is no strict necessity by which an evolutionary account of some process in
cultural history must slip into the mistake of retrospective
determinism or any of the several other defects that Bryant diagnoses
\cite[p.\ 481]{bryant:2004}. Maybe, such mistakes could be avoided,
while still employing evolutionary concepts. But then, specific
theoretical approaches are often in a certain way suggestive. And it
seems that the evolutionary approach is just not very appropriate to
explain the kind of short term cultural transition processes 
that Runciman submits to an evolutionary analysis. Further below,
however, we will see that even for long term cultural development
processes it can be very difficult to draw up a precise evolutionary
description.

Taking Bryant's criticism a step further and linking it with some of our
previous reflections about suitable application scenarios or application
limits of a Darwinian evolutionary theory of culture, we can
conjecture\footnote{To actually demonstrate this conjecture a more detailed
  analysis of Runciman's evolutionary concepts would be necessary.} that
Runciman's basic mistake was to apply the evolutionary scheme to a process
that took place on a time scale which is short enough to fall into the time
horizon of human planning. Now, while the theory of cultural evolution does
take account of intentional or planned human action by treating it as ``directed
mutations'' (kinds of mutations for which there exists no analogon in
biological evolution, where mutations are always random mutations) it does not
explain the single directed mutations itself. Therefore, the theory of
cultural evolution will not yield an appropriate explanation of cultural
development processes that consist, technically speaking, only of one or a few
single ``mutations''.

The other one of Bryant's two case studies of the failure of the evolutionary
theory of culture concerns the possibility of interpreting the genesis of the
Christian religion on the basis of Darwinian cultural evolution. An ambitious
study that follows this approach is David Sloan Wilson's ``Darwin's
Cathedral'' \cite[]{wilson:2002}. Wilson, who is otherwise known for his
collaborate work with Elliott Sober ``Unto Others'', where they make a case
for group selection \cite[]{sober-wilson:1998}, sets out to describe the
mechanism of group selection and to explain why he sees human groups as
adaptive units that are subject to group selection mechanisms. This part,
where Wilson is completely on his own terrain is still the best part of his
book \cite[p.\ 5ff.]{wilson:2002}. But then his reasoning becomes rather naive.
Because group selection is a mechanism that renders functionalistic
explanations\footnote{A {\em functionalistic explanation} is an explanation
  where a phenomenon is not explained by its causes ({\em causal explanation})
  but by its function, e.g.  ``ants are collaborative animals because this
  contributes to the survival of the anthill (or of the ant species etc.)''.
  Functionalistic explanations are never proper explanations because serving
  a certain function is not a sufficient reason for the existence of a certain
  trait.}  plausible (under certain conditions) he hopes to revive a kind of
sociological structural functionalism, just as he and Sober were able to
revive group selection which had formerly fallen into disgrace among
biologists due to its seemingly functionalistic nature \cite[p.\ 
55ff.]{sober-wilson:1998}. But then he never really shows just how the
evolution of religious movements that demand a high degree of dedication and,
in extreme cases, even self sacrifice from their members was due to group
selection mechanisms or to Darwinian evolutionary mechanisms in
general. In this respect his treatment unfortunately remains very vague. Also,
as Bryant contends, Wilson's Sketch of early Christianity almost entirely
rests on the works of the rational choice school of the sociology of religion
like those by Rodney Stark as one of its most prominent representatives
\cite[p.\ 482]{bryant:2004}. Wilson has hardly any insights to offer that go
beyond what rational choice sociologists like Rodney Stark have already said
about the Christian Religion \cite[p.\ 147ff.]{wilson:2002}.\footnote{This is
  also true of Wilson's other examples, the presentation of which to a large extent
  consist of lengthy quotations of what other author's have
  said on the topic.} Therefore, Wilson's ``Darwin's Cathedral'' is again an
example where the evolutionary theory of culture offers just old wine in new
bottles.

In the two examples discussed by Bryant the Darwinian evolutionary
theory of culture was thus not able to provide any new insights. On
the contrary, it led in the case of Runciman even to a somewhat
distorted interpretation of the analyzed historical process. But are
these two examples really symptomatic for the weaknesses of the
evolutionary theory of culture, as Bryant holds, or are they just
examples where the evolutionary theory of culture has been badly
applied?  A complete answer to this question would require carrying
out a systematic survey, which goes beyond the scope of this
book. However, there is a provisional short cut which can help us to
get around this difficulty: Instead of looking at further examples of
evolutionary theory in order to find out how well they deal with
cultural history, we can also look at some examples of cultural
history in order to find out in how far they are evolutionary.  There
is one very prolific recent author whose latest works seem
particularly well suited for such an attempt. This author is Jarred
Diamond who has presented in his books ``Guns, Germs and Steel''
\cite[]{diamond:1997} and ``Collapse'' \cite[]{diamond:2005} a
fascinating and fresh approach to the study of cultural history based
on geological and archeological data. Diamond is an evolutionary
biologist who only lately turned to cultural history. He therefore
knows the Darwinian theory of evolution very well. If we can expect
anyone to transfer Darwinian evolutionary concepts from biology to the
study of human culture then we can expect this the most likely from a
biologist turned historian of culture.\footnote{In fact, Diamond's
  ``Guns, Germs and Steel'' is a favorite reference of adherents of a
  Darwinian evolutionary theory of culture. See, for example,
  \cite[p.\ 104f.]{dennett:2006}.} The first of the two mentioned books
``Guns, Germs and Steel'' covers roughly the last 13 000 years of
human history and tries to answer the question as to why some
civilizations were more successful than others and why ultimately the
Europeans won over most of the other civilizations and not the other
way round. The book thus covers processes within a very large time
horizon, large enough so that the problem of the short time horizon
which rendered Runciman's evolutionary account of a phase in ancient
Greek history implausible will not interfere. The second book is about
past and present environmental catastrophes. The subtitle ``how
societies chose to fail or succeed'' suggests that societies differ in
whether they manage to solve the environmental problems or whether
they fail to do so.  Is there maybe a process of selection going on here?

Let us look at ``Guns, Germs and Steel'' first. Diamond examines the
question why some civilizations develop technologies that make them
stronger than others, why some are faster doing so than others and why
some societies carry germs with them that are lethal to others but not
the other way round. Why, for example, did the Indians die from the
diseases the European invaders brought to them, while the European
invaders did not die from any Indian diseases?  The answer that
Diamond gives, and which is strikingly well supported by the empirical
evidence Diamond relates to, is that this depends primarily on the
habitat a society lives in and the environmental conditions it
faces. Why do some societies have agriculture and others do not?
Because plants that are suitable for agriculture (and there exists
only a small number of plant species for which this is the case) grow
naturally only in some specific areas of the world
\cite[p.\ 131ff.]{diamond:1997}. Why do some societies raise livestock
and others don't? Again, only very specific animal species can be kept
as livestock. And these are not spread all over the globe
\cite[p.\ 157ff.]{diamond:1997}. Why did the technology of agriculture
spread much faster in Eurasia than in America after it had been
discovered? It did so because the Eurasian continent stretches along
an east-west line with roughly similar climate and environmental
conditions, which means that the cultivation of the same crops can
spread easily and without many adjustments along the east-west line
\cite[p.\ 176ff.]{diamond:1997}.  Why are the Australian aborigines not
organized in large hierarchical kingdoms? Because, for the reasons
mentioned above, they did not even get as far as inventing
agriculture. How could they possibly have taken the further steps of
civilizational development that are based on agriculture
\cite[p.\ 295ff.]{diamond:1997}? So, the explanation that Diamond
offers for the civilizational development of societies consists in
describing the natural environmental conditions they live in and in
how far the invention and use of certain cultural techniques requires
certain environmental conditions to be fulfilled. It should be
observed that Diamond only talks about what is required for certain
cultural developments to take place. Very little is said about how
these processes take place and it is completely left open if they
follow a Darwinian evolutionary scheme or not.  The case that gets the
closest to an evolutionary description of the process of the
development of a cultural technique is Diamond's account of the
invention of agriculture.  Agriculture comprises several interlocking
work phases (sowing, harvesting, threshing, grinding and baking), the
suitability of which for the purpose of food production can impossibly
have been known to humans before they had agriculture. Therefore,
agriculture cannot simply have been invented but must have been
introduced stepwise in a gradual process. The question is then, what
the intermediate steps between hunting and gathering and fully fledged
agriculture are. Diamond suggests a slow replacement of hunting and
gathering as means of food production starting with ``accidental''
grain fields on rubbish dumps, continuing with garden keeping by
hunter-gatherers and, finally, resulting in cultivation
\cite[p.\ 104ff.]{diamond:1997}. Is this then an evolutionary process
that Diamond describes? It could be, but neither Diamond's description
nor the empirical data that it is based on could really sufficiently
support this claim.  The processes could also be one of a linear
sequence of inventions building on top of each other with no selection
taking place other than a simple choice of preferred technology. 

Only where Diamond discusses the role that germs played in the world-conquest
of the Europeans, he employs evolutionary theory. But here it is solely
biological evolution that he refers to. The most lethal germs originated from
livestock. Since Europeans (or Eurasians, for that matter) had been raising
livestock for many centuries before they made contact with the cultures on
other continents, the European population had had time enough to evolve
resistance against the associated diseases, but not vice versa
\cite[ch. 11]{diamond:1997}.

The conclusion to be drawn about Diamond's "Guns, Germs and Steel" is that it
provides little evidence for the benefits of an evolutionary
theory of culture. There is no denying that it would surely be possible to
rephrase the whole book in an evolutionary jargon.  But what would be the
point of doing so? And to base an empirically well founded evolutionary
explanation on Diamond's book would require gathering much more empirical data
on the civilizational development processes taking place than is presented by
Diamond.\footnote{The data that would be needed for such an endeavor may not even be
available at all. The elegance of Diamond's approach is among other things due
to the fact that he makes good use of his data, but, at the same time, does
not overinterpret it.  He does not try to answer questions that cannot be
answered with the help of the accessible geographical, biological and
archaeological data, and neither does he indulge in theoretical disputes that
could not be decided by the same empirical data.}

In Diamond's other book that touches on human cultural history the
non-evolutionary character of Diamonds explanations is even more
obvious. Just as if he had read and taken to heart Bryant's criticism
of the evolutionary theory of culture, he avoids the evolutionary
jargon but employs instead non-evolutionary explanatory figures such
as that of ``challenge and response''.  Why did the Norse culture in
Greenland cease to exist after a few hundred years while the Inuit
continued to live in the same hostile environment?  According to
Diamond, who traces back the downfall of the Greenland Norse in great
detail, the Norse and the Inuit faced the same environmental
challenges but the Norse failed to develop an adequate cultural
response to this challenge \cite[p.\ 248ff.]{diamond:2005}. In his
presentation of this processes Diamond does not make any use of
evolutionary assumptions.\footnote{An adherent of evolutionary
  explanations might be inclined to interpret the prevailing of the
  Inuit over the Norse as an instance of selection in the evolutionary
  sense. But then only the terminology but not the concept of
  selection from evolutionary theory would be applied because in
  biology natural selection is the very factor that shapes all
  features of an organism. In analogy, one would have to explain how
  the many particular features of the Inuit and the Norse culture
  (technology, diet, social order etc.) have been shaped by processes
  of selection. This would be a much more ambitious project and it
  would most probably also require to gather much more data about
  these cultures than Diamond had available.} In order to explain the
failure or success of societies in general, Diamond develops a five
tier model that comprises environmental as well as political factors
\cite[p.\ 10-15]{diamond:2005}, rangeing form climate conditions and
and environmental damage to hostile or friendly neighbours and
ultimately the sociatal responses to these hazards. There is nothing
particularly evolutionary about his model. Also, in his whole book,
Diamond strongly emphasizes the role and importance of human
decisions.  For him it is primarily a matter of choice whether human
societies fail or succeed. This emphasize is partly due to Diamond's
political legacy, which is to warn his readers about the dangers of
environmental catastrophes.  Therefore, it is understandable that he
presents the cultural processes that lead to the destruction of the
environment not as anonymous evolutionary processes beyond human
design and intervention. Still, his presentation is very convincing as
it stands.

In neither of his two books on human culture has Diamond had much use
for a Darwinian concept of cultural evolution. Obviously, if a
biologist turns to the study of human culture and is not absolutely
bent on applying evolutionary concepts to human culture, he will quite
naturally end up with the same kind of explanations (like, for
example, explanations in terms of ``challenge and response'') that are
used by historians, archaeologists, social scientists etc. What
distinguishes Diamond from the typical historian or social scientist
is the great importance he attributes to the natural environment in
shaping human culture. Beyond that, there is little in his books that
betrays the biologist. The little use Diamond makes of evolutionary
concepts underlines the impression that Bryant's criticism is not
merely an outcome of the usual prejudices of a group of scientists (in
this case social scientists) against fresh approaches developed by
outsiders but that the use of Darwinian evolutionary concepts for
understanding the development of human societies and cultures is
indeed strongly limited. At any rate, Diamond's books provide
excellent examples for {\em non-evolutionary} explanations of cultural
developments.

Despite all this criticism it should not be forgotten that the theory
of cultural evolution certainly also does have its assets (see section
\ref{culturalEvolutionAssets}). Therefore, summing the discussion of
the evolutionary theory of culture up, it can be concluded that there
is good reason to assume that there are some evolutionary processes in
human history and cultural development that can be understood as
Darwinian evolutionary processes. However, this also means that it is
{\em only} some of the many and varied types of development processes
occurring in human societies that can reasonably be described in
Darwinian evolutionary terms. It is therefore not to be expected that
a Darwinian evolutionary theory of culture will ever attain the rank
of the one great frame paradigm for social sciences as the theory of
evolution does in biology. To call for a unified theory of cultural
evolution, as Mesoudi, Laland and Whiten do, can thus only be regarded
as a mistake. The imminent danger of a unified theory of cultural
evolution is that it induces us to disregard other and possibly better
alternative explanations for cultural development processes.  At any
rate, we must neither assume that all cultural phenomena can be
explained on an evolutionary basis, nor can we assume without further
reason that if we have an evolutionary explanation for a certain
cultural phenomenon that it will be the only possible explanation or
that it really tells us everything there is to know about it. This has
important consequences for the way we have to look at the evolutionary
models of altruism discussed in the following chapter.  While when
applied to biology these models encompass all possibilities how
altruism can evolve that have hitherto been conceived of, we cannot
assume that these models cover all or even just the most important
possibilities of how altruism evolves in culture.  At most, it can be
maintained that the evolutionary models describe mechanisms of the
evolution of altruism that -- under the reserve of its empirical
verification\footnote{See chapter \ref{empiricalResearch}.} -- may be
at work in human culture side by side with other non-Darwinian
mechanisms that produce altruism. For example, one of the major
factors that promote altruistic or cooperative behavior (in a broad
sense) in human societies are the institutions of law and law
enforcement.  This factor is nowhere adequately captured in any of the
common models of the evolution of altruism.\footnote{One would have to
  look in other places such as institutional economics to find models
  that come closest to this.} Of course a Darwinian theory of the
evolution of law and the institutions of law enforcement does not seem
completely inconceivable. But so far no such theory has been
produced. Therefore, the existing evolutionary explanations for
altruism cannot claim to offer a comprehensive answer to the question
how altruism evolves in human societies.

\section{Theory and models}
\label{theoryAndModels}
In the preceding subsections different strata of the Darwinian theory of
evolution have been presented and discussed in some detail. What remains to be
clarified is the place that the models of the evolution of altruism discussed
in the following chapter take within this theoretical framework.

As has been shown, there are basically three types of Darwinian evolutionary
theories. There is the -- well known -- theory of evolution in biology which
can be characterized in contradistinction to the other types as a theory of
genetic evolution of living organisms including humans. Then there is
evolutionary psychology (the successor of sociobiology) which is a theory of
genetic evolution of human nature and behavior in particular. Because
evolutionary psychology relies solely on genetic evolution, it could be
regarded as a branch of the common biological theory of evolution. But because
its claim to explain human behavior and psychology as a part of the genetically
evolved human nature is highly controversial -- much more controversial than
the biological theory of evolution is nowadays among scientists -- it is
advisable to treat it as a different stratum of evolutionary theory. Finally,
there is the theory of cultural evolution which applies Darwinian evolutionary
thinking and evolutionary models to the development of human culture but does
not assume that human culture is determined by the genes.

% Of course, these three types of evolutionary theory represent ``ideal
% types'' and there exist mixed forms like the theory of gene-culture
% co-evolution.  Also, one can regard the different types of Darwinian
% evolutionary theory as concrete instances of an abstract generalized theory
% of evolution.  This generalized theory of evolution describes structures or
% processes in an abstract form without regard of any specific subject matter
% (which could be the evolution of genes or other evolving entities). The
% generalized theory of evolution is instanced by applying it to a specific
% subject matter and potentially adding further principles and conditions
% which are specific to the respective subject matter. Thus, when the
% generalized theory of evolution is ``applied'' to genetic evolution the laws
% of genetics are added to the three ``Darwinian modules'' of the generalized
% theory of evolution.

The models that will be examined in the following chapter are game theoretic
models of evolving strategies. No claim is made about how these strategies are
implemented, e.g. whether they are coded by the genes of an organism or
whether they represent some kind of learned wisdom of human individuals.
Therefore these models should be understood as models within the theoretical
framework of a generalized theory of evolution. In principle, they can be
applied to both genetic and cultural evolution. The ``Darwinian modules''
enter into these models mainly through the replicator dynamics that is used in
these models. Variation or mutation is (except for a very trivial form) not
present in the models presented in the following. Still, they suffice to model
typical selection processes as they occur once a certain ``meaningful'' mutation
has appeared on the scene and challenges the existing types.

The models themselves remain abstract that is, they do not specify how
reproduction, variation and selection takes place. When applying the models to
a biological context, this does not pose a problem because what is meant is
always reproduction of genes, mutation of genes (variation) and selection of
organisms (and thereby indirectly of genes, too). But when applying
evolutionary models to the social sciences, these must be specified
empirically. Such a specification or explanation of what reproduction,
variation and selection means will to some degree be context specific as there
are different kinds of reproduction, variation and selection in cultural
evolution. An abstract evolutionary model can -- in principle -- be applied to
social sciences if there exists at least one context for which the mechanisms
of variation, reproduction and selection can be specified empirically. Whether
the application of the model in this context is meaningful in the sense of
providing new insights and empirically testable hypotheses is yet another
question. In the next chapter different models for the three basic types of
evolutionary altruism will be presented and supplemented by considerations
concerning their interpretation in a cultural context. The question of their
empirical impact will not be considered until the subsequent chapter.

