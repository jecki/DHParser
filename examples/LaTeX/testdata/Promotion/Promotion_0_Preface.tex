
\chapter{Preface}

This book is about computer simulations of the evolution of coopertation and altruism. 
Already at the time when I started writing this book I was quite convinced that
this kind of computer simulations is a pretty {\em worthless} tool for understandaning
and explaining how and why altruism evoloves in nature or among humans. I had at that time already programmed a few simple computer simulations and convinced myself of the notorious arbitrariness of the results that can be produced with computer simulations in the social sicences. The reason why I wrote
a book about it nonetheless was that this book is a Ph.D. thesis and, as it sometimes happens in university philosophy, I was not really free to chose the topic I wanted to write my Ph.D thesis about. Now, as anyone can guess, having to write about some topic that one does not to consider worthwhile is extremely frustrating. The best
I could make out of the situation was to take it as a challenge, a challenge to describe as clearly and in
as much detail as possible just why I do not think that the typical kind of simulations that 
people undertake to study the evolution of cooperation is in any way suited to increase our 
understanding of the reasons why altruism exists beyond the provision of a mere mythology. To me it seems that the reason why this mythology is believed by many philosophers has a lot to do with false respect for the use of mathematical and technological methods in the human sciences. Other than in most natural sciences the use of mathematics in the human sciences (with the possible exception of economics) often does not yield results that reward the effort. This I hold not as a matter of principle -- for there exist no principle reasons why mathematical methods should fail in the social sciences -- but as a matter of fact, testified by the many futile attempts of ameliorating the human sciences by the introduction of a methodology that has been successful in the natural sciences. In this book, I demonstrate in detail for a very specific type of computer simulations, namley, Axelrod-style simulations of the evolution of cooperation that its methodology is misguided. The obvious objection by adherents of the use of computer simulations in the social sciences would be that what has been demonstrated for a certain kind of simulations the importance of which has admittedly been strongly overated for some time must not necessarily be true for other simulations within the field of social sciences. I leave it to the reader to decide whether this objection is justified or whether -- as it appears to me to be the case -- most simulation studies in the social sciences suffer from similar defects as those that I have diagnosed in the case of simulations of the evolution of cooperation. If with this book I am able to convince the reader that for the very specific branch of study that goes under the heading of ``evolution of cooperation'' the methodology of computer simulations is not able to provide us with tenable explanations and if by doing so I am able to convey that a mathematical or computational methodology is not {\em per se} scientific but must -- just like any other methodology -- prove its usefulness by producing empirically tenable results then writing this book will not have been wholly futile.

There are a few people that I feel indepted to and that I would like to thank for having supported me in writing this book. First and foremost, I would like to thank Gerhard Schurz, who was the superviser of my Ph.D. thesis. He provided me with valuable criticism and gave me frequent opportunities to present my ideas on the topic of this book in his research colloquium, which, too,  allowed for valuable and challengeing discussions. Furthermore, I would like to thank Dieter Birnbacher for giving me some very useful and highly appreciated advice concerning the proper use of the English language.

Having expressed my gratitude towards Gerhard Schurz, I'd like to point out (in order to avoid misunderstandings) that Gerhard Schurz holds quite different opinions on some of the topics discussed in this book. Apart from having a more optimistic opinion about the usefullness of computer simulations in the social sciences, he is most notably much more confident regarding the high potential of the evolutionary theory of culture (discussed in chapter \ref{culturalEvolution} in this book) than I am. (Readers who are interested in this topic might want to have a look into his book on the topic of ``Evolution'' which is to appear soon in German language.) My own oppinion regarding the evolutionary theory of culture is that it is a most welcome supplement to other existing explanations for cultural developments. What I have qualms about is the imperialistic attitude of some of the proponents of the evolutionary approach. Led astray by the success of the great overarching theories in the natural sciences like the theory of evolution or the theory of relativity, they seem to believe that if only researchers were willing to submit to some ``unified'' theory of cultural evolution a similar scientific progress as in the natural sciences must ensue. This attitude, however, is severely mistakten. For, so far none of the proposed framework paradigms of cultural evolution is -- in virtue of its proven success (which alone can be the decisive criteria here) -- in a position to render superfluous all other competing paradigms. It is not their being unified that makes scientific theories or paradigms successful but, quite the contrary, their success that eventually leads to the sort of ``unified'' overarching framework theories which some social scientists behold with so much envy in the natural sciences. As long as its success does not qualify a particular competitor theory as a candidate for a unified science all by itself, the call for a unified theory of something (be it cultural evolution or anything else) does not amount to more than an imperialistic appeal to our ignorance of alternative approaches. It is for this reason that in my exposition of the theory of cultural evolution I accentuate its weaknesses much stronger than its potential benefits. This does not mean that I believe that it is devoid of any potential. Also, it should be understood that I consider this issue to be independent from the question of computer simulations as a new methodology in the social sciences.
