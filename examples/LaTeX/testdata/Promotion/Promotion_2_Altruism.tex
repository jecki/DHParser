\chapter{The riddle of altruism}
\label{altruismRiddle}

In this chapter it will be explained why the existence of altruism poses a
philosophical (and scientific) problem. I try to give a precise
definition of altruism that (1) matches our intuitions about altruism and is
-- save for its greater precision -- more or less equivalent to the common
sense meaning of the word ``altruism'', (2) is wide enough to be applied to
both biological and social contexts, (3) can be used in the context of a
Darwinian theory of evolution. In the following chapter
(\ref{generalizedEvolution}), I discuss the generalized theory of
evolution, which is the theoretical framework into which later the computer
models of the evolution of altruism are to be embedded.

\section{Altruism in a hostile world}

It is commonplace that altruism is a highly desirable attitude. Christianity
even declares charity, which is altruism in its highest form, the prime virtue
of man. At the same time moralists of all epochs and cultures could never help
noticing the deplorable lack of altruism, charity and virtuous observance of
the others needs among humans.  It would not need the constant admonitions of
teachears, priests, prophets and philosophers if altruism was so common and
natural a gift as becomes a creature that -- according to the book of genesis
-- is the very image of a benevolent god. The book of genesis has an
explanation for this unpleasant state of affairs for sure: It was primordial
sin that brought evil into this world.

But when we turn from mythology to science we are apt to get the impression
that it is not at all the existence of sin that poses a riddle because the
sciences almost univocally assume a human nature that is thoroughly egoistic,
if not worse. For economists human beings are pure egoists that, given their
preferences, employ their gift of reason solely to the end and purpose of
maximizing utility for themselves.  The same concept of human nature is shared
by many sociologists, especially those that endorse the principle of
methodological individualism.\footnote{{\em Methodological individualism} is
  the doctrine that social phenomena should be explained as the result of the
  actions of individuals \cite[]{heath:2005}. In order to explain the actions
  of the individuals in turn, it is convenient to resort to the assumption of
  utility maximization, i.e.\ rational egoism.}  And if political scientists
do not strictly adhere to the picture of humans as rational egoists then only
because they allow deviations to the worse.  As Machiavelli put it (when
giving the reason for his warning that the duke should reign in blood but not
touch his subjects wives or fortunes): ``For people sooner forget the death of
their own father than the loss of their father's heritage'' \cite[p.\ 69 (my
translation from German, E.A.)]{machiavelli:1532}.  By saying so, Machiavelli
merely put into words a premise that most of the more prudent political
thinkers do at least tacitly presume. Among the human sciences ({\em
  Humanwissenschaften}) it is -- as can be expected -- at best pedagogics that
offers a more optimistic view of human nature and, possibly, also psychology
to some degree. The presupposition that man is by nature an egoistic creature
becomes even more credible when we turn to biology.  For, the theory of
evolution is virtually based on the idea that can -- albeit coarsely -- be
described as the ``survival of the fittest'', a principle that seems to rule
out any non egoistic behavioral traits other than those which are directed
towards the closest relatives right from the start. It is this (supposed)
consequence of evolution that induced T.H. Huxley to coin the equally popular
phrase about ``nature, red in tooth and claw''.\footnote{For the sake of
  fairness it should be noted that neither Herbart Spencer nor Thomas Huxley
  fully endorsed the view that evolution rules out altruism.  However, they
  both thought of ethics and evolution as being antagonistic, so that if
  ethics (that is the Christian ethics of altruism, benevolence and
  compassion) is to prevail then only because the civilizational process of
  society can (in the long run) somehow overcome the iron laws of evolution
  \cite[]{spencer:1907, huxley:1893}. It is the opposite view that is
  advocated here: Evolution can by itself produce altruistic ethics.}  Thus,
from a scientific point of view the question is not how sin came to this
world, but, quite the contrary, how altruism could come into this world and
how it had any chance to survive therein.

Of course the picture of man as a dyed-in-the-wool egoist (if not worse) is
far too bleak and it may justly be objected that this picture is cynical and
contrary to everday experience.  But still the question remains if people
(sometimes) really are altruistic, how can they afford to be so and, if
altruism is desirable, how can people be induced to behave altruistically? It
is no answer to this question that altruism provides so much benefit ``to all
of us''. For, altruism is good for anybody but not for the altruist himself or
herself. And altruism is best for the egoist that benefits from the altruism
of the others but does not give anything in return. So, once again, how can
altruism survive in the long run? To answer the question in this general form
an evolutionary approach seems quite suited. There is a strong similarity
between the question of when and how altruistic behavioral traits can evolve
in animals and the question of how altruistic norms can emerge and be
sustained in human society. But before any explanation can be given, a
clarification of the concept of altruism is necessary. So, what exactly is
altruism?

\section{The definition of altruism}
\label{altruismDefinition}

As has to be expected when using words from everyday language, the terms
``altruism'' and ``cooperation'' turn out to be somewhat ambiguous upon closer
inspection. For example, is any behavior that turns out to be beneficial to
somebody else to be called ``altruistic'' or only when it is done with the
intention to benefit the other person? And what if something is done with the
honest intention to serve somebody else but only on the premise that the other
person will be grateful and eventually return the favor. Is this kind of
mutual exchange of benefits to be called altruism? And, if yes, must we then
count every instance of a successful business transaction as ``altruistic''
because, if carried through freely, it is to the mutual benefit of the
business partners? Part of these ambiguities of the term ``altruism'' stem
from the fact that the common notion of ``altruism'' is closely connected with
moral questions and certain ideals of moral virtue. To avoid confounding the
different meanings and aspects of altruism, an explicit definition of the term
``altruism'' is needed.

In this section ``altruism'' will be defined in view of the general theory of
evolution that will constitute the framework of this examination. In order to
assure that the definition of altruism meets our research interests, the
consequences of this definition will be discussed at some length, thereby
comparing it with the ordinary language understanding of altruism. The
definition of altruism must not only be clear enough to allow the examination
of the problem within an evolutionary approach, it is also important that the
kind of altruism captured by the following definition is the very altruism for
which the ``riddle'' of how and why it came into this world and whether and
under what conditions it has a chance to survive is to be solved.

For the rest of this study the following definition of altruism will be used:
A trait or a type of behavior of an individual is called {\em altruistic} if
it benefits another individual at a cost for the individual itself without
immediate or equal return. Some behavior is thus {\em altruistic},

\begin{enumerate}

\item if it is {\em beneficial} for another individual
\item and if it is {\em costly} for oneself
\item and if an equal {\em return is not guaranteed}
\item and if the {\em altruist chooses} (or, in case of non-intentional animal
behavior, simply if it depends on the altruist) whether the transfer of
benefits takes place

\end{enumerate}

The definition is abstract enough to be applied both to sociological and
biological settings, for it is not required to assume that the individuals can
think or even have a consciousness at all. In a biological setting the costs
of the individuals would be interpreted as {\em fitness costs} in the sense of
adverse effects on the reproduction rate.  Also, the definition is designed to
be wide enough to cover both reciprocal altruism and genuine altruism. {\em
  Reciprocal altruism} is altruism on the premise that the bestowed benefits
will be returned, but with a certain risk that this might not happen. (Only
when the latter is excluded, it is, according to the definition, not called
``altruism'', any more. This would be the case if the return was immediate.)
Thus, even when favors are reciprocated we speak of altruism, but we only
speak of altruism when there exists an opportunity for cheating.

{\em Genuine altruism} on the other hand means that it is sure that the costs
for benefiting other individuals will never be returned. In a biological
setting a certain behavioral trait of an individual is genuinely altruistic if
it helps increasing the reproduction rate of some other individual and at the
same time decreases the reproduction rate of itself. That this is indeed possible
and that therefore genuine altruism can survive in nature despite the fact
that the survival of some phenotypic trait crucially depends on its increasing
the reproduction rate of its bearer over his or her competitors, is one of the
most astonishing results of group selection that will be discussed in chapter
\ref{groupSelection}.

But does the above definition match our common understanding of altruism? The
definition is in some respects rather wide so that it might be disputed that
all of the possible types of behavior that match the definition can
legitimately be called altruism. For example, a carpetbagger investing a high
amount of money into some risky business with the hope of getting a multiple
of his investment back would -- according to the above definition -- have to
be classified as an altruist, although we probably would not call his
financial speculation ``altruism'' in everydays life. We might even hesitate
to speak of ``cooperation'' in this case. The above definition is indeed
counter-intuitive in cases like this one. The problem is not specific to this
definition but it is already apparent in the notion of ``reciprocal altruism''
which for this reason could equally well be called ``reciprocal
egoism''.\footnote{This was suggested by Gerhard Schurz.} The main reason that
-- in spite of these objections -- speaks for a wide definition of altruism is
that it captures all behavioral traits that lead in some form or another to
cooperation. If the less genuine forms of altruism were named ``reciprocal
egoism'' or similar, there would still be the need for a distinction between types
of egoism that encourage cooperation and others that do not. Since we want to
find out what chances of survival altruism and cooperation have in a competitive world, it is therefore advantageous to draw the line between altruism
and egoism in such a way that the realm of altruism more or less matches that
of cooperation. In order to avoid too great a confusion with the common usage
of words, it might help to think of ``reciprocal altruism'' as of a diminutive
of altruism. ``Reciprocal altruism'' is then a kind of altruism that is merely
reciprocal but not more.
% \footnote{Another reason -- albeit less important -- in favor of a wide
%   definition of altruism is the fact that in the branch of science that is
%   studying the evolution of altruism it has become a common standard to
%   bracket altruism with cooperation.}

There are also a few other points that have to be noted about the above
definition of altruism. Although the questions discussed in this study have a
moral connotation as well, the definition above is purely descriptive. The
difference this makes can be explained as follows: When speaking of
``cooperation'' or ``altruism'' the moral connotation usually suggests that
cooperation and altruism are generally good and laudable. But this is not
necessarily the case: The (illegal) pre-arrangement of prices by competitors
on a market, for example, is certainly not a laudable case of cooperation and
it could hardly be labeled ``altruism'' because -- since altruism is commonly
considered laudable -- the word would not be used in cases of cooperation that
seem ethically doubtful in the broader context. The problem that this example
exposes is, however, not a problem specific of the above definition of
altruism, but a fundamental problem of moral philosophy: Moral philosophy
tries to classify human actions and attitudes into categories of good and bad.
But even actions that are generally thought of as being morally laudable can,
when appropriate circumstances are given, turn out to be morally deplorable.
Killing people is generally considered bad and saving lives is good, but for a
soldier in war killing people is a virtue. The problem has to do with the
contextuality of moral attributes. To avoid false conclusions this kind of
contextuality should be borne in mind.

In addition to the fact that altruism is typically considered to be morally
laudable, there exists a more specific reason, why the descriptive definition
of ``altruism'' given above is important for the discussion of ethical
questions. Even if ``altruism'' in a descriptive sense can also be applied to
cases of illegal or antimoral conspiracy, it is hardly imaginable (except may
be for extreme ethical standpoints like Nietzsche's ``Herrenmoral'' or Ayn Rand's ``Objectivism'') that
there can be such a thing as moral conduct that does not involve altruism in
any form. Altruism in a descriptive sense thus seems to be a necessary though
not sufficient condition for morality. Moral conduct typically demands from
the individual to follow certain norms even though this may be costly and even
when no reward is assured. Thus the questions of whether and to what extent
morality has a chance to flourish in a competitive world crucially depends on
the question of whether altruism in the descriptive sense is possible in a
competitive world.

The above definition of altruism does not contain any psychological or
teleological elements such as intention (if humans are meant) or functional
design (if organisms are concerned) directed towards benefiting the other.
This might appear odd at first sight because cases of involuntarily
benefiting others or of benefits which are mere side effects do not seem to be
excluded (as they should). As an example a tree that casts a shadow during a
hot summer day might be taken. The tree's shade is most welcome for humans
resting under it. But is the tree's casting a shadow therefore to be called an
example of biological altruism? The obvious objection would be that the tree
is not designed to provide needful humans or animals with a pleasant shade on
a hot summer day. It is designed to catch sunlight for photosynthesis, the
shade being an unintended side effect. To call this ``altruism'' would surely
overstretch the meaning of the word. This objection would be valid, but
luckily cases like this one are covered by the requirement that altruism
should be costly. There are no extra costs for the tree to cast a shadow. Thus
casting shadows does not count as altruistic according to our
definition. (Biologists sometimes treat this kind of phenomena under the heading of ``byproduct mutualism'', where ``byproduct mutualism'' can be regarded as a degenerate case of altruism \cite[p.\  42]{dugatkin:1997}.\footnote{From an empirical point of view it is quite reasonable to discuss byproduct mutualism in connection with altruism because 1) it is often very hard to distinguish empirically whether some kind of behavior is an instance of, say, reciprocal altruism or merely byproduct mutualism and 2) there is evidence that in many cases byproduct mutualism is a stepping stone in the evolutionary path that leads to the development of altruism. (See chapter \ref{biology} for the empirical examples
  in biology, where this question can often be raised.)}) But if the tree were
an apple tree then its growing fruits would legitimately be called altruistic
because growing fruits involves a cost for the tree. It does not matter here
that the tree is a plant and therefore cannot have intentions.  Presumably,
the tree's growing fruits is a case of reciprocal altruism, as the humans or
animals eating the apples might help the tree to spread its seeds in return.

One might even take this question a step further by arguing that while merely
accidental benefits for others are excluded by the criterion of cost, this
leaves open the possibility for altruistic acts that are not bestowed from a
benefactor on a beneficiary but reaped from the benefactor by force. An
example would be a rabbit that is eaten by a fox. It would not help to
reintroduce the notion of functional design into the definition because a
rabbit is -- in a way -- perfectly well designed to serve as fox food.  Absurd
cases like that of the rabbit that altruistically lends itself to be eaten by
a fox or -- to take an example from social life -- of crime victims that serve
as altruistic benefactors to robbers, thieves and burglars are ruled out by
the fourth criterion, according to which it must depend on the altruist
whether the transfer of benefits takes place.  The criterion is wide enough to
capture altruistic actions by humans as well as animals.  Even though it may
not be apparent at first sight, the criterion can also be applied to inborn
(or genetically determined) altruistic traits as they occur in
mutualisms.\footnote{A {\em mutualism} is an interspecific association of
  different species to their mutual benefit.  An example would be the
  association of hermit crabs and sea anemones.} In this case ``to depend''
means two things: 1) that it is a genetically determined trait of the altruist
that makes the transfer of benefits possible and 2) the altruist could also
exist without this trait.

The choice of costs rather than intentions as a criterion for altruism has the
advantage that it is more objective and that it can be applied equally in
biological and sociological settings without the need for differentiating
between human intentions, animal intentions, mere functional design of
primitive organisms that do not have intentions etc. Furthermore, in a
sociological setting the assumption is certainly
unproblematic that whenever some altruistic act needs a certain effort, it
will not be performed without the intention to perform it.

There is, however, also a downside to neglecting intentions in the definition
of altruism. In everday life, especially when human relations like friendship
and love are concerned, there exists a distinction which is closely connected
to the psychological aspects of altruism and which is at the same time crucial
for the valuation of altruism: The distinction between real or true altruism
on the one hand and false or merely pretended altruism on the other hand.
Altruism is commonly regarded as true only if the benefits one person bestows
unto another are given for the sake of the other person and not merely out of
egoistic motives like prestige or the hope for a reward. In the latter case
the kind of altruism displayed would be regarded as merely pretended and not as
honest. Such psychological subtleties are not covered by the above definition
of altruism, which is designed to be operational in the first place. Still,
should the question arise, the definition of altruism can easily be rendered
more precise, especially so, since the distinction between altruism out of
friendship and opportunistic altruism also leaves its mark in the outer world:
As the psychological findings indicate, the kind of altruism that friendship
evokes is reciprocal only on a long term basis and even defies short term
reciprocity \cite[]{silk:2003}.

As a final remark, it should be noted that there exists a very specifically
philosophical question about altruism, which will only be discussed here
briefly and in the following be left out completely. It is the question
whether true altruism is possible at all. It could be argued that whenever a
person behaves altruistically, he or she does so only because he or she
derives at least an emotional reward of some kind such as, say, personal
satisfaction. But then -- as the argument runs -- the action would not be
truly altruistic any more because it is done for one's own satisfaction.
Indeed, it is difficult to believe that anybody can do anything without at
least some kind of inner reward.  Only a perfect saint might be able to commit
the most gracious acts of altruism and charity and at the same time be wholly
disgusted by what he is doing. If one insists on speaking of true altruism
only where it reaches a level of perfect saintliness then there is no altruism
in this world.  But as long as it is not deliverance that is sought and the
problem of altruism is confined to how and to what extent altruism has a
chance to emerge in natural and cultural evolution, it is safe to assume that
already levels of altruism below perfect saintliness can be morally
satisfactory.
%  The restrictions that the evolutionary process puts onto the
% genesis and survival of altruistic traits (as in fact on any behavioral
% traits) will be discussed in the next section.
