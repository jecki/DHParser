
\chapter{Ergebnis: Das Scheitern von Voegelins Bewusstseinsphilosophie}

\section{Die ungelösten Fundamentalprobleme von Voegelins Ansatz}
\label{Scheitern}
\label{Fundamentalprobleme}

Nach der bisherigen, sehr ins Detail gehenden Erörterung von Voegelins
Bewusstseinsphilosophie soll nun auf einer wiederum etwas allgemeineren Ebene
eine Gesamtbilanz dessen gezogen werden, was Voegelin mit seiner
Bewusstseinsphilosophie erreicht hat. Zur Erinnerung werde ich zunächst noch
einmal in aller Kürze Voegelins Modell politischer Ordnung zusammenfassen,
um dann zu klären, ob aus Voegelins Bewusstseinsphilosophie heraus die Fragen,
die zur Begründung dieses Modells hätten geklärt werden müssen, überzeugend
beantwortet werden können.

Voegelins Modell politischer Ordnung lässt sich in der allerknappsten Form wie
folgt zusammenfassen: Politische Ordnung beruht auf der existenziellen
Grundkonstitution des Menschen. Diese existenzielle Grundkonstitution besteht
in der Spannung zum transzendenten Seinsgrund. Das Wissen um die existenzielle
Spannung zum transzendenten Seinsgrund wird dem Menschen durch mystische
Erfahrung vermittelt. Die politische Ordnung muss aus diesem Wissen um die
existenzielle Spannung zum transzendenten Seinsgrund heraus gestaltet werden.
Dann, und nur dann, kann eine gute und dauerhafte politische Ordnung
entstehen. Ohne auf die weiteren Einzelheiten von Voegelins Deutung der
historischen Entwicklung der Ordnungserfahrung und der Ursachen politischer
Unordnung näher einzugehen, kann bereits an dieser Stelle eine Reihe von
Fragen formuliert werden, die als Prüfstein der Begründungsqualität von
Voegelins Modell politischer Ordnung dienen können:

Die erste und wichtigste Frage, die sich stellt, ist die, ob es einen
transzendenten Seinsgrund überhaupt gibt. Wenn behauptet wird, dass der Mensch
in der Spannung zum Seinsgrund existiert, dann sollte zunächst geklärt werden,
ob ein solcher transzendenter Seinsgrund auch nachweislich vorhanden ist. Zur
Begründung der Annahme der Existenz eines transzendenten Seinsgrundes beruft
sich Voegelin auf die mystische Erfahrung. Diese Begründung bleibt nicht
zuletzt deshalb recht dürftig, weil Voegelin nie genau klärt, was mystische
Erfahrungen sind, und wie man sie machen kann. Über das Wesen mystischer
Erfahrungen wird im nächsten Unterkapitel (Kapitel
\ref{Transzendenzerfahrungen}) noch einiges zu sagen sein. Aber bereits
unabhängig davon, was mystische Erfahrungen nun eigentlich sind und ob es sich
tatsächlich um Erfahrungen von etwas Transzendentem handelt, werfen sie eine
Vielzahl von Fragen auf. So stellt sich z.B. die Frage, wie zwischen richtigen
und falschen Erfahrungen unterschieden werden kann. Diese Frage ist besonders
deshalb prekär, weil offenbar nicht jeder Mensch mystische Erfahrungen hat.
Wie können dann aber die mystischen Erfahrungen des einen Menschen für einen
anderen maßgeblich sein?  Dass sie es sein können ist Voegelins feste
Überzeugung, denn er sagt ausdrücklich, dass alle anderen Menschen zum Zuhören
verpflichtet sind, wenn einem Propheten oder Philosophen eine Erleuchtung
zuteil wird.\footnote{Vgl.  Voegelin, Order and History II, S.  6.}  Aber wie
können diese Menschen, die selbst keine Erleuchtung hatten, zwischen echten
und falschen Propheten unterscheiden?  Dieses Problem hat bereits Thomas
Hobbes aufgeworfen, und Hobbes war -- anders als Voegelin -- der Ansicht, dass
die Weigerung, einem Propheten zu folgen, niemandem als Verstocktheit
ausgelegt werden darf, denn der Betreffende leugnet durch diese Weigerung
nicht Gott, sondern er zweifelt lediglich die Glaubwürdigkeit eines Menschen,
nämlich des Propheten an.\footnote{Vgl. Thomas Hobbes: Leviathan oder Stoff,
  Form und Gewalt eines kirchlichen und bürgerlichen Staates, Frankfurt am
  Main 1998 (erste Auflage 1984), S. 51 (7. Kapitel) / S.  218-219 (26.
  Kapitel).} Voegelin besteht dagegen darauf, dass echte
Transzendenzerfahrungen eines Menschen für alle Menschen autoritativ sind,
aber er zeigt nicht, wie das möglich ist, wenn es doch kein Mittel gibt, um
ihre Echtheit zu prüfen.\footnote{Nicht dass Voegelin diese Probleme nicht
  bemerkt hätte, aber er weicht einer Antwort in der Regel durch
  Weitschweifigkeit oder stillschweigenden Themawechsel aus.  (Vgl. z.B.
  Voegelin, Order and History III, S. 299-303.) -- Noch unzulänglicher:
  Voegelin, Order and History V, S.  26. -- Ebenfalls unbefriedigend:
  Conversations with Eric Voegelin. (ed. R.  Eric O'Connor), Montreal 1980, S.
  23/24.}

Mit diesen die Echtheit der Transzendenzerfahrungen betreffenden Fragen sind
die Probleme von Voegelins Konzeption jedoch noch keineswegs erschöpft. Nicht
weniger problematisch sind die Zusammenhänge zwischen der mystischen Erfahrung
und der konkreten politischen Ordnung. Am auf\/fälligsten erscheint hierbei,
dass zwischen diesen beiden Bereichen augenscheinlich überhaupt kein
Zusammenhang besteht. Wenn die mystische Erfahrung, so wie Voegelin dies
unablässig beteuert, eine unerlässliche Grundlage guter politischer Ordnung
ist, so müsste es doch möglich sein, aus der mystischen Erfahrung heraus
Kriterien zu formulieren, die es erlauben, irgendein konkretes Ensemble
politischer Institutionen zu beurteilen. Warum sind die Institutionen der
amerikanischen Demokratie eher mit der Noese zu vereinbaren als die
Institutionen eines faschistischen Staates wie Spanien unter der Herrschaft
Francos? Es dürfte sehr schwer fallen, hier aus Voegelins metaphysischer
Theorie heraus ein Urteil zu fällen. Zwar hat Voegelin es nicht an (oft sehr
einseitigen) politischen Lagebeurteilungen fehlen lassen, aber selten wird
dabei deutlich, wie diese sich aus den Grundsätzen seiner politischen
Philosophie ergeben.  Voegelin hat noch nicht einmal versucht, aus der Noese
irgendwelche moralischen Prinzipien oder Tugenden herzuleiten. Dergleichen
wäre ihm wohlmöglich bereits als eine dogmatische Entgleisung
erschienen.\footnote{Siehe dazu die Naturrechtskritik, die Voegelin unter
  Rückgriff auf Aristoteles' Begriff der "`phronesis"' in seinem Aufsatz "`Das
  Rechte von Natur"' übt. (Vgl.  Voegelin, Anamnesis, S. 128.) -- Zur
  ethischen Theorie Voegelins vgl. Julian Nida-Rümelin: Das Begründungsproblem
  bei Eric Voegelin. (Typoskript eines Vortrages beim Internationalen
  Eric-Voegelin Symposium in München August 1998, Eric Voegelin-Archiv in
  München).} Aber wie kann mit Hilfe der Noese die gesellschaftliche und
politische Wirklichkeit gestaltet werden, wenn es noch nicht einmal möglich
ist, bestimmte sittliche Prinzipien aus der Noese abzuleiten?\footnote{Hans
  Kelsen legt den Finger auf die Wunde, wenn er feststellt, dass die
  metaphysischen Begriffe, die in der Vergangenheit zur Rechtfertigung
  bestimmter Gerechtigkeitsvorstellungen herangezogen wurden und auf die sich
  auch Voegelin beruft (wie u.a. das platonische {\em agathon}, der
  aristotelische {\em nous}, die thomistische {\em ratio aeterna}), lediglich
  Leerformeln sind, die zur Rechtfertigung jeder beliebigen Sozialordnung
  dienen können.  Vgl.  Kelsen, A New Science, S.  15/16. -- Vgl. auch Hans
  Kelsen: Was ist Gerechtigkeit?, 2. Auflage, Wien 1975.}

% \footnote{Bereits im "`Autoritären Staat"'
%   spricht Voegelin ein verwandtes Problem an. Dort geht es um das
%   rechtstheoretische Problem, inwiefern bestimmte Wirklichkeitsbereiche
%   überhaupt durch ein System von Rechtsnormen erfaßt und normativ geregelt
%   werden können. (Denkbar wäre, dass entweder die Einzelvorgänge der
%   Wirklichkeit zu unterschiedlich und irregulär sind, um durch allgemeine
%   Gesetze erfaßt werden zu können, oder dass die Beurteilung nach allgemeinen
%   Gesetzen dem Gerechtigkeitsempfinden zuwiederläuft.) Voegelin bejaht die
%   Regelbarkeit für den Bereich des bürgerliche Gesetzes, aber er leugnet sie
%   für das Staatsrecht - anknüpfend an Vorstellungen von Carl Schmitt, wonach
%   es der {\it Echtheit} von Politik zuwiderläuft, wenn der Souverän
%   verfassungsmäßigen Bindungen unterliegt.}

Vor allem aber stellt sich die Frage, woraus hervorgeht, dass eine auf die
Noese gegründete Politik moralisch gut und den Erfordernissen des menschlichen
Zusammenlebens angemessen ist. Sofern das moralisch Gute nicht gerade durch
die Noese definiert wird, was zu einer Tautologie führen würde, ist es
keineswegs selbstverständlich, dass das noetisch Wahre mit dem moralisch Guten
zusammenfällt. Umgekehrt liefert Voegelin keinerlei stichhaltige Gründe für
seine These, dass eine gute politische Ordnung ohne die Noese ausgeschlossen
ist. Voegelin scheint hier die anthropologische Annahme zu Grunde zu legen,
dass ein Mensch, dessen Existenz nicht durch das noetische Wissen oder eine
kompakte Vorstufe desselben geordnet ist, notwendigerweise ein Chaot sein
muss.  Aber Voegelin unternimmt nicht einmal ansatzweise den Versuch, die
Richtigkeit dieser willkürlichen und gegenüber ungläubigen Menschen auch sehr
ungerechten Annahme zu demonstrieren.

All die soeben aufgeworfenen Fragen sind nun nicht bloß irgendwelche
Einzelfragen, die sich im Rahmen von Voegelins politischer Theorie nebenbei
auch noch stellen. Vielmehr handelt es sich um Fundamentalprobleme, mit deren
Lösung der gesamte Ansatz von Voegelin steht und fällt. Die Tatsache, dass
Voegelin auf keine dieser Fragen auch nur eine halbwegs schlüssige Antwort
geben kann, erlaubt daher nur eine einzige Schlussfolgerung: Voegelin ist mit
seinem Versuch, die Voraussetzungen politischer Ordnung
bewusstseinsphilosophisch zu begründen, auf der ganzen Linie gescheitert.

\section{Was sind Transzendenzerfahrungen?}
\label{Transzendenzerfahrungen}

In seinen bewusstseinsphilosophischen Schriften beruft Voegelin sich immer
wieder auf Transzendenzerfahrungen. Leider versäumt Voegelin dabei zu klären,
was Transzendenzerfahrungen sind, wie man sie erlangen kann, und ob sie ihren
Namen zu Recht tragen. Will man sich über diesen Zentralbegriff der
Voegelinschen Philosophie Rechenschaft ablegen, so bleibt kein anderer Ausweg,
als ein wenig über das Wesen dieser Erfahrungen zu spekulieren, indem man
verschiedene Möglichkeiten zu ihrer Erklärung gedanklich durchspielt. Im
Wesentlichen sind drei Erklärungsmöglichkeiten zu erwägen: 1) Es gibt
tatsächlich Transzendenzerfahrungen, aber sie sind nur wenigen auserwählten
Individuen zugänglich. 2) Hinter den Transzendenzerfahrungen verbergen sich
bestimmte innere Erlebnisse, die jeder Mensch wenigstens gelegentlich hat. Nur
messen verschiedene Menschen diesen Erlebnissen eine unterschiedliche
Bedeutung für ihr Leben bei. 3) Es gibt keine Transzendenzerfahrungen und
alles, was über sie gesagt wird, ist lediglich leeres Gerede. Diese drei
grundsätzlichen Erklärungsmöglichkeiten sollen nun im Einzelnen etwas näher
beleuchtet werden.

{\it 1. Möglichkeit: Transzendenzerfahrungen in Form von Erleuchtungen
  einzelner Individuen.} Denkbar ist, dass es sich bei den
Transzendenzerfahrungen um besondere Erlebnisse handelt, die nur ganz
bestimmten, man könnte sagen "`religiös privilegierten"' Menschen widerfahren.
Es stellt sich dann die Frage, ob diese Erlebnisse tatsächlich die Folge des
Eindringens eines transzendenten Seins in die Immanenz sind, oder ob es sich
dabei bloß um ein psychisches Phänomen ähnlich einer Geisteskrankheit handelt.
Die erste Variante ist eher unwahrscheinlich, da es für die Existenz einer
transzendenten Seinsphäre keine anderen Anhaltspunkte gibt als eben diese
Erfahrungen selbst, die einer Geisteskrankheit oder einem Drogenrausch unter
Umständen zum Verwechseln ähnlich sehen können.\footnote{Vgl. William James:
  The Varieties of Religious Experience, Cambridge, Massachusetts / London,
  England 1985 (zuerst 1902), S. 11-28.}  Aber selbst wenn es sich um
"`echte"' Transzendenzerfahrungen handeln sollte, steht auf Grund des schon
erwähnten Problems der wahren und falschen Propheten außer Zweifel, dass diese
Erfahrungen außer für den, dem sie widerfahren, für niemanden Autorität
beanspruchen können. Ob Voegelin so interpretiert werden muss, dass die
Transzendenzerfahrungen nur wenigen religiös privilegierten Individuen
zukommen, lässt sich nicht eindeutig bestimmen. Voegelins historischer Ansatz,
nach welchem die "`Seinssprünge"' zunächst als Erleuchtungsereignisse in
einzelnen Individuen stattfinden, legt diese Interpretation
nahe.\footnote{Vgl.  Conversations with Eric Voegelin. (ed. R.  Eric
  O'Connor), Montreal 1980, S. 25. -- Bereits in den "`Politischen
  Religionen"' schreibt Voegelin: "`Die Erneuerung kann in großem Maße nur von
  großen religiösen Persönlichkeiten ausgehen -- aber jedem ist es möglich,
  bereit zu sein und das seine zu tun, um den Boden zu bereiten, aus dem sich
  der Widerstand gegen das Böse erhebt."' (Eric Voegelin: Die politischen
  Religionen, München 1996 (zuerst 1938), S. 6.)} Dem steht jedoch Voegelins
bewusstseinsphilosophischer Ansatz entgegen, der, gerade weil dabei das
Bewusstsein {\it des} Menschen untersucht wird, beinhaltet, dass die
Transzendenzerfahrungen jedem Menschen zugänglich sind, wenn es auch von
Mensch zu Mensch Unterschiede in der Intensität der Erfahrung geben kann.

{\it 2. Möglichkeit: Transzendenzerfahrungen als innere Erlebnisse.} Es könnte
auch der Fall sein, dass es sich bei den Transzendenzerfahrungen, die Voegelin
meint, um innere Empfindungen und Erlebnisse handelt, welche im Seelenleben
eines jeden Menschen vorfindlich sind, auch wenn diesen Empfindungen im
bewussten Denken und Handeln nicht immer eine gleichermaßen bedeutsame Rolle
eingeräumt wird. Zu diesen inneren Empfindungen gehören vermutlich alle Arten
von Gefühlserlebnissen, die aus dem Alltäglichen herausfallen, eine starke
emotionale Besetzung aufweisen und entweder nicht unmittelbar einem äußeren
Anlass zugerechnet werden können oder durch ihre Intensität über diesen Anlass
hinauszuweisen scheinen. Dazu könnten beispielsweise Erinnerungen, Träume,
Tagträume, Drogenerlebnisse, Sex, bestimmte durch Kunst oder Musik
hervorgerufene Empfindungen sowie sich manchmal plötzlich einstellende und
scheinbar grundlose Gefühle von großer Angst oder Freude zählen. Der Versuch,
den Bereich der für Transzendenzerfahrungen in Frage kommenden seelischen
Erlebnisse näher einzugrenzen, begegnet nicht zuletzt deshalb großen
Schwierigkeiten, weil es auch eine Frage des Temperamentes ist, ob jemand
sagt: "`Ich habe eben ein wenig vor mich hingeträumt"', oder ob er erklärt:
"`Ich habe soeben in der Meditation für einige Sekunden das Eindringen der
Transzendenz in mein Bewusstsein erfahren"', oder ob jemand mitteilt: "`Ich
hatte plötzlich ein unerklärliches Angstgefühl"', oder stattdessen vielmehr
behauptet: "`Mir sind gerade die Schauer des Numinosen über den Rücken
gelaufen."' Damit soll nicht gesagt werden, dass die jeweils letztere Form,
die entsprechende Erfahrung in Worte zu fassen, reine Hochstapelei ist. Es ist
nur der Tatbestand festzuhalten, dass derartige Erfahrungen nicht aus sich
selbst heraus ihr Wesen offenbaren, sondern dass dies erst die Folge einer
menschlichen Deutung ist.

Nun spricht nichts dagegen, das eigene Leben im Zeichen irgendwelcher dieser
Erfahrungen zu führen, sofern man das Gefühl hat, dass dadurch das Leben
reicher wird und einen höheren Wert bekommt. Auch mag es für den eigenen
Seelenhaushalt von großer Wichtigkeit sein, auf derartige Erfahrungen
achtzugeben.  Aber gegenüber Voegelins Deutung muss, sofern er diese Art von
Erfahrungen meint, zweierlei festgehalten werden: Erstens können derartige
Erfahrungen zwar Sinn und Wert vermitteln, nicht jedoch Wahrheit. Wahrheit
besteht in der Übereinstimmung von etwas Gemeintem mit der Wirklichkeit.
Wahrheit kommt nicht bereits dadurch zustande, dass etwas als wahr empfunden
oder als höchst real erfahren wird, denn auch Irrtümer werden, solange man von
ihnen überzeugt ist, als wahr empfunden. Grundsätzlich sind innere
Erfahrungen, Evidenzerlebnisse oder Erleuchtungen niemals eine unmittelbare
Quelle von Wahrheit, sondern bestenfalls eine Quelle von Ideen, über deren
Richtigkeit erst eine sorgfältige Prüfung -- möglichst in einem nüchterneren
Seelenzustand! -- befinden muss. Mit anderen Worten: Man sollte sich eine
gesunde Skepsis gerade und besonders auch gegenüber den eigenen Erleuchtungen
bewahren. Zu den Wesensmerkmalen der Wahrheit gehört zudem ihre Objektivität
oder zumindest Intersubjektivität.  Gerade dies ist jedoch für die in Frage
stehenden Erfahrungen nicht gegeben, welche zunächst einmal höchst
individuelle und persönliche Erfahrungen sind.  Daraus, dass diese Erfahrungen
zunächst nur eine Meinung oder ein Wertempfinden, aber nicht zwangsläufig
Wahrheit vermitteln, ergeben sich zwei gewichtige Konsequenzen: Zum einen sind
Konflikte mit der Realität keineswegs ausgeschlossen, so dass es sehr voreilig
ist, die Realität der inneren seelischen Erfahrungen mit der Realität
schlechthin gleichzusetzen.  Zum anderen kann aus der Erfahrung kein
Gehorsamsanspruch abgeleitet werden, da dieser Art der Erfahrung die objektive
Gültigkeit fehlt.  Damit ist natürlich nicht ausgeschlossen, dass die
Erfahrungen eines Einzelnen auf der Basis freiwilliger Anerkennung für Andere
Bedeutsamkeit erlangen können. Zudem kann sich ein Gehorsamsanspruch immer
noch dann ergeben, wenn das Erfahrene aus anderen Gründen diesen Anspruch
rechtfertigt. (Beispiel: Einem Propheten geht im Rahmen einer göttlichen
Vision eine moralische Norm auf, welche sich auch bei anschließender
nüchterner Betrachtung und unter Abwägung aller Eventualitäten als sinnvoll
und akzeptabel erweist.)
  
Zweitens muss gegenüber Voegelin festgehalten werden, dass jene emphatischen
seelischen Erlebnisse -- trotz gelegentlicher überraschender Übereinstimmungen
zwischen unterschiedlichen Individuen -- im Ganzen von einer solch irregulären
Vielfalt sind, dass es unmöglich ist, sie allesamt auf eine Normerfahrung der
existenziellen Spannung zum Grund zu beziehen. Manch einer empfindet eine
existenzielle Spannung zum Grund, ein anderer wiederum fühlt sich in
der Harmonie des Universums aufgehoben, wieder einer spürt vielleicht die
Nähe Gottes in anderen Menschen. Dies sind ganz unterschiedliche Erfahrungen.
Der Versuch, alle Erfahrungen je nach ihrer Nähe zur Normerfahrung der
existenziellen Spannung zum transzendenten Seinsgrund in "`kompaktere"' und
"`differenziertere"' Erfahrungen einzuteilen und so zwischen ihnen eine
Rangfolge herzustellen, dürfte höchstens für sehr nahe verwandte Erfahrungen
durchführbar sein.

{\it 3. Möglichkeit: Transzendenzerfahrungen als soziales Artefakt.} Als
letzte Möglichkeit muss in Erwägung gezogen werden, dass es
Transzendenzerfahrungen überhaupt nicht gibt, und dass auch nicht jene eben
beschriebenen außergewöhnlichen Seelenzustände damit gemeint sind, sondern dass
es sich vielmehr um Einbildungen oder besser um soziale Artefakte handelt, die
bloß geglaubt werden, weil so viele Menschen davon reden, als handele es sich
um eine Selbstverständlichkeit. Der gesellschaftliche Mechanismus, der zu
diesen Artefakten führt, wird sehr treffend durch das Märchen von des Kaisers
neuen Kleidern veranschaulicht, in welchem bekanntlich ein ganzer Hofstaat in
den höchsten Tönen die Farbenpracht der Gewänder eines splitternackten Kaisers
preist. Darauf, dass bei Voegelin die vermeintlichen Transzendenzerfahrungen
nur ein solches Kunstprodukt der Einbildungskraft unter dem Einfluss der
eifrigen Lektüre religiösen Schrifttums sind, deutet die auf\/fällige Tatsache
hin, dass Voegelin niemals von einer eigenen Transzendenzerfahrung berichtet.
Immer nur wird auf Platon, Aristoteles, den heiligen Thomas von Aquin, die
Bibel und auf andere ehrwürdige Schriftstücke verwiesen, von denen Voegelin
versichert, dass sie Ausdruck von echten und unverfälschten
Transzendenzerfahrungen seien. Einen Bericht von eigenen
Transzendenzerfahrungen liefert Voegelin nicht.  Stattdessen müssen
Kindheitserinnerungen als Lückenbüßer herhalten.\footnote{Vgl. Anamnesis, S.
  62-76.} Doch dies wirkt eher wie ein verzweifelter Versuch Voegelins, die
Transzendenzerfahrungen, an deren Wirklichkeit er so gerne glauben wollte, in
irgendeiner Weise auch bei sich selbst vorzufinden. Ohne eigene
Transzendenzerfahrungen aber wird Voegelins Behauptung, dass sich die
Richtigkeit der politischen Philosophien, die er befürwortete, "`empirisch"'
überprüfen ließe,\footnote{Vgl. Voegelin, Neue Wissenschaft der Politik, S.
  96.} vollends unglaubwürdig.

Von den eben aufgeführten drei Deutungsmöglichkeiten erscheint -- auch wenn es
sich nicht mit Sicherheit sagen lässt -- die zweite Deutungsmöglichkeit als die
naheliegendste, da die erste Möglichkeit allzu unwahrscheinlich ist, und man
andererseits auch ungern glauben möchte, dass jemand den größten Teil seines
Lebenswerkes auf pure Einbildungen stützt. Von welcher Deutungsmöglichkeit man
aber auch ausgeht, in keinem Fall lassen sich die weitreichenden Konsequenzen
rechtfertigen, die Voegelin für die politische Ordnung und für die
Politikwissenschaft aus den Transzendenzerfahrungen ziehen möchte.


%%% Local Variables: 
%%% mode: latex
%%% TeX-master: "Main"
%%% End: 

















