
%%% Local Variables: 
%%% mode: latex
%%% TeX-master: "Main"
%%% End: 


\section{"`Zur Theorie des Bewußtseins"'}

\subsection{Voegelins Schrift "`Zur Theorie des Bewußtseins"'}

In seinem Aufsatz "`Zur Theorie des Bewußtseins"'\footnote{Voegelin,
  Anamnesis, S. 37-60.} legt sich Voegelin Rechenschaft über seinen eigenen
philosophischen Standpunkt ab. Einleitend erklärt Voegelin, dass seine
Aufzeichnungen die Ergebnisse anamnetischer Experimente enthielten, doch
bezieht sich dies wohl eher auf die auf diesen Aufsatz folgenden Berichte von
Kindheitserinnerungen. Der Aufsatz selbst zumindest besteht weit überwiegend
aus theoretischer Diskussion.

Es fällt nicht leicht, die Thematik dieses Aufsatzes zu beschreiben,
denn Voegelin reißt darin viele verschiedene Themen an. Den
Hauptthemenschwerpunkt bildet eine Beschreibung der Struktur des
Bewusstseins, seiner Beziehung zur Welt und eine Diskussion der
Möglichkeiten des Bewusstseins dasjenige, was außerhalb des Bewusstseins
liegt, zu erfahren und zu beschreiben. Daneben skizziert Voegelin
umrisshaft eine ontologische Theorie und schließlich versucht Voegelin,
nachdem er der Ontologie das Primat vor der Bewusstseinsphilosophie
eingeräumt hat, das Auftreten der von ihm für falsch oder zu einseitig
gehaltenen Bewusstseinsphilosophien historisch und wissenssoziologisch zu
erklären.

Den Ausgangpunkt für Voegelins Überlegungen bildet eine Kritik der Theorien
des Bewusstseins, die die Bewusstseinsstrommetapher in den Mittelpunkt ihrer
Beschreibung stellen. Voegelin hält dies für eine falsche Akzentuierung: Zwar
strömt das Bewusstsein auch, aber das Strömen ist weder der wesentliche noch
ein alle anderen Bewusstseinsleistungen bedingender Faktor im Bewusstsein. Vor
allem tritt das Strömen nur bei bestimmten Bewusstseinsvorgängen zu Tage wie
etwa beim Hören von Tönen. Bei anderen Bewusstseinstätigkeiten -- Voegelin
beschreibt als Beispiel die inneren Vorgänge beim Betrachten eines Gemäldes --
lässt sich das Strömen des Bewusstseins nur erfassen, wenn die Aufmerksamkeit
von der Hauptsache abgelenkt wird. Voegelin betrachtet es daher als eine
Spekulation, wenn der Bewusstseinsstrom als die Grundform aller
Bewusstseinsvorgänge aufgefasst wird.\footnote{Vgl. Voegelin, Anamnesis,
  S. 37-43.}

Nach Voegelins eigener Vorstellung vom Aufbau des Bewusstseins ist das
Bewusstsein nicht durch zeitliches Fließen sondern thematisch durch
Aufmerksamkeitszuwendung und -abwendung strukturiert. Voegelin nimmt an, dass
es im Bewusstsein ein Aufmerksamkeitsquantum von nicht genau bestimmter aber
beschränkter Größe gibt, welches verschiedenen Bereichen des Bewusstseins in
mehr oder weniger starker Konzentration zugewandt werden kann. Im Bewusstsein
gibt es nun zwei besonders ausgezeichnete Bereiche, Voegelin nennt sie
"`Erhellungsdimensionen"', denen bestimmte Formen der
Aufmerksamkeitszuwendung, "`Erinnerung"' und "`Projektion"', entsprechen.
Diese beiden Erhellungsdimensionen bezeichnet Voegelin daher passenderweise
als "`Vergangenheit"' und "`Zukunft"'. Aus dem Zusammenspiel von
Aufmerksamkeitszuwendung und den Erhellungsdimensionen "`Vergangenheit"' und
"`Zukunft"' leitet sich die Vorstellung der (inneren wie äußeren) Zeit ab.
Voegelin vertritt also, wie es scheint, eine idealistische Auffassung der
Zeit, die derjenigen nicht unähnlich ist, die Augustinus als erster in den
"`Bekenntnissen"' dargelegt hat.\footnote{Aurelius Augustinus: Bekenntnisse,
  Stuttgart 1998, S. 312-330 (Elftes Buch. XIII.15 - XXVIII.38). Man könnte
  geneigt sein, gegen die Logik dieser Art von Zeittheorien einzuwenden, dass
  die Vorgänge innerhalb des Bewusstseins, aus denen die Zeit hervorgeht, doch
  schon die Zeit als solche voraussetzen. Aber in der Tat setzen sie höchstens
  bestimmte (zeitliche) Relationen voraus.  (Voegelin scheint diese Kritik
  jedoch ernst zu nehmen, denn er bringt sie etwas später als Selbsteinwand
  vor; vgl. Voegelin, Anamnesis, S. 54/55.) Die idealistischen Zeittheorien
  scheitern aus einem anderen Grund: Wenn die Zeit ausschließlich eine Form
  oder Leistung des Bewusstseins ist, so ist nicht erklärlich, wie die
  Kommunikation zwischen den "`Bewusstseinen"' verschiedener Menschen zeitlich
  aufeinander abgestimmt erfolgen kann, da die Botschaften des einen an das
  andere Bewusstsein doch durch eine äußere Welt hindurch müssen, in der die
  Zeitrelationen, die der idealistischen Annahme zufolge reine
  Bewusstseinsprodukte sind, verloren gehen müssten. (Das Argument stammt aus:
  Russell, History of Western Philosophy, S. 689, wo es im Zusammenhang mit der
  Diskussion der idealistischen Raum- und Zeittheorie Kants auftaucht. Es
  lässt sich aber unmittelbar auch auf andere idealistische Zeittheorien
  übertragen.)} Da das Bewusstsein insgesamt endlich ist, so ist auch die
Zeitvorstellung aus einem endlichen Vorgang abgeleitet.  Dieser Vorgang ist
der einzige wirkliche Prozess, von dem wir eine innere Erfahrung haben.
Voegelin behauptet, dass dadurch der Bewusstseinsprozess "`zum Modell des
Prozesses überhaupt"'\footnote{Voegelin, Anamnesis, S. 44.}  wird, und dass
all unsere Begriffe von Prozessen nur von diesem einzigen uns zur Verfügung
stehenden Modell abgezogen sind.\footnote{Nur wenige Seiten weiter behauptet
  Voegelin sonderbarerweise genau das Gegenteil: "`...  die {\it Ordnung} des
  Augenblicksbildes in der Dimension, die durch die Erhellung geschaffen wird,
  zur Sukzession eines Prozesses erfordert Erfahrungen von
  bewußtseinstranszendenten Prozessen."'  (Voegelin, Anamnesis, S. 55.)}
Infolge der Endlichkeit dieses Modells entstehen Ausdruckskonflikte bei der
Beschreibung unendlicher Prozesse, wie sie außerhalb des Bewusstseins
gelegentlich vorkommen können. Von diesen Ausdruckskonflikten rühren nach
Voegelins Überzeugung auch die Kantischen Antinomien und die Paradoxe der
Mengenlehre her.\footnote{Vgl.  Voegelin, Anamnesis, S. 44/45. -- Welche
  Paradoxe der Mengenlehre Voegelin meint, geht aus dem Text leider nicht
  hervor.  Wahrscheinlich denkt Voegelin dabei an die Russellsche Antinomie,
  die eine Variante des klassischen Lügnerparadoxons ist und auftritt, wenn
  man versucht die Menge aller Mengen zu bilden, die sich nicht selbst als
  Element enthalten.  (Vgl. auch die Kritik dieser Passage im folgenden
  Abschnitt.)}  Um unendliche Prozesse, die zwar erfahren bzw. erahnt aber
nicht widerspruchsfrei beschrieben werden können, überhaupt in irgendeiner
Weise zu artikulieren, ist es erforderlich, sich der geheimnisvollen
Ausdrucksweise der Mythensymbolik zu bedienen. Den Begriff "`Mythensymbol"'
definiert Voegelin als ein "`finites Symbol, das für einen transfiniten
Prozess `transparent' sein soll."'\footnote{Voegelin, Anamnesis, S. 45.} Eine
genauere Eingrenzung, welches die unendlichen Prozesse sind, die des
Ausdruckes durch die Mythensymbolik bedürfen, gibt Voegelin nicht an.  Seine
Beispiele legen nahe, dass es sich dabei um die Gegenstände handelt, die in
der Religion zum welttranszendenten Bereich gerechnet werden.  Als Beispiele
dieses Gebrauchs der Mythensymbolik führt Voegelin nämlich einige allegorische
Deutungen bekannter Mythensymbole an. So vermittelt etwa die unbefleckte
Empfängnis "`die Erfahrung eines transfiniten geistigen
Anfangs"'\footnote{Voegelin, Anamnesis, S. 45.}.  Voegelin versäumt es leider
zu klären, wie ein transfiniter geistiger Anfang Gegenstand der Erfahrung
werden kann, und ob es jemals einen Menschen gegeben hat, der etwas derartiges
tatsächlich erfahren hat.

Ein größeres Problem im Zusammenhang mit der Mythensymbolik stellt die Frage
der Adäquatheit des Mythos dar. Damit meint Voegelin die Frage, ob ein Mythos
überhaupt eine Erfahrung ausdrückt, und ob er, wenn er es tut, die Erfahrung
richtig zum Ausdruck bringt. Voegelin erläutert dieses Problem am Beispiel
zweier Mythen bei Platon, dem von Platon bewusst als reines
Propagandainstrument konzipierten Mythos der drei Metalle, die in Platons
Dialog {\em Politeia} den drei Sozialklassen zugeordnet werden, und dem Mythos
aus den {\em Nomoi}, in dem die Menschen von den Göttern als Marionetten an
metallenen Fäden gelenkt werden.\footnote{Vgl. Voegelin, Anamnesis, S. 46/47.}
Die beiden einzigen Kriterien, die Voegelin anführt, um den richtigen von dem
von Platon bewusst konstruierten Mythos unabhängig von Platons eigener
Mitteilung zu unterscheiden, sind: 1) Die Übereinstimmung mit Voegelins eigenen
metaphysischen Überzeugungen, denn für Voegelin "`finitisiert"' der zweite
Mythos "`im Marionettensymbol `adäquat' die Erfahrung von der Handlung im
Schnittpunkt der Determinanten, die wir respektive `Ich' und `weltjenseitiges
Sein' nennen"',\footnote{Voegelin, Anamnesis, S. 46. -- An dieser Stelle wird
  nebenbei bemerkt deutlich, wie sehr Voegelin den Erfahrungsbegriff
  strapaziert. Denn, dass das was wir als teilweise Determiniertheit unser
  Handlungen erfahren mögen, Folge der Einwirkungen eines weltjenseitigen
  Seins ist, wird als solches eben nicht erfahren, sondern stellt bestenfalls
  eine metaphysische Deutung dieser Erfahrung dar. Das `weltjenseitiges Sein'
  hier nur eine (ontologische neutrale) Benennung sein soll, wirkt kaum
  glaubwürdig, weil diese Benennung eine {\em weltjenseitige} Determinante in
  einem Zusammenhang suggeriert, in dem es viel plausibler wäre, zunächst die
  Möglichkeit {\em weltdiesseitiger} Determinanten in Erwägung zu ziehen. Der
  Trick der suggestiven Benennung von Phänomenen ist übrigens einer, den
  Voegelin der Heideggerschen Variante der Phänomenologie abgeschaut haben
  könnte.} und 2) der emotionale Eindruck, indem der richtige Mythos "`die
`Schauer' der Transzendenz, das `Numinose' im Sinne Rudolf Ottos
erregt."'\footnote{Voegelin, Anamnesis, S.  46.} Beides sind, wie man unschwer
erkennt, höchst subjektive Kriterien.

Im Zusammenhang mit der Mythensymbolik kommt Voegelin auch auf das Husserlsche
Problem der "`Konstitution der Intersubjektivität"' zu sprechen.\footnote{Vgl.
  Husserl, Cartesianische Meditationen, S. 91ff.  (§42ff.).} Dahinter verbirgt
sich die Frage, woraus hervorgeht, dass die Menschen außerhalb des eigenen
Bewusstseins eigene Wesen mit einem eigenen Bewusstsein sind. Im
philosophischen Gedankenexperiment ist es möglich, sich vorzustellen, dass die
anderen Menschen -- so wie Gestalten im Traume -- nur Hirngespinste des eigenen
Bewusstseins sind, oder dass sie "`Zombies"' sind, die körperlich und von
ihrem Verhalten her Menschen gleichen, in deren Gehirnen jedoch kein
Bewusstsein lebt. Edmund Husserl hat zu zeigen versucht, dass das Ich
bestimmte Objekte des Erfahrungsfeldes als "`alter ego"' konstituieren
kann.\footnote{Vgl.  Husserl, Cartesianische Meditationen, S. 112-116 (§
  50/51).} Voegelin hält dies für ein reines Verwirrspiel. Seiner Ansicht nach
existiert dieses philosophische Problem gar nicht, sondern es ist ein Faktum,
dass das Bewusstsein die anderen Menschen als
"`Nebenbewußtsein"'\footnote{Voegelin, Anamnesis, S. 47.} erfährt. Übrig
bleibt nur ein eher moralphilosophisches Problem, welches darin besteht,
dieses "`Erfahrungsfaktum"'\footnote{Voegelin, Anamnesis, S. 47.} so zum
Ausdruck zu bringen, dass die Mitmenschen als gleichartig anerkannt werden
können. Zur Behandlung dieses Problems greift Voegelin auf die
Mythengeschichte zurück. Es ist nicht ganz klar, warum Voegelin es für
notwendig erachtet, die Lösung im Rahmen der Mythensymbolik zu suchen.  Bei
den anderen Menschen handelt es sich schließlich auch nur um endliche Wesen,
weshalb der zuvor beschriebene Ausdruckskonflikt nicht zum Tragen kommt. Zudem
sind die anderen Menschen nicht welttranszendent, sondern nur in dem trivialen
Sinne bewusstseinstranszendent, in dem auch Tiere und tote Gegenstände
bewusstseinstranszendent sind, weil sie außerhalb und unabhängig vom
Bewusstsein ihres Betrachters eine Eigenexistenz haben.  Um solche Feinheiten
kümmert sich Voegelin jedoch nicht weiter. Der Mythengeschichte meint Voegelin
nun entnehmen zu können, dass alle bisherigen Gleichheitsideen historisch auf
die beiden Mythen der Abstammung aller Menschen von einer Mutter oder der
geistigen Prägung durch ein und denselben Vater (Gottesebenbildlichkeit)
zurückgeführt werden können.\footnote{Vgl.  Voegelin, Anamnesis, S. 47/48.}
Einen möglichen nicht-mythischen Ursprung bestimmter Gleichheitsideen zieht
Voegelin gar nicht erst in Erwägung. Ja er versteigt sich sogar zu der kühnen
Behauptung, dass die erkenntnistheoretischen Probleme der Intersubjektivität
nur innerhalb dieses mythischen Rahmens behandelbar sind.\footnote{Vgl.
  Voegelin, Anamnesis, S.  48.}  Seine Ausführungen zum mythengeschichtlichen
Ursprung der Gleichheitsidee nimmt Voegelin zum Anlass für einen kleinen
Exkurs über einige mythengeschichtliche Einzelprobleme des Konflikts zwischen
Gleichheits- und Gemeinschaftsmythen,\footnote{Vgl.  Voegelin, Anamnesis, S.
  48-50.} der schließlich in einer Klage über den Verlust des Mythos als
Ausdrucksmittel für Transzendenzerfahrungen in der Gegenwart mündet: "`Das
unvermeidliche Ergebnis"', so Voegelin, "`ist das Phänomen der `Verlorenheit'
in einer Welt, die keine Ordnungspunkte mehr im Mythos
hat."'\footnote{Voegelin, Anamnesis, S. 50.} Etwas unvermittelt leitet
Voegelin daraus eine Erklärung für den letzten Weltkrieg ab: "`Die
gesellschaftsdynamisch wichtigsten Symptome sind die `Bewegungen' unserer Zeit
... und die 'großen Kriege': die Kriege nicht nur, wo sie vielleicht ein
positives Wollen zur orgiastischen Entladung verraten, sondern auch dort, wo
sie hingenommen werden müssen, weil die Handlungen, die sie verhindern
könnten, durch die Paralyse des Ordnungswillens, der nur aktiv sein kann, wo
er seinen Sinn in der Ordnung des Gemeinschaftsmythos hat, unmöglich gemacht
werden."'\footnote{Voegelin, Anamnesis, S. 50.} Nun ist aber die Berufung auf
den "`Gemeinschaftsmythos"' ein charakteristisches Merkmal gerade der
faschistischen Ideologien. Wenn Voegelin mit der "`Paralyse des
Ordnungswillens"' auf die Appeasement-Politik Chamberlains gegenüber Hitler
anspielen wollte, dann laufen seine Ausführungen auf den Vorwurf an die
westlichen Demokratien hinaus, sich nicht ihrerseits faschistischer Methoden
bedient zu haben.

Um angemessen über Dinge und Zusammenhänge reden zu können, die mehr sind als
bloß Gegenstände endlicher, innerweltlicher Erfahrung, existiert für Voegelin
neben der Mythensymbolik noch eine philosophisch-begriffliche Alternative in
Form der Prozesstheologie. Sie beschreibt "`die Beziehungen zwischen dem
Bewusstsein, den bewusstseinstranszendenten innerweltlichen Seinsklassen und
dem welttranszendenten Seinsgrund"'\footnote{Voegelin, Anamnesis, S. 50.}.
Dieser Aufgabe ist die Prozesstheologie im Gegensatz zu anderen Ansätzen
innerhalb der Metaphysik deshalb gewachsen, "`weil in ihr zumindest der
Versuch gemacht wird, die bewusstseinstranszendente Weltordnung in einer
`verstehbaren' Sprache zu interpretieren"', nämlich in einer Sprache, die an
"`der einzig `von innen' zugänglichen Erfahrung des
Bewusstseinsprozesses"'\footnote{Voegelin, Anamnesis, S. 51.} orientiert ist.
Wie dies mit der vorherigen Behauptung zu vereinbaren ist, dass gerade dieses
Modell zum Ausdruck der Erfahrung transfiniter Wirklichkeit eher untauglich
sei,\footnote{Vgl. Voegelin, Anamnesis, S.  44/45.} enthüllt Voegelin nicht.
Wahrscheinlich muss man sich die Prozesstheologie als der Mythologie verwandt
vorstellen. Die Prozesstheologie stützt sich auf zwei "`Erfahrungskomplexe"':
Zum einen stützt sie sich auf die "`Erfahrung"', dass die Welt aus mehreren
wesensverschiedenen aber dennoch voneinander abhängigen Seinsstufen aufgebaut
ist, und zum anderen basiert sie auf der in der Meditation zugänglichen
Erfahrung des "`welttranszendenten Seinsgrundes"'.  Werden diese beiden
Erfahrungen kombiniert, so ergibt sich aus der Erfahrung der Abhängigkeit der
Seinsstufen voneinander die "`Nötigung"', sie als Phasen eines Prozesses der
Entfaltung einer identischen Substanz zu betrachten, welcher -- hier kommt die
meditative Erfahrung ins Spiel -- im welttranszendenten Seinsgrund seinen
Ursprung hat. Da die Prozesstheologie unmittelbar auf "`ontologischen
Erfahrungen"' beruht, entzieht sie sich, wie Voegelin glaubt, auch den
ansonsten naheliegenden erkenntnistheoretischen Einwänden Kantischer
Provenienz, wonach es unzulässig ist, Kategorien der innerweltlichen Erfahrung
auf das anzuwenden, was außerhalb aller möglichen Erfahrung
liegt.\footnote{Vgl. Voegelin, Anamnesis, S. 50-54.}

Auf der Grundlage dieser ontologischen Stufentheorie vollzieht Voegelin nun
den Übergang vom Primat der Bewusstseinsphilosophie zum Primat der
Ontologie.\footnote{Dieser Übergang ist übrigens durchaus typisch. In
  ähnlicher Weise ging auch Heidegger, ausgehend von Husserls
  phänomenologischer Bewusstseinsphilosophie, zur Ontologie über. Etwas vom
  Heideggerschen Pathos lässt sich bei Voegelin ebenfalls verspüren, wenn er
  vor dem möglichen Missverständnis warnt, man sei wieder in den "`friedlichen
  Gewässern der Erkenntnistheorie"' (Voegelin, Anamnesis, S. 56.) angelangt.}
Zunächst geht Voegelin jedoch zum Ausgangspunkt seiner
bewusstseinsphilosophischen Überlegungen zurück. Wenn das Bewusstsein durch die
"`Erhellungsdimensionen"' der "`Vergangenheit"' und "`Zukunft"' strukturiert
ist und Zeit als solche dem Bewusstsein nicht unmittelbar gegeben ist, so kann
bezweifelt werden, dass zwischen diesen Erhellungsdimensionen die zeitliche
Beziehung der Sukzession besteht. Das Element der Zeitlichkeit lässt sich aus
diesen Erhellungsdimensionen deshalb nicht ableiten,\footnote{Zuvor scheint
  Voegelin jedoch gerade dies versucht zu haben. (Vgl. Voegelin, Anamnesis,
  S. 44.)} weil es auch vorstellbar ist, dass Erinnerungen und Projektionen nur
Phantasien eines im Augenblickspunkt der Gegenwart verharrenden Bewusstseins
sind. Wie kann man aber einem solchen "`Solipsismus des
Augenblickes"'\footnote{Voegelin, Anamnesis, S. 55.} entgehen? Der einzige
Ausweg besteht für Voegelin in der "`Einsicht, dass das menschliche Bewusstsein
nicht eine Monade ist, welche die Existenzform des Augenblicksbildes hat,
sondern dass es menschliches Bewusstsein ist, d.h. Bewusstsein im Fundament des
Leibes und der Außenwelt."'\footnote{Voegelin, Anamnesis, S. 55.} Voegelin
spricht hier zwar von einer "`Einsicht"', aber diese Einsicht hat eher den
Charakter eines Postulates, denn nachdem Voegelin einmal beim Solipsismus des
Augenblickes angelangt ist, gibt es nichts mehr, woraus das Sein der Zeit und
der Welt mit Gewissheit oder auch nur Wahrscheinlichkeit erschlossen werden
könnte.

Unter dem Gesichtspunkt dieser ontologischen Einsicht ist auch der Begriff des
Bewusstseinsprozesses neu zu deuten. Damit wir die Bewusstseinsvorgänge als
zeitlichen Prozess auffassen können, sind "`Erfahrungen von
bewusstseinstranszendenten Prozessen"'\footnote{Voegelin, Anamnesis, S. 55.}
erforderlich. Wie dies möglich ist, wenn -- wie Voegelin zuvor kategorisch
behauptet hat -- der innere Bewusstseinsprozess seinerseits das einzige Modell
darstellt, mit dem wir bewusstseinstranszendente Prozesse verstehen
können,\footnote{Vgl. Voegelin, Anamnesis, S. 44.} bleibt etwas im Dunkeln.
Voegelin scheint von einer Art wechselseitiger Abhängigkeit zwischen Sein und
Bewusstsein auszugehen, wenn er im folgenden einerseits die physische
Bedingtheit des Bewusstseins betont, zugleich aber der idealistischen Ansicht
Raum gibt, dass das Sein der Dinge von der Beziehung auf ein Bewusstsein
abhängig ist. Es lässt sich nicht leicht feststellen, auf welche Weise
Voegelin bei dieser Argumentation einem Zirkelschluss entgehen will.
Wahrscheinlich zu Recht weist Voegelin jedenfalls darauf hin, dass daraus,
dass das Bewusstsein uns in innerer Erfahrung nur als reines Bewusstsein
gegeben ist, nicht folgt, dass es nichts anderes als reines Bewusstsein ist.
Vielmehr liefert nach Voegelins Überzeugung die innere Erfahrung nur eine
Teilansicht eines untrennbaren materiell-geistigen Seinskomplexes. In der
inneren wie der äußeren Erfahrung bekommt der Mensch jeweils nur den äußersten
Zipfel eines Seins zu fassen, das sich weit über das in der Erfahrung Gegebene
hinaus erstreckt.\footnote{Vgl.  Voegelin, Anamnesis, S. 55-57.}

Aus all diesen Überlegungen zieht Voegelin die Schlussfolgerung, dass die
Bewusstseinsphilosophie keinen geeigneten Anfangspunkt der Philosophie
darstellt. Das Bewusstsein setzt vielmehr das Sein voraus und die Frage des
Anfangs kann nun immer weiter zurückgeschoben werden bis hin zur Frage des
Anfangs der Geschichte des Kosmos. Offenbar trennt Voegelin nicht zwischen der
Frage des erkenntnistheoretischen Ausgangspunktes und der Frage der
historischen Seinsvoraussetzungen des Erkenntnisvermögens. Das klassische 
Problem eines absoluten Anfangs der Philosophie wird Voegelin noch in "`Order
and History V"' beschäftigen.\footnote{Vgl. Voegelin, Order and History V,
  S. 13f.}  Vorerst gelangt Voegelin zu dem Ergebnis, dass das Bewusstsein auf
Grund dieser nie vollständig aufklärbaren Anfangsvoraussetzungen nicht wie
äußere Gegenstände erkannt und beschrieben werden kann, sondern dass es
lediglich durch Besinnung sich selbst und sein eigenes Sein erhellen
kann.\footnote{Vgl.  Voegelin, Anamnesis, S. 57/58.}

Nachdem Voegelin solcherart die "`Kehre"' zur Ontologie vollzogen hat,
behandelt er als letztes Thema dieses Aufsatzes die wissenssoziologische
Frage, wie es zu dem Auftreten der seiner Ansicht nach verfehlten
Bewusstseinsphilosophien kommen konnte. Voegelin liefert eine solche Erklärung
an zwei Stellen seines Aufsatzes. Die erste Erklärung bezieht sich auf den
speziellen Fall der Bewusstseinsstromtheorien, die zweite Erklärung betrifft
die Bewusstseinsphilosophie im Allgemeinen.

In den Bewusstseinsstromtheorien glaubt Voegelin ein "`laizistisches Residuum
der christlichen Existenzvergewisserung in der Meditation"'\footnote{Voegelin,
  Anamnesis, S. 37.} wiederentdecken zu können. Vermutlich auf Grund von
einfühlendem Nachvollzug gelangt Voegelin zu der Auf\/fassung, dass im
Bewusstseinserlebnis des "`Strömens"' der "`Engpaß des Leibes spürbar
wird"'.\footnote{Voegelin, Anamnesis, S.  40. Außer seinen eigenen
  Assoziationen, für die sich in den zeitphilosophischen Texten, auf die
  Voegelin sich bezieht, durchaus einzelne Hinweise finden lassen, führt
  Voegelin noch einen eher aus dem Zusammenhang gegriffenen Gedanken von
  William James (Vgl. William James: Essays in Radikal Empirischem, Anbringe,
  Kassakurses / London, England 1976, S. 19.) und etwas später (Anamnesis,
  S. 42) Bergsons Behandlung der eleatischen Paradoxe an. Bergson ist jedoch
  ein schlechter Gewährsmann, denn seine Behandlung der eleatischen Paradoxe
  scheint auf einer Verwechselung der physikalischen Begriffe von Ort und
  Bewegung zu beruhen: Dass ein Körper sich zu einem bestimmten Zeitpunkt an
  einem ganz bestimmten Punkt im Raum befindet schließt nämlich nicht aus,
  dass er in diesem Punkt einen Bewegungszustand hat.  (Vgl. Henri Bergson:
  Materie und Gedächtnis, Hamburg 1991, S. 184-190.)}  Voegelin schließt
daraus, dass die Bewusstseinsstromtheorien ebenso wie die christliche
Meditation auf eine Form der Transzendenz zielen.  Während die Meditierenden
in der christlichen Meditation jedoch Welttranszendenz suchen, zielen die
Bewusstseinsstromtheorien lediglich auf die bloße Bewusstseinstranszendenz in
Richtung der Leibsphäre hin.\footnote{Vgl. Voegelin, Anamnesis, S. 41-42.}

In einem etwas allgemeineren Rahmen stellt das Auftreten der
Bewusstseinsphilosophie für Voegelin die Reaktion auf eine Krise der Symbole
dar, wie sie alle Kulturen von Zeit zu Zeit heimsucht. Die Symbole, mit denen
die Menschen ihre Transzendenzerfahrungen ausdrücken, tendieren dazu, im Laufe
der Zeit schal und inhaltsleer zu werden. Die daraus resultierende Kulturkrise
kann nur durch die Beseitigung der alten und die Bildung neuer Symbole zum
Ausdruck der Transzendenzerfahrungen behoben werden. Platon war dies als
Antwort auf die Krisis der hellenischen Kultur im 5.Jahrhundert vor Christus
in vorbildlicher Weise gelungen. Die neuzeitliche Philosophie, die mit
Descartes ihren Anfang nimmt, stand nach Voegelins Ansicht vor einer ähnlichen
Aufgabe, doch hat sie ihr Ziel verfehlt, indem sie zwar mit der Tradition
gründlich aufräumte aber zugleich auch die Transzendenzerfahrungen aus dem
Themenkanon der Philosophie ausschloss.\footnote{Vgl. Voegelin, Anamnesis,
  S. 58-60.}

\subsection{Kritik von Voegelins Theorie des Bewusstseins}

Sind Voegelins Überlegungen "`Zur Theorie des Bewußtseins"' überzeugend? Geben
sie die Beziehungen zwischen Sein und Bewusstsein richtig wieder und dürfen
Voegelins Argumente als stichhaltig angesehen werden? Da es kaum möglich ist,
auf alle Einzelheiten der sehr vielfältigen Ausführungen Voegelins
einzugehen, sollen zur genaueren kritischen Untersuchung nur einige Punkte
herausgegriffen werden, die für Voegelins Argumentation von wesentlicher
Bedeutung sind.

Innerhalb von Voegelins eigener Darstellung der Bewusstseinsstruktur findet
sich an zentraler Stelle das Argument, dass sich das Bewusstsein bei dem
Versuch, transfinite Prozesse deskriptiv zu beschreiben, auf Grund seiner
eigenen Endlichkeit unvermeidlich in Widersprüche verwickelt. Das Argument
spielt deshalb eine wesentliche Rolle, weil diese Widersprüche es erforderlich
werden lassen, zur Deutung bestimmter Wirklichkeitsbereiche auf die zwar
subtilen und seelisch sensiblen aber an Klarheit und Objektivität hinter einer
deskriptiven Beschreibung zurückstehenden Instrumente der Mythensymbolik und
der Prozesstheologie zurückzugreifen. Unglücklicherweise steckt gerade in
dieser Passage von Voegelins Darstellung eine Reihe von Fragwürdigkeiten, die
sich nicht ohne weiteres auflösen lassen.

Zunächst einmal ist es zweifelhaft, ob, wie Voegelin es behauptet, der
Bewusstseinsprozess das einzige erfahrene Modell eines Prozesses darstellt.
Prozesse oder, mit anderen Worten, zeitlich ablaufende Vorgänge im weitesten
Sinne erleben wir tagtäglich in der äußeren Erfahrung, z.B. wenn wir ein
fahrendes Auto beobachten. Die äußere Erfahrung eignet sich dabei mindestens
ebenso gut, wenn nicht besser, als die innere Erfahrung, um den Begriff eines
Prozesses zu bilden.  Abgesehen davon ist es aber auch überhaupt nicht
erforderlich, zur Bildung eines Begriffes diesen von irgendeiner Erfahrung
abzuziehen.  Ebenso wie ein großer Teil unseres Wissens nicht aus
unmittelbarer Erfahrung stammt, gibt es auch viele Begriffe, die rein abstrakt
sind.  Alle mathematischen Begriffe gehören zu dieser Klasse. Insbesondere ist
es ohne Probleme möglich, widerspruchsfreie Begriffe von Unendlichkeit zu
bilden. Die Mengenlehre verfügt über mehrere solcher Begriffe.  Freilich
decken diese Begriffe nicht alle Wortbedeutungen von "`unendlich"' ab, und die
"`unendliche Sehnsucht"', von der ein romantischer Dichter schwärmen mag, wird
von der Mengenlehre nicht erfasst, aber es ist nun nicht mehr einleuchtend,
weshalb die Finitheit des Bewusstseinsprozesses bei der Beschreibung von
unendlichen Prozessen zu den von Voegelin unterstellten
Ausdruckskonflikten führen muss. Außerdem scheint sich Voegelin auch
hinsichtlich der Bedeutung der Kantischen Antinomien geirrt zu haben.  Kants
Antinomien beruhen letztlich auf unterschiedlichen Voraussetzungen, die den
einander gegenübergestellten Beweisen und Gegenbeweisen zu Grunde liegen.  Um
Antinomien könnte es sich nur noch dann handeln, wenn diese Voraussetzungen
gleichermaßen notwendig wären.  Aber dies -- und hierin irrt Kant und mit ihm
viele seiner Interpreten und, wie es scheint, leider auch Voegelin -- ist nicht
der Fall.\footnote{Vgl. Immanuel Kant: Kritik der reinen Vernunft, Hamburg
  1976, S. 454-469. Für die Kant-Apologetik stellvertretend: Peter Baumanns:
  Kants Philosophie der Erkenntnis. Durchgehender Kommentar zu den
  Hauptkapiteln der "`Kritik der reinen Vernunft"', Würzburg 1997, S. 742ff.--
  Dass Kant irrt, kann man sich leicht überlegen, wenn man bei den Antinomien
  genau darauf achtet, von welchen expliziten und impliziten Voraussetzungen
  Kant bei seinen Beweisen jeweils ausgeht.  Es würde zu weit führen, dies
  hier im einzelnen auszuführen.}  Was Voegelin schließlich mit den Paradoxen
der Mengenlehre meint, geht aus dem Text leider nicht hervor.  Möglicherweise
meint Voegelin die Russellsche Antinomie, die in der naiven Mengenlehre
auftritt.  Aber erstens handelt es sich nicht um ein Paradox der
Unendlichkeit, und zweitens lässt sie sich mühelos durch eine axiomatische
Fassung der Mengenlehre beseitigen.\footnote{Vgl. Jürgen Schmidt: Mengenlehre
  (Einführung in die axiomatische Mengenlehre). I.  Grundbegriffe, Mannheim
  1966, S. 22-24.}

Voegelins Argument ließe sich im Grundsätzlichen immer noch dann
rechtfertigen, wenn es gelänge zu zeigen, dass bestimmte Wirklichkeitsbereiche
aus anderen Gründen als dem ihrer "`Infinitheit"' einer deskriptiven
Beschreibung unzugänglich sind. Dann müsste die Diskussion um die Fragen
geführt werden, ob es diese Wirklichkeitsbereiche tatsächlich gibt, und wenn
es sie gibt, ob Mythensymbolik oder Prozesstheologie sie erfassen können. So
wie Voegelin argumentiert, bleibt die Notwendigkeit des Gebrauchs dieser
Symbolformen jedoch unbegründet.

Einen weiteren wichtigen Abschnitt, der zwar weniger für Voegelins folgende
Argumentation von Bedeutung ist, aber dafür seine grundsätzliche
philosophische Einstellung widerspiegelt, bildet Voegelins Versuch, das
Problem der Anerkennung der Mitmenschen als gleichartige und gleichwertige
Wesen (in Voegelins Terminologie: das Problem der "`Erfahrung vom
Nebenmenschen"') mit Hilfe der Mythengeschichte zu lösen. Voegelins
Argumentation enthält eine Reihe von Schwachpunkten. Die erste Schwierigkeit
bildet der Begriff des "`Erfahrungsfaktums"'. Obwohl wir, wie auch Voegelin
einräumt, von unseren Mitmenschen keine innere Erfahrung haben, sollen wir
dennoch durch ein Erfahrungsfaktum unmittelbar davon in Kenntnis gesetzt sein,
dass sie ein Innenleben haben. Nun mögen wir zwar intuitiv den Eindruck haben,
dass in unseren Mitmenschen auch ein denkendes und fühlendes Bewusstsein
steckt, aber die Berufung auf die Intuition ist auch dann noch ein schwaches
Argument in der Erkenntnistheorie, wenn sie hochtönend als
"`Fundamentalcharakter"' der "`Transzendenzfähigkeit"'\footnote{Voegelin,
  Anamnesis, S. 47.} des Bewusstseins etikettiert wird. Der Einwand gegen
Husserl, dass sich das Du nicht im Ich konstituiert, ist dagegen durchaus
angebracht, denn das Bewusstsein kann unmöglich durch Konstitution etwas
hervorbringen, was außerhalb seiner selbst existiert. Als geradezu abwegig
erscheint allerdings Voegelins Behauptung, dass dieses erkenntnistheoretische
Problem nur im Rahmen der altertümlichen Gleichheitsmythen behandelt werden
kann. Weder für die Formulierung dieses erkenntnistheoretischen Problems noch
erst recht zu seiner Lösung ist der Rückgriff auf die Mythengeschichte
notwendig oder auch nur hilfreich.

Der zweite Schwachpunkt von Voegelins Argumentation liegt in seiner Annahme,
dass die moralische Gleichheit aller Menschen nur im Rückgriff auf alte Mythen
artikuliert werden kann. Bei der Behandlung dieser moralphilosophischen
Problematik müssen drei unterschiedliche Ebenen klar voneinander getrennt
werden: Die Ebene der Begründung von Werten, die Ebene der Artikulation bzw.
Formulierung der Werte und die Ebene der Vermittlung und Verbreitung der
Werte. Für die Begründung des Gleichheitswertes kann die Mythengeschichte
offensichtlich nicht herangezogen werden. Wenn die moralische Gleichheit der
Menschen nämlich im Sinne einer moralischen Intuition auf einem
"`Erfahrungsfaktum"' beruht,\footnote{Im Bereich der Ethik ist anders als in
  der Erkenntnistheorie die Berufung auf die Intuition unter Umständen
  legitim. Es stellt sich dann nur die Frage, inwieweit intuitiv begründete
  Werte intersubjektive Verbindlichkeit beanspruchen dürfen.} dann besteht die
einzig ehrliche Weise, diesen Wert zu begründen, darin, auf diese Intuition
bzw.  diese Erfahrung hinzuweisen, und gegebenenfalls die Begleitumstände
 zu beschreiben, unter denen sie zustande kommt oder in besonders
deutlicher Weise hervortritt.

Die Formulierung des Gleichheitswertes ist mit und ohne Rückgriff auf Mythen
möglich. Ohne Rückgriff auf die Mythologie kann sie beispielsweise durch die
Worte erfolgen: "`Alle Menschen sind gleich"'.  Bereits mit diesen schlichten
Worten ist der Inhalt der Gleichheitsidee vollständig und ohne jede Mythologie
ausgedrückt. Eine Artikulation unter Rückgriff auf die Mythologie könnte durch
Erzählung der Geschichte von Adam und Eva erfolgen. Allerdings bliebe, wegen
der grundsätzlichen Vieldeutigkeit des Mythos, die Gleichheitsbotschaft dann
möglicherweise undeutlich. Dass man mit dem Hinweis auf den Mythos von Adam
und Eva ebensogut die Ungleichheit begründen kann, führt uns z.B. Sir Robert
Filmer vor Augen, der damit auf eine zu seiner Zeit durchaus übliche Weise das
Gottesgnadentum der Könige rechtfertigte.\footnote{Vgl. Sir Robert Filmer:
  Patriarcha, or the Natural Power of Kings, England 1680.} Analoges gilt für
das Problem der Vermittlung des Gleichheitswertes. Zu der Zeit, als Voegelin
den Aufsatz "`Zur Theorie des Bewußtseins"' niederschrieb, war das
Mythologische einigermaßen in Mode. Nicht zuletzt durch die faschistischen
Bewegungen wurde die Berufung auf den Mythos weidlich missbraucht, weshalb es
aufgeklärten Autoren wie Thomas Mann notwendig erscheinen mochte, dass man den
totalitären Mythen aufgeklärte Mythen entgegen stellen müsse, um das
Verständnis für die Urwahrheiten des menschlichen Zusammenlebens
wiederzuerwecken.\footnote{Den zeitgeschichtlichen Bezug seiner Josephs-Romane
  hat Thomas Mann in einer späteren Selbstdeutung auf die Formel gebracht,
  dass der "`Mythos .. in diesem Buch dem Fascismus aus den Händen genommen"'
  wurde.  (Thomas Mann: Joseph und Seine Brüder (Vortrag in der Library of
  Congress am 17.11.1942), in: Thomas Mann: Essays. Band 5: Deutschland und
  die Deutschen 1938-1945.  (Hrsg.  v. Hermann Kurzke und Stephan Stachorski),
  Frankfurt am Main 1996, S. 185-200 (S. 189).) -- Neben den Josephs-Romanen
  wäre in diesem Zusammenhang auch Thomas Manns biblische Erzählung "`Das
  Gesetz"' zu nennen.} Voegelin klingt freilich wesentlich weniger aufgeklärt,
wenn er wortwörtlich schreibt, dass der Ordnungswille "`nur aktiv sein kann,
wo er seinen Sinn in der Ordnung des Gemeinschaftsmythos
hat"'.\footnote{Voegelin, Anamnesis, S. 50.} Hier scheint eher noch ein Rest
von dem faschistischen Gedankengut durchzuklingen, das Voegelin in seiner
autoritären Phase in den 30er Jahren absorbiert hatte.\footnote{Vgl. Voegelin,
  Autoritärer Staat, a.a.O.} Aus heutiger Sicht muss ein Ordnungswille, der im
"`Gemeinschaftsmythos"' wurzelt, in höchstem Maße suspekt erscheinen.

Wie man sieht, ist also der Rückgriff auf die Mythengeschichte für die
Begründung und Artikulation des Gleichheitsideals in Wirklichkeit keineswegs
erforderlich und höchstens mit Einschränkungen nützlich. Voegelins Argument
dafür, das er im Gegenteil unerlässlich sei, ist historischer Art und besteht
-- wie zuvor ausgeführt -- in der Behauptung, dass alle Gleichheitsideen
Derivate jener beiden Urmythen der Abstammung von einer Mutter oder der
Prägung durch einen Vater sind.\footnote{Dass sich -- so das andere Argument,
  das aus Voegelins Text extrahiert werden kann -- aus dem von Voegelin
  behaupteten Ausdruckskonflikt bei der Artikulation unendlicher Prozesse für
  diesen Fall kein notwendiger Grund für den Gebrauch der Mythensymbolik
  ableiten lässt, wurde bereits erwähnt.}  Inwieweit dies historisch richtig
und zwingend ist, sei dahingestellt. Für Voegelin war diese Vorstellung
wohlmöglich deswegen attraktiv, weil sie ihm erlaubte, eine Analogie zum
Leib-Geist-Dualismus herzustellen, indem der eine Mythos ein leiblicher und
der andere ein geistiger ist. Aber selbst wenn Voegelins historische These
richtig sein sollte, so folgt daraus nicht, dass die Gleichheitsidee niemals
etwas anderes sein kann als ein Derivat dieser Urmythen. Insbesondere kommt es
bei der moralphilosophischen Diskussion der Gleichheitsidee nur auf den Inhalt
und die Begründung dieser Idee an. Diese sind aber von der
Entstehungsgeschichte unabhängig, so dass die Diskussion darüber unbekümmert
um die Geschichte der Mythologie geführt werden kann.

Nicht nur Voegelins Ausführungen zur Mythensymbolik sondern auch seine
Interpretation der Prozesstheologie wirft einige Fragen auf. Vor allem
Voegelins Annahme, dass die Prozesstheologie einen Bereich von "`ontologischen
Erfahrungen"'\footnote{Voegelin, Anamnesis, S. 54.}  auslegt, bedarf der
Klärung. Denn der Begriff der Erfahrung wird mit dieser Annahme stark
überstrapaziert. Das Wissen um die Stufen des Seins ist deskriptives Wissen,
das sich bestenfalls auf die Erfahrung stützt, das aber über die unmittelbare
Erfahrung weit hinaus geht. Auch dass die meditative Erfahrung ein Wissen vom
Seinsgrund vermittelt, muss als höchst zweifelhaft angesehen werden, sofern
"`Seinsgrund"' ein ontologischer Begriff ist und nicht nur ein Name für die
meditative Erfahrung selbst, wie Voegelins spätere Theorie der "`Indizes"' des
Bewusstseins dies nahelegt.\footnote{Siehe Kapitel \ref{IndizesTheorie}} Im
letzteren Fall bestünde dann allerdings auch keine "`Nötigung"' mehr, sondern
es wäre im Gegenteil sogar ganz und gar unmöglich, den ontologischen
Seinsprozess in einem "`Seinsgrund"' außerhab des Bewusstseins entspringen zu
lassen. Wird jedoch auf diese Weise in Zweifel gezogen, dass es eine
privilegierte Klasse ontologischer Erfahrungen gibt, dann bleibt auch die
Prozesstheologie, die Voegelin skizziert, als eine bestimmte ontologische
Theorie in vollem Umfang durch die Erkenntniskritik angreifbar.

Abgesehen von diesen Schwierigkeiten bleibt auch der Sinn und Zweck der
Prozesstheologie im Unklaren. Voegelin zufolge geht die Prozesstheologie aus
von der Frage: "` `Warum ist etwas, warum ist nicht Nichts?'
"'\footnote{Voegelin, Anamnesis, S.51. -- Vgl. Friedrich Schelling: Philosophie
  der Offenbarung, Zwölfte Vorlesung, in: Frank-Peter Hansen (Hrsg.):
  Philosophie von Platon bis Nietzsche, CD-ROM, Berlin 1998, S.37855 / S.72
  (Konkordanz: Friedrich Wilhelm Joseph von Schelling: Werke. Auswahl in drei
  Bänden. Herausgegeben und eingeleitet von Otto Weiß. Leipzig 1907.  Band 3,
  S. 781).} Allerdings unternimmt die Prozesstheologie dann keinen ernsthaften
Versuch, diese Frage zu beantworten. Eher scheint sie darauf hinauszulaufen,
das Gefühl des Staunens bzw. der Verblüffung, das in jener Frage liegt, zu
artikulieren. Wenn sich aber die überwiegende Mehrzahl der Menschen nicht mit
dem erkenntnistheoretischen Befund der Unbeantwortbarkeit dieser Frage
zufrieden geben will, wie Voegelin -- nicht unplausibel --
vermutet,\footnote{Vgl.  Voegelin, Anamnesis, S. 51.} warum sollte sie sich
dann mit der Prozesstheologie, die diese Frage auch nicht beantworten kann,
abspeisen lassen?

Hinsichtlich der Einbettung der Bewusstseinsphilosophie in die Ontologie,
wie sie Voegelin im letzten Abschnitt seines Aufsatzes vollzieht, sind
vor allem die zwei Thesen zu prüfen, dass die ontologische Problematik
die Voraussetzung der Erkenntnistheorie bzw. Bewusstseinsphilosophie
bildet, und dass der Mensch sich auf sein Bewusstsein und sein Wesen nur
orientierend besinnen aber es niemals zu einem Gegenstand äußerer
Beschreibung machen kann.

Die erste dieser Thesen ist auch für Voegelins wissenssoziologische
Erklärungen von Bedeutung, denn nur, wenn sie bejaht wird, kann der "`Versuch
einer `radikalen' Bewusstseinsphilosophie
aufklärungsbedürftig"'\footnote{Voegelin, Anamnesis, S. 58.} erscheinen. In
einer bestimmten Hinsicht kann die Triftigkeit von Voegelins Einwand gegen die
reine Bewusstseinsphilosophie kaum bestritten werden. Die Erklärung der
meisten Bewusstseinsvorgänge dürfte nur schwer möglich sein, ohne auf die
Tatsache zurückzugreifen, dass es sich um das Bewusstsein eines Menschen
handelt, der in einer materiellen Außenwelt lebt. So ist etwa das
gelegentliche Auftreten des Bewusstseinsphänomens "`Hunger"' nur verständlich,
wenn man die Selbstverständlichkeit berücksichtigt, dass das Bewusstsein, in
dem es auftritt, das Bewusstsein eines Lebewesens ist, welches von Zeit zu Zeit
der Speise und des Tranks bedarf. Auch darf wohl behauptet werden, dass
ontologische Fragen insgesamt relevanter sind als nur rein
bewusstseinsphilosophische Probleme, denn der Erhalt und die Wohlfahrt unseres
Lebens hängt von dem ab, was in der Welt geschieht und nicht von dem, was sich
davon im Bewusstsein spiegelt. Insofern spricht für Voegelins kritische
Einstellung gegenüber der reinen Bewusstseinsphilosophie auch eine starke
intuitive Plausibilität.

Aber ist damit auch die Möglichkeit einer reinen, d.h. ausschließlich
introspektiven Beschreibung der Bewusstseinsvorgänge ausgeschlossen? Und muss
die Erkenntnistheorie nun doch, trotz der drohenden Gefahr von
Begründungszirkeln, ein Wissen um die Außenwelt voraussetzen? In dieser
Hinsicht scheint Voegelins These unzureichend begründet zu sein. Auch wenn
viele Bewusstseinsvorgänge losgelöst von der Außenwelt nur schwer zu deuten
sein dürften, so bleibt doch die Möglichkeit, das Bewusstsein als reines
Bewusstsein introspektiv zu beschreiben, immer noch bestehen. Sollte zur
Beschreibung des reinen Bewusstseins als Prozess eine Form von Zeitlichkeit
vorausgesetzt werden müssen, die nicht introspektiv erfahrbar ist, so genügt
es, allein die Existenz dieser Form von Zeitlichkeit zu postulieren, ohne
zugleich auch den Leib und die Geschichte vorauszusetzen. Eine solche
Beschreibung des reinen Bewusstseins würde auch dann keine weiteren
ontologischen Hypothesen voraussetzen, wenn es faktisch substanzidentisch mit
seinem leiblichen Fundament (Gehirn) sein sollte. Dabei ist übrigens die
Hypothese der Substanzidentität zum Verständnis "`der Fundierung von
Bewußtsein in Leib und Materie"'\footnote{Voegelin, Anamnesis, S. 55.} nicht
einmal zwingend erforderlich, denn diese Fundierung könnte auch durch die
Hypothese der kausalen Verursachung von Bewusstseinsphänomenen durch mit
diesen nicht substanzidentische physische Phänomene erklärt werden. Die
Erkenntnistheorie schließlich setzt schon deshalb nicht die Ontologie voraus,
weil die erkenntnistheoretischen Probleme auf einer anderen Ebene, auf der
Ebene der Gültigkeit, liegen als die ontologischen Probleme. Zwar ist das
Faktum, dass es Erkenntnis und Wahrheit gibt, davon abhängig, dass es
Lebewesen gibt, die erkennen können, aber die Gültigkeit von Erkenntnis und
die Antwort auf die Frage, worin Wahrheit besteht und nach welchen Kriterien
sie festgestellt werden kann, hängen nicht von diesen ontischen
Voraussetzungen ab. Am leichtesten lässt sich dies an einem Beispiel
verdeutlichen: Damit der Satz "`Zwei mal zwei ist vier."' existiert, muss es
wenigstens ein intelligentes Wesen geben, welches ihn denkt oder
äußert,\footnote{Manche Philosophen glauben auch, dass Sätze wie dieser in
  einem platonischen Ideenhimmel existieren. Die Sätze würden dann auch
  existieren, wenn es keine Menschen oder nicht einmal eine Welt gäbe.}  und
damit dieses Wesen existiert, müssen weitere ontische Voraussetzungen erfüllt
sein. Die Wahrheit dieses Satzes hängt jedoch von keiner dieser
Voraussetzungen ab.\footnote{Man könnte nun vermuten, dass nicht die Wahrheit
  aber die Bedeutung eines Satzes von ontischen Voraussetzungen, z.B. von der
  Bedeutungsgeschichte der in ihm verwendeten Wörter abhängt. Aber die Art und
  Weise, wie die Bedeutung eines Wortes entstanden ist, stellt keine
  Bedeutungsvoraussetzung des Wortes dar, sondern lediglich eine kausale
  Voraussetzung der Entstehung der Bedeutung des Wortes.}

Wie verhält es sich mit Voegelins zweiter These, dass das Bewusstsein sich
nicht selbst wie einen Gegenstand betrachten kann? Voegelin führt als Grund
für diese These an, dass die Bewusstseinsphilosophie "`ein spätes Ereignis in
der Biographie des Philosophen ist"', welches wiederum ein Ereignis in der
Geschichte seiner Gemeinschaft, in der Geschichte der Menschheit und in der
Geschichte des Kosmos ist. Aber diese Begründung ist wenig stichhaltig und
dürfte eher einer holistischen Überzeugung Voegelins geschuldet sein als auf
rationaler Überlegung beruhen. Jeder noch so profane Gegenstand hat auch seine
Vorgeschichte im Kosmos, und das Wissen über ihn hat eine Vorgeschichte in der
Geschichte des menschlichen Wissens. Dennoch wird niemand bestreiten, dass es
Gegenstände gibt, die vollständig erkannt werden können. Wenn irgendetwas nur
historisch verstanden werden kann, so muss es dafür speziellere Gründe geben.
Und außer dem Kosmos selbst gibt es vermutlich nichts, dessen Erkenntnis die
Geschichte des gesamten Kosmos voraussetzt.

Problematisch ist auch jener Teil von Voegelins Aufsatz, in welchem er das
Auftreten der Bewusstseinsphilosophie in der Neuzeit historisch zu deuten
versucht.\footnote{Vgl. Voegelin, Anamnesis, S. 58ff.} Wie bereits dargelegt,
sind Erkenntnistheorie und mit gewissen Einschränkungen auch die
Bewusstseinsphilosophie legitime Einzeldisziplinen der Philosophie, die nicht
unbedingt als Teilgebiet einer allgemeinen Ontologie behandelt werden müssen.
Ihr Auftreten ist daher bereits wesentlich weniger "`aufklärungsbedürftig"'
als Voegelin meint. Abgesehen davon bleibt es schleierhaft, woher Voegelin
überhaupt die historische Aufgabe der Bewusstseinsphilosophie nimmt, eine neue
Symbolik für religiöse Transzendenzerfahrungen zu suchen. Sofern Voegelin
nicht wie Husserl die Existenz eines historischen Telos voraussetzen will, das
jeden Philosophen verpflichtet, sich mit diesem Problem zu beschäftigen, kann
er den Philosophen kein Versäumnis vorwerfen, wenn sie sich für andere Fragen
als die der Symbolisierung von Transzendenzerfahrungen interessieren. Freilich
hat Voegelin das Recht, den Ausschluss der Besinnung auf
Transzendenzerfahrungen aus dem Themenkanon der Philosophie zu tadeln, wenn er
selbst der Ansicht ist, dass dieses Thema in der Philosophie einen Platz haben
sollte. Allerdings ist zu berücksichtigen, dass andere Philosophen dies
explizit ablehnen, und dass sie dazu mindestens ein ebensogutes Recht haben.
Abgesehen davon existierten auch in der Neuzeit mit Religion und Theologie
durchgängig geistige Disziplinen, die sich mit der Transzendenz beschäftigten
und immer noch beschäftigen.  Insofern ist es ein wenig voreilig, eine
historische Krise der Symbole zu suggerieren.  Schließlich ist anzumerken,
dass gerade die Husserlsche Phänomenologie, welche sich noch am ehesten
angeschickt hat, die Bewusstseinsphilosophie zum allumfassenden
Universalparadigma auszuweiten, sich gegenüber dem religiösen Denken als sehr
aufgeschlossen erwiesen hat.

Alles in allem leidet Voegelins Aufsatz "`Zur Theorie des Bewußtseins"' an
auffällig vielen argumentativen Schwächen. An haltbaren Resultaten ist der
Text außerordentlich arm. Dies mag damit zusammenhängen, dass er eher den
Charakter einer persönlichen Besinnung als den einer philosophischen
Argumentation hat. Von Voegelin war das durchaus intendiert, denn die
philosophische Tätigkeit bestand für ihn vor allem in der meditativen
philosophischen Besinnung. Nur stellt sich dann dem Leser irgendwann die
Frage, warum er sich für die persönlichen Besinnungen des Herrn Voegelin
eigentlich interessieren sollte, eine Frage, die sich noch mehr aufdrängt,
wenn er dann zu den im Folgenden zu besprechenden "`anamnetischen
Experimenten"' Voegelins weiterblättert.

\section{Die "`anamnetischen Experimente"' Voegelins}

Den ersten Teil seines Werkes "`Anamnesis"' schließt Voegelin mit der
Wiedergabe einiger Kindheitserinnerungen ab. Es handelt sich um Schilderungen
intellektueller Erlebnisse seiner Kindheit, in welchen zum erstenmal, in einer
freilich dem zarten Alter entsprechenden Weise, die Fragen auftauchten, welche
Voegelin sich später als Bewusstseinsphilosoph erneut stellte. Da diese
Erinnerungen teilweise erst durch den Versuch wieder zu Tage traten, sich
Rechenschaft über die ersten Anfänge jener Bewusstseinserlebnisse und
Stimmungen abzulegen, die Voegelin später als erwachsener Philosoph
untersuchte, spricht er von "`anamnetischen
Experimente[n]"',\footnote{Voegelin, Anamnesis, S. 61.}  deren Resultate diese
Erinnerungen sind. Unter den manchmal mit Augenzwinkern erzählten Episoden,
die Voegelin aus seiner Kindheit mitteilt, finden sich Stücke wie jenes von
dem Karnevalszug, der in dem Kind eine dunkle Angst erregte, weil sich der
Zug, da ihn einzelne Jecken immer wieder verließen, um in den Seitenstraßen zu
verschwinden, am Ende aufzulösen schien.\footnote{Vgl.  Voegelin, Anamnesis,
  S. 64 (Nr.  3).}  In einer anderen Episode berichtet Eric Voegelin, der
einen Teil seiner Kindheit in Königswinter bei Bonn nahe dem Siebengebirge
verbrachte, von den drei Breibergen, die man vom Ölberg aus sehen kann. Dem
Märchen zufolge muss man sich durch diese Breiberge hindurchfressen, um in das
dahinter liegende Schlaraffenland zu gelangen. Die Angst des Kindes, dabei im
Brei stecken zu bleiben, trübte sehr die Hoffnung auf das
Schlaraffenland.\footnote{Vgl. Voegelin, Anamnesis, S. 65-66 (Nr. 5).}  Andere
Episoden teilen ähnliche Gefühle der Zweifelhaftigkeit des vollkommenen
Glückes mit.\footnote{Vgl. Voegelin, Anamnesis, S. 66 (Nr. 6), S. 73-64
  (Nr.16).}  Voegelin selbst gibt keine Erläuterungen zu den erzählten
Episoden, die ihre Bedeutung für sein späteres Denken erklären
könnten.\footnote{Einige vorsichtige Deutungsversuche unternimmt Barry Cooper.
  Vgl. Barry Cooper: Eric Voegelin and the Foundations of Modern Political
  Science, Columbia and London 1999, S. 204-207.} Womöglich betrachtete
Voegelin diese frühen Erfahrungen als Vorboten der späteren Skepsis des
Politikwissenschaftlers gegenüber der Utopie. In einer weiteren Episode
schildert Voegelin den starken emotionalen Eindruck, den das Märchen vom
Kaiser und der Nachtigall, die durch ihren Gesang den Tod dazu erweicht vom
Kaiser abzulassen, in ihm hinterlassen hat. Später fand Voegelin diese
Stimmung aus seiner Kindheit zwar nicht mehr im Märchen wohl aber beim Anhören
mancher Musikstücke wieder: "`Die Bedeutung, die ein Musikwerk für mich hat,
ist bestimmt durch den Grad, in dem es diese süße Beklemmung zwischen Tod und
Leben wieder erregt."'\footnote{Voegelin, Anamnesis, S. 75 (Nr. 18).}
 
Solcher und ähnlicher Art sind die von Voegelin wiedergegebenen
Kindheitserinnerungen. Doch was soll mit ihrer Mitteilung bewiesen werden? In
den einleitenden Vorbemerkungen zu seinen "`anamnetischen Experimenten"' führt
Voegelin die Thesen aus seinem vorangehenden Aufsatz noch einmal auf: Das
Bewusstsein ist kein Strom, sondern es verfügt über vielfältige
Transzendenzfähigkeiten.  Die Besinnung über das Bewusstsein greift
Bewusstseinserlebnisse des Philosophen auf, die bereits sehr viel früher in
seinem Leben erstmals zu Tage getreten sind. Weiterhin sieht Voegelin in den
frühen Bewusstseinserlebnissen "`Erfahrungseinbrüche"' und
"`Erregungsquellen"', "`aus denen es zu weiterer philosophischer Besinnung
treibt"'. Die Intensität und Emotionalität\footnote{Voegelin spricht wörtlich
  von der "`Natur der Erfahrungseinbrüche"', der "`Art der Erregungen"' und
  der "` `Stimmung' "' des Bewusstseins. (Vgl. Voegelin, Anamnesis, S. 61.)}
solcher "`Erfahrungseinbrüche"' bilden für Voegelin den Maßstab der
Radikalität, d.i.  der Breite und Tiefe einer philosophischen Besinnung.

Sind aber solche Erfahrungen, wie Voegelin sie erzählt, für die Behandlung
bewusstseinsphilosophischer Probleme überhaupt relevant? Sicherlich ist nicht
für jedes philosophische Problem der Rückgang auf die Erfahrung seines ersten
Auftretens erforderlich. Für die Lösung des zenonschen Problems etwa, wie aus
unendlich vielen Einzelschritten ein kontinuierlicher Übergang entstehen
kann,\footnote{Vgl. Voegelin, Anamnesis, S. 71/72 (Nr. 14: Der Laib Brot).}
spielt es sicherlich keine Rolle, wann und wie es zum erstenmal dem
Philosophen, der es behandelt, begegnet ist.  Auch wenn er es erst im
Erwachsenenalter in einem Buch gelesen hat, hindert ihn nichts daran, dieses
Problem angemessen zu erörtern. In welchem Falle ist es dann aber notwendig,
auf die Problemerfahrungen zurückzugehen, und in welchem Fall nicht?
Offensichtlich ist dies dann nicht erforderlich, wenn die Erfahrung nur den
Anlass gibt, über ein philosophisches Problem nachzudenken. Außer wenn es um
Fragen der Selbsterkenntnis geht, hat das {\em Erlebnis} eines philosophischen
Problems jedoch keine andere philosophische Bedeutung als die, ein Anlass des
Nachdenkens zu sein. Dies gilt auch für die Probleme der
Bewusstseinsphilosophie. Für die Lösung beispielsweise des Problems der
Konstitution von Gegenständen ist zwar möglicherweise der Bezug auf innere
Erfahrungen, nicht aber der Rückgang auf die ersten Erfahrungen dieser Art
oder auf das erstmalige Erlebnis, dass es sich hier um ein Problem handelt,
erforderlich. Im übrigen stünde eine Philosophie, die sich nur auf die eigenen
inneren Erlebnisse des Philosophen stützt, vor dem Problem, dass sie bloß
subjektiv gültige Ergebnisse liefern könnte.

Es scheint also, dass Voegelin die Bedeutung von "`Erfahrungseinbrüchen"' für
die Philosophie erheblich überschätzt hat. Deshalb ist es auch ein
zweifelhaftes Unterfangen, die Philosophie an den vermeintlich zu Grunde
liegenden inneren Erlebnissen messen zu wollen, zumal dies die Gefahr birgt,
dass dann in letzter Konsequenz die Heftigkeit und Leidenschaftlichkeit
des Denkens mehr wiegen als die Qualität der Argumente. Eingeräumt werden muss
allerdings, dass die Tiefe einer philosophischen Untersuchung wahrscheinlich
auch durch die Intensität der zur Philosophie motivierenden Erfahrungen
mitbestimmt ist, nur ist die Intensität der Motivation nicht der
Bewertungsmaßstab für philosophische Werke. Unter den zur Philosophie
motivierenden Erlebnissen dürften für die meisten Menschen dabei wohl die
"`Erfahrungseinbrüche"' der Jugend eine größere Rolle spielen als die der
Kindheit. Aber Voegelins "`anamnetische Experimente"' sind vermutlich eher als
Beispiele zu verstehen denn als eine vollzählige Auflistung.

Abgesehen davon bleiben Voegelins "`anamnetische Experimente"' ein wenig
hinter den Erwartungen zurück, die durch seine vorangegangenen Ausführungen
geweckt werden. In den vorangegangenen beiden Abhandlungen kommt der
mystischen Erfahrung des welttranszendenten Seinsgrundes eine zentrale
Bedeutung zu. Für Voegelins These etwa, dass wir das Sein nur verstehen
könnten als einen im welttranszendenten Seinsgrund entspringenden Prozess, ist
diese Erfahrung eine unabdingbare Voraussetzung. Damit diese These glaubhaft
wird, müsste es nämlich schon tatsächlich der Seinsgrund sein, der sich in
dieser Erfahrung zeigt. Sollte es sich nur um irgendein überwältigendes
Meditationserlebnis handeln, welches bloß vor lauter Begeisterung eine
Erfahrung vom Seinsgrund genannt wird, so wäre die These noch unzureichend
begründet, denn der Prozess des Seins kann -- außer für einen philosophischen
Solipsisten -- nicht einer Erfahrung {\em im} Bewusstsein entspringen.
Voegelins "`anamnetische Experimente"' bilden einen der wenigen Anlässe für
Voegelin, von eigenen Erfahrungen zu berichten.  Ein glaubhaftes und
unzweideutiges Transzendenzerlebnis fördern Voegelins "`anamnetische
Experimente"' jedoch nicht zu Tage. Zwar deutet Voegelin in seinen
Einleitenden Bemerkungen zu den anamnetischen Experimenten noch an, dass die
Bewusstseinstranszendenz, die in "`finiter Erfahrung"' in die Welt hinein
führt, nur eine Art von Transzendenz sei, und er führt unter den verschiedenen
Transzendenzerfahrungen, die in der Biographie des Bewusstseins schon lange
vor dem Einsetzen der philosophischen Besinnung vorgegeben sind, auch die
Erfahrung der Transzendenz in den Seinsgrund auf,\footnote{Vgl.  Voegelin,
  Anamnesis, S. 61.} aber in den Kindheitserlebnissen lässt sich dann nichts
mehr davon wiederfinden. Dort ist zwar unter anderem von dem Schlaraffenland
und auch von einer Wolkenburg die Rede, aber einen Hinweis auf irgendetwas,
was auch nur annähernd als transzendente Seinsphäre oder gar als der Grund
allen Seins gelten könnte, sucht man vergebens.

Endlich gibt es noch einen weiteren, mehr psychologischen Grund, der das
Unterfangen, in Kindheitserinngerungen die "`Erregunsquellen"' ausfindig zu
machen, aus denen es im Erwachsenenleben "`zu weiterer philosophischer
Besinnung treibt"',\footnote{Voegelin, Anamnesis, S.62.} fragwürdig erscheinen
lässt. Wenn wir im Erwachsenenalter rückblickend unsere Kindheit betrachten
und uns dabei außerdem noch auf der Suche nach unseren eigenen geistigen
Ursprüngen befinden, dann lässt es sich nicht immer vermeiden, dass wir in
unsere Erinnerung etwas hineininterpretieren, was ursprünglich gar nicht
vorhanden war. Die Gefahr einer Selbstmystifikation ist bei "`anamnetischen
Experimenten"' nur schwer zu umgehen. Es besteht hier übrigens eine Analogie
zu jener größeren historischen "`Anamnese"' der Wiedererweckung verschollenen
Ordnungswissens aus den Quellen der antiken Philosophie und mittelalterlichen
Theologie, wo ebenfalls gelegentlich der Eindruck entsteht, dass bei Voegelin
eine durch und durch moderne existenzialistische Philosophie dem Denken der
Alten aufgestülpt wird.

Voegelins Programm der "`Anamnese"' scheitert im Ganzen also aus drei Gründen:
Erstens ist die Genealogie eines Gedankens für den Gedanken selbst, d.h. für
seinen Inhalt, seine Richtigkeit oder Falschheit, bedeutungslos.
Zweitens führt das Verfahren der "`Anamnese"' mit großer Wahrscheinlichkeit zu
einer Verfälschung der Genealogie. Drittens gelangt man auf diesem Wege
ebensowenig zu jener vermeintlich vorhandenen Transzendenz wie durch die
philosophische Meditation.


%%% Local Variables: 
%%% mode: latex
%%% TeX-master: "Main"
%%% End: 












