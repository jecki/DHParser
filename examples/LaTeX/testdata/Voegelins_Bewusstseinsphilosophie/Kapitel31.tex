
%%% Local Variables: 
%%% mode: latex
%%% TeX-master: "Main"
%%% End: 

\chapter{Voegelins Bewusstseinsphilosophie ("`Anamnesis"' -- Teil I)}
\label{VoegelinsBewusstseinsphilosophie}

In diesem und dem folgenden Kapitel wird Voegelins Bewusstseinsphilosophie
erörtert, wie sie im ersten und dritten Teil seines Werkes "`Anamnesis"'
entfaltet wird. Der zweite Teil von "`Anamnesis"' enthält eher historische und
geschichtsphilosophische Studien und wird daher hier übergangen. Zwar fließen
bei Voegelin Bewusstseinsphilosophie und Geschichtsphilosophie ineinander,
aber schon aus pragmatischen Gründen musste für die vertiefte
Auseinandersetzung mit Voegelins Bewusstseinsphilosophie eine Auswahl
getroffen werden. Zudem wurde zur Verortung von Voegelins
Bewusstseinsphilosophie innerhalb seines Gesamtansatzes schon im vorigen
Kapitel das Nötige gesagt. Weiterhin kehren viele der Motive aus den hier
ausführlich besprochenen Schriften in Voegelins anderen Werken wieder, so dass
man die hier angestellten Überlegungen leicht übertragen kann.

Der erste Teil von Voegelins "`Anamnesis"' enthält neben einer kurzen
Erinnerung an Voegelins Freund Alfred Schütz\footnote{Alfred Schütz (geb. 1989
  in Wien, gest. 1959 in New York), war der Schöpfer einer phänomenologischen
  Soziologie. Sein wohl bekanntestes Werk trägt den Titel: "`Der sinnhafte
  Aufbau der sozialen Welt. Eine Einführung in die verstehende Soziologie"'.},
die -- als eher von biographischem als philosophischen Interesse -- hier
übergangen wird, eine Kritik von Husserls "`Krisis der europäischen
Wissenschaften"'.\footnote{Vgl.  Voegelin, Anamnesis, S. 21-36. -- Edmund
  Husserl: Die Krisis der europäischen Wissenschaften und die transzendentale
  Phänomenologie. Eine Einleitung in die phänomenologische Philosophie,
  Hamburg 1996, im folgenden zitiert als: Husserl, Krisis.}. An diese schließt
sich ein eigenständiger bewusstseinsphilosophischer Entwurf unter dem Titel
"`Zur Theorie des Bewußtseins"' an.\footnote{Vgl. Voegelin, Anamnesis, S.
  37-60.} Den Abschluss des ersten Teils von "`Anamnesis"' bilden dann eine
Reihe von "`anamnetischen Experimenten"',\footnote{Vgl. Voegelin, Anamnesis,
  S. 61-76.} wobei sich hinter dieser geheimnisvollen Bezeichnung jedoch
nicht viel mehr verbirgt als die Erzählung einiger Kindheitserinnerungen
Voegelins.

\section{Voegelin über Husserls "`Krisis der europäischen
  Wissenschaften"'}

\subsection{Husserls Krisis-Schrift}

Bevor auf Voegelins Auseinandersetzung mit Husserls Schrift: "`Die Krisis der
europäischen Wissenschaften und die transzendentale
Phänomenologie"'\footnote{Husserl, Krisis, a.a.O.} eingegangen wird, ist
einiges zu dieser Schrift selbst zu sagen.

Husserls Schrift ist als eine Einführung in die Phänomenologie konzipiert. Sie
entstand aus mehreren Vorträgen, die Husserl im Jahre 1935 gehalten hat. Da
Husserl 1936 in Deutschland nicht mehr publizieren durfte, wurde die Schrift
1936 in der in Belgrad erscheinenden Zeitschrift "`Philosophia"'
veröffentlicht.\footnote{Vgl. dazu die Einleitung von Elisabeth Ströker, in:
  Husserl, Krisis, S.IXff.} Auf diese Fassung, welche Voegelin 1943 in die
Hände bekommen konnte, bezieht sich Voegelin in seinem Brief an Alfred Schütz.
Gegenüber der 1954 in der Reihe {\it Husserliana} erschienenen und um bis
dahin unpubliziertes Material ergänzten Ausgabe ist die
"`Philosophia"'-Fassung um einiges kürzer. Insbesondere wird in der frühen
Fassung die Lebenswelt-Problematik noch kaum angerissen. Dies ist zu
berücksichtigen, da Voegelins Enttäuschung über den wieder nur rein
erkenntnistheoretischen Charakter von Husserls Werk sonst leicht ungerecht
erscheinen könnte.

Husserl hat in seinen einführenden Schriften recht unterschiedliche Zugänge
zur Phänomenologie gegeben. In den "`Cartesianischen Meditationen"'
beispielsweise wird die Phänomenologie durch die Aufgabe motiviert, die Basis
für eine letztbegründete und umfassende philosophische Universalwissenschaft
zu schaffen.\footnote{Vgl. Edmund Husserl: Cartesianische Meditationen. Eine
  Einleitung in die Phänomenologie, Hamburg 1987, S. 8ff.} In seiner letzten
Einführung hingegen wird die Phänomenologie, wie sich schon im Titel andeutet,
durch einen geistigen Notstand motiviert. Dieser geistige Notstand besteht
darin, dass die Weltsicht der Gegenwart fast vollkommen von den
Naturwissenschaften und insbesondere von der Physik als der Leitwissenschaft
dominiert wird.\footnote{Vgl. Husserl, Krisis, S. 3-5 (§ 2).} Husserl
betrachtet dies als ein Verhängnis, weil die Naturwissenschaften nach seiner
Auf\/fassung nicht die wirkliche Welt wiedergeben (welche für Husserl einzig
und allein die Welt der konkret gegebenen Phänomene ist), sondern der
wirklichen Welt mathematische Gestalten unterschieben. Zwar ist Husserl
bereit, die pragmatische Brauchbarkeit dieser Gestalten anzuerkennen, aber er
hält es für einen schweren Fehler, ihnen eine ontologische Beschreibung der
Welt zu entnehmen. Den Irrtum, die Modelle der Naturwissenschaften als
Beschreibungen der Wirklichkeit zu verstehen, bezeichnet Husserl als
"`Physikalismus"'.\footnote{Vgl. Husserl, Krisis, S. 68.} Dass es sich beim
Physikalismus um einen Irrtum handelt, versucht Husserl durch eine suggestive
Beschreibung der historischen Entwicklung des wissenschaftlichen Denkens zu
zeigen. Als Ausweg aus dem Physikalismus preist Husserl die transzendentale
Phänomenologie an. Sie würde es ermöglichen, die Wirklichkeit in ihrer
konkreten phänomenalen Gegebenheit für das Bewusstsein zurückzugewinnen und
die Wissenschaften wieder in angemessener Weise in die Lebenswelt
einzubetten.\footnote{Husserls Kritik des "`Physikalismus"' ist alles andere
  als überzeugend, worauf hier jedoch nicht ausführlich eingegangen werden
  kann.  Die Hauptschwachpunkte seien nur kurz angemerkt: 1. Husserl
  unterstellt, dass die Naturwissenschaft der Natur etwas unterschiebt, was sie
  in Wirklichkeit nicht ist. Da die Naturwissenschaft ihre Ergebnisse jedoch
  experimentell auf die Probe stellt, kann sie der Natur nicht ohne Weiteres
  etwas Falsches unterschieben. Husserls Vorwurf kann sich also höchstens noch
  darauf beziehen, dass die Naturwissenschaft die Erscheinungen nicht für das
  Sein der Natur nimmt. Wird dies jedoch als illegitim angesehen, so stellt
  sich die Frage, ob dann nicht auch die "`eidetische Wesensschau"' des
  Phänomenologen (Vgl.  Edmund Husserl: Die phänomenologische Methode.
  Ausgewählte Texte I. (Hrsg.  von Klaus Held), Stuttgart 1985, S. 101-107.)
  dem Phänomen ein Wesen unterschiebt.  2. Husserls historische
  Darstellungstechnik ist nicht besonders gut dazu geeignet, systematische
  Probleme zu lösen, auch wenn sich aus ihr möglicherweise systematische
  Argumente indirekt entnehmen lassen. Die Feststellung z.B., dass die
  mathematisch-geometrischen Gestalten ursprünglich Methode (nämlich
  Feldmesskunst) waren (Vgl. Husserl, Krisis, S. 52ff.), besagt noch längst
  nicht, dass sie in ihrer entwickelten Form für den Ausdruck ontologischer
  Zusammenhänge untauglich wären. Es sei denn, man nimmt an, dass etwas, was
  einmal Methode gewesen ist, sich niemals zu etwas wesentlich anderem
  entwickeln kann, oder dass Wissenschaften sich grundsätzlich nicht von ihrem
  historischen Ursprung emanzipieren können oder dürfen.}

Allerdings bleibt es in der "`Krisis der europäischen Wissenschaften"' nicht
bei der Kritik am Physikalismus, denn Husserl beabsichtigt, so scheint es, der
Phänomenologie die Weihen einer historischen Mission zu verleihen. Husserl
behauptet dazu, dass sich in der Geistesgeschichte ein "`Telos"' auf\/finden
lasse, wobei das Wort "`Telos"' einen recht vieldeutigen Sinn gewinnt, der
sowohl Ziel und Ursprung als auch Anklänge von Legitimation und Verpflichtung
beinhaltet. Was Husserl in diesem Zusammenhang zu dem Thema der
Geschichtsteleologie zu sagen hat, rückt seine Darstellung in der Tat stark in
die Nähe einer Geschichtsideologie. Husserl zufolge ist dieses Telos nämlich
ein aus der Geschichte ablesbarer höherer Wille, der auf die Entwicklung der
phänomenologischen Philosophie hinzielt.\footnote{Vgl. Husserl, Krisis, S.
  14-19 (§ 6,7).} Dieser Wille darf keineswegs verwechselt werden mit den
Absichten einzelner Philosophen, vielmehr ist er als eine durch den einzelnen
Philosophen "`hindurchgehende Willensrichtung"'\footnote{Husserl, Krisis, S.
  78.} zu verstehen.  Deshalb kann dieser Wille auch nicht den
Selbstzeugnissen dieser Philosophen entnommen werden, sondern muss unter
Zuhilfenahme einer kunstvollen hermeneutischen Interpretationstechnik im
historischen Rückblick aus dem Verborgenen hervorgehoben werden.\footnote{Vgl.
  Husserl, Krisis, S. 62-64 (§ 9 l) ) / S. 77-80 (§ 15), S. 109.}  Entstanden
ist dieser Wille in den beiden "`Urstiftungen"' der antiken griechischen
Philosophie und des philosophischen Neuanfangs durch Descartes.  Diese
Urstiftungen verlangen ihrem Wesen nach (und nicht bloß, wie man denken
könnte, ihrem Namen nach) nach einer "`Endstiftung"',\footnote{Vgl.  Husserl,
  Krisis, S. 79.} für welche aus philosophisch-sachlichen Gründen nur die
transzendentale Phänomenologie in Frage kommt. Urstiftungen und Endstiftungen
sind dabei keine kontingenten historischen Ereignisse, sondern Ausdruck einer
im "`Menschentum"' beschlossenen "`Vernunftentelechie"'.\footnote{Vgl.
  Husserl, Krisis, S. 15.} Die Autorität, die hinter diesem "`Telos"' steht,
ist die Autorität der Geschichte und der Tradition oder, wie es Husserl auch
ausdrückt, der "`Wille der geistigen Vorväter"'\footnote{Husserl, Krisis, S.
  78.}. Philosophieren in der Gegenwart ist nur im reflektierten Rückbezug auf
die Tradition möglich, da jeder Versuch, sich von den Vorurteilen der
Tradition zu lösen, nur unter Rückgriff auf "`Selbstverständlichkeiten"'
erfolgen kann, die wiederum einer Tradition entspringen.\footnote{Vgl.
  Husserl, S. 78-79.} Sind die Philosophen nun aber nicht willens oder in der
Lage, sich der Aufgabe, die ihnen durch das historische Telos gegeben ist, zu
stellen, so würde dies zu den in Husserls Augen erschreckenden Konsequenzen
führen, dass die Geschichte keinen Sinn hätte, dass das europäische
Menschentum keine "`absolute Idee"' in sich trüge und "`ein bloß
anthropologischer Typus wie `China' oder `Indien' "' wäre, und dass das
"`Schauspiel der Europäisierung aller fremden Menschheiten"'\footnote{Husserl,
  Krisis, S. 16.} nicht zum Sinn der Geschichte gehören würde. (Für Husserl,
der seine besten Mannesjahre im Zeitalter der Kolonialherrschaft verlebt hat,
war die "`Europäisierung aller fremden Menschheiten"', wie man sieht, noch
nicht mit der Vorstellung bitteren Unrechts verknüpft, so dass sich ihm auch
nicht die Frage stellte, was wohl die "`fremden Menschheiten"' von seiner
Geschichtsphilosophie halten würden.)  Die Verantwortung der Philosophen ist
denn auch denkbar groß, denn die "`eigentlichen Geisteskämpfe des europäischen
Menschentums als solchen spielen sich als {\it Kämpfe der Philosophien}
ab"'\footnote{Husserl, Krisis, S. 15.  (Hervorhebungen im Original.)}, und die
Philosophen sind gar "`{\it Funktionäre der Menschheit}"'.\footnote{Husserl,
  Krisis, S. 17.  (Hervorhebungen im Original.)}

\subsection{Voegelins Kritik des Husserlschen Geschichtsbildes}

Voegelin teilt seine Kritik an Husserls Krisis-Schrift in einem Brief an
Alfred Schütz mit, den er später in seinem Werk "`Anamnesis"' veröffentlicht
hat.\footnote{Brief an Alfred Schütz über Edmund Husserl, 17.  September 1943,
  in: Voegelin, Anamnesis, S. 21-36.} Er war von Husserls Schrift
einerseits sehr positiv beeindruckt. Ihn überzeugte vor allem Husserls
Darstellung des "`Physikalismus"'. Schließlich hätte er darin auch eine
Bestätigung seiner eigenen Kritik an der Verabsolutierung einzelner
Seinsbereiche sehen können. Auch Husserls Positivismuskritik, seine Klage
darüber, dass die positivistisch reduzierten Wissenschaften keine Orientierung
für die drängenden Lebensfragen der Zeit zu bieten vermöchten, liegt genau auf
Voegelins Linie. Lobend äußert sich Voegelin zudem über die von Husserl an
Descartes herausgearbeitete subtile Differenzierung zwischen transzendentalem
und psychologischem Ego. Enttäuscht war er andererseits von Husserls
fast rein erkenntnistheoretischem Ansatz. Nicht nur, dass Voegelin selbst die
erkenntnistheoretischen Probleme nicht für die wirklich wichtigen
philosophischen Fundamentalprobleme hält, was noch als eine Frage bloßer
Vorlieben abgetan werden könnte, sondern Voegelin ist darüber hinaus der
Ansicht, dass erkenntnistheoretische Fragen nicht isoliert betrachtet werden
können.\footnote{Vgl. Voegelin, Anamnesis, S. 21-22.} Allerdings führt er in
diesem Brief an Schütz keine näheren Gründe dafür an.
 
Den größten Teil des Briefes an Schütz füllt jedoch die Kritik an zwei
Aspekten von Husserls Schrift aus, die Voegelin ganz und gar nicht gefielen:
Die Hinwendung Husserls zur Geschichte und Husserls stiefmütterliche
Behandlung von Descartes' dritter und den folgenden Meditationen. Nicht dass
Voegelin an einer Hinwendung zur Geschichte in der Absicht philosophischer
Selbstbesinnung an sich etwas auszusetzen gehabt hätte, aber in der Art und
Weise, wie sich Husserl des Themas Geschichte in der "`Krisis"'-Schrift
annimmt, konnte Voegelin nur zu gut einige der fatalen Züge wiedererkennen,
die ihm von seiner Auseinandersetzung mit den neuzeitlichen
Geschichtsideologien her wohlbekannt waren. An Descartes verkennt Husserl
nach Voegelins Ansicht vollkommen den Zweck der Meditationen, der
entsprechend der christlichen Tradition, welche Descartes, wie Voegelin meint,
aufgreift, nicht in argumentativer Begründung sondern in meditativer
Besinnung liegt und daher auch nicht argumentativ angreifbar ist.

An Husserls Behandlung der Geschichte missfällt Voegelin nun zweierlei: Zum
einen entspricht die von Husserl vorgenommene Auswahl historisch wichtiger
Epochen (griechische Antike, Neuzeit von Descartes bis Kant, Phänomenologie)
nicht Voegelins Geschmack. Zum anderen lehnt Voegelin die kollektivistischen
Züge von Husserls Geschichtsinterpretation ab.

Die Auswahl historischer Epochen bei Husserl erscheint Voegelin deshalb so
mangelhaft, weil sie nach seiner Ansicht erhebliche Lücken enthält. So ist
weder das christliche Mittelalter in Husserls Darstellung enthalten, noch wird
die Philosophie des Deutschen Idealismus angemessen historisch
gewürdigt.\footnote{In Husserls "`Krisis"' erscheint der Deutsche Idealismus
  nur als Annex zur Philosophie Kants. Vgl. Husserl, Krisis, S. 109-112.} Von
einer ernsthaften Berücksichtigung nicht-europäischer Kulturkreise kann schon
gar keine Rede sein. Damit fallen aber einige Abschnitte der
Menschheitsgeschichte weg, welche Voegelin für überaus bedeutend
hielt.\footnote{Vgl. Voegelin, Anamnesis, S. 22-23.}

Es stellt sich die Frage, ob Voegelins Kritik in diesem Punkt berechtigt ist.
Wäre Husserl verpflichtet gewesen, im Rahmen einer Einleitung in die
Phänomenologie nicht nur die Phasen der Philosophiegeschichte anzusprechen,
die die Vorgeschichte der Phänomenologie bilden, sondern alle Phasen, welche
für die geistige Entwicklung der Menschheit insgesamt bedeutsam waren? Wenn
man nicht gerade einen dogmatischen Holismus vertritt, zu welchem Voegelin
gelegentlich neigt, so würde eine Einleitung in die Phänomenologie es
höchstens erfordern, die Vorgeschichte der Phänomenologie darzustellen, nicht
aber, auf die Geistesgeschichte im Ganzen einzugehen oder auch nur auf
Zusammenhänge zur allgemeinen Geistesgeschichte hinzuweisen.
% Es ist ja auch nicht erforderlich, z.B. in einer Geschichte der
% Naturwissenschaften den Auszug aus Ägypten, den Apostel Paulus oder den
% heiligen Thomas von Aquin zu erwähnen, denn keine dieser Personen und
% Ereignisse hat einen Beitrag zur Entwicklung der Naturwissenschaften
% geleistet.

Husserls Geschichtsdarstellung erscheint jedoch in einem ganz anderen Licht,
wenn man berücksichtigt, dass es Husserl auch und vor allem um den Sinn der
Geschichte überhaupt ging und dass er in der Geschichte ein Telos zu finden
meinte, welches für alle Menschen verbindlich sein sollte und nicht nur für
die Phänomenologie betreibenden Philosophen, wiewohl diese Philosophen durch
ihre schmeichelhafte Führungsrolle als "`Funktionäre der
Menschheit"'\footnote{Husserl, Krisis, S. 17.} noch einmal besonders
hervorgehoben werden. Angesichts dieses hohen geschichtsphilosophischen
Anspruchs tadelt Voegelin zu Recht das armselige Bild der Geistesgeschichte
der Menschheit, welches Husserl zeichnet. Die Missachtung wichtiger Epochen der
Menschheitsgeschichte kann bei diesem Anspruch nicht mehr als thematische
Beschränkung entschuldigt werden.

Doch die Ablehnung von Husserls Geschichtsbild ist noch grundsätzlicher, denn
der Anspruch, das Telos der Geschichte bestimmen zu können, ist unabhängig von
der Tiefe und Vollständigkeit der Geschichtsdarstellung, die diesen Anspruch
untermauern soll, als solcher höchst fragwürdig. Er mündet bei Husserl, so wie
Voegelin es nennt, in eine "`averroistische[..]
Spekulation"'.\footnote{Voegelin, Anamnesis, S. 26.} Unter "`averroistischen
Spekulationen"' versteht Voegelin Varianten des Grundgedankens vom Vorrang des
Allgemeinen vor dem Besonderen. Die ungewöhnliche Bezeichnung leitet Voegelin
vom Namen des mittelalterlichen mohammedanischen Philosophen Averroes ab, der
neben Avicenna einer der bedeutendsten Vermittler des Aristoteles und der
antiken Philosophie war. Durch ihn fand die aristotelische Philosophie Eingang
in das Denken des christlichen Mittelalters. Was Voegelin "`averroistische
Spekulation"' nennt, ist denn auch eine Vorstellung, die schon in der antiken
Philosophie ihre Grundlage hat.\footnote{Vgl. Voegelin, Anamnesis, S. 26.} Es
handelt sich dabei -- soweit man es Voegelins Text entnehmen kann -- um eine
sehr allgemeine und etwas vage metaphysische Vorstellung, nach der es einen
Primat der Wahrheit, des Wertes und der Wirklichkeit des Allgemeinen vor dem
Speziellen, der Klasse vor dem Individuum oder des Ganzen vor dem Teil gibt.
Diese Grundvorstellung kann in den verschiedensten Formen und bezogen auf die
verschiedensten Gegenstände auftauchen. Auf gesellschaftspolitischer Ebene
führt dieser Gedanke sehr rasch zum Kollektivismus. Besonders problematisch
wird die "`averroistische Spekulation"', wenn sie im Verein mit einem
Exklusivitätsprinzip auftritt, nach welchem bestimmte Gruppen oder Individuen
aus dem maßgeblichen Kollektiv ausgeschlossen werden.\footnote{Vgl. Voegelin,
  Anamnesis, S. 26-27. -- Vgl. auch Eric Voegelin: Der autoritäre Staat. Ein
  Versuch über das österreichische Staatsproblem, Wien / New York 1997 (zuerst
  1936), im folgenden zitiert als: Voegelin, Autoritärer Staat, S. 25-26.}

Der averroistisch-spekulative Charakter von Husserls Geschichtsbild wird
besonders deutlich, wenn Husserl das Telos der Geschichte als eine durch den
Einzelnen "`{\it hindurchgehende} Willensrichtung"'\footnote{Husserl, Krisis,
  S. 78.} darstellt. Der Einzelne wird zu einem bloßen Agenten oder Medium
jener höheren Willensrichtung, auf die allein es ankommt. Voegelin spricht
deshalb auch von dem "`kollektivistische[n] Telos"'\footnote{Voegelin,
  Anamnesis, S. 27.} Husserls. Auch die Beschränkung auf ein maßgebliches
Kollektiv, welches dieses Telos vertritt, kommt bei Husserl in der
Einschränkung des eigentlichen Menschentums auf das europäische Menschentum
vor. In historischer Perspektive drückt sich nach Voegelin dieser Gedanke
bei Husserl dadurch aus, dass der überwiegende Teil der Menschheitsgeschichte
schlicht übergangen wird zugunsten der vermeintlich wesentlichen Etappen,
welche die Entfaltung des "`Telos"' verkörpern.\footnote{Vgl. Voegelin,
  Anamnesis, S. 27-28.}

Aber Husserl ist für Voegelin nicht nur "`Fortschrittsphilosoph im besten
Stile der Reichsgründerzeit"'.\footnote{Voegelin, Anamnesis, S. 28.} Darüber
hinaus erblickt Voegelin in Husserls durch die beiden Wendemarken der
Urstiftung und der Endstiftung unterteilten Geschichte jenes
Drei-Phasen-Geschichtsbild, welches, von der christlichen Heilsgeschichte
herstammend, Eingang in so viele Geschichtsideologien der Neuzeit gefunden
hat.  Die messianische Endzeit, die in diesen Geschichtsideologien anders als
in der christlichen Heilslehre nicht überzeitlich sondern geschichtsimmanent
verstanden wird, beginnt bei Husserl mit der Endstiftung. Natürlich hütet sich
Voegelin, Husserl mit gewalttätigen politischen Bewegungen wie dem Kommunismus
oder dem Nationalsozialismus in eine Reihe zu stellen. Aber die
Strukturverwandtschaft von Husserls Geschichtsbild und manchen modernen
Geschichtsideologien scheint ihm doch unverkennbar.\footnote{Vgl. Voegelin,
  Anamnesis, S. 28-31.}

Einige Interpreten der Husserlschen Philosophie versuchen Husserl vor dem
Verdacht der Geschichtsideologie in Schutz zu nehmen, indem sie behaupten,
dass Husserl nur als Phänomenologe innerhalb der "`Epoché"', jener
phänomenologischen Operation der Konzentration auf das Phänomen in seiner
Selbstgegebenheit und unter Absehung von dessen
Wirklichkeitsprätentionen,\footnote{Vgl. Edmund Husserl: Die phänomenologische
  Methode. Ausgewählte Texte I (Hrsg. von Klaus Held), Stuttgart 1985, S.
  141-143.} gesprochen habe. Seine geschichtsphilosophischen Ausführungen
seien daher eher als unverbindliche Besinnungen persönlicher Art auf die ganz
privaten Absichten und Zwecke des Phänomenologen Husserl zu
verstehen.\footnote{Vgl.  Gilbert Weiss: Theorie, Relevanz, Wahrheit. Zum
  Briefwechsel zwischen Eric Voegelin und Alfred Schütz (1938-1959), München
  1997, S. 24-28. -- Vgl. die Einleitung von Elisabeth Ströker in: Husserl,
  Krisis, S. XXIX.} Diese Art der Apologie ist jedoch nicht überzeugend, denn
die phänomenologische Epoché dient nicht minder der Gewinnung
allgemeinverbindlicher Resultate als irgendeine wissenschaftliche
Forschungsmethode. Idealiter liefert sie sogar Ergebnisse von "`apodiktischer
Evidenz"'. Selbst wenn Husserl, ohne es übrigens irgendwo zu erwähnen,
innerhalb der Epoché gesprochen hätte, so würden seine Äußerungen dadurch
keineswegs akzeptabler.  Husserl hätte dann, statt zu behaupten, die
Geschichte habe ein Telos, lediglich behauptet, die Geschichte stelle sich uns
notwendig so dar, als habe sie ein Telos, was aber nicht weniger fragwürdig
wäre.

Man mag einwenden, dass mit diesem recht kritischen Ergebnis das
geistesgeschichtliche Verdienst von Husserls Krisis-Schrift ungenügend
gewürdigt wird. Geistesgeschichtlich gesehen, stellt Husserls Krisis-Schrift
einen höchst bemerkenswerten Versuch einer Verbindung von Traditionalismus und
Rationalismus, von geschichtlichem Denken und systematischer Philosophie, von
religiösem Patriarchalismus und Vernunfterkenntnis dar, eine Synthese, die trotz
der Wilhelminischen Einlassungen Beachtung verdient. Allerdings zeigt
auch gerade Voegelins Kritik, dass diese Synthese nicht aufgeht.

\subsection{Voegelins Einwände gegen die Fortschrittsgeschichte}

Über die Verfehltheit von Ideologien einer geschichtlichen Endzeit lässt sich
Voegelin in seinem Brief an Alfred Schütz nicht weiter aus. (Sie ist ohnehin
offensichtlich genug.) Was hat Voegelin aber daran auszusetzen, die
Geschichte, so wie es bei Husserl geschieht, als eine Geschichte des
Fortschritts zu schreiben? Für Voegelin spielt dabei sowohl ein moralisches
als auch ein eher wissenschaftliches Motiv eine Rolle. Moralisch kritikwürdig
erscheint Voegelin die Inhumanität, die darin liegt, die vergangenen Epochen
und das Streben der damals lebenden Menschen nur als Mittel zum Zweck für die
Gegenwart zu betrachten. Wissenschaftliche Schwierigkeiten entstehen für
Voegelin dadurch, dass vergangene Epochen nicht angemessen verstanden werden
können, wenn in ihnen nur eine Vorstufe der Gegenwart gesehen wird.

Die moralische Problematik der Fortschrittsphilosophie erläutert Voegelin
unter Rückgriff auf Kant. Kant teilte mit vielen anderen Aufklärern die
Ansicht, dass es in der Geschichte einen Fortschritt zum Besseren gibt, so
dass sich der Zustand der menschlichen Gesellschaft immer mehr, wenn auch
niemals endgültig, einem moralischen Optimum (jeder handelt gut und keinem
geschieht ein Unrecht) annähert. Zugleich äußert Kant jedoch auch sein
"`Befremden"' darüber, dass die späteren Generationen von allen Fortschritten
der vorhergehenden profitieren, welche ihrerseits, obwohl sie denselben
Beitrag zum Fortschritt geleistet haben, nicht in gleichem Maße die Vorteile
davon genießen können.\footnote{Vgl. Immanuel Kant: Idee zu einer allgemeinen
  Geschichte in weltbürgerlicher Absicht (Dritter Satz), in: Immanuel Kant:
  Schriften zur Geschichtsphilosophie, Stuttgart 1985, S. 21-39 (S. 25). In
  Voegelins Kant-Interpretation tritt gegenüber Kant eine leichte
  Bedeutungsverschiebung ein. Während Voegelin hier eine Frage des Sinns
  sieht, geht es bei Kant (wenigstens dem Sachzusammenhang nach, wenn auch
  noch andere Motive im Hintergrund eine Rolle spielen mögen) eher um eine
  Frage des materiellen Ausgleichs.  Dies hat natürlich auch Folgen für die
  Interpretation der geistesgeschichtlichen Rolle Kants, die hier jedoch nur
  kurz angedeutet werden können: Es erscheint grundsätzlich fragwürdig, in
  Kants Geschichtsphilosophie (bzw. in den Geschichtsvorstellungen der
  Aufklärer überhaupt) eine "`averroistische Konzeption"' zu sehen. Die
  Geschichtsphilosophie Kants war durchaus keine Geschichtssinntheorie (wie
  die Geschichtsphilosophien des Deutschen Idealismus), denn nicht die
  Geschichte verleiht bei Kant dem Leben und Schaffen des Einzelnen Sinn und
  Wert (und auch nicht die Glückseligkeit, die nur eine Belohnung ist, auf die
  er nach dem Tode hoffen darf), sondern die Erfüllung der Pflicht (meine
  Interpretation). Kants Fortschrittsphilosophie war der Ausdruck der
  optimistischen Hoffnung, dass sich das Gute einmal durchsetzen wird, aber das
  Gute ist bei Kant (noch) nicht dadurch definiert, wer in der Geschichte
  siegreich bleibt.  Die grundsätzliche Möglichkeit, Geschichte als
  Fortschrittsgeschichte zu schreiben, ohne in "`averroistische
  Spekulationen"' zu verfallen, kann auch Voegelin nicht leugnen, sonst müsste
  er sich wegen der Fortschritte der spirituellen Ausbrüche, von denen "`Order
  and History"' handelt, selbst der "`averroistischen Spekulation"'
  bezichtigen. Voegelins Bild einer Kontinuität von der aufklärerischen
  Fortschrittsphilosophie (Geschichtsideologie ist Fortschrittsphilosophie
  minus Humanität plus Endzeitglaube) zu den modernen Geschichtsideologien
  erscheint deshalb teilweise fragwürdig.} Voegelin erblickt in Kants
Befremden eine humane Hemmung, die früheren Generationen nur als Mittel zum
Zweck der Verwirklichung eines geschichtlichen Telos zu sehen, welches bei
Kant in der Vervollkommnung der Vernunftanlagen besteht. Bei Husserl fehlt
diese Humanität und zudem tritt die "`Endstiftung"' anders als Kants
Vervollkommnung des Vernunftgebrauchs tatsächlich in der Geschichte ein.
Diese beiden Punkte markieren für Voegelin den Übergang von der
averroistischen Konzeption aufklärerischer Fortschrittsgeschichte zu den noch
militanteren averroistischen Spekulationen, die sich in den modernen
Geschichtsideologien und, folgt man Voegelin, sogar in seriösen historischen
Untersuchungen wie Otto Gierkes Genossenschaftsrecht finden.  Husserls
Geschichtsbild stellt für Voegelin deshalb eine durchaus zeittypische
Erscheinung dar.\footnote{Vgl.  Voegelin, Anamnesis, S. 28-30.}

Die weniger ethische als wissenschaftliche Problematik dieser Art von
Geschichtsdarstellung besteht für Voegelin darin, dass der Historiker "`die
eigene geistige Position, mit ihrer historischen Bedingtheit,
verabsolutiert"'\footnote{Voegelin, Anamnesis, S. 31.}  und auf die
historischen Fakten nur zurückgreift, um die eigene Position zu stützen, ohne
dabei jemals verstehend in das historische Material einzudringen. Bei Husserl
tritt diese Verabsolutierung in besonders krasser Form auf, da Husserl sich
nach Voegelins Ansicht gegen die Möglichkeit empirischer Kritik systematisch
abschirmt. Voegelin spielt hier wahrscheinlich auf Husserls theoretische
Vorgabe an, dass der Sinn der philosophischen Positionen der Vergangenheit
nicht aus den Selbstzeugnissen der Denker, sondern nur durch die Heraushebung
einer erst rückblickend aus der Gegenwart erkennbaren latenten
"`Willensrichtung"' zu bestimmen sei.\footnote{Vgl.  Voegelin, Anamnesis,
  S. 31. -- Vgl. Husserl, Krisis, S. 78-80.}

Wie sieht für Voegelin aber die Alternative zu diesen Formen von
Geschichtsklitterung aus? Nach Voegelins Überzeugung ist es die Aufgabe des
Historikers, in der Geistesgeschichte "`jede geschichtlich geistige Position
bis zu dem Punkt zu durchdringen, an dem sie in sich selbst ruht, d.h. in dem
sie in den Transzendenzerfahrungen des betreffenden Denkers verwurzelt
ist."'\footnote{Voegelin, Anamnesis, S. 31.} Es kommt weiterhin darauf an,
"`die geistig-geschichtliche Gestalt des andern bis zu ihrem Transzendenzpunkt
zu durchdringen und in solcher Durchdringung die eigene Ausformung der
Transzendenzerfahrung zu schulen und zu klären."'\footnote{Voegelin,
  Anamnesis, S. 31.} Die recht verstandene Geistesgeschichte verfolgt also
zwei Ziele: Verstehen der geistigen "`Gestalten"' der Vergangenheit und
Klärung der eigenen Beziehung zur Transzendenz. \label{Selbstzeugnisse1} Das
Verstehen hat dabei strikt am "`Leitfaden"' der "` `Selbstzeugnisse' der
Denker"'\footnote{Voegelin, Anamnesis, S. 32.} zu erfolgen. Die Klärung des
Selbstverständnisses durch das "`geistesgeschichtliche Verstehen"' zielt
letztlich auf eine "`Kathar[s]is, eine {\it purificatio} im mystischen Sinn,
mit dem persönlichen Ziel der {\it illuminatio} und der {\it unio
  mystica}"'.\footnote{Voegelin, Anamnesis, S. 31.}  Wird dieses
"`geistesgeschichtliche Verstehen"' systematisch ausgeübt, so kann es "`zur
Herausarbeitung von Ordnungsreihen in der geschichtlichen Offenbarung des
Geistes führen."'\footnote{Voegelin, Anamnesis, S. 32.}

% Das "`philosophische
% Ziel"' der Geistesgeschichte besteht nämlich für Voegelin darin, "`jede
% geschichtlich geistige Position bis zu dem Punkt zu durchdringen, an dem sie
% in sich selbst ruht, d.h. in dem sie in den Transzendenzerfahrungen des
% betreffenden Denkers verwurzelt ist."'

Es ist zu berücksichtigen, dass Voegelin dies 1943, also noch lange vor seinem
geschichtlichen Hauptwerk "`Order and History"', geschrieben hat. Voegelins
Ausführungen sind also eher noch als ein frühes Programm zu
verstehen.\footnote{Vgl. Jürgen Gebhardt: Toward the Process of universal
  Mankind. The Formation of Voegelin's Philosophy of History, in: Ellis Sandoz
  (Hrsg.): Eric Voegelins Thought. A critical appraisal, Durham N.C. 1982, S.
  67-86, S. 78.} Dennoch kann die Frage aufgeworfen werden, ob dieses Programm
eine gangbare Alternative zu den von Voegelin abgelehnten "`averroistischen
Konzeptionen"' von Geschichte darstellt. In dieser Hinsicht fällt auf, dass
Voegelins Programm bereits sehr erhebliche Vorentscheidungen über das Wesen
der Geistesgeschichte enthält.  Voegelin unterstellt, dass jeder bedeutsamen
geschichtlichen Gestalt des Geistes eine Transzendenzerfahrung zu Grunde
liegt. Aber nicht alle Denker gründen ihr Denken auf Transzendenzerfahrungen.
Die meisten Philosophen gelangen zu ihren Resultaten durch Überlegungen,
welche mit einer Auslegung von Transzendenzerfahrungen nichts gemein haben. Es
könnte nun behauptet werden, dass die Nichtbeachtung der Transzendenz
ebenfalls eine bestimmte, wenn auch eine deformierte Beziehung zur
Transzendenz repräsentiert. Wird dies behauptet, so wird jedoch gleichzeitig
eine andere methodische Forderung Voegelins vernachlässigt, nämlich die,
"`jede geschichtlich geistige Position bis zu dem Punkt zu durchdringen, an
dem sie in sich selbst ruht"',\footnote{Voegelin, Anamnesis, S. 31.} denn
durch Betrachtung eines nicht-religiösen Denkers unter dem Gesichtspunkt der
Transzendenzerfahrungen werden an diesen Denker völlig heteronome Maßstäbe
herangetragen. Voegelin stellt hier also zwei einander widersprechende
methodische Forderungen auf: Zum einen, die Denker der Vergangenheit strikt
auf Grundlage ihrer Selbstzeugnisse zu erfassen, und zum anderen, jede
geistige Position in der Vergangenheit zwingend als Ausdruck einer
Transzendenzerfahrung zu verstehen.

Betrachtet man dieses frühe historische Programm im Hinblick auf sein späteres
geschichtliches Werk, dann fallen einige Abweichungen auf: So ließ sich etwa
die Forderung, vom Selbstverständnis der Denker der Vergangenheit auszugehen,
nicht durchhalten. Wenigstens für bestimmte philosophische Systeme gibt
Voegelin diese Forderung später auch explizit auf.\footnote{Vgl. Voegelin,
  Anamnesis, S. 310. -- Siehe auch Seite \pageref{Selbstzeugnisse2} in diesem
  Buch.}  Weiterhin kann die scharfe Kritik, die Voegelin etwa an Otto
Gierkes "`phantastischer Vergewaltigung Bodins"'\footnote{Voegelin, Anamnesis,
  S. 30.}  übt, auch gegen Voegelins eigene Klassikerinterpretationen gekehrt
werden, die nicht selten eher kongenial als historisch und philologisch
zuverlässig sind.  Sogar der Vorwurf der "`averroistischen Spekulation"'
könnte gegen Voegelin selbst gerichtet werden, wenn er in seinen späteren
Schriften das menschliche Bewusstsein an einem Prozess partizipieren lässt,
"`durch den die Wahrheit der Realität sich ihrer selbst bewusst
wird"',\footnote{Eric Voegelin: Äquivalenz von Erfahrungen und Symbolen in der
  Geschichte, in: Eric Voegelin, Ordnung, Bewußtsein, Geschichte, Späte
  Schriften (Hrsg. von Peter J. Optiz), Stuttgart 1988, S. 99-126 (S. 123).}
was von Husserls durch die konkreten Philosophen hindurchgehender
Willensrichtung nicht allzu weit entfernt ist.  Voegelins eigene
Geschichtskonstruktion ist in diesen Punkten derjenigen Husserls, die er zu
recht kritisiert, näher, als dies die Entschiedenheit seiner Vorwürfe erwarten
lassen sollte.

% \footnote{Für Voegelin war es offenbar
%   selbstverständlich, daß einen Menschen verstehen heißt, seine
%   Transzendenzerfahrungen nachzuvollziehen. Es gibt hier eine Parallele zur
%   Einleitung von Order and History II., wo Voegelin die Frage nach dem
%   Zusammenhang der Menschheit mit der allen gemeinsamen Mission der Suche nach
%   Wahrheit (Quest for truth) beantwortet. Er behauptet sogar, daß dies die
%   einzige Möglichkeit sei und schließt damit andere denkbare Alternativen
%   (z.B. gleiche Sehsüchte und Wünsche, Nöte und Freuden weltlicher Art) von
%   vornherein aus.}

\subsection{Voegelins Descartes-Deutung}

Dass Voegelin sich große Interpretationsfreiheiten erlaubt, wird auch an
seiner Auseinandersetzung mit Husserls Descartes-Bild deutlich, denn die
rationale und wissenschaftliche Ausrichtung von Descartes' Denken lässt im
Grunde wenig Raum für die Eindrücke von Transzendenzerfahrungen. Für Voegelin
greift Husserls Descartes-Interpretation zu kurz, weil Husserl der Philosophie
des Descartes eine rein erkenntnistheoretische Bedeutung unterstellt, und weil
Husserl nach Voegelins Ansicht den tieferen Sinn von Descartes Gottesbeweis in
der dritten Meditation missversteht.

Was die rein erkenntnistheoretische Deutung des Descartes durch Husserl
betrifft, so gilt dasselbe, was bereits über Voegelins Kritik an Husserls
Geschichtsbild gesagt wurde: Sofern es Husserl um eine Einleitung in die
Phänomenologie geht, ist es sein gutes Recht, die Aspekte der Philosophie von
Descartes herauszugreifen, die für die Phänomenologie von Bedeutung sind, und
dies sind nun einmal die erkenntnistheoretischen. Da Husserls "`Krisis"' aber
noch von wesentlich höheren Aspirationen getragen wird, sind Einwände gegen
das Herausgreifen bestimmter Einzelaspekte der Philosophie des Descartes
grundsätzlich legitim.

Etwas anders verhält es sich jedoch mit Husserls Vernachlässigung des
Gottesbeweises in der dritten Meditation von Descartes "`Meditationen über die
Grundlagen der Philosophie"'. Husserl erwähnt in der "`Krisis"' nur kurz, dass
der Gottesbeweis falsch sei, und geht auf die dritte und die folgenden
Meditationen gar nicht weiter ein.\footnote{Vgl. Husserl, Krisis, S. 82.}  Er
scheint sich hier an eine damals wie heute geläufige Lesart zu halten, nach
der die dritte bis sechste Meditation von Descartes noch durch und durch
scholastisch sind, und philosophisch Belangvolles nur in den ersten beiden
Meditationen zu finden ist.\footnote{Vgl. Bertrand Russell: A History of
  Western Philosophy, London, Sidney, Wellington 1990, im folgenden zitiert
  als: Russell, History of Western Philosophy, S. 550.} Auch Voegelin sieht in
Descartes' Meditationen ein durchaus traditionelles Unternehmen. Für ihn sind
die gesamten "`Meditationen"' des Descartes eine Spielart der christlichen
Meditation, wie sie seit Augustinus insbesondere bei den mystischen Denkern
üblich war. Wenn man Voegelin Glauben schenkt, so war es das Ziel der
Meditationen von Descartes, wie in der christlichen Meditation üblich, in der
Abkehr von der Welt den Kontakt zur Transzendenz als der höchsten Wirklichkeit
zu finden. Das Neue bei Descartes besteht nach Voegelin darin, dass Descartes
-- anders als seine Vorläufer -- die Meditation nicht aus einer Haltung der
Verachtung der Welt heraus unternimmt, sondern in der Absicht, sich durch den
Kontakt zur höchsten Realität der Realität bzw.  der Objektivität der Welt zu
versichern. Der Gottesbeweis in der dritten Meditation ist, Voegelin zufolge,
in diesem Zusammenhang nicht als logische Beweisführung, sondern als
sekundärer, in der Stilform der {\it demonstratio} gefasster Ausdruck der als
solcher unmittelbaren und eines Beweises nicht bedürftigen Gotteserfahrung zu
sehen.\footnote{Vgl. Voegelin, Anamnesis, S. 32-35.}

Voegelins Descartes-Interpretation ergibt sich nicht ganz zwanglos aus dem
Text der "`Meditationes"', da dort von der Gotteserfahrung nur sehr am Rande
die Rede ist. Wäre mit der Erfahrung Gottes schon seine Existenz und darüber
hinaus die Existenz der Welt mitgegeben, so erscheint es ganz und gar unnötig,
dass Descartes versucht, mit Hilfe scholastischer Schlussweisen, deren
Falschheit Voegelin gar nicht bestreitet, zu zeigen, dass die Vorstellung
Gottes anders als alle anderen Vorstellungen die Existenz des Vorgestellten
impliziert.\footnote{Vgl. René Descartes: Meditationen über die Grundlagen der
  Philosophie, Hamburg 1993, S. 36ff.} Darüber hinaus unterscheidet sich
Descartes' Text nicht nur in der Intention der Weltvergewisserung sondern auch
in der Art und Weise der Darstellung recht deutlich von den "`ekstatischen
Konfessionen"'\footnote{So der Titel einer Sammlung mystischer
  Erfahrungsberichte, die teilweise mit den Mitteilungen "`Cloud of
  Unknowing"', die von Voegelin in diesem Zusammenhang angeführt wird,
  vergleichbar sind (z.B. der Auszug aus der Erzählung des Tewekhul-Beg,
  Schüler des Moll\^a-Sch\^ah, über sein mystisches Noviziat, S. 47-49.) in:
  Buber, Martin (Hrsg.): Ekstatische Konfessionen, Leipzig 1921.} von
Mystikern wie etwa dem anonymen Autor der von Voegelin zum Vergleich
herangezogenen "`Cloud of Unknowing"'.\footnote{Vgl. A Book of Contemplation
  wich is called the Cloud of Unknowing, in which a Soul is oned with God.
  (ed. Evelyn Underhill, 2nd ed.  John M. Watkins), London 1922, auf:
  http://www.ccel.org/u/unknowing/cloud.htm (Host: Christian Classics Ethereal
  Library at Calvin College. Zugriff am: 5.4.2000). -- Ein Vergleich mit
  Descartes, wie Voegelin ihn anstellt (vgl. Voegelin, Anamnesis, S. 33)
  bietet sich noch am ehesten an, wenn man das Ende des vierten und das fünfte
  Kapitel dieses Werkes dem Beginn von Descartes' dritter Meditation
  gegenüberstellt. Aber die oberflächlichen Ähnlichkeiten, die sich dabei
  auf\/finden lassen, können kaum über den himmelweiten Unterschied in Inhalt,
  Gegenstand, Absicht, Ausführung und Ziel dieser beiden Werke
  hinwegtäuschen.} Es fällt daher nicht unbedingt leicht, die "`Meditationen"'
unter dieser Art von Literatur einzureihen. Aber selbst wenn man einmal
annimmt, dass die "`Meditationen"' von Descartes nur eine mit anderen Mitteln
vorgenommene Artikulation derselben mystischen Transzendenzerfahrung sind, so
stellt sich die Frage, ob damit irgendetwas gewonnen ist. Wenn schon der
Versuch, die Existenz der Außenwelt argumentativ zu beweisen, fehlgeschlagen
ist, dann kann die Existenz der Außenwelt durch den Rückgriff auf eine
Transzendenzerfahrung ebensowenig begründet werden. Denn es ist zwar
vorstellbar, dass eine Transzendenzerfahrung ein sehr starkes Vertrauen in das
Sein der Welt einflößt, aber außer der bloßen Erfahrung einer Transzendenz ist
nicht auch das Sein der Transzendenz selbst und außerdem noch das objektive
Sein der Welt in dieser Erfahrung mitgegeben, da auch bei einer
Transzendenzerfahrung eine Täuschung denkbar ist, genauso wie eine
Halluzination oder Sinnestäuschung bei der Sinneserfahrung. Die Erfahrung der
Transzendenz, was immer das für eine Erfahrung auch sein mag, bleibt deshalb
für das erkenntnistheoretische Problem der Existenz der Außenwelt ohne Belang.
Bloße Gefühle der Gewissheit, wie sie sich in einer Meditation einstellen
mögen, sind eben noch keine Gewissheit.

Einzuräumen ist jedoch, dass Voegelin einen Schwachpunkt von Husserls Theorie
der "`Konstitution"' des Seins durch Leistungen des Ego trifft, wenn er in
diesem Zusammenhang die Frage aufwirft, woher das Ego die Funktion bekommt,
"`aus der Subjektivität die Objektivität der Welt zu
fundieren"'\footnote{Voegelin, Anamnesis, S. 36. -- Vgl. Husserl,
  Cartesianische Meditationen, S. 84-91 (§ 40-41).}. Husserls
phänomenologische Methode gerät in der Tat an ihre Grenzen, wenn es darum
geht, das Selbstsein anderer Menschen oder auch nur von Dingen zu begründen.

%%% Local Variables: 
%%% mode: latex
%%% TeX-master: "Main"
%%% End: 











