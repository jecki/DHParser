
%%% Local Variables: 
%%% mode: latex
%%% TeX-master: "Main"
%%% End: 

\chapter{Vorwort zur Buchausgabe}

Das Interesse, das der Politikphilosoph Eric Voegelin im deutschsprachigen
Raum findet, scheint in den letzten Jahren mehr und mehr zu wachsen. Des
öfteren schon waren in der jüngsten Zeit Artikel über Eric Voegelin sogar in
den Feuilletons überregionaler Tageszeitungen zu lesen,\footnote{Jüngst zum
  Beispiel im Ressort Literatur und Kunst der Neuen Züricher Zeitung vom 28.
  April 2007: Peter Cornelius Mayer-Tasch: Auf der Suche nach einer Ordnung
  der historischen Ordnungen. Das Hauptwerk des Politikwissenschaftlers und
  Geschichtsphilosophen Eric Voegelin.} was darauf schließen lässt, dass auch
über den engen Kreis von Fachwissenschaftlern hinaus ein breiteres Interesse
an Voegelins politischer Philosophie besteht.  Das gesteigerte Interesse an
Voegelin ist nicht verwunderlich, hat Voegelin sich in seinen Werken doch
intensiv mit dem Verhältnis von Religion und Politik beschäftigt, einem Thema,
das gerade heute wieder besonders aktuell ist.

Leider ist aber gerade die Art und Weise, wie Eric Voegelin das Verhältnis von
Religion und Politik bestimmt, in keiner Weise geeignet, die Schwierigkeiten,
vor denen wir damit in der Gegenwart stehen, zu lösen.  Voegelins ``Lösung''
schafft eher noch zusätzliche Probleme, als dass sie die bestehenden
Schwierigkeiten beseitigt. Weshalb das so ist, will ich an dieser Stelle kurz
erläutern.

Will man die normative Frage untersuchen, in welchem Verhältnis Religion und
Politik zueinander stehen {\em sollten}, so dass weder die politische Ordnung
Schaden nimmt noch die Religion in unnötiger Weise beeinträchtigt wird,
dann muss man zwei verschiedene Unterscheidungen sorgfältig voneinander
trennen:

\begin{enumerate}

\item Die Unterscheidung zwischen {\em richtiger} (bzw. "`wahrer"') und {\em
    falscher} Religion.

\item Die Unterscheidung zwischen {\em guter}, d.h. mit dem Erhalt politischer
  Ordnung verträglicher, und {\em schlechter}, d.h. mit dem Erhalt politischer
  Ordnung unverträglicher Religion.

\end{enumerate}

Die erste dieser beiden Unterscheidungen lässt sich wissenschaftlich-objektiv
überhaupt nicht treffen. Es gibt keine wissenschaftliche Methode, mit der wir
über die Wahrheit und Falschheit von Religionen urteilen könnten. Die
``religiösen Wahrheiten'' sind der Wissenschaft nicht zugänglich und es ist
das Beste, sie aus allen wissenschaftlichen Betrachtungen auszuklammern und
innerhalb der Wissenschaft ihnen gegenüber strikte Neutralität zu wahren.

Die zweite Unterscheidung {\em müssen} wir jedoch treffen, weil dazu eine
praktische Notwendigkeit besteht. Diese Unterscheidung hat eine ethische und
eine empirische Seite. Die ethische Seite betrifft die Werte, die unserer
Überzeugung nach das gesellschaftliche Zusammenleben regeln sollten, und die
durch die politische Ordnung geschützt werden müssen. Die Gestalt der
politischen Ordnung hängt ganz wesentlich davon ab, welche Werte wir als
wichtig erachten.  Vertritt man die Wertfreiheit in der Wissenschaft, dann
lässt sich über die Richtigkeit und Falschheit dieser Werte wissenschaftlich
ebenfalls nicht urteilen. Als Menschen müssen wir uns jedoch für bestimmte
Werte entscheiden und es ist nicht möglich abweichende Werte bei Anderen
uneingeschränkt zu tolerieren. Die empirische Seite der Unterscheidung
betrifft die Frage, wie sich bestimmte religiöse Einstellungen oder bestimmte
religiöse Praktiken kausal auf die politische Ordnung auswirken oder --
ebenfalls eine Frage kausaler Zusammenhänge -- wie gut sie mit der politischen
Ordnung verträglich sind. Die kausalen Zusammenhänge zwischen Religion und
politischer Ordnung sind einer wissenschaftlichen Methodik prinzipiell
zugänglich und ihre wissenschaftliche Untersuchung ist von größtem Interesse.

Es ist wichtig, diese beiden Unterscheidungen auseinander zu halten, weil wir
nur so die ethischen Fragen, über die wir uns streiten müssen, von den
religiös-metaphysischen Fragen, über die wir es nicht ernsthaft zum Streit
kommen lassen sollten, sorgfältig trennen können (siehe dazu auch Kapitel
\ref{Wertbegruendung}). Gerade diese Unterscheidung verwischt Voegelin aber,
was, nebenbei bemerkt, einer der Gründe für den an vielen seiner Schriften so
auffälligen polemisch-intoleranten Zug sein könnte. Voegelin meinte nämlich,
dass es erstens sehr wohl möglich ist, über die Wahrheit und Falschheit
religiöser Überzeugungen ein wissenschaftlich begründetes Urteil zu fällen,
und dass zweitens ein äußerst enger kausaler Zusammenhang zwischen der
Richtigkeit oder Falschheit der religiösen Einstellung und ihrer Wirkung auf
die politische Ordnung besteht. Wie ich in diesem Buch darzulegen versuche,
hat Voegelin sich in beiden Punkten geirrt.

Die Wahrheit oder Falschheit religiöser Überzeugungen schien für Voegelin
davon abhängig zu sein, ob sie Ausdruck intakter oder verdorbener {\em
  religiöser Erfahrungen} sind. Wie soll man aber die Intaktheit oder
Verderbtheit religiöser Erfahrungen wissenschaftlich beurteilen können?
Religiöse Erfahrungen sind doch, sollte man meinen, zunächst einmal sehr
persönliche Erfahrungen jedes Einzelnen, und wenn der eine diese religiösen
Erfahrungen macht und der andere jene, wer wollte damit rechten und sich
anmaßen zu behaupten, die Erfahrungen des einen wären echter und unverdorbener
als die des anderen? Genau das tut Voegelin aber, und die
``wissenschaftliche'' Methodik, deren er sich dazu bedient, besteht unter
anderem in bewusstseinsphilosophischen Untersuchungen, mit denen er meint, die
Natur unserer Transzendenzerfahrungen ergründen zu können. Diese
bewusstseinsphilosophischen Untersuchungen bilden den Hauptgegenstand des
vorliegenden Buches (besonders des \ref{VoegelinsBewusstseinsphilosophie}.
Kapitels), und ich glaube zeigen zu können, dass Voegelin mit seinem
bewusstseinsphilosophischen Unterfangen gescheitert ist. Es gibt keine
wissenschaftliche Methode, die die Bewertung religiöser Erfahrungen erlaubt,
und auch Voegelins Bewusstseinsphilosophie kann das nicht leisten.  Die
Ergebnisse meiner Kritik von Voegelins Bewusstseinsphilosophie sind in Kapitel
\ref{Fundamentalprobleme} und Kapitel \ref{Transzendenzerfahrungen}
zusammengefasst.

Trotz dieses Irrtums wäre Voegelins Werk für die wissenschaftliche
Auseinandersetzung über die Beziehung von Religion und Politik immer noch dann
von großer Bedeutung, wenn Voegelin wenigstens einen Beitrag zur Untersuchung
der kausalen Beziehungen zwischen religiösen Einstellungen und politischer
Ordnung geleistet hätte. Angesichts der theoretischen Grundprämisse Voegelins,
dass die Qualität der politischen Ordnung sehr stark vom Zustand des
religiösen Bewusstseins abhängt, müsste man erwarten, dass Voegelin höchstes
Interesse an der wissenschaftlichen Bestimmung der entsprechenden
Kausalzusammenhänge gehabt hätte. Erstaunlicherweise findet sich in Voegelins
Oevre jedoch so gut wie nirgendwo ein substantieller Beitrag zur
wissenschaftlichen Untersuchung dieser Kausalbeziehungen. Voegelin beschränkt
sich vielmehr fast vollständig auf die geistesgeschichtlich-hermeneutische
Deutung des religiösen und philosophischen Schrifttums bedeutender Epochen der
Menschheitsgeschichte.  Zwar werden in seinem Hauptwerk ``Order and History''
auch immer mal wieder die politischen Zustände der von ihm untersuchten
historischen Gesellschaften referiert, aber die Zusammenhänge zwischen den
politischen Zuständen und dem geistig-religiösen Hintergrund bleiben
weitgehend im Dunkeln, von einer differenzierten Beurteilung dieser
Zusammenhänge ganz zu schweigen.  Abgesehen von Voegelins einseitiger
methodischer Ausrichtung dürfte der Grund dafür ebenfalls in Voegelins
dogmatischer Grundprämisse liegen, dass die Qualität der politischen Ordnung
ganz unmittelbar von ihren spirituellen Grundlagen abhängt. Daher ist für
Voegelin, wenn er nur den Zustand des religiösen Bewusstseins hermeneutisch
bestimmt hat, immer schon alles klar.  Ist das religiöse Bewusstsein
deformiert, dann ergibt sich für Voegelin schon allein daraus, dass die
politische Ordnung nichts taugen kann.  Die Kritik dieser Grundprämisse bildet
ebenfalls ein wichtiges Anliegen dieses Buches (siehe Kapitel
\ref{SpirituellePolitik}), wenn ich auch nicht ganz so ausführlich darauf
eingehe wie auf Voegelins Bewusstseinsphilosophie selbst.

Dieses Buch ist eine überarbeitete Fassung meiner Magisterarbeit über
``Voegelins Bewusstseinsphilosophie (als Grundlage politischer Ordnung)'', die
ich im Jahr 2000 an der Universität Bonn verfasst habe, und die in ihrer
ursprünglichen Form seitdem auf meiner Website abrufbar
war.\footnote{www.eckhartarnbold.de/papers/voegelin} Dass ich mich nun
entschlossen habe, sie sieben Jahre nach ihrer Niederschrift in
überarbeiteter Form als Buch zu veröffentlichen hat verschiedene Gründe.
Einmal habe ich, wie schon oben erwähnt, den Eindruck, dass das Interesse an
Voegelin in letzter Zeit gewachsen ist. Zugleich gilt aber, was ich schon
damals bezüglich der Voegelin-Sekundärliteratur festgestellt habe: Sie ist
ihrem Standpunkt nach oft einseitig und zuweilen, wie mir scheint,
mehr der Gedächnispflege als der wissenschaftlichen Auseinandersetzung
verpflichtet.  Immerhin scheint sich das in jüngster Zeit langsam zu wandeln.
Ein erfreuliches Beispiel bildet in dieser Hinsicht Michael Henkels Studie
``Positivismuskritik und autoritärer Staat. Die Grundlagendebatte in der
Weimarer Staatsrechtslehre und Eric Voegelins Weg zu einer neuen Wissenschaft
der Politik (bis 1938)'', die vor kurzem in der ansonsten nicht immer
höchste Qualität verbürgenden Reihe der vom Münchner Eric-Voegelin-Archiv
herausgegebenen "`Occasional Papers"' erschienen ist.\footnote{Michael Henkel:
  Positivismuskritik und autoritärer Staat. Die Grundlagendebatte in der
  Weimarer Staatsrechtslehre und Eric Voegelins Weg zu einer neuen
  Wissenschaft der Politik (bis 1938), 2. Aufl., München 2005.} Der entschieden
autoritäre Standpunkt, der Voegelins politisches Denken seit 1929 bestimmt,
und der auch nach Voegelins Emigration in die USA in verschleierter Form noch
nachwirkt, wird darin ehrlich angesprochen. Lesenswert erscheinen mir unter
den jüngeren Publikationen auch Hans-Jörg Sigwarts
``Intellektuell-biographische Studien zum Frühwerk Eric
Voegelins''\footnote{Hans-Jörg Sigwart: Das Politische und die Wissenschaft.
  Intellektuell-biographische Studien zum Frühwerk Eric Voegelins, Würzburg
  2005.}, auch wenn sich der Autor Voegelins Standpunkt mehr als notwendig zu
eigen macht. Dezidiert kritische Beiträge sind allerdings trotz der längst
überfällig gewesenen Veröffentlichung von Hans Kelsens Rezension zu Voegelins
``Neuer Wissenschaft der Politik''\footnote{Hans Kelsen: A New Science of
  Politics. Hans Kelsen's Reply to Eric Voegelin's "`New Science of
  Politics"'. A Contribution to the Critique of Ideology (Ed. by Eckhart
  Arnold), Heusenstamm 2004.} bisher eher selten geblieben. Und das, obwohl
viele von Voegelins Ansichten geradezu hanebüchen sind. Deshalb hoffe ich,
dass meine sehr kritische Auseinandersetzung mit Voegelins
Bewusstseinsphilosophie mit dazu beiträgt eine Lücke in der
Voegelin-Sekundärliteratur zu schließen.

Ein weiterer Grund dafür, dass ich mich entschlossen habe, meine Arbeit nun
auch als Buch zu veröffentlichen, besteht darin, dass Internetquellen -- zum
Teil aus guten Gründen -- nach wie vor oft nicht als zitierfähig angesehen
werden. Ich möchte aber anderen Wissenschaftlern die Möglichkeit geben, auf
meine Thesen zu Eric Voegelin Bezug zu nehmen, und dazu ist die
``verbindlichere'' Form einer Buchveröffentlichung immer noch am besten
geeignet.

Nimmt man nach mehreren Jahren eine ältere Arbeit wieder zur Hand, so bleibt
nicht aus, dass man Vieles inzwischen anders schreiben würde. Ich bin daher
für die Buchveröffentlichung die Magisterarbeit noch einmal durchgegangen und
habe sie an vielen Stellen geändert, wo mir der Text unklar oder
missverständlich erschien. Auch habe ich dort, wo es mir aufgefallen ist,
weitere Literaturverweise eingefügt, ohne allerdings die seitdem erschienene
oder damals unberücksichtigt gebliebene Sekundärliteratur systematisch
einzuarbeiten. Der Aufwand dafür hätte zu viel Zeit in Anspruch genommen, ohne
dass die inhaltliche Aussage, um die es mir geht, und die mir nach wie vor
zutreffend erscheint, dadurch wesentlich stärker geworden wäre. Da ich mit
dem, was ich damals über Voegelin geschrieben habe, immer noch fast vollkommen
übereinstimme, und meine Meinung über den Politikwissenschaftler Eric Voegelin
seitdem eher noch kritischer geworden ist, bin ich mit Änderungen im Ganzen
aber eher behutsam geblieben. Vollständig bzw. fast vollständig neu
geschrieben habe ich lediglich die Kapitel \ref{spirituelleSachzwänge} und
\ref{KritikVoegelinsSeinserfahrung}, die mir in der bisherigen Form zum Teil
abwegig bzw. nicht verständlich genug erschienen sind. Beim wiederlesen der
Arbeit ist mir auch aufgefallen, dass ich manchmal zu drastischen
Formulierungen gegriffen habe, wie ich sie heute nicht mehr unbedingt wählen
würde. Trotzdem habe ich mich entschlossen, bei der Überarbeitung den Stil
nicht allzusehr zu glätten, denn ich blicke mit Wehmut auf die glückliche Zeit
zurück, als ich noch nicht wusste, wie sehr man innerhalb akademischer
Institutionen zuweilen gezwungen ist, sich intellektuell zu prostituieren. Die
Art von intellektueller Konzessionslosigkeit, die ich an großen Philosophen
wie Thomas Hobbes, Arthur Schopenhauer, Friedrich Nietzsche oder Ludwig
Wittgenstein so bewundere, und der auch Eric Voegelin auf eine allerdings eher
törichte Weise huldigte, wird im Milieu der akademischen Philosophie ziemlich
wirkungsvoll unterdrückt. Ob es wirklich besser wäre, sie generell zuzulassen,
möchte ich gar nicht unbedingt behaupten, aber schade ist es trotzdem.

Zum Schluss will ich nicht versäumen, Erasmus Scheuer für die Hilfe beim
Korrekturlesen zu danken.

\begin{flushright}Düsseldorf, den 31. August 2007\end{flushright}

\chapter{Einleitung}

% Thema: Bewußtseinsphilosophie als Grundlage politischer Ordnung/über Eric Voegelin

\section{Thema}

Das Thema dieses Buches ist die Bewusstseinsphilosophie Eric
Voegelins.\footnote{Zur Biographie: Eric Voegelin wurde 1901 in Köln geboren.
  1922 Promotion bei Hans Kelsen und Othmar Spann. 1929-38 Privatdozent und
  außerordentlicher Professor für Staats- und Gesellschaftslehre in Wien. 1938
  Flucht vor den Nazis in die USA.  1942-58 Professor of Gouvernment an der
  Lousiana State University in Baton Rouge.  1958 Professor für
  Politikwissenschaft in München. 1969 Rückkehr in die USA.  1974 Senior
  Research Fellow an der Hoover Institution on War, Revolution and Peace in
  Stanford. 1985 Tod. (Angaben aus: Michael Henkel: Eric Voegelin zur
  Einführung, Hamburg 1998, S. 13-35, S. 198-199.)} Sie wird untersucht unter
dem besonderen Aspekt der Begründung politischer Ordnung durch religiöse
Bewusstseinserfahrungen.

Eric Voegelin vertrat die etwas eigentümliche und in der heutigen Zeit im
westlichen Kulturkreis eher befremdlich wirkende Auf\/fassung, dass die
religiösen Erfahrungen des Menschen eine notwendige Grundlage politischer
Ordnung bilden.  Damit ein politisches Gemeinwesen über eine stabile und im
ethischen Sinne gute politische Ordnung verfügt, genügt es nach Voegelins
Ansicht keineswegs, wenn sich diese Ordnung auf ein ausgeklügeltes System von
Institutionen und auf eine wohldurchdachte Verfassung stützt. Für Voegelin
muss die politische Ordnung darüber hinaus tief im religiösen Empfinden der
Bürger verwurzelt sein. Nur dann kann sie eine ausreichende Resistenz
gegenüber inneren und äußeren Anfechtungen entwickeln, und nur dann kann ihr
eine ethische Qualität zugesprochen werden. In diesem Buch soll kritisch
hinterfragt werden, ob die religiöse Erfahrung tatsächlich eine notwendige
Voraussetzung politischer Ordnung bildet und ob eine solche Grundlegung der
politischen Ordnung überhaupt wünschenswert ist.

Wenn für Voegelin die politische Ordnung im religiösen Empfinden oder, um es
in seiner eigenen Terminologie zu formulieren, in den existentiellen
"`Erfahrungen"' der Bürger verwurzelt sein muss, so ist dies bei Voegelin
nicht in der Weise zu verstehen, dass der Staat den religiösen Bereich der
menschlichen Natur für seine Zwecke einspannen soll, wie dies die totalitären
Staaten anstreben. Das religiöse Empfinden geht nicht vom Staat oder vom
gesellschaftlichen Kollektiv aus, sondern es entspringt dem existentiellen
Erleben des Einzelnen, und nach Maßgabe dieses im individuellen Erleben
verankerten religiösen Empfindens muss die politische Ordnung gestaltet werden.
Damit dies funktioniert, ist natürlich die Intaktheit des religiösen
Empfindens von größter Bedeutung. Die intakte "`Ordnungserfahrung"' bildet für
Voegelin nicht nur eine notwendige Voraussetzung (guter) politischer Ordnung,
sie stellt auch eine, zwar nicht unbedingt allein hinreichende, aber doch
stark begünstigende Bedingung dar, gegenüber der alle pragmatischen Probleme
politischer Ordnung, wie z.B. die Einzelheiten der Verfassungsordnung,
vergleichsweise sekundär sind.

Kommt dem Unterschied zwischen intaktem und nicht intaktem
religiös-existentiellen Empfinden eine derartig große Bedeutung zu wie bei
Voegelin, so ergibt sich, dass die politische Philosophie einer
Auseinandersetzung mit der Religion auf der inhaltlichen Ebene der religiösen
Dogmen und Erfahrungen\footnote{Voegelin beschränkt sich auf die Erfahrungen,
  da seinem eher mystischen Religionsverständnis gemäß auch die Dogmen nur
  Ausdruck von religiösen Erfahrungen (und nicht von offenbartem Wissen)
  seien können.} nicht ausweichen kann. Wie kann aber hier zwischen echt und
unecht, zwischen richtig und falsch unterschieden werden?  Voegelin verfolgt
in dieser Frage einen zweifachen Ansatz. Zum einen geht er historisch vor,
indem er sich bemüht, die geschichtlichen Entwicklungsprozesse religiöser
Erfahrung nachzuzeichnen und dabei die "`differenziertesten"' Stufen
religiös-existentiellen Welterlebens ausfindig zu machen. Zum anderen versucht
Voegelin, auf bewusstseinsphilosophischem Wege das Wesen der religiösen bzw.
existentiellen Erfahrungen zu ergründen und in unmittelbarer Selbsterfahrung
nachzuvollziehen. Da letzten Endes auch die historische Beurteilung religiöser
Erfahrungen nur am Maßstab der bewusstseinsphilosophisch ermittelten
Wesensauf\/fassung möglich ist, muss der bewusstseinsphilosophische Ansatz als
der grundlegendere dieser beiden Ansätze angesehen werden. Diesem Buch liegt
daher die Interpretationsannahme zu Grunde, dass die Bewusstseinsphilosophie
Voegelins innerhalb der Systematik seines Gedankengebäudes das Zentrum
einnimmt.\footnote{Diese Annahme entspricht Voegelins Selbstdeutung.  Vgl.
  Eric Voegelin: Anamnesis. Zur Theorie der Geschichte und Politik, München
  1996, im folgenden zitiert als: Voegelin, Anamnesis, S. 7.}

Voegelin geht es nicht darum, empirisch den Zusammenhang zwischen
vorfindlichen politischen Ordnungsgefügen und den sie fundierenden religiösen
Erfahrungen aufzuweisen. Vielmehr verfolgt Voegelin in erster Linie die
normative Absicht, durch die bewusstseinsphilosophische Aufdeckung der
religiösen Erfahrungsquellen die verbindliche Grundlage einer humanen und
totalitarismusresistenten politischen Ordnung für die Gegenwart zu finden,
welche für ihn in vielen westlichen Demokratien, die ihm in Ermangelung
religiöser Grundlagen auf Sand gebaut schienen, noch unzureichend verwirklicht
war. In diesem Buch steht die Untersuchung dieses normativen Aspektes im
Vordergrund. Es geht mir nicht um die Frage, ob Voegelins Modellvorstellung
von politischer Ordnung auf das alte Ägypten oder das römische Kaiserreich
anwendbar ist, sondern es soll versucht werden zu klären, ob Voegelins
Vorstellungen in der heutigen Zeit unter den Bedingungen pluralistischer und
sich entwickelnder multikultureller Gesellschaften noch tragfähig sind und
normative Gültigkeit beanspruchen dürfen. Letzteres ist natürlich nicht nur
eine Frage von Zeitumständen, sondern vor allem eine Frage der
Begründungsqualität.

\section{Methode}

Die Untersuchungsmethode, die in diesem Buch angewandt wird, ist die einer
rationalen Rekonstruktion, d.h. es wird versucht, anhand einzelner Texte
Voegelins dessen Thesen zu rekonstruieren und ihre Begründung kritisch zu
prüfen. Nur am Rande wird dagegen auf philologische und historische Fragen
eingegangen wie die, welche Entwicklung Voegelins Begriffe innerhalb seines
Werkes durchgemacht haben, durch welche Philosophen er geprägt wurde oder
welche zeitgeschichtlichen Umstände auf sein Denken Einfluss genommen haben. Im
Vordergrund steht stattdessen die Frage der Gültigkeit von Voegelins
Theorie.

% Für eine
% rationale Rekonstruktion, die auf die Frage der Gültigkeit einer Theorie
% abzielt, ist die Klärung werk- und zeitgeschichtliche Zusammenhänge der
% Theorie zwar eine Verständnisvoraussetzung, aber ihr kommt kein vordringliches
% thematisches Interesse zu.

% \footnote{Damit soll nicht gesagt werden, daß eine
%   philologische Analyse des Voegelinschen Werkes nicht höchst aufschlußreich
%   könnte, würde eine genaue Untersuchung der Herkunft von Voegelins
%   Denkfiguren und philosophischen Stichworten doch zweifellos zeigen, wie fest
%   Voegelin von seiner geistigen Prägung her an bestimmten philosophischen
%   Strömungen des 19. und 20. Jahrhunderts haftet, und daß er weit weniger aus
%   der Philosophie der Antike schöpft, als daß das seinem auch in der
%   Sekundärliteratur häufig kolportierten Selbstbild entspricht.}

% würde sie doch zweifellos zu Tage fördern, wie sehr Voegelin .
% Wollte man
% einmal versuchen, ausführlich und kritisch nachzuvollziehen, woher Voegelin
% die Denkfiguren und Stichwörter seiner Philosophie bezieht, so würde sich
% zweifellos ein anderes Bild ergeben als das gelegentlich noch in der
% Voegelin-Literatur kolportierte und im wesentlichen seinem Selbstverständnis
% entsprechende Bild des großen Gelehrten, der in gnostisch verwirrter Zeit auf
% dem mühsamen Wege anamnetischer Wiedererinnerung die beinahe verschollenen
% Schätze noetischen Ordnungswissen aus der klassischen Literatur der Antike
% hebt. Eher würde sich das Bild eines Geschichtsphilosophen ergeben, dessen
% intellektueller Horizont zwar einige Jahrtausende der Menschheitsgeschichte
% umfaßt, der von seiner geistigen Prägung her jedoch fest in bestimmten
% philosophischen Strömungen des 19. und 20. Jahrhunderts verwurzelt ist.

Gegen eine derartige Herangehensweise sind von zwei gegensätzlichen Richtungen
her Einwände denkbar. Einerseits könnte eingewandt werden, dass Voegelin
heutzutage keineswegs mehr aktuell und eine theoretische Auseinandersetzung
mit seinen Gedanken daher nicht mehr von Interesse sei. Andererseits könnte
gegen die Methode der rationalen Rekonstruktion und Kritik der Vorwurf erhoben
werden, dass sie, da einem positivistischen Wissenschaftsideal verpflichtet,
dem Denken Voegelins nicht gerecht werden könne.

Der erste Einwand ließe sich dahingehend weiter ausführen, dass Voegelin als
ein typischer Vertreter der Epoche des kalten Krieges inzwischen nurmehr eine
historische Erscheinung sei.\footnote{Dies deutet mit Vorsicht Eugene Webb an.
  Vgl. Eugene Webb: Review of Michael Franz, Eric Voegelin and the Politics of
  Spiritual Revolt: The Roots of Modern Ideology, in: Voegelin Research News,
  Volume III, No. 1, February 1997, auf: \url{http://alcor.concordia.ca/\~{
  }vorenews/v-rnIII2.html} (Host: Eric Voegelin Institute, Lousiana State
  University. Zugriff am: 1.8.2007), im folgenden zitiert als Webb, Review.}
Wenn man heute einen politischen Romantiker wie, um ein beliebiges Beispiel zu
wählen, Konstantin Frantz analysierte, so würde man auch keine Zeit damit
verschwenden, seine weltfremden Träumereien von einem christlichen Europa zu
widerlegen, sondern ihn von vornherein nur unter einer rein geistes- oder
zeitgeschichtlichen Perspektive, also gewissermaßen als ein historisches
Kuriosum betrachten.  Werden derartige Vorbehalte gegen Voegelin auch selten
offen geäußert, so liegen sie doch in der Luft und bilden auch unausgesprochen
einen der Gründe, weshalb Voegelin heutzutage -- trotz des jüngst neuerwachten
Interesses -- im Ganzen eher in Vergessenheit geraten ist. Sollte sich aber
Voegelins Theorie auch als gänzlich unhaltbar erweisen, so scheint mir eine
Auseinandersetzung mit Voegelin auf der Sachebene dennoch lohnend, weil
Voegelins Theorie als ein bestimmter Ansatz quasi-religiöser Politikbegründung
eine geistige Möglichkeit repräsentiert, die unabhängig davon, ob sie gerade
in Mode ist oder nicht, aus grundsätzlichem Interesse der Untersuchung wert
ist. Im übrigen können auch bei politikphilosophischen Grundsatzdiskussionen
Stimmungsumschwünge eintreten, die das, was noch wenige Jahrzehnte zuvor als
abwegig galt, auf einmal wieder naheliegend und vertretbar erscheinen lassen.
Dies gilt umso mehr, als auch die abstruseste Philosophie zur Grundlage
politischen Handelns und politischer Ordnung gemacht werden kann. Und wenn
einmal eine obskure Philosophie gesellschaftlich wirksam geworden ist, so
bleibt der bloße Hinweis auf ihre Abstrusität ohnmächtig, da diese Philosophie
dem Empfinden der meisten Menschen dann ganz natürlich erscheint.

Dem zweiten Einwand liegt die Frage zu Grunde, ob die Methode der rationalen
Rekonstruktion für eine Untersuchung von Voegelins Werk angemessen ist.
Voegelin wünschte sich von seinen Lesern eine ganz bestimmte Lesehaltung, die
weniger durch eine kritisch-rationale Einstellung als durch den meditativen
Nachvollzug seiner Gedanken bestimmt sein sollte, denn er glaubte, eine
besondere Art von Wissenschaft zu verfertigen, bei der es gerade nicht auf das
Aufstellen von Thesen und das kritische Abwägen von Argumenten ankommt. Aber
zugleich beanspruchte Voegelin, mit seinen Schriften die theoretischen
Grundlagen politischer Ordnung zu bestimmen. Ob diese Grundlagen tragfähig
sind, lässt sich jedoch nur überprüfen, indem man sie rational analysiert. Die
Rechtfertigung für meine, dem Denken Voegelins vielleicht etwas fremde,
analytische Herangehensweise liegt also in Voegelins eigener Zielvorgabe, die
geistigen Grundlagen guter politischer Ordnung zu finden. Da eine politische
Ordnung für jeden, der in ihr lebt, verbindliche Geltung haben soll, so muss
ihre Begründung auch intersubjektiv nachvollziehbar sein.  Übrigens nahm
Voegelin für seine Art von Politikwissenschaft in Anspruch, dass sie rationale
Wissenschaft sei. Aber dies beruht, wie noch zu zeigen sein
wird,\footnote{Siehe dazu Seite \pageref{Rationalitaetsbegriff}, besonders
  Fußnote \ref{FussnoteRationalitaet}, sowie grundsätzlich zu Voegelins
  Sprachgebrauch Kapitel \ref{KritikSprache}.} auf einer willkürlichen
Umdeutung des Begriffes der Rationalität.

Anders, als sich dies für die Methode der rationalen Rekonstruktion eigentlich
empfiehlt, erfolgt die Darstellung von Voegelins bewusstseinsphilosophischen
Schriften nicht durch eine Zuspitzung von Voegelins Aussagen zu einzelnen
Thesen, sondern in der Form einer Wiedergabe seines Gedankenganges. Der Grund
hierfür besteht darin, dass Voegelins Texte in hohem Maße einem erzählerischen
Stilprinzip verpflichtet sind und sich daher gegen eine Zuspitzung zu
einzelnen klaren Thesen sträuben. Eine Zusammenfassung in Thesen würde deshalb
bereits ein sehr hohes Maß von Interpretation in Voegelins Texte hineintragen,
so dass nicht mehr leicht zu erkennen wäre, wie die Thesen aus Voegelins
Worten entnommen worden sind. Aus diesem Grund wird der Inhalt eines jeden
untersuchten Textes zunächst ausführlich mit eigenen Worten wiedergegeben, so
dass sich meine Interpretation leicht nachvollziehen lässt. Unmittelbar an die
Darstellung eines jeden Textes oder auch einzelner Textpassagen schließt sich
eine eingehende Kritik dieser Textpassagen an. Mag dieses Verfahren der
intermittierenden Kritik auch einen Eindruck von Voreiligkeit und
Nicht-ausreden-lassen-wollen erwecken, so ist es doch dadurch gerechtfertigt,
dass die untersuchten Texte bezüglich ihrer Entstehungszeit teilweise recht
weit auseinanderliegen und dementsprechend unterschiedliche Fragen aufwerfen.
Außerdem lässt sich eine ins Einzelne gehende Kritik nur schwer an eine
umfassende Darstellung anschließen, nach welcher den Lesern und Leserinnen nur
noch die groben Züge des Gedankenganges im Gedächtnis geblieben sind. Eine
Detail-Untersuchung ist aber beabsichtigt, denn der Wert einer Philosophie
entscheidet sich meiner Ansicht nach weniger an den großen Linien der ihr zu
Grunde liegenden metaphysischen Weltauf\/fassung als an der Qualität ihrer
Durchführung im Detail.

% Schließlich soll nicht
% verhehlt werden, daß in dieser Darstellungsweise meine sehr kritische Meinung
% zu Voegelin zum Ausdruck kommt. Es würde gewiß ein wenig sonderbar erscheinen,
% zunächst in aller Seelenruhe über fünfzig oder sechzig Seiten Voegelins
% Gedankengänge auszubreiten, nur um im Anschluß daran mit der Eröffnung
% aufzuwarten, daß all diese Überlegungen im Übrigen samt und sonders verkehrt
% seien.
 
\section{Quellen und Sekundärliteratur}

Dieses Buch beabsichtigt eine inhaltliche Auseinandersetzung mit den
wichtigsten bewusstseinsphilosophischen Standpunkten Voegelins, beansprucht
aber keinesfalls eine umfassende Darstellung von Voegelins
Bewusstseinsphilosophie zu sein. Bewusstseinsphilosophische Überlegungen
begleiten Voegelins Schaffen von seinen frühesten Schriften bis zu den
spätesten Werken,\footnote{Vgl. etwa das Kapitel über "`Time and Existence"',
  in: Eric Voegelin: On the Form of the american Mind, Baton Rouge / London
  1995, S.23ff.} wobei die Bedeutung der Bewusstseinsphilosophie in Voegelins
Werk im Laufe der Zeit immer mehr zunimmt.  Dabei geht Voegelins
Bewusstseinsphilosophie fließend in seine Geschichtsdeutung und seine
politische Theorie über.\footnote{Besonders deutlich wird dies in der
  Einleitung zu Order and History I. Vgl. Eric Voegelin: Order and History.
  Volume One. Israel and Revelation, Baton Rouge / London 1986 (zuerst: 1956),
  im folgenden zitiert als: Voegelin, Order and History I, S.1-11.} Eine
Vollständigkeit beanspruchende Untersuchung von Voegelins
Bewusstseinsphilosophie müsste all diese Zusammenhänge mit berücksichtigen und
in hohem Maße auch solche Schriften Voegelins einbeziehen, die nicht im
engeren Sinne bewusstseinsphilosophisch genannt werden können.

Aus pragmatischen Gründen beschränkt sich dieses Buch daher auf die
Untersuchung von "`Anamnesis"', dem einzigen ausdrücklich als
bewusstseinsphilosophisch ausgewiesenen größeren Werk, welches Voegelin zu
dieser Thematik selbst veröffentlicht hat. Weiterhin werden aus dem Werk
"`Anamnesis"', das neben bewusstseinsphilosophischen auch eine Reihe von
historischen Aufsätzen Voegelins versammelt, nur die im engeren Sinne
bewusstseinsphilosophischen Aufsätze berücksichtigt, welche den ersten und
dritten Teil dieses Werkes bilden, während der zweite Teil von "`Anamnesis"'
überwiegend historische Probleme behandelt. Durch die Beschränkung auf
"`Anamnesis"' bleiben die späteren Entwicklungen von Voegelins
Bewusstseinsphilosophie außen vor. So wird Voegelins Auseinandersetzung mit
dem Thema "`Egophanie"' (Selbstbezogenheit des modernen Menschen im Gegensatz
zur Gottbezogenheit), welches in "`Order and History IV"' einen so großen Raum
einnimmt,\footnote{Vgl. Voegelin, Order and History IV, S.260ff.} nicht näher
behandelt. Auch der Komplex der "`consciousness-reality-language"' und das
"`paradox of consciousness"', zwei zentrale Begriffe der letzten, in "`Order
and History V"' erreichten Entwicklungsstufe seiner Bewusstseinsphilosophie,
treten in "`Anamnesis"' lediglich in der noch vergleichsweise kruden Form des
dort entwickelten vielschichtigen und paradoxen Realitätsbegriffs
auf.\footnote{Vgl. Eric Voegelin: Order and History. Volume Five. In Search of
  Order, Baton Rouge / London 1987, im folgenden zitiert als: Voegelin, Order
  and History V, S. 14-18.  -- Vgl. Voegelin, Anamnesis, S. 304-305.} Trotz
dieser Einschränkungen umfasst "`Anamnesis"', besonders durch den zeitlichen
Abstand der darin aufgenommenen Texte, eine große Spannbreite von Voegelins
bewusstseinsphilosophischem Denken und kann daher als durchaus repräsentativ
für Voegelins gesamte Bewusstseinsphilosophie angesehen werden. Die Kritik an
Voegelin, die in diesem Buch anhand einzelner bewusstseinsphilosophischer
Schriften entwickelt wird, ist zu einem großen Teil von grundsätzlicher Art,
so dass sie sich leicht auf andere Schriften Voegelins übertragen lässt. Einer
allzu großen Fixierung auf bloße Einzelaspekte von Voegelins
Bewusstseinsphilosophie wird dadurch entgegengewirkt, dass im ersten Teil des
Buches ein Gesamtüberblick über die politische und historische Philosophie
Voegelins gegeben wird, in welche die Bewusstseinsphilosophie eingebettet ist.

Die Lage der Sekundärliteratur zu Eric Voegelin und zu seiner
Bewusstseinsphilosophie ist nicht in jeder Hinsicht günstig. Zwar gibt es über
Eric Voegelin und besonders zu seinem Hauptwerk "`Order and History"' schon
ein beachtliches Schrifttum,\footnote{Vgl. Geoffrey L. Price: Recent
  International Scholarship on Voegelin and Voegelinian Themes. A Brief
  Topical Bibliography, in: Stephen A. McKnight / Geoffry L. Price (Hrsg.):
  International and Interdisciplinary Perspectives on Eric Voegelin, Missouri
  1997, S. 189-214. -- Eine regelmäßig aktualisierte Bibliographie enthalten
  die Voegelin-Research News des Eric Voegelin Insitute der Louisiana State
  University, http://alcor.concordia.ca/\~{ }vorenews/} aber gerade zu
Voegelins Bewusstseinsphilosophie sind Einzeluntersuchungen noch recht dünn
gesät.\footnote{Eine erschöpfende Darstellung der mittleren Schaffensperiode,
  einschließlich der Bewusstseinsphilosophie des ersten Teils von Anamnesis
  liefert Barry Cooper. Vgl. Barry Cooper: Eric Voegelin and the Foundations
  of Modern Political Science, Columbia and London 1999, S. 161ff. -- Für die
  spätere Schaffensperiode, insbesondere "`Order and History V"': Vgl. Michael
  P. Morrissey: Consciousness and Transcendence. The Theology of Eric
  Voegelin, Notre Dame 1994, S. 117ff. -- Meist wird die
  Bewusstseinsphilosophie jedoch nur im Rahmen einer anderen Thematik
  mitbehandelt. Vgl.  beispielsweise: Petropulos, William: The Person as
  `Imago Dei'. Augustine and Max Scheler in Eric Voegelins `Herrschaftslehre'
  and `The Political Religions', München 1997, S. 35-38.} Hinsichtlich dieser
Seite von Voegelins Werk herrscht noch ein Forschungsdefizit, zu dessen
Behebung dieses Buch ebenfalls einen Beitrag leisten möchte. Darüber hinaus
leidet die Sekundärliteratur zuweilen an einer gewissen Einseitigkeit, die,
wie es scheint, dadurch zustande kommt, dass sie zu einem großen Teil von
überzeugten Anhängern Voegelins bestritten wird, während die vorhandenden und
möglichen Gegner Voegelins ihn offenbar mehr oder weniger ignorieren. Nicht
selten wird recht unkritisch das Selbstbild Voegelins, des großen Gelehrten,
der in gottvergessener Zeit in den Tiefen der Geschichte auf Wahrheitssuche
geht, kolportiert und geradezu eifersüchtig gegen Einwände
verteidigt.\footnote{Deutlich wird dies etwa an den heftigen Reaktionen auf
  Eugene Webbs maßvolle Voegelin-Kritik.  -- Vgl. Thomas J.  Farrell: The Key
  Question. A critique of professor Eugene Webbs recently published review
  essay on Michael Franz's work entitled "'Eric Voegelin and the Politics of
  Spiritual Revolt: The Roots of Modern Ideology"', in: Voegelin Research
  News, Volume III, No.2, April 1997, auf: http://alcor.concordia.ca/\~{
  }vorenews/v-rnIII2.html -- Maben W. Poirier: VOEGELIN-- A Voice of the Cold
  War Era ...? A COMMENT on a Eugene Webb review, in: Voegelin Research News,
  Volume III, No.5, October 1997, auf: http://alcor.concordia.ca/\~{
  }vorenews/v-rnIII5.html (Host jeweils: Eric Voegelin Institute, Lousiana
  State University. Zugriff am: 1.8.2007).}  Freilich ist Voegelin nicht ganz
unschuldig daran, dass sein Werk unter die Zeloten gefallen ist, sah er doch
selbst in Ansichten, die zu seiner Denkweise im Gegensatz standen, die
"`Rhetorik deformierter Existenz"' am Werk, und empfahl er einmal sogar, dem
"`verführerischen Zwang [für den modernen Menschen, E.A.], sich selbst zu
deformieren"', mit den kräftigen Worten eines altägyptischen Dichters
entgegenzutreten, die da lauten: "`Siehe, mein Name wird übel riechen durch
dich // mehr als der Gestank von Voegelmist // an Sommertagen, wenn der Himmel
heiß ist"'.\footnote{Vgl. Eric Voegelin: Äquivalenz von Erfahrungen und
  Symbolen in der Geschichte, in: Eric Voegelin: Ordnung, Bewußtsein,
  Geschichte, Späte Schriften (Hrsg. von Peter J. Optiz), Stuttgart 1988,
  S.99-126 (S.105).}  Erst neuerlich sind einige Beiträge erschienen, die sich
mit der gebotenen Distanz und zum Teil auch sehr kritisch mit Voegelin
auseinandersetzen. Nach wie vor scheint dies aber eher noch die Ausnahme zu
sein.\footnote{Als ein durchaus typisches Beispiel für die Art von
  Sekundärliteratur, die sich fast nur aus Bestandsaufnahme zusammensetzt,
  aber so gut wie keine kritische Diskussion enthält, sei hier nur das
  folgende herausgegriffen: Glenn Hughes (Ed.): The Politics of the Soul. Eric
  Voegelin on Religious Experience, Lanham / Boulder / New York / Oxford 1999.
  -- Als Beispiele der Voegelin-Kritik seien herausgegriffen: Mit
  gesellschaftskritischem Akzent: Richard Faber: Der Prometheus-Komplex.  Zur
  Kritik der Politotheologie Eric Voegelins und Hans Blumenbergs, Königshausen
  1984. -- Ideologiekritisch vor allem gegenüber Voegelins Gnosis-Begriff:
  Albrecht Kiel: Säkularisierung als Geschichte des Unheils.  Die
  Gleichsetzung von Rationalität und Ordnung mit Katholizität in der
  Geschichtsphilosophie Eric Voegelins, in: Albrecht Kiel: Gottesstaat und Pax
  Americana. Zur Politischen Theologie von Carl Schmitt und Eric Voegelin,
  Cuxhaven und Dartford 1998, S.95-118. -- Erhebliche Zweifel an der
  philologischen Genauigkeit Voegelins meldet Zdravko Planinc an: Zdravko
  Planinc: The Uses of Plato in Voegelin's Philosophy of Consciousness:
  Reflections prompted by Voegelin's Lecture, "`Structures of Consciousness"',
  in: Voegelin-Research News, Volume II, No.  3, September 1996, auf:
  http://alcor.concordia.ca/\~{ }vorenews/v-rnII3.html (Host: Eric Voegelin
  Institute, Lousiana State University. Zugriff am: 1.8.2007). -- Sehr
  kritisch gegenüber Voegelin: Hans Kelsen: A New Science of Politics. Hans
  Kelsen's Reply to Eric Voegelin's "`New Science of Politics"'. A
  Contribution to the Critique of Ideology (Ed. by Eckhart Arnold),
  Heusenstamm 2004, im folgenden zitiert als: Kelsen, A New Science of
  Politics. -- Ähnlich Voegelin-kritisch auch: Eckhart Arnold: Eric Voegelin
  als Schüler Hans Kelsens, erscheint voraussichtlich Wien 2007. -- Mit
  kritischen Akzenten: Michael Henkel: Positivismuskritik und
  autoritärer Staat. Die Grundlagendebatte in der Weimarer Staatsrechtslehre
  und Eric Voegelins Weg zu einer neuen Wissenschaft der Politik (bis 1938),
  München 2005.}

Außer der Sekundärliteratur zu Eric Voegelin wird auch philosophische
Literatur zu den Themen, die Voegelin in seinen bewusstseinsphilosophischen
Texten anspricht, herangezogen. Hier besteht allerdings die Schwierigkeit,
dass es in der Philosophie kein Expertentum gibt, und dass man daher je
nachdem, auf welche Schule man zurückgreift, zu einer sehr unterschiedlichen
Ansicht des Gegenstandes gelangen kann.  In diesem Buch wurden vor allem die
Autoren zu Rate gezogen, die auch Voegelin in seinen Schriften anspricht. Dies
bereitet für die Untersuchung des ersten Teils von "`Anamnesis"' keine
Probleme, da klar ist, dass Voegelin sich hier vornehmlich mit der
Phänomenologie auseinandersetzt.  Schwieriger ist dies jedoch für dritten Teil
von "`Anamnesis"' zu realisieren, da Voegelin hier wesentlich selbständiger
vorgeht.  Weiterhin werden solche Autoren mit einbezogen, die von Voegelin zwar
nicht immer ausdrücklich erwähnt werden, auf die er sich jedoch
stillschweigend zu beziehen scheint.

\section{Aufbau}

Das Buch ist in drei Teile untergliedert. Der erste Teil (Kapitel
\ref{Grundzuege}) gibt einen Grundriss von Voegelins politischer Philosophie.
Ziel ist es, die Hauptthesen von Voegelins politischer Philosophie
darzustellen sowie seinen methodischen Ansatz zu bestimmen. Insbesondere soll
gezeigt werden, wie und an welcher Stelle bewusstseinsphilosophische
Voraussetzungen in Voegelins politisches Denken eingehen. In diesem Teil
beziehe ich mich überwiegend auf Voegelins "`Neue Wissenschaft der
Politik"',\footnote{Eric Voegelin: Die Neue Wissenschaft der Politik. Eine
  Einführung, München 1959, im folgenden zitiert als: Voegelin, Neue
  Wissenschaft der Politik.} da dieser Schrift unter Voegelins Werken am
ehesten der Charakter einer Programmschrift eigen ist. Dabei werden von
vornherein auch kritische Einwände gegen Voegelins Auf\/fassungen diskutiert.
Die Kritik dient nicht zuletzt dazu, den Problemhorizont abzustecken, der bei
der Untersuchung von Voegelins Bewusstseinsphilosophie berücksichtigt werden
muss. Inhaltlich stimmt meine Kritik, soweit sie Voegelins "`Neue Wissenschaft
der Politik"' betrifft, vielfach mit der von Hans Kelsen geübten
überein,\footnote{Vgl. Kelsen, A New Science of Politics, a.a.O.} von der ich
allerdings erst nach der Niederschrift meiner damaligen Magisterarbeit
Kenntnis bekommen hatte.  Auf Übereinstimmungen zu Kelsens Deutung habe ich,
wo sie mir besonders aufgefallen sind, nun in den Fußnoten hingewiesen. Neben
der "`Neuen Wissenschaft der Politik"' beziehe ich mich auch kritisch auf
Voegelins geschichtsphilosophische Prämissen, wie sie von Voegelin vor allem
in den Einleitungen zu den Einzelbänden von "`Order and History"' entfaltet
werden.

Im zweiten Teil, der seiner Länge wegen auf mehrere Kapitel verteilt ist
(Kapitel \ref{VoegelinsBewusstseinsphilosophie}, \ref{politischeRealitaet},
\ref{Scheitern}), werden ausführlich Voegelins bewusstseinsphilosophische
Schriften dargestellt und einer eingehenden Detailkritik unterzogen. Den
Abschluss des zweiten Teils (Kapitel \ref{Scheitern}) bildet die Diskussion
einiger Grundprobleme von Voegelins Bewusstseinsphilosophie, wobei die
kritische Betrachtung von Voegelins Begriff der (religiösen) Erfahrung im
Zentrum steht. Es gilt dabei kritisch Bilanz zu ziehen, ob der in Voegelins
Denken zentrale Begriff der "`Erfahrung"' hinreichend durch die
bewusstseinsphilosophischen Überlegungen Voegelins begründet und erläutert
ist, um für das Verständnis und die Gestaltung politischer Ordnung fruchtbar
gemacht werden zu können.

Im letzten, mehr essayistisch gehaltenen Teil des Buches (Kapitel
\ref{SpirituellePolitik}), wird schließlich auf einer etwas allgemeineren
Ebene die Frage angesprochen, ob gute politische Ordnung einer religiösen
Grundlage bedarf. Dabei wird zu zeigen versucht, dass eine
religiös-spirituelle Grundlegung der Politik, wie sie Voegelin vorschwebte,
sowohl aus grundsätzlichen Überlegungen als auch insbesondere unter den
Bedingungen einer pluralistischen und zunehmend multikulturellen Gesellschaft
vor erheblichen Schwierigkeiten steht. Zugleich wird die Frage aufgeworfen, ob
eine rein säkulare, durch Konsens bestimmte Grundlegung politischer Ordnung
auf Basis eines Gesellschaftsvertrages denkbar ist, und ob daher politische
Ordnung des transzendenten Bezuges nicht ohnehin gänzlich entraten kann. Eine
Frage, die ich ziemlich uneingeschränkt bejahe.

Den Abschluss des Buches (Kapitel \ref{WasBleibt}) bilden schließlich einige
Gedanken zu der Frage, ob Eric Voegelins kulturhistorischer Ansatz der
gegenwärtigen Politikwissenschaft noch etwas mitzuteilen hat, selbst wenn die
meisten seiner Ansichten inzwischen als überholt gelten müssen.

% Wenn man so will ist dies nichts weiter als eine liberale
% Selbstvergewisserung. Da Theorie der liberalen Demokratie hierzulande zur Zeit
% sowieso die herrschende Meinung wiedergibt, kann dieser Teil eher knapp
% ausfallen. Die Grundthesen des dritten Teils lauten kurz gefaßt:

% \begin{itemize}
% \item Wenn die Bewußtseinsphilosophie kein objektives Wissen über die
%   Ordnung des Seins vermitteln kann, so kann sie auch auch keine Grundlage
%   politischer Ordnung bilden.
% \item Wenn Politik auf Transzendenz gegründet wird, dann wird die Religiosität
%   zu einer Angelegenheit der politischen Öffentlichkeit. Dies wirft Probleme
%   für die Religionsfreiheit und Toleranz auf.
% \item Es ist (insbesondere in einer multikulturellen Gesellschaft)
%   aussichtsreicher Konsens auf der Ebene der Werte als auf der Ebene der
%   Wertbegründung zu suchen.
% \item Die Notwendigkeit politischer Ordnung entsteht aus dem Umstand, daß
%   Menschen einander in die Quere kommen können, und deshalb Abmachungen
%   treffen müssen, damit dies nicht geschieht. Politik hat daher ihrem Wesen
%   nach mehr mit der niederen, materiellen Sphäre des unumgehbaren
%   Notwendigkeiten zu tun als mit der geistigen Sphäre. Es ist daher ein
%   Fehler, von der Politik den Ausdruck spiritueller Wahrheit zu
%   verlangen.
% \item Die Trennung von Religion und Politik zu fordern, bedeutet weder die
%   Religion zu leugnen noch sie auf die Privatsphäre zu begrenzen, denn
%   zwischen der politischen Öffentlichkeit und der Privatsphäre gibt es eine
%   Reihe weiterer Öffentlichkeiten (etwa die der religiösen
%   Glaubensgemeinschaften) in denen der Ausdruck und die Pflege der
%   Spiritualität in kollektiver Form möglich ist.
% \item Solange die Mehrheit der Bürger von der politischen Ordnung nicht den
%   Ausdruck ihrer religiösen Überzeugungen erwartet, kann eine nicht
%   spirituelle Grundlegung der Politik Legitimität entfalten.
% \item Da das Letztbegründungsproblem in der Ethik ohnehin noch nicht gelöst
%   ist, steht hinsichtlich der ethischen Qualität der politischen Ordnung die
%   Vertragstheorie, welche sich diesem Problem entzieht, nicht schlechter da,
%   als eine religiöse Grundlegung politischer Ordnung, welche dieses Problem
%   verschiebt. 
% \end{itemize}
 
%  Abgesehen von "`Anamnesis"' ist Voegelins
% Bewußtseinsphilosophie eher über sein gesamtes Werk verteilt als in bestimmten
% Schriften zusammengefaßt.\footnote{{\bf wichtigste bewußtseinsphilosophische
%     Passagen aufzählen aus: Form d. am.  Geistes, Rasse U. Staat,
%     Briefwechsel, Einleitung OH I,II, Anamnesis OH IV, OH V, Sammelband
%     Ordnung, Bewußtsein Geschichte, Aufsätze?}} Bei den zu untersuchenden
% Primärtexten beschränke ich mich auf den ersten und dritten Teil von
% "`Anamnesis"' sowie den Anfang von "`Order and History V"'. Diese Beschränkung
% ist teils inhaltlich und teils pragmatisch begründet. Inhaltlich habe ich
% versucht, mich auf solche Texte zu beschränken, in denen vorwiegend der
% normative Aspekt der Grundlagen politischer Ordnung zur Geltung kommt.

% \footnote{Dies wirft selbst
%   wiederum eine philosophische Grundsatzfrage auf. Wahrscheinlich besteht
%   hierin der wesentliche Unterschied zwischen Platon und den Neu-Platonisten,
%   daß für Platon der Weg zur Erkenntnis des Höchsten, der über alle
%   wissenschaftlichen Profanerkenntnise führt, einen eigenen Wert hat und eine
%   notwendige Voraussetzung zur Erkenntnis des höchsten bildet, während sich
%   die Neu-Platonisten nur noch auf das Höchste konzentrieren und dem
%   vermeintlich niederen kein Interesse mehr entgegen bringen. Ich halte es
%   hier mit dem Wort aus Goethes Faust: "`Willst du das Unendliche erreichen,
%   so schreite nur im Endlichen nach allen Seiten."' Oder anders gesagt: In
%   jeder Wahrheit steckt die höchste Wahrheit. Voegelin war zweifellos eher
%   Neu-Platonist.}

%%% Local Variables: 
%%% mode: latex
%%% TeX-master: "Main"
%%% End: 

















