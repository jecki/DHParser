%%% Local Variables: 
%%% mode: latex
%%% TeX-master: t
%%% End: 

\documentclass[11pt,a4paper, german]{book}
\usepackage{ae}
\usepackage{ngerman}
\usepackage[utf8x]{inputenc}
\usepackage{ucs}
\usepackage[T1]{fontenc}
\usepackage{t1enc}
\usepackage{type1cm}
%\usepackage{babel}


%\usepackage{fancyhdr}
%\pagestyle{fancy}

%\setlength{\headheight}{15pt}
%\addtolength{\headheight}{\baselineskip}

%\fancyfoot{}

%\renewcommand{\chaptermark}[1]{\markboth{#1}{}}
%\renewcommand{\sectionmark}[1]{\markright{#1}}
%\lhead{\leftmark}
%\rhead{\thepage}

%\renewcommand{\headrulewidth}{0pt}
%\renewcommand{\footrulewidth}{0pt}

%\addtolength{\textwidth}{-1cm}
%\addtolength{\headwidth}{\marginparsep}
%\addtolength{\headwidth}{\marginparwidth}

%\setlength{\oddsidemargin}{0cm}
%\setlength{\evensidemargin}{0cm}

%Thanks to Daniel Ferrante
%http://olympus.het.brown.edu/~danieldf/latex/
\usepackage{eso-pic}
\usepackage{color}
\usepackage{rotating}
\makeatletter
  \AddToShipoutPicture{%
    \setlength{\@tempdimb}{.5\paperwidth}%
    \setlength{\@tempdimc}{.5\paperheight}%
    \setlength{\unitlength}{1pt}%
    \put(\strip@pt\@tempdimb,\strip@pt\@tempdimc){%
      \makebox(0,0){\rotatebox{45}{\textcolor[gray]{0.92}{\fontsize{5cm}{5cm}\selectfont{Entwurf}}}}
    }
}
\makeatother


\pagestyle{plain}

\sloppy

% K21355

\begin{document}

\title{Religiöses Bewusstsein und Politische Ordnung. Eine Kritik von Eric
  Voegelins Bewusstseinsphilosophie} 

\author{Eckhart Arnold} 

\date{31. August 2007}


\begin{titlepage}

\setlength{\parindent}{0em}
%\setlength{\parskip}{3cm}
\begin{flushleft} 

{\Large Eckhart Arnold}

\setlength{\parskip}{2cm}
{\huge\bf Religiöses Bewusstsein}

\setlength{\parskip}{0.5cm}
{\huge\bf und}

\setlength{\parskip}{0.5cm}
{\huge\bf Politische Ordnung}

\setlength{\parskip}{1.0cm}
{\huge Eine Kritik von Eric Voegelins}

\setlength{\parskip}{0.5cm}
{\huge Bewusstseinsphilosophie}

\end{flushleft} 

\begin{center} 

\setlength{\parskip}{11cm}
{\footnotesize www.eckhartarnold.de} 
\end{center}

\end{titlepage}

%\maketitle

\newpage

\pagenumbering{roman} \tableofcontents

\newpage

\pagenumbering{arabic}

\setcounter{page}{1}


%%% Local Variables: 
%%% mode: latex
%%% TeX-master: "Main"
%%% End: 

\chapter{Einleitung}

% Thema: Bewußtseinsphilosophie als Grundlage politischer Ordnung/über Eric Voegelin

\section{Thema}

Das Thema dieser Arbeit ist die Bewußtseinsphilosophie Eric
Voegelins.\footnote{Zur Biographie: Eric Voegelin wurde 1901 in Köln geboren.
  1922 Promotion bei Hans Kelsen und Othmar Spann. 1929-38 Privatdozent für
  Staatslehre und Soziologie in Wien. 1938 Flucht vor den Nazis in die USA.
  1942-58 Professor of Gouvernment an der Lousiana State University in Baton
  Rouge.  1958 Professor für Politikwissenschaft in München. 1969 Rückkehr in
  die USA.  1974 Senior Research Fellow an der Hoover Institution on War,
  Revolution and Peace in Stanford. 1985 Tod. (Angaben aus: Michael Henkel:
  Eric Voegelin zur Einführung, Hamburg 1998, S.13-35., S.198-199.)} Sie wird
untersucht unter dem besonderen Aspekt der Begründung politischer Ordnung
durch religiöse Bewußtseinserfahrungen.

Eric Voegelin vertrat eine höchst eigentümliche und in der heutigen Zeit im
westlichen Kulturkreis geradezu befremdlich wirkende Auf\/fassung von den
religiösen Erfahrungen des Menschen als der notwendigen Grundlage politischer
Ordnung. Damit ein politisches Gemeinwesen über eine stabile und im ethischen
Sinne gute politische Ordnung verfügt, genügt es nach Voegelins Ansicht
keineswegs, wenn sich diese Ordnung auf ein ausgeklügeltes System von
Institutionen und auf eine wohldurchdachte Verfassung stützt. Für Voegelin muß
die politische Ordnung darüber hinaus tief im religiösen Empfinden der Bürger
verwurzelt sein. Nur dann kann sie eine ausreichende Resistenz gegenüber
inneren und äußeren Anfechtungen entwickeln, und nur dann kann ihr eine
ethische Qualität zugesprochen werden. In dieser Arbeit soll kritisch
hinterfragt werden, ob die religiöse Erfahrung tatsächlich eine notwendige
Voraussetzung politischer Ordnung bildet und ob eine solche Grundlegung der
politischen Ordnung überhaupt wünschenswert ist.

Wenn für Voegelin die politische Ordnung im religiösen Empfinden oder,
um es in seiner eigenen Terminologie zu formulieren, in den
existentiellen "`Erfahrungen"' der Bürger verwurzelt sein muß, so ist
dies natürlich nicht in der Weise zu verstehen, daß der Staat den
religiösen Bereich der menschlichen Natur für seine Zwecke einspannen
soll, wie dies die totalitären Staaten anstreben. Das religiöse
Empfinden geht nicht vom Staat oder vom gesellschaftlichen Kollektiv
aus, sondern es entspringt dem existentiellen Erleben des Einzelnen, und
nach Maßgabe dieses im individuellen Erleben verankerten religiösen
Empfindens muß die politische Ordnung gestaltet werden.  Damit dies
funktioniert, ist natürlich die Intaktheit des religiösen Empfindens von
größter Bedeutung. Die intakte "`Ordnungserfahrung"' bildet für Voegelin
nicht nur eine notwendige Voraussetzung (guter) politischer Ordnung, sie
stellt auch eine, zwar nicht allein hinreichende, aber doch stark
begünstigende Bedingung dar, gegenüber der alle pragmatischen Probleme
politischer Ordnung, wie z.B. die Einzelheiten der Verfassungsordnung,
vergleichsweise sekundär sind.

Kommt dem Unterschied zwischen intaktem und nicht intaktem
religiös-existentiellen Empfinden eine derartig große Bedeutung zu wie
bei Voegelin, so ist klar, daß eine rein funktionalistische Definition
des Religiösen (z.B. durch die gesellschaftliche Funktion, die die
Religion übernimmt) nicht ausreicht. Vielmehr ist es erforderlich, sich
auf die inhaltliche Ebene der religiösen Dogmen und
Erfahrungen\footnote{Voegelin beschränkt sich auf die Erfahrungen, da
  seinem mystischen Religionsverständnis gemäß auch die Dogmen nur
  Ausdruck von religiösen Erfahrungen (und nicht von geoffenbartem
  Wissen) seien können.} zu begeben.  Wie kann aber hier zwischen echt
und unecht, zwischen richtig und falsch unterschieden werden?  Voegelin
verfolgt in dieser Frage einen zweifachen Ansatz. Zum einen geht er
historisch vor, indem er sich bemüht, die geschichtlichen
Differenzierungsprozesse religiöser Erfahrung nachzuzeichnen und dabei
die differenziertesten Stufen religiös-existentiellen Welterlebens
ausfindig zu machen. Zum anderen versucht Voegelin, auf
bewußtseinsphilosophischem Wege das Wesen der religiösen bzw.
existentiellen Erfahrungen zu ergründen und in unmittelbarer
Selbsterfahrung nachzuvollziehen. Da letzten Endes auch die historische
Beurteilung religiöser Erfahrungen nur am Maßstab der
bewußtseinsphilosophisch ermittelten Wesensauf\/fassung möglich ist, muß
der bewußtseinsphilosophische Ansatz als der grundlegendere dieser
beiden Ansätze angesehen werden. Dieser Arbeit liegt daher die
Interpretationsannahme zu Grunde, daß die Bewußtseinsphilosophie
Voegelins innerhalb der Systematik seines Gedankengebäudes das Zentrum
einnimmt.\footnote{Diese Annahme entspricht Voegelins Selbstdeutung.
  Vgl. Eric Voegelin: Anamnesis. Zur Theorie der Geschichte und Politik,
  München 1996, im folgenden zitiert als: Voegelin, Anamnesis, S.7.}

Voegelin geht es nicht nur darum, empirisch den Zusammenhang zwischen
vorfindlichen politischen Ordnungsgefügen und den sie fundierenden
religiösen Erfahrungen aufzuweisen. Auch wenn derartige Untersuchungen
in seinem zum größten Teil geschichtlichen Oevre bei weitem überwiegen,
so verfolgt Voegelin ebensosehr die normative Absicht, durch die
bewußtseinsphilosophische Aufdeckung der religiösen Erfahrungsquellen
die verbindliche Grundlage einer humanen und totalitarismusresistenten
politischen Ordnung für die Gegenwart zu finden, welche für Voegelin in
den westlichen Demokratien, die ihm in Ermangelung religiöser Grundlagen
auf Sand gebaut schienen, noch unzureichend verwirklicht war. In dieser
Arbeit steht die Untersuchung des normativen Aspektes im Vordergrund. Es
geht mir nicht um die Frage, ob Voegelins Modellvorstellung von
politischer Ordnung auf das alte Ägypten oder das römische Kaiserreich
anwendbar ist, sondern es soll versucht werden herauszufinden, ob
Voegelins Vorstellungen in der heutigen Zeit unter den Bedingungen
pluralistischer und sich entwickelnder multikultureller Gesellschaften
noch (oder vielleicht gerade) tragfähig sind und normative Gültigkeit
beanspruchen dürfen. Letzteres ist natürlich nicht nur eine Frage von
Zeitumständen, sondern vor allem eine Frage der Begründungsqualität.

\section{Methode}

Die Untersuchungsmethode, die in dieser Arbeit angewandt wird, ist die einer
rationalen Rekonstruktion, d.h. es wird versucht, anhand einzelner Texte
Voegelins seine Thesen zu rekonstruieren und ihre Begründung kritisch zu
prüfen. Nur am Rande wird dagegen auf philologische und historische Fragen
eingegangen wie die, welche Entwicklung Voegelins Begriffe innerhalb seines
Werkes durchgemacht haben, durch welche Philosophen er beeinflußt wurde oder
welche zeitgeschichtlichen Umstände auf sein Denken Einfluß genommen haben. Im
Vordergrund steht statt dessen die Frage der Gültigkeit von Voegelins
Theorie.

% Für eine
% rationale Rekonstruktion, die auf die Frage der Gültigkeit einer Theorie
% abzielt, ist die Klärung werk- und zeitgeschichtliche Zusammenhänge der
% Theorie zwar eine Verständnisvoraussetzung, aber ihr kommt kein vordringliches
% thematisches Interesse zu.

% \footnote{Damit soll nicht gesagt werden, daß eine
%   philologische Analyse des Voegelinschen Werkes nicht höchst aufschlußreich
%   könnte, würde eine genaue Untersuchung der Herkunft von Voegelins
%   Denkfiguren und philosophischen Stichworten doch zweifellos zeigen, wie fest
%   Voegelin von seiner geistigen Prägung her an bestimmten philosophischen
%   Strömungen des 19. und 20. Jahrhunderts haftet, und daß er weit weniger aus
%   der Philosophie der Antike schöpft, als daß das seinem auch in der
%   Sekundärliteratur häufig kolportierten Selbstbild entspricht.}

% würde sie doch zweifellos zu Tage fördern, wie sehr Voegelin .
% Wollte man
% einmal versuchen, ausführlich und kritisch nachzuvollziehen, woher Voegelin
% die Denkfiguren und Stichwörter seiner Philosophie bezieht, so würde sich
% zweifellos ein anderes Bild ergeben als das gelegentlich noch in der
% Voegelin-Literatur kolportierte und im wesentlichen seinem Selbstverständnis
% entsprechende Bild des großen Gelehrten, der in gnostisch verwirrter Zeit auf
% dem mühsamen Wege anamnetischer Wiedererinnerung die beinahe verschollenen
% Schätze noetischen Ordnungswissen aus der klassischen Literatur der Antike
% hebt. Eher würde sich das Bild eines Geschichtsphilosophen ergeben, dessen
% intellektueller Horizont zwar einige Jahrtausende der Menschheitsgeschichte
% umfaßt, der von seiner geistigen Prägung her jedoch fest in bestimmten
% philosophischen Strömungen des 19. und 20. Jahrhunderts verwurzelt ist.

Gegen eine derartige Herangehensweise sind von zwei gegensätzlichen Richtungen
her Einwände denkbar. Einerseits könnte eingewandt werden, daß Voegelin
heutzutage keineswegs mehr aktuell und eine theoretische Auseinandersetzung
mit seinen Gedanken daher nicht mehr von Interesse sei. Andererseits könnte
gegen die Methode der rationalen Rekonstruktion und Kritik der Vorwurf erhoben
werden, daß sie, da einem positivistischen Wissenschaftsideal verpflichtet,
dem Denken Voegelins nicht gerecht werden könne.

Der erste Einwand ließe sich dahingehend weiter ausführen, daß Voegelin als
ein typischer Vertreter der Epoche des kalten Krieges inzwischen nurmehr eine
historische Erscheinung sei.\footnote{Dies deutet mit Vorsicht Eugene Webb an.
  Vgl. Eugene Webb: Review of Michael Franz, Eric Voegelin and the Politics of
  Spiritual Revolt: The Roots of Modern Ideology, in: Voegelin Research News,
  Volume III, No. 1, February 1997, auf:
  http://vax2.concordia.ca/\~{ }vorenews/v-rnIII2.html (Host: Eric Voegelin
  Institute, Lousiana State University. Zugriff am: 5.3.2000).} Wenn man heute
einen politischen Romantiker wie, um ein beliebiges Beispiel zu wählen,
Konstantin Frantz analysierte, so würde man auch keine Zeit damit
verschwenden, seine weltfremden Träumereien von einem christlichen Europa zu
widerlegen, sondern ihn von vornherein nur unter einer rein geistes- oder
zeitgeschichtlichen Perspektive, also gewissermaßen als ein historisches
Kuriosum betrachten. Werden derartige Vorbehalte gegen Voegelin auch selten
offen geäußert, so liegen sie doch in der Luft des wissenschaftlichen
Zeitgeistes und bilden auch unausgesprochen einen der Gründe, weshalb Voegelin
heutzutage weitgehend in Vergessenheit geraten ist. Sollte sich aber Voegelins
Theorie auch als gänzlich unhaltbar erweisen, so scheint mir eine
Auseinandersetzung mit Voegelin auf der Sachebene dennoch lohnend, weil
Voegelins Theorie als ein bestimmter Ansatz quasi-religiöser Politikbegründung
eine geistige Möglichkeit repräsentiert, die unabhängig davon, ob sie gerade
in Mode ist oder nicht, aus grundsätzlichem Interesse der Untersuchung wert
ist. Im übrigen können auch bei politikphilosophischen Grundsatzdiskussionen
Stimmungsumschwünge eintreten, die das, was noch wenige Jahrzehnte zuvor als
abwegig galt, auf einmal wieder naheliegend und vertretbar erscheinen lassen.
Dies gilt umso mehr, als auch die abstruseste Philosophie zur Grundlage
politischen Handelns und politischer Ordnung gemacht werden kann. Und wenn
einmal eine obskure Philosophie gesellschaftlich wirksam geworden ist, so
bleibt der bloße Hinweis auf ihre Abstrusität ohnmächtig, da diese Philosophie
dem Empfinden der meisten Menschen dann ganz natürlich erscheint.

Dem zweiten Einwand liegt die Frage zu Grunde, ob die Methode der rationalen
Rekonstruktion für eine Untersuchung von Voegelins Werk angemessen ist.
Voegelin wünschte sich von seinen Lesern eine ganz bestimmte
Lesehaltung, die weniger durch eine kritisch-rationale Einstellung als durch
den meditativen Nachvollzug seiner Gedanken bestimmt sein sollte, denn er
glaubte, eine besondere Art von Wissenschaft zu verfertigen, bei der es gerade
nicht auf das Aufstellen von Thesen und das kritische Abwägen von Argumenten
ankommt. Aber zugleich beanspruchte Voegelin, mit seinen Schriften die
theoretischen Grundlagen politischer Ordnung zu bestimmen. Ob diese Grundlagen
tragfähig sind, läßt sich jedoch nur überprüfen, indem man sie rational
analysiert. Die Rechtfertigung für meine, dem Denken Voegelins vielleicht
etwas fremde, analytische Herangehensweise, liegt also in Voegelins eigener
Zielvorgabe, die geistigen Grundlagen guter politischer Ordnung zu finden. Da
eine politische Ordnung für jeden, der in ihr lebt, verbindliche Geltung haben
soll, so muß ihre Begründung auch intersubjektiv nachvollziehbar sein.
Übrigens nahm Voegelin für seine Art von Politikwissenschaft in Anspruch,
daß sie rationale Wissenschaft sei. Aber dies beruht, wie noch zu zeigen sein
wird, auf einer willkürlichen Umdeutung des Begriffes der Rationalität.

Anders, als sich dies für die Methode der rationalen Rekonstruktion eigentlich
empfiehlt, erfolgt die Darstellung von Voegelins bewußtseinsphilosophischen
Schriften nicht durch eine Zuspitzung von Voegelins Aussagen auf einzelne
Thesen, sondern in der Form einer Wiedergabe seines Gedankenganges. Der Grund
hierfür besteht darin, daß Voegelins Texte in hohem Maße einem erzählerischen
Stilprinzip verpflichtet sind und sich daher gegen eine Zuspitzung auf
einzelne klare Thesen sträuben. Eine Zusammenfassung in Thesen würde deshalb
bereits ein sehr hohes Maß von Interpretation in Voegelins Texte hineintragen,
so daß nicht mehr leicht zu erkennen wäre, wie die Thesen aus Voegelins Worten
entnommen worden sind. Aus diesem Grund wird der Inhalt eines jeden
untersuchten Textes zunächst ausführlich mit eigenen Worten wiedergegeben, so
daß sich meine Interpretation leicht nachvollziehen läßt. Unmittelbar an die
Darstellung eines jeden Textes oder auch einzelner Textpassagen schließt sich
eine eingehende Kritik dieser Textpassagen an. Mag dieses Verfahren der
intermittierenden Kritik auch einen Eindruck von Voreiligkeit und
Nicht-ausreden-lassen-wollen erwecken, so ist es doch dadurch gerechtfertigt,
daß die untersuchten Texte bezüglich ihrer Entstehungszeit teilweise recht
weit auseinanderliegen und dementsprechend unterschiedliche Fragen aufwerfen.
Außerdem läßt sich eine ins Einzelne gehende Kritik nur schwer an eine
umfassende Darstellung anschließen, nach welcher dem Leser nur noch die groben
Züge des Gedankenganges im Gedächnis geblieben sind. Eine Detail-Untersuchung
ist aber beabsichtigt, denn der Wert einer Philosophie enscheidet sich weniger
an den großen Linien der ihr zu Grunde liegenden metaphyischen Weltauf\/fassung
als an der Qualität ihrer Durchführung im Detail. 

% Schließlich soll nicht
% verhehlt werden, daß in dieser Darstellungsweise meine sehr kritische Meinung
% zu Voegelin zum Ausdruck kommt. Es würde gewiß ein wenig sonderbar erscheinen,
% zunächst in aller Seelenruhe über fünfzig oder sechzig Seiten Voegelins
% Gedankengänge auszubreiten, nur um im Anschluß daran mit der Eröffnung
% aufzuwarten, daß all diese Überlegungen im Übrigen samt und sonders verkehrt
% seien.
 
\section{Quellen und Sekundärliteratur}

Eine umfassende Darstellung von Voegelins Bewußtseinsphilosophie würde,
soll es nicht bei einer bloßen Übersicht bleiben, den Umfang einer
Magisterarbeit sprengen. Bewußtseinsphilosophische Überlegungen
begleiten Voegelins Schaffen von seinen frühesten Schriften bis zu den
spätesten Werken,\footnote{Vgl. etwa das Kapitel über "`Time and
  Existence"', in: Eric Voegelin: On the Form of the american Mind,
  Baton Rouge / London 1995, S.23ff.} wobei die Bedeutung der
Bewußtseinsphilosophie in Voegelins Werk im Laufe der Zeit immer mehr
zunimmt.  Dabei geht Voegelins Bewußtseinsphilosophie fließend in seine
Geschichtsdeutung und seine politische Theorie über.\footnote{Besonders
  deutlich wird dies in der Einleitung zu Order and History I. Vgl. Eric
  Voegelin: Order and History. Volume One. Israel and Revelation, Baton
  Rouge / London 1986 (zuerst: 1956), im folgenden zitiert als:
  Voegelin, Order and History I, S.1-11.} Eine Vollständigkeit
beanspruchende Untersuchung von Voegelins Bewußtseinsphilosophie müßte
all diese Zusammenhänge mitberücksichtigen und in hohem Maße auch solche
Schriften Voegelins einbeziehen, die nicht im engeren Sinne
bewußtseinsphilosophisch genannt werden können.

Aus pragmatischen Gründen beschränkt sich diese Arbeit daher auf die
Untersuchung von "`Anamnesis"', dem einzigen ausdrücklich als
bewußtseinsphilosophisch ausgewiesenen größeren Werk, welches Voegelin zu
dieser Thematik selbst veröffentlicht hat. Weiterhin werden aus dem Werk
"`Anamnesis"', das eine Reihe von bewußtseinsphilosophischen und historischen
Aufsätzen Voegelins versammelt, nur die im engeren Sinne
bewußtseinsphilosophischen Schriften berücksichtigt, welche den ersten und
dritten Teil dieses Werkes bilden, während der zweite Teil von
"`Anamnesis"' überwiegend historische Probleme behandelt. Durch die
Beschränkung auf "`Anamnesis"' bleiben die späteren Entwicklungen von
Voegelins Bewußtseinsphilosophie außen vor. So wird Voegelins
Auseinandersetzung mit dem Thema "`Egophanie"' (Selbstbezogenheit des modernen
Menschen im Gegensatz zur Gottbezogenheit), welches in "`Order and History
IV"' einen so großen Raum einnimmt,\footnote{Vgl. Voegelin, Order and History
  IV, S.260ff.} nicht näher behandelt. Auch der Komplex der
"`consciousness-reality-language"' und das "`paradox of consciousness"', zwei
zentrale Begriffe der letzten, in "`Order and History V"' erreichten
Entwicklungsstufe seiner Bewußtseinsphilosophie, treten in "`Anamnesis"'
lediglich in der noch vergleichsweise kruden Form des dort entwickelten
vielschichtigen und paradoxen Realitätsbegriffes auf.\footnote{Vgl. Eric
  Voegelin: Order and History. Volume Five. In Search of Order, Baton Rouge /
  London 1987, im folgenden zitiert als: Voegelin, Order and History V,
  S14-18.  - Vgl. Voegelin, Anamnesis, S.304-305.} Trotz dieser
Einschränkungen umfaßt "`Anamnesis"', besonders durch den zeitlichen Abstand
der darin aufgenommenen Texte, eine große Spannbreite von Voegelins
bewußtseinsphilosophischem Denken und kann daher als durchaus repräsentativ
für Voegelins gesamte Bewußtseinsphilosophie angesehen werden.  Die Kritik an
Voegelin, die in dieser Arbeit anhand einzelner bewußtseinsphilosophischer
Schriften entwickelt wird, ist zu einem großen Teil von grundsätzlicher Art,
so daß sie sich leicht auf andere Schriften Voegelins übertragen läßt. Die
Beschränkung der Untersuchung auf einige wenige Texte ist nicht zuletzt
dadurch begründet, daß es eher durch eine eingehende Darstellung möglich ist,
Voegelins Schriften gerecht zu werden, die sich durch eine Fülle des
verarbeiteten Materials und einen verblüffenden Reichtum an interessanten
Nebengedanken und beiläufigen Überlegungen auszeichnen, als durch einen
notwendigerweise oberflächlich bleibenden Gesamtüberblick. Einer allzu großen
Fixierung auf bloße Einzelaspekte von Voegelins Bewußtseinsphilosophie wird
dadurch entgegengewirkt, daß im ersten Teil der Arbeit ein
Gesamtüberblick über die politische und historische Philosophie Voegelins
gegeben wird, in welche die Bewußtseinsphilosophie eingebettet ist.

Die Lage der Sekundärliteratur zu Eric Voegelin und zu seiner
Bewußtseinsphilosophie ist nicht in jeder Hinsicht günstig. Zwar gibt es über
Eric Voegelin und besonders zu seinem Hauptwerk "`Order and History"' schon
ein beachtliches Schrifttum,\footnote{Vgl. Geoffrey L. Price: Recent
  International Scholarship on Voegelin and Voegelinian Themes. A Brief
  Topical Bibliography, in: Stephen A. McKnight / Geoffry L. Price (Hrsg.):
  International and Interdisciplinary Perspectives on Eric Voegelin, Missouri
  1997, S.189-214. - Eine regelmäßig aktualisierte Bibliographie enthalten die
  Voegelin-Research News des Eric Voegelin Insitute der Louisiana State
  University, http://vax2.concordia.ca/\~{ }vorenews/} aber gerade zu Voegelins
Bewußtseinsphilosophie sind Einzeluntersuchungen noch recht dünn
gesät.\footnote{Eine erschöpfende Darstellung der mittleren Schaffensperiode,
  einschließlich der Bewußtseinsphilosophie des ersten Teils von Anamnesis
  liefert Barry Cooper. Vgl. Barry Cooper: Eric Voegelin and the Foundations
  of Modern Political Science, Columbia and London 1999, S.161ff. - Für die
  spätere Schaffensperiode, insbesondere "`Order and History V"': Vgl. Michael
  P. Morrissey: Consciousness and Transcendence. The Theology of Eric
  Voegelin, Notre Dame 1994, S.117ff. - Meist wird die Bewußtseinsphilosophie
  jedoch nur im Rahmen einer anderen Thematik mitbehandelt. Vgl.
  beispielsweise: Petropulos, William: The Person as `Imago Dei'. Augustine
  and Max Scheler in Eric Voegelins `Herrschaftslehre' and `The Political
  Religions', München 1997, S.35-38.} Hinsichtlich dieser Seite von Voegelins
Werk herrscht noch ein erhebliches Forschungsdefizit, zu dessen Behebung auch
diese Arbeit einen Beitrag leisten möchte. Darüber hinaus leidet die
Sekundärliteratur zuweilen an einer gewissen Einseitigkeit, die, wie es
scheint, dadurch zustande kommt, daß sie zu einem großen Teil von überzeugten
Anhängern Voegelins bestritten wird, während die vorhandenden und möglichen
Gegner Voegelins ihn offenbar mehr oder weniger ignorieren. Nicht selten wird
recht unkritisch das Selbstbild Voegelins, des großen Gelehrten, der in
gottvergessener Zeit in den Tiefen der Geschichte auf Wahrheitssuche geht,
kolportiert und geradezu eifersüchtig gegen Einwände
verteidigt.\footnote{Deutlich wird dies etwa an den heftigen Reaktionen auf
  Eugene Webbs maßvolle Voegelin-Kritik.  - Vgl. Thomas J.  Farrell: The Key
  Question. A critique of professor Eugene Webbs recently published review
  essay on Michael Franz's work entitled "'Eric Voegelin and the Politics of
  Spiritual Revolt: The Roots of Modern Ideology"', in: Voegelin Research
  News, Volume III, No.2, April 1997, auf:
  http://vax2.concordia.ca/\~{ }vorenews/v-rnIII2.html - Maben W. Poirier:
  VOEGELIN-- A Voice of the Cold War Era ...? A COMMENT on a Eugene Webb
  review, in: Voegelin Research News, Volume III, No.5, October 1997, auf:
  http://vax2.concordia.ca/\~{ }vorenews/V-RNIII5.HTML (Host jeweils: Eric
  Voegelin Institute, Lousiana State University. Zugriff am: 5.3.2000).}
Freilich ist Voegelin nicht ganz unschuldig daran, daß sein Werk unter die
Zeloten gefallen ist, sah er doch selbst in Ansichten, die zu seiner Denkweise
im Gegensatz standen, die "`Rhetorik deformierter Existenz"' am Werk, und
empfahl er einmal sogar, dem "`verführerischen Zwang [für den modernen
Menschen, E.A.], sich selbst zu deformieren"', mit den einem altägyptischen
Dichter entnommenen Worten entgegenzutreten: "`Siehe, mein Name wird übel
riechen durch dich // mehr als der Gestank von Voegelmist // an Sommertagen,
wenn der Himmel heiß ist"'.\footnote{Vgl. Eric Voegelin: Äquivalenz von
  Erfahrungen und Symbolen in der Geschichte, in: Eric Voegelin: Ordnung,
  Bewußtsein, Geschichte, Späte Schriften (Hrsg. von Peter J. Optiz),
  Stuttgart 1988, S.99-126 (S.105).} Insgesamt scheint ein gewisser Mangel
zwar nicht an einzelnen kritischen Tönen aber an kritischer Auseinandersetzung
mit Voegelin zu bestehen.\footnote{Als ein durchaus typisches Beispiel für
  diese Art von Sekundärliteratur, die fast nur aus Bestandsaufnahme, aber so
  gut wie gar nicht aus kritischer Diskussion besteht sein hier nur das
  folgende herausgegriffen: Glenn Hughes (Ed.): The Politics of the Soul. Eric
  Voegelin on Religious Experience, Lanham / Boulder / New York / Oxford 1999.
  - Als Beispiele der Voegelin-Kritik seien herausgegriffen: Mit
  gesellschaftskritischem Akzent: Richard Faber: Der Prometheus-Komplex.  Zur
  Kritik der Politotheologie Eric Voegelins und Hans Blumenbergs, Königshausen
  1984. - Ideologiekritisch vor allem gegenüber Voegelins Gnosis-Begriff:
  Albrecht Kiel: Säkularisierung als Geschichte des Unheils.  Die
  Gleichsetzung von Rationalität und Ordnung mit Katholizität in der
  Geschichtsphilosophie Eric Voegelins, in: Albrecht Kiel: Gottesstaat und Pax
  Americana. Zur Politischen Theologie von Carl Schmitt und Eric Voegelin,
  Cuxhaven und Dartford 1998, S.95-118. - Erhebliche Zweifel an der
  philologischen Genauigkeit Voegelins meldet Zdravko Planinc an: Zdravko
  Planinc: The Uses of Plato in Voegelin's Philosophy of Consciousness:
  Reflections prompted by Voegelin's Lecture, "`Structures of Consciousness"',
  in: Voegelin-Research News, Volume II, No.  3, September 1996, auf:
  http://vax2.concordia.ca/\~{ }vorenews/v-rnII3.html (Host: Eric Voegelin
  Institute, Lousiana State University. Zugriff am: 5.3.2000).}

Außer der Sekundärliteratur zu Eric Voegelin wird auch philosophische
Literatur zu den Themen, die Voegelin in seinen
bewußtseinsphilosophischen Texten anspricht, herangezogen. Hier besteht
allerdings die Schwierigkeit, daß es in der Philosophie kein Expertentum
gibt und daß man daher je nachdem, auf welche Schule man zurückgreift,
zu einer sehr unterschiedlichen Ansicht des Gegenstandes gelangen kann.
In dieser Arbeit wurden vor allem die Autoren zu Rate gezogen, die auch
Voegelin in seinen Schriften anspricht. Dies bereitet für die
Untersuchung des ersten Teils von Anamnesis keine Probleme, da klar ist,
daß Voegelin sich hier vornehmlich mit der Phänomenologie
auseinandersetzt.  Schwieriger ist dies jedoch für den dritten Teil von
"`Anamnesis"', da Voegelin hier bereits wesentlich selbständiger
vorgeht.  Weiterhin werden solche Autoren miteinbezogen, die von
Voegelin zwar nicht immer ausdrücklich erwähnt werden, auf die er sich
jedoch stillschweigend zu beziehen scheint.

\section{Aufbau}

Die Arbeit ist in drei Teile untergliedert. Der erste Teil gibt einen Grundriß
von Voegelins politischer Philosophie. Ziel ist es, die Hauptthesen von
Voegelins politischer Philosophie darzustellen sowie seinen methodischen
Ansatz zu bestimmen. Insbesondere soll gezeigt werden, wie und an welcher
Stelle bewußtseinsphilosophische Voraussetzungen in sein politisches Denken
eingehen. In diesem Teil beziehe ich mich überwiegend auf Voegelins "`Neue
Wissenschaft der Politik"',\footnote{Eric Voegelin: Die Neue Wissenschaft der
  Politik. Eine Einführung, München 1959, im folgenden zitiert als: Voegelin,
  Neue Wissenschaft der Politik.} da dieser Schrift unter Voegelins Werken am
ehesten der Charakter einer Programmschrift eigen ist. Dabei werden von
vornherein auch kritische Einwände gegen Voegelins Auf\/fassungen diskutiert.
Die Kritik dient nicht zuletzt dazu, den Problemhorizont abzustecken, der bei
der Untersuchung von Voegelins Bewußtseinsphilosophie berücksichtigt werden
muß.

Im zweiten Teil werden ausführlich Voegelins bewußtseinsphilosophische
Schriften dargestellt und einer eingehenden Detail-Kritik unterzogen. Den
Abschluß des zweiten Teils bildet die Diskussion einiger Grundprobleme von
Voegelins Bewußtseinsphilosophie, wobei die kritische Betrachtung von
Voegelins Begriff der (religiösen) Erfahrung im Zentrum steht. Es gilt dabei
kritisch Bilanz zu ziehen, ob der in Voegelins Denken zentrale Begriff der
Erfahrung hinreichend durch die bewußtseinsphilosophischen Überlegungen
Voegelins begründet und erläutert ist, um für das Verständnis und die
Gestaltung politischer Ordnung fruchtbar gemacht werden zu können.

Im letzten, mehr essayistisch gehaltenen Teil der Arbeit wird schließlich auf
einer etwas allgemeineren Ebene die Frage angesprochen, ob gute politische
Ordnung einer religiösen Grundlage bedarf. Dabei wird zu zeigen versucht, daß
eine religiös-spirituelle Grundlegung der Politik, wie sie Voegelin
vorschwebte, sowohl aus grundsätzlichen Überlegungen als auch insbesondere
unter den Bedingungen einer pluralistischen und zunehmend multikulturellen
Gesellschaft vor erheblichen Schwierigkeiten steht. Zugleich wird die Frage
aufgeworfen, ob eine rein säkulare, durch Konsens bestimmte Grundlegung
politischer Ordnung auf Basis eines Gesellschaftsvertrages denkbar ist, und ob
daher politische Ordnung des transzendenten Bezuges nicht ohnehin gänzlich
entraten kann.

% Wenn man so will ist dies nichts weiter als eine liberale
% Selbstvergewisserung. Da Theorie der liberalen Demokratie hierzulande zur Zeit
% sowieso die herrschende Meinung wiedergibt, kann dieser Teil eher knapp
% ausfallen. Die Grundthesen des dritten Teils lauten kurz gefaßt:

% \begin{itemize}
% \item Wenn die Bewußtseinsphilosophie kein objektives Wissen über die
%   Ordnung des Seins vermitteln kann, so kann sie auch auch keine Grundlage
%   politischer Ordnung bilden.
% \item Wenn Politik auf Transzendenz gegründet wird, dann wird die Religiosität
%   zu einer Angelegenheit der politischen Öffentlichkeit. Dies wirft Probleme
%   für die Religionsfreiheit und Toleranz auf.
% \item Es ist (insbesondere in einer multikulturellen Gesellschaft)
%   aussichtsreicher Konsens auf der Ebene der Werte als auf der Ebene der
%   Wertbegründung zu suchen.
% \item Die Notwendigkeit politischer Ordnung entsteht aus dem Umstand, daß
%   Menschen einander in die Quere kommen können, und deshalb Abmachungen
%   treffen müssen, damit dies nicht geschieht. Politik hat daher ihrem Wesen
%   nach mehr mit der niederen, materiellen Sphäre des unumgehbaren
%   Notwendigkeiten zu tun als mit der geistigen Sphäre. Es ist daher ein
%   Fehler, von der Politik den Ausdruck spiritueller Wahrheit zu
%   verlangen.
% \item Die Trennung von Religion und Politik zu fordern, bedeutet weder die
%   Religion zu leugnen noch sie auf die Privatsphäre zu begrenzen, denn
%   zwischen der politischen Öffentlichkeit und der Privatsphäre gibt es eine
%   Reihe weiterer Öffentlichkeiten (etwa die der religiösen
%   Glaubensgemeinschaften) in denen der Ausdruck und die Pflege der
%   Spiritualität in kollektiver Form möglich ist.
% \item Solange die Mehrheit der Bürger von der politischen Ordnung nicht den
%   Ausdruck ihrer religiösen Überzeugungen erwartet, kann eine nicht
%   spirituelle Grundlegung der Politik Legitimität entfalten.
% \item Da das Letztbegründungsproblem in der Ethik ohnehin noch nicht gelöst
%   ist, steht hinsichtlich der ethischen Qualität der politischen Ordnung die
%   Vertragstheorie, welche sich diesem Problem entzieht, nicht schlechter da,
%   als eine religiöse Grundlegung politischer Ordnung, welche dieses Problem
%   verschiebt. 
% \end{itemize}
 
%  Abgesehen von "`Anamnesis"' ist Voegelins
% Bewußtseinsphilosophie eher über sein gesamtes Werk verteilt als in bestimmten
% Schriften zusammengefaßt.\footnote{{\bf wichtigste bewußtseinsphilosophische
%     Passagen aufzählen aus: Form d. am.  Geistes, Rasse U. Staat,
%     Briefwechsel, Einleitung OH I,II, Anamnesis OH IV, OH V, Sammelband
%     Ordnung, Bewußtsein Geschichte, Aufsätze?}} Bei den zu untersuchenden
% Primärtexten beschränke ich mich auf den ersten und dritten Teil von
% "`Anamnesis"' sowie den Anfang von "`Order and History V"'. Diese Beschränkung
% ist teils inhaltlich und teils pragmatisch begründet. Inhaltlich habe ich
% versucht, mich auf solche Texte zu beschränken, in denen vorwiegend der
% normative Aspekt der Grundlagen politischer Ordnung zur Geltung kommt.

% \footnote{Dies wirft selbst
%   wiederum eine philosophische Grundsatzfrage auf. Wahrscheinlich besteht
%   hierin der wesentliche Unterschied zwischen Platon und den Neu-Platonisten,
%   daß für Platon der Weg zur Erkenntnis des Höchsten, der über alle
%   wissenschaftlichen Profanerkenntnise führt, einen eigenen Wert hat und eine
%   notwendige Voraussetzung zur Erkenntnis des höchsten bildet, während sich
%   die Neu-Platonisten nur noch auf das Höchste konzentrieren und dem
%   vermeintlich niederen kein Interesse mehr entgegen bringen. Ich halte es
%   hier mit dem Wort aus Goethes Faust: "`Willst du das Unendliche erreichen,
%   so schreite nur im Endlichen nach allen Seiten."' Oder anders gesagt: In
%   jeder Wahrheit steckt die höchste Wahrheit. Voegelin war zweifellos eher
%   Neu-Platonist.}

%%% Local Variables: 
%%% mode: latex
%%% TeX-master: "Main"
%%% End: 



















%%% Local Variables: 
%%% mode: latex
%%% TeX-master: "Main"
%%% End: 

\chapter{Die Grundzüge von Voegelins Philosophie}

\section{Voegelins theoretischer Ansatz}
\label{Grundzuege}

Voegelins theoretischer Ansatz ist durch drei zentrale Elemente bestimmt: 1)
Einer entschieden kritischen Frontstellung gegen das positivistische Ideal der
wertfreien, empirisch-tatsachenorientierten Wissenschaft, 2) Voegelins eigenem
Ideal einer normativ-ontologischen Ordnungswissenschaft, das religiöse
Erfahrungen als politisch ordnungsstiftendes Moment und zugleich als Grundlage
normativ-wissenschaftlicher Urteile einbezieht und 3) einer
Geschichtsphilosophie, der zufolge es eine Entwicklung von primitiven
("`kompakten"') zu immer "`differenzierteren"' religiösen Ordnungserfahrungen
gibt, die aber -- mit verhängnisvollen politischen Folgen -- unterbrochen
werden kann durch Phasen, in denen verfälschte Ordnungserfahrungen
("`Gnosis"') dominieren. Die Neuzeit ist für Voegelin eine solche Phase
verhängnisvoller Ordnungsstörung.  Diese drei Elemente von Voegelins
theoretischen Ansatz sollen in diesem Kapitel näher erläutert werden.

\subsection{Die Kritik des Positivismus}

Eric Voegelin entwickelte seinen eigenen wissenschaftlichen Ansatz in
ausdrücklicher Opposition zu den herkömmlichen Vorgehensweisen in den
Sozialwissenschaften, wobei er sich insbesondere gegen die "`szientistischen"'
Ansätze in den Gesellschaftswissenschaften wandte. Diese kritische Seite der
wissenschaftlichen Neuorientierung, die Voegelin in der "`Neue[n] Wissenschaft
der Politik"' vornimmt, betrifft das an den Naturwissenschaften orientierte
Methodenideal des Positivismus sowie die von Max Weber aufgestellte Forderung
der Wertfreiheit der Wissenschaft.

Voegelin unternimmt in der "`Neuen Wissenschaft der Politik"' nicht die
Auseinandersetzung mit einer bestimmten, elaborierten positivistischen
Wissenschaftstheorie. Es geht ihm vielmehr um die Charakterisierung der
geistesgeschichtlichen Strömung des Positivismus und um die Kritik des
geistigen Klimas, welches diese Strömung in den Gesellschaftswissenschaften
hervorgerufen hat. Den Begriff des Positivismus fasst Voegelin dabei
vergleichsweise weit. Seiner skizzenhaften historischen Darstellung zufolge
ging der Positivismus aus der Rezeption der Newtonschen Physik durch die
Aufklärer hervor und lief von diesem Ausgangspunkt über Auguste Comte als
seinem ersten vorläufigen Höhepunkt fort bis zur Entwicklung der Methodologie
am Ende des 19. und Anfang des 20. Jahrhunderts. Die Methodologie trägt als
eine skeptische und ideologiekritische Erscheinung allerdings auch schon den
Keim der Gegenbewegung in sich.\footnote{Vgl. Voegelin, Neue Wissenschaft der
  Politik, S.24-31. -- Eine ausführliche Darstellung von Voegelins
  Positivismuskritik in: Barry Cooper: Eric Voegelin and the Foundations of
  Modern Political Science, Columbia and London 1999, S. 67ff. -- Eine
  sorgfältige Analyse von Voegelins Positivismuskritik in der "`Neuen
  Wissenschaft der Politik"', sowie eine wohlbegründete Zurückweisung von
  dessen völlig überzogenen Vorwürfen bei: Kelsen, A New Science of Politics,
  a.a.O., S. 11ff. -- Vgl. auch Eckhart Arnold: Nachwort: Voegelins "`Neue
  Wissenschaft"' im Lichte von Kelsens Kritik, in: Kelsen, ebd., S. 109-137,
  im folgenden zitiert als: Arnold, Nachwort zu Kelsens Voegelin-Kritik, S.
  119-122.}
  
Das positivistische Denken führt nach Voegelins Ansicht dazu, dass als
Untersuchungsgegenstand der Wissenschaft nur noch dasjenige zugelassen wird,
was sich mit einem bestimmten Kanon quasi-naturwissenschaftlicher Methoden
erfassen lässt. Hierdurch wird in Voegelins Augen die Relevanzordnung der
Wissenschaft geradezu umgekehrt. Denn anstatt dass das Thema bzw. die
wissenschaftliche Fragestellung vorgegeben ist und der Wissenschaftler sich
nun nach den geeigneten Methoden zur Bearbeitung dieses Themas umsieht, gibt
nach dem positivistischen Wissenschaftsverständnis der Methodenkatalog vor,
welche Fragen überhaupt gestellt werden können.\footnote{Vgl. Voegelin, Neue
  Wissenschaft der Politik, S. 24. -- Dies ist vielleicht der einzige Punkt,
  in dem Voegelins Positivismuskritik m.E. ein berechtigtes Anliegen
  formuliert, auch wenn Voegelin es versäumt, seine Kritik durch Beispiele zu
  untermauern. Zur jüngeren Kritik an der "`methodengetriebenen"' Wissenschaft
  vgl.  Donald Green / Ian Shapiro: The Pathologies of Rational Choice Theory.
  A Critique of Applications in Political Science, New Haven \& London 1994,
  sowie neuerlich: Ian Shapiro: The Flight from Reality in the Human Sciences,
  Princeton 2005, im folgenden zitiert als: Shapiro, Flight from Reality.} Nun
gibt es aber Fragen, die nicht nur für das Leben des Einzelnen von größter
Bedeutung sind, sondern deren Beantwortung auch die Gestalt der Gesellschaft
und die Form der politischen Ordnung entscheidend prägt, welche aber unter den
methodologischen Vorgaben des Positivismus kaum angemessen untersucht werden
können. Hierzu gehören beispielsweise die Frage nach dem, was moralisch gut
und richtig ist, oder auch die Frage nach dem Sinn des Lebens oder dem Sinn
der Welt im Ganzen. Es ist offensichtlich, dass jede Gesellschaft vor der
Herausforderung steht, auf die erste dieser Fragen eine gemeinverbindliche
Antwort zu geben. Und die Antwort auf die zweite Frage ist ersichtlich
wenigstens dort von öffentlicher Bedeutung, wo die Religion einen großen
Einfluss auf die Politik ausübt. Nach dem positivistischen Verständnis ist es
jedoch unmöglich, auf irgendeine dieser Fragen eine objektive Antwort zu
geben.  Gegenstand der Wissenschaft kann nach positivistischer Auf\/fassung
daher bestenfalls sein, welche Antworten Gesellschaften oder einzelne Menschen
auf diese Fragen geben oder unter welchen Bedingungen Menschen dazu neigen,
derartige Fragen aufzuwerfen und dann auf diese oder jene Weise zu
beantworten. Ausgeschlossen ist jedoch die Erörterung moralischer oder
spiritueller Fragen durch die Wissenschaft selbst.  Solche Fragen dürfen vom
Forscher, wenn überhaupt, dann höchstens privat und nach Feierabend gestellt
werden.\footnote{Gerade dies ist es, was Max Weber mit gar nicht schlechten
  Gründen fordert. Vgl. Max Weber: Der Sinn der "`Wertfreiheit"' der
  soziologischen und ökonomischen Wissenschaften, in: Max Weber: Gesammelte
  Aufsätze zur Wissenschaftslehre, Tübingen 1988, im folgenden zitiert als:
  Weber, Wissenschaftslehre, S. 489-540 (S. 492/493).}

Genau dies hält Voegelin aber für einen untragbaren Zustand. Wichtiger
vielleicht noch als das Problem der massenhaften Anhäufung irrelevanter
Nebensächlichkeiten, welches Voegelin dem Positivismus ebenfalls
vorwirft,\footnote{Vgl. Voegelin, Neue Wissenschaft der Politik, S. 27. -- Vgl.
  auch die Diskussion des Relevanzproblems im Briefwechsel zwischen Voegelin
  und Alfred Schütz, der in dieser Frage eine klarere Auf\/fassung vertritt,
  in: Eric Voegelin / Alfred Schütz / Leo Strauss / Aron Gurwitsch:
  Briefwechsel über "`Die Neue Wissenschaft der Politik"' (Hrsg. von Peter J.
  Opitz), München 1993, 55ff. Schütz geht von einer bloß relativen Relevanz
  wissenschaftlicher Themen in Bezug auf bestimmte Fragestellungen aus.
  Voegelin gerät dagegen in Schwierigkeiten bei dem Versuch den Anspruch
  absoluter Relevanz von bestimmten Fragestellungen begründen will.} ist es
daher, dass einige der relevantesten Fragen der Menschheit vom Positivismus
unter ein wissenschaftstheoretisches Tabu gestellt werden. Wie soll, so könnte
man im Sinne Voegelins fragen, Politikwissenschaft möglich sein, wenn sie auf
die Frage, ob der Faschismus dem Kommunismus vorzuziehen sei oder beiden
vielleicht der Liberalismus, nur mit einem Achselzucken oder allenfalls mit
dem (natürlich wertfrei zu haltenden) Hinweis auf die möglichen Folgen der
Entscheidung antworten kann?  Das Problem der Möglichkeit von Werturteilen in
der Wissenschaft wird von Voegelin besonders detailliert an Max Weber
herausgearbeitet.

Max Weber vertrat die Ansicht, dass Wertfragen nicht objektiv beantwortet
werden können. Daher kann auch nicht wissenschaftlich über die Richtigkeit und
Falschheit von Werten befunden werden. Wertfragen müssen vielmehr entschieden
werden. Weber hat diese Art von Entscheidungen, die letztlich ohne rationale
Anhaltspunkte getroffen werden müssen, Voegelin zufolge auch des öfteren als
"`dämonisch"' charakterisiert.\footnote{Vgl. Voegelin, Neue Wissenschaft der
  Politik, S. 33/34. -- Zu der recht komplexen Beziehung Voegelins zu Webers
  Werk, auf die hier nicht ausführlich eingegangen werden kann, vgl. Peter J.
  Opitz: Max Weber und Eric Voegelin, in: Eric Voegelin: Die Größe Max Webers.
  (Hrsg. von Peter J.  Opitz), München 1995, S. 105-133.} Dennoch spielen Werte
in der Wissenschaft in einem anderen Zusammenhang bei Max Weber durchaus eine
Rolle.  Die Auswahl des Gegenstandes der Wissenschaft erfolgt nämlich
wertgesteuert durch die wissenschaftliche Interessenrichtung des
Forschers.\footnote{Vgl.  Max Weber: Die "`Objektivität"'
  sozialwissenschaftlicher und sozialpolitischer Erkenntnis, in: Weber,
  Wissenschaftslehre, S. 146-214.  (S. 175-185).} Dies ist einer der Punkte, an
denen Voegelins Kritik am Wertfreiheitsdogma ansetzt.  Wenn Werte nicht
wissenschaftlich begründet werden können, sie aber gleichzeitig die
Voraussetzung zur "`Konstitution des Gegenstandes der
Wissenschaft"'\footnote{Voegelin, Neue Wissenschaft der Politik, S. 37.}
bilden, dann gibt es ebenso viele Wissenschaften, wie es Werte gibt. Das
Ergebnis wäre ein Relativismus als Konsequenz der
Objektivitätsforderung.\footnote{Vgl.  Voegelin, Neue Wissenschaft der
  Politik, S. 37.} Für Voegelin hat Max Weber damit das Prinzip der wertfreien
Wissenschaft ad absurdum geführt. Voegelin anerkennt durchaus das historische
Verdienst Max Webers, welches für ihn darin besteht, dass Max Weber das
Problematische des Positivismus reflektiert hat, wenn er auch immer noch in
den positivistischen Tabus seiner Zeit befangen blieb. Dass Max Weber der
Durchbruch zu einer umfassenden, d.h. auch Werte mit einschließenden
politischen Ordnungswissenschaft nicht gelungen ist, erklärt sich Voegelin mit
eben dieser Befangenheit Webers und damit, dass Weber bei seinen historischen
Studien genau die Epochen und Denker ausließ, bei denen er auf eine solche
Ordnungswissenschaft hätte stoßen können, nämlich die Epochen der griechischen
Antike und des vorreformatorischen Christentums, in denen Denker wie Platon
und Aristoteles oder, im anderen Fall, Thomas von Aquin über eine politische
Ordnungswissenschaft verfügten.\footnote{Vgl.  Voegelin, Neue Wissenschaft der
  Politik, S.41.}

Es gibt jedoch einen Punkt, über den Voegelin bei seiner Auseinandersetzung
mit Max Weber mit einer gewissen Ungeduld hinweggeht. Dass Max Weber Werte für
rational unbegründbar hält, hängt nicht bloß mit dem ungünstigen historischen
Umstand zusammen, dass er in einer positivistisch geprägten Epoche lebte, noch
kann es ernsthaft dadurch erklärt werden, dass Max Weber die
Auseinandersetzung mit dem christlichen Mittelalter und Vertretern der
Ordnungswissenschaft wie Thomas von Aquin, Platon und Aristoteles peinlichst
vermieden hätte.\footnote{Vgl. zu dieser wenig glaubwürdigen Unterstellung
  Voegelins auch Kelsen, A New Science of Politics, a.a.O., S.  27.} Vielmehr
besaß Max Weber sachliche Gründe für die Annahme, dass sich Werte nicht
rational begründen lassen. Bisher, und dies gilt auch noch 80 Jahre nach Max
Weber, ist es noch niemandem gelungen, die Gültigkeit irgendeiner moralischen
Norm vollständig zu beweisen. Das Einzige, was erreicht worden ist, ist die
Rückführung moralischer Normen auf andere Normen.  Irgendwann einmal gelangt
man auf diese Weise jedoch zu einer obersten Norm (z.B. Menschenliebe,
kategorischer Imperativ oder dergleichen), die sich nicht auf weitere Normen
zurückführen lässt. Es kann nun behauptet werden, dass diese Norm ein "`Faktum
der Vernunft"'\footnote{Immanuel Kant: Kritik der praktischen Vernunft,
  Hamburg 1990, S.36.} oder ein Befehl Gottes ist oder dass sie von Natur aus
gilt. Aber all das sind lediglich eloquente Beteuerungen ihrer Gültigkeit und
keine rationalen Begründungen. Max Weber hielt Wertfragen deshalb
wissenschaftlich nicht für entscheidbar. Mit gutem Grund sprach er sich daher
dagegen aus, in der Wissenschaft Werturteile zu fällen, und nicht bloß, weil
er ein Opfer des positivistischen Zeitgeistes gewesen wäre.\footnote{Vgl.  Max
  Weber: Die "`Objektivität"' sozialwissenschaftlicher und sozialpolitischer
  Erkenntnis, in: Weber, Wissenschaftslehre, S. 146-214 (S. 151-157). -- Vgl.
  Max Weber: Der Sinn der "`Wertfreiheit"' der soziologischen und ökonomischen
  Wissenschaften, in: Weber, Wissenschaftslehre, S. 489-540 (S. 508). -- Zur
  Diskussion über die Wertfreiheit in den Sozialwissenschaften: Hans Albert /
  Ernst Topitsch (Hrsg.): Werturteilsstreit, Darmstadt 1971, im folgenden
  zitiert als: Albert/Topitsch, Werturteilsstreit. Darin besonders deutlich
  gegen die Möglichkeit wissenschaftlicher Wertbegründung: Walter Dubislav:
  Zur Unbegründbarkeit der Forderungssätze, S. 439-454. Noch am ehesten mit
  Voegelins Auf\/fassung vergleichbar: Jürgen Habermas: Erkenntnis und
  Interesse, S. 334-364 (S. 337).} Daher ist es auch kaum anzunehmen, dass Max
Weber seine Auf\/fassungen zur Werturteilsproblematik hätte revidieren müssen,
wenn er die Zeitalter von Aristoteles oder von Thomas von Aquin in seinen
Forschungen stärker berücksichtigt hätte, denn weder Aristoteles noch Thomas
von Aquin haben eine Methode gefunden, mit der Wertfragen wissenschaftlich
entschieden werden können.

Ebensowenig stimmt es, dass die interessengeleitete bzw. wertbezogene Auswahl
der Gegenstände wissenschaftlicher Forschung -- Voegelin spricht hier mit einer
sehr unklaren phänomenologischen Terminologie von der "`Konstitution des
Gegenstandes der Wissenschaft"'\footnote{Voegelin, Neue Wissenschaft der
  Politik, S. 37.} -- zum Relativismus führt. Von Relativismus kann nur dann
die Rede sein, wenn es sich um widersprüchliche Aussagen zu ein und demselben
Gegenstand handelt, die alle vom jeweiligen Standpunkt aus gleichermaßen
berechtigt erscheinen. Da sich die Wertbezogenheit aber nur auf die Auswahl
der Untersuchungsgegenstände bezieht und nicht die Aussagen über diese
Gegenstände selbst und die Verfahren zur Prüfung der Richtigkeit der Aussagen
betrifft, wird die Gefahr des Relativismus vermieden.\footnote{Vgl. Ernest
  Nagel: Der Einfluß von Wertorientierungen auf die Sozialforschung, in:
  Albert/Topitsch, Werturteilsstreit, S. 237-250 (S. 237-239). Eine m.E.
  Voegelins Sichtweise ähnelnde Auf\/fassung vertritt dagegen Hans Albert, in:
  Hans Albert: Kritische Vernunft und menschliche Praxis, Stuttgart 1977, S.
  71f. Albert scheint jedoch zu übersehen, dass die Normierungen, die den
  Erkenntnisprozess durchsetzen, ausschließlich von der Art der hypothetischen
  Imperative Kants sind, welche kein eigentliches Wertbegründungsproblem
  aufwerfen.} Max Weber gerät auch nicht in einen Widerspruch, wenn er
ungeachtet der Wertfreiheit der Wissenschaft den Marxismus wissenschaftlich
kritisiert, denn das Wertesystem des Marxismus bleibt von dieser Kritik
unberührt. Lediglich die Sachaussagen des Marxismus, also etwa Aussagen über
den vermuteten Verlauf der künftigen Geschichte, können wissenschaftlich
kritisiert werden.

Die Positivismuskritik Voegelins bedarf ebenfalls einer gewissen
Differenzierung: Das zentrale Motiv wenigstens des Neupositivismus ist nicht
so sehr die Forderung nach Imitation der naturwissenschaftlichen Methoden in
allen Wissensbereichen.\footnote{Vgl. andererseits die Schwächen, die Shapiro
  bei den Anhängern des logisch-empiristischen Wissenschaftsideals in den
  Gesellschaftswissenschaften diagnostiziert, in: Shapiro, a.a.O., S. 23ff.}
Dem Neupositivismus geht es vielmehr darum, dass Erkenntnis nur möglich ist,
wenn sich Kriterien anführen lassen, die es erlauben, die Richtigkeit oder
Falschheit von Behauptungen festzustellen.\footnote{Vgl. Richard von Mises:
  Kleines Lehrbuch des Positivismus. Einführung in die empiristische
  Wissenschaftsauf\/fassung, Frankfurt am Main 1990 (zuerst: Den Haag 1939),
  S. 135ff. -- Ob Voegelins Kritik tatsächlich auch auf den Neupositivismus,
  wie er vom Wiener Kreis um Moritz Schlick entwickelt wurde, abzielt, lässt
  sich der Darstellung in der "`Neuen Wissenschaft der Politik"' nicht
  unmittelbar entnehmen. Aber schwerlich kann Voegelin nur Auguste Comte im
  Auge haben, der zu der Zeit, als die "`Neue Wissenschaft der Politik"'
  erschien, in der philosophischen und wissenschaftlichen Diskussion keine
  Rolle mehr spielte.  (Vgl. Robert A.  Dahl: The Science of politics: New and
  Old, in: World Politics Vol. VII (April 1955), S. 484-489.)} Nur dann lässt
sich überhaupt zwischen echter Erkenntnis und bloßer Meinung unterscheiden.
Zumindest bezogen auf wissenschaftliche Erkenntnis ist kaum zu bestreiten,
dass diese Forderung berechtigt ist, wobei allerdings über die erforderliche
Strenge der Prüfungskriterien unterschiedliche Auf\/fassungen bestehen können.
Die Bemerkungen, die Voegelin zu dem Problem der Überprüfung
wissenschaftlicher Erkenntnisse in der "`Neue[n] Wissenschaft der Politik"'
fallen lässt, sind wenig erhellend. Sie besagen kaum mehr, als dass die
Ergebnisse einer wissenschaftlichen Untersuchung die Erwartungen des
Wissenschaftlers erfüllen müssen.\footnote{Vgl. Voegelin, Neue Wissenschaft
  der Politik, S.23. Voegelin äußert sich dort mit Worten wie diesen: "`Wenn
  die Methode das anfangs nur trübe Geschaute zu wesenhafter Klarheit gebracht
  hat, dann war sie adäquat; ..."'. -- Interessanterweise kritisiert Voegelin
  eben diesen Grundsatz, dass die Wahrheit der Prämissen durch das Ergebnis
  der Untersuchung gerechtfertigt wird, einige Jahre später bei Hegel auf das
  Schärfste. Vgl.  Eric Voegelin: Wissenschaft, Politik und Gnosis, München
  1959, im folgenden zitiert als: Voegelin, Wissenschaft, Politik und Gnosis,
  S. 55.} Dies garantiert jedoch noch keine Erkenntnis und könnte
schlimmstenfalls sogar auf die bloße Bestätigung der Vorurteile des Forschers
hinauslaufen.

Ist Voegelins Kritik des Positivismus daher zwar in mancher Hinsicht
unzulänglich, so wird in diesem Zusammenhang doch zumindest seine
Grundintention deutlich, die auf die Schaffung der Politikwissenschaft als
einer umfassenden Ordnungswissenschaft zielt, welche sich auch den Wert- und
Sinnfragen nicht verschließt.  Wie sieht nun diese umfassende
Ordnungswissenschaft aus?

\subsection{Politikwissenschaft als Ordnungswissenschaft}

Voegelin verfolgt mit seiner Politikwissenschaft sowohl eine rein
theoretische als auch eine normative Absicht. Zum einen stellt er Prinzipien
zur Analyse bestehender politischer Ordnungen auf, zum anderen glaubt er,
Kriterien angeben zu können, mit denen über den Wert einer politischen Ordnung
objektiv entschieden werden kann. Beidem liegt jedoch ein und dieselbe
dogmatische Vorstellung vom Wesen politischer Ordnung zu Grunde: Politische
Ordnung ist Voegelin zufolge stets ein Abbild der Seinsordnung, wie sie von
der jeweiligen Gesellschaft in spiritueller Erfahrung erlebt wird.

\subsubsection{"`Artikulation"' und "`Repräsentation"' als Grundfunktionen
  politischer Ordnung}

Im Zentrum des nicht-normativen Teils des politikwissenschaftlichen
Programms der "`Neuen Wissenschaft der Politik"' stehen die Begriffe
der Artikulation, der Repräsentation und der Erfahrung. Unter
Artikulation versteht Voegelin den Prozess der Entstehung einer
politischen Gesellschaft. Voegelin spricht auch davon, dass sich eine
Gesellschaft "`zur historischen Existenz"' artikuliert.\footnote{Vgl.
  Voegelin, Neue Wissenschaft der Politik, S. 61., S. 67.} Den Ausdruck
"`Artikulation"' gebraucht Voegelin deshalb, weil die Symbole, mit denen
eine Gesellschaft ihr Selbstverständnis ausdrückt, in seinen Augen
bereits einen wesentlichen Teil der gesellschaftlichen Wirklichkeit
ausmachen und dadurch die politische Gemeinschaft recht eigentlich erst
hervorbringen.\footnote{Vgl. Voegelin, Neue Wissenschaft der Politik,
  S. 50.}  Voegelin bezeichnet diesen Komplex von Symbolen, die das
Selbstverständnis einer Gesellschaft ausdrücken, auch als
"`Symbolismus"'.\footnote{Der Ausdruck "`Symbolismus"' könnte an Ernst
  Cassirers "`Philosophie der symbolischen Formen"' angelehnt sein. Auch
  Cassirer hält die Fähigkeit, Symbole zu bilden, für eine menschliche
  Leistung {\it sui generis} und betrachtet den Begriff der
  "`symbolischen Form"' daher als einen irreduziblen Grundbegriff der
  Humanwissenschaften, ohne allerdings so weitreichende Konsequenzen zu
  ziehen wie Voegelin.  Vgl. Ernst Cassirer: Versuch über den Menschen.
  Einführung in eine Philosophie der Kultur, Hamburg 1996, S. 49.} In dem
Ausdruck "`Artikulation"' klingt darüber hinaus etwas von Voegelins
spezifischem Verständnis von politischer Ordnung an. Voegelin zufolge
äußern die politischen Gesellschaften nicht bloß irgendein beliebiges
Selbstverständnis, sondern durch ihre Ordnung "`artikulieren"' sie
zugleich ihr Verständnis der Ordnung des Seins.\footnote{In der
  "`History of Political Ideas"' spricht Voegelin noch etwas plastischer
  von "`Evokation"'.  (Vgl.  Voegelin, "`Introduction"' zur "`History of
  Political Ideas"', S. 23ff.)  Möglicherweise erschien Voegelin der
  Ausdruck "`Evokation"' später zu relativistisch, indem dieses Wort
  suggeriert, dass die "`evozierte"' Realität ein gesellschaftliches
  Artefakt ist und nicht Ausdruck von spirituellen Erfahrungen.}

Zur Artikulation, d.h. zur Entstehung und Erhaltung einer politischen
Gesellschaft gehört aber auch eine Form herrschaftlicher Organisation dieser
Gesellschaft. Diesen Aspekt beschreibt Voegelin mit dem Begriff der
Repräsentation. Unter Repräsentation versteht Voegelin, abweichend von der im
politikwissenschaftlichen Kontext üblichen Bedeutung von Repräsentation als
demokratischer Volksvertretung, eine herrschaftliche Vertretung beliebiger Art
im politischen Handeln der Gesellschaft.\footnote{Vgl.  Voegelin, Neue
  Wissenschaft der Politik, S. 60 unten, S. 61 oben, wo sehr deutlich wird,
  dass Voegelin mit "`Repräsentation"' eigentlich eher Herrschaft als
  Repräsentation im Sinne einer ganz bestimmten, nämlich demokratischen Form
  der Herrschaftsbestellung meint. -- Die Verworrenheiten von Voegelins
  Repräsentationsbegriff sind mit größter Klarheit von Hans Kelsen aufgedeckt
  worden. Vgl. Kelsen, A New Science of Politics, a.a.O., S. 29-76. -- Vgl.
  auch Arnold, Nachwort zu Kelsens Voegelin-Kritik, S. 122ff.} Voegelin
unterscheidet drei Ebenen der Repräsentation: deskriptive Repräsentation,
existenzielle Repräsentation und transzendente Repräsentation.  Unter
deskriptiver Repräsentation versteht Voegelin ein beliebiges System
institutioneller Regelungen, welches handlungsbevollmächtigte Vertreter einer
politischen Gemeinschaft hervorbringt. Ausgeschlossen bleiben auf dieser Ebene
noch Fragen wie die nach der Legitimität, Effizienz und auch der tieferen
Wahrheit eines solchen Systems.\footnote{Vgl. Voegelin, Neue Wissenschaft der
  Politik, S. 57.  -- Auch wenn Voegelin es anfangs so erscheinen lässt, deckt
  sich sein Begriff deskriptiver Repräsentation nicht mit dem in der
  (westlichen) Politikwissenschaft üblichen Begriff von Repräsentation, denn
  das herkömmliche Verständnis von Repräsentation umfasst auch den
  Legitimitätsaspekt und beschränkt den Begriff andererseits auf die
  demokratische Repräsentation, so dass es ein klarer Missbrauch dieses
  Ausdruckes wäre, so wie Voegelin es tut, im Falle des Sowjetsystems von
  repräsentativer Regierung zu reden. Vgl. dazu Kelsen, A New Science of
  Politics, S. 41.} Von existenzieller Repräsentation spricht Voegelin, wenn
eine "`deskriptive Repräsentation"' vorliegt, durch die eine Herrschaft
hervorgebracht wird, deren Anordnungen Gehorsam finden und die in der Lage
ist, die vitalen Bedürfnisse einer politischen Gesellschaft (also Schutz nach
außen und Sicherheit im Inneren) zu garantieren. Der Begriff entspricht
weitgehend dem, was man üblicherweise eine legitime Herrschaft
nennt.\footnote{Vgl. Voegelin, Neue Wissenschaft der Politik, S. 77.}

Bis zu diesem Punkt bietet Voegelin, abgesehen von seiner eigenwilligen
Terminologie, nichts Ungewöhnliches. Ein grundlegend neuer Aspekt tritt jedoch
mit der dritten Bedeutungsebene von Voegelins Repräsentationsbegriff, der
"`transzendenten Repräsentation"', hinzu. Die transzendente Repräsentation
bezieht sich nicht mehr nur auf Regierung und Herrschaft, sondern auf die
politische Ordnung einer Gesellschaft im Ganzen. Alle politischen
Gesellschaften halten ihrem Selbstverständnis nach ihre eigene politische
Ordnung für die wahre Ordnung. Voegelin fasst dies so auf, dass die
politischen Gesellschaften durch ihre politische Ordnung eine höhere Wahrheit
repräsentieren.\footnote{Vgl.  Voegelin, S. 81ff.} So glaubten etwa die
Menschen in Mesopotamien oder im alten Ägypten, dass sich in ihrer politischen
Ordnung die Ordnung des Kosmos widerspiegele bzw. fortsetze. Doch ist dies
nicht die einzige Möglichkeit der Wahrheitsrepräsentation. In Platons idealem
Staat etwa repräsentiert die politische Ordnung eine Wahrheit, derer der
Philosoph im Inneren seiner Seele gewahr wird. Voegelin sieht deshalb im
Auftreten Platons den Durchbruch zu einem neuen Typus von transzendenter
Repräsentation.\footnote{Vgl. Voegelin, Neue Wissenschaft der Politik, S. 93.}

\subsubsection{Der Begriff der "`Erfahrung"' als Zentralbegriff von Voegelins
  Theorie politischer Ordnung}

Eine zentrale Stellung kommt in diesem Zusammenhang dem Begriff der Erfahrung
zu. Die Wahrheit, die die politischen Gesellschaften repräsentieren, beruht
nach Voegelins Ansicht auf einer spirituellen Erfahrung der Ordnung des Seins.
Dieser Begriff der Erfahrung verlangt auf Grund seiner großen Bedeutung für
Voegelins Verständnis von politischer Ordnung eine etwas eingehendere
Untersuchung.\footnote{Zur Genese des Begriffes der Erfahrung im Werk Eric
  Voegelins: Vgl. Peter J. Opitz: Rückkehr zur Realität: Grundzüge der
  politischen Philosophie Eric Voegelins, in: Peter J.  Opitz /
  Gregor Sebba (Hrsg.): The Philosophy of Order. Essays on History,
  Consciousness and Politics, Stuttgart 1981, S. 21-73.}

Erfahrung spielt bereits in Voegelins frühesten Schriften eine Rolle, in
welchen der Nachvollzug der seelischen Hintergründe und motivierenden
Erfahrungen zu den grundlegenden Methoden des Verständnisses philosophischer
Texte gehört.\footnote{Deutlich wird dies etwa in: Eric Voegelin: On the Form
  of the american Mind, Baton Rouge / London 1995, S. 23ff.} Voll entfaltet
und für das Verständnis politischer Ordnungen fruchtbar gemacht wird dieser
Begriff jedoch erst mit "`Order and History"'. Der Begriff der Erfahrung ist
nicht nur einer der wichtigsten Begriffe bei Voegelin, sondern angesichts der
in ihn eingehenden theoretischen Voraussetzungen zugleich auch einer der
anspruchsvollsten Begriffe Voegelins.

Um Verwechselungen zu vermeiden, soll zunächst geklärt werden, was
"`Erfahrung"' bei Voegelin nicht bedeutet: "`Erfahrung"' bedeutet bei Voegelin
nicht wissenschaftliche Empirie. Die wissenschaftliche Empirie bezieht sich
auf deutlich abgrenzbare und klar beschreibbare Sinneserfahrungen. Erfahrung
im Sinne Voegelins meint dagegen eher ein schwer fassbares inneres Erleben.
Zwar spricht Voegelin an einer Stelle davon, dass man sich zur empirischen
Überprüfung auf die Erfahrung zu beziehen habe, aber dies geschieht wohl
vornehmlich aus dem Wunsch heraus, seinen eigenen Begriff von Erfahrung an die
Stelle der wissenschaftlichen Empirie treten zu lassen, und nicht weil diese
beiden Begriffe irgendetwas gemeinsam hätten.\footnote{Vgl. Voegelin, Neue
  Wissenschaft der Politik, S.96.} Dass die Erfahrungen, von denen Voegelin
spricht, ihrerseits nicht als Kriterium für die Überprüfung wissenschaftlicher
Theorien taugen, wird besonders daran deutlich, dass Voegelin zwar einerseits
von einer Überprüfung an der Erfahrung spricht, dass dann aber, wenn sich bei
bestimmten Menschen die Erfahrungen, die er meint, nicht einstellen, die
Betreffenden kurzerhand für verstockt und ihre Erfahrungen für deformiert
erklärt werden.\footnote{Vgl.  auch Ted V.  McAllister: Revolt against
  modernity. Leo Strauss, Eric Voegelin \& the Search For a Postliberal Order,
  Kansas 1995, S. 172.  (McAllister scheint das Problematische an Voegelins
  Erfahrungsbegriff freilich nicht recht zu sehen.)}  Unter solchen
Bedingungen ist eine Überprüfung anhand der Erfahrung natürlich
ausgeschlossen.  Weiterhin meint Voegelin (um auch dieses mögliche
Missverständnis auszuschließen), wenn er von "`Ordnungserfahrung"' spricht,
niemals den Komplex täglicher Lebenserfahrungen, die wir in einer geordneten
gesellschaftlichen Umwelt haben.  Dies würde Voegelins Konzeption geradezu auf
den Kopf stellen, denn für Voegelin resultiert die gesellschaftliche Ordnung
aus der Ordnungserfahrung und nicht umgekehrt.

Was bedeutet aber nun "`Erfahrung"', wenn es sich nicht um Sinneserfahrungen
handelt? Mit "`Erfahrung"' meint Voegelin hauptsächlich ein bestimmtes inneres
Erleben mystisch-religiöser Art.  Diese Erfahrung existiert in
unterschiedlichen Varianten. So unterscheidet Voegelin die Erfahrungen
hinsichtlich ihres Niveaus nach kompakten und differenzierten Erfahrungen. Auf
dem kompakten Erfahrungsniveau, welches vor allem für die älteren
kosmologischen Gesellschaften charakteristisch ist, ist die Erfahrung als
inneres Erlebnis noch gar nicht bewusst und unauflöslich mit dem allgemeinen
Daseinsgefühl verwoben. Der Gegensatz "`kompakt-differenziert"' wird bei der
Darstellung von Voegelins Geschichtsphilosophie noch genauer erörtert werden.
Es sei nur soviel vorweggenommen, dass Voegelin an eine historische
Entwicklungstendenz hin zu immer differenzierteren Erfahrungen glaubte. Obwohl
nämlich die Erfahrung als inneres Erlebnis eine höchst individuelle, ja
geradezu intim persönliche Angelegenheit darstellt, ist sie dennoch durch ein
erstaunliches Maß von gesellschaftsinterner Einförmigkeit gekennzeichnet. Alle
Individuen einer Gesellschaft haben bei Voegelin offenbar die gleichen
seelischen Erlebnisse, solange bis ein Prophet oder Philosoph kommt und ihnen
eine neue Art des seelischen Empfindens nahe bringt.\footnote{Nicht eindeutig
  lässt sich übrigens die Frage klären, ob und wodurch sich bei Voegelin die
  Erfahrung des Propheten von den Erfahrungen seiner Anhänger unterscheidet:
  Gibt es (1.) keinen Unterschied oder besteht (2.) bloß ein Unterschied der
  Intensität oder liegt (3.) auch hier ein Unterschied der Differenziertheit
  vor? Für das Letztere spricht, dass es bei Voegelin gelegentlich den
  Anschein hat, als sei nur eine gesellschaftliche Elite starker Seelen der
  differenziertesten Erfahrung fähig. Vgl. Voegelin, Neue Wissenschaft der
  Politik, S. 172-174. Die Theorie, die Voegelin an dieser Stelle vertritt,
  ist übrigens in jeder Hinsicht unglaubwürdig: 1. Unsicherheit ist -- anders
  als Voegelin behauptet -- gewiss nicht das Wesen des Christentums. Jeder
  christliche Priester wird uns ganz im Gegenteil bestätigen, dass gerade der
  Glaube der unsicheren, zufälligen und schutzlosen Existenz des Menschen Halt
  und Sicherheit zu geben vermag. (Vgl. dazu auch Kelsen, A New Science of
  Politics, a.a.0., S. 87/88.) 2. Die Annahme, dass die gnostischen
  Turbulenzen des Mittelalters und der Neuzeit ursprünglich dadurch verursacht
  wurden, dass mit der Ausbreitung und Zunahme höherer Bildung im Zuge der
  Verstädterung im Spätmittelalter zunehmend auch Unberufene mit der vollen
  und, wie Voegelin glaubt, nur für ganz starke Seelen erträglichen Wahrheit
  des Christentum in Berührung kommen, kann man kaum ernst nehmen.
  Möglicherweise entspringt der Geistes- und Seelenaristokratismus, der dabei
  zum Ausdruck kommt, Voegelins Beschäftigung mit den Schriften des
  George-Kreises in den 20er und 30er Jahren. Zu Voegelins George-Rezeption
  vgl. Thomas Hollweck: Der Dichter als Führer. Dichtung und Repräsentanz in
  Voegelins frühen Arbeiten, München 1996.}

Wenn die Erfahrung ein inneres Erleben ist, so stellt sich die Frage, was dort
eigentlich erlebt wird. Was ist der Inhalt der Erfahrung? Dies ist eine Frage,
bei deren Beantwortung auch Voegelin vor großen Schwierigkeiten stand.
Oberflächlich könnte der Eindruck entstehen, dass je nach historischem
Erfahrungsniveau unterschiedliche Dinge erfahren werden: In der kosmischen
Erfahrung wird der Kosmos erfahren, auf differenzierterem Erfahrungsniveau
dagegen wird die Transzendenz erfahren.\footnote{Siehe dazu auch Fußnote
  \ref{FNErfahrung}.} Aber Voegelin wollte es nicht bei einem unvermittelbaren
Gegensatz zwischen den verschiedenen Erfahrungstypen bewenden lassen. In
seinen Augen ist die Erfahrung der Transzendenz schon auf kompaktem Niveau
unbewusst mitgegenwärtig. Will man die übergreifenden Gemeinsamkeiten der
unterschiedlichen Erfahrungstypen herausstellen, so lässt sich in etwa
festhalten, dass nach Voegelins Vorstellung in jedem Falle das Sein im Ganzen
und die Stellung des Menschen im Sein erfahren wird, nur dass auf kompaktem
Erfahrungsniveau das Sein als sinnhaft geordneter Kosmos erlebt wird, während
es auf differenziertem Niveau als Stufenfolge immanenter und transzendenter
Seinsstufen erfasst wird.  Entscheidend ist, dass in jedem Falle die
ontologische Ordnung ein und desselben Seins erfahren wird.\footnote{Vgl.
  Voegelin, Anamnesis, S. 305.}

\subsubsection{Von der Ordnungserfahrung zur politischen Ordnung}

Die Beziehung zwischen Erfahrung und Ordnung stellt sich bei Voegelin in der
Weise dar, dass die spirituelle Erfahrung die Quelle der politischen Ordnung
ist. Man könnte etwas überspitzt sagen, dass bei Voegelin die Erfahrung die
Basis bildet, während die politischen Ideen und Institutionen den Überbau
verkörpern, wobei Voegelin allerdings nicht gänzlich leugnet, dass es auf der
Ebene pragmatischer Politik auch Probleme gibt, deren Lösung weitgehend
unabhängig von dem ist, was sich auf der spirituellen Basis-Ebene abspielt.
Ordnung existiert dabei auf insgesamt drei Ebenen: Als Ordnung des Seins, als
Ordnung der Seele und als Ordnung der Gesellschaft.\footnote{Als viertes
  könnte noch die "`Ordnung der Geschichte"' hinzugefügt werden, die in der
  Regel allerdings keine Voraussetzung der politischen Ordnung einer
  Gesellschaft darstellt (außer m.E. in dem denkbaren Fall sich selbst primär
  geschichtlich legitimierender Gesellschaften), sondern sich für Voegelin
  umgekehrt aus der Abfolge politischer Ordnungen in der Geschichte ergibt.}
Die Reihenfolge dieser Ordnungen ist nicht umkehrbar: Zunächst existiert die
Ordnung des Seins. Diese wird vom Menschen erfahren und verleiht ihm dadurch
eine Ordnung der Seele, welche sich wiederum auf die Ordnung der Gesellschaft
auswirkt.  Voegelin scheint es für ausgeschlossen zu halten, dass es zu diesem
Weg, eine Ordnung der Seele und eine Ordnung der Gesellschaft zu erlangen,
eine legitime Alternative gibt.\footnote{Vgl. Voegelin, Anamnesis, S. 349.} Es
handelt sich dabei um einen der vielen dogmatischen Grundsätze von Voegelins
Theorie, die er voraussetzt aber niemals begründet.

Die Ordnungserfahrungen unterschiedlicher Kulturen sind nun allerdings häufig
höchst gegensätzlich beschaffen und stehen oft in unvereinbarem Gegensatz
zueinander.\footnote{Vgl. Voegelin, Neue Wissenschaft der Politik, S. 86-90.}
Voegelin hält es dennoch für möglich, politische Ordnungen nach ihrer
Wertigkeit zu unterscheiden.\footnote{Vgl.  Voegelin, Neue Wissenschaft der
  Politik, S. 91.} Maßstab hierfür sind nicht irgendwelche moralischen Ideale,
etwa Humanität oder Gerechtigkeit, sondern die spirituelle Erfahrung selbst.
Hieraus ergibt sich die normative Zielsetzung. Um zur normativ richtigen
politischen Ordnung zu gelangen, ist es erforderlich, eine spirituelle
Empfindsamkeit auf höchstem Niveau zu kultivieren, was Voegelin in Anlehnung
an Henri Bergson als das "`Öffnen der Seele"' bezeichnet.\footnote{Vgl. Henri
  Bergson: Die beiden Quellen der Moral und der Religion, Olten 1980, S. 33-36.
  Bei Voegelin wird das "`Öffnen der Seele"' im Gegensatz zu Bergson jedoch
  eher sensitiv als schöpferisch gedacht.} Dieses "`Öffnen der Seele"'
ermöglicht es, die Ordnung des Seins angemessen zu erfahren und dadurch einen
"`autoritativ"' gültigen Maßstab für die richtige politische Ordnung zu
gewinnen.\footnote{Vgl. Voegelin, Order and History II, S. 6/7.}

\subsubsection{Probleme der Voegelinschen Konzeption politischer Ordnung}

An dieser Stelle steht die normative Konzeption Voegelins vor einem
schwierigen Problem: Wie kann gültig zwischen richtiger und falscher Erfahrung
von der Ordnung des Seins unterschieden werden? In der "`Neuen Wissenschaft
der Politik"' gibt Voegelin auf diese Frage keine zufriedenstellende Antwort.
Der Hinweis auf das unterschiedliche, kompakte oder differenzierte Niveau von
Erfahrungen hilft kaum weiter, da gerade das Niveau aus der jeweiligen Sicht
der unterschiedlichen Erfahrungen gegensätzlich beurteilt werden
dürfte.\footnote{Das Problem der Relativität der Wertungen wird von Voegelin
  zwar öfters angesprochen und manchmal auch eine Weile verfolgt (z.B.
  anlässlich der Interpretation von Aristoteles in: Eric Voegelin: Order and
  History. Volume Three. Plato and Aristotle, Baton Rouge / London 1986
  (zuerst: 1957), im folgenden zitiert als: Voegelin, Order and History III,
  S.299-302.), aber niemals glaubhaft gelöst.}  Zweifelhaft ist auch Voegelins
Voraussetzung, dass es eine Ordnung des Seins gibt, die jenseits
naturgesetzlicher Bestimmtheit einen sinnhaften Zusammenhang der Dinge
herstellt. Selbst wenn eine solche Ordnung des Seins objektiv vorhanden wäre,
so ist noch längst nicht geklärt, ob diese Ordnung des Seins auch dazu
geeignet ist, die Grundlage der politischen Ordnung einer Gesellschaft
abzugeben. Woher können wir die Sicherheit nehmen, dass die Normen, die aus der
Ordnung des Seins abgeleitet sind, moralisch akzeptabel und mit den
praktischen Erfordernissen der Politik verträglich sind? Ohne eine schlüssige
Antwort auf diese Frage dürfte Voegelins normatives Programm, welches die
Revitalisierung eines spirituell-religiös eingebundenen Politikverständnisses
in der heutigen Zeit anstrebt, kaum bei der Gestaltung der gesellschaftlichen
und politischen Praxis hilfreich sein.

Aber nicht nur die normative Seite von Voegelins politikwissenschaftlichem
Ansatz bereitet Schwierigkeiten. Auch Voegelins analytische Konzeption ruht
auf einer Reihe von sehr anspruchsvollen und längst nicht restlos geklärten
Voraussetzungen. Schon die Annahme, dass alle politischen Ordnungen auf einer
metaphysischen Erfahrung der Ordnung des Seins beruhen müssen, fordert
Widerspruch heraus, denn offensichtlich kommen die liberalen Demokratien ohne
eine derartige Grundlage aus. Voegelin versucht diese Tatsache zu leugnen,
indem er entweder die Situation des Liberalismus als höchst gefährdet und
prekär darstellt, weil ihm eine solche Grundlage fehlt,\footnote{Vgl. Eric
  Voegelin: Der Liberalismus und seine Geschichte, in: Karl Forster (Hrsg.):
  Christentum und Liberalismus, München 1960, S. 13-42 (S. 35-42).} oder den
liberalen Demokratien unterstellt, in verschleierter Form (als "`common
sense"') doch ein solches metaphysisches Ordnungswissen zu
konservieren.\footnote{Vgl. Voegelin, Anamnesis, S. 352-354. Vgl. Voegelin,
  Neue Wissenschaft der Politik, S. 259.}  Hier zeigt sich wiederum die an
Voegelins spirituellem Erfahrungsbegriff zu Tage getretene
erkenntnistheoretische Gefahr des normativ-ontologischen Ansatzes. Wird
nämlich ein und dieselbe Theorie zur Sacherklärung wie zur normativen Kritik
verwandt, so liegt es nahe, diejenigen Fälle, die der Theorie widersprechen
könnten, nicht als falsifizierende Gegenbeispiele in Betracht zu ziehen,
sondern als illegitime Fälle der normativen Kritik zu
unterwerfen.\footnote{Dies wird auch besonders an Voegelins Behandlung der
  Philosophiegeschichte deutlich.  Philosophien, die Voegelin nicht als
  Artikulation von Seinserfahrungen deuten kann, verwirft er in der Regel als
  törichte oder gefährliche Abirrungen.}

Darüber hinaus wirft Voegelins Erfahrungsbegriff als methodisches
Analysewerkzeug einige Probleme auf. Die Aufgabe des Verstehens politischer
Ordnung besteht für Voegelin darin, in die Erfahrungen der entsprechenden
Gesellschaft oder, wenn es sich um die Untersuchung einer politischen Theorie
handelt, in die Erfahrung des entsprechenden Theoretikers einzudringen.
Voegelin unterscheidet dieses Eindringen in die motivierenden Erfahrungen klar
von der bloßen Rekonstruktion einer Theorie auf der Ebene der politischen
Ideen.\footnote{Vgl.  Voegelin, Neue Wissenschaft der Politik, S. 115-117,
  S. 176. -- Auf Seite 176 schreibt Voegelin: "`Es wird .. nicht überflüssig
  sein, sich des Prinzips zu erinnern, daß die Substanz der Geschichte auf der
  Ebene der Erlebnisse, nicht auf der Ebene der Ideen zu finden ist"'.} Die
Ideen sind nur die Oberfläche, während die Erfahrungen das seelische Innere
repräsentieren, auf das es eigentlich ankommt. Um also beispielsweise die
politische Theorie Platons zu verstehen, genügt es nicht zu untersuchen,
welche Institutionen und Gesetze Platon vorschlägt. Vielmehr ist danach zu
fragen, welche motivierenden inneren Erlebnisse Platons Denken zu Grunde
liegen. Aber wie kann man den seelischen Erfahrungshintergrund einer Theorie
sicher rekonstruieren? Das Verfahren, welches Voegelin zur Ergründung der
Erfahrungen anwendet, scheint eines der Innervation mit anschließender
Selbstauslegung zu sein. Naturgemäß sind daher die Ergebnisse, zu denen
Voegelin gelangt, stark subjektiv gefärbt. So findet Voegelin beispielsweise
bei Thomas von Aquin die tiefste und reinste Transzendenzerfahrung, während er
der Reformation eine echte Erfahrungsgrundlage offenbar nicht in gleichem Maße
zubilligen will.\footnote{Vgl. Voegelin, Neue Wissenschaft der Politik,
  S. 188.} Oft lässt sich bereits die Auswahl der von Voegelin als relevant
eingestuften und zur Deutung herangezogenen Quellen nicht ohne weiteres
nachvollziehen. Indessen muss eingeräumt werden, dass die Beweggründe des
Handelns und Denkens von Menschen in der Tat häufig in inneren seelischen
Regungen bestehen, die als solche niemals nach außen dringen, so dass man in
derartigen Fällen entweder auf jede Chance des Verstehens ganz verzichten oder
einen Versuch auf dem unsicheren Wege psychologischer Einfühlung wagen muss.
Nur bleibt es dann immer noch grundsätzlich fragwürdig, ob für das Verständnis
einer Theorie das Verständnis des Theoretikers und seiner Motive überhaupt
eine notwendige Voraussetzung bildet. Die Wahrheit oder Falschheit einer
Theorie entscheidet sich schließlich nicht an den Motiven des Erfinders der
Theorie.
% \footnote{Für Voegelin gelten als vollwertige politische Theorien von
%   vornherein nur diejenigen Theorien, die als Ausdruck von
%   Transzendenzerlebnissen verstanden werden können. Unter dieser Vorgabe
%   wären die seelischen Hintergründe in der Tat relevant.}

Im Ganzen beruht Voegelins politikwissenschaftliches Paradigma mit seiner
starken Betonung der religiös-mystischen Erfahrung in hohem Maße auf
bewusstseinsphilosophischen oder sogar theologischen Voraussetzungen. Dass es
Voegelin kaum gelingt, diese Voraussetzungen hinreichend plausibel zu
begründen, wird auch bei der Analyse seiner Bewusstseinsphilosophie in Kapitel
\ref{VoegelinsBewusstseinsphilosophie} deutlich werden.

\section{Voegelins Geschichtsdeutung}

Eine sehr große Bedeutung misst Voegelin der Geschichte und der
Geschichtlichkeit der menschlichen Existenz bei. Voegelin folgt damit nicht
nur einer Mode seiner Zeit, sondern die Untersuchung der historischen
Entwicklung der politischen Ordnungsvorstellungen bildet neben der
Erfahrungsanalyse auch einen wichtigen Bestandteil von Voegelins Methode
des Verständnisses politischer Ordnungsvorstellungen. Eine politische
Ordnungsvorstellung verstehen heißt bei Voegelin, ihre motivierenden
Erfahrungen aufzudecken und sie auf ihre historische Urform zurückzuführen. Um
ein vollständiges Bild von Voegelins politischem Denken zu geben, ist es daher
unerlässlich, auch auf seine Geschichtsdeutung einzugehen. Dabei soll die
Darstellung von Voegelins Geschichtsphilosophie, d.h. seiner Grundvorstellung
vom Ablauf und der Bedeutung der Geschichte im Vordergrund stehen.

Voegelin hat seine Geschichtsdeutung neben der "`History of Political Ideas"'
vor allem in den fünf Bänden von "`Order and History"', seinem eigentlichen
wissenschaftlichen Hauptwerk, entfaltet.\footnote{Angaben im
  Literaturverzeichnis.} Die philosophischen Grundlagen dieser
Geschichtsdeutung legt Voegelin dabei insbesondere in den Einleitungen zu den
Einzelbänden dar.  "`Order and History"' lässt sich am ehesten als eine
Geschichte der spirituellen Entwicklung der Menschheit in theologischer
Absicht charakterisieren. Um eine Geschichte der spirituellen Entwicklung
handelt es sich, weil Voegelin sich darin fast ausschließlich der Deutung von
religiösen und (stets spirituell interpretierten) philosophischen
Weltauf\/fassungen widmet, während der Zusammenhang dieser Weltauf\/fassungen
mit der gesellschaftlichen und institutionellen Ordnung nur am Rande behandelt
wird.  Von einer theologischen Absicht bei diesem Unternehmen kann man deshalb
sprechen, weil Voegelin sich nicht darauf beschränkt, die unterschiedlichen
religiösen Vorstellungswelten darzustellen, sondern das Ziel verfolgt
aufzuzeigen, wie sich aus der geistigen Entwicklung der Menschheit nach und
nach so etwas wie spirituelle Wahrheit herausschält. Diese Geschichtsdeutung
wird von Voegelin durch eine zum Teil an anderer Stelle dargelegte historische
Metaphysik überwölbt, nach welcher die Geschichte als ein theogonischer Prozess
aufzufassen ist, in dessen Verlauf sich ein transzendentes ewiges Sein in der
Zeit verwirklicht, indem es über das Medium des menschlichen Bewusstseins in
die Immanenz eindringt.\footnote{Zu Voegelins Geschichtsphilosophie: Vgl.
  Eugene Webb: Eric Voegelin. Philosopher of History, Seattle and London 1981.
  -- Vgl.  Jürgen Gebhardt: Toward the process of universal mankind: The
  formation of Voegelin's philosophy of history, in: Ellis Sandoz (Hrsg.):
  Eric Voegelins Thought. A critical appraisal, Durham N.C. 1982, S. 67-86.}

\subsection{Geschichte als Geschichte der spirituellen Entwicklung der Menschheit}

Voegelins Geschichtsphilosophie ist im wesentlichen die einer Entwicklungs-
und Fortschrittsgeschichte. Anders als die Fortschrittsgeschichten
beispielsweise der Aufklärer ist Voegelins Fortschrittsgeschichte jedoch eine
Geschichte des spirituellen und nicht des moralischen oder technischen
Fortschritts. Dieser Geschichte des spirituellen Fortschritts liegt allerdings
ein ahistorischer Kern in Form einer religiös-existenzialistischen Metaphysik
zu Grunde, die Voegelins Auf\/fassung von der existenziellen Situation des
Menschen widerspiegelt: Der Mensch findet sich in einer Welt wieder, deren
Sinn er nicht kennt. Er ist sich zwar dunkel bewusst, dass er in dieser Welt
eine Rolle zu spielen hat, die er nicht selbst bestimmen darf, dennoch weiß
er nicht, was für eine Rolle dies ist. Anfänglich kann er diese Rolle nur
schwach ahnen, doch er glaubt, dass sie etwas mit der Ordnung des Seins zu tun
hat.\footnote{Vgl. Voegelin, Order and History I, S. 1/2.} Das Trachten des
Menschen zielt nun darauf ab, diese Ordnung zu finden und seine eigene
Existenz sowie die Ordnung der Gesellschaft in Einklang mit ihr zu bringen,
weil er hofft, dadurch seiner flüchtigen Existenz etwas mehr Dauerhaftigkeit
zu verleihen. Die Suche nach der Ordnung des Seins bestimmt nun die
Entwicklung der Geschichte. Die Dynamik dieser Entwicklung entspringt den
wechselnden Auf\/fassungen davon, was die richtige Ordnung des Seins ist. Da
der Mensch die wahre Ordnung des Seins nur ahndungsvoll spüren kann, ist es
ihm nicht möglich, seiner Auf\/fassung von der richtigen Ordnung anders
Ausdruck zu verleihen als dadurch, dass er sein Gefühl der Ordnung bzw. seine
Ordnungserfahrung durch Analogien in Symbole fasst. Aus den solcherart
artikulierten Ordnungsauf\/fassungen bestehen die bereits erwähnten
Symbolismen.  In der geschichtlichen Abfolge der Symbolismen glaubt Voegelin
nun einen Fortschritt erkennen zu können, der, wie Voegelin es nennt, von
"`kompakteren"' zu "`differenzierteren"' Symbolismen
führt.\footnote{\label{FNErfahrung} Die Einleitung von Order and History I legt
  die Auf\/fassung nahe, dass es stets dieselbe Seinserfahrung ist, die nur
  unterschiedlich vollkommen artikuliert wird. (Vgl. Voegelin, Order and
  History I, S.1-11.) Im Schlusskapitel von "`Anamnesis"' unterscheidet
  Voegelin dann unterschiedlich differenzierte Ordnungserfahrungen. Nur das
  Sein (bzw. die Realität) bleibt dasselbe, und sogar dies gilt nur unter
  Einschränkungen.  (Vgl. Voegelin, Anamnesis, S. 286ff.)}

Der Fortgang von einem Symbolismus zum nächsten tritt oft plötzlich und
sprunghaft infolge neuer spiritueller Erlebnisse einzelner Personen ein, die
Voegelin als "`spirituelle Ausbrüche"' bezeichnet und die sich von den
Menschen, denen sie widerfahren, auf den Rest der Gesellschaft übertragen
(sofern dieser nicht gerade an einer törichten Verstocktheit leidet). Wenn der
Übergang sehr plötzlich eintritt und der Unterschied zwischen dem alten und
dem neuen Symbolismus besonders groß ist, dann spricht Voegelin von einem
"`Sprung im Sein"'.\footnote{Vgl. Voegelin, Order and History I, S. 123. Dort
  definiert Voegelin den Begriff "`Sprung im Sein"' als die "`Entdeckung des
  transzendenten Seins als die Quelle der Ordnung im Menschen und der
  Gesellschaft"' (meine Übersetzung, E.A.). Nach dieser Definition dürfte es
  (wenigstens innerhalb der Geschichte einer Zivilisation) eigentlich nur
  einen einzigen "`Sprung im Sein"' geben. Allerdings gebraucht Voegelin den
  Ausdruck auch häufig als Synonym für "`spiritueller Ausbruch"'.} Ein solcher
"`Sprung im Sein"' fand zum Beispiel statt, als Moses das Volk Israel aus
Ägypten führte, denn dabei wurde -- abgesehen davon, dass dieses Ereignis die
Geburtsstunde des Monotheismus war -- der kosmische Symbolismus, welcher
typischerweise mit einer zyklischen Geschichtsauf\/fassung verbunden ist,
durch den völlig neuartigen {\em historischen Symbolismus}
ersetzt.\footnote{Vgl.  Voegelin, Order and History I, S. 116ff.}  Später hat
Voegelin diese Auf\/fassung allerdings teilweise revidiert, nachdem er
festgestellt hatte, dass unabhängig von diesem einmaligen Ereignis auch andere
Völker auf die Idee gekommen waren, die Geschichte nicht nur zyklisch zu
betrachten.\footnote{Vgl.  Voegelin, Anamnesis, S. 79ff. -- Vgl. Eric Voegelin:
  Order and History.  Volume Four. The Ecumenic Age, Baton Rouge / London 1986
  (zuerst: 1974), im folgenden zitiert als: Voegelin, Order and History IV, S.
  7-13.} Weitere wichtige Übergänge sind für Voegelin unter anderem die
Entwicklung vom Judentum zum Christentum und der Übergang von der Mythologie
zur Philosophie im antiken Griechenland. Wenn Voegelin in diesen Übergängen
einen Fortschritt sieht, so stellt sich natürlich die Frage, was die
nachfolgenden Symbolismen gegenüber den vorhergehenden als überlegen
auszeichnet.  Warum ist die mosaische Religion der ägyptischen überlegen, und
was verleiht dem Christentum vor dem Judentum den Vorzug, fortschrittlicher zu
sein?  Voegelin versucht meist, solche Wertungen mit der Behauptung der
größeren Differenziertheit des seiner Ansicht nach besseren Symbolismus zu
begründen.  Da dem Konzept der Differenzierung für Voegelins Vorstellung vom
Fortschritt (oder auch gelegentlichem Rückschritt) der Geschichte eine
zentrale Bedeutung zukommt, soll es etwas ausführlicher untersucht werden.

\subsection{Exkurs: Die Begriffe "`Kompaktheit"' und "`Differenzierung"'}
\label{KompaktDifferenziert}
Das Begriffpaar "`kompakt-differenziert"' ist neben dem Begriff der
(spirituellen) Erfahrung ein weiteres großes Feigenblatt der Voegelinschen
Geschichtsphilosophie, denn mit diesem Begriffspaar kaschiert Voegelin eine
Reihe von Begründungsproblemen, Unklarheiten und fragwürdigen Voraussetzungen.
Der Gegensatz "`kompakt-differenziert"' kann zunächst auf einer rein formalen
Ebene verstanden werden. Aber es zeigt sich rasch, dass die formale Bedeutung
nicht ausreicht, um alle Funktionen dieses Gegensatzpaares zu rechtfertigen.

Auf der formalen Ebene bedeutet Differenzierung das Auseinandertreten von
zuvor wesensmäßig nicht unterschiedenem Sein in unterschiedliche Seinsklassen.
Insbesondere im Auseinandertreten von immanentem weltlichen und transzendentem
göttlichen Sein sieht Voegelin einen bedeutenden Differenzierungsfortschritt.
In etwas stärkerer Anlehnung an die Bewusstseinsphilosophie kann die
Essenz der formalen Bedeutungsebene dieses Begriffspaares in etwa
folgendermaßen wiedergegeben werden: Zunächst findet sich der Mensch vor einer
verwirrenden Vielfalt von Bewusstseinserlebnissen wieder. Erst nach und nach
und unter großen Unsicherheiten lernt der Mensch das Innere vom Äußeren, das
Transzendente vom Immanenten und die immanenten Dinge voneinander zu
unterscheiden.\footnote{Vgl. Order and History I, S. 3.} Soweit beschreibt der
Begriff der Differenziertheit lediglich gewisse phänomenale Eigenschaften von
Weltauf\/fassungen.

Unzulänglich bleibt der rein formale Differenzierungsbegriff, weil es auf der
phänomenalen Ebene oft der Willkür überlassen bleibt, was als kompakt und was
als differenziert bezeichnet wird. So vertritt Voegelin beispielsweise die
Ansicht, dass der Monotheismus wegen der deutlicheren Erkenntnis des
welttranszendenten Charakters des Göttlichen differenzierter ist als der
Polytheismus.\footnote{Vgl. Eric Voegelin: Die geistige und politische Zukunft
  der westlichen Welt (Hrsg. von Peter J. Opitz und Dietmar Herz), München
  1996, S. 25.} Aber ebensogut könnte man behaupten, dass der Polytheismus
differenzierter ist, weil im Polytheismus die vielfältigen Funktionen des
undifferenziert Göttlichen des Monotheismus deutlich auf eine Vielzahl von
Göttern verteilt sind. Das Begiffspaar "`kompakt-differenziert"' drückt auf
dieser Bedeutungsebene ähnlich wie viele andere abstrakte Begriffspaare (z.B.
"`formal-material"') lediglich einen Unterschied aus, ohne diesen zu
qualifizieren.

Damit die Unterscheidung zwischen kompakten und differenzierten Symbolismen zu
wertenden Vergleichen, wie Voegelin sie anstellt, herangezogen werden kann,
müssen noch weitere Voraussetzungen erfüllt sein: Die verglichenen Symbolismen
müssen sich auf denselben Gegenstand beziehen, damit ein Vergleich
stattfinden kann, und es muss ein gültiger Bewertungsmaßstab vorhanden sein, um
die Bewertung der Symbolismen durchzuführen.

Die Voraussetzung, dass sich die verglichenen Symbolismen auf den gleichen
Gegenstand beziehen müssen, ist nicht schon dann erfüllt, wenn beide
Symbolismen eine Antwort auf dieselbe Herausforderung geben, etwa auf die
Frage nach dem Sinn der Welt oder nach dem Wesen Gottes. An einem Beispiel
lässt sich dies verdeutlichen: Voegelin ist der Ansicht, dass die christliche
Gnadenlehre gegenüber dem bei Platon und Aristoteles vorherrschenden
Gottesverständnis eine Differenzierung darstellt, da bei den griechischen
Philosophen Gott bloß Ziel menschlicher Sehnsucht ist, während nach
christlichem Verständnis Gott dieser Sehnsucht durch die Gnade auch entgegen
kommt.\footnote{Vgl. Voegelin, Neue Wissenschaft der Politik, S. 113-114.}
Gegen dieses Argument liegt freilich der Einwand nahe, dass es sich hier um
unterschiedliche Gottesvorstellungen handelt, und dass nach der griechischen
Vorstellung Gott nun einmal nicht die Eigenschaft der Gnade besitzt. Das
christliche Verständnis scheint also eher eine Modifikation als eine
Differenzierung darzustellen. Der Gebrauch des Ausdruckes "`Differenzierung"'
kann genaugenommen nur dann als voll gerechtfertigt betrachtet werden, wenn
der differenziertere Symbolismus nichts anderes ausdrückt als das, was im
kompakteren Symbolismus bereits gemeint aber noch unvollkommen ausgedrückt
ist. Es braucht wohl kaum dargelegt werden, dass dies im Einzelfall äußerst
schwierig nachzuweisen sein dürfte, sofern man nicht dogmatisch unterstellt,
dass ohnehin alle Symbolismen nur denselben vorgegebenen Bestand von
Erfahrungen ausdrücken, die genau zu kennen sich der Interpret zudem anmaßen
muss.\footnote{Vgl. auch Eugene Webb: Philosophers of Consciousness. Polanyi,
  Lonergan, Voegelin, Ricoeur, Girard, Kierkegaard, Seatle and London 1988,
  S. 126ff.}

Nicht weniger dunkel bleibt, woher Voegelin die Bewertungsmaßstäbe nimmt, nach
denen er die verschiedenen Symbolismen beurteilt. Selbst wenn man bei dem eben
angeführten Beispiel einmal annimmt, dass sich beide Gottesvorstellungen
nachweisbar auf dieselbe religiöse Erfahrung stützen, so fehlt immer noch
jeder Anhaltspunkt, aus dem heraus die christliche Gnadenlehre im
inhaltlich-wertenden Sinne als differenzierterer Ausdruck der zugrunde
liegenden religiösen Erfahrung beurteilt werden kann als die Gottesvorstellung
der griechischen Philosophen. Wie auch bei anderen methodischen Problemen hat
es den Anschein, dass Voegelin glaubt, diese Frage im Einzelfall ad-hoc, durch
genaues Hinschauen und ein wenig Genialität in evidenter Weise beantworten zu
können.

\subsection{Der Sinn der Geschichte}

Wenn Voegelin Religionen, Philosophien und auch die konkreten politischen
Ordnungen als Ausdruck einer Suche nach der Ordnung des Seins deutet, so ist
dies nicht bloß die heuristische Voraussetzung eines besonders einfühlsamen
Geistesgeschichtlers. Voegelin ist vielmehr fest davon überzeugt, dass es eine
objektive, sinngebende und wertvermittelnde höhere Ordnung des Seins gibt. Die
Suche nach dieser Ordnung betrachtet Voegelin als das historische Projekt der
Menschheit. Durch dieses menschheitliche Projekt der Suche nach Ordnung wird
für Voegelin zu allererst die Einheit der Menschheit und der Sinn der
Geschichte hergestellt.\footnote{Vgl. Eric Voegelin: Order and History.
  Volume Two. The World of the Polis, Baton Rouge / London 1986 (zuerst:
  1957), im folgenden zitiert als: Voegelin, Order and History II, S.1-7. Auch
  wenn Voegelin leugnet, dass es einen erkennbaren Sinn der Geschichte geben
  kann, so scheint Voegelins Geschichtsphilosophie dennoch wenigstens so etwas
  wie den vorläufigen Sinn der Geschichte beschreiben zu wollen. Anders als
  auf den Sinn der Geschichte bezogen lassen sich Äußerungen wie die, dass die
  menschliche Existenz in Gesellschaft eine Geschichte habe, weil sie eine
  Dimension der Spiritualität habe (Vgl. ebd., S. 2), kaum verstehen. Denn
  Geschichte im Sinne einer Abfolge wechselnder Gesellschaftszustände gäbe es
  ja auch ohne die Spiritualität. -- Siehe auch die Anmerkungen zu den
  kollektivistischen Zügen von Voegelins Geschichtsphilosophie auf Seite
  \ref{HistorischerKollektivismus} und Seite
  \ref{KritikHistorischerKollektivismus} dieses Buches.}  Voegelin leugnet
allerdings entschieden, dass dem Menschen das Ziel der Geschichte bekannt
werden kann und dass ihr Ausgang vorhersagbar wäre.\footnote{Implizit gibt es
  jedoch auch in Voegelins Geschichtsphilosophie ein Ende der Geschichte, denn
  über das optimal differenzierte Ordnungswissen hinaus ist keine weitere
  Steigerung von Ordnungswissen mehr denkbar (und alle anderen geschichtlichen
  Entwicklungen sind politische Profangeschichte, für die Voegelin sich nicht
  interessiert).  Den bisherigen geistigen Höhepunkt der geschichtlichen
  Entwicklung stellt für Voegelin jedenfalls das christliche Mittelalter dar.}
Hierin setzt sich Voegelin in ausdrücklichen Gegensatz zu manchen
Geschichtsphilosophien der Hegelschen Machart, durch die er ansonsten durchaus
beeinflusst ist. Weiterhin vertritt Voegelin einen konsequenten
individualistischen Vorbehalt, was die Verkörperung des Geistes in der
Geschichte angeht. Geist verkörpert sich bei Voegelin in der Geschichte
niemals durch kollektive Gebilde wie den Staat oder die Nation. Vielmehr
dringt der Geist ausschließlich über das konkrete Bewusstsein des menschlichen
Individuums in die Geschichte ein.\footnote{Vgl.  Voegelins gegen Hegel
  gerichtete Bemerkungen, in: Eric Voegelin: "`Structures of Consciousness"',
  in: Voegelin-Research News Volume II, No 3, September 1996, auf:
  http://alcor.concordia.ca/\~{ }vorenews/v-rnII3.html (Host: Eric Voegelin
  Institute, Lousiana State University), im folgenden zitiert als: Voegelin,
  Structures of Consciousness, Abschnitt I (1).}  Partikularismen gegenüber,
seien sie nationaler oder anderer Art, ist Voegelin eher abgeneigt.  Ohne die
Realität partikulärer Gebilde zu leugnen, bleibt für Voegelin zumindest vor
der Geschichte die höchste Gemeinschaft stets die ganze
Menschheit.\footnote{Vgl. Voegelin, Order and History II, S. 1-20.} In diesem
Bezug auf die Menschheit, und zwar nicht nur auf die gegenwärtige Menschheit,
sondern auf die Menschheit in ihrer gesamten Geschichte, kommt ein moralisches
Anliegen Voegelins zum Ausdruck, welches auch seine Kritik an der
aufklärerischen Fortschrittsgeschichte motiviert. Die Fortschrittsgeschichte
entwertet in Voegelins Augen die vergangene Menschheit, indem sie in der
Vergangenheit nur die Vorstufe und das Mittel zum Zweck der Gegenwart
erblickt. Sie vergisst dabei, dass die vergangenen Menschen auch einmal eine
Gegenwart hatten, die sie genauso durchleben mussten, wie die heute lebenden
Menschen ihre Gegenwart bewältigen müssen.\footnote{Vgl. Voegelin, Order and
  History II, S. 3.}  Freilich stellt sich die Frage, wie Voegelin nun
seinerseits derartige Entwertungen vermeiden will. Und an Voegelins Kritik der
Neuzeit als einem gnostischen Zeitalter wird deutlich, dass auch Voegelin
ganze historische Epochen verdammen konnte.

Mit dem vierten Band von "`Order and History"' tritt ein Bruch in Voegelins
historischem Programm ein. Dieser Bruch resultiert zu einem Teil aus der
Feststellung, dass die Geschichte nicht linear, sondern in vielfältigen
Verzweigungen, Verästelungen und unabhängig nebeneinander herlaufenden
Entwicklungssträngen verläuft.\footnote{Vgl. Voegelin, Order and History IV,
  S. 1-6.} Dieses Faktum war Voegelin schon zuvor bewusst, wenn er auch dessen
Ausmaß unterschätzte, und er es daher in der Konzeption von "`Order and
History"' zunächst eher vernachlässigt hat. Zum anderen Teil kommt der Bruch
durch die Entdeckung zustande, dass die Idee einer fortschreitenden, nicht
zyklischen geschichtlichen Entwicklung keineswegs einzigartig mit dem "`Sprung
im Sein"' zur Zeit von Moses verbunden ist, sondern sehr häufig bereits im
Rahmen kosmischer Symbolismen auftritt.  Voegelin geht nun noch stärker als
zuvor davon aus, dass es in der Praxis zu einer Verschränkung kosmischer und
nach-kosmischer Symbolismen und weniger zu einer deutlichen Ablösung des einen
durch den anderen kommt.\footnote{Vgl.  Order and History IV, S. 7-12.} Seine
Grundvorstellung von der Geschichte als Prozess der zunehmenden Differenzierung
spiritueller Erfahrungen behält Voegelin jedoch bei. In dieser Hinsicht bleibt
der Bruch weniger dramatisch, als es die Einleitung von "`Order and History
IV"' zunächst vermuten lässt.

Voegelins Geschichtsphilosophie ist eingebettet in eine kosmische
Geschichtsmetaphysik, der zufolge die Geschichte die Verwirklichung eines
ewigen Seins in der Zeit darstellt.\footnote{Vgl. Voegelin, Anamnesis, S.
  254ff.} Dieser Prozess, den Voegelin gelegentlich bis in die Naturgeschichte
und die Evolution zurückverlängert,\footnote{Vgl. Voegelin, Structures of
  Consciousness, Abschnitt I (2).} verursacht und bestimmt auch die
Menschheitsgeschichte, indem das ewige Sein im menschlichen Bewusstsein als
anziehender transzendenter Pol wirkt. Diese Geschichtsmetaphysik ist wenig
überzeugend: Da die Menschheitsgeschichte nicht erkennbar auf einen Punkt
zuläuft, sondern sich vielfältig verzweigt, wäre es sehr viel naheliegender,
im "`ewigen Sein"' eine dem menschlichen Bewusstsein entspringende Vorstellung
und nicht ein in das Bewusstsein eindringendes transzendentes Sein zu
vermuten. Ohnehin verwundert es ein wenig, dass sich das transzendente Sein zu
seiner immanenten Verwirklichung als Ort gerade die Erde -- ein Staubkorn im
Weltall, wie man es sich unbedeutender gar nicht vorstellen kann -- ausgesucht
hat, wo doch das ganze Universum zur Verfügung gestanden hätte. (Oder hängt
die Verwirklichung der Transzendenz in der Immanenz etwa von Wasser,
Kohlenstoff und günstigen Temperaturbedingungen ab?)  Voegelin wandelt mit
seiner Geschichtsmetaphysik ersichtlich auf den Spuren der
Geschichtsphilosophien des deutschen Idealismus, wonach die Geschichte ein
Prozess ist, in welchem der Geist zu sich selbst kommt.  Nicht anders als die
Philosophen des Deutschen Idealismus verliert sich Voegelin dabei in
metaphysische Spekulationen ohne Maß und Zügel. Bei Voegelin ist es jedoch im
Unterschied zu den Philosophen des Deutschen Idealismus nicht der Geist
sondern die "`Realität"', die sich im Menschen selbst erhellt.  Voegelin
begründet dies damit, dass der Mensch Teil der Realität ist, und dass
folglich, wenn der Mensch die Realität erkennt, diese sich selbst zwar nicht
geradewegs erkennt, aber doch erhellt.\footnote{Vgl.  Voegelin, Structures in
  Consciousness, Abschnitt I (2)-(3).} Die Logik dieser Begründung ist
ungefähr die folgende: Wenn ich durch meine Heimatstadt spazieren gehe und die
Häuser anschaue, dann erblickt, da ich ja ein Teil meiner Heimatstadt bin, die
Stadt sich selbst. Wie man sieht handelt es sich um eine Trivialität, die nur
durch die philosophisch-abstrakte Ausdrucksweise den Schein tiefsinniger
Bedeutsamkeit annimmt.

So sehr Voegelin im übrigen auch die Historizität der menschlichen Existenz
betont, es bleibt hier eine schwer überbrückbare Spannung zu den mehr
ahistorischen Zügen seiner Philosophie bestehen. Sowohl Voegelins
existenzialistisches Menschenbild als auch seine Seinsmetaphysik und seine
Bewusstseinsphilosophie sind wesentlich ahistorisch.\footnote{Vgl. zur
  unreflektierten Ahistorizität von Voegelins prozesstheologischer
  Geschichtsdeutung die Diskussion über Voegelins Vortrag über "`Ewiges Sein
  in der Zeit"'. Dort insbesondere Baumgartners treffende Einwürfe, in: Helmut
  Kuhn / Franz Wiedmann (Hrsg.): Die Philosophie und die Frage nach dem
  Fortschritt, München 1964, S. 340.} Hinzu kommt Voegelins Neigung, sich
gelegentlich recht unbekümmert über die Jahrhunderte hinweg mit Philosophen
und Propheten auseinanderzusetzen als wären es Zeitgenossen. Nicht selten
trägt er dabei in anachronistischer Weise Konzepte in die Interpretation der
Klassiker hinein, die der Philosophie des 19. und 20.  Jahrhunderts entnommen
sind.\footnote{Vgl. Zdravko Planinc: The Uses of Plato in Voegelin's
  Philosophy of Con\-s\-cious\-ness: Reflections prompted by Voegelin's
  Lecture, "`Structures of Con\-s\-cious\-ness"', in: Voegelin-Re\-search News
  Volume II, No 3, September 1996, auf: http:\-//alcor.concordia.ca/\~{
  }vorenews/v-rnII3.html (Host: Eric Voegelin Institute, Lousiana State
  University).} Wenn dies nicht immer sogleich auf\/fällt, so hängt das auch
damit zusammen, dass Voegelin ein wenig zögerlich war, die moderne Herkunft
seiner Ideen auch stets anzuerkennen. Wird versucht, die Beziehung zwischen
den historischen und ahistorischen Zügen von Voegelins Philosophie näher zu
bestimmten, so lässt sich dabei feststellen, dass -- ebenso wie in Bezug auf
politische Ordnung -- auch hinsichtlich der Geschichte die
Bewusstseinsphilosophie und die Seinsmetaphysik die theoretische Grundlage
bilden, auf der die geschichtlichen Prozesse von Voegelin gedeutet werden.

\section{Gnosisbegriff und Zeitkritik}

Da sich für Voegelin gute politische Ordnung auf ein richtiges Verständnis der
höheren Ordnung des Seins und auf eine wohlausgebildete Ordnung der Seele
gründet, so liegt es für ihn natürlich nahe, den Ursprung politischer
Unordnung in spiritueller Desorientierung zu suchen. Der Gnosisbegriff, mit
dem Voegelin lange Zeit die Formen spiritueller Desorientierung beschrieb,
bildet zugleich das Hauptinstrument seiner politischen Gegenwartskritik, einer
Kritik, die sich auf die gesamte Neuzeit, insbesondere aber auf das 20.
Jahrhundert bezieht. Die Ursprünge des Gnosisbegriffs reichen zurück bis zu
Voegelins Schrift über die "`politischen Religionen"' von 1938,\footnote{Eric
  Voegelin: Die politischen Religionen, München 1996 (zuerst 1938).} einer
Kampfschrift gegen den Nationalsozialismus, die er geradezu panikartig
verfasste, nachdem er mit einer gewissen Verspätung dessen Gefährlichkeit
erkannt hatte.\footnote{Vgl.  Eckhart Arnold: Eric Voegelin als Schüler Hans
  Kelsens, a.a.O., S. 4-5 (die Seitenzahlen folgen der Entwurfsfassung).}
Ausgehend von dem dort noch verwendeten Begriff der "`politischen Religion"'
bildet Voegelin später seine Theorie von der Neuzeit als einem Zeitalter der
wiedererwachten Gnosis, welche ihren prägnantesten Ausdruck in den totalitären
Herrschaftsformen des 20. Jahrhunderts gefunden hat. Voegelins Gnosistheorie
kann daher auch als seine Form der Auseinandersetzung mit dem Phänomen des
Totalitarismus verstanden werden, zumal sie merklich durch den
zeitgeschichtlichen Kontext ihrer Entstehung geprägt ist.

Unter Gnosis versteht Voegelin eine spirituelle Desorientierung im
Zusammenhang mit der Erfahrung der Transzendenz, die unter bestimmten
Bedingungen den Glauben mit sich führen kann, dass ein geschichtlicher
Endzustand von vollendeter Glückseligkeit innerhalb einer absehbaren Zeit die
gegenwärtige schlechte Welt ablösen kann.\footnote{Zur Definition des
  Begriffes bei Voegelin: Vgl. Dante Germino: Eric Voegelin on the Gnostic
  Roots of Violence, München 1998, im folgenden zitiert als: Germino, Voegelin
  on the Gnostic Roots of Violence, S. 26. -- Vgl. Voegelin, Neue Wissenschaft
  der Politik, S. 169-171. -- Vgl. Voegelin, Order and History IV, S. 18-27. --
  Vgl. Voegelin, Wissenschaft, Politik und Gnosis, S. 17-19.} Diese recht
allgemeine Bedeutung erlaubt es Voegelin, den Begriff der Gnosis von den
üblicherweise mit diesem Namen bezeichneten häretischen Glaubensströmungen des
Vor- und Frühchristentums und des Mittelalters auf die chiliastischen
politischen Bewegungen der Neuzeit zu übertragen. Die Gnosis stellt in
Voegelins Augen insofern ein spirituelles Missverständnis dar, als sie auf
einer seiner Ansicht nach falschen Vorstellung vom Wesen der Transzendenz
beruht. Zwar liegt der Gnosis dieselbe religiöse Erfahrungssubstanz zugrunde
wie dem Christentum, nämlich die differenzierende Erfahrung eines
transzendenten göttlichen Seins, aber in der Gnosis wird das transzendente
Sein als so überwältigend erlebt, dass die Immanenz in die Sinnlosigkeit
absinkt und dem Welthass verfällt. Die Spannung zwischen Immanenz und
Transzendenz, welche für Voegelin recht eigentlich die Realität des
menschlichen Existierens ausmacht, wird dadurch zu Gunsten einer
einseitigen, geradezu tagträumerischen Fixierung auf das transzendente
Ziel aufgelöst.\footnote{Vgl. Voegelin, Order and History IV, S. 19/20.}

Ist das Gefühl für die Spannung zwischen Transzendenz und Immanenz aber erst
einmal verlorengegangen, so kann an der Stelle des transzendenten Zieles
leicht auch irgendein immanenter Weltgehalt untergeschoben werden, der dann
alle Attribute des göttlichen Seins erbt und zum Gegenstand eines äußerst
unheiligen Götzenkultes erhoben wird. Dieser Fall von "`gnostischem
Immanentismus"' ist in Voegelins Augen höchst charakteristisch für die gesamte
Neuzeit.\footnote{Vgl. Voegelin, Neue Wissenschaft der Politik, S. 175-180,
  S. 229ff.} Voegelin trifft innerhalb dessen, was er als Gnosis bezeichnet,
noch allerlei Einzelunterscheidungen, die ihm aber letzten Endes nur dazu
dienen, recht wahllos alle politischen und geistigen Strömungen der Neuzeit,
denen in irgendeiner Weise nachgesagt werden kann, dass sie ein Ideal
vertreten, unter dem Begriff des gnostischen Immanentismus zu versammeln. So
nennt Voegelin als gnostische Bewegungen Progressivismus, Liberalismus,
Humanismus, Marxismus, Kommunismus, Faschismus, Psychoanalyse und je nach
Bedarf noch einige mehr.\footnote{Vgl. Voegelin, Neue Wissenschaft der
  Politik, S. 176. -- Vgl. Germino, Voegelin on the Gnostic Roots of Violence,
  S. 27.} Angesichts dieser breit gefächerten Auswahl neuzeitlicher gnostischer
Bewegungen verwundert es nicht, dass Voegelin im Gnostizismus das Wesen der
Moderne überhaupt erblickt.

Der gnostische Immanentismus führt, da dem Menschen das transzendente
Ordnungskorrektiv verlorengeht, in letzter Instanz zur Selbstvergottung des
Menschen. Neben Auguste Comte ist Friedrich Nietzsche Voegelins Hauptbeispiel
und zugleich sein wichtigster Gewährsmann für diese These.\footnote{Vgl.
  Voegelin, Neue Wissenschaft der Politik, S. 182-184.} Voegelin lässt sich
nicht dadurch irritieren, dass es säkularistische Philosophien gibt, die nicht
die Konsequenz der Selbstvergottung ziehen. Im Zweifelsfall interpretiert
Voegelin solche scheinbar harmlosen Säkularismen als Schritte auf dem Wege,
der unvermeidlich zur Selbstvergottung des Menschen führt. So ist Voegelin
beispielsweise überzeugt, dass der Liberalismus ohne spirituelle Basis mit
innerer Logik zum Kommunismus führt.\footnote{Vgl. Eric Voegelin: Der
  Liberalismus und seine Geschichte, in: Karl Forster (Hrsg.): Christentum und
  Liberalismus, München 1960, S. 28-31 (S. 11-42).} Plausibel wird diese
Auf\/fassung freilich nur, wenn man das Dogma zu Grunde legt, dass der Mensch
nicht nicht-religiös sein kann, und dass dementsprechend die Säkularisierung
nicht zum Verschwinden religiöser Absolutheitsansprüche, sondern bloß zu
deren Fehlbesetzung führen kann.

Die Vollendung der menschlichen Selbstvergottung und Herrschsucht erreicht der
gnostische Immanentismus im politischen Bereich in den totalitären
Herrschaftsformen. Der Totalitarismus ist für Voegelin ein unmittelbares
Resultat der antichristlichen Unterdrückung der "`Wahrheit der Seele"' sowie
des Irrglaubens, dass in der Geschichte mit Hilfe politischer Aktion ein
Endzustand glückseliger Verklärtheit herbeigeführt werden kann. Voegelin hat
allerdings nie ernsthaft versucht, den Zusammenhang von religiösen Abirrungen
und gewalttätiger Politik auf der Ebene des konkreten politischen Geschehens
detailliert nachzuzeichnen. Seine Erklärung verbleibt auf einer rein
hermeneutisch-geistesgeschichtlichen Ebene und ihre Glaubwürdigkeit hängt von
einer empirisch nicht überprüften Einschätzung der Bedeutsamkeit des
religiösen Faktors im politischen Geschehen ab.

Auch aus anderen Gründen ist der Gnosisbegriff zur Beschreibung chiliastischer
politischer Bewegungen ungünstig gewählt: Der historischen Gnosis, deren
Gewaltpotential sich kaum mit dem ihrer rechtgläubigen Verfolger messen kann,
wird Voegelin am allerwenigsten gerecht. Es genügt eben nicht, den
vermeintlichen Welthass oder irgendwelche spirituellen Missverständnisse (aus
Sicht der eigenen Religiosität!) festzustellen, um daraus auf die latente
Gewalttätigkeit der Gnosis zu schließen.\footnote{Vgl. Eric Voegelin: Das Volk
  Gottes. Sektenbewegungen und der Geist der Moderne (Hrsg. von Peter
  J.Opitz), München 1994, S. 74. Dort schreibt Voegelin über die häretischen
  Bewegungen des späten Mittelalters: "`Da die eschatologische Gewalt jenseits
  von Gut und Böse liegt, und da der Krieg für die Welt des Lichtes eine
  transzendentale geistige Operation ist, in der die Mächte der Finsternis aus
  dem Kosmos entfernt werden, werden sich die Gläubigen zwangsläufig [{\em
    sic!}]  in einer Gründlichkeit der Vernichtung ergehen, die von der Warte
  der Realität aus als Bestialität und Grausamkeit erscheint."' -- Viel zu
  sehr vereinfacht wird dies auch von Dante Germino. Vgl. Germino, Eric
  Voegelin on the Gnostic Roots of Violence, S. 28. Darüber hinaus lässt sich
  Germinos These, dass in expressiver Gewalt, also in einer Form von {\em
    Hooliganism}, das Wesen der totalitären Gewalt besteht, nicht leicht mit
  der administrativen und planvollen Form der Durchführung totalitärer
  Massenmorde vereinbaren.  Bei bestimmten Tätergruppen -- etwa Gestalten wie
  Eichmann oder jenen "`ganz normalen Männern"' (C.Browning), die als
  Befehlsempfänger die Morde durchführten -- kann man ein expressives Moment
  ihrer Gewalttaten nur schwer auf\/finden.} Dass es gerade die Vergöttlichung
eines Teilinhaltes der Welt sein soll, die die exzessive Gewalttätigkeit nach
sich zieht, ist nicht besonders einleuchtend, da ja auch im Namen des
transzendenten Gottes in Form von Kreuzzügen, Progromen oder Hexenverfolgungen
so mancher Exzess der Gewalt stattfand, wobei sich das vorreformatorische
Christentum -- von Voegelin romantisch als letzter Hort intakten
Ordnungswissens verklärt -- zuweilen auf eine ausgesprochen unschöne Weise
hervorgetan hat. Die Gefahr dürfte wohl eher von dem Irrglauben, über eine
absolute, umfassende und für alle Menschen verpflichtende Wahrheit zu
verfügen, ausgehen als vom Säkularismus oder der Gnosis. Voegelins Konzept der
Politischen Religion ist ebenso wie sein Begriff der Gnosis nicht nur allzu
undifferenziert, sondern er verfehlt darüber hinaus auch das Wesentliche,
indem er die Neigung zur Gewalttätigkeit nicht primär als eine Frage der Form
(fanatisch oder tolerant), sondern als eine Frage des Inhalts des Glaubens
auf\/fasst, womit er sich auf die Ebene religiös-konfessioneller Polemik
begibt.\footnote{Es verwundert daher auch nicht, dass Voegelin -- vermutlich
  von Carl Schmitt dazu inspiriert -- zum Verständnis der politischen
  Bewegungen der Neuzeit die Lektüre des Werkes "`Adversus Haereses"' des
  Kirchenvaters Iraeneus (2.Jh. nach Christus!)  empfiehlt.  Vgl. Voegelin,
  Neue Wissenschaft der Politik, S.178. -- Nicht ganz zu unrecht wird Voegelin
  von Albrecht Kiel als "`katholischer Fundamentalist"' gesehen. Vgl.
  Albrecht Kiel: Gottesstaat und Pax Americana. Zur Politischen Theologie von
  Carl Schmitt und Eric Voegelin, Cuxhaven und Dartford 1998, S. 3, S. 95ff.}
Als Analysewerkzeug zum Verständnis der chiliastischen politischen Bewegungen
der Neuzeit entwertet Voegelin seinen Gnosis-Begriff dadurch, dass er auf ihn
zurückgreift, um einer wenig qualifizierten politischen Polemik Ausdruck zu
verleihen.\footnote{Vgl.  beispielsweise Voegelin, Neue Wissenschaft der
  Politik, 6.Kapitel, S.  224-259. -- Als eine Form von politischem Moralismus
  weisen sich Voegelins Äußerungen an dieser Stelle dadurch aus, dass er den
  Sachproblemcharakter schwieriger politischer Entscheidungsdilemmata leugnet,
  indem er ihre Lösung zu einer jedem Einsichtigen völlig selbstverständlichen
  Banalität stilisiert (Ebda.  S. 236-238), wodurch er im zweiten Schritt die
  Verfehlung jener vermeintlich eindeutig und offensichtlich richtigen
  Lösungen auf billige Weise moralischen Makeln der Entscheidungsträger
  anlasten kann. -- Dass der Gnosis-Begriff als solcher, wenn er mit Augenmaß
  eingesetzt wird, zur Analyse {\em bestimmter} moderner Geistesströmungen
  auch durchaus sinnvoll eingesetzt werden kann, führt Micha Brumlik vor. Vgl.
  Micha Brumlik: Die Gnostiker.  Der Traum von der Selbsterlösung des
  Menschen, Frankfurt am Main 1992.}

% Voegelin
%   geht auch bei der Untersuchung von fremden Theorien des öfteren so vor, daß
%   er die Theorie zunächst als völlig abwegig darstellt, um dann ihre Erfindung
%   für "`aufklärungsbedürftig"' zu erklären und ungesäumt zur klinischen
%   Diagnose des Falles überzugehen.}

Als Totalitarismustheorie gehört Voegelins Gnosistheorie insgesamt noch eher
einer Phase der (natürlich legitimen) emotionalen Auseinandersetzung und
inneren Abwehr des Phänomens an. Sie erscheint als eine Erklärung des
Totalitarismus, wie sie in den fünfziger Jahren nicht untypisch war: Der
Totalitarismus wird mit einer historisch weitausholenden Fundamentalerklärung,
als deren Haupterklärungsmoment ideologische Faktoren fungieren, erfasst und
als die Folge des Abfalls von der Religion interpretiert. Besonders in dieser
Hinsicht ähnelt Voegelins Gnosiskonzept jenen gerade in der Nachkriegszeit
populären Säkularisierungstheorien, wie sie Hermann Lübbe eingehend untersucht
hat.\footnote{Vgl. Hermann Lübbe: Säkularisierung.  Geschichte eines
  ideenpolitischen Begriffs, München 1965, S. 108ff. -- Stärker philosophisch
  als zeitgeschichtlich orientiert: Hans Blumenberg: Die Legitimität der
  Neuzeit. Erneuerte Ausgabe, Frankfurt am Main 1996. (Vgl. S. 138.)} Die
Erklärung Lübbes für das Auftreten der Säkularisierungstheorien lässt sich
allerdings nur in Teilen auf Voegelin übertragen, da Voegelin als Emigrant
nicht unbedingt apologetische Absichten hatte. Für Voegelin stellte im
Gegenteil gerade die Ignoranz der Nachkriegsgesellschaft, die sich in
Verdrängung, Verklärung und darin äußerte, dass gestandene Nazi-Schufte als
angesehene Bürger gelten und in einzelnen Fällen sogar einflussreiche Posten
besetzen konnten, einen Hinweis darauf dar, dass die tieferen existenziellen
Bedingungen der politischen Katastrophe noch fortdauerten.\footnote{Vgl.
  Voegelins Vorlesung über "`Hitler und die Deutschen"', S. 1ff. (Typoskript im
  Eric-Voegelin-Archiv in München.) -- Vgl. Eric Voegelin: Die deutsche
  Universität und die Ordnung der deutschen Gesellschaft, in: Die deutsche
  Universität im Dritten Reich.  Eine Vortragsreihe der Universität München,
  München 1966, S. 241-282 (S. 241ff.).}

Wenn Voegelin die verschiedensten geistigen Strömungen der Neuzeit mehr oder
weniger unterschiedslos als Gnosis identifiziert, so ist auch dies wohlmöglich
die Folge eines Methodenmissbrauchs. Voegelin verwendet bei der Untersuchung
der Geistesgeschichte häufig die Technik der Suche nach
"`Strukturverwandtschaften"'. Eine Strukturverwandtschaft scheint dabei nicht
viel mehr zu sein als eine sich in irgendeiner Weise aufdrängende Analogie
zwischen zwei oder mehreren Theorien. So erkennt Voegelin zwischen dem in die
Phasen des Vaters, des Sohnes und des heiligen Geistes unterteilten
Geschichtsbild von Joachim Fiori und dem Dreistadiengesetz des Auguste Comte
eine solche Strukturverwandtschaft.  Zu diesen beiden Theorien steht wieder
die nationalsozialistische Ideologie vom Dritten Reich in der Beziehung einer
Strukturverwandtschaft.\footnote{Vgl. Voegelin, Neue Wissenschaft der Politik,
  S. 157-162.}  Wie Hans Kelsen treffend heraus gestellt hat, ist dieser
Befund jedoch ziemlich trivial und historisch bedeutungslos.\footnote{Vgl.
  Kelsen, A New Science of Politics, a.a.O., S. 77ff.} Dass Problem der
Methode der Strukturverwandtschaften besteht nämlich darin, dass die
Feststellung von Strukturverwandtschaften es allein noch nicht erlaubt, auf
kausale Zusammenhänge oder auch nur auf eine historische Traditionslinie zu
schließen. So wie Voegelin von dieser Methode Gebrauch macht, lädt sie zu
extremen Versimplifizierungen geradezu ein.\footnote{Ein extremes Beispiel
  einer solchen Simplifizierung liefert Voegelin in einem Brief an Alfred
  Schütz, wo er den Abwurf der Atombombe zu einem Ausfluss des Phänomenalismus
  (nach Voegelins Wortgebrauch die Auf\/fassung, dass nur den
  (natur-)wissenschaftlich erfassbaren Phänomenen substantielle Wirklichkeit
  zukommt) stilisiert.  Vgl. Gilbert Weiss: Theorie, Relevanz und Wahrheit.
  Zum Briefwechsel zwischen Eric Voegelin und Alfred Schütz, München 1997, S.
  46-49. -- Ähnlich willkürlich wie die Strukturverwandtschaften scheint das
  Kriterium der Äquivalenz von Erfahrungen zu sein. Vgl. dazu Eric Voegelin:
  Äquivalenz von Erfahrungen und Symbolen in der Geschichte, in: Eric
  Voegelin, Ordnung, Bewußtsein, Geschichte, Späte Schriften (Hrsg. von Peter
  J. Optiz), Stuttgart 1988, S.  99-126 (S. 105-106 / S. 110).}

Gehört Voegelins Gnosistheorie auch ohne Zweifel zu den schwächeren Seiten
seiner Politikwissenschaft, so liegt in der grundsätzlichen Frage, ob der
Verlust der Spiritualität nicht auch die Gefahr eines Wertverfalls nach sich
zieht, indem mit der Spiritualität auch das innere Empfinden für den Sinn und
die Bedeutung des Lebens verlorengeht, ein Vorbehalt, wie er sich auch heute
noch manchem religiösen Menschen aufdrängen mag. Unter diesem Aspekt ist daher
die Frage an Voegelins Bewusstseinsphilosophie zu richten, ob sie diesen
Vorbehalt rechtfertigen kann.  Gelingt es Voegelin -- so wird im Folgenden zu
fragen sein -- zu zeigen, dass die Spiritualität im menschlichen Bewusstsein
verankert ist, und dass ihre Leugnung oder ihr Verlust zu existenzieller
Unsicherheit und möglicherweise (gefährlichen) kompensatorischen
Gegenreaktionen führt?

%%% Local Variables: 
%%% mode: latex
%%% TeX-master: "Main"
%%% End: 

































%%% Local Variables: 
%%% mode: latex
%%% TeX-master: "Main"
%%% End: 

\chapter{Voegelins Bewusstseinsphilosophie ("`Anamnesis"' -- Teil I)}
\label{VoegelinsBewusstseinsphilosophie}

In diesem und dem folgenden Kapitel wird Voegelins Bewusstseinsphilosophie
erörtert, wie sie im ersten und dritten Teil seines Werkes "`Anamnesis"'
entfaltet wird. Der zweite Teil von "`Anamnesis"' enthält eher historische und
geschichtsphilosophische Studien und wird daher hier übergangen. Zwar fließen
bei Voegelin Bewusstseinsphilosophie und Geschichtsphilosophie ineinander,
aber schon aus pragmatischen Gründen musste für die vertiefte
Auseinandersetzung mit Voegelins Bewusstseinsphilosophie eine Auswahl
getroffen werden. Zudem wurde zur Verortung von Voegelins
Bewusstseinsphilosophie innerhalb seines Gesamtansatzes schon im vorigen
Kapitel das Nötige gesagt. Weiterhin kehren viele der Motive aus den hier
ausführlich besprochenen Schriften in Voegelins anderen Werken wieder, so dass
man die hier angestellten Überlegungen leicht übertragen kann.

Der erste Teil von Voegelins "`Anamnesis"' enthält neben einer kurzen
Erinnerung an Voegelins Freund Alfred Schütz\footnote{Alfred Schütz (geb. 1989
  in Wien, gest. 1959 in New York), war der Schöpfer einer phänomenologischen
  Soziologie. Sein wohl bekanntestes Werk trägt den Titel: "`Der sinnhafte
  Aufbau der sozialen Welt. Eine Einführung in die verstehende Soziologie"'.},
die -- als eher von biographischem als philosophischen Interesse -- hier
übergangen wird, eine Kritik von Husserls "`Krisis der europäischen
Wissenschaften"'.\footnote{Vgl.  Voegelin, Anamnesis, S. 21-36. -- Edmund
  Husserl: Die Krisis der europäischen Wissenschaften und die transzendentale
  Phänomenologie. Eine Einleitung in die phänomenologische Philosophie,
  Hamburg 1996, im folgenden zitiert als: Husserl, Krisis.}. An diese schließt
sich ein eigenständiger bewusstseinsphilosophischer Entwurf unter dem Titel
"`Zur Theorie des Bewußtseins"' an.\footnote{Vgl. Voegelin, Anamnesis, S.
  37-60.} Den Abschluss des ersten Teils von "`Anamnesis"' bilden dann eine
Reihe von "`anamnetischen Experimenten"',\footnote{Vgl. Voegelin, Anamnesis,
  S. 61-76.} wobei sich hinter dieser geheimnisvollen Bezeichnung jedoch
nicht viel mehr verbirgt als die Erzählung einiger Kindheitserinnerungen
Voegelins.

\section{Voegelin über Husserls "`Krisis der europäischen
  Wissenschaften"'}

\subsection{Husserls Krisis-Schrift}

Bevor auf Voegelins Auseinandersetzung mit Husserls Schrift: "`Die Krisis der
europäischen Wissenschaften und die transzendentale
Phänomenologie"'\footnote{Husserl, Krisis, a.a.O.} eingegangen wird, ist
einiges zu dieser Schrift selbst zu sagen.

Husserls Schrift ist als eine Einführung in die Phänomenologie konzipiert. Sie
entstand aus mehreren Vorträgen, die Husserl im Jahre 1935 gehalten hat. Da
Husserl 1936 in Deutschland nicht mehr publizieren durfte, wurde die Schrift
1936 in der in Belgrad erscheinenden Zeitschrift "`Philosophia"'
veröffentlicht.\footnote{Vgl. dazu die Einleitung von Elisabeth Ströker, in:
  Husserl, Krisis, S.IXff.} Auf diese Fassung, welche Voegelin 1943 in die
Hände bekommen konnte, bezieht sich Voegelin in seinem Brief an Alfred Schütz.
Gegenüber der 1954 in der Reihe {\it Husserliana} erschienenen und um bis
dahin unpubliziertes Material ergänzten Ausgabe ist die
"`Philosophia"'-Fassung um einiges kürzer. Insbesondere wird in der frühen
Fassung die Lebenswelt-Problematik noch kaum angerissen. Dies ist zu
berücksichtigen, da Voegelins Enttäuschung über den wieder nur rein
erkenntnistheoretischen Charakter von Husserls Werk sonst leicht ungerecht
erscheinen könnte.

Husserl hat in seinen einführenden Schriften recht unterschiedliche Zugänge
zur Phänomenologie gegeben. In den "`Cartesianischen Meditationen"'
beispielsweise wird die Phänomenologie durch die Aufgabe motiviert, die Basis
für eine letztbegründete und umfassende philosophische Universalwissenschaft
zu schaffen.\footnote{Vgl. Edmund Husserl: Cartesianische Meditationen. Eine
  Einleitung in die Phänomenologie, Hamburg 1987, S. 8ff.} In seiner letzten
Einführung hingegen wird die Phänomenologie, wie sich schon im Titel andeutet,
durch einen geistigen Notstand motiviert. Dieser geistige Notstand besteht
darin, dass die Weltsicht der Gegenwart fast vollkommen von den
Naturwissenschaften und insbesondere von der Physik als der Leitwissenschaft
dominiert wird.\footnote{Vgl. Husserl, Krisis, S. 3-5 (§ 2).} Husserl
betrachtet dies als ein Verhängnis, weil die Naturwissenschaften nach seiner
Auf\/fassung nicht die wirkliche Welt wiedergeben (welche für Husserl einzig
und allein die Welt der konkret gegebenen Phänomene ist), sondern der
wirklichen Welt mathematische Gestalten unterschieben. Zwar ist Husserl
bereit, die pragmatische Brauchbarkeit dieser Gestalten anzuerkennen, aber er
hält es für einen schweren Fehler, ihnen eine ontologische Beschreibung der
Welt zu entnehmen. Den Irrtum, die Modelle der Naturwissenschaften als
Beschreibungen der Wirklichkeit zu verstehen, bezeichnet Husserl als
"`Physikalismus"'.\footnote{Vgl. Husserl, Krisis, S. 68.} Dass es sich beim
Physikalismus um einen Irrtum handelt, versucht Husserl durch eine suggestive
Beschreibung der historischen Entwicklung des wissenschaftlichen Denkens zu
zeigen. Als Ausweg aus dem Physikalismus preist Husserl die transzendentale
Phänomenologie an. Sie würde es ermöglichen, die Wirklichkeit in ihrer
konkreten phänomenalen Gegebenheit für das Bewusstsein zurückzugewinnen und
die Wissenschaften wieder in angemessener Weise in die Lebenswelt
einzubetten.\footnote{Husserls Kritik des "`Physikalismus"' ist alles andere
  als überzeugend, worauf hier jedoch nicht ausführlich eingegangen werden
  kann.  Die Hauptschwachpunkte seien nur kurz angemerkt: 1. Husserl
  unterstellt, dass die Naturwissenschaft der Natur etwas unterschiebt, was sie
  in Wirklichkeit nicht ist. Da die Naturwissenschaft ihre Ergebnisse jedoch
  experimentell auf die Probe stellt, kann sie der Natur nicht ohne Weiteres
  etwas Falsches unterschieben. Husserls Vorwurf kann sich also höchstens noch
  darauf beziehen, dass die Naturwissenschaft die Erscheinungen nicht für das
  Sein der Natur nimmt. Wird dies jedoch als illegitim angesehen, so stellt
  sich die Frage, ob dann nicht auch die "`eidetische Wesensschau"' des
  Phänomenologen (Vgl.  Edmund Husserl: Die phänomenologische Methode.
  Ausgewählte Texte I. (Hrsg.  von Klaus Held), Stuttgart 1985, S. 101-107.)
  dem Phänomen ein Wesen unterschiebt.  2. Husserls historische
  Darstellungstechnik ist nicht besonders gut dazu geeignet, systematische
  Probleme zu lösen, auch wenn sich aus ihr möglicherweise systematische
  Argumente indirekt entnehmen lassen. Die Feststellung z.B., dass die
  mathematisch-geometrischen Gestalten ursprünglich Methode (nämlich
  Feldmesskunst) waren (Vgl. Husserl, Krisis, S. 52ff.), besagt noch längst
  nicht, dass sie in ihrer entwickelten Form für den Ausdruck ontologischer
  Zusammenhänge untauglich wären. Es sei denn, man nimmt an, dass etwas, was
  einmal Methode gewesen ist, sich niemals zu etwas wesentlich anderem
  entwickeln kann, oder dass Wissenschaften sich grundsätzlich nicht von ihrem
  historischen Ursprung emanzipieren können oder dürfen.}

Allerdings bleibt es in der "`Krisis der europäischen Wissenschaften"' nicht
bei der Kritik am Physikalismus, denn Husserl beabsichtigt, so scheint es, der
Phänomenologie die Weihen einer historischen Mission zu verleihen. Husserl
behauptet dazu, dass sich in der Geistesgeschichte ein "`Telos"' auf\/finden
lasse, wobei das Wort "`Telos"' einen recht vieldeutigen Sinn gewinnt, der
sowohl Ziel und Ursprung als auch Anklänge von Legitimation und Verpflichtung
beinhaltet. Was Husserl in diesem Zusammenhang zu dem Thema der
Geschichtsteleologie zu sagen hat, rückt seine Darstellung in der Tat stark in
die Nähe einer Geschichtsideologie. Husserl zufolge ist dieses Telos nämlich
ein aus der Geschichte ablesbarer höherer Wille, der auf die Entwicklung der
phänomenologischen Philosophie hinzielt.\footnote{Vgl. Husserl, Krisis, S.
  14-19 (§ 6,7).} Dieser Wille darf keineswegs verwechselt werden mit den
Absichten einzelner Philosophen, vielmehr ist er als eine durch den einzelnen
Philosophen "`hindurchgehende Willensrichtung"'\footnote{Husserl, Krisis, S.
  78.} zu verstehen.  Deshalb kann dieser Wille auch nicht den
Selbstzeugnissen dieser Philosophen entnommen werden, sondern muss unter
Zuhilfenahme einer kunstvollen hermeneutischen Interpretationstechnik im
historischen Rückblick aus dem Verborgenen hervorgehoben werden.\footnote{Vgl.
  Husserl, Krisis, S. 62-64 (§ 9 l) ) / S. 77-80 (§ 15), S. 109.}  Entstanden
ist dieser Wille in den beiden "`Urstiftungen"' der antiken griechischen
Philosophie und des philosophischen Neuanfangs durch Descartes.  Diese
Urstiftungen verlangen ihrem Wesen nach (und nicht bloß, wie man denken
könnte, ihrem Namen nach) nach einer "`Endstiftung"',\footnote{Vgl.  Husserl,
  Krisis, S. 79.} für welche aus philosophisch-sachlichen Gründen nur die
transzendentale Phänomenologie in Frage kommt. Urstiftungen und Endstiftungen
sind dabei keine kontingenten historischen Ereignisse, sondern Ausdruck einer
im "`Menschentum"' beschlossenen "`Vernunftentelechie"'.\footnote{Vgl.
  Husserl, Krisis, S. 15.} Die Autorität, die hinter diesem "`Telos"' steht,
ist die Autorität der Geschichte und der Tradition oder, wie es Husserl auch
ausdrückt, der "`Wille der geistigen Vorväter"'\footnote{Husserl, Krisis, S.
  78.}. Philosophieren in der Gegenwart ist nur im reflektierten Rückbezug auf
die Tradition möglich, da jeder Versuch, sich von den Vorurteilen der
Tradition zu lösen, nur unter Rückgriff auf "`Selbstverständlichkeiten"'
erfolgen kann, die wiederum einer Tradition entspringen.\footnote{Vgl.
  Husserl, S. 78-79.} Sind die Philosophen nun aber nicht willens oder in der
Lage, sich der Aufgabe, die ihnen durch das historische Telos gegeben ist, zu
stellen, so würde dies zu den in Husserls Augen erschreckenden Konsequenzen
führen, dass die Geschichte keinen Sinn hätte, dass das europäische
Menschentum keine "`absolute Idee"' in sich trüge und "`ein bloß
anthropologischer Typus wie `China' oder `Indien' "' wäre, und dass das
"`Schauspiel der Europäisierung aller fremden Menschheiten"'\footnote{Husserl,
  Krisis, S. 16.} nicht zum Sinn der Geschichte gehören würde. (Für Husserl,
der seine besten Mannesjahre im Zeitalter der Kolonialherrschaft verlebt hat,
war die "`Europäisierung aller fremden Menschheiten"', wie man sieht, noch
nicht mit der Vorstellung bitteren Unrechts verknüpft, so dass sich ihm auch
nicht die Frage stellte, was wohl die "`fremden Menschheiten"' von seiner
Geschichtsphilosophie halten würden.)  Die Verantwortung der Philosophen ist
denn auch denkbar groß, denn die "`eigentlichen Geisteskämpfe des europäischen
Menschentums als solchen spielen sich als {\it Kämpfe der Philosophien}
ab"'\footnote{Husserl, Krisis, S. 15.  (Hervorhebungen im Original.)}, und die
Philosophen sind gar "`{\it Funktionäre der Menschheit}"'.\footnote{Husserl,
  Krisis, S. 17.  (Hervorhebungen im Original.)}

\subsection{Voegelins Kritik des Husserlschen Geschichtsbildes}

Voegelin teilt seine Kritik an Husserls Krisis-Schrift in einem Brief an
Alfred Schütz mit, den er später in seinem Werk "`Anamnesis"' veröffentlicht
hat.\footnote{Brief an Alfred Schütz über Edmund Husserl, 17.  September 1943,
  in: Voegelin, Anamnesis, S. 21-36.} Er war von Husserls Schrift
einerseits sehr positiv beeindruckt. Ihn überzeugte vor allem Husserls
Darstellung des "`Physikalismus"'. Schließlich hätte er darin auch eine
Bestätigung seiner eigenen Kritik an der Verabsolutierung einzelner
Seinsbereiche sehen können. Auch Husserls Positivismuskritik, seine Klage
darüber, dass die positivistisch reduzierten Wissenschaften keine Orientierung
für die drängenden Lebensfragen der Zeit zu bieten vermöchten, liegt genau auf
Voegelins Linie. Lobend äußert sich Voegelin zudem über die von Husserl an
Descartes herausgearbeitete subtile Differenzierung zwischen transzendentalem
und psychologischem Ego. Enttäuscht war er andererseits von Husserls
fast rein erkenntnistheoretischem Ansatz. Nicht nur, dass Voegelin selbst die
erkenntnistheoretischen Probleme nicht für die wirklich wichtigen
philosophischen Fundamentalprobleme hält, was noch als eine Frage bloßer
Vorlieben abgetan werden könnte, sondern Voegelin ist darüber hinaus der
Ansicht, dass erkenntnistheoretische Fragen nicht isoliert betrachtet werden
können.\footnote{Vgl. Voegelin, Anamnesis, S. 21-22.} Allerdings führt er in
diesem Brief an Schütz keine näheren Gründe dafür an.
 
Den größten Teil des Briefes an Schütz füllt jedoch die Kritik an zwei
Aspekten von Husserls Schrift aus, die Voegelin ganz und gar nicht gefielen:
Die Hinwendung Husserls zur Geschichte und Husserls stiefmütterliche
Behandlung von Descartes' dritter und den folgenden Meditationen. Nicht dass
Voegelin an einer Hinwendung zur Geschichte in der Absicht philosophischer
Selbstbesinnung an sich etwas auszusetzen gehabt hätte, aber in der Art und
Weise, wie sich Husserl des Themas Geschichte in der "`Krisis"'-Schrift
annimmt, konnte Voegelin nur zu gut einige der fatalen Züge wiedererkennen,
die ihm von seiner Auseinandersetzung mit den neuzeitlichen
Geschichtsideologien her wohlbekannt waren. An Descartes verkennt Husserl
nach Voegelins Ansicht vollkommen den Zweck der Meditationen, der
entsprechend der christlichen Tradition, welche Descartes, wie Voegelin meint,
aufgreift, nicht in argumentativer Begründung sondern in meditativer
Besinnung liegt und daher auch nicht argumentativ angreifbar ist.

An Husserls Behandlung der Geschichte missfällt Voegelin nun zweierlei: Zum
einen entspricht die von Husserl vorgenommene Auswahl historisch wichtiger
Epochen (griechische Antike, Neuzeit von Descartes bis Kant, Phänomenologie)
nicht Voegelins Geschmack. Zum anderen lehnt Voegelin die kollektivistischen
Züge von Husserls Geschichtsinterpretation ab.

Die Auswahl historischer Epochen bei Husserl erscheint Voegelin deshalb so
mangelhaft, weil sie nach seiner Ansicht erhebliche Lücken enthält. So ist
weder das christliche Mittelalter in Husserls Darstellung enthalten, noch wird
die Philosophie des Deutschen Idealismus angemessen historisch
gewürdigt.\footnote{In Husserls "`Krisis"' erscheint der Deutsche Idealismus
  nur als Annex zur Philosophie Kants. Vgl. Husserl, Krisis, S. 109-112.} Von
einer ernsthaften Berücksichtigung nicht-europäischer Kulturkreise kann schon
gar keine Rede sein. Damit fallen aber einige Abschnitte der
Menschheitsgeschichte weg, welche Voegelin für überaus bedeutend
hielt.\footnote{Vgl. Voegelin, Anamnesis, S. 22-23.}

Es stellt sich die Frage, ob Voegelins Kritik in diesem Punkt berechtigt ist.
Wäre Husserl verpflichtet gewesen, im Rahmen einer Einleitung in die
Phänomenologie nicht nur die Phasen der Philosophiegeschichte anzusprechen,
die die Vorgeschichte der Phänomenologie bilden, sondern alle Phasen, welche
für die geistige Entwicklung der Menschheit insgesamt bedeutsam waren? Wenn
man nicht gerade einen dogmatischen Holismus vertritt, zu welchem Voegelin
gelegentlich neigt, so würde eine Einleitung in die Phänomenologie es
höchstens erfordern, die Vorgeschichte der Phänomenologie darzustellen, nicht
aber, auf die Geistesgeschichte im Ganzen einzugehen oder auch nur auf
Zusammenhänge zur allgemeinen Geistesgeschichte hinzuweisen.
% Es ist ja auch nicht erforderlich, z.B. in einer Geschichte der
% Naturwissenschaften den Auszug aus Ägypten, den Apostel Paulus oder den
% heiligen Thomas von Aquin zu erwähnen, denn keine dieser Personen und
% Ereignisse hat einen Beitrag zur Entwicklung der Naturwissenschaften
% geleistet.

Husserls Geschichtsdarstellung erscheint jedoch in einem ganz anderen Licht,
wenn man berücksichtigt, dass es Husserl auch und vor allem um den Sinn der
Geschichte überhaupt ging und dass er in der Geschichte ein Telos zu finden
meinte, welches für alle Menschen verbindlich sein sollte und nicht nur für
die Phänomenologie betreibenden Philosophen, wiewohl diese Philosophen durch
ihre schmeichelhafte Führungsrolle als "`Funktionäre der
Menschheit"'\footnote{Husserl, Krisis, S. 17.} noch einmal besonders
hervorgehoben werden. Angesichts dieses hohen geschichtsphilosophischen
Anspruchs tadelt Voegelin zu Recht das armselige Bild der Geistesgeschichte
der Menschheit, welches Husserl zeichnet. Die Missachtung wichtiger Epochen der
Menschheitsgeschichte kann bei diesem Anspruch nicht mehr als thematische
Beschränkung entschuldigt werden.

Doch die Ablehnung von Husserls Geschichtsbild ist noch grundsätzlicher, denn
der Anspruch, das Telos der Geschichte bestimmen zu können, ist unabhängig von
der Tiefe und Vollständigkeit der Geschichtsdarstellung, die diesen Anspruch
untermauern soll, als solcher höchst fragwürdig. Er mündet bei Husserl, so wie
Voegelin es nennt, in eine "`averroistische[..]
Spekulation"'.\footnote{Voegelin, Anamnesis, S. 26.} Unter "`averroistischen
Spekulationen"' versteht Voegelin Varianten des Grundgedankens vom Vorrang des
Allgemeinen vor dem Besonderen. Die ungewöhnliche Bezeichnung leitet Voegelin
vom Namen des mittelalterlichen mohammedanischen Philosophen Averroes ab, der
neben Avicenna einer der bedeutendsten Vermittler des Aristoteles und der
antiken Philosophie war. Durch ihn fand die aristotelische Philosophie Eingang
in das Denken des christlichen Mittelalters. Was Voegelin "`averroistische
Spekulation"' nennt, ist denn auch eine Vorstellung, die schon in der antiken
Philosophie ihre Grundlage hat.\footnote{Vgl. Voegelin, Anamnesis, S. 26.} Es
handelt sich dabei -- soweit man es Voegelins Text entnehmen kann -- um eine
sehr allgemeine und etwas vage metaphysische Vorstellung, nach der es einen
Primat der Wahrheit, des Wertes und der Wirklichkeit des Allgemeinen vor dem
Speziellen, der Klasse vor dem Individuum oder des Ganzen vor dem Teil gibt.
Diese Grundvorstellung kann in den verschiedensten Formen und bezogen auf die
verschiedensten Gegenstände auftauchen. Auf gesellschaftspolitischer Ebene
führt dieser Gedanke sehr rasch zum Kollektivismus. Besonders problematisch
wird die "`averroistische Spekulation"', wenn sie im Verein mit einem
Exklusivitätsprinzip auftritt, nach welchem bestimmte Gruppen oder Individuen
aus dem maßgeblichen Kollektiv ausgeschlossen werden.\footnote{Vgl. Voegelin,
  Anamnesis, S. 26-27. -- Vgl. auch Eric Voegelin: Der autoritäre Staat. Ein
  Versuch über das österreichische Staatsproblem, Wien / New York 1997 (zuerst
  1936), im folgenden zitiert als: Voegelin, Autoritärer Staat, S. 25-26.}

Der averroistisch-spekulative Charakter von Husserls Geschichtsbild wird
besonders deutlich, wenn Husserl das Telos der Geschichte als eine durch den
Einzelnen "`{\it hindurchgehende} Willensrichtung"'\footnote{Husserl, Krisis,
  S. 78.} darstellt. Der Einzelne wird zu einem bloßen Agenten oder Medium
jener höheren Willensrichtung, auf die allein es ankommt. Voegelin spricht
deshalb auch von dem "`kollektivistische[n] Telos"'\footnote{Voegelin,
  Anamnesis, S. 27.} Husserls. Auch die Beschränkung auf ein maßgebliches
Kollektiv, welches dieses Telos vertritt, kommt bei Husserl in der
Einschränkung des eigentlichen Menschentums auf das europäische Menschentum
vor. In historischer Perspektive drückt sich nach Voegelin dieser Gedanke
bei Husserl dadurch aus, dass der überwiegende Teil der Menschheitsgeschichte
schlicht übergangen wird zugunsten der vermeintlich wesentlichen Etappen,
welche die Entfaltung des "`Telos"' verkörpern.\footnote{Vgl. Voegelin,
  Anamnesis, S. 27-28.}

Aber Husserl ist für Voegelin nicht nur "`Fortschrittsphilosoph im besten
Stile der Reichsgründerzeit"'.\footnote{Voegelin, Anamnesis, S. 28.} Darüber
hinaus erblickt Voegelin in Husserls durch die beiden Wendemarken der
Urstiftung und der Endstiftung unterteilten Geschichte jenes
Drei-Phasen-Geschichtsbild, welches, von der christlichen Heilsgeschichte
herstammend, Eingang in so viele Geschichtsideologien der Neuzeit gefunden
hat.  Die messianische Endzeit, die in diesen Geschichtsideologien anders als
in der christlichen Heilslehre nicht überzeitlich sondern geschichtsimmanent
verstanden wird, beginnt bei Husserl mit der Endstiftung. Natürlich hütet sich
Voegelin, Husserl mit gewalttätigen politischen Bewegungen wie dem Kommunismus
oder dem Nationalsozialismus in eine Reihe zu stellen. Aber die
Strukturverwandtschaft von Husserls Geschichtsbild und manchen modernen
Geschichtsideologien scheint ihm doch unverkennbar.\footnote{Vgl. Voegelin,
  Anamnesis, S. 28-31.}

Einige Interpreten der Husserlschen Philosophie versuchen Husserl vor dem
Verdacht der Geschichtsideologie in Schutz zu nehmen, indem sie behaupten,
dass Husserl nur als Phänomenologe innerhalb der "`Epoché"', jener
phänomenologischen Operation der Konzentration auf das Phänomen in seiner
Selbstgegebenheit und unter Absehung von dessen
Wirklichkeitsprätentionen,\footnote{Vgl. Edmund Husserl: Die phänomenologische
  Methode. Ausgewählte Texte I (Hrsg. von Klaus Held), Stuttgart 1985, S.
  141-143.} gesprochen habe. Seine geschichtsphilosophischen Ausführungen
seien daher eher als unverbindliche Besinnungen persönlicher Art auf die ganz
privaten Absichten und Zwecke des Phänomenologen Husserl zu
verstehen.\footnote{Vgl.  Gilbert Weiss: Theorie, Relevanz, Wahrheit. Zum
  Briefwechsel zwischen Eric Voegelin und Alfred Schütz (1938-1959), München
  1997, S. 24-28. -- Vgl. die Einleitung von Elisabeth Ströker in: Husserl,
  Krisis, S. XXIX.} Diese Art der Apologie ist jedoch nicht überzeugend, denn
die phänomenologische Epoché dient nicht minder der Gewinnung
allgemeinverbindlicher Resultate als irgendeine wissenschaftliche
Forschungsmethode. Idealiter liefert sie sogar Ergebnisse von "`apodiktischer
Evidenz"'. Selbst wenn Husserl, ohne es übrigens irgendwo zu erwähnen,
innerhalb der Epoché gesprochen hätte, so würden seine Äußerungen dadurch
keineswegs akzeptabler.  Husserl hätte dann, statt zu behaupten, die
Geschichte habe ein Telos, lediglich behauptet, die Geschichte stelle sich uns
notwendig so dar, als habe sie ein Telos, was aber nicht weniger fragwürdig
wäre.

Man mag einwenden, dass mit diesem recht kritischen Ergebnis das
geistesgeschichtliche Verdienst von Husserls Krisis-Schrift ungenügend
gewürdigt wird. Geistesgeschichtlich gesehen, stellt Husserls Krisis-Schrift
einen höchst bemerkenswerten Versuch einer Verbindung von Traditionalismus und
Rationalismus, von geschichtlichem Denken und systematischer Philosophie, von
religiösem Patriarchalismus und Vernunfterkenntnis dar, eine Synthese, die trotz
der Wilhelminischen Einlassungen Beachtung verdient. Allerdings zeigt
auch gerade Voegelins Kritik, dass diese Synthese nicht aufgeht.

\subsection{Voegelins Einwände gegen die Fortschrittsgeschichte}

Über die Verfehltheit von Ideologien einer geschichtlichen Endzeit lässt sich
Voegelin in seinem Brief an Alfred Schütz nicht weiter aus. (Sie ist ohnehin
offensichtlich genug.) Was hat Voegelin aber daran auszusetzen, die
Geschichte, so wie es bei Husserl geschieht, als eine Geschichte des
Fortschritts zu schreiben? Für Voegelin spielt dabei sowohl ein moralisches
als auch ein eher wissenschaftliches Motiv eine Rolle. Moralisch kritikwürdig
erscheint Voegelin die Inhumanität, die darin liegt, die vergangenen Epochen
und das Streben der damals lebenden Menschen nur als Mittel zum Zweck für die
Gegenwart zu betrachten. Wissenschaftliche Schwierigkeiten entstehen für
Voegelin dadurch, dass vergangene Epochen nicht angemessen verstanden werden
können, wenn in ihnen nur eine Vorstufe der Gegenwart gesehen wird.

Die moralische Problematik der Fortschrittsphilosophie erläutert Voegelin
unter Rückgriff auf Kant. Kant teilte mit vielen anderen Aufklärern die
Ansicht, dass es in der Geschichte einen Fortschritt zum Besseren gibt, so
dass sich der Zustand der menschlichen Gesellschaft immer mehr, wenn auch
niemals endgültig, einem moralischen Optimum (jeder handelt gut und keinem
geschieht ein Unrecht) annähert. Zugleich äußert Kant jedoch auch sein
"`Befremden"' darüber, dass die späteren Generationen von allen Fortschritten
der vorhergehenden profitieren, welche ihrerseits, obwohl sie denselben
Beitrag zum Fortschritt geleistet haben, nicht in gleichem Maße die Vorteile
davon genießen können.\footnote{Vgl. Immanuel Kant: Idee zu einer allgemeinen
  Geschichte in weltbürgerlicher Absicht (Dritter Satz), in: Immanuel Kant:
  Schriften zur Geschichtsphilosophie, Stuttgart 1985, S. 21-39 (S. 25). In
  Voegelins Kant-Interpretation tritt gegenüber Kant eine leichte
  Bedeutungsverschiebung ein. Während Voegelin hier eine Frage des Sinns
  sieht, geht es bei Kant (wenigstens dem Sachzusammenhang nach, wenn auch
  noch andere Motive im Hintergrund eine Rolle spielen mögen) eher um eine
  Frage des materiellen Ausgleichs.  Dies hat natürlich auch Folgen für die
  Interpretation der geistesgeschichtlichen Rolle Kants, die hier jedoch nur
  kurz angedeutet werden können: Es erscheint grundsätzlich fragwürdig, in
  Kants Geschichtsphilosophie (bzw. in den Geschichtsvorstellungen der
  Aufklärer überhaupt) eine "`averroistische Konzeption"' zu sehen. Die
  Geschichtsphilosophie Kants war durchaus keine Geschichtssinntheorie (wie
  die Geschichtsphilosophien des Deutschen Idealismus), denn nicht die
  Geschichte verleiht bei Kant dem Leben und Schaffen des Einzelnen Sinn und
  Wert (und auch nicht die Glückseligkeit, die nur eine Belohnung ist, auf die
  er nach dem Tode hoffen darf), sondern die Erfüllung der Pflicht (meine
  Interpretation). Kants Fortschrittsphilosophie war der Ausdruck der
  optimistischen Hoffnung, dass sich das Gute einmal durchsetzen wird, aber das
  Gute ist bei Kant (noch) nicht dadurch definiert, wer in der Geschichte
  siegreich bleibt.  Die grundsätzliche Möglichkeit, Geschichte als
  Fortschrittsgeschichte zu schreiben, ohne in "`averroistische
  Spekulationen"' zu verfallen, kann auch Voegelin nicht leugnen, sonst müsste
  er sich wegen der Fortschritte der spirituellen Ausbrüche, von denen "`Order
  and History"' handelt, selbst der "`averroistischen Spekulation"'
  bezichtigen. Voegelins Bild einer Kontinuität von der aufklärerischen
  Fortschrittsphilosophie (Geschichtsideologie ist Fortschrittsphilosophie
  minus Humanität plus Endzeitglaube) zu den modernen Geschichtsideologien
  erscheint deshalb teilweise fragwürdig.} Voegelin erblickt in Kants
Befremden eine humane Hemmung, die früheren Generationen nur als Mittel zum
Zweck der Verwirklichung eines geschichtlichen Telos zu sehen, welches bei
Kant in der Vervollkommnung der Vernunftanlagen besteht. Bei Husserl fehlt
diese Humanität und zudem tritt die "`Endstiftung"' anders als Kants
Vervollkommnung des Vernunftgebrauchs tatsächlich in der Geschichte ein.
Diese beiden Punkte markieren für Voegelin den Übergang von der
averroistischen Konzeption aufklärerischer Fortschrittsgeschichte zu den noch
militanteren averroistischen Spekulationen, die sich in den modernen
Geschichtsideologien und, folgt man Voegelin, sogar in seriösen historischen
Untersuchungen wie Otto Gierkes Genossenschaftsrecht finden.  Husserls
Geschichtsbild stellt für Voegelin deshalb eine durchaus zeittypische
Erscheinung dar.\footnote{Vgl.  Voegelin, Anamnesis, S. 28-30.}

Die weniger ethische als wissenschaftliche Problematik dieser Art von
Geschichtsdarstellung besteht für Voegelin darin, dass der Historiker "`die
eigene geistige Position, mit ihrer historischen Bedingtheit,
verabsolutiert"'\footnote{Voegelin, Anamnesis, S. 31.}  und auf die
historischen Fakten nur zurückgreift, um die eigene Position zu stützen, ohne
dabei jemals verstehend in das historische Material einzudringen. Bei Husserl
tritt diese Verabsolutierung in besonders krasser Form auf, da Husserl sich
nach Voegelins Ansicht gegen die Möglichkeit empirischer Kritik systematisch
abschirmt. Voegelin spielt hier wahrscheinlich auf Husserls theoretische
Vorgabe an, dass der Sinn der philosophischen Positionen der Vergangenheit
nicht aus den Selbstzeugnissen der Denker, sondern nur durch die Heraushebung
einer erst rückblickend aus der Gegenwart erkennbaren latenten
"`Willensrichtung"' zu bestimmen sei.\footnote{Vgl.  Voegelin, Anamnesis,
  S. 31. -- Vgl. Husserl, Krisis, S. 78-80.}

Wie sieht für Voegelin aber die Alternative zu diesen Formen von
Geschichtsklitterung aus? Nach Voegelins Überzeugung ist es die Aufgabe des
Historikers, in der Geistesgeschichte "`jede geschichtlich geistige Position
bis zu dem Punkt zu durchdringen, an dem sie in sich selbst ruht, d.h. in dem
sie in den Transzendenzerfahrungen des betreffenden Denkers verwurzelt
ist."'\footnote{Voegelin, Anamnesis, S. 31.} Es kommt weiterhin darauf an,
"`die geistig-geschichtliche Gestalt des andern bis zu ihrem Transzendenzpunkt
zu durchdringen und in solcher Durchdringung die eigene Ausformung der
Transzendenzerfahrung zu schulen und zu klären."'\footnote{Voegelin,
  Anamnesis, S. 31.} Die recht verstandene Geistesgeschichte verfolgt also
zwei Ziele: Verstehen der geistigen "`Gestalten"' der Vergangenheit und
Klärung der eigenen Beziehung zur Transzendenz. \label{Selbstzeugnisse1} Das
Verstehen hat dabei strikt am "`Leitfaden"' der "` `Selbstzeugnisse' der
Denker"'\footnote{Voegelin, Anamnesis, S. 32.} zu erfolgen. Die Klärung des
Selbstverständnisses durch das "`geistesgeschichtliche Verstehen"' zielt
letztlich auf eine "`Kathar[s]is, eine {\it purificatio} im mystischen Sinn,
mit dem persönlichen Ziel der {\it illuminatio} und der {\it unio
  mystica}"'.\footnote{Voegelin, Anamnesis, S. 31.}  Wird dieses
"`geistesgeschichtliche Verstehen"' systematisch ausgeübt, so kann es "`zur
Herausarbeitung von Ordnungsreihen in der geschichtlichen Offenbarung des
Geistes führen."'\footnote{Voegelin, Anamnesis, S. 32.}

% Das "`philosophische
% Ziel"' der Geistesgeschichte besteht nämlich für Voegelin darin, "`jede
% geschichtlich geistige Position bis zu dem Punkt zu durchdringen, an dem sie
% in sich selbst ruht, d.h. in dem sie in den Transzendenzerfahrungen des
% betreffenden Denkers verwurzelt ist."'

Es ist zu berücksichtigen, dass Voegelin dies 1943, also noch lange vor seinem
geschichtlichen Hauptwerk "`Order and History"', geschrieben hat. Voegelins
Ausführungen sind also eher noch als ein frühes Programm zu
verstehen.\footnote{Vgl. Jürgen Gebhardt: Toward the Process of universal
  Mankind. The Formation of Voegelin's Philosophy of History, in: Ellis Sandoz
  (Hrsg.): Eric Voegelins Thought. A critical appraisal, Durham N.C. 1982, S.
  67-86, S. 78.} Dennoch kann die Frage aufgeworfen werden, ob dieses Programm
eine gangbare Alternative zu den von Voegelin abgelehnten "`averroistischen
Konzeptionen"' von Geschichte darstellt. In dieser Hinsicht fällt auf, dass
Voegelins Programm bereits sehr erhebliche Vorentscheidungen über das Wesen
der Geistesgeschichte enthält.  Voegelin unterstellt, dass jeder bedeutsamen
geschichtlichen Gestalt des Geistes eine Transzendenzerfahrung zu Grunde
liegt. Aber nicht alle Denker gründen ihr Denken auf Transzendenzerfahrungen.
Die meisten Philosophen gelangen zu ihren Resultaten durch Überlegungen,
welche mit einer Auslegung von Transzendenzerfahrungen nichts gemein haben. Es
könnte nun behauptet werden, dass die Nichtbeachtung der Transzendenz
ebenfalls eine bestimmte, wenn auch eine deformierte Beziehung zur
Transzendenz repräsentiert. Wird dies behauptet, so wird jedoch gleichzeitig
eine andere methodische Forderung Voegelins vernachlässigt, nämlich die,
"`jede geschichtlich geistige Position bis zu dem Punkt zu durchdringen, an
dem sie in sich selbst ruht"',\footnote{Voegelin, Anamnesis, S. 31.} denn
durch Betrachtung eines nicht-religiösen Denkers unter dem Gesichtspunkt der
Transzendenzerfahrungen werden an diesen Denker völlig heteronome Maßstäbe
herangetragen. Voegelin stellt hier also zwei einander widersprechende
methodische Forderungen auf: Zum einen, die Denker der Vergangenheit strikt
auf Grundlage ihrer Selbstzeugnisse zu erfassen, und zum anderen, jede
geistige Position in der Vergangenheit zwingend als Ausdruck einer
Transzendenzerfahrung zu verstehen.

Betrachtet man dieses frühe historische Programm im Hinblick auf sein späteres
geschichtliches Werk, dann fallen einige Abweichungen auf: So ließ sich etwa
die Forderung, vom Selbstverständnis der Denker der Vergangenheit auszugehen,
nicht durchhalten. Wenigstens für bestimmte philosophische Systeme gibt
Voegelin diese Forderung später auch explizit auf.\footnote{Vgl. Voegelin,
  Anamnesis, S. 310. -- Siehe auch Seite \pageref{Selbstzeugnisse2} in diesem
  Buch.}  Weiterhin kann die scharfe Kritik, die Voegelin etwa an Otto
Gierkes "`phantastischer Vergewaltigung Bodins"'\footnote{Voegelin, Anamnesis,
  S. 30.}  übt, auch gegen Voegelins eigene Klassikerinterpretationen gekehrt
werden, die nicht selten eher kongenial als historisch und philologisch
zuverlässig sind.  Sogar der Vorwurf der "`averroistischen Spekulation"'
könnte gegen Voegelin selbst gerichtet werden, wenn er in seinen späteren
Schriften das menschliche Bewusstsein an einem Prozess partizipieren lässt,
"`durch den die Wahrheit der Realität sich ihrer selbst bewusst
wird"',\footnote{Eric Voegelin: Äquivalenz von Erfahrungen und Symbolen in der
  Geschichte, in: Eric Voegelin, Ordnung, Bewußtsein, Geschichte, Späte
  Schriften (Hrsg. von Peter J. Optiz), Stuttgart 1988, S. 99-126 (S. 123).}
was von Husserls durch die konkreten Philosophen hindurchgehender
Willensrichtung nicht allzu weit entfernt ist.  Voegelins eigene
Geschichtskonstruktion ist in diesen Punkten derjenigen Husserls, die er zu
recht kritisiert, näher, als dies die Entschiedenheit seiner Vorwürfe erwarten
lassen sollte.

% \footnote{Für Voegelin war es offenbar
%   selbstverständlich, daß einen Menschen verstehen heißt, seine
%   Transzendenzerfahrungen nachzuvollziehen. Es gibt hier eine Parallele zur
%   Einleitung von Order and History II., wo Voegelin die Frage nach dem
%   Zusammenhang der Menschheit mit der allen gemeinsamen Mission der Suche nach
%   Wahrheit (Quest for truth) beantwortet. Er behauptet sogar, daß dies die
%   einzige Möglichkeit sei und schließt damit andere denkbare Alternativen
%   (z.B. gleiche Sehsüchte und Wünsche, Nöte und Freuden weltlicher Art) von
%   vornherein aus.}

\subsection{Voegelins Descartes-Deutung}

Dass Voegelin sich große Interpretationsfreiheiten erlaubt, wird auch an
seiner Auseinandersetzung mit Husserls Descartes-Bild deutlich, denn die
rationale und wissenschaftliche Ausrichtung von Descartes' Denken lässt im
Grunde wenig Raum für die Eindrücke von Transzendenzerfahrungen. Für Voegelin
greift Husserls Descartes-Interpretation zu kurz, weil Husserl der Philosophie
des Descartes eine rein erkenntnistheoretische Bedeutung unterstellt, und weil
Husserl nach Voegelins Ansicht den tieferen Sinn von Descartes Gottesbeweis in
der dritten Meditation missversteht.

Was die rein erkenntnistheoretische Deutung des Descartes durch Husserl
betrifft, so gilt dasselbe, was bereits über Voegelins Kritik an Husserls
Geschichtsbild gesagt wurde: Sofern es Husserl um eine Einleitung in die
Phänomenologie geht, ist es sein gutes Recht, die Aspekte der Philosophie von
Descartes herauszugreifen, die für die Phänomenologie von Bedeutung sind, und
dies sind nun einmal die erkenntnistheoretischen. Da Husserls "`Krisis"' aber
noch von wesentlich höheren Aspirationen getragen wird, sind Einwände gegen
das Herausgreifen bestimmter Einzelaspekte der Philosophie des Descartes
grundsätzlich legitim.

Etwas anders verhält es sich jedoch mit Husserls Vernachlässigung des
Gottesbeweises in der dritten Meditation von Descartes "`Meditationen über die
Grundlagen der Philosophie"'. Husserl erwähnt in der "`Krisis"' nur kurz, dass
der Gottesbeweis falsch sei, und geht auf die dritte und die folgenden
Meditationen gar nicht weiter ein.\footnote{Vgl. Husserl, Krisis, S. 82.}  Er
scheint sich hier an eine damals wie heute geläufige Lesart zu halten, nach
der die dritte bis sechste Meditation von Descartes noch durch und durch
scholastisch sind, und philosophisch Belangvolles nur in den ersten beiden
Meditationen zu finden ist.\footnote{Vgl. Bertrand Russell: A History of
  Western Philosophy, London, Sidney, Wellington 1990, im folgenden zitiert
  als: Russell, History of Western Philosophy, S. 550.} Auch Voegelin sieht in
Descartes' Meditationen ein durchaus traditionelles Unternehmen. Für ihn sind
die gesamten "`Meditationen"' des Descartes eine Spielart der christlichen
Meditation, wie sie seit Augustinus insbesondere bei den mystischen Denkern
üblich war. Wenn man Voegelin Glauben schenkt, so war es das Ziel der
Meditationen von Descartes, wie in der christlichen Meditation üblich, in der
Abkehr von der Welt den Kontakt zur Transzendenz als der höchsten Wirklichkeit
zu finden. Das Neue bei Descartes besteht nach Voegelin darin, dass Descartes
-- anders als seine Vorläufer -- die Meditation nicht aus einer Haltung der
Verachtung der Welt heraus unternimmt, sondern in der Absicht, sich durch den
Kontakt zur höchsten Realität der Realität bzw.  der Objektivität der Welt zu
versichern. Der Gottesbeweis in der dritten Meditation ist, Voegelin zufolge,
in diesem Zusammenhang nicht als logische Beweisführung, sondern als
sekundärer, in der Stilform der {\it demonstratio} gefasster Ausdruck der als
solcher unmittelbaren und eines Beweises nicht bedürftigen Gotteserfahrung zu
sehen.\footnote{Vgl. Voegelin, Anamnesis, S. 32-35.}

Voegelins Descartes-Interpretation ergibt sich nicht ganz zwanglos aus dem
Text der "`Meditationes"', da dort von der Gotteserfahrung nur sehr am Rande
die Rede ist. Wäre mit der Erfahrung Gottes schon seine Existenz und darüber
hinaus die Existenz der Welt mitgegeben, so erscheint es ganz und gar unnötig,
dass Descartes versucht, mit Hilfe scholastischer Schlussweisen, deren
Falschheit Voegelin gar nicht bestreitet, zu zeigen, dass die Vorstellung
Gottes anders als alle anderen Vorstellungen die Existenz des Vorgestellten
impliziert.\footnote{Vgl. René Descartes: Meditationen über die Grundlagen der
  Philosophie, Hamburg 1993, S. 36ff.} Darüber hinaus unterscheidet sich
Descartes' Text nicht nur in der Intention der Weltvergewisserung sondern auch
in der Art und Weise der Darstellung recht deutlich von den "`ekstatischen
Konfessionen"'\footnote{So der Titel einer Sammlung mystischer
  Erfahrungsberichte, die teilweise mit den Mitteilungen "`Cloud of
  Unknowing"', die von Voegelin in diesem Zusammenhang angeführt wird,
  vergleichbar sind (z.B. der Auszug aus der Erzählung des Tewekhul-Beg,
  Schüler des Moll\^a-Sch\^ah, über sein mystisches Noviziat, S. 47-49.) in:
  Buber, Martin (Hrsg.): Ekstatische Konfessionen, Leipzig 1921.} von
Mystikern wie etwa dem anonymen Autor der von Voegelin zum Vergleich
herangezogenen "`Cloud of Unknowing"'.\footnote{Vgl. A Book of Contemplation
  wich is called the Cloud of Unknowing, in which a Soul is oned with God.
  (ed. Evelyn Underhill, 2nd ed.  John M. Watkins), London 1922, auf:
  http://www.ccel.org/u/unknowing/cloud.htm (Host: Christian Classics Ethereal
  Library at Calvin College. Zugriff am: 5.4.2000). -- Ein Vergleich mit
  Descartes, wie Voegelin ihn anstellt (vgl. Voegelin, Anamnesis, S. 33)
  bietet sich noch am ehesten an, wenn man das Ende des vierten und das fünfte
  Kapitel dieses Werkes dem Beginn von Descartes' dritter Meditation
  gegenüberstellt. Aber die oberflächlichen Ähnlichkeiten, die sich dabei
  auf\/finden lassen, können kaum über den himmelweiten Unterschied in Inhalt,
  Gegenstand, Absicht, Ausführung und Ziel dieser beiden Werke
  hinwegtäuschen.} Es fällt daher nicht unbedingt leicht, die "`Meditationen"'
unter dieser Art von Literatur einzureihen. Aber selbst wenn man einmal
annimmt, dass die "`Meditationen"' von Descartes nur eine mit anderen Mitteln
vorgenommene Artikulation derselben mystischen Transzendenzerfahrung sind, so
stellt sich die Frage, ob damit irgendetwas gewonnen ist. Wenn schon der
Versuch, die Existenz der Außenwelt argumentativ zu beweisen, fehlgeschlagen
ist, dann kann die Existenz der Außenwelt durch den Rückgriff auf eine
Transzendenzerfahrung ebensowenig begründet werden. Denn es ist zwar
vorstellbar, dass eine Transzendenzerfahrung ein sehr starkes Vertrauen in das
Sein der Welt einflößt, aber außer der bloßen Erfahrung einer Transzendenz ist
nicht auch das Sein der Transzendenz selbst und außerdem noch das objektive
Sein der Welt in dieser Erfahrung mitgegeben, da auch bei einer
Transzendenzerfahrung eine Täuschung denkbar ist, genauso wie eine
Halluzination oder Sinnestäuschung bei der Sinneserfahrung. Die Erfahrung der
Transzendenz, was immer das für eine Erfahrung auch sein mag, bleibt deshalb
für das erkenntnistheoretische Problem der Existenz der Außenwelt ohne Belang.
Bloße Gefühle der Gewissheit, wie sie sich in einer Meditation einstellen
mögen, sind eben noch keine Gewissheit.

Einzuräumen ist jedoch, dass Voegelin einen Schwachpunkt von Husserls Theorie
der "`Konstitution"' des Seins durch Leistungen des Ego trifft, wenn er in
diesem Zusammenhang die Frage aufwirft, woher das Ego die Funktion bekommt,
"`aus der Subjektivität die Objektivität der Welt zu
fundieren"'\footnote{Voegelin, Anamnesis, S. 36. -- Vgl. Husserl,
  Cartesianische Meditationen, S. 84-91 (§ 40-41).}. Husserls
phänomenologische Methode gerät in der Tat an ihre Grenzen, wenn es darum
geht, das Selbstsein anderer Menschen oder auch nur von Dingen zu begründen.

%%% Local Variables: 
%%% mode: latex
%%% TeX-master: "Main"
%%% End: 













%%% Local Variables: 
%%% mode: latex
%%% TeX-master: "Main"
%%% End: 


\section{"`Zur Theorie des Bewußtseins"'}

\subsection{Voegelins Schrift "`Zur Theorie des Bewußtseins"'}

In seinem Aufsatz "`Zur Theorie des Bewußtseins"'\footnote{Voegelin,
  Anamnesis, S. 37-60.} legt sich Voegelin Rechenschaft über seinen eigenen
philosophischen Standpunkt ab. Einleitend erklärt Voegelin, dass seine
Aufzeichnungen die Ergebnisse anamnetischer Experimente enthielten, doch
bezieht sich dies wohl eher auf die auf diesen Aufsatz folgenden Berichte von
Kindheitserinnerungen. Der Aufsatz selbst zumindest besteht weit überwiegend
aus theoretischer Diskussion.

Es fällt nicht leicht, die Thematik dieses Aufsatzes zu beschreiben,
denn Voegelin reißt darin viele verschiedene Themen an. Den
Hauptthemenschwerpunkt bildet eine Beschreibung der Struktur des
Bewusstseins, seiner Beziehung zur Welt und eine Diskussion der
Möglichkeiten des Bewusstseins dasjenige, was außerhalb des Bewusstseins
liegt, zu erfahren und zu beschreiben. Daneben skizziert Voegelin
umrisshaft eine ontologische Theorie und schließlich versucht Voegelin,
nachdem er der Ontologie das Primat vor der Bewusstseinsphilosophie
eingeräumt hat, das Auftreten der von ihm für falsch oder zu einseitig
gehaltenen Bewusstseinsphilosophien historisch und wissenssoziologisch zu
erklären.

Den Ausgangpunkt für Voegelins Überlegungen bildet eine Kritik der Theorien
des Bewusstseins, die die Bewusstseinsstrommetapher in den Mittelpunkt ihrer
Beschreibung stellen. Voegelin hält dies für eine falsche Akzentuierung: Zwar
strömt das Bewusstsein auch, aber das Strömen ist weder der wesentliche noch
ein alle anderen Bewusstseinsleistungen bedingender Faktor im Bewusstsein. Vor
allem tritt das Strömen nur bei bestimmten Bewusstseinsvorgängen zu Tage wie
etwa beim Hören von Tönen. Bei anderen Bewusstseinstätigkeiten -- Voegelin
beschreibt als Beispiel die inneren Vorgänge beim Betrachten eines Gemäldes --
lässt sich das Strömen des Bewusstseins nur erfassen, wenn die Aufmerksamkeit
von der Hauptsache abgelenkt wird. Voegelin betrachtet es daher als eine
Spekulation, wenn der Bewusstseinsstrom als die Grundform aller
Bewusstseinsvorgänge aufgefasst wird.\footnote{Vgl. Voegelin, Anamnesis,
  S. 37-43.}

Nach Voegelins eigener Vorstellung vom Aufbau des Bewusstseins ist das
Bewusstsein nicht durch zeitliches Fließen sondern thematisch durch
Aufmerksamkeitszuwendung und -abwendung strukturiert. Voegelin nimmt an, dass
es im Bewusstsein ein Aufmerksamkeitsquantum von nicht genau bestimmter aber
beschränkter Größe gibt, welches verschiedenen Bereichen des Bewusstseins in
mehr oder weniger starker Konzentration zugewandt werden kann. Im Bewusstsein
gibt es nun zwei besonders ausgezeichnete Bereiche, Voegelin nennt sie
"`Erhellungsdimensionen"', denen bestimmte Formen der
Aufmerksamkeitszuwendung, "`Erinnerung"' und "`Projektion"', entsprechen.
Diese beiden Erhellungsdimensionen bezeichnet Voegelin daher passenderweise
als "`Vergangenheit"' und "`Zukunft"'. Aus dem Zusammenspiel von
Aufmerksamkeitszuwendung und den Erhellungsdimensionen "`Vergangenheit"' und
"`Zukunft"' leitet sich die Vorstellung der (inneren wie äußeren) Zeit ab.
Voegelin vertritt also, wie es scheint, eine idealistische Auffassung der
Zeit, die derjenigen nicht unähnlich ist, die Augustinus als erster in den
"`Bekenntnissen"' dargelegt hat.\footnote{Aurelius Augustinus: Bekenntnisse,
  Stuttgart 1998, S. 312-330 (Elftes Buch. XIII.15 - XXVIII.38). Man könnte
  geneigt sein, gegen die Logik dieser Art von Zeittheorien einzuwenden, dass
  die Vorgänge innerhalb des Bewusstseins, aus denen die Zeit hervorgeht, doch
  schon die Zeit als solche voraussetzen. Aber in der Tat setzen sie höchstens
  bestimmte (zeitliche) Relationen voraus.  (Voegelin scheint diese Kritik
  jedoch ernst zu nehmen, denn er bringt sie etwas später als Selbsteinwand
  vor; vgl. Voegelin, Anamnesis, S. 54/55.) Die idealistischen Zeittheorien
  scheitern aus einem anderen Grund: Wenn die Zeit ausschließlich eine Form
  oder Leistung des Bewusstseins ist, so ist nicht erklärlich, wie die
  Kommunikation zwischen den "`Bewusstseinen"' verschiedener Menschen zeitlich
  aufeinander abgestimmt erfolgen kann, da die Botschaften des einen an das
  andere Bewusstsein doch durch eine äußere Welt hindurch müssen, in der die
  Zeitrelationen, die der idealistischen Annahme zufolge reine
  Bewusstseinsprodukte sind, verloren gehen müssten. (Das Argument stammt aus:
  Russell, History of Western Philosophy, S. 689, wo es im Zusammenhang mit der
  Diskussion der idealistischen Raum- und Zeittheorie Kants auftaucht. Es
  lässt sich aber unmittelbar auch auf andere idealistische Zeittheorien
  übertragen.)} Da das Bewusstsein insgesamt endlich ist, so ist auch die
Zeitvorstellung aus einem endlichen Vorgang abgeleitet.  Dieser Vorgang ist
der einzige wirkliche Prozess, von dem wir eine innere Erfahrung haben.
Voegelin behauptet, dass dadurch der Bewusstseinsprozess "`zum Modell des
Prozesses überhaupt"'\footnote{Voegelin, Anamnesis, S. 44.}  wird, und dass
all unsere Begriffe von Prozessen nur von diesem einzigen uns zur Verfügung
stehenden Modell abgezogen sind.\footnote{Nur wenige Seiten weiter behauptet
  Voegelin sonderbarerweise genau das Gegenteil: "`...  die {\it Ordnung} des
  Augenblicksbildes in der Dimension, die durch die Erhellung geschaffen wird,
  zur Sukzession eines Prozesses erfordert Erfahrungen von
  bewußtseinstranszendenten Prozessen."'  (Voegelin, Anamnesis, S. 55.)}
Infolge der Endlichkeit dieses Modells entstehen Ausdruckskonflikte bei der
Beschreibung unendlicher Prozesse, wie sie außerhalb des Bewusstseins
gelegentlich vorkommen können. Von diesen Ausdruckskonflikten rühren nach
Voegelins Überzeugung auch die Kantischen Antinomien und die Paradoxe der
Mengenlehre her.\footnote{Vgl.  Voegelin, Anamnesis, S. 44/45. -- Welche
  Paradoxe der Mengenlehre Voegelin meint, geht aus dem Text leider nicht
  hervor.  Wahrscheinlich denkt Voegelin dabei an die Russellsche Antinomie,
  die eine Variante des klassischen Lügnerparadoxons ist und auftritt, wenn
  man versucht die Menge aller Mengen zu bilden, die sich nicht selbst als
  Element enthalten.  (Vgl. auch die Kritik dieser Passage im folgenden
  Abschnitt.)}  Um unendliche Prozesse, die zwar erfahren bzw. erahnt aber
nicht widerspruchsfrei beschrieben werden können, überhaupt in irgendeiner
Weise zu artikulieren, ist es erforderlich, sich der geheimnisvollen
Ausdrucksweise der Mythensymbolik zu bedienen. Den Begriff "`Mythensymbol"'
definiert Voegelin als ein "`finites Symbol, das für einen transfiniten
Prozess `transparent' sein soll."'\footnote{Voegelin, Anamnesis, S. 45.} Eine
genauere Eingrenzung, welches die unendlichen Prozesse sind, die des
Ausdruckes durch die Mythensymbolik bedürfen, gibt Voegelin nicht an.  Seine
Beispiele legen nahe, dass es sich dabei um die Gegenstände handelt, die in
der Religion zum welttranszendenten Bereich gerechnet werden.  Als Beispiele
dieses Gebrauchs der Mythensymbolik führt Voegelin nämlich einige allegorische
Deutungen bekannter Mythensymbole an. So vermittelt etwa die unbefleckte
Empfängnis "`die Erfahrung eines transfiniten geistigen
Anfangs"'\footnote{Voegelin, Anamnesis, S. 45.}.  Voegelin versäumt es leider
zu klären, wie ein transfiniter geistiger Anfang Gegenstand der Erfahrung
werden kann, und ob es jemals einen Menschen gegeben hat, der etwas derartiges
tatsächlich erfahren hat.

Ein größeres Problem im Zusammenhang mit der Mythensymbolik stellt die Frage
der Adäquatheit des Mythos dar. Damit meint Voegelin die Frage, ob ein Mythos
überhaupt eine Erfahrung ausdrückt, und ob er, wenn er es tut, die Erfahrung
richtig zum Ausdruck bringt. Voegelin erläutert dieses Problem am Beispiel
zweier Mythen bei Platon, dem von Platon bewusst als reines
Propagandainstrument konzipierten Mythos der drei Metalle, die in Platons
Dialog {\em Politeia} den drei Sozialklassen zugeordnet werden, und dem Mythos
aus den {\em Nomoi}, in dem die Menschen von den Göttern als Marionetten an
metallenen Fäden gelenkt werden.\footnote{Vgl. Voegelin, Anamnesis, S. 46/47.}
Die beiden einzigen Kriterien, die Voegelin anführt, um den richtigen von dem
von Platon bewusst konstruierten Mythos unabhängig von Platons eigener
Mitteilung zu unterscheiden, sind: 1) Die Übereinstimmung mit Voegelins eigenen
metaphysischen Überzeugungen, denn für Voegelin "`finitisiert"' der zweite
Mythos "`im Marionettensymbol `adäquat' die Erfahrung von der Handlung im
Schnittpunkt der Determinanten, die wir respektive `Ich' und `weltjenseitiges
Sein' nennen"',\footnote{Voegelin, Anamnesis, S. 46. -- An dieser Stelle wird
  nebenbei bemerkt deutlich, wie sehr Voegelin den Erfahrungsbegriff
  strapaziert. Denn, dass das was wir als teilweise Determiniertheit unser
  Handlungen erfahren mögen, Folge der Einwirkungen eines weltjenseitigen
  Seins ist, wird als solches eben nicht erfahren, sondern stellt bestenfalls
  eine metaphysische Deutung dieser Erfahrung dar. Das `weltjenseitiges Sein'
  hier nur eine (ontologische neutrale) Benennung sein soll, wirkt kaum
  glaubwürdig, weil diese Benennung eine {\em weltjenseitige} Determinante in
  einem Zusammenhang suggeriert, in dem es viel plausibler wäre, zunächst die
  Möglichkeit {\em weltdiesseitiger} Determinanten in Erwägung zu ziehen. Der
  Trick der suggestiven Benennung von Phänomenen ist übrigens einer, den
  Voegelin der Heideggerschen Variante der Phänomenologie abgeschaut haben
  könnte.} und 2) der emotionale Eindruck, indem der richtige Mythos "`die
`Schauer' der Transzendenz, das `Numinose' im Sinne Rudolf Ottos
erregt."'\footnote{Voegelin, Anamnesis, S.  46.} Beides sind, wie man unschwer
erkennt, höchst subjektive Kriterien.

Im Zusammenhang mit der Mythensymbolik kommt Voegelin auch auf das Husserlsche
Problem der "`Konstitution der Intersubjektivität"' zu sprechen.\footnote{Vgl.
  Husserl, Cartesianische Meditationen, S. 91ff.  (§42ff.).} Dahinter verbirgt
sich die Frage, woraus hervorgeht, dass die Menschen außerhalb des eigenen
Bewusstseins eigene Wesen mit einem eigenen Bewusstsein sind. Im
philosophischen Gedankenexperiment ist es möglich, sich vorzustellen, dass die
anderen Menschen -- so wie Gestalten im Traume -- nur Hirngespinste des eigenen
Bewusstseins sind, oder dass sie "`Zombies"' sind, die körperlich und von
ihrem Verhalten her Menschen gleichen, in deren Gehirnen jedoch kein
Bewusstsein lebt. Edmund Husserl hat zu zeigen versucht, dass das Ich
bestimmte Objekte des Erfahrungsfeldes als "`alter ego"' konstituieren
kann.\footnote{Vgl.  Husserl, Cartesianische Meditationen, S. 112-116 (§
  50/51).} Voegelin hält dies für ein reines Verwirrspiel. Seiner Ansicht nach
existiert dieses philosophische Problem gar nicht, sondern es ist ein Faktum,
dass das Bewusstsein die anderen Menschen als
"`Nebenbewußtsein"'\footnote{Voegelin, Anamnesis, S. 47.} erfährt. Übrig
bleibt nur ein eher moralphilosophisches Problem, welches darin besteht,
dieses "`Erfahrungsfaktum"'\footnote{Voegelin, Anamnesis, S. 47.} so zum
Ausdruck zu bringen, dass die Mitmenschen als gleichartig anerkannt werden
können. Zur Behandlung dieses Problems greift Voegelin auf die
Mythengeschichte zurück. Es ist nicht ganz klar, warum Voegelin es für
notwendig erachtet, die Lösung im Rahmen der Mythensymbolik zu suchen.  Bei
den anderen Menschen handelt es sich schließlich auch nur um endliche Wesen,
weshalb der zuvor beschriebene Ausdruckskonflikt nicht zum Tragen kommt. Zudem
sind die anderen Menschen nicht welttranszendent, sondern nur in dem trivialen
Sinne bewusstseinstranszendent, in dem auch Tiere und tote Gegenstände
bewusstseinstranszendent sind, weil sie außerhalb und unabhängig vom
Bewusstsein ihres Betrachters eine Eigenexistenz haben.  Um solche Feinheiten
kümmert sich Voegelin jedoch nicht weiter. Der Mythengeschichte meint Voegelin
nun entnehmen zu können, dass alle bisherigen Gleichheitsideen historisch auf
die beiden Mythen der Abstammung aller Menschen von einer Mutter oder der
geistigen Prägung durch ein und denselben Vater (Gottesebenbildlichkeit)
zurückgeführt werden können.\footnote{Vgl.  Voegelin, Anamnesis, S. 47/48.}
Einen möglichen nicht-mythischen Ursprung bestimmter Gleichheitsideen zieht
Voegelin gar nicht erst in Erwägung. Ja er versteigt sich sogar zu der kühnen
Behauptung, dass die erkenntnistheoretischen Probleme der Intersubjektivität
nur innerhalb dieses mythischen Rahmens behandelbar sind.\footnote{Vgl.
  Voegelin, Anamnesis, S.  48.}  Seine Ausführungen zum mythengeschichtlichen
Ursprung der Gleichheitsidee nimmt Voegelin zum Anlass für einen kleinen
Exkurs über einige mythengeschichtliche Einzelprobleme des Konflikts zwischen
Gleichheits- und Gemeinschaftsmythen,\footnote{Vgl.  Voegelin, Anamnesis, S.
  48-50.} der schließlich in einer Klage über den Verlust des Mythos als
Ausdrucksmittel für Transzendenzerfahrungen in der Gegenwart mündet: "`Das
unvermeidliche Ergebnis"', so Voegelin, "`ist das Phänomen der `Verlorenheit'
in einer Welt, die keine Ordnungspunkte mehr im Mythos
hat."'\footnote{Voegelin, Anamnesis, S. 50.} Etwas unvermittelt leitet
Voegelin daraus eine Erklärung für den letzten Weltkrieg ab: "`Die
gesellschaftsdynamisch wichtigsten Symptome sind die `Bewegungen' unserer Zeit
... und die 'großen Kriege': die Kriege nicht nur, wo sie vielleicht ein
positives Wollen zur orgiastischen Entladung verraten, sondern auch dort, wo
sie hingenommen werden müssen, weil die Handlungen, die sie verhindern
könnten, durch die Paralyse des Ordnungswillens, der nur aktiv sein kann, wo
er seinen Sinn in der Ordnung des Gemeinschaftsmythos hat, unmöglich gemacht
werden."'\footnote{Voegelin, Anamnesis, S. 50.} Nun ist aber die Berufung auf
den "`Gemeinschaftsmythos"' ein charakteristisches Merkmal gerade der
faschistischen Ideologien. Wenn Voegelin mit der "`Paralyse des
Ordnungswillens"' auf die Appeasement-Politik Chamberlains gegenüber Hitler
anspielen wollte, dann laufen seine Ausführungen auf den Vorwurf an die
westlichen Demokratien hinaus, sich nicht ihrerseits faschistischer Methoden
bedient zu haben.

Um angemessen über Dinge und Zusammenhänge reden zu können, die mehr sind als
bloß Gegenstände endlicher, innerweltlicher Erfahrung, existiert für Voegelin
neben der Mythensymbolik noch eine philosophisch-begriffliche Alternative in
Form der Prozesstheologie. Sie beschreibt "`die Beziehungen zwischen dem
Bewusstsein, den bewusstseinstranszendenten innerweltlichen Seinsklassen und
dem welttranszendenten Seinsgrund"'\footnote{Voegelin, Anamnesis, S. 50.}.
Dieser Aufgabe ist die Prozesstheologie im Gegensatz zu anderen Ansätzen
innerhalb der Metaphysik deshalb gewachsen, "`weil in ihr zumindest der
Versuch gemacht wird, die bewusstseinstranszendente Weltordnung in einer
`verstehbaren' Sprache zu interpretieren"', nämlich in einer Sprache, die an
"`der einzig `von innen' zugänglichen Erfahrung des
Bewusstseinsprozesses"'\footnote{Voegelin, Anamnesis, S. 51.} orientiert ist.
Wie dies mit der vorherigen Behauptung zu vereinbaren ist, dass gerade dieses
Modell zum Ausdruck der Erfahrung transfiniter Wirklichkeit eher untauglich
sei,\footnote{Vgl. Voegelin, Anamnesis, S.  44/45.} enthüllt Voegelin nicht.
Wahrscheinlich muss man sich die Prozesstheologie als der Mythologie verwandt
vorstellen. Die Prozesstheologie stützt sich auf zwei "`Erfahrungskomplexe"':
Zum einen stützt sie sich auf die "`Erfahrung"', dass die Welt aus mehreren
wesensverschiedenen aber dennoch voneinander abhängigen Seinsstufen aufgebaut
ist, und zum anderen basiert sie auf der in der Meditation zugänglichen
Erfahrung des "`welttranszendenten Seinsgrundes"'.  Werden diese beiden
Erfahrungen kombiniert, so ergibt sich aus der Erfahrung der Abhängigkeit der
Seinsstufen voneinander die "`Nötigung"', sie als Phasen eines Prozesses der
Entfaltung einer identischen Substanz zu betrachten, welcher -- hier kommt die
meditative Erfahrung ins Spiel -- im welttranszendenten Seinsgrund seinen
Ursprung hat. Da die Prozesstheologie unmittelbar auf "`ontologischen
Erfahrungen"' beruht, entzieht sie sich, wie Voegelin glaubt, auch den
ansonsten naheliegenden erkenntnistheoretischen Einwänden Kantischer
Provenienz, wonach es unzulässig ist, Kategorien der innerweltlichen Erfahrung
auf das anzuwenden, was außerhalb aller möglichen Erfahrung
liegt.\footnote{Vgl. Voegelin, Anamnesis, S. 50-54.}

Auf der Grundlage dieser ontologischen Stufentheorie vollzieht Voegelin nun
den Übergang vom Primat der Bewusstseinsphilosophie zum Primat der
Ontologie.\footnote{Dieser Übergang ist übrigens durchaus typisch. In
  ähnlicher Weise ging auch Heidegger, ausgehend von Husserls
  phänomenologischer Bewusstseinsphilosophie, zur Ontologie über. Etwas vom
  Heideggerschen Pathos lässt sich bei Voegelin ebenfalls verspüren, wenn er
  vor dem möglichen Missverständnis warnt, man sei wieder in den "`friedlichen
  Gewässern der Erkenntnistheorie"' (Voegelin, Anamnesis, S. 56.) angelangt.}
Zunächst geht Voegelin jedoch zum Ausgangspunkt seiner
bewusstseinsphilosophischen Überlegungen zurück. Wenn das Bewusstsein durch die
"`Erhellungsdimensionen"' der "`Vergangenheit"' und "`Zukunft"' strukturiert
ist und Zeit als solche dem Bewusstsein nicht unmittelbar gegeben ist, so kann
bezweifelt werden, dass zwischen diesen Erhellungsdimensionen die zeitliche
Beziehung der Sukzession besteht. Das Element der Zeitlichkeit lässt sich aus
diesen Erhellungsdimensionen deshalb nicht ableiten,\footnote{Zuvor scheint
  Voegelin jedoch gerade dies versucht zu haben. (Vgl. Voegelin, Anamnesis,
  S. 44.)} weil es auch vorstellbar ist, dass Erinnerungen und Projektionen nur
Phantasien eines im Augenblickspunkt der Gegenwart verharrenden Bewusstseins
sind. Wie kann man aber einem solchen "`Solipsismus des
Augenblickes"'\footnote{Voegelin, Anamnesis, S. 55.} entgehen? Der einzige
Ausweg besteht für Voegelin in der "`Einsicht, dass das menschliche Bewusstsein
nicht eine Monade ist, welche die Existenzform des Augenblicksbildes hat,
sondern dass es menschliches Bewusstsein ist, d.h. Bewusstsein im Fundament des
Leibes und der Außenwelt."'\footnote{Voegelin, Anamnesis, S. 55.} Voegelin
spricht hier zwar von einer "`Einsicht"', aber diese Einsicht hat eher den
Charakter eines Postulates, denn nachdem Voegelin einmal beim Solipsismus des
Augenblickes angelangt ist, gibt es nichts mehr, woraus das Sein der Zeit und
der Welt mit Gewissheit oder auch nur Wahrscheinlichkeit erschlossen werden
könnte.

Unter dem Gesichtspunkt dieser ontologischen Einsicht ist auch der Begriff des
Bewusstseinsprozesses neu zu deuten. Damit wir die Bewusstseinsvorgänge als
zeitlichen Prozess auffassen können, sind "`Erfahrungen von
bewusstseinstranszendenten Prozessen"'\footnote{Voegelin, Anamnesis, S. 55.}
erforderlich. Wie dies möglich ist, wenn -- wie Voegelin zuvor kategorisch
behauptet hat -- der innere Bewusstseinsprozess seinerseits das einzige Modell
darstellt, mit dem wir bewusstseinstranszendente Prozesse verstehen
können,\footnote{Vgl. Voegelin, Anamnesis, S. 44.} bleibt etwas im Dunkeln.
Voegelin scheint von einer Art wechselseitiger Abhängigkeit zwischen Sein und
Bewusstsein auszugehen, wenn er im folgenden einerseits die physische
Bedingtheit des Bewusstseins betont, zugleich aber der idealistischen Ansicht
Raum gibt, dass das Sein der Dinge von der Beziehung auf ein Bewusstsein
abhängig ist. Es lässt sich nicht leicht feststellen, auf welche Weise
Voegelin bei dieser Argumentation einem Zirkelschluss entgehen will.
Wahrscheinlich zu Recht weist Voegelin jedenfalls darauf hin, dass daraus,
dass das Bewusstsein uns in innerer Erfahrung nur als reines Bewusstsein
gegeben ist, nicht folgt, dass es nichts anderes als reines Bewusstsein ist.
Vielmehr liefert nach Voegelins Überzeugung die innere Erfahrung nur eine
Teilansicht eines untrennbaren materiell-geistigen Seinskomplexes. In der
inneren wie der äußeren Erfahrung bekommt der Mensch jeweils nur den äußersten
Zipfel eines Seins zu fassen, das sich weit über das in der Erfahrung Gegebene
hinaus erstreckt.\footnote{Vgl.  Voegelin, Anamnesis, S. 55-57.}

Aus all diesen Überlegungen zieht Voegelin die Schlussfolgerung, dass die
Bewusstseinsphilosophie keinen geeigneten Anfangspunkt der Philosophie
darstellt. Das Bewusstsein setzt vielmehr das Sein voraus und die Frage des
Anfangs kann nun immer weiter zurückgeschoben werden bis hin zur Frage des
Anfangs der Geschichte des Kosmos. Offenbar trennt Voegelin nicht zwischen der
Frage des erkenntnistheoretischen Ausgangspunktes und der Frage der
historischen Seinsvoraussetzungen des Erkenntnisvermögens. Das klassische 
Problem eines absoluten Anfangs der Philosophie wird Voegelin noch in "`Order
and History V"' beschäftigen.\footnote{Vgl. Voegelin, Order and History V,
  S. 13f.}  Vorerst gelangt Voegelin zu dem Ergebnis, dass das Bewusstsein auf
Grund dieser nie vollständig aufklärbaren Anfangsvoraussetzungen nicht wie
äußere Gegenstände erkannt und beschrieben werden kann, sondern dass es
lediglich durch Besinnung sich selbst und sein eigenes Sein erhellen
kann.\footnote{Vgl.  Voegelin, Anamnesis, S. 57/58.}

Nachdem Voegelin solcherart die "`Kehre"' zur Ontologie vollzogen hat,
behandelt er als letztes Thema dieses Aufsatzes die wissenssoziologische
Frage, wie es zu dem Auftreten der seiner Ansicht nach verfehlten
Bewusstseinsphilosophien kommen konnte. Voegelin liefert eine solche Erklärung
an zwei Stellen seines Aufsatzes. Die erste Erklärung bezieht sich auf den
speziellen Fall der Bewusstseinsstromtheorien, die zweite Erklärung betrifft
die Bewusstseinsphilosophie im Allgemeinen.

In den Bewusstseinsstromtheorien glaubt Voegelin ein "`laizistisches Residuum
der christlichen Existenzvergewisserung in der Meditation"'\footnote{Voegelin,
  Anamnesis, S. 37.} wiederentdecken zu können. Vermutlich auf Grund von
einfühlendem Nachvollzug gelangt Voegelin zu der Auf\/fassung, dass im
Bewusstseinserlebnis des "`Strömens"' der "`Engpaß des Leibes spürbar
wird"'.\footnote{Voegelin, Anamnesis, S.  40. Außer seinen eigenen
  Assoziationen, für die sich in den zeitphilosophischen Texten, auf die
  Voegelin sich bezieht, durchaus einzelne Hinweise finden lassen, führt
  Voegelin noch einen eher aus dem Zusammenhang gegriffenen Gedanken von
  William James (Vgl. William James: Essays in Radikal Empirischem, Anbringe,
  Kassakurses / London, England 1976, S. 19.) und etwas später (Anamnesis,
  S. 42) Bergsons Behandlung der eleatischen Paradoxe an. Bergson ist jedoch
  ein schlechter Gewährsmann, denn seine Behandlung der eleatischen Paradoxe
  scheint auf einer Verwechselung der physikalischen Begriffe von Ort und
  Bewegung zu beruhen: Dass ein Körper sich zu einem bestimmten Zeitpunkt an
  einem ganz bestimmten Punkt im Raum befindet schließt nämlich nicht aus,
  dass er in diesem Punkt einen Bewegungszustand hat.  (Vgl. Henri Bergson:
  Materie und Gedächtnis, Hamburg 1991, S. 184-190.)}  Voegelin schließt
daraus, dass die Bewusstseinsstromtheorien ebenso wie die christliche
Meditation auf eine Form der Transzendenz zielen.  Während die Meditierenden
in der christlichen Meditation jedoch Welttranszendenz suchen, zielen die
Bewusstseinsstromtheorien lediglich auf die bloße Bewusstseinstranszendenz in
Richtung der Leibsphäre hin.\footnote{Vgl. Voegelin, Anamnesis, S. 41-42.}

In einem etwas allgemeineren Rahmen stellt das Auftreten der
Bewusstseinsphilosophie für Voegelin die Reaktion auf eine Krise der Symbole
dar, wie sie alle Kulturen von Zeit zu Zeit heimsucht. Die Symbole, mit denen
die Menschen ihre Transzendenzerfahrungen ausdrücken, tendieren dazu, im Laufe
der Zeit schal und inhaltsleer zu werden. Die daraus resultierende Kulturkrise
kann nur durch die Beseitigung der alten und die Bildung neuer Symbole zum
Ausdruck der Transzendenzerfahrungen behoben werden. Platon war dies als
Antwort auf die Krisis der hellenischen Kultur im 5.Jahrhundert vor Christus
in vorbildlicher Weise gelungen. Die neuzeitliche Philosophie, die mit
Descartes ihren Anfang nimmt, stand nach Voegelins Ansicht vor einer ähnlichen
Aufgabe, doch hat sie ihr Ziel verfehlt, indem sie zwar mit der Tradition
gründlich aufräumte aber zugleich auch die Transzendenzerfahrungen aus dem
Themenkanon der Philosophie ausschloss.\footnote{Vgl. Voegelin, Anamnesis,
  S. 58-60.}

\subsection{Kritik von Voegelins Theorie des Bewusstseins}

Sind Voegelins Überlegungen "`Zur Theorie des Bewußtseins"' überzeugend? Geben
sie die Beziehungen zwischen Sein und Bewusstsein richtig wieder und dürfen
Voegelins Argumente als stichhaltig angesehen werden? Da es kaum möglich ist,
auf alle Einzelheiten der sehr vielfältigen Ausführungen Voegelins
einzugehen, sollen zur genaueren kritischen Untersuchung nur einige Punkte
herausgegriffen werden, die für Voegelins Argumentation von wesentlicher
Bedeutung sind.

Innerhalb von Voegelins eigener Darstellung der Bewusstseinsstruktur findet
sich an zentraler Stelle das Argument, dass sich das Bewusstsein bei dem
Versuch, transfinite Prozesse deskriptiv zu beschreiben, auf Grund seiner
eigenen Endlichkeit unvermeidlich in Widersprüche verwickelt. Das Argument
spielt deshalb eine wesentliche Rolle, weil diese Widersprüche es erforderlich
werden lassen, zur Deutung bestimmter Wirklichkeitsbereiche auf die zwar
subtilen und seelisch sensiblen aber an Klarheit und Objektivität hinter einer
deskriptiven Beschreibung zurückstehenden Instrumente der Mythensymbolik und
der Prozesstheologie zurückzugreifen. Unglücklicherweise steckt gerade in
dieser Passage von Voegelins Darstellung eine Reihe von Fragwürdigkeiten, die
sich nicht ohne weiteres auflösen lassen.

Zunächst einmal ist es zweifelhaft, ob, wie Voegelin es behauptet, der
Bewusstseinsprozess das einzige erfahrene Modell eines Prozesses darstellt.
Prozesse oder, mit anderen Worten, zeitlich ablaufende Vorgänge im weitesten
Sinne erleben wir tagtäglich in der äußeren Erfahrung, z.B. wenn wir ein
fahrendes Auto beobachten. Die äußere Erfahrung eignet sich dabei mindestens
ebenso gut, wenn nicht besser, als die innere Erfahrung, um den Begriff eines
Prozesses zu bilden.  Abgesehen davon ist es aber auch überhaupt nicht
erforderlich, zur Bildung eines Begriffes diesen von irgendeiner Erfahrung
abzuziehen.  Ebenso wie ein großer Teil unseres Wissens nicht aus
unmittelbarer Erfahrung stammt, gibt es auch viele Begriffe, die rein abstrakt
sind.  Alle mathematischen Begriffe gehören zu dieser Klasse. Insbesondere ist
es ohne Probleme möglich, widerspruchsfreie Begriffe von Unendlichkeit zu
bilden. Die Mengenlehre verfügt über mehrere solcher Begriffe.  Freilich
decken diese Begriffe nicht alle Wortbedeutungen von "`unendlich"' ab, und die
"`unendliche Sehnsucht"', von der ein romantischer Dichter schwärmen mag, wird
von der Mengenlehre nicht erfasst, aber es ist nun nicht mehr einleuchtend,
weshalb die Finitheit des Bewusstseinsprozesses bei der Beschreibung von
unendlichen Prozessen zu den von Voegelin unterstellten
Ausdruckskonflikten führen muss. Außerdem scheint sich Voegelin auch
hinsichtlich der Bedeutung der Kantischen Antinomien geirrt zu haben.  Kants
Antinomien beruhen letztlich auf unterschiedlichen Voraussetzungen, die den
einander gegenübergestellten Beweisen und Gegenbeweisen zu Grunde liegen.  Um
Antinomien könnte es sich nur noch dann handeln, wenn diese Voraussetzungen
gleichermaßen notwendig wären.  Aber dies -- und hierin irrt Kant und mit ihm
viele seiner Interpreten und, wie es scheint, leider auch Voegelin -- ist nicht
der Fall.\footnote{Vgl. Immanuel Kant: Kritik der reinen Vernunft, Hamburg
  1976, S. 454-469. Für die Kant-Apologetik stellvertretend: Peter Baumanns:
  Kants Philosophie der Erkenntnis. Durchgehender Kommentar zu den
  Hauptkapiteln der "`Kritik der reinen Vernunft"', Würzburg 1997, S. 742ff.--
  Dass Kant irrt, kann man sich leicht überlegen, wenn man bei den Antinomien
  genau darauf achtet, von welchen expliziten und impliziten Voraussetzungen
  Kant bei seinen Beweisen jeweils ausgeht.  Es würde zu weit führen, dies
  hier im einzelnen auszuführen.}  Was Voegelin schließlich mit den Paradoxen
der Mengenlehre meint, geht aus dem Text leider nicht hervor.  Möglicherweise
meint Voegelin die Russellsche Antinomie, die in der naiven Mengenlehre
auftritt.  Aber erstens handelt es sich nicht um ein Paradox der
Unendlichkeit, und zweitens lässt sie sich mühelos durch eine axiomatische
Fassung der Mengenlehre beseitigen.\footnote{Vgl. Jürgen Schmidt: Mengenlehre
  (Einführung in die axiomatische Mengenlehre). I.  Grundbegriffe, Mannheim
  1966, S. 22-24.}

Voegelins Argument ließe sich im Grundsätzlichen immer noch dann
rechtfertigen, wenn es gelänge zu zeigen, dass bestimmte Wirklichkeitsbereiche
aus anderen Gründen als dem ihrer "`Infinitheit"' einer deskriptiven
Beschreibung unzugänglich sind. Dann müsste die Diskussion um die Fragen
geführt werden, ob es diese Wirklichkeitsbereiche tatsächlich gibt, und wenn
es sie gibt, ob Mythensymbolik oder Prozesstheologie sie erfassen können. So
wie Voegelin argumentiert, bleibt die Notwendigkeit des Gebrauchs dieser
Symbolformen jedoch unbegründet.

Einen weiteren wichtigen Abschnitt, der zwar weniger für Voegelins folgende
Argumentation von Bedeutung ist, aber dafür seine grundsätzliche
philosophische Einstellung widerspiegelt, bildet Voegelins Versuch, das
Problem der Anerkennung der Mitmenschen als gleichartige und gleichwertige
Wesen (in Voegelins Terminologie: das Problem der "`Erfahrung vom
Nebenmenschen"') mit Hilfe der Mythengeschichte zu lösen. Voegelins
Argumentation enthält eine Reihe von Schwachpunkten. Die erste Schwierigkeit
bildet der Begriff des "`Erfahrungsfaktums"'. Obwohl wir, wie auch Voegelin
einräumt, von unseren Mitmenschen keine innere Erfahrung haben, sollen wir
dennoch durch ein Erfahrungsfaktum unmittelbar davon in Kenntnis gesetzt sein,
dass sie ein Innenleben haben. Nun mögen wir zwar intuitiv den Eindruck haben,
dass in unseren Mitmenschen auch ein denkendes und fühlendes Bewusstsein
steckt, aber die Berufung auf die Intuition ist auch dann noch ein schwaches
Argument in der Erkenntnistheorie, wenn sie hochtönend als
"`Fundamentalcharakter"' der "`Transzendenzfähigkeit"'\footnote{Voegelin,
  Anamnesis, S. 47.} des Bewusstseins etikettiert wird. Der Einwand gegen
Husserl, dass sich das Du nicht im Ich konstituiert, ist dagegen durchaus
angebracht, denn das Bewusstsein kann unmöglich durch Konstitution etwas
hervorbringen, was außerhalb seiner selbst existiert. Als geradezu abwegig
erscheint allerdings Voegelins Behauptung, dass dieses erkenntnistheoretische
Problem nur im Rahmen der altertümlichen Gleichheitsmythen behandelt werden
kann. Weder für die Formulierung dieses erkenntnistheoretischen Problems noch
erst recht zu seiner Lösung ist der Rückgriff auf die Mythengeschichte
notwendig oder auch nur hilfreich.

Der zweite Schwachpunkt von Voegelins Argumentation liegt in seiner Annahme,
dass die moralische Gleichheit aller Menschen nur im Rückgriff auf alte Mythen
artikuliert werden kann. Bei der Behandlung dieser moralphilosophischen
Problematik müssen drei unterschiedliche Ebenen klar voneinander getrennt
werden: Die Ebene der Begründung von Werten, die Ebene der Artikulation bzw.
Formulierung der Werte und die Ebene der Vermittlung und Verbreitung der
Werte. Für die Begründung des Gleichheitswertes kann die Mythengeschichte
offensichtlich nicht herangezogen werden. Wenn die moralische Gleichheit der
Menschen nämlich im Sinne einer moralischen Intuition auf einem
"`Erfahrungsfaktum"' beruht,\footnote{Im Bereich der Ethik ist anders als in
  der Erkenntnistheorie die Berufung auf die Intuition unter Umständen
  legitim. Es stellt sich dann nur die Frage, inwieweit intuitiv begründete
  Werte intersubjektive Verbindlichkeit beanspruchen dürfen.} dann besteht die
einzig ehrliche Weise, diesen Wert zu begründen, darin, auf diese Intuition
bzw.  diese Erfahrung hinzuweisen, und gegebenenfalls die Begleitumstände
 zu beschreiben, unter denen sie zustande kommt oder in besonders
deutlicher Weise hervortritt.

Die Formulierung des Gleichheitswertes ist mit und ohne Rückgriff auf Mythen
möglich. Ohne Rückgriff auf die Mythologie kann sie beispielsweise durch die
Worte erfolgen: "`Alle Menschen sind gleich"'.  Bereits mit diesen schlichten
Worten ist der Inhalt der Gleichheitsidee vollständig und ohne jede Mythologie
ausgedrückt. Eine Artikulation unter Rückgriff auf die Mythologie könnte durch
Erzählung der Geschichte von Adam und Eva erfolgen. Allerdings bliebe, wegen
der grundsätzlichen Vieldeutigkeit des Mythos, die Gleichheitsbotschaft dann
möglicherweise undeutlich. Dass man mit dem Hinweis auf den Mythos von Adam
und Eva ebensogut die Ungleichheit begründen kann, führt uns z.B. Sir Robert
Filmer vor Augen, der damit auf eine zu seiner Zeit durchaus übliche Weise das
Gottesgnadentum der Könige rechtfertigte.\footnote{Vgl. Sir Robert Filmer:
  Patriarcha, or the Natural Power of Kings, England 1680.} Analoges gilt für
das Problem der Vermittlung des Gleichheitswertes. Zu der Zeit, als Voegelin
den Aufsatz "`Zur Theorie des Bewußtseins"' niederschrieb, war das
Mythologische einigermaßen in Mode. Nicht zuletzt durch die faschistischen
Bewegungen wurde die Berufung auf den Mythos weidlich missbraucht, weshalb es
aufgeklärten Autoren wie Thomas Mann notwendig erscheinen mochte, dass man den
totalitären Mythen aufgeklärte Mythen entgegen stellen müsse, um das
Verständnis für die Urwahrheiten des menschlichen Zusammenlebens
wiederzuerwecken.\footnote{Den zeitgeschichtlichen Bezug seiner Josephs-Romane
  hat Thomas Mann in einer späteren Selbstdeutung auf die Formel gebracht,
  dass der "`Mythos .. in diesem Buch dem Fascismus aus den Händen genommen"'
  wurde.  (Thomas Mann: Joseph und Seine Brüder (Vortrag in der Library of
  Congress am 17.11.1942), in: Thomas Mann: Essays. Band 5: Deutschland und
  die Deutschen 1938-1945.  (Hrsg.  v. Hermann Kurzke und Stephan Stachorski),
  Frankfurt am Main 1996, S. 185-200 (S. 189).) -- Neben den Josephs-Romanen
  wäre in diesem Zusammenhang auch Thomas Manns biblische Erzählung "`Das
  Gesetz"' zu nennen.} Voegelin klingt freilich wesentlich weniger aufgeklärt,
wenn er wortwörtlich schreibt, dass der Ordnungswille "`nur aktiv sein kann,
wo er seinen Sinn in der Ordnung des Gemeinschaftsmythos
hat"'.\footnote{Voegelin, Anamnesis, S. 50.} Hier scheint eher noch ein Rest
von dem faschistischen Gedankengut durchzuklingen, das Voegelin in seiner
autoritären Phase in den 30er Jahren absorbiert hatte.\footnote{Vgl. Voegelin,
  Autoritärer Staat, a.a.O.} Aus heutiger Sicht muss ein Ordnungswille, der im
"`Gemeinschaftsmythos"' wurzelt, in höchstem Maße suspekt erscheinen.

Wie man sieht, ist also der Rückgriff auf die Mythengeschichte für die
Begründung und Artikulation des Gleichheitsideals in Wirklichkeit keineswegs
erforderlich und höchstens mit Einschränkungen nützlich. Voegelins Argument
dafür, das er im Gegenteil unerlässlich sei, ist historischer Art und besteht
-- wie zuvor ausgeführt -- in der Behauptung, dass alle Gleichheitsideen
Derivate jener beiden Urmythen der Abstammung von einer Mutter oder der
Prägung durch einen Vater sind.\footnote{Dass sich -- so das andere Argument,
  das aus Voegelins Text extrahiert werden kann -- aus dem von Voegelin
  behaupteten Ausdruckskonflikt bei der Artikulation unendlicher Prozesse für
  diesen Fall kein notwendiger Grund für den Gebrauch der Mythensymbolik
  ableiten lässt, wurde bereits erwähnt.}  Inwieweit dies historisch richtig
und zwingend ist, sei dahingestellt. Für Voegelin war diese Vorstellung
wohlmöglich deswegen attraktiv, weil sie ihm erlaubte, eine Analogie zum
Leib-Geist-Dualismus herzustellen, indem der eine Mythos ein leiblicher und
der andere ein geistiger ist. Aber selbst wenn Voegelins historische These
richtig sein sollte, so folgt daraus nicht, dass die Gleichheitsidee niemals
etwas anderes sein kann als ein Derivat dieser Urmythen. Insbesondere kommt es
bei der moralphilosophischen Diskussion der Gleichheitsidee nur auf den Inhalt
und die Begründung dieser Idee an. Diese sind aber von der
Entstehungsgeschichte unabhängig, so dass die Diskussion darüber unbekümmert
um die Geschichte der Mythologie geführt werden kann.

Nicht nur Voegelins Ausführungen zur Mythensymbolik sondern auch seine
Interpretation der Prozesstheologie wirft einige Fragen auf. Vor allem
Voegelins Annahme, dass die Prozesstheologie einen Bereich von "`ontologischen
Erfahrungen"'\footnote{Voegelin, Anamnesis, S. 54.}  auslegt, bedarf der
Klärung. Denn der Begriff der Erfahrung wird mit dieser Annahme stark
überstrapaziert. Das Wissen um die Stufen des Seins ist deskriptives Wissen,
das sich bestenfalls auf die Erfahrung stützt, das aber über die unmittelbare
Erfahrung weit hinaus geht. Auch dass die meditative Erfahrung ein Wissen vom
Seinsgrund vermittelt, muss als höchst zweifelhaft angesehen werden, sofern
"`Seinsgrund"' ein ontologischer Begriff ist und nicht nur ein Name für die
meditative Erfahrung selbst, wie Voegelins spätere Theorie der "`Indizes"' des
Bewusstseins dies nahelegt.\footnote{Siehe Kapitel \ref{IndizesTheorie}} Im
letzteren Fall bestünde dann allerdings auch keine "`Nötigung"' mehr, sondern
es wäre im Gegenteil sogar ganz und gar unmöglich, den ontologischen
Seinsprozess in einem "`Seinsgrund"' außerhab des Bewusstseins entspringen zu
lassen. Wird jedoch auf diese Weise in Zweifel gezogen, dass es eine
privilegierte Klasse ontologischer Erfahrungen gibt, dann bleibt auch die
Prozesstheologie, die Voegelin skizziert, als eine bestimmte ontologische
Theorie in vollem Umfang durch die Erkenntniskritik angreifbar.

Abgesehen von diesen Schwierigkeiten bleibt auch der Sinn und Zweck der
Prozesstheologie im Unklaren. Voegelin zufolge geht die Prozesstheologie aus
von der Frage: "` `Warum ist etwas, warum ist nicht Nichts?'
"'\footnote{Voegelin, Anamnesis, S.51. -- Vgl. Friedrich Schelling: Philosophie
  der Offenbarung, Zwölfte Vorlesung, in: Frank-Peter Hansen (Hrsg.):
  Philosophie von Platon bis Nietzsche, CD-ROM, Berlin 1998, S.37855 / S.72
  (Konkordanz: Friedrich Wilhelm Joseph von Schelling: Werke. Auswahl in drei
  Bänden. Herausgegeben und eingeleitet von Otto Weiß. Leipzig 1907.  Band 3,
  S. 781).} Allerdings unternimmt die Prozesstheologie dann keinen ernsthaften
Versuch, diese Frage zu beantworten. Eher scheint sie darauf hinauszulaufen,
das Gefühl des Staunens bzw. der Verblüffung, das in jener Frage liegt, zu
artikulieren. Wenn sich aber die überwiegende Mehrzahl der Menschen nicht mit
dem erkenntnistheoretischen Befund der Unbeantwortbarkeit dieser Frage
zufrieden geben will, wie Voegelin -- nicht unplausibel --
vermutet,\footnote{Vgl.  Voegelin, Anamnesis, S. 51.} warum sollte sie sich
dann mit der Prozesstheologie, die diese Frage auch nicht beantworten kann,
abspeisen lassen?

Hinsichtlich der Einbettung der Bewusstseinsphilosophie in die Ontologie,
wie sie Voegelin im letzten Abschnitt seines Aufsatzes vollzieht, sind
vor allem die zwei Thesen zu prüfen, dass die ontologische Problematik
die Voraussetzung der Erkenntnistheorie bzw. Bewusstseinsphilosophie
bildet, und dass der Mensch sich auf sein Bewusstsein und sein Wesen nur
orientierend besinnen aber es niemals zu einem Gegenstand äußerer
Beschreibung machen kann.

Die erste dieser Thesen ist auch für Voegelins wissenssoziologische
Erklärungen von Bedeutung, denn nur, wenn sie bejaht wird, kann der "`Versuch
einer `radikalen' Bewusstseinsphilosophie
aufklärungsbedürftig"'\footnote{Voegelin, Anamnesis, S. 58.} erscheinen. In
einer bestimmten Hinsicht kann die Triftigkeit von Voegelins Einwand gegen die
reine Bewusstseinsphilosophie kaum bestritten werden. Die Erklärung der
meisten Bewusstseinsvorgänge dürfte nur schwer möglich sein, ohne auf die
Tatsache zurückzugreifen, dass es sich um das Bewusstsein eines Menschen
handelt, der in einer materiellen Außenwelt lebt. So ist etwa das
gelegentliche Auftreten des Bewusstseinsphänomens "`Hunger"' nur verständlich,
wenn man die Selbstverständlichkeit berücksichtigt, dass das Bewusstsein, in
dem es auftritt, das Bewusstsein eines Lebewesens ist, welches von Zeit zu Zeit
der Speise und des Tranks bedarf. Auch darf wohl behauptet werden, dass
ontologische Fragen insgesamt relevanter sind als nur rein
bewusstseinsphilosophische Probleme, denn der Erhalt und die Wohlfahrt unseres
Lebens hängt von dem ab, was in der Welt geschieht und nicht von dem, was sich
davon im Bewusstsein spiegelt. Insofern spricht für Voegelins kritische
Einstellung gegenüber der reinen Bewusstseinsphilosophie auch eine starke
intuitive Plausibilität.

Aber ist damit auch die Möglichkeit einer reinen, d.h. ausschließlich
introspektiven Beschreibung der Bewusstseinsvorgänge ausgeschlossen? Und muss
die Erkenntnistheorie nun doch, trotz der drohenden Gefahr von
Begründungszirkeln, ein Wissen um die Außenwelt voraussetzen? In dieser
Hinsicht scheint Voegelins These unzureichend begründet zu sein. Auch wenn
viele Bewusstseinsvorgänge losgelöst von der Außenwelt nur schwer zu deuten
sein dürften, so bleibt doch die Möglichkeit, das Bewusstsein als reines
Bewusstsein introspektiv zu beschreiben, immer noch bestehen. Sollte zur
Beschreibung des reinen Bewusstseins als Prozess eine Form von Zeitlichkeit
vorausgesetzt werden müssen, die nicht introspektiv erfahrbar ist, so genügt
es, allein die Existenz dieser Form von Zeitlichkeit zu postulieren, ohne
zugleich auch den Leib und die Geschichte vorauszusetzen. Eine solche
Beschreibung des reinen Bewusstseins würde auch dann keine weiteren
ontologischen Hypothesen voraussetzen, wenn es faktisch substanzidentisch mit
seinem leiblichen Fundament (Gehirn) sein sollte. Dabei ist übrigens die
Hypothese der Substanzidentität zum Verständnis "`der Fundierung von
Bewußtsein in Leib und Materie"'\footnote{Voegelin, Anamnesis, S. 55.} nicht
einmal zwingend erforderlich, denn diese Fundierung könnte auch durch die
Hypothese der kausalen Verursachung von Bewusstseinsphänomenen durch mit
diesen nicht substanzidentische physische Phänomene erklärt werden. Die
Erkenntnistheorie schließlich setzt schon deshalb nicht die Ontologie voraus,
weil die erkenntnistheoretischen Probleme auf einer anderen Ebene, auf der
Ebene der Gültigkeit, liegen als die ontologischen Probleme. Zwar ist das
Faktum, dass es Erkenntnis und Wahrheit gibt, davon abhängig, dass es
Lebewesen gibt, die erkennen können, aber die Gültigkeit von Erkenntnis und
die Antwort auf die Frage, worin Wahrheit besteht und nach welchen Kriterien
sie festgestellt werden kann, hängen nicht von diesen ontischen
Voraussetzungen ab. Am leichtesten lässt sich dies an einem Beispiel
verdeutlichen: Damit der Satz "`Zwei mal zwei ist vier."' existiert, muss es
wenigstens ein intelligentes Wesen geben, welches ihn denkt oder
äußert,\footnote{Manche Philosophen glauben auch, dass Sätze wie dieser in
  einem platonischen Ideenhimmel existieren. Die Sätze würden dann auch
  existieren, wenn es keine Menschen oder nicht einmal eine Welt gäbe.}  und
damit dieses Wesen existiert, müssen weitere ontische Voraussetzungen erfüllt
sein. Die Wahrheit dieses Satzes hängt jedoch von keiner dieser
Voraussetzungen ab.\footnote{Man könnte nun vermuten, dass nicht die Wahrheit
  aber die Bedeutung eines Satzes von ontischen Voraussetzungen, z.B. von der
  Bedeutungsgeschichte der in ihm verwendeten Wörter abhängt. Aber die Art und
  Weise, wie die Bedeutung eines Wortes entstanden ist, stellt keine
  Bedeutungsvoraussetzung des Wortes dar, sondern lediglich eine kausale
  Voraussetzung der Entstehung der Bedeutung des Wortes.}

Wie verhält es sich mit Voegelins zweiter These, dass das Bewusstsein sich
nicht selbst wie einen Gegenstand betrachten kann? Voegelin führt als Grund
für diese These an, dass die Bewusstseinsphilosophie "`ein spätes Ereignis in
der Biographie des Philosophen ist"', welches wiederum ein Ereignis in der
Geschichte seiner Gemeinschaft, in der Geschichte der Menschheit und in der
Geschichte des Kosmos ist. Aber diese Begründung ist wenig stichhaltig und
dürfte eher einer holistischen Überzeugung Voegelins geschuldet sein als auf
rationaler Überlegung beruhen. Jeder noch so profane Gegenstand hat auch seine
Vorgeschichte im Kosmos, und das Wissen über ihn hat eine Vorgeschichte in der
Geschichte des menschlichen Wissens. Dennoch wird niemand bestreiten, dass es
Gegenstände gibt, die vollständig erkannt werden können. Wenn irgendetwas nur
historisch verstanden werden kann, so muss es dafür speziellere Gründe geben.
Und außer dem Kosmos selbst gibt es vermutlich nichts, dessen Erkenntnis die
Geschichte des gesamten Kosmos voraussetzt.

Problematisch ist auch jener Teil von Voegelins Aufsatz, in welchem er das
Auftreten der Bewusstseinsphilosophie in der Neuzeit historisch zu deuten
versucht.\footnote{Vgl. Voegelin, Anamnesis, S. 58ff.} Wie bereits dargelegt,
sind Erkenntnistheorie und mit gewissen Einschränkungen auch die
Bewusstseinsphilosophie legitime Einzeldisziplinen der Philosophie, die nicht
unbedingt als Teilgebiet einer allgemeinen Ontologie behandelt werden müssen.
Ihr Auftreten ist daher bereits wesentlich weniger "`aufklärungsbedürftig"'
als Voegelin meint. Abgesehen davon bleibt es schleierhaft, woher Voegelin
überhaupt die historische Aufgabe der Bewusstseinsphilosophie nimmt, eine neue
Symbolik für religiöse Transzendenzerfahrungen zu suchen. Sofern Voegelin
nicht wie Husserl die Existenz eines historischen Telos voraussetzen will, das
jeden Philosophen verpflichtet, sich mit diesem Problem zu beschäftigen, kann
er den Philosophen kein Versäumnis vorwerfen, wenn sie sich für andere Fragen
als die der Symbolisierung von Transzendenzerfahrungen interessieren. Freilich
hat Voegelin das Recht, den Ausschluss der Besinnung auf
Transzendenzerfahrungen aus dem Themenkanon der Philosophie zu tadeln, wenn er
selbst der Ansicht ist, dass dieses Thema in der Philosophie einen Platz haben
sollte. Allerdings ist zu berücksichtigen, dass andere Philosophen dies
explizit ablehnen, und dass sie dazu mindestens ein ebensogutes Recht haben.
Abgesehen davon existierten auch in der Neuzeit mit Religion und Theologie
durchgängig geistige Disziplinen, die sich mit der Transzendenz beschäftigten
und immer noch beschäftigen.  Insofern ist es ein wenig voreilig, eine
historische Krise der Symbole zu suggerieren.  Schließlich ist anzumerken,
dass gerade die Husserlsche Phänomenologie, welche sich noch am ehesten
angeschickt hat, die Bewusstseinsphilosophie zum allumfassenden
Universalparadigma auszuweiten, sich gegenüber dem religiösen Denken als sehr
aufgeschlossen erwiesen hat.

Alles in allem leidet Voegelins Aufsatz "`Zur Theorie des Bewußtseins"' an
auffällig vielen argumentativen Schwächen. An haltbaren Resultaten ist der
Text außerordentlich arm. Dies mag damit zusammenhängen, dass er eher den
Charakter einer persönlichen Besinnung als den einer philosophischen
Argumentation hat. Von Voegelin war das durchaus intendiert, denn die
philosophische Tätigkeit bestand für ihn vor allem in der meditativen
philosophischen Besinnung. Nur stellt sich dann dem Leser irgendwann die
Frage, warum er sich für die persönlichen Besinnungen des Herrn Voegelin
eigentlich interessieren sollte, eine Frage, die sich noch mehr aufdrängt,
wenn er dann zu den im Folgenden zu besprechenden "`anamnetischen
Experimenten"' Voegelins weiterblättert.

\section{Die "`anamnetischen Experimente"' Voegelins}

Den ersten Teil seines Werkes "`Anamnesis"' schließt Voegelin mit der
Wiedergabe einiger Kindheitserinnerungen ab. Es handelt sich um Schilderungen
intellektueller Erlebnisse seiner Kindheit, in welchen zum erstenmal, in einer
freilich dem zarten Alter entsprechenden Weise, die Fragen auftauchten, welche
Voegelin sich später als Bewusstseinsphilosoph erneut stellte. Da diese
Erinnerungen teilweise erst durch den Versuch wieder zu Tage traten, sich
Rechenschaft über die ersten Anfänge jener Bewusstseinserlebnisse und
Stimmungen abzulegen, die Voegelin später als erwachsener Philosoph
untersuchte, spricht er von "`anamnetischen
Experimente[n]"',\footnote{Voegelin, Anamnesis, S. 61.}  deren Resultate diese
Erinnerungen sind. Unter den manchmal mit Augenzwinkern erzählten Episoden,
die Voegelin aus seiner Kindheit mitteilt, finden sich Stücke wie jenes von
dem Karnevalszug, der in dem Kind eine dunkle Angst erregte, weil sich der
Zug, da ihn einzelne Jecken immer wieder verließen, um in den Seitenstraßen zu
verschwinden, am Ende aufzulösen schien.\footnote{Vgl.  Voegelin, Anamnesis,
  S. 64 (Nr.  3).}  In einer anderen Episode berichtet Eric Voegelin, der
einen Teil seiner Kindheit in Königswinter bei Bonn nahe dem Siebengebirge
verbrachte, von den drei Breibergen, die man vom Ölberg aus sehen kann. Dem
Märchen zufolge muss man sich durch diese Breiberge hindurchfressen, um in das
dahinter liegende Schlaraffenland zu gelangen. Die Angst des Kindes, dabei im
Brei stecken zu bleiben, trübte sehr die Hoffnung auf das
Schlaraffenland.\footnote{Vgl. Voegelin, Anamnesis, S. 65-66 (Nr. 5).}  Andere
Episoden teilen ähnliche Gefühle der Zweifelhaftigkeit des vollkommenen
Glückes mit.\footnote{Vgl. Voegelin, Anamnesis, S. 66 (Nr. 6), S. 73-64
  (Nr.16).}  Voegelin selbst gibt keine Erläuterungen zu den erzählten
Episoden, die ihre Bedeutung für sein späteres Denken erklären
könnten.\footnote{Einige vorsichtige Deutungsversuche unternimmt Barry Cooper.
  Vgl. Barry Cooper: Eric Voegelin and the Foundations of Modern Political
  Science, Columbia and London 1999, S. 204-207.} Womöglich betrachtete
Voegelin diese frühen Erfahrungen als Vorboten der späteren Skepsis des
Politikwissenschaftlers gegenüber der Utopie. In einer weiteren Episode
schildert Voegelin den starken emotionalen Eindruck, den das Märchen vom
Kaiser und der Nachtigall, die durch ihren Gesang den Tod dazu erweicht vom
Kaiser abzulassen, in ihm hinterlassen hat. Später fand Voegelin diese
Stimmung aus seiner Kindheit zwar nicht mehr im Märchen wohl aber beim Anhören
mancher Musikstücke wieder: "`Die Bedeutung, die ein Musikwerk für mich hat,
ist bestimmt durch den Grad, in dem es diese süße Beklemmung zwischen Tod und
Leben wieder erregt."'\footnote{Voegelin, Anamnesis, S. 75 (Nr. 18).}
 
Solcher und ähnlicher Art sind die von Voegelin wiedergegebenen
Kindheitserinnerungen. Doch was soll mit ihrer Mitteilung bewiesen werden? In
den einleitenden Vorbemerkungen zu seinen "`anamnetischen Experimenten"' führt
Voegelin die Thesen aus seinem vorangehenden Aufsatz noch einmal auf: Das
Bewusstsein ist kein Strom, sondern es verfügt über vielfältige
Transzendenzfähigkeiten.  Die Besinnung über das Bewusstsein greift
Bewusstseinserlebnisse des Philosophen auf, die bereits sehr viel früher in
seinem Leben erstmals zu Tage getreten sind. Weiterhin sieht Voegelin in den
frühen Bewusstseinserlebnissen "`Erfahrungseinbrüche"' und
"`Erregungsquellen"', "`aus denen es zu weiterer philosophischer Besinnung
treibt"'. Die Intensität und Emotionalität\footnote{Voegelin spricht wörtlich
  von der "`Natur der Erfahrungseinbrüche"', der "`Art der Erregungen"' und
  der "` `Stimmung' "' des Bewusstseins. (Vgl. Voegelin, Anamnesis, S. 61.)}
solcher "`Erfahrungseinbrüche"' bilden für Voegelin den Maßstab der
Radikalität, d.i.  der Breite und Tiefe einer philosophischen Besinnung.

Sind aber solche Erfahrungen, wie Voegelin sie erzählt, für die Behandlung
bewusstseinsphilosophischer Probleme überhaupt relevant? Sicherlich ist nicht
für jedes philosophische Problem der Rückgang auf die Erfahrung seines ersten
Auftretens erforderlich. Für die Lösung des zenonschen Problems etwa, wie aus
unendlich vielen Einzelschritten ein kontinuierlicher Übergang entstehen
kann,\footnote{Vgl. Voegelin, Anamnesis, S. 71/72 (Nr. 14: Der Laib Brot).}
spielt es sicherlich keine Rolle, wann und wie es zum erstenmal dem
Philosophen, der es behandelt, begegnet ist.  Auch wenn er es erst im
Erwachsenenalter in einem Buch gelesen hat, hindert ihn nichts daran, dieses
Problem angemessen zu erörtern. In welchem Falle ist es dann aber notwendig,
auf die Problemerfahrungen zurückzugehen, und in welchem Fall nicht?
Offensichtlich ist dies dann nicht erforderlich, wenn die Erfahrung nur den
Anlass gibt, über ein philosophisches Problem nachzudenken. Außer wenn es um
Fragen der Selbsterkenntnis geht, hat das {\em Erlebnis} eines philosophischen
Problems jedoch keine andere philosophische Bedeutung als die, ein Anlass des
Nachdenkens zu sein. Dies gilt auch für die Probleme der
Bewusstseinsphilosophie. Für die Lösung beispielsweise des Problems der
Konstitution von Gegenständen ist zwar möglicherweise der Bezug auf innere
Erfahrungen, nicht aber der Rückgang auf die ersten Erfahrungen dieser Art
oder auf das erstmalige Erlebnis, dass es sich hier um ein Problem handelt,
erforderlich. Im übrigen stünde eine Philosophie, die sich nur auf die eigenen
inneren Erlebnisse des Philosophen stützt, vor dem Problem, dass sie bloß
subjektiv gültige Ergebnisse liefern könnte.

Es scheint also, dass Voegelin die Bedeutung von "`Erfahrungseinbrüchen"' für
die Philosophie erheblich überschätzt hat. Deshalb ist es auch ein
zweifelhaftes Unterfangen, die Philosophie an den vermeintlich zu Grunde
liegenden inneren Erlebnissen messen zu wollen, zumal dies die Gefahr birgt,
dass dann in letzter Konsequenz die Heftigkeit und Leidenschaftlichkeit
des Denkens mehr wiegen als die Qualität der Argumente. Eingeräumt werden muss
allerdings, dass die Tiefe einer philosophischen Untersuchung wahrscheinlich
auch durch die Intensität der zur Philosophie motivierenden Erfahrungen
mitbestimmt ist, nur ist die Intensität der Motivation nicht der
Bewertungsmaßstab für philosophische Werke. Unter den zur Philosophie
motivierenden Erlebnissen dürften für die meisten Menschen dabei wohl die
"`Erfahrungseinbrüche"' der Jugend eine größere Rolle spielen als die der
Kindheit. Aber Voegelins "`anamnetische Experimente"' sind vermutlich eher als
Beispiele zu verstehen denn als eine vollzählige Auflistung.

Abgesehen davon bleiben Voegelins "`anamnetische Experimente"' ein wenig
hinter den Erwartungen zurück, die durch seine vorangegangenen Ausführungen
geweckt werden. In den vorangegangenen beiden Abhandlungen kommt der
mystischen Erfahrung des welttranszendenten Seinsgrundes eine zentrale
Bedeutung zu. Für Voegelins These etwa, dass wir das Sein nur verstehen
könnten als einen im welttranszendenten Seinsgrund entspringenden Prozess, ist
diese Erfahrung eine unabdingbare Voraussetzung. Damit diese These glaubhaft
wird, müsste es nämlich schon tatsächlich der Seinsgrund sein, der sich in
dieser Erfahrung zeigt. Sollte es sich nur um irgendein überwältigendes
Meditationserlebnis handeln, welches bloß vor lauter Begeisterung eine
Erfahrung vom Seinsgrund genannt wird, so wäre die These noch unzureichend
begründet, denn der Prozess des Seins kann -- außer für einen philosophischen
Solipsisten -- nicht einer Erfahrung {\em im} Bewusstsein entspringen.
Voegelins "`anamnetische Experimente"' bilden einen der wenigen Anlässe für
Voegelin, von eigenen Erfahrungen zu berichten.  Ein glaubhaftes und
unzweideutiges Transzendenzerlebnis fördern Voegelins "`anamnetische
Experimente"' jedoch nicht zu Tage. Zwar deutet Voegelin in seinen
Einleitenden Bemerkungen zu den anamnetischen Experimenten noch an, dass die
Bewusstseinstranszendenz, die in "`finiter Erfahrung"' in die Welt hinein
führt, nur eine Art von Transzendenz sei, und er führt unter den verschiedenen
Transzendenzerfahrungen, die in der Biographie des Bewusstseins schon lange
vor dem Einsetzen der philosophischen Besinnung vorgegeben sind, auch die
Erfahrung der Transzendenz in den Seinsgrund auf,\footnote{Vgl.  Voegelin,
  Anamnesis, S. 61.} aber in den Kindheitserlebnissen lässt sich dann nichts
mehr davon wiederfinden. Dort ist zwar unter anderem von dem Schlaraffenland
und auch von einer Wolkenburg die Rede, aber einen Hinweis auf irgendetwas,
was auch nur annähernd als transzendente Seinsphäre oder gar als der Grund
allen Seins gelten könnte, sucht man vergebens.

Endlich gibt es noch einen weiteren, mehr psychologischen Grund, der das
Unterfangen, in Kindheitserinngerungen die "`Erregunsquellen"' ausfindig zu
machen, aus denen es im Erwachsenenleben "`zu weiterer philosophischer
Besinnung treibt"',\footnote{Voegelin, Anamnesis, S.62.} fragwürdig erscheinen
lässt. Wenn wir im Erwachsenenalter rückblickend unsere Kindheit betrachten
und uns dabei außerdem noch auf der Suche nach unseren eigenen geistigen
Ursprüngen befinden, dann lässt es sich nicht immer vermeiden, dass wir in
unsere Erinnerung etwas hineininterpretieren, was ursprünglich gar nicht
vorhanden war. Die Gefahr einer Selbstmystifikation ist bei "`anamnetischen
Experimenten"' nur schwer zu umgehen. Es besteht hier übrigens eine Analogie
zu jener größeren historischen "`Anamnese"' der Wiedererweckung verschollenen
Ordnungswissens aus den Quellen der antiken Philosophie und mittelalterlichen
Theologie, wo ebenfalls gelegentlich der Eindruck entsteht, dass bei Voegelin
eine durch und durch moderne existenzialistische Philosophie dem Denken der
Alten aufgestülpt wird.

Voegelins Programm der "`Anamnese"' scheitert im Ganzen also aus drei Gründen:
Erstens ist die Genealogie eines Gedankens für den Gedanken selbst, d.h. für
seinen Inhalt, seine Richtigkeit oder Falschheit, bedeutungslos.
Zweitens führt das Verfahren der "`Anamnese"' mit großer Wahrscheinlichkeit zu
einer Verfälschung der Genealogie. Drittens gelangt man auf diesem Wege
ebensowenig zu jener vermeintlich vorhandenen Transzendenz wie durch die
philosophische Meditation.


%%% Local Variables: 
%%% mode: latex
%%% TeX-master: "Main"
%%% End: 













%%% Local Variables: 
%%% mode: latex
%%% TeX-master: "Main"
%%% End: 

\chapter{"`Was ist politische Realität?"' (Anamnesis - Teil III)}  
\label{politischeRealitaet}

Der Aufsatz "`Was ist politische Realität?"', mit dem Voegelin sein Werk
"`Anamnesis"' beschließt, stellt eine umfassende Grundsatzarbeitet über das
Wesen politischer Realität, so wie Voegelin es sah, und die Grundlagen einer
diese Realität adäquat beschreibenden Politikwissenschaft dar. Der Aufbau und
die Argumentation des Aufsatzes sind einigermaßen verwickelt, denn obwohl
Voegelin im Vorwort zu "`Anamnesis"' diesem Aufsatz "`eine umfassende und
vorerst befriedigende Neuformulierung der Philosophie des
Bewußtseins"'\footnote{Voegelin, Anamnesis, S. 8.} attestiert, verraten häufige
Wiederholungen, begriffliche Unklarheiten und gelegentliche Selbstkorrekturen
innerhalb des Aufsatzes, dass sich Voegelin seiner Sache keineswegs sicher
war.  Daher gebe ich zunächst eine kurze Übersicht über die wichtigsten
Themenkomplexe, die sich aus Voegelins Aufsatz extrahieren lassen, bevor
dessen Inhalt im Einzelnen dargestellt und kritisiert wird.

Der wohl wichtigste Themenkomplex dieses Aufsatzes bezieht sich auf den
Begriff der Realität. "`Realität"' ist bei Voegelin ein Inbegriff
absoluter metaphysischer Wahrheiten, die die Welt im Ganzen und die Stellung
des Menschen in der Welt betreffen.  Das Wissen um diese metaphysischen
Wahrheiten ("`Ordnungswissen"') wird dem Menschen durch ein inneres Gefühl
("`Ordnungserfahrung"') vermittelt. Eine politische Ordnung kann nur dann eine
gute politische Ordnung sein, wenn sie sich auf dieses Ordnungswissen
gründet.

Der zweite Themenkomplex betrifft Voegelins sprachphilosophische
Ausführungen.  Voegelin war der Ansicht, dass die Wörter, mit denen die
Ordnungserfahrung artikuliert wird, sich nicht wie gewöhnliche Wörter
auf etwas Gegebenes beziehen, das ihre Bedeutung ist, sondern dass sie
"`Indizes"' sind, die etwas über die innere Verfassung und über
besondere Erfahrungen des Bewusstseins vermelden.

Der dritte Themenkomplex behandelt die Beziehungen, die zwischen
unterschiedlich niveauvollen Formen des Ordnungswissens bestehen. Voegelin
zufolge können die Ordnungserfahrungen in einzelnen Fällen klarer oder weniger
klar und damit das ihnen korrespondierende Ordnungswissen niveauvoller
("`differenzierter"') oder weniger niveauvoll ("`kompakter"')
ausfallen.\footnote{Siehe auch die Ausführungen zu den Begriffen der
  Kompaktheit und Differenziertheit in Kapitel \ref{KompaktDifferenziert}.}
Dennoch betreffen sie stets dieselbe Realität. Voegelin glaubt, dass es eine
geschichtliche Entwicklung von einem kompakteren zu einem immer
differenzierteren "`Ordnungswissen"' gibt.

Der vierte Themenkomplex bezieht sich auf den Verlust und das tragische
In-Vergessenheit-Geraten des Ordnungswissens. Voegelin unterscheidet
nicht nur zwischen kompaktem und differenziertem Ordnungswissen, sondern auch
zwischen Philosophien, die überhaupt Ausdruck von Ordnungserfahrungen sind,
und solchen Philosophien, die lediglich aus dogmatischer Begriffsklauberei
und leerer Spekulation bestehen. Zwar bleibt die Realität immer dieselbe,
aber sie kann in Vergessenheit geraten und das Ordnungswissen
schlimmstenfalls durch Ideologien verdrängt werden. Voegelin bezeichnet dieses
Phänomen als "`Realitätsverlust"', und er hält es für die Ursache von
politischen Katastrophen wie z.B. den Totalitarismus.


\section{Naturwissenschaft und Politikwissenschaft} 

Im einleitenden Teil seines Aufsatzes stellt Voegelin die Behauptung
auf, dass die Politische Wissenschaft von einer fundamental anderen Art
sei als die Naturwissenschaften, so dass die Politikwissenschaft nach
Voegelins Ansicht nicht zu einem durchgängig logisch zusammenhängenden
System von Aussagen ausgebaut werden kann.  Die Gründe hierfür sind für
Voegelin prinzipieller Natur: 1. Der Gegenstandsbereich der
Politikwissenschaft ist bereits durch nicht-wissenschaftliche
Interpretationen besetzt. 2. Der Gegenstand (Politik) wird durch
Interpretationen des Gegenstandes selbst geformt. 3.  Unterschiedliche
Interpretationen der Politik, seien sie nun wissenschaftlicher oder
unwissenschaftlicher Art, streiten einander ihren Wahrheitsanspruch ab
und betrachten sich gegenseitig nur als Störfaktor innerhalb des
Gegenstandsbereiches, indem sie beispielsweise gegen die jeweils andere
Interpretation den Ideologievorwurf erheben.\footnote{Vgl. Voegelin,
  Anamnesis, S. 284-285.}

Aus all dem schließt Voegelin, dass die Beziehung von Wissen und Gegenstand in
der Politikwissenschaft von grundsätzlich anderer Art ist als in den
Naturwissenschaften und dass daher die Politikwissenschaft auch eine besondere
Art von Wissen hervorbringen muss, welches Voegelin als "`noetische
Interpretation"' bezeichnet.\footnote{Vgl. Voegelin, Anamnesis, S. 287.}

Die Gründe, die Voegelin andeutet, legen jedoch nur sehr bedingt die
Konsequenz der Wesensverschiedenheit von Politikwissenschaft und
Naturwissenschaft nahe.\footnote{Ich untersuche hier nur die Gründe, die
  Voegelin für diese These anführt. Eine Untersuchung, ob diese These, für die
  gewiss bessere Argumente ins Feld geführt werden können, grundsätzlich
  richtig ist, würde an dieser Stelle zu weit führen. Für die jüngere
  Diskussion dazu vgl. Shapiro, Flight from Reality, a.a.O.} Der erste Grund
gibt eine Bedingung wieder, die in genau derselben Weise auch für die
Naturwissenschaft gilt, stößt sie doch ebenfalls auf schon vorhandene
Deutungen der Natur, bei denen es sich, je nachdem, um praktisch nützliche
Kenntnisse oder um abergläubische Vorstellungen handeln kann. Der dritte Grund
besagt lediglich, dass es bei den Deutungen der Politik anders als innerhalb
der Naturwissenschaften nicht nur eine durch unterschiedliche
wissenschaftliche Lager, sondern auch eine durch unterschiedliche politische
Lager bestimmte Konkurrenz gibt. Zudem kann man wohl konstatieren, dass die
Lagerkämpfe in den Gesellschaftswissenschaften zuweilen noch etwas
unversöhnlicher ausgetragen werden als in den Naturwissenschaften, weil es in
den Gesellschaftswissenschaften viel schwieriger ist, empirisch zwischen
konkurrierenden Theorien zu entscheiden. Auch haben politikwissenschaftliche
Theorien typischerweise einen weniger formalen und deduktiven Stil und
Charakter als naturwissenschaftliche Theorien. Es würde jedoch zu weit gehen,
aus diesen Unterschieden zu folgern, dass die Politikwissenschaft ein
grundsätzlich anderer Typus von Wissenschaft ist als die Naturwissenschaften.
Allein der zweite Grund könnte diese Konsequenz rechtfertigen. Allerdings
erläutert Voegelin weder, ob und wie infolge dieser Selbstbezüglichkeit die
konventionelle Theoriebildung Gefahr läuft zu scheitern, noch zeigt er, wie
die "`noetische Interpretation"' derartige Probleme vermeidet. Aus dem ersten
Teil von Voegelins Aufsatz ergeben sich also keine stichhaltigen Gründe für
die Vorteile oder die Notwendigkeit des noetischen Verfahrens.

\section{Voegelins Begriff der Realität}
 
Im zweiten Teil seines Aufsatzes beschäftigt sich Voegelin mit dem Wesen und
der Rolle der noetischen Interpretation. Voegelin beginnt zunächst mit einigen
dogmatischen Voraussetzungen über den Ursprung politischer Ordnung.  Dann
entwickelt er am Beispiel des Aristoteles den Begriff der "`noetischen
Exegese"' der Realitätserfahrung und versucht die komplizierte Beziehung
zwischen der noetischen Exegese und der vergleichsweise primitiveren
mythischen Auslegung zu bestimmen. Darauf geht Voegelin auf die Schwächen der
aristotelischen Philosophie ein und leitet zu seiner eigenen Fortführung der
aristotelischen Exegese über, in deren Zentrum ein höchst eigentümlicher
Begriff der "`Realität"' steht. Schließlich geht Voegelin auf das Thema des
"`Realitätsverlustes"' und der seiner Ansicht nach daraus resultierenden
politischen Unordnung ein.
 
\subsection{Die "`Spannung zum Grund"' als Ursprung der Ordnung}

Politische Ordnung entspringt Voegelin zufolge in letzter Instanz einer
inneren Erfahrung des Menschen, der Erfahrung, geordnet zu sein "`durch die
Spannung zum göttlichen Grund seiner Existenz"'.\footnote{Voegelin, Anamnesis,
  S. 287.} Von dieser Erfahrung "`strahlen"' in einer nicht näher
spezifizierten Weise "`die Interpretationen gesellschaftlicher Ordnung
aus"'.\footnote{Voegelin, Anamnesis, S. 287.} Da diese Erfahrung nicht
gegenständlich ist (ähnlich, vermutlich, wie auch eine Stimmung oder das
Lebensgefühl eines Menschen nicht gegenständlich sind), kann es "`kein
sogenanntes intersubjektives Wissen"'\footnote{Voegelin, Anamnesis, S. 287.}
von der richtigen Ordnung geben, was Voegelin später jedoch nicht im
Geringsten daran hindert, strikt auf der intersubjektiven Verbindlichkeit
dieser Ordnung zu bestehen.\footnote{Vgl. beispielsweise Voegelin,
  Anamnesis, S. 348-350.}  Im Ringen um einen angemessenen Ausdruck für diese
innere Erfahrung, welches Anlass für die verschiedensten Interpretationen der
richtigen Ordnung gibt, erblickt Voegelin den Ursprung von "`Spannungen in der
politischen Realität"'.\footnote{Voegelin, Anamnesis, S. 288.} So vielfältig
die Interpretationen der Ordnung auch sind, so ist ihnen doch gemeinsam, dass
sie alle nur von einem Grund der Ordnung ausgehen, selbst dann, wenn, wie zur
Zeit des Aristoteles, das Faktum einer Interpretationsvielfalt schon bekannt
ist.  Daraus schließt Voegelin, dass es auch {\em tatsächlich} nur einen
Ordnungsgrund gibt. Dieser Schluss ist jedoch aus mehreren Gründen fragwürdig:
Erstens lässt sich der Befund des Glaubens an einen einzigen (transzendenten)
Grund schwer mit polytheistischen Religionen oder mit naturphilosophischen
Elementelehren, die mehr als ein Element annehmen (z.B. die vier Elemente
Feuer, Wasser, Erde, Luft bei Empedokles), vereinbaren.\footnote{Es kann
  berechtigterweise in Zweifel gezogen werden, ob es bei den Elementelehren
  der Vorsokratiker um die Bestimmung eines {\it Ordnungs-}grundes geht. Aber
  im Zusammenhang der Voegelinschen Interpretation der Philosophiegeschichte
  wäre diese Annahme konsequent.} Zweitens unterscheiden sich die
Interpretationen, die einen einzigen Grund annehmen, zum Teil sehr stark
voneinander hinsichtlich der Eigenschaften dieses Grundes. Es bleibt daher
sehr fraglich, ob in den unterschiedlichen Interpretionen derselbe Grund
gemeint ist.  Drittens folgt daraus, dass es den Glauben an einen einzigen
Grund gibt, weder dass dieser Grund existiert, noch dass es auch in
Wirklichkeit nur ein einziger ist.

\subsection{Die "`noetische"' Exegese bei Aristoteles} 

Voegelin geht nun in einiger Ausführlichkeit auf die Metaphysik des
Aristoteles ein. Aristoteles hat nach Voegelins Auf\/fassung als einer der
ersten Philosophen eine umfassende "`noetische Exegese"' des Bewusstseins
geliefert. Die "`noetische Exegese"' folgt historisch auf die rein mythische
Deutung der Ordnung. Sie entsteht, wenn das Bewusstsein des Menschen entdeckt
und infolge dessen der Grund der Ordnung in der inneren Erfahrung und nicht
mehr im Kosmos gesucht wird. Die Auslegung der Bewusstseinserfahrung ist es,
was Voegelin "`noetische Exegese"' nennt.\footnote{Vgl. Voegelin, Anamnesis,
  S. 288.  Wörtlich spricht Voegelin davon, dass die "`noetische Exegese"' den
  "`Logos"' des Bewusstseins auslegt.}
 
Woraus entspringt das Bedürfnis nach einer noetischen Exegese? Voegelins
Aristoteles-Interpretation zufolge lebt der Mensch, der den Grund seiner
Existenz nicht kennt, in einem Zustand der Angst.\footnote{Dass Aristoteles
  nicht eigentlich von "`Angst"' spricht, erklärt Voegelin kurzerhand damit,
  dass es in der griechischen Sprache kein entsprechendes Wort gegeben habe.
  (Vgl. Voegelin, Anamnesis, S. 288.)  An dieser Stelle lässt sich der
  Eindruck schwer vermeiden, dass Voegelin in anachronistischer Weise einen
  Schlüsselbegriff des modernen Existentialismus in die Deutung der
  klassischen Philosophie hineinträgt.}  Diese Angst ist zugleich eine
metaphysisch sehr informative Angst, denn sie enthält "`das Wissen des
Menschen um seine Existenz aus einem Seinsgrund, der nicht der Mensch selbst
ist."'\footnote{Voegelin, Anamnesis, S. 289.} Nun möchte der Mensch diesen
Seinsgrund verständlicherweise näher kennenlernen.  Deshalb strebt er nach
Wissen. Dieses Streben hat die Form eines suchenden Begehrens, es hat die
Richtung auf den Seinsgrund hin, und es wird am anderen Ende vom Seinsgrund
durch eine eigenständige Anziehungskraft -- über die dieser gemäß Aristoteles
verfügt -- unterstützt. \label{Rationalitaetsbegriff} Die Richtung dieser Suche
bezeichnet Voegelin als "`Ratio"'. Unter "`rational"' versteht Voegelin daher
völlig abweichend vom üblichen Wortgebrauch in etwa das, was Bergson (nach
Voegelins Interpretation) mit der "`Offenheit der Seele"' meint, also eine
besonders ausgeprägte spirituelle
Sensibilität.\footnote{\label{FussnoteRationalitaet} Vgl.  Voegelin, Anamnesis,
  S. 289.  -- Vgl. auch Eric Voegelin: In Search of the Ground, in:
  Conversations with Eric Voegelin.  (ed. R. Eric O'Connor), Montreal 1980, S.
  1-20 (S. 4-5). -- Hier führt Voegelin anhand von Aristoteles aus, dass von
  Rationalität nur die Rede sein kann, wenn nicht bloß das Mittel in Bezug auf
  den Zweck sondern auch der Zweck selbst rational ist, wozu die
  Zweck-Mittel-Ketten irgendwann einmal zum Nous (göttlicher Geist) als dem
  höchsten Zweck führen müssen. Dieses Argument liefert zwar eine Definition
  von Nous, beweist aber weder dessen Existenz noch die Identität des so
  definierten Nous mit dem transzendenten Seinsgrund, der sich (mutmaßlich) in
  mystischen Erfahrungen zeigt. -- Auf das Grundproblem, welches die legitime
  Bedeutung umstrittener Ausdrücke wie z.B.  "`Ratio"' ist, kann an dieser
  Stelle nicht ausführlich eingegangen werden. Zwei Anmerkungen erscheinen mir
  jedoch angesichts der von Voegelin verfolgten semantischen Strategie
  notwendig: 1. Die legitime Wortbedeutung ist nicht notwendigerweise die
  historisch ursprünglichste Bedeutung dieses Wortes, da sich auch Wörter und
  Begriffe entwickeln können.  Daher wäre es falsch zu sagen: Aristoteles hat
  als erster von "`Ratio"' gesprochen, also müssen wir uns an das halten, was
  Aristoteles damit gemeint hat. 2. Wenn man ein Wort in einer anderen als der
  üblichen Bedeutung verwenden will, so muss man entweder darauf achten, die
  neue Bedeutung so zu wählen, dass das semantische Feld des Wortes erhalten
  bleibt (z.B. rational ist immer etwas, was jedermann durch Nachdenken
  einsichtig werden kann), oder man muss das gesamte semantische Feld
  abändern, was möglicherweise eine Lawine von Redefinitionen nach sich zieht.
  Bei beiden Punkten spielt es keine Rolle, wie fehlgeleitet der herrschende
  Sprachgebrauch ist.  Im übrigen ist immer Abhilfe durch die Einführung neuer
  Begriffe möglich.}  Voegelin führt nun noch weiter aus, wie sich bei
Aristoteles die Beziehung zwischen menschlichem Wissen und göttlichem
Seinsgrund als eine Form von "`Partizipation"', d.i. der Teilhabe des Menschen
am göttlichen Seinsgrund, darstellt. Obwohl Voegelin den Begriff der
Partizipation im folgenden für seine eigenen Überlegungen übernimmt, werden
weder die genaue Bedeutung dieses Begriffes noch die Bedingungen der
Möglichkeit eines derartigen Vorgangs von Voegelin näher bestimmt.  Der
Verzicht auf die Klärung dieses Begriffes ist um so verwunderlicher, als
Voegelin feststellt, dass in Aristoteles' Überlegungen an dieser Stelle noch
sehr massiv mythische Denkweisen Eingang gefunden haben.
 
Diese Feststellung führt Voegelin zu einem neuen Thema, nämlich der
grundsätzlichen Frage nach der Beziehung von Mythos und noetischer Exegese.
Voegelin zufolge beruht der Mythos auf einem eigenen Typ von Welterfahrung,
den er im Gegensatz zur noetischen Erfahrung als "`Primärerfahrung"'
bezeichnet. Für gewöhnlich ersetzt bzw. "`differenziert"' die noetische
Erfahrung die Primärerfahrung.  Aber es gibt eine Ausnahme, bei der dies, wie
Voegelin meint, nicht möglich ist: Die Erfahrung der Wesensgleichheit aller
Menschen. Diese Ausnahme berührt zugleich eines der Fundamentalprobleme der
gesamten philosophischen Konzeption Voegelins, nämlich das Problem, wie die
noetischen Erfahrungen, obwohl sie kein intersubjektives Wissen
zulassen,\footnote{Vgl. Voegelin, Anamnesis, S. 287.} dennoch für alle Menschen
gültig sein können. Nach Voegelins Ansicht geht die universelle Gültigkeit
noetischer Erfahrung aus der Wesensgleichheit aller Menschen hervor, welche
ihrerseits Gegenstand der mythischen Primärerfahrung ist. Offensichtlich ist
die Ersetzung dieser Primärerfahrung durch eine noetische Erfahrung nicht
möglich, denn dies würde zu einem Begründungszirkel führen. Voegelin
übersieht, wenn er so argumentiert, jedoch mehrere Schwierigkeiten: Erstens
würde das Problem der Universalität der noetischen Erfahrung nur auf das
Problem der Universalität des Mythos verschoben werden, so dass sich auf einer
anderen Ebene genau dasselbe Gültigkeitsproblem wieder stellt. Zweitens folgt
aus der grundsätzlichen Wesensgleichheit aller Menschen nicht, dass die
Menschen auch hinsichtlich ihrer religiösen Erfahrungen gleich sind, oder dass
die religiöse Erfahrung eines Menschen verbindlich für einen anderen Menschen
sein kann. Drittens lässt sich die Wesensgleichheit aller Menschen prinzipiell
nicht mythisch begründen, denn Mythen können höchstens etwas veranschaulichen
aber niemals begründen.
 
Doch damit ist noch nicht alles über den komplizierten Zusammenhang von
noetischer Exegese und Mythos gesagt. Wird versucht, die tieferen Beziehungen
dieser beiden Auslegungsweisen zu einander und zur Wirklichkeit zu ergründen,
so findet man sich Voegelin zufolge zunächst vor einer Reihe von Aporien
wieder, die aufgelöst werden müssen: Die erste Aporie beruht darauf, dass
sowohl die noetische Erfahrung als auch andere Auslegungsweisen, seien sie nun
mythischer oder dichterischer oder philosophischer Art, Formen der
Partizipation darstellen.  Gleichzeitig wird das Wort "`Partizipation"' aber
auch als Selbstbezeichnung allein der noetischen Erfahrung verwendet.
Voegelin übersieht, dass hier offenbar ein Wort in zweierlei Bedeutung
gebraucht wird. Anstatt durch die Einführung eines neuen Wortes oder durch ein
qualifizierendes Adjektiv Klarheit zu schaffen,\footnote{Dazu ist es
  keineswegs notwendig, wie Voegelin unter (2) (Anamnesis, S. 292.) sagt,
  "`das Partizipieren des Philosophen ... von den anderen Fällen zu
  dissoziieren und ihm kognitive Qualität zuzuschreiben"'.  Zum
  "`Dissoziieren"' genügt es hinsichtlich des von Voegelin aufgeworfenen
  logischen Problems, dass die Fälle überhaupt unterschieden werden können,
  was offenbar gegeben ist, denn wenn zwischen noetischer und nicht-noetischer
  Auslegung unterschieden werden kann, dann kann auch zwischen noetischer
  Auslegung und der Klasse unterschieden werden, die die noetische und
  nicht-noetische Auslegung (und möglicherweise noch weitere Übergangsformen)
  enthält.} zieht Voegelin die falsche Schlussfolgerung, dass die
Partizipation als Spezies unter sich selbst als Genus fiele. Voegelin krönt
seinen logischen Fehler durch die kategorische Feststellung, dass "`die Logik
der Gegenstände und ihrer Klassifikation"'\footnote{Voegelin, Anamnesis, S.
  293.}  nicht auf den Realitätsbereich des Partizipierens anwendbar sei.
Übrigens glaubte Voegelin auch sonst recht häufig, vor einem tieferen Rätsel
zu stehen, wenn er in Wirklichkeit bloß mehrdeutige Ausdrücke vor sich hatte.
Als ein Opfer seiner anti-nominalistischen Vorurteile erkannte er in diesen
Vieldeutigkeiten nicht eine sprachliche Ungenauigkeit, wie sie durch eine
saubere begriffliche Unterscheidung leicht bereinigt werden kann, sondern er
vermutete in derartigen Vieldeutigkeiten oftmals einen tieferen Sinn und damit
ein schwieriges philosophisches Problem,\footnote{An prominenter Stelle
  liefert dafür die Diskussion des Begriffes der Geschichte in "`Order and
  History I"' ein Beispiel. (Vgl. Voegelin, Order and History I, S. 126-133.)
  Voegelin hätte sich einen Großteil seiner mühevollen Erörterungen sparen
  können, wenn er von vornherein klar zwischen Geschichte und
  Geschichtsbewusstsein bzw.  zwischen Geschichte und religiöser
  Heilsgeschichte unterschieden hätte. Denn es ist durchaus nichts Absurdes
  daran zu sagen, dass die alten Ägypter, wie jedes Volk, eine Geschichte
  hatten aber keine Heilsgeschichte wie das Volk Israel, während es in der Tat
  falsch wäre zu behaupten, Israel habe eine Geschichte, Ägypten aber nicht.}
während es sich in Wirklichkeit bloß um den klassischen Fall eines
philosophischen Scheinproblems handelt.

Trotz der völlig misslungenen Herleitung seines Gedankens lässt sich aus
Voegelins Worten immerhin entnehmen, worauf er hinaus will. Im Folgenden
versteht Voegelin das Wort "`vergegenständlichen"' nicht mehr im Sinne von
"`klassifizieren"', sondern im Sinne von "`zum Gegenstand einer Untersuchung
machen"'. Diese Form von Vergegenständlichung ist eine Voraussetzung
wissenschaftlicher Erkenntnis, aber sie schneidet gleichzeitig die Möglichkeit
eines existentiellen Verstehens ab.\footnote{Voegelin scheint hier einen
  Verstehensbegriff zu Grunde zu legen, wie er sich z.B. auch bei Karl Jaspers
  als Begriff der "`existenziellen Kommunikation"' findet. Vgl. Jaspers,
  Philosophie II, S. 51, S. 58. -- Vgl. auch Jeanne Hersch: Karl Jaspers. Eine
  Einführung in sein Werk, 4. Aufl., München 1990, S. 31-35. -- Es gibt
  zahlreiche Berührungspunkte zwischen dem Denken Voegelins und der
  Philosophie Jaspers', auf die hier jedoch nicht ausführlich eingegangen
  werden kann. Einige Bemerkungen über die nicht weniger gravierenden
  Unterschiede sind jedoch dringend angebracht: Bei Voegelin wird die
  Existenzphilosophie um eine politische Militanz verschärft, die geeignet
  ist, einige ihrer Botschaften geradezu ins Gegenteil zu verkehren. So glaubt
  Voegelin, die Öffnung zur Transzendenz ebenso einfordern zu können wie die
  existentielle Kommunikation, die zudem auf Basis von Bedingungen zu erfolgen
  hat, welche Voegelin vorschreibt (Anerkennung der Existenz des und einer
  liebenden Beziehung zum transzendenten Sein). Das Scheitern der
  existenziellen Kommunikation auf Basis der geöffneten Seele bedeutet für
  Voegelin nicht bloß ein existenzielles Misslingen von individueller Tragik,
  sondern es begründet -- wenn man Voegelins Polemik ernst nimmt, wie es u.a.
  Poirier tut (Vgl. Poirier, a.a.O.) -- den Vorwurf eines schuldhaften
  Vergehens, welches in letzter Instanz die politische Untragbarkeit des
  Scheiternden nach sich zieht.} Hieraus ergibt sich für Voegelin, dass die
noetische und die nicht-noetische Auslegung sich nicht gegenseitig
"`vergegenständlichen"' können, ohne dass etwas dabei verloren ginge, weil
beide Formen des Partizipierens und damit derselben existentiellen
Betroffenheit sind, die aus der Berührung mit der Transzendenz hervorgeht. Aus
dem Blickwinkel der noetischen Auslegung darf also anderen Auslegungsformen
der Rang der Partizipation nicht abgesprochen werden.
 
Aber auch wenn der Mythos daher nicht gänzlich der Unwahrheit verfällt, so
wird doch, wie Voegelin meint, aus der noetischen Exegese heraus ein
Wahrheitsgefälle sichtbar. Die Entwicklung von niederer Wahrheit zu höherer
Wahrheit nennt Voegelin das "`Feld der Geschichte"'. Diese Entwicklung findet
zunächst im Bewusstsein einzelner Menschen statt, die eine vollkommenere
Ausdrucksform für die Partizipation und damit eine höhere Wahrheit finden. Da
diese neue Ausdrucksform jedoch zur Infragestellung nicht bloß der bisherigen
persönlichen Überzeugungen des Denkers, sondern auch der gesellschaftlich
tradierten Ausdrucksformen führt, erlangt sie gesellschaftliche
Bedeutung.\footnote{Vgl. Voegelin, S. 294.}
 
Diesen komplexen Beziehungen zwischen noetischer Exegese und anderen
Auslegungsformen versucht Voegelin nun bei Aristoteles nachzuspüren.
Aristoteles nimmt in seiner Metaphysik auf zwei geistige Traditionen Bezug:
Auf die Mythologie und auf die Philosophie von den Vorsokratikern bis Platon.
Üblicherweise werden diese beiden Traditionen als zwei unterschiedliche, ja
gegensätzliche Diskurstypen innerhalb der hellenischen Geisteskultur
betrachtet, wobei die Philosophie der Vorsokratiker demselben nicht-mythischen
Diskurstyp zugehört wie die der späteren Philosophen einschließlich Platon und
Aristoteles. Diese Sichtweise entspricht ja auch der Selbstwahrnehmung der
antiken griechischen Philosophen einschließlich des Aristoteles.\footnote{Vgl.
  Luc Brisson: Einführung in die Philosophie des Mythos. Antike, Mittelalter
  und Renaissance. Band I, Darmstadt 1996, S. 13-19 / S. 52-53.} Und Voegelin
leugnet keineswegs, dass Aristoteles sich mit den Vorsokratikern auf einer
argumentativ-diskursiven Ebene auseinandersetzt.  Aber Voegelin glaubt,
Aristoteles in diesem Punkt besser als dieser sich selbst zu verstehen, und
hält ihm daher für sein "` `Sich-Einlassen' "'\footnote{Voegelin, Anamnesis,
  S. 296.} auf eine argumentative Auseinandersetzung mit den Vorsoktratikern
einen Mangel an Exaktheit vor. Für Voegelin hat Aristoteles nämlich nicht
gebührend berücksichtigt, dass er selbst sich bereits auf einer höheren Stufe
der "`Bewußtseinshelle"' befand, während die Vorsokratiker nur über ein
"`Partizipationswissen geringerer Deutlichkeit"'\footnote{Ebd.} verfügten. Im
Grunde hätte Aristoteles es nämlich gar nicht nötig gehabt, den Ansichten der
Vorsokratiker Argumente entgegenzusetzen, da er ja "`weiß .., daß in seiner
noetischen Erfahrung der Nous das adäquate Symbol für den Grund ist, und ..
sich diese Wahrheit daher nicht durch ein Argument zu beweisen
[braucht]."'\footnote{Ebd.}

Während für Voegelin also Aristoteles (und Platon) von den Vorsokratikern
durch eine Erfahrungsstufe getrennt sind, scheint ihm der Unterschied zwischen
Philosophie und Mythos andererseits weniger fundamental. Auf allen Stufen, vom
Mythos über die Vorsokratiker bis zu Aristoteles, geht es Voegelin zufolge um
eine Erfahrung der "`Partizipation"' und um deren Artikulation in Symbolen.
Was sich von Stufe zu Stufe (also zunächst von der Stufe des Mythos zu
der der vorsokratischen Philosophie und dann von dieser zur
Platonisch-Aristotelischen) ändert, ist die Erfahrung, die von Mal zu Mal
"`differenzierter"' wird. Durch diese Steigerung entsteht die Geschichte. Und
zwar entsteht dabei nicht, wie man denken könnte, irgendeine bestimmte
Geschichte, etwa die Geschichte der religiösen oder philosophischen
"`Erfahrungen"', sondern es entsteht die Geschichte schlechthin, denn
Geschichte wird "`durch das Bewußtsein konstituiert, so daß der Logos des
Bewußtseins darüber entscheidet, was geschichtlich relevant ist, und was
nicht."'\footnote{Voegelin, Anamnesis, S.  299.}  Voegelin fügt hinzu, dass
die Zeit, in der sich die Geschichte abspielt, keineswegs "`die der Außenwelt
ist, ... sondern die dem Bewußtsein immanente Dimension des Begehrens und
Suchens nach dem Grund."'\footnote{Ebd.}  Da ferner
alle Menschen nach dem Grund suchen, ist die solcherart durch das Bewusstsein
konstituierte Geschichte "`universell-menschlich"'\footnote{Ebd.}.
 
Es fällt schwer, diese Äußerungen über die Geschichte nachzuvollziehen. Denn
entweder man versteht sie als Aussagen über das, was konventionellerweise als
Geschichte bezeichnet wird, also etwa über die politische Geschichte. Dann
sind Voegelins Aussagen schlicht falsch, denn die politische Geschichte spielt
sich natürlich in der äußeren Zeit ab, und der "`Logos des Bewußtseins"' kann
so wenig über das entscheiden, was geschichtlich relevant ist, wie er über das
entscheiden kann, was geschehen ist. Oder man versteht Voegelins Äußerungen
als Definition von "`Geschichte"'. Dann bleibt die so definierte Geschichte
jedoch für alle, die nicht Anhänger der Voegelinschen oder einer ähnlichen
Philosophie sind, völlig irrelevant.  "`Universell-menschlich"' ist diese
Geschichte höchstens ihrem eigenen Anspruch nach, ähnlich, wie auch manche
Religionen sich selbst als "`universell-menschlich"' verstehen, ohne es
jedoch, da es ihrer eine Vielzahl gibt, jemals wirklich zu sein.

\subsection{Der Begriff der politischen Realität}
 
Voegelin leitet nun mit einer Kritik an Aristoteles über zu seiner eigenen
noetischen Exegese. Die größte Schwäche von Aristoteles' noetischer Exegese
erblickt Voegelin darin, dass Aristoteles an zentraler Stelle immer wieder auf
den sehr missverständlichen Ausdruck "`Ousia"' zurückgreift. Voegelin zufolge
ist dieser Ausdruck noch der mythischen Primärerfahrung verhaftet und bezieht
sich auf die "`fraglos, selbstverständlich und überzeugend uns
entgegentretende Wirklichkeit der `Dinge' "'.\footnote{Voegelin, Anamnesis,
  S. 301.} Dieser mythische Überhang, den Voegelin in diesem Falle offenbar
nicht wie im Falle der mythisch begründeten Wesensgleichheit aller Menschen
für sachlich notwendig hält, rächte sich historisch, indem spätere
Philosophen, bei welchen die noetische Erfahrung so weit in den Vordergrund
gerückt war, dass die mythische Primärerfahrung fast völlig verblassen musste,
die aristotelische "`Ousia"' als Gegenstandsbezeichnung missverstanden und
begrifflich-philosophische Spekulationen daran knüpften. Das Missverständnis
des Aristoteles ist Voegelin zufolge die Ursache für den theologischen und
philosophischen Dogmenstreit über Fragen wie die der Unsterblichkeit der
Seele, der Beweisbarkeit der Existenz Gottes oder der Endlichkeit oder
Unendlichkeit der Welt. Das historische Unheil vollendet sich für Voegelin mit
der Aufklärung und dem Positivismus, die nicht nur, was noch zu rechtfertigen
wäre, die dogmatischen Argumente der mittelalterlichen Philosophie angreifen,
sondern die auch die höhere Realität des Partizipierens des Menschen am
transzendenten Seinsgrund leugnen. Dies zieht nach Voegelins Überzeugung auf
individueller Ebene die psychopathologische Erscheinung des "`realitätslosen
Existierens"' und auf gesellschaftlicher Ebene den Totalitarismus nach sich.
Voegelin illustriert diese Zusammenhänge mit einzelnen Beispielen aus der
schönen Literatur, worin Wirklichkeitsverlust und Sprachlosigkeit thematisiert
werden.\footnote{Vgl. Voegelin, Anamnesis, S. 302-303.}

Wenn die noetische Exegese des Aristoteles also in einigen Punkten noch
unvollkommen oder wenigstens missverständlich ist, dann stellt sich natürlich
die Frage, wie sie besser durchgeführt werden kann. Voegelin versucht dies,
indem er statt der problematischen "`Ousia"' des Aristoteles den Begriff der
Realität in den Mittelpunkt seiner Überlegungen stellt. "`Realität"' wird
gewöhnlicherweise als der Inbegriff all dessen verstanden, was tatsächlich
vorhanden ist, im Gegensatz zu dem, was bloß in der Vorstellung oder der
Phantasie existiert.  Voegelin gebraucht dieses Wort in einem anderen Sinne.
Für ihn ist "`Realität"' ein Inbegriff bestimmter metaphysischer
Seinszusammenhänge, die er in dem Satz zusammenfasst: "`Eine Realität, genannt
Mensch, bezieht sich, innerhalb eines umgreifend Realen, durch die Realität
des Partizipierens, genannt Bewusstsein, erfahrungs- und bildhaft auf die
Termini des Partizipierens als Realitäten"'.\footnote{Voegelin, Anamnesis, S.
  304.} Der wesentliche Teil dieser Aussage liegt in dem Wort
"`Partizipieren"' und darin, dass zu den "`Termini des Partizipierens"'
(denjenigen Dingen, die aneinander partizipieren) auch der "`göttliche Grund"'
gehört, dessen Existenz Voegelin, wie üblich, ohne weitere Begründung als
vermeintliches Erfahrungsfaktum voraussetzt. Weiterhin spielt es für Voegelin
eine große Rolle, dass der Vorgang der Partizipation und die partizipierenden
Bestandteile ("`Termini des Partizipierens"') einen untrennbaren
Gesamtzusammenhang bilden. Wollte man also beispielsweise nur von Gott bzw.
dem göttlichen Grund reden, ohne auch auf die Beziehung des Menschen zu Gott
einzugehen, so würde man sich aus Voegelins Perspektive wohl eines
gedanklichen Fehlers oder wenigstens einer Ungenauigkeit schuldig machen.
Voegelin ist um die Wahrung dieses Gesamtzusammenhangs so ängstlich besorgt,
dass er es sogar für unumgänglich hält, das Wort "`Realität"' vieldeutig zu
gebrauchen, derart dass es zugleich sowohl den Gesamtzusammenhang als auch
jeden einzelnen Bestandteil des Zusammenhanges und darüber hinaus auch noch
die "`Symbole"' bezeichnet, die zur Artikulation des Gesamtzusammenhanges oder
seiner Bestandteile gebraucht werden.\footnote{Vgl.  Voegelin, Anamnesis, S.
  305, S. 307. -- Dass Voegelin die Vieldeutigkeit des Wortes "`Realität"' für
  notwendig erklärt, verwundert umso mehr, als er sie selber durch den
  Gebrauch unterschiedlicher und sich auf jeweils andere Aspekte beziehende
  Ausdrücke ("`Realität"', "`Partizipation"', "`Termini des Partizipierens"')
  zu umgehen weiß. (Vgl. auch: Voegelin, Order and History V, S.16-18. Hier
  tritt an die Stelle des vieldeutigen Realitätsbegriffs der Komplex von
  Bewusstsein-Realität-Sprache, dessen einzelne Elemente ebenfalls
  terminologisch eindeutig gekennzeichnet sind.)  Vermutlich haben wir es hier
  wieder mit dem sprachlichen Problem der vieldeutigen Ausdrücke zu tun,
  welches Voegelin so viel unnötiges Kopfzerbrechen bereitete.} Die
"`Realität"' des Partizipierens und seiner "`Termini"' ist immer und in
gleichbleibender Weise vorhanden, unabhängig davon, auf welchem Niveau
(noetisch oder prä-noetisch) sie erlebt und artikuliert wird. Sie bleibt als
Realität selbst dann noch gegenwärtig, wenn sie geleugnet wird.  Etwas
irritierend wirkt es auf den ersten Blick, dass Voegelin trotz dieser
ausdrücklichen Erklärung wenige Zeilen weiter nicht mehr von der Konstanz der
Realität ausgeht, sondern davon spricht, dass die "`Realität"' zugleich
konstant und veränderlich ist.\footnote{Vgl. Voegelin, Anamnesis, S. 306.}
Vielleicht muss man sich das Partizipieren ähnlich der Beziehung der
Verwandtschaft zwischen verwandten Menschen vorstellen, die auch dann noch
vorhanden ist, wenn die Verwandten kein Wort miteinander reden, die aber
dadurch stark intensiviert werden kann, dass die Verwandten wieder anfangen,
miteinander zu kommunizieren, indem sie beispielsweise Geburtstagsgrüße oder
Weihnachtskarten austauschen. Auch die Partizipation kann intensiviert werden,
wenn sich die Menschen ihrer bewusst werden und sie auf das Niveau
"`noetischer Erfahrung"' heben.  Einen derartigen Zusammenhang scheint
Voegelin im Auge zu haben, wenn er von der gleichzeitigen Konstanz und
Veränderlichkeit der Partizipation spricht.  Im ganzen repräsentiert der
Begriff der Realität in Voegelins Gedankengebäude jedoch das Unveränderliche
gegenüber den sich wandelnden Erfahrungen und ihren unterschiedlichen
Artikulationen.\footnote{Vgl.  auch Vgl. Eric Voegelin: Äquivalenz von
  Erfahrungen und Symbolen in der Geschichte, in: Eric Voegelin, Ordnung,
  Bewußtsein, Geschichte, Späte Schriften (Hrsg. von Peter J. Opitz),
  Stuttgart 1988, S. 99-126 (S. 107-108 / S. 111-112.).}
 
Ein schwerwiegendes Missverständnis ist es Voegelin zufolge, wenn auf Grund
einer plötzlichen und sehr intensiven Steigerung der Partizipationserfahrung
irrtümlich geglaubt wird, der Mensch und die Welt selbst hätten sich nun in
ihrem Wesen verwandelt. In diesem Missverständnis glaubt Voegelin die Ursache
sowohl der aufklärerischen Fortschrittsidee als auch von apokalyptischen
Visionen und Endzeithoffnungen entdecken zu können.\footnote{Vgl. Voegelin,
  Anamnesis, S. 307.} Die Behauptung, dass eine plötzlich intensivierte
Partizipationserfahrung die Ursache dieser Phänomene sei, verblüfft ein wenig,
da Voegelin unmittelbar zuvor noch das politische Unheil aus der Leugnung der
metaphysischen Seinsrealität abgeleitet hat.\footnote{Vgl. Anamnesis,
  S. 302/303.} Besonders deutlich wird diese Unstimmigkeit bei Voegelins
Deutung der aufklärerischen Fortschrittsidee: Wenn die Aufklärung die Leugnung
der metaphysischen Realitätserfahrung par exellence verkörpert, wie kann dann
die aufklärerische Fortschrittsidee zugleich Ausdruck des Überschießens dieser
Realitätserfahrung sein?
 
Der Überschwang durchbrechender neuer Realitätserfahrung kann weiterhin dazu
führen, dass Bewusstsein und Realität, die nach Voegelins Auffassung im
Verhältnis eines Teils zum Ganzen stehen, irrtümlich für vollidentisch gehalten
werden. Diese Gefahr deutet sich schon bei Aristoteles an, wenn er, an
Parmenides anknüpfend, Denken und Gedachtes miteinander identifiziert. Bei
Hegel, der wiederum auf Aristoteles zurückgreift, wird dann der göttliche
Grund in das Bewusstsein hineingezogen, womit für Voegelin der schwerwiegende
Tatbestand gnostischer Spekulation erfüllt ist.\footnote{Vgl. Voegelin,
  Anamnesis, S. 307-309.}
 
Ausgehend von seiner Vorstellung davon, was Realität in Wahrheit ist, stellt
Voegelin nun einige methodologische Grundsätze hinsichtlich der Interpretation
von unterschiedlichen Deutungen der Realität ("`Realitätsbildern"') auf.
Selbstredend scheint Voegelin auch hier wieder vorauszusetzen, dass
Mythologie, Religion und Philosophie samt und sonders solche
"`Realitätsbilder"' verkörpern. Zunächst müssen daher die "`Realitätsbilder"'
als Ausdruck jener von Voegelin als wahr und gültig erkannten "`Realitätsform
des Partizipierens"'\footnote{Voegelin, Anamnesis, S. 309.} verstanden werden.
Wenn alle "`Realitätsbilder"' als Ausdruck jener einen "`Realitätsform"'
verstanden werden, so hat dies Voegelin zufolge den wissenschaftsökonomischen
Vorteil, dass sich daraus unmittelbar eine Erklärung für die oft überraschende
Übereinstimmung räumlich und zeitlich unabhängig voneinander entstandener
"`Realitätsbilder"' ergibt, ohne dass "`okkasionelle
Theorien"'\footnote{Voegelin, Anamnesis, S. 310.} zur Deutung solcher
Übereinstimmungen gefunden werden müssen. Dies ist ein für Voegelins
Verhältnisse überraschend einleuchtendes Argument. Voegelin unterschlägt dabei
jedoch, dass jener wissenschaftsökonomische Vorteil dadurch wieder aufgehoben
wird, dass nun "`okkasionelle Theorien"' zur Erklärung von Abweichungen
zwischen "`Realitätsbildern"', die es ja auch gibt, erfunden werden müssen.
\label{Selbstzeugnisse2} Ein weiterer methodologischer Grundsatz, den Voegelin
in diesem Zusammenhang aufstellt, besteht darin, dass "`Realitätsentwürfe, die
sich als Systeme geben"'\footnote{Ebd.} am Maßstab der Realität, mit welcher
selbstredend die von Voegelin als wahr und richtig erkannte Realität des
Partizipierens gemeint ist, untersucht werden müssen. Es genügt nicht, sie nur
auf Grundlage ihrer eigenen Voraussetzungen zu verstehen. Voegelin vertritt
also wenigstens in Bezug auf bestimmte "`Realitätsentwürfe"' inzwischen genau
den gegenteiligen Grundsatz zu der in seinem Brief über Husserl aufgestellten
Forderung, die Selbstzeugnisse eines Denkers zur strikten Grundlage der
Interpretation seiner Philosophie zu nehmen.\footnote{Vgl. Voegelin,
  Anamnesis, S. 32. -- Siehe auch Seite \pageref{Selbstzeugnisse1} in diesem
  Buch.}
 
Nachdem Voegelin noch einmal kurz das Thema des "`Realitätsverlustes"'
gestreift hat, kommt er auf auf die Möglichkeit der "`periagogé"', der
inneren Umkehr, zu sprechen, durch die sich jeder Mensch auch in
realitätsverlassener Zeit von falschen "`Ersatzrealitäten"' reinigen kann. Als
Beispiel zieht Voegelin hier die Entwicklung von Albert Camus heran, in dessen
intellektuellem Werdegang er vorbildhaft die inneren Kämpfe verkörpert sieht,
die nach Voegelins Ansicht ein Mensch in der heutigen Zeit durchleben muss,
"`der im Widerstand gegen die Zeit seine Wirklichkeit als Mensch gewinnen
will."'\footnote{Voegelin, Anamnesis, S. 313.}

\section{Kritik von Voegelins Realitätsbegriff}
 
Wie überzeugend ist nun Voegelins Vorstellung von Realität, von der
Notwendigkeit ihrer Anerkennung und von den Gefahren ihres Verlustes? Hier
stellt sich erstens die Frage der metaphysischen Wahrheit von Voegelins
Realitätsvorstellung: Gibt es wirklich ein transzendentes Sein, und beugt es
sich tatsächlich gnädig zum liebend hingerissenen Menschen hinab? Zweitens
stellt sich die Frage der Begründbarkeit von Voegelins Realitätsbild: Woher
wissen wir, dass die Realität so beschaffen ist, wie Voegelin es sagt? Kann die
Übereinstimmung mit der inneren Erfahrung auch dann noch ein hinreichendes
Kriterium für die Wahrheit des Voegelinschen Realitätsbildes sein, wenn man
wie Voegelin zugibt, dass es in Bezug auf diesen Gegenstand von einander
abweichende innere Erfahrungen gibt? Drittens stellt sich die Frage, ob der
Realitätsverlust, so wie ihn Voegelin versteht, in der Tat mit Notwendigkeit
oder wenigstens Wahrscheinlichkeit politische Unordnung nach sich zieht, und
ob umgekehrt die Anerkennung der Voegelinschen Seinsrealität für die
Errichtung politischer Ordnung in irgendeiner Weise vorteilhaft ist. Es
empfiehlt sich, die letzte dieser Fragen zuerst zu untersuchen, denn von der
Beantwortung dieser Frage hängt es ab, ob den anderen Fragen nur eine
theoretische Bedeutung oder auch eine praktisch-politische Dringlichkeit
zukommt.

\subsection{Die Verwechselung von gewöhnlichem und spirituellem Realitätsverlust}

Auf die Unklarheiten, die sich durch die unterschiedlichen Formen von
Realitätsverlust, von denen Voegelin spricht, ergeben, wurde bereits
hingewiesen. An dieser Stelle ist daher vor allem die grundsätzliche Frage zu
stellen, ob Realitätsverlust im Voegelinschen Sinne das politische Chaos nach
sich zieht?  Bei oberflächlicher Betrachtung könnte man geneigt sein, diese
Frage ohne jedes Zögern zu bejahen. Wenn die Bürger und insbesondere die
Politiker das Gefühl für die Grenzen ihrer Möglichkeiten verlieren, vollkommen
unrealistische Wünsche hegen oder gar utopisch-weltfremde Vorstellungen davon
haben, was überhaupt möglich ist, dann steht allerdings zu befürchten, dass
eine chaotische Politik dabei herauskommt.  Nach genauerer Untersuchung von
Voegelins Äußerungen stellt sich jedoch heraus, dass es gar nicht dies ist,
was er mit Realitätsverlust meint.  Unter Realitätsverlust versteht Voegelin
vielmehr die Nicht-Anerkennung einer bestimmten metaphysischen Seinsrealität
und insbesondere des "`Partizipierens"' des Menschen am transzendenten
göttlichen Seinsgrund.  Zur besseren Unterscheidung kann das, was Voegelin
unter Realitätsverlust versteht, als spiritueller Realitätsverlust bezeichnet
werden. Wie vehält sich nun der spirituelle Realitätsverlust zum gewöhnlichen
Realitätsverlust? Zieht ein spiritueller Realitätsverlust auch einen
Realitätsverlust auf pragmatischer Ebene nach sich? Diese Annahme ist wenig
einleuchtend. Warum sollte denn beispielsweise ein Mensch, der nicht an die
Existenz eines transzendenten Seins glaubt, weniger als andere Menschen dazu
in der Lage sein, die Grenzen des Möglichen zutreffend einzuschätzen?
Interessanterweise findet sich in Voegelins gesamten Werk kein einziger
stichhaltiger empirischer Beleg, der diese Annahme stützen könnte. Umgekehrt
spricht ebensowenig dafür, dass jemand, der über ein hohes Maß an spirituellem
Realitätssinn verfügt, bessere Voraussetzungen für das Verständnis oder die
Gestaltung der politischen Wirklichkeit mitbringt.  Voegelin leugnet auch
keineswegs, dass die richtige Gesinnung und eine erfolgreiche pragmatische
Politik nicht ein- und dasselbe sind. Wozu ist dann aber der richtige
spirituelle Realitätssinn überhaupt wichtig? Wenn Voegelin im Zusammenhang mit
dem Thema "`Realitätsverlust"' immer wieder auf die totalitären Herrschaften
anspielt, so liegt der Grund wohl darin, dass Voegelin sich von einer
Verbreitung des Empfindens für die spirituelle Realität eine besondere
immunisierende Wirkung gegen den Totalitarismus und totalitäre Demagogie
erhoffte. Womöglich ging Voegelin davon aus, dass die in der "`Spannung zum
Grund"' lebenden Menschen schon deshalb nicht auf den Totalitarismus
hereinfallen würden, weil die totalitäre Propaganda, der durch politische
Bildung und Aufklärung auf der Sachebene so schwer beizukommen ist, dann ihrem
innersten Lebensgefühl widersprechen würde.  Eine oberflächliche Plausibilität
kann man Voegelins Überlegung nicht absprechen. Nur vernachlässigt Voegelin
völlig, dass auch andere Existenzweisen als nur die Existenz in der "`Spannung
zum Grund"' oder ihre kompakten Vorstufen dies leisten können. Hier wäre etwa
an die Existenzweise eines Atheisten mit humanen moralischen Grundsätzen zu
denken. Voegelins Menschenkenntnis und psychologisches Einfühlungsvermögen
erweisen sich hier als außerordentlich engstirnig.
 
Die Schwierigkeiten, die bei Voegelin entstehen, wenn er die Notwendigkeit und
Geeignetheit spirituellen Wahrheitsbesitzes zur Bewältigung der
pragmatisch-politischen Realität begründen will, können auch als ein
theologisches Problem seines mystischen Gottesverständnisses gedeutet werden.
Denn dass das Leben nach den Gesetzen Gottes auch das pragmatisch klügste bzw.
richtigste ist, ergibt sich aus der konventionellen christlichen
Gottesauffassung zwanglos dadurch, dass Gott als allmächtiges Wesen die
Unterwerfung des Menschen honorieren kann, und dass er als gütiges und
allwissendes Wesen von vornherein vom Menschen nur fordert, was gut für ihn
ist. In Voegelins mystisch ausgedünntem Gottesverständnis bleibt von Gott
jedoch nur ein transzendentes Sein übrig (welches zudem bloß uneigenständiger
Pol einer Beziehung ist). Die Attribute der Allmacht und Allwissenheit
sind dadurch keineswegs mehr selbstverständlich gegeben. Lediglich
die Güte ist -- der von Voegelin beschriebenen Erfahrung des Hingezogenseins
nach zu urteilen -- noch vorhanden (wenn sie sich auch als Sirenengesang
erweisen kann, wie es die transzendente Variante der Gnosis vor Augen führt,
die sich bei Voegelin nicht auf einen falschen Gott sondern auf das richtige
transzendente Sein in der falschen Weise bezieht). Es fehlt bei diesem
ohnmächtigen transzendenten Sein aber jede Gewähr, dass die spirituell
richtige, nach der "`Spannung zum Seinsgrund"' ausgerichtete Existenzweise
auch in pragmatischer Hinsicht die richtige ist. Sie könnte ja auch genau das
Gegenteil davon sein.

\subsection{Die Zirkularität der Begründung von Voegelins Realitätsbegriff}

Als nicht weniger problematisch als der Zusammenhang von spirituellem
Realitätsverlust und politischem Chaos erweist sich die Begründungsproblematik
von Voegelins Realitätsbegriff. Woher kann man wissen, dass das, was Voegelin
über die metaphysische Seinsrealität sagt, wahr ist? Aus Voegelins
Gedankengang heraus müsste darauf die Antwort gegeben werden, dass sich diese
Wahrheit aus der Erfahrung ergibt, wobei unter Erfahrung nicht die
Sinneserfahrung sondern entweder jenes innere Erleben der "`noetischen"'
Erfahrung oder die mythische "`Primärerfahrung"' zu verstehen ist. Hier stellt
sich jedoch ein unlösbares Problem: Indem Voegelin zugibt, dass es
unterschiedliche Erfahrungen gibt, denen unterschiedliche Realitätsbilder
entsprechen, wie kann dann die Erfahrung noch ein Kriterium für die Wahrheit
(oder größere "`Differenziertheit"') einer bestimmten Auf\/fassung der
"`Realität"' abgeben? Auf diese Frage gibt Voegelins Bewusstseinsphilosophie
keine Antwort.  

% Auch der Begriff der Differenziertheit kann zur Beantwortung
% dieser Frage nicht herangezogen werden, denn dazu müsste er, soll ein
% Zirkelschluß vermieden werden, unabhängig von den Begriffen der Realität und
% der Erfahrung definiert werden. Damit bleibt aber nur noch ein rein formaler
% Differenzierungsbegriff übrig, der, wie im ersten Teil dieser Arbeit bereits
% ausgeführt, kaum zu wertenden Vergleichen herangezogen werden
% kann.\footnote{Zumindest liefert Voegelin keinerlei Anhaltspunkte dafür, wie
%   dies geschehen könnte. (Rein theoretisch ist es natürlich denkbar, daß der
%   Begründungsregreß an dieser Stelle oder an irgendeiner späteren mit einem
%   sinnvollen Kriterium abbricht, nur muß dieses Kriterium dann auch angegeben
%   werden und als sinnvoll oder evident ausgewiesen sein.)}

An anderer Stelle, in seinem Aufsatz "`Äquivalenz von Erfahrungen und Symbolen
in der Geschichte"', behauptet Voegelin, dass sich seine Aussagen über das
Wesen der Realität geschichtlich überprüfen lassen.\footnote{Eric Voegelin:
  Äquivalenz von Erfahrungen und Symbolen in der Geschichte, in: Eric
  Voegelin, Ordnung, Bewußtsein, Geschichte, Späte Schriften (Hrsg. von Peter
  J. Optiz), Stuttgart 1988, S. 99-126 (S. 109).} Die Aussagen dürfen nach
Voegelins Ansicht dann als gültig angesehen werden, wenn sie sich auf die
Geschichte beziehen, ohne "`einen erheblichen Teil des geschichtlichen Feldes
ignorieren oder im Dunkeln lassen"'\footnote{Ebd.} zu müssen, und wenn sie
"`erkennbar äquivalent mit den Symbolen [sind], die unsere Vorgänger in der
Suche nach der Wahrheit der menschlichen Existenz geschaffen
haben"'.\footnote{Ebd.} Dieses Prüfungskriterium ist offensichtlich zirkulär,
weil bereits zuvor bekannt sein müsste, welche Symbole der "`Vorgänger"' echte
Erfahrungssymbole sind, welche allein in die Prüfung einbezogen werden
dürfen.\footnote{Zur Zirkularität von Voegelins Begründung der Wahrheit
  bestimmter Symbolismen besonders deutlich: Vgl. Eugene Webb: Philosophers of
  Consciousness. Polanyi, Lonergan, Voegelin, Ricoeur, Girard, Kierkegaard,
  Seattle and London 1988, S. 126ff.} Dieser Zirkelschluss lässt sich auch
nicht zu einem hermeneutischen Verstehenszirkel erweitern, denn abgesehen
davon, dass der hermeneutische Zirkel höchstens die innere Folgerichtigkeit
der schrittweise verfeinerten Deutung gewährleistet, treten in Voegelins
Geschichtsbild zwei Symboltraditionen auf (die Tradition der echten Symbole
und die Tradition der Entgleisungen), die höchstwahrscheinlich beide die
Grundlage eines hermeneutischen Zirkels mit jeweils symmetrischen Stärken und
Schwächen bilden können. Darüber hinaus sind die Kriterien, die Voegelin
anführt, nur dann ihrem Zweck angemessen, wenn bereits zuvor als
metaphysisches Postulat vorausgesetzt wird, dass die Geschichte der Ausdruck
des Prozesses der Realität des Partizipierens ist, und dass die Symbole
Ausdruck der menschlichen Erfahrung des Partizipierens sind. Am Schluss des
Aufsatzes über die "`Äquivalenz von Erfahrungen und Symbolen in der
Geschichte"' gibt Voegelin dies auch ganz ungeniert zu.\footnote{Vgl. Eric
  Voegelin: Äquivalenz von Erfahrungen und Symbolen in der Geschichte, a.a.O.,
  S. 126.} Damit kann aber von einer historischen Prüfbarkeit seiner Aussagen
über die Realität keine Rede mehr sein.

Im Ergebnis stellt sich also heraus, dass es bereits {\it innerhalb} der
Voegelinschen Theorie weder möglich ist, die Realitätsadäquatheit von
Erfahrungen festzustellen, noch die Richtigkeit von Realitätsauf\/fassungen,
einschließlich der Realitätsauf\/fassung, die Voegelin selbst vertritt, zu
beurteilen.

\subsection{Die Fragwürdigkeit von Voegelins Seinserfahrung} 
\label{KritikVoegelinsSeinserfahrung}

Es bleibt schließlich zu überlegen, ob Voegelins Vorstellung von Realität
überhaupt der Wahrheit entspricht. Die richtige Art, diese Frage anzugehen,
bestünde zweifellos darin, zunächst zu untersuchen, ob ein transzendentes Sein
überhaupt existiert, und dann zu klären, ob es sich in der von Voegelin
behaupteten Beziehung zum Menschen befindet. Dieses Vorgehen würde jedoch
genau auf das hinauslaufen, was Voegelin als dogmatisches Missverständnis von
Symbolen, die Erfahrungen beschreiben, kritisiert. Da nun aber, unabhängig von
der Berechtigung eines solchen Vorwurfs, die Frage von Interesse ist, ob
Voegelin wenigstens nach seinen eigenen Maßstäben Recht behält, so empfiehlt
sich der Versuch, Voegelins Ansatz einmal naiv nachzuvollziehen, und über die
Frage zu meditieren, ob die Realität tatsächlich so erfahren wird, wie
Voegelin sie beschreibt. Auf diese Weise lässt sich außerdem klären, ob die
recht kritische Sicht von Voegelins Philosophie nur der in diesem Buch
verwendeten rationalistischen Methode zuzuschreiben ist, oder ob auch eine dem
Ideal der immanenten Kritik verpflichtete Herangehensweise zu kritischen
Resultaten kommen könnte. Im Folgenden erlaube ich mir daher das Protokoll
einer philosophischen Meditation über eine der Schlüsselpassagen aus Voegelins
Werk "`Anamnesis"' wiederzugeben.

Voegelin beschreibt die Erfahrung der Realität an einer Stelle seines
Vortrages "`Ewiges Sein in der Zeit"' mit den folgenden Worten:
 \begin{quote}\label{ZitatSeinserfahrung}
   Wie immer es um den Menschen als das Subjekt der Erfahrung bestellt sein
   möge, so erfährt er seelisch eine Spannung zwischen zwei Seinspolen, deren
   einer, genannt der zeitliche, in ihm selbst liegt, während der andere
   außerhalb seiner selbst liegt, jedoch nicht als Gegenstand im zeitlichen
   Sein identifiziert werden kann, sondern als ein Sein jenseits alles
   zeitlichen Seins der Welt erfahren wird. Vom zeitlichen Pol her wird die
   Spannung als ein liebendes und hoffendes Drängen zur Ewigkeit des
   Göttlichen erfahren; vom Pol des ewigen Seins her als ein gnadenhaftes
   Anrufen und Eindringen. Im Verlauf der Erfahrung wird weder das ewige Sein
   als ein Objekt in der Zeit gegenständlich, noch wird die erfahrende Seele
   aus ihrem zeitlichen in ewiges Sein transfiguriert; vielmehr ist der
   Verlauf zu charakterisieren als ein Sich-Ordnen und Sich-Ordnen-Lassen der
   Seele durch ihr liebendes Sich-Öffnen für das Eindringen des ewigen
   Seins.\footnote{Vgl. Voegelin, Anamnesis, S. 265. -- Vgl. Peter J. Opitz:
     Rückkehr zur Realität: Grundzüge der politischen Philosophie Eric
     Voegelins, in: Peter J.  Opitz / Gregor Sebba (Hrsg.): The Philosophy of
     Order. Essays on History, Consciousness and Politics, Stuttgart 1981,
     S. 57/58.}
 \end{quote}
 Wird die Realität tatsächlich in dieser Weise erfahren? Auf diese Frage ist
 natürlich nur eine subjektive Antwort möglich, aber für meinen Teil kann ich
 diese Frage doch ziemlich klar verneinen: Die Realität wird nicht als ein
 Partizipieren erfahren, in dessen Verlauf ein sich gnädig herabbeugendes
 transzendentes Sein in die liebend sich entgegendrängende Seele des Menschen
 eindringt. Die Welt fühlt sich einfach nicht so an, wie Voegelin es
 beschreibt! Schon die Zusammenstellung von Lieben und "`Sich-Ordnen-Lassen"'
 mutet, wie ich finde, grotesk an, und die Rede vom "`Eindringen des ewigen
 Seins"' in die sich öffnende und liebend entgegendrängende Seele kommt mir
 persönlich etwas geschmacklos vor. Kurzum, auch bei den intensivsten
 Meditationsbemühungen komme ich nicht dazu vom Pol des ewigen Seins her ein
 gnadenhaftes Anrufen und Eindringen zu erfahren.  Und ehrlich gesagt bin ich
 dem ewigen Sein, respektive Gott recht dankbar dafür, dass es mir die
 Peinlichkeit solcher Begegnungen erspart.

 Bei der Lektüre von Voegelin drängt sich mir häufig eine Frage auf, über die
 ich jedesmal den Kopf schütteln muss: Wollte Voegelin allen Ernstes den
 Menschen, die derartige Empfindungen nicht teilen, eine geschlossene Seele
 und eine existentielle Deformation ihrer selbst vorwerfen? Sind Menschen, die
 diese spezielle Art von Religiosität nicht für sich bejahen können, die sie
 vielleicht auch bewusst und explizit ablehnen tatsächlich {\em politisch}
 gefährlich und eine Bedrohung für die öffentliche Ordnung? Kaum zu fassen,
 dass Voegelin dergleichen ernsthaft als Wissenschaft verkaufen konnte! Und
 ebensowenig zu fassen, dass es Leute zu geben scheint, die ihm das
 abkaufen.\footnote{Darüber hinaus kann man die Frage aufwerfen, ob derartige
   mystische Ergüsse wirklich ein Zeichen besonderer seelischer Sensitivität
   sind, zu der nur Wenige in vollem Maße fähig sind, wie Voegelin wohl meinte
   (Vgl. Voegelin, Neue Wissenschaft der Politik, S. 172-174), oder ob sie
   nicht eher eine gewisse Form intellektueller Einfalt zur Voraussetzung
   haben. Dem psychologischen Scharfblick Tolstojs ist die Einsicht zu
   verdanken, dass das mystische Denken nicht, wie man vielleicht voreilig
   vermuten möchte, eine besondere Tiefe und Empfänglichkeit des Geistes und
   der Vorstellungskraft voraussetzt, sondern im Gegenteil auch auf einer
   ausgeprägten Oberflächlichkeit derselben beruhen kann. So charakterisiert
   Tolstoj in "`Anna Karenina"' die Hinwendung des betrogenen Alexej Karenin
   zu einer gerade in Mode gekommenen mystischen Richtung des Christentums mit
   folgenden Worten: "`Es fehlte ihm, gleich Lydia Iwanowna und den anderen
   Leuten, die derselben neuen Auffassung huldigten, jegliche Tiefe der
   Vorstellungskraft, jener geistigen Fähigkeit, dank welcher die durch die
   Phantasie hervorgerufenen Bilder mit dem Vorstellungskomplex und zugleich
   mit der Wirklichkeit im Einklang bleiben. Er sah nichts unmögliches und
   Absurdes in dem Gedanken, daß der Tod, der nur für die Ungläubigen
   existierte, für ihn nicht vorhanden sei und daß, da er den vollkommenen
   Glauben besaß, dessen Maß er im übrigen selbst bestimmte, auch für die
   Sünde in seiner Seele kein Raum sei und er daher schon hier auf Erden des
   Heils teilhaftig werde."' (Leo N.  Tolstoi: Anna Karenina, München 1992, S.
   511.) Besonderes der spätere Voegelin scheint mir eine ähnliche Entwicklung
   durchgemacht zu haben, wie Alexej Karenin in dem Roman (nur aus anderen
   Gründen, versteht sich).}
 
 Natürlich geben die vorstehenden Bemerkungen nichts weiter als meine
 persönliche Einstellung zum Thema "`Mystische Erfahrungen"' wieder. Der
 einzige Grund, aus dem ich sie hier anführe, besteht darin, dass für Voegelin
 im Zentrum der Philosophie immer eine persönliche Besinnung stehen sollte,
 bzw. wie Voegelin es selbst ausdrückte, dass die "`Basis für die Behandlung
 der philosophischen Problematik ..  selbstverständlich immer die
 Meditationspraxis sein"'\footnote{Franz-Martin Schmölz (Hrsg.): Das
   Naturrecht in der politischen Theorie, Wien 1963, S. 137. -- Der Band gibt
   die Vorträge und Diskussion einer Tagung zu dem Thema Naturrecht wieder.
   Die zitierte Äußerung Voegelins fällt in der Diskussion.} müsse. Ich bin
 dieser Forderung Voegelins an dieser Stelle einmal gefolgt erstens, um mir
 nicht mangelnde hermeneutische Sensibilität vorwerfen lassen zu müssen und
 zweitens, um zu zeigen, dass selbst wenn man die "`Meditationspraxis"' zur
 "`Basis für die Behandlung der philosophischen Problematik"'\footnote{Ebda.}
 nimmt, man noch längst nicht zu denselben Ergebnissen kommen muss wie Eric
 Voegelin.

 Voegelins intellektueller Kardinalfehler besteht darin, dass er nicht bereit
 ist, den rein religiösen Charakter seiner eigenen Vorstellung von der
 höchsten Realität einzugestehen und die Konsequenzen daraus zu ziehen. Er
 weigert sich zuzugeben, dass die Wahrheit seiner Auf\/fassung von der
 höchsten Realität wissenschaftlich nicht greifbar ist. Statt entsprechend
 behutsam damit umzugehen, setzt er die Wahrheit seiner Realitätsauf\/fassung
 absolut und zieht sie ohne Umstände als Verständnisgrundlage und als
 Bewertungsmaßstab aller anderen Weltanschauungen heran. Deutlich wird dies
 immer wieder an Urteilen wie diesem: "`Unter den Erfahrungen des
 Partizipierens schließlich hat die noetische dadurch ihren besonderen Rang,
 daß sie die Spannung zum göttlichen Grund nicht nur als Sachstruktur des
 Bewußtseins, sondern als die Grundspannung aller Realität, die nicht selbst
 der göttliche Grund ist, zur Klarheit bringt."'\footnote{Voegelin, Anamnesis,
   S. 304.} Sinnvoll ist ein solches Urteil nur, wenn als gegeben
 vorausgesetzt wird, dass "`die Spannung zum göttlichen Grund"' in der Tat
 "`die Grundspannung aller Realität"' ist, was aber, gerade weil es
 unterschiedlich erlebt wird, niemand mit Sicherheit behaupten kann.  Unter
 der Hand gerät Voegelin daher auch seine eigene Philosophie zu einem jener
 geschlossenen Dogmensysteme, die sich mit Hilfe intellektueller Tricks gegen
 jede Kritik abschirmen. Zwar beschreibt Voegelin die Realität als offen, aber
 seine Beschreibung der Realität ist ihrerseits ganz und gar nicht offen. Zu
 den intellektuellen Tricks, mit denen Voegelin seine Philosophie zu einem
 geschlossenen System abriegelt, gehört unter anderem die im folgenden zu
 beschreibende Theorie der sprachlichen Indizes, mit der er seinen, wie schon
 festgestellt wurde, sehr eigenwilligen Sprachgebrauch rechtfertigt.

\section{Die Theorie der sprachlichen Indizes}
\label{IndizesTheorie}
Die Theorie der sprachlichen Indizes beschreibt die sprachlichen
Eigentümlichkeiten der verbalen Wiedergabe noetischer Erfahrungen. Es geht
dabei um das Problem, die Besonderheit noetischer Beschreibungen zu erfassen,
denn rein äußerlich unterscheidet sich die sprachliche Wiedergabe echter
noetischer Erfahrungen durch nichts von der Sprache dogmatischer Metaphysik.
Außerdem versucht Voegelin, mit seiner Theorie der sprachlichen Indizes seinen
eigenen philosophischen Sprachgebrauch zu rechtfertigen und insbesondere das
Definitionsrecht bestimmter Begriffe (Welt, Mensch, Geschichte etc.) für sich
zu reklamieren.

Voegelin beginnt zunächst mit einer knappen Zusammenfassung der wichtigsten
Züge seines Realitätsbegriffs. Daran anknüpfend stellt er seine Theorie der
sprachlichen Indizes als eines Ausdruckes der (noetischen) Erfahrungen dieser
Realität vor. Schließlich zieht Voegelin aus dieser Theorie eine Reihe von
Schlussfolgerungen in Bezug auf die Politikwissenschaft, die menschliche Natur
und die Deutung der Geschichte.

Realität ist für Voegelin eine komplexe Beziehung von Mensch, Dingen und
Seinsgrund. Diese Beziehung wird vom Menschen nicht beobachtet, sondern "`
`von innen' "'\footnote{Vgl. Voegelin, Anamnesis, S. 316.} erfahren.
Ungeachtet dessen bleiben der Mensch und sein Leben jedoch in äußere
Zusammenhänge eingeordnet. Der Seinsgrund ragt durch das Bewusstsein des
Menschen in die Welt hinein, aber der Mensch kann sich nicht durch das
bewusste Partizipieren am Seinsgrund über die Welt hinausheben. (Voegelin baut
hier dem "`gnostischen"' Missverständnis der Möglichkeit einer Erlösung durch
Wissen vor.)  In der Erfahrung des Partizipierens gewinnen wir, Voegelin
zufolge, gültige "`Einsichten"' nicht nur in das Partizipieren selbst, sondern
auch in die Termini des Partizipierens, also beispielsweise in das Wesen des
Menschen und den Seinsgrund. Noetisches Wissen ist der unmittelbar den
"`Bewegungen"' des Partizipierens entspringende Ausdruck dieser
Einsichten.\footnote{Vgl.  Voegelin, Anamnesis, S. 315-316.}

An diesem Punkt führt Voegelin seine Theorie der sprachlichen Indizes ein.
Voegelin greift für diese Theorie eine Denkfigur auf, die er bereits in seinem
Aufsatz über die Struktur des Bewusstseins herangezogen hat, in welchem er die
These vertritt, dass das Bewusstsein nicht zeitlich, sondern durch
Erhellungsdimensionen strukturiert sei, die dann als "`Zukunft"' und
"`Vergangenheit"' sprachlich gekennzeichnet oder, wie Voegelin nun sagen
würde, {\it indiziert} werden.\footnote{Vgl. Voegelin, Anamnesis, S. 44. --
  Auf die philosophiehistorischen Zusammenhänge von Voegelins Ausführungen
  gehe ich, da ich eine {\em systematische} Kritik von Voegelins
  Bewusstseinsphilosophie beabsichtige, in diesem Buch meist nicht weiter ein.
  Trotzdem sei an dieser Stelle eine Vermutung geäußert: Voegelins
  Index-Theorie scheint dem Modell der Husserlschen Phänomenologie
  nachgebildet zu sein. Nur will Voegelin dann gewissermaßen den
  phänomenologischen Charakter als exklusives {\em Merkmal} der Beschreibung
  {\em noetischer} Erfahrungen verstanden wissen, statt, wie in der
  Phänomenologie als einen besonderen {\em Beschreibungsmodus} aller
  Erfahrungen. Durch diesen eklektischen Rückgriff auf eine Denkfigur aus
  einer durchgearbeiteten philosophischen Theorie handelt sich Voegelin dann
  jede Menge Inkonsequenzen und Widersprüche ein. Siehe dazu meine Kritik von
  Voegelins Theorie der sprachlichen Indizes in Abschnitt \ref{KritikSprache}
  weiter unten.} Die Theorie der Indizes besagt, dass die sprachlichen
Ausdrücke, mit denen die noetischen Erfahrungen artikuliert werden, nicht
gegenständlich als Aussagen über etwas sondern als Kennzeichnung von inneren
Erfahrungen bzw.  Erlebnissen verstanden werden müssen. Dies gilt, obwohl
diese sprachlichen Ausdrücke ihrer äußeren Form nach gegenstandsförmlich sind.
So wäre also etwa der Satz: "`In der noetischen Erfahrung dringt der
transzendente Seinsgrund in das Bewußtsein ein"' nicht als Aussage über das
transzendente Sein und das menschliche Bewusstsein zu verstehen, sondern als
Kennzeichnung einer inneren Erfahrung des Eindringens, die offenbar von
solcher Intensität und Eigenart ist, dass zu ihrem angemessenen Ausdruck vom
"`Eindringen des transzendenten Seins"' gesprochen werden muss. Warum aber
müssen die noetischen Erfahrungen überhaupt gegenständlich ausgedrückt werden,
wenn dies doch so missverständlich ist?  Voegelin glaubt, dass es zum
gegenständlichen Ausdruck keine Alternative gibt, "`weil das Bewußtsein
gegenstandsförmlich ist"'.\footnote{Voegelin, Anamnesis, S. 316.} Unter der
Gegenstandsförmlichkeit des Bewusstseins versteht Voegelin dabei, dass
"`Bewußtsein [..] immer Bewußtsein-von-Etwas ist"'\footnote{Voegelin,
  Anamnesis, S. 307.}

Schwere Fehler und Missverständnisse ergeben sich nach Voegelins Ansicht, wenn
Ausdrücke, die Indizes von Bewusstseinserfahrungen sind, unabhängig von diesen
Erfahrungen als Begriffe für etwas eigenständig Seiendes verwendet werden.
Voegelin illustriert dies an einer Reihe von Beispielen. So gibt es für
Voegelin "`weder eine immanente Welt noch ein transzendentes Sein als
Entitäten"',\footnote{Voegelin, Anamnesis, S. 316.} vielmehr sind die Ausdrücke
"`immanent und "`transzendent"' Indizes, welche Bereichen der Erfahrung
zugeteilt werden. Nach Voegelins Überzeugung ist es daher unsinnig, über die
Existenz von transzendentem oder immanentem Sein zu streiten. Weiterhin ist
Voegelin der Ansicht, dass der Ausdruck Mensch wenigstens in bestimmter
Hinsicht einen Index der Erfahrung darstellt, denn unter "`Mensch"' ist auch
"`der immanente Pol der existenziellen Spannung zum Grund zu
verstehen"'.\footnote{Voegelin, Anamnesis, S. 317.} Da außerdem nach Voegelins
Ansicht auch der Ausdruck "`Philosophie"' ein Index der Erfahrung ist, so
glaubt Voegelin folgern zu können, dass es unmöglich ist, den Menschen im
Rahmen einer philosophischen Anthropologie ausschließlich als welt-immanentes
Wesen zu verstehen. In der Vernachlässigung dieses Grundsatzes in der
Anthropologie erblickt Voegelin nicht bloß einen philosophischen
Irrtum, wie er beim Nachdenken schon einmal unterlaufen könnte, sondern eine
Form von Realitätsverlust.\footnote{Vgl. Voegelin, Anamnesis, S. 316-317.}

Aus der Theorie der sprachlichen Indizes folgt für Voegelin eine Reihe von
Konsequenzen, die überwiegend bereits gewonnene Einsichten bekräftigen und
vertiefen. Die erste Konsequenz ergibt sich hinsichtlich des Begriffes der
Wissenschaft. "`Wissenschaft"' ist für Voegelin ebenfalls ein Index. Sie
entdeckt "`sich selbst als das Strukturwissen von Realität, wenn die
Selbsterhellung des Bewußtseins und seiner Ratio sich historisch
ereignet"',\footnote{Voegelin, Anamnesis, S. 318. -- Dass Wissenschaft
  ebenfalls ein Index sein soll, verblüfft auf den ersten Blick, denn
  Wissenschaft ist primär eine menschliche Tätigkeit und nicht etwas, das
  erfahren wird, so dass man an dieser Stelle einen Kategorienfehler
  Voegelins vermuten muss. Es sei denn, Voegelin wollte die recht abwegige
  Ansicht vertreten, dass Wissenschaft in erster Linie aus der Selbsterfahrung
  des wissenschaftlichen Denkens entsteht.} wobei in Erinnerung zu rufen ist,
dass Voegelin unter "`Ratio"' die zum Seinsgrund hin geöffnete Seele versteht
und nicht etwa Vernunft oder Verstand im gewöhnlichen Sinne. Dieses
historische Ereignis hat, Voegelin zufolge, bei Platon und Aristoteles
stattgefunden, deren Noese "`die Indizes Wissenschaft ({\it episteme}) und
Theorie ({\it theoria}) entwickelt hat."'\footnote{Voegelin, Anamnesis,
  S. 318.} Selbst die moderne Naturwissenschaft verdankt nach Voegelins Ansicht
ihren Wissenschaftscharakter weniger dem Erfolg ihrer Methoden als vielmehr
der Tatsache, dass ihre Methoden mit der "`Ratio der Noese verträglich
sind."'\footnote{Voegelin, Anamnesis, S. 318.} Erst die Noese legt nämlich die
"`Welt"', welche wiederum ein sprachlicher Index des Bewusstseins ist, als ein
von mythischen und anderen Glaubenselementen gereinigtes Feld für die
Bearbeitung durch die Naturwissenschaft frei.\footnote{Vgl. Voegelin,
  Anamnesis, S. 318.}

Um über Partizipationserfahrungen angemessen reden zu können, genügen
allerdings nicht allein die sprachlichen Indizes, welche diese Erfahrungen
selbst ausdrücken. Es ist darüber hinaus eine Art von Begriffen notwendig, mit
denen {\it über} diese Erfahrungen gesprochen werden kann. Diese Begriffe
bezeichnet Voegelin als Typenbegriffe. Als historische Beispiele für
Typenbegriffe führt Voegelin die Ausdrücke "`philodoxos"' und "`sophistes"'
von Platon und die Ausdrücke "`philosophos"' und "`philomythos"' von
Aristoteles an. Unter seinen eigenen Begriffen rechnet Voegelin unter anderem
die Begriffe der "`kompakten und differenzierten Erfahrungen"' und der
"`noetischen und revelatorischen Transzendenzerfahrungen"' zu den
Typenbegriffen.\footnote{Vgl. Voegelin, Anamnesis, S. 319.} Die
Erforderlichkeit von Typenbegriffen wird besonders dann akut, wenn infolge
geistesgeschichtlicher Differenzierungsprozesse die kompakteren
Partizipationserfahrungen in eine Rolle relativer Unwahrheit gedrängt werden,
so dass ihr symbolischer Selbstausdruck nicht mehr zählt und Begriffe gefunden
werden müssen, um die kompakten Erfahrungen angemessen bezeichnen zu können.

Im \label{HistorischerKollektivismus}Zusammenhang mit der geschichtlichen
Entwicklung von Partizipationserfahrungen kommt Voegelin auf das Problem der
Beziehung des überindividuellen Prozesses der Geschichte zum individuellen
Bewusstsein zu sprechen, welches nach Voegelins Auffassung durch seine
Transzendenzerfahrungen der Träger dieses Prozesses ist. Für Voegelin gibt es
Bewusstsein ausschließlich in der Form des konkreten Bewusstseins einzelner
Individuen. Es ist "`diskret real"'.\footnote{Voegelin, Anamnesis, S.320.}
Wie können aber die individuellen Transzendenzerfahrungen der vielen diskret
realen Bewusstseine innerhalb eines sinnhaften historischen Prozesses oder
Feldes der Geschichte verortet werden, von dessen Existenz Voegelin nach wie
vor überzeugt ist?\footnote{Auch nachdem Voegelin die Auffassung einer
  linearen Geschichtsentwicklung aufgegeben hat (Vgl. Eric Voegelin:
  Historiogenesis, in: Voegelin, Anamnesis, S. 79-116.), hält er dennoch daran
  fest, dass das "`Feld der Geschichte"' prozesshaft geordnet ist.  (Vgl. Eric
  Voegelin: Ewiges Sein in der Zeit, in: Voegelin, Anamnesis, S. 254-280.)}
Voegelin beantwortet diese Frage damit, dass in den Transzendenzerfahrungen der
vielen Bewusstseine stets ein und derselbe transzendente Seinsgrund erfahren
wird: "`Geschichte wird zu einem strukturell verstehbaren Feld der Realität
durch die Präsenz des einen Grundes, an dem alle Menschen
partizipieren..."'.\footnote{Voegelin, Anamnesis, S. 320.} Keinesfalls kann
dagegen die Geschichte (wie etwa bei Hegel) als die Entfaltung eines
kollektiv-überindividuellen oder gar absoluten Bewusstseins verstanden werden,
da hierbei vollkommen ignoriert wird, dass Bewusstsein nur als das Bewusstsein
einzelner Menschen vorkommt.\footnote{Vgl.  Voegelin, Anamnesis, S. 320-321.}

Schließlich weist Voegelin noch auf die problematischen Folgen hin, die
aus dem unsachgemäßen Gebrauch von Typenbegriffen entstehen.
Typenbegriffe dürfen, so scheint es Voegelin aufzufassen, legitimerweise
nur dann eingesetzt werden, wenn ihr Gebrauch durch eine eigene
noetische Erfahrung gedeckt ist, durch welche allein die weniger
differenzierten Erfahrungen richtigerweise als Typen von relativ
geringerem Wahrheitsgrad erkannt werden können. Dazu muss außerdem hinter
jedem Typus die je eigene Erfahrungsgrundlage dieses Typus erkannt
werden. (Voegelin greift hier auf das bereits in der "`Neuen
Wissenschaft der Politik"' entwickelte Prinzip zurück, dass die
Erfahrungen und nicht die Ideen "`die Substanz der Geschichte"' bilden.)
Die Vernachlässigung dieser Prinzipien führt nach Voegelins Ansicht zu
unersprießlichen Dogmenstreitereien zwischen sich gegenseitig
typisierend einordnenden Meinungen, die bis zum allgemeinen
Ideologieverdacht ausarten können, ohne dass jemals die entscheidende
Ebene der Transzendenzerfahrungen auch nur in den Blick
gerät.\footnote{Vgl. Voegelin, S. 321-323.} Voegelin gesteht sich nicht
ein, dass seine Theorie auch nur eine weitere Position in der
wissenschaftlichen Auseinandersetzung der Theorien darstellt (was auch
dann der Fall wäre, wenn sie tatsächlich und als einzige von allen
Theorien wahr wäre), und dass er durch seine polemischen Ausfälle selbst
nicht wenig zum allgemeinen Ideologieverdacht beiträgt.

\section{Kritik von Voegelins Sprachtheorie}
\label{KritikSprache}

Die Theorie der sprachlichen Indizes erweist sich in vielerlei Hinsicht als
höchst unglaubwürdig und zweifelhaft. Dies beginnt schon mit den
Voraussetzungen der Theorie: Voegelins Theorie der sprachlichen Indizes
stellt eine Theorie über die Bedeutung bestimmter sprachlicher Ausdrücke dar.
Sie besagt, dass bestimmte sprachliche Äußerungen, obwohl sie von ihrer Form
her Aussagen über Gegenstände sind, dennoch eine andere Bedeutung haben, die
Bedeutung eines reinen Ausdruckes von inneren Bewusstseinserfahrungen. Warum
wird aber der Ausdruck von inneren Erlebnissen in die Form gegenständlicher
Aussagen gepresst? Voegelins Antwort lautet: Das Bewusstsein ist
gegenständlich, und weil das Bewusstsein gegenstandsförmlich ist, können
Bewusstseinserfahrungen nicht anders als in der uneigentlichen Form
gegenständlicher Aussagen artikuliert werden. Gegen diese Antwort liegen die
Einwände jedoch auf der Hand: Ungeachtet der Gegenstandsförmlichlichkeit oder
Nicht-Gegenstandsförmlichkeit des Bewusstseins ist es ohne Weiteres möglich,
den Ausdruck von Erfahrungen als solchen sprachlich kenntlich zu machen und
von Aussagen über Dinge zu unterscheiden, indem man z.B. Sätze von der Form
"`Ich hatte die Erfahrung, dass..."' oder "`Ich hatte ein Gefühl, als ob..."'
bildet. Voegelin macht es ja selber vor, wenn er in der weiter oben bereits
zitierten Passage\footnote{Siehe Seite \pageref{ZitatSeinserfahrung}}
schreibt: "`Vom zeitlichen Pol her wird die Spannung als ein liebendes und
hoffendes Drängen zur Ewigkeit des Göttlichen erfahren"'.\footnote{Voegelin,
  Anamnesis, S. 265.} Niemand wird das als eine gegenständliche Aussage
missverstehen. Wie dieses Beispiel gleichfalls vor Augen führt, wird die
Möglichkeit, Erfahrungen als Erfahrungen sprachlich zu artikulieren, auch
nicht durch die "`gegenständliche"' Subjekt-Prädikats-Form eingeschränkt,
welche die Grammatik den Sätzen unserer Sprache
vorschreibt.\footnote{Derartiges deutet Voegelin in seinem Aufsatz "`Ewiges
  Sein in der Zeit"' an, worin die Theorie der sprachlichen Indizes ebenfalls
  angesprochen wird. Vgl. Voegelin, Anamnesis, S. 266.} Abgesehen von diesen
Einwänden kann es gar nicht ohne Weiteres als ausgemacht gelten, dass das
Bewusstsein in jeder Hinsicht als gegenstandsförmlich aufzufassen ist.  Zwar
haben die meisten Bewusstseinsvorgänge (z.B. Wahrnehmen, Denken, Fühlen) die
Form intentionaler Akte, indem sich in ihnen ein Subjekt durch einen
Bewusstseinsakt auf einen Gegenstand des Bewusstseins bezieht.  Aber wie
verhält es sich beispielsweise mit Stimmungen? Zudem wäre es auch gar nicht
ausdenklich, wie ein rein gegenstandsförmliches Bewusstsein
Transzendenzerfahrungen haben könnte, sofern diese Erfahrungen
ungegenständlich sind.
% Wenn sie aber doch gegenständlich
% sind\footnote{Was Voegelin jedoch eindeutig leugnet vgl. Voegelin, Anamnesis,
%   S.287.}, warum darf dann das in ihnen Erfahrene nicht wie ein Gegenstand
% diskutiert werden?

Doch Voegelin geht nicht nur von falschen Voraussetzungen aus. Seine Theorie
wirkt auch deshalb unglaubwürdig, weil er sich selbst nicht an die von ihm
gezogenen Grenzen hält. So bestreitet Voegelin zwar entschieden, dass eine
Diskussion über die Existenz von immanenter Welt und transzendentem Sein als
Entitäten sinnvoll ist, aber wenn das transzendente Sein als reiner Index des
Bewusstseins verstanden werden müsste, dann wäre Voegelins Realitätsbegriff in
der zuvor von ihm beschriebenen Form kaum noch haltbar. Voegelin behauptet ja
gerade, dass die Realität des Partizipierens (am transzendenten Seinsgrund)
auch dann noch bestehen bleibt, wenn die Erfahrung des Partizipierens
verlorengegangen ist oder geleugnet wird. Wenn aber der Seinsgrund nur Index
des Bewusstseins wäre, dann würde es auch kein Partizipieren ohne die
Bewusstseinserfahrung des Partizipierens geben können.

Schon von vornherein ließe sich gegen die Theorie der sprachlichen Indizes
eben jener ontologische Vorbehalt geltend machen, den Voegelin am Ende seines
Aufsatzes "`Zur Struktur des Bewußtseins"' gegenüber der reinen
Bewusstseinsphilosophie vertritt, dass es in erster Linie auf das Sein und
nicht auf das Bewusstsein ankommt.\footnote{Vgl. Voegelin, Anamnesis, S. 56.}
Wie leicht sich Voegelins Theorie der sprachlichen Indizes aushebeln lässt,
wenn man die Indizes als Kennzeichnung reiner Erfahrungsbereiche auf\/fasst,
kann an Voegelins Behauptung demonstriert werden, dass man den Menschen
innerhalb einer philosophischen Anthropologie nicht angemessen als
welt-immanentes Wesen verstehen könne. Diese Behauptung stellt sich bei
genauerem Hinsehen als weit anspruchsloser heraus, als sie auf den ersten Blick
erscheint. Denn da "`welt-immanent"' für Voegelin lediglich ein Index der
Erfahrung ist, so beinhaltet diese Behauptung nur, dass nicht geleugnet werden
darf, dass es Menschen gibt, die innere Erlebnisse haben, zu deren Ausdruck
sie sich genötigt fühlen, Worte wie "`immanent"' und "`transzendent"' zu
verwenden.\footnote{Für den Fall, dass Voegelin so interpretiert werden
  müsste, dass nach seiner Theorie alle Menschen Transzendenzerlebnisse
  hätten, kann statt "`daß es Menschen gibt, die innere Erlebnisse haben,..."'
  genausogut "`daß alle Menschen innere Erlebnisse haben,..."' eingesetzt
  werden. Das nachfolgende Argument bleibt dann immer noch gültig. Allerdings
  hätte dann auch der Materialist, der das transzendente Sein leugnet
  Transzendenzerlebnisse, was ihn jedoch nicht hindern muss ihre Wirklichkeit
  zu leugnen.} Dies nicht zu leugnen dürfte allerdings auch dem
hartgesottensten Materialisten keinerlei Sorgen bereiten, da er dadurch ja
noch längst nicht genötigt ist zuzugeben, dass es ein transzendentes Sein
tatsächlich gibt. Ja er könnte unter Berufung auf Voegelins Theorie der
sprachlichen Indizes sogar ausdrücklich darauf verweisen, dass es illegitim
sei, von einer inneren Erfahrung, die als Erfahrung von Transzendenz
sprachlich indiziert wird, auf die Existenz eines transzendenten Seins zu
schließen. Voegelins Theorie gleicht daher -- um ein Bild von Schopenhauer zu
entlehnen -- einer Grenzfeste, die zwar uneinnehmbar ist, deren Besatzung aber
auch nicht in der Lage ist auszubrechen, so dass man sie getrost im Hinterland
zurücklassen kann.

Wenn die sprachlichen Indizes überhaupt irgendeinem Zweck dienen sollen,
so sind wir also gezwungen, hinter ihnen die Existenz von Entitäten
anzunehmen, auf welche sie verweisen. Voegelins Theorie der sprachlichen
Indizes hätte dann immer noch dadurch ihren guten Sinn, dass sie es verbietet,
sich bei der Diskussion über die Indizes von den Erfahrungen zu lösen, in
denen diese Entitäten mutmaßlich zum Vorschein kommen. Zur
erkenntnistheoretischen Rechtfertigung von Aussagen über die Transzendenz taugt
die Theorie der sprachlichen Indizes dann allerdings nicht mehr.

Kaum noch rechtfertigen lässt sich Voegelins Theorie der sprachlichen Indizes
jedoch dort, wo seine Indizes mit herkömmlichen Begriffen konkurrieren, wie
dies bei dem Begriff der Wissenschaft der Fall ist. Zwar ist es erfreulich zu
hören, dass die Methoden der modernen Naturwissenschaft "`mit der Ratio der
Noese verträglich sind"'.\footnote{Voegelin, Anamnesis, S. 318.} Aber da das
Gelingen der Naturwissenschaft selbstverständlich in keiner Weise davon
abhängt, ob ihre Methoden mit der bewusst gewordenen existentiellen Spannung
zum Grund vereinbar sind, so ist es -- jedenfalls soweit es um die
Naturwissenschaften geht -- wenig sinnvoll, die Definition des Begriffes
Wissenschaft an die "`Platonisch-Aristotelische Noese"' zu knüpfen, zumal sich
die experimentelle Naturwissenschaft von der platonischen und aristotelischen
{\it episteme} sehr erheblich unterscheidet. Nicht ganz unzweifelhaft
erscheint auch die These, dass die Beseitigung "`mythische[r],
revelatorische[r] oder ideologische[r]
Wahrheitshypotheken"'\footnote{Voegelin, Anamnesis, S. 318.}  durch die Noese
eine historische Ermöglichungsbedingung der Naturwissenschaft darstellt. Die
Anfänge der Naturwissenschaft fallen bereits in prä-noetische Zeit. So wurde
die Entwicklung der Astronomie durch das kosmologische Weltbild nicht etwa
behindert, sondern eher noch gefördert. Und bereits am Beispiel des Thales
lässt sich -- stellvertretend für die Vorsokratiker insgesamt --
veranschaulichen, dass die Entgöttlichung der Welt, anders als dies
gelegentlich zu hören ist,\footnote{Vgl. Eric Voegelin: Die geistige und
  politische Zukunft der westlichen Welt (Hrsg. von Peter J.  Opitz und
  Dietmar Herz), München 1996, S. 26/27.} keine notwendige Voraussetzung für
die Entfaltung unbefangenen naturwissenschaftlichen Forschergeistes darstellt,
denn Thales hinderte die Überzeugung, dass alles von Göttern erfüllt sei,
nicht daran, dieser Deutung die materialistische Erklärung hinzuzufügen, dass
alles auf und aus Wasser sei. Monotheismus oder Atheismus ist entgegen der
Entgöttlichungsthese keinesfalls eine notwendige Voraussetzung für die
Entstehung von Wissenschaft.

Aber auch wenn man sich nicht auf die Naturwissenschaften beschränkt, so kann
Voegelins Definitionsversuch, nach welchem Wissenschaft dasjenige ist, was
sich als "`das Strukturwissen von Realität"' infolge der sich historisch
ereignenden "`Selbsterhellung des Bewußtseins"' selbst
entdeckt,\footnote{Voegelin, Anamnesis, S. 318.} nicht ohne Weiteres
überzeugen. Ob irgendeine menschliche Erkenntnisaktitivität als Wissenschaft
eingestuft werden kann oder nicht, hängt weder von dem Selbstverständnis
derjenigen ab, die diese Aktivität ausüben (auch Alchemisten, Astrologen und
Naturheiler hielten und halten sich schließlich für Wissenschaftler), noch
hängt es von den historischen Rahmenbedingungen ab, unter denen diese
Erkenntnisaktivität entstanden ist. Entscheidend ist einzig und allein die
Frage, ob bei dieser Erkenntnisaktivität eine Welterkenntnis von objektiver
und nachprüfbarer Gültigkeit herauskommt. Eine Vorentscheidung über ein
bestimmtes, etwa mathematisch-naturwissenschaftliches Wissenschaftsmodell ist
mit diesem Kriterium noch nicht getroffen, so dass Voegelins Vorbehalten
gegenüber einer zu engen Wissenschaftsauf\/fassung Rechnung getragen werden
kann. Nur wenn die historische {\it episteme} des Aristoteles nicht schon per
definitionem mit Wissenschaft gleichgesetzt wird, lässt sich außerdem die
wichtige wissenschaftshistorische Frage aufwerfen, ob und in welchem Maße die
aristotelische {\it episteme} tatsächlich Wissenschaft ist.

Auf \label{KritikHistorischerKollektivismus} die Fragwürdigkeit der noetischen
Definition von Voegelins Begriff der Geschichte wurde bereits im
vorhergehenden Abschnitt hingewiesen. Voegelins Vorwurf gegen die
hegelianischen Geschichtskonstruktionen, dass sie fälschlicherweise ein reales
Kollektivbewusstseins zu Grunde legen würden, während Bewusstsein in
Wirklichkeit nur "`diskret real"' vorkomme, ist dagegen vollkommen berechtigt.
Nur stellt sich die Frage, ob Voegelin nicht seinerseits auf einer anderen
Ebene den hegelianischen Geschichtskonstruktionen nahekommt, wenn er darauf
besteht, dass "`Geschichte .. zu einem strukturell verstehbaren Feld der
Realität durch die Präsenz des einen Grundes"'\footnote{Voegelin, Anamnesis,
  S.320. -- Vgl. auch Voegelin, Order and History IV, S.305.} wird, welches
sich nicht in Einzelvorstellungen auflösen ließe. Hier ist anzumerken, dass
erstens nach wie vor jeder Anhaltspunkt für die Richtigkeit der Annahme fehlt,
dass es einen transzendenten Seinsgrund gibt, und dass dieser ein einziger ist.
Zweitens operiert Voegelin mit der falschen Alternative, dass entweder ein
gemeinsamer Grund existieren müsse, oder nur "`jeder ein privates -- im
klassischen Sinne von `idiotisches' -- Bewußtsein für sich
selbst"'\footnote{Voegelin, Anamnesis, S.320.} hätte. Auch wenn es keinen
gemeinsamen transzendenten Seinsgrund gibt, so können doch die diskret realen
"`Bewusstseine"' durch Miteinander-Reden, durch Einfühlung, Mitleid und
teilnehmende Freude, durch gemeinsame Erlebnisse und gemeinsames Handeln auf
das Schönste zu einander in Kontakt treten. Und drittens bleibt hinsichtlich
der Bedeutung der Geschichte für den Menschen und die Menschheit anzumerken,
dass, solange in der Geschichte nicht irgendeine Form religiöser Erbauung
gesucht wird,\footnote{Vgl. dazu Poppers an den Theologen Karl Barth
  anknüpfende Kritik der theogonischen Geschichtsdeutung, in: Karl Popper: Die
  offene Gesellschaft und ihre Feine. Band II. Falsche Propheten: Hegel, Marx
  und die Folgen, 7.Aufl., Tübingen 1992, S. 316-328.} niemandem etwas
entgeht, wenn sich herausstellt, dass Geschichte kein "`Feld der Realität [ist]
..., an dem alle Menschen partizipieren"'.\footnote{Voegelin, Anamnesis,
  S. 320.}

Im Ganzen stellt Voegelins Theorie der Indizes ein sehr fragwürdiges
Unterfangen dar. Sie geht nicht nur von falschen Voraussetzungen bezüglich der
Natur des menschlichen Bewusstseins und der Sprache aus, sondern sie führt als
ein Verfahren der Begriffsklärung oft zu recht willkürlichen Definitionen
zentraler Begriffe wie z.B. Wissenschaft, Geschichte, Rationalität, Realität.
Es fällt nicht leicht, sich hierbei des Eindrucks zu erwehren, dass Voegelin
versucht, höchst strittige Sachfragen (wie z.B. ob Rationalität in
der Spannung zum Grund besteht, ob Realität in erster Linie spirituelle
Realität ist, ob Geschichte sich nicht in der Zeit sondern in der
Bewusstseinsdimension des Begehrens und der Suche nach dem Grund abspielt etc.)
durch Definitionen vorzuentscheiden und ihre Diskussion dadurch zu verhindern,
dass er von vornherein alle zentralen Begriffe für sich reklamiert, so dass
die Formulierung von Kritik erheblich erschwert wird.

%%% Local Variables: 
%%% mode: latex 
%%% TeX-master: "Main" 
%%% End:







%%% Local Variables: 
%%% mode: latex
%%% TeX-master: "Main"
%%% End: 

\subsection{Die Stufen des Ordnungswissens}

Im vierten, mit "`Die Spannungen in der Wissensrealität"'\footnote{Voegelin,
  Anamnesis, S.323.} betitelten Abschnitt seines Aufsatzes untersucht Voegelin
die Beziehung zwischen den verschiedenen Stufen des Ordnungswissens, wozu die
pränoetische, die noetische und die Verfallsstufe des Wissens von der
richtigen Ordnung zählen.

Nachdem Voegelin in einer kurzen Einleitung noch einmal auf die Weisen des
Ordnungswissens und ihre gegenseitige Beziehung im Prozess der noetischen
Differenzierung eingegangen ist, stellt er zunächst in groben Zügen die
historische Entwicklung von prä-noetischem über das noetische Ordnungswissen
bis hin zur Entgleisung des Ordnungswissens in der Gegenwart dar. Daraufhin
erörtert Voegelin ausführlich, wie seiner Ansicht nach das verlorengegangene
Ordnungswissen in der Gegenwart wieder hergestellt werden kann. Als Letztes
geht Voegelin mit dem mystischen Denken Jean Bodins und Henri Bergsons auf
zwei mögliche historische Anknüpfungspunkte jüngeren Datums zur
Wiedergewinnung des Ordnungswissens ein.

Wie bereits ausgeführt, tritt Ordnungswissen Voegelin zufolge zunächst in
einer prä-noetischen, mythischen Form auf. Durch die Noese wird das
prä-noetische Ordnungswissen ergänzt und vertieft. Erst auf der Stufe des
noetischen Ordnungswissens wird sich der Mensch seiner Existenz in der
"`Spannung zum Grund"' bewußt, so daß das Ordnungswissen nun einer expliziten
Kontrolle aus dem Wissen um die "`Spannung zum Grund"' heraus unterliegt
("`explizit-rationale Kontrolle"'\footnote{Voegelin, Anamnesis, S.325. - Es
  ist ziemlich wahrscheinlich, daß Voegelin in dieser Passage (S.323-325.) den
  Ausdruck "`rational"' doppeldeutig gebraucht: einmal im Sinne seiner
  Definition von "Ratio"' als "`der Spannung des Bewußtseins zum Grund"'
  (S.289.), dann aber auch in dem - dem gewöhnlichen Wortgebrauch
  näherkommenden - Sinn von "`begriffliches Wissen"' bzw. "`begriffliches
  Denken"'.}). Unter Rückgriff auf das bewußt gewordene Wissen von der
"`Spannung zum Grund"' ist es zwar möglich, das prä-noetische oder auch das
entgleiste Ordnungswissen in Frage zu stellen. Dennoch bleibt das noetische
Ordnungswissen auf die prä-noetischen Wissensbestände, die es erweitert aber
nicht ersetzt, sachlich angewiesen. Auch auf der gesellschaftlichen Ebene, wo
sich das noetische zum kompakten Ordnungswissen ähnlich wie die Theologie zum
Volksglauben verhält, kann es das prä-noetische Wissen niemals gänzlich
verdrängen.\footnote{Vgl. Voegelin, Anamnesis, S.323-325.}

Die historische Entwicklung des Ordnungswissens, die Voegelin im folgenden
skizziert, kann als eine Bewegung in dialektischen Dreischritten gedeutet
werden: Auf eine Phase heilen Ordnungswissens prä-noetischer oder noetischer
Art folgt als deren Antithese eine dogmatische Entgleisung oder, wie Voegelin
auch sagt, eine "`Parekbasis"' falschen Ordnungswissens, welche dann durch die
Noese aufgehoben wird. Aber auch das noetische Ordnungswissen entgleist zum
Dogmatismus, so daß eine weitere Noese - denn ein höheres als das noetische
Ordnungswissen gibt es nicht - vonnöten ist, die wiederum diese Entgleisung 
aufhebt.

In dieser Weise folgt nach Voegelins Geschichtsbild auf die griechische
Mythologie und die "`Parekbasis der Sophistik"' die "`klassische
Noese"'\footnote{Voegelin, Anamnesis, S.325.} der nach-sokratischen
Philosophie. Nicht zuletzt, weil sie politisch mit der Polisgesellschaft für
das in den Eroberungszügen Alexanders des Großen unterlegene Modell optierte,
war der klassischen Noese kein langfristiger Erfolg beschieden, und sie
entgleiste ihrerseits "`zur philosophischen Dogmatik der
Schulen"'.\footnote{Voegelin, Anamnesis, S.326.} Obwohl zur "`Dogmatik der
Schulen"' herabgekommen, taugte, wenn man Voegelins Worten Glauben schenken
darf, das als "`Parekbasis einer Noese ...  charakteriesierte
Phänomen"'\footnote{Voegelin, Anamnesis, S.326.} dennoch dazu, gegenüber der
jüdisch-christlichen Offenbarungsreligion als "`Repräsentant der
Noese"'\footnote{Voegelin, Anamnesis, S.326.} zu fungieren. Da im
jüdisch-christlichen Kontext der Übergang von der vor-noetischen
Offenbarungsweisheit zum noetischen Ordnungswissen weniger schroff verlief,
kam es dabei nicht wie im antiken Griechenland zum radikalen Bruch mit dem
traditionellen Ordnungswissen. Vielmehr verschmolz die Philosophie mit dem
traditionellen Ordnungswissen der Offenbarung zur Theologie. In dieser Form
hat die Noese zwar bei der Bekämpfung der Häresien (zur der Voegelin schon in
früheren Schriften der katholischen Kirche das volle historische Recht
zuerkennt\footnote{Vgl. Eric Voegelin: Das Volk Gottes. Sektenbewegungen
  und der Geist der Moderne (Hrsg. von Peter J. Opitz), München 1994, S.31-34.
  - Voegelin setzt in dieser Schrift recht kritiklos voraus, daß die
  katholische Kirche objektiv die Wahrheit des Geistes verkörpert. Eine
  legitime Auflehnung gegen die katholische Kirche ist dann kaum noch denkbar.
  Allerdings hätte die katholische Kirche seiner Ansicht nach besser daran
  getan, sich die Häresien einzuverleiben, als sie gewaltsam zu
  bekämpfen.}) Großes geleistet, aber zugleich stand sie der Entwicklung der
Naturwissenschaften und einer von der religiösen Orthodoxie unabhängigen
Geschichtsdeutung ("`Freilegung der Realitätsbereiche von Welt und
Geschichte"'\footnote{Voegelin, Anamnesis, S.327.}) im Wege. Dadurch hat sie
nicht wenig dazu beigetragen, jene Revolte gegen den Grund heraufzubeschwören,
welche in Voegelins Augen das charakteristische Merkmal der Neuzeit ist. Da
bisher die neuzeitliche Auseinandersetzung mit der Theologie als "`Spiel von
dogmatischer Position und Opposition"'\footnote{Voegelin, Anamnesis, S.327.}
und somit auf einer reinen Parekbasis ohne mystischen Erfahrungsgehalt
stattgefunden hat, ist nun der nächste historisch-dialektische Schritt zu
leisten, durch welchen das noetische Ordnungswissen wiederhergestellt wird.

Die Wiederherstellung des noetischen Ordnungswissens gestaltet sich
deshalb so schwierig, weil sie nicht auf der Ebene der Dogmatik in den
Streit mit der "`Revolte gegen den Grund"' eintreten darf, sondern sich
durch die Neu-Erschließung des Bewußtseins als dem Ordnungszentrum über
alle Dogmatik erheben muß. Zur "`Revolte gegen den Grund"' rechnet
Voegelin die verschiedensten philosophisch-weltanschaulichen Strömungen
der Gegenwart. Wörtlich zählt er "`die Ideologien des Positivismus,
Marxismus, Historismus, Szientismus, Behaviorismus, ..  Psychologisieren
und Soziologisieren, .. welt-intentionale Methodologien und
Phänomenologien"'\footnote{Voegelin, Anamnesis, S.328.} auf.  Alle diese
geistigen Strömungen dienen seiner Ansicht nach vor allem einem Zweck:
Das Empfinden für die existenzielle Spannung zum Grund im eigenen
Inneren abzutöten, und durch die Entwicklung einer "`Obsessivsprache"'
zu verhindern, daß die Frage nach dem Grund überhaupt aufkommen kann.
Nach Voegelins Überzeugung gibt es in diesen geistigen Strömungen
nichts, woran es sich im Interesse des noetischen Ordnungsdenkens lohnt
anzuknüpfen. Immerhin räumt Voegelin später ein, daß der
"`ideologische[n] Revolte"' historisch gesehen eine wertvolle Funktion
im Kampf gegen den "`Sozialterror der [theologischen]
Orthodoxie"'\footnote{Voegelin, Anamnesis, S.329.} zukam.

Die Abkehr von der "`Revolte gegen den Grund"' wird nach Voegelins
Einschätzung bislang überwiegend von diversen Traditionalismen und
Konservativismen getragen. In einem weiteren großen Rundumschlag zählt
Voegelin hier einen ganzen Reigen von rechts-liberalen bis konservativen
Erneuerungsbewegungen seiner Zeit auf. Alle diese Erneuerungsbewegungen leiden
nach Voegelins Ansicht allerdings darunter, daß sie versuchen, die "`Revolte
gegen den Grund"' auf der gleichen Ebene zu bekämpfen, weshalb Voegelin sie
auch als "`Sekundärideologien"' (zu den Primärideologien der "`Revolte gegen
den Grund"') bezeichnet. Ihr oberflächliches Vorgehen führt dazu, daß die
Sekundärideologien lediglich zu älteren Dogmatiken zurückkehren. Aber nicht
zuletzt deshalb, weil die Revolte sich teilweise zu Recht gegen die älteren
Dogmatiken aufgelehnt hat, darf die Umkehr nicht bloß eine Rückkehr sein,
sondern sie muß zur "`Wissensrealität"' der bewußt gemachten existentiellen
"`Spannung zum Grund"' vordringen.\footnote{Vgl. Voegelin, Anamnesis, S.329.}

Woran soll man aber dann anknüpfen, wenn der Rückbezug auf die Tradition nur
wieder in eine "`Sekundärideologie"' mündet? Eine Möglichkeit besteht für
Voegelin darin, sich am Vorgehen des schon einmal als Vorbild
angeführten Albert Camus zu orientieren, der auf die griechische Mythologie
zurückgeht.\footnote{Vgl. Voegelin, Anamnesis, S.312-313.} Nach Voegelins
Interpretation bedient sich Camus deshalb des griechischen Mythos, weil im "`
`Blödsinn' der Zeit .. keine Heimat für den Menschen"'\footnote{Voegelin,
  Anamnesis, S.330.} zu finden ist. Camus' Lebensweg, der sich in Voegelins
Deutung ein wenig wie das Gleichnis vom verlorenen Sohn ausnimmt, führt
beispielhaft die Entwicklung von der Revolte "`gegen die Präsenz des Lebens in
der Spannung zum göttlichen Grund"'\footnote{Voegelin, Anamnesis, S.330.} bis
zur demütigen Wiedereinkehr in ein Leben, "`geordnet durch die liebende
Spannung der Existenz zum göttlichen Grund"'\footnote{Voegelin, Anamnesis,
  S.313.} vor. Zwar hat Camus das letzte Stadium nicht mehr erreichen können
(da er vorher durch einen Autounfall ums Leben kam), aber Voegelin glaubt
dennoch den Tagebüchern von Camus mit einiger Sicherheit entnehmen zu
können, welchen Fortgang dessen geistige Entwicklung genommen
hätte.\footnote{Vgl. Voegelin, Anamnesis, S.312., S.330.}

Nicht weniger schwierig als die individuelle Umkehr stellt sich das Problem
dar, wie die Politische Wissenschaft wieder auf den richtigen Weg gebracht
werden kann. Von der Politischen Wissenschaft, wie sie an den Universitäten
überwiegend betrieben wird, ist nach Voegelins Auffassung wenig zu erwarten,
da diese Politische Wissenschaft sich vorwiegend mit den verschiedenen
politischen Institutionen der Gegenwart beschäftigt, welche nach Voegelins
Ansicht sämtlich vom falschen Ordnungsverständnis durchseucht sind. Von
welcher Seite dürfen dann aber Antworten auf die Fundamentalfragen politischer
Ordnung erhofft werden? Voegelin glaubt, daß Hinweise zur Beantwortung dieser
Fragen noch am ehesten in der schönen Literatur bei Autoren wie z.B. Robert
Musil und Hermann Broch oder, als Alternative und zur Ergänzung der schönen
Literatur, in den Altertumswissenschaften zu finden sind, da sich die
Altertumswissenschaften mit den noch unverdorbenen prä-dogmatischen
Wissensrealitäten beschäftigen. Voegelin vertritt die Ansicht, daß die
wissenschaftliche Untersuchung prä-dogmatischer Ordnungssymbole "`eine
Bewegung auf die Noese hin"'\footnote{Voegelin, Anamnesis, S.332.}
herbeiführt. Die florierenden Altertumswissenschaften künden daher nach seiner
optimistischen Überzeugung von einer Gegenbewegung historischen Ausmaßes gegen
die dogmatischen Scheinrealitäten.  Voegelin räumt allerdings
ein, daß "`die Bewegung noch nicht bewußt zentriert ist."'\footnote{Voegelin,
  Anamnesis, S.332.} Nichts desto trotz erwartet Voegelin für die Zukunft
bedeutsame "`Durchbrüche"' zur Noese, wenn er auch verständlicherweise keine
genauen Vorhersagen darüber riskiert, wann genau und wo sich diese
Durchbrüche ereignen werden.

Zu guter Letzt geht Voegelin noch auf die wichtigsten historischen Residuen
der Noese ein. Diese sind seiner Ansicht nach im klassischen Altertum und in
der mittelalterlichen und neuzeitlichen Mystik zu finden.\footnote{Vgl.
  Voegelin, Anamnesis, S.333.} Hinsichtlich der klassischen Noese entwickelt
Voegelin nur noch einmal die Genese ihres Mißverständnisses als dogmatische
Metaphysik, ohne die klassische Noese selbst ein weiteres Mal zu behandeln.
Als bedeutsame Vertreter der Mystik erwähnt Voegelin Jean Bodin und Henri
Bergson, wobei Voegelin jedoch nur auf Bodin ausführlich eingeht.

Das Verständnis der klassischen Philosophie wird nach Voegelins Ansicht
erheblich durch die anti-metaphysische Einstellung der neuzeitlichen
Philosophie getrübt. Die Anwendung der neuzeitlichen Metaphysikkritik
auf die klassische Philosophie beruht jedoch auf einem Mißverständnis.
Nicht nur wurde das unter dem Titel "`Metaphysik"' bekannte Werk des
Aristoteles erst sehr viel später so genannt,\footnote{Der Titel
  "`Metaphysik"' stammt von Andronikus von Rhodos, der im 1.Jh. v. Chr.
  die Werke des Aristotels herausgab. Er betitelte das Werk, in dem
  Aristoteles seine prima philosophia entwickelt, als "`Metaphysik"',
  weil es auf das Buch der "`Physik"' folgte.  Vgl. den Artikel über
  Metaphysik von Th. Kobusch in: Joachim Ritter / Karlfried Gründer:
  Historisches Wörterbuch der Philosophie. Band 5: L-Mn, Basel /
  Stuttgart 1980, S.1186-1279 (S.1188).}  auch die Entwicklung des
Terminus "`Metaphysik"' als eines philosophischen Fachbegriffes findet,
Voegelins Darstellung zufolge, erst im Mittelalter statt. Den
bedeutsamsten Beitrag zum dogmatischen Mißverständnis der klassischen
Philosophie hat Thomas von Aquin geleistet, der, obwohl er nach
Voegelins Urteil ebenfalls von größter Offenheit der Seele war, in
Aristoteles' "`Metaphysik"' offenbar nicht die Noese eines Ko-Noetikers
erkannte\footnote{Dies ist meine Interpretation.  Voegelin geht nicht
  darauf ein, wie es zu diesem erstaunlichen Mißverständnis kommen
  konnte, wo doch Thomas von Aquin erstens ein großer Kenner der
  aristotelischen Philosophie und zweitens, so wie Voegelin ihn sonst
  beurteilt, ein Denker von der gleichen noetischen Höhe wie Aristoteles
  war.} und sie rein dogmatisch als "`eine Wissenschaft von {\it primae
  causae}, von {\it principia maxime universalia}, und von Substanzen
{\it quae sunt maxime a materia speratae}"'\footnote{Voegelin,
  Anamnesis, S.333.} verstand, wofür Thomas von Aquin von Voegelin,
ganz entgegen dessen sonstiger Hochschätzung für diesen
mittelalterlichen Gelehrten, scharf kritisiert wird. Nachdem die
Metaphysik einmal in dieser Form mißverstanden worden war, wurde sie in
der Folge als rein dogmatisches Geschäft weiterbetrieben. Es ist daher
nach Voegelins Ansicht den aufgeklärten Kritikern der Metaphysik auch
gar kein Vorwurf dafür zu machen, daß sie diese Form der Metaphysik
angreifen. Nur muß bei der Rezeption der Metaphysikkritik darauf
geachtet werden, daß sich die Metaphysikkritik des 18.Jahrhunderts
lediglich gegen die dogmatische Metaphysik richten kann nicht aber gegen
die klassische Philosophie. So hatte Kant in erster Linie die Metaphysik
Christian Wolfs im Visier, weshalb, wie Voegelin nahelegt, die
klassische Philosophie von Kant eher irrtümlicherweise vernachlässigt
wird und von seiner Metaphysikkritik in Wirklichkeit unberührt
bleibt.\footnote{Vgl. Voegelin, Anamnesis, S.335.}

Die mystische Religionsauffassung Jean Bodins geht, Voegelins Deutung zufolge,
auf die Neoplatonisten der Renaissance und auf Dionysius Areopagita,
insbesondere auf dessen Begriff der {\it conversio}, der Hinwendung zu Gott
zurück. Im Kern besagt Bodins mystische Religionsauffassung, daß die wahre
Religion nicht irgendeine bestimmte religiöse Lehre sei, sondern allein in der
Hinwendung zu Gott bestehe. Die Mystik kommt nach Voegelins Interpretation bei
Bodin auf zweierlei Weise zum Tragen. Zum einen erlaubt sie ihm, die
Geschichte zu verstehen. Zum anderen bildet die mystische Religionsauffassung
den Ausganspunkt für Bodins Bekenntnis zur Toleranz. Nach dem Geschichtsbild,
das Voegelin in Bodins "`Lettre à Jean Bautru"' ausgesprochen findet,
verläuft die Geschichte einzelner Gesellschaften zyklisch, indem zunächst ein
von Gott auserwählter Prophet sein Volk durch die Vermittlung von
Offenbarungswissen zu neuen Höhen geistiger Wahrheit führt, worauf jedoch,
sofern sich das störrische Volk für die prophetische Botschaft überhaupt
empfänglich gezeigt hat, im Laufe der Zeit die wahre Religion, die es gelehrt
wurde, mehr und mehr zu einem Buchstabenglauben erstarrt. Ein ganz analoges
Geschichtsbild findet Voegelin in Bergsons "`Deux Sources de la Morale et de
la Religion"' wieder.\footnote{Vgl. Voegelin, Anamnesis, S.336-337.} Freilich
darf bezweifelt werden, ob jenes ein wenig romantische Geschichtsbild, nach
welchem das Schicksal eines Volkes oder einer Zivilisation im wesentlichen von
der Frische des religiösen Glaubens und der Führung durch erleuchtete
Propheten abhängt, wirklich ein starkes Argument für die
erkenntnisaufschließende Kraft der Noese darstellt.

Überzeugender wirken dagegen die Konsequenzen, die sich aus der
mystischen Religionsauffassung für die Toleranz ergeben. Für Bodin
stellt nach der Interpretation, die Voegelin aus dem "`Lettre à Jean
Bautru"' in Verbindung mit Bodins "`Colloquium Heptaplomeres"' gewinnt,
der Rückgang auf die Mystik eine Möglichkeit dar, religiöse Toleranz zu
begründen. Die Wahrheit der Religion liegt in der mystischen Hinwendung
zu Gott selbst. Die göttliche Wahrheit als solche ist unaussprechlich,
und jeder sprachliche Ausdruck dieser Wahrheit trägt lediglich
behelfsmäßigen Charakter. Da es unsinnig ist, sich im Namen religiöser
Dogmen, die bloß sprachliche Notbehelfe sind, zu bekämpfen, ergibt sich
aus der Unaussprechlichkeit der Wahrheit Gottes das Gebot der
Toleranz.\footnote{Vgl. Voegelin, Anamnesis, S.337. - Voegelins Text
  bedarf hier ein wenig der Interpretation, da nicht ganz deutlich wird,
  wie sich das "`Wesen der Toleranz"' aus "`einer Balance zwischen den
  Bereichen des Schweigens und des Ausdrucks in der Wissensrealität"'
  ergibt.} Voegelin führt Bodins Gedanken des Unfaßbaren (das
"`Ineffabile"') noch weiter aus, wobei er eine Übereinstimmung mit
Thomas von Aquins {\it tetragrammaton} (nach Thomas der höchste und
umfassendste Gottesname) und mit seiner eigenen Vorstellung von der
"`existenziellen Spannung zum Grund"' feststellt.\footnote{Vgl.
  Voegelin, Anamnesis, S.337-338.} Dieses Ergebnis verblüfft ein wenig,
denn im Hinblick auf Voegelins im ersten Abschnitt seines Aufsatzes
entwickelten Realitätsbegriff erscheint die Deckungsgleichtheit von
Voegelins und Bodins Vorstellung der mystischen Realität eher
fragwürdig: Während sich bei Bodin (nach Voegelins eigener Deutung) das
"`Ineffabile"' als das eigentlich Wahre von seinem sprachlichen Ausdruck
als einem nur unbeholfenen Hinweis deutlich unterscheidet, hat Voegelin
zuvor noch mit größtem Nachdruck darauf bestanden, daß die Symbole, die
die Realität ausdrücken, den unabtrennbaren Teil eines
Gesamtzusammenhanges von Realität bilden, der die Termini des
Partizipierens, das Partizipieren selbst und ausdrücklich auch die
Symbole umfaßt.\footnote{Vgl. Voegelin, Anamnesis, S.305-307.}

Weiterhin führt Voegelin aus, daß die "`Einsicht in das Wissen vom Ineffabile
.. erhebliche Bedeutung für das Verständnis einer großen Klasse von
Ordnungsphänomenen"'\footnote{Voegelin, Anamnesis, S.338.} habe, die über das
"`Wesen der Toleranz"' noch hinausgehen. So sei es für die kompakte Erfahrung
des "`Ineffabilen"' charakteristisch, daß die symbolische Ausdrucksform
Sakralcharakter gewinne, so daß weitere Differenzierungen der Erfahrungen nur
noch als Kommentare zum Sakraltext auftreten könnten. Inwiefern die
Feststellung dieses "`Ordnungsphänomens"' aus der "`Einsicht in das Wissen vom
Ineffabile"' fließt, bleibt ein Rätsel, zumal Voegelin schon
wenig später feststellt, daß es auch im Bereich der Ideologie das Phänomen
ideologischer Klassiker mit "`anschließender kommentatorischer und
apologetischer Literatur"' gibt.\footnote{Vgl. Voegelin, S.338-340.}

\subsection{Kritik von Voegelins Bodin- und Camus-Deu\-tung}

Da dies bereits im ersten Teil dieser Arbeit geschehen ist, erübrigt es sich,
an dieser Stelle noch einmal ausführlich auf Voegelins Geschichtsbild und
seine polemische Zeitkritik einzugehen. Auch Voegelins Empfehlung an die
Politikwissenschaft, das Wissen von politischer Ordnung lieber durch die
Lektüre der modernen Klassiker der schönen Literatur oder durch Vertiefung in
das Altertum zu erweitern als durch das Studium der politischen Institutionen
der Gegenwart, soll an dieser Stelle nicht weiter nachgegangen werden.
Ohnenhin ist der Eintritt des von Voegelin prophezeiten großen noetischen
Durchbruchs in den dreißig Jahren, die seit der Publikation seines Aufsatzes
verflossen sind, bisher ausgeblieben, so daß diese Empfehlungen, die in
Antizipation eines solchen Durchbruches ausgesprochen wurden, nicht
mehr allzu aktuell sind.

Ebenfalls keine besonders weittragende Bedeutung kommt Voegelins These zu, daß
die Metaphysikkritik der aufklärerischen Philosophie auf einem Mißverständnis
beruht, soweit sie sich nicht nur gegen die neuzeitliche Metaphysik richtet,
sondern auch gegen die klassische Philosophie gewendet wird. Es ist kaum
anzunehmen, daß die antike Philosophie, wenn sie, wie Voegelin dies fordert,
als Kryptomystik interpretiert worden wäre, in der aufklärerischen Kritik
besser abgeschnitten hätte als die Leibniz-Wolffsche Metaphysik oder die
Einsichten des "`Geistersehers"' Swedenborg.\footnote{Vgl. Immanuel Kant:
  Träume eines Geistersehers, erläutert durch Träume der Metaphysik, in:
  Frank-Peter Hansen (Hrsg.): Philosophie von Platon bis Nietzsche, CD-ROM,
  Berlin 1998, S.23599ff. / S.70ff. (Zweiter Teil, Zweites Hauptstück:
  Ekstatische Reise eines Schwärmers durch die Geisterwelt.) (Konkordanz:
  Immanuel Kant: Werke in zwölf Bänden. Herausgegeben von Wilhelm Weischedel.
  Frankfurt am Main 1977. Band 2, S.970ff.). - Daß die
  aufklärerisch-positivistische Religionskritik nicht notwendigerweise auf
  dogmatischen Mißverständnissen beruht und daher den Hinweis auf die
  religiöse Erfahrung keineswegs zu fürchten braucht, verdeutlicht Ayers
  erkenntnistheoretische Kritik der Berufung auf die religiöse Erfahrung. Vgl.
  Alfred J. Ayer: Language, Truth and Logic, New York [u.a.]  1982, S.157-158.
  - Zur Ersetzung der Religion durch die reine Religiösität in der modernen
  Religionsapologetik seit Schleiermacher: Vgl. Hans Albert: Kritischer
  Rationalismus. Vier Kapitel zur Kritik illusionären Denkens, Tübingen 2000,
  S.147ff.}

Lohnender erscheint es, auf Voegelins Deutung der Vorbilder Camus und Bodin
einzugehen. Führt Camus' Lebensweg tatsächlich jene Entwicklung von der
Auflehnung gegen Gott bis zur demütigen Unterwerfung unter Gott vor, die
Voegelin mit solchem Entzücken registriert? Liefert Bodins mystische
Religiosität wirklich eine optimale Begründung der Toleranz?

Legt man für die Interpretation von Camus' geistiger Entwicklung seine beiden
großen Essays "`Der Mythos von Sisyphos"'\footnote{Albert Camus: Der Mythos
  von Sisyphos. Ein Versuch über das Absurde, Hamburg 1998 (zuerst 1942), im
  folgenden zitiert als: Camus, Mythos von Sisyphos.} und der "`Mensch in der
Revolte"'\footnote{Albert Camus: Der Mensch in der Revolte. Essays, Hamburg
  1997 (zuerst 1951), im folgenden zitiert als: Camus, Mensch in der Revolte.}
zu Grunde, so ergibt sich das Bild einer Entwicklung, die weniger dramatisch
verläuft, als sie bei Voegelin erscheint. Die metaphysische Auflehnung, von
der der "`Mythos von Sisiphos"' handelt, richtet sich gegen die Absurdität der
Welt, sie richtet sich nicht primär gegen Gott, und die Absurdität ist auch
kein Resultat der Abwesendheit Gottes, die durch den Glauben behoben werden
könnte. Gott ist ebenso eine Lüge wie all die falschen Tröstungen, bei denen
die verschiedenen existentialistischen Philosophien am Ende doch wieder
herauskommen.\footnote{Vgl. Camus, Mythos von Sisyphos, S.39ff.} Der Glaube,
so könnte man sagen, ist ein Beruhigungsmittel für den Geist, welches der am
Leben Leidende stolz verschmäht. Nicht umsonst erwählt Camus den Empörer
Sisyphus zum Helden seines Essays.

Welche Entwicklung findet nun im Übergang zu "`Der Mensch in der Revolte
statt"'? Der wesentliche Unterschied zwischen diesen beiden Essays
besteht darin, daß der "`Mythos von Sisyphos"' ein rein metaphysisches
Thema behandelt: den Menschen in der Auseinandersetzung mit der absurden
Welt. "`Der Mensch in der Revolte"' erweitert diese Thematik ins
Politische, was natürlich Konsequenzen für die Deutung des Absurden nach
sich zieht. Angesichts des philosophisch gerechtfertigten Massenmordes,
wie er sich in den totalitären Staaten vollzog, war Camus seine
Philosophie des Absurden, deren stolzer Inidivualismus auch eine
Gleichgültigkeit gegen die Moral implizierte, offenbar nicht mehr
geheuer.\footnote{Vgl. Camus, Mensch in der Revolte, S.9-18 (Einleitung:
  Das Absurde und der Mord).} Es kommt zwar nicht zu einer Revision aber
zu einer Präzisierung seines früheren Standpunktes. Zu der Auflehnung
tritt das Bewußtsein der humanen Verantwortlichkeit hinzu. Die
Auflehnung ist nun nicht mehr eine Auflehnung gegen die metaphysische
Unsinnigkeit der Welt, die sich aus Kontingenzerfahrungen speist,
sondern sie wandelt sich zur Revolte gegen die Unsinnigkeit menschlichen
Leidens, deren Ursprung recht konkrete moralische und politische
Erfahrungen sind.

"`Der Mensch in der Revolte"' stellt daher auch in erster Linie eine
philosophische Auseinandersetzung mit der linksgerichteten Variante des
Totalitarismus und eine scharfe Abrechnung mit dessen intellektuellen
Verherrlichern im Westen dar. Den Kommunismus deutet Camus als Ausfluß
von Nihilismus und Gottesmord.\footnote{Vgl. Camus, Mensch in der
  Revolte, S154ff.} In diesen Punkten berührt sich Camus' Deutung am
stärksten mit der Theorie Voegelins und anderen christlichen-religiösen
Erklärungen des Totalitarismus. Allerdings handelt es sich dabei um eine
der weniger überzeugenden Passagen von Camus' Essay. Kein Kommunist oder
Sozialist muß sich getroffen fühlen, wenn Camus der kommunistischen
Revolution eine Genealogie von Nihilisten und Dandys unterschiebt. Der
Kommunismus war kein Nihilismus. Er trat mit handfesten materiellen
Glücksversprechungen an, und es können kaum die Motive einer exklusiven
künstlerischen Bohème gewesen sein, die ihm seine Massenunterstützung
sicherten. Hier entsteht bei Camus ein politisches Weltbild, das
zusammengesetzt ist aus literarischen Referenzen - ähnlich, wie es nicht
selten bei Voegelin geschieht. Trotz dieser Schwächen behält Camus in
der Sache recht. Im Jahre 1951 konnte man unmöglich noch den Kommunismus
(und gar den Kommunismus stalinscher Prägung) unterstützen, der seinen
Kredit durch das Scheitern seiner Prophezeiung wie durch seine
Verbrechen längst verspielt hatte, auch wenn es noch nicht jeder
wahrhaben wollte.\footnote{Vgl. Camus, Mensch in der Revolte, S.238ff. -
  Als Beispiel einer sehr weitgehenden intellektuellen Apologie des
  Kommunismus aus dieser Zeit: Vgl. Maurice Merlau-Ponty: Humanismus und
  Terror, Frankfurt am Main 1990 (entstanden 1946/47), S68ff.}

Zu welchem Ergebnis gelangt nun Camus? Bleibt, wenn der Gottesmord mit
innerer Logik zum Terror führt, als die einzig akzeptable Haltung nur
noch die "`die liebende Spannung der Existenz zum göttlichen
Grund"'\footnote{Voegelin, Anamnesis, S.313.} übrig? War die
metaphysische Auflehnung des "`Mythos von Sysiphus"' nur ein dummer
Jungenstreich eines unreifen Philosophen? Dies läßt sich kaum behaupten.
"`Der Mensch in der Revolte"' schließt mit dem Gegensatz von Revolte und
Revolution. Beide sind Ausdruck einer Auflehnung, aber die Revolte
anerkennt ein Gesetz des "`Maßes"' und damit Grenzen der eigenen
Unfehlbarkeit, des eigenen Rechtes, über andere zu verfügen, und
besonders des eigenen Rechtes, für politische Zwecke Gewalt auszuüben.
Die Revolution dagegen bedeutet die Entgleisung der Revolte in einen
schrankenlosen Terror, der der Anmaßung entspringt, mit einer abstrakten
Philosophie das menschliche Leben in seiner Totalität erfassen zu
können, so daß keine humanen Vorbehalte mehr möglich sind, da durch die
abstrakte Totalität schon alles berücksichtigt ist. Der Gedanke des
Maßes, welcher, repräsentiert durch den Mythos von Nemesis, den dritten
Teil des von Voegelin als Indiz für die Entwicklung Camus'
herangezogenen Werkprogrammes bildet,\footnote{Vgl. Voegelin, Anamnesis,
  S.330. - Vgl.  Albert Camus: Tagebücher 1935-1951, Hamburg 1997,
  S.465.}  verweist zwar auf eine Realität, die sich - ganz im Sinne
Voegelins - nicht uneingeschränkt und nicht ungestraft manipulieren
läßt, aber dieser Gedanke hebt die Revolte nicht auf.\footnote{Vgl.
  Albert Camus: Tagebuch März 1951 - Dezember 1959, Hamburg 1997, S.32.
  Dort schreibt Camus: "`Maß. Sie halten es für die Lösung des
  Widerspruchs. Es kann nichts anderes sein als die Bestätigung des
  Widerspruchs und der heroische Entschluß, sich daran zu halten und ihn
  zu überleben."'} Die Annerkennung des christlichen Gottes, jenes
Herren, dem man "`Ja!"' und "`Danke!"' sagen muß, ganz gleich, welches
Leid den Menschen geschieht, hätte für Camus ebenso die Aufkündigung der
Solidarität mit den Menschen bedeutet, wie das unkritische Lob der
Revolution von Seiten der Intellektuellen.\footnote{Ebd., S.263:
  "`Nemesis.  Wesentliche Komplizität von Marxismus und Christianismus
  (weiterentwickeln).  Deswegen bin ich gegen beide."'} Hier liegt ein
fundamentaler Gegensatz zu Voegelin vor. Während Voegelin die
Menschenliebe ohne die Partizipation an ein und demselben göttlichen
Grund für unmöglich hält,\footnote{Vgl. Eric Voegelin: In Search of the
  Ground, in: Conversations with Eric Voegelin.  (ed. R. Eric O'Connor),
  Montreal 1980, S.1-20 (S.10).}  schließt für Camus gerade die
Menschenliebe die Liebe zum transzendenten Sein aus.

Hinsichtlich Voegelins Bodin-Interpretation ist zunächst einmal festzuhalten,
daß Bodin die religiöse Toleranz fast ausschließlich mit
politisch-pragmatischen Gründen rechtfertigt.  In seinen "`Six Livres de la
République"' rät Bodin von der gewaltsamen Unterdrückung etablierter
Religionen und Sekten ab, da dies kriegerische Auseinandersetzungen
heraufbeschwören kann. Allenfalls durch sein eigenes Vorbild und durch Anreize
soll der Herrscher die Untertanen zum Übertritt zur wahren Religion bewegen,
mit welcher Bodin eine der konkreten Religionen meint, ohne sie jedoch zu
nennen.\footnote{Vgl. Jean Bodin: Sechs Bücher über den Staat. Buch IV-VI.
  (Hrsg. von P.C. Mayer-Tasch), München 1986, S.150-151.} Auch im "`Colloquium
Heptaplomeres"' ändert sich an der wesentlich pragmatischen Begründung der
Toleranz nicht viel. Die Frage der Wahrheit der Religion bleibt im Streit der
Sieben gänzlich unentschieden.  Einigkeit herrscht am Ende der Diskussion
lediglich darüber, daß jeder Mensch in eine religiöse Gemeinschaft eingebunden
sein muß, und daß es das Beste ist, wenn die Existenz jeder bestehenden
Glaubensgemeinschaft geduldet wird, sofern sie dazu bereit ist, den anderen
ihren Glauben nicht streitig zu machen.\footnote{Vgl. Jean Bodin: Colloquium
  of the Seven about Secrets of the Sublime.  Colloquium Heptaplomeres de
  Rerum Sublimium Arcanis Abditis, Princton 1975 (im folgenden zitiert als:
  Bodin, Heptaplomeres), S.473.} Kaum Anhaltspunkte bietet Bodin für
eine Interpretation, nach welcher die Toleranz durch die Differenz zwischen
der unaussprechlichen Wahrheit und dem symbolischen Ausdruck der Religion
begründet wird. Zwar taucht im Gespräch der Sieben einmal der Vorschlag auf,
auf die natürliche Religion als der ursprünglichen Religion zurückzugehen,
doch wird dieser Vorschlag als nicht praktikabel abgelehnt.\footnote{Bodin,
  Heptaplomeres, S.462f.} Mit der natürlichen Religion meinte Bodin dabei den
alt-israelischen Glauben, dem die unterschiedlichen monotheistischen
Glaubensrichtungen entsprungen sind, die bei Bodin so schwer vermittelbar
gegeneinander stehn. Auch hier ist also von einer konkreten Religion die Rede
und nicht von einem mystischen Wahrheitskern als Grundlage aller
Religionen.\footnote{Wollte man den Hinweis Bodins auf die ursprüngliche
  Religion weiter ausbauen, so käme man wohl zu einer deistischen Begründung
  der Toleranz, wie sie im Denken der Aufklärung (z.B.  in Lessings "`Nathan
  der Weise"') zu finden ist. Vgl. Voegelin, Anamnesis, S.380 (Anmerkung 19).}
Daß bei Bodin die Forderung der Toleranz weit eher einer Einsicht in die
politischen Notwendigkeiten als einer tiefen Überzeugung entspringt, geht auch
aus dem Schönheitsfehler hervor, daß Bodin nur die bestehenden Konfessionen
einbezieht und von einer Glaubens- und Gewissensfreiheit im heutigen Sinne bei
ihm keine Rede sein kann.\footnote{Vgl. Georg Roellenbleck: Der Schluß des
  "`Heptaplomeres"' und die Begründung der Toleranz bei Bodin, in: Horst
  Denzer (Hrsg.): Jean Bodin.  Verhandlungen der internationalen Bodin Tagung
  in München, München 1973, S.53-67 (S.66.). - Nach Roellenblecks Deutung ist
  die Toleranzbegründung im "`Heptaplomeres"' allerdings nicht nur Ausdruck
  politischen Kalküls, sondern sie steht auch in Zusammenhang mit einem
  tieferen Harmoniegedanken. Dennoch fällt es angesichts der Grenzen von
  Bodins Toleranzbegriff und angesichts der rein politisch begründeten
  Toleranz in den "`Six Livres de la République"' schwer, hierin mehr als
  nur eine nachträgliche ideologische Verschönung (die subjektiv ehrlich
  gemeint sein mag) zu sehen.}

Unabhängig davon, wie treffend Voegelins Interpretation die Überlegungen
Bodins wiedergibt, kann die Frage aufgeworfen werden, ob Voegelins
Bodin-Interpretation ihrerseits einen gangbaren Weg zur Begründung der
Toleranz aufzeigt. Der Hinweis auf die unaussprechliche Wahrheit, die
hinter jedem symbolischen Ausdruck liegt, begründet einerseits eine
überaus starke Form religiöser Toleranz, indem jede andere religiöse
Überzeugung als gleichwertvoll wie die eigene anzusehen ist, da sie sich
auf dieselbe Wahrheit bezieht. Andererseits enthält Voegelins Begründung
der Toleranz zum Teil ähnliche Schönheitsfehler wie die Überlegungen von
Bodin, indem Atheisten von der Toleranz auch bei Voegelin ausgenommen zu
sein scheinen.\footnote{Ähnlich ernüchternd wirken Voegelins an anderer
  Stelle geäußerte Bemerkungen darüber, daß ein fruchtbarer Dialog mit
  Indern oder Chinesen nur denkbar wäre, wenn diese Völker sich zuvor
  bereit finden, fleißig die Philosophie von Platon und Aristoteles zu
  studieren, damit eine Grundlage für das Gespräch vorhanden ist. Vgl.
  Conversations with Eric Voegelin. (ed. R. Eric O'Connor), Montreal
  1980., S.70/71.} Abgesehen davon stellt sich das Problem der Toleranz
in vollem Maße erst dann, wenn es nicht gelingt, in irgend einer Weise
eine Übereinstimmung zu einer fremden Auffassung herzustellen.
Beispielsweise stellt es sich dann, wenn man sich mit einer Anschauung
oder Lebensweise konfrontiert sieht, die einem, ohne daß sie einen
selber tangieren müßte, in jeder Hinsicht verkehrt und widerwärtig
vorkommt. Gerade in einer solchen Situation ist Toleranz gefragt,
während es ein Leichtes ist, tolerant zu sein, wenn man feststellt, daß
man im Grunde einer Meinung ist.

\subsection{Der Leib-Geist-Dualismus in der Theorie der Politik}

Im vorletzten Abschnitt seines umfangreichen Aufsatzes untersucht Voegelin
verschiedene Grundsatzfragen einer Theorie der Geschichte und der Politk.
Zunächst klärt Voegelin, daß ein vernüftiger Ansatz in der politischen Theorie
beide Seiten der menschlichen Natur, die leibliche und die geistige
berücksichtigen muß. Im Anschluß daran entwickelt Voegelin den Begriff des
Sozialfeldes, welcher im Wesentlichen ein etwas allgemeinerer Begriff von
Gemeinschaft ist, und stellt schließlich einige grundsätzliche Überlegungen zu
dem Zusammenhang von Sozialfeldern und geschichtlicher Entwicklung an.

Der Mensch ist ein Wesen, das ein Bewußtsein und einen Leib hat. Dies gilt
natürlich nicht nur für den Menschen als Einzelwesen, sondern auch für die
"`soziale Existenz"'\footnote{Voegelin, Anamnesis, S.340.} des Menschen. Nach
Voegelins Ansicht ist es die Leiblichkeit des Menschen, die bedingt, daß jede
Gesellschaft über Herrschaftsinstitutionen verfügen muß, die Ordnung im
Inneren und Sicherheit nach Außen schaffen. Die Untersuchung der pragmatischen
Probleme muß daher in der Politischen Wissenschaft einen breiten Raum
einnehmen. Aber Ordnung kann zugleich nur vom Bewußtsein ausgehen, weshalb
diese Seite der menschlichen Natur in der Politischen Wissenschaft nicht
vernachlässigt werden darf.

Voegelin glaubt nun spezifische Irrtümer verschiedener Ansätze des politischen
Denkens aus der Vernachlässigung jeweils eines Teils der menschlichen Natur
erklären zu können. Diese Vernachlässigung scheint für Voegelin nicht bloß ein
theoretisch-wissenschaftlicher Fehler zu sein, sondern er erblickt darin 
"`Krankheitsbilder, an denen das pneumopathische Phänomen des
Realitätsverlustes, der Verdunkelung von Sektoren der
Realität"'\footnote{Voegelin, Anamnesis, S.341.} zu erkennen ist.

Aus der Vernachlässigung der Leiblichkeit resultieren nach Voegelins
Überzeugung alle Formen von Utopien, wobei Voegelin mit feinen
Unterscheidungen nicht allzu kleinlich verfährt, so daß in diese Kategorie
alles von der aufklärerischen Fortschrittsidee bis zum Dritten Reicht fällt,
und auch Karl Marx sich anscheinend wieder einmal für Nietzsches Übermenschen
verantworten muß.\footnote{Vgl. Voegelin, Anamnesis, S.341. - Zwar spricht
  Voegelin nur allgemein vom Übermenschen ("`sei es der von Marx oder von
  Nietzsche"'), aber die Frage stellt sich, wo Marx denn jemals den
  Übermenschen gepredigt hat.}

Auf die Vernachlässigung der Geistnatur des Menschen sind nach Voegelins
Ansicht die Gesellschaftsvertragstheorien zurückzuführen. Voegelin unternimmt
jedoch nicht den geringsten Versuch, zu erklären, inwiefern sich die
Vertragstheorien allein auf die Leiblichkeit des Menschen beschränken. Sein
lapidarer Verweis auf Platons Staat hilft nicht weiter, da Platon im Staat
zwar die sophistische Vertragstheorie skizziert aber nicht eigens
widerlegt.\footnote{Vgl. Platon: Der Staat, Stuttgart 1997, S.126 (359a). -
  Die Vertragstheorie wird dort im Zusammenhang einer umfassenden Wiedergabe
  sophistischer Lehren über die Gerechtigkeit aufgeführt. Allerdings widerlegt
  Sokrates diese Lehren nicht im Einzelnen, sondern er geht statt dessen
  sogleich zur Konstruktion des idealen Staates über. Wir erfahren daher nicht
  die Gründe, die gegen die Vetragstheorie sprechen. Allenfalls könnte man aus
  dem Bau des idealen Staates indirekt solche Gründe ableiten.}

Schon zuvor hat Voegelin darauf hingewiesen, daß es kein Kollektivbewußtsein
gibt. Sofern man sich jedoch im Klaren darüber bleibt, daß sich
gesellschaftliches Handeln oder Denken stets aus den Handlungen und Gedanken
einzelner Individuen zusammensetzt, ist es im Sinne einer abkürzenden
Ausdrucksweise legitim, davon zu sprechen, daß "`jede Gesellschaft die Symbole
hervorbringt, durch die sie ihre Erfahrung von Ordnung
ausdrückt."'\footnote{Voegelin, Anamnesis, S.342.} Werden bestimmte
Ordnungserfahrungen und -symbole von einer Gruppe von Menschen geteilt und zur
Grundlage ihrer Handlungsweisen gemacht, so spricht Voegelin von einem
"`sozialen Feld"'. Soziale Felder können dauerhaft und institutionell
verfestigt sein, dann handelt es sich um "`Gesellschaften"', sie können aber
auch rein ideeller Natur sein, wie z.B. die "`ideologischen
Sozialfelder"'.\footnote{Voegelin, Anamnesis, S.342.} Darüber hinaus schließen
sich die Zugehörigkeiten zu manchen Sozialfeldern gegenseitig aus, andere
nicht. Voegelin bringt die Exklusivität von Sozialfeldern irrtümlich mit der
Frage in Zusammenhang, ob die Sozialfelder in der Leiblichkeit oder nur im
Bewußtsein fundiert sind. So glaubt Voegelin, daß die organisierten
Gesellschaften (Staaten) auf Grund ihres Fundamentes in der Leiblichkeit
wechselseitig exklusiv sein müssen. Aber in Wirklichkeit hängt dies von der
Gestaltung des Staatsbürgerschaftrechtes ab, das eine Doppelstaatsbürgerschaft
zulassen kann oder nicht. Umgekehrt schließen viele Religionen die
gleichzeitige Zugehörigkeit zu einer anderen Religion aus, obwohl eine
Religion doch gewiß eher ein "`Feld des Bewußtseins"' ist. Das Leibfundament
spielt für die Exklusivität also keine Rolle.\footnote{Vgl. Voegelin,
  Anamnesis, S.342-343.}  Probleme befürchtet Voegelin, wenn es innerhalb
einer Gesellschaft mehr als nur das eine tragende Sozialfeld gibt. Die
pluralistische Demokratie erscheint ihm daher als ein "`prekäre[r]
Kompromiß"'.\footnote{Voegelin, Anamnesis, S.342.}

Ein weiteres wissenschaftliches Problem, das Voegelin in diesem Zusammenhang
aufwirft, ist die Frage, welches die Sozialfelder sind, innerhalb derer
historische Prozesse stattfinden. Die Nationalstaaten sind als Einheiten
offenbar zu klein, da die geschichtlichen Entwicklungsprozesse meist weit über
den Rahmen einzelner Nationalstaaten hinausreichen. Arnold Toynbee betrachtete
die Zivilisationen als diejenigen Einheiten, welche eine umfassende
geschichtliche Betrachtung in den Blick nehmen muß. Dagegen wendet Voegelin
jedoch ein, daß es auch multizivilisatorische Reiche gibt, die auf diese Weise
nicht erfaßt werden können. Voegelin bezeichnet solche
zivilisationsübergreifenden Sozialfelder mit dem von Herodot übernommenen
Begriff der Oikoumene. Weiterhin vertritt Voegelin die These, daß (in der
Zeit, als er den Aufsatz verfaßte) die Oikoumene global geworden sei, und er
befürchtet, daß diese global gewordene Oikoumene das "`potentielle
Organisationsfeld für ideologische Imperien"' sei.\footnote{Voegelin,
  Anamnesis, S. 344.}

Von dem Thema "`Oikoumene"' dazu angeregt, kommt Voegelin noch einmal auf die
Themen Menschheit und Geschichte zu sprechen. Im Wesentlichen wiederholt Voegelin
allerdings nur, was er zu diesen Themen bereits zuvor geäußert hat. Mensch und
Menschheit sind "`nicht Gegenstände der Außenwelt, über die man
selbst-gewisse, empirische Aussagen machen könnte; vielmehr sind sie Symbole,
die von konkreten Menschen ... als Ausdruck für den menschlich-repräsentativen
Charakter ihrer Erfahrung vom Grund gefunden wurden"'.\footnote{Voegelin,
  Anamnesis, S.344.} Die Symbole "`Mensch"' und "`Menschheit"' legen das
"`Wissen von der Menschenwesentlichkeit"' aus, welches erfahren wird, "`Wenn
das Bewußtsein von Ordnung durch die existentielle Spannung zum Grund die
Helle der noetischen und pneumatischen Erfahrungen
erreicht"'.\footnote{Voegelin, Anamnesis, S.344.} Es fällt nicht leicht, dies
mit Voegelins vorheriger Behauptung zu vereinbaren, daß die noetische
Erfahrung ihren repräsentativen Charakter nur unter der Voraussetzung einer
ausschließlich durch die kosmische Primärerfahrung begründbaren
Wesensgleichheit aller Menschen erhält.\footnote{Vgl. Voegelin, Anamnesis,
  S.291. Eine konsistente Interpretation beider Textstellen ist nur möglich,
  wenn man annimmt, daß der "`menschlich-repräsentative Charakter"' der
  Erfahrung allein durch Rekurs auf die kosmische Primärerfahrung gegeben ist,
  daß er aber, sobald sich die noetische Erfahrung eingestellt hat, durch
  dieselben Symbole ("`Mensch"' und "`Menschheit"') zum Ausdruck gebracht
  wird, wie die "`Menschenwesentlichkeit"' (Existenz des Menschen in der
  "`Spannung zum Grund"' als dem Wesen des Menschen), ohne daß jedoch die
  "`Menschenwesentlichkeit"' die Primärerfahrung in ihrer Funktion der
  Begründung des repräsentativen Charakters der Erfahrung ablösen könnte
  (weshalb es sich in diesem Falle auch nicht um eine Differenzierung von
  Erfahrung handelt). - Wie auch immer man die beiden Textstellen (S.290/291,
  S.344/345) in ihrer Beziehung zueinander deuten mag, falsch ist Voegelins
  Ansicht aus den bereits genannten Gründen in jedem Falle.} Geschichte ist
für Voegelin die Geschichte der auf den transzendenten Seinsgrund bezogenen
Menschheit. Weiterhin unterscheidet Voegelin die "`universale
Menschheit"'\footnote{Voegelin, Anamnesis, S.345.}, zu der auch alle schon
gestorbenen und noch zukünftig lebenden Menschen gehören, von der
"`kontemporanen Kulturmenschheit"'\footnote{Voegelin, Anamnesis, S.344.}, die
nur die in der Gegenwart lebenden Menschen umfaßt. Trivialerweise ist nur die
kontemporane Kulturmenschheit ein Feld möglicher Organisation, während die
universale Menschheit lediglich im "`Interpretationsfeld"' Geschichte
auftaucht. Voegelin weist außerdem noch einmal nachdrücklich daraufhin, daß
das "`Bewußtsein von der existentiellen Spannung zum Grund ... ontisch über
alle immanent-zeitlichen Prozesse der Geschichte
hinaus[ragt]."'\footnote{Voegelin, Anmanesis, S.345} Vermutlich, weil die
Spannung zum Grund zur Selbsterfahrung des Menschen gehört, schließt Voegelin
im folgenden, daß das Symbol "`Geschichte"' (so wie Voegelin es versteht) aus
dem Wissen von der Spannung zum Grund stammt,\footnote{Voegelin spricht an der
  entsprechenden Stelle (S.345 unter (8)) von "`Menschwesentlichkeit"', doch
  scheint dies lediglich ein weiteres Synonym Voegelins für die Spannung zum
  Grund zu sein, welches sich dadurch erklärt, daß für Voegelin die
  existentielle Spannung zum Grund das Wesens des Menschen ausmacht.} und daß
eine Interpretation der Ordnung der Existenz einer Gesellschaft zunächst auf
die "`Akte des Selbstverständnisses"' eingehen muß, "`um von diesem Zentrum
her die Ramifikationen in die Ordnung der Gesamtexistenz zu
verfolgen."'\footnote{Voegelin, Anamnesis, S.345.} Weiterhin stellt Voegelin
fest, daß nur die vergangene Geschichte interpretiert werden kann, daß es aber
unmöglich ist, den "`Sinn der Geschichte"' in die unvorhersagbare Zukunft
hinein zu verfolgen. Eschatologien können daher auch nur als
mythisch-symbolischer Ausdruck der "`Spannung zur Ewigkeit des
Grundes"'\footnote{Voegelin, Anamnesis, S.346.} einen Sinn beanspruchen. Sie
dürfen nicht als noetische Analyse oder empirische Aussage mißverstanden
werden.

\subsection{Kritik: Die Unerheblichkeit des Leib-Geist Dualismus}

Es ist ziemlich offensichtlich und bedarf daher keiner ausführlichen
Erörterungen, daß der Versuch, mit Hilfe des Leib-Geist Dualismus politisches
Denken und soziale Gebilde zu charakterisieren, nicht gerade einen Glücksgriff
theoretisch-wissenschaftlicher Erklärungskunst darstellt. Voegelin scheint
sich nicht recht im Klaren darüber zu sein, daß die Ausdrücke "`Leiblichkeit"'
und "`Leibfundament"' in den Zusammenhängen, in denen er sie verwendet, kaum
mehr als Metaphern sind, die er zudem in einer recht willkürlichen und kaum
nachvollziehbaren Weise mit verschiedenen politischen Theorieansätzen und
Denkweisen assoziert. Zwar mag es sein, daß Voegelin die politischen Utopien
zu Recht für bedenklich erachtet. Aber mit Hinweis darauf, daß in der Utopie
das Bewußtsein von seiner Leiblichkeit freigesetzt werde, läßt sich das
utopische Denken ebensowenig erledigen, wie man dem Faschismus einen
entscheidenden Schlag versetzt hat, wenn es einem gelingt, die Gnosis
philosophisch zu widerlegen. Daß andererseits die
Gesellschaftsvertragstheorien als Folge der Reduktion des Menschen auf seine
Leiblichkeit und ihre Begierden zu betrachten sind, leuchtet schon deshalb
nicht ein, weil der Abschluß eines Vertrages Vernunft und damit Geist und
Bewußtsein voraussetzt. Voegelin hoffte wohl, mit Hilfe der ontologischen
Unterscheidung Theorieansätze und Ideologien, die ihm mißfielen, bereits bei
einem besonders elementaren Fehlschluß ertappen zu können (so wie man manchmal
insgeheim hofft, bei einem Philosophen, dessen Ansichten man verabscheut,
irgendwo einen offensichtlichen Widerspruch anzutreffen). Aber aus dem
Leib-Geist Dualismus lassen sich keinerlei gültige Kriterien zur Beurteilung
politischer Theorien ableiten.

Auf Voegelins wissenschaftliche Programmatik, die er in diesem Abschnitt
seines Aufsatzes entwickelt, und auf die Begriffe Menschheit und Geschichte
braucht an dieser Stelle kein weiteres Mal eingegangen zu werden, da Voegelin
kaum etwas Neues zu seinen bisherigen Begriffsbestimmungen hinzufügt.
Lediglich in bezug auf den Geschichtsbegriff ist festzuhalten, daß Voegelin
offenbar versucht, sein methodisches Prinzip, daß "`sich jede Untersuchung von
Ordnung auf die Akte des Selbsverständnisses [einer Gesellschaft] zu
konzentrieren"'\footnote{Voegelin, Anamnesis, S.345.} habe, bereits aus dem
Begriff der Geschichte als eines universalen Interpretationsfeldes, welches
die gesamte vergangene und zukünftige Menschheit in ihrer Beziehung zum
Seinsgrund umfaßt, herzuleiten. Wenn jener methodische Grundsatz auch als
regulatives Prinzip der Forschung durchaus sinnvoll sein kann, so erscheint
dennoch der Versuch, diesen Grundsatz a priori abzuleiten und damit zum Dogma
zu erheben, als überaus fragwürdig.

%  Anzumerken ist nur,
% daß Voegelin, der an anderer Stelle darauf insistiert hat, daß die Bedeutung
% von Wörtern keinesfalls beliebig gewählt werden darf, selbst einen recht
% freizügigen Gebrauch von der nominalistischen Befugnis macht, die Wortbedeutung
% nach Gutdünken festzulegen, wenn er darauf besteht, daß "`Mensch"' und
% "`Menschheit"' Symbole für den "`menschlich-repräsentativen Charakter"' der
% "`Erfahrung vom Grund"'\footnote{Voegelin, Anamnesis, S.344.} sein sollen. Im
% Vergleich dazu erscheint Platons Definition des Menschen als "`federloser
% Zweibeiner"' denn doch einleuchtender und durch ihre Vorbehaltslosigkeit
% auch humaner. 

%  Angemerkt
% sei nur, daß Voegelins Prinzip, daß "`sich jede Untersuchung von Ordnung auf
% die Akte des Selbsverständnisses [einer Gesellschaft] zu
% konzentrieren"'\footnote{Voegelin, Anamnesis, S.345.} habe, als regulatives
% Prinzip durchaus sinnvoll sein kann (und in Voegelins umfangreichen
% historischen Studien auch ihren Erfolg bewiesen hat), daß Voegelin jedoch den
% Fehler begeht, diesen Grundsatz a priori aus seinem abstrusen
% Geschichtsbegriff abzuleiten, womit er ihn zu einem Dogma verhärtet.

\subsection{"`Common Sense"' als kompaktes Ordnungswissen} 

Im Schlußteil seines Aufsatzes faßt Voegelin nach einer weiteren
ausführlichen Wiederholung seiner bisherigen Ergebnisse seine Ansichten von
politischer Ordnung und politischer Ordnungswissenschaft in Form eines knappen
Modelles zusammen. Außerdem knüpft Voegelin noch einmal an sein
Ausgangsproblem des Unterschiedes zwischen politikwissenschaftlicher und
naturwissenschaftlicher Methode an, wobei er zu dem Ergebnis kommt, daß die
Politikwissenschaft, soweit es nicht um die noetische Ergründung der
Seinsrealität geht, lediglich aus "`Common Sense"'-Einsichten bestehen kann.
In diesem Zusammenhang stellt Voegelin auch die These auf, daß der "`Common
Sense"' eine kompakte Form noetischen Ordnungswissens darstelle.

Voegelins Modell politischer Ordnung gestaltet sich folgendermaßen: Ordnung
existiert auf den drei Ebenen der Ordnung des Bewußtseins, der Ordnung der
Gesellschaft und der Ordnung der Geschichte.\footnote{Als vierte Ebene spielt
  bei Voegelin häufig noch eine mehr oder weniger eigenständige Ebene der
  Ordnung des Seins eine Rolle, die hier jedoch nicht eigens aufgeführt wird.
  Es besteht jedoch (trotz der vergleichsweise idealistischen Tendenz dieses
  Aufsatzes) kein Zweifel daran, daß für Voegelin die Ordnung des Bewußtseins
  auf die Erfahrung einer Ordnung des Seins zurückgeht.} Diese drei Ebenen
bilden eine Reihenfolge, die nicht verändert werden kann. So geht zwar die
Geschichte aus der Abfolge gesellschaftlicher Ordnungen hervor, aber es wäre
ein schwerer Fehler, die gesellschaftliche Ordnung nach Maßgabe einer
Geschichtsphilosophie gestalten zu wollen. Als einen weiteren Grundsatz stellt
Voegelin das Prinzip auf, daß keine dieser Ebenen unabhängig von den anderen
ist (was genaugenommen so verstanden werden muß, daß die nachfolgenden Ebenen
abhängig von den vorhergehenden sind, da andernfalls dieses Prinzip der
Forderung der Unumkehrbarkeit der Reihenfolge der Ordnungsebenen
widerspräche). Abgesehen von diesen Ordnungsebenen und ihrer Abfolge
untereinander muß eine politische Theorie die Hierarchie der Seinsstufen vom
materiellen Sein bis zum Bewußtsein berücksichtigen, nach welcher die höheren
Seinsstufen in den niedrigeren fundiert sind, während die niedrigeren
Seinsstufen durch die höheren organisiert werden.\footnote{Vgl. Voegelin,
  Anamnesis, S.349-350. - Vgl. auch Eric Voegelin: Vernunft: Die klassische
  Erfahrung, in: Eric Voegelin: Ordnung, Bewußtsein, Geschichte. Späte
  Schriften.  (Hrsg. von Peter J. Optiz), Stuttgart 1988, S.127-164 (S.162).}

Welchem Zweck dient dieses Modell, außer daß es in knapper Form die Essenz von
Voegelins Ordnungsphilosophie zusammenfaßt? Voegelin glaubt, daß irrige
Theorien politischer Ordnung an Verstößen gegen die Prinzipien dieses Modells
leicht erkannt werden können. Dies zeigt sich für Voegelin an manchen
Geschichtsphilosophien, was Voegelin jedoch nicht an einem Beispiel, sondern
nur unter Berufung auf einen anderen Kritiker der Geschichtsspekulationen
demonstriert.\footnote{Vgl. Voegelin, Anamnesis, S.350-351.} Auch in der
Auffassung, man könne in der Politikwissenschaft so wie in der
Naturwissenschaft allgemeine Gesetze finden, erblickt Voegelin einen Verstoß
gegen sein Modell, ohne daß er jedoch deutlich erklärt, weshalb hier ein
Verstoß vorliegt. Jedenfalls vertritt Voegelin die Ansicht, daß die Politische
Wissenschaft, sofern sie sich mit politischen Institutionen
und nicht mit der Noese beschäftigt, lediglich "`Common Sense"'-Einsichten zu
Tage fördern kann. Zu derartigen "`Common Sense"'-Einsichten gehört
beispielsweise die Feststellung, "`daß Macht die Tendenz hat, von ihrem
Besitzer mißbraucht zu werden"', oder auch die Einsicht, "`daß Kabinette nicht
mehr als eine gewisse Zahl von Mitgliedern haben sollten, weil über eine Zahl
von etwa zwanzig hinaus Beratungen und Entscheidungen schwierig
werden."'\footnote{Voegelin, Anamnesis, S.351.} 

Dabei stellt der "`Common Sense"' für Voegelin weit mehr dar als bloß eine
Ansammlung von Faustregeln des politischen Alltagsverstandes. Voegelin
vertritt vielmehr die These, daß der "`Common Sense"' eine kompakte Form
des noetischen Ordnungswissens ist - nicht anders als dies auf seine Weise der
Mythos ist, nur daß der "`Common Sense"' die Noese voraussetzt, während der
Mythos ihre Vorstufe bildet. Voegelin begründet seine These im Rückgriff auf
den schottischen Philosophen Thomas Reid. Reid verstand unter "`Common Sense"'
den Teil der Vernunft ({\it reason}), der, ausschließlich aus selbst-evidenten
Elementarwahrheiten zusammengesetzt, auch den ungebildetesten Menschen noch zu
Gebote steht. Voegelin interpretiert dies dahingehend, daß für Thomas Reid der
"`Common Sense"' ein kompakter Typus von Rationalität (im Sinne der Spannung
des Bewußtseins zum transzendenten Seinsgrund) sei. Demnach unterscheiden sich
der "`Common Sense"' und die aristotelische Noese nur durch die
"`Bewußtseinshelle"' (das Wissen darum, daß das Bewußtsein sich in einer
Spannung zum Grund befindet).

Voegelin glaubt weiterhin, daß der "`Common Sense"', dessen Philosophie im
18.Jahrhundert "`eben noch rechtzeitig, um nicht von der ideologischen
Dogmatik gebrochen zu werden"',\footnote{Voegelin, Anamnesis, S.353.}
entstanden ist, ein "`echtes Residuum der Noese"'\footnote{Voegelin,
  Anamnesis, S.354.} darstellt. Daraus erklärt sich für Voegelin die
"`Widerstandskraft des anglo-amerikanischen Kulturbereiches gegen die
Ideologien"'. Langfristig jedoch reicht in Voegelins Augen die
Widerstandskraft des "`Common Sense"' gegen die Ideologien nicht aus, so daß
die Wiedergewinnung der "`Helle des noetischen
Bewußtseins"'\footnote{Voegelin, Anamnesis, S.354.} unerläßlich ist.
  
\subsection{Kritik: "`Common Sense"' ist kein Ordnungswissen}

Da die Kritik an Voegelins Modell der Entstehung politischer Ordnung
größtenteils bereits vorweggenommen wurde, genügt es, an dieser Stelle
die wichtigsten Einwände kurz zusammenzufassen: Gegen Voegelins
Forderung, daß die Ordnung der Gesellschaft stets aus einer durch
besondere mystische Erfahrungen bestimmten Ordnung des Bewußtseins
hervorzugehen hat, ist einzuwenden, daß es keinerlei Anhaltspunkte dafür
gibt, daß dieser Weg (und nur dieser Weg) zu einer guten und
erfolgreichen politischen Ordnung führt. Überzeugend erscheint dagegen
Voegelins Ablehnung einer geschichtsphilosophischen Ableitung des
gesellschaftlichen Ordnungsentwurfes, auch wenn Voegelins Argumentation
in diesem Punkt ein wenig umständlich wirkt.\footnote{Wesentlich klarer
  argumentiert im Vergleich zu Voegelin sein Zeitgenosse Karl Popper
  gegen die Geschichtsphilosophien und insbesondere gegen die
  Vorhersagbarkeit der Geschichte: Vgl. Karl Popper: Das Elend der
  Historizismus, Tübingen 1987 (6.Aufl.), S.XI-XII.} Als recht
dogmatisch und theoretisch überaus fragwürdig erweist sich Voegelins
Forderung nach der Berücksichtigung der ontologischen Stufenhierarchie
in der politischen Theorie. Abgesehen davon, daß die Rechtfertigung
ontologischer Stufentheorien viele schwierige philosophische Probleme
aufwirft,\footnote{Eine wesentliche Schwierigkeit ontologischer
  Stufentheorien besteht darin, daß häufig angenommen wird, daß die
  Gesetze, die auf den niederen Ebenen gelten, auf den höheren Ebenen
  aus Kraft gesetzt sind, so daß beispielsweise die körperliche Ebene
  deterministisch, der Geist aber frei gedacht wird. Gerade dies ist
  jedoch nicht möglich, denn wenn alle physischen Vorgänge
  deterministisch ablaufen, kann auch der Geist, dessen Wirken auf
  physischen Vorgängen beruht (bzw. der im Körper "`fundiert"' ist),
  auch dann nicht frei sein, wenn sich das Geistige nicht im Physischen
  erschöpft: Grundsätzlich können auf den höhreren Ebenen nur weitere
  Gesetze hinzukommen, aber nicht die auf den niederen Stufen
  bestehenden Gesetze aufgehoben werden.} und die ontologischen
Stufentheorien darüber hinaus jederzeit Gefahr laufen, durch neue
naturwissenschaftliche Erkenntnisse umgestoßen zu werden, ist es kaum
ersichtlich (und auch Voegelin demonstriert es nirgendwo in einer
einleuchtenden Weise), wie aus der ontologischen Stufenhierarchie heraus
eine Kritik an einer politischen Theorie geübt werden kann. Da die
ontologischen Zusammenhänge auf der Ebene der Politik keine unmittelbare
Wirksamkeit entfalten, kann wahrscheinlich jede politische Theorie ohne
Aufgabe ihrer zentralen Forderungen an irgendeine vorgegebene Ontologie
angepaßt werden. So dürfte es vermutlich keinem Marxisten Probleme
bereiten, gegebenenfalls zuzugeben, daß der Mensch ein
leiblich-geistiges Wesen ist, dessen Geist im Leib fundiert ist. Im
Übrigen können gegen politische Ideologien wie den Faschismus oder den
Kommunismus wahrlich bessere Vorwürfe erhoben werden als die
Nicht-Berücksichtigung des Aufbaus des Seinsbereiches Mensch. Alles in
allem erscheint Voegelins Modell zur Beurteilung politischer
Ordnungsentwürfe und theoretischer Erklärungsansätze in der
Politikwissenschaft wenig tauglich.

Ernstzunehmender als Voegelins Modell politischer Ordnung ist seine
Be\-haup\-tung, daß die Politische Wissenschaft lediglich "`Common
Sense"'-Einsichten zu Tage fördern kann. Es kann in der Tat nicht
bestritten werden, daß die politischen Wissenschaften in dem Bestreben,
möglichst allgemeine Grundsätze zu finden, und prognosefähige Theorien
aufzustellen, längst nicht dieselben Erfolge feiern können wie die
Naturwissenschaften. Dies mag damit zusammenhängen, daß insbesondere bei
historisch-politischen Untersuchungen die Allgemeinheit fast immer durch
einen Verlust an Genauigkeit erkauft wird.\footnote{So verläuft
  beispielsweise die Transition von einer Diktatur zur einer Demokratie
  in unterschiedlichen Staaten trotz gewisser Ähnlichkeiten
  unterschiedlich, so daß eine allgemeine Transitionstheorie, je mehr
  Transitionen in unterschiedlichen Ländern sie beschreiben soll, desto
  größere Abweichungen zu den einzelnen Transitionen aufweisen wird. In
  den Naturwissenschaften verhält sich dies anders: Ein Stein, der in
  Moskau zu Boden fällt, fällt auf die gleiche Weise wie ein Stein in
  Budapest. Und aus der allgemeinen Gravitationstheorie lassen sich die
  Planetenbewegungen ebenso exakt ableiten wie die Fallgesetze.} Die
Gründe, die Voegelin gegen die Möglichkeit größerer Verallgemeinerungen
in der Politikwissenschaft anführt, sind jedoch alles andere als
überzeugend. Weshalb der "`Bau des Seinsbereich Mensch"' es verbietet
nach allgemeinen Gesetzen des menschlichen Verhaltens zu suchen, bleibt
völlig im Dunkeln.\footnote{Abgesehen davon verwechselt Voegelin an der
  entsprechenden Stelle (Voegelin, Anamnesis, S.352: "`Jeder
  Versuch...Mensch"') offenbar Axiome mit Naturgesetzen.  (Von der
  sauberen Unterscheidung zwischen normativen und deskriptiven Theorien
  und einer fairen Beurteilung der Absichten szientistischer Ansätze
  ganz zu schweigen.)} Im übrigen kann nicht von vornherein behauptet
werden, daß die Schwierigkeiten, auf die die Formulierung
allgemeingültiger Theorien in den Gesellschaftswissenschaften stoßen,
auch in aller Zukunft unbehebbar bleiben werden. Es gibt daher auch
keinen zwingenden Grund, weshalb die Naturwissenschaften nicht als
Vorbild betrachtet werden sollten. Umso mehr gilt dies, als der Versuch,
naturwissenschaftliche Denkweisen für die Politikwissenschaft fruchtbar
zu machen, völlig ungefährlich ist. Denn wenn er gelingt, ist die
Wissenschaft ein großes Stück weiter, und wenn er scheitert, haben
lediglich einige Forscher ihre Zeit verschwendet. Gewiß ist ein solcher
Versuch jedoch keine "`potentielle Quelle gesellschaftlicher Unordnung,
insofern er Störungen des rationalen Bewußtseins in anderen Menschen
bewirken kann"'.\footnote{Voegelin, Anamnesis, S.352.} Aber auch wenn
man zugibt, daß die Gesetzmäßigkeiten, die die Politikwissenschaft
bisher gefunden hat, von einer recht unspektakulären Art sind, so muß
dies nicht unbedingt bedeuten, daß Gesetzmäßigkeiten, wie etwa jene
vielzitierte Feststellung Lord Actons, daß Macht korrumpiert, nur
Erkenntnisse des gesunden Menschenverstandes darstellen. Dies wörtlich
zu behaupten, hieße zu verkennen, daß sich derartige Einsichten nicht
selten erst als Folge des gedanklichen Bewußtwerdens langer und
schmerzlicher historischer Erfahrungen durchsetzen.

Mehr als fragwürdig muß Voegelins Versuch genannt werden, den "`Common Sense"'
unter Rückgriff auf Thomas Reid als eine kompakte Form noetischen
Ordnungswissens zu interpretieren. Als Thomas Reid den "`Common Sense"' als
einen Teil der Vernunft bestimmte, dachte er natürlich nicht im Traum an
irgendetwas, was der Voegelinschen Ratio (als "`Sachstruktur"' des in der
"`existentiellen Spannung zum Grund"' stehenden Bewußtseins) auch nur in
irgendeiner Weise nahegekommen wäre.  Für Thomas Reid besteht der "`Common
Sense"' aus denjenigen elementaren Prinzipien des menschlichen
Urteilsvermögens, die es erlauben, selbst-evidente Zusammenhänge zu
beurteilen.  Die Vernunft (reason) umfaßt darüber alle Schlußfolgerungen, die
mit Hilfe dieser selbstverständlichen Wahrheiten gezogen werden
können.\footnote{Vgl.  Thomas Reid: Essays on the intellectual powers of man.
  (Ed. A.D.  Wooz\-ley), London 1941, S.329ff. - Zur Schule der Schottischen
  Philosophen vgl. James Mc\-Cosh: The Schottish Philosophy, 1875 (ed.  1995
  by James Fieser) auf: ""http:""//""socserv2"".""socsci"".""mcmaster"".""ca""/""\~{ }econ""/""ugcm""/""3ll3""/""mccosh""/""scottishphilosophy.pdf"" 
  (Archive for the history of
  economic thought, McMaster University, Hamilton, Canada; letzter Zugriff am:
  30.3.2005).  Zu Thomas Reid: Ebd., S.178-209.}  Mystische Erfahrungen oder
noetisches Ordnungswissen spielen bei Thomas Reid keine Rolle. Dementsprechend
erscheint es auch recht fragwürdig, den "`Common Sense"' Thomas Reids als eine
kompakte Form der aristotelischen Noese einzustufen.\footnote{Mit seiner
  These, daß der "`Common Sense"' eine kompakte Form der aristotelischen Noese
  sei, landet Voegelin allerdings insofern noch einen Zufallstreffer, als der
  Philosophie des Aristoteles tatsächlich eine gewisse Nähe zum "`Common
  Sense"' eigen ist, was sich am deutlichsten in der Lehre von der goldenen
  Mitte darstellt.  (Vgl. auch Russel, History of Western Philosophy, S.176.)
  Mit der Noese hat dies jedoch nichts zu tun, denn sonst müßte eine ähnliche
  Nähe zum "`Common Sense"' auch bei Platon, den Voegelin als Noetiker dem
  Aristoteles an die Seite stellt, zum Vorschein kommen.  Eine Verwandtschaft
  mit dem "`Common Sense"' kann der Philosophie Platons im Gegensatz zu der
  des Aristoteles jedoch kaum nachgesagt werden.}  Es stellt sich natürlich
die Frage, wie Voegelin ein derartig grober Interpretationsfehler unterlaufen
konnte.  Möglicherweise hat Voegelin Gedanken, die ihm bei der Lektüre von
Thomas Reid in den Sinn kamen, mit der Aussage dieses Philosophen verwechselt.
Vielleicht ist Voegelin an dieser Stelle aber auch das Opfer seiner eigenen
Definitionswillkür geworden, indem er zuerst den Ausdruck "`Ratio"' ganz
entgegen der üblichen Wortbedeutung definiert hat, und er dann offenbar
vergißt, daß Thomas Reid als ein Philosoph des 18.Jahrhunderts den Ausdruck
"`reason"' natürlich noch in einem ganz gewöhnlichen Sinne gebraucht.

% (Mißverständnisse
% dieser Art sind in der Philosophiegeschichte nicht selten und führen manchmal
% zu fruchtbaren neuen Entdeckungen.)

Als historischer Mythos muß auf Grund dieses Interpretationsfehlers die These
Voegelins betrachtet werden, daß sich die Noese im 18.Jahrhundert in den
"`Common Sense"' geflüchtet habe, um auf diese Weise das Heraufkommen der
Ideologien zu überwintern und den anglo-amerikanischen Kulturbereich gegen
totalitäre Versuchungen zu immunisieren. Ohnehin hätte - um diese These
einigermaßen glaubhaft zu begründen - der "`Common Sense"' als soziales
Phänomen näher eingegrenzt werden müssen, wofür der Hinweis auf eine
philosophische Schule des 18. und frühen 19.Jahrhunderts nicht ausreicht.

\subsection{Fazit}

Insgesamt stellt Voegelins Aufsatz "`Was ist politische Realität?"' zwar
einen grandiosen Versuch dar, unter Rückgriff auf unterschiedliche
philosophische Disziplinen von der politischen Philosphie bis zur
Sprachphilosophie und unter breitem Einbezug der Tradition
abendländischen politischen Denkens das Wesen der politischen
Wirklichkeit zu bestimmen. Zugleich ist jedoch festzuhalten, daß dieser
ambitionierte Versuch fast gänzlich gescheitert ist.  Verantwortlich
dafür sind vorwiegend die kaum zu übersehenden argumentativen Schwächen
in Voegelins Aufsatz. Darüber hinaus entsteht im Vergleich zu den frühen
Schriften aus dem ersten Teil von "`Anamnesis"' der Eindruck, daß
Voegelin sich auf dem besten Wege zu einer zunehmenden dogmatischen
Verhärtung seiner Position befindet. Während Voegelins Aufsatz "`Zur
Theorie des Bewußtseins"' noch mit einem Reichtum an originellen
Einfällen und Gedankenblitzen glänzt, wiederholt Voegelin in seinem
Grundsatzreferat "`Was ist politische Realität"' nurmehr
gebetsmühlenartig dieselben Dogmen. Dabei steht die Schärfe von
Voegelins immer wieder durchbrechender Polemik in einem eigentümlichen
Kontrast zu dem Mangel an einer stichhaltigen Begründung seiner eigenen
Auffassungen.

% \footnote{Gegenüber dem Vorwurf argumentativer
%   Schwächen könnte eingewandt werden, daß dieser Vorwurf fehlginge, da für
%   Voegelin politische Theorie weniger argumentativ begründet als vielmehr
%   erzählerisch glaubhaft vermittelt werden sollte.  (Darauf deuten Voegelins
%   eigene Interpretationen politischer Theorien sowie seine Ablehnung des
%   "`dogmatischen"' Spiels von theoretischen Positionen und Gegenpositionen
%   hin.) Ich lasse dahingestellt, ob Voegelins philosophische Schriften als
%   Erzählungen mehr hergeben, aber bevor eine Erzählung zur Begründung einer
%   politischen Ordnungsvorstellung dienlich sein kann, müßte erst einmal
%   geklärt werden, wie überhaupt irgend etwas durch eine Erzählung schlüssig
%   begründet werden kann. Zwar hat Voegelin dieses Problem gelegentlich
%   angesprochen (Vgl. Voegelin, Order and History V, S.26.), aber er hat
%   niemals eine auch nur halbwegs überzeugende Antwort darauf gegeben.}

% Dabei gibt sich
% Voegelin jedoch nicht die geringste Mühe zu erklären, weshalb dies der Fall
% ist.  wo doch das Schließen eines Vertrages Vernunft (und damit Geist)
% voraussetzt. Würde die Leiblichkeit genügen, dann müßte man Vertragschlüsse
% zumindest bei Säugetieren beobachten können. Voegelin stellt lediglich die
% klinische Diagnose, daß dergleichen 

% , wobei sie sich
% der dogmatischen Schulphilosophie - diese, wie man annehmen muß, mit neuem
% Sinn erfüllend\footnote{Voegelin erklärt an dieser Stelle nicht ausführlich,
%   wie das als "`Dogmatik"' und "`Parekbasis einer Noese"' charakterisierte
%   Phänomen auf einmal wieder zum "`Repräsentant der Noese"' (S.326.) werden
%   kann.} - als Ausdrucksmittel bediente.

% Statt dessen stellt er sogleich die klinische Diagnose, daß das "`Auftreten
% der sogennanten Vertragstheorien ... auf das umfassendere Syndrom geistiger
% Störung einer Gesellschaft aufmerksam mach."'\footnote{Voegelin, Anamnesis,
%   S.341.} Im Übrigen verweist Voegelin auf das zweite Buch von Platons {\it
%   Politeia}, wo seiner Ansicht nach schon "`alles Wesentliche zur Sache ...
% gesagt wurde."'\footnote{Voegelin, Anamnesis, S.342.} Der Verweis führt jedoch
% in die Leere, da im zweiten Buch der Politeia zwar die sophistische
% Vertragstheorie skizziert aber nicht eigens wiederlegt wird.\footnote{Vgl.
%   Platon: Der Staat, Stuttgart 1997, S.126. (359a).}


% \footnote{Voegelins
%   Haltung erinnert unweigerlich an seinen einmal gegen Marx geäußerten Vorwurf
%   des "`intellektuellen Schwindels"', welchen zu begründen er sich ebensowenig
%   Mühe gegeben hat, wie seine professorale Kritik an einem abgewandelten
%   Schillerzitat bei Hegel den Vorwurf der "`Fälschung"' rechtfertigt. (Vgl.
%   Wissenschaft, Politik und Gnosis) - Auf dem selben Niveau seine
%   Wagner-Kritik in: Eric Voegelin: Die deutsche Universität und die Ordnung
%   der deutschen Gesellschaft, in: Die deutsche Universität im Dritten Reich.
%   Eine Vortragsreihe der Universität München, München 1966, S.241-282
%   (S.252-253). In ästhetischer Hinsicht mag Voegelins Kritik ja berechtigt
%   sein, aber was ist an dem Gebrauch von Aliterationen in einem Opern-Libretto
%   politisch so verdächtig?}


%%% Local Variables: 
%%% mode: latex 
%%% TeX-master: "Main" 
%%% End:
























\chapter{Ergebnis: Das Scheitern von Voegelins Bewusstseinsphilosophie}

\section{Die ungelösten Fundamentalprobleme von Voegelins Ansatz}
\label{Scheitern}
\label{Fundamentalprobleme}

Nach der bisherigen, sehr ins Detail gehenden Erörterung von Voegelins
Bewusstseinsphilosophie soll nun auf einer wiederum etwas allgemeineren Ebene
eine Gesamtbilanz dessen gezogen werden, was Voegelin mit seiner
Bewusstseinsphilosophie erreicht hat. Zur Erinnerung werde ich zunächst noch
einmal in aller Kürze Voegelins Modell politischer Ordnung zusammenfassen,
um dann zu klären, ob aus Voegelins Bewusstseinsphilosophie heraus die Fragen,
die zur Begründung dieses Modells hätten geklärt werden müssen, überzeugend
beantwortet werden können.

Voegelins Modell politischer Ordnung lässt sich in der allerknappsten Form wie
folgt zusammenfassen: Politische Ordnung beruht auf der existenziellen
Grundkonstitution des Menschen. Diese existenzielle Grundkonstitution besteht
in der Spannung zum transzendenten Seinsgrund. Das Wissen um die existenzielle
Spannung zum transzendenten Seinsgrund wird dem Menschen durch mystische
Erfahrung vermittelt. Die politische Ordnung muss aus diesem Wissen um die
existenzielle Spannung zum transzendenten Seinsgrund heraus gestaltet werden.
Dann, und nur dann, kann eine gute und dauerhafte politische Ordnung
entstehen. Ohne auf die weiteren Einzelheiten von Voegelins Deutung der
historischen Entwicklung der Ordnungserfahrung und der Ursachen politischer
Unordnung näher einzugehen, kann bereits an dieser Stelle eine Reihe von
Fragen formuliert werden, die als Prüfstein der Begründungsqualität von
Voegelins Modell politischer Ordnung dienen können:

Die erste und wichtigste Frage, die sich stellt, ist die, ob es einen
transzendenten Seinsgrund überhaupt gibt. Wenn behauptet wird, dass der Mensch
in der Spannung zum Seinsgrund existiert, dann sollte zunächst geklärt werden,
ob ein solcher transzendenter Seinsgrund auch nachweislich vorhanden ist. Zur
Begründung der Annahme der Existenz eines transzendenten Seinsgrundes beruft
sich Voegelin auf die mystische Erfahrung. Diese Begründung bleibt nicht
zuletzt deshalb recht dürftig, weil Voegelin nie genau klärt, was mystische
Erfahrungen sind, und wie man sie machen kann. Über das Wesen mystischer
Erfahrungen wird im nächsten Unterkapitel (Kapitel
\ref{Transzendenzerfahrungen}) noch einiges zu sagen sein. Aber bereits
unabhängig davon, was mystische Erfahrungen nun eigentlich sind und ob es sich
tatsächlich um Erfahrungen von etwas Transzendentem handelt, werfen sie eine
Vielzahl von Fragen auf. So stellt sich z.B. die Frage, wie zwischen richtigen
und falschen Erfahrungen unterschieden werden kann. Diese Frage ist besonders
deshalb prekär, weil offenbar nicht jeder Mensch mystische Erfahrungen hat.
Wie können dann aber die mystischen Erfahrungen des einen Menschen für einen
anderen maßgeblich sein?  Dass sie es sein können ist Voegelins feste
Überzeugung, denn er sagt ausdrücklich, dass alle anderen Menschen zum Zuhören
verpflichtet sind, wenn einem Propheten oder Philosophen eine Erleuchtung
zuteil wird.\footnote{Vgl.  Voegelin, Order and History II, S.  6.}  Aber wie
können diese Menschen, die selbst keine Erleuchtung hatten, zwischen echten
und falschen Propheten unterscheiden?  Dieses Problem hat bereits Thomas
Hobbes aufgeworfen, und Hobbes war -- anders als Voegelin -- der Ansicht, dass
die Weigerung, einem Propheten zu folgen, niemandem als Verstocktheit
ausgelegt werden darf, denn der Betreffende leugnet durch diese Weigerung
nicht Gott, sondern er zweifelt lediglich die Glaubwürdigkeit eines Menschen,
nämlich des Propheten an.\footnote{Vgl. Thomas Hobbes: Leviathan oder Stoff,
  Form und Gewalt eines kirchlichen und bürgerlichen Staates, Frankfurt am
  Main 1998 (erste Auflage 1984), S. 51 (7. Kapitel) / S.  218-219 (26.
  Kapitel).} Voegelin besteht dagegen darauf, dass echte
Transzendenzerfahrungen eines Menschen für alle Menschen autoritativ sind,
aber er zeigt nicht, wie das möglich ist, wenn es doch kein Mittel gibt, um
ihre Echtheit zu prüfen.\footnote{Nicht dass Voegelin diese Probleme nicht
  bemerkt hätte, aber er weicht einer Antwort in der Regel durch
  Weitschweifigkeit oder stillschweigenden Themawechsel aus.  (Vgl. z.B.
  Voegelin, Order and History III, S. 299-303.) -- Noch unzulänglicher:
  Voegelin, Order and History V, S.  26. -- Ebenfalls unbefriedigend:
  Conversations with Eric Voegelin. (ed. R.  Eric O'Connor), Montreal 1980, S.
  23/24.}

Mit diesen die Echtheit der Transzendenzerfahrungen betreffenden Fragen sind
die Probleme von Voegelins Konzeption jedoch noch keineswegs erschöpft. Nicht
weniger problematisch sind die Zusammenhänge zwischen der mystischen Erfahrung
und der konkreten politischen Ordnung. Am auf\/fälligsten erscheint hierbei,
dass zwischen diesen beiden Bereichen augenscheinlich überhaupt kein
Zusammenhang besteht. Wenn die mystische Erfahrung, so wie Voegelin dies
unablässig beteuert, eine unerlässliche Grundlage guter politischer Ordnung
ist, so müsste es doch möglich sein, aus der mystischen Erfahrung heraus
Kriterien zu formulieren, die es erlauben, irgendein konkretes Ensemble
politischer Institutionen zu beurteilen. Warum sind die Institutionen der
amerikanischen Demokratie eher mit der Noese zu vereinbaren als die
Institutionen eines faschistischen Staates wie Spanien unter der Herrschaft
Francos? Es dürfte sehr schwer fallen, hier aus Voegelins metaphysischer
Theorie heraus ein Urteil zu fällen. Zwar hat Voegelin es nicht an (oft sehr
einseitigen) politischen Lagebeurteilungen fehlen lassen, aber selten wird
dabei deutlich, wie diese sich aus den Grundsätzen seiner politischen
Philosophie ergeben.  Voegelin hat noch nicht einmal versucht, aus der Noese
irgendwelche moralischen Prinzipien oder Tugenden herzuleiten. Dergleichen
wäre ihm wohlmöglich bereits als eine dogmatische Entgleisung
erschienen.\footnote{Siehe dazu die Naturrechtskritik, die Voegelin unter
  Rückgriff auf Aristoteles' Begriff der "`phronesis"' in seinem Aufsatz "`Das
  Rechte von Natur"' übt. (Vgl.  Voegelin, Anamnesis, S. 128.) -- Zur
  ethischen Theorie Voegelins vgl. Julian Nida-Rümelin: Das Begründungsproblem
  bei Eric Voegelin. (Typoskript eines Vortrages beim Internationalen
  Eric-Voegelin Symposium in München August 1998, Eric Voegelin-Archiv in
  München).} Aber wie kann mit Hilfe der Noese die gesellschaftliche und
politische Wirklichkeit gestaltet werden, wenn es noch nicht einmal möglich
ist, bestimmte sittliche Prinzipien aus der Noese abzuleiten?\footnote{Hans
  Kelsen legt den Finger auf die Wunde, wenn er feststellt, dass die
  metaphysischen Begriffe, die in der Vergangenheit zur Rechtfertigung
  bestimmter Gerechtigkeitsvorstellungen herangezogen wurden und auf die sich
  auch Voegelin beruft (wie u.a. das platonische {\em agathon}, der
  aristotelische {\em nous}, die thomistische {\em ratio aeterna}), lediglich
  Leerformeln sind, die zur Rechtfertigung jeder beliebigen Sozialordnung
  dienen können.  Vgl.  Kelsen, A New Science, S.  15/16. -- Vgl. auch Hans
  Kelsen: Was ist Gerechtigkeit?, 2. Auflage, Wien 1975.}

% \footnote{Bereits im "`Autoritären Staat"'
%   spricht Voegelin ein verwandtes Problem an. Dort geht es um das
%   rechtstheoretische Problem, inwiefern bestimmte Wirklichkeitsbereiche
%   überhaupt durch ein System von Rechtsnormen erfaßt und normativ geregelt
%   werden können. (Denkbar wäre, dass entweder die Einzelvorgänge der
%   Wirklichkeit zu unterschiedlich und irregulär sind, um durch allgemeine
%   Gesetze erfaßt werden zu können, oder dass die Beurteilung nach allgemeinen
%   Gesetzen dem Gerechtigkeitsempfinden zuwiederläuft.) Voegelin bejaht die
%   Regelbarkeit für den Bereich des bürgerliche Gesetzes, aber er leugnet sie
%   für das Staatsrecht - anknüpfend an Vorstellungen von Carl Schmitt, wonach
%   es der {\it Echtheit} von Politik zuwiderläuft, wenn der Souverän
%   verfassungsmäßigen Bindungen unterliegt.}

Vor allem aber stellt sich die Frage, woraus hervorgeht, dass eine auf die
Noese gegründete Politik moralisch gut und den Erfordernissen des menschlichen
Zusammenlebens angemessen ist. Sofern das moralisch Gute nicht gerade durch
die Noese definiert wird, was zu einer Tautologie führen würde, ist es
keineswegs selbstverständlich, dass das noetisch Wahre mit dem moralisch Guten
zusammenfällt. Umgekehrt liefert Voegelin keinerlei stichhaltige Gründe für
seine These, dass eine gute politische Ordnung ohne die Noese ausgeschlossen
ist. Voegelin scheint hier die anthropologische Annahme zu Grunde zu legen,
dass ein Mensch, dessen Existenz nicht durch das noetische Wissen oder eine
kompakte Vorstufe desselben geordnet ist, notwendigerweise ein Chaot sein
muss.  Aber Voegelin unternimmt nicht einmal ansatzweise den Versuch, die
Richtigkeit dieser willkürlichen und gegenüber ungläubigen Menschen auch sehr
ungerechten Annahme zu demonstrieren.

All die soeben aufgeworfenen Fragen sind nun nicht bloß irgendwelche
Einzelfragen, die sich im Rahmen von Voegelins politischer Theorie nebenbei
auch noch stellen. Vielmehr handelt es sich um Fundamentalprobleme, mit deren
Lösung der gesamte Ansatz von Voegelin steht und fällt. Die Tatsache, dass
Voegelin auf keine dieser Fragen auch nur eine halbwegs schlüssige Antwort
geben kann, erlaubt daher nur eine einzige Schlussfolgerung: Voegelin ist mit
seinem Versuch, die Voraussetzungen politischer Ordnung
bewusstseinsphilosophisch zu begründen, auf der ganzen Linie gescheitert.

\section{Was sind Transzendenzerfahrungen?}
\label{Transzendenzerfahrungen}

In seinen bewusstseinsphilosophischen Schriften beruft Voegelin sich immer
wieder auf Transzendenzerfahrungen. Leider versäumt Voegelin dabei zu klären,
was Transzendenzerfahrungen sind, wie man sie erlangen kann, und ob sie ihren
Namen zu Recht tragen. Will man sich über diesen Zentralbegriff der
Voegelinschen Philosophie Rechenschaft ablegen, so bleibt kein anderer Ausweg,
als ein wenig über das Wesen dieser Erfahrungen zu spekulieren, indem man
verschiedene Möglichkeiten zu ihrer Erklärung gedanklich durchspielt. Im
Wesentlichen sind drei Erklärungsmöglichkeiten zu erwägen: 1) Es gibt
tatsächlich Transzendenzerfahrungen, aber sie sind nur wenigen auserwählten
Individuen zugänglich. 2) Hinter den Transzendenzerfahrungen verbergen sich
bestimmte innere Erlebnisse, die jeder Mensch wenigstens gelegentlich hat. Nur
messen verschiedene Menschen diesen Erlebnissen eine unterschiedliche
Bedeutung für ihr Leben bei. 3) Es gibt keine Transzendenzerfahrungen und
alles, was über sie gesagt wird, ist lediglich leeres Gerede. Diese drei
grundsätzlichen Erklärungsmöglichkeiten sollen nun im Einzelnen etwas näher
beleuchtet werden.

{\it 1. Möglichkeit: Transzendenzerfahrungen in Form von Erleuchtungen
  einzelner Individuen.} Denkbar ist, dass es sich bei den
Transzendenzerfahrungen um besondere Erlebnisse handelt, die nur ganz
bestimmten, man könnte sagen "`religiös privilegierten"' Menschen widerfahren.
Es stellt sich dann die Frage, ob diese Erlebnisse tatsächlich die Folge des
Eindringens eines transzendenten Seins in die Immanenz sind, oder ob es sich
dabei bloß um ein psychisches Phänomen ähnlich einer Geisteskrankheit handelt.
Die erste Variante ist eher unwahrscheinlich, da es für die Existenz einer
transzendenten Seinsphäre keine anderen Anhaltspunkte gibt als eben diese
Erfahrungen selbst, die einer Geisteskrankheit oder einem Drogenrausch unter
Umständen zum Verwechseln ähnlich sehen können.\footnote{Vgl. William James:
  The Varieties of Religious Experience, Cambridge, Massachusetts / London,
  England 1985 (zuerst 1902), S. 11-28.}  Aber selbst wenn es sich um
"`echte"' Transzendenzerfahrungen handeln sollte, steht auf Grund des schon
erwähnten Problems der wahren und falschen Propheten außer Zweifel, dass diese
Erfahrungen außer für den, dem sie widerfahren, für niemanden Autorität
beanspruchen können. Ob Voegelin so interpretiert werden muss, dass die
Transzendenzerfahrungen nur wenigen religiös privilegierten Individuen
zukommen, lässt sich nicht eindeutig bestimmen. Voegelins historischer Ansatz,
nach welchem die "`Seinssprünge"' zunächst als Erleuchtungsereignisse in
einzelnen Individuen stattfinden, legt diese Interpretation
nahe.\footnote{Vgl.  Conversations with Eric Voegelin. (ed. R.  Eric
  O'Connor), Montreal 1980, S. 25. -- Bereits in den "`Politischen
  Religionen"' schreibt Voegelin: "`Die Erneuerung kann in großem Maße nur von
  großen religiösen Persönlichkeiten ausgehen -- aber jedem ist es möglich,
  bereit zu sein und das seine zu tun, um den Boden zu bereiten, aus dem sich
  der Widerstand gegen das Böse erhebt."' (Eric Voegelin: Die politischen
  Religionen, München 1996 (zuerst 1938), S. 6.)} Dem steht jedoch Voegelins
bewusstseinsphilosophischer Ansatz entgegen, der, gerade weil dabei das
Bewusstsein {\it des} Menschen untersucht wird, beinhaltet, dass die
Transzendenzerfahrungen jedem Menschen zugänglich sind, wenn es auch von
Mensch zu Mensch Unterschiede in der Intensität der Erfahrung geben kann.

{\it 2. Möglichkeit: Transzendenzerfahrungen als innere Erlebnisse.} Es könnte
auch der Fall sein, dass es sich bei den Transzendenzerfahrungen, die Voegelin
meint, um innere Empfindungen und Erlebnisse handelt, welche im Seelenleben
eines jeden Menschen vorfindlich sind, auch wenn diesen Empfindungen im
bewussten Denken und Handeln nicht immer eine gleichermaßen bedeutsame Rolle
eingeräumt wird. Zu diesen inneren Empfindungen gehören vermutlich alle Arten
von Gefühlserlebnissen, die aus dem Alltäglichen herausfallen, eine starke
emotionale Besetzung aufweisen und entweder nicht unmittelbar einem äußeren
Anlass zugerechnet werden können oder durch ihre Intensität über diesen Anlass
hinauszuweisen scheinen. Dazu könnten beispielsweise Erinnerungen, Träume,
Tagträume, Drogenerlebnisse, Sex, bestimmte durch Kunst oder Musik
hervorgerufene Empfindungen sowie sich manchmal plötzlich einstellende und
scheinbar grundlose Gefühle von großer Angst oder Freude zählen. Der Versuch,
den Bereich der für Transzendenzerfahrungen in Frage kommenden seelischen
Erlebnisse näher einzugrenzen, begegnet nicht zuletzt deshalb großen
Schwierigkeiten, weil es auch eine Frage des Temperamentes ist, ob jemand
sagt: "`Ich habe eben ein wenig vor mich hingeträumt"', oder ob er erklärt:
"`Ich habe soeben in der Meditation für einige Sekunden das Eindringen der
Transzendenz in mein Bewusstsein erfahren"', oder ob jemand mitteilt: "`Ich
hatte plötzlich ein unerklärliches Angstgefühl"', oder stattdessen vielmehr
behauptet: "`Mir sind gerade die Schauer des Numinosen über den Rücken
gelaufen."' Damit soll nicht gesagt werden, dass die jeweils letztere Form,
die entsprechende Erfahrung in Worte zu fassen, reine Hochstapelei ist. Es ist
nur der Tatbestand festzuhalten, dass derartige Erfahrungen nicht aus sich
selbst heraus ihr Wesen offenbaren, sondern dass dies erst die Folge einer
menschlichen Deutung ist.

Nun spricht nichts dagegen, das eigene Leben im Zeichen irgendwelcher dieser
Erfahrungen zu führen, sofern man das Gefühl hat, dass dadurch das Leben
reicher wird und einen höheren Wert bekommt. Auch mag es für den eigenen
Seelenhaushalt von großer Wichtigkeit sein, auf derartige Erfahrungen
achtzugeben.  Aber gegenüber Voegelins Deutung muss, sofern er diese Art von
Erfahrungen meint, zweierlei festgehalten werden: Erstens können derartige
Erfahrungen zwar Sinn und Wert vermitteln, nicht jedoch Wahrheit. Wahrheit
besteht in der Übereinstimmung von etwas Gemeintem mit der Wirklichkeit.
Wahrheit kommt nicht bereits dadurch zustande, dass etwas als wahr empfunden
oder als höchst real erfahren wird, denn auch Irrtümer werden, solange man von
ihnen überzeugt ist, als wahr empfunden. Grundsätzlich sind innere
Erfahrungen, Evidenzerlebnisse oder Erleuchtungen niemals eine unmittelbare
Quelle von Wahrheit, sondern bestenfalls eine Quelle von Ideen, über deren
Richtigkeit erst eine sorgfältige Prüfung -- möglichst in einem nüchterneren
Seelenzustand! -- befinden muss. Mit anderen Worten: Man sollte sich eine
gesunde Skepsis gerade und besonders auch gegenüber den eigenen Erleuchtungen
bewahren. Zu den Wesensmerkmalen der Wahrheit gehört zudem ihre Objektivität
oder zumindest Intersubjektivität.  Gerade dies ist jedoch für die in Frage
stehenden Erfahrungen nicht gegeben, welche zunächst einmal höchst
individuelle und persönliche Erfahrungen sind.  Daraus, dass diese Erfahrungen
zunächst nur eine Meinung oder ein Wertempfinden, aber nicht zwangsläufig
Wahrheit vermitteln, ergeben sich zwei gewichtige Konsequenzen: Zum einen sind
Konflikte mit der Realität keineswegs ausgeschlossen, so dass es sehr voreilig
ist, die Realität der inneren seelischen Erfahrungen mit der Realität
schlechthin gleichzusetzen.  Zum anderen kann aus der Erfahrung kein
Gehorsamsanspruch abgeleitet werden, da dieser Art der Erfahrung die objektive
Gültigkeit fehlt.  Damit ist natürlich nicht ausgeschlossen, dass die
Erfahrungen eines Einzelnen auf der Basis freiwilliger Anerkennung für Andere
Bedeutsamkeit erlangen können. Zudem kann sich ein Gehorsamsanspruch immer
noch dann ergeben, wenn das Erfahrene aus anderen Gründen diesen Anspruch
rechtfertigt. (Beispiel: Einem Propheten geht im Rahmen einer göttlichen
Vision eine moralische Norm auf, welche sich auch bei anschließender
nüchterner Betrachtung und unter Abwägung aller Eventualitäten als sinnvoll
und akzeptabel erweist.)
  
Zweitens muss gegenüber Voegelin festgehalten werden, dass jene emphatischen
seelischen Erlebnisse -- trotz gelegentlicher überraschender Übereinstimmungen
zwischen unterschiedlichen Individuen -- im Ganzen von einer solch irregulären
Vielfalt sind, dass es unmöglich ist, sie allesamt auf eine Normerfahrung der
existenziellen Spannung zum Grund zu beziehen. Manch einer empfindet eine
existenzielle Spannung zum Grund, ein anderer wiederum fühlt sich in
der Harmonie des Universums aufgehoben, wieder einer spürt vielleicht die
Nähe Gottes in anderen Menschen. Dies sind ganz unterschiedliche Erfahrungen.
Der Versuch, alle Erfahrungen je nach ihrer Nähe zur Normerfahrung der
existenziellen Spannung zum transzendenten Seinsgrund in "`kompaktere"' und
"`differenziertere"' Erfahrungen einzuteilen und so zwischen ihnen eine
Rangfolge herzustellen, dürfte höchstens für sehr nahe verwandte Erfahrungen
durchführbar sein.

{\it 3. Möglichkeit: Transzendenzerfahrungen als soziales Artefakt.} Als
letzte Möglichkeit muss in Erwägung gezogen werden, dass es
Transzendenzerfahrungen überhaupt nicht gibt, und dass auch nicht jene eben
beschriebenen außergewöhnlichen Seelenzustände damit gemeint sind, sondern dass
es sich vielmehr um Einbildungen oder besser um soziale Artefakte handelt, die
bloß geglaubt werden, weil so viele Menschen davon reden, als handele es sich
um eine Selbstverständlichkeit. Der gesellschaftliche Mechanismus, der zu
diesen Artefakten führt, wird sehr treffend durch das Märchen von des Kaisers
neuen Kleidern veranschaulicht, in welchem bekanntlich ein ganzer Hofstaat in
den höchsten Tönen die Farbenpracht der Gewänder eines splitternackten Kaisers
preist. Darauf, dass bei Voegelin die vermeintlichen Transzendenzerfahrungen
nur ein solches Kunstprodukt der Einbildungskraft unter dem Einfluss der
eifrigen Lektüre religiösen Schrifttums sind, deutet die auf\/fällige Tatsache
hin, dass Voegelin niemals von einer eigenen Transzendenzerfahrung berichtet.
Immer nur wird auf Platon, Aristoteles, den heiligen Thomas von Aquin, die
Bibel und auf andere ehrwürdige Schriftstücke verwiesen, von denen Voegelin
versichert, dass sie Ausdruck von echten und unverfälschten
Transzendenzerfahrungen seien. Einen Bericht von eigenen
Transzendenzerfahrungen liefert Voegelin nicht.  Stattdessen müssen
Kindheitserinnerungen als Lückenbüßer herhalten.\footnote{Vgl. Anamnesis, S.
  62-76.} Doch dies wirkt eher wie ein verzweifelter Versuch Voegelins, die
Transzendenzerfahrungen, an deren Wirklichkeit er so gerne glauben wollte, in
irgendeiner Weise auch bei sich selbst vorzufinden. Ohne eigene
Transzendenzerfahrungen aber wird Voegelins Behauptung, dass sich die
Richtigkeit der politischen Philosophien, die er befürwortete, "`empirisch"'
überprüfen ließe,\footnote{Vgl. Voegelin, Neue Wissenschaft der Politik, S.
  96.} vollends unglaubwürdig.

Von den eben aufgeführten drei Deutungsmöglichkeiten erscheint -- auch wenn es
sich nicht mit Sicherheit sagen lässt -- die zweite Deutungsmöglichkeit als die
naheliegendste, da die erste Möglichkeit allzu unwahrscheinlich ist, und man
andererseits auch ungern glauben möchte, dass jemand den größten Teil seines
Lebenswerkes auf pure Einbildungen stützt. Von welcher Deutungsmöglichkeit man
aber auch ausgeht, in keinem Fall lassen sich die weitreichenden Konsequenzen
rechtfertigen, die Voegelin für die politische Ordnung und für die
Politikwissenschaft aus den Transzendenzerfahrungen ziehen möchte.


%%% Local Variables: 
%%% mode: latex
%%% TeX-master: "Main"
%%% End: 


















\chapter{Die Schlüsselfrage: Braucht Politik spirituelle Grundlagen?}
\label{SpirituellePolitik}
Nachdem im zweiten Teil dieses Buches Voegelins Bewusstseinsphilosophie
überwiegend unter einer philosophischen Perspektive betrachtet wurde, soll nun
wieder die Beziehung zur Politik hergestellt werden. Voegelin ging es mit
seinen bewusstseinsphilosophischen Untersuchungen nicht bloß um die seelische
Ordnung der menschlichen Einzelexistenz, sondern vor allem auch um die
politische Ordnung der Gesellschaft. Nicht umsonst trägt die große
bewusstseinsphilosophische Abhandlung, die den Schlussteil seines Werkes
"`Anamnesis"' bildet, die Frage nach der politischen Realität im Titel. Im
folgenden wird zunächst versucht zu klären, inwiefern für Voegelin spirituelle
Erfahrungen die Voraussetzung guter politischer Ordnung bilden und von welcher
Gestalt eine politische Ordnung ist, die die spirituelle Erfahrung nach
Voegelins Maßstäben in angemessener Weise berücksichtigt. Anschließend wird,
losgelöst von Voegelins Theorie, in Bezug auf einige Grundfragen politischer
Ordnung untersucht, ob politische Ordnung ohne eine spirituelle Grundlage
auskommen kann.

\section{Spirituelle Wahrheit und politische Ordnung bei Voegelin}

Bereits bei der Untersuchung von Voegelins Bewusstseinsphilosophie fiel auf,
dass die Zusammenhänge zwischen den spirituellen Erfahrungsgrundlagen
politischer Ordnung und der politischen Ordnung selbst, die sich als
rechtliche und institutionelle Ordnung einer Gesellschaft konkretisiert,
merkwürdig im Dunkeln bleiben. Zwar lässt Voegelin keine Gelegenheit aus, um
vor den verhängnisvollen Folgen zu warnen, die ein Verlust des
Erfahrungskontakts zum transzendenten Seinsgrund nach seiner Überzeugung
unweigerlich nach sich zieht, aber diese Warnungen sind wissenschaftlich kaum
präziser, als es die pauschale Behauptung wäre, dass alles Unheil unserer Zeit
eine Folge der menschlichen Gottlosigkeit sei. Alles läuft bei Voegelin
letztlich auf die anthropologische These hinaus, dass der Mensch des Kontakts
zum Seinsgrund bedarf, um seine Existenz zu ordnen, und dass keine politische
Ordnung erzielt werden kann, wenn nicht sowohl auf der Seite der Herrschenden
als auch auf der Seite der Beherrschten der in genau dieser Weise existentiell
geordnete Menschentypus dominiert. Man mag einwenden, dass Voegelin im Rahmen
seiner bewusstseinsphilosophischen Untersuchungen aus Gründen der thematischen
Beschränkung diese Punkte nur habe andeuten können. Doch auch in seinen
anderen Schriften beschäftigt sich Voegelin fast ausschließlich mit der
geistigen Seite politischer Ordnung und fast nie mit dem Zusammenhang der
geistigen Grundlagen und der konkreten Machtordnung, wiewohl er an der
Ansicht, dass zwischen beiden Bereichen eine unmittelbare Beziehung besteht,
offenbar keinerlei Zweifel hat.

Wie wenig erklärende Kraft Voegelins Theorie hat, wenn er tatsächlich einmal
versucht, mit ihr die Ursachen gesellschaftlicher Unordnung zu beschreiben,
lässt sich am besten an einem Beispiel nachvollziehen: In seinem 1959
gehaltenen Vortrag "`Die geistige und politische Zukunft der westlichen Welt"'
stellt Voegelin ein "`Gesetz der westlichen Ordnung"' auf, welches besagt,
dass es drei "`Autoritätsquellen"' der Ordnung gibt, erstens die
herrscherliche Macht, zweitens die Vernunftphilosophie und drittens die
religiöse Offenbarung, und dass Ordnung herrscht, solange diese
Autoritätsquellen relativ autonom voneinander bleiben, Unordnung aber dann,
wenn sie zusammenfallen.\footnote{ Eric Voegelin: Die geistige und politische
  Zukunft der westlichen Welt. (Hrsg. von Peter J. Opitz und Dietmar Herz),
  München 1996, im folgenden zitiert als: Voegelin, Zukunft der westlichen
  Welt, S. 21-23.} Verfolgt man nun Voegelins weitere Ausführungen zu diesem
Gesetz, so springen einige Merkwürdigkeiten ins Auge: Zunächst einmal
unternimmt Voegelin keinen Versuch zu erklären, weshalb aus dem Zusammenfallen
der drei Autoritätsquellen gesellschaftliche oder politische Unordnung
resultiert. Solange Voegelin dies nicht demonstriert, liefert sein Gesetz nur
eine völlig willkürliche Definition von "`Unordnung"', die mit innerem
Unfrieden, chaotischen Zuständen oder tyrannischen Übergriffen des Staates gar
nichts zu tun haben muss.\footnote{Dies gilt umso mehr, als nach Voegelins
  "`Gesetz"' auch im Reich des Kaisers Justinian, anhand von dessen {\it
    constitutio imperatoria majestas} Voegelin sein "`Gesetz"' entwickelt,
  größte Unordnung geherrscht haben müsste, da ja der Kaiser alle drei
  Autoritätsquellen in seiner Person vereinte. (Vgl. ebd.)} Des weiteren stützt
sich Voeglins folgende Argumentation nur marginal auf das gerade erst
aufgestellte Gesetz. Es zeigt sich, dass es Voegelin keineswegs auf die
Autonomie der Autoritätsquellen ankommt, -- sonst müsste er ja auch die
Trennung von Staat und Kirche und die Abspaltung der Naturwissenschaft von der
Philosophie befürworten\footnote{Vgl. ebd., S. 31, S. 34.} -- sondern darauf,
dass christlicher Glaube und Philosophie (mit der selbstverständlich nur die
von Voegelin favorisierten Richtungen der Philosophie gemeint
sind\footnote{Vgl.  ebd., S. 35.}) einen bestimmenden Einfluss auf Gesellschaft
und Politik erlangen. Dafür ist Voegelin sogar bereit, bemerkenswerte
Einschränkungen der demokratischen Rechte hinzunehmen.  So fordert er "`sehr
energisch mit Parteiverboten"'\footnote{Ebd., S. 33. -- Nicht immer hat
  Voegelin derart drastische Forderungen aufgestellt. Aber wenn schon nicht
  mit Verboten, dann sollte zumindest durch starke informelle Mechanismen
  sichergestellt werden, dass religiös ungeeignete Leute von einer politischen
  Führungsrolle effektiv ausgeschlossen bleiben.}  gegen Parteien
"`antichristlicher oder antiphilosophischer Art"'\footnote{Ebd.} vorzugehen.
Der Grund für diese radikale Forderung liegt dabei einzig in Voegelins
vorgefasster Meinung, dass die westlichen Demokratien sich nur halten können,
wenn die Regierung im christlichen Geiste über eine weitgehend christliche
Bevölkerung regiert.\footnote{Vgl. ebd., S.  32-33.}  Warum sie nur unter
dieser Bedingung funktionieren können, dafür liefert Voegelin trotz seiner
historisch weitausholenden Erörterungen keinerlei Begründung.

Ob die Berücksichtigung der spirituellen Erfahrung für die Herstellung
politischer Ordnung überhaupt irgendwelche Vorteile erbringt, lässt sich nicht
zuletzt deshalb nur schwer klären, weil Voegelin niemals deutlich mitteilt,
von welcher Gestalt eine optimal erfahrungsbegründete politische Ordnung sein
würde.\footnote{Wie ich an anderer Stelle ausgeführt habe, würde Voegelins
  Vorgaben am ehesten eine theokratische politische Ordnung, etwa so wie sie
  im heutigen Iran existiert, entsprechen. Vgl. Arnold, Nachwort zu Kelsens
  Voegelin-Kritik, S. 125-127.}  Versucht man sich hilfsweise an Voegelins
historische Beispiele zu halten, dann erhält man ein recht irritierendes Bild.
So ist für Voegelin beispielsweise im christlichen Mittelalter vor der
Reformation mit der Trennung von geistlicher und weltlicher Autorität bei der
gleichzeitigen Legitimation und Gestaltung der weltlichen Ordnung nach
religiösen Prinzipien eine optimale Verwirklichung spirituell
erfahrungsbegründeter politischer Ordnung gegeben. Dies gilt umso mehr, als
nach Voegelins Einschätzung im mittelalterlichen Christentum die bislang
größte Erfahrungshelle des Ordnungswissens erreicht worden ist.  Gleichzeitig
herrscht jedoch mit der hierarchischen Gesellschaftsform und dem feudalen
Herrschaftssystem im Mittelalter eine politische Ordnung, die alles andere als
human und gerecht ist. Ein weiteres irritierendes Beispiel stellt die
politische Philosophie Platons dar. Für Voegelin war Platon ein Philosoph von
größter Offenheit der Seele und höchster spiritueller Empfindsamkeit. Aber die
politisch-institutionelle Ordnung, die Platon im "`Staat"' entworfen hat,
bildet geradezu das Musterbeispiel einer totalitären
Schreckensutopie.\footnote{Vgl. dazu die bekannte Kritik in: Karl Popper: Die
  offene Gesellschaft und ihre Feinde. Band I. Der Zauber Platons, 7.Aufl.,
  Tübingen 1992, S. 104ff. -- Poppers Deutung ist freilich nicht unumstritten.
  Außer einem in der Tat verfälschenden Platon-Zitat auf dem Umschlag (in der
  Taschenbuchausgabe: S. 9.) wird ihm unter anderem vorgeworfen, sich bei
  seiner Kritik an dem personenzentrierten Ansatz Platons zu sehr auf den
  "`Staat"' zu konzentrieren, und den stärker institutionellen Ansatz der
  "`Gesetze"' zu vernachlässigen. (Vgl. August Benz: Popper, Platon und das
  "`Fundamentalproblem der politischen Theorie"': eine Kritik, in: Zeitschrift
  für Politik 1999.) Von dieser Kritik unberührt bleibt allerdings Poppers
  massiver Vorwurf der Inhumanität gegen Platon.}  Hält man sich diese
Beispiele vor Augen, so erscheint es geradezu absurd, dass Voegelin der
Wiedererlangung einer spirituellen Realitätserfahrung vermittels der Öffnung
der Seele eine so große Bedeutung beimisst. Eher müsste man den Schluss ziehen,
dass für gute politische Ordnung ein niedriges spirituelles Niveau von Vorteil
ist. Gewiss, die soeben gegebenen Beispiele sind Extrembeispiele, denn
Voegelin befürwortete auch die amerikanische Demokratie, die in der Tat eine
sehr erfolgreiche Verwirklichung humaner und gerechter politischer Ordnung
darstellt, und nach Voegelins Ansicht steht der politischen Philosophie
Platons die des Aristoteles, welche wesentlich vernünftiger ist, in nichts
nach. Aber immer noch stellt sich dann die Frage, ob empirisch überhaupt eine
Korrelation zwischen dem Niveau der spirituellen Erfahrung und der Güte der
politischen Ordnung festgestellt werden kann.

Dieser Frage, ob politische Ordnung überhaupt eine spirituelle Grundlage
benötigt, um eine erfolgreiche und gerechte politische Ordnung zu sein, soll
im folgenden nachgegangen werden.

\section{Gibt es spirituelle Sachzwänge?}
\label{spirituelleSachzwänge}

Dasjenige, was Voegelins Philosophie der politischen Ordnung so hoffnungslos
anachronistisch erscheinen lässt, ist die Tatsache, dass er spezifische
religiöse Glaubensüberzeugungen als objektiv wahr voraussetzt. Voegelin hält
es für eine unbestreitbare Tatsache, dass es einen transzendenten Seinsgrund
gibt und dass wir zu diesem transzendenten Seinsgrund in einer Beziehung
stehen, die er mit Ausdrücken wie "`Spannung zum Grund"' beschreibt. Dass
Voegelin seine religiösen Vorstellungen stets nur mit vagen und undeutlichen
Worten umschreibt, darf nicht darüber hinwegtäuschen, dass er ihre Wahrheit
streng dogmatisch voraussetzt. Zudem bildet der Vorwurf der Ignoranz und der
Fehldeutung dieser religiösen Wahrheiten seine beinahe einzige Erklärung für
jede Art politischer Unordnung. Es ist daher nicht verkehrt zu sagen, dass
Voegelin diese religiösen Wahrheiten für eine Art von spirituellen Sachzwängen
hält, die die Politik berücksichtigen muss, z.B.  indem sie -- wie
schon zitiert -- "`sehr energisch mit Parteiverboten"'\footnote{Voegelin,
  Zukunft der westlichen Welt, S. 33.}  gegen Parteien "`antichristlicher oder
antiphilosophischer Art"'\footnote{Ebd.} durchgreift.

Die Vorstellung, dass die Politik spirituelle Sachzwänge berücksichtigen muss,
mag nun heutzutage in der westlichen Welt vollkommen anachronistisch
erscheinen. Die Tatsache aber, dass diese Vorstellung in dem sicherlich
größten Teil der Menschheitsgeschichte als selbstverständlich galt, wenn nicht
gar eine bestimmende Rolle gespielt hat, und dass ein Politikwissenschaftler
wie Voegelin sie in der Gegenwart artikulieren und ernst genommen werden
konnte, lässt es nicht ungeraten erscheinen, sie einmal ganz naiv als Frage zu
formulieren und zu untersuchen, ob es nicht vielleicht tatsächlich spirituelle
Sachzwänge der ein oder anderen Art gibt. Hierbei sind drei Varianten der
These zu unterscheiden:

1. Der ersten denkbaren Variante zufolge gibt es {\em objektive spirituelle
  Sachzwänge}, soll heißen: Wenn die Menschen das transzendente Sein nicht in
der gebührenden Weise berücksichtigen, so zieht dies ernste Konsequenzen
seitens des transzendenten Seins nach sich. Sintfluten, Schwefelregen und
andere Unbill, mit der zornige Götter die zuchtlose Menschenbrut zu strafen
pflegen, sind dann mindestens die Folge. Da die Härte solcher Strafen nicht
immer nur die jeweiligen Missetäter, sondern unter Umständen die gesamte
Gemeinschaft trifft, wäre es auch eine Aufgabe der Politik, durch Maßnahmen,
die dem Erhalt der spirituellen Volksgesundheit dienen, die Bevölkerung zu
schützen.

Es gibt allerdings mindestens zwei starke Gründe, die dagegen sprechen, dass
diese Art von objektiven spirituellen Sachzwängen tatsächlich existiert: 1)
Naturkatastrophen haben natürliche Ursachen, über die wir dank der
Wissenschaft inzwischen ziemlich gut bescheid wissen. Würden Naturkatastrophen
tatsächlich eine Folge der Vernachlässigung der Transzendenz sein, dann müsste
man dagegen erwarten, dass ihr Eintreten eher moralischen als natürlichen
Gesetzmäßigkeiten folgt. 2) Unabhängig davon, welche religiösen Überzeugungen
man für die wahren hält, werden Ungläubige von großen Katastrophen nicht
deutlich häufiger getroffen als Gläubige. Das bedeutet aber, dass man sich
durch die richtige Religionspolitik gar nicht vor derartigen Katastrophen
schützen kann. Objektive spirituelle Sachzwänge der oben angedeuteten Art,
denen die Politik Rechnung tragen könnte oder müsste, scheint es also nicht zu
geben.

2. Von dieser, zugegeben recht abergläubischen Vorstellung, existiert eine
modernere, psychologisierte Variante. Dieser Variante zufolge zieht eine
gestörte Beziehung zum transzendenten Seinsgrund keine Konsequenzen seitens
des transzendenten Seinsgrundes nach sich, sondern der Mensch macht sich
dadurch das Leben selbst unerträglich, weil er einer gesunden Beziehung zur
Transzendenz zutiefst bedarf. Ebenso wie bei der ersten Variante der These
wird vorausgesetzt, dass es eine objektive Wahrheit bezüglich der Transzendenz
gibt. Nur bilden nicht Naturkatastrophen die Folge der Vernachlässigung oder
Fehlinterpretation dieser Wahrheit, sondern sie führt in den seelischen
Wahnsinn. Zur Unterscheidung von der ersten Variante der These der Existenz
spiritueller Sachzwänge könnte man in diesem Fall von {\em subjektiven
  spirituellen Sachzwängen} sprechen. Sehr eindrucksvoll ist die Vorstellung,
dass es subjektive spirituelle Sachzwänge gibt, von Dostojewski literarisch
gestaltet worden, z.B. in dem Roman "`Die Dämonen"', wo sie -- ein
Gegenentwurf zu Turgenjews "`Väter und Söhne"' -- als Abwärtsbewegung von der
Generation der Väter zu der der Söhne in Erscheinung tritt: Stepan
Trofimowitsch Werchowenski, der Repräsentant der Vätergeneration, ist ein im
Grunde genommen noch sehr liebenswerter, aber natürlich naiver und
vertrottelter liberaler Romantiker. Die fatalen Konsequenzen des Verlustes der
religiösen Bindungen treten erst in seinem gefühlskalten und skrupellosen Sohn
Pjotr Stepanowitsch Werchowenski hervor, der sich von dem "`dämonischen"'
Nikolai Stawrogin zum Mord inspiriren lässt.\footnote{Vgl. Fjodor M.
  Dostojewski: Die Dämonen, 20.Aufl., München 1996.}  Neben dem Verlust
religiöser Bindungen hat Dostojewski auch ihrer Fehlbesetzung literarisch
eindrucksvoll in der Figur des Großinquisitors Gestalt verliehen.  Den
absoluten Tiefpunkt religiöser Abirrung verkörperte für Dostojewski nämlich
die katholische Kirche, deren Prinzip der Großinquisitor vertritt, während
Dostojeweski die sozialrevolutionären Bewegungen seiner Zeit im Vergleich dazu
als verzeihliche Kindereien und nachgerade eine Folge der Zerstörung der
Glaubenssubstanz durch die katholische Kirche zu entschuldigen bereit war.
Konsequenterweise ist in der Legende vom Großinquisitor\footnote{Vgl. Fjodor
  Dostojewskij: Die Brüder Karamasow, Frankfurt am Main 2006, S. 397-427.} ein
dritter Weg zwischen dem richtigen Glauben, d.i. der Wahrheit und der Freiheit
Jesu, und dem falschen Glauben der Inquisitionsgerichte nicht vorgesehen. Die
aufklärerische Vorstellung des autonomen, sich und sein Schicksal selbst
bestimmenden Menschen war für Dostojewski ein Unding. Wie wirkungsmächtig
diese von Dostojeweski literarisch gestaltete Vorstellung war, sieht man
daran, dass sie bei Voegelin in sehr ähnlicher Form wieder auftritt.  Nur dass
bei Voegelin die Rollen leicht vertauscht sind, nimmt doch bei Voegelin gerade
das katholische Christentum vor der Reformation (also ausgerechnet jener
Epoche, die das Wüten des Großinquisitors Torquemada, dem Vorbild von
Dostojewskis Großinquisitor, gesehen hatte) den Platz authentischer
Glaubenserfahrung ein, den Dostojewski dem orthodoxen Christentum vorbehalten
hatte.

Gegen diese schwächere, psychologisierte Variante der These von den
"`spirituellen Sachzwängen"' gibt es ebenfalls gravierende Einwände, auch wenn
die Situation hier schon ein wenig komplizierter ist. Ähnlich wie im Fall der
Naturkatastrophen stimmt es einfach nicht, dass seelische Gesundheit ein
Privileg nur von solchen Leuten ist, die über die richtige religiöse
"`Erfahrungsbasis"' verfügen. Seelisches Wohlbefinden kann sich bei den
Anhängern der allerverschiedensten religiösen Überzeugungen mit entsprechend
unterschiedlichen "`Bewusstseinserfahrungen"' einstellen. Es kommt bei der
Religion nur darauf an, was für wen passt. Und das kann für jeden etwas
anderes sein. Etwas komplizierter liegt der Fall bei der psychologisierten
Variante aber insofern, als die Existenz irgendwelcher Kausalzusammenhänge
zwischen der Religiosität und dem seelischen Wohlbefinden sowie zwischen der
Religiosität und dem moralischen und unter Umständen auch dem politischen
Wohlverhalten sehr wohl anzunehmen ist. Nur, dass das persönliche Wohlbefinden
und das sittliche Wohlverhalten von dem religiösen Wahrheitsgrad der
weltanschaulichen Einstellung abhängen sollen, ist eine These, die die
Wissenschaft unmöglich bestätigen kann, da sich wissenschaftlich nicht über
religiöse Wahrheiten urteilen lässt.
 
3. Die dritte Variante der These trägt diesem Einwand Rechnung. Sie setzt
daher auch nicht mehr irgendwelche religiösen Überzeugungen als wahr voraus
oder spricht -- im Sinne Voegelins -- ganz spezifischen religiösen Erfahrungen
eine höhere Adäquatheit zu als anderen. Behauptet wird lediglich, dass es
kausale Beziehungen zwischen der religiösen Einstellung und dem politischen
Verhalten von Menschen geben kann, denen die Politik Rechnung tragen sollte.
Genaugenommen handelt es sich dann aber nicht mehr um "`spirituelle
Sachzwänge"', sondern lediglich um {\em anthropologische Sachzwänge}. Die
Politik muss die Tatsache der Religiosität als einer wesentlichen Eigenschaft
des Menschen berücksichtigen, nicht zuletzt deshalb, weil religiöse
Institutionen einen erheblichen politischen Einfluss ausüben können. Das ist
aber etwas ganz anderes, als wenn die Politik unmittelbar religiöse
Glaubenssätze zu berücksichtigen hätte.

Fasst man das Verhältnis von Religion und Politik ausschließlich im Sinne
dieser dritten Variante auf, so hat das bedeutende Konsequenzen für den Umgang
mit potentiell gefährlichen religiösen oder weltanschaulichen Gruppierungen.
Die Auseinandersetzung mit solchen Gruppierungen braucht nun nicht mehr um
religiöse Wahrheitsansprüche geführt zu werden. Sie kann wesentlich
pragmatischer als eine Auseinandersetzung um deren äußeres Verhalten und
säkulare Einstellung (d.h. ihre Einstellung gegenüber dem Rest der
Gesellschaft und ihre Haltung zum gesellschaftlichen Zusammenleben, nicht aber
ihre Einstellung zu religiösen und metaphysischen Fragen im engeren Sinne)
ausgetragen werden. Dementsprechend braucht, wenn überhaupt, auch erst dann
"`sehr energisch mit Parteiverboten zugegriffen werden"',\footnote{Voegelin,
  Die geistige und politische Zukunft der westlichen Welt, a.a.O., S. 33.}
wenn derartige Gruppierungen sich verfassungsfeindlich betätigen und nicht
bereits dann, wenn sie weltanschaulich nicht konform gehen. Dieser subtile
Unterschied markiert, beiläufig bemerkt, die Grenze zwischen dem, was man die
"`wehrhafte Demokratie"' nennt, und einem autoritären Staat.

Welcher Art die kausalen Beziehungen zwischen religiöser Einstellung und
politischem Verhalten sind, darüber kann man im Einzelnen sehr
unterschiedliche Theorien aufstellen. Einer besonders unter Anhängern
Voegelins populären Ansicht zufolge führt der Verlust religiöser Bindungen
dazu, dass eine Art religiöses Vakuum entsteht, das dann von politischen
Ideologien aufgefüllt werden kann. In diesem Sinne ist z.B. Voegelins Rede von
der "`positivistischen Destruktivität"'\footnote{Eric Voegelin: The New
  Science of Politics. An Introduction, Chicago \& London 1987 (zuerst: 1952),
  S. 4.} zu verstehen, denn der Positivismus ist als solcher zwar keineswegs
totalitär, bereitet aber nach Voegelins Verständnis den totalitären Ideologien
geistig den Boden, indem er ein geistig-religiöses Vakuum hinterlässt, das
derartige Ideologien dann widerstandslos besetzen können. Allerdings gibt es
gute Gründe diese "`Vakuumtheorie"' anzuzweifeln. Wäre sie wahr, dann hätten
sich ja z.B. die religiös stark gebunden Bevölkerungskreise im 3. Reich in
Deutschland durch eine auffällig große Resistenz gegenüber dem
Nationalsozialismus auszeichnen müssen. Von Ausnahmen besonders in den
katholischen Bevölkerungsteilen abgesehen, war das aber nicht unbedingt der
Fall. Und was die "`positivistische Destruktivität"' betrifft, so hätte man
erwarten müssen, dass gerade die positivistischen Philosophenschulen wegen des
von ihnen erzeugten geistig-religiösen Vakuums in besonderem Maße anfällig für
totalitäre Ideologien gewesen sind. Ein flüchtiger Blick in die
historisch-biographischen Materialien etwa zum Wiener Kreis\footnote{Vgl.
  Friedrich Stadler: Studien zum Wiener Kreis. Ursprung, Entwicklung und
  Wirkung des Logischen Empirismus im Kontext, Frankfurt am Main 1997.} legt
aber viel eher die Vermutung nahe, dass die philosophische Strömung des
Neupositivismus -- von den kommunistischen Sympathien einzelner
positivistischer Philosophen wie z.B.  Otto Neurath abgesehen -- ganz im
Gegenteil außergewöhnlich resistent gegen die totalitäre Versuchung geblieben
ist. Im Ganzen dürften die kausalen Beziehungen zwischen dem religiösen
Hintergrund und der politischen Haltung also sehr viel komplizierter sein als
dies die "`Vakuumtheorie"' nahelegt.

Zusammenfassend lässt sich festhalten: Spirituelle Sachzwänge im Sinne
religiöser Wahrheiten oder Tatsachen, seien dies nun göttliche Strafen oder
eine vermeintlich unleugbare "`Spannung zum Grund"', die das politische
Handeln berücksichtigen müsste, gibt es nicht. Was die Politik berücksichtigen
muss ist allein das anthropologische Faktum, dass die
meisten Menschen religiös sind. Je nachdem welche Vorstellung man von der
Religiosität und insbesondere der Dominanz religiöser Motive für das
menschliche Handeln hat, könnte damit aber immer noch die These vereinbar
sein, dass die Politik es sich nicht erlauben kann, den religiösen Bereich
unbesetzt zu lassen. Dass dem nicht so ist, dafür soll in den folgenden
Abschnitten argumentiert werden.

\section{Bedarf die Legitimation der politischen Ordnung einer religiösen Komponente?} 

Ein entscheidendes Problem einer jeden politischen Ordnung, bei welchem der
Rückgriff auf religiöse Wahrheiten naheliegend erscheinen könnte, ist das
Problem der Legitimation der politischen Ordnung. Die Legitimation erfüllt
eine zweifache Funktion. Zum einen soll sie die grundsätzliche Zustimmung der
Herrschaftsunterworfenen zur Herrschaftsordnung sicherstellen. Zum anderen
dient sie der Motivation von Einsatzbereitschaft für den eigenen
Herrschaftsverband, was besonders im Kriegsfall von großer Bedeutung ist. Es
stellt sich nun die Frage, ob eine Legitimation politischer Ordnung ohne
Inanspruchnahme der menschlichen Religiosität möglich ist, und ob sie genügend
Intensität erreicht, um die Stabilität des politischen Systems auch in
Krisenzeiten zu gewährleisten.

Die heutzutage in der westlichen Welt übliche Form der Legitimation ist die
einer Gesellschaftsvertragstheorie. Die Gesellschaftsvertragstheorie
legitimiert dabei sowohl die Existenz eines Staates überhaupt als auch im
besonderen die demokratische Herrschaftsform. Die Existenz des Staates wird
dadurch legitimiert, dass ohne Staat der Einzelne vor Übergriffen von
seinesgleichen auf sein Leben und Vermögen keinen Augenblick sicher ist, so
dass die Menschen ohne Staat ohnehin nichts Besseres tun könnten, als durch
Vertrag einen Staat zu gründen, der sie voreinander beschützt. Die
demokratische Herrschaftsform wird dadurch legitimiert, dass sie diejenige
Herrschaftsform ist, zu der die Menschen in einem auf Basis freier Zustimmung
geschlossenen Vertrag am ehesten ihre Zustimmung geben könnten, da sie ihnen
nicht nur vor den Übergriffen der Mitbürger sondern auch vor dem
Machtmissbrauch des Herrschers die größte Sicherheit bietet.\footnote{Ich
  beziehe mich hier in erster Linie auf die Hobbessche
  Gesellschaftsvertragstheorie unter Berücksichtigung der Lockeschen Kritik
  dieses Modells.}

Die Gesellschaftsvertragstheorien rechtfertigen die Existenz des Staates und
die demokratische Herrschaftsform, indem sie sich rational einleuchtender
Argumente bedienen. Der Sinn des Staates wird dabei hinreichend durch den
Zweck der Schaffung innerer Sicherheit erklärt, ein Zweck der, so sollte man
meinen, im Eigeninteresse eines jeden Menschen liegt. Eine zusätzliche
Legitimation, etwa durch göttliche Autorität, könnte innerhalb dieses
Gedankenganges sogar problematisch erscheinen, denn, wenn es nicht schon
genügend rationale Gründe gäbe, um die Existenz des Staates zu legitimieren,
dann hätte der Staat ohnehin kein Existenzrecht, und alle weiteren
Rechtfertigungen seiner Existenz müssten als Ideologie verworfen werden.

Eine Legitimation politischer Ordnung ohne religiösen Bezug scheint also
grundsätzlich möglich zu sein, wenn man voraussetzt, dass die Menschen
vernünftig genug sind, um ihre eigenen Interessen zu erkennen.  Erweist sich
diese Art der Legitimation aber auch als krisenfest, wenn die politische
Ordnung vor besonderen Herausforderungen steht? Sind die liberalen Demokratien
im Falle eines Krieges in der Lage, ohne die Mobilisierung religiöser Energien
in genügendem Maße Opferbereitschaft für sich zu motivieren? Und handeln sie
sich bei der Auseinandersetzung mit ideologischen Bewegungen im Inneren nicht
einen entscheidenden Nachteil dadurch ein, dass sie die religiösen Gefühle der
Bürger unangetastet lassen müssen (und wollen)?\footnote{Vgl. Joachim Fest:
  Die schwierige Freiheit. Über die offene Flanke der offenen Gesellschaft,
  Berlin 1993, S. 38ff.}

Gegen die erste dieser Befürchtungen kann eingewandt werden, dass auch in den
liberalen Demokratien angesichts äußerer Herausforderungen gesellschaftliche
Mechanismen wirksam werden, die die Abwehrbereitschaft der demokratischen
Gesellschaft erheblich stärken. So macht sich im Falle eines Krieges oft eine
Art von innerem Zusammenrücken der Gesellschaft bemerkbar, das sich
beispielsweise in einer schlagartigen Zunahme der Beliebtheitswerte der
jeweiligen Regierung äußern kann. Auch haben sich beispielsweise im Zweiten
Weltkrieg die Soldaten, die auf Seiten der liberalen Demokratien kämpften,
nicht weniger tapfer geschlagen als die Armeen der totalitären Regime, was
beweist, dass im Ernstfall durch quasi-religiöse Sinnversprechungen keine
wesentlichen Vorteile zu erzielen sind. Weit entfernt davon, eine
Schwachstelle der liberalen Ordnung zu offenbaren, können äußere
Herausforderungen diese Ordnung sogar erheblich stärken.

Ebensowenig zwingend ist das Argument, dass ein rein rational legitimiertes
System keine ausreichende Immunität gegen die verführerische Kraft
chiliastischer politischer Bewegungen im Inneren entwickeln könnte.  Zumindest
ist nicht unmittelbar ersichtlich, wie eine spirituelle oder religiöse
Legitimationskomponente hier Abhilfe schaffen könnte. Jede Form der
Legitimation kann zusammenbrechen, wenn das politische System, das durch sie
legitimiert wird, sich als erfolglos erweist oder wenn sie durch eine vom
Geist der Zeit als überzeugender empfundene Legitimation herausgefordert wird.
Dies würde auch für eine Legitimation auf Basis der existentiellen Spannung
zum transzendenten Seinsgrund gelten, ganz gleich, welches Maß philosophischer
Wahrheit diese Legitimation für sich beanspruchen dürfte. Zudem ließe sich die
Überlegung anstellen, das gerade spirituelle Legitimationskomponenten ein
Einfallstor für Ideologien darstellen könnten, da durch sie dem
Irrationalismus bereits öffentlicher Glaubwürdigkeitskredit eingeräumt wird.

% Andererseits gibt es durchaus gravierende Einwände, die gegen religiöse oder
% spirituelle Legitimationskomponenten sprechen. So stellt sich in einer
% pluralistischen Gesellschaft die nicht unerhebliche Frage, woher die
% spirituelle Wahrheit zur Legitimation der politischen Ordnung bezogen werden
% soll. Und unabhängig davon, kann eine spirituelle Legitimation
% ernsthaft nur dann gefordert werden, wenn auch irgendeine spirituelle
% Wahrheit vorweisbar ist, in deren Namen die Legitimation vorgenommen wird.
% Ohne spirituelle Wahrheit kann es keine spirituelle Legitimation geben, auch
% wenn sie noch so nützlich wäre.

Auch wenn das Problem der hinreichenden Legitimation politischer Ordnung mit
großen Unsicherheiten behaftet ist (da sich nicht bloß die Frage stellt,
wodurch eine politische Ordnung philosophisch gerechtfertigt ist, sondern vor
allem, wann eine politische Ordnung als gerechtfertigt empfunden wird) scheint
der Rückgriff auf religiöse Überzeugungen oder spirituelle Erfahrungen für die
Legitimation politischer Ordnung nicht unbedingt erforderlich zu sein.

\section{Wertbegründung und -konsens in der pluralistischen
  Gesellschaft} 
\label{Wertbegruendung}

Ein wichtiges Argument, welches für die Religion und besonders für eine
stärkere Geltung der Religion im gesellschaftlichen Leben angeführt
werden könnte, beruht auf dem philosophischen Problem der
Letztbegründung ethischer Werte. Dieses Argument lautet in etwa wie
folgt: Keine Gesellschaft, so könnte argumentiert werden, kann ohne
einen Satz verbindlicher ethischer Grundwerte existieren. Es wäre aber
absurd, diese Grundwerte, die absolut gelten müssen, zur Disposition
eines Konsensfindungsverfahrens zu stellen, sei dies nun eine
verfassungsgebende Versammlung oder auch nur ein gedachter
Gesellschaftsvertrag, zumal dann immer noch die Gültigkeit des
Verfahrens als Wertvoraussetzung übrig bliebe. Zugleich zeigt die
Philosophiegeschichte, dass alle Versuche einer rein säkularen
Letztbegründung ethischer Werte zum Scheitern verurteilt sind. Mit
anderen Worten: Wenn es Gott nicht gäbe, dann wäre alles erlaubt. Also
muss das religiöse Bewusstsein in der Gesellschaft mindestens noch so wach
sein, dass die Verbindlichkeit der Grundwerte anerkannt wird.

Stimmt dieses Argument, und ist die Religiosität damit tatsächlich
unverzichtbar?  An diesem Ergebnis scheint kein Weg vorbeizuführen, denn wenn
eine philosophische Letztbegründung der Ethik nicht möglich ist, dann bleibt
als einzige Form der Wertbegründung ein ethischer Dezisionismus übrig, d.h.
jeder wählt sich seine Werte selbst aus, und wenn jemand die Wahl trifft,
überhaupt keine Werte zu beachten, dann ist dies genauso möglich. Diese
theoretische Konsequenz ist gewiss sehr ernüchternd. Aber kann die Religion
überhaupt Abhilfe schaffen? Das ist wiederum mehr als zweifelhaft, denn durch
eine religiöse Wertbegründung würde das Begründungsproblem nicht gelöst,
sondern nur auf die Religion verschoben werden. Dadurch dürfte das
Begründungsproblem aber eher noch komplizierter werden, da außer den Werten
nun auch die Wahrheit des religiösen Glaubens, der die Werte begründet, auf
dem Prüfstand steht.  Zwischen konkurrierenden religiösen Glaubensüberzeugungen
objektiv zu entscheiden ist aber unmöglich. Die Anerkennung einer Religion
beruht letzten Endes auf einem Glaubensakt und damit nicht weniger auf einer
persönlichen Entscheidung als die sittlichen Werte nach der Theorie des
ethischen Dezisionismus.

Eine Letztbegründung oder gar ein regelrechter Beweis ethischer Werte scheint
also unmöglich zu sein. Die universelle Verbindlichkeit bestimmter Werte lässt
sich daher bestenfalls auf Basis eines Konsenses erreichen, auch wenn dies dem
Charakter ethischer Werte als unverfügbare Werte zu widersprechen scheint.
Dabei dürfte es höchstwahrscheinlich sogar aussichtsreicher sein, den Konsens
auf der Ebene der Werte als auf der Ebene der philosophischen oder religiösen
Begründung der Werte zu suchen. Denn darüber, dass töten oder stehlen
verwerflich ist, lässt sich gewiss leichter eine Einigung erzielen als über
die Frage, ob Allah oder der liebe Gott oder die philosophische Vernunft der
legitime moralische Gesetzgeber ist. Und dort, wo unversöhnliche
Wertauffassungen aufeinanderprallen, würde es eine Einigung erst recht
erschweren, wenn der Streit zuerst auf der metaphysischen bzw. existentiellen
Ebene entschieden werden soll. Von großer Bedeutung ist dabei, dass die
Akzeptanz von Werten nicht zwingend durch die existentielle Haltung eines
Menschen bedingt ist, sondern dass sie auch auf der bloßen Einsicht in die
Nützlichkeit eines Wertes für das gesellschaftliche Zusammenleben beruhen
kann. Weiterhin können Werte auch deshalb akzeptiert werden, weil sie im
Dialog mit anderen vereinbart worden sind. Ein Wertkonsens ist daher
grundsätzlich auch ohne einen einheitlichen spirituellen Erfahrungshintergrund
der Beteiligten denkbar.

Auch in der Frage der Wertbegründung und des gesellschaftlichen Konsenses über
bestimmte Grundwerte lautet daher das Ergebnis, dass der Rückgriff auf die
Spiritualität keinesfalls notwendig und in der Regel eher hinderlich als
förderlich ist.

\section{Sinngebung durch die politische Ordnung?}

Aus Voegelins Sicht müsste ein ethischer Wertkonsens jedoch als ein höchst
brüchiges Fundament der gesellschaftlichen Ordnung beurteilt werden, sofern er
sich nicht auf einen einheitlichen spirituellen Erfahrungshintergrund stützen
kann. Dies hängt unter anderem damit zusammen, dass Voegelin die
Dialogmöglichkeiten zwischen Menschen mit unterschiedlichem spirituellem
Erfahrungshintergrund überaus skeptisch beurteilt, was sogar soweit führt,
dass Menschen ohne spirituelle Erfahrungen von Voegelin als potentielle
Ordnungsstörer eingestuft werden. Ähnliche Auffassungen kehren auch bei
manchen Anhängern Voegelins wieder. So wurde die Ansicht, dass die Gegensätze
zwischen Menschen, die an die Existenz eines transzendenten Seins glauben, und
Menschen, die sie bestreiten, weitgehend unversöhnlich bleiben müssen, solange
über diese "`Schlüsselfrage"' nicht Einigkeit erzielt worden ist, unlängst von
Thomas J. Farrell bekräftigt, der in diesem Zusammenhang die Leugnung der
Existenz eines transzendenten Seins unter Berufung auf prominente Psychologen
wie C. G. Jung und ganz auf der Linie Voegelins als eine Art Geisteskrankheit
deutet. Allerdings räumt auch Farrell ein, dass es Profanbereiche gibt,
innerhalb derer ein fruchtbarer Dialog zwischen Menschen, die jene
"`Schlüsselfrage"' unterschiedlich beantworten, möglich ist.\footnote{Vgl.
  Thomas J.  Farrell: The Key Question. A critique of professor Eugene Webbs
  recently published review essay on Michael Franz's work entitled "'Eric
  Voegelin and the Politics of Spiritual Revolt: The Roots of Modern
  Ideology"', in: Voegelin Research News, Volume III, No.2, April 1997, auf:
  http://alcor.concordia.ca/\~{ }vorenews/v-rnIII2.html (Host: Eric Voegelin
  Institute, Lousiana State University. Zugriff am: 1.8.2007).} Die Frage
stellt sich nun, ob die politische Ordnung zu diesen Profanbereichen des
menschlichen Lebens gehört.

Damit ist zugleich eine Grundfrage des Wesens politischer Ordnung
angeschnitten: Ist die (in der Neuzeit stets durch den Staat
repräsentierte) politische Ordnung nur ein Mittel zu bestimmten Zwecken
wie etwa der Schaffung innerer und äußerer Sicherheit, oder ist sie
darüber hinaus Ausdruck einer historischen Suche nach Ordnung, die mit
dem Sinn der Welt und dem Sinn des Lebens in Zusammenhang steht? Im
ersteren Fall kann die politische Ordnung voll und ganz dem
Profanbereich zugeordnet werden, so dass eine Einigung über alle
wesentlichen Prinzipien der politischen Ordnung auch zwischen Menschen
mit unterschiedlicher Offenheit der Seele im Bereich des Möglichen
liegt. Nur im letzteren Fall müsste zunächst eine gesellschaftlich
verbindliche Entscheidung über die metaphysische Schlüsselfrage der
Existenz transzendenten Seins getroffen werden.

Welche dieser beiden grundverschiedenen Wesensauffassungen politischer Ordnung
ist nun aber die richtigere? Um diese Frage zu beantworten, empfiehlt es sich,
von unterschiedlichen Funktionen des Politischen auszugehen, einer
Friedenssicherungsfunktion und einer spirituellen Funktion, und dann zu
klären, in welcher Beziehung diese Funktionen zueinander stehen, d.h. 
insbesondere, ob die politische Ordnung die Friedenssicherungsfunktion nur
erfüllen kann, wenn sie auch spirituelle Funktionen erfüllt. Sollte sich
herausstellen, dass sich beide Funktionen trennen lassen, dann kann als
Nächstes die Frage gestellt werden, welche der beiden Funktionen die für
die politische Ordnung wesentlichere ist, und ob es nicht günstiger
wäre, die andere Funktion innerhalb eines anderen Rahmens zu erfüllen,
also etwa die spirituellen Ziele nicht auf der Ebene der politischen Ordnung
sondern im Rahmen privater religiöser Vereinigungen zu verfolgen.  

Geht man zunächst einmal davon aus, dass die Stiftung inneren Friedens die
Kernfunktion politischer Ordnung ist, so kann man überlegen, was mindestens zu
einer politischen Ordnung gehören muss, damit sie diese Kernfunktion erfüllen
kann. Sicherlich sind für die Erfüllung der Kernfunktion der Friedenssicherung
Herrschaftsinstitutionen notwendig, die die Einhaltung des Friedens
garantieren. Weiterhin müssen sich die Herrschaftsinstitutionen auf die
Loyalität oder wenigstens den regelmäßigen Gehorsam der Bürger stützen können.
Eine politische Ordnung, die die Kernfunktion der Friedenssicherung erfüllen
soll, bedarf daher auch einer Legitimation, wozu mindestens eine politische
Philosophie oder Herrschaftsideologie vorhanden sein muss, die den Bürgern den
Zweck der politischen Ordnung erklärt. Dann könnte eingewandt werden, dass ein
echter Frieden noch gar nicht vorhanden ist, solange nicht auch Gerechtigkeit
herrscht. Es wären also auch noch Vorkehrungen für die Gerechtigkeit zu
treffen usw. . Führt man diese Überlegungen weiter fort, so gelangt man
irgendwann einmal zu einer politischen Mindestordnung, die alles umfasst, was
notwendig ist, um die Kernfunktion der Friedenssicherung zu erfüllen. Gehört
zu dieser Mindestordnung bereits die Funktion der
Sinnvermittlung?\footnote{Unter Sinnvermittlung ist zu verstehen, dass die
  politische Ordnung in ihrer Gestalt Ausdruck der in spiritueller Erfahrung
  erlebten sinnhaften Seinsordnung ist, die sie zugleich ihren Mitgliedern
  weitervermittelt. Dies trifft die Intention Voegelins besser als der (an
  sich verständlichere) Ausdruck Sinngebung, da nach Voegelins Verständnis die
  politische Ordnung keinesfalls die Quelle des Sinns ist, sondern idealiter
  in die sinnhafte Gesamtordnung der Welt eingebettet ist.} Nach den
Überlegungen der vorhergehenden Abschnitte ist dies wahrscheinlich nicht der
Fall, denn die politische Ordnung bedarf des Rückgriffs auf die spirituelle
Erfahrung weder zur Legitimation noch um der Begründung verbindlicher Werte
willen, noch ist die spirituelle Erfahrung bei der Bewältigung politischer
Probleme von Vorteil. Also ist die Sinnvermittlungsfunktion, wenn überhaupt,
eine rein optionale Funktion politischer Ordnung, soweit unter politischer
Ordnung die eben angedeutete Mindestordnung zu verstehen ist. Neben der
Sinnvermittlungsfunktion sind noch weitere solcher optionaler Funktionen
politischer Ordnung denkbar (z.B. Sozialstaatlichkeit\footnote{Historisch
  hatte die Entwicklung des Sozialstaates natürlich durchaus einiges mit der
  Sicherung des inneren Friedens zu tun, aber für die theoretische Frage, ob
  und warum der Staat sozialstaatliche Aufgaben übernehmen soll, spielen
  historisch-kontingente Tatsachen nur bedingt eine Rolle.}). Solche
Funktionen der politischen Ordnung zuzurechnen ist dann empfehlenswert, wenn
ihre Erfüllung am ehesten oder sogar einzig und allein auf der Ebene der
politischen Ordnung möglich ist und wenn dabei keine gravierenden Nachteile
entstehen. Nun kann die Sinnvermittlungsfunktion aber zweifellos auch anders
als im Rahmen der politischen Ordnung erfüllt werden. Die Vermittlung von
Lebensinn, die sinnhafte Deutung der Welt und die Erfüllung menschlicher
Transzendenzbedürfnisse kann, wenn schon nicht individuell, so doch auf jeden
Fall im Rahmen von Kirchen und Religionsgemeinschaften geleistet werden. Es
tut der Spiritualität also keinerlei Abbruch, wenn ihr nur ein Platz außerhalb
der politischen Ordnung angewiesen wird, während andererseits nicht einzusehen
ist, welche Vorteile es haben soll, wenn sie der politischen Ordnung
aufgebürdet wird.

Die Tatsache, dass die Spiritualität keineswegs darunter leiden muss, wenn
sie nicht als Bestandteil der politischen Ordnung betrachtet wird,
scheint Voegelin zu übersehen, wenn er es den
Gesellschaftsvertragstheorien zum Vorwurf macht, dass sie sich nur auf
die leibliche Seite des Menschen konzentrieren und die geistige Seite
des Menschen vernachlässigen.\footnote{Vgl. Voegelin, Anamnesis,
  S. 341/342.} Seinem Vorwurf liegt ein fundamentales Missverständnis des
Zweckes politischer Ordnung zu Grunde. Die Notwendigkeit politischer
Ordnung entsteht letztlich aus dem Umstand, dass Menschen einander in die
Quere kommen können und deshalb Abmachungen treffen müssen, damit dies
nicht geschieht. Politik hat daher ihrem Wesen nach mehr mit der
niederen, materiellen Sphäre der unumgehbaren Notwendigkeiten zu tun als
mit der geistigen Sphäre. Es ist deshalb ein Irrtum, von der politischen
Ordnung den Ausdruck spiritueller Wahrheit zu verlangen.  Und der
Verzicht darauf bedeutet keinesfalls eine Leugnung des Geistes, da
gerade nach den liberalen Gesellschaftvertragstheorien die politische
Ordnung gar nicht beansprucht, das ganze Wesen des Menschen zu erfassen.

Umgekehrt wäre es höchst prekär, religiöse Erfahrungen zu einer Angelegenheit
von politischer Bedeutung zu erklären. Denn wenn die politische Ordnung auf
eine Erfahrung der Transzendenz gegründet wird, dann wird die Religiosität zu
einer Frage der politischen Ordnung. Sie dürfte dann nicht mehr im Belieben
des Einzelnen stehen, was erhebliche Probleme für die Religionsfreiheit und
Toleranz aufwirft. Dann wäre es in der Tat nur konsequent, nach dem Irrenarzt
zu rufen, wenn es Menschen geben sollte, die es wagen, die Transzendenz zu
leugnen.  Ansonsten ist die Leugnung der Transzendenz eine sehr harmlose
"`Krankheit"', denn sie beeinträchtigt weder das Lebensglück der Befallenen,
noch hindert sie sie daran, die Rechte ihrer Mitbürger zu
respektieren.\footnote{Dasselbe Argument gilt umgekehrt auch für analoge
  Versuche, die Religion oder die Religiosität als eine psychopathologische
  Erscheinung zu verstehen. Vgl. dazu die sehr vernünftigen Ausführungen von
  Eugene Webb, in: Webb, Review, a.a.O.}

Als Gesamtergebnis lässt sich festhalten, dass weder die politische Ordnung
der Transzendenzerfahrungen bedarf, noch die Realisierung bzw.  der Ausdruck
der Transzendenzerfahrungen durch die politische Ordnung geschehen muss. Da
andererseits die Forderung der Berücksichtigung spiritueller Erfahrungen bei
der Gestaltung politischer Ordnung erhebliche ethische Bedenken hinsichtlich
der Toleranz aufwirft, so ergibt sich, dass Transzendenzerfahrungen bei der
Gestaltung der politischen Ordnung besser keine Rolle spielen sollten. Kurzum:
Wenn es Gott gäbe, müsste man ihn ignorieren -- wenigstens in der Politik.

\chapter{Was bleibt von Eric Voegelin?}
\label{WasBleibt}

Nachdem sich Eric Voegelins Bewusstseinsphilosophie für die Erforschung der
geistigen Grundlagen guter politischer Ordnung als so wenig haltbar erwiesen
hat, erscheint es mir angebracht, einige Überlegungen dazu anzustellen, welche
Rolle Eric Voegelin in der heutigen wissenschaftlichen und politischen
Diskussion noch spielen kann, und in welcher Richtung die Auseinandersetzung
über sein Werk fortzuführen wäre. Dazu werde ich im folgenden drei Aspekte der
Frage der Aktualität von Voegelins Werk ansprechen: 1. Welche Bedeutung kommt
Voegelins Philosophie zu? 2. Wie aktuell sind seine Vorstellungen politischer
Ordnung und politischer Unordnung? 3. Gibt es dennoch ein
politikwissenschaftliches Vermächtnis Eric Voegelins, das fortzuführen sich
lohnt.

\section{Zum Charakter von Voegelins Philosophie}

Die Philosophie Eric Voegelins halte ich, wie aus den bisherigen Ausführungen
sicherlich hervorgegangen ist, nicht für sonderlich geglückt. Dabei halten
nicht nur die Ergebnisse seiner Philosophie einer kritischen Prüfung nicht
stand, auch die Art seines Philosophierens ist keinesfalls nachahmenswert.
Voegelins Philosophie, und dies gilt sowohl für seine Geschichtsphilosophie
als auch für seine Bewusstseinsphilosophie, ist eine überaus {\em dogmatische
  Philosophie}, sie stellt außerdem eine hochgradig {\em monologische
  Philosophie} dar, und darüber hinaus erscheint sie über weite Strecken als
das, was Karl Popper sehr treffend "`{\em orakelnde Philosophien}"' genannt
hat.\footnote{Vgl. Karl Popper: Die offene Gesellschaft und ihre Feine.  Band
  II. Falsche Propheten: Hegel, Marx und die Folgen, 7. Aufl., Tübingen 1992,
  S. 262ff.}

Eine dogmatische Philosophie ist eine Philosophie, die ein Weltbild
artikuliert, ohne es zu begründen. Während eine kritische Philosophie
versucht, ihre Thesen durch Argumente zu begründen, findet bei einer
dogmatischen Philosophie gar keine oder nur eine tautologische Begründung
statt oder eine Begründung durch Voraussetzungen, die ihrerseits nicht weniger
begründungsbedürftig sind als die begründeten Thesen. Damit ist nicht gesagt,
dass dogmatische Philosophien notwendigerweise schlechte Philosophien sind,
denn die Bildung und Ausgestaltung eines Weltbildes (oder auch nur einer
Unternehmensphilosophie) ist weder eine triviale noch eine unbedeutende
Aufgabe, aber dogmatische Philosophien können nur in begrenztem Maße
Objektivität für sich in Anspruch nehmen. Und genau in diesem Sinne ist
Voegelins Philosophie eine hochdogmatische Philosophie. Deutlich wird dies
immer wieder an metaphysischen Voraussetzungen wie Annahme der Existenz eines
transzendenten Seinsgrundes, der Ontologie der Seinsstufen, der Auffassung der
Geschichte als eines theogonischen Prozesses usw. . Voegelins Philosophie wird
dadurch nicht uninteressanter und die Darstellung seines Weltbildes ist ihm
einige Male auch in einer ästhetisch und rhetorisch ansprechenden Weise
gelungen.\footnote{Dies gilt besonders für die Einleitungen von Order and
  History I und II. (Vgl. Voegelin, Order and History I, S. 1ff. -- Vgl.
  Voegelin, Order and History II, S. 1-20.) -- Fast noch schöner hat es aber
  Thomas Hollweck gesagt: Vgl. Thomas Hollweck: Truth and Relativity: On the
  Historical Emergence of Truth, in: Opitz, Peter J. / Sebba, Gregor (Hrsg.):
  The Philosophy of Order. Essays on History, Consciousness and Politics,
  Stuttgart 1981, S. 125-136. -- Für meinen Geschmack jedoch eher misslungen
  und an eine schlechte Predigt erinnernd: Eric Voegelin: Ewiges Sein in der
  Zeit, in: Voegelin, Anamnesis, S. 254-280.} Aber Voegelins Philosophie ist
eben auch nicht mehr als eine Philosophie. Sie ist nicht {\it die}
Philosophie, und es steht jedem Menschen frei, sich zu ihr zu bekennen oder
sie abzulehnen.

Als problematischer stellt sich der monologische Charakter von Voegelins
Philosophie dar. Auch diese Eigenschaft hat Voegelins Philosophie mit der
Philosophie anderer Denker gemeinsam. Der monologische Charakter findet sich
bei Voegelin sowohl auf der Ebene des Philosophierens als auch auf der Ebene
seiner philosophischen Doktrin. Auf der Ebene des Philosophierens äußert sich
der monologische Charakter in Voegelins heftigen polemischen Ausfällen, in
seiner Weigerung, mit jedem, der seine Grundüberzeugungen nicht teilt, auch
nur ein Wort zu reden,\footnote{Vgl. Conversations with Eric Voegelin. (ed. R.
  Eric O'Connor), Montreal 1980, S. 58ff.} und in der fast paranoiden
Vorstellung einer Ansteckungsgefahr, die von der vermeintlichen Krankheit
deformierter Existenz ausgeht, welche er hinter den von ihm unerwünschten
Philosophien allzeit vermutete. Wichtiger noch als diese etwas schrulligen
Äußerungen eines leidenschaftlichen intellektuellen Temperamentes ist die
Rolle des monologischen Prinzips innerhalb von Voegelins Doktrin.
Philosophische Wahrheit wird für Voegelin immer von Einzelnen erfahren und
dann sprachlich an Andere weitervermittelt, was ein überaus schwieriger Prozess
ist, da die Erfahrung, die in gewisser Weise auch eine
Verständnisvoraussetzung bildet, im Anderen durch die sprachliche Vermittlung
erst angeregt werden muss.\footnote{Vgl. auch William C.  Harvard, Jr.: Notes
  on Voegelin's contributions to political theory, in: in: Ellis Sandoz
  (Hrsg.): Eric Voegelins Thought. A critical appraisal, Durham N.C. 1982,
  S. 87-124 (S. 112-113).} Nach dieser Vorstellung von philosophischer Wahrheit
ist es unmöglich, dass Wahrheit im philosophischen Dialog gefunden wird, denn
die Erfahrung eines Menschen kann logischerweise nicht durch Argumente eines
anderen Menschen korrigiert werden. Die typische Gesprächssituation, die
Voegelins Philosophie zu Grunde liegt, ist daher nicht der Dialog unter
Gleichgestellten, sondern stets das belehrende Gespräch, in welchem die Rollen
von Lehrer und Schülern, von Führer und Gefolgsleuten, von Prophet und Jüngern
klar verteilt sind. Problematisch erscheint am monologischen Charakter von
Voegelins Philosophie, dass eine legitime Pluralität von Weltanschauungen
dadurch theoretisch ebenso ausgeschlossen ist, wie die gegenseitige
Befruchtung gegensätzlicher Standpunkte. Pluralismus war für Voegelin beinahe
gleichbedeutend mit Verwirrung, und ein Philosoph, der die Wahrheit
existentiell erfahren hat, kann sich durch andere Standpunkte höchstens noch
beirren lassen.

Für den heikelsten Punkt halte ich allerdings die philosophische
Geheimniskrämerei, zu der Voegelin nicht immer aber in seinen späteren
Schriften immer häufiger neigt. Ein philosophischer Geheimniskrämer ist
jemand, der das Rätsel und das Gefühl des Geheimnisvollen mehr liebt als die
Lösung der Rätsel. Voegelin hat sich in mehrfacher Weise der philosophischen
Geheimniskrämerei befleißigt. Dies beginnt mit Voegelins oft unklarer und
vieldeutiger Ausdrucksweise, es geht fort über die nicht wenigen technischen
Mängel seiner Philosophie, unter denen insbesondere die Schlussfehler der {\it
  petitio principii}, der {\it aquivocatio} und des {\it non sequitur} einen
prominenten Platz einnehmen, und der Höhepunkt ist erreicht, wenn Voegelin
sich auf Paradoxien und Mysterien beruft. Ich gebe zu, dass dies eine höchst
subjektive Kritik ist, und wer in Hegel einen großen Philosophen sieht, der
wird Voegelin wegen seiner Denkfehler gewiss nicht tadeln wollen. Aber mir
scheint, dass ein Philosoph, der sich auf ein Mysterium beruft, mit demselben
Misstrauen betrachtet werden sollte, wie ein Politiker, der sich auf sein
Ehrenwort beruft. Nicht dass von vornherein ausgeschlossen werden kann, dass
es in der Welt Mysterien gibt. Aber bei einem Mysterium hat alles Denken ein
Ende, und unter der Berufung auf Mysterien lässt sich jede beliebige
Behauptung aufstellen. Deshalb sollte zuerst eine genaue Prüfung stattfinden,
bevor die Annahme akzeptiert wird, dass ein Mysterium vorliegt. In dieser
Hinsicht scheint mir Voegelin in der Tat mehr als voreilig gewesen zu sein,
wenn er etwa von einem Paradox des Bewusstseins spricht, obwohl die Tatsache,
dass das Bewusstsein die Welt wahrnehmen kann, von der es selbst zugleich ein
Teil ist, doch bestenfalls eine staunenswerte Besonderheit aber gewiss kein
Paradoxon ist.\footnote{Vgl.  Voegelin, Order and History V, S. 14-15.}
Ebensowenig kann ich mich zu der Ansicht durchringen, dass, wie Voegelin uns
im letzten Band von "`Order and History"' weismachen will, das Wort "`Es"' in
dem Satz "`Es regnet"' auf eine geheimnisvolle "`Es-Realität"' verweist, die
die Partner im Sein: Gott, Welt, Mensch und Gesellschaft
umgreift.\footnote{Vgl. Voegelin, Order and History V, S. 16.} Vielleicht gibt
es Menschen, die in solchen Philosophemen den tiefsten Ausdruck ihres
ureigensten Welterlebens finden können. Für meinen Teil scheint mir jedoch,
dass Voegelin hier alle guten Grundsätze des klaren Denkens in den Wind
schlägt.

Was bleibt aber von Voegelins Philosophie, wenn sie tatsächlich so sehr mit
Irrtümern und Denkfehlern gespickt ist? Sie bleibt immer noch der
Ausdruck einer bestimmten und, wenn man sich an die besseren von Voegelins
Schriften hält, zuweilen reichen und tiefen Weltanschauung. Wenn man die
Aufgabe der Philosophie nicht nur, wie es die analytische Philosophie in der
Tradition des Neupositivismus tut, in der Beantwortung wissenschaftlich
klärbarer Fragen sieht, sondern auch in der Artikulation und Verständigung
über weltanschauliche Überzeugungen, dann ist das immerhin etwas.

\section{Zur Frage der Aktualität von Voegelins Ordnungsentwurf}

Die Frage der Aktualität von Voegelins Ordnungsentwurf bedarf keiner langen
Erörterungen, da die Antwort hierauf eindeutig ausfällt, und sie sich auch in
der wissenschaftlichen Voegelin-Debatte mehr und mehr durchzusetzen
scheint.\footnote{Vgl. Webb, Review, a.a.O.} Voegelins Vorstellung von
politischer Ordnung ist in hohem Maße bedingt und beeinflusst durch das
Zeitalter der Ideologien und des Totalitarismus, in welchem sie entstanden
ist. Voegelin hatte selbst vor dem Nationalsozialismus fliehen müssen. Nicht
minder gegenwärtig waren ihm die Verbrechen der kommunistischen Regime und die
Menschheitsbedrohung durch das atomare Wettrüsten. Unter solchen Bedingungen
kann ein Gefühl von Sicherheit nur schwer aufkommen, und dies erklärt zum Teil
Voegelins polemischen Eifer, welcher sich womöglich einem Gefühl der
Dringlichkeit verdankt, das nach dem Ende des kalten Krieges nicht mehr
unmittelbar verständlich wirkt. Die Zeitumstände erklären auch einiges von
dem, was man die metaphysische Überhöhung des Politischen bei Voegelin nennen
könnte. Aus heutiger Sicht mag es sehr befremdlich und unwissenschaftlich
wirken, die metaphysische Kategorie des Bösen in die Politikwissenschaft
einführen zu wollen.\footnote{Vgl. Voegelin, Eric: Die politischen Religionen,
  München 1996 (zuerst 1938).} Aber um mit einer Erscheinung wie dem
Nationalsozialismus fertig zu werden erscheint dieser Versuch, wiewohl
wissenschaftlich fragwürdig, doch nicht ganz unverständlich.  Vor dem
zeithistorischen Hintergrund ist es daher sehr wohl nachvollziehbar, dass
Voegelin sich nicht auf die Frage beschränkte, welches die geeignetsten
politischen Institutionen für einen guten Staat sind, sondern dem Übel an die
Wurzel gehen wollte und nach den metaphysischen Bedingungen wahrer politischer
Ordnung fragte.

Indes leben wir heute mit einer liberalen politischen Ordnung, die seit
über fünfzig Jahren stabil ist, und die auch keine Risse aufzuweisen scheint,
obwohl sich die Gesellschaft gegenüber den Fünfziger Jahren gewiss noch
weiter säkularisiert hat, was Voegelins Grundthesen über die Ursachen
politischer Unordnung doch sehr zweifelhaft erscheinen lässt. Dazu
vermitteln Voegelins Äußerungen über politische Ordnung nicht selten den
Eindruck, dass es Voegelin weit eher darauf ankam, eine wahre politische
Ordnung (nach den Maßstäben seiner privatreligiösen Überzeugungen) zu
finden als eine im moralischen und pragmatischen Sinne gute politische
Ordnung. Als recht gravierend fällt dabei ins Gewicht, dass Voegelin in
seinem metaphysischen Eifer oft hart an der Grenze zum religiösen
Fanatismus operiert. Seine Suche nach der wahren Ordnung beschwört
dadurch die "`entgegengesetzte Gefahr"' (John H.  Herz\footnote{Vgl.
  John H. Herz: Politischer Realismus und politischer Idealismus.  Eine
  Untersuchung von Theorie und Wirklichkeit, Meisenheim am Glan 1959.
  Mit der "`entgegengesetzten Gefahr"' bezeichnet Herz die besonders dem
  politischen Idealismus inhärente Gefahr bei der Bekämpfung politischer
  Missstände durch das Mittel der Bekämpfung genau den entgegengesetzten
  Missstand herbeizuführen. (Beispiel: Die Bekämpfung des
  kapitalistischen Ausbeutungssystems mündet in die kommunistische
  Diktatur.)}) der religiösen und weltanschaulichen Intoleranz herauf.
Nicht zuletzt aus diesem Grund ist uns heutzutage bei der Suche nach
guter politischer Ordnung mit etwas "`altmodischem Liberalismus"' weitaus
besser gedient als mit Voegelins metaphysischen Rezepturen.

\section{Was sollte dennoch bleiben?}

Wenn Voegelins Philosophie nichts hergibt, und seine politische
Ordnungsvorstellung nichts taugt, wäre es dann nicht besser, Eric Voegelin
ganz zu vergessen? Zwei wichtige Gründe lassen es, trotz aller Kritik,
wünschenswert erscheinen, Eric Voegelin dem drohenden Vergessen zu entreißen.
Zum einen ist da Voegelins imposante Gelehrsamkeit. Voegelins Interpretationen
der Klassiker des politischen Denkens fallen zwar häufig sehr eigenwillig aus
 -- nicht zuletzt deshalb, weil sich Voegelin meist nur auf ganz bestimmte und
scheinbar willkürlich ausgewählte Textpassagen bezieht. Aber Voegelins Auswahl
vollzieht sich fast immer vor dem Hintergrund einer profunden Kenntnis des
Gesamtwerkes. Wenn Voegelin daher auch die falsche Quelle ist, um sich über
die Klassiker des politischen Denkens zu informieren, so dürften Kenner eines
Denkers, den Voegelin behandelt hat, bei Voegelin oft einen
ausgefallenen Kommentar auf hohem intellektuellen Niveau finden. 

Zweitens gilt es die bedeutende kulturwissenschaftliche Horizonterweiterung
festzuhalten, die die Politologie durch Eric Voegelin erfahren hat. Darin
besteht, wenn man so will, das eigentliche Vermächtnis des
Politikwissenschaftlers Voegelin und in dieser Hinsicht ist Voegelin gerade in
der heutigen Zeit von großer Aktualität, denn der Umgang mit fremden Kulturen
erfordert auch für die Politik und die politische Theorie nicht nur eine
Kenntnis der Gesetze und Spielregeln von Diplomatie und Außenpolitik, sondern
auch ein Verständnis dieser Kulturen selbst.  Dabei ist allerdings zu hoffen,
dass Voegelins Denken als Vorbild für ein einfühlendes Verständnis fremder
Kulturen dient, und nicht im Fahrwasser von Samuel Huntingtons "`Zusammenprall
der Kulturen"' zur düsteren Prophetie unüberbrückbarer Gegensätze missbraucht
wird.\footnote{Zur Aktualität Voegelins im Zusammenhang mit Huntingtons
  Theorie: Vgl.  Michael Henkel: Eric Voegelin zur Einführung, Hamburg 1998,
  S. 167.}

Von entscheidender Bedeutung für die Nachwirkung Eric Voegelins dürfte es
jedoch sein, dass die Diskussion um Voegelins Werk mit der notwendigen
kritischen Distanz geführt wird, was die bisherige Sekundärliteratur zu Eric
Voegelin eher vermissen lässt. Der sicherste Weg, Eric Voegelin zu einem
Nischendasein in den Zirkeln religiöser Sektierer zu verdammen, besteht darin,
seine Ressentiments, besonders seinen an Don Quichotte gemahnenden Kampf gegen
echte und vermeintliche Gnostiker in allen Formen und Farben, zu einem
unverzichtbaren Wesensbestandteil seiner Wissenschaft zu
erklären.\footnote{Vgl.  Maben W. Poirier: VOEGELIN-- A Voice of the Cold War
  Era ...? A COMMENT on a Eugene Webb review, in: Voegelin Research News,
  Volume III, No.5, October 1997, auf: http://alcor.concordia.ca/\~{
  }vorenews/v-rnIII5.html (Host: Eric Voegelin Institute, Lousiana State
  University. Zugriff am: 1.8.2007).}  Gerade hier wäre es notwendig, eine
kritische Sonderung vorzunehmen, was bei Voegelin Wissenschaft und was
Vorurteil ist. Immerhin sind zu einer auch kritischen Auseinandersetzung mit
Voegelin mittlerweile schon einige interessante Beiträge
erschienen.\footnote{z.B. Michael Henkel: Positivismuskritik und autoritärer
  Staat.  Die Grundlagendebatte in der Weimarer Staatsrechtslehre und Eric
  Voegelins Weg zu einer neuen Wissenschaft der Politik (bis 1938), München
  2005. -- Sehr deutlich auch: Hans Kelsen: A New Science of Politics. Hans
  Kelsen's Reply to Eric Voegelin's "`New Science of Politics"'. A
  Contribution to the Critique of Ideology (Ed. by Eckhart Arnold),
  Heusenstamm 2004.}  Ein weiterer wichtiger Punkt, der zur Entmystifikation
Voegelins beitragen könnte, ist die Erforschung von Voegelins Biographie,
insbesondere seiner frühen Jahre. Wie einige der inzwischen erschienenen
Studien zu dieser Phase von Voegelins Biographie
zeigen,\footnote{Herausgegriffen seien hier nur: Michael Henkel, a.a.O. --
  Hans-Jörg Sigwart: Das Politische und die Wissenschaft.
  Intellektuell-biographische Studien zum Frühwerk Eric Voegelins, Würzburg
  2005. -- Claus Heimes: Antipositivistische Staatslehre. Eric Voegelin und
  Carl Schmitt zwischen Wissenschaft und Ideologie, München 2004. -- Eckhart
  Arnold: Eric Voegelin als Schüler Hans Kelsens, erscheint voraussichtlich
  Wien 2007.} verlief Voegelins geistige Entwicklung in dieser Zeit viel
spannungsgeladener als seine Autobiographie dies vermuten lässt, denn Voegelin
war gerade in jungen Jahren von jenen irrationalistischen Strömungen der
Geisteskultur der Zwanziger und Dreißiger Jahre nicht wenig beeinflusst, gegen
deren politische Auswüchse er dann später wissenschaftlich zu Felde zog.
Allerdings herrscht, was diesen Teil von Voegelins Biographie angeht
sicherlich immer noch Diskussionsbedarf.

Schließlich könnte eine zukünftige Voegelin-Forschung auch davon profitieren,
wenn sie sich von der, wie es scheint, insgesamt immer noch zu engen Fixierung
auf die Deutung von Voegelins Werk selbst lösen und sich vermehrt den von
Voegelin untersuchten Sachfragen zuwenden würde. Dazu gehört beispielsweise
die Frage nach dem Verhältnis von Religion und Politik oder auch die Frage
möglicher kultureller Vorbedingungen des Gelingens demokratischer
Ordnung. Dies wäre, sollte man meinen, ganz in Voegelins Sinne.

%%% Local Variables: 
%%% mode: latex 
%%% TeX-master: "Main" 
%%% End: 











 
%%% Local Variables: 
%%% mode: latex
%%% TeX-master: Main.tex
%%% End: 

\newpage

\chapter{Literatur}

\setlength{\parindent}{0ex}


\setlength{\parskip}{5ex}

{\large I. Schriften von Eric Voegelin}

\setlength{\parskip}{3ex}

{\bf Voegelin, Eric}: Anamnesis. Zur Theorie der Geschichte und Politik,
München 1966.  % UB 66/5523

\setlength{\parskip}{1.5ex}

{\bf Voegelin, Eric}: Autobiographical Reflections (ed. Ellis Sandoz), Baton
Rouge and London 1996.

{\bf Voegelin, Eric / Schütz, Alfred / Strauss, Leo / Gurwitsch, Aron}:
Briefwechsel über "`Die Neue Wissenschaft der Politik"'. (Hrsg. von Peter
J. Opitz), Freiburg/München 1993.  % 94/3274

{\bf Voegelin, Eric}: On the Form of the american Mind, Baton Rouge / London
1995.

{\bf Voegelin, Eric}: "`Die spielerische Grausamkeit der Humanisten"'. Eric
Voegelins Studien zu Niccolò Machiavelli und Thomas Morus. (Hrsg. von
D. Herz), München 1995.

{\bf Voegelin, Eric}: Die Größe Max Webers. (Hrsg. von Peter J. Opitz),
München 1995.  % UB 96/7988

{\bf Voegelin, Eric}: What is history and other late unpublished writings,
Baton Rouge and London 1989.

{\bf Voegelin, Eric}: Der Liberalismus und seine Geschichte, in: Karl Forster
(Hrsg.): Christentum und Liberalismus, München 1960, S.11-42.

{\bf Voegelin, Eric}: Ordnung, Bewußtsein, Geschichte. Späte Schriften.
(Hrsg. von Peter J. Optiz), Stuttgart 1988.

{\bf Voegelin, Eric}: Order and History. Volume One. Israel and Revelation,
Baton Rouge / London 1986 (zuerst: 1956).

{\bf Voegelin, Eric}: Order and History. Volume Two. The World of the Polis,
Baton Rouge / London 1986 (zuerst: 1957).

{\bf Voegelin, Eric}: Order and History. Volume Three. Plato and Aristotle,
Baton Rouge / London 1986 (zuerst: 1957).

{\bf Voegelin, Eric}: Order and History. Volume Four. The Ecumenic Age, Baton
Rouge / London 1986 (zuerst: 1974).

{\bf Voegelin, Eric}: Order and History. Volume Five. In Search of Order,
Baton Rouge / London 1987.

{\bf Voegelin, Eric}: Die politischen Religionen, München 1996 (zuerst 1938).

{\bf Voegelin, Eric}: The New Science of Politics. An Introduction, Chicago \&
London 1987 (zuerst: 1952).

{\bf Voegelin, Eric}: Der autoritäre Staat. Ein Versuch über das
österreichische Staatsproblem, Wien / New York 1997 (zuerst 1936).

{\bf Voegelin, Eric}: "`Structures of Consciousness"' (ed. Zdravko Planinc),
in: Voegelin-Research News Volume II, No 3, September 1996, auf:
http:""//""alcor.concordia.ca/\~{ }vorenews/v-rnII3.html (Host: Eric Voegelin
Institute, Lousiana State University).

{\bf Voegelin, Eric}: Die deutsche Universität und die Ordnung der deutschen
Gesellschaft, in: Die deutsche Universität im Dritten Reich. Eine
Vortragsreihe der Universität München, München 1966, S.241-282.

{\bf Voegelin, Eric}: Das Volk Gottes. Sektenbewegungen und der Geist der
Moderne (Hrsg. von Peter J. Opitz), München 1994.

{\bf Voegelin, Eric}: Die Neue Wissenschaft der Politik. Eine Einführung,
München 1959.  % UB 59/4508

{\bf Voegelin, Eric}: Wissenschaft, Politik und Gnosis, München 1959.

{\bf Voegelin, Eric}: Die geistige und politische Zukunft der westlichen
Welt. (Hrsg. von Peter J. Opitz und Dietmar Herz), München 1996 
(zuerst als Vortrag im Amerikahaus München 1959).

Conversations with Eric Voegelin. (ed. R. Eric O'Connor), Montreal 1980.


\setlength{\parskip}{5ex}

{\large II. Über Eric Voegelin}

\setlength{\parskip}{3ex}

{\bf Arnold, Eckhart}: Eric Voegelin als Schüler Hans Kelsens, erscheint
voraussichtlich Wien 2007.

\setlength{\parskip}{1.5ex}

{\bf Arnold, Eckhart}: Nachwort: Voegelins "`Neue Wissenschaft"' im Lichte von
Kelsens Kritik, in: Hans Kelsen: A New Science of Politics. Hans Kelsen's
Reply to Eric Voegelin's "`New Science of Politics"'. A Contribution to the
Critique of Ideology (Ed. by Eckhart Arnold), Heusenstamm 2004, S. 109-137.

{\bf Cooper, Barry}: Eric Voegelin and the Foundations of Modern Political
Science, Columbia and London 1999.

{\bf Dahl, Robert A.}: The Science of politics: New and Old, in: World
Politics Vol. VII (April 1955), S.484-489.

{\bf Faber, Richard}: Der Prometheus-Komplex. Zur Kritik der Politotheologie
Eric Voegelins und Hans Blumenbergs, Königshausen 1984.

{\bf Farrell, Thomas J.}: The Key Question. A critique of professor Eugene
Webbs recently published review essay on Michael Franz's work entitled "'Eric
Voegelin and the Politics of Spiritual Revolt: The Roots of Modern Ideology"',
in: Voegelin Research News, Volume III, No.2, April 1997, auf:
http://alcor.concordia.ca/\~{ }vorenews/v-rnIII2.html (Host: Eric Voegelin
Institute, Lousiana State University. Zugriff am: 1.8.2007).

{\bf Germino, Dante}: Eric Voegelin on the Gnostic Roots of Violence, München
1998.

{\bf Heimes, Claus}: Antipositivistische Staatslehre. Eric Voegelin und Carl
Schmitt zwischen Wissenschaft und Ideologie, München 2004.

{\bf Henkel, Michael}: Eric Voegelin zur Einführung, Hamburg 1998.

{\bf Henkel, Michael}: Positivismuskritik und autoritärer Staat. Die
Grundlagendebatte in der Weimarer Staatsrechtslehre und Eric Voegelins Weg zu
einer neuen Wissenschaft der Politik (bis 1938), 2. Aufl., München 2005
(zuerst: 2003).

{\bf Hollweck, Thomas}: Der Dichter als Führer. Dichtung und Repräsentanz in
Voegelins frühen Arbeiten, München 1996.

{\bf Hughes, Glenn} (Ed.) The Politics of the Soul. Eric Voegelin on Religious
Experience, Lanham / Boulder / New York / Oxford 1999.

{\bf Kelsen, Hans}: A New Science of Politics. Hans Kelsen's Reply to Eric
Voegelin's "`New Science of Politics"'. A Contribution to the Critique of
Ideology (Ed. by Eckhart Arnold), Heusenstamm 2004.

{\bf Kiel, Albrecht}: Gottesstaat und Pax Americana. Zur Politischen Theologie
von Carl Schmitt und Eric Voegelin, Cuxhaven / Dartford 1998.

{\bf Mayer-Tasch, Peter Cornelius}: Auf der Suche nach einer Ordnung der
historischen Ordnungen. Das Hauptwerk des Politikwissenschaftlers und
Geschichtsphilosophen Eric Voegelin, in: Neue Züricher Zeitung, 28. April
2007, Ressort Literatur und Kunst.

{\bf McAllister, Ted V.}: Revolt against modernity. Leo Strauss, Eric Voegelin
\& the Search For a Postliberal Order, Kansas 1995.

{\bf McKnight, Stephen A. / Geoffrey L. Price} (Hrsg.): International and
Interdisciplinary Perspectives on Eric Voegelin, Missouri 1997.

{\bf Morrissey, Michael P.}: Consciousness and Transcendence. The Theology of
Eric Voegelin, Notre Dame 1994.

{\bf Nida-Rümelin, Julian}: Das Begründungsproblem bei Eric
Voegelin. (Typoskript eines Vortrages beim Internationeln Eric-Voegelin
Symposium in München August 1998, Eric Voegelin-Archiv in München)

{\bf Opitz, Peter J. / Sebba, Gregor} (Hrsg.): The Philosophy of Order. Essays
on History, Consciousness and Politics, Stuttgart 1981.

{\bf Opitz, Peter J.}:: Rückkehr zur Realität: Grundzüge der
  politischen Philosophie Eric Voegelins, in: Peter J.  Opitz /
  Gregor Sebba (Hrsg.): The Philosophy of Order. Essays on History,
  Consciousness and Politics, Stuttgart 1981, S. 21-73.

{\bf Petropulos, William}: The Person as `Imago Dei'. Augustine and Max
Scheler in Eric Voegelins `Herrschaftslehre' and `The Political Religions',
München 1997.

{\bf Poirier, Maben W.}: VOEGELIN-- A Voice of the Cold War Era ...? A COMMENT
on a Eugene Webb review, in: Voegelin Research News, Volume III, No.5, October
1997, auf: http://alcor.concordia.ca/\~{ }vorenews/v-rnIII5.html (Host:
Eric Voegelin Institute, Lousiana State University. Zugriff am: 1.8.2007).

{\bf Sandoz, Ellis} (Hrsg.): Eric Voegelin's significance for the modern
mind, Lousiana 1991.

{\bf Sandoz, Ellis} (Hrsg.): Eric Voegelin's Thought. A critical appraisal,
Durham N.C. 1982.

{\bf Schmölz, Franz-Martin} (Hrsg.): Das Naturrecht in der politischen
Theorie, Wien 1963.

{\bf Sigwart, Hans-Jörg}: Das Politische und die Wissenschaft.
Intellektuell-biographische Studien zum Frühwerk Eric Voegelins,
Würzburg 2005.

{\bf Webb, Eugene}: Eric Voegelin. Philosopher of History, Seattle and London
1981.

{\bf Webb, Eugene}: Philosophers of Consciousness. Polanyi,
  Lonergan, Voegelin, Ricoeur, Girard, Kierkegaard, Seatle and London 1988.

{\bf Webb, Eugene}: Review of Michael Franz, Eric Voegelin and the
Po\-li\-tics of Spiritual Revolt: The Roots of Modern Ideology, in:
Voe\-ge\-lin Research News, Volume III, No. 1, February 1997, auf:
http:""//""alcor"".""concordia"".ca/\~{ }vorenews/v-rnIII2.html (Host: Eric
Voegelin Institute, Lousiana State University. Zugriff am: 1.8.2007).

{\bf Weiss, Gilbert}: Theorie, Relevanz und Wahrheit. Zum Briefwechsel
zwischen Eric Voegelin und Alfred Schütz (1938-1959), München 1997.



\setlength{\parskip}{5ex}

{\large III. Weitere Sekundärliteratur}

\setlength{\parskip}{3ex}

{\bf Albert, Hans}: Kritischer Rationalismus. Vier Kapitel zur Kritik
illusionären Denkens, Tübingen 2000.

\setlength{\parskip}{1.5ex}

{\bf Albert, Hans}: Kritische Vernunft und menschliche Praxis, Stuttgart 1984.

{\bf Aristoteles}: Metaphysik. Schriften zur Ersten Philosophie (Hrsg. und
übersetzt von Franz F. Schwarz), Stuttgart 1984.

{\bf Augustinus, Aurelius}: Bekenntnisse, Stuttgart 1998.

{\bf Ayer, Alfred J.}: Language, Truth and Logic, New York [u.a.] 1982.

{\bf Baumanns, Peter}: Kants Philosophie der Erkenntnis. Durchgehender
Kommentar zu den Hauptkapiteln der "`Kritik der reinen Vernunft"', Würzburg
1997.

{\bf Benz, August}: Popper, Platon und das "`Fundamentalproblem der
politischen Theorie"': eine Kritik, in: Zeitschrift für Politik 1999.

{\bf Bergson, Henri}: Materie und Gedächnis, Hamburg 1991.

{\bf Bergson, Henri}: Die beiden Quellen der Moral und Religion, Olten 1980.

{\bf Blumenberg, Hans}: Die Legitimität der Neuzeit. Erneuerte Ausgabe,
Frankfurt am Main 1996.

{\bf Bodin, Jean}: Colloquium of the Seven about Secrets of the
Sublime. Colloquium Heptaplomeres de Rerum Sublimium Arcanis Abditis, Princton
1975.

{\bf Bodin, Jean}: Sechs Bücher über den Staat. Buch IV - VI. (Hrsg. von
P.C. Mayer-Tasch), München 1986.

{\bf Brisson, Luc}: Einführung in die Philosophie des Mythos. Antike,
Mittelalter und Renaissance. Band I, Darmstadt 1996.

{\bf Brumlik, Micha}: Die Gnostiker. Der Traum von der Selbsterlösung des
Menschen, Frankfurt am Main 1992.

{\bf Buber, Martin} (Hrsg.): Ekstatische Konfessionen, Leipzig 1921. 

{\bf Camus, Albert}: Der Mensch in der Revolte. Essays, Hamburg 1997 (zuerst
1951).

{\bf Camus, Albert}: Der Mythos von Sisyphos. Ein Versuch über das Absurde,
Hamburg 1998 (zuerst 1942).

{\bf Camus, Albert}: Tagebücher 1935-1951, Hamburg 1997.

{\bf Camus, Albert}: Tagebuch März 1951 - Dezember 1959, Hamburg 1997.

{\bf Cassirer, Ernst}: Versuch über den Menschen. Einführung in eine
Philosophie der Kultur, Hamburg 1996.

{\bf Denzer, Horst} (Hrsg.): Jean Bodin. Verhandlungen der internationalen
Bodin Tagung in München, München 1973.

{\bf Descartes, René}: Meditationen über die Grundlagen der Philosophie,
Hamburg 1993.

{\bf Dostojewskij, Fjodor}: Die Brüder Karamasow, Frankfurt am Main 2006

{\bf Dostojewski, Fjodor M.}: Die Dämonen, 20.Aufl., München 1996.

{\bf Fest, Joachim}: Die schwierige Freiheit. Über die offene Flanke der
offenen Gesellschaft, Berlin 1993.

{\bf Filmer, Sir Robert}: Patriarcha, or the Natural Power of Kings, England
1680.

{\bf Forster, Karl} (Hrsg.): Christentum und Liberalismus, München
1960. (Studien und Berichte der katholischen Akademie in Bayern.)

{\bf Green, Donald / Shapiro, Ian}: The Pathologies of Rational Choice Theory.
A Critique of Applications in Political Science, New Haven \& London 1994.

{\bf James, William}: The Varieties of religious Experience, Cambridge,
Massachusetts / London, England 1985 (zuerst 1902).

{\bf Hersch, Jeanne}: Karl Jaspers. Eine Einführung in sein Werk, 4. Aufl.,
München 1990.

{\bf Herz, John H.}: Politischer Realismus und politischer Idealismus.  Eine
Untersuchung von Theorie und Wirklichkeit, Meisenheim am Glan 1959.

{\bf Hobbes, Thomas}: Leviathan oder Stoff, Form und Gewalt eines kirchlichen
und bürgerlichen Staates, Frankfurt am Main 1998 (erste Auflage 1984).

{\bf Husserl, Edmund}: Die Krisis der europäischen Wissenschaften und die
transzendentale Phänomenologie, Hamburg 1996.

{\bf Husserl, Edmund}: Cartesianische Meditationen, Hamburg 1987.

{\bf Husserl, Edmund}: Die phänomenologische Methode. Ausgewählte Texte
I. (Hrsg. von Klaus Held), Stuttgart 1985.

{\bf Kuhn, Helmut} (Hrsg.): Die Philosophie und die Frage nach dem
Fortschritt, München 1964.

{\bf Kant, Immanuel}: Kritik der praktischen Vernunft, Hamburg 1990.

{\bf Kant, Immanuel}: Kririk der reinen Vernunft, Hamburg 1976.

{\bf Kant, Immanuel}: Schriften zur Geschichtsphilosophie, Stuttgart 1985.

{\bf Kant, Immanuel}: Träume eines Geistersehers, erläutert durch Träume der
Metaphysik, in: Frank-Peter Hansen (Hrsg.): Philosophie von Platon bis
Nietzsche, CD-ROM, Berlin 1998. (folgt: Immanuel Kant: Werke in zwölf Bänden.
Herausgegeben von Wilhelm Weischedel, Frankfurt am Main 1977. Band 2).

{\bf Kelsen, Hans}: Was ist Gerechtigkeit?, 2. Auflage, Wien 1975.

{\bf Landgrebe, Ludwig}: Phänomenologie und Geschichte, Gütersloh 1967.

% {\bf Lévinas, Emmanual}: Wenn Gott ins Denken einfällt. Diskurse über die
% Betroffenheit von Transzendenz, München, 4. Auflage 2004.

{\bf Lübbe, Hermann}: Säkularisierung. Geschichte eines ideenpolitischen
Begriffs, München 1965.

{\bf Mann, Thomas}: Essays. Band 5: Deutschland und die Deutschen 1938-1945.
(Hrsg. von Hermann Kurzke und Stephan Stachorski), Frankfurt am Main 1996.

{\bf McCosh, James}: The Schottish Philosophy, 1875, (ed. 1995 by James
Fieser) auf: ""http:""//""socserv2"".""socsci"".""mcmaster"".""ca""/\~{ }econ/""ugcm""/""3ll3""/""mccosh""/""scottishphilosophy.pdf"" 
(Archive for
the history of economic thought, McMaster University, Hamilton, Canada;
letzter Zugriff am: 30.3.2005).

{\bf Merlau-Ponty, Maurice}: Humanismus und Terror, Frankfurt am Main 1990
(entstanden 1946/47).

{\bf Mises, Richard von}: Kleines Lehrbuch des Positivismus. Einführung in die
empiristische Wissenschaftsauf\/fassung, Frankfurt am Main 1990 (zuerst: Den
Haag 1939).

{\bf Platon}: Der Staat, Stuttgart 1997

{\bf Popper, Karl R.}: Das Elend des Historizismus, 6.Aufl., Tübingen 1987.

{\bf Popper, Karl R.}: Die offene Gesellschaft und ihre Feinde. Band I. Der
Zauber Platons, 7.Aufl., Tübingen 1992.

{\bf Popper, Karl R.}: Die offene Gesellschaft und ihre Feinde. Band II.
Falsche Propheten: Hegel, Marx und die Folgen, 7.Aufl., Tübingen 1992.

{\bf Reid, Thomas}: Essays on the intellectual powers of
man. (Ed. A.D. Woozley), London 1941.

{\bf Reid, Thomas}: An Inquiry into the human mind on the principles of common
sense, Edinburgh 1997. 

{\bf Ritter, Joachim / Gründer, Karlfried}: Historisches Wörterbuch der
Philosophie. Band 5: L-Mn, Basel / Stuttgart 1980.

{\bf Russell, Bertrand}: A History of Western Philosophy, London / Sydney /
Wellington 1990.

{\bf Schelling, Friedrich Wilhelm}: Philosophie der Offenbarung, in:
Frank-Peter Hansen (Hrsg.): Philosophie von Platon bis Nietzsche, CD-ROM,
Berlin 1998. (folgt der Ausgabe: Friedrich Wilhelm Joseph von Schelling:
Werke. Auswahl in drei Bänden. Herausgegeben und eingeleitet von Otto Weiß.
Leipzig 1907. Band 3.)

{\bf Shapiro, Ian}: The Flight from Reality in the Human Sciences, Princeton
2005.

{\bf Stadler, Friedrich}: Studien zum Wiener Kreis. Ursprung, Entwicklung und
Wirkung des Logischen Empirismus im Kontext, Frankfurt am Main 1997.

{\bf Tolstoi, Leo N.}: Anna Karenina, München 1992.

{\bf Topitsch, Ernst} (Hrsg.): Werturteilsstreit, Darmstadt 1971.

{\bf Weber, Max}: Gesammelte Aufsätze zur Wissenschaftslehre. (Hrsg. von
Johannes Winckelmann), Tübingen 1988. 

A Book of Contemplation wich is called the Cloud of Unknowing, in which a Soul
is oned with God. (ed. Evelyn Underhill, 2nd ed.  John M. Watkins), London
1922, auf: http://www.ccel.org/u/unknowing/cloud.htm (Host: Christian Classics
Ethereal Library at Calvin College. Zugriff am: 1.8.2007).


% McCosh http:///www.utm.edu/research/iep/text/mccosh/mccosh.htm

%%% Local Variables: 
%%% mode: latex
%%% TeX-master: "Main"
%%% End: 


\newpage

\chapter*{Zusammenfassung}

\setlength{\parindent}{0cm}

\setlength{\parskip}{0.5cm}

{\bf Zusammenfassung in deutscher Sprache:} Eric Voegelin glaubte, dass eine
moralisch akzeptable und langfristig erfolgreiche (und das hieß für den
Emigranten Voegelin vor allem: totalitarismusresistente) politische Ordnung
nur auf Grundlage einer gesunden Religiosität der Bürger und insbesondere des
politischen Führungspersonals errichtet werden kann. Der Frage, wie eine
gesunde Religiosität bzw. ein gesundes Transzendenzbewusstein beschaffen sein
muss, versuchte Voegelin sowohl durch geistesgeschichtliche als auch durch
bewusstseinsphilosophische Untersuchungen nachzuspüren. In diesem Buch wird
die Bewusstseinsphilsophie Voegelins und die sich darauf gründende politische
Ordnungsvorstellung einer eingehenden Kritik unterzogen. Im Ergebnis führt
dies zu einer Absage an die politische Theologie Voegelinscher oder auch
anderer Prägung und zu einem entschiedenen Plädoyer für die Trennung von Religion
und Politik.

{\bf English Abstract:} Eric Voegelin believed that a morally acceptable and
in the long run successful political order (which meant for the emigrant
Voegelin primarily an order that is resistant to totalitarianism) can only be
built on the foundation of a healthy religiosity of the citizens and the
political leaders. The question of what a healthy religiosity or a healthy
consciousness of the transcendent is was examined by Voegelin by recurring to
intellectual history and to the philosophy of consciousness. In this book a
detailed criticism not only of Voegelin's philosophy of consciousness but also
of the concept of political order based on this philosophy will be given. This
results in a rejection of political theology of a Voegelinian or other brand
and a resolute defense of the separation of religion and politics.

{\bf Über den Autor:} Eckhart Arnold (Jahrgang 1972) hat in Bonn 
Politische Wissenschaften, Öffentliches Recht und Philosophie studiert. Nach
seinem Magisterabschluss im Jahr 2000 hat er zunächst für mehrere Jahre
an der Erfurt School of Public Policy in der "`Entwicklung multimedialer 
Lehrmethoden im Bereich Public Policy"' gearbeitet. Anschließend war er als 
wissenschaftlicher Mitarbeiter am Lehrstuhl für Theoretische Philosophie in
Düsseldorf tätig, wo er vor kurzem seine Dissertation zu dem Thema
"`Explaining Altruism. A Simulation-Based Approach and its Limits"' 
fertig gestellt hat. Seit Oktober 2007 ist er wissenschaftlicher Mitarbeiter
an der Universität Bayreuth im Studiengang "`Philosophy \& Economics"'.



\end{document}










