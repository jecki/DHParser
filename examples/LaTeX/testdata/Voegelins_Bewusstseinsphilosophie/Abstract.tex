\newpage

\chapter*{Zusammenfassung}

\setlength{\parindent}{0cm}

\setlength{\parskip}{0.5cm}

{\bf Zusammenfassung in deutscher Sprache:} Eric Voegelin glaubte, dass eine
moralisch akzeptable und langfristig erfolgreiche (und das hieß für den
Emigranten Voegelin vor allem: totalitarismusresistente) politische Ordnung
nur auf Grundlage einer gesunden Religiosität der Bürger und insbesondere des
politischen Führungspersonals errichtet werden kann. Der Frage, wie eine
gesunde Religiosität bzw. ein gesundes Transzendenzbewusstein beschaffen sein
muss, versuchte Voegelin sowohl durch geistesgeschichtliche als auch durch
bewusstseinsphilosophische Untersuchungen nachzuspüren. In diesem Buch wird
die Bewusstseinsphilsophie Voegelins und die sich darauf gründende politische
Ordnungsvorstellung einer eingehenden Kritik unterzogen. Im Ergebnis führt
dies zu einer Absage an die politische Theologie Voegelinscher oder auch
anderer Prägung und zu einem entschiedenen Plädoyer für die Trennung von Religion
und Politik.

{\bf English Abstract:} Eric Voegelin believed that a morally acceptable and
in the long run successful political order (which meant for the emigrant
Voegelin primarily an order that is resistant to totalitarianism) can only be
built on the foundation of a healthy religiosity of the citizens and the
political leaders. The question of what a healthy religiosity or a healthy
consciousness of the transcendent is was examined by Voegelin by recurring to
intellectual history and to the philosophy of consciousness. In this book a
detailed criticism not only of Voegelin's philosophy of consciousness but also
of the concept of political order based on this philosophy will be given. This
results in a rejection of political theology of a Voegelinian or other brand
and a resolute defense of the separation of religion and politics.

{\bf Über den Autor:} Eckhart Arnold (Jahrgang 1972) hat in Bonn 
Politische Wissenschaften, Öffentliches Recht und Philosophie studiert. Nach
seinem Magisterabschluss im Jahr 2000 hat er zunächst für mehrere Jahre
an der Erfurt School of Public Policy in der "`Entwicklung multimedialer 
Lehrmethoden im Bereich Public Policy"' gearbeitet. Anschließend war er als 
wissenschaftlicher Mitarbeiter am Lehrstuhl für Theoretische Philosophie in
Düsseldorf tätig, wo er vor kurzem seine Dissertation zu dem Thema
"`Explaining Altruism. A Simulation-Based Approach and its Limits"' 
fertig gestellt hat. Seit Oktober 2007 ist er wissenschaftlicher Mitarbeiter
an der Universität Bayreuth im Studiengang "`Philosophy \& Economics"'.

