\chapter{Die Schlüsselfrage: Braucht Politik spirituelle Grundlagen?}
\label{SpirituellePolitik}
Nachdem im zweiten Teil dieses Buches Voegelins Bewusstseinsphilosophie
überwiegend unter einer philosophischen Perspektive betrachtet wurde, soll nun
wieder die Beziehung zur Politik hergestellt werden. Voegelin ging es mit
seinen bewusstseinsphilosophischen Untersuchungen nicht bloß um die seelische
Ordnung der menschlichen Einzelexistenz, sondern vor allem auch um die
politische Ordnung der Gesellschaft. Nicht umsonst trägt die große
bewusstseinsphilosophische Abhandlung, die den Schlussteil seines Werkes
"`Anamnesis"' bildet, die Frage nach der politischen Realität im Titel. Im
folgenden wird zunächst versucht zu klären, inwiefern für Voegelin spirituelle
Erfahrungen die Voraussetzung guter politischer Ordnung bilden und von welcher
Gestalt eine politische Ordnung ist, die die spirituelle Erfahrung nach
Voegelins Maßstäben in angemessener Weise berücksichtigt. Anschließend wird,
losgelöst von Voegelins Theorie, in Bezug auf einige Grundfragen politischer
Ordnung untersucht, ob politische Ordnung ohne eine spirituelle Grundlage
auskommen kann.

\section{Spirituelle Wahrheit und politische Ordnung bei Voegelin}

Bereits bei der Untersuchung von Voegelins Bewusstseinsphilosophie fiel auf,
dass die Zusammenhänge zwischen den spirituellen Erfahrungsgrundlagen
politischer Ordnung und der politischen Ordnung selbst, die sich als
rechtliche und institutionelle Ordnung einer Gesellschaft konkretisiert,
merkwürdig im Dunkeln bleiben. Zwar lässt Voegelin keine Gelegenheit aus, um
vor den verhängnisvollen Folgen zu warnen, die ein Verlust des
Erfahrungskontakts zum transzendenten Seinsgrund nach seiner Überzeugung
unweigerlich nach sich zieht, aber diese Warnungen sind wissenschaftlich kaum
präziser, als es die pauschale Behauptung wäre, dass alles Unheil unserer Zeit
eine Folge der menschlichen Gottlosigkeit sei. Alles läuft bei Voegelin
letztlich auf die anthropologische These hinaus, dass der Mensch des Kontakts
zum Seinsgrund bedarf, um seine Existenz zu ordnen, und dass keine politische
Ordnung erzielt werden kann, wenn nicht sowohl auf der Seite der Herrschenden
als auch auf der Seite der Beherrschten der in genau dieser Weise existentiell
geordnete Menschentypus dominiert. Man mag einwenden, dass Voegelin im Rahmen
seiner bewusstseinsphilosophischen Untersuchungen aus Gründen der thematischen
Beschränkung diese Punkte nur habe andeuten können. Doch auch in seinen
anderen Schriften beschäftigt sich Voegelin fast ausschließlich mit der
geistigen Seite politischer Ordnung und fast nie mit dem Zusammenhang der
geistigen Grundlagen und der konkreten Machtordnung, wiewohl er an der
Ansicht, dass zwischen beiden Bereichen eine unmittelbare Beziehung besteht,
offenbar keinerlei Zweifel hat.

Wie wenig erklärende Kraft Voegelins Theorie hat, wenn er tatsächlich einmal
versucht, mit ihr die Ursachen gesellschaftlicher Unordnung zu beschreiben,
lässt sich am besten an einem Beispiel nachvollziehen: In seinem 1959
gehaltenen Vortrag "`Die geistige und politische Zukunft der westlichen Welt"'
stellt Voegelin ein "`Gesetz der westlichen Ordnung"' auf, welches besagt,
dass es drei "`Autoritätsquellen"' der Ordnung gibt, erstens die
herrscherliche Macht, zweitens die Vernunftphilosophie und drittens die
religiöse Offenbarung, und dass Ordnung herrscht, solange diese
Autoritätsquellen relativ autonom voneinander bleiben, Unordnung aber dann,
wenn sie zusammenfallen.\footnote{ Eric Voegelin: Die geistige und politische
  Zukunft der westlichen Welt. (Hrsg. von Peter J. Opitz und Dietmar Herz),
  München 1996, im folgenden zitiert als: Voegelin, Zukunft der westlichen
  Welt, S. 21-23.} Verfolgt man nun Voegelins weitere Ausführungen zu diesem
Gesetz, so springen einige Merkwürdigkeiten ins Auge: Zunächst einmal
unternimmt Voegelin keinen Versuch zu erklären, weshalb aus dem Zusammenfallen
der drei Autoritätsquellen gesellschaftliche oder politische Unordnung
resultiert. Solange Voegelin dies nicht demonstriert, liefert sein Gesetz nur
eine völlig willkürliche Definition von "`Unordnung"', die mit innerem
Unfrieden, chaotischen Zuständen oder tyrannischen Übergriffen des Staates gar
nichts zu tun haben muss.\footnote{Dies gilt umso mehr, als nach Voegelins
  "`Gesetz"' auch im Reich des Kaisers Justinian, anhand von dessen {\it
    constitutio imperatoria majestas} Voegelin sein "`Gesetz"' entwickelt,
  größte Unordnung geherrscht haben müsste, da ja der Kaiser alle drei
  Autoritätsquellen in seiner Person vereinte. (Vgl. ebd.)} Des weiteren stützt
sich Voeglins folgende Argumentation nur marginal auf das gerade erst
aufgestellte Gesetz. Es zeigt sich, dass es Voegelin keineswegs auf die
Autonomie der Autoritätsquellen ankommt, -- sonst müsste er ja auch die
Trennung von Staat und Kirche und die Abspaltung der Naturwissenschaft von der
Philosophie befürworten\footnote{Vgl. ebd., S. 31, S. 34.} -- sondern darauf,
dass christlicher Glaube und Philosophie (mit der selbstverständlich nur die
von Voegelin favorisierten Richtungen der Philosophie gemeint
sind\footnote{Vgl.  ebd., S. 35.}) einen bestimmenden Einfluss auf Gesellschaft
und Politik erlangen. Dafür ist Voegelin sogar bereit, bemerkenswerte
Einschränkungen der demokratischen Rechte hinzunehmen.  So fordert er "`sehr
energisch mit Parteiverboten"'\footnote{Ebd., S. 33. -- Nicht immer hat
  Voegelin derart drastische Forderungen aufgestellt. Aber wenn schon nicht
  mit Verboten, dann sollte zumindest durch starke informelle Mechanismen
  sichergestellt werden, dass religiös ungeeignete Leute von einer politischen
  Führungsrolle effektiv ausgeschlossen bleiben.}  gegen Parteien
"`antichristlicher oder antiphilosophischer Art"'\footnote{Ebd.} vorzugehen.
Der Grund für diese radikale Forderung liegt dabei einzig in Voegelins
vorgefasster Meinung, dass die westlichen Demokratien sich nur halten können,
wenn die Regierung im christlichen Geiste über eine weitgehend christliche
Bevölkerung regiert.\footnote{Vgl. ebd., S.  32-33.}  Warum sie nur unter
dieser Bedingung funktionieren können, dafür liefert Voegelin trotz seiner
historisch weitausholenden Erörterungen keinerlei Begründung.

Ob die Berücksichtigung der spirituellen Erfahrung für die Herstellung
politischer Ordnung überhaupt irgendwelche Vorteile erbringt, lässt sich nicht
zuletzt deshalb nur schwer klären, weil Voegelin niemals deutlich mitteilt,
von welcher Gestalt eine optimal erfahrungsbegründete politische Ordnung sein
würde.\footnote{Wie ich an anderer Stelle ausgeführt habe, würde Voegelins
  Vorgaben am ehesten eine theokratische politische Ordnung, etwa so wie sie
  im heutigen Iran existiert, entsprechen. Vgl. Arnold, Nachwort zu Kelsens
  Voegelin-Kritik, S. 125-127.}  Versucht man sich hilfsweise an Voegelins
historische Beispiele zu halten, dann erhält man ein recht irritierendes Bild.
So ist für Voegelin beispielsweise im christlichen Mittelalter vor der
Reformation mit der Trennung von geistlicher und weltlicher Autorität bei der
gleichzeitigen Legitimation und Gestaltung der weltlichen Ordnung nach
religiösen Prinzipien eine optimale Verwirklichung spirituell
erfahrungsbegründeter politischer Ordnung gegeben. Dies gilt umso mehr, als
nach Voegelins Einschätzung im mittelalterlichen Christentum die bislang
größte Erfahrungshelle des Ordnungswissens erreicht worden ist.  Gleichzeitig
herrscht jedoch mit der hierarchischen Gesellschaftsform und dem feudalen
Herrschaftssystem im Mittelalter eine politische Ordnung, die alles andere als
human und gerecht ist. Ein weiteres irritierendes Beispiel stellt die
politische Philosophie Platons dar. Für Voegelin war Platon ein Philosoph von
größter Offenheit der Seele und höchster spiritueller Empfindsamkeit. Aber die
politisch-institutionelle Ordnung, die Platon im "`Staat"' entworfen hat,
bildet geradezu das Musterbeispiel einer totalitären
Schreckensutopie.\footnote{Vgl. dazu die bekannte Kritik in: Karl Popper: Die
  offene Gesellschaft und ihre Feinde. Band I. Der Zauber Platons, 7.Aufl.,
  Tübingen 1992, S. 104ff. -- Poppers Deutung ist freilich nicht unumstritten.
  Außer einem in der Tat verfälschenden Platon-Zitat auf dem Umschlag (in der
  Taschenbuchausgabe: S. 9.) wird ihm unter anderem vorgeworfen, sich bei
  seiner Kritik an dem personenzentrierten Ansatz Platons zu sehr auf den
  "`Staat"' zu konzentrieren, und den stärker institutionellen Ansatz der
  "`Gesetze"' zu vernachlässigen. (Vgl. August Benz: Popper, Platon und das
  "`Fundamentalproblem der politischen Theorie"': eine Kritik, in: Zeitschrift
  für Politik 1999.) Von dieser Kritik unberührt bleibt allerdings Poppers
  massiver Vorwurf der Inhumanität gegen Platon.}  Hält man sich diese
Beispiele vor Augen, so erscheint es geradezu absurd, dass Voegelin der
Wiedererlangung einer spirituellen Realitätserfahrung vermittels der Öffnung
der Seele eine so große Bedeutung beimisst. Eher müsste man den Schluss ziehen,
dass für gute politische Ordnung ein niedriges spirituelles Niveau von Vorteil
ist. Gewiss, die soeben gegebenen Beispiele sind Extrembeispiele, denn
Voegelin befürwortete auch die amerikanische Demokratie, die in der Tat eine
sehr erfolgreiche Verwirklichung humaner und gerechter politischer Ordnung
darstellt, und nach Voegelins Ansicht steht der politischen Philosophie
Platons die des Aristoteles, welche wesentlich vernünftiger ist, in nichts
nach. Aber immer noch stellt sich dann die Frage, ob empirisch überhaupt eine
Korrelation zwischen dem Niveau der spirituellen Erfahrung und der Güte der
politischen Ordnung festgestellt werden kann.

Dieser Frage, ob politische Ordnung überhaupt eine spirituelle Grundlage
benötigt, um eine erfolgreiche und gerechte politische Ordnung zu sein, soll
im folgenden nachgegangen werden.

\section{Gibt es spirituelle Sachzwänge?}
\label{spirituelleSachzwänge}

Dasjenige, was Voegelins Philosophie der politischen Ordnung so hoffnungslos
anachronistisch erscheinen lässt, ist die Tatsache, dass er spezifische
religiöse Glaubensüberzeugungen als objektiv wahr voraussetzt. Voegelin hält
es für eine unbestreitbare Tatsache, dass es einen transzendenten Seinsgrund
gibt und dass wir zu diesem transzendenten Seinsgrund in einer Beziehung
stehen, die er mit Ausdrücken wie "`Spannung zum Grund"' beschreibt. Dass
Voegelin seine religiösen Vorstellungen stets nur mit vagen und undeutlichen
Worten umschreibt, darf nicht darüber hinwegtäuschen, dass er ihre Wahrheit
streng dogmatisch voraussetzt. Zudem bildet der Vorwurf der Ignoranz und der
Fehldeutung dieser religiösen Wahrheiten seine beinahe einzige Erklärung für
jede Art politischer Unordnung. Es ist daher nicht verkehrt zu sagen, dass
Voegelin diese religiösen Wahrheiten für eine Art von spirituellen Sachzwängen
hält, die die Politik berücksichtigen muss, z.B.  indem sie -- wie
schon zitiert -- "`sehr energisch mit Parteiverboten"'\footnote{Voegelin,
  Zukunft der westlichen Welt, S. 33.}  gegen Parteien "`antichristlicher oder
antiphilosophischer Art"'\footnote{Ebd.} durchgreift.

Die Vorstellung, dass die Politik spirituelle Sachzwänge berücksichtigen muss,
mag nun heutzutage in der westlichen Welt vollkommen anachronistisch
erscheinen. Die Tatsache aber, dass diese Vorstellung in dem sicherlich
größten Teil der Menschheitsgeschichte als selbstverständlich galt, wenn nicht
gar eine bestimmende Rolle gespielt hat, und dass ein Politikwissenschaftler
wie Voegelin sie in der Gegenwart artikulieren und ernst genommen werden
konnte, lässt es nicht ungeraten erscheinen, sie einmal ganz naiv als Frage zu
formulieren und zu untersuchen, ob es nicht vielleicht tatsächlich spirituelle
Sachzwänge der ein oder anderen Art gibt. Hierbei sind drei Varianten der
These zu unterscheiden:

1. Der ersten denkbaren Variante zufolge gibt es {\em objektive spirituelle
  Sachzwänge}, soll heißen: Wenn die Menschen das transzendente Sein nicht in
der gebührenden Weise berücksichtigen, so zieht dies ernste Konsequenzen
seitens des transzendenten Seins nach sich. Sintfluten, Schwefelregen und
andere Unbill, mit der zornige Götter die zuchtlose Menschenbrut zu strafen
pflegen, sind dann mindestens die Folge. Da die Härte solcher Strafen nicht
immer nur die jeweiligen Missetäter, sondern unter Umständen die gesamte
Gemeinschaft trifft, wäre es auch eine Aufgabe der Politik, durch Maßnahmen,
die dem Erhalt der spirituellen Volksgesundheit dienen, die Bevölkerung zu
schützen.

Es gibt allerdings mindestens zwei starke Gründe, die dagegen sprechen, dass
diese Art von objektiven spirituellen Sachzwängen tatsächlich existiert: 1)
Naturkatastrophen haben natürliche Ursachen, über die wir dank der
Wissenschaft inzwischen ziemlich gut bescheid wissen. Würden Naturkatastrophen
tatsächlich eine Folge der Vernachlässigung der Transzendenz sein, dann müsste
man dagegen erwarten, dass ihr Eintreten eher moralischen als natürlichen
Gesetzmäßigkeiten folgt. 2) Unabhängig davon, welche religiösen Überzeugungen
man für die wahren hält, werden Ungläubige von großen Katastrophen nicht
deutlich häufiger getroffen als Gläubige. Das bedeutet aber, dass man sich
durch die richtige Religionspolitik gar nicht vor derartigen Katastrophen
schützen kann. Objektive spirituelle Sachzwänge der oben angedeuteten Art,
denen die Politik Rechnung tragen könnte oder müsste, scheint es also nicht zu
geben.

2. Von dieser, zugegeben recht abergläubischen Vorstellung, existiert eine
modernere, psychologisierte Variante. Dieser Variante zufolge zieht eine
gestörte Beziehung zum transzendenten Seinsgrund keine Konsequenzen seitens
des transzendenten Seinsgrundes nach sich, sondern der Mensch macht sich
dadurch das Leben selbst unerträglich, weil er einer gesunden Beziehung zur
Transzendenz zutiefst bedarf. Ebenso wie bei der ersten Variante der These
wird vorausgesetzt, dass es eine objektive Wahrheit bezüglich der Transzendenz
gibt. Nur bilden nicht Naturkatastrophen die Folge der Vernachlässigung oder
Fehlinterpretation dieser Wahrheit, sondern sie führt in den seelischen
Wahnsinn. Zur Unterscheidung von der ersten Variante der These der Existenz
spiritueller Sachzwänge könnte man in diesem Fall von {\em subjektiven
  spirituellen Sachzwängen} sprechen. Sehr eindrucksvoll ist die Vorstellung,
dass es subjektive spirituelle Sachzwänge gibt, von Dostojewski literarisch
gestaltet worden, z.B. in dem Roman "`Die Dämonen"', wo sie -- ein
Gegenentwurf zu Turgenjews "`Väter und Söhne"' -- als Abwärtsbewegung von der
Generation der Väter zu der der Söhne in Erscheinung tritt: Stepan
Trofimowitsch Werchowenski, der Repräsentant der Vätergeneration, ist ein im
Grunde genommen noch sehr liebenswerter, aber natürlich naiver und
vertrottelter liberaler Romantiker. Die fatalen Konsequenzen des Verlustes der
religiösen Bindungen treten erst in seinem gefühlskalten und skrupellosen Sohn
Pjotr Stepanowitsch Werchowenski hervor, der sich von dem "`dämonischen"'
Nikolai Stawrogin zum Mord inspiriren lässt.\footnote{Vgl. Fjodor M.
  Dostojewski: Die Dämonen, 20.Aufl., München 1996.}  Neben dem Verlust
religiöser Bindungen hat Dostojewski auch ihrer Fehlbesetzung literarisch
eindrucksvoll in der Figur des Großinquisitors Gestalt verliehen.  Den
absoluten Tiefpunkt religiöser Abirrung verkörperte für Dostojewski nämlich
die katholische Kirche, deren Prinzip der Großinquisitor vertritt, während
Dostojeweski die sozialrevolutionären Bewegungen seiner Zeit im Vergleich dazu
als verzeihliche Kindereien und nachgerade eine Folge der Zerstörung der
Glaubenssubstanz durch die katholische Kirche zu entschuldigen bereit war.
Konsequenterweise ist in der Legende vom Großinquisitor\footnote{Vgl. Fjodor
  Dostojewskij: Die Brüder Karamasow, Frankfurt am Main 2006, S. 397-427.} ein
dritter Weg zwischen dem richtigen Glauben, d.i. der Wahrheit und der Freiheit
Jesu, und dem falschen Glauben der Inquisitionsgerichte nicht vorgesehen. Die
aufklärerische Vorstellung des autonomen, sich und sein Schicksal selbst
bestimmenden Menschen war für Dostojewski ein Unding. Wie wirkungsmächtig
diese von Dostojeweski literarisch gestaltete Vorstellung war, sieht man
daran, dass sie bei Voegelin in sehr ähnlicher Form wieder auftritt.  Nur dass
bei Voegelin die Rollen leicht vertauscht sind, nimmt doch bei Voegelin gerade
das katholische Christentum vor der Reformation (also ausgerechnet jener
Epoche, die das Wüten des Großinquisitors Torquemada, dem Vorbild von
Dostojewskis Großinquisitor, gesehen hatte) den Platz authentischer
Glaubenserfahrung ein, den Dostojewski dem orthodoxen Christentum vorbehalten
hatte.

Gegen diese schwächere, psychologisierte Variante der These von den
"`spirituellen Sachzwängen"' gibt es ebenfalls gravierende Einwände, auch wenn
die Situation hier schon ein wenig komplizierter ist. Ähnlich wie im Fall der
Naturkatastrophen stimmt es einfach nicht, dass seelische Gesundheit ein
Privileg nur von solchen Leuten ist, die über die richtige religiöse
"`Erfahrungsbasis"' verfügen. Seelisches Wohlbefinden kann sich bei den
Anhängern der allerverschiedensten religiösen Überzeugungen mit entsprechend
unterschiedlichen "`Bewusstseinserfahrungen"' einstellen. Es kommt bei der
Religion nur darauf an, was für wen passt. Und das kann für jeden etwas
anderes sein. Etwas komplizierter liegt der Fall bei der psychologisierten
Variante aber insofern, als die Existenz irgendwelcher Kausalzusammenhänge
zwischen der Religiosität und dem seelischen Wohlbefinden sowie zwischen der
Religiosität und dem moralischen und unter Umständen auch dem politischen
Wohlverhalten sehr wohl anzunehmen ist. Nur, dass das persönliche Wohlbefinden
und das sittliche Wohlverhalten von dem religiösen Wahrheitsgrad der
weltanschaulichen Einstellung abhängen sollen, ist eine These, die die
Wissenschaft unmöglich bestätigen kann, da sich wissenschaftlich nicht über
religiöse Wahrheiten urteilen lässt.
 
3. Die dritte Variante der These trägt diesem Einwand Rechnung. Sie setzt
daher auch nicht mehr irgendwelche religiösen Überzeugungen als wahr voraus
oder spricht -- im Sinne Voegelins -- ganz spezifischen religiösen Erfahrungen
eine höhere Adäquatheit zu als anderen. Behauptet wird lediglich, dass es
kausale Beziehungen zwischen der religiösen Einstellung und dem politischen
Verhalten von Menschen geben kann, denen die Politik Rechnung tragen sollte.
Genaugenommen handelt es sich dann aber nicht mehr um "`spirituelle
Sachzwänge"', sondern lediglich um {\em anthropologische Sachzwänge}. Die
Politik muss die Tatsache der Religiosität als einer wesentlichen Eigenschaft
des Menschen berücksichtigen, nicht zuletzt deshalb, weil religiöse
Institutionen einen erheblichen politischen Einfluss ausüben können. Das ist
aber etwas ganz anderes, als wenn die Politik unmittelbar religiöse
Glaubenssätze zu berücksichtigen hätte.

Fasst man das Verhältnis von Religion und Politik ausschließlich im Sinne
dieser dritten Variante auf, so hat das bedeutende Konsequenzen für den Umgang
mit potentiell gefährlichen religiösen oder weltanschaulichen Gruppierungen.
Die Auseinandersetzung mit solchen Gruppierungen braucht nun nicht mehr um
religiöse Wahrheitsansprüche geführt zu werden. Sie kann wesentlich
pragmatischer als eine Auseinandersetzung um deren äußeres Verhalten und
säkulare Einstellung (d.h. ihre Einstellung gegenüber dem Rest der
Gesellschaft und ihre Haltung zum gesellschaftlichen Zusammenleben, nicht aber
ihre Einstellung zu religiösen und metaphysischen Fragen im engeren Sinne)
ausgetragen werden. Dementsprechend braucht, wenn überhaupt, auch erst dann
"`sehr energisch mit Parteiverboten zugegriffen werden"',\footnote{Voegelin,
  Die geistige und politische Zukunft der westlichen Welt, a.a.O., S. 33.}
wenn derartige Gruppierungen sich verfassungsfeindlich betätigen und nicht
bereits dann, wenn sie weltanschaulich nicht konform gehen. Dieser subtile
Unterschied markiert, beiläufig bemerkt, die Grenze zwischen dem, was man die
"`wehrhafte Demokratie"' nennt, und einem autoritären Staat.

Welcher Art die kausalen Beziehungen zwischen religiöser Einstellung und
politischem Verhalten sind, darüber kann man im Einzelnen sehr
unterschiedliche Theorien aufstellen. Einer besonders unter Anhängern
Voegelins populären Ansicht zufolge führt der Verlust religiöser Bindungen
dazu, dass eine Art religiöses Vakuum entsteht, das dann von politischen
Ideologien aufgefüllt werden kann. In diesem Sinne ist z.B. Voegelins Rede von
der "`positivistischen Destruktivität"'\footnote{Eric Voegelin: The New
  Science of Politics. An Introduction, Chicago \& London 1987 (zuerst: 1952),
  S. 4.} zu verstehen, denn der Positivismus ist als solcher zwar keineswegs
totalitär, bereitet aber nach Voegelins Verständnis den totalitären Ideologien
geistig den Boden, indem er ein geistig-religiöses Vakuum hinterlässt, das
derartige Ideologien dann widerstandslos besetzen können. Allerdings gibt es
gute Gründe diese "`Vakuumtheorie"' anzuzweifeln. Wäre sie wahr, dann hätten
sich ja z.B. die religiös stark gebunden Bevölkerungskreise im 3. Reich in
Deutschland durch eine auffällig große Resistenz gegenüber dem
Nationalsozialismus auszeichnen müssen. Von Ausnahmen besonders in den
katholischen Bevölkerungsteilen abgesehen, war das aber nicht unbedingt der
Fall. Und was die "`positivistische Destruktivität"' betrifft, so hätte man
erwarten müssen, dass gerade die positivistischen Philosophenschulen wegen des
von ihnen erzeugten geistig-religiösen Vakuums in besonderem Maße anfällig für
totalitäre Ideologien gewesen sind. Ein flüchtiger Blick in die
historisch-biographischen Materialien etwa zum Wiener Kreis\footnote{Vgl.
  Friedrich Stadler: Studien zum Wiener Kreis. Ursprung, Entwicklung und
  Wirkung des Logischen Empirismus im Kontext, Frankfurt am Main 1997.} legt
aber viel eher die Vermutung nahe, dass die philosophische Strömung des
Neupositivismus -- von den kommunistischen Sympathien einzelner
positivistischer Philosophen wie z.B.  Otto Neurath abgesehen -- ganz im
Gegenteil außergewöhnlich resistent gegen die totalitäre Versuchung geblieben
ist. Im Ganzen dürften die kausalen Beziehungen zwischen dem religiösen
Hintergrund und der politischen Haltung also sehr viel komplizierter sein als
dies die "`Vakuumtheorie"' nahelegt.

Zusammenfassend lässt sich festhalten: Spirituelle Sachzwänge im Sinne
religiöser Wahrheiten oder Tatsachen, seien dies nun göttliche Strafen oder
eine vermeintlich unleugbare "`Spannung zum Grund"', die das politische
Handeln berücksichtigen müsste, gibt es nicht. Was die Politik berücksichtigen
muss ist allein das anthropologische Faktum, dass die
meisten Menschen religiös sind. Je nachdem welche Vorstellung man von der
Religiosität und insbesondere der Dominanz religiöser Motive für das
menschliche Handeln hat, könnte damit aber immer noch die These vereinbar
sein, dass die Politik es sich nicht erlauben kann, den religiösen Bereich
unbesetzt zu lassen. Dass dem nicht so ist, dafür soll in den folgenden
Abschnitten argumentiert werden.

\section{Bedarf die Legitimation der politischen Ordnung einer religiösen Komponente?} 

Ein entscheidendes Problem einer jeden politischen Ordnung, bei welchem der
Rückgriff auf religiöse Wahrheiten naheliegend erscheinen könnte, ist das
Problem der Legitimation der politischen Ordnung. Die Legitimation erfüllt
eine zweifache Funktion. Zum einen soll sie die grundsätzliche Zustimmung der
Herrschaftsunterworfenen zur Herrschaftsordnung sicherstellen. Zum anderen
dient sie der Motivation von Einsatzbereitschaft für den eigenen
Herrschaftsverband, was besonders im Kriegsfall von großer Bedeutung ist. Es
stellt sich nun die Frage, ob eine Legitimation politischer Ordnung ohne
Inanspruchnahme der menschlichen Religiosität möglich ist, und ob sie genügend
Intensität erreicht, um die Stabilität des politischen Systems auch in
Krisenzeiten zu gewährleisten.

Die heutzutage in der westlichen Welt übliche Form der Legitimation ist die
einer Gesellschaftsvertragstheorie. Die Gesellschaftsvertragstheorie
legitimiert dabei sowohl die Existenz eines Staates überhaupt als auch im
besonderen die demokratische Herrschaftsform. Die Existenz des Staates wird
dadurch legitimiert, dass ohne Staat der Einzelne vor Übergriffen von
seinesgleichen auf sein Leben und Vermögen keinen Augenblick sicher ist, so
dass die Menschen ohne Staat ohnehin nichts Besseres tun könnten, als durch
Vertrag einen Staat zu gründen, der sie voreinander beschützt. Die
demokratische Herrschaftsform wird dadurch legitimiert, dass sie diejenige
Herrschaftsform ist, zu der die Menschen in einem auf Basis freier Zustimmung
geschlossenen Vertrag am ehesten ihre Zustimmung geben könnten, da sie ihnen
nicht nur vor den Übergriffen der Mitbürger sondern auch vor dem
Machtmissbrauch des Herrschers die größte Sicherheit bietet.\footnote{Ich
  beziehe mich hier in erster Linie auf die Hobbessche
  Gesellschaftsvertragstheorie unter Berücksichtigung der Lockeschen Kritik
  dieses Modells.}

Die Gesellschaftsvertragstheorien rechtfertigen die Existenz des Staates und
die demokratische Herrschaftsform, indem sie sich rational einleuchtender
Argumente bedienen. Der Sinn des Staates wird dabei hinreichend durch den
Zweck der Schaffung innerer Sicherheit erklärt, ein Zweck der, so sollte man
meinen, im Eigeninteresse eines jeden Menschen liegt. Eine zusätzliche
Legitimation, etwa durch göttliche Autorität, könnte innerhalb dieses
Gedankenganges sogar problematisch erscheinen, denn, wenn es nicht schon
genügend rationale Gründe gäbe, um die Existenz des Staates zu legitimieren,
dann hätte der Staat ohnehin kein Existenzrecht, und alle weiteren
Rechtfertigungen seiner Existenz müssten als Ideologie verworfen werden.

Eine Legitimation politischer Ordnung ohne religiösen Bezug scheint also
grundsätzlich möglich zu sein, wenn man voraussetzt, dass die Menschen
vernünftig genug sind, um ihre eigenen Interessen zu erkennen.  Erweist sich
diese Art der Legitimation aber auch als krisenfest, wenn die politische
Ordnung vor besonderen Herausforderungen steht? Sind die liberalen Demokratien
im Falle eines Krieges in der Lage, ohne die Mobilisierung religiöser Energien
in genügendem Maße Opferbereitschaft für sich zu motivieren? Und handeln sie
sich bei der Auseinandersetzung mit ideologischen Bewegungen im Inneren nicht
einen entscheidenden Nachteil dadurch ein, dass sie die religiösen Gefühle der
Bürger unangetastet lassen müssen (und wollen)?\footnote{Vgl. Joachim Fest:
  Die schwierige Freiheit. Über die offene Flanke der offenen Gesellschaft,
  Berlin 1993, S. 38ff.}

Gegen die erste dieser Befürchtungen kann eingewandt werden, dass auch in den
liberalen Demokratien angesichts äußerer Herausforderungen gesellschaftliche
Mechanismen wirksam werden, die die Abwehrbereitschaft der demokratischen
Gesellschaft erheblich stärken. So macht sich im Falle eines Krieges oft eine
Art von innerem Zusammenrücken der Gesellschaft bemerkbar, das sich
beispielsweise in einer schlagartigen Zunahme der Beliebtheitswerte der
jeweiligen Regierung äußern kann. Auch haben sich beispielsweise im Zweiten
Weltkrieg die Soldaten, die auf Seiten der liberalen Demokratien kämpften,
nicht weniger tapfer geschlagen als die Armeen der totalitären Regime, was
beweist, dass im Ernstfall durch quasi-religiöse Sinnversprechungen keine
wesentlichen Vorteile zu erzielen sind. Weit entfernt davon, eine
Schwachstelle der liberalen Ordnung zu offenbaren, können äußere
Herausforderungen diese Ordnung sogar erheblich stärken.

Ebensowenig zwingend ist das Argument, dass ein rein rational legitimiertes
System keine ausreichende Immunität gegen die verführerische Kraft
chiliastischer politischer Bewegungen im Inneren entwickeln könnte.  Zumindest
ist nicht unmittelbar ersichtlich, wie eine spirituelle oder religiöse
Legitimationskomponente hier Abhilfe schaffen könnte. Jede Form der
Legitimation kann zusammenbrechen, wenn das politische System, das durch sie
legitimiert wird, sich als erfolglos erweist oder wenn sie durch eine vom
Geist der Zeit als überzeugender empfundene Legitimation herausgefordert wird.
Dies würde auch für eine Legitimation auf Basis der existentiellen Spannung
zum transzendenten Seinsgrund gelten, ganz gleich, welches Maß philosophischer
Wahrheit diese Legitimation für sich beanspruchen dürfte. Zudem ließe sich die
Überlegung anstellen, das gerade spirituelle Legitimationskomponenten ein
Einfallstor für Ideologien darstellen könnten, da durch sie dem
Irrationalismus bereits öffentlicher Glaubwürdigkeitskredit eingeräumt wird.

% Andererseits gibt es durchaus gravierende Einwände, die gegen religiöse oder
% spirituelle Legitimationskomponenten sprechen. So stellt sich in einer
% pluralistischen Gesellschaft die nicht unerhebliche Frage, woher die
% spirituelle Wahrheit zur Legitimation der politischen Ordnung bezogen werden
% soll. Und unabhängig davon, kann eine spirituelle Legitimation
% ernsthaft nur dann gefordert werden, wenn auch irgendeine spirituelle
% Wahrheit vorweisbar ist, in deren Namen die Legitimation vorgenommen wird.
% Ohne spirituelle Wahrheit kann es keine spirituelle Legitimation geben, auch
% wenn sie noch so nützlich wäre.

Auch wenn das Problem der hinreichenden Legitimation politischer Ordnung mit
großen Unsicherheiten behaftet ist (da sich nicht bloß die Frage stellt,
wodurch eine politische Ordnung philosophisch gerechtfertigt ist, sondern vor
allem, wann eine politische Ordnung als gerechtfertigt empfunden wird) scheint
der Rückgriff auf religiöse Überzeugungen oder spirituelle Erfahrungen für die
Legitimation politischer Ordnung nicht unbedingt erforderlich zu sein.

\section{Wertbegründung und -konsens in der pluralistischen
  Gesellschaft} 
\label{Wertbegruendung}

Ein wichtiges Argument, welches für die Religion und besonders für eine
stärkere Geltung der Religion im gesellschaftlichen Leben angeführt
werden könnte, beruht auf dem philosophischen Problem der
Letztbegründung ethischer Werte. Dieses Argument lautet in etwa wie
folgt: Keine Gesellschaft, so könnte argumentiert werden, kann ohne
einen Satz verbindlicher ethischer Grundwerte existieren. Es wäre aber
absurd, diese Grundwerte, die absolut gelten müssen, zur Disposition
eines Konsensfindungsverfahrens zu stellen, sei dies nun eine
verfassungsgebende Versammlung oder auch nur ein gedachter
Gesellschaftsvertrag, zumal dann immer noch die Gültigkeit des
Verfahrens als Wertvoraussetzung übrig bliebe. Zugleich zeigt die
Philosophiegeschichte, dass alle Versuche einer rein säkularen
Letztbegründung ethischer Werte zum Scheitern verurteilt sind. Mit
anderen Worten: Wenn es Gott nicht gäbe, dann wäre alles erlaubt. Also
muss das religiöse Bewusstsein in der Gesellschaft mindestens noch so wach
sein, dass die Verbindlichkeit der Grundwerte anerkannt wird.

Stimmt dieses Argument, und ist die Religiosität damit tatsächlich
unverzichtbar?  An diesem Ergebnis scheint kein Weg vorbeizuführen, denn wenn
eine philosophische Letztbegründung der Ethik nicht möglich ist, dann bleibt
als einzige Form der Wertbegründung ein ethischer Dezisionismus übrig, d.h.
jeder wählt sich seine Werte selbst aus, und wenn jemand die Wahl trifft,
überhaupt keine Werte zu beachten, dann ist dies genauso möglich. Diese
theoretische Konsequenz ist gewiss sehr ernüchternd. Aber kann die Religion
überhaupt Abhilfe schaffen? Das ist wiederum mehr als zweifelhaft, denn durch
eine religiöse Wertbegründung würde das Begründungsproblem nicht gelöst,
sondern nur auf die Religion verschoben werden. Dadurch dürfte das
Begründungsproblem aber eher noch komplizierter werden, da außer den Werten
nun auch die Wahrheit des religiösen Glaubens, der die Werte begründet, auf
dem Prüfstand steht.  Zwischen konkurrierenden religiösen Glaubensüberzeugungen
objektiv zu entscheiden ist aber unmöglich. Die Anerkennung einer Religion
beruht letzten Endes auf einem Glaubensakt und damit nicht weniger auf einer
persönlichen Entscheidung als die sittlichen Werte nach der Theorie des
ethischen Dezisionismus.

Eine Letztbegründung oder gar ein regelrechter Beweis ethischer Werte scheint
also unmöglich zu sein. Die universelle Verbindlichkeit bestimmter Werte lässt
sich daher bestenfalls auf Basis eines Konsenses erreichen, auch wenn dies dem
Charakter ethischer Werte als unverfügbare Werte zu widersprechen scheint.
Dabei dürfte es höchstwahrscheinlich sogar aussichtsreicher sein, den Konsens
auf der Ebene der Werte als auf der Ebene der philosophischen oder religiösen
Begründung der Werte zu suchen. Denn darüber, dass töten oder stehlen
verwerflich ist, lässt sich gewiss leichter eine Einigung erzielen als über
die Frage, ob Allah oder der liebe Gott oder die philosophische Vernunft der
legitime moralische Gesetzgeber ist. Und dort, wo unversöhnliche
Wertauffassungen aufeinanderprallen, würde es eine Einigung erst recht
erschweren, wenn der Streit zuerst auf der metaphysischen bzw. existentiellen
Ebene entschieden werden soll. Von großer Bedeutung ist dabei, dass die
Akzeptanz von Werten nicht zwingend durch die existentielle Haltung eines
Menschen bedingt ist, sondern dass sie auch auf der bloßen Einsicht in die
Nützlichkeit eines Wertes für das gesellschaftliche Zusammenleben beruhen
kann. Weiterhin können Werte auch deshalb akzeptiert werden, weil sie im
Dialog mit anderen vereinbart worden sind. Ein Wertkonsens ist daher
grundsätzlich auch ohne einen einheitlichen spirituellen Erfahrungshintergrund
der Beteiligten denkbar.

Auch in der Frage der Wertbegründung und des gesellschaftlichen Konsenses über
bestimmte Grundwerte lautet daher das Ergebnis, dass der Rückgriff auf die
Spiritualität keinesfalls notwendig und in der Regel eher hinderlich als
förderlich ist.

\section{Sinngebung durch die politische Ordnung?}

Aus Voegelins Sicht müsste ein ethischer Wertkonsens jedoch als ein höchst
brüchiges Fundament der gesellschaftlichen Ordnung beurteilt werden, sofern er
sich nicht auf einen einheitlichen spirituellen Erfahrungshintergrund stützen
kann. Dies hängt unter anderem damit zusammen, dass Voegelin die
Dialogmöglichkeiten zwischen Menschen mit unterschiedlichem spirituellem
Erfahrungshintergrund überaus skeptisch beurteilt, was sogar soweit führt,
dass Menschen ohne spirituelle Erfahrungen von Voegelin als potentielle
Ordnungsstörer eingestuft werden. Ähnliche Auffassungen kehren auch bei
manchen Anhängern Voegelins wieder. So wurde die Ansicht, dass die Gegensätze
zwischen Menschen, die an die Existenz eines transzendenten Seins glauben, und
Menschen, die sie bestreiten, weitgehend unversöhnlich bleiben müssen, solange
über diese "`Schlüsselfrage"' nicht Einigkeit erzielt worden ist, unlängst von
Thomas J. Farrell bekräftigt, der in diesem Zusammenhang die Leugnung der
Existenz eines transzendenten Seins unter Berufung auf prominente Psychologen
wie C. G. Jung und ganz auf der Linie Voegelins als eine Art Geisteskrankheit
deutet. Allerdings räumt auch Farrell ein, dass es Profanbereiche gibt,
innerhalb derer ein fruchtbarer Dialog zwischen Menschen, die jene
"`Schlüsselfrage"' unterschiedlich beantworten, möglich ist.\footnote{Vgl.
  Thomas J.  Farrell: The Key Question. A critique of professor Eugene Webbs
  recently published review essay on Michael Franz's work entitled "'Eric
  Voegelin and the Politics of Spiritual Revolt: The Roots of Modern
  Ideology"', in: Voegelin Research News, Volume III, No.2, April 1997, auf:
  http://alcor.concordia.ca/\~{ }vorenews/v-rnIII2.html (Host: Eric Voegelin
  Institute, Lousiana State University. Zugriff am: 1.8.2007).} Die Frage
stellt sich nun, ob die politische Ordnung zu diesen Profanbereichen des
menschlichen Lebens gehört.

Damit ist zugleich eine Grundfrage des Wesens politischer Ordnung
angeschnitten: Ist die (in der Neuzeit stets durch den Staat
repräsentierte) politische Ordnung nur ein Mittel zu bestimmten Zwecken
wie etwa der Schaffung innerer und äußerer Sicherheit, oder ist sie
darüber hinaus Ausdruck einer historischen Suche nach Ordnung, die mit
dem Sinn der Welt und dem Sinn des Lebens in Zusammenhang steht? Im
ersteren Fall kann die politische Ordnung voll und ganz dem
Profanbereich zugeordnet werden, so dass eine Einigung über alle
wesentlichen Prinzipien der politischen Ordnung auch zwischen Menschen
mit unterschiedlicher Offenheit der Seele im Bereich des Möglichen
liegt. Nur im letzteren Fall müsste zunächst eine gesellschaftlich
verbindliche Entscheidung über die metaphysische Schlüsselfrage der
Existenz transzendenten Seins getroffen werden.

Welche dieser beiden grundverschiedenen Wesensauffassungen politischer Ordnung
ist nun aber die richtigere? Um diese Frage zu beantworten, empfiehlt es sich,
von unterschiedlichen Funktionen des Politischen auszugehen, einer
Friedenssicherungsfunktion und einer spirituellen Funktion, und dann zu
klären, in welcher Beziehung diese Funktionen zueinander stehen, d.h. 
insbesondere, ob die politische Ordnung die Friedenssicherungsfunktion nur
erfüllen kann, wenn sie auch spirituelle Funktionen erfüllt. Sollte sich
herausstellen, dass sich beide Funktionen trennen lassen, dann kann als
Nächstes die Frage gestellt werden, welche der beiden Funktionen die für
die politische Ordnung wesentlichere ist, und ob es nicht günstiger
wäre, die andere Funktion innerhalb eines anderen Rahmens zu erfüllen,
also etwa die spirituellen Ziele nicht auf der Ebene der politischen Ordnung
sondern im Rahmen privater religiöser Vereinigungen zu verfolgen.  

Geht man zunächst einmal davon aus, dass die Stiftung inneren Friedens die
Kernfunktion politischer Ordnung ist, so kann man überlegen, was mindestens zu
einer politischen Ordnung gehören muss, damit sie diese Kernfunktion erfüllen
kann. Sicherlich sind für die Erfüllung der Kernfunktion der Friedenssicherung
Herrschaftsinstitutionen notwendig, die die Einhaltung des Friedens
garantieren. Weiterhin müssen sich die Herrschaftsinstitutionen auf die
Loyalität oder wenigstens den regelmäßigen Gehorsam der Bürger stützen können.
Eine politische Ordnung, die die Kernfunktion der Friedenssicherung erfüllen
soll, bedarf daher auch einer Legitimation, wozu mindestens eine politische
Philosophie oder Herrschaftsideologie vorhanden sein muss, die den Bürgern den
Zweck der politischen Ordnung erklärt. Dann könnte eingewandt werden, dass ein
echter Frieden noch gar nicht vorhanden ist, solange nicht auch Gerechtigkeit
herrscht. Es wären also auch noch Vorkehrungen für die Gerechtigkeit zu
treffen usw. . Führt man diese Überlegungen weiter fort, so gelangt man
irgendwann einmal zu einer politischen Mindestordnung, die alles umfasst, was
notwendig ist, um die Kernfunktion der Friedenssicherung zu erfüllen. Gehört
zu dieser Mindestordnung bereits die Funktion der
Sinnvermittlung?\footnote{Unter Sinnvermittlung ist zu verstehen, dass die
  politische Ordnung in ihrer Gestalt Ausdruck der in spiritueller Erfahrung
  erlebten sinnhaften Seinsordnung ist, die sie zugleich ihren Mitgliedern
  weitervermittelt. Dies trifft die Intention Voegelins besser als der (an
  sich verständlichere) Ausdruck Sinngebung, da nach Voegelins Verständnis die
  politische Ordnung keinesfalls die Quelle des Sinns ist, sondern idealiter
  in die sinnhafte Gesamtordnung der Welt eingebettet ist.} Nach den
Überlegungen der vorhergehenden Abschnitte ist dies wahrscheinlich nicht der
Fall, denn die politische Ordnung bedarf des Rückgriffs auf die spirituelle
Erfahrung weder zur Legitimation noch um der Begründung verbindlicher Werte
willen, noch ist die spirituelle Erfahrung bei der Bewältigung politischer
Probleme von Vorteil. Also ist die Sinnvermittlungsfunktion, wenn überhaupt,
eine rein optionale Funktion politischer Ordnung, soweit unter politischer
Ordnung die eben angedeutete Mindestordnung zu verstehen ist. Neben der
Sinnvermittlungsfunktion sind noch weitere solcher optionaler Funktionen
politischer Ordnung denkbar (z.B. Sozialstaatlichkeit\footnote{Historisch
  hatte die Entwicklung des Sozialstaates natürlich durchaus einiges mit der
  Sicherung des inneren Friedens zu tun, aber für die theoretische Frage, ob
  und warum der Staat sozialstaatliche Aufgaben übernehmen soll, spielen
  historisch-kontingente Tatsachen nur bedingt eine Rolle.}). Solche
Funktionen der politischen Ordnung zuzurechnen ist dann empfehlenswert, wenn
ihre Erfüllung am ehesten oder sogar einzig und allein auf der Ebene der
politischen Ordnung möglich ist und wenn dabei keine gravierenden Nachteile
entstehen. Nun kann die Sinnvermittlungsfunktion aber zweifellos auch anders
als im Rahmen der politischen Ordnung erfüllt werden. Die Vermittlung von
Lebensinn, die sinnhafte Deutung der Welt und die Erfüllung menschlicher
Transzendenzbedürfnisse kann, wenn schon nicht individuell, so doch auf jeden
Fall im Rahmen von Kirchen und Religionsgemeinschaften geleistet werden. Es
tut der Spiritualität also keinerlei Abbruch, wenn ihr nur ein Platz außerhalb
der politischen Ordnung angewiesen wird, während andererseits nicht einzusehen
ist, welche Vorteile es haben soll, wenn sie der politischen Ordnung
aufgebürdet wird.

Die Tatsache, dass die Spiritualität keineswegs darunter leiden muss, wenn
sie nicht als Bestandteil der politischen Ordnung betrachtet wird,
scheint Voegelin zu übersehen, wenn er es den
Gesellschaftsvertragstheorien zum Vorwurf macht, dass sie sich nur auf
die leibliche Seite des Menschen konzentrieren und die geistige Seite
des Menschen vernachlässigen.\footnote{Vgl. Voegelin, Anamnesis,
  S. 341/342.} Seinem Vorwurf liegt ein fundamentales Missverständnis des
Zweckes politischer Ordnung zu Grunde. Die Notwendigkeit politischer
Ordnung entsteht letztlich aus dem Umstand, dass Menschen einander in die
Quere kommen können und deshalb Abmachungen treffen müssen, damit dies
nicht geschieht. Politik hat daher ihrem Wesen nach mehr mit der
niederen, materiellen Sphäre der unumgehbaren Notwendigkeiten zu tun als
mit der geistigen Sphäre. Es ist deshalb ein Irrtum, von der politischen
Ordnung den Ausdruck spiritueller Wahrheit zu verlangen.  Und der
Verzicht darauf bedeutet keinesfalls eine Leugnung des Geistes, da
gerade nach den liberalen Gesellschaftvertragstheorien die politische
Ordnung gar nicht beansprucht, das ganze Wesen des Menschen zu erfassen.

Umgekehrt wäre es höchst prekär, religiöse Erfahrungen zu einer Angelegenheit
von politischer Bedeutung zu erklären. Denn wenn die politische Ordnung auf
eine Erfahrung der Transzendenz gegründet wird, dann wird die Religiosität zu
einer Frage der politischen Ordnung. Sie dürfte dann nicht mehr im Belieben
des Einzelnen stehen, was erhebliche Probleme für die Religionsfreiheit und
Toleranz aufwirft. Dann wäre es in der Tat nur konsequent, nach dem Irrenarzt
zu rufen, wenn es Menschen geben sollte, die es wagen, die Transzendenz zu
leugnen.  Ansonsten ist die Leugnung der Transzendenz eine sehr harmlose
"`Krankheit"', denn sie beeinträchtigt weder das Lebensglück der Befallenen,
noch hindert sie sie daran, die Rechte ihrer Mitbürger zu
respektieren.\footnote{Dasselbe Argument gilt umgekehrt auch für analoge
  Versuche, die Religion oder die Religiosität als eine psychopathologische
  Erscheinung zu verstehen. Vgl. dazu die sehr vernünftigen Ausführungen von
  Eugene Webb, in: Webb, Review, a.a.O.}

Als Gesamtergebnis lässt sich festhalten, dass weder die politische Ordnung
der Transzendenzerfahrungen bedarf, noch die Realisierung bzw.  der Ausdruck
der Transzendenzerfahrungen durch die politische Ordnung geschehen muss. Da
andererseits die Forderung der Berücksichtigung spiritueller Erfahrungen bei
der Gestaltung politischer Ordnung erhebliche ethische Bedenken hinsichtlich
der Toleranz aufwirft, so ergibt sich, dass Transzendenzerfahrungen bei der
Gestaltung der politischen Ordnung besser keine Rolle spielen sollten. Kurzum:
Wenn es Gott gäbe, müsste man ihn ignorieren -- wenigstens in der Politik.

\chapter{Was bleibt von Eric Voegelin?}
\label{WasBleibt}

Nachdem sich Eric Voegelins Bewusstseinsphilosophie für die Erforschung der
geistigen Grundlagen guter politischer Ordnung als so wenig haltbar erwiesen
hat, erscheint es mir angebracht, einige Überlegungen dazu anzustellen, welche
Rolle Eric Voegelin in der heutigen wissenschaftlichen und politischen
Diskussion noch spielen kann, und in welcher Richtung die Auseinandersetzung
über sein Werk fortzuführen wäre. Dazu werde ich im folgenden drei Aspekte der
Frage der Aktualität von Voegelins Werk ansprechen: 1. Welche Bedeutung kommt
Voegelins Philosophie zu? 2. Wie aktuell sind seine Vorstellungen politischer
Ordnung und politischer Unordnung? 3. Gibt es dennoch ein
politikwissenschaftliches Vermächtnis Eric Voegelins, das fortzuführen sich
lohnt.

\section{Zum Charakter von Voegelins Philosophie}

Die Philosophie Eric Voegelins halte ich, wie aus den bisherigen Ausführungen
sicherlich hervorgegangen ist, nicht für sonderlich geglückt. Dabei halten
nicht nur die Ergebnisse seiner Philosophie einer kritischen Prüfung nicht
stand, auch die Art seines Philosophierens ist keinesfalls nachahmenswert.
Voegelins Philosophie, und dies gilt sowohl für seine Geschichtsphilosophie
als auch für seine Bewusstseinsphilosophie, ist eine überaus {\em dogmatische
  Philosophie}, sie stellt außerdem eine hochgradig {\em monologische
  Philosophie} dar, und darüber hinaus erscheint sie über weite Strecken als
das, was Karl Popper sehr treffend "`{\em orakelnde Philosophien}"' genannt
hat.\footnote{Vgl. Karl Popper: Die offene Gesellschaft und ihre Feine.  Band
  II. Falsche Propheten: Hegel, Marx und die Folgen, 7. Aufl., Tübingen 1992,
  S. 262ff.}

Eine dogmatische Philosophie ist eine Philosophie, die ein Weltbild
artikuliert, ohne es zu begründen. Während eine kritische Philosophie
versucht, ihre Thesen durch Argumente zu begründen, findet bei einer
dogmatischen Philosophie gar keine oder nur eine tautologische Begründung
statt oder eine Begründung durch Voraussetzungen, die ihrerseits nicht weniger
begründungsbedürftig sind als die begründeten Thesen. Damit ist nicht gesagt,
dass dogmatische Philosophien notwendigerweise schlechte Philosophien sind,
denn die Bildung und Ausgestaltung eines Weltbildes (oder auch nur einer
Unternehmensphilosophie) ist weder eine triviale noch eine unbedeutende
Aufgabe, aber dogmatische Philosophien können nur in begrenztem Maße
Objektivität für sich in Anspruch nehmen. Und genau in diesem Sinne ist
Voegelins Philosophie eine hochdogmatische Philosophie. Deutlich wird dies
immer wieder an metaphysischen Voraussetzungen wie Annahme der Existenz eines
transzendenten Seinsgrundes, der Ontologie der Seinsstufen, der Auffassung der
Geschichte als eines theogonischen Prozesses usw. . Voegelins Philosophie wird
dadurch nicht uninteressanter und die Darstellung seines Weltbildes ist ihm
einige Male auch in einer ästhetisch und rhetorisch ansprechenden Weise
gelungen.\footnote{Dies gilt besonders für die Einleitungen von Order and
  History I und II. (Vgl. Voegelin, Order and History I, S. 1ff. -- Vgl.
  Voegelin, Order and History II, S. 1-20.) -- Fast noch schöner hat es aber
  Thomas Hollweck gesagt: Vgl. Thomas Hollweck: Truth and Relativity: On the
  Historical Emergence of Truth, in: Opitz, Peter J. / Sebba, Gregor (Hrsg.):
  The Philosophy of Order. Essays on History, Consciousness and Politics,
  Stuttgart 1981, S. 125-136. -- Für meinen Geschmack jedoch eher misslungen
  und an eine schlechte Predigt erinnernd: Eric Voegelin: Ewiges Sein in der
  Zeit, in: Voegelin, Anamnesis, S. 254-280.} Aber Voegelins Philosophie ist
eben auch nicht mehr als eine Philosophie. Sie ist nicht {\it die}
Philosophie, und es steht jedem Menschen frei, sich zu ihr zu bekennen oder
sie abzulehnen.

Als problematischer stellt sich der monologische Charakter von Voegelins
Philosophie dar. Auch diese Eigenschaft hat Voegelins Philosophie mit der
Philosophie anderer Denker gemeinsam. Der monologische Charakter findet sich
bei Voegelin sowohl auf der Ebene des Philosophierens als auch auf der Ebene
seiner philosophischen Doktrin. Auf der Ebene des Philosophierens äußert sich
der monologische Charakter in Voegelins heftigen polemischen Ausfällen, in
seiner Weigerung, mit jedem, der seine Grundüberzeugungen nicht teilt, auch
nur ein Wort zu reden,\footnote{Vgl. Conversations with Eric Voegelin. (ed. R.
  Eric O'Connor), Montreal 1980, S. 58ff.} und in der fast paranoiden
Vorstellung einer Ansteckungsgefahr, die von der vermeintlichen Krankheit
deformierter Existenz ausgeht, welche er hinter den von ihm unerwünschten
Philosophien allzeit vermutete. Wichtiger noch als diese etwas schrulligen
Äußerungen eines leidenschaftlichen intellektuellen Temperamentes ist die
Rolle des monologischen Prinzips innerhalb von Voegelins Doktrin.
Philosophische Wahrheit wird für Voegelin immer von Einzelnen erfahren und
dann sprachlich an Andere weitervermittelt, was ein überaus schwieriger Prozess
ist, da die Erfahrung, die in gewisser Weise auch eine
Verständnisvoraussetzung bildet, im Anderen durch die sprachliche Vermittlung
erst angeregt werden muss.\footnote{Vgl. auch William C.  Harvard, Jr.: Notes
  on Voegelin's contributions to political theory, in: in: Ellis Sandoz
  (Hrsg.): Eric Voegelins Thought. A critical appraisal, Durham N.C. 1982,
  S. 87-124 (S. 112-113).} Nach dieser Vorstellung von philosophischer Wahrheit
ist es unmöglich, dass Wahrheit im philosophischen Dialog gefunden wird, denn
die Erfahrung eines Menschen kann logischerweise nicht durch Argumente eines
anderen Menschen korrigiert werden. Die typische Gesprächssituation, die
Voegelins Philosophie zu Grunde liegt, ist daher nicht der Dialog unter
Gleichgestellten, sondern stets das belehrende Gespräch, in welchem die Rollen
von Lehrer und Schülern, von Führer und Gefolgsleuten, von Prophet und Jüngern
klar verteilt sind. Problematisch erscheint am monologischen Charakter von
Voegelins Philosophie, dass eine legitime Pluralität von Weltanschauungen
dadurch theoretisch ebenso ausgeschlossen ist, wie die gegenseitige
Befruchtung gegensätzlicher Standpunkte. Pluralismus war für Voegelin beinahe
gleichbedeutend mit Verwirrung, und ein Philosoph, der die Wahrheit
existentiell erfahren hat, kann sich durch andere Standpunkte höchstens noch
beirren lassen.

Für den heikelsten Punkt halte ich allerdings die philosophische
Geheimniskrämerei, zu der Voegelin nicht immer aber in seinen späteren
Schriften immer häufiger neigt. Ein philosophischer Geheimniskrämer ist
jemand, der das Rätsel und das Gefühl des Geheimnisvollen mehr liebt als die
Lösung der Rätsel. Voegelin hat sich in mehrfacher Weise der philosophischen
Geheimniskrämerei befleißigt. Dies beginnt mit Voegelins oft unklarer und
vieldeutiger Ausdrucksweise, es geht fort über die nicht wenigen technischen
Mängel seiner Philosophie, unter denen insbesondere die Schlussfehler der {\it
  petitio principii}, der {\it aquivocatio} und des {\it non sequitur} einen
prominenten Platz einnehmen, und der Höhepunkt ist erreicht, wenn Voegelin
sich auf Paradoxien und Mysterien beruft. Ich gebe zu, dass dies eine höchst
subjektive Kritik ist, und wer in Hegel einen großen Philosophen sieht, der
wird Voegelin wegen seiner Denkfehler gewiss nicht tadeln wollen. Aber mir
scheint, dass ein Philosoph, der sich auf ein Mysterium beruft, mit demselben
Misstrauen betrachtet werden sollte, wie ein Politiker, der sich auf sein
Ehrenwort beruft. Nicht dass von vornherein ausgeschlossen werden kann, dass
es in der Welt Mysterien gibt. Aber bei einem Mysterium hat alles Denken ein
Ende, und unter der Berufung auf Mysterien lässt sich jede beliebige
Behauptung aufstellen. Deshalb sollte zuerst eine genaue Prüfung stattfinden,
bevor die Annahme akzeptiert wird, dass ein Mysterium vorliegt. In dieser
Hinsicht scheint mir Voegelin in der Tat mehr als voreilig gewesen zu sein,
wenn er etwa von einem Paradox des Bewusstseins spricht, obwohl die Tatsache,
dass das Bewusstsein die Welt wahrnehmen kann, von der es selbst zugleich ein
Teil ist, doch bestenfalls eine staunenswerte Besonderheit aber gewiss kein
Paradoxon ist.\footnote{Vgl.  Voegelin, Order and History V, S. 14-15.}
Ebensowenig kann ich mich zu der Ansicht durchringen, dass, wie Voegelin uns
im letzten Band von "`Order and History"' weismachen will, das Wort "`Es"' in
dem Satz "`Es regnet"' auf eine geheimnisvolle "`Es-Realität"' verweist, die
die Partner im Sein: Gott, Welt, Mensch und Gesellschaft
umgreift.\footnote{Vgl. Voegelin, Order and History V, S. 16.} Vielleicht gibt
es Menschen, die in solchen Philosophemen den tiefsten Ausdruck ihres
ureigensten Welterlebens finden können. Für meinen Teil scheint mir jedoch,
dass Voegelin hier alle guten Grundsätze des klaren Denkens in den Wind
schlägt.

Was bleibt aber von Voegelins Philosophie, wenn sie tatsächlich so sehr mit
Irrtümern und Denkfehlern gespickt ist? Sie bleibt immer noch der
Ausdruck einer bestimmten und, wenn man sich an die besseren von Voegelins
Schriften hält, zuweilen reichen und tiefen Weltanschauung. Wenn man die
Aufgabe der Philosophie nicht nur, wie es die analytische Philosophie in der
Tradition des Neupositivismus tut, in der Beantwortung wissenschaftlich
klärbarer Fragen sieht, sondern auch in der Artikulation und Verständigung
über weltanschauliche Überzeugungen, dann ist das immerhin etwas.

\section{Zur Frage der Aktualität von Voegelins Ordnungsentwurf}

Die Frage der Aktualität von Voegelins Ordnungsentwurf bedarf keiner langen
Erörterungen, da die Antwort hierauf eindeutig ausfällt, und sie sich auch in
der wissenschaftlichen Voegelin-Debatte mehr und mehr durchzusetzen
scheint.\footnote{Vgl. Webb, Review, a.a.O.} Voegelins Vorstellung von
politischer Ordnung ist in hohem Maße bedingt und beeinflusst durch das
Zeitalter der Ideologien und des Totalitarismus, in welchem sie entstanden
ist. Voegelin hatte selbst vor dem Nationalsozialismus fliehen müssen. Nicht
minder gegenwärtig waren ihm die Verbrechen der kommunistischen Regime und die
Menschheitsbedrohung durch das atomare Wettrüsten. Unter solchen Bedingungen
kann ein Gefühl von Sicherheit nur schwer aufkommen, und dies erklärt zum Teil
Voegelins polemischen Eifer, welcher sich womöglich einem Gefühl der
Dringlichkeit verdankt, das nach dem Ende des kalten Krieges nicht mehr
unmittelbar verständlich wirkt. Die Zeitumstände erklären auch einiges von
dem, was man die metaphysische Überhöhung des Politischen bei Voegelin nennen
könnte. Aus heutiger Sicht mag es sehr befremdlich und unwissenschaftlich
wirken, die metaphysische Kategorie des Bösen in die Politikwissenschaft
einführen zu wollen.\footnote{Vgl. Voegelin, Eric: Die politischen Religionen,
  München 1996 (zuerst 1938).} Aber um mit einer Erscheinung wie dem
Nationalsozialismus fertig zu werden erscheint dieser Versuch, wiewohl
wissenschaftlich fragwürdig, doch nicht ganz unverständlich.  Vor dem
zeithistorischen Hintergrund ist es daher sehr wohl nachvollziehbar, dass
Voegelin sich nicht auf die Frage beschränkte, welches die geeignetsten
politischen Institutionen für einen guten Staat sind, sondern dem Übel an die
Wurzel gehen wollte und nach den metaphysischen Bedingungen wahrer politischer
Ordnung fragte.

Indes leben wir heute mit einer liberalen politischen Ordnung, die seit
über fünfzig Jahren stabil ist, und die auch keine Risse aufzuweisen scheint,
obwohl sich die Gesellschaft gegenüber den Fünfziger Jahren gewiss noch
weiter säkularisiert hat, was Voegelins Grundthesen über die Ursachen
politischer Unordnung doch sehr zweifelhaft erscheinen lässt. Dazu
vermitteln Voegelins Äußerungen über politische Ordnung nicht selten den
Eindruck, dass es Voegelin weit eher darauf ankam, eine wahre politische
Ordnung (nach den Maßstäben seiner privatreligiösen Überzeugungen) zu
finden als eine im moralischen und pragmatischen Sinne gute politische
Ordnung. Als recht gravierend fällt dabei ins Gewicht, dass Voegelin in
seinem metaphysischen Eifer oft hart an der Grenze zum religiösen
Fanatismus operiert. Seine Suche nach der wahren Ordnung beschwört
dadurch die "`entgegengesetzte Gefahr"' (John H.  Herz\footnote{Vgl.
  John H. Herz: Politischer Realismus und politischer Idealismus.  Eine
  Untersuchung von Theorie und Wirklichkeit, Meisenheim am Glan 1959.
  Mit der "`entgegengesetzten Gefahr"' bezeichnet Herz die besonders dem
  politischen Idealismus inhärente Gefahr bei der Bekämpfung politischer
  Missstände durch das Mittel der Bekämpfung genau den entgegengesetzten
  Missstand herbeizuführen. (Beispiel: Die Bekämpfung des
  kapitalistischen Ausbeutungssystems mündet in die kommunistische
  Diktatur.)}) der religiösen und weltanschaulichen Intoleranz herauf.
Nicht zuletzt aus diesem Grund ist uns heutzutage bei der Suche nach
guter politischer Ordnung mit etwas "`altmodischem Liberalismus"' weitaus
besser gedient als mit Voegelins metaphysischen Rezepturen.

\section{Was sollte dennoch bleiben?}

Wenn Voegelins Philosophie nichts hergibt, und seine politische
Ordnungsvorstellung nichts taugt, wäre es dann nicht besser, Eric Voegelin
ganz zu vergessen? Zwei wichtige Gründe lassen es, trotz aller Kritik,
wünschenswert erscheinen, Eric Voegelin dem drohenden Vergessen zu entreißen.
Zum einen ist da Voegelins imposante Gelehrsamkeit. Voegelins Interpretationen
der Klassiker des politischen Denkens fallen zwar häufig sehr eigenwillig aus
 -- nicht zuletzt deshalb, weil sich Voegelin meist nur auf ganz bestimmte und
scheinbar willkürlich ausgewählte Textpassagen bezieht. Aber Voegelins Auswahl
vollzieht sich fast immer vor dem Hintergrund einer profunden Kenntnis des
Gesamtwerkes. Wenn Voegelin daher auch die falsche Quelle ist, um sich über
die Klassiker des politischen Denkens zu informieren, so dürften Kenner eines
Denkers, den Voegelin behandelt hat, bei Voegelin oft einen
ausgefallenen Kommentar auf hohem intellektuellen Niveau finden. 

Zweitens gilt es die bedeutende kulturwissenschaftliche Horizonterweiterung
festzuhalten, die die Politologie durch Eric Voegelin erfahren hat. Darin
besteht, wenn man so will, das eigentliche Vermächtnis des
Politikwissenschaftlers Voegelin und in dieser Hinsicht ist Voegelin gerade in
der heutigen Zeit von großer Aktualität, denn der Umgang mit fremden Kulturen
erfordert auch für die Politik und die politische Theorie nicht nur eine
Kenntnis der Gesetze und Spielregeln von Diplomatie und Außenpolitik, sondern
auch ein Verständnis dieser Kulturen selbst.  Dabei ist allerdings zu hoffen,
dass Voegelins Denken als Vorbild für ein einfühlendes Verständnis fremder
Kulturen dient, und nicht im Fahrwasser von Samuel Huntingtons "`Zusammenprall
der Kulturen"' zur düsteren Prophetie unüberbrückbarer Gegensätze missbraucht
wird.\footnote{Zur Aktualität Voegelins im Zusammenhang mit Huntingtons
  Theorie: Vgl.  Michael Henkel: Eric Voegelin zur Einführung, Hamburg 1998,
  S. 167.}

Von entscheidender Bedeutung für die Nachwirkung Eric Voegelins dürfte es
jedoch sein, dass die Diskussion um Voegelins Werk mit der notwendigen
kritischen Distanz geführt wird, was die bisherige Sekundärliteratur zu Eric
Voegelin eher vermissen lässt. Der sicherste Weg, Eric Voegelin zu einem
Nischendasein in den Zirkeln religiöser Sektierer zu verdammen, besteht darin,
seine Ressentiments, besonders seinen an Don Quichotte gemahnenden Kampf gegen
echte und vermeintliche Gnostiker in allen Formen und Farben, zu einem
unverzichtbaren Wesensbestandteil seiner Wissenschaft zu
erklären.\footnote{Vgl.  Maben W. Poirier: VOEGELIN-- A Voice of the Cold War
  Era ...? A COMMENT on a Eugene Webb review, in: Voegelin Research News,
  Volume III, No.5, October 1997, auf: http://alcor.concordia.ca/\~{
  }vorenews/v-rnIII5.html (Host: Eric Voegelin Institute, Lousiana State
  University. Zugriff am: 1.8.2007).}  Gerade hier wäre es notwendig, eine
kritische Sonderung vorzunehmen, was bei Voegelin Wissenschaft und was
Vorurteil ist. Immerhin sind zu einer auch kritischen Auseinandersetzung mit
Voegelin mittlerweile schon einige interessante Beiträge
erschienen.\footnote{z.B. Michael Henkel: Positivismuskritik und autoritärer
  Staat.  Die Grundlagendebatte in der Weimarer Staatsrechtslehre und Eric
  Voegelins Weg zu einer neuen Wissenschaft der Politik (bis 1938), München
  2005. -- Sehr deutlich auch: Hans Kelsen: A New Science of Politics. Hans
  Kelsen's Reply to Eric Voegelin's "`New Science of Politics"'. A
  Contribution to the Critique of Ideology (Ed. by Eckhart Arnold),
  Heusenstamm 2004.}  Ein weiterer wichtiger Punkt, der zur Entmystifikation
Voegelins beitragen könnte, ist die Erforschung von Voegelins Biographie,
insbesondere seiner frühen Jahre. Wie einige der inzwischen erschienenen
Studien zu dieser Phase von Voegelins Biographie
zeigen,\footnote{Herausgegriffen seien hier nur: Michael Henkel, a.a.O. --
  Hans-Jörg Sigwart: Das Politische und die Wissenschaft.
  Intellektuell-biographische Studien zum Frühwerk Eric Voegelins, Würzburg
  2005. -- Claus Heimes: Antipositivistische Staatslehre. Eric Voegelin und
  Carl Schmitt zwischen Wissenschaft und Ideologie, München 2004. -- Eckhart
  Arnold: Eric Voegelin als Schüler Hans Kelsens, erscheint voraussichtlich
  Wien 2007.} verlief Voegelins geistige Entwicklung in dieser Zeit viel
spannungsgeladener als seine Autobiographie dies vermuten lässt, denn Voegelin
war gerade in jungen Jahren von jenen irrationalistischen Strömungen der
Geisteskultur der Zwanziger und Dreißiger Jahre nicht wenig beeinflusst, gegen
deren politische Auswüchse er dann später wissenschaftlich zu Felde zog.
Allerdings herrscht, was diesen Teil von Voegelins Biographie angeht
sicherlich immer noch Diskussionsbedarf.

Schließlich könnte eine zukünftige Voegelin-Forschung auch davon profitieren,
wenn sie sich von der, wie es scheint, insgesamt immer noch zu engen Fixierung
auf die Deutung von Voegelins Werk selbst lösen und sich vermehrt den von
Voegelin untersuchten Sachfragen zuwenden würde. Dazu gehört beispielsweise
die Frage nach dem Verhältnis von Religion und Politik oder auch die Frage
möglicher kultureller Vorbedingungen des Gelingens demokratischer
Ordnung. Dies wäre, sollte man meinen, ganz in Voegelins Sinne.

%%% Local Variables: 
%%% mode: latex 
%%% TeX-master: "Main" 
%%% End: 










