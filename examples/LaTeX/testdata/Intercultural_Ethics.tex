%%% Local Variables:
%%% mode: latex
%%% TeX-master: t
%%% End:

\documentclass[12pt,a4paper,ngerman]{article}
\usepackage{microtype}
\usepackage{ae}
%\usepackage[latin1]{inputenc}
\usepackage[utf8x]{inputenc}
%\usepackage{unicode}
\usepackage[T1]{fontenc}
\usepackage{t1enc}
\usepackage{type1cm}
\usepackage[german,ngerman, english, USenglish]{babel}
\usepackage{graphics}
\usepackage{natbib}

\usepackage{ifpdf}
\ifpdf
\usepackage{xmpincl}
\usepackage[pdftex]{hyperref}
\hypersetup{
    colorlinks,
    citecolor=black,
    filecolor=black,
    linkcolor=black,
    urlcolor=black,
    bookmarksopen=true,     % Gliederung öffnen im AR
    bookmarksnumbered=true, % Kapitel-Nummerierung im Inhaltsverzeichniss anzeigen
    bookmarksopenlevel=1,   % Tiefe der geöffneten Gliederung für den AR
    pdfstartview=FitV,       % Fit, FitH=breite, FitV=hoehe, FitBH
    pdfpagemode=UseOutlines, % FullScreen, UseNone, UseOutlines, UseThumbs 
}
\includexmp{Intercultural_Ethics}
\pdfinfo{
  /Author (Eckhart Arnold)
  /Title (Moral Judgments of Foreign Cultures and Bygone Epochs)
  /Subject (A philosophical examination of the legitimacy of moral judgements that concern other cultural or historical contents than one's own epoch or culture)
  /Keywords (Ethical Relativism, Intercultural Ethics, Historical Ethics)
}
\fi

\pagestyle{myheadings}

\sloppy

\begin{document}
\selectlanguage{\english}
\pagenumbering{roman}

\title{Moral Judgments of Foreign Cultures and Bygone Epochs. A Two-Tier Approach}
\author{Eckhart Arnold}
\date{October 10th 2006}

\maketitle

\begin{abstract}

  {\em English}: In this paper the ethical problem will be discussed
  how moral judgments of foreign cultures and bygone epochs can be
  justified.  After ruling out the extremes of moral absolutism
  (judging without any reservations by the standards of one's own
  culture and epoch) and moral relativism (judging only by the
  respective standards of the time and culture in question) the
  following solution to the dilemma is sought: A distinction has to be
  made between judging the norms and institutions in power at a
  certain place and time and judging people acting within the social
  institutions of their time and culture. While the former may be
  judged rigorously, only taking into account the objective
  possibilities for having other institutions at a certain development
  stage, the latter should be judged against the background of the
  common sense morals of the respective time and culture.

  {\em German}: In diesem Aufsatz wird das ethische Problem erörtert,
  wie moralische Urteile über fremde Kulturen und vergangene Epochen
  gerechtfertigt werden können. Nach Ausschluss der Extreme des
  moralischen Absolutismus (uneingeschränkte Beurteilung nach den
  Maßstäben der eigenen Moral) und des moralischen Relativismus
  (Beurteilung ausschließlich nach den Maßstäben der beurteilten
  Kultur oder Epoche) wird die folgende Lösung angestrebt: Es muss
  eine Unterscheidung getroffen werden zwischen der Beurteilung der
  Normen und der Insitutionen, die an einem bestimmten Ort und zu
  einer bestimmten Zeit in Kraft sind, und der Beurteilung der
  Menschen, die innerhalb der gesellschaftlichen Institutionen ihrer
  Zeit und Kultur handeln. Während die ersteren rigoros beurteilt
  werden dürfen, nur unter den Einschränkungen, die sich aus den
  objektiven Möglichkeiten für andere Institutionen in einem
  bestimmten Entwicklungsstadium ergeben, sollten die letzteren stets
  vor dem Hintergrund der ``common sense''-Moral der jeweiligen Zeit
  und Kultur beurteilt werden.

%% For example, we may legitimately
%%   despise the Gladiator fights in ancient Rome, but it would be
%%   pointless to condemn the ancient Romans for enjoying Gladiator
%%   fights, for they only did what was common and appreciated at their
%%   time. Or, we may legitimately criticize the discrimination of woman
%%   in many of today's Islamic countries, but we cannot as easily decry
%%   the men behaving in accordance with the discriminatory customs that
%%   they have grown up with and do not know any better.

\end{abstract}

\fontsize{12}{18}
\selectfont

\newpage

\tableofcontents

\newpage

\setcounter{page}{1}
\pagenumbering{arabic}

\section{Exposition of the Problem}

The problem that I would like to address in this paper is how we can
form sound moral judgments of actions that take place outside of our
own historical and cultural context. Strictly speaking, there are two
different problems, one concerning historical judgments and one
concerning judgments of other cultures. But there is a strong logical
similarity between both types of moral judgments insofar as they both
concern judgments about something that takes place in a life context
different from our own.

It is, I believe, easy to see that this is indeed a problem in the sense that
the historical or cultural context does make a difference for our moral
judgments. For example, when Alexander the Great conquered the city of Tyros
he crucified all remaining men in the city and sold the women and children as
slaves \cite[p. 239]{fox:1973}. Yet, despite the severe violation of human
rights during his conquests historians usually do not tend to place Alexander
in the same league with dictators like Saddam Hussein or Kim Jong-il.  Or, to
take another example, it is reported that some tribes in the highlands of New
Guinea honor newly deceased relatives by devouring their corpses \cite[p.
151]{diamond:2005}. Abhorrent as it may seem to us, there would be no point in
blaming the high-landers of New Guinea for following a revered ancient custom.
 
Thus, there are many cases where a certain amount of cultural or historical
moral relativism seems appropriate. It is simply a fact that values change
over time and differ between cultures. If we do not take account of this fact
in our ethical convictions, we risk to become hopelessly parochial or to slip
into absurdities. On the other hand, the opposite standpoint, a complete
cultural and historical relativism, would be equally unsound. For, to take an
extreme example, there is certainly no way of justifying the atrocities that
communist or fascist regimes committed in the last century on the grounds that
the allowance of licentious manslaughter was common at that time.

Obviously, we can neither leave historical and cultural contexts
aside when forming moral judgments nor must we fully submit to these
contexts. The right solution has to be a golden mean somewhere
 between these extremes. 


\section{Preliminary: The meaning of moral reasoning and discussion in
 face of the impossibility of proving the truth of moral statements}

Although this paper is intended to give an answer to a certain ethical
questions, it is unavoidable to say a few words about {\em meta-ethics},
simply because there is no consensus at all among philosophers
concerning the right method of ethical investigations. Therefore, I will
briefly explain my own point of view concerning the right method of
ethical investigation. 

There exists, in moral philosophy, a problem that is apt to discourage any
kind of ethical reasoning. This is the problem of finding an ultimate
foundation (``Letztbegr\"undung'') for moral values or precepts. In spite of
many attempts during the last 2500 years no philosopher has ever managed to
solve the problem. It would lead too far to enter into the discussion of some
of these attempts here. But the fact that most of them can very easily be
disproven suggests the conclusion that no solution to the foundation problem
of ethics exists, although a positive prove that the foundation problem cannot
be solved does not exist either.\footnote{The logical distinction between is
  and ought does not imply the impossibility of founding ethics, because it
  does not exclude the possibility that what ought to be might be derived in
  some other way than from what is.}

Some people deny that we need to take this fact too seriously by pointing to
seemingly similar foundation problems in the realm of logic and epistemology,
most notably the problem of induction, which, despite the fact that it has not
been solved, never leads to any controversy among scientists, who happily
employ induction to justify their theories. But there
exists an important difference. While there is the theoretical problem of
justifying induction, nobody ever claims in practice that induction cannot be
relied on. This is not the case for the ethical foundation problem, for, as
the examples before have shown, here we are indeed confronted with a
considerable dissent concerning even the most basic of our moral values.

If the meaning of an ethical discourse cannot be any more to find ultimate
reasons why certain values are right and others are wrong, what then could be
the meaning of an ethical discourse and how should it proceed? Without
entering into too much detail here, the answer is that the meaning of ethical
reasoning can primarily consist only in either the determination
(``Festlegung'') of one's own moral will and, furthermore, in the attempt to
influence the moral will of others. That is to say that moral reasoning is
primarily of rethorical character. Only in a secondary sense, that is when a
certain number of normative premises have already been accepted (without any
reason as they must), can ethical reasoning gain the character of a rational
inquiry concerning such questions as whether a certain action is good or bad
according to the premises or what other imperatives follow from the premises
etc. .

The range of premises that needs to be decided upon does not only
encompass concrete values, but, more importantly, also the formal or
logical principles of our ethical systems. These, as well, depend on a
moral decision for their validity. There is no a priori normative
necessity why the system of our morals should be in any way logically
conclusive or why our morals should be systematized at all.
Theoretically, also a system of morals that allows murder when it
rains and forbids it when the sun shines is possible, absurd as it may
seem. However, it must be admitted that at least a certain amount of
systemacity and conclusiveness is a {\em meta-ethical} constant across
all cultures and throughout all ages. How far reaching the
meta-ethical consensus is, is up to empirical science to decide.
(However, no matter what degree of consensus anthropologists might
determine, if someday anybody seriously does not want to adhere to
this consensus, he (or she) cannot be proven wrong by the fact that
such a consensus had hitherto existed, and the consensus is broken
from that time onward.)

Once it has been acknowledged that there exists no a priori necessity why our
ethics should be strictly logical or systematic in a particular way, but that
this too depends on our moral will to have it that way, this has a somewhat
liberating effect on our moral reasoning. For example, we will not any more be
compelled to force our moral intuitions under certain supposedly a priori
principles of morals at any price. (This is what happened to Kant when he
believed that he could decide any moral question by his formula of the
categorical imperative.)  The more formal and logical principles of our ethics
can be weighted against material principles, and we will feel free to allow a
certain amount of inconclusiveness in our moral opinions, if this is more akin
to our moral intuition. We will see that it is hard to arrive at a sound moral
solution to the problem of the judgments of bygone epochs and foreign cultures
without accepting at least some tensions in our judgments.

If we exclude (by moral decision) completely absurd ethical systems, then the
usual case will be that of an ethical system that is generated by (1) meta
ethical decisions that set the logics and formal principles for the subsequent
ethical reasoning, (2) ethical decisions that fill the system with material
values and (3) conclusions and inferences drawn with the help of the
acknowledged formal principles from (1). This raises the question, at what
point do the ethical decisions, especially from (2) enter into our ethical
system? Without discussing this question here, I will assume that ethical
decisions may enter our system of morals at any level of abstraction. We may
decide to adhere to certain more or less abstract values like honesty or love
of man (``Menschenliebe''), but we may also decide to judge a singular case in
a certain way and then adjust our more abstract precepts accordingly, if the
judgment in the singular case does not match the judgment according to the
precepts under which the case must be subsumed. This allows for the well known
method of the ``reflective equilibrium'' to be employed in order to determine
the values we want to adhere. The method works roughly as follows: One starts
with an arbitrary set of values which deems the inquirer reasonable. Then one
looks for example cases where these values come into play. If the judgment by
our values does not match our moral intuition in the example case, we can
either assume our intuition to be wrong or we can conclude that our values
were mistaken and adjust them so that they match our intuition in the
particular case.\footnote{The latter somewhat resembles the procedure of
  falsification of a theory in science, though there is no analog in science
  to the former. Regarding moral intuitions it can be assumed that we have
  moral intuitions in particular situations as well as intuitions of values.
  Our intuitions need not necessarily be clear cut and free from
  contradictions.  However, if we decide on a contradiction free ethical
  system we will probably be forced to neglect some of our intuitions. Which
  of them is a matter of decision, just as it is a matter of decision to take
  into account moral intuitions at all.}

Therefore, in the following examination examples will be used as test
cases in order to ``check'' the proposed scheme of forming such
judgments. Also, as it should be clear by now that moral philosophy is
all about postulating and cannot be anything else, certain moral
values and insitutions like {\em world responsibility} (see below)
will be postualted liberally in the following. It should be understood
that these reflect my own moral oppinions. I would be a liar to claim
any objectivity for them, although I hope they are suggestive enough to
convince others to advocate the same values in the future.


\section{Breaking up the Question: Judgments
of Institutions and Judgments of People}

Moral judgments can be formed with different goals in mind. They can be formed
for the purpose of conflict resolution, which is the case when a judge decides
a lawsuit. Or they can be formed merely with the aim of gaining a well reasoned
moral opinion on some subject matter. This is the goal of historians when they
judge historical persons and their actions. The former requires that we
reach definite and unambiguous solutions, while the latter allows some amount
of ambiguity. If it is just for the sake of forming an opinion, we may look at
the issue from different angles without reducing the different perspectives to
a single ultimate decision. The following discussion is primarily concerned
with well reasoned moral opinions. How the cases where definite decisions must
be made are to be dealt with will only briefly be considered later, in the
concluding paragraphs of this paper.

What then are the reference points that we should look out for in
order to form well reasoned moral judgments of strange cultures and
bygone epochs, if we are to avoid the extremes of imposing our set of
values ({\em moral absolutism}) and {\em moral relativism} alike? The
solution that I would like to propose is to make a fundamental
difference between the judgment of social institutions, including
moral codes, and the judgment of people acting within the social
institutions of their time and culture. While the former may be
valuated rigorously, only taking into account the objective
possibilities for having other institutions at a certain development
stage, the latter should be judged against the background of the moral
common sense of the respective time and culture.

\subsection{Judgments of Institutions and Moral Systems}

When looking at moral systems or social institutions abstractly, we do
not need to take into account in how far it can be expected from a
human being to emancipate herself or himself from traditional moral
prejudices and to rise above the level of his or her surrounding.
Under this perspective we therefore do not need to have any hesitation
to judge rigorously according to our own ethical standards. The reason
why we should do so is simply that morals matter. Moral rules regulate
how people should treat each other and it is a matter of great
importance how people are treated -- anywhere in this world.  More
emphatically we could say that there exists some such thing as a {\em
  world responsibility} which compells us and at the same time
entitles us to take up a stance on what happens to human beings
anytime and anywhere in this world.\footnote{The idea of {\em world
    responsibility} is borrowed from the the total responsibility for
  everything that some strata of the philosophy of existentialism
  assume.} On a mythical level our world responsibility is the
expression of the unity of mankind that is of the moral bonds that
connect any human in this world with any other human being. If we
assume {\em world responsibility} in this sense we cannot suspend our
moral judgment merely on behalf of the remoteness of context -- at
least not when important matters are at stake.

There should be only two restrictions to the rigour of moral judgment
in this case: limits of possibilities and limits of importance.
``Limits of possibility'' describe the fact that certain morally
approvable goals may not be feasible in some contexts. Take, for
example, the introduction of liberal democracy. This form of
government (most probably) cannot exist if not certain prerequisites
concerning social structure, economic prosperity, educational level
and the like are met \cite[p. 438ff.]{schmidt:2000}. Moreover, in
order to install a liberal democracy, a good deal of technical
knowledge about institutional arrangements and procedures is needed, a
technical knowledge that is in its fully developed form a relatively
recent invention.  Therefore, it would be absurd to make a moral point
of the absence of liberal democracy in, say, medieval Europe. The same
holds true for the intercultural case, although it is a little less
obvious there. For, if the technical knowledge required to realize some
moral goal exists somewhere in this world then it should be readily
available anywhere.  But there can still be objective limits of
possibilities that preclude the realization of this or other moral
goods in a certain context. In this case we cannot simply judge
according to our own moral standards, which tacitly rely on the
existence of certain ``objective possibilities'' \cite[]{weber:1906}.

Regarding the limits of possibilities as a restriction of moral
judgment, there is a danger of mistakenly or dishonestly assuming
limits of possibility where really are none. The problem of
determining objective possibilities or the limits thereof is, however,
more an epistemological problem than one of moral philosophy. It is
precisely the problem that historians and social scientists face when
they want to assess the ``objective possibility'' (Max Weber) of
historical developments. As our knowledge of the laws that govern
social developments or the course of history is extremely limited,
determining the ``objective possibilities or impossibilities'' of
social development is quite a difficult task. The techniques by which
social scientists help themselves out when they want to assess the
``objective possibilities'' that a given historical situation offers
include the comparison with similar situations at a different place or
time, or looking at the alternatives that were (or are) under
discussion among the actors within these situations, presuming that
something that was seriously considered by the contemporaries was
probably not totally unrealistic. Roughly speaking, anything that ever
existed represents a possibility, but it may still not be a viable
alternative in a given situation, and conversely, some possibilities
may never have been realized or even thought of and still be realistic
alternatives.

In the intercultural context the question is frequently raised whether
the adoption of certain values, for example modern values like human
rights or religious tolerance or democratic government, is compatible
with a certain cultural background, say Islamic culture. This is an
important question concerning ``objective possibilities'', because if
there really was such an incompatibility of modern values and cultural
tradition, then demanding the the adoption of modern values would
entail nothing less than the abandonment of a culture. To answer the
question, whether the adoption of modern values is compatible with
retaining the traditional culture, a comparison with our own culture
might help. There was indeed a time when Christian occidental culture
posed quite a contrast to the above mentioned ``modern values''.
However, the propagation of these values through the movement of
enlightenment and ultimately their adoption did not lead to the
abandonment of Christian occidental culture but only to a
transformation of this culture. There is no reason why a similar
transformation should be inaccessible to other cultures, although we
will potentially have to face the fact that the members of other cultures
may perhaps not {\em want} to adopt modern values. But since there is
an objective possibility of consolating Islamic culture with modern
values, we do not need to have any hesitations about critizising the
insufficient observance of, say, the human rights in many Islamic
countries today.

The other restriction for the judgment of moral systems and
institutions of foreign cultures or past epochs concerns limits of
importance of the subject matter at hand. The ``importance of the
subject matter'' depends on the rank of the moral values concerned and
on the level of involvedness, which in turn depends on spatial and
temporal distance and the strength or weakness of social or just
empathetic ties. We can call the principle according to which the
importance of a moral subject matter decreases with remoteness the
{\em principle of locality}. A good example for the employment of this
principle are burial rites. In most countries (including western
countries) these are strictly regulated by the law and strong feelings
are involved with regard to the appropriateness of the respective
ceremonial proceedings. Yet, although the burial rites in different
countries may strongly contradict each other, this is hardly a matter
of intercultural controversy. As their regulation by law testifies,
this does by no means entail that they are morally neutral.

There exists, however, a difference here between the
intercultural case and the historical case. In the historical case the moral
importance may indeed decrease until almost nothing is left. Historians do not
really need to argue about the human rights violations that occurred during
Alexander's conquests, if only because there are other aspects of these
happenings that are of much greater historical interest. But in contemporary
times, if in some place of the world severe violations of human
rights occur then the moral aspect cannot be ignored.

Thus we could say that the importance of a moral questions is the
smaller the farther away it occurs and the lower the rank of the
values involved, but that when basic values are concerned it may never
become so small as to render the answer completely unimportant.  The
latter may be understood as a consequence of our {\em world
  responsibility}.

With these restrictions moral judgments of strange cultures and
distant epochs according to one's own set of values represent the
upper limit up to which a rigorous moral absolutism (i.e. the
unanimous application or imposition of one's own values in any
context) is sensible. However, it is only so, when we judge abstractly
about moral systems or about institutions.  When we judge the actions
of concrete people this is still too much, because we have to take
into account the unavoidable limitations of human nature and
especially the fact that anybody's perspective is necessarily limited
by the time and culture he or she is born into and lives in. This will
be the topic of the following.

\subsection{Judgments of People and their Actions}


% As has been shown previously, there exits no ultimate foundation for ethics.
% But since it is impossible to live without any rules, we need to pick some set
% values and consider it as the valid set of values. Of course, a different set
% of values could have been picked as well, and the varying systems of morals in
% different times and cultures do, in a sense, represent such alternative
% choices. But it is hardly possible to accept all these different sets of
% values on an equal footing, not unless we do not wish to take any of them
% serious any more. This, however, raises another question, the question of
% fairness when we judge what people did in former times or what people do in
% other places of the world today.

People in different countries and in different historical epochs act in
accordance with the most diverse systems of norms and values. But it is hardly
possible to accept all these different sets of values on an equal footing, not
unless we do not wish to take any of them serious any more. This, however,
raises the question of fairness when we form moral judgments about what people
did in former times or what people do in other places of the world.

The answer proposed here is that we should judge the actions of
concrete people against the background of the moral common sense of
their respective culture or historical period.\footnote{This idea as
  well as the following discussion of ``\"Ubermoral'' is strongly
  inspired by Hermann L\"ubbe's treatment of ``political moralism''
  \cite[]{luebbe:1987}.} This simple answer may at first sight appear
like plain moral relativism, but it is not.  ``Moral common sense''
can be described as the morals that are common knowledge and in effect
over a longer period of time.\footnote{This definition is, of course,
  not very strict, but only intended as a rough explanation to
  supplement the verbal intuition the phrase ``moral common sense''
  suggests.}  Moral common sense as a criteria frees us from the
necessity to take account of such sets of moral rules that are only
transitory or that remain partial even within one society or that are
in the long run not compatible with the neccessities of every day's
life. This is especially the case for morals that may be characterized
as the outcome of fanatism. Fanatism is an exceptional state of mind
that can hardly be kept up over a longer period of time, and it is to
its full extend often only adopted by a subgroup of the society.  It
may, for a certain while, act as a kind of ``\"Ubermoral'' that
overshadows the common sense moral, but it will never fully replace
the common sense moral, although it must be assumed that it can
influence the subsequent development of the moral common sense to a
certain degree. An example for this kind of ``\"Ubermoral'' are the
morals embodied in the ideologies of totalitarian states. Typically,
the totalitarian morals are so excessive that before they have
pervaded the whole society they are either broken down or have, before
long, been watered down to a much more common sense like version of
themselves. That the Nazis made some attempts to hide the mass
extinction of the Jews from the rest of the populace bears proof of
the fact that they were aware of the existence of a another set of
morals according to which genocide is a crime. If they chose to rather
adhere to Nazi morals they can -- even under the variant of moral
relativism advocated here -- be held fully responsible for this
choice.

The line of reasoning in the previous paragraph does, of course, rest
on the optimistic empirical assumption that ``fanatical morals'' are
normally not evolutionary stable. But if this is true then we can
safely rule out fanatical morals without risking to be ``unfair'' to
the people acting according to a fanatical set of morals. For, neither
do we demand that they act according to an enlightened set of morals
that they cannot realistically be expected to take account of (or 
even just be aware of), nor are we, by taking recourse to the (context
dependent) moral common sense, forced to accept the most unreasonable
moral excesses.

But is the {\em criteria of moral common sense} really sufficient?
Several problems this criteria raises suggest that it is too liberal
and therefore must be restricted some more:

\begin{enumerate}

\item The criteria is {\em ambiguous}: There may be situations where several
  common sense morals are in conflict with each other. Also, the common sense
  moral is continuously changing. According to which common sense moral
  shall we then form our judgments?

\item The criteria is {\em conservative}: If we slavishly stick to the {\em
    criteria of moral common sense} then we would always have to give
  bad marks to those people that are ahead of their time. Moral progress
  would be practically forbidden.
  
\item The criteria is {\em insufficient} in cases, where the traditional
  morals allow or even demand grave moral vices: While fanatism may be only
  short lived, atavisms and superstitions can be an unquestioned part of a
  moral tradition. An extreme example is that of genital mutilation of girls
  practiced in some regions of Africa \cite[]{amnesty:2004}. The practice is so
  abhorrent that any abstract principle of moral judgment that does not allow
  to banish it, must be considered insufficient.

\end{enumerate} 

{\bf 1)} The first objection does not necessarily call for a
restriction of the criteria of moral common sense, but for a further
decision on whether it should be applied liberally or in a more strict
way. A liberal application would mean that any of the several
conflicting common sense morals should be accepted. That is, if some
action is right according to one of these different common sense
morals, we are not entitled to criticize the person committing it any
more. This may lead to contradictions in the sense that possibly
opposing actions must both be accepted as morally legitimate.
(Borrowing a metaphor from politics we could say that as outside
observers we ought to follow a policy of non intervention when
different common sense morals conflict.)

The other way to resolve the conflict between several competing common sense
norms, would be to just pick the one that deems us the best (according to our
own values) as reference.  One might object that this solution essentially
breaks the moral relativism to which we have confined ourselves when judging
the actions of people. But, after all, we have only introduced a limited
relativism to avoid unfair moral judgments. The sort of judgments to be
excluded on behalf of their unfairness are primarily those where we would
implicitly demand from people to become moral inventors in case their
conventional morals should prove unacceptable to our enlightened standards.
But if we confine what we may call {\em the justified demand of moral
  self-reflection} to the respectable systems of common sense morals competing
within the context under discussion then the unfairness is much smaller and
may to this extent be justified by our urge not to give in to a full fledged
moral relativism. Of course, whether we ought to choose a liberal or a strict
application of the criteria of moral common sense, may depend on the
particular circumstances, especially the moral importance of the subject
matter in question.\footnote{It should be emphasized that even if we chose the
  liberal application of the criteria of moral common sense, we still need not
  include fanatism in the previously described sense, because fanatism does
  not even count as common sense moral.}

{\bf 2)} The second objection can only be met by extending our
criteria of moral common sense, so that it also includes progressive
morals (from our own point of view). Unfortunately, we can now hardly
argue for a strict application of the criteria in the above (1) sense
any more, because it would seem unfair to expect from the majority of
people the appreciation of the progressive point of view right away.
What we have gained is only that we are not forced to condemn the
progressivists as a consequence of our own criteria.  This may in
effect lead to ``tragical situations'', situations where conflicting
values clash without even a theoretical possibility of
resolution.\footnote{Usually, there are good reasons for avoiding
  ``tragical situations'' in any system of ethics: Tragical situations
  are often just a bad excuse for not taking a stance or for already
  having chosen the wrong side in the past. More importantly, tragical
  situations are essentially a type of ethical contradiction and
  contradictions should by and large be avoided. What appears as a
  contradiction in an ethical system is in practice a matter that is
  decided by the right of the strongest.  Normally, we do not want
  that. But if there is really no sensible way to resolve an ethical
  conflict it might in certain exceptional cases even be the most
  humane choice to accept tragical situations and thereby the decision
  according to the right of the strongest.  For, then the inferior is
  still spared from additional moral humiliation of having been
  illegitimately wrong.}

{\bf 3)} The third objection could appear to be the most crucial one,
because it seems to force us to dilute our criteria of moral common
sense by other criteria like the criterion of moral importance, which
otherwise should -- due to its relatively strong subjectivity -- only be
applied as a lower rank criteria. But if we think about it a little
longer then we might also come to the conclusion that it is especially
the case of superstitions and atavisms where the two-tier approach to
moral judgments of institutions and norm systems as such and the people acting
within these systems pays off. The best way to overcome superstitious
customs is by education and tenacious convincing. A moralizing attitude
is in danger of producing the adverse effect. The two-tier approach
allows us to condemn the practice itself without reacting with moral
reproach against the very people that need to be convinced. 

If we keep in mind that, following our two-tier approach,
the social institutions as such should still be judged rigorously, then
the relatively weak criteria of moral common sense may, with the
qualifications made above, be morally satisfactory for the judgment of
concrete people and their actions.

\section{Objections and Refinement}

The two-tier approach to moral judgments concerning foreign cultures
and bygone epochs permits a multifaceted and -- as I hope -- much more
balanced view than a single set of criteria would. Still, it is open to
many objections, the most obvious of which is that it introduces too
many and too grave contradictions into our moral reasoning. For example,
we can be forced to condemn a certain action taking place in different
cultural context because it contradicts one or more of our core
values, and at the same time we cannot criticize the person performing
this action because he or she acts according to accepted moral standards
of his or her culture. I believe that tolerating this kind of
contradictions is a lesser evil than either laissez-faire moral
relativism or the intercultural arrogance of moral absolutism. (Of
course, a certain dose of both relativism as well as Western arrogance
is still present in my approach.)

When forming an opinion we can be content with a
multifaceted view and, most probably, this is even better than a single
sided view. But when we have to take decisions then these must be
unanimous. The problem becomes urgent, for example, when we have to
decide on how to deal with immigrant subcultures that bring their own
traditional values, some of which might come into conflict with moral
standards of the host society. There can be only one law in one country,
so that at least when the conflict comes down to legal matters, we will
probably have to revert to a solution that is more in the spirit of
moral absolutism. Still, our judgments will be more reasoned if we keep
in mind that the problem as such is not as simple. 

Quite the opposite becomes true, when we are concerned with intercultural
dialogue. One can hardly start profitable a dialogue on the basis of a claim
of moral superiority. A dialogue can only succeed when the partners talk to
each other on an equal footing, which requires an attitude that may be termed
the {\em willing relativism of dialogue}. This does not mean that we are not
allowed to stand by our moral convictions, but prima facie we will look at the
convictions of the others as equally respectable.

Summing it up, the two-tier approach to moral judgments expounded here will
in many concrete situations have to be resolved to a more univocal point of
view or judgment. However, putting the step of resolving last (in situations
where this is necessary) has the heuristic advantage to allow more well
reasoned judgments over the alternative of deciding definitely on a system of
values first. It allows us to criticize moral standards that we strongly
reject without having to react irritated against the people who comply with
them. The moral judgments arrived at by the two-tier approach will therefore
probably be more satisfactory than otherwise. The latter does, of course,
depend on our moral intuitions or, rather, the self determination of our own
moral will.


\newpage

\bibliographystyle{plainnat}

\bibliography{Intercultural_Ethics}

\end{document}
