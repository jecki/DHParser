
%%% Local Variables:
%%% mode: latex
%%% TeX-master: t
%%% End:

\documentclass[12pt,a4paper,ngerman]{article}
\usepackage{ae}
%\usepackage[latin1]{inputenc}
\usepackage[utf8x]{inputenc}
\usepackage[T1]{fontenc}
\usepackage{t1enc}
\usepackage{type1cm}
\usepackage[ngerman, german]{babel}

\usepackage{ifpdf}
\ifpdf
\usepackage{xmpincl}
\usepackage[pdftex]{hyperref}
\hypersetup{
    colorlinks,
    citecolor=black,
    filecolor=black,
    linkcolor=black,
    urlcolor=black,
    bookmarksopen=true,     % Gliederung öffnen im AR
    bookmarksnumbered=true, % Kapitel-Nummerierung im Inhaltsverzeichniss anzeigen
    bookmarksopenlevel=1,   % Tiefe der geöffneten Gliederung für den AR
    pdfstartview=FitV,       % Fit, FitH=breite, FitV=hoehe, FitBH
    pdfpagemode=UseOutlines, % FullScreen, UseNone, UseOutlines, UseThumbs 
}
\includexmp{Kants_Friedensschrift}
\pdfinfo{
  /Author (Eckhart Arnold)
  /Title (Eine unvollendete Aufgabe: Die politische Philosophie von Kants
  Friedensschrift)
  /Subject (Eine philosophische Auseinandersetzung zur Bedeutung von Kants Friedensschrift in der Gegenwart)
  /Keywords (Immanuel Kant, Zum Ewigen Frieden, Weltfrieden, Kantsches Theorem)
}
\fi

\sloppy

\begin{document}

\title{Eine unvollendete Aufgabe: Die politische Philosophie von Kants
  Friedensschrift\\}

\author{Eckhart Arnold}
\date{Muğla, 7. Oktober 2004}
\maketitle

\begin{quote}
{\em erschienen in: Nebil Reyhani (Hrsg.): Immanuel Kant. Essays
Presented at the Mu\u{g}la University International Kant
Symposium, Vadi Yay\i nlar\i  Verlag, Ankara 2006, S. 496-512.}
\end{quote}


\begin{abstract}
  {\bf English}: In this essay Kant's ``perpetual peace'' is
  interpreted as a {\em realistic utopia}. Kant's ``perpetual peace''
  remains an utopia even today in the sense that the described
  perpetual world peace is still a long way to go from today's state of
  world politics. But Kant also tries to show that this state is
  possible under realistic assumptions. Therefore this essay examines
  the question, if Kant's basic assumptions -- such as for example the
  assumption that democracies are generally non aggressive -- are
  still valid in the light of the political experiences of the two
  centuries that have elapsed since the publication of the ``perpetual
  peace'' and how the realization of Kant's utopia can best be promoted
  in today's situation.

  {\bf Deutsch}: In diesem Aufsatz wird Kants Friedensschrift als eine {\em
    realistische Utopie} gedeutet. Eine Utopie ist Kants Friedensschrift auch
  heute noch in dem Sinne, dass der darin beschriebene dauerhafte Weltfrieden
  vom gegenwärtigen Zustand der Weltpolitik noch denkbar weit entfernt
  ist. Aber Kant versucht auch zu zeigen, dass dieser Zustand unter
  realistischen Bedingungen möglich ist. Der Aufsatz widmet sich daher der
  Frage, ob Kants grundlegende Voraussetzungen -- wie z.B. die Annahme, dass
  Demokratien im Allgemeinen nicht aggressiv sind -- im Lichte der politischen
  Erfahrungen der zwei Jahrhunderte, die seit der Veröffentlichung der
  Friedensschrift verstrichen sind, immer noch als zutreffend angesehen werden
  können, und wie der Verwirklichung von Kants Utopie in der heutigen
  Situation am besten zugearbeitet werden kann.

\end{abstract}

\newpage

\tableofcontents

\section{Einleitung}

Kants berühmte Aussprüche, dass jeder Philosoph seine Philosophie auf
den Trümmern der Systeme seiner Vorläufer aufbaut, und dass man nicht
die Philosophie, wohl aber das Philosophieren lehren könne, scheinen
sich mittlerweile auch für seine eigene Philosophie bewahrheitet zu
haben: Kants Philosophie wirkt inzwischen veraltet. Dieser Eindruck
muss sich einstellen, wenn man die modernen philosophischen Debatten
verfolgt, worin Kant in der Erkenntnistheorie fast überhaupt keine, in
der Ethik allenfalls als Inspirator noch eine Rolle spielt.  Kants
"`Kritik der reinen Vernunft"' -- zweifellos eine der bedeutendsten
erkenntnistheoretischen Leistungen ihrer Zeit -- ist mit den jüngeren
Entwicklungen in den Naturwissenschaften, insbesondere in der Physik
nur noch schwer vereinbar.\footnote{Wenn Kants Theorie der Mathematik
  und insbesondere seine Theorie von Raum und Zeit richtig wäre,
  hätten weder nicht-euklidische Geometrien konzipiert noch erst recht
  die auf einer nicht-euklidischen Geometrie beruhende
  Relativitätstheorie empirisch bestätigt werden können. Ähnliches
  gilt für die Quantentheorie. Andererseits gibt es auch in der
  modernen Erkenntnistheorie Ansätze, die zumindest weitläufig in ein
  kantianisches Paradigma eingeordnet werden können, wie z.B. Bertrand
  Russells (vergleichsweise wenig bekanntes) Spätwerk "`Human
  Knowledge"'.} Wer heute eine Antwort auf die Frage "`Was kann ich
wissen?"' sucht, wird sich eher der modernen analytischen Philosophie
zuwenden als bei Kant nachzuschlagen. In ähnlicher Weise verlieh Kant
zwar in der "`Grundlegung zur Metaphysik der Sitten"' und der "`Kritik
der praktischen Vernunft"' dem Gedanken der Menschenwürde und des
unveräußerlichen Wertes eines jeden einzelnen Menschen klassischen
Ausdruck, scheiterte jedoch bei dem Versuch, dass zentrale Prinzip
seiner Moralphilosophie, den kategorischen Imperativ, philosophisch zu
beweisen.\footnote{An der entscheidenden Stelle in der "`Kritik der
  praktischen Vernunft"' (§7 des ersten Teils des ersten Buches des
  ersten Hauptstückes) beruft sich Kant, anstatt den nach aufwendiger
  Vorbereitung nunmehr fälligen Beweis zu liefern, nur noch auf ein
  Faktum der Vernunft.  Schon vorher unterläuft Kant in seiner
  Argumentation ein schwerwiegender Fehler: Die Unterscheidung
  zwischen kategorischem Imperativ und hypothetischen Imperativen ist
  nicht -- wie Kant voraussetzt -- ausschöpfend. Ein Satz wie "'Esst
  keine Bohnen"' (Empedokles) ist zweifellos ein Imperativ, aber er
  ist weder ein hypothetischer Imperativ im Sinne einer
  Zweck-Mittel-Verknüpfung noch entspricht er offensichtlich Kants
  kategorischem Imperativ, von dem Kant sagt, dass es nur einen
  einzigen geben könne (obwohl er selbst davon drei keineswegs
  äquivalente Formulierungen liefert).}  Heutzutage scheint Kants
Moralphilosophie weitgehend, wenn auch nicht vollständig, vom
Utilitarismus verdrängt worden zu sein.

Es gibt jedoch ein Feld des philosophischen Denkens, auf dem Kants
Philosophie nach wie vor von unbestrittener Aktualität ist. Man könnte
sogar sagen, dass Kants Philosophie auf diesem Gebiet heute aktueller
ist, als sie es in der Zeit unmittelbar nach seinem Tod gewesen
ist. Die Rede ist von Kants politischer Philosophie, besonders seiner
Theorie der internationalen Beziehungen, wie er sie am prägnantesten
in der Schrift "`Zum ewigen Frieden"'\footnote{Immanuel Kant: Zum
ewigen Frieden. Ein philosophischer Entwurf (1795), in: Kants
Werke. Akademie-Textausgabe. Band VIII, Berlin 1968,
S.341-386, im Folgenden zitiert als: Kant, Zum Ewigen Frieden.}
ausgedrückt hat. Kants Friedenskonzept soll im Folgenden sowohl in
seiner historischen Bedeutung als auch hinsichtlich seiner heutigen
Gültigkeit im Lichte der historischen Erfahrungen der letzten 200
Jahre untersucht werden.

\section{Kants Friedensschrift als {\em realistische Utopie}}

Zunächst zur historischen Bedeutung: Zwei wichtige Neuerungen sind es,
die die politische Philosophie von Kants Friedensschrift auszeichnen:
Die erste Neuerung besteht darin, dass Kant den Krieg als solchen als
moralischen Skandal betrachtete. Dies ist bei Kant nun nicht nur im
Sinne einer Anklage oder eines moralischen Appells zu verstehen, wie
wir ihn auch bei Vorgängern wie z.B. in der Friedensschrift des
Erasmus von Rotterdam antreffen.\footnote{Vgl. Erasmus von Rotterdam:
  Querela pacis undique gentium ejectae profligataeque (Die Klage des
  Friedens, der von allen Völkern verstoßen und vernichtet wurde), in:
  Erasmus von Rotterdam: Ausgewählte Schriften. Fünfter Band, Hrsg.
  von Werner Welzig, Darmstadt 1968, S.359-451 (S.419).} Vielmehr
ergibt sich für Kant aus dieser Wertung eine unmittelbare
Pflichtaufgabe zur Beendigung dieses Kriegszustandes.  Das eben ist
neu: Dass Kant den Krieg nicht mehr als schicksalhaftes Ereignis
versteht, dessen Wiederkehr so unvermeidlich ist wie die von
Naturkatastrophen, sondern als menschengemachtes Übel, das durch eine
menschliche Anstrengung überwunden werden kann und muss.

Die Vorstellung, dass der Krieg endgültig überwindbar ist, kann man
sehr wohl als eine politische Utopie bezeichnen. Aber es gibt einen
wichtigen Unterschied zwischen Kants Utopie und den Utopien anderer
politischer Philosophien oder Weltanschauungen. Kants Utopie -- dies
ist die zweite wesentliche Neuerung -- ist eine {\em realistische
  Utopie}. Eine Utopie kann als realistisch bezeichnet werden, wenn
zwei Bedingungen erfüllt sind: Erstens muss der vorgestellte utopische
Zustand unter realistischen Bedingungen, d.h. insbesondere ohne dass
man beispielsweise einen dramatischen Wandel der menschlichen Natur
zum sittlich Besseren voraussetzt, als möglich erachtet werden können.
Zweitens muss es einen denkbaren Entwicklungspfad vom gegenwärtigen
Zustand zum utopischen Zustand geben. Kants Friedenskonzept
berücksichtigt beide Bedingungen: Der ersten Bedingung trägt Kant
Rechnung, indem er den zu erringenden Frieden durch einen föderalen
Friedensbund stützen will. Ein möglicher Entwicklungspfad ist für Kant
dadurch gegeben, dass sich der Friedensbund, angefangen von einer
Föderation einzelner Republiken, schrittweise zu einem Weltfriedensbund
erweitern kann.\footnote{Vgl.  Kant, Zum Ewigen Frieden, a.a.O.,
  Zweiter Definitivartikel, S.356.}

Wenn diese beiden Aspekte von Kants Friedensschrift als Neuerungen
bezeichnet werden, so ist dies natürlich insoweit zu differenzieren,
als auch vor Kant immer wieder Friedenspläne der
unterschiedlichsten Art von den Philosophen entworfen worden sind.
Gerade im Zeitalter der Aufklärung fand über diese Frage eine rege
Diskussion statt, an die Kant unmittelbar anknüpft.\footnote{Zu den
  Vorläufern von Kants Friedensschrift vgl. Georg Cavallar: Pax
  Kantiana. Systematisch-historische Untersuchung des Entwurfs "`Zum
  ewigen Frieden"' (1795) von Immanuel Kant, Wien/Köln/Weimar 1992,
  S.23ff. - Eine Anthologie von Friedensentwürfen deutscher
  Intellektueller aus der Zeit Kants enthält: Anita Dietze / Walter
  Dietze: Ewiger Friede? Dokumente einer deutschen Diskussion um 1800,
  München 1989.} Kaum ein Philosoph vor Kant hat die Friedensfrage
aber mit solcher Klarheit erörtert und dabei zugleich die vielfältigen
intellektuellen Fallstricke vermieden, die in Form eines weltfremden
Utopismus oder eines resignativen Realismus oder -- indem die Not auf
eine sehr unangemessene Weise zur Tugend gemacht wird -- naiver
Kriegsverherrlichung lauern.

\section{Die Kernelemente von Kants Friedenskonzept}

Wie begründet Kant nun die Aufgabe und Möglichkeit eines "`ewigen
Friedens"', und wie trägt er dabei den beiden oben angeführten
Bedingungen für eine {\em realistische Utopie} Rechnung? Kant hat das
Problem des Weltfriedens in mehreren seiner kritischen bzw.
nachkritischen Schriften aus unterschiedlichen Perspektiven
(moralisch, rechtlich und historisch betrachtend)
behandelt.\footnote{Die Schriften, in denen Kant die Frage des
  Weltfriedens behandelt oder anspricht, sind (in chronologischer
  Reihenfolge): Immanuel Kant: Idee zu einer Geschichte in
  weltbürgerlicher Absicht (1784), in: Kants Werke.
  Akademie-Textausgabe. Band VIII, Berlin 1968, S.15-31, im Folgenden
  zitiert als: Kant, Idee zu einer Geschichte in weltbürgerlicher
  Absicht. - Immanuel Kant: Über den Gemeinspruch: Das mag in der
  Theorie richtig sein, taugt aber nicht für die Praxis (1793), in:
  Kants Werke.  Akademie-Textausgabe. Band VIII, Berlin 1968,
  S.273-313, im Folgenden zitiert als: Kant, Über den Gemeinspruch. -
  Kant, Zum ewigen Frieden (1795), a.a.O. - Immanuel Kant: Die
  Metaphysik der Sitten (1797), in: Kants Werke.
  Akademie-Textausgabe.  Band VI, Berlin 1968, S.203-493, im Folgenden
  zitiert als: Kant, Metaphysik der Sitten.}  Insgesamt lässt sich
dabei eine Reihe von Kernelementen bestimmen, aus denen sich Kants
Friedenskonzept zusammensetzt. Diese Kernelemente können, auch wenn
sie natürlich untereinander zusammenhängen, grob unterteilt werden in
moralisch-sittliche Forderungen zur Zähmung der Kriegspolitik,
institutionelle Maßnahmen zur Sicherung des Friedens, und historische
Entwicklungstendenzen, die diesem Ziel entgegen kommen.

\subsection{Die sittliche Pflicht zum Frieden}

Zu den moralisch-sittlichen Forderungen Kants zählt an erster Stelle
die grundsätzliche Pflicht des Menschen, aus dem anarchischen, durch
Gewalt und ungehinderte Rechtsverletzung geprägten Naturzustand
heraus zu treten und in einen Rechtszustand einzutreten. Anders als bei
seinen Vorläufern wird der Gesellschaftsvertrag bei Kant also nicht
nur als eine prudentiell begründete Maßnahme zur Sicherung des
Friedens (Hobbes) oder zur Verbesserung der Rechtspflege (Locke)
gesehen, sondern als sittliche Pflicht verstanden. Ist dies einmal
akzeptiert,\footnote{Wenn man, wie ich es oben getan habe, die
  Moralphilosophie Kants als unbegründet ablehnt, stellt sich
  natürlich die Frage, wie die ethischen Prinzipien, die Kants
  Friedensschrift zugrunde liegen, dann gerechtfertigt werden können.
  (Ich bin Gerrit Steunebrink für den Hinweis auf dieses Problem
  dankbar.) Zu diesem Problem nehme ich eine pragmatische Haltung ein:
  Da es bisher noch keine allgemein geteilte Lösung des
  Letztbegründungsproblems in der Ethik gibt, ist es am sinnvollsten
  mit solchen Grundsätzen zu beginnen, die in hohem Maße als
  konsensfähig gelten können. Den Grundsatz, dass Frieden sein soll,
  halte ich in hohem Maße für konsensfähig. Dass es darüber hinaus
  auch eine Pflicht gibt, (durch geeignete Institutionen) einen
  Zustand zu schaffen, in dem der Frieden dauerhaft gesichert wird,
  dürfte nicht gleichermaßen konsensfähig sein, lässt sich aber -
  entsprechend den Überlegungen Kants - aus dem vorhergehenden Prinzip
  ableiten.} so erscheint es nur konsequent, auch in den Beziehungen
zwischen den Staaten nicht nur eine Pflicht zur gegenseitigen Achtung
der Rechte anzunehmen, sondern zusätzlich eine Pflicht zur
Institutionalisierung des Rechtszustandes.\footnote{Vgl.  Kant: Zum
  ewigen Frieden, a.a.O., Zweiter Definitivartikel, S.354f.  - Die
  Forderung, den Rechtszustand zwischen den Staaten zu
  institutionalisieren, fällt allerdings auch bei Kant vergleichsweise
  schwächer aus als die, den Naturzustand zwischen den Menschen zu
  überwinden.} Im weiteren Sinne zählen zu den moralischen Forderungen
aber auch die friedenspolitischen Grundsätze, die Kant in den
"`Präliminarartikeln"' zum "`ewigen Frieden"' aufstellt, und die alle
den gemeinsamen Zweck verfolgen, die Politik des Staates auf den
Frieden und nicht auf den Krieg hin auszurichten.\footnote{Vgl. Kant, Zum
  ewigen Frieden, a.a.O., Erster Abschnitt, S.343-348.}

\subsection{Institutionelle Maßnahmen zur Friedenssicherung}

Will man sich nicht auf (fruchtlose) moralische Appelle beschränken,
so stellt sich die Frage, durch welche politischen und
institutionellen Maßnahmen der Frieden herbeigeführt bzw. gesichert
werden kann? Hier führt Kant nicht nur Maßnahmen an, die auf die
Gestaltung der zwischenstaatlichen Ordnung zielen, sondern auch
solche, die die innere Verfassung der Staaten betreffen. Unter den
letzteren ist die wichtigste die Einführung der republikanischen
Verfassungsordnung in allen Staaten, da diese Staatsform nach Kants
Überzeugung am stärksten zum Frieden geneigt ist. Daneben ist ein
weiteres wichtiges Prinzip Kants das der freien politischen
Öffentlichkeit, von der Kant sich eine sittlich disziplinierende
Wirkung erhoffte. Für die außenpolitische Absicherung des Friedens
sollte ein föderativer Friedensbund sorgen.

Wie dieser Friedensbund genau zu gestalten ist, lässt sich nicht
zweifelsfrei bestimmen, da Kants Vorstellungen in diesem Punkt
nicht eindeutig sind. Erwägt Kant in der Schrift "`Über den Gemeinspruch"'
noch die Bildung einer "`weltbürgerlichen Verfassung"', also eines
Weltstaates, als die logisch konsequente Lösung,\footnote{Vgl. Kant,
  Über den Gemeinspruch, a.a.O., III. Vom Verhältnis der Theorie zur
  Praxis im Völkerrecht, S.310f.} so ist spätestens in der
Friedensschrift ausdrücklich nur noch von einer Föderation die
Rede,\footnote{Vgl. Kant, Zum ewigen Frieden, a.a.O., Zweiter
  Definitivartikel, S.355-357.} wobei der entscheidende Unterschied
darin besteht, dass eine Föderation anders als der Weltstaat nicht
über ein Gewaltmonopol und folglich auch nicht über eine zentrale
Durchsetzungsmacht verfügt. Ob Kants Friedenskonzept trotz dieser
Einschränkung noch glaubwürdig ist, wird noch zu erörtern sein.  Nicht
unwichtig sind in diesem Zusammenhang auch die Gründe, aus denen Kant
die "`weltbürgerliche Verfassung"' auf einen Friedensbund beschränkt
wissen will. In erster Linie spielen hier pragmatische Überlegungen
eine Rolle dergestalt, dass bei Großstaaten, und erst recht bei einem
Weltstaat, die Gefahr, in eine despotische Herrschaft umzukippen,
größer ist als bei überschaubaren Republiken. Anders als die Autoren
der "`Federalist Papers"', die diese Frage im Zusammenhang mit der
amerikanischen Verfassungsdiskussion beinahe zeitgleich mit Kant
erörterten,\footnote{Vgl. James Madison: Objections to the Proposed
  Constitution From Extend of Territory Answered. From the New York
  Packes. Friday, November 30, 1787, auf: The Avalon Project at Yale
  Law School. The Federalist Papers: No. 14, unter:
  www.yale.edu/lawweb/avalon/federal/fed14.htm .} sieht Kant hier
offenbar keine Vermittlungsmöglichkeit in der Bildung eines föderalen
Bundesstaates mit Zentralgewalt, aber limitierten
Kompetenzen.\footnote{Vgl. Kant, Zum ewigen Frieden, a.a.O., Zweiter
  Definitivartikel, S.354ff. - Vgl. Kant, Metaphysik der Sitten,
  a.a.O., Rechtslehre § 61, S.350f.  - Vgl. Jürgen Habermas: Hat die
  Konstitutionalisierung des Völkerrechts noch eine Chance?, in:
  Jürgen Habermas: Der gespaltene Westen, Frankfurt am Main 2004,
  S.113-193, im Folgenden zitiert als: Habermas,
  Konstitutionalisierung des Völkerrechts, S.127.}  Neben dieser
pragmatischen Argumentation schimmert aber auch bei Kant zuweilen
schon eine grundsätzlichere Begründung durch, indem Kant, gestützt auf
einen logischen Begriffsdogmatismus, aus der vermeintlichen inneren
Widersprüchlichkeit von Souveränitätseinschränkungen durch das
Völkerrecht als eines Rechts souveräner Staaten die Illegitimität
solcher Souveränitätseinschränkungen ableitet.\footnote{Vgl. Kant,
  Friedensschrift, Zweiter Abschnitt, 2., S.354. - Habermas
  substituiert für diese begriffliche Konstruktion das sehr viel
  sinnvollere Argument, dass der Einzelne beim Übergang vom
  Naturzustand zum Staat nichts zu verlieren hat, während er beim
  Übergang von der Staatenwelt zum Weltstaat diejenige Sicherheit und
  Freiheit aufs Spiel setzt, die ihm der Nationalstaat bereits
  gewährt. Vgl. Habermas, Konstitutionalisierung des Völkerrechts,
  a.a.O., S.128-131.}

% Wenn diese Begründung heute nur noch wenig
%einleuchtend erscheint, selbst wenn man (aus anderen Gründen) der
%Vorstellung eines Weltstaates skeptisch gegenübersteht, dann hängt
%dies nicht zuletzt damit zusammen, dass einer der wichtigsten
%Hinderungsgründe, der nach Kant der (demokratischen) Regierung von
%Großstaaten entgegensteht, nämlich die große geographische Ausdehnung,
%mit der Entwicklung moderner Verkehrs- und Kommunikationsmittel
%inzwischen weggefallen ist.\footnote{Vgl. Kant, Metaphysik der Sitten,
%  Erster Teil, 2.Teil, 2.Abschnitt, § 61.} 

\subsection{Historische Gesetzmäßigkeiten, die den Frieden fördern}

Die von Kant vorgeschlagenen institutionellen Maßnahmen zur
Friedenssicherung stützen sich zum Teil unmittelbar auf bestimmte
politische und historische Gesetzmäßigkeiten, die Kant als gegeben
voraussetzt. Zu diesen Voraussetzungen zählt erstens die Annahme,
dass Republiken grundsätzlich friedliebend sind, zweitens die
Überzeugung, dass eine freie politische Öffentlichkeit eine
disziplinierende Wirkung auf die Regierung ausübt, und drittens
glaubte Kant an die pazifizierende Kraft von Handelsbeziehungen, indem
sich durch ökonomische Verflechtungen zwischen den Staaten starke
Eigeninteressen entwickeln, die möglichen kriegerischen Absichten
entgegen stehen.

Der friedliebende Charakter von Republiken wird von Kant damit
begründet, dass in einer Republik diejenigen, die über Krieg und
Frieden entscheiden, auch diejenigen sind, die die Lasten des Krieges
tragen müssen, so dass sie nicht leichtfertig in einen Krieg eintreten
werden. Ob diese Annahme, die von der modernen Politikwissenschaft
auch als das "`Kantsche Theorem"' bezeichnet wird,\footnote{Vgl.
  Ernst-Otto Czempiel: Kants Theorem und die zeitgenössische Theorie
  der internationalen Beziehungen, im Folgenden zitiert als: Czempiel,
  Kants Theorem, in: Matthias Lutz-Bachmann / James Bohmann (Hrsg.):
  Frieden durch Recht. Kants Friedensidee und das Problem einer neuen
  Weltordnung, Frankfurt am Main 1996, S.300-323 (S.300).} sich
empirisch bestätigen lässt wird noch zu untersuchen sein.

%Schon jetzt sei jedoch
%darauf hingewiesen, dass dieses Gesetz ungerechte Kriege nicht
%grundsätzlich ausschließt. Lediglich solche Kriege, die sich nicht für
%eine Mehrheit lohnen, werden nach diesem Gesetz wahrscheinlich nicht
%statt finden. Wie wir später sehen werden muss dies noch weiter
%differenziert werden.

Die Bedeutung der politischen Öffentlichkeit liegt für Kant in
zweierlei: Zum einen ermöglicht die Rede- und Meinungsfreiheit die
öffentliche Entwicklung von Friedenskonzepten, auf die ein Herrscher
gegebenenfalls zurückgreifen kann. Vorausgesetzt wird dabei von Kant
offenbar eine grundsätzliche Friedensneigung der politischen
Intellektuellen (der "`Philosophen"'), wie sie von der Administration
und dem Beraterstab des Herrschers nicht ohne weiteres zu erwarten
ist. Ohne solche Friedenskonzepte, die sich eben vornehmlich in
einer freien politischen Öffentlichkeit entwickeln werden, wäre ein
Staat nur für den Krieg aber nicht für den Frieden geistig
gerüstet,\footnote{Vgl. Kant, Friedensschrift, Anhang, II., S.368ff.}
da es an Vorstellungen darüber fehlt, wie der Frieden gefördert und
auf welche Weise er erhalten werden kann.

Daneben kann der politischen Öffentlichkeit noch eine weitere Funktion
zugesprochen werden, die im Zusammenhang mit Kants Publizitätsprinzip
steht. Kants Publizitätsprinzip besagt: "`Alle auf das Recht anderer
Menschen bezogene Handlungen, deren Maxime sich nicht mit der
Publizität verträgt, sind unrecht."'\footnote{Kant, Friedensschrift,
  Anhang, II., S.368.} Dieses Prinzip lässt sich zu der Überlegung
ausbauen, dass in einem Staat, in dem die Regierung auf eine freie
öffentliche Meinung Rücksicht nehmen muss, die politische
Öffentlichkeit eine besonders in kriegspolitischer Hinsicht mäßigende
und disziplinierende Wirkung auf das Regierungshandeln
ausübt.\footnote{Vgl. Jürgen Habermas: Kants Idee des ewigen Friedens
  - aus dem historischen Abstand von zweihundert Jahren, im Folgenden
  zitiert als: Habermas, Kants Idee des ewigen Friedens, in: Matthias
  Lutz-Bachmann / James Bohmann (Hrsg.), a.a.O., S.7-24 (S.15-17). -
  Die Kontrollfunktion der öffentlichen Meinung und ihre Rolle im
  politischen Willensbildungsprozess treten bei Kant natürlich noch
  nicht so deutlich hervor wie in der Fortführung des Gedankens bei
  Habermas. Bei Kant ist das Publizitätsprinzip in erster Linie ein
  Kriterium für die Moralität politischer Maximen. (Ich danke Karel
  Mom für den Hinweis auf diese Differenzierungen.)}

Die dritte von Kant angenommene Gesetzmäßigkeit betrifft die
friedensfördernde Wirkung des Welthandels. Kant glaubt, dass der
Kriegsgeist mit dem Handelsgeist schlecht zusammen passt, unter
anderem deshalb, weil die mit Krieg verbundene Störung des Handels dem
wohlverstandenen Eigeninteresse handelführender und von
Import und Export abhängiger Staaten widersprechen muss.

\section{Ist Kants Friedenskonzept noch gültig ?}

Wenn nun die Gültigkeit von Kants Friedenskonzept erörtert werden
soll, dann empfiehlt es sich, in umgekehrter Reihenfolge vorzugehen,
und zunächst die von Kant angenommenen historisch-politischen
Gesetzmäßigkeiten zu untersuchen, zumal diese die Sachgrundlage seiner
politischen Gestaltungsvorschläge bilden, und es von ihnen abhängt,
welche Chance auf Verwirklichung seine moralische Forderung der Schaffung
eines dauerhaften Friedens überhaupt haben kann.

\subsection{Das "`Kantsche Theorem"' auf dem Prüfstand}

Die wichtigste Frage ist in diesem Zusammenhang zweifellos, ob
Demokratien tatsächlich, wie Kant dies behauptet, friedliebender als
Staaten mit anderen Regierungssystemen sind. Diese Frage ist schon des
öfteren empirisch untersucht und gerade in jüngerer Zeit ausgiebig
diskutiert worden, so dass hier auf bestehende Ergebnisse
zurückgegriffen werden kann. Der empirische Befund besagt, dass
Demokratien sich gegenüber anderen Demokratien überaus friedlich, um
nicht zu sagen, geradezu pazifistisch verhalten.\footnote{Für diesen
  Befund vgl. Bruce Russet: Grasping the Democratic Peace. Principles
  fo a Post-Cold War World, Princeton University Press, Princeton /
  New Jersy 1993, im Folgenden zitiert als Russet: Grasping the
  Democratic Peace, S.3ff. - Vgl. auch Czempiel, Kants Theorem,
  S.302ff. - Für eine detaillierte Auflistung der Kriege des 19. und
  20. Jahrhundert vgl. Singer, J. David / Small, Melvin: The Wages of
  War 1816-1965. A Statistical Handbook, New York / London / Sidney /
  Toronto 1972, S.383-398.}  Man müsste in der neuren Geschichte schon
sehr weit zurück gehen und den Demokratiebegriff übermäßig stark
ausweiten, um überhaupt auf einen handfesten Krieg zwischen zwei
demokratischen Staaten zu stoßen.\footnote{Als mögliche Ausnahmen
  kämen beispielsweise in Frage: Der amerikanisch-britische Krieg von
  1812, der amerikanische Bürgerkrieg (1861-65), der Burenkrieg (1899)
  und einige weitere. Vgl. dazu die Diskussion mit weiteren Beispielen
  in: Russet, Grasping Democratic Peace, S.16ff. - Die Kriege, die
  zwischen demokratischen Stadtstaaten im antiken Griechenland geführt
  wurden, taugen nur sehr begrenzt als Gegenbeispiel, da sich die
  Demokratien im antiken Griechenland hinsichtlich ihrer
  institutionellen Ordnung und ihrer normativen Voraussetzungen zu
  sehr von den modernen liberalen Demokratien unterscheiden.} Zugleich
zeigt der empirische Befund aber auch, dass Demokratien sich gegenüber
nicht demokratischen Staaten keineswegs weniger aggressiv verhalten
als diese Staaten untereinander.\footnote{Vgl. Czempiel, Kants
  Theorem, S.302ff.}

Geht man von dem "`Kantschen Theorem"' aus, dass Demokratien sich
deshalb friedliebender verhalten, weil in der Demokratie diejenigen,
die über den Krieg entscheiden, zugleich auch die vom Krieg betroffenen
sind, dann ist dieser Befund erklärungsbedürftig.  Insbesondere muss
jede Erklärung dafür, warum Demokratien sich gegenüber
Nicht-Demokratien unter Umständen doch aggressiv gebärden, auch
angeben können, weshalb sie es untereinander wiederum nicht tun.

Denkbar ist, dass das "`Kantsche Theorem"' nur eine begrenzte
Gültigkeit hat, weil auch in modernen Demokratien die Voraussetzung,
dass die über den Krieg Entscheidenden auch die zuerst Betroffenen
sind, nur teilweise eingelöst ist. In der Tat kann dies aus drei
Gründen der Fall sein: Erstens mischen sich in die
Entscheidungsprozesse in der Demokratie immer auch mehr oder weniger
starke Partikularinteressen ein, und gerade außenpolitische
Entscheidungen werden so gut wie niemals basisdemokratisch getroffen.
Zweitens werden bestimmte Bevölkerungsgruppen (junge Männer,
Berufssoldaten, Steuerzahler) in den meisten Fällen stärker und auf
andere Weise vom Krieg betroffen sein als andere. Beides zusammen kann
bereits dazu führen, dass die Gruppe der Entscheidenden und die der
Kriegsbetroffenen auseinander rücken. Schließlich können durch eine
entsprechende Steuerpolitik auch die Kosten des Krieges sehr
ungleichmäßig auf unterschiedliche Bevölkerungsgruppen abgewälzt
werden. Alles in allem ist es also sehr plausibel, davon auszugehen,
dass nur im Idealfall die Voraussetzung von Kants Theorem ohne
Einschränkungen gegeben ist.\footnote{Vgl. Czempiel, Kants Theorem,
  S.312-314.} Als Erklärung reicht dies dennoch nicht hin, denn diese
Erklärung würde für Konflikte zwischen Demokratien ebenso gelten, die
aber -- so der empirische Befund -- so gut wie nie in einen Krieg
ausarten.

Eine weitere Erklärung könnte darin bestehen, dass Kants
Theorem Lücken aufweist, d.h. dass auch dann, wenn die
Voraussetzungen dieses Gesetzes gegeben sind, immer noch andere Gründe
eine Neigung zu Kriegen, die nicht der Verteidigung dienen, in der
Demokratie hervorrufen können. In der Tat lässt sich in Kants
Theorem eine entsprechende Lücke finden. Wenn nämlich die
militärischen Kräfteverhältnisse zwischen einer Demokratie und einem
anderen Staat derartig unausgewogen sind, dass kaum einer der Bürger
der Demokratie fürchten muss, durch einen Krieg negativ betroffen zu
werden, dann ist es sehr wohl denkbar, dass auch ein demokratisches
Mehrheitsvotum zugunsten des Krieges ausfällt. Auch Kants
Publizitätsprinzip wird dies kaum effektiv verhindern können, da, wenn
schon eine Mehrheit am Krieg interessiert ist, auch die Vorwände,
unter denen er geführt wird, von der öffentlichen Meinung akzeptiert
werden dürften. Mit dieser Lücke lässt sich womöglich eine ganze Reihe
"`demokratischer"' Aggressionskriege erklären. Zu denken wäre hier
beispielsweise an die Epoche des Kolonialimperialismus, der ja auch
von demokratisch regierten Staaten ausging.\footnote{Man mag
  einwenden, dass Staaten wie Großbritannien und Amerika im
  19.Jahrhundert, sowie die dritte französische Republik nicht oder
  nur in einem höchst eingeschränkten Maße demokratisch waren, indem
  ein Großteil der Bevölkerung (Frauen, Schwarze in den amerikanischen
  Südstaaten) noch vom Wahlrecht ausgeschlossen waren. In moralischer
  Hinsicht ist dieser Einwand auch vollkommen zutreffend. Für die
  Frage allerdings, ob Demokratien aufgrund ihrer inneren Struktur
  kriegsgeneigter sind als andere Herrschaftsformen, kommt es mehr auf
  das Vorhandensein von demokratischen Entscheidungsstrukturen als auf
  die Breite der Umsetzung des demokratischen Ideals an.} Wie bereits
gesagt, schränkt die Feststellung dieser Lücke in Kants Theorem die
Gültigkeit dieser Gesetzmäßigkeit in keiner Weise ein, verweist aber
darauf, dass mit Kants Gesetz noch nicht alle inneren Bedingungen der
Friedensfähigkeit von Demokratien gegeben sind. Um zu verhindern, dass
Demokratien ungerechte Kriege führen, müssen noch weitere
Voraussetzungen erfüllt sein. Zu denken wäre hier neben einer
wachsamen politischen Öffentlichkeit an Verfassungsschranken sowie an
wirksame internationale Regime.

Sicherlich spielt die oben beschriebene Lücke in "`Kants Theorem"' für
die Erklärung ungerechter Kriege von Demokratien eine große Rolle.
Aber auch mit der Aufdeckung dieser Lücke ist die vergleichsweise
höhere Aggressionsbereitsschaft von Demokratien gegenüber
Nicht-Demokratien nicht erklärbar, da ungleiche militärische
Kräfteverhältnisse als kriegsbegünstigendes Motiv auch zwischen
Demokratien wirksam werden müssten.

Hinsichtlich der Erklärung dieser Differenz seien drei Vermutungen
angestellt. Zum einen kann vermutet werden, dass
zwischen Demokratien eine gewisse Affinität aufgrund der gleichartigen
politischen Systeme besteht. Die Entscheidungsweisen, der Politikstil
des anderen demokratischen Staates sind bekannt; das stiftet
Vertrauen. Ebensogut könnten aber auch historisch kontingente Gründe
eine Rolle spielen wie z.B., dass nach dem Zweiten Weltkrieg die
meisten Demokratien durch Bündnisse und Verträge untereinander
verbunden waren, wodurch die Gefahr möglicher wechselseitiger
Aggressionen -- ganz im Sinne der intendierten Wirkung von Kants
Friedensbund -- von vornherein vermieden wurde. Schließlich könnte man
versucht sein, in Anlehnung an Huntingtons These vom "`Clash of
Civilisations"' darüber spekulieren, ob die demokratische
Friedensneigung nicht auf eine kulturelle Affinität zurückzuführen
sei, die dem ebenfalls kontingenten historischen Umstand geschuldet
ist, dass die meisten Demokratien dem westlichen Kulturkreis
zuzurechnen sind.\footnote{Vgl. zu den hier angeführten Vermutungen:
    Russet, Grasping the Democratic Peace, S.24ff. - Hinsichtlich
    einer sich auf Huntington stützenden Erklärung wäre natürlich in
    Rechnung zu stellen, dass kulturelle Affinität weder den Ersten
    noch den Zweiten Weltkrieg verhindern oder auch nur eindämmen
    konnte, was für die relative Unbedeutsamkeit des kulturellen
    Faktors in diesem Zusammenhang spricht.}

Alles in allem zeigt sich aber, dass die These von der Friedlichkeit
der Demokratien zwar in einigen Punkten der Differenzierung bedarf, in
ihrer Grundsubstanz durch den starken empirischen Befund aber
bestätigt wird. Dementsprechend erscheint auch der Ansatz, den
Weltfrieden durch eine Politik globaler Demokratisierung zu fördern,
grundsätzlich sinnvoll.

\subsection{Politische Öffentlichkeit und Welthandel als friedensfördernde Faktoren}

Wie verhält es sich mit den beiden anderen Faktoren, der politischen
Öffentlichkeit und dem Handel, die die Friedensneigung von Staaten
nach Kants Ansicht positiv beeinflussen? Was die politische
Öffentlichkeit betrifft, so steht heutzutage außer Frage, wie wichtig
die freie und öffentliche Meinungsbetätigung für die Kontrolle der
Herrschaft, die Aufdeckung von Missständen im politischen und
gesellschaftlichen System und für die Ideengenerierung ist. Aber ist
die politische Öffentlichkeit notwendig friedensfördernd? Kants
Überzeugung ruht auf einem moralischen Intellektualismus, der oft in
Zweifel gezogen wird. Die Argumente, die dabei gegen Kant ins Feld
geführt werden, sind unter anderem folgende: Kant überschätze die
Aufgeklärtheit und Einsichtigkeit der Menschen. Spätestens seit der
Entwicklung der Massenmedien habe sich gezeigt, in wie hohem Maße die
öffentliche Meinung manipulierbar sei, was sich insbesondere im
Zusammenhang mit ideologischen Massenbewegungen verheerend auswirken
könne. Schließlich sei auch die öffentliche Meinung fehlbar und von
Eigeninteressen durchsetzt.\footnote{Vgl. Hans Ebeling: Kants "`Volk
  von Teufeln"', der Mechanismus der Natur und die Zukunft des
  Unfriedens, in: Klaus-Michael Kodalle (Hrsg.): Der Vernunft-Frieden.
  Kants Entwurf im Widerstreit, Würzburg 1996, S.87-94 (S.92-93). -
  Vgl. Habermas, Kants Idee des ewigen Friedens, a.a.O., S.15-17.}

In der Tat erscheint Kants Vorstellung von den Philosophen oder
politischen Intellektuellen, die in öffentlicher Diskussion, aber
unabhängig vom Tagesgeschehen und ohne unmittelbar damit verbundene
Eigeninteressen die Grundsätze ausbilden, die eine kluge Politik sich
dann gegebenenfalls zu Herzen nehmen kann,\footnote{Vgl. Kant,
Friedensschrift, Anhang, II., S.368-369.}  etwas naiv. Derartige
Diskussionen finden oft eher in einer wenig beachteten
Fachöffentlichkeit statt, was freilich nicht bedeutet, dass sie auf
Dauer wirkungslos bleiben müssen. Die politische Öffentlichkeit im
engeren Sinne, wie sie von Tagespresse, Fernsehen, Rundfunk und
Internet gebildet wird, ist dagegen sehr viel stärker durch das
Tagesgeschehen geprägt, und sie ist -- anders kann es auch gar nicht
sein -- im Wesentlichen Ausdruck der unterschiedlichen Eigeninteressen
und der vielfältigen moralischen Standpunkte in einer Gesellschaft.
Dadurch sind der Wirkung der politischen Öffentlichkeit als
moralisches Tribunal natürliche Grenzen gesetzt. Noch schwerer wiegt,
dass man sich auf die Aufgeklärtheit der Meinungsführer, Philosophen
und Intellektuelle eingeschlossen, keineswegs verlassen kann. Sie
unterliegt vielmehr Schwankungen und Moden, wie die Popularität
irrationaler Strömungen in der kontinentaleuropäischen Philosophie der
Zwischenkriegszeit -- von Julien Benda so treffend als der "`Verrat
der Intellektuellen"' angeprangert\footnote{Vgl. Julien Benda: La
Trahison des clercs, Paris 1977 (zuerst 1927), S.195ff.} -- ebenso vor
Augen führt wie der heute in arabischen Ländern gerade unter
oppositionellen Intellektuellen sehr populäre islamische
Fundamentalismus.

Aber nicht alle der oben aufgeführten Einwände gegen Kant erweisen
sich als durchschlagend. Insbesondere der Hinweis auf die
Manipulierbarkeit der Öffentlichkeit durch moderne Medien und auf den
Sog ideologischer Massenbewegungen unterschlägt die wesentliche
Voraussetzung Kants, dass die Meinungsbetätigung frei sein muss. Dem
müsste man unter heutigen Bedingungen allerdings noch die Einschränkung
hinzufügen, dass die Medienlandschaft nicht monopolisiert sein darf.
Ist dies aber gegeben, dann können auch die modernsten Medien kaum zu
breit angelegter Massenmanipulation missbraucht werden.
Ideologische Massenindokrination als charakteristisches Merkmal
(totalitärer) Diktaturen widerspricht nicht Kants Ansichten über
die Bedeutung der freien politischen Öffentlichkeit für die Zähmung
des Krieges, sondern unterstreicht ganz im Gegenteil die Wichtigkeit
einer pluralistischen Öffentlichkeit.

Auch hinsichtlich der friedensfördernden Wirkungen des Handels kann
Kants Ansicht mit einigen Differenzierungen als nach wie vor bestätigt
angesehen werden. Ohne systematische empirische Untersuchungen zu
dieser Frage\footnote{Vgl. dazu Ernst-Otto Czempiel:
  Friedensstrategien. Eine systematische Darstellung außenpolitischer
  Theorien von Machiavelli bis Madariaga, 2. Aufl., Westdeutscher
  Verlag, Opladen / Wiesbaden 1998, im Folgenden zitiert als Czempiel:
  Friedensstrategien, S.226.} kann zumindest festgehalten werden, dass
enge Handelsbeziehungen eine Interessenlage schaffen, die als ein
Faktor unter anderen in Richtung auf Friedenserhaltung wirken
kann.\footnote{Vgl. Czempiel, Friedensstrategien, S.225.}
Andererseits darf die Wirkung dieses Faktors nicht überschätzt werden.
Die Wichtigkeit, die Handelsinteressen in den politischen
Entscheidungszentren beigemessen wird, kann -- unabhängig von ihrer
objektiven Bedeutung -- größer oder geringer ausfallen. So hinderten
beispielsweise die ausgeprägten Wirtschaftsbeziehungen, die zwischen
den europäischen Großmächten am Vorabend des Ersten Weltkriegs
herrschten, diese Mächte keineswegs daran, in diesen Krieg
einzutreten, und einen auch für die späteren Sieger wirtschaftlich
überaus ruinösen Krieg vier Jahre lang fortzuführen.

\subsection{Was kann die UNO als Friedensbund leisten?}

Was die Faktoren betrifft, die den Frieden fördern können (Demokratie,
Öffentlichkeit, Welthandel), scheint Kant also -- trotz einiger
Einschränkungen und Differenzierungen -- durch die empirische
Politikwissenschaft sowie durch die Geschichte im Wesentlichen
bestätigt worden zu sein. Wie verhält es sich aber mit den
institutionellen Maßnahmen, die Kant zur Förderung des Friedens
vorschlug. Wie gezeigt wurde, lehnte Kant die aus seinen
moralphilosophischen Voraussetzungen heraus eigentlich logisch
erscheinende Forderung nach der Bildung eines Weltstaates ab und
beschränkte sich auf die Empfehlung, einen eher losen Friedensbund zu
gründen. Hat sich diese Empfehlung verwirklichen lassen? Und wichtiger
noch: Kann der Friedensbund seinen Zweck, Frieden zu schaffen und zu
sichern, erfolgreich erfüllen?

Das Experiment eines Friedensbundes, d.h. einer staatenübergreifenden
Organisation, die dem (ernstgemeinten) Ziel dient, Frieden zwischen
den beteiligten Staaten zu stiften, ist seit Kants Vorschlag sowohl
auf regionaler als auch auf globaler Ebene mehrfach unternommen
worden. Auf globaler Ebene ist dieses Konzept zweimal, erst in der
Gestalt des Völkerbundes und nun durch die UNO umgesetzt worden.
Aber obwohl Kant in dieser Hinsicht der größte Erfolg beschieden war,
und sein Friedensbund weitgehend seinen Vorstellungen entsprechend in
die Wirklichkeit umgesetzt worden ist (ein Glück wie es nur den
wenigsten Philosophen mit ihren Vorschlägen zur Verbesserung der
Politik beschieden ist), hat sich die erwünschte Folge, der
Weltfrieden, dennoch nicht eingestellt. Für einige Kritiker Kants ist
sein Friedenskonzept damit offensichtlich gescheitert.\footnote{Vgl.
  Ebeling, a.a.O., S.88.} Andere ziehen daraus die Konsequenz, dass
Kants Friedensbund ohne militärische Durchsetzungsmacht eben doch zu
schwach sei, um den Frieden zu garantieren, und dass man daher seinen
Friedensbund in Richtung einer Weltregierung bzw. eines Weltstaates
weiterdenken müsse, so unrealistisch dies auf absehbare Zeit
erscheint.

Wie schon zuvor bei der Frage, wie "`Kants Theorem"' mit der Tatsache
demokratischer Aggressionskriege vereinbar ist, kann allerdings auch
hier bezweifelt werden, ob der Befund der sehr begrenzten Wirksamkeit
der Friedensinstrumente der Weltorganisation UNO Kants Theorie
widerspricht. Zwar weist der von Kant konzipierte Friedensbund in der
Tat die Schwäche auf, dass er nur sehr lose konstruiert ist.
Andererseits bildet für Kant die Demokratisierung der Staaten eine
weitere Voraussetzung für den "`ewigen Frieden"', und man kann -- im
Lichte des oben angeführten empirischen Befundes -- zugestehen, dass
zwischen Demokratien ein loser Friedensbund genügen und die Bildung
eines Superstaates unnötig sein würde, da Demokratien sich
untereinander von vornherein friedlich verhalten. Kants Theorie wäre
also durch die nur begrenzte Fähigkeit der UNO zur Friedenssicherung
nicht widerlegt, da die andere wichtige Voraussetzung Kants, nämlich
die Demokratisierung der Staaten, im Weltmaßstab noch zu wenig erfüllt
ist.\footnote{Vgl. Ottfried Höffe: Die Vereinten Nationen im Lichte
  Kants, in: Ottfried Höffe (Hrsg.): Immanuel Kant: Zum ewigen
  Frieden, Berlin 1995, S.245-272 (S.254).} Einen
eindrucksvollen Beleg für die Schlüssigkeit von Kants Friedenskonzept
liefert der Erfolg der Europäischen Union bei der Stiftung
innereuropäischen Friedens, denn für die in der Europäischen Union
versammelten Staaten trifft die Voraussetzung der demokratischen
Verfassungsform zu, während sie ebenso wenig wie die UNO über eine
zentrale Durchsetzungsmacht verfügt.

Aber eine solche Rettung der bloßen Theorie Kants bliebe sehr
unbefriedigend, denn sie befreit natürlich nicht von der Frage, wie
denn der Weltfrieden gefördert werden kann, wenn nun einmal nicht
alle Staaten Demokratien sind, und wenn die UNO unter den
gegenwärtigen Bedingungen nicht in der Lage ist, ihre
friedenssichernde Aufgabe umfassend zu erfüllen.  Hier scheint ein
Dilemma vorzuliegen: Je demokratischer die Staaten der Welt sind, um
so weniger Weltstaat wäre notwendig, um Frieden zu schaffen. Umgekehrt
müsste eine effektive Weltorganisation um so stärker sein, je weniger
demokratische Staaten es gibt. Aber gerade dann ist es
unwahrscheinlich, dass eine Weltorganisation, die weitgehend auf
Prinzipien des Konsenses und der Übereinstimmung aufgebaut sein muss,
diese Stärke erlangen kann.

Wie kann dann aber in der gegenwärtigen Situation der Weltfrieden
gefördert werden? Die Pflicht, auf dieses Ziel hinzuarbeiten, ist -- 
folgt man Kant -- ja auch dann unbestreitbar, wenn wir es nie ganz
erreichen können.\footnote{Was aber hinsichtlich des Weltfriedens,
  anders als hinsichtlich anderer Ziele wie der vollständigen und
  endgültigen Ausmerzung des Bösen in der Welt, immerhin denkbar ist.
  So gesehen kann man Kant zustimmen, dass die Philosophie in diesem
  Punkt ihren Chiliasmus haben darf. Vgl. Kant, Idee zu einer
  Geschichte in weltbürgerlicher Absicht, Achter Satz, S.27.} Wenn man
einmal von philosophisch zwar immer legitimen aber politisch sehr
unrealistischen Spekulationen über die Wünschbarkeit eines Weltstaates
absieht, dann zeichnen sich in der gegenwärtigen Situation zwei
Alternativen ab,\footnote{Diese beiden Alternativen beziehen sich nur
  auf die kurzfristig naheliegenden Politikoptionen. Für eine
  wesentlich weiter ausgreifende Diskussion der möglichen Entwicklung
  des zukünftigen System der Weltpolitik vgl. Habermas,
  Konstitutionalisierung des Völkerrechts, a.a.O., S.133ff. - Habermas
  befürchtet neben den beiden hier diskutierten Alternativen noch eine
  dritte Möglichkeit, nämlich die Aufspaltung der Weltgesellschaft in
  kulturalistisch definierte, sich gegenseitig feindlich gegenüber
  stehende Großräume im Sinne Carl Schmitts (und Huntingtons). Vgl.
  ebda., S.187ff.} die beide in Anspruch nehmen können, dem Ziel des
Friedens und der Sicherheit der Welt zu dienen. Die erste Alternative
bestünde darin, Sicherheitspolitik nur strikt unter dem Dach und nach
den Vorgaben der UNO zu betreiben. Dies schließt natürlich ein, dass
auch alle Schwächen und Nachteile der Sicherheitsmechanismen der UNO
in Kauf genommen werden müssen, was insbesondere bedeutet, dass
aufgrund der Vetoposition einzelner ständiger Mitglieder des
UNO-Sicherheitsrates der Weltgemeinschaft in vielen Fällen, in denen
im Interesse des Friedens ein Eingreifen geboten wäre, die Hände
gebunden sind.  Gleichzeitig müsste daher auf jeden Fall an der
Verbesserung der Sicherheitsmechanismen der UNO gearbeitet werden,
wobei man sich jedoch keinen Illusionen darüber hingeben darf, hier in
kurzer Zeit Umwälzendes zu erreichen. Die andere Alternative bestünde
darin, die UNO und den Sicherheitsrat gegebenenfalls beiseite zu
lassen, und sich auf effektivere Bündnisse und Allianzen (wie z.B. die
NATO) zu stützen, um dort, wo es notwendig erscheint, und woran
mächtige Staaten ein Interesse nehmen (denn ohne solche Interessen
findet weder mit noch ohne UNO-Sicherheitsrat ein Eingreifen
statt\footnote{Die Frage ist nicht, ob irgendwelche Staaten bei
  friedensschaffenden Einsätzen egoistische Interessen verfolgen oder
  nicht, sondern nur, ob ohne Rücksicht auf die egoistischen
  Interessen der Einsatz moralisch geboten ist oder nicht.
  Dementsprechend genügt der Nachweis eigennütziger Motive allein
  nicht, um die moralische Verwerflichkeit des kriegerischen
  Engagements eines Staates zu demonstrieren. Ein anderer, häufig
  ebenso unberechtigter Vorwurf ist der der Inkonsequenz. Denn auch
  wenn es inkonsequent ist in einem Fall einzugreifen, in einem
  anderen, ähnlich gelagerten aber nicht, so ist es doch immer noch
  besser -- sofern der Eingriff überhaupt geboten ist -- wenigstens in
  einem Fall zu helfen als in gar keinem.}), zügig einzugreifen.

Um zu zeigen, dass die Entscheidung zwischen diesen beiden
Alternativen keineswegs eindeutig ausfallen muss, selbst wenn man
sich, ganz im Geiste Kants, allein am Leitfaden des moralisch
Gebotenen orientiert, sei das Problem an Hand von zwei Beispielen aus
der jüngeren Vergangenheit illustriert. Das erste Beispiel ist der
Kosovo-Einsatz der NATO im Jahre 1998. Das zweite Beispiel der Irak
Krieg von 2003. Beim Kosovo-Einsatz hat die NATO durch eine Serie von
Luftangriffen Serbien zur Beendigung des Krieges gegen die
mehrheitlich albanische Provinz Kosovo gezwungen. Der Kosovo-Krieg
wurde damit erfolgreich gestoppt, eine humanitäre Katastrophe, wie sie
zuvor von der UNO in Bosnien nicht hatte verhindert werden können,
konnte abgewendet werden.\footnote{Für einen knappen historischen
  Abriss vgl. Malte Wellhausen: Humanitäre Intervention. Probleme der
  Anerkennung des Rechtsinstituts unter besonderer Berücksichtigung
  des Kosovo-Konflikts, Baden-Baden 2002, S.183-192. - {\em
    Nachträgliche Ergänzung (1. 11. 2015): Anders als das damals in
    der öffentlichen und auch einem Großteil der wissenschaftlichen
    bundesdeutschen Diskussion erschien, wird die Frage, ob der Kosovo
    Einsatz vergleichbare Verbrechen wie im Bosnien-Krieg verhindert
    hat, von der historischen Forschung inzwischen sehr viel
    differenzierter betrachtet. Ein regelrechter Plan zur ethnischen
    Säuberung des Kosovo (der sogenannte ``Hufeisenplan'') ließ sich
    nicht nachweisen. Möglicherweise hat der damalige
    Verteidigungsminister Rudolf Scharping gelogen als behauptet hat,
    über entsprechende Hinweise zu verfügen. Das Haager
    Kriegsverbrechertribunal hat die Hinweise, die er vorlegen konnte,
    später als unzureichend angesehen, um den Hufeisenplan in die
    Anklage gegen Milošević aufzunehmen. Damit ist aber zumindest
    weniger klar, ob der Einsatz eine humanitäre Katastrophe von
    solchem Ausmaß, dass sie zur Rechtfertigung des Einsatzes hätte
    taugen können, tatsächlich verhindert hat. Zugleich stellte sich
    durch den Einsatz das Problem, wie man nun wiederum die Gewalt der
    albanischen UÇK kontrollieren konnte, nachdem die serbischen
    militärischen Kräfte erfolgreich ausgeschaltet worden waren. Das
    Fatale an der Situation im zerfallenden Jugoslawien bestand darin,
    dass auf allen Seiten nationalistische Kräfte die Politik
    bestimmten. Opfer- und Täterrollen waren damit weniger eindeutig
    festgelegt, als das die öffentliche Meinung in Deutschland sich
    damals vorstellen wollte. Vgl. Marie-Janine Calic: Geschichte
    Jugoslawiens im 20. Jahrhundert, Beck Verlag München, 2. Auflage
    2014, S. 306 ff.}} Die moralische Frage, ob dieser Einsatz erlaubt
und geboten gewesen ist, lässt sich ziemlich klar zugunsten dieses
Einsatzes entscheiden,\footnote{Es gibt nur wenige Stimmen, die dem
  widersprechen, so etwa mit einer (kaum überzeugenden) kantianischen
  Argumentationsfigur Reinhard Merkel: Das Elend der Beschützten. Der
  NATO-Angriff ist illegal und moralisch verwerflich, in: Dieter
  S. Lutz (Hrsg.): Der Kosovo-Krieg.  Rechtliche und rechtsethische
  Aspekte, Baden-Baden 2000, S.227-232 (S.230-231). - {\em
    Nachträgliche Ergänzung (1. 11. 2015): Siehe auch die Ergänzung
    zur vorhergehenden Fußnote. Die Lage ist komplizierter gewesen und
    damit auch die moralische Beurteilung weniger eindeutig. Dabei
    deuten sich auch Fallstricke der rein philosophischen, d.h. zu
    sehr an Prinzipien orientierten ethischen Argumentation an: Damit
    die Prinzipien greifen können, muss oft eine Eindeutigkeit der
    empirischen Sachlage gegeben sein, wie sie in der (historischen)
    Wirklichkeit leider selten gegeben ist.}}  mit einer
Einschränkung: Der Einsatz war nicht durch den Sicherheitsrat
autorisiert worden und stellte damit formal gesehen eine illegitime
Aggression gegen einen souveränen Staat (Serbien)
dar.\footnote{Vgl. Hermann Weber: Die NATO-Aktion war unzulässig, in:
  Dieter S. Lutz, a.a.O., S.65-71. - Auf dasselbe Ergebnis läuft auch
  die differenzierte rechtliche Würdigung bei Malte Wellhausen hinaus:
  Vgl. Malte Wellhausen, a.a.O., S.200ff. - Möglichkeiten einer
  völkerrechtlichen Rechtfertigung des Einsatzes sieht dagegen Knut
  Ipsen: Vgl. Knut Ipsen: Der Kosovo-Einsatz -- llegal?
  Gerechtfertigt? Entschuldbar?, in: Dieter S. Lutz, a.a.O.,
  S.101-105.} Stellt man sich auf den grundsätzlichen und moralisch
überaus plausiblen Standpunkt, dass friedensschaffende Militäreinsätze
nur von der UNO autorisiert stattfinden sollten, dann muss man
erklären können, warum man in diesem Fall bereit gewesen wäre, den
Kosovo-Albanern ein so grausames Schicksal zuzumuten, wie es die
Bosnier unter den Augen hilfloser UNO-Blauhelmtruppen zuvor schon
hatten erdulden müssen.\footnote{Die Frage, ob den Kosovo-Albanern
  vergleichbar grausame ethnische Säuberungen drohten, lässt sich - da
  sie durch den NATO-Einsatz effektiv verhindert wurden - nicht mit
  letzter Sicherheit klären.  Deutliche Anzeichen und nicht zuletzt
  der Präzedenzfall Bosnien sprachen aber dafür. Vgl. Malte
  Wellhausen, a.a.O., S.183-192. - Für eine etwas skeptischere
  Beurteilung der humanitären Gesamtbilanz des Einsatzes vgl. Hans
  Joachim Heintze: Gibt es ein Recht auf humanitäre Intervention? Das
  Völkerrecht nach dem Kosovo-Krieg, in: Ulrich Albrecht, Michael
  Kalmon, Sabine Riedel, Paul Schäfer (Hrsg.): Das Kosovo-Dilemma.
  Schwache Staaten und Neue Kriege als Herausforderung des
  21.Jahrhunderts, Münster 2002, S.165-181. {\em Nachträgliche
    Ergänzung (1. 11. 2015): Aus Sicht der neueren historischen
    Forschung vgl. dazu: Calic, Geschichte Jugoslawiens im
    20. Jahrhundert, a.a.O., S. 306ff.}}

Ist also die Sicherung des Friedens, soweit sie militärisches
Eingreifen erfordert, bei entscheidungsfähigen und schlagkräftigen
Allianzen wie der NATO oder einer ad hoc "`Koalition der Willigen"'
unter Führung einzelner Großmächte wie Amerika doch besser aufgehoben?
Dagegen spricht wiederum das andere Beispiel, der Irak-Krieg von 2003.
Der Irak-Krieg kann geradezu als das Paradebeispiel eines ungerechten
demokratischen Krieges gelten. Die möglichen legitimen Kriegsgründe:
Schutz vor Massenvernichtungswaffen und Krieg gegen den Terror, haben
sich, wie angesichts der dünnen nachrichtendienstlichen Grundlage
abzusehen war, als falsch herausgestellt.\footnote{Die erfreuliche
  Folge des Krieges, dass die Diktatur Saddam Husseins beseitigt
  worden ist, genügt als Rechtfertigung kaum.} Der Irak-Krieg ist
damit nicht nur, wie der Generalsekretär der UNO Kofi Annan im
Nachhinein angemerkt hat,\footnote{Meldung auf Spiegel-Online vom 16.
  September 2004 unter:
  http://www.spiegel.de/politik/ausland/0,1518,318253,00.html (Zugriff
  am: 12. Oktober 2004).} formal rechtswidrig, sondern auch moralisch
ungerechtfertigt gewesen.\footnote{Vgl. Herfried Münkler: Der neue
  Golfkrieg, Hamburg 2003, S. 123ff.}

Für welche Lösung soll man sich angesichts dieses Dilemmas
entscheiden? Für eine vergleichsweise effektivere aber unter
moralischen Gesichtspunkten fehleranfällige Pazifizierungs- und
Demokratisierungspolitik unter der Führung Amerikas oder für einen,
wenn man es so nennen will, "`UNO-Friedensprozess"', der ebenfalls von
Supermächten (den Mitgliedern des Sicherheitsretes und besonders
Amerikas) dominiert wird, aber durch die Statuten und
Entscheidungsprozeduren der UNO eine höhere Gewähr dafür bietet nicht
durch machtpolitische Einzelinteressen oder Selbsttäuschungen
unterlaufen zu werden, wenn er auch durch Einzelinteressen blockiert
werden kann? Die Frage ist, wie die unterschiedlichen Beispiele
gezeigt haben dürften, nicht von vornherein in der einen oder der
anderen Richtung zu beantworten. Grundsätzliche Überlegungen sprechen
jedoch dafür, dass ein UNO-Friedensprozess sehr viel eher im Sinne von
Kants Friedensidee ist. In der UNO sind bestimmte, an sittlichen
Prinzipien orientierte Formen der Konfliktlösung zwischen Staaten
institutionalisiert. Ob diese Formen mit Leben erfüllt werden, hängt
freilich von der Kooperationsbereitschaft der einzelnen Länder und
insbesondere der großen Mächte ab. Dennoch ist mit der UNO ein Kontext
geschaffen wurden, innerhalb dessen außenpolitische Streitfragen
artikuliert werden können.\footnote{Vgl. Czempiel, Friedensstrategien,
  109ff.} In Übereinstimmung mit Kants Publizitätsprinzip erschwert
dies zumindest die offene Verfolgung ruchloser Kriegspolitik. Damit
verkörpert die UNO eine Art sittlichen Basis-Konsens in
institutionalisierter Form, an den selbst nach einer längeren Phase
faktischer Außerkraftsetzung ihrer Sicherheitsmechanismen im kalten
Krieg eine rasche Wiederanknüpfung möglich war. Zwar kann auch eine
nationale Politik im Alleingang sittlichen Zielen verpflichtet sein,
aber sie wird kaum den gleichen Grad an Legitimität erreichen können
und für die anderen Staaten weniger berechenbar und dadurch
bedrohlicher erscheinen.

Ein solches grundsätzliches Bekenntnis zur UNO schließt nicht aus,
dass Abweichungen davon in Einzelfällen wie dem Kosovo-Krieg geboten
sind.\footnote{Günstiger wäre es natürlich formaljuristischen
  Missständen wie dem sittlich gebotenen, rechtlich aber verwehrten
  Kosovo-Einsatz durch Reformen der UNO und des Völkerrechts
  abzuhelfen. An ernstzunehmenden Vorschlägen dazu, die beispielsweise
  auf den Aufbau regionaler Autorisierungsmechanismen setzen, mangelt
  es nicht: Vgl. Malte Wellhausen, a.a.O., S.240ff. - Vgl. Winrich
  Kühne: Humanitäre NATO-Einsätze ohne Mandat? Ein Diskussionsbeitrag
  zur Fortentwicklung der UNO-Charta, in: Dieter S.  Lutz, a.a.O.,
  S.73-99. - Allerdings ist es unwahrscheinlich, dass selbst nach
  weiteren Reformen das Sicherheitsregime der UNO vollkommen genug
  ist, um vergleichbare Probleme in Zukunft auszuschließen.}  Dies
einzuräumen ist zweifellos inkonsequent, aber wer kann und wer will
schon Philosoph genug sein, um -- wie es Kant in anderem Zusammenhang
in seiner Abhandlung "`Über ein vermeintliches Recht aus Menschenliebe
zu lügen"' unglaublicherweise vorgemacht hat\footnote{Vgl. Immanuel
  Kant: Über ein vermeintliches Recht aus Menschenliebe zu lügen, in:
  Kants Werke. Akademie Ausgabe. Band VIII, a.a.O., S.423-430
  (S.425-430).}  -- bloß um der logischen Folgerichtigkeit willen
Menschenleben zu opfern?

\subsection{Kants vernünftiger Moralismus}

Es könnte als ein Verstoß gegen ein fest etabliertes Grundprinzip der
Moralphilosophie, das Prinzip, dass vom Sein kein Weg zum Sollen
führt, verstanden werden, wenn nun zuletzt die Frage untersucht wird,
ob die von Kant eingeforderte Pflicht zum Frieden im Lichte der
untersuchten Tatsachen noch haltbar ist. Aber das grundsätzlich
richtige Prinzip, dass vom Sein kein Weg zum Sollen führt, wird in der
Praxis so gut wie immer durch (oft implizite) Normen, sogenannte
"`Brückenprinzipien"', eingeschränkt, die eine Verknüpfung zwischen
beiden Bereichen herstellen, was solange logisch einwandfrei bleibt,
wie diese Normen ihrerseits nicht vom Sein her begründet
werden.\footnote{Vgl. Gerhard Schurz: The Is-Ought Problem. An
  Investigation in Philosophical Logic, Dordrecht/Boston/London 1997,
  S.279-285.} Gerade in der politischen Philosophie ist es sinnvoll
ein derartiges Brückenprinzip zugrunde zu legen, das im Folgenden das
{\em Realitätsadäquatheitsprinzip} genannt werden soll. Das
Realitätsadäquatheitsprinzip besagt, dass keine politischen
Zielsetzungen verfolgt werden sollten, deren Erreichung sehr
unwahrscheinlich oder gar unmöglich ist, und die damit unrealistisch
sind. Dieses Prinzip findet seine Rechtfertigung darin, dass die
Kosten (an Geld und mehr noch an Menschenleben), die bei der
Verfolgung unrealistischer Ziele anfallen, sich niemals auszahlen
werden.

Wie ist Kants Friedenskonzept unter der Berücksichtigung des
Realitätsadäquatheitsprinzips zu beurteilen? Eingangs wurde Kants
Friedensidee als eine {\em realistische Utopie} charakterisiert. Nach
der eingehenden Untersuchung der einzelnen Elemente von Kants
Friedenskonzept lässt sich diese Einschätzung zweifellos aufrecht
erhalten. Nicht unentscheidend ist dabei, dass der Weg zum Weltfrieden
schrittweise gegangen werden kann, und dass dabei Teilerfolge erzielt
werden können.  (Insofern könnte man sagen, dass die Kosten für die
Verfolgung dieses Ziels immer kontrollierbar bleiben.) Im Zweifelsfall
kann sich Kant daher immer darauf zurückziehen, dass auch eine
ewig währende Annäherung an den "`ewigen Frieden"' noch sinnvoll und
dann selbstverständlich auch moralisch geboten ist. Freilich hängt
dies auch mit davon ab, welcher Weg zum Frieden gegangen wird. Von den
beiden oben skizzierten Möglichkeiten scheint in dieser Hinsicht der
Weg über eine Stärkung der UNO der unproblematischere zu sein, da hier
nicht die Gefahr besteht, eben die Friedenschancen, die man gewinnen
möchte, durch eine für die anderen weltpolitischen Mitspieler
möglicherweise bedrohlich erscheinende Politik des Alleingangs zu
verspielen. Ein möglicher Einwand könnte allerdings lauten, dass damit
die Hoffnung auf eine friedliche und sichere Welt auf absehbare Zeit
begraben wird, da die UNO als Friedensstifter bislang oft
wirkungslos geblieben sei.

Insgesamt erweist sich Kants Friedenskonzept aber als hinreichend
realitätsadäquat, damit eine Pflicht zum Frieden bejaht werden kann.
Aber, so könnte man nun noch einwenden, ergibt sich daraus bereits
eine Pflicht zum Frieden? Würde es nicht genügen den Frieden als Ideal
zu postulieren, dass zu verfolgen wünschenswert sei, ohne gleich eine
moralische Pflicht daraus zu machen? Diese Frage führt zurück zu den
ethischen Grundprämissen von Kants politischer Philosophie, sowie auch
auf die weitere, bereits erwähnte Frage, ob es neben der Pflicht, die
natürlichen Rechte anderer zu achten, und die Gesetze, die vom Staat
zum Schutz dieser Rechte erlassen werden, zu befolgen, auch die
weitergehende Pflicht gibt, Institutionen wie beispielsweise einen
Staat zu schaffen, die die natürlichen Rechte schützen, sofern solche
Institutionen nicht vorhanden sind. Ohne diese Fragen -- die eine sehr
viel ausführlichere Behandlung, als sie hier möglich ist, verdienen
würden -- an dieser Stelle näher zu untersuchen, soll doch wenigstens
eine Überlegung angeführt werden, die zeigt das Kants Prämissen
schlüssig sind. Würde man nämlich -- wie dies in der nationalistischen
Staatsphilosophie nach Kant üblich wurde\footnote{Für ein
  einschlägiges Beispiel vgl. Georg Friedrich Wilhelm Hegel:
  Grundlinien der Philosophie des Rechts (Hrsg. von Johannes
  Hofmeister), 4.Aufl., Hamburg 1995, §333-§337, S.285-287. - Man kann
  Hegel nicht ohne Weiteres damit entschuldigen, dass er an dieser
  Stelle nur (gewissermaßen wertfrei) den Ist-Zustand beschreiben
  wollte. Wäre dies der Fall gewesen, so hätte Hegel es nicht nötig
  gehabt, diejenigen, die sich für die Herrschaft des Rechts auch in
  den internationalen Beziehungen einsetzen, als Moralisten zu
  diffamieren.} -- das auch von Kant befürwortete Prinzip der
staatlichen Souveränität soweit ausdehnen, dass es geradezu verbietet,
dass sich der Staat an internationalen Friedensregimen beteiligt, die
seine Souveränität einschränken, dann ließe sich der Widerspruch nicht
vermeiden, dass die Einzelnen als Bürger des Staates zur Gesetzestreue
verpflichtet wären, in ihrer Gesamtheit aber von allen rechtlichen
Bindungen frei blieben, ja als Soldaten unter Umständen sogar dazu
verpflichtet sein könnten, sich für die organisierte Rechtlosigkeit
mit ihrem Leben einzusetzen. Kurz gesagt, wenn die Rechtstreue vom
Einzelnen gefordert werden kann, dann erst Recht vom Staat als Ganzes.
Dass der Staat als Ganzes dabei unter anderen Rahmenbedingungen
agiert, nämlich in einer Welt, die nicht wiederum durch einen
übergeordneten Gesetzgeber geordnet ist, und dass dies zuweilen auch
den Gebrauch kriegerischer Gewalt erfordert und rechtfertigt, steht
auf einem anderen Blatt und widerspricht diesem Prinzip nicht.

\section{Der "`ewige Frieden"' als unvollendete Aufgabe}

Sowohl die moralphilosophischen Prämissen als auch die
Sachvoraussetzungen von Kants "`ewigem Frieden"' erweisen sich also
als nach wie vor aktuell. Dass Kants "`ewiger Frieden"' immer noch
eine unvollendete Aufgabe ist, hängt mit dem Zustand der Weltpolitik
zusammen, der von einem dauerhaft gesicherten Weltfrieden leider auch
heute noch weit entfernt ist. Kants "`ewiger Frieden"' ist aber auch
in dem positiven Sinne eine unvollendete Aufgabe, dass es sich immer
noch lohnt für dieses Ziel zu kämpfen. Die theoretischen Grundlagen
von Kants Friedenskonzept bleiben auch nach 200 Jahren in ihren
Kernpunkten gültig. Anders als andere politische Utopien, die -- oft
erst nach großen Opfern -- schließlich ad acta gelegt wurden, hat
Kants Utopie damit den Test der Zeit bestanden.


\section{Literatur}

{\small

\setlength{\parindent}{0ex} \setlength{\parskip}{1.5ex}

Benda, Julien: La Trahison des clercs, Bernard Grasset, Paris 1977
(zuerst 1927).

{\em Calic, Marie-Janine: Geschichte
  Jugoslawiens im 20. Jahrhundert, Beck Verlag München, 2. Auflage
  2014. (Literaturhinweis ergänzt am 1.11.2015.)}

Cavallar, Georg: Pax Kantiana. Systematisch-historische
Untersuchung des Entwurfs "`Zum ewigen Frieden"' (1795) von Immanuel
Kant, Böhlau Verlag, Wien/Köln/Weimar 1992.

Czempiel, Ernst-Otto: Friedensstrategien. Eine systematische
Darstellung außenpolitischer Theorien von Machiavelli bis Madariaga,
2. Aufl., Westdeutscher Verlag, Opladen / Wiesbaden 1998.
% soz y 300 c 998

Czempiel, Ernst-Otto: Kants Theorem und die zeitgenössische
Theorie der internationalen Beziehungen, in: Matthias Lutz-Bachmann /
James Bohmann (Hrsg.): Frieden durch Recht. Kants Friedensidee und das
Problem einer neuen Weltordnung, Suhrkamp, Frankfurt am Main 1996,
S.300-323.

Dietze, Anita / Dietze, Walter: Ewiger Friede? Dokumente einer
deutschen Diskussion um 1800, C.H. Beck Verlag, München 1989.

Ebeling, Hans: Kants "`Volk von Teufeln"', der Mechanismus der
Natur und die Zukunft des Unfriedens, in: Klaus-Michael Kodalle
(Hrsg.): Der Vernunft-Frieden.  Kants Entwurf im Widerstreit,
Könighausen \& Neumann, Würzburg 1996, S.87-94.

Erasmus von Rotterdam: Querela pacis undique gentium ejectae
profligataeque (Die Klage des Friedens, der von allen Völkern
verstoßen und vernichtet wurde), in: Erasmus von Rotterdam:
Ausgewählte Schriften. Fünfter Band, Hrsg.  von Werner Welzig,
Wissenschaftliche Buchgesellschaft, Darmstadt 1968, S.359-451.

Habermas, Jürgen: Kants Idee des ewigen Friedens - aus dem
historischen Abstand von zweihundert Jahren, in: Matthias
Lutz-Bachmann / James Bohmann (Hrsg.): Frieden durch Recht. Kants
Friedensidee und das Problem einer neuen Weltordnung, Suhrkamp,
Frankfurt am Main 1996, S.7-24.

Habermas, Jürgen: Hat die Konstitutionalisierung des Völkerrechts noch
eine Chance?, in: Jürgen Habermas: Der gespaltene Westen, Suhrkamp,
Frankfurt am Main 2004, S.113-193.

Hackel, Volker Marcus: Kants Friedensschrift und das
Völkerrecht, Duncker \& Humblot, Berlin 2000.

Hegel, Georg Friedrich Wilhelm: Grundlinien der Philosophie des Rechts
(Hrsg. von Johannes Hofmeister), 4.Aufl., Felix Meiner Verlag, Hamburg
1995.

Heintze, Hans Joachim: Gibt es ein Recht auf humanitäre Intervention?
Das Völkerrecht nach dem Kosovo-Krieg, in: Ulrich Albrecht, Michael
Kalmon, Sabine Riedel, Paul Schäfer (Hrsg.): Das Kosovo-Dilemma.
Schwache Staaten und Neue Kriege als Herausforderung des
21.Jahrhunderts, Verlag Westfälisches Dampfboot, Münster 2002,
S.165-181.

Höffe, Ottfried: Die Vereinten Nationen im Lichte Kants, in:
Ottfried Höffe (Hrsg.): Immanuel Kant: Zum ewigen Frieden,
Akademie-Verlag, Berlin 1995, S.245-272.

Ipsen, Knut: Der Kosovo-Einsatz -- llegal?  Gerechtfertigt?
Entschuldbar?, in: Dieter S. Lutz (Hrsg.): Der Kosovo-Krieg.
Rechtliche und rechtsethische Aspekte, Baden-Baden 2000, S.101-105.

Kant, Immanuel: Zum ewigen Frieden. Ein philosophischer Entwurf
(1795), in: Kants Werke. Akademie-Textausgabe. Band VIII, de
Gruyter\&Co, Berlin 1968, S.341-386.

Kant, Immanuel: Über den Gemeinspruch: Das mag in der Theorie
richtig sein, taugt aber nicht für die Praxis (1793), in: Kants Werke.
Akademie-Textausgabe. Band VIII, de Gruyter\&Co, Berlin 1968, S.273-313.

Kant, Immanuel: Idee zu einer Geschichte in weltbürgerlicher
Absicht (1784), in: Kants Werke. Akademie-Textausgabe. Band VIII, de
Gruyter\&Co, Berlin 1968, S.15-31.

Kant, Immanuel: Über ein vermeintliches Recht aus Menschenliebe zu
lügen, in: Kants Werke. Akademie Ausgabe. Band VIII, de Gruyter\&Co,
Berlin 1968, S.423-430.

Kant, Immanuel: Die Metaphysik der Sitten (1797), in: Kants
Werke. Akademie-Textausgabe. Band VI, de Gruyter\&Co, Berlin 1968,
S.203-493.

Kühne, Winrich: Humanitäre NATO-Einsätze ohne Mandat? Ein
Diskussionsbeitrag zur Fortentwicklung der UNO-Charta, in: Dieter S.
Lutz (Hrsg.): Der Kosovo-Krieg. Rechtliche und rechtsethische Aspekte,
Baden-Baden 2000, S.73-99.

Madison, James: Objections to the Proposed Constitution From Extend of
Territory Answered. From the New York Packes. Friday, November 30,
1787, auf: The Avalon Project at Yale Law School. The Federalist
Papers: No. 14, unter: www.yale.edu/lawweb/avalon/federal/fed14.htm
(Zugriff am: 20.10.2004).

Merkel, Reinhard: Das Elend der Beschützten. Der NATO-Angriff ist
illegal und moralisch verwerflich, in: Dieter S. Lutz (Hrsg.): Der
Kosovo-Krieg.  Rechtliche und rechtsethische Aspekte, Baden-Baden
2000, S.227-232.

Münkler, Herfried: Der neue Golfkrieg, Rowohlt Verlag, Hamburg 2003.

Russet, Bruce: Grasping the Democratic Peace. Principles for a
Post-Cold War World, Princeton University Press, Princeton / New Jersy
1993.
% soz y 300 r 968

Singer, J. David / Small, Melvin: The Wages of War 1816-1965. A
Statistical Handbook, John Wiley \& Sons, New York / London / Sidney /
Toronto 1972.

Schurz, Gerhard: The Is-Ought Problem. An Investigation in
Philosophical Logic, Kluwer Academic Publishers,
Dordrecht/Boston/London 1997, S.279-285.

Weber, Hermann: Die NATO-Aktion war unzulässig, in: Dieter S. Lutz
(Hrsg.): Der Kosovo-Krieg. Rechtliche und rechtsethische Aspekte,
Baden-Baden 2000, S.65-71.

Wellhausen, Malte: Humanitäre Intervention. Probleme der Anerkennung
des Rechtsinstituts unter besonderer Berücksichtigung des
Kosovo-Konflikts, Nomos Verlagsgesellschaft, Baden-Baden 2002.

}

\end{document}
