%%% Local Variables:
%%% mode: latex
%%% TeX-master: t
%%% End:

\documentclass[12pt,a4paper,ngerman]{article}
\usepackage{microtype}
\usepackage{ae}
%\usepackage[latin1]{inputenc}
\usepackage[utf8x]{inputenc}
%\usepackage{unicode}
\usepackage[T1]{fontenc}
\usepackage{t1enc}
\usepackage{type1cm}
\usepackage[german,ngerman, english, USenglish]{babel}
\usepackage{graphics}
\usepackage{natbib}

\usepackage{ifpdf}
\ifpdf
\usepackage{xmpincl}
\usepackage[pdftex]{hyperref}
\hypersetup{
    colorlinks,
    citecolor=black,
    filecolor=black,
    linkcolor=black,
    urlcolor=black,
    bookmarksopen=true,     % Gliederung öffnen im AR
    bookmarksnumbered=true, % Kapitel-Nummerierung im Inhaltsverzeichniss anzeigen
    bookmarksopenlevel=1,   % Tiefe der geöffneten Gliederung für den AR
    pdfstartview=FitV,       % Fit, FitH=breite, FitV=hoehe, FitBH
    pdfpagemode=UseOutlines, % FullScreen, UseNone, UseOutlines, UseThumbs 
}
\pdfinfo{
  /Author (Eckhart Arnold)
  /Title (Eric Voegelin als Schueler Hans Kelsens)
  /Subject (Eine intellektuelle Biographie Eric Voegelins mit Fokus auf Voegelins Beziehung zu Hans Kelsen)
  /Keywords (Eric Voegelin, Hans Kelsen, Reine Rechtslehre, Politische Theologie, Autoritaerer Staat)
}
\includexmp{VoegelinKelsen}
\fi

\pagestyle{myheadings}

\sloppy

\begin{document}
\selectlanguage{\ngerman}

\title{Eric Voegelin als Schüler Hans Kelsens}
\author{Eckhart Arnold}
\date{20. März 2007}

%\maketitle

\begin{titlepage}
\begin{center}
\ \\[4cm]{\Huge Eric Voegelin als Schüler Hans Kelsens}\\[2cm]
{\Large Eckhart Arnold}\\[3cm]
{\small erchienen in: Robert Walter, Clemens Jabloner, Klaus Zeleny (Hrsg.): Der Kreis um Hans Kelsen. Die Anfangsjahre der Reinen Rechtslehre, Wien 2007.}
\end{center}

\end{titlepage}

%\begin{abstract}

%{\bf [Zusammenfassung]}

%\end{abstract}

%\fontsize{12}{18}
%\selectfont

\newpage

\pagenumbering{roman}
\tableofcontents

\setcounter{page}{1}
\pagenumbering{arabic}


\section{Einleitung}

Wenn von Eric Voegelin als Schüler Hans Kelsens die Rede ist, dann ist
dies gleich mit einer wichtigen Einschränkung zu versehen. Unter einem
"`Schüler"' stellt man sich gewöhnlich jemanden vor, der vom Denken
des "`Lehrers"' tief geprägt ist, der eventuell auch an dessen Werk
anknüpft, es weiterführt oder auch abwandelt und in einer neuen
Richtung fortbildet. In diesem Sinne ist Eric Voegelin kaum als
Schüler Hans Kelsens zu verstehen. Zwar hat er bei Kelsen
studiert und sich in den 20er und 30er Jahren in mehreren Aufsätzen
und Büchern intensiv mit der Reinen Rechtslehre Kelsens sowie
dessen Staatsphilosophie auseinandergesetzt. Aber Hans Kelsen bleibt
für Voegelin nur ein Einfluss unter zahlreichen anderen, und gerade
als Philosoph hat sich Voegelin schließlich in einer von Kelsen sehr
verschiedenen, ja gegensätzlichen Richtung entwickelt. In Voegelins
entscheidenderen späteren Werken, insbesondere in seinem
geschichtsphilosophischen Hauptwerk "`Ordnung und Geschichte"' ist von
Kelsens Einfluss kaum noch etwas zu spüren. Wenn daher in diesem
Kapitel Voegelin als Schüler Hans Kelsens besprochen wird, dann
bedeutet dies, dass Eric Voegelins Werk thematisiert wird, soweit
Voegelin eben Schüler Hans Kelsens ist, d.h. der Schwerpunkt liegt auf
den Schriften, in denen sich Voegelin mit der Reinen Rechtslehre
Kelsens auseinandersetzt, womit der größere und wichtigere Teil von
Voegelins Werk ausgespart bleibt.

Aber gerade dadurch, dass Voegelin kein typischer "`Schüler"' ist, sondern von
Anfang an einen selbstständigen Standpunkt vertreten hat, wird seine
Auseinandersetzung mit Kelsens Reiner Rechtslehre besonders interessant.
Nicht unbedingt typisch für das Lehrer-Schüler Verhältnis ist auch, dass
Kelsen sich umgekehrt mit Voegelins eigener "`Lehre"' auseinandergesetzt und
sie einer eingehenden, wenn auch sehr ablehnenden Kritik gewürdigt hat. Auch
darauf soll im Folgenden kurz eingegangen werden.

\section{Voegelins Leben und Werk}

Eric Voegelin wird am 3. Januar 1901 in Köln geboren. Seine Familie zieht aber
bereits 1910 nach Wien, wo Voegelin im Jahr 1919 das Studium an der
staatswissenschaftlichen Fakultät der Universität Wien aufnimmt.\footnote{Die
  biographischen Angaben stützen sich überwiegend auf: Michael Henkel: Eric
  Voegelin zur Einführung, Hamburg 1998, S.  13-35, S. 198-199. Weiterhin
  wurden Akten aus dem Archiv der Universität Wien hinzugezogen, die mir das
  Hans Kelsen Institut dankenswerter Weise zur Verfügung gestellt hat.} Der
Kanon von Lehrveranstaltungen, der in erster Linie juristische,
volkswirtschaftliche und soziologische Fächer umfasst, deckt Voegelins
intellektuelle Interessen allein kaum ab.  Nebenher beginnt er sich intensiv
mit moderner Literatur und Philosophie zu beschäftigen.\footnote{Voegelins
  frühe Studie über Frank Wedekind legt Zeugnis davon ab. (Eric Voegelin:
  Wedekind. Ein Beitrag zur Soziologie der Gegenwart, München 1996.) -- Vgl.
  auch Thomas Hollweck, Der Dichter als Führer. Dichtung und Repräsentanz in
  Voegelins frühen Arbeiten, München 1996.} Ohnehin spielte sich ein
wesentlicher Teil des wissenschaftlichen und geistigen Lebens im Wien der
Nachkriegszeit außerhalb der Universität in privaten Seminaren und
Gelehrtenzirkeln ab. Voegelin hatte Zugang zu gleich vier solcher Kreise, dem
Seminar Othmar Spanns, den Privatseminaren Hans Kelsens und Ludwig von Mises'
und dem von Friedrich August von Hayek und Josef Herbert Fürth gegründeten
``Geist-Kreis''.\footnote{Vgl. Johannes Feichtinger: Wissenschaft zwischen den
  Kulturen.  Österreichische Hochschullehrer in der Emigration 1933-1945,
  Frankfurt am Main 2001, S. 35.} Sicherlich ist die geistige Breite von
Voegelins späterem Schaffen auch dem prägenden Einfluss dieser weltanschaulich
sehr unterschiedlich ausgerichteten Diskussionsrunden auf den jungen
Nachwuchswissenschaftler zuzuschreiben. 1922 promoviert Eric Voegelin bei
Othmar Spann {\em und} Hans Kelsen. Daran zeigt sich bereits, dass Voegelin,
der von einem starken Ehrgeiz erfüllt war, seine akademischen Lehrer bewusst
auswählt und sich dabei an die ersten Adressen hält. In den folgenden Jahren
unternimmt Eric Voegelin umfangreiche Studienreisen. Zunächst begibt er sich
nach Berlin, wo er unter anderem den Althistoriker Eduard Meyer hört. Nachdem
er im Herbst 1923 kurzzeitig als wissenschaftliche Hilfskraft an den
Lehrkanzeln für Staats- und Verwaltungsrecht in Wien tätig war,\footnote{Vgl.
  Günther Winkler: Geleitwort, in: Eric Voegelin: Der autoritäre Staat. Ein
  Versuch über das österreichische Staatsproblem (Hrsg.  von Günther Winkler),
  2.  Auflage, Wien / New York 1997 (1936), S. V-XXXII, im folgenden zitiert
  als: Winkler, Geleitwort, S. V.} verbringt Voegelin mit einem Stipendium der
Rockefeller-Stiftung ab 1924 drei Jahre im Ausland, zunächst zwei Jahre in den
USA, dann ein Jahr in Frankreich an der Pariser Sorbonne.  In den USA sind
seine wichtigsten Stationen die Columbia University, die University of
Wisconsin und Harvard. Er studiert dort unter anderem bei John Dewey, Alfred
North Whitehead und John R. Commons, der letztere ein bedeutender Vertreter
der Institutionen- und Arbeitsökonomie.  Später verbringt Voegelin noch ein
Semester in Heidelberg, wo er Karl Jaspers und Alfred Weber hört.

Aus seiner Studienreise in die USA geht Voegelins erstes größeres Werk "`Über
die Form des Amerikanischen Geistes"' hervor, eine inhaltlich breit gestreute,
am ehesten noch als geistesgeschichtlich zu bezeichnende Studie, in der
Voegelin verschiedene angelsächsische und amerikanische Denker behandelt.
%  in der
% Voegelin erstmals seine spätere hermeneutische Standardmethode der Suche nach
% Formverwandtschaften -- später spricht er auch von
% "`Strukturverwandtschaften"' -- erprobt, eine Methode, bei der nach
% Übereinstimmungen und Zusamenhängen nicht so sehr des Inhalts als vielmehr der
% tieferliegenden Vorstellungen und Denkstrukturen zwischen historisch oder vom
% Fachgebiet her zum Teil weit entfernten Denkern gesucht wird. Wie der Titel
% andeutet, ist Voegelins Buch von gewissen zeittypischen Vorstellungen von
% Nationalcharakter und nationalem Denkstil nicht unbeeinflusst. Auch in seinen
% späteren umfangreichen ideengeschichtlichen Studien wird Voegelin mehr oder
% weniger selbstverständlich davon ausgehen, dass man das Wesen bzw. den
% "`Geist"' einer Kultur vollumfänglich durch das Studium ihrer bedeutenden
% Philosophen oder Propheten erfassen kann.
Mit diesem Buch habilitiert sich Voegelin in Wien im Jahr 1929,
zunächst nur für Gesellschaftslehre. Erst ab 1931 wird seine
Lehrbefugnis, wie schon vorher erhofft, auch auf die allgemeine
Staatslehre ausgedehnt.\footnote{Günther Winkler vermutet, dass
  "`vielleicht auch weil Kelsen zweiter Gutachter war"' (Winkler,
  Geleitwort, S. V.) die Lehrbefugnis für Staatslehre zunächst
  verweigert worden war. Die Quellen geben jedoch keinerlei
  Anhaltspunkte, die diese Vermutung stützen könnten. Das von Kelsen
  und Othmar Spann unterschriebene Referat über Voegelins
  Habilitationsgesuch vom 21. Mai 1928, in welchem noch davon
  ausgegangen wird, dass Voegelin sich in erster Linie für allgemeine
  Staatslehre habilitieren möchte, äußert sich im Gegenteil bei sehr
  wenigen Vorbehalten streckenweise geradezu euphorisch über Voegelin.
  Vgl. die Abschrift vom Referat über das Habilitationsgutachten des
  Dr. Erich Voegelin, Archiv der Universität Wien.} 1929 wird Voegelin
erneut Assistent bei Hans Kelsen und ab 1930 -- Kelsen war nach seiner
aus politischen Gründen erzwungenen Entlassung als Verfassungrichter
nach Köln gewechselt -- bei Adolf Merkel.\footnote{Vgl. Winkler,
  Geleitwort, S. V-VI.} Entsprechend seinem Fachgebiet widmet er sich
in diesen Jahren Jahren vorwiegend Fragen der Staatsphilosophie. Er
arbeitet an einer Herrschaftslehre, die er jedoch nicht
veröffentlicht, und publiziert zu juristischen und
staatswissenschaftlichen sowie ideengeschichtlichen Themen. Nebenher
setzt er sich aber auch mit den verschiedensten zeitgenössischen
Geistesströmungen auseinander. So fällt in diese Zeit unter anderem
auch eine intensive Beschäftigung mit den Schriften des
George-Kreises.\footnote{Vgl. Hollweck, Der Dichter als Führer,
  a.a.O., S. 7-9, S. 27ff. Der George-Kreis war eine Künstler- und
  Gelehrtensekte, die Ende des 19. Jahrhunderts um den Dichter Stefan
  George als ihre Zentralfigur entstand, und in den 30er Jahren wieder
  zerfiel. Vgl. Stefan Breuer: Ästethischer Fundamentalismus. Stefan
  George und der deutsche Antimodernismus, Darmstadt 1995.} Als
wichtigste Veröffentlichungen verfasst Voegelin seine beiden
"`Rasse-Bücher"', "`Die Rassenidee in der Geistesgeschichte"' und
"`Rasse und Staat"',\footnote{Eric Voegelin: Rasse und Staat, Tübingen
  1933. -- Eric Voegelin: Die Rassenidee in der Geistesgeschichte von
  Ray bis Carus, Berlin 1933.} die beide 1933 erscheinen, und in denen
er sich ausführlich mit den damals durch den Nationalsozialismus immer
populärer werdenden Rassentheorien und dem Antisemitismus
auseinandersetzt. Aus heutiger Sicht wirkt es etwas irritierend, wenn
Voegelin von seiner Kritik die Rassentheorien von Ludwig-Ferdinand
Clauß und von Othmar Spann ausnimmt, weil sie den Rassenbegriff nicht
nur physisch sondern auch geistig, etwa als "`Seelenartung"' (Clauß),
bestimmten.\footnote{Vgl.  Voegelin, Rasse und Staat, a.a.O., S.
  92ff.} Andererseits konnte damals auch ein denkbar unverfänglicher
Rezensent wie Helmut Plessner Voegelins Buch über "`Rasse und Staat"'
hochschätzen, gerade weil Voegelin den Rassentheorien durch eine
gründliche wissenschaftliche Kritik den Boden entzog, anstatt sich auf
moralische Empörung zu beschränken.\footnote{Vgl.  Helmut Plessner:
  Rechtsphilosophie und Gesellschaftslehre, Besprechung von Rasse und
  Staat von Eric Voegelin, Tübingen 1933, in: Zeitschrift für
  Öffentliches Recht, XIV, 1934, S. 407-414. -- Andererseits wurde das
  Werk von nationalsozialistisch eingestellten Rezensenten gar nicht
  unbedingt als Angriff auf ihre Ideologie verstanden, sondern
  höchstens wegen seiner "`Glaubenslosigkeit"' kritisiert. (Vgl.
  Hans-Jörg Sigwart: Das Politische und die Wissenschaft.
  Intellektuell-biographische Studien zum Frühwerk Eric Voegelins,
  Würzburg 2005, S. 225-228.) Die Tatsache, dass Voegelin auf eine
  Anfrage des völkischen Philosophen Ernst Krieck bezüglich des Werkes
  "`Rasse und Staat"' mit einem Schreiben antwortet, das Sigwart als
  "`Initiativ-Bewerbung"' beurteilt (ebd., S.227, Fußnote 162), legt
  die Vermutung nahe, dass auch Voegelin selbst dieses Werk damals
  nicht als einen Affront gegen den Nationalsozialismus verstanden
  wissen wollte.}

Voegelins äußere Karriere geht dabei vorerst noch den erwarteten Gang, auch
wenn er auf eine feste Universitätsanstellung noch warten muss.
Immerhin wird Voegelin 1935 vom Bundespräsidenten der Titel eines
außerordentlichen Professors verliehen. Ein Jahr später veröffentlicht er
sein Buch "`Der Autoritäre Staat"', worin er die Verfassung des autoritären
Österreich nach dem Dollfuß-Putsch beschreibt.  Daneben werden in diesem Werk
aber auch die philosophischen Grundlagen moderner Staaten eruiert, und
Voegelin nutzt die Gelegenheit zu einer umfassenden und sehr kritischen
Abrechnung mit der Reinen Rechtslehre und der ihr entsprechenden liberalen
Staatsauffassung.\footnote{Eric Voegelin: Der Autoritäre Staat.  Ein Versuch
  über das österreichische Staatsproblem, Wien 1936, im folgenden zitiert als:
  Voegelin, Autoritärer Staat.} In seiner Autobiographie hat Eric Voegelin
sehr viel später behauptet, dass der "`Autoritäre Staat"' ein Versuch gewesen
sei nachzuweisen, "`dass ein autoritärer Staat, der radikale Ideologien in
Schach zu halten vermag, noch die beste Möglichkeit zur Verteidigung der
Demokratie darstellt"'.\footnote{Eric Voegelin: The collected Works of Eric
  Voegelin, Volume 34.  Autobiographical Reflections. Revised Edition.  (Ed.
  by Ellis Sandoz), Columbia and London 2006, im folgenden zitiert als:
  Voegelin, Autobiographical Reflections, S. 69.} Diese Behauptung, die in der
Voegelin-Sekundärliteratur gelegentlich kolportiert wird,\footnote{Vgl.  Claus
  Heimes: Antipositivistische Staatslehre. Eric Voegelin und Carl Schmitt
  zwischen Wissenschaft und Ideologie, München 2004, S. 43. -- Vgl.  Andreas
  Krasemann: Eric Voegelins politiktheoretisches Denken in den Frühschriften,
  Erfurt 2002, auf:
  http://www.db-thueringen.de/servlets/DerivateServelets/Derivate-1408/krasemann.html
  (Zugriff: 8.5.2006), S. 110 (krasemann-ch3.html\#Seite110).  -- Richtig wird
  der Sachverhalt dagegen von Michael Henkel beurteilt.  Vgl. Michael Henkel:
  Positivismuskritik und autoritärer Staat. Die Grundlagendebatte in der
  Weimarer Staatslehre und Eric Voegelins Weg zu einer neuen Wissenschaft der
  Politik (bis 1938), München 2005, S.  62/63.} ist nicht bloß wegen des
offensichtlichen inneren Widerspruchs kaum glaubwürdig.  Vielmehr lässt
Voegelin im "`Autoritären Staat"' wenig Zweifel daran, dass er die autoritäre
oder totalitäre Staatsform für moderne Staaten für die weitaus angemessenste
hält, während er die liberale Demokratie und den pluralistischen Parteienstaat
lediglich als das Symptom einer chaotischen Übergangszeit erachtet. Dabei
knüpft er sehr stark an das totalitäre Staatsdenken von Autoren wie Carl
Schmitt, Ernst Rudolf Huber, Ernst Jünger und Moussolini an, die im
"`Autoritären Staat"' eine auffällig positive Rezeption
erfahren.\footnote{Vgl. Voegelin, Autoritärer Staat, S. 7-54.}

In den Sog rechter Ideologien ist Voegelin etwa ab 1929
geraten.\footnote{Der Einfluss rechter Ideologien zeigt sich erstmals
  deutlich in seinem Kommentar zur österreichischen Verfassungsreform
  von 1929. Vgl. Eric Voegelin: The Austrian Constitutional Reform of
  1929, in; The collected Works of Eric Voegelin. Volume 8. Published
  Essays 1929-1933. (Ed. Thomas W. Heilke and John von Heyking),
  Columbia and London 2003, S. 148-179.} Den Nationalsozialismus hat
er jedoch von Anfang abgelehnt, wenn auch zunächst nur sehr
behutsam.\footnote{Voegelins Einstellung zum Nationalsozialismus {\em
    vor} 1938 ist bisher noch nicht gründlich erforscht worden.
  Hans-Jörg Sigwart erwähnt in einer Fußnote, dass die Korrespondenz
  aus Voegelins Nachlass zeige, dass Voegelin sich noch Ende 1933
  relativ intensiv um berufliche Kontakte nach Deutschland bemüht hat.
  (Vgl. Hans-Jörg Sigwart: Das Politische und die Wissenschaft.
  Intellektuell-biographische Studien zum Frühwerk Eric Voegelins,
  Würzburg 2005, S. 227-228, Fußnote 162.) In dem zuvor schon
  erwähnten Schreiben Voegelins an Ernst Krieck werden von Voegelin
  laut Sigwart "`seine Familienverhältnisse als `einwandfrei'
  bezeichnet `im Sinne der Ansprüche, die heute an deutsche Abkunft
  gestellt werden' "'. Sigwart führt weiterhin aus, dass Voegelin
  betont, "`dass er in keiner besonders engen `politischen und
  wissenschaftlichen Beziehung' zu Hans Kelsen stünde ... und damit
  eine `Belastung' anspricht, die ihn `schon in mancherlei Weise
  geschädigt' habe."'  (ebd.)} Und bis zur Veröffentlichung der
"`Politischen Religionen"' im Jahr 1938 deutet in seinen
Schriften nur wenig daraufhin, dass er ihn auch als Gefahr
hinreichend ernst genommen hätte.\footnote{Vgl.  Sigwart, a.a.O., S.
  227/228. -- In seiner Autobiographie erwähnt Voegelin sogar, dass er
  nach dem Anschluss in einem Zustand ohnmächtiger Wut kurzzeitig mit
  dem Gedanken gespielt hat, der nationalsozialistischen Partei
  beizutreten. (Vgl.  Voegelin, Autobiographical Reflections, S. 70.)}
Von den politischen Umwälzungen wurde Voegelin dann überrollt. Schon
wenige Wochen nach dem Anschluss Österreichs an das Deutsche Reich am
12.  März 1938 wurde zahlreichen Dozenten der Universität Wien,
darunter Eric Voegelin, die Lehrbefugnis entzogen.\footnote{Voegelin
  selbst vermutet in einem Brief an Walter Gurian als Gründe für seine
  Entlassung seine Beziehungen zum Schuschnigg-Regime sowie sein
  Engagement in der Vaterländischen Front. Eine entsprechende Passage
  des Briefes vom 26. November 1938 wird von Gerhard Wagner und
  Gilbert Weiss zitiert in: Alfred Schütz / Eric Voegelin: Eine
  Freundschaft, die ein Leben gehalten hat.  Briefwechsel 1938-1959.
  (Hrsg. von Gerhard Wagner und Gilbert Weiss), Konstanz 2004, S. 9.}
Als die Gestapo seines Passes habhaft zu werden versucht, flieht Eric
Voegelin aus Österreich und emigriert über die Schweiz in die
Vereinigten Staaten von Amerika. 

Unmittelbar vor dem Anschluss entsteht Voegelins Schrift über die
"`Politischen Religionen"'.\footnote{Eric Voegelin: Die politischen
  Religionen, München 1993 (1938).} In diesem Werk legt Voegelin seine
einflussreiche, wenn auch nicht unbedingt originelle Charakterisierung der
totalitären Bewegungen als politische Religionen vor. Von sehr philosophischem
Charakter und mit starken polemischen Akzenten markiert die Schrift zugleich
eine deutliche Stilwende, die Voegelins weiteres Schaffen bestimmen wird. Auch
hinsichtlich der Anknüpfung an die christliche Modernitätskritik leitet sie
eine neue Schaffensphase Voegelins ein, in der der Bezug auf die Religiosität
des Menschen eine immer größere Bedeutung gewinnt.

Im Gegensatz zu vielen anderen Emigranten hatte Voegelin den Vorteil, dass er
durch seine frühere Studienreise mit den Verhältnissen in Amerika bereits
vertraut war. Dennoch dauert es eine Weile, bis er nach Zwischenstationen an
der Harvard University, am Bennington College in Vermont und an der University
of Alabama schließlich mit einer Professur für Political Science an der
Lousiana State University in Baton Rouge Fuß fassen kann.\footnote{Vgl.
  Johannes Feichtinger: Wissenschaft zwischen den Kulturen, a.a.O., S.
  335-338.} Im Laufe der Zeit wird Amerika für Voegelin jedoch zu seiner
Wahlheimat. 1944 nimmt er die amerikanische Staatsbürgerschaft an, und er
verwendet für seinen Vornamen fortan die amerikanische Schreibweise. In den
40er Jahren entsteht auch die mehrbändige "`History of Political
Ideas"',\footnote{Eric Voegelin: History of Political Ideas, in: Paul
  Caringella et al. (Ed.): The collected works of Eric Voegelin, Volumes
  19-26, Baton Rouge, 1997.} die Voegelin jedoch unveröffentlicht lässt, da er
mit der üblichen Art Ideengeschichte zu schreiben, an die er sich in diesem
Werk noch anschließt, unzufrieden ist.  Erst mit der 1952 erschienen "`New
Science of Politics"'\footnote{Eric Voegelin: The New Science of Politics. An
  Introduction, Chicago and London, 1987.} gelingt ihm der (sich schon in den
"`Politischen Religionen"' ankündigende) Durchbruch zu einer "`Neuen"'
Politischen Wissenschaft, der zufolge politische Ordnung stets auf den
religiösen Erfahrungen einer Gesellschaft beruht, und politische Unordnung
durch Störungen der religiösen Erfahrung zu erklären ist.  Diese Störungen der
religiösen Erfahrungen bezeichnet Voegelin nun nicht mehr als "`Politische
Religionen"' sondern als "`gnostisch"',\footnote{"`Gnosis"' meint dabei einen
  unter anderem in den Sektenbewegungen des Frühchristentums stark vertretenen
  Religionstypus, der durch eine radikale und umfassende Weltablehnung in
  Verbindung mit oft sehr ausgeprägten Erlösungshoffnungen bestimmt ist. Ein
  -- im Gegensatz zu Voegelins "`New Science of Politics"' -- wissenschaftlich
  ernst zu nehmender Versuch, die gnostischen Züge einiger moderner
  Geistesströmungen herauszuarbeiten, findet sich in: Micha Brumlik, Die
  Gnostiker. Der Traum von der Selbsterlösung des Menschen, Frankfurt am Main
  1992.} und er lässt wenig Zweifel daran, dass er beinahe alle modernen
Geistesströmungen in diesem Sinne für gnostisch hält.\footnote{Vgl. Voegelin,
  New Science of Politics, S.  162ff.} Auf dieser Einschätzung beruht auch
Voegelins berühmt-berüchtigte Charakterisierung der Neuzeit als eines
gnostischen Zeitalters.\footnote{Vgl.  Hans Blumenberg: Die Legitimität der
  Neuzeit. Erneuerte Ausgabe, Frankfurt am Main 1996, S. 138.}

Mit ihrer Betonung der religiösen Erfahrung als Voraussetzung der politischen
Ordnung legt die "`New Science of Politics"' darüber hinaus die theoretischen
Grundlagen zu Voegelins voluminösen Hauptwerk "`Order and History"', dessen
erste drei Bände ("`Israel and Revelation"', "`The World of the Polis"' und
"`Plato and Aristotle"') in den Jahren 1956 und 1957 erscheinen.\footnote{Eric
  Voegelin: Order and History. Volume One. Israel and Revelation, Baton Rouge
  / London, reprint 1981 (1956). -Volume Two. The World of The Polis, reprint
  1980 (1957). -Volume Three.  Plato and Aristotle, reprint 1983 (1957).} In
diesen ersten drei Bänden wird die Menschheitsgeschichte als eine Geschichte
des religiösen Fortschritts gedeutet, der sich in zeitlich weit
auseinanderliegenden Sprüngen vollzieht, hervorgerufen durch die
Transzendenzerfahrungen einzelner bedeutender Philosophen und Propheten. Diese
Transzendenzerfahrungen werden -- dadurch ist die Richtung des Fortschritts
bestimmt -- mit der Zeit immer "`differenzierter"', wenn auch die Trennung
zwischen Transzendenz und Immanenz nie aufgehoben werden kann. Den
differenzierter werdenden Transzendenzerfahrungen entsprechen dabei sich
steigernde Niveaus politischer Ordnung.
% Allerdings beschränkt sich Veogelin in seinem
% geschichtsphilosophischen Hauptwerk weitgehend auf die inhaltliche
% Zusammenfassung und Deutung ihm wichtig erscheinender relgiöser und
% politischer Schriften, so dass der Zusammenhang zwischen religiösen
% Erfahrungen und der politischen Ordnung höchstens am Rande berührt wird.
% Problematisch ist auch, dass Voegelin die Kategorien religiöser Wahrheit und
% Unwahrheit für wissenschaftlich objektivierbar hält. So verkörpert für ihn das
% Christentum objektiv eine höhere Wahrheit als das Judentum und das Judentum
% wiederum eine höhrere Wahrheit als die Religion des alten
% Ägypten.\footnote{Vgl. Eric Voegelin: Ordnung und Geschichte 4, Die Welt der
%   Polis Gesellschaft, Mythos und Geschichte, München 2002, S. 41/42. Voegelin
%   schreibt an dieser Stelle: "`Was an Nathans Weisheit gültig ist ..., ist der
%   Respekt vor jeglicher Ordnung und vor jeglicher Wahrheit über die Ordnung;
%   denn jede Gesellschaft ... ringt um die Einstimmung in die Seinsordnung.
%   Dieser Respekt darf jedoch nicht zu einer Toleranz verkommen, die die
%   Unterschiede im Rang sowohl hinsichtlich der Suche nach Wahrheit als auch im
%   Hinblick auf die erlangte Einsicht unbeachtet lässt."' Voegelin spricht hier
%   von religiöser Wahrheit und er glaubt offenbar sie bewerten zu können. In
%   diesem Sinne tadelt einige Zeilen zuvor Arnold Toynbees vergleichsweise
%   tolerantere Haltung mit den Worten: "`Dieselbe Unentschiedenheit
%   hinsichtlich der theoretischen Streitfragen kennzeichnet die Haltung
%   Toynbees. Wenn er die von ihm vertretene Gleichrangigkeit der vier
%   `Hochreligionen' verteidigt, flüchtet er sich in das generelle menschliche
%   Unvermögen, in geistigen Angelegenheiten Wahrheit zu erkennen. Dies ist eine
%   einnehmende Demut -- leider läßt sich intellektuelle Demut zuweilen nur
%   schwer von intellektueller Drückebergerei unterscheiden."'}

Im Jahr 1958 tritt ein erneuter, wenn auch weniger folgenschwerer Wandel in
Voegelins Leben ein. Er kehrt zurück nach Europa und wird Professor für
Politikwissenschaft an der Ludwig-Maximilians-Universität München, wo er das
Institut für Politische Wissenschaft aufbaut.  Zeitweise bringt es Voegelin zu
einer gewissen Berühmtheit, wozu beigetragen haben mag, dass er die
nationalsozialistische Vergangenheit Deutschlands im Hörsaal sehr
kompromisslos zum Thema macht.\footnote{Vgl. dazu die Vorlesungsreihe über
  "`Hitler und die Deutschen"'.  Eric Voegelin: The collected Works of Eric
  Voegelin. Volume 31.  Hitler and the Germans (translated and edited by
  Detlev Clemens and Brendan Purcell), Columbia and London 1999.} Als er 1969
emeritiert wird, gilt er jedoch schon fast als vergessen. Sein wichtigstes
Werk aus dieser Zeit ist die 1966 erschienene Sammlung
bewusstseinsphilosophischer Aufsätze unter dem Titel
"`Anamnesis"'.\footnote{Eric Voegelin: Anamnesis. Zur Theorie der Geschichte
  und Politik München 1966.} Die jüngeren Kapitel dieses Werkes zeigen
Voegelin auf dem Weg zu einer immer mehr um religiöse Transzendenzerfahrungen
kreisenden Philosophie.\footnote{Vgl.  Eric Voegelin: Was ist politische
  Realität?, in: Voegelin, Anamnesis, a.a.O., S. 283-354.}

Nach seiner Emeritierung kehrt Voegelin wieder nach Amerika zurück. Dort
bleibt er weiterhin wissenschaftlich und beratend an der Hoover Institution on
War, Revolution and Peace in Stanford tätig. Der vierte und vorletzte Band
seines Hauptwerkes "`Order and History"' mit dem Titel "`The Ecumenic
Age"'\footnote{Eric Voegelin: Order and History.  Volume Four. The Ecumenic
  Age, Baton Rouge / London, reprint 1980 (1974).} erscheint 1974. Voegelin
gibt darin die Vorstellung eines linearen Geschichtsverlaufs, die noch für die
ersten drei Bände von "`Order and History"' bestimmend gewesen ist, endgültig
auf. An der Vorstellung, dass religiöse Transzendenzerlebnisse das Agens der
Menschheitsgeschichte sind, hält er jedoch weiterhin fest. Der fünfte und
letzte Band seines Hauptwerkes ("`In Search of Order"'\footnote{Eric Voegelin:
  Order and History V. In Search of Order, Baton Rouge / London 1986.}) wird
erst posthum erscheinen.  Voegelin stirbt am 19. Januar 1985.



\section{Die Reine Rechtslehre in Voegelins frühen Schriften}

Mit der Reinen Rechtslehre, die er intensiv studiert hat, setzt sich
Voegelin bis zu seiner Emigration 1938 in vielen seiner Schriften
auseinander. Von Anfang an wahrt er dabei Distanz zur
"`Schule"'. In den frühen Schriften (bis 1935) ist seine Haltung zur
Reinen Rechtslehre dabei stets eine sympathisch-kritische. Die Reine
Rechtslehre bildet für Voegelin immer wieder den Ausgangs- und
Bezugspunkt in der Diskussion um eine allgemeine Staatslehre, die aber
-- das war Voegelins Anspruch -- weit über eine bloße Rechtslehre
hinausgehen müsste. Dabei ist ihm natürlich die von Kelsen postulierte
Identität von Staat und Recht von Anfang an ein Dorn im Auge. Aber
erst im "`Autoritären Staat"' (1936) formuliert Voegelin eine
umfassende Kritik der Reinen Rechtslehre. Mit dieser Fundamentalkritik
schließt Voegelin seine wissenschaftliche Beschäftigung mit der
Reinen Rechtslehre ab.

Die erste und bis zur Veröffentlichung des "`Autoritären Staates"' zugleich
ausführlichste Auseinandersetzung mit der Reinen Rechtslehre liefert Voegelin
in seinem Aufsatz {\em Reine Rechtslehre und Staatslehre} (1924)\footnote{Eric
  Voegelin: Reine Rechtslehre und Staatslehre, in: Zeitschrift für
  öffentliches Recht, IV, Wien 1925, S. 80-131.}. Bereits in dieser frühen
Veröffentlichung liegt der Standpunkt Voegelins zur Reinen Rechtslehre in
allen wesentlichen Zügen fest.\footnote{Vgl. Dietmar Herz: Das Ideal einer
  objektiven Wissenschaft von Recht und Staat. Zur Kritik Eric Voegelins an
  Hans Kelsen, S. 30. -- Wenn Herz schreibt: "`Die später von Voegelin
  vorgebrachte Kritik an Kelsens Theorie bringt gegenüber dem Aufsatz von 1924
  nur wenig Neues."', so stimmt das allerdings höchstens mit Ausnahme des
  "`Autoritären Staates"'.}  Voegelin setzt sich darin mit den philosophischen
Begründungsproblemen der Reinen Rechtslehre auseinander. Vor allem dient ihm
dieser Aufsatz aber zur Bestimmung des Verhältnisses von Rechts- und
Staatslehre. Für die Reine Rechtslehre legt er dabei vor allem Kelsens
"`Hauptprobleme der Staatsrechtslehre"'\footnote{Hans Kelsen: Hauptprobleme
  der Staatsrechtslehre.  Entwickelt aus der Lehre vom Rechtssatze.  Nachdruck
  der 2.Auflage von 1923, Aalen 1960.} unter Einbeziehung der jüngeren
Diskussion Anfang der 20er Jahre zu Grunde.

Das zentrale Begründungsproblem der Reinen Rechtslehre stellt nach
Voegelin das Problem der Rechtsdefinition dar, worunter er die Frage
versteht, wodurch sich der positive "`Gesetzesinhalt als Recht
qualifiziert"'.\footnote{Voegelin, Reine Rechtslehre und Staatslehre,
  a.a.O., S. 81.} Nach Voegelin ist es nun nicht möglich das Recht von
anderen Normensystemen (wie der Moral) allein durch eine immanente
Untersuchung der Normen und ihrer Beziehungen untereinander zu
unterscheiden. Ohne das Kriterium der Zwangsbewehrtheit, das in
Kelsens "`Hauptproblemen der Staatsrechtslehre"' noch nicht zur
Rechtsdefinition herangezogen wird\footnote{Vgl. Kelsen,
  Hauptprobleme der Staatsrechtslehre, a.a.O., S. 217. -- Vgl. dagegen
  die Einleitung zur zweiten Auflage, ebd., S. X-XII.} und zudem, wie
Voegelin anmerkt, nicht in allen Rechtsgebieten gleichermaßen
plausibel erscheint, bietet die Reine Rechtslehre nach Voegelins
Darstellung als einzige Lösung an, diejenigen Normen als Recht zu
qualifizieren, die von einem Staatsorgan oder auf Grund der
Ermächtigung durch ein Staatsorgan erlassen worden sind. Dadurch
verstrickt sich die Reine Rechtslehre jedoch in einen Zirkelschluss,
weil nach der Reinen Rechtslehre ja nur der Inhalt der Rechtsordnung
über die Organqualität entscheiden kann.\footnote{Vgl. Voegelin, Reine
  Rechtslehre und Staatslehre, S. 85.} Auch nach der ausführlichen
Diskussion der Beiträge von Sander, Schreier und Kaufmann, bleibt
Voegelin bei der Schlussfolgerung, dass das Problem der
Rechtsdefinition innerhalb der Rechtslehre nicht gelöst werden kann.
Dazu bedarf die Rechtslehre vielmehr der Einbettung in eine von der
Rechtslehre unabhängige empirische Staatslehre.

Um nun die Möglichkeit einer solchen Staatslehre auszuloten, nachdem
Kelsens Reine Rechtslehre die schon in den Staatsrechtslehren Georg
Jellineks, Paul Labands und Carl Friedrich Gerbers angelegte
Entwicklung der Reduktion der Staatslehre auf die Normlogik zu ihrem
logischen Abschluss gebracht hat, greift Voegelin schließlich auf die
sehr viel älteren historischen Staatstheorien von Friedrich Christoph
Dahlmann und Georg Waitz zurück. Diese etwas komplizierte und
historisch recht weit ausholende Operation\footnote{Herz beurteilt
  Voegelins Rückgriff auf den "`organischen Liberalismus"' des 19.
  Jahrhunderts recht treffend als einen "`gewagten Anachronismus"'
  (Herz, a.a.O., S.  64.).} ist wahrscheinlich dadurch bedingt, dass
Voegelin Kelsens Polemik gegen bestimmte "`soziologische"'
Staatsbegriffe, insofern sie zeitgenössische Theorien betrifft,
durchaus zustimmt.\footnote{Vgl.  Voegelin, Reine Rechtslehre und
  Staatslehre, S. 129. -- Vgl. Kelsen, juristischer und soziologischer
  Staatsbegriff, S. 45ff. (§ 7-11), S.  140 (§ 23).} Bei Dahlmann und
Waitz findet Voegelin einen Staatsbegriff, der sich vorwiegend auf
Symbole (das Symbol des Königs, des Volkes etc.) stützt, und dadurch
nach Voegelins Einschätzung weniger angreifbar für Kelsens Argumente
gegen soziologische Staatsbegriffe ist.

Als Gesamtergebnis deutet sich so in Voegelins Aufsatz ein eigenes
staatswissenschaftliches Forschungsprogramm an: Da die Reine
Rechtslehre das Problem der Rechtsdefinition nicht lösen kann, muss
sie eingebettet werden in eine übergreifende Staatslehre, die im
wesentlichen eine Theorie der zentralen politischen Symbole ist.

Ohne wesentliche Änderungen taucht dieselbe Deutung der Reinen Rechtslehre mit
Bezug auf die gleichen Autoren wie in seinem früheren Aufsatz auch in dem
Beitrag {\em Zur Lehre von den Staatsformen}\footnote{Eric Voegelin: Zur Lehre
  von den Staatsformen, in: Zeitschrift für öffentliches Recht VI, Wien 1927,
  S. 572-608.} (1927) auf. Sie wird dort jedoch nur auf wenigen Seiten
berührt.\footnote{Vgl. Voegelin, a.a.O., S. 589-594.} In erster Linie setzt
sich Voegelin hier mit der "`Staatslehre"' Jellineks auseinander.

In dem Aufsatz {\em Kelsen's Pure Theory of Law}\footnote{Eric
  Voegelin: Kelsens Pure Theory of Law, in: The collected Works of
  Eric Voegelin. Volume 7. Published Essays 1929-1928. (Ed.  Thomas W.
  Heilke and John von Heyking), Columbia and London 2003, S.  182-192,
  zuerst in: Political Quarterly 42, no. 2 (1927), S.  268-76.} (1927)
erläutert Voegelin die Reine Rechtslehre in der Form, die Kelsen ihr
in seiner "`Allgemeinen Staatslehre"' von 1925 gegeben hat, für ein
amerikanisches Publikum.  Da Voegelin bei seinen amerikanischen Lesern
keinerlei Kenntnis der Materie voraussetzt, hat seine Darstellung eher
mitteilenden und beschreibenden als kommentierenden Charakter.
Voegelin erläutert darin in knapper Form die für die Reine Rechtslehre
grundlegende Unterscheidung zwischen Sein und Sollen, den Rechtssatz,
die Parallelität von Kausalität und Zurechnung, sowie die Frage der
Verfassungskontinuität und des Verfassungswandels.\footnote{Vgl.
  Voegelin, Kelsen's Pure Theory of Law, a.a.O., S. 184-188}
Bemerkenswert ist Voegelins abschließendes Urteil: Die Reine
Rechtslehre bedeutet für ihn nicht nur einen wichtigen Fortschritt der
juristischen Technik, sondern auch eine Entwicklung von der
halb-absolutistischen Philosophie des deutschen Reiches hin zum Geist
der neuen Demokratie.\footnote{Vgl. Voegelin, Kelsen's Pure Theory of
  Law, a.a.O., S. 191.}

In {\em Die Souveränitätstheorie Dickinsons und die Reine
  Rechtslehre}\footnote{Eric Voegelin: Die Souveränitätstheorie Dickinsons und
  die Reine Rechtslehre, in: Zeitschrift für öffentliches Recht VIII,
  Frankfurt 1969 (Wien, Berlin 1929), S. 413-434.} (1929) vergleicht Voegelin
die Reine Rechtslehre mit der geistig verwandten aber weniger philosophisch
konzipierten Souveränitätstheorie Dickinsons. Im Unterschied zur Reinen
Rechtslehre liegt bei Dickinson der Akzent nicht auf den Normen sondern auf
den Institutionen. Beiden Theorien gemeinsam ist jedoch der juristische
Blickwinkel. Daraus ergeben sich trotz der unterschiedlichen philosophischen
Fundierung erstaunlich weitreichende Übereinstimmungen. Offenbar, so folgert
Voegelin, motiviert die juristische Sichtweise eine ganz bestimmte und daher
übereinstimmende Auffassung des Gegenstandes.

So gehen beide Theorien von der Einheit einer hierarchisch
strukturierten Rechtsordnung aus, nur dass sich für die Reine
Rechtslehre die Einheit aus einem philosophischen Prinzip, der Einheit
der Erkenntniskonstitution ergibt, während sie bei Dickinson eher
pragmatisch begründet wird. An die Stelle der Grundnorm tritt bei
Dickinson entsprechend seines institutionalistischen Ansatzes die
Souveränität, verstanden als die oberste Recht setztende Instanz.
Geradezu frappierend erscheint nach Voegelins Darstellung die
Übereinstimmung beider Theorien hinsichtlich der doppelten Abgrenzung
der rein rechtlich normativen Betrachtungsweise gegen das Verfahren
kausalwissenschaftlicher Untersuchung einerseits und gegen ethische
oder politische Bewertungen andererseits.\footnote{Vgl.  Voegelin,
  Souveränitätstheorie Dickinsons, S. 420-421.} Beide grenzen das
positive Recht strikt von Gerechtigkeit im ethischen Sinne ab, und so
wie Kelsen Recht und Staat miteinander identifiziert, warnt auch
Dickinson vor Hypostasierungen des Staates, bei denen dem Staat und
seinen Institutionen ein eigenes Wesen unterstellt wird jenseits und
über die Rechtsordnung hinaus.\footnote{Vgl.  Voegelin,
  Souveränitätstheorie Dickinsons, S. 429-431.}

Deutliche Unterschiede beider Ansätze werden erst auf dem Gebiet des
Völkerrechts sichtbar. Voegelin zufolge tendieren Kelsen und die Reine
Rechtslehre dazu, aus logischen und systematischen Gründen die Existenz einer
völkerrechtlichen Ordnung zu unterstellen, obwohl ihr faktisch nur eine sehr
begrenzte Wirksamkeit zukommt. Umgekehrt ist Dickinson für Voegelin dazu
genötigt, die in Form eines wirksamen Völkergewohnheitsrechts tatsächlich
vorhandenen Elemente des Völkerrechts zu bagatellisieren, da es oberhalb der
souveränen Staaten keine Institution mehr gibt, die Völkerrecht setzen und
Verstöße ahnden könnte.\footnote{Vgl. Voegelin, Souveränitätstheorie
  Dickinsons, S. 422ff.} Insgesamt ist Voegelins Vergleich zwischen Dickinsons
Souveränitätstheorie und der Reinen Rechtslehre überaus erhellend, da er --
wie Voegelin eingangs auch andeutet -- zeigt, dass das rechtspositivistische
Paradigma als solches nicht zwingend an eine bestimmte Philosophie
(Neukantianismus) gebunden ist oder von einer spezifischen
Wissenschaftstradition (z.B. der deutschen Staatsrechtslehre) abhängt.
Allerdings zieht Voegelin diese mögliche Schlussfolgerung selbst nicht in
voller Schärfe.

Als Fortführung der Gedankengänge seines früheren Aufsatzes über "`Reine
Rechtslehre und Staatslehre"' kann in vielerlei Hinsicht Voegelins {\em Die
  Einheit des Rechts und das soziale Sinngebilde Staat}\footnote{Eric
  Voegelin: Die Einheit des Rechtes und das soziale Sinngebilde Staat, in:
  Revue Internationale de la Théorie du Droit 5, 1930, S.  58-89.} (1930)
verstanden werden. Ebenso wie in seinem früheren Aufsatz geht Voegelin von
einer internen Kritik der Reinen Rechtslehre über zu Fragen einer allgemeinen
Staatswissenschaft. Zum weit überwiegenden Teil ist der Aufsatz jedoch der
Kritik der Reinen Rechtslehre, insbesondere dem Problem der Einheit der
Rechtsordnung und dem Gültigkeitsbegriff gewidmet gewidmet. Den Ausgangspunkt
bildet zunächst eine knappe Darstellung der Rechtstheorie
Bierlings,\footnote{Voegelin stützt auf Bierlings Werk "`Zur Kritik der
  juristischen Grundbegriffe"' (I. Bd., 1877), sowie dessen
"`Juristische
  Prinzipienlehre"', Band I, 1894.} einer Vorläufertheorie der Reinen
Rechtslehre, die Voegelin nutzt, um die beiden Problematiken der Einheit und
der Gültigkeit der Rechtsordnung zu entwickeln.\footnote{Vgl. Voegelin, Die
  Einheit des Rechts und das soziale Sinngebilde Staat, a.a.O., S. 58-61.} Die
Einheit der Rechtsordnung wird bei Bierling wie später in der Reinen
Rechtslehre durch den Delegationszusammenhang hergestellt. Nach Voegelin
liefert die Reine Rechtslehre (ebenso wie Bierling) jedoch nur ein sehr
vereinfachtes Bild der Beziehung zwischen den Rechtsakten bzw. zwischen Normen
und Rechtsakten, weil "`in der Regel eine Beziehung zwischen zwei
materiell-rechtlichen Akten herausgegriffen und als das allgemeingültige
Muster der Beziehung zwischen Rechtsakten überhaupt hingestellt
wird."'\footnote{Eric Voegelin, Die Einheit des Rechts und das soziale
  Sinngebilde Staat, a.a.O., S. 65.} Zudem ist durch das allzu simple Bild
eines hierarchischen Stufenbaus der in Wirklichkeit sehr viel kompliziertere
Delegationszusammenhang gar nicht angemessen zu erfassen. Die Gültigkeit eines
Rechtsaktes hängt nicht in jedem Fall nur von einer Ermächtigungsnorm
höherer
Stufe ab, also etwa die Gültigkeit eines Gesetzes von der Ermächtigung des
Parlaments zur Gesetzgebung durch die Verfassung, sondern unter Umständen von
einem komplizierten Verfahren, dass in seinen Einzelheiten durch Vorschriften
geregelt ist, die in der Normenhierarchie gar nicht unbedingt oberhalb der
durch den Rechtsakt gesetzten Norm angesiedelt sind.\footnote{Vgl. Voegelin,
  Die Einheit des Rechts und das soziale Sinngebilde Staat, a.a.O., S. 69/70.}
Anhand verschiedener Beispiele zeigt Voegelin, dass die Terminologie der
Reinen Rechtslehre zu undifferenziert ist, um solche und andere Feinheiten
wiederzugeben. Er folgt damit einer Linie von Kritik, wie sie gerne aus
rechtsempirischer Sicht an der Reinen Rechtslehre geübt wird, und die Anstoß
an der dezidierten kategorialen Sparsamkeit der Reinen Rechtslehre
nimmt.\footnote{Vgl. Günther Winkler: Rechtswissenschaft und Rechtserfahrung,
  Wien / New York, S. 41ff.}

Um ein differenzierteres Bild der Rechtsordnung zu liefern, unterscheidet
Voegelin drei Bedeutungsebenen, eine Ebene des "`formale[n]
Aktzusammenhangs"', eine Ebene der "`sachlichen Sinngehalte"' und eine Ebene
des "`normative[n] Sinn[s]"'.\footnote{Voegelin, Die Einheit des Rechts und
  das soziale Sinngebilde Staat, a.a.O., S. 81.} Auf der formalen Ebene werden
die Rechtsnormen als "`Deutungsschemata"' charakterisiert, durch die bestimmte
Handlungen bzw. "`Akte"' einen rechtlichen Sinn erhalten.  Die
"`Deutungsschemata"' wiederum werden in "`Aktreihen"' erzeugt bzw.
angewendet. Voegelin motiviert seine Terminologie mit der Absicht, alle
"`axiologischen"', d.h. moralische Wertentscheidungen ausdrückenden
Konnotationen zu vermeiden, die seiner Ansicht nach bei der Verwendung von
Begriffen wie "`Rechtserzeugung"', "`Rechtsanwendung"' etc. unvermeidlich
miteinfließen.\footnote{Vgl. Voegelin, Die Einheit des Rechts und das soziale
  Sinngebilde Staat, a.a.O., S. 64.} Damit streicht Voegelin aber auf der
"`formalen Ebene"' auch jeden präskriptiven Gehalt des Rechts. Insofern
verwundert es nicht, wenn er dann feststellt, dass sich das Problem der
(normativen) Geltung auf dieser Ebene nicht behandeln lässt.\footnote{Vgl.
  Voegelin, Die Einheit des Rechts und das soziale Sinngebilde Staat, a.a.O.,
  S. 71.} Aber auch auf der nächst höheren Schicht, der des "`inhaltlichen"'
bzw.  "`sachlichen Sinns"', lässt es sich nach Voegelin nicht lösen, da diese
Ebene, die von Voegelin nur sehr knapp skizziert wird, lediglich die
Bedingungsbeziehungen der Akte untereinander beschreibt, also etwa dass die
Lesung eines Gesetzesentwurfes im Parlament die Antragstellung
voraussetzt.\footnote{Vgl. ebd.} Erst auf der normativen Ebene lässt sich die
Geltungsfrage sinnvoll behandeln. Allerdings zeigt sich für Voegelin, dass
sich auf dieser Ebene die Rechtsordnung hinsichtlich der Geltungsfrage nicht
mehr als abgeschlossener Zusammenhang rekonstruieren lässt. Als wesentlichen
Grund liefert er eine Kritik des Kriteriums der Zwangsbewehrtheit. Nach
Voegelin kann das Kriterium der Zwangsbewehrtheit keine Eigenschaft aller
Rechtsnormen sein, da diejenigen Normen, die das Vorgehen der
zwangverhängenden Instanzen regeln aus systematischen Gründen nicht in
demselben Maße zwangsbewehrt sein können, wie die Normen, die das Verhalten
der Bürger regeln sollen. Zumindest gilt dies für die höchsten Instanzen im
Zwangssystem, deren eventuelles Fehlverhalten von keiner anderen Instanz mehr
sanktioniert werden kann.\footnote{Vgl. Voegelin, Die Einheit des Rechts und
  das soziale Sinngebilde Staat, a.a.O., S. 74-76.} Die Möglichkeit, auf der
normativen Ebene die Einheit der Rechtsordnung und, ausgehend von der
Grundnorm, damit auch ihre Geltung durch den Delegationszusammenhang
herzustellen, scheint Voegelin nicht zu sehen, wohl weil er den
Delegationszusammenhang schon auf der "`formalen"' Ebene verbucht hat. So
kommt Voegelin vielmehr sogar zu dem Ergebnis, dass die von ihm
unterschiedenen Bedeutungsebenen teilweise disparat sind und sich allenthalben
"`Sinnlücken"' auftun.\footnote{Vgl. Voegelin, Die Einheit des Rechts und das
  soziale Sinngebilde Staat, a.a.O., S. 82.} Anders als in seinen früheren
Aufsätzen zweifelt Voegelin damit erstmals auch die innere Folgerichtigkeit
der Reinen Rechtslehre an. Hier könnte man natürlich die Frage stellen, ob die
"`Sinnlücken"' nicht eher ein Artefakt von Voegelins nicht in jeder Hinsicht
glücklicher Differenzierung der Rechtsordnung in unterschiedliche
Bedeutungsebenen sind, als dass sie auf Schwächen in der Architektur der
Reinen Rechtslehre verweisen.
%Auf die möglichen Einwände gegen Voegelins
%Behandlung der Reinen Rechtslehre soll jedoch weiter unten bei der
%Beschreibung von Voegelins "`Autoritärem Staat"' noch ausführlicher
%eingegangen werden.
Voegelins beschließt seinen Aufsatz über "`Die Einheit des Rechts und das
soziale Sinngebilde Staat"' mit einer kurzen, an Max Webers Begriffen
orientierten soziologischen Betrachtung der Normativität der Rechtsordnung und
der Legitimität der Herrschaftsordnung.

In seinem Aufsatz {\em Die Verfassungslehre von Carl Schmitt. Versuch einer
  konstruktiven Analyse ihrer staatstheoretischen Prinzipien}\footnote{Eric
  Voegelin: Die Verfassungslehre von Carl Schmitt. Versuch einer konstruktiven
  Analyse ihrer staatstheoretischen Prinzipien, in: Zeitschrift für
  öffentliches Recht XI, 1931, S. 89-109.} (1931) geht Voegelin beiläufig auch
auf die Reine Rechtslehre ein. Voegelin verteidigt dabei die Reine Rechtslehre
gegen verschiedene Missverständnisse von Carl Schmitt und wirft ihm insgesamt
vor, seine Ablehnung der Reinen Rechtslehre "`in die Form der Ablehnung einer
inhaltlich-normativen Rechtslehre"'\footnote{Voegelin, Die Verfassungslehre
  von Carl Schmitt, S. 91.} zu kleiden. Zu dieser Zeit vertritt Voegelin noch
die Ansicht, dass die Reine Rechtslehre nicht auf dieser Ebene, sondern
höchstens an ihren ungeklärten Voraussetzungen scheitert. Er stimmt Carl
Schmitt denn auch darin zu, dass die Geltung des Verfassungsrechts auf
vorrechtlichen Voraussetzungen beruht. Aus diesem Grund lässt sich auch das
Prinzip der Methodenreinheit nach Voegelins Ansicht auf dem Gebiet der
Staatslehre nicht aufrecht erhalten.  Hatte Voegelin in seinem früheren
Aufsatz "`Reine Rechtslehre und Staatslehre"' der reinen Staatsrechtslehre ein
empirisches, historisch-politisches Staatsverständnis entgegengehalten, so
fasst er die Kritik am Prinzip der Methodenreinheit nun auch in Form eines
philosophischen Arguments. Die Methodenreinheit kann nicht aufrecht erhalten
werden, da der Gegenstand der Staatslehre (wie auch anderer
Geisteswissenschaften) "`unabhängig vom Erkenntniszusammenhang der
Wissenschaft Züge der Eigenkonstitution aufweist."'\footnote{Voegelin, Die
  Verfassungslehre von Carl Schmitt, S. 91.}  Bei diesem Argument Voegelins
wird jedoch nicht deutlich, weshalb gerade die "`Eigenkonstitution"' des
Gegenstandes eine methodenreine wissenschaftliche Untersuchung unmöglich
macht. Das Argument erscheint umso weniger überzeugend als die Reine
Rechtslehre den Aspekt der "`Eigenkonstitution"' ihres Gegenstandes in einem
gewissen Sinne, nämlich dem der dynamischen Erzeugung des Rechts durch Recht
setzende Organe, sogar ausdrücklich zum Thema macht.  Neben dieser Kritik hebt
Voegelin aber auch eine Reihe von Verdiensten der Reinen Rechtslehre hervor,
wie die "`genaue Durcharbeitung der normativen Sphäre und ihre Loslösung von
anderen Problemkreisen"', die "`Kritik an den politischen Elementen in der
herrschenden Staatslehre"' und "`die ganz außerordentliche Hebung des Niveaus
in der Rechtstheorie"'.\footnote{Voegelin, Die Verfassungslehre von Carl
  Schmitt, a.a.O., S. 92.} Angesichts dieser immer noch recht positiven
Bewertung der Reinen Rechtslehre in Voegelins Aufsatz über Carl Schmitts
Verfassungslehre ist es nicht ganz verständlich, wenn Günther Winkler in
diesem Aufsatz den Anlass zu einem Gesinnungswandel Kelsens gegenüber Voegelin
vermutet.\footnote{Vgl. Günther Winkler: Eric Voegelin und Hans Kelsen.
  Geistesgeschichtliche Notizen über eine wissenschaftliche Schüler-
  Lehrerbeziehung in der "`Wiener Schule der reinen Rechtslehre"', Archiv der
  Universität Wien, S. 17.} Dass die ebenfalls recht positive Beurteilung von
dessen Antipoden Schmitt durch Voegelin\footnote{Vgl. Winkler, Geleitwort, S.
  XXIV.} dafür hinreichend gewesen sein könnte, erscheint wenig plausibel.

Ebenfalls nur beiläufig geht Voegelin in seinem Buch {\em Rasse und
  Staat}\footnote{Eric Voegelin: Rasse und Staat, Tübingen 1933, S.
  5-7.}  (1933) auf die Reine Rechtslehre ein. Sein Standpunkt ist
nach wie vor derselbe: Der Reinen Rechtslehre werden große Verdienste
auf dem Gebiet der Rechtsinterpretation bescheinigt und zumindest auf
diesem engeren Gebiet hält Voegelin die saubere Abgrenzung von der
kausalwissenschaftlichen Untersuchungsebene einerseits und von
ethischen und politischen Bewertungen andererseits für bewahrenswerte
Leistungen. In der Staatslehre kann der Ansatz der Reinen Rechtslehre
aber höchstens der erste Schritt zu einer umfassenden Staatslehre
sein, die besonders eine "`Staatsideenlehre"' beinhalten und sich auf
eine philosophische Anthropologie stützen müsste, um beispielsweise
die in der Reinen Rechtslehre dogmatisch vorausgesetzte
"`Rechtssphäre"' auf "`ihre Wurzeln im Wesen des
Menschen"'\footnote{Voegelin, Rasse und Staat, a.a.O., S. 7.}
zurückzuführen.

\section{Voegelins Kritik der Reinen Rechtslehre im "`Autoritären Staat"'}

Die ausführlichste und gründlichste Auseinandersetzung mit der Reinen
Rechtslehre findet sich in Voegelins "`Autoritärem Staat"'. Alle kritischen
Argumente aus den früheren Aufsätzen sind darin zusammengefasst.  Zugleich
geht Voegelins Kritik der Reinen Rechtslehre weit über seine schon früher
geäußerten Einwände hinaus und gewinnt den Charakter einer grundsätzlichen
Ablehnung. Sie ist sehr deutlich nicht nur wissenschaftlich, sondern auch
politisch motiviert.

Voegelins Auseinandersetzung mit Kelsens Reiner Rechtslehre spielt sich dabei
auf mehreren Ebenen ab, (1) einer allgemein philosophischen, soweit es um
erkenntnistheoretische, ontologische und anthropologische Voraussetzungen der
Reinen Rechtslehre geht, (2) einer im engeren Sinne rechtsphilosophischen, die
die Frage betrifft, inwieweit die Reine Rechtslehre ihren Gegenstand, die
Rechtsordnung, richtig erfasst und (3) einer historisch politischen, auf der
Voegelin die reine Rechtslehre als Symptom einer typisch österreichischen
Administrativ-Staatlichkeit deutet, die, wie Voegelin es sieht, erst mit dem
sich durch das Dollfuß-Regime vollziehenden Übergang zu einem echt
politischen Staatswesen (im Sinne Carl Schmitts) überwunden wird. Auf alle
Ebenen dieser Kritik soll im folgenden eingegangen werden.

\subsection{Voegelins Kritik der philosophischen Grundlagen der Reinen
  Rechtslehre}

Voegelins Kritik setzt zunächst mit einer philosophischen Einordnung der
Reinen Rechtslehre an. Die Grundprämissen der Reinen Rechtslehre sind
Voegelins Ansicht nach motiviert durch spezifisch positivistische bzw.
neukantianische metaphysische und erkenntnistheoretische
Voraussetzungen.\footnote{Vgl. Voegelin, Autoritärer Staat, S.  102-108.}
Diese Voraussetzungen sind die Prinzipien der Methodenreinheit, der
Gegenstandseinheit und der Konstitution des Gegenstandes durch das
Erkenntnissubjekt. Unter Methodenreinheit ist dabei zu verstehen, dass alle
Erkenntnisgegenstände eines Gegenstandsbereiches durch eine bestimmte
Erkenntnismethode zu erfassen sind. Im Extremfall kann sich diese Forderung
sogar auf die gesamte Wirklichkeit beziehen, sofern sie als ein
Gegenstandsbereich aufgefasst wird. Dies ist nach Voegelins Ansicht im
Neukantianismus der Fall, der die gesamte Wirklichkeit durch ein vorgegebenes
System von Kategorien (bzw. Anschauungsformen) erfassen will.\footnote{Vgl.
  Voegelin, Autoritärer Staat, S. 104/105.} Das Prinzip der Gegenstandseinheit
besagt komplementär dazu, dass alle Gegenstandsbereiche in sich geschlossen
und von anderen soweit trennbar sind, dass sie sich eben "`methodenrein"'
erfassen lassen. Beide Prinzipien gehen wiederum schlüssig aus dem Prinzip der
Konstitution des Erkenntnisgegenstandes durch das Erkenntnissubjekt hervor,
denn wenn der Erkenntnisgegenstand durch das Erkenntnissubjekt (nach Maßgabe
apriorischer Kategorien) allererst konstituiert wird, dann ist dadurch
sichergestellt, dass er auch methodenrein (mit denselben Kategorien) erkannt
werden kann.

Diese Prinzipien kehren in der Reinen Rechtslehre in einer spezifisch
abgewandelten, auf das Recht als den Gegenstand der Reinen Rechtslehre
bezogenen Form wieder. Das Prinzip der Methodenreinheit findet sich in der
Reinen Rechtslehre darin wieder, dass sie das Recht rein als Normordnung
betrachtet, unter striktem Ausschluss aller kausalwissenschaftlichen
("`rechtssoziologischen"') Fragen, die die Rechtsordnung betreffen, wie etwa
die Frage, warum bestimmte Gesetze erlassen werden, welche Wirkung ein Gesetz
auf das Verhalten der Bürger ausüben wird etc. . Aus dem Prinzip der
Gegenstandseinheit wird die Einheit der Rechtsordnung, die alle Rechtsgebiete
innerhalb der einen staatlichen Rechtsordnung umfasst, so dass die Reine
Rechtslehre sogar einen prinzipiellen Unterschied zwischen öffentlichem und
privaten Recht leugnet. Das Prinzip der Konstitution des Gegenstandes durch
das Erkenntnissubjekt geht vor allem in die Lehre von der Grundnorm ein, mit
der die Rechtsordnung gewissermaßen verankert wird.  Es wirkt sich darüber
hinaus indirekt auf die Reine Rechtslehre aus, indem sie die historischen und
empirischen Gegebenheiten, von denen sie faktisch immer noch abhängt, gar
nicht erst thematisiert,\footnote{Voegelin, Autoritärer Staat, S. 109/110. }
da diese Abhängigkeit von empirischen Voraussetzungen nicht zum Prinzip der
Konstitution des Gegenstandes durch das Erkenntnissubjekt passt.

Bei seiner Kritik der Reinen Rechtslehre geht Voegelin nun so vor, dass er
zunächst die Schwächen des Neukantianismus herausarbeitet, und dann zeigt, wie
sich diese Probleme auch auf die Reinen Rechtslehre auswirken, und dort zu
spezifischen Schwierigkeiten führen. Voegelins Kritik am Neukantianismus ist
sehr grundsätzlicher Natur. So hält er dem Prinzip der Konstitution des
Erkenntnisgegenstandes durch das Erkenntnissubjekt entgegen, dass die
Erkenntnisgegenstände nicht in erster Linie durch das Erkenntnissubjekt
konstituiert werden, sondern sich aus vorwissenschaftlichen Zusammenhängen
ergeben. Insbesondere sind die Gegenstandsbereiche, ihre jeweilige Abgrenzung
zueinander und -- für Voegelin besonders wichtig -- die Frage, welche
Gegenstände überhaupt relevante Untersuchungsgegenstände sind, durch
vorwissenschaftliche (lebensweltliche) Zusammenhänge bestimmt. Daher ist es
von vorherein auch keinesfalls sichergestellt, ob und in welchem Maße ein
bestimmter Gegenstandsbereich (oder "`Seinsausschnitt"', wie Voegelin es
nennt) methodenrein erkannt und von anderen Gegenstandsbereichen isoliert
betrachtet werden kann.\footnote{Vgl.  Voegelin, Autoritärer Staat, S. 106-108
  (§ 4).}
% Dergleichen kann der Fall sein, muss es aber
%nicht und ist in jedem Fall erst im Verlaufe der Untersuchung
%festzustellen, so dass man die Methodenreinheit oder die Erkennbarkeit
%mittels bestimmter vorgegebener Kategorien nicht a priori voraussetzen
% darf.

Besonders problematisch wirkt sich für Voegelin dabei aus, dass die
Erkenntnisprinzipien des Neukantianismus (und speziell des Positivismus)
am Maßstab der Naturwissenschaften gebildet und daher für die
Sozialwissenschaften nur sehr bedingt brauchbar sind. Ein wesentlicher
Unterschied zwischen dem Erkenntnisbereich der Naturwissenschaften und
dem der Sozialwissenschaften besteht für Voegelin nämlich darin, dass
die Gegenstände in den Sozialwissenschaften nicht nur, wie der
Neukantianismus annimmt, einer (epistemischen) Konstitution durch das
Erkenntnissubjekt unterliegen, sondern zugleich auch Produkt einer
"`Realkonstitution"' sind, indem sie z.B. durch sprachliche Akte und
entsprechende Handlungen hervorgebracht ("`konstituiert"')
werden.\footnote{Vgl. Voegelin, Autoritärer Staat, S. 108f. (§ 5).}
Wie bereits zuvor angemerkt, ist es allerdings fraglich, ob dieser
Irrtum in der Reinen Rechtslehre überhaupt auftritt, und ob er, wenn
er auftritt, die Ursache der ihr von Voegelin vorgeworfenen Irrtümer,
insbesondere der Identifikation von Staat und Recht, ist. Eine weitere
Folge, die im Positivismus Kelsens, wie Voegelin es sieht, sehr
deutlich zu Tage tritt, ist die, dass der Mensch allein als "`Naturwesen"'
betrachtet wird (ein schlimmer "`faux pas"' aus Voegelins Sicht!), und die
"`Geistnatur"' des Menschen, die für Voegelin auch und besonders eine
spirituelle Seite einschließt, ignoriert wird.\footnote{Vgl. Voegelin,
  Autoritärer Staat, S. 120.} Auch bei diesem Hinweis, der in
Voegelins späterer Philosophie oft als Totschlagargument gegen
Positivisten jeglicher Couleur eingesetzt wird, bleibt es sehr
zweifelhaft, inwiefern er als Argument gegen die Reine Rechtslehre
überhaupt greift.

Es würde an dieser Stelle zu weit führen, Voegelins philosophische
Kritik am Neukantianismus ausführlich zu untersuchen und die Frage zu
erörtern, ob Voegelin die erkenntnistheoretischen Voraussetzungen des
Neukantianismus richtig beschreibt und inwieweit seine Kritik den
Neukantianismus und den (davon zu unterscheidenden) Neupositivismus
trifft. Dazu nur soviel: Voegelin bemüht sich nicht eigens, die Fehler
innerhalb der philosophischen "`Systeme"' des Neukantianismus zu
suchen (was im Rahmen seiner staatswissenschaftlichen, aber nicht
primär philosophischen Untersuchung aber auch kaum verlangt werden
kann), sondern er kritisiert den Neukantianismus von einem anderen
Standpunktpunkt aus. Sein eigener Standpunkt ist ersichtlich von
Edmund Husserls Krisis-Schrift\footnote{Edmund Husserl: Die Krisis der
  europäischen Wissenschaften und die transzendentale Phänomenologie,
  3. Auflage, Hamburg 1996.}  beeinflusst, wobei aber auch Heidegger
und Jaspers -- beide werden im Text erwähnt\footnote{Vgl.  Voegelin,
  Autoritärer Staat, S. 105.} -- Pate gestanden haben könnten. Damit
knüpft er an philosophische Strömungen an die, mit jeweiligen
Differenzierungen, das Erkenntnissubjekt wieder stärker dem Sein
unterordnen und überhaupt der philosophischen Lebensbetrachtung vor
der Erkenntnistheorie den Vorzug geben. Zumindest im Hinblick auf den
transzendentalphilosophischen Ansatz der (neu)kantianischen Philosophie
trifft Voegelins Kritik: Die Annahme, dass die Erkenntisgegenstände
durch Kategorien und Anschauungsformen geordnet sind, impliziert immer
auch ziemlich starke ontologische Voraussetzungen, die in der Regel
ungerechtfertigt bleiben. So impliziert beispielsweise Kants Annahme,
dass die Welt der Sinneserfahrungen (die "`Erscheinungen"')
notwendigerweise kausal geordnet ist, dass auch die Dinge selbst
zumindest soweit vorstrukturiert sind, dass sie auf eine kausal
geordnete Welt von Erscheinungen abgebildet werden können. Die -- im
Folgenden zu erörternde -- Frage bleibt jedoch, ob sich diese Mängel
der neukantianischen Erkenntnistheorie in der Reinen Rechtslehre
überhaupt auswirken.

\subsection{Immanente Kritik: Grenzen der "`Reinheit"' der Rechtlehre}

Voegelins Ansicht nach wirken sich die Schwächen der neutkantianischen
erkenntnistheoretischen Grundlage sehr konkret in der Reinen
Rechtslehre aus. Im Rückgriff auf seine Fundamentalkritik der
neukantianischen Erkenntnisprinzipien leitet er eine ganze Reihe von
Einwänden gegen die Reine Rechtslehre ab. Teilweise sind diese
Einwände funamentalkritischer Natur und zielen vor allem auf die
politisch-ideologiekritischen Implikationen der Reinen Rechtslehre und
die liberale Staatsauffassung Kelsens ab. Aber ein Teil der Kritik
Voegelins kann auch als immanente Kritik verstanden werden, die
Voegelin in Fortführung seiner früheren Aufsätze zur Reinen
Rechtslehre entwickelt. Dieser Teil von Voegelins Kritik soll zunächst
untersucht werden. Er lässt sich dahingehend zusammenfassen, dass die
Reinheit der Rechtslehre kaum aufrecht erhalten werden kann, sondern
an mindestens fünf Punkten Gefahr läuft, durchbrochen zu werden:

1) Es gelingt der Reinen Rechtslehre nicht, sich bloß auf die
Untersuchung von Normen zu beschränken, sie muss als Zweites auch die
Akte hinzuziehen, in denen Normen gesetzt (bzw. angewendet oder
vollzogen) werden. Die Akte bilden aber einen zusätzlichen
Gegenstandsbereich, dessen Einbeziehung die Gegenstandseinheit und
damit die Reinheit der Rechtslehre gefährdet.\footnote{Vgl. Voegelin,
  Autoritärer Staat, S. 111-112 (§8).} Nur mit der durchaus
fragwürdigen, da auf starken neukantianischen metaphysischen
Voraussetzungen beruhenden Hilfskonstruktion der Grundnorm lassen sich
die Akte, wenn überhaupt, in der normativen Sphäre
verankern.\footnote{Vgl. Voegelin, Autoritärer Staat, S.  115-116 (§
  11).}

2) Die Rechtsanwendung ist in hohem Maße geprägt und beeinflusst durch
die Rechtsdogmatik, die schon dadurch auf sie einwirkt, dass die
Richter von Rechtswissenschaftlern ausgebildet
werden.\footnote{Vgl. Voegelin, Autoritärer Staat, S. 110-111 (§ 7).} Die
Rechtsdogmatik ist aber im Delegationszusammenhang gar nicht explizit
vorgesehen. Das System aus Akten und Normen, das den
Delegationszusammenhang bildet, ist damit nicht geschlossen.

3) Der eigentliche Gegenstand der Reinen Rechtslehre, das Recht, ist
historisch-empirisch vorgegeben. Die Reine Rechtslehre übernimmt es,
gewissermaßen naiv, von der Rechtsdogmatik. Nun beziehen sich aber die
traditionell überlieferten Rechtsgebiete wie Strafrecht, Zivilrecht,
öffentliches, also Verfassungs- und Verwaltungsrecht auf sehr unterschiedliche
Lebensbereiche, in denen jeweils eigene Bedingungen gelten, und die sich zum
Teil auch nur unterschiedlich gut rechtlich regeln lassen.\footnote{Vgl.
  Voegelin, Autoritärer Staat, S.  124.}  (Hier zeigen sich wieder die
erkenntnisrelevanten ontologischen Vorbedingungen verschiedener
Gegenstandsbereiche, die von der neukantianischen Erkenntnistheorie nach
Voegelins Ansicht nicht genügend berücksichtigt werden.) Die Rechtsgebiete,
die im Grunde eine Vielzahl von eigenen Rechtsordnungen bilden, können nur
mühsam durch die "`Einführung des Staates"' als normsetzender Instanz bzw. der
staatlichen Verfassung als oberster Norm zu einer Einheit der Rechtsordnung
zusammengefasst werden.\footnote{Vgl.  Voegelin, Autoritärer Staat, S. 110-112
  (§ 7, § 8).}

4) Die Zwangsbewehrtheit lässt sich als definierendes Merkmal von Rechtsnormen
nicht aufrecht erhalten. Nur ein Teil der Rechtsnormen ist überhaupt
zwangsbewehrt. Besonders die Normen des Verfassungsrechts sind zum großen Teil
schon deshalb davon ausgenommen, weil die Verfassung das Verhalten der
Staatsgewalt regelt, neben der Staatsgewalt aber niemand mehr vorhanden ist,
der ihr mögliches Fehlverhalten wiederum durch Zwang unterbinden oder
bestrafen könnte.  (Dieser Irrtum der Reinen Rechtslehre beruht Voegelins
Ansicht nach ebenfalls auf der Vernachlässigung der ontologischen
Eigengesetzlichkeiten der durch unterschiedliche Rechtsgebiete geregelten
Lebensbereiche.) Fällt die Zwangsbewehrtheit als definierendes Merkmal weg, so
bleibt auch der Ausweg verschlossen, die Einheit der Rechtsordnung in ihrem
Charakter als Zwangsordnung zu sehen.\footnote{Vgl. Voegelin, Autoritärer
  Staat, S.  122-126 (§ 17).}

5) Zur genauen Klärung der Bedeutung vieler Wörter und Begriffe, die in den
Gesetzestexten vorkommen, muss vielfach auf nicht-juristische
"`Voraussetzungswissenschaften"'\footnote{Voegelin, Autoritärer Staat, S.
  145.} zurückgegriffen werden. Dies gilt besonders für das Verfassungsrecht
(im Gegensatz zum Bürgerlichen Gesetzbuch enthält die Verfassung ja kaum
Definitionen), das einer "`soziologischen"' Staatslehre als
"`Voraussetzungswissenschaft"' bedarf.\footnote{Vgl. Voegelin, Autoritärer
  Staat, S.  144-147.}

Von diesen fünf Einwänden können die ersten drei die Reine Rechtslehre nur
sehr bedingt gefährden. Dass die Reine Rechtslehre besonders in der
Rechtsdynamik neben Normen auch Akte mit einbezieht, bedeutet insofern keinen
Bruch mit der rein normativen Betrachtungsweise als die Akte nur hinsichtlich
ihrer normativen Bedeutung thematisiert werden.\footnote{Vgl. Hans Kelsen:
  Reine Rechtslehre, Nachdruck der 2.  Auflage, Wien 1992 (1960), im folgenden
  zitiert als: Kelsen, Reine Rechtslehre, S. 2. -- Voegelin konnte die 2.
  Auflage der Reinen Rechtslehre, auf die hier verwiesen wird, und die erst
  1960 erschienen ist, natürlich noch nicht kennen. Für die inhaltliche
  Diskussion, um die es hier geht, ist dieser Anachronismus jedoch
  hinnehmbar.} Die Rechstlehre fragt etwa, ob für einen bestimmten Akt eine
Ermächtigungsgrundlage vorhanden ist, ob den Verfahrensvorschriften genüge
getan wurde, und ob der Akt damit gültig oder ungültig ist. Aber sie fragt
nicht, wie der Akt kausal zu Stande gekommen ist, d.h. aus welchen Ursachen der
Akt so und nicht anders ausgefallen ist. Insofern erscheint es auch nicht ganz
treffend, wenn etwa Günther Winkler in seiner Darstellung von Voegelins Kritik
schreibt, Kelsen sehe "`die Akte bloß als beliebige Inhalte von
Rechtsnormen"', klammere sie "`wegen seines einseitigen kategorialen
Denkansatzes aus der Rechtsbetrachtung aus"' und verweise sie "`als
Seinsphänomen in die {\em Soziologie}"'.\footnote{Winkler, Geleitwort, S.
  XXVII, Hervorhebung im Original.} Richtig ist vielmehr, dass Kelsen in der
Reinen Rechtslehre die Akte einzig unter rechtlich-normativen Gesichtspunkten
betrachtet. Damit sind sie aber weder "`beliebige Inhalte von Rechtsnormen"'
noch werden sie aus der Rechtsbetrachtung ausgeklammert. Ausgeklammert wird
lediglich die Frage ihres kausalen Zustandekommens.

Die Lehre von der Grundnorm, die Voegelin in diesem Zusammenhang kritisch
erörtert, gehört in der Tat zu den schwerer fassbaren philosophischen
Voraussetzungen der Reinen Rechtslehre.  Voegelin hält die Grundnorm für die
Reine Rechtslehre deshalb für unerlässlich, weil seiner Ansicht nach der
Delegationszusammenhang nur dann rein normativer Natur sein kann, wenn er in
einer Norm und nicht in einem Akt (z.B.  der Verfassungsgebung) verankert ist.
Zugleich stellt er es so dar, als sei die Lehre von der Grundnorm ohne
spezifisch kantianisch-transzendentalphilosophische Voraussetzungen überhaupt
nicht zu verstehen. In Wirklichkeit drückt die Grundnorm aber nur die
Behauptung der normativen Gültigkeit der Rechtsordnung aus.\footnote{Vgl.
  Kelsen, Reine Rechtslehre, S. 223ff. 
%- Einer vereinfachenden
%  Interpretationen der Reinen Rechtslehre zu Folge ist die Grundnorm schlicht
%  identisch mit der obersten Norm einer Rechtsordnung, also in der Regel mit
%  der Verfassung.  Vgl.  Michael Baurmann: Der Markt der Tugend. Recht und
%  Moral in der liberalen Gesellschaft. Eine soziologische Untersuchung,
%  Tübingen 1996, S.  75/76.
}  Die Abhängigkeit von der sehr
voraussetzungsreichen kantianischen Philosophie ist dabei nicht allzu stark,
zumal Kelsen, auch wenn er eine Analogie zwischen der Annahme der Grundnorm
und dem transzendentalphilosophischen Verfahren Kants herstellt,\footnote{Vgl.
  Kelsen, Reine Rechtslehre, S. 205.}  die Grundnorm gerade deshalb
ausdrücklich als eine bloß "`gedachte"' Voraussetzung
apostrophiert,\footnote{Vgl. Kelsen, Reine Rechtslehre, S. 204ff.}  weil er --
anders als Kant und die meisten kantianischen Philosophien -- Normen
grundsätzlich nicht für begründbar (auch nicht durch die Vernunft begründbar)
hält.

Mit der Trennung von kausaler und normativer Ebene erledigt sich auch
der zweite Einwand: Die Einwirkung der Rechtsdogmatik auf die
Rechtsprechung gehört der Sphäre der Kausalbeziehungen an und betrifft
damit lediglich die für die Reine Rechtslehre unwesentliche Frage, aus
welchen Ursachen ein Rechtsakt so oder anders ausfällt. Ein
Rechtsgutachten oder ein wissenschaftlicher Aufsatz, den ein
Rechtswissenschaftler verfasst, hat als solcher ja noch keinerlei
rechtliche Wirkung, sondern gibt lediglich die Rechtsmeinung des
Verfassers wieder.\footnote{Vgl. Kelsen, Reine Rechtslehre, S. 352ff.}
Dass solche Rechtsmeinungen für die Rechtsanwender
Entscheidungsgrundlage werden können, ist dem Kausalzusammenhang
zuzuordnen.\footnote{Schwierigkeiten könnte der Reinen Rechtslehre
  höchstens der denkbare Fall bereiten, in dem die Vernachlässigung
  rechtsdogmatisch üblicher aber nicht gesetzlich fixierter
  Auslegungsregeln in einer richterlichen Entscheidung als
  Rechtsfehler betrachtet werden würde.}

Der dritte Einwand, der die Unterschiedlichkeit der verschiedenen
Rechtsgebiete und der darin zu regelnden Lebensbereiche betrifft, lässt sich
sehr leicht damit beantworten, dass die verschiedenen Rechtsgebiete allesamt
im Delegationszusammenhang erfasst und in der Verfassung verankert sind.
Voegelin ist sich dessen natürlich bewusst, seine Kritik zielt vor allem
darauf, dass die Reine Rechtslehre den Staat seiner Ansicht nach in einem
umfassenderen, insbesondere auch vorrechtlichen Sinne voraussetzen muss, als
sie dies zugesteht. So ist es wohl zu verstehen, wenn Voegelin die
"`Einführung des Staates"' in die Reine Rechtslehre als "`ein dem
Erkenntnisprozeß vorangehender Akt existentieller
Relevanzfeststellung"'\footnote{Voegelin, Autoritärer Staat, S. 111.}
beschreibt.
Nun bezieht sich die Reine Rechtslehre in der Tat auf einige elementare
empirische Voraussetzungen, wie die Tatsache, dass es überhaupt eine
Rechtsordnung gibt, dass es rechtssetzende Institutionen und rechtsanwendende
Instanzen gibt etc., Tatsachen, die ja -- selbstverständlich wie sie seien
mögen -- nicht a priori hergeleitet werden können. Dass die Reine Rechtslehre
sich auf empirische Phänomene bezieht bedeutet jedoch noch nicht, dass sie
diese Phänomene auch in ihrer ganzen empirischen Bedeutungsfülle voraussetzen
müsste, wie Voegelin es offenbar für notwendig hält. Es ist ja gerade das
Grundanliegen der Reinen Rechtslehre, das Recht ausschließlich unter dem
normativ-rechtlichen Gesichtspunkt zu untersuchen, wobei sie von allen anderen
Bezügen, empirisch-kausalen ebenso wie ethischen und politischen,
abstrahiert.\footnote{Vgl. Kelsen, Reine Rechtslehre, S. 78f.}  In dieser
Hinsicht ist der Rückgriff auf den Staat bzw. die Verfassung in der Reinen
Rechtslehre nicht problematischer als der Rückgriff auf andere Rechtsgebiete
und beinhaltet keine Inkonsequenz. Die unterschiedliche Regelbarkeit
verschiedener Lebensbereiche durch Rechtsnormen schließlich ist eine
empirische bzw. rechtssoziologische Frage, die die Reine Rechtslehre nur
bedingt berührt. Im Zweifelsfall hat die geringere rechtliche Regelbarkeit
bestimmter Lebensbereiche nur zur Folge, dass den Rechtsanwendern in diesen
Bereichen größere Ermessenspielräume verbleiben. Für die Reine Rechtslehre ist
das unproblematisch.

Gravierender sind der vierte und fünfte Punkt, an denen Voegelin die Reinheit
der Rechtslehre bedroht sieht. Besonders der Zwangsbewehrtheit als
wesentliches Merkmal von Rechtsnormen widmet Voegelin einen recht ausführlichen
Abschnitt.\footnote{Vgl. Voegelin, Autoritärer Staat, S.  122-126 (§ 17). --
  Vgl. Kelsen, Reine Rechtslehre, S. 34ff.}  Voegelins Haupteinwand lautet,
dass es in jeder Rechtsordnung auch Rechtsnormen gibt, ja geben muss, die
nicht zwangsbewehrt sind. Vor allem gilt dies für die Normen des
Verfassungsrechts, die das Handeln der Staatsorgane regeln. Denn da irgend
eine Instanz die höchste zwingende Instanz sein muss, wer sollte dann wiederum
die Normbefolgung dieser Instanz erzwingen? Die Normbefolgung der höchsten
Staatsorgane wird daher laut Voegelin auch nicht durch Normkontrolle sondern,
wenn überhaupt, durch "`flexible Situationskontrolle"' ausgeübt, indem
beispielsweise mehrere Kammern an der Gesetzgebung beteiligt werden, die
aufeinander eine Art von Kontrolle ausüben.  
%Aus Kelsenscher Sicht könnte man
%dagegen nun einwenden, dass, da die an der Situationskontrolle beteiligten
%Instanzen ja nicht gehalten sind ihre Kompetenzen nach verfassungsrechtlichen
%Gesichtspunkten auszuüben (sondern nach politischen), solche
%Verfassungsnormen, die außer durch "`flexible Situationskontrolle"' in keiner
%Weise zwangsbewehrt sind, rechtlich irrelevant bleiben. Bei einem solchen
%Einwand würde allerdings indirekt zugestanden, dass es verfassungspolitische
%Probleme geben könnte, die mit rechtlichen Mitteln nicht in den Griff zu
%bekommen sind.
%Doch ist es gar nicht einmal notwendig, diesen Einwand zu erheben, denn es
%gibt kein grundsätzliches ("`ontologisches"') Hindernis, warum
%Verfassungsrecht
%nicht zwangsbewehrt sein könnte, kann man es doch der Normenkontrolle durch
%eine Verfassungsgerichtsbarkeit unterwerfen, die zumindest die rechtliche
%Nichtigkeit von nicht verfassungskonformen Akten der Staatsorgane feststellen
%kann.
Gerade im Hinblick auf diese Problematik wurde aber - nicht zuletzt von Kelsen -
eine normenkontrollierende Verfassungsgerichtsbarkeit entworfen, die zumindest
die rechtliche Nichtigkiet von nicht verfassungskonformen Akten der
Staatsorgane feststellen kann. (Inwieweit man diese Art der Sanktion noch
unter den Begriff "`Zwang"' fassen kann, sei hier einmal dahin
gestellt.\footnote{Vgl. Kelsen, Reine Rechtslehre, S. 52/53, S. 55ff.})
Voegelin zieht die Verfassungsgerichtsbarkeit in diesem Zusammenhang jedoch
gar nicht ernsthaft in Betracht. Die Gründe dafür sind weder besonders
philosophischer noch von pragmatischer Natur,\footnote{Ein pragmatischer Grund
  gegen die Verfassungsgerichtsbarkeit lässt sich indirekt Voegelins später im
  Text folgender Erörterung des Grundrechtschutzes in der autoritären
  Verfassung entnehmen. Voegelin betrachtet den verbliebenen Grundrechtsschutz
  als einen Fremdkörper der autoritären Verfassung und empfiehlt dem
  Verfassungsgericht dringend sich für unzuständig zu erklären, sollte der
  entsprechende Fall auftreten.  Eine Auseinandersetzung zwischen Regierung
  und Verfassungsgericht könne nur mit der "`Kompromittierung"' des Gerichts
  ausgehen. (Voegelin, Autoritärer Staat, S. 274.) Voegelins Argument ist
  jedoch nur in einem autoritären Kontext schlüssig, wo die Regierung
  tatsächlich die Macht hat, das Verfassungsgericht nach Belieben zu
  kompromittieren. Also lässt sich auch daraus kein allgemeingültiger Einwand
  gegen die Möglichkeit einer Normenkontrolle im "`Seinsbereich"' (Voegelin)
  des Verfassungsrechts ableiten.}  sondern in dem von Carl Schmitt
beeinflussten normativen Politikverständnis Voegelins zu
suchen.\footnote{Siehe dazu auch weiter unten die Ausführungen zur politischen
  Motivation Voegelins} Ganz im Sinne von Carl Schmitt polemisiert Voegelin
mit Bezug auf den österreichischen Verfassungsgerichtshof gegen die
"`Entpolitisierung der politischen Instanzen durch ihre Unterordnung unter
eine sanktionierende Instanz"'.\footnote{Voegelin, Autoritärer Staat, S.  126.
  -- Im Zusammenhang der Diskussion der Prüfung von Verordnungen durch den
  Bundesgerichtshof nach der autoritären Verfassung von 1934 spricht Voegelin
  später ähnlich polemisch vom "`Weiterleben der Verfassungsideologie von
  1920"' (Voegelin, Autoritärer Staat, S.  275).} Dass Voegelin so sehr darauf
beharrt, dass das Handeln der höchsten Staatsorgane nur begrenzt rechtlich
regelbar sein soll, hat also weniger mit der Natur des durch das
Verfassungsrecht geregelten "`Seinsbereiches"' zu tun als vielmehr mit einem
ganz bestimmten Politikverständnis Voegelins, demzufolge die staatliche
Gewalt, soll sie ihres "`politischen"' Charakters nicht verlustig gehen, nicht
durch ein sanktionsbewehrtes Verfassungsrecht gezügelt werden darf.

Mit dem fünften Punkt, dass die Reine Rechtslehre die Rechtswissenschaft der
allgemeinen Staatswissenschaft wie auch anderer Wissenschaften als
"`Voraussetzungswissenschaften"' bedarf, trifft Voegelin allerdings einen
Punkt, auf den man in Kelsens Darstellungen der Reinen Rechtslehre nicht so
leicht eine Antwort findet. Man könnte sich im Extremfall sogar vorstellen,
dass sich durch die Aushöhlung zentraler staatsrechtlicher Begriffe
(etwa durch eine politisch entsprechend interessierte Staatslehre) der
Charakter der Verfassung ändert.\footnote{Man überlege etwa wie
  unterschiedlich sich ein pluralistischer Demokratiebegriff und ein
  Demokratiebegriff im Sinne "`homogener"' Volksdemokratie sich auf die
  Möglichkeit der Vertiefung der Europäischen Integration aus Sicht des
  Verfassungsrechts auswirken können. Spätestens wenn sich ein bestimmter
  Demokratiebegriff endgültig gegen einen anderen durchgesetzt hat, könnte man
  von einem stillschweigenden Verfassungswandel ausgehen (soweit die
  Interpretation der Verfassung vom Demokratiebegriff abhängt).} Dieser Fall
wäre nicht mehr der zuvor beschriebenen bloß kausalen Einwirkung der
Rechtsdogmatik auf richterliche Auslegung durch die Erarbeitung von
"`Entscheidungsgrundlagen"' vergleichbar. Denn über die Ansichten von
Rechtswissenschaftlern kann sich ein Richter hinwegsetzen, aber wohl kaum über
den wissenschaftlichen Sprachgebrauch bzw. das allgemeine Sprachverständnis.

Wenn Voegelin somit auf eine Lücke im Gedankengebäude der Reinen Rechtslehre
aufmerksam macht, dann freilich nicht, um sie zu schließen, sondern gerade
weil er die Rechtslehre wieder ausdrücklich an eine philosophische und von
"`politisch-ethischen Postulaten"' (Kelsen) keineswegs freie Staatslehre
binden wollte. Eine erste Nutzanwendung aus dieser Art staatswissenschaftlich
informierter Rechtsinterpretation zieht Voegelin in seinem "`Autoritären
Staat"' sogleich selbst, wenn er es in einer später im Text folgenden Passage
unternimmt, den Verfassungsübergang von 1934 auch juristisch als legalen
Vorgang zu rechtfertigen, was auf Grundlage der Interpretationsprinzipien der
Reinen Rechtslehre, wie Voegelin selbst einräumt, nicht zu machen
ist.\footnote{Diese reichlich fragwürdige Rechtfertigungskonstruktion, bei der
  Voegelin sich eifrig auf "`Sinnlinien"', "`Sinnphänomene"' und
  "`Seinsbereiche"' beruft, klingt dann etwa so: "`Was immer geschah, war
  gültiges Verfassungsrecht -- die Versuche der Deutung vom B.-VG.  1920/29
  her waren, da dessen Funktion als Ordnung des Seinsbereiches sich stetig
  abschwächte im gleichen Maße zunehmend inadäquat."' (Voegelin, Autoritärer
  Staat, S. 180.)}

Voegelin ist aber wohl zuzubilligen, dass er mit dem Hinweis auf die
semantischen Voraussetzungen der Rechtsauslegung auf ein Problem gestoßen ist,
dass in der Reinen Rechtslehre weitgehend unberücksichtigt geblieben
ist.\footnote{Vgl. dazu auch Voegelins Ausführungen in: Voegelin, Autoritärer
  Staat, S.  143-150 (§ 23), die keineswegs alle abzulehnen sind.}

\subsection{Fundamentalkritik: Angriff auf Kelsens Positivismus}

Neben den eben diskutierten eher immanenten Kritikpunkten, gibt es noch eine
Reihe von Aspekten, in denen Voegelin die Reine Rechtslehre sowie die
politische Philosophie Kelsens, soweit sie über die Rechtslehre hinaus geht,
noch viel grundsätzlicher angreift. Zu diesen zählen, (1) die Trennung von
Rechtslehre und Rechtsoziologie,\footnote{Vgl. Voegelin, Autoritärer Staat, S.
  113-115 (§ 10), S. 116/117, S. 119.} (2) Kelsens Identifizierung von
natürlicher und juristischer Person und dessen Rückführung subjektiver auf
objektive Rechte,\footnote{Vgl. Voegelin, Autoritärer Staat, S. 121-122.} (3)
die Lehre von der Identität von Staat und Recht,\footnote{Vgl. Voegelin, S.
  122 (§ 16).}  (4) Kelsens Vorstellungen einer
Weltrechtsordnung\footnote{Vgl. Voegelin, Autoritärer Staat, S. 126-127 (§
  18).} und ganz besonders (5) Kelsens Ideologiekritik.\footnote{Vgl.
  Voegelin, Autoritärer Staat, S.  116-118 (§ 12).}  Voegelins Diskussion
dieser Aspekte von Kelsens Philosophie ist dadurch charakterisiert, dass sie
überwiegend polemisch ausfällt, und dass Voegelin sich vergleichsweise wenig
mit Kelsens Argumenten auseinandersetzt.  Vielmehr leitet er seine Kritik
weitgehend aus seiner eigenen Einordnung von Kelsens Philosophie als
"`positivistische Metaphysik"' ab. Sehr deutlich zeigt sich hier die Abkehr
von der aufgeschlossenen Haltung gegenüber der Reinen Rechtslehre in Voegelins
früheren Schriften.

Die Trennung von Rechtslehre und "`Soziologie"' bei Kelsen wird von Voegelin
so aufgefasst, dass die Rechtslehre alle legitimen Elemente einer
geisteswissenschaftlichen Gesellschaftslehre bei sich monopolisiert, während
die Soziologie nur als kausale "`Naturwissenschaft"' zurückbleibt.
Ausgeschlossen wird damit eine umfassende Gesellschaftswissenschaft, die
zugleich Geisteswissenschaft aber nicht bloß Normwissenschaft im engeren Sinne
der Reinen Rechtslehre ist.\footnote{Vgl. Voegelin, Autoritärer Staat, S.
  120.}  Nun hat Kelsen die Gesellschaftswissenschaft, soweit sie nicht
Rechtslehre ist, in der Tat als eine Kausalwissenschaft ähnlich den kausalen
Naturwissenschaften verstanden. Aber die Reine Rechtslehre ist dabei --
unabhängig von Kelsens tatsächlichen Ansichten über die Natur der Soziologie
oder das Wesen des Menschen -- mühelos mit der Ansicht vereinbar, dass der
Mensch nicht bloß ein materielles sondern auch ein geistiges Wesen ist, und
dass die Soziologie als Kausalwissenschaft auch geistige Kausalzusammenhänge
untersucht. (Jede psychologische Gesetzmäßigkeit beschreibt ja im Grunde einen
geistigen Kausalzusammenhang.)
% Ebenso mühelos ist die Reine Rechtslehre mit einer
% hermeneutischen Auffassung von den Gessellschaftswissenschaften
% vereinbar, nach der die Hauptaufgabe der Gesellschaftswissenschaften
% im verstehenden Nachvollzug der vom Menschen geschaffenen
% Symbolsysteme oder dergleichen besteht.
Voegelins Kritik an Kelsens materialistischer Auffassung der Soziologie trifft
kein wesentliches, sondern, wenn überhaupt, nur ein sehr kontingentes und
zeitgebundenes Moment von Kelsens Philosophie.

Anders verhält es sich mit Kelsens Ablehnung solcher Arten von Geistes- und
Gesellschaftswissenschaften, bei denen explizit oder implizit beansprucht
wird, aus der Untersuchung der Gesellschaft, der Geschichte oder auch der
Geistnatur des Menschen verbindliche ethische Konsequenzen wissenschaftlich
abzuleiten. Dieses Stück Kelsenscher Ideologiekritik gehört sicherlich zu den
Kernelementen seiner Philosophie. Voegelin vertrat in enger Anknüpfung an die
Anthropologie Max Schelers nun gerade einen solchen Typus von
geisteswissenschaftlicher Anthropologie.\footnote{Max Scheler legte großen
  Wert darauf, dass der Mensch als Geistwesen in allen Seinsschichten,
  einschließlich dem transzendenten Seinsgrund verwurzelt ist. (Vgl. Max
  Scheler: Die Stellung des Menschen im Kosmos, 14. Auflage, Bonn 1998 (1928),
  S. 38ff., S. 87ff.) Wenn Voegelin sich auf die philosophische Anthropologie
  beruft, dann meint er vor allem die von Max Scheler und nicht die von
  Schelers Nachfolgern Plessner oder Gehlen, die den Menschen wieder
  konsequent als Naturwesen bestimmen.}  Nur verteidigt er diesen Ansatz nicht
direkt, indem er Kelsens dagegen gerichtete Argumente angreift, sondern
indirekt, indem er Kelsens wissenschaftstheoretische Auffassungen über das
Verhältnis von Rechtslehre und Soziologie in einer Weise darstellt, die sie in
sich widersprüchlich und hoffnungslos verworren erscheinen
lässt.\footnote{Vgl.  Voegelin, Autoritärer Staat, S. 114. Im Anschluss an
  eine wenig faire Kelsen-Interpretation kann Voegelin an dieser Stelle dann
  bequem das Fazit ziehen: "`Daß im Spiel dieser wechselnden Bedeutungen eine
  innersystematisch widerspruchsfreie Begründung der Methodenfragen von
  Naturwissenschaft, Geisteswissenschaft, Soziologie und Normwissenschaft auch
  nur versucht würde, ist der Lage der Sache nach unmöglich. Wir müssen uns
  mit der Darstellung des instrumentalen Charakters der
  Wissenschaftseinteilung im Dienste der Metaphysik begnügen."'  (Voegelin,
  Autoritärer Staat, S. 114.)}

Gegen Kelsens Kritik des Begriffs der natürlichen Person in der Rechtslehre
und gegen dessen Identifizierung des Staates mit der Rechtsordnung wendet
Voegelin ein, dass auf diese Weise "`notwendig alle Rechtsbegriffe
verschwinden [müssen], die ihre Legitimation aus der Behauptung einer vom
Norminhalt unabhängigen einzelmenschlichen oder sozialen geistigen Substanz
schöpfen"'.\footnote{Vgl. Voegelin, Autoritärer Staat, S.121.}  Was mit diesem
Einwand aber verkannt wird, ist die abstrahierende, eben rein auf den
rechtlichen Gehalt bezogene Betrachtungsweise von Kelsens Reiner Rechtslehre.
Diese schließt es keineswegs aus, bestimmte ethische Überzeugungen
hinsichtlich
einer "`einzelmenschlichen oder sozialen geistigen Substanz"' zu vertreten.
Aber wenn diese ethischen Überzeugungen Eingang in das Rechtssystem finden
sollen, dann sollte dies dadurch geschehen, dass sie explizit kodifiziert
werden, etwa durch einen Grundrechtekatalog, nicht dadurch, dass sie aus
irgendwelchen im Gesetzestext vorkommenden oder in der Rechtsdogmatik
gebräuchlichen Begriffen abgeleitet werden.

Dasselbe gilt natürlich für die Frage einer Weltrechtsordnung. Ob eine solche
überhaupt eingerichtet werden soll, ist eine Wertfrage, die nicht dadurch
präjudiziert wird, dass in der Reinen Rechtslehre eine Rechtsordnung nur dann
als vollkommen (im rechtstechnischen Sinne) bezeichnet wird, wenn der
individuelle Rechtsverletzer bestraft wird anstatt das Kollektiv, dem er
angehört. Eine solche Präjudizierung wirft Voegelin der Reinen Rechtslehre
jedoch vor, und er ist auf keinen Fall bereit zu akzeptieren, dass es als
ideologisch gelten soll, wenn jemand meint, "`dass die Forderung nach Ordnung
der Verfassungssphäre durch Zwangsnormen und nach Erledigung politischer
Gegensätze zwischen Staaten durch gerichtsförmiges Verfahren unter der
Sanktion einer übergeordneten Macht die Zerstörung der politischen Substanz
der Staaten bedeute"'.\footnote{Voegelin, Autoritärer Staat, S. 127.} Indem
Voegelin solcherart einen zentralen Topos rechtshegelianischer Staatsideologie
aufgreift, bestätigt er jedoch nur die Kelsensche Ideologiekritik.\footnote{Zu
  Kelsens Kritik der speziell mit dem Souveränitätsbegriff zusammenhängenden
  Ablehnung des Völkerrechts vgl.  Hans Kelsen: Das Problem der Souveränität
  und die Theorie des Völkerrechts.  Beitrag zu einer Reinen Rechtslehre,
  Nachdruck der 2.Auflage von 1928, Aalen 1960, S. 196ff.}

Die Ideologiekritik Kelsens geht Voegelin denn auch besonders scharf
an. Hat Voegelin Kelsens rein juristische Deutung des Begriffs der
"`natürlichen Person"' schon als "`Auflösung der
Person"'\footnote{Voegelin, Autoritärer Staat, S. 121.} und den
Staatsbegriff der Reinen Rechtslehre als "`Auflösung des
Staates"'\footnote{Voegelin, Autoritärer Staat, S. 122.}
gebrandmarkt, so spricht er in Bezug auf die Ideologiekritik gar von
einem "`System der metaphysischen Kampfbegriffe"',\footnote{Voegelin,
  Autoritärer Staat, S. 116.}  womit er neben dem von Kelsen
allerdings recht häufig erhobenen Ideologievorwurf den Vorwurf
"`Scheinprobleme"' zu behandeln oder im Rahmen der Rechtslehre
"`Soziologie"' zu betreiben meint. Da Voegelins eigener, normativ
ontologischer Standpunkt, unter das Ideologie-Verdikt Kelsens fällt,
musste er dessen Ideologiekritik natürlich ablehnen. Aber das allein
erklärt noch nicht die aggressive Schärfe dieser Ablehnung. Sie wird
nur verständlich wenn man den politischen Hintergrund von Voegelins
"`Autoritärem Staat"' berücksichtigt.

\subsection{Die politische Motivation von Voegelins Kritik}

Voegelin betrachtete die Reine Rechtslehre -- durchaus mit einigem Recht --
als die Rechts- und Staatslehre der Österreichischen Republik par excellence
und genau als diese bekämpft er sie im "`Autoritären Staat"'.  Voegelins
eigene Staatsauffassung erscheint im "`Autoritären Staat"' sehr stark bestimmt
durch das nationalstaatliche Model, angereichert mit Vorstellungen, die seiner
Rezeption der autoritären (Dollfuß, Hauriou) und totalitären Staatstheorie
(Moussolini, Carl Schmitt, Ernst Rudolf Huber, Ernst Jünger)
entspringen.\footnote{Vgl.  Voegelin, Autoritärer Staat, S. 7-54. -- Sigwart
  bezeichnet die Carl Schmitt-Rezeption im "`Autoritären Staat"' als
  "`fundamental-kritisch"' (Sigwart, a.a.O., S. 202). Dagegen spricht jedoch
  die von ihm selbst herausgearbeitete Übernahme zahlreicher grundlegender
  Kategorien Schmitts durch Voegelin. Kritische Töne fehlen Voegelins
  Schmitt-Rezeption im "`Autoritären Staat"' fast vollkommen. Man kann es kaum
  als Kritik deuten, wenn Voegelin anmerkt, dass der Begriff des "`totalen
  Staates"' eher ein politisches Symbol als ein wissenschaftlicher Begriff
  sei. Dies gilt umso mehr, als Voegelin ausdrücklich sagt, dass "`die
  glückliche Prägung Schmitts zu einer politischen Funktion des Ausdrucks
  geführt [hat], die gleichfalls den systematischen Rahmen sprengt, den ihr
  Urheber um ihn gezogen hat."' Voegelin betrachtet den Begriff des "`totalen
  Staates"' bei Carl Schmitt also sehr wohl als wissenschaftlichen Begriff,
  der erst durch seinen Erfolg nachträglich ein politisches Symbol geworden
  ist.} Naturgemäß muss der österreichische Staat nach diesem Maßstab prekär
erscheinen. Das gilt schon für die Monarchie, in der die "`seelische Formung
der Bevölkerung des Territoriums zu einem politischen Volk ..  überhaupt nicht
erreicht"'\footnote{Voegelin, Autoritärer Staat, S.  2.} wird.  Zudem konnte
die "`staatliche Machtorganisation"' in der Monarchie nie dasselbe
"`Autoritätsgewicht"' erwerben, wie die anderer westeuropäischer Nationen,
weil "`das politische Volk, das die Machtorganisation als den Ausdruck seines
politischen Existenzwillens erlebt, sich nicht entwickelt
hat."'\footnote{Voegelin, Autoritärer Staat, S. 2.} Die Folge dieser
mangelhaft ausgeprägten Nationalstaatlichkeit besteht darin, dass die
österreichische Politik keinen wirklich politischen, sondern nur einen
gleichsam "`administrativen"' und "`apolitischen"' Stil besitzt.  Wurde in der
Monarchie dieser "`apolitische Charakter"' wenigstens noch "`durch die
autoritären Züge ausgewogen .., die das politische Gebilde aus den Quellen des
`Reiches' hatte, vor allem durch die Person des
`Kaisers'"',\footnote{Voegelin, Autoritärer Staat, S. 3.}  so ist die
Staatlichkeit der Republik nur noch administrativer Natur und "`die
herrschende Verfassungslehre dieses politischen Gebildes, die reine
Rechstlehre, zeigt in idealtypisch vollkommener Weise die Züge, die wir eben
als die des 'administrativen Stils' herausgearbeitet
haben."'\footnote{Voegelin, Autoritärer Staat, S. 3.}  Das ist durchaus etwas
verächtlich gemeint, spricht Voegelin an anderer Stelle doch von "`der
Typenlehre Max Webers und der reinen Rechtslehre Kelsens"' sogar als von
Versuchen, "`auf methodisch verschiedener Basis die Verfallserscheinung der
'Legalität' zu rationalisieren."'\footnote{Voegelin, Autoritärer Staat, S.
  157.}  Erst mit der Errichtung des autoritären Staates ändert sich die
Situation nach Voegelins Einschätzung wieder zum besseren, denn "`es sind
existentielle Schritte in der Staatswerdung Österreichs geschehen, in dem
Sinne, daß die obersten Staatsorgane durch die politische Situation
legitimiert als die Träger des Willens zur Existenz des Staates Österreich
entscheidend auftraten."'\footnote{Voegelin, Autoritärer Staat, S. 3.}

Um Voegelins Polemik gegen die österreichische
"`Administrativstaatlichkeit"' richtig einschätzen zu können, muss
kurz auf den Politikbegriff von Carl Schmitt eingegangen werden, der
Voegelin merklich beeinflusst hat.\footnote{Vgl. Michael Henkel:
  Positivismuskritik und autoritärer Staat, a.a.O., S. 44ff. -- Vgl.
  Claus Heimes: Antipositivistische Staatslehre, S.  35ff.} Zu den
wesentlichen Komponenten von Carl Schmitts
Politikbegriff\footnote{Carl Schmitt: Der Begriff des Politischen,
  Hamburg 1933. -- Vgl. Carl Schmitt: Politische Theologie.  Vier
  Kapitel zur Lehre von der Souveränität, 7. Auflage, Berlin 1996
  (1922), S. 19f.} gehören: 1.  Die Unterscheidung von Freund und
Feind. 2. Die Souveränität als eine in keiner Weise gebundene Macht.
3. Die Betonung der tatsachenschaffenden Entscheidung.  Auf Voegelin
haben davon offenbar die 2. und 3. Komponente abgefärbt.\footnote{Vgl.
  dazu auch Sigwart, a.a.O., S. 161ff. -- Meine Deutung unterscheidet
  sich allerdings von der Sigwarts. Insbesondere scheint mir Sigwart
  den Einfluss Max Webers auf Voegelin und besonders auf dessen
  Wissenschaftsethos stark zu überschätzen.} Man sieht leicht, dass
die 2.  Komponente mit dem Prinzip der Verfassungsstaatlichkeit, das
nur einen durch Verfassungsnormen gebundenen Souverän zulässt, schwer
vereinbar ist.  Daher rührt die Polemik gegen den Verfassungsstaat als
einen bloß administrativen und nicht wirklich politischen Staat. Und
daher rührt insbesondere Voegelins Vorwurf gegen Kelsen und die Reine
Rechtslehre, das Verfassungsrecht zum Privatrecht zu
machen,\footnote{Vgl.  Voegelin, Autoritärer Staat, S.  126.}  und dem
Staat keine höhere "`Dignität als einem Briefmarkensammlerverein"'
zuzugestehen.\footnote{Voegelin, Autoritärer Staat, S. 127.} Und aus
dem u.a.  von Carl Schmitt übernommenen politischen Existentialismus
erklärt es sich auch, wenn Voegelin Kelsen "`Normfetischismus"'
vorwirft und ihm, weil er für den Verfassungsstaat eintrat, eine
"`Angst vor dem Ungewissen"' unterstellt.\footnote{Vgl.  Voegelin,
  Autoritärer Staat, S. 153.  Voegelin steigert sich an dieser Stelle
  geradezu in eine Mythologie von "`Chaos"' und "`Ordnung"' hinein,
  deren Vernachlässigung er dem Rechtspositivismus ankreidet.  -- Vgl.
  auch Sandro Chignola, "`Fetishism with the Norm"' and Symbols of
  Politics. Eric Voegelin between Sociology and Rechtswissenschaft
  (1924-1938), S. 59ff.  Sandro Chignola scheint die politische
  Brisanz und den ideologischen Hintergrund von Voegelins Vorwurf des
  "`Normfetischismus"' allerdings nicht zu bemerken. Dieses Versäumnis
  ist nicht uncharakteristisch für einen großen Teil der
  Voegelin-Sekundärliteratur, die Voegelins Denken ganz in seinen
  eigenen Kategorien deutet und nicht zuletzt dadurch häufig jede
  kritische Distanz vermissen lässt.}

Vor diesem Hintergrund muss nun auch die von Voegelin vorgenommene Einordnung
von Kelsens Reiner Rechtslehre in die österreichische Verfassungsgeschichte
verstanden werden, die zu den schwächsten Teilen seiner Auseinandersetzung mit
Hans Kelsen im "`Autoritären Staat"' gehört. Wie bereits angemerkt, sieht
Voegelin in Kelsens reiner Rechtslehre die Verfassungslehre eines
"`apolitischen"', bloß "`administrativen"' Staates.  Voegelin geht sogar
soweit, die Reine Rechtslehre in die Tradition eines politischen Quietismus
der österreichischen Staatswissenschaften einzuordnen, der noch in die Zeit
der Restauration zurückreicht,\footnote{Vgl. Voegelin, Autoritärer Staat, S.
  128.}  und der sich auf Grund der Tatsache, dass die österreich-ungarische
Monarchie als Vielvölkerstaat eine anderen europäischen Staaten vergleichbare
Entwicklung zum Nationalstaat nicht durchmachen konnte, in einer dem positiven
Staats- und Verwaltungsrecht verhaftet bleibenden Staatslehre fortsetzte.
Diese Einordnung von Kelsens Lehre in einen Zusammenhang "`unpolitischer"'
Staatslehren verwundert ein wenig, wirft Voegelin ihr an zahlreichen Stellen
doch ihren vermeintlich eminent politischen Charakter und ihren Einsatz als
Waffe im politischen Meinungsstreit vor.\footnote{Vgl.  Voegelin, Autoritärer
  Staat, S.  113 (§ 10), S.  116-118 (§ 12), S.  125-127 (§ 17, § 18).} Zudem
sind geistig verwandte rechstpositivistische Strömungen ja auch in anderen
Staaten unter ganz anderen zeithistorischen Bedingungen
aufgekommen.\footnote{Voegelin selbst hatte mit der Souveränitätstheorie
  Dickinsons eine geistig verwandte amerikanische Theorie einige Jahre früher
  ausführlich besprochen (siehe oben).} Zur Krönung seiner Polemik gegen die
Reine Rechtslehre versteigt Voegelin sich schließlich zu einem völlig
deplatzierten Vergleich zwischen Kelsen und dem rassistischen
Staatswissenschaftler Ludwig Gumplowicz, den er auf der allzu dünnen Grundlage
anstellt, dass beide in der einen oder anderen Form einen Naturalismus
vertreten hätten.\footnote{Vgl.  Voegelin, Autoritärer Staat, S.  135.  Dort
  schreibt Voegelin wörtlich: "`Die naturwissenschaftliche Rassentheorie hat
  bei ihm [Gumplowicz] eine ganz ähnliche Funktion der Geistzerstörung wie
  später bei Kelsen die materialistische Geschichtsauffassung."'}

Zusammenfassend ist zu Voegelins Auseinandersetzung mit der Reinen Rechtslehre
im "`Autoritären Staat"' wohl folgendes zu sagen: Die philosophisch
erkenntnistheoretische Einordnung von Kelsens Reiner Rechtslehre in den
Neukantianismus und ihre Kritik aus einer empirischen Perspektive (die bei
Voegelin zwar ontologisch verbrämt auftritt, aber durch einen später folgenden
Hinweis auf die empirisch ausgerichteten Rechtswissenschaftler Felix Stoerk
und Friedrich Tezner immerhin geistesgeschichtlich plausibel unterfüttert
wird\footnote{Vgl.  Voegelin, Autoritärer Staat, S. 136-143.}) ist
aufschlussreich und zeigt einige vielleicht nicht genügend reflektierte
Voraussetzungen der Reinen Rechtslehre auf. Weiterhin trägt Voegelin eine
Reihe immanenter Kritikpunkte gegen die Reine Rechtslehre zusammen, die
zumindest ernsthafte Berücksichtigung verdienen, wenn sie sich auch
größtenteils aus dem System der Reinen Rechtslehre heraus schlüssig
beantworten lassen. Weniger überzeugend und keinesfalls immer fair erscheint
dagegen Voegelins politische Beurteilung der Reinen Rechtslehre. Nicht nur,
dass er nicht immer sauber zwischen der Reinen Rechtslehre als solcher und dem
politischen Engagement ihres Erfinders Hans Kelsen unterscheidet, auch die
Rückführung von (vermeintlichen) politischen Irrtümern auf weltanschauliche
Fehlleistungen ("`positivistische Geistzerstörung"') ist in ihrer
undifferenzierten Form nicht nachvollziehbar und vertieft eine politische
Auseinandersetzung ebenso gewollt wie unnötig ins Grundsätzliche.  Nicht, dass
Voegelin einen anderen politischen Standpunkt einnahm als Kelsen und ihn von
diesem Standpunkt aus kritisiert, ist ihm vorzuwerfen. Doch welchen Standpunkt
Voegelin einnahm, und aus welchen Gründen er es tat, hinterlässt einen
zweideutigen Eindruck. Zwar darf nicht vergessen werden, dass der autoritäre
Staat Österreichs, für den Voegelin sich engagiert hat, anders als
Nazi-Deutschland nicht exzessiv verbrecherisch war.  (Insofern erübrigt sich
eine moralische Kritik, wie sie bei deutschen Gelehrten dieser Zeit wie Carl
Schmitt oder Arnold Gehlen angebracht sein könnte.)  Aber Voegelins Anknüpfung
an die autoritäre und totalitäre Staatsphilosophie seiner Zeit führt dazu,
dass er die österreichische Demokratie nicht bloß aus pragmatischen Gründen
ablehnte, etwa weil sie schlecht funktioniert hätte,\footnote{So die Deutung
  Sigwarts, die mir verfehlt erscheint (Vgl. Sigwart, a.a.O., S. 214).
  Insbesondere irrt sich Sigwart, wenn er glaubt, dass Voegelin "`am Postulat
  der Werturteilsfreiheit"' festhält (ebd.).  Die politische Stellungnahme
  zugunsten des Schuschnigg-Regimes ist in Voegelins "`Autoritärem Staat"'
  deutlich herauszuhören und auch die ätzende Kritik an der Reinen Rechtslehre
  -- Voegelin spricht in Bezug auf die Reine Rechtslehere immerhin von einem
  "`System der Kampfbegriffe"' (Voegelin, Autoritärer Staat, S. 116) und einer
  "`Verfallserscheinung"' (Voegelin, Autoritärer Staat, S. 157) -- ist in
  hohem Maße politisch motiviert.}  sondern vor allem aus prinzipiellen
Erwägungen, weil sie seinen damaligem Auffassungen von der Substanz und dem
Wesen eines modernen Nationalstaates widersprach.\footnote{Dass es mit England
  und Frankreich erfolgreiche moderne europäische Demokratien gab, erklärte
  Voegelin sich damit, dass auch diese Demokratien "`Totalitätselemente"'
  enthielten und im übrigen ihre "`Totalitätskämpfe"' schon hinter sich bzw.
  noch vor sich hätten.  (Vgl.  Voegelin, Autoritärer Staat, S. 4.) } Als eine
Apologie in eigener Sache muss man Voegelins spätere Behauptung werten, der
"`Autoritäre Staat"' sei sein erster größerer Versuch gewesen, die "`Rolle der
Ideologien linker wie rechter Couleur"' zu erfassen.\footnote{Vgl.  Voegelin,
  Autobiographical Reflections, S. 69.}  Die linken Ideologien behandelt
Voegelin im "`Autoritären Staat"' gar nicht und an die rechten Ideologien
knüpft er mit gewissen Vorbehalten sogar an.  Kritisiert wird allein die
liberale Staatsauffassung Hans Kelsens.\footnote{Erfreulich klar wird dies von
  Henkel hervorgehoben.  Vgl.  Henkel, Positivismuskritik und autoritärer
  Staat, a.a.O., S.  63.}


\section{Kelsens Voegelin-Kritik}

War bisher die Rede davon, wie Kelsens Reine Rechtslehre von Eric Voegelin
beurteilt wurde, so soll nun wenigstens kurz die andere Seite beleuchtet
werden, nämlich wie Hans Kelsen Voegelins Philosophie einschätzte. Die
Diskussion dreht sich dabei nicht mehr um die Reine Rechtslehre, sondern
diesmal um Voegelins spätere politische Philosophie, wie er sie in der 1952
erschienen "`Neuen Wissenschaft der Politik"' dargelegt hat. Zu diesem Werk
hat Hans Kelsen eine ausführliche Kritik verfasst, die er jedoch
unveröffentlicht ließ.  Erst fünfzig Jahre später ist sie herausgegeben
worden.\footnote{Hans Kelsen: A New Science of Politics. Hans Kelsen's Reply
  to Eric Voegelin's "`New Science of Politics"'. A Contribution to the
  Critique of Ideology (Ed. by Eckhart Arnold), Heusenstamm 2004.}

Nicht zuletzt, weil Kelsens Kritik so lange unveröffentlicht blieb, ranken
sich darum einige Legenden, auf die kurz eingegangen werden muss. Die erste
dieser Legenden hat Voegelin selbst in die Welt gesetzt. Kelsen hatte Voegelin
seine Kritik zugeschickt, wie er es üblicherweise tat, wenn er zu der Theorie
eines zeitgenössischen Wissenschaftlers eine kritische Abhandlung
verfasste.\footnote{Vgl.  Métall, Hans Kelsen, S. 68.} In seiner
Autobiographie nun behauptet Voegelin, dass er Kelsen zunächst brieflich und
später noch deutlicher durch gemeinsame Bekannte gewarnt habe, dass Kelsens
"`Auffassung von den historischen und politischen Problemen, die mit der
Thematik verknüpft sind, unzureichend"' wäre, und dass "`eine Veröffentlichung
eher Kelsens als seinen Ruf schädigen würde."'\footnote{Voegelin,
  Autobiographical Reflections, S. 81.}  Voegelin knüpft an diese Bemerkung
die Vermutung, dass Kelsen aufgrund seiner Warnungen die Kritik
unveröffentlicht gelassen habe.  Diese Vermutung wird von Günther Winkler in
seinem Geleitwort zur Neuauflage von Voegelins "`Autoritärem Staat"' von 1997
dankbar aufgegriffen.  Kelsen habe, so heißt es bei Winkler, seine Kritik
möglicherweise deshalb unveröffentlicht gelassen, weil er eingesehen habe,
dass er als Positivist bei geisteswissenschaftlichen Themen nicht mitreden
könne.\footnote{Vgl. Winkler, Geleitwort, S. XXVI. -- Vgl. auch Erika
  Weinzierl: Historical Commentary on the Period (translated from German by
  Fred Lawrence), in: The collected Works of Eric Voegelin.  Volume 4. The
  Authoritarian State. An essay on the Problem of the Austrian State.
  (Translated by Ruth Hein, edited by Gilbert Weiss), Columbia and London
  1999, S. 10-38 (S. 30).} Und Winkler fügt noch eine zweite Legende hinzu,
die vermutlich auf eine Bemerkung Voegelins in einem Brief an Kelsen zurück
geht, und die besagt, dass Hans Kelsen darüber verstimmt gewesen wäre, dass
sein Schüler Voegelin sich von ihm, dem "`Lehrer"', wissenschaftlich (wie auch
politisch) emanzipiert hätte.\footnote{Vgl.  ebd., S.  XXVII-XXIX, S
  XXXI-XXXII.  -- Vgl. den Brief Voegelins and Hans Kelsen vom 10. Februar
  1954.}

Die erste dieser Legenden ist schon deshalb sehr fragwürdig, weil Kelsen ja
auch nach Voegelins Hinweis auf seine vermeintliche Inkompetenz weiterhin
eifrig zu geisteswissenschaftlichen Themen gearbeitet und veröffentlicht hat.
Die Gründe für die Nichtveröffentlichung der Kritik sind nicht geklärt,
könnten aber damit zusammenhängen, dass die Voegelin-Kritik Teil eines
größeren Werkes mit dem Titel "`Religion without God"' hätte werden sollen,
das Kelsen schließlich unveröffentlicht ließ, weil er von der zentralen These
des Werkes, dass Religion einen Götterglauben voraussetzt, nicht mehr
überzeugt war.\footnote{Vgl.  Métall, S. 91.}  Um eine unabhängige
Veröffentlichung der Voegelin-Kritik hat er sich dann nicht mehr bemüht.

Was die zweite Legende betrifft, so gibt es nur einen einzigen
Anhaltspunkt, der sie stützt. Dieser Anhaltspunkt ist ein, wenn man so
will, etwas giftiger Brief, den Kelsen an William Rappard vom Institut
Universitaire des Hautes Etudes in Genf geschrieben hat, nachdem
Voegelin, den Kelsen zuvor noch wärmstens empfohlen
hatte,\footnote{Vgl. den Brief von Hans Kelsen and William Rappard,
  Wien, 10. Mai 1930, Archiv der Universität Wien.} dort als Dozent
abgelehnt worden worden war. Kelsen schreibt darin, dass ihn das
"`Urteil nicht allzusehr überrascht"' habe. Dann heißt es:

\begin{quote}
  "`Herr Dr. V. [Voegelin] hat in der letzten Zeit eine
  wissenschaftliche Richtung eingeschlagen, für die ich auch kein
  rechtes Verständnis mehr habe.  Seine Gedankenführung verliert immer
  mehr an Klarheit und seine Phraseologie stellt an den aufmerksamen
  Leser und Hörer Anforderungen, die zu erfüllen, man nicht eigentlich
  mehr verpflichtet ist. Herr Dr.  V. gerät leider immer mehr unter
  den Einfluss gewisser geistiger Strömungen, die in Deutschland ...
  stetig zunehmen, und in denen es Mode wird, an Stelle exakter
  Begriffsbestimmung und empirischer Untersuchung nur an Gefühlen
  orientierte, romantische Konstruktion zu setzen."'\footnote{Brief
    von Hans Kelsen an William Rappard, Köln, 16. Dezember 1930,
    Archiv der Universität Wien.}
\end{quote}

Günther Winkler vermutet, dass dieser Brief Ausdruck eines Einstellungswandels
Kelsens gegenüber Voegelin ist und unmittelbar durch Voegelins Aufsatz über
die Verfassungslehre Carl Schmitts hervorgerufen wurde.\footnote{Vgl. Winkler,
  Eric Voegelin und Hans Kelsen. , a.a.O., S. 17. -- Vgl. Winkler, Geleitwort,
  S. XIII, S. XXVII-XXVIII. -- Vgl. Eric Voegelin, Die Verfassungslehre von
  Carl Schmitt, a.a.O.} Sehr viel wahrscheinlicher ist aber, dass es der
Einfluss des George-Kreises auf Voegelin war, der Kelsen zu der oben zitierten
Bemerkung veranlasst hat. Voegelins Begeisterung für die Schriften des
George-Kreises befand sich in den Jahren 1929/30 auf einem Höhepunkt und
schlägt sich zu dieser Zeit sowohl in seinem Stil als auch in seinen Ansichten
nieder.\footnote{Vgl.  Thomas Hollweck: Der Dichter als Führer, S. 32. -- Vgl.
  Eric Voegelin: Max Weber, Rede vor der Wiener Soziologischen Gesellschaft
  vom 14. Juni 1930, in: Eric Voegelin: Die Größe Max Webers (hrsg. von Peter
  J. Opitz, München 1995, S.  29-47. Dort schreibt Eric Voegelin auf Seite 32
  wörtlich: "`Im Tiefpunkt der Zerrüttung, der auch die Sprache verfallen war,
  beginnt die allmähliche Neugewinnung des Bildungsgutes in Philosophie und
  Geschichte und ersteht der Schöpfer der Sprache in Stefan George. Das Wunder
  der wiederholten Erneuerung wird viel beredet, und die Besten glauben an die
  ewige Jugend unseres Volkes als sein auszeichnendes Glück vor den anderen
  Völkern. Nötig ist die Gabe gewiß, und heute mehr als je zuvor."'}  Schon
einige Monate zuvor hatten sich Leopold von Wiese und Marianne Weber
anlässlich von Voegelins Rede über Max Weber ausgesprochen kritisch über
diesen Zug von Voegelin geäußert.\footnote{Vgl.  den Brief von Leopold von
  Wiese an Eric Voegelin vom 21. Juni 1930, in: Eric Voegelin: Die Größe Max
  Webers, a.a.O., S. 48-50. Dort schreibt Leopold von Wiese: "`Aber in einem
  Punkte glaube ich pedantisch sein zu müssen, weil ich in ihrer Duldung eine
  große allgemeine Gefahr sehe. Das ist die romantische Klage über die
  angeblich zersetzenden und auflösenden Wirkungen des Verstandes. Gerade Max
  Weber hätte, wie sie ja zwischen den Zeilen andeuten, Ihnen ganz gehörig
  widersprochen. Wenn diese Anschuldigung des Verstandes und die
  Verherrlichung des bloßen Glaubens eine Einzelerscheinung wäre, würde ich
  keineswegs widersprechen. Aber das große Unglück unser Geisteskultur in
  Deutschland hängt ja mit diesem von 90\% aller jüngeren Leute vorgetragenen
  Jammern über den Verstand zusammen. Was heute von Leuten, deren Beruf und
  Aufgabe es ist, der Wissenschaft zu dienen, gesündigt wird in Anklagen über
  den Wissenschaftsgeist und in unbewußter Verherrlichung der
  Unwissenschaftlichkeit, das schreit zum Himmel. Und ich kann für meine
  Person nicht die Hand reichen, um diese Untergrabungen des Denkens zu
  kultivieren."' (S.  49). -- In ganz ähnlichem Sinne schreibt Marianne Weber
  an Eric Voegelin: "`Ich finde zwar manche Ihrer Deutungen, die, wie mir
  scheint, stark von Wolters Buch beeinflußt sind, dessen Deutungen M.W.s {\em
    sehr} verzerrt sind und mir {\em sehr} auf die Nerven fallen, schief.
  Offenbar stehen Sie z.Z.  stark unter dem Einfluß von St.  George und sind
  von dorther, d.h.  von dessen Wertungen u. `Gestalt' her orientiert? Bei
  M.W. von `Glaubenslosigkeit' oder Lähmung des Handelns durch den Verstand zu
  sprechen, geht m.E.'s wirklich nicht, ..."' (Brief an Eric Voegelin vom 3.
  Juli 1930, ebd., S. 57-58.  Hervorhebungen in Original).)}  Wenn Hans Kelsen
in seinem Brief also auf die Auswüchse von Voegelins George-Begeisterung
angespielt haben sollte, dann war er jedenfalls nicht der Einzige, der sie
ungenießbar fand.

Was nun die inhaltliche Auseinandersetzung betrifft, so geht es um Folgendes:
Voegelin hatte in der zunächst als Vorlesungsreihe gehaltenen "`New Science of
Politics"' (1952)\footnote{Eric Voegelin: The New Science of Politics. An
  Introduction, Chicago 1987 (1952).} eine neue Theorie der Repräsentation
entwickelt. (Der ursprüngliche Titel des Buches lautete daher auch "`Truth and
Representation"'.\footnote{Vgl. das Vorwort von Dante Germino zur Neuausgabe
  der ``New Science of Politics'', in: Voegelin, The New Science of Politics,
  a.a.O., S. v-ix (S. v).}) Umrahmt wird diese Theorie von einer scharfen
Positivismuskritik und Voegelins bekannter Gnosistheorie, der zufolge der
Totalitarismus durch das Überhandnehmen gnostischer politischer Bewegungen in
der Neuzeit zu erklären sei. Die Positivismuskritik knüpft nahtlos an
Voegelins Vorwürfe "`positivistischer Geistzerstörung"' im "`Autoritären
Staat"' an, nur dass Voegelin diesmal von "`destruktivem Positivismus"' redet
und sich nicht mehr vornehmlich auf die Reine Rechtslehre
bezieht.\footnote{Voegelin, The New Science of Politics, a.a.O., S. 3ff.} Sein
Angriff richtet sich besonders gegen die Forderung einer wertfreien
Wissenschaft, wie sie Max Weber ausführlich begründet hat, und die auch von
Hans Kelsen vertreten wird.  Voegelin glaubt, dass es Alternativen zur
wertfreien Wissenschaft gibt. Er verweist dazu auf die klassische politische
Wissenschaft von Platon und Aristoteles, sowie auf das christliche Denken
eines Thomas von Aquin.\footnote{Vgl. Voegelin, The New Science of Politics,
  a.a.O., S. 13ff.}

Die in der Neuen Wissenschaft der Politik ausgebreitete Repräsentationstheorie
Voegelins hat mit der in der Politikwissenschaft üblichen Bedeutungen des
Wortes "`Repräsentation"' nicht viel gemein.  Insbesondere geht es Voegelin
dabei nicht um den Begriff der repräsentativen, d.h. demokratischen Regierung.
Am ehesten handelt es sich noch um eine Theorie der Legitimation.  Voegelin
unterscheidet drei Arten von Repräsentation bzw.  Legitimation: "`Deskriptive
Repräsentation"' (Legitimation durch Verfahren),\footnote{Vgl. Voegelin, The
  New Science of Politics, a.a.O., S. 31ff.} "`Existentielle Repräsentation"'
(Legitimation durch wirksame Herrschaftsausübung und machtpolitische
Erfolge)\footnote{Vgl. Voegelin, The New Science of Politics, a.a.O., S.
  36ff.} und "`Wahrheitsrepräsentation"' (religiöse
Legitimation).\footnote{Vgl. Voegelin, The New Science of Politics, a.a.O., S.
  52ff.}  Diese drei Arten von Repräsentation bzw.  Legitimation sind nicht
als einander ausschließende Alternativen zu verstehen, sondern sie bauen
aufeinander auf.  Was die profanen Typen dieser Hierarchie angeht, also die
der "`deskriptiven"' und der "`existentiellen Repräsentation"', so knüpft
Voegelin teilweise an seine autoritären Vorstellungen aus den 30er Jahren an.
Die "`deskriptive Repräsentation"' wird am Beispiel des demokratischen
Wahlverfahrens deutlich abfällig beschrieben,\footnote{Vgl. Voegelin, The New
  Science of Politics, a.a.O., S. 32.  Dennoch hatte sich Voegelin in Amerika
  inzwischen mit der Demokratie angefreundet, auch wenn er sie sich ein wenig
  zurecht interpretierte. Vgl. Voegelin, Anamnesis, a.a.O., S. 351ff.} und
hinsichtlich der existentiellen Repräsentation spricht Voegelin davon, dass
ein Volk zu historischem Handeln bereit wird,\footnote{Voegelin bezeichnet ein
  solches Volk oder Herrschaftsgebilde dann in seiner eigenwilligen
  Terminologie als "`articulate for action"' (Voegelin, The New Science of
  Politics, a.a.O., S. 47) oder als "`in form for action in history"' (S.
  36).}  sobald es in der Lage ist, sich in einer historisch nachhaltigen
Weise militärisch bemerkbar zu machen.\footnote{Vgl. Voegelin, New Science of
  Politics, a.a.O., S. 36ff.} Neu ist daran nur, dass der Staat nicht mehr als
zentrale Bezugskategorie auftaucht, sondern Völker oder Kulturräume (z.B.  die
westliche Welt) oder ganze Kulturepochen. Eine sehr viel wichtigere Neuerung
gegenüber Voegelins früheren Schriften stellt dagegen die überaus starke
Betonung der Bedeutung, ja Notwendigkeit religiöser Legitimation unter dem
Titel der "`Repräsentation von Wahrheit"' dar.\footnote{Voegelin, The New
  Science of Politics, a.a.O., S. 52ff.} Sehr deutlich kommt dabei das
normative Anliegen Voegelins zur Geltung: Nicht nur darauf, dass die
politische Ordnung sich auf irgendwelche kollektiv verbindlichen Symbole als
Ausdruck gemeinsamer religiöser Überzeugungen stützt, kommt es an, es muss
auch die richtige religiöse Wahrheit sein, auf die sie sich stützt.  Was die
richtige religiöse Wahrheit ist, und wie man zu ihr gelangen kann, ist eine
Frage die Voegelin dann bis zum Ende seines Lebens beschäftigt
hat.\footnote{Vgl. z.B. Voegelin, Anamnesis, a.a.O., S.  283ff.}

Die Folge, die eintritt, wenn statt intakter religiöse Erfahrungen
"`deformierte"' Erfahrungen den Kern der repräsentierten Wahrheit bilden, ist
der "`Gnostizismus"'.  Gemeint ist damit eine umfassende religiös motivierte
Weltverdammung in Kombination mit ausgeprägten Erlösungshoffnungen. Werden
diese Erlösungshoffnungen zum Gegenstand einer politischen Programmatik, dann
resultieren daraus sehr destruktive, im schlimmsten Fall totalitäre politische
Bewegungen. Voegelins wichtigstes historisches Beispiel ist der
Puritanismus.\footnote{Vgl. Voegelin, The New Science of Politics, a.a.O., S.
  133ff. Das Beispiel verdeutlicht die Willkür von Voegelin
  geistesgeschichtlichen Beurteilungen, denn gerade die von Voegelin so
  geschätzte politische Ordnung der Vereinigten Staaten von Amerika kann man
  ja in gewisser Weise zu den kulturellen Leistungen des Puritanismus zählen.}
In der jüngeren Zeit zählt Voegelin nun beinahe alle modernen politischen
Bewegungen zur Gnosis, also nicht nur, was noch nachvollziehbar wäre,
Faschismus und Kommunismus, sondern unter anderem auch diejenigen liberalen
Kräfte im Amerika seiner Zeit, die sich seiner Ansicht nach dem Kommunismus
nicht entschieden genug entgegen stellen.\footnote{Vgl.  Voegelin, New Science
  of Politics, S.  162ff.}

Wie antwortet Kelsen auf diesen Entwurf einer "`Neuen Wissenschaft der
Politik"'? Zunächst einmal stellt Kelsen fest, dass es sich keineswegs um eine
neue, sondern in Wirklichkeit um den Rückfall in eine sehr alte, nämlich
religiöse Form der Politikbegründung handelt.\footnote{Vgl. Kelsen, A New
  Science of Politics, a.a.O., S. 12.} Voegelins Positivismuskritik hält er
entgegen, dass sie ein Zerrbild des Positivismus entwirft (z.B. stützen sich
nicht alle positivistischen Philosophien auf die Methoden der mathematischen
Naturwissenschaften, wie Voegelin behauptet; der Rechtspositivismus Kelsens
ist ein Gegenbeispiel).\footnote{Vgl. Kelsen, A New Science of Politics,
  a.a.O., S. 13.}  Zudem ist die Alternative, die Voegelin skizziert, eine
normative Ordnungswissenschaft, die zur Wertbegründung auf das platonische
"`Agathon"' oder die thomistische "`ratio aeterna"' zurückgreift, kaum
haltbar, da sowohl das platonische "`Agathon"' als auch die "`ratio aeterna"'
Leerformeln sind, die man mit welchen Werten auch immer auf\/füllen
kann.\footnote{Vgl. Kelsen, A New Science of Politics, a.a.O., S. 15/16, S.
  21/22.}

Ebenso kritisch äußert sich Kelsen zu Voegelins "`Repräsentationstheorie"'.
Neben der unüblichen Verwendung des Ausdrucks "`Repräsentation"' durch
Voegelin merkt er kritisch an, dass Voegelins "`existentielle Repräsentation"'
ausgesprochen militante bzw. sogar militaristische Vorstellungen von Politik
verkörpert.\footnote{Vgl. Kelsen, A New Science of Politics, S. 64.}  Der
Rückgriff auf in diesem Zusammenhang im Grunde abwegige antike Quellen wie die
Geschichte der Langobarden von Paulus Diaconus, um den Begriff der
existentiellen Repräsentation zu motivieren, erscheint Kelsen dabei als
Ausdruck eines unnötigen wissenschaftlichen Imponiergehabes.\footnote{Vgl.
  Kelsen, A New Science of Politics, a.a.O., S. 50/51.} Besonders deutlich
lehnt Kelsen, wie zu erwarten, Voegelins Ausführungen zur
"`Wahrheitsrepräsentation"' ab.\footnote{Vgl. Kelsen, A New Science of
  Politics, a.a.O., S. 53ff.}  Der heikle Punkt ist, dass Voegelin so etwas
wie eine objektive religiöse Wahrheit voraussetzt, eine wissenschaftlich
natürlich unhaltbare Voraussetzung.

Sehr hart geht Kelsen auch mit Voegelins Gnosis-Theorie ins Gericht.  Es
gelingt ihm mühelos zahlreiche Irrtümer und Fehlinterpretationen der
historischen Quellen bei Voegelin nachzuweisen.\footnote{Vgl.  Kelsen, A New
  Science of Politics, a.a.O., S. 76ff.} Wenn Voegelin am Schluss seines
Werkes den Gnosis-Vorwurf dann ebenso wahllos wie aggressiv gegen die
Politiken von Roosevelt und Truman, gegen Rüstungskontrolle und kollektive
Sicherheit richtet, dann ist das für Kelsen nicht anders als wenn "`auf dem
niedrigsten Niveau politischer Propaganda diejenigen, die mit der eigenen
Meinung nicht übereinstimmen, als Kommunisten beschimpft
werden."'\footnote{Kelsen, A New Science of Politics, a.a.O., S. 107.
  [Übersetzung von mir, E.A.] In der Tat scheint Voegelin die geistige
  Atmosphäre der McCarthy Ära geschätzt zu haben. Vgl. dazu Eric Voegelin: Die
  geistige und politische Zukunft der westlichen Welt, München 1996 (1959), S.
  33/34.}

Angesichts der Tatsache, dass Kelsens Antwort auf Voegelins "`Neue
Wissenschaft der Politik"' durch und durch eine Fundamentalkritik darstellt,
könnte man geneigt sein, sie vor allem als den Ausdruck sehr gegensätzlicher
philosophischer Haltungen, der agnostischen Überzeugung Kelsens einerseits und
der zunehmend religiösen Einstellung Voegelins andererseits, aufzufassen. Aber
damit würde man Kelsens Kritik nicht gerecht werden. Denn selbst wenn man die
Frage der Existenz transzendenter Wahrheiten als eine nie endgültig zu
beantwortende philosophische Menschheitsfrage wertet, so wird man doch
erwarten dürfen, dass ein wissenschaftliches Werk, welches auf eine religiöse
Glaubensgrundlage aufbaut, wenigstens in seinen profanen Teilen die üblichen
Maßstäbe argumentativer Schlüssigkeit, sachlicher Genauigkeit und
hermeneutischer Sorgfalt berücksichtigt. In dieser Hinsicht gelingt es Kelsens
sehr detaillierter Kritik zahlreiche Schwächen von Voegelins "`Neuer
Wissenschaft der Politik"' herauszuarbeiten.\footnote{Vgl. Eckhart Arnold:
  Voegelins "`Neue Wissenschaft"' im Lichte von Kelsens Kritik, Nachwort zu:
  Hans Kelsen: A New Science of Politics. Hans Kelsen's Reply to Eric
  Voegelin's "`New Science of Politics"'. A Contribution to the Critique of
  Ideology (Ed. by Eckhart Arnold), Heusenstamm 2004, S. 111/112, S. 118ff.}
Und was die transzendenten Wahrheiten selbst betrifft, so ist, wenn der
Begriff des transzendenten Seins schon in die Wissenschaft eingeführt wird,
die Frage keineswegs mehr vorlaut, ob die Existenz einer solchen
transzendenten Seinsphäre auch bewiesen werden kann.

Eine Antwort Voegelins auf Kelsens ausführliche Kritik ist leider nicht
erhalten. Die letzte Aussprache zwischen Kelsen und Voegelin über diese Kritik
fand nur mündlich statt.\footnote{Das Treffen fand am 23. August 1954 in
  Cambridge, Massachussetts statt, wie aus Voegelins knapper brieflicher
  Mitteilung an Alfred Schütz vom 24. August 1954 hervor geht. Der Brief ist
  abgedruckt in: Alfred Schütz / Eric Voegelin: Eine Freundschaft, die ein
  Leben gehalten hat.  Briefwechsel 1938-1959, S. 504. -- Winkler schreibt in
  seinem Geleitwort, dass Kelsen, "`offensichtlich schwer getroffen"' von
  Voegelins Kritik an seiner Schrift "`Was ist Gerechtigkeit?"', die Kontakte
  beendet hätte. (Vgl. Winkler, Geleitwort, S.  XXV.) Es bleibt unklar, woraus
  Winkler schließt, dass Kelsen die Kontakte beendet hätte. Dagegen spricht
  gerade, dass Voegelin und Kelsen sich am 23. August noch einmal getroffen
  haben.  Zudem stammt der letzte Brief des Briefwechsels von Kelsen. Der
  Brief ist auf den 27. Juli 1954 datiert und lag offenbar dem Manuskript von
  Kelsens ausführlicher Kritik an Voegelins "`New Science of Politics"' bei.}
Dennoch ist der kurze Briefwechsel, der sich zuvor zwischen Kelsen und
Voegelin entsponnen hat, wissenschaftlich aufschlussreich, denn er enthält
eine Kritik Voegelins an Kelsens Schrift "`Was ist Gerechtigkeit?"', die ein
Schlaglicht auf die unterschiedlichen Denkweisen beider wirft.  Nirgendwo
sonst tritt der Gegensatz zwischen Kelsens und Voegelins wissenschaftlicher
Herangehensweise derart zugespitzt auf. Es lohnt sich daher kurz darauf
einzugehen.

In seiner Schrift "`Was ist Gerechtigkeit?"'\footnote{Hans Kelsen: Was
  ist Gerechtigkeit?, 2. Auflage, Wien 1975.} untersucht Hans Kelsen
die klassischen Gerechtigkeitstheorien der abendländischen Philosophie
und kommt zu dem Ergebnis, dass es keiner einzigen gelingt, die
Richtigkeit der vertretenen Gerechtigkeitsvorstellung zu begründen.
Eingeleitet wird die Schrift von einer kurzen Beschreibung des
Gesprächs zwischen Jesus und Pontius Pilatus, in dessen Verlauf
Pontius Pilatus die berühmte Frage stellt: "`Was ist Wahrheit?"' Nach
Kelsens Interpretation geht es in dem Gespräch um
Gerechtigkeit.\footnote{Kelsen, Was ist Gerechtigkeit?, a.a.O., S. 1.}
Voegelin setzt ihm dagegen in seinem Brief unter Heranziehung
theologischer Kommentare aufwendig auseinander, dass es in dem
Gespräch nicht um Gerechtigkeit sondern um "`Wahrheit"'
geht.\footnote{Brief von Eric Voegelin an Hans Kelsen, Baton Rouge, 7.
  März 1954, Eric-Voegelin-Archiv München.} Hat Voegelin Kelsen, wie
er es in einer höflich umwundenen Form durchblicken lässt, bereits auf
der ersten Seite bei einem groben Schnitzer ertappt? Dazu ist zu
sagen, dass Voegelin unter theologischen Gesichtspunkten sicherlich
Recht hat, es geht um "`Wahrheit"'. Nur ist es für diese Art
religiöser "`Wahrheit"' charakteristisch, dass sie eine umfassende
"`Wahrheit"' ist, die unter anderem auch eine Moral- und
Gerechtigkeitskomponente umfasst. Da es Kelsen aber gerade auf das
Problem der Begründung von Gerechtigkeit ankommt, ist es legitim, wenn
er diesen Aspekt aussondert und die Begegnung zwischen Jesus und
Pilatus so deutet, dass es dabei um die "`wahre"' Gerechtigkeit
geht.\footnote{Diese Intention Kelsens wird von Winkler nicht genügend
  berücksichtigt, wenn er schreibt, Voegelin habe Kelsen eine
fehlerhafte
  Deutung der historischen Quellen nachgewiesen. Vgl. Günther Winkler:
  Die Reine Rechtslehre als Dekonstruktionismus?
  Geistesgeschichtliche Notizen zu einer grundlegenden Kontroverse
  zwischen Kelsen und Voegelin, in: Verfassungsstaatlichkeit.
  Festschrift für Klaus Stern zum 65.  Geburtstag (Hrsg. von Joachim
  Burmeister), München 1997, S. 122.}

Der Gegensatz, der sich hier zwischen Voegelin und Kelsen auftut, ist der
zwischen zwei unterschiedlichen, aber gleichermaßen legitimen
Herangehensweisen.  Kelsens Herangehensweise ist analytisch, indem er
untersucht inwieweit ein bestimmter historischer Standpunkt Antwort auf eine
bestimmte Frage, in diesem Fall die Frage nach der Gerechtigkeit, geben kann.
Dafür ist es unerlässlich, diesen Standpunkt im Hinblick auf die zu
untersuchende Sachfrage zu interpretieren.  Voegelin geht dagegen von einer
hermeneutischen Methode aus, bei der es darauf ankommt, einen Standpunkt so zu
verstehen, wie er gemeint ist, einschließlich, wie man vielleicht sogar
fordern müsste, der ganzen Verworrenheit, mit der er gemeint
ist.\footnote{Einschränkend muss allerdings darauf hingewiesen werden, dass
  Voegelin sich bei der Interpretation geistesgeschichtlicher Quellen gerade
  über die Autorintentionen häufig in einer kaum vertretbaren Weise
  hinwegsetzt.  Vgl. dazu Kelsen, A New Science of Politics, a.a.O., S.
  95/96.} Nun glaubt Voegelin aber, dass man gerade mit dieser hermeneutischen
Herangehensweise auch eine Antwort auf die Frage nach der Gerechtigkeit geben
kann, soweit sich überhaupt eine Antwort geben lässt. Voegelin meint nämlich,
dass die großen Philosophen, deren Unfähigkeit Gerechtigkeit zu begründen
Kelsen in seiner Abhandlung nachweist, in Wirklichkeit niemals einen solchen
Versuch unternommen haben. Vielmehr ist Gerechtigkeit das Ergebnis einer Art
von Seelenforschung, die Voegelin folgendermaßen beschreibt: "`Das Instrument
des Findens ist die Seele des Finders. ... Die Seelen sind, wenn sie
historisch manifest werden, die Seelen der großen Propheten, Nomotheten,
Philosophen und Heiligen. Und der Grund, warum man ihnen folgen soll ist nicht
in einer Norm zu finden, ... sondern im respondieren der verwandten
Seelen."'\footnote{Brief von Eric Voegelin an Hans Kelsen, Baton Rouge, 7.
  März 1954, Eric-Voegelin-Archiv München.} Aber hier liegt ein Trugschluss
vor.  Selbst wenn Voegelin recht damit hätte, dass dies die Art und Weise ist,
wie Gerechtigkeitsvorstellungen entstehen, so bleibt immer noch offen, warum
man sie als autoritativ gültig betrachten sollte. Eine Frage, die man allein
mit hermeneutischen Methoden nicht beantworten kann, sondern nur argumentativ.
(Und der Hinweis auf das "`respondieren der verwandten Seelen"' ist sicherlich
kein besonders stichhaltiges Argument.) Abgesehen davon bleibt der konkrete
Inhalt der Gerechtigkeitsvorstellung bei Voegelins Bemerkung einigermaßen
leer, ganz wie Kelsen dies Voegelin in seiner Kritik der "`Neuen Wissenschaft
der Politik"' schon vorgeworfen hat.\footnote{Vgl.  Kelsen, A New Science of
  Politics, a.a.O., S. 63.} Faktisch ist Voegelins Ethik daher genauso
relativistisch wie die von Kelsen, nur dass Voegelin dies nicht zugibt und
sich in dieser Lage für ein autoritätsgestütztes Modell der Moralbegründung
entscheidet, während Kelsen sich für die Werte der Aufklärung und liberale
Gerechtigkeitsideale entschieden hat.\footnote{Kelsen, Was ist Gerechtigkeit?,
  S. 40-43.}


\section{Eine letzte Begegnung: Kelsen, Voegelin und das Naturrecht}

Zu einer letzten Begegnung zwischen Kelsen und Voegelin kam es anlässlich
eines Symposiums über "`Naturrecht"', das 1962 in Salzburg abgehalten wurde,
organisiert von Franz-Martin Schmölz, einem Schüler Voegelins.\footnote{Vgl.
  Franz-Martin Schmölz (Hrsg.): Das Naturrecht in der politischen Theorie,
  Wien 1963, Vorwort.} Schmölz hatte neben anderen sowohl Kelsen als auch
Voegelin eingeladen, wobei Kelsen den ersten Vortrag hielt, Voegelin den
zweiten. Der Gegensatz könnte kaum größer sein.  Kelsens Vortrag besteht zum
größten Teil aus einer historischen Untersuchung von
Naturrechtslehren.\footnote{Hans Kelsen: Die Grundlage der Naturrechtslehre,
  in: Schmölz, a.a.O., S. 1-38.} Seine zentrale These lautet, dass das
Naturrecht ohne theologische Voraussetzungen nicht zu begründen ist. Wer die
theologische Voraussetzung ablehnt, für den kann daher auch das Naturrecht
nicht bindend sein. Kelsen beginnt bei Aristoteles, der nach seiner
Interpretation noch keine eigentliche Naturrechtslehre entwickelt hat. Erst
der Aristoteliker Thomas von Aquin wird -- unter expliziten theologischen
Voraussetzungen -- eine Naturrechtslehre vertreten. Die Erörterung wird von
Kelsen bis zu zeitgenössischen theologisch begründeten Naturrechtstheorien
fortgeführt. In jedem Fall zeigt sich für ihn, dass die theologische
Voraussetzung unerlässlich ist.  Das aufklärerische Vernunftrecht, wie man es
bei Kant findet, kann Kelsen durch eine theologische Lesart Kants mit seiner
These vereinbaren.

Voegelin legt eine vollkommen gegensätzliche Interpretation
vor.\footnote{Eric Voegelin: Das Rechte von Natur, in Schmölz, a.a.O.,
  S. 38-51.} Er lobt Kelsens Vortrag als eine sorgfältige
dogmengeschichtliche Untersuchung, lässt aber wenig Zweifel daran,
dass auf diese Weise das Wesentliche nicht erfasst wird. Das
Wesentliche sind nach Voegelins Ansicht nicht die Dogmen, sondern die
(seelischen) "`Erfahrungen"' davon, was das "`Rechte von Natur"' ist.
Ganz im Gegensatz zu Kelsen findet er ein Naturrecht in diesem Sinne
schon bei Aristoteles. Die Art von "`Naturrecht"', die er aus
Aristoteles' Nikomachischer Ethik heraus holt, gestaltet sich
folgendermaßen: Ein ein für allemal feststehendes Naturrecht, das man
in Form von Gesetzen fassen könnte, gibt es nicht. Was das Rechte von
Natur ist, dafür sind im konkreten Fall die "`repräsentativen
Menschen"' maßgeblich. Ein solcher "`repräsentativer Mensch"' oder
{\em spoudaios}, wie Voegelin einen aristotelischen Ausdruck
verwendend auch gerne sagt, gibt die richtige ethische Haltung vor,
die die gewöhnlichen Menschen als verbindlich anzuerkennen
verpflichtet sind. Es handelt sich um dieselbe Auffassung, die
Voegelin auch an anderen Stellen vertreten hat.\footnote{Vgl.
  Voegelin, Order and History. Volume Three, a.a.O., S. 299ff.} Auf
die naheliegende Frage, woran man einen "`repräsentativen Menschen"'
erkennt, und wie man ihn ggf. von einem Menschen unterscheidet, der
nur behauptet, ein "`repräsentativer Mensch"' zu sein, es aber nicht
ist, findet sich in Voegelins Vortrag allerdings keine Antwort.

In der sich an die Vorträge anschließenden Diskussion bekräftigt Kelsen noch
einmal im Gegensatz zu Voegelins Aristoteles-Interpretation, dass ein
veränderliches Naturrecht kein Naturrecht ist.\footnote{Vgl. Schmölz, a.a.O.,
  S. 128.} Voegelin hält dagegen, dass es ihm auch gar nicht um so etwas wie
absolute Werte gegangen sei. Gleichzeitig behauptet er aber, dass es objektive
Kriterien gäbe, nach denen man entscheiden könne, ob Marx oder ob Aristoteles
Recht hätte.\footnote{Vgl. Schmölz, a.a.O., S. 129-132.} Allerdings wird
Voegelin an dieser Stelle merklich undeutlich. Der Grund dafür dürfte darin
liegen, dass diese vermeintlich "`objektiven Kriterien"' bei Voegelin in
letzter Instanz auf innere Erlebnisse zurückgeführt werden, die nur
bewusstseinsphilosophisch zu erfassen sind, und die in einem rein
argumentativen Diskurs gar nicht ohne weiteres vermittelt werden können. In
diesem Sinne ist es zu verstehen, wenn Voegelin wenig später fordert, dass die
"`Basis für die Behandlung der philosophischen Problematik ..
selbstverständlich immer die Meditationspraxis sein"' müsse.\footnote{Schmölz,
  a.a.O., S. 137.}

\section{Schluss}

Inwiefern kann man Voegelin nach all dem als Schüler Hans Kelsens betrachten?
Sicherlich nicht in dem Sinne, dass Voegelin als Schüler Hans Kelsens die
Lehre seines Meisters ausgelegt und gegebenenfalls fortentwickelt hätte. Aber
insofern als zu jeder Philosophie auch ihre Kritik gehört und einen
wesentlichen Anteil daran hat, sie weiter zu entwickeln, kann man in Voegelins
Kritik an der Reinen Rechtslehre sehr wohl einen wichtigen Beitrag zur
"`Schule"' sehen. Dass gilt besonders auch für die Kritik der Reinen
Rechtslehre im "`Autoritären Staat"', die umfassendste und gründlichste
Auseinandersetzung Voegelins mit der Reinen Rechtslehre. Dass diese Kritik
teils politisch motiviert ist und überaus polemisch ausfällt, tut ihrer
wissenschaftlichen Bedeutung keinen Abbruch, denn der scharfe Ton ist zum Teil
auch durch starke Argumente gedeckt (was man von den beinahe habituellen
Ausfälligkeiten gegen missliebige Auffassungen, die sich in vielen von
Voegelins nach der Emigration entstandenen Schriften finden, nicht immer
behaupten kann).

In der späteren wissenschaftlichen Auseinandersetzung zwischen
Voegelin und Kelsen über die "`Neue Wissenschaft der Politik"' kann
von einem Lehrer-Schüler Verhältnis keine Rede mehr sein. Auch nicht
in dem Sinne, dass Kelsens Voegelin-Kritik dadurch motiviert wäre,
dass sein "`Schüler"' Voegelin der "`Schule"' entlaufen wäre. Vielmehr
geht es um gegensätzliche philosophische Standpunkte (indem Kelsens
agnostischer Standpunkt Voegelins religiöser Orientierung
gegenübersteht) und um eine gegensätzliche politische Ausrichtung:
Hier das mit religiöser Note versehene aber immer noch eher
autoritäre Denken Voegelins, dort der aufgeklärte Liberalismus
Kelsens. Im übrigen war Voegelin zu der Zeit, als er die "`Neue
Wissenschaft der Politik"' schrieb, schon dabei seine eigene Schule zu
gründen, eine Schule, deren Angehörige viel entschiedener auf ihren
Meister eingeschworen waren (und zum Teil noch sind), als dies bei
Kelsen jemals der Fall gewesen ist.

\newpage

\section{Bibliographie}

\subsection{Voegelin über Kelsen (chronologisch geordnet)}

\setlength{\parindent}{0ex}

\setlength{\parskip}{3ex}

{\bf Voegelin, Eric}: Reine Rechtslehre und Staatslehre, in: Zeitschrift für
öffentliches Recht IV, 1925, S. 80-131.

{\bf Voegelin, Eric}: Zur Lehre von den Staatsformen, in: Zeitschrift für
öffentliches Recht VI, 1927, S. 268-276.

{\bf Voegelin, Eric}: Kelsens Pure Theory of Law, in: The collected
Works of Eric Voegelin. Volume 7.  Published Essays 1929-1928. (Ed.
Thomas W. Heilke and John von Heyking), Columbia and London 2003, S.
182-192, zuerst in: Political Quarterly 42, no. 2 (1927), S. 268-76.

{\bf Voegelin, Eric}: Die Souveränitätstheorie Dickinsons und die Reine
Rechtslehre, Zeitschrift für öffentliches Recht VIII, 1928, S. 413-434.

{\bf Voegelin, Eric}: Die Einheit des Rechtes und das soziale
Sinngebilde Staat, in: Revue Internationale de la Théorie du Droit 5,
1930, S. 58-89.

{\bf Voegelin, Eric}: Die Verfassungslehre von Carl Schmitt. Versuch einer
konstruktiven Analyse ihrer staatstheoretischen Prinzipien, in: Zeitschrift
für öffentliches Recht XI, 1931, S. 89-109.

{\bf Voegelin, Eric}: Rasse und Staat, Tübingen 1933.

{\bf Voegelin, Eric}: Der Autoritäre Staat. Versuch über das österreichische
Staatsproblem, Wien 1936.


\subsection{Andere}

\setlength{\parindent}{0ex}

\setlength{\parskip}{3ex}

{\bf Arnold, Eckhart}: Voegelins "`Neue Wissenschaft"' im Lichte von Kelsens
Kritik, Nachwort zu: Hans Kelsen: A New Science of Politics. Hans Kelsen's
Reply to Eric Voegelin's "`New Science of Politics"'. A Contribution to the
Critique of Ideology (Ed. by Eckhart Arnold), Heusenstamm 2004.

%{\bf Baurmann, Michael}: Der Markt der Tugend. Recht und Moral in der
%liberalen Gesellschaft. Eine soziologische Untersuchung, Tübingen
%1996.

{\bf Blumenberg, Hans}: Die Legitimität der Neuzeit. Erneuerte Ausgabe,
Frankfurt am Main 1996.

{\bf Breuer, Stefan}: Ästethischer Fundamentalismus. Stefan George und der
deutsche Antimodernismus, Darmstadt 1995.

{\bf Brumlik, Micha}: Die Gnostiker. Der Traum von der Selbsterlösung
des Menschen, Frankfurt am Main 1992.

{\bf Chignola, Sandro}: "`Fetishism"' with the Norm and Symbols of Politics.
Eric Voegelin between Sociology and "`Rechtswissenschaft"', 1924-1938, München
1999.

{\bf Feichtinger, Johannes}: Wissenschaft zwischen den Kulturen.
Österreichische Hochschullehrer in der Emigration 1933-1945, Frankfurt am Main
2001.

{\bf Heimes, Claus}: Antipositivistische Staatslehre. Eric Voegelin und Carl
Schmitt zwischen Wissenschaft und Ideologie, München 2004.

{\bf Henkel, Michael}: Eric Voegelin zur Einführung, Hamburg 1998.

{\bf Henkel, Michael}: Positivismuskritik und autoritärer Staat. Die
Grundlagendebatte in der Weimarer Staatsrechtslehre und Eric Voegelins Weg zu
einer neuen Wissenschaft der Politik (bis 1938), München 2005.

{\bf Hollweck, Thomas}: Der Dichter als Führer. Dichtung und Repräsentanz
  in Voegelins frühen Arbeiten, München 1996.

{\bf Husserl, Edmund}: Die Krisis der europäischen Wissenschaften und die
transzendentale Phänomenologie, 3. Auflage, Hamburg 1996.

{\bf Kelsen, Hans}: Was ist Gerechtigkeit?, 2. Auflage, Wien 1975.

{\bf Kelsen, Hans}: Die Grundprobleme der Naturrechtslehre, in: Franz-Martin
Schmölz (Hrsg.): Das Naturrecht in der politischen Theorie, Wien 1963, S.
1-38.

{\bf Kelsen, Hans}: Hauptprobleme der Staatsrechtslehre. Entwickelt aus der
Lehre vom Rechtssatze.  Nachdruck der 2. Auflage von 1923, Aalen 1960.

{\bf Kelsen, Hans}: A New Science of Politics. Hans Kelsen's Reply to Eric
Voegelin's "`New Science of Politics"'. A Contribution to the Critique of
Ideology (Ed. by Eckhart Arnold), Heusenstamm 2004.

{\bf Kelsen, Hans}: Das Problem der Souveränität und die Theorie des
Völkerrechts. Beitrag zu einer Reinen Rechtslehre, Nachdruck der 2.Auflage von
1928, Aalen 1960.

{\bf Kelsen, Hans}: Reine Rechtslehre, Nachdruck der 2. Auflage, Wien 1992
(1960).

{\bf Kelsen, Hans}: Der soziologische und der juristische Staatsbegriff.
Kritische Untersuchung des Verhältnisses von Staat und Recht, Neudruck der 2.
Auflage von 1928, Aalen 1962.

{\bf Krasemann, Andreas}: Eric Voegelins politiktheoretisches Denken in den
Frühschriften, Erfurt 2002, auf:
http://www.db-thueringen.de/servlets/DerivateServelets/Derivate-1408/krasemann.html
(Zugriff: 8.5.2006).

{\bf Métall, Rudolf Aladár}: Hans Kelsen, Leben und Werk, Wien 1969.

{\bf Plessner, Helmut}: Rechtsphilosophie und Gesellschaftslehre,
Besprechung von Rasse und Staat von Eric Voegelin, Tübingen 1933, in:
Zeitschrift für Öffentliches Recht, XIV, 1934, S. 407-414.

{\bf Scheler, Max}: Die Stellung des Menschen im Kosmos, 14. Auflage, Bonn
1998 (1928).

{\bf Schmitt, Carl}: Der Begriff des Politischen, Hamburg 1933.

{\bf Schmitt, Carl}: Politische Theologie. Vier Kapitel zur Lehre von
der Souveränität, 7. Auflage, Berlin 1996 (1922).

{\bf Schmölz, Franz-Martin} (Hrsg.): Das Naturrecht in der politischen
Theorie, Wien 1963.

{\bf Schütz, Alfred / Voegelin, Eric}: Eine Freundschaft, die ein Leben
ausgehalten hat. Briefwechsel 1938-1959. (Hrsg. von Gerhard Wagner und Gilbert
Weiss), Konstanz 2004.

{\bf Sigwart, Hans-Jörg}: Das Politische und die Wissenschaft.
Intellektuell-biographische Studien zum Frühwerk Eric Voegelins,
Würzburg 2005.

{\bf Voegelin, Eric}: Anamnesis. Zur Theorie der Geschichte und
Politik, München 1966.

{\bf Voegelin, Eric}: The Austrian Constitutional Reform of 1929, in;
The collected Works of Eric Voegelin. Volume 8.  Published Essays
1929-1933. (Ed. Thomas W. Heilke and John von Heyking), Columbia and
London 2003, S. 148-179.

{\bf Voegelin, Eric}: Die Größe Max Webers (Hrsg. von Peter J.
Opitz), München 1995.

{\bf Voegelin, Eric}: History of Political Ideas, in: Paul Caringella
et al. (Ed.): The collected works of Eric Voegelin, Volumes 19-26,
Columbia and London, 1997.

{\bf Voegelin, Eric}: Order and History. Volume One. Israel and
Revelation, Baton Rouge / London, reprint 1981 (1956).

{\bf Voegelin, Eric}: Order and History. Volume Two. The World of the
Polis, Baton Rouge / London, reprint 1980 (1957).

{\bf Voegelin, Eric}: Order and History. Volume Three. Plato and
Aristotle, Baton Rouge / London, reprint 1983 (1957).

{\bf Voegelin, Eric}: Order and History. Volume Four. The Ecumenic
Age, Baton Rouge / London, reprint 1980 (1974).

{\bf Voegelin, Eric}: Order and History. Volume Five. In Search of
Order, Baton Rouge / London 1986.

{\bf Voegelin, Eric}: Ordnung und Geschichte 4, Die Welt der Polis
Gesellschaft, Mythos und Geschichte, München 2002. [Anmerkung: Die deutsche
Ausgabe ist in zehn Bänden erschienen und folgt daher in der Nummerierung
nicht dem amerikanischen Original. Der vierte Band der deutschen Ausgabe ist
ein Auszug aus "`Volume Two"' der amerikanischen Ausgabe.]

{\bf Voegelin, Eric}: Rasse und Staat, Tübingen 1933.

{\bf Voegelin, Eric}: Die Rassenidee in der Geistesgeschichte von Ray
bis Carus, Berlin 1933.

{\bf Voegelin, Eric}: Der autoritäre Staat. Ein Versuch über das
österreichische Staatsproblem, Wien 1936. -- (2. Auflage, hrsg. von
Günther Winkler, Wien / New York 1997).

{\bf Voegelin, Eric}: Das Rechte von Natur, in: Franz-Martin Schmölz (Hrsg.):
Das Naturrecht in der politischen Theorie, Wien 1963, S. 38-51.

{\bf Voegelin, Eric}: Die politischen Religionen, München 1993 (1938).

{\bf Voegelin, Eric}: The New Science of Politics. An Introduction, Chicago
and London, 1987 (1952).

{\bf Voegelin, Eric}: Wedekind. Ein Beitrag zur Soziologie der
Gegenwart, München 1996 (1921).

{\bf Voegelin, Eric}: The collected Works of Eric Voegelin. Volume 4.
The Authoritarian State. An essay on the Problem of the Austrian
State. (Translated by Ruth Hein, edited by Gilbert Weiss), Columbia
and London 1999.

{\bf Voegelin, Eric}: The collected Works of Eric Voegelin. Volume 7.
Published Essays 1929-1928. (Ed. Thomas W. Heilke and John von
Heyking), Columbia and London 2003.

{\bf Voegelin, Eric}: The collected Works of Eric Voegelin. Volume 8.
Published Essays 1929-1933. (Ed. Thomas W. Heilke and John von
Heyking), Columbia and London 2003.

{\bf Voegelin, Eric}: The collected Works of Eric Voegelin. Volume 9.
Published Essays 1934-1939. (Ed. Thomas W. Heilke), Columbia and London 2001.

{\bf Voegelin, Eric}: The collected Works of Eric Voegelin. Volume 31.
Hitler and the Germans (translated and edited by Detlev Clemens and Brendan
Purcell), Columbia and London 1999.

{\bf Voegelin, Eric}: The collected Works of Eric Voegelin, Volume 34.
Autobiographical Reflections. Revised Edition. (Ed. by Ellis Sandoz),
Columbia and London 2006.

{\bf Voegelin, Eric}: Die geistige und politische
  Zukunft der westlichen Welt, München 1996 (1959).

{\bf Weinzierl, Erika}: Historical Commentary on the Period
(translated from German by Fred Lawrence), in: The collected Works
of Eric Voegelin. Volume 4. The Authoritarian State. Ab essay on
the Problem of the Austrian State. (Translated by Ruth Hein, edited
by Gilbert Weiss), Columbia and London 1999, S. 10-38.

{\bf Winkler, Günther}: Erich Voegelin und Hans Kelsen.
Geistesgeschichtliche Notizen über eine wissenschaftliche Schüler- und
Lehrerbeziehung in der "`Wiener Schule der Reinen Rechtslehre"',
Archiv der Universität Wien.

{\bf Winkler, Günther}: Geleitwort, in: Eric Voegelin: Der
autoritäre Staat. Ein Versuch über das österreichische Staatsproblem
(Hrsg. von Günther Winkler), 2.  Auflage, Wien / New York 1997
(1936), S. V-XXXII.

{\bf Winkler, Günther}: Rechtswissenschaft und Rechtserfahrung, Wien /
New York 1994.

{\bf Winkler, Günther}: Die Reine Rechtslehre als
Dekonstruktionismus? Geistesgeschichtliche Notizen zu einer
grundlegenden Kontroverse zwischen Kelsen und Voegelin, in:
Verfassungsstaatlichkeit. Festschrift für Klaus Stern zum 65.
Geburtstag (Hrsg. von Joachim Burmeister), München 1997.


\subsection{Archivquellen}

Abschrift Referat über das Habilitationsgutachten des Dr. Erich
Voegelin, 21. Mai 1928, unterzeichnet in Maschinenschrift mit den
Namen Kelsen und Spann, Archiv der Universität.

Brief von Hans Kelsen and William Rappard, Wien, 10. Mai 1930, Archiv
der Universität Wien.

Brief von Hans Kelsen an William Rappard, Köln, 16. Dezember 1930, Archiv
der Universität Wien.

Brief von Eric Voegelin an Hans Kelsen, Baton Rouge, 10. Februar 1954,
Eric-Voegelin-Archiv München.

Brief von Eric Voegelin an Hans Kelsen, Baton Rouge, 7. März 1954,
Eric-Voegelin-Archiv München.

Brief von Hans Kelsen and Eric Voegelin, Newport, 27. Juli 1954, Hans Kelsen
Institut Wien.


\end{document}
