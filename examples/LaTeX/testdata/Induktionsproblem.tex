\documentclass[12pt, a4paper]{article}
\usepackage{microtype}
\usepackage[utf8x]{inputenc}
\usepackage{ucs} % unicode
\usepackage[T1]{fontenc}
\usepackage{t1enc}
\usepackage{type1cm}
\usepackage{setspace}

\usepackage{amsmath, amssymb}
\usepackage{natbib}

\usepackage{eurosym}  
\usepackage{graphicx}
\usepackage{rotating}

\usepackage{ifpdf}
\ifpdf
\usepackage{xmpincl}
\usepackage[pdftex]{hyperref}
\hypersetup{
    colorlinks,
    citecolor=black,
    filecolor=black,
    linkcolor=black,
    urlcolor=black,
    bookmarksopen=true,     % Gliederung öffnen im AR
    bookmarksnumbered=true, % Kapitel-Nummerierung im Inhaltsverzeichniss anzeigen
    bookmarksopenlevel=1,   % Tiefe der geöffneten Gliederung für den AR
    pdfstartview=FitV,       % Fit, FitH=breite, FitV=hoehe, FitBH
    pdfpagemode=UseOutlines, % FullScreen, UseNone, UseOutlines, UseThumbs 
}
\includexmp{Induktionsproblem}
\pdfinfo{
  /Author (Eckhart Arnold)
  /Title (Can the Best-Alternative-Justification solve Hume's Problem? On  the Limits of a Promising Approach)
  /Subject (Demonstration that the Best-Alternative-Justification proposed by Gerhard Schurz cannot fully solve the problem of induction)
  /Keywords (Evolution of Cooperation, Social Simulations, History of Simulations)
}
\fi

\numberwithin{equation}{section}

\sloppy

\begin{document} 
%\doublespacing

\title{Can the Best-Alternative-Justification solve Hume's Problem?
\\On the Limits of a Promising Approach}


\author{
{\large Eckhart Arnold}
\\{\normalsize University of Stuttgart, SimTech-Cluster}
\\{\small Eckhart.Arnold@philo.uni-stuttgart.de}
}

\date{February, 22nd 2010}

\maketitle

\begin{center}
{\em published in: Philosophy of Science 2010, Vol. 77, No 4, pp. 584-593.}
\end{center}

\begin{abstract}
In a recent {\em Philosophy of Science} article Gerhard Schurz
proposes meta-inductivistic prediction strategies 
as a new approach to Hume's Problem \cite[]{schurz:2008}. This comment
examines the limitations of Schurz's approach. 
It can be proven that
the meta-inductivist approach does not work any more if the meta-inductivists
have to face an infinite number of alternative predictors. With this
limitation it remains doubtful whether the meta-inductivist can
provide a full solution to the problem of induction.
\end{abstract}

\newpage

\tableofcontents 

\section{Introduction}

In a recent article on ``The Meta-inductivist's Winning Strategy in the
Prediction Game: A New Approach to Hume's Problem'' \cite[]{schurz:2008}
Gerhard Schurz proposes the {\em Best-Alternative-Justification} as a new
approach to the problem of induction. As acknowledged by Schurz, the
original idea goes back to Hans Reichenbach
\cite[]{reichenbach:1935, reichenbach:1938}. But Schurz furnishes this
idea with a new technical approach relying on a certain class of
prediction strategies which he calls ``meta-inductivists''. Given that
all attempts to solve the problem of induction have hitherto failed
\cite[]{howson:2000}, what should induce us to reconsider the sceptical
conclusion that there is no solution to the problem of induction? Prima
facie, Gerhard Schurz has a convincing answer to this question: Most of
the proposed solutions to the problem of induction tried to prove the
{\em reliability} of the inductive procedure. But Schurz, following
Reichenbach, merely tries to show the {\em optimality} of a specific
inductive strategy, namely his ``meta-inductivist'' strategy.
Demonstrating the optimality of an inductive prediction strategy is a
less ambitious task than demonstrating its reliability, because for an
inductive strategy to be reliable one would have to prove that it works
in any possible world. But an inductive strategy that is merely optimal
is allowed to fail in some possible worlds, as long as in the worlds
where it does not work all other possible prediction strategies are bound
to fail, too. Schurz does not raise the claim that he has solved the
problem of induction literally, but the closing paragraph of his article
suggests that he believes that he has at least provided a very good
candidate for a solution to the problem of induction \cite[p.
304]{schurz:2008}.\footnote{Regarding this claim, see also an earlier
presentation of Schurz' ideas on the 5th GAP conference \cite[p.
256]{schurz:2003}.}

Schurz discusses the problem of induction within the technical framework
of prediction games, where a number of players have to predict the next
event in a series of binary valued (0 or 1) or real valued (any real
number from 0.0 to 1.0) world events. By proving two theorems regarding
the optimality of the prediction strategies of the ``weighted meta
inductivist'' and the ``collective weighted meta inductivist'', Schurz is
at least able to give a partial solution for the problem of induction
that accounts for the case of finitely many prediction strategies. But,
as shall be demonstrated in the following, Schurz' technical approach
meets insurmountable limits once one tries to pass from a finite number
to an infinite number of prediction strategies.
This raises the philosophical question whether an optimality result
demonstrated for a finite number of prediction strategies might suffice
to answer the problem of induction. If not, then providing a full
solution to Hume's problem remains an open challenge.

In the following, I am going to briefly restate Schurz' central results and
then demonstrate that the results for prediction games with finitely many
strategies cannot be extended to prediction games with infinitely many
strategies. Finally, there will be a brief discussion of open questions
regarding Schurz' answer to Hume's problem.


\section{Schurz' basic approach}

The technical framework within which Schurz derives his results consists of 
a series of events which can either be 1 or 0 ({\em binary prediction game}) or
take any real value in the closed interval from 0.0 to 1.0 ({\em real valued
prediction game}) and of a set of prediction strategies that have to get as many
predictions of these events right as possible (in the binary prediction game) or that
have to predict the coming events as closely as possible (in the real valued
prediction game).

Two kinds of prediction strategies occur in Schurz' framework: {\em Ordinary
predictors} that predict the next event by some arbitrary algorithm without
looking at any of the other predictors and {\em meta-inductivists} that may --
although they do not have to -- base their own predictions on any of the ordinary
predictor's predictions. The ordinary predictor's predictions are considered to
be at least {\em output-accessible}, i.e.~the meta-inductivists get informed
about the ordinary predictor's predictions before they place their own
predictions. In order to chose between the ordinary predictors, the
meta-inductivists may take into account the predictor's success rates as well as
their previous predictions. 

The ordinary predictors are symbolized by Schurz with capital ``$P$''s with an
added index, e.g. $P_1$, $P_2$, etc. The meta-inductivists are symbolized as $xMI_i$,
where $x$ is a place holder for a string of characters that indicates the type of
meta-inductivist and $i$ is, again, an index. There exists a kind of canonical
ordinary predictor that Schurz calls the {\em object-inductivist} and which he
denotes as $OI$. The object-inductivist's algorithm takes either -- in the
real valued prediction game -- the mean value of all past events as its next prediction
or -- in the binary valued prediction game -- the kind of event that had the higher
frequency in the past.

If the sequence of events is a {\em random} sequence and if we exclude demonic
predictors that know the coming world event ahead of time, there exists no 
strategy that can do better than the object inductivist. The meta-inductivist
will consequently chose the object inductivist among the ordinary predictors, or,
if no object inductivist is present, it will predict according to the object
inductivist's algorithm by itself. At any rate the meta-inductivist will either
be as good or better than any of the ordinary predictors.

However, in order for the meta-inductivist to be a ``winning strategy'' of the
sort that is required to provide a solution to the problem of induction, it must
also be optimal or, at least, approximately optimal in a ``deceiving''
world, where the event sequence or the other predictors or both ``demonically''
conspire against the meta-inductivist. The situation can be described as a game
with the following rules:

\begin{enumerate}

  \item In each round, first, the ordinary predictors predict
  what the next event in the event sequence will be. For making their
  predictions the ordinary predictors have access to the following information:

  \begin{enumerate}
    \item Complete knowledge about the past of the game, i.e.~the past event
    sequence and the past predictions by all other ordinary and
    meta-inductivist predictors.

    \item \label{deceivers}{\em Deceiving predictors} know if and by which
    meta-inductivists their output will be accessed (see point \ref{MIs}). 
    However, they have to deliver their predictions before the
    meta-inductivists do. As a consequence, deceivers that always
    have a reliable forenowledge of the predictions of the meta-inductivists cannot
    exist. For, the case may arise where a deceiver would have to base its
    evaluation of which prediction to deliver on a meta-inductivist's prediction
    which is in turn based on the very results of this evaluation. Now, assume a
    deceiver $A$ predicts 0 whenever a meta-inductivist $MI$ is going to predict
    1 and 1 if $MI$ is going to predict 0 and at the same time $MI$ predicts 0
    when $A$ has predicted 0 and 1 when $A$ has predicted 1. Then, the prediction of
    the deceiver would be undefined.
    
    \item Deceiving predictors may have the capability of clairvoyance that is,
    in a {\em non deceiving world} they know beforehand what the next event will
    be. For reasons similar as in \ref{deceivers}, permanent clairvoyants
    cannot exist in a deceiving world.
  \end{enumerate}
  
  \item \label{MIs} Then, the meta-inductivists make their predictions. In doing
  so, they may access the ``output'', i.e.~the predictions, of any non
  meta-inductivist. Also, the meta-inductivists may -- just like the ordinary
  predictors -- take into account the complete information about all past events
  and predictions.
  
  \item \label{world} Finally, the world event occurs. In a {\em deceiving}
  world, the world event may depend in an arbitrary way on what predictions the
  predictors have made.

  \item A meta-inductivist ``wins'', i.e. is long run optimal, if in the long run
  the success loss of MI as compared to the best player at the given time
  converges to zero or a negative number.\footnote{I am indepted to an anonymous
  referee for the precise formulation.} (Or, simply put, if in the long run it
  performs at least almost as good as the best player.) Otherwise, the
  meta-inductivist ``looses'' the game.
 
\end{enumerate}


\section{Schurz' central results}

Within the just sketched framework, is it possible to find a provably
optimal meta-inductivist strategy? Schurz believes it is and he presents two
important theorems regarding the optimality of certain types of
meta-inductivists. But he also relates an argument \cite[p.
298]{schurz:2008} by Cesa-Bianchi and Lugosi \cite[p.
67]{cesaBianchi-lugosi:2006} that points out certain limits. The
argument is the following:

\begin{quote}
{\bf Impossibility Theorem 1}: In the binary prediction game, a single
meta-inductivist cannot be optimal in all possible worlds.

The proof of this theorem can informally be stated as follows: Assume a
world with one meta-inductivist and two alternative predictors, one of which always
predicts $1$ and the other always $0$. Whatever the sequence of world events
is, the succes rates of the two assumed alternative predictors will always add up
to 1. It follows that in any round the success rate of either the one or the other
of the two assumed alternative predictors is at least 50\%.  Now, if the sequence
of world events is a demonic sequence that always delivers the event that was
not predicted by the meta-inductivist, the meta-inductivist's success rate will
always remain 0\%. Thus, in any round the meta-inductivist's success
rate is significantly lower than that of the best player, which means that the
meta-inductivist's strategy is not optimal.
\end{quote}

Because of this, Schurz understands the different types of
meta-inductivists which he develops in section four 
through to section six of his article \cite[p. 285-296]{schurz:2008} as
only partial solutions to the problem of induction. They are optimal in
(large) classes of prediction games but not in all prediction games.

%  Because of this, the
% types of different meta-inductivists that Schurz develops in section four
% through to section six of his article \cite[p. 285-296]{schurz:2008} are
% understood by him only as partial solutions to the problem of induction
% that are optimal in certain (large) classes of prediction games but not
% in all prediction games.

The ``impossibility theorem'' stated before does not cover the real
valued prediction game.  And indeed it can be demonstrated for the real valued
prediction game that a meta-inductivist that averages over the
alternative predictor's predictions weighted by the alternative
predictor's relative success-advantage over the meta-inductivist will
quickly approximate the maximal success rate of the alternative
predictors. This is Schurz' theorem 4 \cite[p. 297]{schurz:2008}, which
shall be termed ``optimality theorem 1'' here:

\begin{quote}
{\bf Optimality Theorem 1} for a weighted average
metainductivist $wMI$ in the real valued prediction game:
\begin{enumerate}
  \item {\em Long run}: The success rate $suc_n$ of $wMI$ approximates the
  alternative predictors maximal success rate $maxsuc_n(P)$, i.e.
  $lim_{n\rightarrow\infty} (maxsuc_n(P) - suc_n(wMI)) = 0$.
  \item {\em Short run}: In any round $n$ for the success rate of $wMI$ holds:
  $suc_n(wMI) \geq maxsuc_n(P) - \sqrt{m/n}$ with $m$ being the number of
  alternative predictors.
\end{enumerate}
\end{quote}

Here $P$ denotes the set of all alternative predictors and $maxsuc_n(P)$ denotes
the maximal success rate of the alternative predictors in round $n$. For the
precise definition of the strategy of the weighted average meta-inductivist and
the proof of the theorem, see Schurz' paper \cite[p. 296ff.]{schurz:2008}.

As can be seen, the long run success of the meta-inductivists does not
depend on how many alternative predictors are present. Only in the short
run, the number or alternative predictors $m$ is important in so far as
the more alternative predictors are present, the more ``distracted'' the
weighted average metainductivist can become in the short run.

Schurz is able to prove a similar theorem for
 the binary valued prediction game. The
impossibility theorem stated earlier precludes that this will work for a single
meta-inductivist in the binary prediction game. But, as Schurz is able to
demonstrate, the mean success rate of a {\em collective of meta-inductivists}
$cwMI$ can approximate the maximal success of the alternative predictors -- be
they as demonic as they may -- almost equally well as the weighted average
meta-inductivist can in the real valued prediction game. The more $cwMI$
predictors are present, the better the approximation. Or, more
precisely:

\begin{quote}
{\bf Optimality Theorem 2} for the {\em mean success rate} of a collective of
$k$ collective weighted-average meta-inductivists $cwMI$ in the binary
prediction game:
\begin{enumerate}
  \item {\em Long run}: The mean success rate $meansuc_n$ of the
  collective meta-inductivists $\frac{1}{2k}$-approximates the alternative
  predictor's maximal
  success rate $maxsuc_n$, i.e. $lim_{n\rightarrow\infty} (maxsuc_n(P) -
  meansuc_n(cwMI)) \leq \frac{1}{2k}$.
  \item {\em Short run}: In any round $n$ for the mean success rate of $cwMI$
  holds: $meansuc_n(wMI) \geq maxsuc_n(P) - \sqrt{m/n} - \frac{1}{2k}$ with $m$
  being the number of alternative predictors.
\end{enumerate}
\end{quote}

For the proof of this theorem and for the precise algorithm of the
collective weighted-average meta-inductivists ($cwMI$), see Schurz'
article \cite[p. 297-299]{schurz:2008}. Again, the long run success of
the collective meta-inductivists (with regard to their mean success
rates!) does not depend on the number alternative predictors. This
renders the finding non-trivial. For no matter how large the number or
alternative predictors is, their maximal success can be approximated by a
comparatively smaller collective of meta-inductivists, the precise number
of which is only determined by what level of approximation is regarded as
satisfactory. It is only in the short run that a large number of possibly
demonic alternative predictors can effectively deceive the collective
meta-inductivists.

\section{Limitations of Schurz' approach: Confinement to the finite}

\subsection{Why Schurz' approach cannot be extended to the infinite case}
\label{finiteness}

Schurz explicitly restricts his investigation to prediction games with finitely
many prediction strategies \cite[p. 284]{schurz:2008}.  In this case the number
of meta-inductivists may even be much smaller than the number of alternative
predictors. It will now be shown why a similar argument cannot be
made in case the number of alternative predictors is infinite and why, therefore,
Schurz' optimality argument is confined to the finite. 

\begin{quote}
{\bf Impossibility Theorem 2}: If there is an infinite number of
alternative predictors, then even a collective of meta-inductivists 
cannot perform approximately optimal in all possible worlds in the
binary prediction game.

{\em Proof:} Consider the following scenario: Let there be an arbitrary
number of meta-inductivists. As there are only two possible events,
namely 1 and 0, at least half of the meta-inductivists predicts the same
event. Obviously, in each round there exists an event that is {\em not}
picked by a majority of meta-inductivists. Now, assume a demonic
world where the world event is always an event that is not predicted by a
majority of meta-inductivists. Then the average success rate of the
meta-inductivists never exceeds 50\%.

Now we only need to show that there can exist at least one (demonic)
predictor that achieves a higher success rate. For this purpose, split
the (infinite) set of alternative predictors into two infinite sets in
the first round. The predictors in the first set predict 1, the
predictors in the other set predict 0. In the following rounds take the
(infinite) set of predictors that have always predicted true so far,
split it into two infinite sets and again let all predictors from the
first set predict 1 and all predictors from the second set predict 0. In
any round of the game there is thus an infinite number of predictors left
that has a success rate of 100\%. Since the meta-inductivist's average
success is significantly lower (smaller or equal 50\%), their strategy is
not optimal.
\end{quote}

And, as may be expected, there is a similar impossibility theorem for the real
valued prediction game:

\begin{quote}
{\bf Impossibility Theorem 3}: If there is an infinite number of alternative
predictors then no meta-inductivist $xMI$ can  
be approximately optimal in all possible worlds in the real valued prediction
game.

{\em Proof:} Assume a demonic world, where the world event is always 0 or 1,
whichever of these two numbers is further away from the predicion that $xMI$
makes. As to the infinite number of alternative predictors: In the first round,
let half of them predict 1 and the other half 0. In the following rounds let
half of the alternative predictors that have always predicted correctly so far
predict 1 and the other half 0. Then at any point in time $n$, there exist some
predictors with complete success, while the average success of $xMI$
does not exceed 50\%.
\end{quote}

% But why, one might ask, should we consider the average success of a collective of
% meta-inductivists, anyway? After all, when drawing inductive inferences, what we
% are after is most probably one prediction at a time and not a whole number of
% different predictions (by different members of a collective of meta-inductivists)
% at a time. This is a good question and we will come back to that in a
% while.\label{irreducibilityArgument} Also, it is a question that does not only
% concern the infinite case but that is specific for the collective approach in the
% binary prediction game and that can be raised for the finite case just as well.
% In order to understand that there is no easy way around this question it suffices
% to recall that taking any single meta-inductivist as our reference
% meta-inductivist instead of considering the whole collective would not solve the
% problem, because a single meta-inductivist can always be beaten (in the binary
% prediction game) and the same would be true for a rule that picks a new reference
% meta-inductivist from the collective meta-inductivists in every round.

Hence, the conclusion: Neither in the binary nor in the real
valued prediction game exists an optimal meta-inductive strategy if the
number of alternative predictors is infinite.

\subsection{A sidenote: Limitations of ``one favorite'' meta-inductivists}

Just how difficult it is to design a meta-inductivist that covers all possible
or at least all desirable scenarios becomes apparent when considering
a limitation of Schurz' avoidance meta-inductivist, which is the most universal
type in a series of ``one-favorite meta-inductivists'' that Schurz develops in
sections four to six of his article. 

As Schurz proves mathematically (his theorem 3), the avoidance
meta-inductivist ($aMI$) $\epsilon$-approximates the maximal success of
the {\em non-deceiving} alternative predictors. However, this proof
does not cover all strategies that we might intuitively consider as
non-deceivers. For example, a strategy that starts as a deceiver and
switches to a non-deceiving clairvoyant prediction algorithm only later
in the game (after it has been classified as a deceiver by $aMI$) will
remain classified as a deceiver by $aMI$. Intuitively, though, we would
probably not consider it a deceiver any more after it has switched to a
non-deceiving clairvoyant algorithm. Further below it will be
demonstrated that this can even happen accidentally for a predictor that
never deceives (in an intuitive sense).

This limitation is a consequence of the fact that Schurz' definition of
``deception'' is purely extensional. It is based on the predictor's overt
behaviour and not on the deceptive or non-deceptive algorithm the
predictor uses: ``A non-MI-player $P$ (and the strategy played by $P$) is
said to {\em deceive} (or to be a {\em deceiver}) {\em at time n} iff
$suc_n(P) - suc_n(P|\epsilon MI) > \epsilon_d$'' \cite[p.
293]{schurz:2008}, where $\epsilon_d$ is the ``deception-threshold'' and
$suc_n(P|\epsilon MI)$ is $P$'s conditional success-rate when $\epsilon
MI$ has $P$ as a favorite. So, contrary to what we might intuitively
think, it is not a necessary condition for being a deceiver to base
the predictions on what favorites the meta-inductivists have.

As Schurz himself notices ``even an object-strategy (such as OI) may
become a deceiver, namely, when a demonic stream of events deceives the
object-strategy''\cite[p. 293]{schurz:2008}. This is of course to be
understood in terms of Schurz' previous definition of deception,
because the algorithm that, say, $OI$ uses is the same as in a
non-demonic world and would intuitively not be considered as deceptive. 

% While Schurz does not offer any explicit definition of what
% ``deception'' by a demonic stream of events means, it must, once
% again, be assumed that the definition is purely extensional. Because
% only then the possibility is excluded by definition that a
% random stream of world events accidentally deceives an $OI$, i.e.
% without being itself a ``deceptive'' stream of world events.

But then there is a finite probability that an $OI$ will be classified as
a deceiver by $aMI$ even though the stream of world events is not demonic
in the sense that the events are computed from the predictions made by
the predictors. For there is a finite probability that a random stream of
world events accidentally mimics a demonic stream of world events up to
round $k$ so that $OI$ appears as a deceiver up to round $k$. If $k$ is
sufficiently large then $aMI$ classifies $OI$ as deceiver. And it will
only reevaluate $OI$'s status if $OI$ lowers its unconditional success
rate. ``For a player P who is recorded as a deceiver will be
'stigmatized' by aMI as a deceiver as long as P does not decrease his
unconditional success rate (since P's aMI-conditional success rate is
frozen as long as aMI does not favor P)''\cite[p. 295]{schurz:2008}. As
$OI$'s success rate reflects the frequency of random world events in the
binary prediction game, it is unlikely that it significantly lowers its
success rate at a later stage in the game.

In such a situation $aMI$ would fail to $\epsilon$-approximate the
maximal success of $OI$ even though no deception was ever intended and
the conditions under which this situation can occur are completely
natural (i.e. non demonic world, no supernatural abilities like
clairvyoance, etc.). Thus, if $aMI$ can fail to be optimal even with
respect to $OI$ under completely natural circumstances, the optimality
result concerning the performance of $aMI$ with regards to all
non-deceivers may not quite deliver what we expect. For example, we
cannot not say that $aMI$ is optimal save for demonic conditions or
deception, if decpetion is understood in an intuitive sense as
described above.

\section{Open Questions}

Schurz' ``New approach to Hume's problem'' is both a research program on
prediction strategies and a proposed answer to the problem of induction.
As a research program it can be pursued more or less independently of its
claim to offer a new approach to Hume's problem. Regarding this
claim, the central question is of course: Can meta-in\-ductivists solve
the problem of induction? This question can be broken down into three
separate questions.

\begin{enumerate}
  \item How is Schurz' model related to the problem of induction,
  if the latter is understood as the problem of justifying scientific
  inference? 
  \item Does it matter that no
  single meta-inductivist but only a collective of meta-inductivists
  constitutes an optimal strategy in the (binary) prediction game? 
  \item Is it acceptable that the best alternative justification
  works only when the number of predictors is finite?
\end{enumerate}

In order to better understand these questions, each of them shall
briefly be discussed:

1) {\em Justification of scientific inference}: Schurz answers the
problem of induction within a technical framework that consists of a
highly stylized world that produces a sequence of (binary or real-valued)
events at discrete time intervals. Further interpretative work might be
necessary to show that the problem of induction that has been solved
within Schurz' technical framework matches the problem(s) of induction
that occur in the real world. In particular, the kind of induction in
Schurz' model is the induction from previous events to the next event -- in
contradistinction to the inductive or abductive inference from a number
of single instances to a general rule. For the justification of
scientific inference, the latter type of induction seems to be even more
important. After all, we would like to know whether we can rely on a law
of nature if it has been confirmed in a finite number of instances and
never disconfirmed. Thus, the question would be whether and how Schurz'
answer to Hume's problem can also be transferred to this case.

2) {\em Admissibility of collectives of meta-inductivists}: In the
binary-valued game a {\em collective} of meta-inductivists is required to
assure optimality in the prediction game. Now, if we are looking for a
reliable method of induction to the next event, then we are faced with
the somewhat puzzling fact that we do not get a single proposal but a
multitude of proposals instead. As follows from the impossibility theorem
for single meta-inductivists (impossibility theorem 1 above), it is
impossible to melt down the different predictions of the collective of
meta-inductivists to a single prediction without becoming vulnerable to
deception by a demonic world. There is no grave problem involved if the
prediction game is considered in a decision theoretical framework
\cite[p. 301ff.]{schurz:2008}. For then the primary goal would not be to
get the next prediction right, but rather to derive as much utility as
possible from a number of predictions. However, this implies a shift of
emphasis from the original problem of induction to a closely related
decision theoretical problem.

3) {\em Confinement to the finite}: If only a finite number of prediction
strategies is taken into account, then we exclude the overwhelming
majority of {\em possible} prediction strategies from the game right from
the beginning. For, given that the sequence of events is infinite, there
exists an uncountably large number of posssible event sequences. And even
if we take into consideration only those prediction strategies which can
be described by an algorithm, there still remains a countable infinity of
possible alternative prediction strategies. Unfortunately, neither a
single meta-inductivist nor a collective of meta-inductivists can perform
optimal in all possible worlds if the number of alternative predictors is
infinite (see section \ref{finiteness}).

Schurz deliberately restricts his investigation to
prediction games with finetely many players, because he makes ``the
realistic assumption that xMI has finite computational means.'' \cite[p.
284]{schurz:2008}. But in order to justify this restriction one would
need to show that an infinite number of alternative players is
impossible, rather than arguing that xMI cannot deal with an infinite
number of alternative players. Otherwise the notice ``that xMI has
finite computational means'' merely amounts to admitting that under
this ``realistic assumption'' xMI simply cannot always perform
optimal.

As there is no logical contradiction involved in the assumption of an
infinite number of alternative players, the only grounds upon which it
could be defended would be empirical grounds. But then it is hard to
see how the general impossibility of an infinite number of alternative
players can be demonstrated without silently or explicitly making use of
inductive inference.

Very simply put, there seems to be no good reason why a theoretical
framwork for answering Hume's problem that allows for ``clairvoyants'', 
``deceivers'', ``demonic'' streams of world events, should not also
admit infinite sets of alternative predictors. Surely, that an xMI which
has only finite computational means does not work under this
condition is not a sufficient reason. 

\section{Conclusion} 

Summing it up, what can be achieved with prediction games with respect to
inductive inferences is a non-trivial result which shows that if only a
finite number of prediction strategies are involved there exists with the
``(collective) weighted average meta-inductivist'' a strategy that is an
approximately best strategy in all possible worlds. However, it can also
be demonstrated that no approximately best strategy exists if an infinite
number of alternative prediction strategies is involved. And there is
some reason to believe that for a full solution to the problem of
induction the restriction to a finite number of alternative prediction
methods is insufficient. If this is true, then the best alternative
justification cannot offer a full solution to the problem of induction.

%\section{References} 
\bibliographystyle{chicago}
\bibliography{Induktionsproblem}

\end{document}
