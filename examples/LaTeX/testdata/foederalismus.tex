\documentclass[a4paper,12pt]{article} 
\usepackage{ae}
\usepackage[german, ngerman]{babel}
\usepackage[utf8x]{inputenc}
\usepackage{ucs}
\usepackage[T1]{fontenc}
\usepackage{t1enc}
\usepackage{type1cm}

\usepackage{ifpdf}
\ifpdf
\usepackage{xmpincl}
\usepackage[pdftex]{hyperref}
\hypersetup{
    colorlinks,
    citecolor=black,
    filecolor=black,
    linkcolor=black,
    urlcolor=black,
    bookmarksopen=true,     % Gliederung öffnen im AR
    bookmarksnumbered=true, % Kapitel-Nummerierung im Inhaltsverzeichniss anzeigen
    bookmarksopenlevel=1,   % Tiefe der geöffneten Gliederung für den AR
    pdfstartview=FitV,       % Fit, FitH=breite, FitV=hoehe, FitBH
    pdfpagemode=UseOutlines, % FullScreen, UseNone, UseOutlines, UseThumbs 
}
\includexmp{foederalismus}
\pdfinfo{
  /Author (Eckhart Arnold)
  /Title (Hauptvertreter des Föderalismusgedankens in Deutschland von der Neuzeit bis zum Ende des 19.Jahrhunderts)
  /Subject (Deutsche Theoretiker des Föderalismus)
  /Keywords (Föderalismus, Deutschland, Constantin Frantz, Karl Georg Winkelblech, Althusius, Immanuel Kant)
}
\fi

\sloppy

\begin{document}

% \catcode`\ä = \active \catcode`\ö = \active \catcode`\ü = \active
% \catcode`\ß = \active \catcode`\Ä = \active \catcode`\Ö = \active
% \catcode`\Ü = \active

% \defä{"a} \defö{"o} \defü{"u} \defß{"s} \defÄ{"A} \defÖ{"O}
% \defÜ{"U}

\begin{titlepage}

\title{Hauptvertreter des Föderalismusgedankens in
Deutschland von der Neuzeit bis zum Ende des
19.Jahrhunderts}

\setlength{\parindent}{0em}

\begin{center}{\large {\bf Seminar für politische Wissenschaften der
Universität Bonn}}\end{center}

\setlength{\parskip}{2cm}

\begin{center}{\Large Hauptvertreter des Föderalismusgedankens in
Deutschland von der Neuzeit bis zum Ende des
19.Jahrhunderts}\end{center}

\setlength{\parskip}{3cm}

Referat zum\\Hauptseminar:\\Legitimationsprobleme des Föderalismus
am Beispiel traditionsreicher Bundesstaaten\\Sommer
1997\\[0.5cm]Leitung: Dr. S. Fröhlich

\setlength{\parskip}{1cm}

vorgelegt von:\\[0.5cm]Eckhart Arnold\\5.Fachsemester, Magister
% \\Kaiserstr.  57, 53113 Bonn\\Tel.: 0228 / 26 44 01

\setlength{\parskip}{1cm}

Bonn, den 15. August 1997

\end{titlepage}

\pagenumbering{roman} \tableofcontents

\newpage

\pagenumbering{arabic} \setcounter{page}{1}

\section{Einleitung}

In dieser Arbeit soll anhand einiger ausgewählter Vertreter die
Entwicklung der Föderalismusidee im politischen Denken von der
Neuzeit bis zum 19. Jahrhundert nachgezeichnet werden.

Unter Föderalimus wird heute ein Staatsstrukturprinzip verstanden,
nach welchem ein Gesamtstaat {\em regional} in Gliedstaaten
unterteilt ist, wobei sowohl auf Gesamtstaatsebene als auch auf
Gliedstaatsebene wesentliche Elemente der Staatlichkeit
vorkommen. Heute wird der Föderalismus scharf vom Korporationismus
unterschieden, bei dem die Glieder Stände oder
Berufsgenossenschaften sind. Allerdings hat sich dieser moderne
Föderalismusbegriff erst mit der Zeit herausgeschält, und deshalb
ist es wohl berechtigt, wenn in dieser Arbeit ausführlich auch
solche Konzepte dargestellt werden, die man heute eher als
Ständestaat bezeichnen würde. Daraus, daß Föderalismus im
wesentlichen ein Staatsstrukturprinzip ist und keine "`große
politische Idee"' ergibt sich bereits, daß der Föderalismus in der
politischen Diskussion häufig eher ein Anhängsel bildet zu den
großen Fragen der Zeit. So stellt sich denn die Frage von
föderalistischem oder zentralistischem Staatsaufbau im
16./17.Jahrhundert im Zusammanhang mit dem Problem des staatlichen
Machterhalts im Zeichen konfessioneller Spaltung. In der Aufklärung
wird Föderalismus dann (unter anderem) als ein Mittel zu dem Zweck
gesehen, die staatliche Macht im Inneren zu zähmen oder
kriegerische Aggressionen zwischen Staaten einzudämmen. Bei einigen
Denkern des 19.Jahrhundert wiederum spielt der Föderalismus im
Zusammenhang mit dem Problem sozialer Gerechtigkeit eine wichtige
Rolle. Es wird versucht, in dieser Arbeit solche Zusammenhänge zu
verdeutlichen. Daher wurden vor allem solche Vertreter
des Föderalismusgedankens ausgewählt, die in gewisser Weise
represäntativ für eine Strömung ihrer Epoche sind, und in deren
Werk der Föderalismus eine zentrale Rolle spielt. Auch kam es vor
allem darauf an, dem Föderalismus in der politischen
Ideengeschichte nachzuspühren und nicht die staatsrechtliche
Entwicklung des Föderalismus in Deutschland zu verfolgen. Für die
Aufklärungsepoche weicht diese Arbeit davon ab, nur deutsche
Vertreter zu besprechen. Die wesentlichen Impulse zur
Weiterentwicklung des Föderalismusgedankens kamen zu dieser Zeit
von John Locke, David Hume und Montesqieu. Viele ihrer Argumente
finden sich gebündelt wieder in den "`Federalist papers"'. Um nun
nicht diese Autoren einzeln darzustellen und auch wegen der großen
Bedeutung der "`Federalist Papers"' wird deshalb an dieser Stelle
eine knappe Darstellung der wichtigsten Gedanken der "`Federalist
Papers"' gegeben.

An Literatur wurden soweit wie möglich die Werke der besprochenen
Autoren verwendet, daneben ausgewählte Sekundärliteratur und
historische Standardwerke.


\section{Föderalismusdiskussion in der frühen Neuzeit: Föderalismus oder
Souveränitätslehre}

\subsection{Der Föderalismusgedanke im Werk des Johannes Althusius}

Föderalistische Auffassungen vom Aufbau der Gesellschaft und des
Staates waren unter den Denkern des ausgehenden Mittelalters
durchaus verbreitet. Sie standen meist im Zusammenhang mit
bestimmten theologischen Ideen, insbesondere mit der Vorstellung
des Weltganzen als eines hierarchisch gerodneten, vielfach
geschichteten Kosmos. Durch die Reformation erhielt der
Föderalismusgedanke - vertreten vor allem durch die Gruppe der
Föderaltheologen - wesentlichen Auftrieb, mußten sich doch die
reformierten Fürstentümer des deutschen Reiches gegenüber der
katholischen kaiserlichen Oberhoheit behaupten. Aber auch in der
politischen Wirklichkeit des 16.Jahrhunderts spielte der
Föderalismus eine bedeutende Rolle. Nicht zuletzt baute der
Augsbuger Religionsfrieden 1555 mit dem Prinzip des {\em cuius
regio eius religio} bei gleichzeitiger Anerkennung der beiden
Hauptkonfessionen im Reich auf der faktisch vorhandenen
föderlistischen Struktur des deutschen Reiches auf. Dieser in die
ständische Gesellschaftsordnung des 16.Jahrhunderts eingebette
Föderalismus wie er im Deutschen Reich herrschte, ist es, den
Johannes Althusius in seinem politik\/theoretischen Hauptwerk, der
"`Politica methodice digesta"', ebensowohl empirisch beschreibt wie
propagiert.

\subsubsection{Biographische Skizze}

Johannes Althusius wurde im Jahre 1557 in Diedenhausen, einem
kleinen westfälischen Dorf, geboren. Möglicherweise ein
uneheliches Fürstenkind erhielt Johannes Althusius eine optimale
Föderung durch seinen Landesherren, den Grafen von Sayn und
Wittgenstein. Er ermöglichte Althusius Rechtswissenschaften zu
studieren, zunächst in Köln, dann in Basel und Genf. Insbesondere
die Aufenthalte in den letzteren beiden Städten vermittelten
Althusius prägende Bildungseinflüsse durch den neuzeitlichen
Humanismus sowie durch den Calvinismus. Unmittelbar nach Abschluß
seines Studiums im Jahre 1586 wurde er an die Universität Herborn
berufen, wo er rasch aufstieg. In der folgenden Zeit verfasste er
zahlreiche, meist juristische Schriften. Der Durchbruch gelang ihm
jedoch mit seinem zuerst im Jahre 1603 erschienen umfassenden
politikwissenschaftlichen Werk: "`Politica methodice digesta et
exemplis sacris et profanis illustrata"', welches Althusius weithin
bekannt machte und ihm schließlich 1604 das Amt eines Syndikus der
calvinistischen Stadt Emden einbrachte. Der Magistrat erhoffte sich
durch die Berufung von Althusius eine wirkungsvollere Vertretung
gegenüber der ostfriesischen Landesherrschaft. Althusius
enttäuschte die emdener Ratsherren nicht. Es gelang nach zähen
Verhandlungen entgültige Vereinbarungen zwischen der Stadt Emden
und dem Grafen von Ostfriesland zu erzielen. Bis zu seinem Tod im
Jahre 1638 blieb Althusius - tortz lukrativer Angebote aus anderen
Städten - in Emden, wo er weiterhin sowohl politisch als auch
wissenschaftlich tätig war. Im Jahre 1617 veröffentlichte er mit
den "`Dicaeologicae libri tres totum et universum jus"' ein
umfassendes Kompendium der Rechtslehre seiner Zeit\footnote{Zur
Biographie von Althusius vgl. Carl Joachim
Friedrich\cite{friedrich1}: Johannes Althusius und sein Werk im
Rahmen der Entwicklung der Theorie von der Politik, Berlin 1975. -
Vgl. Erik Wolf (Hrsg.)\cite{wolf}: Johannes
Althusius. Grundbegriffe der Politik. Aus "`Politica methodice
digesta"'.1603, Frankfurt/M 1948, S.42-44.}.

\subsubsection{Der Entwurf der "`Politica methodice digesta"'}

Obwohl Althusius ein ausgebildeter Jurist war, bestand er doch
darauf, daß die Politikwissenschaft neben der Rechtswissenschaft
und auch neben anderen Wissenschaften, wie Theologie oder
Philosophie einen eigenständigen Platz einnehmen müsse. Im
Unterschied zur Rechtswissenschaft beschreibt die
Politikwissenschaft nach Althusius' Auffassung die
Lebenssachverhalte, deren normative Regelung die Jurisprudenz
behandelt. In diesem Sinne einer (eher) empirischen, die
Wirklichkeit beschreibenden Wissenschaft ist auch Althusius
"`Politica"' zu verstehen\footnote{Vgl. die Eineitung von
Althusius, in: Wolf (Hrsg.)\cite{wolf}, a.a.O., S.3-12.}.

Was für eine große Rolle der Föderalismus in Althusius Werk spielt,
wird schon daran deutlich, daß Althusius nicht mit einer Darstellung
der zentralen und wichtigsten Institutionen des Staates beginnt,
sondern daß er zunächst beschreibt, wie die einzelnen Individuen
kleine Gemeinschaften bilden, wie diese Gemeinschaften sich zu
größeren Einheiten zusammenfügen, und wie sich zu guter Letzt über
mehrere Zwischenstufen hinweg aus solchen Einheiten schließlich der
Staat konstituiert. Die aus natürlichen Gemeinschaften auf eine
gewissermaßen organische Weise zusammengestzte Gesellschaft bezeichnet
Althusius als {\em consociatio symbiotica}. An der untersten Stelle
stehen dabei die einfachen und privaten Lebensgenossenschaften, womit
Althusius vor allem Familie und Hausgemeinschaften meint. Sie sind
nach Althusius naturgegeben und ihr Zusammenhalt beruht auf
gegenseitigem Vertrauen und
Hilfsbereitschaft.\footnote{Vgl. Wolf(Hrsg.)\cite{wolf}, a.a.O.,
S.18ff.} Daneben enstehen privatrechtliche Genossenschaften, die
freiwillig und unter Umständen auch nur auf bestimmte Zeit geschlossen
werden. Zu diesen gehören vor allem die Berufsgenossenschaften. Die
privatrechtlichen Genossenschaften verfügen über eigenes Recht und
eigenes Eigentum. Durch gemeinsames Handeln und Erleben, sowie
Einmütigkeit ({\em concordia}) und gegenseitiges Wohlwollen wird die
Verbundenheit der weitgehend gleichberechtigten Mitglieder dieser
Genossenschaften hergestellt. Es gibt nun verschiedene
Klassifizierungen dieser Genossenschaften. Unter anderem können sie
den Ständegruppen von Adel, Klerus oder drittem Stand zugeordnet
werden\footnote{Vgl.  Wolf(Hrsg.)\cite{wolf}, a.a.O., S.25ff.}.

Über den einfachen Lebensgenossenschaften und den privatrechtlichen
Genossenschaften stehen die besonderen und allgemeinen politischen
Gemeinschaften. Die politischen Gemeinschaften umfassen stets
ein bestimmtes Territorium und schließen alle privatrechtlichen
Genossenschaften und Lebensgemeinschaften dieses Territoriums in
sich ein. Anders als die privtrechtlichen Genossenschaften bestehen
die politischen Gemeinschaften zumindest prinzipiell zeitlich
unbegrenzt, und die Mitgliedschaft in ihnen ist nicht freiwillig,
denn sie beruht auf einem Bund ({\em pactus}), was ausschließlich
Gegenseitigkeit von Garantien und Verpflichtungen meint, und nicht
auf einem Vertrag ({\em contractus}), der freiwillig geschlossen
werden kann. Die politischen Gemeinschaften unterteilt Althusius
wiederum in besondere bzw. engere und allgemeinere politische
Gemeinschaften. Der wesentliche Unterschied zwischen diesen beiden
Gemeinschaftstypen besteht darin, daß die engeren politischen
Gemeinschaften (zu denen Dörfer, Gemeinden und Städte zählen)
unmittelbar von den einzelnen Bürgern gebildet werden, während in
den allgemeineren politischen Gesellschaften die Bürger nur
mittelbar als Angehörige besonderer Gemeinschaften vertreten
sind.\footnote{Vgl. Johannes Althusius\cite{althusius}:
Politica. Faksimiledruck der 3.Auflage Herborn 1614, Meisenheim am
Glan 1961, Cap. 9, Rn 5, S.168.}  Die allgemeinste politische
Gemeinschaft ist der Gesamtstaat.

Der Gesamtstaat setzt sich sowohl aus politischen Gemeinschaften
als auch aus privatrechtlichen Genossenschaften zusammen.
(Insbesondere die Ständeversammlungen spielen hierbei eine große
Rolle). Die Herrschaftsgewalt ({\em potestas universalis
imperandi}) in der allgemeinsten politischen Gemeinschaft wird von
ihren Gliedern getragen. Hieren stellt sich Althusius in bewußten
Gegensatz zur Souveränitätslehre Bodins,\footnote{Vgl.
Althusius\cite{althusius}, a.a.O., Cap. 9, Rn 20, S.170.} nach der
die Herschaftsgewalt allein dem Monarchen bzw. der Majestät
zukommt. Althusius schließt sich jedoch insoweit der
Souveränitätslehre an, als auch er von der Unteilbarkeit dieser
Gewalt ausgeht. Auch erlaubt Althusius die Delegation der
Herrschatfsgewalt an eine Vertretung, doch warnt er ausdrücklich
davor die gesamte Macht einer einzelnen Person in die Hände zu
legen.\footnote{Vgl. Althusius\cite{althusius}, a.a.O., Cap. 9, Rn
19, S.170.} Diejenigen, welche die Herrschaftsgewalt ausüben,
bleiben an die Grenzen der ihnen übertragegenen Vollmachten
gebunden. Überschreiten sie diese, so erlischt damit automatisch
der Gerhorsamsanspruch gegenüber den Untertanen.\footnote{Vgl.
Althusius\cite{althusius}, a.a.O., Cap. 18, Rn 41, S.289. - Zum
Staats- und Gesellschaftsaufbau bei Althusius vgl. auch: Peter
Jochen Winters: Die >>Politik<< des Johannes Althusius und ihre
zeitgenössischen Quellen. Zur Grundlegung der politischen
Wissenschaft im 16. und im beginnenden 17.Jahrhundert, Freiburg im
Breisgau 1963, S.170ff. - Vgl.  Otto v. Gierke\cite{gierke}:
Johannes Althusius und die Entwicklung der naturrechtlichen
Staatstheorien. Nachdruck der 3.Auflage von 1913, Meisenheim am
Glan 1958, S.226-263.}

\paragraph{Würdigung}

Es wäre müßig darauf hinzuweisen, daß der theoretische Ansatz des
Althusius, der von der Ständegesellschaft des ausgehenden
Mittelalters bzw. der frühen Neuzeit ausgeht, den heutigen
Politikvorstellungen nicht mehr entspricht.  Dennoch gibt es einige
Aspekte, die seinen Ansatz nach wie vor beachtenswert erscheinen
lassen.

Einmal kommt Althusius' Begriff der {\em consociatio symbiotica}
und des föderlistisch gegliederten Staates dem
Subsidiaritätsprinzip sehr nahe. Althusius hebt immer wieder
deutlich hevor, daß die kleineren Gemeinschaften und insbesondere
die privatrechtlichen Genossenschaften ihre inneren Verhältnisse
selbst regeln. Weiterhin erscheint dieser Begriff unter
legitimatorischen Gesichtspunkten interessant. Indem Althusius
nämlich den gewissermaßen vorstaatlichen Gesellschaftsaufbau in den
Staatsaufbau einbezieht, werden Loyalitäten und Gruppenidentitäten
unterhalb der Gesamtstaatsebene automatisch
mitberücksichtigt. Dadurch kann die Interessenstruktur der
Gesellschaft wahrscheinlich akkurater wiedergegeben werden als
durch eine politische Repräsentation allein auf Gesamtstaatsebene.
In diesem Punkt scheint Althusius' Politikauffassung gegenüber
manchen Gesellschaftsvertragstheorien überlegen, die von abstrakten
Individuen ausgehen. Andererseits darf nicht vergessen werden, daß
die {\em consociatio symbiotica} des Althusius eine Gesellschaft
von Ungleichen beschreibt. Zwar gibt es keinen unmittelbaren
Widerspruch zum Gedanken der Gleichheit der Menschen, aber die
normative Forderung nach gleichberechtigter politischer
Partizipation aller Individuen läßt sich aus dem System des
Althusius nicht ganz zwanglos ableiten.

Die wesentliche mit Althusius föderlistischer Konzeption
verbundene Frage ist jedoch, ob ein solcher föderal gegliederter
Staat lebensfähig und stabil sein kann. Sie soll im folgenden im
Vergleich mit der etwa zur selben Zeit entstandenen
Souveränitätslehre Bodins erörtert werden.

\subsubsection{Der Gegensatz zur Souveränitätslehre Bodins}

Jean Bodin (1529-1596) hatte bereits 27 Jahre, bevor Althusius seine
"`Politica"' veröffentlichte, mit den "`six livres de la R\'epublique"' ein
staatstheoretisches Werk geschaffen, dessen Grundsätze auf einen strikten
Antiföderlismus hinausliefen. Unter dem Eindruck der französischen
Religionsunruhen im 16.Jahrhundert fordert Bodin in diesem Werk, daß die
höchste Macht im Staat ungeteilt und unbeschränkt seien müsse.\footnote{Vgl.
  Jean Bodin\cite{bodin}: Les six Livres de la R\'epublique avec l'Apologie de
  R.Herjin. Faksimiledruck der Ausgabe Paris 1583, o.O. (Meisenheim am Glan?),
  1961, Livre Premi\`er, Chap. VIII, S.122ff.} Diese unbeschränkte und
ungeteilte Staatsmacht, die Bodin als {\em souverainit\'e} bezeichnet, sieht
er als definierendes Wesensmerkmal des Staates an. Insbesondere schließt für
Bodin die Souveränität eines Staates aus, daß einzelne Gruppen innerhalb des
Staates über eigene Herrschaftsmacht verfügen, da sonst der Bürgerkrieg
droht.\footnote{Vgl Bodin\cite{bodin}, a.a.O., Livre Quatri\`eme, Chap. VII,
  S.634ff.}

Bodins Theorie setzte sich mit dem Siegeszug des Absolutismus in
der europäischen Staatenwelt auch unter den Staatstheoretikern
durch, während Althusius in Vergessenheit geriet. Es stellt sich
die Frage, inwieweit dies zu Recht geschah, d.h. ob Althusius
föderalistische Konzeption unter den Bedingungen der
konfessionellen Spaltung in der Tat nicht tragfähig war.

Betrachtet man die historische Entwicklung des deutschen Reiches in
der Zeit zwischen dem Augsburger Religionsfrieden 1555 und dem
Beginn des Dreißigjährigen Krieges, so springt als einer der
wesentlichen Vorgänge, die zum Dreißigjährigen Krieg geführt haben,
der Prozeß der schleichenden Säkularisierung zahlreicher
Kirchengüter ins Auge, in welchem protestantische Fürsten - meist
mit sanftem Druck - diese Güter in die Hand ihrer Familien
brachten. Zu Beginn des 17.Jahrhunderts hatte sich die Situation
schon soweit zugespitzt, daß ein Kompromiß nur noch unter
schwierigen Bedingungen möglich gewesen wäre, denn ein Nachgeben
hätte für die katholische Seite bedeutet, auf ihre im Augsburger
Religionsfrieden wohlverbrieften Rechte zu verzichten, für die
protestantischen Fürsten aber geheißen, Güter, in deren Besitz sie
nun schon seit einer geraumen Zeit waren, wieder aufzugeben. Die
dadurch entstehende Konfliktlage wurde durch das Ausfallen
zentraler Entscheidungsinstanzen\footnote{So wurde die Entscheidung
des Reichskammergerichtes im sogenanten "`Vierklosterstreit"' von
protestantischer Seite schließlich nicht mehr anerkannt, was
allerdings nur einer der Kulminationspunkte des länger schwelenden
und langsam exkalierenden Streits war. Vgl. Heinrich
Lutz\cite{lutz}: Das Ringen um die deutsche Einheit und die
krichliche Erneuerung. Von Maximilian I. bis zum westfälischen
Frieden, Berlin 1983, S.362-363.} zusätzlich verschärft. Einen
Kompromiß - ohne Krieg - zu erzwingen, war der Habsburger Kaiser
nicht mächtig genug\footnote{Zur Vorgeschichte des Dreißigjährigen
Krieges vgl. Heinrich Lutz\cite{lutz}, a.a.O., S.358ff., S.393ff.}.

Im Ausbruch des Dreißigjährigen Kriges kann geradezu eine
Bestätigung von Bodins Befürchtungen gesehen. Gleichzeitig
offenbart dieser geschichtliche Vorgang eine entscheidende Schwäche
von Althusius Theorie: Althusius hatte den föderalen Staat statisch
konzipiert. Seine Stabilität beruht letzlich auf dem
Machtgleichgewicht der unterschiedlichen Gruppen und auf der
Einhaltung der Rechte. Was geschehen soll, wenn eine der
Großgruppen ständig an Stärke zunimmt, so daß sich die andere
schließlich vital bedroht fühlen muß, dafür ist aus Althusius'
"`Politica"' kein Rat zu holen. Innerhalb von Bodins Ansatz stellt
sich dieses Problem dagegen nicht, da es für Bodin von Anfang an
darauf ankommt, der souveränen Majestät alle Macht zu sichern, so
daß bei einem inneren Streit die Majestät in jedem Fall entscheiden
kann.

\section{Föderalismus und Machtkontrolle - Föderalistische Entwürfe der
Aufklärungsepoche}

\subsection{Die Argumente der "`Federalist Papers"'}

Das Vorbild der vereinigten Staaten und die "`Federalist Papers"',
die gewissermaßen die politische Philosophie zum amerikanischen
Bundesstaat liefern, haben - wenn auch mit erstaunlicher
Verzögerung - einen erheblichen Einfluß auf das politische Denken
in Europa ausgeübt. Der heutige Föderalismusbegriff entspricht im
wesentlichen dem Föderalismuskonzept das in den "`Federalist
Papers"' beschrieben wird. Wegen der außerordentlichen Bedeutung der
"`Federalist Papers"' für diese Entwicklung des
Föderalismusgedankens soll daher kurz auf die wichtigsten Argumente
dieses Werkes eingegangen werden.

Die "`Federalist Papers"' enstanden in den Jahren 1787/1788 im
Vorfeld der Bildung des amerikanischen Bundesstaates. Sie bestehen
aus einer Serie von Zeitungsartikeln, in denen ihre Autoren
Alexander Hamilton, James Madison und John Jay, für die Bildung
eines echten amerikanischen Bundesstaates mit gestärkter
Zentralgewalt warben.\footnote{Zur Geschichte der "`Federalist
Papers"' vgl. die Einleitung von Barbara Zehnpfennig in: Alexander
Hamilton / James Madison / John Jay\cite{hamilton}: Die Federalist
Papers. Übersetzt, eingeleitet und mit Anmerkungen versehen von
Barbara Zehnpfennig, Darmstadt 1993, S.1ff.} Während ein Teil der
Artikel speziell die amerikanischen Verhältnisse betrifft,
behandelt ein anderer Teil der Artikel die verschiedenen Aspekte
eines föderalistischen Staates in grundsätzlicher Weise. Die
wesentlichen Merkmale eines föderalistischen Staates (im Vergleich
zu einem Staatenbund) sind dabei folgende:
\begin{itemize}
\item {\em Außenpolitische Vorteile}: Ein Bundesstaat ist nach
außen hin mächtiger und geschlossener. Er kann daher die
(prinzipell gleichgerichteten) sicherheits- und handelspolitischen
Interessen der Gliedstaaten erfolgreicher wahrnehmen, als dies die
Einzelstaaten alleine oder innerhalb einer losen, gegenüber
den Spaltungsbemühungen äußerer Mächte anfälligen Konföderation leisten
könnten.\footnote{Vgl. Hamilton / Madison / Jay \cite{hamilton},
a.a.O., Nr.3, S.61ff., Nr.4, S.64ff., Nr.5, S.69ff., Nr.11,
S.100ff.}

\item {\em Friedenspolitische Vorzüge}: Die Aufgabe der
einzelstaatlichen Souveränität bewirkt den Abbau von
Drohpotentialen und gegenseitiger Gefährdung der
Einzelstaaten. Dadurch werden außerdem die Verteidigungsausgaben
gesenkt, und die Einzelstaaten gewinnen einen weniger
militaristischen Charakter. Im Vergleich zu einem Vertrags- und
Bündnissystem von Einzelstaaten garantiert ein Bundesstaat die
größere Stabilität.\footnote{Vgl. Hamilton / Madison / Jay
\cite{hamilton}, a.a.O., Nr.7, S.78ff., Nr.8, S.83ff., Nr.10,
S.93ff.} (Dies ist vor allem deshalb interessant, weil die
Möglichkeit einer stabilen Zwischenlösung zwischen Staatenbund und
Zentralstaat oft geleugnet worden ist.)

\item {\em Zusätzliche Machtkontrolle durch vertikale
Gewaltenteilung}: Der einzelne Bürger ist sowohl vor der Machtanmaßung oder
sogar dem Machtmißbrauch sowohl des Einzelstaates als auch des Bundesstaates
geschützt, da die jeweils andere Ebene ein Gegengewicht bilden
kann. Der Föderalismus erscheint damit neben der funktionalen
Gewaltentrennung im Staat als ein zusätzliches Mittel um die
Gewaltenteilung zu implementieren.\footnote{Vgl. Hamilton / Madison
/ Jay \cite{hamilton}, a.a.O., Nr.45, S.289ff., Nr.51, S.319ff.}

\item {\em Umittelbare Beziehung des Bürgers zum Bundesstaat als
Hauptwesensmerkmal}: Die Autoren des "`Federalist"' gehen recht
ausführlich auf zahlreiche Einzelheiten der Kompetenzverteilung
zwischen Bundesstaat und Gliedstaaten ein. Ein entscheidendes
Merkmal ist, daß der einzelne Bürger in unmittelbare Beziehung zum
Gesamtstaat tritt, indem er unmittelbar die Bundesregierung wählt
und sich umgekehrt die Maßnahmen der Bundesregierung und die
Gesetzgebung des Bundesstaates unmittelbar auf die einzelnen Bürger
anstatt ausschließlich auf die Gleidstaaten beziehen. Hierin liegt
ein wesentlicher Unterschied zu vielen älteren
Föderalismuskonzeptionen, nach denen die Glieder nur als Kollektive
in der zentralen Einheit vertreten sind.\footnote{Vgl. Hamilton /
Madison / Jay \cite{hamilton}, a.a.O., Nr. 15, S.122ff.}
\end{itemize}


\subsection{Supranationaler Föderalismus und Friedenssicherung - Die Utopie
Kants} Der Gedanke, daß die Bildung eines föderalen Staates
militärische Aggressionen zwischen den Gliedstaaten verhindern
kann, spielt in Immanuel Kants\footnote{Zur Biographie: Immanuel
Kant (1724-1804) verbrachte fast sein ganzes Leben in
Königsberg. Nachdem er 1781 sein philosophisches Hauptwerk {\em
Kritik der reinen Vernunft} veröffentlicht hatte, verfasste er auch
wieder verstärkt wissenschaftliche und politische Schriften. Unter
seinen politischen Schriften sind außer der hier dargestellten
Friedensschrift noch die {\em Ideen zu einer Geschichte in
Weltbürgerlicher Absicht} (1784) und die {\em Metaphysik der
Sitten} (1797) hervorzuheben.} 1795 erschienener Schrift "`Zum
ewigen Frieden"'\footnote{Immanuel Kant\cite{kant}: Zum ewigen
Frieden. Ein philosophischer Entwurf, Stuttgart 1991.} eine
zentrale Rolle. Kant sieht nämlich in der Bildung eines
Weltstaatenbundes die einzige realistische Möglichkeit, um den
Frieden zwischen den Staaten dauerhaft zu sichern.

Bei seiner Argumentation für den Weltstaatenbund muß Kant zwei (theoretische)
Probleme bewältigen. Einmal scheint das Konzept eines Weltstaatenbundes der
völkerrechtlichen Forderung nach Unabhängigkeit und Selbstbestimmung der
souveränen Einzeltaaten zu widersprechen. Mit welchem Recht könnte daher
Bildung eines Weltstaates gefordert werden, wenn dies nur unter Einschränkung
elementarer völkerrechtlicher Prinzipien möglich ist? Die zweite Schwierigkeit
besteht darin zu zeigen, daß die Bildung eines Weltstaates überhaupt
herbeiführbar ist, da nun einmal die bestehenden Staaten kaum freiwillig ihre
Souveränität abgeben werden.

Das erste Problem ist moralischer Natur. Kant beantwortet es, indem er darauf
besteht, daß die Freiheit der Staaten sich nicht auf das Recht zu zügellosem
Gewaltgebrauch erstreckt. Zügelloser Gewaltgebrauch kann aber allein durch
freiwillige Selbstkontrolle eines souveränen Staates nicht effektiv verhindert
werden. Daher sind die Staaten moralisch verpflichtet, aus dem anarchischen
Naturzustand, in welchem sie sich untereinander befinden, herauszutreten, was
durch die Bildung eines Gesamtstaates geschehen könnte. Diese Verpflichtung
der Staaten, einen dauerhaften Friedenszustand zu schaffen, ähnelt Kant
zufolge der moralischen Vepflichtung einzelner Menschenen, die in einem
Hobbeschen Naturzustand leben, durch die Bildung eines Staates einen
Rechtszustand zu schaffen.\footnote{Vgl. Kant\cite{kant}, Zweiter
  Definitivartikel zum ewigen Frieden. Das Völkerrecht soll auf einem
  Föderalism freier Staaten gegründet sein, S.16-18.}

Für das zweite Problem bietet sich als Königsweg der Föderalismus an. Ein
Welteinheitsstaat würde zwar das Friedensproblem in idealer Weise lösen, aber
ein solcher Staat ist unrealistisch und auch nicht unbedingt wünschenswert, da
die Eigenständigkeit der Völker auch bei Kant durchaus einen hohen Wert
darstellt. Ein Weltstaatenbund würde jedoch bei den einzelnen Staaten auf
weniger Widerstand stoßen. Denkbar wäre, daß eine einzelne und sehr mächtige
Republik dazu die Initiative ergreift, denn Republiken sind nach Kants
Auffassung naturgemäß weniger angriffslustig\footnote{Vgl. Kant\cite{kant},
  Erster Definitivartikel zum ewigen Frieden. Die bürgerliche Verfassung in
  jedem Staate soll republikanisch sein, S.10-15.} und würden daher eine
Eigenständigkeit, die sich positiv in der Möglichkeit erschöpft, Kriege vom
Zaun zu brechen, weniger hoch schätzen als etwa Monarchien. Diese Repulik
könnte dann das Zentrum eines sich nach und nach ausbreitenden
Weltstaatenbundes abgeben.\footnote{Vgl. Kant\cite{kant}, S.19-21.}

Kant geht nicht näher auf die technischen Details eines solchen
Weltföderalismus ein. Dies ist verständlich, denn eine wesenliche Vorausstzung
für diese erhoffte Entwicklung bestand für Kant darin, daß sich die
republikansiche - heute würde man sagen "`demokratische"' - Staatsform
druchsetzten müßte, was zu seiner Zeit in Europa und in der Welt nur für die
wenigsten Staaten galt. Daher lag für Kant auch der "`Weltstaatenbund"' noch
in einiger Ferne. Wie ist die Kantsche Utopie eines "`Weltstaatenbundes"'
jedoch heute zu bewerten, nachdem die Demokratie sich inzwischen in einigen
Teilen der Erde durchgesetzt hat?

Zwei Faktoren fallen unmittelbar auf, die Kants Annahmen zu widersprechen
scheinen. Einmal hat sich gezeigt, daß auch Demokratieen sich sehr stark gegen
die Aufgabe ihrer Souveränität sträuben. Dies erschwert die Bildung
supranationaler Föderationen selbst bei ausschließlicher Beteiligung von
Demokratien. Zum zweiten hat sich erwiesen, daß sich Demokratien zumindest
gegenüber nicht-demokratischen Staaten oft nicht weniger aggressiv verhalten
als autokratisch geführte Staaten\footnote{Vgl. Ernst-Otto
  Czempiel\cite{czempiel}: Kants Theorem. Oder: Warum sind Demokratien (noch
  immer) nicht friedlich, in: Zeitschrift für Internationale Beziehungen,
  3.Jahrgang (1996), Heft 1, S.79-101. - Czempiels eigene Hypothese zur
  Erklärung dieses (empirisch ziemlich gesichert festgestellten) Phänomens
  läuft darauf hinaus, daß auch in den heutigen Demokratieen noch keine
  genügende Kongruenz zwischen denjenigen, die die Kreigslasten zu tragen
  haben, und denjenigen, die eine Kriegsentscheidung beeinflussen, besteht.}.
Dies ist ein Faktor, der der Bildung eines Weltstaatenbundes entgegensteht, da
ein solcher Bund unterschiedliche Staatssysteme umfassen müßte. Dennoch zeigen
Prozesse, wie die Entstehung der Vereinigten Staaten von Amerika oder die
Europäische Integration, die beide mit einer weitgehend erfolgreichen inneren
Befriedung verbunden waren, daß Kants Grundidee der Friedensstiftung durch
Bildung einer Föderation eine realistische Grundlage hat.

\section{19.Jahrhundert: Föderalismuskonzepte im Spannnungsfeld von
sozialer und nationaler Frage}

\subsection{Karl Georg Winkelblech}

Eng mit den sozialen Fragen seiner Zeit verknüpft ist das
Föderalismuskonzept von Karl Georg Winkelblech.\footnote{Zur
Biographie: Winkelblech wure 1810 in Ensheim bei Wörstadt in
Rheinhessen geboren. Er studierte Pharmazie und Chemie. Sein
Interesse an sozialen Fragen wurde geweckt als er 1843 in Norwegen
bei einem Fabrikbesuch mit dem Elend der Arbeiter konfrontiert
wurde. Winkelblech beschäftigte sich daraufhin intensiv mit
sozialen und wirtschaftlichen Fragen. In den Jahren 48/49 versuchte
er (allerdings ohne allzu großen Erfolg) Handerwerker und Arbeiter
für seine Ideen zu gewinnen. Winkelblech starb im Jahre 1865. -
Vgl. Ernst Deuerlein: Föderalismus. Die historischen und
philosophischen Grundlagen der föderativen Prinzips, München 1972,
S.102-106.} Bei Winkelblech gewinnt das Wort Föderalismus eine
umfassende Bedeutung. Er bezeichnet damit eine politische,
wirtschaftliche und soziale Ordnung. Den Ausgangspunkt seiner
Überlegungen bildet die {\em föderale} Eigentumsordnung. Dieser
Begriff bezieht sich auf das Eigentum an natürlichen Resourcen
({\em Naturkräfte}). Die föderale Eigentumsordnung erlaubt jedem
Menschen genau soviel Besitz an natürlichen Resourcen wie er mit
seiner eigenen Arbeitskraft verwerten kann. Winkelblech grenzt
diese Eigentumsordnung von der liberalen und kommunistischen
Eigentumsordnung ab. Der liberalen Eigentumsordnung wirft
Winkelblech vor, daß sie es Einzelnen erlaubt ihr Eigentum an den
natürlichen Resourcen unbegrenzt zu erweitern. Dies führt, da die
menschliche Arbeitskraft ohne natürliche Resourcen völlig nutzlos
ist, zu einer geradezu sklavischen Abhängigkeit. Die föderale
Eigentumsordnung betrachtet Winkelblech als ein Ideal, welches zwar
nie ganz verwirklicht werden kann, an das aber eine Annährung
möglich ist und - dies fordert die christliche Gerechtigkeit -
angestrebt werden muß.

Dieses Ziel soll bei Winkelblech im Rahmen einer demokratischen
politischen Ordnung verwirklicht werden. Das Regierungssystem teilt
sich in gesetzgebende und Regierungskörperschaften, wobei die
vollziehenden Organe den gesetzgebenden Organen untergeordnet
sind. An der Spitze der Regierung steht ein Wahlkaiser, der für
begrenzte Zeit aus dem Kreise eines Verdienstadels gewählt
wird. Das gesamte politische System ist föderalistisch gegliedert,
neben die Reichskammer treten Provinzialkammern, die in die
Reichsgesetzgebung eingebunden sind.

Winkelblech betont ausdrücklich, daß es keine Trennung zwischen
politischer und sozialer Ordnung geben dürfe, womit er vor allem
meint, daß Privatwirtschaft und Privateigentum nicht völlig
sakrosant seien dürfen und durch interventionsstaatliche Maßnahmen
angetastet werden können. Winkelblech skizziert in seinem Werk auch
ein in hohem Maße interventionsstaatliches Wirtschaftsmodell, in
dem manche Bereiche ausschließlich öffentlich bewirtschaftet
werden, und in dem die privatwirtschaftlichen Bereiche in Zünften
gegliedert sowie durch eine Erwerbsordnung genau geregelt sind. Die
Erwerbsordnung legt unter anderem Grenzen fest, bis zu denen
private Betriebe wachsen dürfen.\footnote{Vgl. Karl Georg
Winkelblech (alias Karl Marlo)\cite{winkelblech}: Untersuchungen
über die Organisation der Arbeit oder System der
Weltökonomie. Ersten Bandes erste Abteilung. Historischer Teil,
Kassel 1850, S.347-383.}

Winkelblechs Vorschläge sind nie in die Tat umgesetzt worden. Die
Arbeiterbewegung, auf die er vor allem hatte Einfluß nehmen wollen,
tendierte eher in eine rein sozialistische Richtung. Ob die Umsetzung
seiner Ideen die sozialen Probleme der Mitte des 19.Jahrhunderts
wirksam hätte bekämpfen können, kann deshalb bestenfalls hypothetisch
entschieden werden. Es wäre die Frage zu stellen, ob Winkelblechs
zünftliche Wirtschaftsordnung ebenfalls jene Entwicklungsdynamik
entfaltet hätte, durch die der Industriekapitalismus mit seiner
raschen Kapazitätsausweitung nach der Jahrhundertmitte den Pauperismus
als das drängenste soziale Problem schließlich zum Verschwinden
brachte.\footnote{Zum Pauperismus vgl. Hans-Ulrich
Wehler\cite{wehler}: Deutsche Gesellschaftsgeschichte. Zweiter
Band. Von der Reformära bis zur industriellen und politischen
"`Deutschen Doppelrevolution"'. 1815-1845/49, München 1987,
S.281-296.}

\subsection{Constantin Frantz}

\subsubsection{Kurzbiographie}

Ebenso wie Karl Georg Winkelblech und wie viele andere bedeutende
Intelektuelle seiner Zeit ging Constantin Frantz aus einem
protestantischen Pfarrhaus hervor. Er wurde 1817 zu Oberbörnecke in
der Nähe von Halberstadt in der preußischen Provinz Sachsen
geboren. Im Jahre 1836 begann Frantz an der Universität Halle
Mathematik und Physik zu studieren. Obwohl er schließlich über ein
mathematisches Thema promovierte, scheint ihn die Beschäftigung mit
der exakten Naturwissenschaft nicht wesentlich geprägt zu haben. Dafür
hörte Frantz, nachdem er 1839 nach Berlin gewechselt war, nebenher mit
großem Interesse philosophische und historische Vorlesungen - letztere
bei Leopold Ranke.\footnote{Leopold v. Ranke(1795-1886), einer der
bekanntesten deutschen Historiker, entickelte eine
empirisch-quellenkritische Historiographie mit staatsfreundlich
konservativer Tendenz. Zu seinen Grunddogmen gehörte das
Individualitätsprinzip, nach welchem jeder Staat und jede Epoche ihren
eigenen Gesetzen gehorcht, die nicht übertragbar sind. Dieser Gedanker
findet sich auch bei Frantz wieder.} In philosophischer Hinsicht hat
ihn, nach einer kurzen hegelianischen Phase, am nachhaltigsten die
religiös gefärbte Gedankenwelt Schellings\footnote{Schelling
(1775-1854) ist neben Fichte und Hegel einer der Hauptvertreter der
philosophischen Schule des {\em Deutschen Idealismus}. Er erfand die
{\em Identitätsphilosophie} nach welcher Natur und Geist eins sind. In
seiner zunehmend religiös verbrämten Spätphilosophie, von der Frantz
offenbar stark beeinflußt worden ist, deutet er die menschliche
Geschichte als Heilsprozeß, in welchem sich Gott mit den Menschen
vereinigt.}  beeinflußt. Etwa zur selben Zeit begann Frantz damit,
kleinere Schriften und Zeitungsartikel zu Themen der Politik und
Philosophie zu schreiben. Durch eine philosophische Schrift auf ihn
aufmerksam geworden, stellte ihn 1844 der preußische Kultusminister
Eichhorn im Ministerium ein. Seitdem hoffte Frantz auf eine politische
Karriere. Diese schien sich zunächst auch recht günstig zu entwickeln
und Frantz blieb - abgesehen von einer kleinen Unterbrechung durch die
Revolution - bis 1862 im preußischen Staatsdienst. Während dieser Zeit
publizierte Frantz unermüdlich und brüskierte, obwohl seine Schriften
von durchaus konservativem Geist waren, durch seine eigenwilligen
Meinungen immer wieder Vorgesetzte un Gönner. Dies und die Tatsache,
daß die ihm angebotenen Posten oft nicht seinen ehrgeizigen
Erwartungen entsprachen, trugen 1862 mit zu seinem Ausscheiden aus dem
Staatsdienst bei. Seitdem hielt sich Frantz als freier Publizist über
Wasser. Mit seinen politischen Ansichten geriet Frantz mehr und mehr
in eine Außenseiterposition, zumal er als strikter Befürworter einer
großdeutschen Lösung das 1871 entstandene Bismarckreich entschieden
ablehnte. Constantin Frantz starb 1891 in einem kleinen Ort in der
Nähe von Dresden.\footnote{Zur Biographie vgl.: Eugen
Stamm\cite{stamm}: Ein berühmter Unberühmter. Neue Studien über
Konstantin Frantz und den Föderalismus, Konstanz 1948, S.143-157. -
Vgl.: Paulus Franciscus Hermanus Lauxtermann\cite{lauxtermann}:
Romantik und Realismus im Werk eines politischen Außenseiters, Utrecht
1979, S.9-21.}

Unter den zahlreichen Publikationen von Constantin Frantz sind
besonders hervorzuheben: Seine beiden Denkschriften "`Polen,
Preussen, Deutschland"'(1848) und "`Von der deutschen
Föderation"'(1850er Jahre), worin Frantz sein
deutschlandpolitisches Konzept einer mulitnationalen,
mitteleuropäischen Födertion entwirft; das 1859 entstandene Werk
"`Untersuchungen über das europäische Gleichgewicht"', worin er
sich mit dem Zerfall des europäischen Gleichgewichtssystems
auseinandersetzt und die Zunkunftsvision eines Systems von vier
Weltmächten (Amerkia, Rußland, England, Frankreich) entwirft;
schließlich Frantz' 1879 entstandenes Werk "`Der Föderalismus, als
das leitende Prinzip für die soziale, staatliche und internationale
Organisation [...]"', in welchem Frantz den Begriff des
Föderalismus als ein umfassendes, Staat, Gesellschaft und
Wirtschaft prägendes Prinzip bestimmt\footnote{Zu den wichtigsten Schriften von Frantz mit ausführlichen Zitaten: Lauxtermann\cite{lauxtermann}, a.a.O. }.

Auch wenn Frantz' Schriften sich fast immer auf aktuelle politische
Vorgänge bezogen, soll im folgenden weniger die Gedankenentwicklung
von Constantin Frantz und die historische Porblematik seiner
Entwürfe thematisiert werden, sondern es wird versucht, die mehr
oder weniger gleich bleibenden Grundideen seines
Föderalismuskonzeptes darzustellen unter der Fragestellung, ob sich
daraus möglicherweise Anknüpfungspunkte für die gegenwärtige
Diskussion ergeben. Einer zeitgemäßen Interpretation von Frantz
stellen sich jedoch einige Schwierigkeiten in den Weg, die vor
allem auf Frantz' Denk- und Argumentationsstil beruhen. So sind
Frantz' Schriften durchsetzt mit Romantizismen\footnote{Dazu
gehören bei Frantz' unter anderem: Der Rückgriff auf ein mythisch
verklärtes Mittelalter, vor allem auf ein idealisiertes heiliges
römisches Reich deutscher Nation; die Vorstellung, daß alle Völker
bzw. Nationen ein je eigenes Prinzip verkörpern und einen je
eigenen und nur sehr bedingt wandelbaren Wesenscharakter haben; der
Glaube, daß jeder Staat eine besondere Idee verkörpert
bzw. verkörpern soll, was letzten Endes auf eine Sakralisierung der
Politik hinausläuft; das organische Staatsdenken mit seiner
deskriptiven wie normativen Überbetonung von geographischen und
historischen Bestimmungsfaktoren; schließlich der Glaube an einen
christlichen Missionsauftrag des Abendlandes und insbesondere
Deutschlands.} und gelegentlich durch einen entschiedenen
Antisemitismus stark verunziert.\footnote{Der Antisemitismus von
Constantin Frantz, der leider nicht auf das zu seiner Zeit auch
unter Gebildeten durchaus geläufige Ressentiment beschränkt bleibt
sondern - religiös begründet - mit seiner Theorie vom christlichen
Missionsauftrag Deutschlands in plausiblem Zusammenhang steht, soll
hier nicht weiter breit getreten werden. Zur Kostprobe immerhin:
Constantin Frantz\cite{frantz-foederalismus}: Der Föderalismus als
das leitende Prinzip für die sociale, staatliche und internationale
Organisation, unter besonderer Bezugnahme auf Deutschland, Mainz
1879, S.352ff.}

\subsubsection{Deutschland als mitteleuropäische Föderation}

Mehrfach hat Frantz Entwürfe für die staatliche Zukunft
Deutschlands vorgelegt, denen allen gemeinsam ist, daß Deutschland
darin als eine offene mitteleuropäische Föderation beschrieben
wird. So empfielt Frantz etwa in der 1860 erschienen Denkschrift
"`Dreiunddreißig Sätze über den deutschen
Bund"'\footnote{Vgl. Lauxtermann, a.a.O., S.58ff. - Frantz zog mit
dieser Schrift ebenso wie mit den zuvor erschienen "`Untersuchungen
über das europäische Gleichgewicht"' die Konsequenzen aus dem
Krimkrieg, der für ihn den Zerfall des pentarchischen
Gleichgewichtssystems markierte.} einen deutschen Staat zu bilden,
in welchem Preußen, Österreich und die restlichen Bundesgebiete als
drei gleichberechtigte Partner vertreten seien sollten. Dieser
Staat sollte jedoch nicht als nach außen abgeschlossener
Nationalstaat entstehen (ohnehin würde er bereits mehrere
Nationalitäten umfassen), sondern es sollte zumindest kleineren
Nachbarstaaten ermöglicht werden, sich dieser Föderation ebenfalls
anzuschließen.\footnote{Ob allerdings Belgien oder die Schweiz
begeistert gewesen wären sich als Juniorpartner einer deutschen
Föderation anzuschließen ist fraglich. Auch wäre die von Frantz
angebotene Alternative zum Deutschen Reich wohl wesentlich
schwieriger zu verwirklichen gewesen als das
Bismarckreich. Vgl. Lauxtermann\cite{lauxtermann}, a.a.O.,
S.66-78. - Vgl. auch Golo Mann\cite{mann}: Deutsche Geschichte des
19. und 20.Jahrhunderts, Frankfurt /M 1992, S.392f.}

Schon in früheren Schriften hatte Frantz - damals allerdings noch
stärker aus preußischer Sicht - Vorschläge zu einer
deutsche-österreichisch-preußischen Föderation
vorgelegt.\footnote{Vgl. Constantin Frantz\cite{frantz-foederation}
: Von der deutschen Föderation, Siegburg 1980 (zuerst Berlin
1851).} Bemerkenswert sind insbesondere seine Denkschriften zur
Polenfrage, worin er die Bildung einer preußisch-polnischen
Föderation empfiehlt. Frantz sieht in einer solchen Föderation eine
Möglichkeit die berechtigten nationalen Ansprüche der polnischen
Bevölkerung (freilich nicht die demokratischen Ansprüche der
polnischen Freiheitsbewegung) zum beiderseitigen Vorteil mit den
Interessen Preußens zu versöhnen.\footnote{Constantin
Frantz\cite{frantz-polen}: Polen, Preussen und Deutschland. Ein
Beitrag zur Reorganisation Europas. Faksimiledruck der Ausgabe
Halberstadt 1848, Siegburg 1969. - Constantin Frantz: Betrachtungen
über den Polonismus im Großherzogtum Posen und die damit
zusammenhängenden politischen Verhältnisse, abgedruckt ebda.,
S.61ff. - Frantz Polen-Schrift muß freilich vor dem Hintergrund der
Zeit interpretiert werden, in welcher neben einer nur sehr
kurzlebigen Polenbegeisterung bei den Liberalen die Politik der
preußischen Regierung durch den rücksichtslosesten
Interessenegoismus bestimmt wurde. Heutzutage würde eine solche
Schrift wegen der darin immer noch deutlich zum Ausdruck kommenden
Überheblichkeit Empörung hervorrufen.}

Daß der Föderalismus für Frantz nicht bloß ein Mittel zur Lösung
politischer Gestaltungsfragen in Mitteleuropa sondern auch
Selbstzweck ist, geht besonders daraus hervor, daß Frantz auch nach der
Reichsgründung entschieden an seinen föderalistischen Plänen für
Deutschland festhielt und das Bismarcksche Reich unter diesem
Aspekt scharf kritisierte.\footnote{Vgl. Constantin
Frantz\cite{frantz-foederalismus}: Der Föderalismus als das
leitende Prinzip für die sociale, staatliche und internationale
Organisation, unter besonderer Bezugnahme auf Deutschland, Mainz
1879, S.220ff.}

Aus den verschiedenen Föderalismus-Entwürfen lassen sich in etwa
folgende wesentliche Merkmale herausfiltern:
\begin{itemize}

\item Der Föderalismus soll das Nationalitätenproblem lösen, da in
einer Föderation jede ihr angehörende Nationalität durch das
föderale Prinzip ihre Eigenständigkeit bewahren kann. Frantz
erkennt sehr deutlich die Ungerechtigkeiten und das
Konfliktpotential, welches demgegenüber der Rückgriff auf das
Nationalstaatsprinzip
heraufbeschwört\footnote{Vgl. Frantz\cite{frantz-foederation}: Von
der deutschen Föderation, a.a.O., S.87-122.}.

\item Ein föderaler Großstaat bietet gegenüber kleineren nach außen
hin abgeschlossenen Nationalstaaten große wirtschaftliche Vorteile,
die sich aus der gegenseitigen Ergänzung der in ihm vereinten
unterschiedlichen Wirtschaftsregionen ergeben. Für Frantz bildet
Deutschland in geographischer, historischer und wirtschaftlicher
Hinsicht eher ein Netzwerk interdependenter Regionen. Diesem
organischen Gebilde würde ein föderaler Staat eher entsprechen als ein
vermeintlich gewaltsam konstruierter zentralistischer
Staat.\footnote{Vgl. Constantin Frantz\cite{frantz-deutschland}:
Deutschland und der Föderalismus, Hellerhau 1917, S.38ff.}

\item Weiterhin geht Frantz davon aus, daß eine große
mitteleuropäische Föderation sich gegenüber den Nachbarn Rußland und
Frankreich sowie auch auf internationaler Bühne besser behaupten
kann. Gleichzeitig unterstellt Frantz, daß eine solche Föderation von
Natur aus weniger aggressiv ist und deshalb weniger Argwohn unter den
anderen Staaten hervorrufen wird. Das Letztere erscheint nicht
unbedingt plausibel, und es stellt sich daher die Frage, ob die
Frantzsche Föderation, die immer noch scharf gegen Rußland und
Frankreich abgegrenzt bleibt, wirklich eine bessere friedenspolitische
Perspektive geboten hätte als der Bismarcksche
Machtstaat.\footnote{Vgl. Lauxtermann\cite{lauxtermann}, S.58ff.}

\item Über diesen mitteleuropäischen Föderalimus hinausgehend malt
sich Frantz noch einen föderalen Weltvölkerbund aus. Seine eng an
Novalis\footnote{Vgl. Novalis\cite{novalis}: Die Christenheit oder
Europa. Und andere philosophische Schriften, Köln 1996, S.23-43.}
anknüpfenden Ideen, die auf eine Art christlicher Zwangsbeglückung der
Welt unter deutscher Führung hinauslaufen, sind allerdings ebenso
unrealistisch wie
indiskutabel.\footnote{Vgl. Frantz\cite{frantz-deutschland}:
Deutschland und der Föderalismus, a.a.O., S.154-216.}

\end{itemize}

Abgesehen von seinen Ausflügen in die politische Romantik
erscheinen Frantz' Gedanken in mancherlei Hinsicht
zukunftsweisend. Dies gilt besonders für seine Relativierung der
Bedeutung des Nationalstaates und auch für den von ihm
nahegelegten Politikstil, welcher - im Gegensatz zum rein
egoistischen Machtstaatsdenken - bei vernünftiger Berücksichtigung
des Eigeninteresses die Rechte anderer anerkennt (was am
deutlichsten in seinen Schriften zur Polenfrage zum Ausdruck
kommt). Dieser Eindruck relativiert sich jedoch wieder, wenn man
untersucht, wie Frantz sich den inneren Aufbau seines föderalen
Staates denkt.

\subsubsection{Ständischer Föderalismus statt repräsentative Demokratie} 

Man würde zweifellos ein falsches Bild von Konstantin Frantz
bekommen, sähe man in ihm nur den Visionär einer europäischen
Friedenslösung. Abgesehen davon war Konstantin Frantz nämlich ein
entschiedener Reaktionär. Bereits 1846 hatte er - damals ein
Angestellter des preußischen Kultusministeriums - ein scharfes
Pamphlet gegen die ``Constitutionellen'' verfasst. Unter den
konservativen Theoretikern seiner Zeit war es üblich Liberalismus
und Demokratie als eine verschärfte Form von Staatsabsolutismus zu
betrachten und dementsprechend zu brandmarken. Auf dieser Schiene
fährt auch Konstantin Frantz, wenn er den gesellschaftlichen
Föderalismus als heilsames Prinzip dem demokratischen
Repräsentativsystem entgegenstellt.

In seinem späten Werk zum Föderalismus weitet Frantz diesen Begriff
zu einem umfassenden, die soziale wie die staatliche und
internationale Ordnung umfassenden Prinzip aus. Frantz Ideen zur
internationalen Ordnung wurden bereits dargestellt. Seine
Vorschläge zur Gestaltung einer sozialen Gesellschafts- und
Wirtschaftsordnung lehnen sich an Winkelblech an und werden daher
hier nicht noch einmal besprochen. Bleibt die Frage, wie Frantz
sich das föderale politische System denkt. Die wichtigsten
grundsätzlichen Kritikpunkte,\footnote{Frantz entwickelt seine
Kritik am politischen System des Kaiserreichs gegen das er einige
sehr treffende Einwände erhebt (z.B.: Unausgewogenheit des Föderalismus
durch die preußische Dominanz, ungenügende politische Kontrolle des
Militärs). Hier sollen allerdings nur Frantz' prinzipielle
Vorstellungen vom Design des politischen System dargelegt werden.}
die Constantin Frantz gegen das repräsentative System anführt sind:
\begin{enumerate}

\item Echte Repräsentation ist unmöglich, da der Volkswille
ohnehin nicht delegiert werden kann. Frantz untermauert dies
durch eine Rechnung, die zeigen soll, daß bei Entscheidungen eines
Repräsentativorganes in der Regel nur eine Minderheit des
Volkes ihren Willen erhält. 

\item Die gesetzgebenden Organe werden von Leuten gewählt, die
nichts von der Gesetzgebung verstehen (nämlich vom Volk). Aus
diesem Grund hat eine Volkswahl auch höchstens innerhalb der
kleinsten Einheiten (Städte, Gemeinden) sinn.

\item Das Volk stellt beim Wahlakt keinen lebendigen Körper
(d.h. keine durch organisierte Körperschaften strukturierte
Gemeinschaft) mehr da, sondern einen bloßen Menschenhaufen. Dieser
könne aber weder sinnvoll ein Repräsentationsorgan wählen noch
durch den Wahlakt eine glaubhafte Legitimation bereitstellen. Die
Vertreter sollten nach Frantz aus der Mitte von Körperschaften
hervorgehen, da die Körperschaften über einen einheitlichen Geist
verfügen, der auch ihre Vertreter durchdringt.

\end{enumerate}

Constantin Frantz empfiehlt aus all diesen Gründen ein mehrstufiges
System, bei welchem die Vetretungsorgane der umfassenderen Ebenen
von Deligierten der jeweils niedrigeren Ebene gebildet werden. Nur
auf Kreis- oder Gemeindeebene sollen Wahlen stattfinden. Außerdem
schlägt Constantin Frantz eine zweite Kammer vor, die sich aus
Vertretern der Stände und Berufsgenossenschaften zusammensetzten
soll. Nach Frantz' Vorstellung von Föderalismus ist der
Staatsbürger also nicht Bürger zweier Staaten (Gliedstaat und
Gesamtstaat), sondern steht zumindest hinsichtlich seiner
politischen Partizipationsrechte nur zur untersten Einheit des
mehrstufigen Staates in unmittelbarer
Beziehung.\footnote{Vgl. Constantin Franz\cite{frantz-deutschland}:
Deutschland und der Föderalismus, a.a.O., S.7-37.}

Es erübrigt sich, im Einzelnen gegen diese Auffassungen nun den
Katalog von Standardargumenten für die repräsentative Demokratie
und das allgemeine Wahlsystem herunterzubeten. Auffällig ist, wie
gering Constantin Frantz die politische Mündigkeit des einzelnen
Bürgers veranschlagt. Auch ordnet Frantz den Bürger völlig dem
Kollektiv von Gemeinde bzw. Berufsgenossenschaft unter, so als wäre
mit dieser Zugehörigkeit auch der politische Standpunkt schon
vorgegeben. Mit diesem autoritären und kollektivistischen Zug
erinnert Frantz ständischer Föderalismus nicht wenig an
Althusius. Nur klingen solche Vorstellungen am Ende des 19.Jahrhunderts
nicht so plausibel wie zu Beginn des 17.Jahrhunderts, weshalb denn
auch Frantz empört forden muß, was Althusius noch mit gelassender
Selbstverständlichkeit beschreiben konnte.


\section{Schlußbetrachtung}

An welche der hier dargestellten Werke könnte oder sollte die
heutige Föderalismusdiskussion auf der Suche nach historischen
Vorbildern anknüpfen? 

Am ehesten bieten sich hierfür ohne Zweifel die {\em Federalist
Papers} an. Einmal entsprechen sie, da sie den Föderlismus in
Verbindung mit der liberalen Demokratie zum Gegenstand haben, am
meisten der heutigen Situation. Zum zweiten lassen sich zahlreiche
der Argumente der {\em Federalist Papers} fast unmittelbar auf die
gegenwärtige Lage der Europäischen Union übertragen.

Aber auch Althusius Theorie trägt einige Züge, die ihn heute noch
oder wieder interessant erscheinen lassen könnten. Althusius
anthropologische Auffassung, daß der Mensch sich nur in einer
Gemeinschaft mit anderen Menschen verwirklichen könnte, und daß ein
natürlicher Altruismus existiert, der zumindest innerhalb kleinerer
Gruppen für das Gemeinschaftsleben nutzbar gemacht werden kann,
erinnert nicht wenig an heutige kommunitaristische Ideen.

Weniger als Vorbild geeignet erscheint dagegen Constantin
Frantz. Seine politischen Entwürfe tragen immer einen gewissen
autokratischen Zug. Man gewinnt bei Frantz öfters den Eindruck, als
wolle er den Betroffenen ihre Rollen zuweisen, ohne ihnen zu
erlauben, ihre Interessen und Ziele selbst zu definieren.

\newpage

\begin{thebibliography}{99}

\bibitem{althusius} Johannes {\bf Althusius}:
Politica. Faksimiledruck der 3.Auflage. Herborn 1614, Meisenheim am
Glan 1961.

\bibitem{bodin} Jean {\bf Bodin}: Les six Livres de la République
avec l'Apologie de R.Herjin. Faksimiledruck der Ausgabe Paris 1583,
o.O. (Meiesnheim am Glan?) 1961.

\bibitem{franz-verfassung} Constantin {\bf Frantz}: Über die
Gegenwart und Zukunft der preussischen Verfassung. Neudruck der
Ausgabe Halberstadt 1846, Siegburg 1975.

\bibitem{frantz-polen} Constantin {\bf Frantz}: Polen, Preussen und
Deutschland. Ein Beitrag zur Reorganisation Europas. Faksimiledruck
der Ausgabe Halberstadt 1848, Siegburg 1969.

\bibitem{frantz-foederation} Constantin {\bf Frantz}: Von der
deutschen Föderation, Siegburg 1980. (Neudruck der Ausgabe Berlin
1851).

\bibitem{frantz-foederalismus} Constantin {\bf Frantz}: Der
Föderalismus als das leitende Prinzip für die sociale, staatliche
und internationale Organisation, unter besonderer Bezugnahme auf
Deutschland, Mainz 1879.

\bibitem{frantz-deutschland} Constantin {\bf Frantz}: Deutschland
und der Föderalismus, Hellerhau 1917. (Auszug aus dem zuvor
genannten Werk, hrsg. von Franz Blei)

\bibitem{hamilton} Alexander {\bf Hamilton} / James {\bf Madison} /
John {\bf Jay}: Die Federalist Papers. Übersetzt, eingeleitet und
mit Anmerkungen versehen von Barbara Zehnpfennig, Darmstadt 1993.

\bibitem{kant} Immanuel {\bf Kant}: Zum ewigen Frieden. Ein
philosophischer Entwurf, Stuttgart 1991. (Erscheinungsdatum der
Erstausgabe: 1795)

\bibitem{novalis} {\bf Novalis} (Georg Philipp Friedrich von
Hardenberg): Die Christenheit oder Europa. Und andere
philosophische Schriften, Köln 1996.

\bibitem{winkelblech} Karl Georg {\bf Winkelblech} (alias Karl
Marlo): Untersuchungen über die Organisation der Arbeit oder System
der Weltökonomie. Erster Band, Kassel 1850, Zweiter Band, Kassel
1857.

\bibitem{wolf} Erik {\bf Wolf} (Hrsg.): Johannes
Althusius. Grundbegriffe der Politik. Aus "`Politica methodice
digesta"'.1603, Frankfurt/M 1948.



\bibitem{czempiel} Ernst-Otto {\bf Czempiel}: Kants Theorem. Oder:
Warum sind Demokratien (noch immer) nicht friedlich, in:
Zeitschrift für Internationale Beziehungen (ZiB), 3.Jahrgang
(1996), Heft 1, S.79-101.

\bibitem{deuerlein} Ernst {\bf Deuerlein}: Föderalismus. Die
historischen und philosophischen Grundlagen der föderativen
Prinzips, München 1972.

\bibitem{friedrich1} Carl Joachim {\bf Friedrich}: Johannes
Althusius und sein Werk im Rahmen der Entwicklung der Theorie von
der Politik, Berlin 1975.

\bibitem{gierke} Otto v. {\bf Gierke}: Johannes Althusius und die
Entwicklung der naturrechtlichen Staatstheorien. Nachdruck der
3.Auflage von 1913, Meisenheim am Glan 1958.

\bibitem{lauxtermann} Paulus Franciscus Hermanus {\bf Lauxtermann}:
Romantik und Realismus im Werk eines politischen Außenseiters,
Utrecht 1979.

\bibitem{lutz} Heinrich {\bf Lutz}: Das Ringen um die deutsche
Einheit und die krichliche Erneuerung. Von Maximilian I. bis zum
westfälischen Frieden, Berlin 1983.

\bibitem{mann} Golo {\bf Mann}: Deutsche Geschichte des
19. und 20.Jahrhunderts, Frankfurt /M 1992.

\bibitem{stamm} Eugen {\bf Stamm}: Ein berühmter Unberühmter. Neue
Studien über Konstantin Frantz und den Föderalismus, Konstanz 1948.

\bibitem{wehler} Hans-Ulrich {bf Wehler}: Deutsche
Gesellschaftsgeschichte. Zweiter Band. Von der Reformära bis zur
industriellen und politischen "`Deutschen
Doppelrevolution"'. 1815-1845/49, München 1987.

\bibitem{winters} Peter Jochen {\bf Winters}: Die >>Politik<< des
Johannes Althusius und ihre zeitgenössischen Quellen. Zur
Grundlegung der politischen Wissenschaft im 16. und beginnenden
17.Jahrhundert, Freiburg im Breisgau 1963.

\end{thebibliography}

\end{document}



