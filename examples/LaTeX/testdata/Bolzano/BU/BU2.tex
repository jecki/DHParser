\clearpage\ifnackt\else\linenumbers\fi%
%\PdUsec[Zu RW\,I]{[RW I (§ 1 -- 177)]}\seitenwohne{121}\ignorespaces
\RWsec{I}{1}{177}\noindent\seitenwohne{121}\noindent
Als man vor etwa sechs und dreißig Jahren die Errichtung eigener Lehrkanzeln unter der Benennung \danf{Kanzeln der Religionswissenschaft} im österreichischen Kaiserstaate angeordnet hatte: glaubte der Erste, der an der philosophischen Facultät zu Prag auf einen solchen Lehrstuhl erhoben wurde, B., schon aus dem bloßen Namen der Wissenschaft, die er zu lehren hätte, schließen zu dürfen, er habe die Bestimmung, einen in wissenschaftlicher Form gehaltenen Unterricht in derjenigen Religion, welche er, und diejenigen, die ihn angestellt hatten, als die vollkommenste ansahen, zu ertheilen. Der Name: Religionswissenschaft, schien ihm in keiner anderen als nur in der Bedeutung, wo er die Wissenschaft der vollkommensten Religion bezeichnet, der Name einer Wissenschaft zu seyn, die es unwidersprechlich verdienet, daß Jeder, der einer höheren Bildung genießt, ihr einen Theil seiner Zeit und seiner Kräfte widme. Und daß dieß Letztere eben keine sehr unrichtige Ansicht sey, dürften uns unsere Leser wohl zugeben, auch ohne daß wir sie erst auf jene Grundsätze verweisen, welche die \danf{Wissenschaftslehre} zur Beurtheilung, ob der Begriff einer Wissenschaft zweckmäßig sey oder nicht, aufgestellt hat. Denn welcher Gegenstand hat wohl mehr Ansprüche auf unsere Wißbegierde als die Religion? und sollen wir uns mit der Erkenntniß einer andern als der vollkommensten genügen lassen? \par 
Dieses ist nun der Begriff, den der Vf.\ auch bei der Ausarbeitung seines \danf{Lehrbuchs} zu Grunde gelegt hat. Da er jedoch eigene Erklärungen auch von den Begriffen gibt, welche er mit den Ausdrücken: Wissenschaft und vollkommenste Religion, verbindet: so müssen wir, um ein vollständiges Urtheil über den Zweck, den er sich in diesem Buche gesetzt hat, fällen zu können, auch noch vernehmen, wie diese Erklärungen lauten. \par 
Von dem Worte: Wissenschaft, sagt der Vf., er verstehe darunter hier \danf{einen Inbegriff aller über einen gewissen \seitenw{122} Gegenstand bekannter, merkwürdiger Behauptungen, die so geordnet sind, daß sie bei Jedem, der sie in dieser Anordnung durchdenkt, nicht nur die Überzeugung von ihrer Wahrheit bewirken, sondern ihn auch den Grund dieser Wahrheit, so oft es möglich ist, einsehen lassen.} Diese Erklärung beweiset uns durch ihre Abweichung von dem, was wir in der später geschriebenen Wissenschaftslehre lesen, daß der Vf.\ in seinen Begriffen nicht stehen geblieben sey; daß er insonderheit später etwas genauer unterschieden habe zwischen der Wissenschaft an sich und ihrer Darstellung in einem Lehrbuche; ingleichen daß er die Angabe des Grundes (die Nachweisung des objectiven Zusammenhanges) später nicht mehr so allgemein wie früher für jede, sondern nur für diejenigen Darstellungen einer Wissenschaft gefordert, die für Geübtere, für den Gelehrten bestimmt sind. Die Ansicht, die allein Einfluß auf seine Darstellungsweise übte, daß es ihm obliege, sich nicht mit solchen Beweisen, die bloße Gewißmachungen sind, zu begnügen, sondern daß er bestrebt seyn müsse, den objectiven Grund jeder behaupteten Wahrheit, wie möglich, nachzuweisen; daß er ferner auch jeden Begriff, welchen er aufstelle, zur möglichsten Deutlichkeit erheben müsse. Diese Ansicht besaß B.\ schon sehr frühe, und sein ganzes Buch ist darnach eingerichtet, und gibt dieß Streben auf jeder Seite kund. Ob nun die Leser dieß als eine Tugend ansehen wollen, das müssen wir freilich nur ihrem Ermessen anheim stellen, und gestehen unverholen, daß wir im widrigen Falle nicht hoffen, sie werden an B.'s Lehrbuche der Religionswissenschaft ein besonderes Wohlgefallen finden. \par 
Über den zweiten Ausdruck in seiner obigen Erklärung sagt B., daß er diejenige Religion die vollkommenste nenne, welche für unsere Tugend und Glückseligkeit die zuträglichste oder mindestens doch so zuträglich ist, daß keine andere sie übertrifft. Gegen diese Erklärung kann nun wohl Niemand etwas einzuwenden haben, der sie nicht mißverstehet. Denn es ist keineswegs von einer bloß zeitlichen, noch weniger von einer bloß sinnlichen Glückseligkeit die Rede, sondern ganz allgemein von Allem, was uns in Zeit und Ewigkeit glückselig machen kann; es wird auch, weil nicht von Glückseligkeit \seitenw{123} allein, sondern von Tugend und Glückseligkeit gesprochen wird, vorausgesetzt, daß keine Art von Glückseligkeit gesucht werde, die sich nicht mit den Gesetzen der Tugend verträgt; es ist endlich nicht die Meinung, daß eine Religion zuträglich seyn könne, ohne wahr zu seyn; aber auch eben so wenig, daß wir diese Wahrheit derselben immer im Stande seyn müßten, ohne die Dazwischenkunft eines göttlichen Zeugnisses zu erkennen. \par 
Doch eine neue Bedenklichkeit erregt der Umstand, daß B.\ auch den Begriff der Religion auf eine eigene Weise erklärt, und in einem gewissen Betrachte denselben erweitert, in einem andern aber wenigstens scheinbar beenget. Das Erste, indem er zur Religion nicht bloß Lehren, die Gott und unsere Verhältnisse und Pflichten gegen Gott betreffen, sondern überhaupt alle diejenigen Lehren gezählt wissen will, die einen Einfluß auf unsre Tugend und Glückseligkeit haben; das Andere, indem er den ganzen Inhalt der Religion nur auf sittliche, \dh\  (nach seiner eigenthümlichen Erklärung) nur auf solche Lehren beschränket, für oder wider deren Annahme sich irgend ein in der Natur des Menschen liegendes Interesse erhebet. \par 
Was nun zuerst die Erweiterung anlangt, so mögen wir über die Zweckmäßigkeit derselben urtheilen, wie wir wollen; wir mögen die \RWpar{I}{21} angezogenen Gründe vom ersten bis zum letzten verwerfen; dieß Alles kann uns -- wie sehr leicht einzusehen ist -- nicht im Geringsten berechtigen, diejenige Religion, die der Vf.\ uns in seinem Buche als die vollkommenste erweiset, nicht dafür anzuerkennen, oder den von ihm geführten Beweis nicht überzeugend zu finden. Denn entspricht diese Religion den Forderungen, die der Vf.\ an die vollkommenste macht, auch dann noch, wenn zu dem Inbegriffe der Fragen, die eine solche uns zu beantworten hat, ein Mehreres, als man sonst insgemein thut, gezählt wird: um wie viel sicherer muß sie in dieser Prüfung bestehen, wenn man seine Forderungen minder hoch stellt? --\par 
Ein Anderes wäre es, wenn der Vf.\ Lehren aus dem Gebiete der Religion ausschlöße, die man bisher dazu gerechnet. Doch dieses geschieht durch seine Forderung, daß alle \seitenw{124} Lehren der Religion ein sittliches Interesse haben müßten, gar nicht. Denn diesen Beisatz hat er zu keinem anderen Zwecke gemacht, als um uns einen Bestandtheil in dem Begriffe der Religion, den wir von jeher allgemein mit diesem Worte verbanden, zum deutlichen Bewußtseyn zu bringen. Denn unstreitig gibt es eine ganze Menge von Wahrheiten, die Niemand von uns zur Religion zählt, obgleich sie von der größten Wichtigkeit bald für unsere Glückseligkeit, bald selbst für unsere Tugend sind, ja überdieß in dem innigsten Zusammenhange mit unsern Begriffen von Gott und unsern Verhältnissen und Pflichten gegen ihn stehen. Wie wichtig ist es \zB , zu wissen, daß eintretende \DruckVariante{Fäulniß}{Faülniß} ein fast nie \DruckVariante{täuschendes}{taüschendes} Merkmal des Todes sey; oder daß \DruckVariante{häufiger}{haüfiger} Genuß von Branntwein oder Opium den Menschen an Leib und Seele verderbe \usw\ Wie wichtig ist für denjenigen, der noch nicht fest überzeugt von Gottes Güte ist, die Frage, ob man die mancherlei Einrichtungen in der Natur, welche zum Wohle der Lebendigen dienen, als das Werk einer freiwirkenden Ursache ansehen dürfe oder nicht; ob es auch unendliche Reihen von Ursachen geben könne? \umA\  Alle diese Lehren und Fragen, meint nun B., zähle nur deßhalb Niemand zur Religion, weil sich bei all ihrer Wichtigkeit doch kein entschiedenes Interesse für oder wider sie in eines jeden Menschen Brust erhebe. Hierin nun irre er sich oder nicht: so ist doch so viel gewiß, und wird durch den ganzen Inhalt des Buches (besonders durch den letzten Theil desselben) auf's Kläreste erwiesen, daß der Verf. durch diese eigenthümliche Ansicht sich nicht verleiten ließ, auch nur eine einzige Lehre, welche von wohl unterrichteten Personen, von Theologen selbst zur Religion gezählt wird, aus dem Gebiete derselben zu verbannen. \par 
Ohne Besorgniß also, daß die Eigenthümlichkeiten, die der Vf.\ in den bisher vernommenen Erklärungen an den Tag gelegt hat, sein Werk uns ungenießbar machen werde, können wir hören, was Weiteres er uns zu sagen habe. Wir wissen schon, es sey das Christenthum, und zwar das Christenthum nach der katholischen Auffassungsweise, welches er für die vollkommenste Religion erkläret; und den Beweis dieser hochwichtigen Behauptung stützet er darauf, daß diese \seitenw{125} Religion uns von Gott selbst geoffenbart oder (was er für gleichgeltend hält) bestätiget sey. \par 
Hier ist vor Allem nöthig, zu untersuchen, in welchem Sinne der Vf.\ die Redensart, daß eine Religion von Gott geoffenbart oder bestätiget sey, nehme? Er zählet (Bd. I. \RWpar{I}{23--28}) vier Bedeutungen auf, in denen man das Wort Offenbaren nehme; erkläret aber, daß er nur an der letzten, oder derjenigen festhalten wolle, in der es gleichgeltend mit Bestätigen, oder Zeugenschaftablegen ist. In dieser Bedeutung, sagt er, heiße eine Lehre von Gott geoffenbart, sobald wir an gewissen sinnlichen Zeichen abnehmen können, es sey der Wille Gottes, daß wir sie glauben, weil er sie selbst für wahr erkennet. Erschöpfet nun diese Erklärung in der That schon Alles, was sich die Menschen insgemein vorstellen, wenn sie eine Lehre von Gott geoffenbart oder -- (denn mit diesem Ausdrucke glauben sie wechseln zu dürfen) -- dem ersten Verkündiger derselben eingegeben nennen? Diese Frage können wir in der That nicht umhin zu verneinen; denn nebst demjenigen, was schon B.\ verlangt, daß Gott von einer solchen Lehre uns seinen Willen, sie \DruckVariante{gläubig}{glaübig} anzunehmen, durch bestimmte sinnliche Zeichen zu erkennen gegeben habe, stellt man sich meistentheils vor, daß Gott auch schon die Sätze selbst, aus welchen diese Lehre bestehet, auf eine übernatürliche Weise, durch seine unmittelbare Einwirkung in dem Gemüthe des ersten Verkündigers derselben hervorgebracht habe. Das ist es aber, was unser Vf.\ gar nicht für nothwendig erklärt; indem er behauptet, daß eine Lehre den Namen einer göttlich geoffenbarten, ja auch den einer den Menschen eingegebenen Lehre ansprechen könne, wenn sich gleich nicht erweisen läßt, daß irgend eine übernatürliche oder unmittelbare Wirksamkeit Gottes bei ihrer Entstehung in dem Gemüthe des ersten Verkündigers -- (einen dergleichen es nach seinen Ansichten nicht einmal immer gibt) -- oder bei sonst einer andern Gelegenheit Platz gegriffen habe. Er wagt es selbst in Bezug auf das Christenthum nicht zu behaupten und darthun zu wollen, daß auch nur einige Lehren desselben auf eine so übernatürliche Weise entstanden wären; sondern er ist \seitenw{126} in seinem Beweise für die Göttlichkeit dieser Religion bemühet, uns zu zeigen, daß wir jenen Umstand völlig dahin gestellt lassen, und doch auf das Festeste überzeugt werden können, Gott habe durch Zeichen zu erkennen gegeben, daß er die christliche Lehre als eine von ihm selbst bezeugte von uns will angenommen wissen. -- Es kömmt nun darauf an, ob Alles, was uns B.\ an mehreren Orten (vornehmlich Bd. I. \RWpar{I}{34} 169--176.) über diesen Gegenstand zu bedenken gibt, hinreiche, uns zu der Einsicht zu bringen, daß und wienach wir in der That nicht das Geringste verlieren, wenn das Christenthum nur eine Offenbarung in der von ihm aufgestellten Bedeutung, nämlich nur eine Lehre ist, deren Wahrheit uns Gott bestätiget hat, wobei es dahin gestellt bleibt, durch welche Mittel die erste Vorstellung von jeder einzelnen Lehre desselben unter uns Menschen veranlasset worden sey: oder ob wir fortwährend darauf bestehen, daß uns der übernatürliche Ursprung wenigstens einiger christlichen Lehren nachgewiesen werde, weil es uns mindestens als eine Schmälerung des Werthes und der Wichtigkeit des Christenthums erscheinet, wenn dem Gedanken Raum gelassen wird, daß Gott bei dessen Einführung vielleicht durchweg nur lauter natürliche Kräfte und Werkzeuge angewandt habe. Nur in dem ersten Falle kann uns der in B.'s Lehrbuche befindliche Beweis eine vollkommene Befriedigung gewähren: wer sich in dem zweiten Falle befindet, dem leistet jener Beweis nicht Alles, was er zu seiner Beruhigung und zur Ehre des Christenthums fordern zu müssen glaubet. Wir erachten für unsere Pflicht, dieses mit aller Offenheit vorauszusagen: aber mit eben der Offenheit sey es uns auch erlaubt, zu gestehen, daß wir nicht recht begreifen, wie irgend Jemand, der die genannten §§. durchdacht hat, noch ferner der Meinung zugethan bleiben könne, daß eine unmittelbare oder übernatürliche Wirksamkeit Gottes zu einer Offenbarung nothwendig, und, falls sie doch wirklich irgendwo Statt gefunden hat, erkennbar sey. Denn, um nur von dem Letzteren zu reden, worauf das Erstere schon von selbst wegfällt: die Erwägung des einzigen Umstands, daß wir nicht alle Naturkräfte kennen, reicht hin, um einzusehen, daß wir von keiner wahrgenommenen Erschei\seitenw{127}nung, wenn sie auch noch so außerordentlich wäre, behaupten dürfen, sie könne durch keine Kraft der Natur vermittelt worden seyn. -- Was aber die höhere Wichtigkeit anlangt, die eine Religion in unsern Augen erhält, sobald wir uns vorstellen, daß sie durch eine unmittelbare oder übernatürliche Thätigkeit Gottes eingeführt worden sey: so wollen und können wir allerdings nicht in Abrede stellen, daß es mit dem Einflusse, den diese Vorstellung auf unsere Sinnlichkeit ausübt, seine völlige Richtigkeit habe; und finden eben in diesem Umstande den Grund, warum das Christenthum selbst in mehreren seiner Lehren von übernatürlichen oder unmittelbaren Wirkungen Gottes redet; wie namentlich bei den Sacramenten. Wir rechtfertigen dergleichen Lehren als bildliche, durch die Bemerkung, daß wir uns jene Gnaden und Wohlthaten Gottes, welche das Christenthum von uns als übernatürliche betrachtet wissen will, niemal zu groß, zu wichtig vorstellen können; nie zu viel Eifer an den Tag legen können, um ihrer theilhaft zu werden, und nie auch zu viel Sorgfalt anwenden können, um sie, wenn sie uns einmal zu Theil geworden sind, zu bewahren. Aus dieser einfachen Bemerkung, deren Richtigkeit Jedem einleuchten muß, folgt, daß wir ganz recht daran thun, und jener Vorstellung des Christenthums als eines Bildes, \dh\  nur zu dem Zwecke zu bedienen, um jene Gefühle und Entschließungen in uns hervorzubringen, welche der Natur der Sache in der That angemessen sind; keineswegs aber folgt hieraus die Nothwendigkeit, in diesem Bilde einen Begriff der Sache, wie sie an sich ist, zu finden, und dessen objective Wahrheit vertheidigen zu sollen. Vielmehr wird Jeder, der nur einigermaßen hierüber nachdenken will, durch seine eigene Vernunft mit aller Deutlichkeit einsehen, daß der wahre Werth einer Gabe nie von den Mitteln, durch welche sie von Gott herbeigeführt wird, sondern lediglich von ihrer inneren Beschaffenheit abhängt, daß wir somit \zB\ sicherlich keine Ursache hätten, gewisse Lehren des Christenthums darum geringer zu schätzen, weil es nicht unmöglich wäre, daß Gott, welchem die ganze Natur zu Gebote stehet, sich bloßer natürlicher Mittel bedienet habe, um diesen Lehren ihre Entstehung zu geben. \seitenw{128} \par 
Überdieß müssen wir bemerken, daß B.\ das Vorhandenseyn nicht bloß einiger, sondern unendlich vieler unmittelbarer Einwirkungen Gottes auf die geschaffenen Wesen nicht nur nicht \DruckVariante{läugne}{laügne}, sondern ganz ausdrücklich behaupte, und mit sehr triftigen Gründen erweise. (III. \RWpar{III}{144}) Hiedurch ist Rath vorhanden auch für denjenigen, der sich nicht losreißen kann von der Ansicht, daß der Werth des Christenthums, und die Wichtigkeit der in demselben uns dargebotenen Gnadenmittel denn doch verlöre, wenn nicht erwiesen wäre, daß eine unmittelbare Wirksamkeit Gottes schon bei der Einführung des erstern statt gefunden habe, und bei der Ausspendung der letzteren noch fortwährend statt finde. Denn was B.\ in seinem Beweise für die Wahrheit des Christenthums nur nicht behauptet, weil die sich selbst überlassene Vernunft des Menschen nach seiner Ansicht kein sicheres Kennzeichen hat, um unmittelbare Einwirkungen Gottes von mittelbaren in einem bestimmten Falle zu unterscheiden, das \DruckVariante{läugnet}{laügnet} er doch nicht; und das können wir, nachdem uns durch den im Lehrbuche gelieferten Beweis erst klar geworden ist, daß Gott die Wahrheit des Christenthums bestätiget habe, jetzt erst als eine in diesem Christenthume selbst geoffenbarte (in unsern Augen nun nicht mehr bloß bildlich auszulegende) Lehre und Wahrheit annehmen. Gewiß, wenn es nicht eine bloße Frage der Neugier für uns Menschen ist, auf welche Weise Gott bei Einführung des Christenthums wirksam gewesen sey, und auf welche Weise er noch täglich bei dem Gebrauche der h. Sakramente zu wirken fortfährt; wenn der Mensch wirklich ein Recht hat, diese Frage an Gott zu stellen: so gibt es offenbar keine Art, es sicherer zu beweisen, daß jene Wirksamkeit eine unmittelbare und übernatürliche sey, als wenn man zuvörderst ohne der sich selbst überlassenen Vernunft hierüber ein Urtheil anzumaßen, nach B.'s Anleitung darthut, daß man am Christenthume jedenfalls eine Lehre, deren Wahrheit Gott bestätiget hat, besitze, worauf man dann erst die bejahende Antwort auf jene Frage aus Gottes unfehlbarem Munde selbst herholt. \par 
Und so hätten wir nun gesehen, daß selbst Personen, die das Christenthum als eine übernatürliche und unmittelbare \seitenw{129} Anstalt Gottes aufgefaßt wissen wollen, nicht die geringste Ursache haben, den in B.'s Lehrbuche geführten Beweis im Voraus als einen, der ihnen nichts nützen könne, von sich zu stoßen, sobald sie nur dessen Behauptung von der bildlichen Natur einiger Lehren des Christenthums enger, als es B.\ thut, beschränken. Lasset uns demnach die Beschaffenheit dieses Beweises mit allem Fleiße prüfen, doch nur, nachdem wir erst noch vernommen, wie der Vf.\ daraus, daß das Christenthum, besonders nach der katholischen Auffassungsweise, als eine uns von Gott gegebene Offenbarung angesehen werden könne, den Schluß ziehen will, daß es die vollkommenste Religion für uns sey. \par 
Daraus allein, daß uns Gott etwas geoffenbart hat, fließt ohne Zweifel nur, daß es wahr sey und von uns angenommen und befolgt zu werden verdiene. Wer aber wäre berechtiget zu verlangen, daß Gott, so oft er es für gut befindet, sich einem einzelnen, oder auch einer ganzen Gesellschaft mehrerer Menschen zu offenbaren, immer einen vollständigen religiösen Lehrbegriff zum Inhalte dieser Offenbarung erwähle? Müßte es uns nicht genug seyn, wenn Gott uns zuweilen auch nur mit einer einzigen richtigen Wahrheit bekannt machen wollte? Um so mehr Ursache haben wir aber freilich, Gottes allwaltender Fürsorge für unser sterbliches Geschlecht zu danken, wenn wir beinahe nicht Ein Volk auf Erden antreffen, das nicht in dem Besitze eines gewissen mehr oder weniger vollständig ausgebildeten religiösen Lehrbegriffes wäre, über dessen Wahrheit es eine Bestätigung von Gott selbst empfangen zu haben glaubet. Untersuchen wir nun, ob irgend einer dieser religiösen Lehrbegriffe auch für uns als ein von Gott bestätigter angesehen werden könne; und halten wir (wie B.) dafür, hiezu gehöre als Merkmal, daß dieser Lehrbegriff einen jeden anderen, den wir um seinetwillen verlassen mußten, an sittlicher Zuträglichkeit übertreffe; findet sich endlich (wie dieser Fall hier gleicherweise statt hat), daß der religiöse Lehrbegriff, den wir für den uns geoffenbarten erklären, unter Anderem auch den Satz aufstellt, daß eine jede durch unsre bloße Vernunft erreichbare religiöse Wahrheit einschließlich auch \seitenw{130} die seinige sey: wie sollten wir da nicht berechtiget werden, zu sagen, daß dieser Lehrbegriff der vollkommenste sey, den es für uns nur gibt? Wo doch sollten wir Lehren, die eine höhere sittliche Vollkommenheit hätten, herzubekommen hoffen; wenn Alles, was durch die bloße Vernunft eingesehen werden kann, ohnehin schon zu dieser Religion gehöret; hinsichtlich solcher Lehren aber, die nur auf Gottes Zeugniß angenommen werden können, keine andere Religion etwas aufzuweisen hat, das sittlich zuträglicher wäre? \par 
Der Beweis nun, der für die Göttlichkeit des Christenthums, und zwar nach dessen katholischer Auffassungsweise in dem Lehrbuche geführt wird, beruhet lediglich auf den zwei folgenden Sätzen: \par 
1) Daß wir eine jede Lehre als eine von Gott bestätigte ansehen dürfen und müssen, wenn wir zwei Kennzeichen an ihr gewahren: wenn wir sie a) nach der gewissenhaftesten Prüfung als eine uns sittlich zuträgliche zu erkennen glauben, und überdieß b) gewisse außerordentliche Begebenheiten mit ihr verbunden finden, an denen wir keinen Nutzen ihres Vorhandenseyns absehen könnten, sollten sie uns nicht eben als Zeichen der göttlichen Bestätigung jener Lehren dienen. \par 
2) Daß der christliche, namentlich der katholische Lehrbegriff diese beiden Merkmale einer göttlichen Offenbarung in der That an sich trage. \par 
Den ersten oder allgemeinen Satz erweiset der Vf.\ in dem \RWHSfmt{vierten Hauptstücke des ersten Bandes}\RWHS{Viertes Hauptstück}{135--177} (\RWpar{I}{135--177}), welches die Überschrift: \danf{\RWHSfmt{Möglichkeit und Kennzeichen einer Offenbarung}} führet, auf eine doppelte Art: erst durch Berufung auf die Aussprüche des gemeinen Menschenverstandes, dann durch eine versuchte Auseinandersetzung seiner inneren Gründe. Gesetzt, Jemand vermöchte ihm in diesem letzteren Beweise nicht zu folgen, so könnte demselben doch schon der erste genügen, obgleich wir nicht bergen wollen, daß für einen solchen Zweck,
nämlich um auch Personen zu befriedigen, die noch sehr ungeübt \seitenw{131} in eigenem Nachdenken sind, eine etwas weitere Auseinandersetzung des dort nur Angedeuteten zu wünschen wäre. \par 
Was B.\ Aussprüche des gemeinen Menschenverstandes nenne, und unter welchen Bedingungen er denselben einen so hohen Grad der Verlässigkeit beilege, wissen die Leser bereits aus der \danf{Wissenschaftslehre.} Hier nun zeigt er (\RWpar{I}{141}) (doch eben dieses ist es, was viel ausführlicher hätte geschehen sollen) daß alle Menschen, Gelehrte und Ungelehrte, so verschieden auch ihre religiösen Begriffe sonst waren, durch ihr Benehmen von jeher stillschweigend an den Tag gelegt hätten, sie fühlten sich gedrungen, eine jede Lehre als eine von Gott bestätigte anzuerkennen, sobald sie gegen die sittliche Zuträglichkeit (Heiligkeit) derselben nichts einzuwenden hätten, und überdieß nicht \DruckVariante{läugnen}{laügnen} könnten, daß gewisse außerordentliche Begebenheiten zu ihrer Entstehung, Erhaltung oder Ausbreitung mitgewirkt hätten. Daß es sich wirklich so verhalte, daß man nie etwas Mehreres verlanget habe, daß selbst diejenigen Gelehrten, die von der Nothwendigkeit übernatürlicher, ja unmittelbarer Wirkungen Gottes zur Bestätigung einer Offenbarung sprachen, am Ende doch sich mit Ereignissen begnügten, welche nur einen hohen Grad der Außerordentlichkeit hatten; daß aber auch die außerordentlichsten Thaten, die Jemand verrichtete, nicht hinreichend waren, ihm das Vertrauen der Vernünftigen zu verschaffen, wenn seine Lehre eine unheilige war; daß man in einem solchen Falle, wenn man sich jene Ereignisse nicht anders zu erklären wußte, lieber behauptete, daß sie des Teufels Blendwerke wären, als daß man zugegeben hätte, Gott sey es, der uns die Wahrheit einer solchen Lehre durch diese außerordentlichen Erscheinungen bestätigen wolle; daß man, mehr oder weniger sich deutlich bewußt der Regel, nach der man verfahre, auch bei den täglichen Ereignissen im Leben beurtheilet habe, ob man in ihnen einen ermunternden oder abmahnenden Fingerzeig Gottes anzuerkennen habe oder nicht: das Alles sind Thatsachen, die strenge erwiesen werden könnten. Was aber, ohne erst vieler Beweise zu bedürfen, einleuchten muß, ist, ob das eigenthümliche Urtheil des gemeinen Menschenverstandes, das sich in diesen Thatsachen ausspricht, ganz den Bedingungen gemäß \seitenw{132} sey, welche bei solchen Urtheilen eintreten müssen, wenn sie verlässig seyn sollen. Es betrifft offenbar eine Sache, die sich durch bloße Vernunft, verbunden allenfalls mit gewissen Jedem aus uns zu Gebote stehenden Erfahrungen entscheiden lassen muß; und es lautet in der That so, dieses Urtheil, daß es nicht unserer Sinnlichkeit schmeichelt, sondern vielmehr sie beschränket. \danf{Denn weil die Menschen} (sind B.'s eigene Worte) \danf{gestehen, daß eine Lehre nur dann als eine wahre göttliche Offenbarung angesehen werden dürfe, wenn ihre Annahme sittlich zuträglich ist: so schneiden sie hiedurch sich selbst die Gelegenheit ab, je etwas Unsittliches unter dem Vorwande zu thun, daß es durch eine göttliche Offenbarung erlaubt worden sey; sie legen im Gegentheil sich die Nothwendigkeit auf, so Manches, wogegen sich ihre Sinnlichkeit \DruckVariante{sträubt}{straübt}, als einen von Gott selbst an sie ergangenen Befehl zu erkennen und zu befolgen.} \par
Doch diesen ersten Beweis, den der Vf.\ für seine Theorie von den Kennzeichen einer Offenbarung liefert, führt er hier nur nebenbei auf: in einem wissenschaftlichen Werke wollte und mußte er die Wahrheit einer so wichtigen Lehre aus ihrem objectiven Grunde darthun, und dieß geschieht in dem zweiten Beweise, zu dessen Zergliederung wir nun übergehen. \par
Indem wir eine Lehre für geoffenbart erklären, behaupten wir, aus gewissen sinnlichen Zeichen entnommen zu haben, daß Gott diese Lehre von uns geglaubt wissen wolle. Wir erlauben uns also auf einen Zweck zu schließen, welchen Gott bei gewissen von uns wahrgenommenen Ereignissen in der Welt (jenen Zeichen) habe. Die Lehre von den Kennzeichen einer göttlichen Offenbarung beruhet sonach wesentlich auf der allgemeineren Lehre von der Art und Weise, wie wir aus gegebenen Ereignissen oder Einrichtungen in der Welt den bei denselben zu Grunde liegenden göttlichen Zweck erkennen. Hiemit also macht B.\ \RWpar{I}{143} den Anfang. In der \danf{Wissenschaftslehre} werden (Bd. III. §. 386. u. 379.\WLi{III}{386}\WLi{III}{379}) die noch viel allgemeineren Fragen, wie Zwecke aus gegebenen Handlungen, und Ursachen überhaupt aus gegebenen Wirkungen beurtheilt werden können, besprochen; wohin wir also \seitenw{133} diejenigen Leser verweisen, die diese Lehre noch tiefer begründet sehen wollen. \par
Hier eine freie, vielleicht auch einigermaßen verbesserte Darstellung seiner dießfälligen Lehre: wie sie für unsern gegenwärtigen Zweck genüget. Jedes Ding in der Welt, ja jedes einzelne Ereigniß in derselben kommt nur durch Gottes nähere oder entferntere Wirksamkeit oder Zulassung zu Stande, und bei jedem bezwecket er etwas, das er auch in der That erreichet; weil es bei Gott keine Zwecke, wenigstens keine im strengsten Sinne des Wortes so zu nennende gibt, die er nicht eben darum, weil er sie hat, auch erreichte. Dieser Zweck nun bestehet bei jedem Dinge oder Ereignisse in nichts Anderem als in derjenigen Wirkung, die unter allen, welche dasselbe -- es versteht sich durch den Hinzutritt verschiedener möglicher Umstände und Verhältnisse -- hervorbringen kann, die allerzuträglichste für das Wohl der Geschaffenen ist. So viel bedarf, glauben wir, nicht weiter erwiesen zu werden. Mit eben dem Grade der Zuversicht also, mit dem wir urtheilen können, daß eine gewisse Wirkung die zuträglichste sey, die ein in Rede stehender Gegenstand hervorbringen könnte, und daß die Umstände, die dazu nothwendig sind, sämmtlich in dem Bereiche der Möglichkeit liegen: mit eben dem Grade der Zuversicht dürfen wir urtheilen, daß diese Wirkung auch der wirkliche Zweck des Gegenstandes sey, und eben deßhalb erwarten, daß die noch fehlenden Umstände sich alsbald einstellen, worauf denn jene Wirkung, Gottes beabsichtigter Zweck, in der That eintreten werde. Wir sehen \zB\ Jemand einer Versuchung erliegen, und vermuthen nach dieser Regel, Gott habe den Fall dieses Menschen nur darum zugelassen, damit er, seine Schwäche kennen lernend, in Zukunft aufmerksamer auf sich selbst werde; \usw\ In dem besonderen Falle, wo wir durch unsere eigene Thätigkeit beitragen können, daß jene Umstände, die zur Hervorbringung einer von uns als Gottes Zweck vermutheten Wirkung erforderlich sind, alle zusammenkommen; in diesem besonderen Falle wird es, sobald sich die Sache auch nach der reiflichsten Überlegung uns noch als wahrscheinlich darstellt, unsere Pflicht, und eine nicht bloß wahrscheinliche, sondern gewisse Pflicht, alles zu thun, \seitenw{134} was wir zur Herbeiführung jener Wirkung nur immer für dienlich erachten. Kommt dann dieselbe wirklich zu Stande, so lehrt uns der Erfolg selbst, daß sie von Gott beabsichtiget war. Erreichen wir aber nicht, was wir in der guten Meinung, daß es Gott wolle, anstrebten: so haben wir weder Ursache, uns über unser Bestreben Vorwürfe zu machen; noch auch nur \Hgkorr{Ursache}{Ursachen} zu wähnen, daß dasselbe fruchtlos, oder den Absichten Gottes entgegen gewesen; sondern wir dürfen uns vielmehr dem doppelten Troste hingeben: erstens, daß dasjenige, was die Weisheit Gottes statt des von uns betriebenen Erfolges herbeigeführt hat, ohne Zweifel etwas viel Besseres sey; und zweitens, daß unsere scheinbar vergeblichen Anstrengungen doch so gewiß als sie wohlmeinend waren, auch Gottes Wohlgefallen hatten und durch ersprießliche Folgen gesegnet werden sollen. \par
So viel im Allgemeinen. Wenden wir dieß nun auf eine uns vorliegende einzelne Lehre oder auch auf einen Inbegriff mehrerer an. \par
1. So lange wir an einer Lehre keine sittliche Zuträglichkeit für uns gewahren, \dh\  nicht hoffen können, daß wir durch ihre Annahme sittlich vollkommener würden: so lange können wir die Frage, zu welchem Zwecke uns Gott mit ihr bekannt werden lassen, gewiß nicht damit beantworten, \danf{daß wir sie \DruckVariante{gläubig}{glaübig} annehmen sollen.} Denn selbst in dem Falle, den wir gleich als den günstigsten kennen lernen werden, nämlich wenn wir unsere Bekanntschaft mit dieser Lehre (ihre Entstehung oder die Richtung unserer Aufmerksamkeit auf sie) einem Zusammenflusse der ungewöhnlichsten Ereignisse verdanken, von denen wir so ganz und gar nicht anzugeben wissen, wozu sie dienen sollen, wenn nicht um uns zu ihrer \Hgkorr{gläubigen}{glaubigen} Annahme zu bestimmen: selbst in diesem Falle sind wir zu einem solchen Schlusse nicht berechtigt. Denn der einzige Grund, der für einen solchen Schluß spräche, ist, daß die in Rede stehende Lehre einen sittlichen Nutzen habe, den wir nicht einzusehen vermögen. Aber wer sieht nicht, daß dieser Grund vollkommen aufgewogen wird durch die entgegenstehende Möglichkeit, daß eben so wohl auch die Ereignisse, die uns \seitenw{135} mit jener Lehre bekannt gemacht, einen noch anderweitigen uns unbekannten Nutzen in der Welt haben können? \par
2. Doch wie die außerordentliche Entstehungsart einer Lehre für sich allein, so kann auch ihre sittliche Zuträglichkeit allein und als solche noch keinen Grund darbieten, sie für geoffenbart zu erklären. Was wir mit Recht schließen, so oft wir eine uns sittlich zuträgliche Lehre antreffen, ist, daß es sofort uns obliege, zu untersuchen, ob es nicht Gründe gibt, die ihre Wahrheit darthun. Soll aber die Lehre von Gott uns geoffenbart heißen, so müssen diese Gründe mindestens theilweise in dem Umstande liegen, daß wir aus irgend einem Ereignisse entnehmen, Gott verlange die \DruckVariante{gläubige}{glaübige} Annahme jener Lehre von uns, weil er sie selbst für wahr erkennet. Ein Ereigniß, welchem wir einen so ganz besonderen Zweck von Seite Gottes zuzuschreiben uns wirklich \par
gedrungen fühlen sollten, muß so beschaffen seyn, daß wir, wenn dieser Zweck nicht gelten sollte, sonst keinen anderen vermuthen könnten. Es muß also irgend ein außerordentliches, noch niemal wahrgenommenes Ereigniß seyn; denn Ereignisse, die gewöhnlichen Gesetzen folgen, lassen auch einen, wie immer unbekannten, doch nur gewöhnlichen Zweck vermuthen. Hat das Ereigniß gedient, die Lehre zum Vorschein, oder mindestens zu unsrer Kunde zu bringen: so geben wir eine befriedigende Erklärung, wozu es da gewesen ist, wenn wir sagen, es habe uns ein Zeichen seyn sollen, daß Gott die Wahrheit dieser Lehre bezeuge. Im widrigen Fall bleibt es uns unerklärt. \par
3. Wahr ist es freilich, daß wir uns in der Beurtheilung jener sittlichen Zuträglichkeit, so leicht sie auch zu beurtheilen ist, doch geirrt haben konnten: allein da es einerseits außer diesen zwei Umständen keinen anderen gibt, aus dem wir schließen könnten, daß Gott zu uns gesprochen habe; und da wir andererseits auch keinen hinreichenden Grund zu der Behauptung haben, daß Gott zu uns nie sprechen wolle: so sind wir berechtigt, mit aller Zuversicht vorauszusetzen, er werde nie zulassen, daß wir zu unserm eigenen sittlichen Nachtheile irren, so oft wir mit aller Vorsicht in der Beurtheilung des sittlichen \seitenw{136} Nutzens jener Lehre verfahrend nur dann erst annehmen, daß Gott gesprochen habe, nachdem wir vergeblich versucht, den Zweck der außerordentlichen Ereignisse, durch die wir die Lehre kennen gelernt, auf eine andere Art zu erklären. So gewiß uns also unser Bewußtseyn bezeugt, daß wir in redlicher Absicht geforscht, so gewiß können wir seyn, daß wir an einer Lehre, die sich auf solche Weise uns dargestellt hat, eine wahre göttliche Offenbarung besitzen. --\par
Dieses in wenigen Worten die Theorie B.'s, von der wir glauben, daß es für einen Unbefangenen kaum möglich sey, ihr seinen Beifall zu versagen. Wohl trägt der Vf.\ bei dieser Gelegenheit noch manches Andere vor; so wird \zB\ \RWpar{I}{145} von dem wohl schon für sich klaren Begriffe der sittlichen Zuträglichkeit eine genaue Erklärung gegeben, bei welcher er bis zu dem Begriffe des obersten Sittengesetzes hinaufsteigt; so wird \RWpar{I}{147} der Jedem hinlänglich bekannte Begriff des Außerordentlichen in seine Bestandtheile zerlegt; so wird \RWpar{I}{149} auseinandergesetzt, wie und wodurch es eigentlich geschehe, daß außerordentliche Begebenheiten zur Bestätigung einer göttlichen Offenbarung so ganz besonders taugen; und \RWpar{I}{147} in welcher Verbindung sie mit der zu bestätigenden Lehre stehen müssen: in allen diesen Untersuchungen könnten Fehler unterlaufen seyn, oder die Leser könnten mindestens dergleichen Fehler hier wahrzunehmen meinen, sich von der Richtigkeit seiner dießfälligen Behauptungen nicht überzeuget fühlen, ohne daß dadurch der Beweis des Hauptsatzes, nämlich des oben ausgesprochenen Lehrsatzes von den Kennzeichen einer Offenbarung an seiner Beweiskraft verlieren müßte. Denn was sittlich zuträglich, was außerordentlich sey, in welcher Verbindung eine außerordentliche Begebenheit mit einer Lehre stehen müsse, wenn gesagt werden soll, daß sie zu ihrer Bestätigung diene: das Alles weiß Jeder, auch ohne sich der Bestandtheile, aus denen er diese Begriffe zusammensetzt, mit einer solchen Deutlichkeit bewußt zu seyn, daß er darüber Rede und Antwort zu geben vermöchte. Unser Verf. aber nimmt diese Ausdrücke und Redensarten nicht etwa in einer neuen, sondern in ihrer gewöhnlichen \seitenw{137} durch den bisherigen Sprachgebrauch hinlänglich festgesetzten Bedeutung. Somit bedarf es zur Beurtheilung der Richtigkeit oder Unrichtigkeit seiner Behauptungen über die Kennzeichen einer Offenbarung keineswegs erst eines Eingehens in diese Begriffbestimmungen, die er nur nothwendig findet, um einen Unterricht zu ertheilen, welcher den Namen eines philosophischen ansprechen könne. \par
Um was es sich aber hier in der That handelt, das ist die Frage, wie viele und welche Lehren aus der natürlichen Religion von uns als sicher und unbezweifelbar vorausgesetzt werden müssen, um des Vf., \dh\  im Wesentlichen nur die oben vorgetragenen Schlüsse zwingend zu finden? Hier ist nun nicht zu \DruckVariante{läugnen}{laügnen}, daß gleicherweise wie in unserer Darstellung, auch in B.'s Vortrage seines dießfälligen Beweises das Daseyn Gottes, als eines Wesens, welches der Grund von allen übrigen ist, und Allmacht, Allwissenheit und vollkommenste Heiligkeit hat, vorausgesetzt worden sey. Gleichwohl wenn wir in eine genaue Zergliederung der hier gebrauchten Schlüsse eingehen, wenn wir in dieser Beziehung sie nur noch einmal in unsern Gedanken wiederholen wollen, so werden wir finden, daß B.\ im III. Bd. §. 40.\RWi{III}{40} nicht mit Unrecht angemerkt habe, es sey nicht schlechterdings nöthig, das Daseyn eines solchen Gottes mit einer unerschütterlichen Gewißheit anerkannt zu haben, um die obige Lehre von den Kennzeichen einer Offenbarung nur in so weit richtig zu finden, als hinreicht, um durch eben diese Kennzeichen geleitet im Christenthume oder vielleicht auch noch in irgend einer anderen Religion eine wahre göttliche Offenbarung zu sehen. Wir werden einsehen, es könne recht füglich zu diesem Zwecke genügen, wenn Jemand nur so weit gekommen ist, daß er das Daseyn eines Wesens, wie wir Gott denken, nicht unmöglich, \dh\  (nach den Erklärungen, die unsere Leser aus der Wissenschaftslehre schon kennen) nur nicht mit irgend einer bekannten Wahrheit im Widerspruche findet. Ist uns ein solches Wesen nur möglich (in der von dem Vf.\ sogenannten problematischen Bedeutung dieses Wortes, möglich), dann kann das Daseyn einer \seitenw{138} Religion, wie die christliche, selbst uns zu einem Beweise dienen, daß dasselbe wirklich vorhanden sey, und daß es uns eben durch die außerordentlichen Ereignisse, welchen diese Religion ihre Entstehung, Erhaltung und Ausbreitung verdanket, sein Daseyn und seine Fürsorge für unser Geschlecht auf das Augenscheinlichste zu erkennen gegeben habe. Die Sache verhält sich nämlich ohngefähr eben so, wie wenn wir auf das Vorhandenseyn einiger Menschen oder andrer vernünftiger Wesen an einem Orte schließen, an welchem wir Spuren einer zweckvollen Thätigkeit gewahren. Bedarf es zu einem solchen Schlusse wohl eines Mehreren, als daß wir Wesen der Art, als wir sie zur Erklärung des Wahrgenommenen annehmen wollen, an diesem Orte nur eben nicht unmöglich finden; und daß wir ohne deren Voraussetzung jene Erscheinungen nicht zu erklären wissen? \par
Allein noch auf Einen Umstand müssen wir aufmerksam machen. Aus Feindschaft gegen das Christenthum, wie gegen jede andere Religion, welche sich für geoffenbart ausgibt; zum Theile aber auch aus Veranlassung jener falschen Begriffe, die man sich insgemein von der Beschaffenheit der zu einer Offenbarung nöthigen Wirksamkeit Gottes gebildet hatte, indem man wähnte, daß hiezu irgend ein unmittelbares oder doch übernatürliches Einwirken Gottes gehöre, haben gar manche Gelehrte eine Menge mehr oder weniger scheinbarer Einwürfe gegen die Möglichkeit einer Offenbarung, auch wenn das Daseyn Gottes selbst nicht in Zweifel gesetzt wird, zum Vorschein gebracht. Wenn uns nun diese Einwürfe, mindestens einige gegründet schienen, wenn wir die Möglichkeit einer Offenbarung bezweifelten, dann könnten wir auch an ihre Wirklichkeit nicht glauben; daher denn in unserer obigen Darstellung auch eine ausdrückliche Erwähnung dieser Möglichkeit geschah. Sehr nöthig war es also, daß jene Einwürfe auch von dem Vf.\ beachtet und beseitiget wurden. Damit beschäftiget er sich in dem noch übrigen Theile dieses Hauptstücks. (\RWpar{I}{152--163}) Indessen so viele Einwürfe er hier auch anführt, so ist es immer möglich, daß dem Leser noch andere einfallen oder daß er noch andere hier nicht beantwortete, irgendwo antreffe. Wir hoffen jedoch, wer nur \seitenw{139} die hier besprochenen (ohnehin die \DruckVariante{täuschendsten}{taüschendsten}, die bisher vorgebracht wurden) sammt Allem, was zu ihrer Widerlegung gesagt wird, gehörig durchdacht hat, der wird auch jeden anderen Einwurf sich leicht selbst zu beantworten wissen. Sind ja doch alle diese Einwürfe nur aus unrichtigen Begriffen von dem, was eine Offenbarung sey, oder zu leisten habe, hervorgegangen, und verschwinden von selbst, sobald man die richtigeren Begriffe, die der Vf.\ hierüber schon im Vorhergehenden aufstellte, kennen gelernt hat. Was mit gelehrter Miene von Kategorien und Denkgesetzen, denen durch eine Offenbarung übersinnlicher Gegenstände widersprochen werden müßte; von dem Gesetze der Stätigkeit, das sie verletzen würde; von der Unfähigkeit der menschlichen Sprache zum Ausdruck ihrer Lehren vorgebracht wurde, das Alles kann wahrlich nur Menschen, die noch sehr ungeübt im Denken sind, verwirren. Daß Gott auf unsere Sinne (nämlich unmittelbar) nicht einwirken könne; daß der erste Verkündiger einer Offenbarung unfehlbar seyn müßte; daß Wunder (unmittelbare Wirkungen Gottes) keine Erkennbarkeit für uns haben, auch nicht historisch beglaubigt werden können \umA\  sind Einwürfe, die bei B.'s Theorie gar nicht Platz greifen können; was eben darthun dürfte, daß seine Lehre eine genauere Prüfung von Seite der Sachkundigen verdiene. Diese müßten jedoch ihre Aufmerksamkeit auch schon auf des Vf.\ schärfere Bestimmung des Begriffs der Möglichkeit (\RWpar[\BUparformat{136--8.}]{I}{136--138}) und auf die Behauptung richten, daß nur eine problematische Möglichkeit einer Offenbarung erwiesen zu werden brauche, und daß die vollkommne Möglichkeit einer Sache, deren Verwirklichung nur von Gott selbst abhängt, auch ihre Wirklichkeit schon zur Folge habe. \par
Eine besondere Art von Einwürfen gegen die Möglichkeit einer höheren Offenbarung nimmt man bekanntlich daher, daß sie dem menschlichen Geschlechte nicht nothwendig, nicht einmal zuträglich wäre. Mit der Beseitigung dieser Einwürfe und dem Beweise des Gegentheils war der Vf.\ schon in dem vorhergehenden, nämlich dem \RWHSfmt{dritten Hauptstücke}\RWHS{Drittes Hauptstück}{96--134} (\RWpar{I}{95--134}) beschäftigt, dem er die Überschrift: \seitenw{140} \danf{\RWHSfmt{von der Nothwendigkeit einer Offenbarung},} gegeben. Es könnte ein Verstoß gegen die gute Ordnung scheinen, daß dieses Hauptstück dem von der Möglichkeit und den Kennzeichen einer Offenbarung vorhergehet; denn wenn die Nothwendigkeit einer Sache bereits erwiesen ist, was braucht es da noch vieler Beweise für ihre Möglichkeit? So spräche man mit Recht, wenn B.\ unter Nothwendigkeit hierorts verstände, was man im strengsten Sinne so nennt; allein er sagt uns \RWpar{I}{97} ausdrücklich, daß er unter Nothwendigkeit einer Offenbarung nichts Anderes verstehe als eine hohe Erwünschlichkeit derselben, und auch diese wolle er hier behaupten und beweisen nur mit dem beschränkenden Beisatze: \danf{so viel wir Menschen die Sache zu beurtheilen vermögen;} wodurch bedeutet werden soll, daß Gott wohl Gründe haben könne, um derentwillen er, ohne ungerecht oder grausam, ja nur ungütig zu seyn, einzelnen Menschen oder auch ganzen Völkern und Zeitaltern das Geschenk einer göttlichen Offenbarung verweigert. Eine so zu verstehende Nothwendigkeit läßt sich nun allerdings leicht darthun, ehe man noch ihre Möglichkeit außer Zweifel gesetzt hat. \par
Der Vf.\ gehet auf folgende Art zu Werke: Er setzet auseinander \par
1) wie nothwendig eine höhere Offenbarung selbst für die Gebildetsten unsers Geschlechtes sey, indem er aufzählt, welche Dunkelheiten die natürliche Religion auch in ihrer vollkommensten Gestalt hat; er zeigt dann \par
2) wie noch viel größer das Bedürfniß einer Offenbarung für das menschliche Geschlecht im Ganzen sey; welches er wieder auf eine doppelte Weise darthut: \par
a) aus der Geschichte der Verirrungen, in welche nicht nur die große Menge der Menschen, sondern auch die Gelehrten zu allen Zeiten verfielen; b) aus der Natur der Sache; indem er die Schwierigkeiten darlegt, die einer allgemeinen Verbreitung der natürlichen Religion in einiger Reinheit und Vollständigkeit entgegen stehen würden, wenn keine Offenbarung dabei zu Hülfe käme. \seitenw{141} 3) Endlich folgt eine Prüfung der vornehmsten Einwürfe, die gegen die Nothwendigkeit einer Offenbarung vorgebracht worden sind. \par
Wenn nun Schreiber dieses vermeinet, daß das Verfahren des Vf.\ in diesem Hauptstücke oder sonst irgendwo etwas des Beifalls, ja der Nachahmung Würdiges darbiete: so ist es (um dieß hier ein für allemal zu sagen) jederzeit nur der Gedankengang, die Anlage, nie die wirkliche Ausführung, die immer zu kurz und unvollständig ist, und in historischen Partien vollends nur wenig auf sie verwendete Mühe an den Tag legt. \par
Da wir, wenn keine Offenbarung wäre, begreiflicher Weise nur an die natürliche Religion -- (so nennt der Vf.\ eben diejenige, deren Lehren ohne ein göttliches Zeugniß erkannt werden können) -- uns halten müßten: so handelte es sich bei dieser Untersuchung vor Allem darum, welche Darstellung der natürlichen Religion hier zu Grunde gelegt werden müsse, da Verschiedene den Inhalt derselben auch verschiedentlich bestimmen? So viel ist offenbar, nur wenn wir erweisen, daß die natürliche Religion selbst in ihrer vollkommensten Gestalt (die sie vielleicht nur eben durch Hülfe mancher göttlichen Offenbarung erreichte) noch mangelhaft genug sey, um den Wunsch nach einer höheren Belehrung übrig zu lassen, werden wir die Nothwendigkeit dieser letztern so einleuchtend, als an sich möglich ist, darthun. Aber wo ist nun diese vollkommenste Gestalt der natürlichen Religion wohl anzutreffen? Durfte der Vf.\ so eitel seyn, zu glauben, der Inbegriff nur eben derjenigen religiösen Lehrsätze, die ihm nach seinem eigenen philosophischen Systeme als die einzig richtigen und durch unsre bloße Vernunft erweislichen erscheinen, bilde den Inhalt der natürlichen Religion in ihrer vollkommensten Gestalt? So dachte er keineswegs, sondern er meinte, daß selbst der gelehrteste Mensch, auf so viel neue der übrigen Menschheit verborgene Ansichten er durch sein Nachdenken auch gekommen seyn möchte, ihnen doch schwerlich ein unbedingtes Vertrauen schenken dürfe, daß er vielmehr immer nur das als eine ausgemachte Wahrheit ansehen könne, worüber auch andre, ja \seitenw{142} alle Menschen mit ihm zusammenstimmen; daß er somit geneigt seyn werde, zum Inhalte der natürlichen Religion in ihrer vollkommensten Gestalt, mindestens zu dem gewissen Inhalte derselben eben nichts Mehreres zu zählen, als was man nach \RWpar{I}{20} u. \RWpar[64.]{I}{64} die natürliche Religion des menschlichen Geschlechtes überhaupt nennen könnte. Nur diese ist es denn, die der Vf.\ bei seinen Untersuchungen in diesem Hauptstücke und auch sonst überall, namentlich in der Folge bei der Beurtheilung der einzelnen Lehren des Christenthums zu Grunde gelegt hat. Können wir uns beschweren, daß er zu viel voraussetze, wenn er dieß thut? Wer könnte sicherer gehen, als wer nichts Anderes annimmt als das, worüber alle Menschen auf Erden einverstanden sind? In der That aber hätte es wenig auf sich, wenn wir auch mancher Behauptung, die der Vf.\ in diesem Hauptstücke aufstellt, nicht beistimmen könnten: es ist genug, wenn wir durch Alles, was theils hier, theils in der Folge erst, bei Untersuchung der sittlichen Zuträglichkeit des christlichen, namentlich des katholisch-christlichen Lehrbegriffes gesagt wird, zu der Überzeugung gelangen, daß eine Religion, die solche Zuthaten zu der natürlichen hinzufügt, durch ihre allgemeinere Verbreitung unter den Menschen einen überwiegend wohlthätigen Einfluß auf die Verbesserung ihrer Sitten und auf ihr Wohlseyn ausüben würde; ja es ist schon genug, wenn wir nur nach einer gewissenhaften Prüfung des Einflusses, welchen die Annahme dieser Religion auf unsere eigene Sittlichkeit verspricht, alle Ursache haben, zu hoffen, daß sie für uns sich zuträglich erweisen werde. \par
Indessen wollen wir doch noch mit wenigen Worten erzählen, welche Mängel es vornehmlich sind, die der Vf.\ an der natürlichen Religion in ihrer vollkommensten Gestalt gewahr wird, um derentwillen er eine höhere Offenbarung auch für den Weisesten erwünschlich findet. Schon die Lehre von Gott, meinet er, sey in der natürlichen Religion mit vielen Dunkelheiten umhüllt, da es unserm Verstande so schwer fällt, sich ein unendliches Wesen zu denken. -- Unsere Unsterblichkeit, die endlose Fortdauer unseres Wesens mit Bewußtseyn und Rückerinnerung, läßt sich nach ihm nicht einmal für \seitenw{143} einen gewissen Lehrsatz der natürlichen Religion des menschlichen Geschlechtes erklären, weil doch so haüfig daran gezweifelt wurde und noch wird, und weil die große Menge der Menschen, die eine Fortdauer annahm, sie eigentlich nicht als eine durch die Vernunft erweisliche, sondern geoffenbarte Wahrheit annahm. -- Noch weniger fühlt sich die menschliche Vernunft im Stande, die wichtige Frage von der Vergebung der Sünden aus sich selbst zu entscheiden; was der Vf.\ theils aus den höchst unzulänglichen Sühnund Genugthuungsmitteln, die man bei allen Völkern antrifft, theils aus dem Streite der Weltweisen über diesen Gegenstand darthut. -- Nicht minder groß findet B.\ unsere natürliche Unfähigkeit in der Erklärung des Ursprungs sowohl als auch des Zweckes der mannigfaltigen Übel, die wir in der Welt antreffen; und diese Unfähigkeit beweiset er theils aus den Verirrungen, in welche die Menschen durch das Bestreben einer Erklärung dieser Übel geriethen, theils aus den Mängeln, welche selbst die gelungensten Theodiceen haben. -- In ihrer Ethik, sagt der Vf.\ weiter, ist die natürliche Religion vergleichungsweise noch am Vollkommensten; doch könnte auch hier eine Offenbarung wichtige Dienste leisten, indem sie strittige oder sehr schwierige Pflichten bestimmter festsetzen und als göttliche Gebote darstellen könnte. Besonders arm aber ist (nach des Vf.\ Meinung) die natürliche Religion in ihrer Asketik oder Tugendmittellehre, \zB\ um dem an der Möglichkeit seiner Besserung beinahe schon verzweifelnden Gewohnheitssünder neuen Muth einzuflößen \usw\ Kann man nun wohl in irgend einer dieser Behauptungen etwas Übertriebenes nachweisen? -- Vernehmen wir aber noch kurz, wie der Vf.\ zum Beschlusse dieses Hauptstücks (\RWpar{I}{123}) die Frage, \danf{ob das Bedürfniß einer Offenbarung für den Gebildeten oder den Ungebildeten größer sey, und von wem es lebhafter empfunden werde?} beantwortet. Der Ungebildete, sagt er, hat einen größeren Schaden davon, wenn ihm keine Offenbarung zu Theil wird; der Nutzen aber, den der Gebildete aus dem Besitze einer Offenbarung zu ziehen vermag, ist größer. Gleichwohl ist die Empfindung jenes Schadens bei dem Ungebildeten und \seitenw{144} sein Wunsch nach einer Offenbarung schwächer; der Gebildete aber wird sein an sich kleineres Bedürfniß in der That stärker fühlen. -- Kann Jemand widersprechen? \par
Unsere Leser kennen nun ohngefähr das Wichtigste, was der Vf.\ für seine Theorie von der Möglichkeit und den Kennzeichen einer Offenbarung vorbringt; sie sind insofern auch schon im Stande, zu beurtheilen, was mit derselben, wenn sie die Probe bestehet, gewonnen seyn werde? Wir werden dann, wenn untersucht werden soll, ob eine Religion göttlich geoffenbart sey, nicht mehr nach übernatürlichen Ereignissen, wohl gar nach Wundern, die nur durch Gottes unmittelbare Wirksamkeit entstanden seyn könnten, fragen, sondern es wird uns als \DruckVariante{äußeres}{aüßeres} Merkmal genügen, wenn diese Religion ihre Entstehung, Erhaltung oder Ausbreitung einem Zusammenflusse mehrerer ungewöhnlicher Ereignisse verdanket. Wir werden eben deßhalb durchaus nicht nöthig haben, über den eigentlichen Hergang bei diesen Ereignissen ein bestimmtes Urtheil zu fällen, wenn nur einleuchtend ist, daß etwas Ungewöhnliches geschehen seyn müsse, der Erfolg mag nun auf diese oder auf eine andere Weise herbeigeführt worden seyn; wir werden es nimmer als einen gegen die Göttlichkeit einer Lehre zu erhebenden Einwurf betrachten, daß auch menschliche Irrthümer und Leidenschaften zu ihrer Entstehung oder Ausbreitung Einiges beigetragen haben; wir werden uns aber auch nimmer mit einem bloßen Beweise des außerordentlichen oder wenn man will, selbst übernatürlichen Ursprunges einer Lehre, so überzeugend derselbe auch dargethan sey, begnügen, sondern verlangen, daß man uns auch noch die sittliche Zuträglichkeit dieser Lehre und also auch ihre Vernunftmäßigkeit beweise; wir werden, wenn es der religiösen Lehrbegriffe mehrere gibt, die durch sehr ungewöhnliche Ereignisse entstanden, erhalten oder ausgebreitet wurden, die Frage, welcher aus diesen Lehrbegriffen als uns geoffenbart angesehen werden dürfe und müsse, nicht durch historische, wohl gar linguistische Untersuchungen entscheiden wollen, sondern wir werden begreifen, daß dieses lediglich von der sittlichen Zuträglichkeit dieser Lehrbegriffe, also von einer \seitenw{145} rein psychologischen oder genauer noch, von einer moralischen Erörterung abhange; wir werden endlich auch nicht mehr besorgen, daß sich aus Unvollkommenheit unsrer naturwissenschaftlichen, historischen oder auch philosophischen Kenntnisse irgend ein Irrthum in unsre religiösen Überzeugungen eingeschlichen habe, und die Nothwendigkeit einer Abänderung derselben nur für den Fall, dann aber auch mit Freuden anerkennen, wenn man uns zeigt, daß eine Lehre, die wir bisher als der Vernunft zuwider, verwarfen, gar wohl auch eine vernünftige Auffassung und einen sittlichen Gebrauch verstatte. -- Sollten einige dieser rasch vorgetragenen Folgerungen unseren Lesern nicht sogleich einleuchten, so bitten wir das Gesagte nur wiederholt durchzugehen oder nöthigenfalls die weitere Auseinandersetzung des Buches selbst zu berathen; und wir hoffen, auch sie werden zu Ansichten übergehen, welche man, ohne sie eben noch deutlich ausgesprochen zu haben, zu unserer Zeit schon allenthalben ahnet. \par
Da es vornehmlich nur das vierte und dritte Hauptstück des ersten Bandes waren, mit deren Inhalte wir unsere Leser bis jetzt bekannt gemacht, so müssen wir, ehe wir weiter gehen, ihnen doch auch noch berichten, was in den beiden früheren Hauptstücken vorkommt. Das \RWHSfmt{erste Hauptstück}\RWHS{Erstes Hauptstück}{9--59} führet die Überschrift: \danf{\RWHSfmt{von dem Begriffe der Religion, ihren verschiedenen Arten und dem pflichtmäßigen Verhalten gegen sie}} (\RWpar{I}{9--59}), und enthält außer dem, was wir demselben schon entnommen haben, nur eine einzige Erklärung noch, welche von einigem Einflusse auf das Folgende seyn kann; es ist dieß die Erklärung des Begriffes einer Gesellschaftsreligion (\RWpar{I}{22}), welche nur darum nicht ohne Wichtigkeit ist, weil B.\ die katholische Religion für eine solche Gesellschaftsreligion hält, und daher auch nur nach jener Erklärung bestimmt, was man zum Inhalte derselben zu zählen oder nicht zu zählen habe. Er sagt aber, daß man, sich haltend an den Sprachgebrauch und an die bei ähnlichen Wortbildungen befolgte Regel unter der Religion einer gewissen Gesellschaft füglich nichts Anderes verstehen dürfe als einen Inbegriff aller derjenigen religiösen Lehren, zu welchen sich alle oder doch fast alle \seitenw{146} Mitglieder dieser Gesellschaft, für welche jene Lehren Verständlichkeit und religiöse Wichtigkeit haben, bekennen. Wir sehen in der That nicht ein, wie man ihm hierin widersprechen könnte? Denn wie nothwendig der beschränkende Beisatz sey, daß man bei einer jeden Lehre nur auf das Urtheil derer zu achten habe, für welche sie Verständlichkeit und religiöse Wichtigkeit hat, muß Jedem einleuchtend werden, sobald er sich erinnert, daß man in jeder religiösen Gesellschaft auch Kinder und Personen habe, die auf einer noch sehr niedrigen Stufe der Ausbildung stehen, und die man eben deßhalb billig mit Belehrungen verschonet, für welche sie noch kein Bedürfniß haben. \par
Was nun sonst noch in diesem Hauptstücke vorkommt, namentlich Alles, was \RWpar{I}{37--59} über das pflichtmäßige Verhalten beigebracht wird, das wir ein Jeder theils gegen unsere eigenen, theils gegen die religiösen Überzeugungen unserer Mitmenschen zu beobachten haben; ingleichen über die gewöhnlichsten Verstöße, welche wir uns in dieser Hinsicht zu Schuld kommen lassen: dieß Alles stehet da ohne irgend eine Verbindung mit dem Folgenden, und ohne alle Anwendung auf dasselbe. Gleichwohl befindet sich gerade in dieser Abtheilung des Buches, und zwar in \RWpar{I}{38} eine \DruckVariante{Äußerung}{Aüßerung}, die, in so unschuldiger Absicht sie der Vf.\ auch mittheilt, doch einen unsäglichen Schaden ihm oder vielmehr seinem Systeme verursachet hat, weil man von daher Veranlassung nahm, seinen Charakter zu verdächtigen, und zu behaupten, daß es in seinem ganzen Systeme ihm nirgends um Wahrheit, sondern statt deren nur um Nützlichkeit, oder vielmehr um noch was Ärgeres zu thun sey! Es ist die \DruckVariante{Äußerung}{Aüßerung}, daß wir, wenn einmal überwiegende Gründe für eine Meinung da sind, und wenn wir überdieß deutlich erkennen, daß unsre Annahme derselben auch für den Fall, wenn sie doch irrig wäre, gar keine Nachtheile brächte, -- wohl daran thun, wenn wir die Aufmerksamkeit unsers Geistes von den ihr entgegenstehenden Zweifeln und Einwürfen (die ihre überwiegende Wahrscheinlichkeit doch immer nicht aufheben könnten) abziehen. Wir wollen nicht eine Sylbe zur Vertheidigung dieser Vorschrift sagen; sie sey so falsch und \seitenw{147} so verwerflich, als man nur immer will; sie verdiene im vollsten Sinne des Worts eine Ausgeburt der Hölle (wie Krug sie genannt hat) zu heißen: nur so viel erklären wir hier wiederholt, nicht die geringste Anwendung von dieser Regel macht der Vf.\ in seinem folgenden Beweise für den Katholicismus. Und so viel muß auch Jedem, der nur die Augen offen hat, einleuchten: in einem solchen Beweise ist nicht einmal die entfernteste Gelegenheit vorhanden, Gebrauch von jener Regel zu machen. Denn nicht, so lange man die Frage, ob der Katholicismus auch sittlich zuträglich sey, und als von Gott bestätiget angesehen werden könne, noch untersucht, höchstens erst nach Beendigung unsers Beweises könnten wir, wenn etwa derselbe noch einige Fragen zurückließe, die wir uns nicht zu beantworten wissen, von denen wir gleichwohl mit aller Deutlichkeit erkennen, daß ihre Beantwortung ihn auf keinen Fall umzustoßen vermöge, von der besagten Regel die Anwendung machen, daß wir nicht länger uns bei der Erörterung jener müßigen Fragen aufhaltend, unsere Aufmerksamkeit auf andre nützliche Gegenstände richten. \par
Wichtiger ist das \RWHSfmt{zweite Hauptstück},\RWHS{Zweites Hauptstück}{64--95} das \RWpar{I}{64--95} einen \danf{\RWHSfmt{kurzen Abriß der natürlichen Religion}} liefert. Es werden aber hier nicht bloß solche Lehren, die B., weil sie von allen oder doch fast allen Menschen, welche darüber nachgedacht haben, einstimmig angenommen werden, zu den entschiedenen Wahrheiten der natürlichen Religion des menschlichen Geschlechtes zählet, aufgeführt, sondern er theilt hier gelegenheitlich auch mehrere Ansichten mit, die er bloß für die seinigen ausgeben kann. Von jenen sowohl als von diesen bemühet er sich, damit sein Vortrag den Namen eines wissenschaftlichen verdiene, die objectiven Gründe, auf denen sie etwa beruhen dürften, in Kürze anzudeuten. Obgleich er nun, wie wir schon sagten, in seinem Beweise für das Christenthum nur lauter solche Lehren der natürlichen Religion voraussetzt, die den entschiedenen Wahrheiten derselben beigezählt werden können: so befindet sich doch \seitenw{148} auch unter dem, was er hier nur als seine subjective Ansicht vorträgt, gar Manches, das wir, wenn wir uns erst von dessen Richtigkeit überzeugten, sehr wohl benützen können, um unsern Beweis für das Christenthum noch tiefer zu begründen, oder was sonst eine anderweitige Anwendung zuläßt. Wir wollen Einiges davon dem Leser vorlegen. \par
Was in der Einschaltung (\RWpar{I}{60--63}) über die kritische Philosophie und über die neueste Art des Philosophirens in Deutschland überhaupt gesagt ist, hat nur den Zweck, den Vf.\ einigermaßen darüber zu rechtfertigen, daß er sich weder an jene noch diese angeschlossen, sondern ein eigenes System befolgt hat. Dieß können wir hier um so mehr übergehen, da unsere Leser das Wichtigste schon aus unserer von der \danf{Wissenschaftslehre} ihnen gegebenen Übersicht kennen. \par
Von Gott gibt der Vf.\ \RWpar{I}{66} die Erklärung, daß er das unbedingt Wirkliche sey; eine Erklärung, aus welcher sofort einleuchtet, daß dessen Daseyn nicht objectiv begründet werden könne, daß aber wohl eine Gewißmachung dieses Daseyns versucht werden dürfe und müsse. So wenig nun auch der \RWpar{I}{67} gelieferte Beweis in dieser Hinsicht zu wünschen übrig läßt: so scheint er dem Vf.\ doch nicht genug gethan zu haben, weil er in der später geschriebenen \danf{Athanasia} eine noch umständlichere Beweisführung (in der ersten Auflage S.\,292 ff.\ATi{1}{292}, in der zweiten S.\,321 ff.\ATi{2}{321 ff.}) versuchte. Das Merkwürdige in beiden Beweisen ist, daß der Vf.\ nicht, wie es insgemein geschieht, voraussetzt, daß eine Reihe von Ursachen oder Bedingungen nie in's Unendliche fortgehen könne, vielmehr (\RWpar{I}{68}) die Irrigkeit dieser Voraussetzung selbst darthut. -- Eigenthümlich ist auch sein Begriff von der Allvollkommenheit Gottes, welche er \RWpar{I}{74} dahin erklärt, daß Gott alle Kräfte, die neben einander möglich sind, und eine jede in jenem höchsten Grade, in welchem sie neben den übrigen möglich ist, vereinigt. Durch diese Erklärung wird eine Menge Schwierigkeiten beseitigt, die man sonst in der Lehre von Gottes Eigenschaften fand, indem man die eine derselben mit der andern, \zB\ die Güte mit der Gerechtigkeit \udgl\  \seitenw{149} im Widerspruche glaubte. Was vollends von Schelling und Hegel gegen die Lehre von göttlichen Eigenschaften vorgebracht wurde, beruhet auf so unklaren (in der Wissenschaftslehre berichtigten) Begriffen, und widerspricht sich selbst so offenbar, daß es gar keiner besonderen Widerlegung bedurfte. -- Unter den \danf{Folgerungen,} die der Vf.\ \RWpar{I}{81} aus Gottes Eigenschaften ziehet, liest man, es müsse zu aller Zeit eine unendliche Menge von Geschöpfen, die der Glückseligkeit empfänglich sind, gegeben haben und geben; keines derselben könne als bloßes Mittel für die übrigen dienen; dagegen lasse sich nicht verlangen, weder daß Gott alle mit gleichen Kräften ursprünglich ausgerüstet habe, noch daß er die gleichgeschaffenen auf gleiche Weise behandle; wohl aber, daß er auf jede sittlich gute Handlung eine Erhöhung, auf jede sittlich böse eine Verminderung der Glückseligkeit erfolgen lasse \usw\ -- Vorsichtiger als gewöhnlich geht der Vf.\ \RWpar{I}{83} zu Werke, wenn er erweisen will, daß \danf{auch die Beschaffenheit der Welt das Daseyn Gottes und seine Eigenschaften bestätiget,} \dh\  wenn er den sogenannten physikotheologischen Beweis in seinen Grundzügen darlegt. Nur dann, behauptet er nämlich, sind wir \danf{berechtiget, anzunehmen, daß ein gewisser Gegenstand das Werk eines vernünftigen Wesens und von demselben zu einem bestimmten Zwecke hervorgebracht sey, wenn wir a) keinen Grund finden, das Daseyn und die Einrichtungen dieses Gegenstandes für an sich nothwendig zu halten; auch b) keine Unmöglichkeit darin erkennen, daß ein Wesen, wie jenes, dem wir das Daseyn desselben zuschreiben wollen, vorhanden sey, ihn hervorgebracht und so eingerichtet habe; wenn wir vielmehr c) wahrnehmen, daß der Gegenstand durch seine Einrichtungen zur Hervorbringung jener Wirkung, die wir als seinen Zweck angeben wollen, in der That tauge, daß ferner d) diese Wirkung ein Erfolg solcher Art sey, daß ihn ein Wesen, wie das unsrige, sehr wohl beabsichtigen konnte; und wenn er endlich e) bei einer andern Einrichtung nicht mehr im Stande wäre, eben dieselbe Wirkung hervorzubringen. Je größer sodann die Anzahl jener nicht an sich nothwendigen Beschaffenheiten \seitenw{150} des Gegenstands ist, und je weniger er, wenn auch nur einige derselben anders wären, zur Hervorbringung der angegebenen Wirkung noch ferner tauglich bliebe: mit desto größerer Zuversicht können wir sagen, daß er von jenem vernünftigen Wesen und nur zu diesem Zwecke hervorgebracht sey.} \par
Die Lehre von der Unsterblichkeit erachtet B., wie wir schon wissen, für keinen ganz unbezweifelbaren Lehrsatz in der natürlichen Religion des menschlichen Geschlechtes; indessen gibt er uns \RWpar{I}{85} auf 12 Seiten eine gedrängte Übersicht seiner wichtigsten Gründe für die Einfachheit unserer Seele und ihre endlose Fortdauer mit Bewußtseyn und Rückerinnerung; wobei sich jedoch von selbst verstehet, daß diese Gründe in der \danf{Athanasia} viel vollständiger ausgeführt und mit verschiedenen andern vermehrt sind.\BUfootnote{%
	Es that uns leid, zu sehen, daß auch Hr. Prof. Beneke, einer der gründlichsten und scharfsinnigsten Forscher unserer Zeit, sich bisher noch nicht hat entschließen können, die Athanasia einer sorgfältigeren Durchsicht zu würdigen. In seinem neuesten Werke: \danf{System der Metaphysik und Religionsphilosophie} (Berlin, 1840), finden wir nämlich S.\,417 zwar eine Erwähnung der Athanasia, es wird auch sogar als eine Probe, wie Bolzano die Einfachheit der Seele erweisen wolle, eine Stelle aus S.\,37 der 2ten Aufl. herausgehoben; allein Hr. Beneke rügt, daß diese Stelle statt der Einfachheit bloß die Einheit (Einerleiheit oder Identität) der Seele darthue. Dieß hat nun seine völlige Richtigkeit; da es aber aus dem Zusammenhange deutlich hervorgehet, daß und warum Bolzano an jenem Orte nicht von der Einfachheit, sondern der Einerleiheit der Seele rede; und da auch Hr. Beneke selbst gestehet, daß Bolzano \seitenw{151} dort nicht die Einfachheit, sondern die Einheit folgere: wie können wir uns erklären, daß er von jener Stelle gleichwohl verlangt, sie solle die Einfachheit darthun; wenn wir nicht annehmen dürfen, er habe den ganzen Zusammenhang derselben übersehen? Und wie hätte auch ein so geübter Denker, wenn er \Druckfehlerkorr{den}{deu} dießfälligen Abschnitt aufmerksam durchgelesen hätte, \DruckVariante{geäußert}{geaüßert}, daß Bolzano die Unsterblichkeit durch die Einheit, diese aber \danf{der Hauptsache nach mit denselben Gründen wie M. Mendelsohn} erweise; während hier in der That zuerst die Einfachheit der Seele und aus dieser ihre endlose Fortdauer, dann aber in einer zweiten Abtheilung erst ihre Einheit oder Einerleiheit dargethan wird? -- Wie sehr wäre doch zu wünschen, daß sich gerade Männer, wie Beneke, Reinhold, Fichte, Drobisch, Erdmann entschlößen, Bolzanos philosophische Ansichten genauer kennen zu lernen!
}\par
Für die praktischen Wissenschaften: Moral und Rechtslehre könnte nichts wichtiger seyn, als der dem Vf.\ eigenthümliche Begriff des obersten Sittengesetzes, wenn er als richtig anerkannt werden sollte; daß es nämlich sey eine praktische Wahrheit, aus welcher alle übrigen praktischen Wahrheiten durch einen bloß theoretischen Untersatz, wie die Folgen aus ihrem Grunde, objectiv abfolgen. Durch diesen Begriff erkennt man sofort eine Menge Formeln, mit denen man sich bisher hingehalten hatte, als offenbar verfehlt und \seitenw{151} untauglich, der Wissenschaft zu einer festen Grundlage zu dienen. Wird man aber erst allgemein anerkennen, daß es eine solche praktische Grundwahrheit gebe: dann wird man sich wohl auch bald darüber vereinigen, wie diese Wahrheit laute; möchte man immerhin die \RWpar{I}{88} gegebene Anleitung zu ihrer Aufsuchung als eine unrichtige verlassen, oder (was wir jedenfalls im Voraus zugestehen) ihre Ausführung a. a. O. viel zu kurz und zu mangelhaft finden. Das Ergebniß der Untersuchungen sey endlich dasselbe, das B.\ herausbringt, oder es laute so wesentlich anders, daß auch in den wenigen Begriffen und Lehrsätzen, welche er noch \RWpar{I}{94} u. \RWpar[95.]{I}{95} aus der natürlichen Asketik aushebt, Einiges abgeändert werden müßte: doch stehet immer noch nicht zu besorgen, dieß werde nöthigen, von dem Vf.\ auch in demjenigen Theile seiner Behauptungen, auf denen der Beweis seines Hauptsatzes ruhet, abzugehen. Denn in dem ganzen Werke zieht er nicht eine einzige zu diesem Beweise dienende Folgerung aus seinem Princip, welche wir nicht auch als eine schon durch das Urtheil des gemeinen Menschenverstandes entschiedene Wahrheit betrachten müßten; und wenn er insbesondere (im dritten Haupttheile) dazu kommt, den sittlichen Nutzen der einzelnen Lehren des Katholicismus zu zeigen -- (bei welcher Gelegenheit man am Ehesten noch eine durch seine eigenthümliche Ansicht vom \seitenw{152} obersten Sittengesetze veranlaßte Abweichung von dem, was Andere wahr finden werden, besorgen könnte) -- was ist es denn eigentlich, worein er den sittlichen Nutzen der hier betrachteten Lehren setzet? Wir gehen Alles durch, und finden vom Anfang bis zum Ende nichts Anderes, als daß er sagt, bald eine gewisse Lehre mache uns geneigter, das zu erfüllen, was wir auf irgend eine Weise als unsere Pflicht erkannten; bald sie erhöhe unseren Eifer in der Entwicklung und Ausbildung unserer Kräfte und Anlagen; bald sie mache uns mäßiger in der Befriedigung unserer sinnlichen Triebe, und setze den allzu hohen Werth, den sinnliche Ergötzungen in unsern Augen haben, herab; bald sie verstärke die Liebe zu unsern Mitmenschen, bald sie erfülle unsere Herzen mit den Gefühlen der Freudigkeit, der Liebe, der Ehrfurcht und des Vertrauens zu Gott \ua\  Ähnliches. Haltet Ihr Wirkungen von einer solchen Art für sittlich zuträglich, dann ist es leerer Vorwand, wenn Ihr behauptet, mit dem Vf.\ Euch nicht vereinigen zu können im Schlußsatz seines Werkes, weil seine Ansicht vom obersten Sittengesetze schon eine andere als die Eurige ist! 
Eine ganz ähnliche Bewandtniß hat es mit der eigenthümlichen Ansicht, welche B.\ von der Freiheit des Menschen schon \RWpar{I}{15} \ua\  a. O. des Lehrbuches aufstellt, vermöge deren er sich für seine eigene Person zwar zu dem strengsten Indeterminismus bekennet, zugleich aber anmerkt, daß der Determinismus zu denselben praktischen Folgen führe. Hier liegt es also schon in des Vf.\ eigenem Geständnisse, daß man der einen oder der andern Meinung beitreten könne, ohne mit ihm zerfallen zu müssen. In dem achten Abschnitte der Athanasia wird auch, obwohl in gedrängter Kürze, doch unserm Dafürhalten nach, mit vieler Klarheit vor Augen gelegt, wie dieses komme. \par

\RWsec{II}{1}{83}
Und so können wir denn hoffentlich immer noch Hand in Hand mit unseren Lesern zur Betrachtung des zweiten oder besonderen Satzes, der in B.'s Schlusse für die Wahrheit des Katholicismus den Untersatz bildet, übergehen. \seitenw{153} Dieser bestehet aus den zwei Theilen. Der christliche, und näher noch der katholische Lehrbegriff besitzt die höchste sittliche Zuträglichkeit für uns, und -- er hat seine Entstehung, Erhaltung und Ausbreitung einem Zusammenflusse außerordentlicher Ereignisse zu danken. Aus Rücksichten auf seine Zuhörer fand sich B.\ veranlaßt, den Beweis des zweiten Theiles jenem des ersten vorauszuschicken. Somit beschäftiget er sich in dem zweiten Haupttheile des Werkes (der dessen zweiten Band bildet) mit dem Beweise, daß bei Entstehung, Erhaltung und Ausbreitung des Christenthums außerordentliche Begebenheiten mitgewirkt haben; in dem dritten Hauptstücke aber (der seines Umfanges wegen in zwei Bände zerfällt) prüft der Vf.\ die sittliche Zuträglichkeit des Christenthums nach der katholischen Auffassung desselben. Der Leser brauchte sich nun begreiflich eben nicht an diese Ordnung des Buches zu halten, er könnte den dritten und vierten Band voraus nehmen, und mit dem zweiten schließen. Will er aber dem Vortrage B.'s genau folgen, so erinnern wir ihn, nicht zu übersehen, daß dieser ihm keineswegs zumuthe, die außerordentlichen Begebenheiten, von welchen der zweite Theil erzählt, als eigentliche Zeichen, die zur Bestätigung des Christenthumes dienen, anzuerkennen, ehe er sich noch von der sittlichen Zuträglichkeit eines der christlichen Lehrbegriffe überzeugt hat. \par
Um es jedoch schon vor der Hand einigermaßen zu rechtfertigen, warum der Vf.\ seine Aufmerksamkeit jetzt gerade auf das Christenthum richte, beginnt er in dem \RWHSfmt{ersten Hauptstücke des zweiten Bandes}\RWHS{Erstes Hauptstück}{4--12} (\RWpar{II}{4--12}) mit einem von ihm sogenannten \danf{\RWHSfmt{äußeren oder Autoritäts-Beweise für die sittliche Zuträglichkeit der christlichen und näher noch des katholischen Lehrbegriffes}.} Diesen Beweis können die Leser, wenn es ihnen beliebt, ganz übergehen, da der Vf.\ selbst nicht will, daß sie mit diesem sich zufrieden stellend, die Untersuchung der christlichen Lehrbegriffe nach ihrer inneren Beschaffenheit und ihrem Verhältnisse zu sich selbst unterlassen; wie es denn auch sogar für den Fall, wenn wir uns von der Wahrheit eines dieser Lehrbegriffe überzeugt halten könnten, ehe wir noch die Lehren desselben \seitenw{154} im Einzelnen kennen gelernt, mindestens von dem Augenblicke an, da dieß geschehen ist, unsere Pflicht würde, dieß Einzelne kennen zu lernen und uns zu befragen, wozu es sich anwenden lasse. \par
Inzwischen ist die Art der Untersuchung, die der Vf.\ hier anstellt, im Allgemeinen, nämlich die Untersuchung, wie viel sich für oder wider die Wahrheit eines Satzes, ohne ein Eingehen in seine inneren Gründe, bloß aus Betrachtung derer, die sich für oder wider ihn erkläret haben, entscheiden lasse, -von solcher Wichtigkeit und so vielfältiger Anwendung, daß es B.\ wohl nicht mit Unrecht der Mühe werth erachtet, schon in der \danf{Wissenschaftslehre} §. 390.\WLi{III}{390} eine eigene Anweisung, wie solche Untersuchungen eingerichtet werden sollen, zu ertheilen. Bei der Anwendung auf den vorliegenden Fall kann jedoch abermal nur dem Gedanken selbst, nicht seiner unvollständigen Ausführung (auf nicht mehr als 27 Seiten) einiger Werth zuerkannt werden. Wir wollen aber, statt den bloßen Gang, den der Vf.\ bei dieser Untersuchung einschlägt, des Näheren zu bezeichnen, lieber nur einen einzigen Gedanken, der uns besonders richtig dünkt, ausheben. Bei einer Gesellschaftsreligion, wie die katholische, kann es sich sehr leicht ergeben, daß wir neben den Lehren, die wir mit einer solchen Allgemeinheit verbreitet finden, daß sie als wirkliche Lehren dieser Gesellschaft selbst betrachtet werden müssen, noch einer Menge von Vorurtheilen begegnen, die -- ob ihnen gleich von Manchen widersprochen wird -- von Andern nur allzu eifrig vertheidiget werden, und wie auf die Sittlichkeit so auch auf die Wohlfahrt der Gesellschaft eine verderbliche Einwirkung ausüben. Hiedurch und durch den nachtheiligen Einfluß, welchen auch andere, gar nicht zur Religion gehörige Umstände haben, \zB\ die in einem Lande herrschenden bürgerlichen sowohl als kirchlichen Gesetze u. v. A. kann es geschehen, daß sich ein Volk, an dessen religiösem Lehrbegriffe nicht das Geringste auszustellen ist, dennoch in einem sittlich und physisch schlimmern Zustande befinden kann als ein anderes, in dessen Religion sich einige offenbar falsche Sätze nachweisen lassen. \seitenw{155} \par
Auch das \RWHSfmt{zweite Hauptstück}:\RWHS{Zweites Hauptstück}{13--30} \danf{\RWHSfmt{über die Natur der historischen Erkenntnisse, besonders in Hinsicht auf Wunder}} (\RWpar{II}{13--30}), könnten wir überschlagen; denn das Einzige, was von einem Einflusse auf den hier zu führenden Beweis seyn könnte, ist der Lehrsatz des \RWpar{II}{27}, daß auch Wunder (nämlich außerordentliche Begebenheiten) historisch beglaubigt werden können, und der des \RWpar{II}{30}, daß historische Urtheile von einer solchen Art, wie sie der Glaube an eine Offenbarung fordert, eben den Grad der Gewißheit, wie Urtheile a priori ersteigen können. Aber beide Wahrheiten dürften wohl einem Jeden, der die hier aufgestellte Theorie von den Kennzeichen einer Offenbarung einmal begriffen und angenommen hat, von selbst einleuchten, sollte er sich auch ihre Gründe nicht deutlich auseinander zu setzen wissen. Unseren Lesern sind diese letzteren großentheils schon bekannt. Alle historischen Urtheile gründen zuletzt sich auf gewisse unmittelbare Wahrnehmungen, zu deren Erklärung wir ein in der Vergangenheit stattgefundenes Ereigniß aus mehreren möglichen andern, nach einem bloßen Schlusse der Wahrscheinlichkeit annehmen. Da kann es sich nun, so groß auch die innere Unwahrscheinlichkeit eines von den Geschichtschreibern uns erzählten Ereignisses seyn mag, doch sehr wohl ergeben, daß dasselbe einen sehr hohen Grad der Verlässigkeit ersteige, sobald nur (was gar nicht unmöglich ist) die Umstände von einer solchen Art sind, daß jede andre Annahme, durch welche wir uns unsre vorhandenen Wahrnehmungen erklären wollten (\zB\ daß die Erzähler sich alle geirrt oder gelogen \usw ) eine noch weit größere Unwahrscheinlichkeit haben. In der That aber ist es, um nur gewiß zu werden, daß eine Lehre ihre Verbreitung ungewöhnlichen Ereignissen verdanke, nie nöthig, über den eigentlichen Hergang dieser Ereignisse zu entscheiden, und somit auch nie nöthig, daß unter den mehreren Annahmen, welche wir zur Erklärung unsrer unmittelbaren Wahrnehmungen machen, Eine sich nachweisen lasse, die alle übrigen als die vergleichungsweise wahrscheinlichste übertrifft. Und eben darauf und auf die Betrachtung, daß Gott nicht zulassen könne, daß wir zu unserem Nachtheile \seitenw{156} irren, wenn wir nach unserer besten Einsicht annehmen, er habe uns etwas als wahr bezeuget, -- auf diese beiden Umstände gründet sich unsre Behauptung, daß wir von dem Besitze einer göttlichen Offenbarung sicherer überzeugt seyn können, als es selbst Mathematiker von ihren Lehrsätzen sind. \par
Was in diesem Hauptstücke sonst noch vorkommt, die näheren Bestimmungen der Erfordernisse zu einem glaubwürdigen Zeugnisse (z.B.\ daß hiezu nicht eben Wahrheitsliebe gehöre), was über Zeugenmehrheit und Zeugenwiderspruch, über die Art, wie man die Echtheit und Unverfälschtheit eines Buchs zu untersuchen habe, gesagt wird, -- kann vielleicht die Beachtung derjenigen verdienen, die über dergleichen Gegenstände Theorien aufstellen wollen, für den uns vorliegenden Zweck ist es ganz gleichgültig. \par
Eben dieß Urtheil fällen wir unbedenklich auch über das ganze \RWHSfmt{dritte Hauptstück},\RWHS{Drittes Hauptstück}{31--54} das \danf{\RWHSfmt{von der Echtheit, Unverfälschtheit und historischen Glaubwürdigkeit der Bücher des neuen Bundes}} (\RWpar{II}{31--54}) handelt. Denn dieß sind Untersuchungen, von welchen nach des Vf.\ und unserer hierin ganz mit ihm einstimmigen Ansicht der Beweis für die Wahrheit des Christenthums überhaupt und des Katholicismus insbesondere nicht abhängig gemacht werden soll und kann. Hiezu kömmt noch, daß wir diesem Hauptstücke -- so beifällig es auch von einigen Recensenten beurtheilet wurde -- nicht einmal den Vorzug nachrühmen dürfen, den wir den beiden ersteren zuzugestehen geneigt wären, nämlich daß sie bei aller Mangelhaftigkeit der Ausführung doch einige neue beachtenswerthe Gedanken darbieten. So genüge es denn, hier nur im Kurzen den Plan anzugeben, nach welchem B., nachdem er zuvor die für die Echtheit und Unverfälschtheit jener Bücher sprechenden Gründe angeführt hat, ihre historische Glaubwürdigkeit darzuthun sucht. 1) Die Schriftsteller neuen Bundes (behauptet er zuerst) hatten nicht nur volle Gelegenheit, sondern auch allen Antrieb, sich selbst Kenntniß von den Begebenheiten, die wir von ihnen erfahren wollen, zu verschaffen; sie waren überdieß weder so leichtgläubig, noch solche Schwärmer, als Manche aus \seitenw{157} ihnen machen wollen; es ist auch nicht eben unbegreiflich, auf welche Weise sie die Reden Jesu, welche sie uns erzählen, unverfälscht zu überliefern vermochte. 2) Sie erzählen ferner verständlich und in der ernstesten Absicht. 3) Sie berichten in der That Wahrheit. Dieses Letzte erhellet \par
a) aus jenem Beifalle, den ihre Schriften vorzugsweise vor andern Geschichtsbüchern über denselben Gegenstand erhielten; \par
b) aus der hohen innern Vortrefflichkeit, die Beides, die religiöse Lehre des neuen Bundes sowohl als auch die in diesen Büchern erzählte Geschichte Jesu besitzt; c) aus der Vergleichung der Erzählungen, die wir in den verschiedenen Büchern dieser Sammlung über ein und dasselbe Ereigniß antreffen; d) aus dem Mangel eines hinlänglichen Beweggrundes zur Lüge bei den Verfassern dieser Bücher; e) aus den Beweisen der Aufrichtigkeit und Wahrheitsliebe, die sie uns geben; endlich f) aus der Vergleichung ihrer Erzählungen mit den Erzählungen anderer Geschichtschreiber. (S.\,105--144). \par
So ohngefähr, meinte B., müsse man sich über diesen Gegenstand erklären, wenn man im Allgemeinen ihn besprechen wolle; aber eben damit ist der Wissenschaft wenig gedient. Jedes besondre Ereigniß im Leben Jesu, jede seiner einzelnen Reden und Thaten hat ihren eigenen bald größeren, bald geringeren Grad der Glaubwürdigkeit, indem die inneren sowohl als auch die äußeren Gründe der Wahrscheinlichkeit, die für oder wider ein jegliches dieser Ereignisse sprechen, verschieden sind. Diese Grade der Glaubwürdigkeit nun auf recht unbefangene Weise zu untersuchen, wäre eine Aufgabe, durch deren bisher noch nicht vollendete Lösung die Wissenschaft -- nicht zwar der Religionswissenschaft -- wohl aber einer der wichtigsten Zweige der Geschichtskunde, -die Lehre von den Thaten und Schicksalen merkwürdiger Personen einen höchst dankenswerthen Zuwachs zur Vollkommenheit erhielte. Solche Äußerungen erinnern wir uns von B.\ vernommen zu haben vor mehr als dreißig Jahren! \par
\gliederungslinie \seitenw{158}\par
Das einzige \RWHSfmt{Hauptstück}\RWHS{Viertes Hauptstück}{55--75} in diesem Bande also, dessen wir wesentlich bedürfen, ist das \RWHSfmt{vierte}, in welchem die \danf{\RWHSfmt{Beweise für das Vorhandenseyn des äußeren Merkmals einer Offenbarung am Christenthume}} (\RWpar{II}{55--75}) aufgestellt werden. Beachten wir aber auch hier eine Sonderung, die der Vf.\ nicht ohne Grund gemacht hat. Er trägt nämlich unter der Überschrift: \danf{Erste Abtheilung. Allgemeine Beweise,} erst einige solche Beweise für das Vorhandenseyn des äußeren Merkmales einer Offenbarung am Christenthume vor, bei welchen die historische Glaubwürdigkeit der Bücher des neuen Bundes gar nicht vorausgesetzt zu werden braucht, von denen er überhaupt glaubt, daß sie ganz unumstößlich und für Jedermann, der sie mit unbefangenem Gemüthe überdenkt, befriedigend seyn müßten. Nachdem dieß geschehen, läßt er sich erst herbei, in der zweiten Abtheilung unter der Überschrift: \danf{Einzelne Wunder,} Ereignisse aufzuführen, deren Erzählung den Büchern des neuen Bundes entnommen ist. \par
In der ersten Abtheilung (\RWpar{II}{56--61}), liefert er fünf Beweise: 1) aus dem Daseyn der Bibel, 2) aus der Predigt der Apostel, 3) aus dem Glauben der ersten Christen, 4) aus dem Betragen der Feinde des Christenthums, und endlich 5) aus dem Urtheile der Gelehrten unserer Zeit. Der letzte ist wohl der unwiderstehlichste; daher wir uns begnügen, nur diesen den Lesern vorzulegen. \danf{In unsrer neuesten Zeit,} sagt der Vf., \danf{ward die Geschichte der Entstehung und Ausbreitung des Christenthums nicht einmal, sondern unzählige Male der strengsten Prüfung unterworfen; häufig mit der entschiedensten Neigung, Alles, was einem Wunder ähnlich sähe, und diese Religion als eine göttliche beurkunden würde, im voraus zu bezweifeln und \DruckVariante{abzuläugnen}{abzulaügnen}: und das Ergebniß dieser Prüfungen war, wie folgt: \par
1) Der größere Theil dieser Gelehrten, und zwar gerade die Classe derjenigen, die ihre Prüfung mit der gewissenhaftesten Unbefangenheit unternommen zu haben scheinen, die auch durch andere Untersuchungen sich als die bescheidensten \seitenw{159} Forscher und als die gründlichsten Beurtheiler des Alterthums bewiesen hatten, entschieden für das Christenthum; entschieden, daß bei der Entstehung und Ausbreitung desselben allerdings viele und unläugbare Wunder (nämlich gerade diejenigen, die uns die Bücher des neuen Bundes erzählen) statt gefunden hätten. \par
2) Nur ein sehr kleiner Theil war es, der wider das Christenthum entschied, \dh\  der läugnete, daß bei dessen Entstehung oder Ausbreitung Wunder von der Art statt gefunden hätten, wie sie die Bücher des neuen Bundes erzählen. Bei einer näheren Betrachtung zeigt sich jedoch, daß auch schon dasjenige, was diese Gelehrten zugaben, vollkommen hinreichend sey, die Wahrheit des Christenthums als einer göttlichen Offenbarung zu beurkunden. Diese Gelehrten behaupten nämlich im Grunde nichts Anderes, als daß es mit dieser oder jener einzelnen Begebenheit, die uns die Bücher des neuen Bundes erzählen, \zB\ mit der Auferstehung Jesu, nicht seine Richtigkeit habe; sie behaupten ferner, daß bei Entstehung und Ausbreitung des Christenthums nirgends Wunder in der Bedeutung, die man bisher in den Schulen als die allein gültige ansieht, nämlich nicht übernatürliche oder unmittelbare Wirkungen Gottes statt gefunden hätten. Aber sie sehen sich Alle genöthiget, zuzugestehen, daß sich zu Gunsten des Christenthums sehr viele ungewöhnliche Ereignisse, die auch ganz anders hätten erfolgen können, ergeben hätten.} -- \danf{Und so ist denn,} (schließt der Vf.) \danf{die Sache des Christenthums durch einen Zeitraum von achtzehn Jahrhunderten von Feinden und Freunden geprüft worden, und nicht nur jene, die für, sondern auch jene, die gegen dasselbe entschieden, behaupteten (die Letzteren ohne es zu wissen und zu wollen) Dinge, aus welchen folgt, daß diese Religion das äußere Kennzeichen einer göttlichen Offenbarung habe. Wer sich demnach nicht weiser dünken will als alle diese Personen, der wird auch zugeben müssen, daß sie in einem Stücke, darin sie alle übereinstimmen, sich nicht geirrt haben. Wer aber auch behaupten wollte, sie hätten sich dennoch geirrt, der müßte eben in \seitenw{160} diesem Irrthume ein Ereigniß annehmen, welches so ungewohnlich ist, und so sehr zum Vortheil des Christenthums dient, daß man es abermal als ein die Göttlichkeit dieser Religion erweisendes Wunder ansehen müßte.} -- Wir wollten wissen, wer gegen diesen Beweis etwas Haltbares vorzubringen vermöge? Oder wollte man vielleicht sich an demselben nur darum nicht genügen lassen, weil er so kurz und einfach ist? \par
In der zweiten Abtheilung (\RWpar{II}{62--73}) werden folgende Wunder betrachtet: 1) Eine das Christenthum (nämlich die schnelle Verbreitung desselben) betreffende Weissagung; 2) Weissagungen Jesu, das Volk der Juden betreffend; 3) Merkwürdiges Ereigniß bei der versuchten Wiedererbauung des jüdischen Tempels; 4) Einige Wunder, welche uns die Apostelgeschichte erzählt (die Begebenheit am Pfingstfest, die Bekehrung Pauli \ua\ ); 5) Wunder Jesu (a. allgemeine Berichte von wunderthätigen Krankenheilungen Jesu, b. einige einzelne Wunder); 6) Die Auferstehung Jesu; 7) Messianische Weissagungen. -- Deutlich genug leuchtet aus Wahl, Anordnung und Behandlung dieser Stoffe hervor, wie der vf. der Ansicht lebe, daß ein aufgeklärt Denkender immer geneigter seyn werde, das Geschehenseyn eines Ereignisses zuzugestehen und somit auch in demselben ein göttliches Zeichen anzuerkennen, wenn er begreifen kann, durch welcher natürlichen Kräfte Zusammenwirkung es etwa dürfte herbeigeführt worden seyn, als wenn das Gegentheil der Fall ist. Von dieser Art sind offenbar die Wunder 1, 2, 3 und 7; denn daß es B.\ nicht in den Sinn komme, in Nr. 7 zu einer messianischen Weissagung zu verlangen, daß der alttestamentliche Schriftsteller vorhergewußt habe, hier schreibe er etwas, das einst auf diese und diese Art in Erfüllung gehen werde: das brauchen wir nicht erst zu erinnern. Das Factum der Auferstehung Jesu aber will er aus einem doppelten, bisher noch nicht genau unterschiedenen Gesichtspunkte betrachtet wissen; dem einen, wenn wir es lediglich als ein zum Beweise des Christenthums dienendes Zeichen gebrauchen; dem andern, wenn wir darin einen factischen Beweis für die Fort\seitenw{161}dauer unsrer Persönlichkeit auch nach dem Tode antreffen wollen. In dem ersten Falle genügt es zu zeigen, daß, wie man auch immer sich den eigentlichen Hergang der Sache vorstellen möchte, immer doch irgend ein höchst ungewöhnliches Zusammentreffen vieler sehr zufälliger Umstände vorausgesetzt werden muß. -- Dieses zu zeigen, ist nun eine so leichte Sache, daß B.\ das hier in Rede stehende Ereigniß wohl nicht mit Unrecht den größten und entscheidendsten Wundern des Christenthums beizählt. \par
Eine ganz andere Behandlungsweise des Gegenstandes ist in dem zweiten Falle nöthig, wo untersucht werden muß, ob und in welchem Grade es sich historisch gewiß machen lasse, daß Jesus a) am Kreuze wirklich gestorben, b) am dritten Tage sich wieder lebendig dargestellt habe, und dieß zwar c) nicht mehr in einem gewöhnlichen sterblichen Menschenleibe, sondern in einem verklärten, für eine höhere Stufe des Daseyns berechneten Leibe. Wir glauben, ein bescheidener Leser werde, sobald er sich auch nur den Titel des Buches angesehen hat, nicht fordern, daß der Vf.\ seine jungen Zuhörer mit einem jeden Einwurfe, der unsere zuversichtliche Voraussetzung eines solchen Factums beeinträchtigen kann, vertraut gemacht haben solle. Wie dem auch sey, so wird es doch uns bei dieser Gelegenheit erlaubt seyn, unseren Lesern eine ganz hieher gehörige Betrachtung B.'s mitzutheilen, die wir in einem von ihm vor mehreren Jahren geschriebenen Briefe an eine sehr verständige Frau (eine seiner ehemaligen Schülerinnen, wenn wir nicht irren C. L. geb. R.\editorischeanmerkung{%
	Caroline Lieblein, geb. \v{R}eho\v{r} (1792--1871), zunächst Schülerin, später Freundin und im Alter Betreuerin Bolzanos.}) gelesen. Es ist doch, äußerte er daselbst, unwidersprechlich, daß nicht nur die Apostel und alle ersten Christen, sondern daß auch noch in der Folge durch alle Jahrhunderte der christlichen Zeitrechnung hindurch bis auf den heutigen Tag Millionen Anhänger Jesu geglaubt, durch ihn sey uns Menschen erfüllet worden, was wir von jeher als den stärksten und sinnenfälligsten Beweis unserer Unsterblichkeit uns gewünschet hatten, daß nämlich irgend einer aus unsern Brüdern nach seinem Hinscheiden wieder zurückkehre, um uns das Daseyn eines Lebens jenseits der Gräber zu bezeugen. So Viele dieß glaubten, so Viele stellten sich vor, daß es ein nur durch Gottes besondere Ver\seitenw{162}anstaltung herbeigeführtes Ereigniß gewesen, welches den Zweck gehabt, unsern Glauben an Unsterblichkeit zu befestigen, \dh\  sie sahen darin eine wahre göttliche Offenbarung. Ist es nun wohl gedenkbar, daß unter einer so zahllosen Menge von Menschen nicht auch sehr Viele gewesen, die bei der Annahme dieses Glaubens mit aller ihnen nur möglichen Vorsicht, nach ihrer besten Einsicht, auf das Gewissenhafteste verfuhren? Wenn aber dieß geschah, können wir einer allwaltenden Fürsehung Gottes dann wohl zumuthen, daß jene Menschen getäuscht worden wären, getäuscht in irgend einem Punkte, der nur von einiger Wichtigkeit ist, \dh\  in Betreff dessen sie sich beklagen könnten, wenn es sich damit nicht wirklich so verhielte, wie sie vertrauend auf Gottes Zeugniß angenommen hatten? Könnte Gott zulassen, daß wir in Folge dessen, was sich in jenen Tagen zutrug, an eine Fortdauer unserer Persönlichkeit und an ein ewiges Leben glauben, wenn es ein solches nicht wirklich gäbe? -- Gesetzt also auch in unsern jetzigen Tagen, bei einer so weiten Entfernung in der Zeit und bei den unvollständigen Berichten, die uns erübrigen, könnten wir nicht über einen jeden uns wichtig scheinenden Umstand bei diesem Ereignisse zu einer ganz zweifellosen Entscheidung gelangen; gesetzt nicht einmal darüber könnten wir uns vollkommen sicher stellen, ob der Erstandene seinen Freunden in einem verklärten oder in dem noch immer der Sterblichkeit unterworfenen irdischen Menschenleibe erschienen sey: wird es uns nicht erlaubt seyn, zu behaupten, daß selbst in diesem letzteren Falle nichts abzuändern sey an unserm Glauben, daß es ein anderes Leben, und einen Zustand höherer Vollkommenheit in demselben gebe? Werden wir nicht ohngefähr so schließen dürfen: Falls jener Leib, in dem sich der Auferstandene zeigte, noch nicht die ganze Beschaffenheit hatte, welche die Leiber der Seligen im andern Leben haben: so geschah dieß ohne Zweifel, weil nur ein Leib wie der seinige Wahrnehmbarkeit für unsere Sinne gehabt, oder, weil schon ein solcher hingereicht hatte, die hier allein beabsichtigte Überzeugung hervorzubringen? Was es auch seyn mochte, das sich in jenen Tagen zutrug, es war von der. Art, daß es hingereicht hatte, mehrere Hunderte unserer Brüder durch \seitenw{163} ihre eigenen Sinne zu überzeugen, daß \danf{der getödtet war, als Sieger über Tod und Hölle wieder hervorgegangen sey aus dem Grabe, und sich gen Himmel aufgeschwungen habe:} können wir da noch zweifelhaft bleiben, zu welchem Zwecke dieß Alles geschehen sey, und welche Wahrheit es sey, die Gott den Menschen hiemit habe verbürgen wollen? \par
Dieß ohngefähr der Gedankengang in jenem Schreiben! Kehren wir nun nur noch ein einziges Mal zu dem zweiten Bande zurück, um zu erfahren, wie der Vf.\ am Schlusse (\RWpar{II}{74}) die Frage über den Zusammenhang jener außerordentlichen Ereignisse, die bisher aufgezählt wurden, mit dem katholischen Lehrbegriffe erledigt. Sehr kurz und einleuchtend ist seine Erklärung (Nr. 1), daß nicht nur bei dem katholischen, sondern bei allen Lehrbegriffen, die ihre Entstehung und Ausbreitung der richtigen oder unrichtigen Meinung verdanken, daß sie nur eine weitere Entwicklung der Lehre Jesu wären, ein Zusammenhang mit den erwähnten Ereignissen statt finde, der ganz so innig ist, als man ihn zwischen einer Lehre und den Ereignissen, die sie bestätigen sollen, nur immer zu verlangen berechtiget ist. Was er dann (Nr. 2) weiter noch sagt, daß der katholische Lehrbegriff mit jenen Ereignissen in einer noch genaueren Verbindung stehe, als andere christliche Lehrbegriffe, das können wir gerne dahin gestellt lassen. Man wird es bestreiten, wer aber kann das Erstere in Abrede stellen? \par
\gliederungslinie\par


\RWsec{IIIa}{1}{166}
Somit wird nun Alles nur darauf ankommen, von welchem christlichen Lehrbegriffe es sich nach einer unbefangenen Prüfung zeige, daß er nebst dem allen gemeinschaftlich zukommenden äußeren, auch noch das sogenannte innere Merkmal einer göttlichen Offenbarung, nämlich die größte sittliche Zuträglichkeit besitze, dergestalt, daß wir bei jedem einzelnen Punkte eingestehen müssen, wir fühlen uns zu der Erfüllung unserer Pflichten stärker ermuntert, wenn wir so glauben, wie hier, als wenn wir glauben, was anderwärts \seitenw{164}\par
gelehrt wird. Wohl zu bemerken ist also, daß es sich hier keineswegs darum handelt, zu untersuchen, ob dieser oder jener Lehrbegriff für alle Menschen sittliche Zuträglichkeit habe; noch weniger darum, ob eine gewisse Lehre nicht etwa durch Mißverstand oder Mißbrauch verderblich werden könne, vielleicht auch schon geworden: sondern es handelt sich jetzt lediglich darum, zu beurtheilen, ob nur wir selbst hoffen können, durch eine solche Lehre, und zwar bei jener eigenthümlichen Weise, wie gerade wir sie uns zu deuten und sie anzuwenden verständen, an sittlicher Vollkommenheit zu gewinnen; ingleichen ob es keine andere Ansicht über denselben Gegenstand gebe, deren Annahme uns einen größeren sittlichen Nutzen verspräche? \par
Da es jedoch einleuchtend ist, daß wir eine Lehre nie gläubig annehmen könnten, wenn wir bemerkten, daß sie mit irgend einer bekannten Wahrheit in einem Widerspruche stehe: so folgt, daß wir eben so sehr, wie wir verpflichtet sind, die sittliche Zuträglichkeit einer jeden Lehre zu untersuchen, auch ihre Vernunftmäßigkeit zu prüfen berechtiget seyen. Erinnern wir uns indessen an das, was gleich im Anfange dieses Aufsatzes über die bildliche Natur so mancher religiöser Lehren gesagt worden ist: so wird uns klar, daß es nur Übereilung wäre, wenn wir sofort eine jede Lehre, die in ihrem wörtlichen Sinne auf einen Widerspruch hinausführt, als widervernünftig und somit auch als falsch und nicht von Gott geoffenbart verwerfen wollten; weil sehr wohl möglich wäre, daß sie bloß bildlich zu nehmen sey, und in diesem bildlichen Sinne sich vollkommen rechtfertigen lasse. Zu diesem letzteren wird aber nichts Mehreres erfordert, als daß die Vorstellung, die Sache verhalte sich, wie diese Lehre angibt, zweckmäßige Gefühle und Willensentschließungen in uns erwecke, \dh\  sittliche Zuträglichkeit für uns besitze. Hieraus ergibt sich denn, daß wir bei Untersuchung der Vernunftmäßigkeit einer Lehre ein ungünstiges Ergebniß, und somit ein Verwerfungsurtheil derselben nie für begründet ansehen dürfen, wenn wir nicht erst noch erwägen, ob sie nicht bildlich sich auffassen lasse und welche Brauchbarkeit sie dann beurkunde. Die Empfehlung dieser Vorsicht ist um so nothwendiger, da \seitenw{165} wir es nie oder doch nur in den seltensten Fällen der Offenbarung selbst zumuthen dürfen, daß sie (uns) ihre bildlichen Lehren als solche ausdrücklich bezeichne; zumal wenn sie uns in der Form einer Gesellschaftsreligion gegeben ist. Denn was diejenigen Mitglieder der Gesellschaft betrifft, die so viel Einsicht haben, um in der wörtlichen Auslegung einer Lehre einen Widerspruch zu gewahren, so haben sie eben deßhalb auch Einsicht genug, um es sich selbst zu sagen, daß ihnen hier ein bloßes Bild vorliege: für Andere aber, die noch zu ungeübt im Denken sind, und noch zu wenig Vorkenntnisse besitzen, um etwas Anstößiges in dieser Lehre auch bei einer ganz wörtlichen Auffassung zu finden, dürfte es nicht beirrend für sie seyn, wenn ihnen zugerufen würde: \danf{Es ist ein bloßes Bild, was ihr hier vor euch sehet; nur um euere Gefühle und Entschließungen zu beleben, sollt ihr euch vorstellen, die Sache wäre so; sie ist es aber nicht in der Wirklichkeit?} \par
Mit solchen Gesinnungen nun gehe ein Jeder an die Prüfung der mancherlei christlichen Lehrbegriffe; und wir können ihm zwar nicht voraussagen, ob das Ergebniß seiner Prüfung der Katholicismus oder sonst eine andere christliche Religion seyn werde; aber mit eben dem Grade der Zuversicht, mit dem wir ihm zutrauen, daß er gewissenhaft verfahre, vertrauen wir auch, daß Gott ihn nicht zu seinem sittlichen Nachtheile werde einen Fehlgriff thun lassen. Anlangend den Vf., so wissen wir schon, es sey der katholisch christliche Lehrbegriff, den er uns als den sittlich vollkommensten darstellt. Hierin nun mag man ihm beistimmen oder nicht: so ist man, glauben wir, mindestens nicht berechtiget, zu denken, er habe das ohne die innerste durch eine vorausgegangene Prüfung derselben Art, wie er von Andern sie fordert, gewordene Überzeugung gethan; vielmehr ersieht man aus dem ganzen Buche deutlich genug, warum es nur eben der Katholicismus sey, der ihn befriedige, und begreifen, daß er, wenn nicht mit diesem, um so weniger mit irgend einem andern christlichen Lehrbegriffe sich hätte befreunden können. \par
Aber die Frage ist, ob, was der Vf.\ uns hier als Katholicismus vorlegt, auch echter Katholicismus \seitenw{166} sey? Das hat man wirklich schon bestritten, und mehr als ein katholischer Recensent hat B.\ des Neologismus und einer Verflachung der katholischen Dogmen beschuldigt. Wir müssen also genau untersuchen, wie es sich hiemit verhalte! \par
Bedenken erregt schon die eigenthümliche Erklärung des Vfs. von dem Begriffe einer Gesellschafts-Religion, und die Behauptung, daß der Katholicismus eine solche sey. Denn schwerlich werden unsere Theologen geneigt seyn, zuzugeben, daß kein Satz eher als ein katholisches Dogma angesehen werden könne, als bis auch alle Laien, für welche er Verständlichkeit und religiöse Wichtigkeit hat, ihn angenommen haben. Da jedoch der Vf.\ \RWpar[Bd. III. §. 2.]{III}{2} äußert, daß man schon dann, wenn eine Lehre nur sich in allen öffentlich gebrauchten Lehrbüchern findet, voraussetzen könne, daß sie auch von der gesammten Geistlichkeit und dem übrigen Theile der Gesellschaft angenommen werde: so sieht man, daß er eben nicht gesonnen ist, sich seiner eigenthümlichen Erklärung zu bedienen, um etwa Lehren, die ihm unwillkommen sind, aus dem Verzeichnisse der katholischen Dogmen zu streichen. Doch wir müssen die Einrichtung, welche er seinem \danf{dritten Haupttheile} nämlich \danf{der systematischen Darstellung der Lehre des Katholicismus nach ihrer inneren Vortrefflichkeit} gegeben, näher in's Auge fassen. \par
Seinem eigenen Geständnisse nach hätte B.\ hier bei einem jeden Satze, den er als eine katholische Lehre aufführt, nachweisen sollen, daß sich derselbe in den gebräuchlichsten Lehrbüchern der katholischen Kirche deutlich ausgesprochen oder mindestens in dem daselbst Gesagten stillschweigend vorausgesetzt finde. Das hat er aber keineswegs gethan, sondern statt dessen es vorgezogen, bei einer jeden Lehre unter der sonderbaren Überschrift: \danf{Historischer Beweis dieser Lehre,} gewöhnlich nur gewisse Bibeltexte zu liefern, aus welchen die katholischen Theologen die in Rede stehende Lehre gefolgert wissen wollen. Zu diesem Verfahren -- wir können es uns leicht denken -- mochten ihn mancherlei Gründe bestimmen. Für seine Schüler wäre die Anführung so vieler oft sehr langer Stellen aus den verschiedensten Lehrbüchern, ja auch die \seitenw{167} bloße Verweisung auf diese Stellen ganz ohne Zweck und Nutzen gewesen. Durch eine nicht allzugehäufte Anführung jener Schrifttexte dagegen wurden sie nicht nur mit einer Menge herrlicher Bibelstellen vertraut, deren sie sich in ihrem ganzen künftigen Leben zu ihrer eigenen sowohl als Anderer Erbauung bedienen konnten; sondern sie lernten so auch einigermaßen verstehen, wie eine jede Lehre aufkam, und welche Mittel Gott angewandt habe, um dieselbe mit der Zeit allgemein herrschend zu machen. Wie lehrreich können oft solche Betrachtungen werden, und welche Beweise bieten sie dar für das Vorhandenseyn einer Fürsehung, die für das sittliche Fortschreiten der ganzen Menschheit, und darum vornehmlich auch für die Entwicklung der religiösen Begriffe unter uns sorget! \par
Allein so gut dieß Alles für seine Schüler seyn mochte, unsere Leser, wollen sie sicher gehen, befinden sich nun in der Nothwendigkeit, was der Vf.\ unterließ, wenn sie es nicht etwa als Theologen schon längst gethan, selbst über sich zu nehmen, und in den Lehrbüchern, welche in der katholischen Kirche die angesehensten sind, Nachforschung anzustellen. Und was werden sie dann wohl finden? Erstlich, daß der Vf.\ von denjenigen Lehren der katholischen Kirche, welche als Dogmen im engeren Sinne betrachtet werden, kaum eine einzige ganz unberührt gelassen habe; dann aber (was sie vielleicht nicht erwartet hätten), daß er unter die Lehren des Katholicismus auch einige Lehrsätze aufgenommen habe, die man in den gewöhnlichen Lehrbüchern nicht antrifft. Es sind dieß, wie sich zeigt, Sätze, welche B.\ als ausgemachte Vernunftwahrheiten betrachtet, die, wenn sie auch in unsern Lehrbüchern bisher nicht vorgetragen wurden, doch seiner Meinung nach verdienten, daselbst ausdrücklich aufgestellt zu werden. Obgleich er nun, wenn er sich einen solchen Zusatz erlaubt, in dem historischen Beweise es gewöhnlich eingestehet; und selbst im Unterlassungsfalle voraussetzen dürfte, daß hieraus keine Irrung hervorgehen werde: so können wir doch nicht umhin, in diesem Verfahren, (zu welchem er offenbar nur durch den Wunsch verleitet wurde, solchen Wahrheiten mehr Aufmerksamkeit zu verschaffen, und es je eher je lieber dahin zu bringen, daß man sie allgemein aufnehme) -- etwas \seitenw{168} Unschickliches zu finden; und wir sind uns gewiß, daß er, der eben nicht an dem Fehler der Selbstgefälligkeit leidet, sich wünschte, dieß und so manches Andere in seinem Buche wäre anders! -- So viel ist jedenfalls gewiß: wenn sich ergeben sollte, daß ein Leser des Lehrbuchs mit einem dieser Artikel aus was immer für einem Grunde nicht einverstanden wäre: so gäbe dieß wohl einen Grund, B., nicht aber den katholischen Lehrbegriff anzuklagen. \par
Wird es nun wohl nach dem, was wir so eben über des Vfs. Zweck bei allen mit der Überschrift: \danf{historischer Beweis,} versehenen §§. gesagt, nothwendig seyn, zu bemerken, daß man hier niemals einen Beweis der Wahrheit einer besprochenen Lehre zu suchen habe? Daß es nur lächerlich wäre, wenn man hier aus dem Umstande, weil\editorischeanmerkung{öffnende Anführungszeichen weggelassen} mancher Text nicht sagt, was man aus ihm gefolgert hat, einen Vorwurf für den Vf.\ ableiten wollte? Wird es nöthig seyn, zu erinnern, daß es überhaupt unmöglich sey, aus dem Inhalte dieser §§. je einen gegründeten Einwurf gegen B.'s Beweis für die Wahrheit des Katholicismus zu entbehren, weil sie durchaus nichts, was als ein Vordersatz zu diesem Beweise gehört, enthalten? -- Für unsern Zweck also können wir diese §§. gänzlich bei Seite setzen und ungelesen lassen; besonders da wir, um uns zu versichern, ob uns B.\ die Lehre des Katholicismus getreulich darstelle, ohnehin ganz andere Mittel anwenden können und müssen. \par
Nicht minder entbehrlich für unser Vorhaben sind auch noch die §§., die unter der Überschrift: \danf{Wirklicher Nutzen dieser Lehre,} eine ohngefähre Schätzung desjenigen Nutzens versuchen, den eine in Rede stehende Lehre des Katholicismus seit der Zeit ihres Aufkommens bis auf den heutigen Tag gestiftet haben dürfte; wobei der Vf.\ zwar nie vergißt, auch des Schadens, den sie durch Mißverstand, Mißbrauch, erhobene Streitigkeiten \udgl\  veranlasset hat, zu erwähnen, aber zu zeigen sucht, daß jener diesen wohl auch bisher schon überwogen habe. Solche Erörterungen können allerdings die Hoffnung, daß ein Glaube, der sich an Tausenden heilsam erwiesen hat, auch uns ersprießlich seyn werde, erzeugen; sie können überdieß manchen Zweifel gegen die leitende Fürsehung Gottes entfernen; sie können \seitenw{169} endlich (was wohl das Wichtigste ist) uns eigene Regeln der Vorsicht, die bei dem Vortrage dieser Lehren zu beobachten sind, an die Hand geben: allein die Frage, ob der katholische Lehrbegriff bei der bestimmten Art, wie eben wir ihn auszulegen und anzuwenden wüßten, unserer eigenen Sittlichkeit einen Vorschub verspreche; diese Frage, auf die es eigentlich hier allein ankommt, können wir mit einer hinlänglichen Sicherheit entscheiden, ohne uns in jene historische Untersuchungen einzulassen. Möchten wir also gegen dasjenige, was B.\ in den besagten §§. ohnehin nur sehr mangelhaft andeutet, noch so viel einzuwenden haben: in der Meinung, daß wir hiedurch seinen Beweis für die Wahrheit des Katholicismus erschüttern, könnten wir solche Einwendungen wahrlich nicht vorbringen, wenn wir nicht an den Tag legen wollten, daß wir den innern Zusammenhang des Buches nicht verstehen. \par
So sind es denn nun die §§., welche von der Vernunftmäßigkeit und von dem sittlichen Nutzen einer jeden Lehre sprechen, in deren Prüfung der Leser nothwendig eingehen muß, doch so, daß er auch hier, wenn eine Lehre bei ihrer wörtlichen Auffassung \zB\ Ereignisse voraussetzt, die ihm nicht hinlänglich erwiesen scheinen, nicht sogleich berechtiget seyn wird, zu klagen, daß man ihm zumuthe, sich in religiösen Dingen mit einer bloßen Wahrscheinlichkeit zu begnügen; wenn anders die nähere Untersuchung zeigt, daß jene Lehre auch bei ihrer bildlichen Auffassung noch eine unbestreitbare sittliche Brauchbarkeit für ihn behalte. \par
So viel im Allgemeinen; ein Mehreres können wir nur sagen, wenn unsere Leser uns verstatten, sie zu den einzelnen in diesem Theile enthaltenen Lehren selbst zu begleiten. Wir versprechen hiebei uns immer kurz zu fassen und nur das Wichtigste zu berühren. \par
Die erste Lehre, die ihnen hier begegnet, ist die \danf{\RWHSfmt{Lehre des Katholicismus von den Erkenntnißquellen seiner Lehren},}\RWHS{Erstes Hauptstück}{3--20} mit deren Betrachtung sich der Vf.\ in dem ganzen \RWHSfmt{ersten Hauptstücke} (\RWpar{IIIa}{3--20}) beschäftigt. Hier heißt es \RWpar{IIIa}{3}, daß die Katholiken über die Frage, nach \seitenw{170} welcher Regel man sich in der Annahme seiner religiösen Meinungen zu richten habe, nicht durchaus gleichförmig dächten (Nr. 1.); daß aber alle darüber einig wären, denjenigen Theil ihrer Meinungen als eine von Gott selbst bestätigte Wahrheit betrachten zu dürfen, worüber alle gleichförmig denken (Nr. 2.); weil Christus selbst der Kirche einen in ihren allgemeinen Entscheidungen sie unfehlbar machenden Beistand des h. Geistes, und dieß zwar unbedingt versprochen habe (Nr. 3 u. 5.); doch nur in Dingen, welche die Religion betreffen, so daß Disciplinarvorschriften, rein wissenschaftliche und bloß geschichtliche Fragen ausdrücklich ausgenommen werden. (Nr. 6.) Es gebe deßhalb ein Fortschreiten in der Entwicklung des katholischen Lehrbegriffes, und kein Buch sey noch vorhanden, das diesen Lehrbegriff schon in der vollständigsten Entwicklung, welche ihm je zu Theil werden soll, enthielte. (Nr. 7. u. 8.) Indessen sey doch die h. Schrift (alten und neuen Bundes) ein Buch, welches die meisten Lehren enthält und bei gehörigem Gebrauche in einem so hohen Grade zu unsrer Belehrung und Erbauung geeignet ist, daß wir es nicht mit Unrecht als ein durch Gottes, oft bis auf die Worte sich erstreckende Eingebung geschriebenes Werk verehren. (Nr. 8.) \par
Was sagen nun wohl unsere Leser zu dieser ersten Lehre des Katholicismus? Der Umstand, daß weder die Regel, der päpstlichen Entscheidung sich zu fügen, noch sonst eine andere mit allgemeiner Übereinstimmung gelehrt wird, gewähret uns, das muß ihnen wohl einleuchten, den wichtigen Vortheil, daß wir in jedem vorkommenden Falle gerade das thun können, was die Vernunft uns als das passendste darstellt. -- Ob aber Nr. 2. wahr sey, das können und sollen die Leser nach B.'s Wunsche freilich erst beurtheilen, nachdem sie alle Lehren des Katholicismus im Einzelnen geprüft und der Vernunft gemäß und sittlich zuträglich befunden. -- Die Meinung, daß Jesus die Kirche gestiftet und ihr den Vorzug der Unfehlbarkeit selbst versprochen habe, brauchen sie keineswegs von Seite ihrer objectiven Richtigkeit erst mühsam zu untersuchen. Möchte es immerhin historisch unerweislich seyn, \seitenw{171} daß es die Absicht Jesu gewesen, eine religiöse Gesellschaft der Art zu stiften, wie in der Folge die katholische Kirche geworden: kann uns das hindern, diese Kirche gleichwohl in denjenigen ihrer Einrichtungen, die etwas unläugbar Gutes sind, als seine Anstalt (durch ihn veranlasset) zu betrachten, und auch zu glauben, daß ihr Lehrbegriff unter der Leitung des göttlichen Geistes zu Stande gekommen sey, wenn wir ihn durchaus vernunftgemäß und sittlich ersprießlich finden? -Daß dieser Beistand des göttlichen Geistes der Kirche unbedingt, \dh\  auch für den Fall verheißen sey, wenn einige ihrer Glieder bei der Annahme ihrer religiösen Ansichten nicht mit gehöriger Gewissenhaftigkeit verfahren: das könnte uns höchstens dann anstößig seyn, wenn wir vergäßen, daß dieser Beistand eine Wohlthat ist, die nicht demjenigen, der ihrer unwürdig ist, sondern den übrigen zu statten kömmt; ingleichen, daß ein an jene Bedingung gebundenes Versprechen so gut als gar keines wäre, weil sie gewiß nie von allen Seiten und in ihrem ganzen Umfange erfüllt wird. -- Was aber einige unserer Leser vielleicht im Ernste wünschten, ist, daß die Gabe der Unfehlbarkeit, wenn sie denn wirklich der Kirche beiwohnen soll, sich doch auf Mehreres, sich namentlich auch auf die Verordnungen erstrecken möchte, die ihre Vorsteher geben, und die durch ihre Unzweckmäßigkeit oft so viel Gutes verhindert und so viel Unheil herbeigeführt haben! Den bloßen Wunsch nun, daß dieß anders wäre, dürfen wir eben nicht verdammen, gehe nur Keiner so weit, es mit Entschiedenheit von Gott zu fordern, was ohngefähr eben so unbillig wäre, als wenn wir von Gott verlangen wollten, daß er auch allen weltlichen Obrigkeiten (deren Anordnungen oft von nicht minderer Wichtigkeit sind) die Gabe der Unfehlbarkeit ertheilet hätte. Im Buche selbst werden sie noch umständlicher auseinander gesetzt finden, daß und warum ihrem Verlangen auf keinen Fall entsprochen werden durfte. -- Da es nur Menschen, die auf einer schon etwas höheren Stufe der Bildung stehen, einleuchtet, wie nöthig ein allmähliches Fortschreiten in der Entwicklung des christlichen Lehrbegriffs sey, wenn er die vollkommenste Religion für alle Zeitalter seyn. und bleiben soll: so dürfen wir nicht begehren, \seitenw{172} daß sich die dießfällige Lehre der Katholiken in jedem Volkskatechismus finde. -- Was aber von der h. Schrift gelehrt wird, kann nur demjenigen zu glauben schwer fallen, der bei dem Worte Eingebung an eine unmittelbare göttliche Einwirkung auf das Gemüth der h. Schriftsteller denket. Wer die Begriffe des Vf.\ angenommen, kann der wohl anstehen zu sagen, es habe nur durch einen Zusammenfluß der seltensten Umstände dahinkommen können, daß wir \zB\ in den Büchern des alten Bundes so viele Stellen lesen, die auf Ereignisse in des Herrn Jesu Leben deuten, ohne daß die Verfasser derselben nur im Geringsten vorherwissen mochten, daß und wie nach Jesus an ihm einst in Erfüllung gehen würde, ja daß sie nur überhaupt von etwas Zukünftigem hier reden? \par
Doch nun zu dem wichtigsten, d.h. dem zweiten Punkte, der den Katholicismus von jeder anderen Religion wesentlich unterscheidet, noch einmal zurückzukehren, so fordern wir Jeden unserer Leser auf, eine der folgenden drei Behauptungen B.'s, wenn er es vermag, zu widerlegen: 1) daß den Lehrsatz von der Unfehlbarkeit des Gesammtglaubens keine Gesellschaftsreligion entbehren könne, wenn wir sie ansehen sollen als eine Offenbarung; 2) daß wir auch ohne irgend eine göttliche Weisung dazu zu besitzen, nicht besser thun können, als nur dasjenige mit höchster Zuversicht zu ergreifen, worüber Millionen, ja Alle, die von denselben Grundansichten mit uns ausgehen, sich vereiniget haben; 3) daß nur der erwähnte Lehrsatz des Katholicismus vollkommen entspreche dem entschiedenen Grundsatze, daß Gott bei einem jeden sittlich guten Zwecke, zu dessen Herbeiführung wir Menschen selbst etwas thun können, erst uns vorhalten müsse, alles das Unsrige zu leisten; dann aber, was wir aus Unvermögenheit oder auch Trägheit nicht geleistet haben, zum Besten des Ganzen durch seine eigene Dazwischenkunft vollenden dürfe. -- Man lese doch das ganze Hauptstück, oder vergleiche mindestens die §§. 6, 14, 17, 26 u. 28.\RWi{IIIa}{6}\RWi{IIIa}{14}\RWi{IIIa}{17}\RWi{IIIa}{26}\RWi{IIIa}{28}: und beantworte sich dann selbst die Frage, ob nicht vielleicht gerade in gewissen unwiderleglichen Bemerkungen, welche hier (freilich durchgehends nur mehr angedeutet als gehörig ausgeführt) anzutreffen sind, der wahre Grund liege, weßhalb \seitenw{173} B.'s Lehrbuch von einigen protestantischen Gottesgelehrten mit nur zu sichtbarer Erbitterung beurtheilt, ja als ein gefährliches Buch bezeichnet worden sey? \par
\gliederungslinie\par
In dem \RWHSfmt{zweiten Hauptstücke}\RWHS{Zweites Hauptstück}{31--234} dieses dritten Haupttheiles, welches als das weitläufigste von \RWpar[\BUparformat{31}. bis \BUparformat{234}.]{III}{31--234} reichet, behandelt der Vf.\ die \danf{\RWHSfmt{christkatholische Dogmatik},} indem er, nach einigen allgemeinen Bemerkungen über \danf{Begriff,\editorischeanmerkung{Öffnende Anführungszeichen ergänzt.} Vernunftmäßigkeit und sittlichen Nutzen der Geheimnißlehren} (\RWpar{IIIa}{32--34}), die wichtigsten theoretischen Lehren des Katholicismus unter folgenden sechs Abtheilungen: \par
1) Lehren von Gott, \par
2) von der Welt und den Geschöpfen, \par
3) vom Menschen insbesondere, \par
4) von den Verhältnissen Gottes zu uns, \par
5) von den Verhältnissen der Geschöpfe unter einander, \par
6) von den Belohnungen und Strafen, \par
zusammenstellt, ohne behaupten zu wollen, daß es nicht eine bessere Anordnung derselben geben könne. \par
In dem ersten Abschnitte: \danf{Lehren von Gott,} (\RWpar{IIIa}{35--140}) trägt er die Lehren vor: vom Daseyn eines in seinem Wesen nur einzigen Gottes, von dessen Allvollkommenheit, Unbegreiflichkeit, unendlichem Verstande, unendlich vollkommnem Willen, und zwar von dessen Allmacht, Freiheit und Heiligkeit (welche letztere er wieder zerlegt in die Arten: Gerechtigkeit, Wahrhaftigkeit, Unparteilichkeit, Güte und Gnade), von Gottes unendlicher Seligkeit, Körperlosigkeit, Allgegenwart und Ewigkeit, von Gottes dreifacher Persönlichkeit und endlich von dessen Rathschlüssen. \par 
In allen diesen Lehren, wenn wir die vorletzte für jetzt noch ausnehmen, wird man kaum etwas Anstößiges finden, so ferne man nur die von dem Vf.\ aufgestellten Erklärungen festhält, und von dem Vorurtheile läßt, als ob die Unendlichkeit Gottes forderte, daß jede seiner Eigenschaften \seitenw{174} unbegrenzt sey. So folgt aus der Allvollkommenheit Gottes wohl allerdings, daß seine Erkenntnißkraft, weil diese durch keine andere Kraft begrenzt wird, unbegrenzt, also allwissend sey; für Gottes Wollkraft aber und die mit ihr verbundene Kraft nach Außen zu wirken, bestehet unläugbar eine doppelte Schranke: die eine in seiner Erkenntnißkraft, wornach er nichts wollen kann, was sittlich böse wäre, \dh\  wovon die Vernunft erklärt, daß man es nicht wollen solle; die andre im absolut Unmöglichen, wornach er nichts wollen kann, was einer reinen Begriffswahrheit widerspricht. Nennen wir Gottes Willen in der ersten Beziehung heilig, so sind Gerechtigkeit, Güte \usw\ nur als besondere Arten der Heiligkeit zu betrachten; und es bestehet \zB\ die Gerechtigkeit Gottes darin, daß er nichts thut, wovon sogar wir Menschen selbst einzusehen vermöchten, daß es dem Wohle des Ganzen nicht zuträglich, oder überhaupt, nicht sittlich gut wäre; die Güte darin, daß er seinen Geschöpfen gar manche Wohlthaten erweiset, die nicht aus bloßer Gerechtigkeit folgen, \dh\  von denen wir nicht erkennen, daß das Wohl des Ganzen sie fordre, die aber in der That doch diesem, oder dem Sittengesetze entsprechend seyn müssen \usw\ \par
\gliederungslinie\par 
In der Lehre \danf{von Gottes dreifacher Persönlichkeit} (\RWpar{IIIa}{110--135}) begegnen wir freilich einer Lehre, die schon unzählige Male von Schwachen sowohl als von Übelgesinnten der aufgelegtesten Widersprüche mit der Vernunft beschuldigt worden ist. Mit welchem Rechte, wird sich bald zeigen. Statt der etwas weitläufigeren Darstellung, die \RWpar{IIIa}{110} zu lesen ist, wird es für unseren Zweck an dem Auszuge genügen, den der Vf.\ selbst in seinem Aufsatze: \danf{Mein Glaube} (abgedruckt in dem Buche: \danf{Religionsbekenntnisse zweier Vernunftfreunde.} Sulzbach, 1835) geliefert. \danf{Durch die Dreieinigkeitslehre,} heißt es daselbst S.\,83 ff., \danf{sollen wir erfahren, \par
1) daß Gottes sämmtliche Wohlthaten und Werke überhaupt unter drei Arten gebracht werden können, dergestalt, daß wir die Werke der einen Art Gott als Vater, jene \seitenw{175} der andern Gott als Sohn, jene der dritten endlich Gott als heiligem Geiste zuschreiben dürfen. Ohne ein allgemeines Kennzeichen anzugeben, aus welchem wir sogleich im Voraus abnehmen könnten, zu welcher dieser drei Arten des göttlichen Wirkens und Wollens jedes einzelne seiner Werke gehöre, wird uns nur beispielsweise gesagt, daß wir die Schöpfung, Erhaltung und Regierung des ganzen Weltalls Gott als dem Vater, die Erlösung des Menschengeschlechtes Gott als dem Sohne, die Erleuchtung, Heiligung und Beseligung jedes einzelnen Menschen Gott als dem heil. Geiste zu verdanken hätten. \par
2) Vater, Sohn und h. Geist sind also nicht etwa drei bloße Namen eines und desselben Gottes, welchen kein innerer, durch sie bezeichneter Unterschied in Gott zu Grunde läge; auch sind es nicht Namen von drei bloßen Wirkungsarten Gottes: sondern es sind drei in Gott selbst vorhandene Gründe, innere und reale, d. i. Wirklichkeit habende Gründe, aus welchen jene verschiedenen Wirkungen nur in der Zeit hervorgehen; es sind drei göttliche Personen. \par
3) Jeder von diesen drei Personen ist Gottheit, ist Allwissenheit, Allmacht, Heiligkeit, kurz jede der sogenannten natürlichen Prädikate der Gottheit beizulegen; und gleichwohl kann man nicht sagen, daß es drei Götter, drei Allwissende oder Allmächtige gebe; sondern dasselbe göttliche Wesen, dieselbe Erkenntnißkraft, dieselbe Wollkraft \usw\, welche im Vater ist, befindet sich auch im Sohne und im h. Geiste. \par
4) Bei aller Gleichheit in den genannten Stücken aber finden doch auch gewisse Unterschiede, und zwar selbst innere, unter den drei Personen des einigen Wesens der Gottheit statt. Die Person des Vaters nämlich hat keinen weiteren Grund ihres Daseyns, jene des Sohnes aber hat den Grund ihres Daseyns im Vater, jene des h. Geistes endlich in beiden. Vom Sohne sagen wir, er sey gezeuget vom Vater, und zwar von Ewigkeit; vom h. Geiste aber, er gehe aus vom Vater und vom Sohne, gleichfalls von Ewigkeit. \seitenw{176} \par
5) Obgleich der Sohn sowohl als auch der h. Geist einen Grund ihres Daseyns haben, so sind sie doch nicht Geschöpfe, nicht einmal sind sie abhängig von dem Vater, oder geringer als er zu nennen.} \par
Jeder Theolog wird uns zugeben, daß wenn nur diese fünf Punkte sich mit der Vernunft vollkommen vereinigen lassen, die wenigen etwa noch übrigen Bestimmungen der Kirche keiner ferneren Schwierigkeit unterliegen. Wir aber sagen, daß man nur folgende zwei Bemerkungen fest zu halten brauche, um jeden Schein eines Widerspruchs aus den fünf obigen Artikeln verschwinden zu sehen. Die erste ist, daß in dem Begriffe, welchen uns die Theologen mit dem Worte: Person, verbinden lehren, durchaus nichts liege, was uns verhinderte, zu denken, daß sich in einem einzigen Wesen mehrere Personen unterscheiden lassen; wie denn selbst ein und derselbe Mensch gar oft der Personen mehrere, \zB\ die eines Gelehrten, eines Staatsbürgers, eines Familienglieds \usw\ in sich vereiniget. Die andre Bemerkung ist, daß es kein Widerspruch sey zu lehren, der Sohn und der Geist hätten den Grund ihres Daseyns im Vater, und seyen doch eben so ewig als dieser; völlig so, wie auch die Rathschlüsse Gottes den Grund ihres Daseyns in den Erkenntnissen Gottes haben, und darum doch nicht als etwas in der Zeit erst Entstandenes, sondern Ewiges angesehen werden müssen. Die eigenthümliche Art, wie der Vf.\ \RWpar{IIIa}{132} die Wirkungskreise der drei Personen in Gott von einander scheidet, wenn er dem Vater alle diejenigen Anstalten, welche zum Wohle aller Geschöpfe erforderlich sind; dem Sohne die Anstalten, welche das Wohl des ganzen menschlichen Geschlechts betreffen, dem h. Geiste endlich diejenigen göttlichen Einwirkungen, die auf die Seele jedes einzelnen Menschen zu seiner Erleuchtung, Heiligung und Beseligung erfolgen, zuschreibt, diese Erklärung wird wohl, wenn wir sie annehmen, uns das Verständniß des Zusammenhanges zwischen den einzelnen Artikeln der ganzen Lehre erleichtern: zu dem Zwecke aber, um sie nur frei von jedem Widerspruche und der Vernunft gemäß zu finden, genügen völlig die zwei obigen Bemerkungen, und es bedarf \seitenw{177} durchaus keiner von jenen künstlichen, dem Dogma nur Gewalt anthuenden Deutungen, welche von Tertullians Zeitalter an bis auf den heutigen Tag (\zB\ von Schelling, Hegel u. A.) versucht worden sind. \par
Denn was zuerst jene Unterscheidung dreier Arten des göttlichen Wirkens betrifft, wer sollte in diese sich nicht zu finden wissen, auch wenn er den von dem Vf.\ angedeuteten Eintheilungsgrund gar nicht genehmigen wollte? Sind Schöpfung -- Erlösung -- Heiligung nicht deutlich genug unterschieden? Wenn aber diesen verschiedenen Arten des göttlichen Wirkens entsprechend auch drei Personen in Gott unterschieden werden: ist man hiezu nicht wenigstens eben so sehr berechtiget, wie zu der Unterscheidung verschiedener Personen in einem und ebendemselben Menschen? Dürfen wir nicht persönliche Wirkungen annehmen, wo immer wir Wirkungen antreffen, die durch Verstand und Willen erfolgen? So wenig wir aber drei Wesen nöthig haben, um drei Personen in Gott zu unterscheiden; so wenig brauchen wir eine dreifache Erkenntniß- oder Wollkraft in ihm vorauszusetzen, weil eben dieselbe Erkenntnißkraft, welche beurtheilt, wie Gott als Vater, auch beurtheilt, wie er als Sohn und Geist zu wirken habe \usw\ Wo bleibt also nun der abgeschmackte Einwurf, daß Eins nicht Drei und Drei nicht Eins seyn könne? Daß aber der Sohn sein Daseyn vom Vater, der Geist sein Daseyn von Vater und Sohn habe, begreifet sich ja schon aus dem Umstande, weil auch die Werke des Sohnes, wie die Erlösung, jene des Vaters, wie die Weltschöpfung; die Werke des Geistes aber, die Erleuchtung, Heiligung und Beseligung eines jeden Einzelnen aus uns, die Werke des Vaters und des Sohnes zugleich voraussetzen. Ist ferner auch gegen die Ewigkeit des Sohnes und des Geistes bloß deßhalb, weil sie einen Grund ihres Daseyns haben, nichts einzuwenden: so ist es noch leichter zu begreifen, daß sie nicht Geschöpfe genannt werden dürfen, weil sie ja keine vom Vater verschiedene Wesen, sondern Personen in einerlei Wesen sind. Abhängig aber von ihrem Grunde, und von geringerem Range als er, mag eine Folge allerdings gar oft genannt werden: warum gleichwohl die Kirche \seitenw{178} nicht zulassen wolle, daß wir den Sohn und Geist abhängig und geringer als den Vater nennen, begreift sich aus dem sittlichen Nutzen der Lehre. \par
Auch diesen wollen wir hier noch wörtlich so mittheilen, wie der Vf.\ ihn in dem erwähnten Aufsatze auf einer einzigen Seite (S.\,87) darstellt: \danf{Schon daß die Lehre vom Dreieinigen Geheimnisse einschließt, und hiedurch uns Gelegenheit gibt, inne zu werden, wie wenig wir Alle, und wären wir auch die Weisesten, weise genug sind, um Gott zu begreifen, schon dieses betrachte ich als einen sittlichen Vortheil derselben. Ein anderer ist, daß uns durch diese Lehre ein deutlicherer Begriff von den mancherlei Wohlthaten Gottes zu Theil wird. Wir hören hier nämlich, daß jede der göttlichen Personen, \dh\  jeder der drei in Gott vorhandenen Gründe des Wirkens sich für uns thätig bewiesen habe von Anbeginn, und auch in Ewigkeit fort thätig sich zu beweisen fortfahren werde. Wie sollte uns dieses nicht erheben, wie nicht mit reger Dankbarkeit gegen Gott erfüllen! Wie geeignet sind ferner auch selbst die Namen, welche Gott in diesem Geheimnisse sich anzunehmen würdigt, um die Gefühle der Ehrfurcht, der Liebe und des Vertrauens in uns zu erwecken und zu erhalten! Gott, unser Schöpfer, Erhalter und Regierer, ist also auch ein Vater, und ist es nicht bloß zu seinem eingeborenen Sohne, sondern auch zu uns Allen, welche dieser Sohn seine Brüder genannt hat. Und der Geist Gottes will seine Wohnung aufschlagen in unsern Herzen, und will uns weihen zu seinem lebendigen Tempel, daß unsere Glieder Glieder des Geistes Gottes würden! Und daß wir ja nicht fürchten, die Einrichtungen in der Welt, welche das Beste unsers Geschlechtes, um so gewisser diejenigen, die nur das Beste eines Einzelnen aus uns bezwecken, dürften als etwas höchst Unbedeutendes verschwinden, weil auch wir selbst und unser ganzes Geschlecht, verglichen mit dem unendlichen All, verschwinden: so wird uns eingeprägt, daß sich Gott als Sohn und Geist nicht geringer denn als Vater erweise; daß er mit gleicher Allmacht, Weisheit und Güte, wie für das Ganze für jeden \seitenw{179} Einzelnen sorge. Wie viele Berührungspunkte endlich für die Erinnerung an Gott bietet uns diese Lehre nicht durch die vielen, zum Theile selbst von sinnlichen Gegenständen entlehnten Begriffe und Benennungen dar! wie sehr erleichtert sie uns hiedurch jenes beständige Denken an Gott, ohne welches, wie schon der Heide Seneca erkannte, Niemand ein wahrhaft guter Mensch wird!} -- Wir zweifeln, ob auch nur Einer derjenigen, welche die Trinitätslehre als eine im günstigsten Fall nutzlose Speculation verwarfen, diese wichtigen Vortheile derselben je in Erwägung gezogen habe? \par
Aus der Lehre \danf{von Gottes Rathschlüssen} (\RWpar{IIIa}{136--140}) entfernt der Vf.\ alles Anstößige, indem er a) vorbeugend dem Gedanken, daß Gottes unbedingte Rathschlüsse gar keinen Grund hätten, erkläret, daß ihr vornehmster Grund nur eben nicht in dem Wesen, welches sie unmittelbar betreffen, sondern in der Beförderung des Wohles Anderer liege; dann b) den Unterschied zwischen wirkenden und bloß zulassenden Rathschlüssen Gottes nicht in dem größeren oder geringeren Antheile sucht, den Gottes Thätigkeit an deren Verwirklichung nimmt; sondern bloß darin, daß jene etwas an sich selbst Gutes, diese dagegen etwas betreffen, das an sich böse ist, aber nicht verhindert werden kann, ohne irgend ein größeres Übel. Dieß Letztere wird uns um so begreiflicher, je mehr der Vf.\ schon im Vorhergehenden (und in der \danf{Wissenschaftslehre}) dafür gesorgt hat, die irrige Vorstellung zu entfernen, als ob alle reine Begriffswahrheiten, wie alle Lehrsätze der reinen Mathematik und Physik nur eben auf Gottes Belieben beruhten. Einer Menge hieher einschlagender Verirrungen wird endlich bloß dadurch vorgebeugt, daß der Vf.\ die verschiedenen Bedeutungen, in denen wir das Wort Bestimmung nehmen, genauer unterscheidet. Hierüber müssen wir jedoch, um nicht zu weitlaüfig zu werden, auf das Buch selbst verweisen. \par
In dem zweiten Abschnitte: \danf{Katholische Kosmologie,} (\RWpar{IIIa}{141--166}) ist die Lehre \danf{von Gottes \seitenw{180} Schöpfung und Weltregierung,} wie auch die \danf{von den Engeln} der nüchternen Vernunft so zusagend, daß wir kaum eine Einwendung von Seite unserer Leser besorgen; zumal da Jeder, der sich mit dem Wortlaute einiger Artikel in der letzteren, wie etwa des von den bösen Geistern, nicht vereinigen könnte, gewiß alles Anstößige verschwinden sehen wird, sobald er zu einer ihm unverwehrten bildlichen Auffassung seine Zuflucht genommen. Daß der Vf.\ das Schaffen nicht als einen in der Zeit vor sich gegangenen Act, sondern als ein beständiges Daseyn der begrenzten Substanzen durch das beständige Wollen der Einen unendlichen Substanz sich denkt, bestreite und widerlege man, wenn man's vermag; man nenne es auch, selbst ohne dieß zu vermögen, eine für Ungebildete schwer zu fassende, deßhalb gefährliche Lehre: als eine Ketzerei sie zu bezeichnen, dürfte man sich nicht einfallen lassen. Denn Jeder muß doch einsehen, daß das Anstößige, das diese Lehre für Ungebildete hat, lediglich davon herrührt, daß Solche gewohnt sind, die Wirkung immer später als ihre Ursache (nämlich nicht die vollständige, sondern nur eine bloße Theilursache) eintreten zu sehen; daß es dagegen dem geübteren Denker sehr leicht wird, dieß zu begreifen, und daß es die Gefühle der Bewunderung und Dankbarkeit, welche er Gott zu zollen wünschet, gar sehr erhöhe, wenn er sich vorstellen darf, daß die Welt nicht erst sechstausend, auch nicht sechstausendmal sechstausend Jahre bestehet, sondern von Ewigkeit ist, und ebenso auch dem Raume nach in das Unendliche sich ausdehnt. \par
In der Lehre \danf{vom letzten Weltzwecke} nimmt der Vf.\ vielleicht mit Unrecht an, daß die Verherrlichung Gottes mit einer solchen Allgemeinheit als dieser letzte Zweck bezeichnet werde; doch wenn man erst des Vf.\ weitere Auslegung vernommen, daß dieses nämlich nur der höchste Zweck, den wir uns vorzustellen hätten, nicht aber der objectiv höchste wäre, daß dieser letztere vielmehr in der möglich größten Beförderung der Tugend und Glückseligkeit der geschaffenen Wesen liege, dann wird sich wohl Niemand an diese Lehre stoßen. Aus der \danf{Lehre von den Engeln} (\RWpar{IIIa}{159--166}) wollen wir nur hervorheben, daß der Vf.\ sich erlaubt, \seitenw{181} auch ihnen gewisse Leiber zuzuerkennen, und der entgegenstehenden Lehre schon genug gethan zu haben meinet, wenn er nur beisetzt: es seyen keine sinnlichen Leiber, und ihre Vergnügungen seyen deßhalb nicht unsre sinnlichen. Das ist nun wirklich schon genug, weil es für jenen Zweck, um dessentwillen so gelehret wird, vollkommen hinreicht. Dieß nämlich ist kein anderer als zu verhüten, daß die sinnlichen Vergnügungen uns nicht um so werthvoller erschienen, wenn wir uns vorstellten, daß sich auch höhere Geister auf solche Art ergötzen. \par
\gliederungslinie\par

\noindent\RWsec{IIIb}{167}{303}\noindent
Den dritten Abschnitt: \danf{christkatholische Anthropologie} (\RWpar{IIIb}{167--196}) eröffnet eine kurze Lehre \danf{von den Bestandtheilen und Kräften des Menschen.} Um mit dem Wenigen, was hier eigentliche Glaubenslehre ist, sich einverstehen zu können, genügt es nur in irgend einem Sinne einen Gegensatz zwischen Seele und Leib zuzugestehen, und die Fortdauer unsers Bewußtseyns nach dem Tode nur nicht für unmöglich zu halten? Wer nämlich meint, -- und Viele meinen es -- daß die Gründe, die unsre bloße Vernunft für die Unsterblichkeit der Seele darbietet, zwar ihre Möglichkeit zeigen, keineswegs aber hinreichen, dieser \danf{erwünschlichsten aller Wahrheiten} denjenigen Grad der Gewißheit zu ertheilen, nach welchem es uns verlanget: der muß wohl die Versicherung, die uns das Christenthum hierüber gibt, als eine der größten Wohlthaten, welche uns durch eine Offenbarung nur immer zu Theil werden können, betrachten. Allein selbst wer so unglücklich wäre, die Fortdauer unsers Bewußtseyns nach dem Tode für etwas schlechthin Unmögliches zu halten, wer somit die hier in Rede stehende christliche Lehre in ihrem Wortsinne nicht für wahr annehmen könnte, der könnte und müßte sie doch als eine bildliche Vorstellung, die uns das Leben verannehmlicht und den Tod ungemein erleichtert, in Ehren halten. Nur für denjenigen, dem der Gedanke an eine Fortdauer mit Rückerinnerung ein furchtbarer Gedanke ist, kennet das Christenthum keinen Trost: doch einem Solchen wagen wir zuzurufen, daß es in seiner sittlichen Verfassung irgend ein wesentliches Gebrechen geben müsse, und erst wenn \seitenw{182} er dieses verbessert, werde auch er die Lehre von der Unsterblichkeit wohlthätig finden. Nicht so hart dürfen wir Jene beurtheilen, welche die Fortdauer ihrer Persönlichkeit nicht eben fürchten, aber in diesem Glauben auch gar nichts Erfreuliches, namentlich gar keine Ermunterung zu einem desto vollkommneren Wandel auf Erden anerkennen wollen. Diese fragen wir nur, wie lange sie eigentlich schon so denken, und ob denn ihre Gesinnung auch schon alle Proben bestanden? Übrigens wollen wir ihnen, was für sie keinen Werth hat, auch eben darum nicht aufdringen. Vielleicht kommt noch die Zeit, wo auch sie anders denken lernen. \par
Was man gewöhnlich die Lehre \danf{von der Abstammung aller Menschen von einem einzigen Paare} nennt, bezeichnet und faßt B.\ auf als eine Lehre \danf{von aller Menschen wesentlicher Gleichheit} (\RWpar{IIIb}{172--176}), d.h. als den Satz, daß alle Menschen wesentlich gleiche Rechte und Ansprüche auf den Genuß der irdischen Güter haben. Denn diese wichtige Wahrheit gründet sich, wie er bemerkt, nicht auf die erweisliche Abstammung aller Menschen von einem einzigen Paare, sondern einzig darauf, daß alle Menschen eine wesentlich gleiche Empfänglichkeit für den Genuß der erwähnten Güter haben. Wir wollen sehen, ob man dieß widerlegen werde? \par
Die dem Menschen (dem unverdorbenen) ertheilte Erlaubniß, sich als das Ebenbild Gottes auf Erden zu betrachten, stützt der Vf.\ darauf, daß der Mensch unter allen (sichtbaren) Erdbewohnern die meiste Ähnlichkeit mit Gott besitze. \par
Daß aber der Zustand, in welchem sich unser Geschlecht bis auf den heutigen Tag befindet, noch gar nicht von der Art sey, daß man damit zufrieden seyn könnte; daß vielmehr eine große sittliche sowohl als physische Verdorbenheit unter uns herrsche, betrachtet B.\ als eine Wahrheit, die keine vernünftige Religion verhehlen dürfe; und zeigt \RWpar{IIIb}{182}, daß das Christenthum diese Verdorbenheit mit den lebhaftesten Farben schildere; aber auch darauf dringe, daß an der Verbesserung dieses Zustands unaufhörlich gearbeitet werde. Um unsern Glauben an die Möglichkeit eines Besserwerdens zu beleben, und zugleich auch uns einige Winke zu geben, worauf \seitenw{183} wir zu diesem Zwecke vornehmlich hinwirken müssen, stellt es B.'s Ansichten nach die drei innig zusammenhängenden Lehren von der Vollkommenheit des ursprünglichen Zustandes des Menschen, vom Sündenfalle und von der Erbsünde (\RWpar{IIIb}{183--196}) auf, von denen B.\ immer zu behaupten pflegte, daß sie zu den schönsten und wichtigsten gehören, die eine Religion nur immer aufstellen könne. Was einige Schwierigkeit bei diesen Lehren verursacht, ist einzig die ungeschickte Auffassung derselben, wenn man \zB\ die Ereignisse, die nach dem Wortsinne darin vorausgesetzt werden, daß unser ganzes Geschlecht seinen Anfang von einem einzigen Paare genommen, daß dieses erste Paar mit so ausgezeichneten Kräften und Vorzügen des Leibes und der Seele ausgeschmückt gewesen, von seiner hohen Stufe der Vollkommenheit plötzlich durch eine einzige Sünde herabgestürzt sey \usw\, über jeden Zweifel geschichtlich sicher gestellt sehen will. Denn dieß sind freilich wohl Erzählungen, deren historische Wahrheit nach dem Urtheile Vieler noch nie erwiesen worden ist. Scheinet doch der Verfasser des alten Aufsatzes, aus dem wir sie gemeinhin schöpfen, nicht einmal selbst gewollt zu haben, daß man seine Worte als die Erzählung eines wirklichen Ereignisses auslege. Allein was nöthigt uns, über dieß Alles erst zu entsch\erg{eid}en, wenn alle praktischen Folgerungen, welche die Kirche aus diesen Ereignissen gezogen wissen will, unwidersprechlich wahr sind und bleiben, wie es auch immer sich mit dem eigentlichen Hergange selbst verhalte? Was ist der Zweck der Erzählungen, die wir 1 Mos. 2, 7 ff. lesen? welche Folgerungen sollen wir ans ihnen ziehen? und welche Wahrheiten sollen sie uns versinnlichen? Die wichtigsten sind gewiß nur folgende: daß Gott schon bei der Entstehung unsers Geschlechtes die väterlichste Fürsorge für uns getragen, und uns hervorgebracht habe auf eine Weise, welche den Adel unsrer sittlichen Natur beurkundet; daß die gegenwärtige Schwäche und Hinfälligkeit unsers Leibes (durch welche \zB\ das Gebären mit so großen Schmerzen und Gefahren verknüpft ist u. m. a.) keineswegs auf unabänderlichen Naturgesetzen beruhe, daß wir im Gegentheil nur durch unsre eigenen Laster und Thorheiten dergleichen Übel herbeigeführt \seitenw{184} haben, durch Rückkehr zur Tugend und Weisheit aber hoffen dürften, uns von denselben wenigstens theilweise wieder zu befreien; daß nur die Sünde den Tod in die Welt gebracht habe, weil nur sie es sey, welche ihm alles Schreckliche gibt und den Tod erst zum Tode macht; daß wir durch eine jede Abweichung vom Gesetze Gottes nicht nur uns selbst und unsrer nächsten Umgebung, sondern auch unsern entferntesten Nachkommen schaden können, so zwar, daß sich die unseligen Folgen einer einzigen bösen That zuweilen auf ganze Geschlechter erstrecken; daß der starke Hang zur Sinnlichkeit, den wir in uns verspüren, und unsre große Reizbarkeit zum Bösen nur durch uns selbst allmählich herbeigeführt worden; daß aber diese Fehler, nachdem sie einmal vorhanden sind, von Seite Gottes eine härtere Behandlung, eine Behandlung, die fast das Ansehen hat, als zürne er uns, nothwendig für uns machen; daß der erste Schritt zum Bösen insgemein gar viele andere nach sich zieht; daß es eben deßhalb nicht unbillig sey, wenn Gott gleich diesen ersten Schritt oft mit der äußersten Strenge ahndet, \usw\ Ist nun dieß Alles nicht wahr, und bleibt es nicht wahr, auch wenn es nie ein Paradies, ja nie einen Adam und eine Eva gegeben hätte? -- --\par
Wir müssen aber bemerken, daß der Vf.\ die Lehre vom Sündenfall mit einem Artikel vermehrte, den man in den gewöhnlichen Dogmatiken nicht antrifft, nämlich, daß nicht nur die erste Sünde geschadet, sondern daß auch alle folgenden Sünden und Thorheiten, die von irgend einem Menschen begangen worden sind, oder noch jetzt begangen werden, zur Vermehrung des Übels auf Erden mitgewirkt haben und noch immer mitwirken. Wir besorgen nicht, es werde irgend Jemand diese Wahrheit in Abrede stellen: oder behaupten, daß sie zur Lehre vom Sündenfall nicht gehöre. -- In der Lehre von der Erbsünde endlich behauptet B., diese sey nicht eine eigentliche (wirkliche) Sünde, d.h. nicht eine freiwillige Übertretung des Sittengesetzes, sondern nur Sünde in jener weitern Bedeutung, in der man darunter etwas Gott Mißfälliges, das seinen Ursprung in der Sünde hat, verstehet. Von diesem Mißfallen (das ohnehin bloß bildlich verstanden seyn will, weil eigentliches Gefallen oder Mißfallen \seitenw{185} in Gott nicht durch Geschöpfe erregt werden kann) behauptet der Vf., daß nur die beiden Grenzen desselben angegeben wären; \danf{a) es sey größer, als daß Gott dem Menschen, der die Erbsünde noch an sich hat, jene höheren übernatürlichen Seligkeiten angedeihen lassen könnte, die er dem menschlichen Geschlechte, da es noch ohne Sünde war, zugedacht hatte; b) es sey nicht so groß, daß es den heiligen Gott in der Behandlung der Menschen von jener Regel der Gerechtigkeit abbringen könnte, zufolge welcher er nur den Bösen bestraft, und jeden Tugendhaften belohnt.} \par
Unschwer läßt sich nun rechtfertigen, sowohl daß es eine solche Erbsünde gebe, als auch, daß einzelne Personen, namentlich Jesus derselben nicht unterworfen gewesen, ingleichen, daß wir durch Aufnahme in die christliche Kirche von ihr befreiet werden. Denn in der That, die für uns selbst so wie für unsere Mütter höchst schmerzliche und gefahrvolle Weise, auf die wir zur Welt geboren werden; die vielen Krankheiten, denen wir ausgesetzt sind, welche den größten Theil von uns schon in den ersten Jahren unseres Lebens wieder dahin raffen; die vielen übrigen Leiden und ungünstigen Verhältnisse, die daran Ursache sind, daß unter Hunderten kaum Einer aus uns Gelegenheit findet, seine geistigen Kräfte und Anlagen gehörig zu entwickeln; die große Reizbarkeit zum Bösen, welche so frühzeitig schon sich bei uns Allen äußert und in Verbindung mit den üblen Beispielen, die wir um uns sehen, uns wie mit Allgewalt zu ähnlichen Sünden und Thorheiten fortreißt; die verderblichen Einrichtungen, die in unsere bürgerlichen Verfassungen so tiefe Wurzeln geschlagen haben, daß fast nicht abzusehen ist, wie wir uns je wieder von ihnen losreißen werden; die noch weit schrecklicheren religiösen Irrthümer, der Wahn des Polytheismus u. A., in deren Fesseln der größte Theil des menschlichen Geschlechtes noch immerwährend schmachtet; das feindliche Verhältniß, in welchem nicht nur die ganze thierische Schöpfung, sondern auch leblose Wesen, ja alle Elemente zu uns, den angeblichen Herrn dieser Erde stehen; endlich die traurige Erfahrung, daß der Mensch so, wie er nun einmal ist, gute Tage gar nicht zu tragen vermag, weil \seitenw{186} sie ihn alsbald nur übermüthig, ausschweifend, träge und lasterhaft machen: dieß Alles bezeuget es nicht nur allzu sehr, daß unser Geschlecht das nicht geworden sey, wozu es ursprünglich wohl eine Anlage gehabt; und folgt hieraus nicht, daß Gott uns einer strengern Behandlung unterwerfen müsse, als es diejenige gewesen wäre, die er uns konnte angedeihen lassen, wenn wir nie gesündiget hätten? Kann somit nicht mit vollem Rechte gesagt werden, daß wir ein Gegenstand göttlichen Mißfallens seyen? daß sich an einem Jeden aus uns, schon von Geburt an, etwas befinde, das Gott mißfällig ist, und weil es seinen Grund in den Sünden und Thorheiten unserer Vorfahren hat, eine von ihnen auf uns vererbte Sünde gar füglich heißen mag? -- Daß aber doch Einzelne von dieser Erbsünde ausgenommen seyn können, daß namentlich an Jesu Christo gewiß nichts Gott Mißfälliges anzutreffen gewesen, daß vielmehr Alles, was sich an ihm von seiner Geburt an wahrnehmen ließ, ihn uns als einen Gegenstand des ausgezeichnetsten göttlichen Wohlgefallens darstellt: wer wollte das bestreiten? Wer wollte endlich nicht zugeben, daß auch alle diejenigen, die so glücklich sind, in die Gesellschaft der christlichen Kirche als Mitglieder aufgenommen zu werden, sofort dem freudigen Gedanken Raum geben dürfen, daß sie nicht ferner Gott mißfällig seyen wegen desjenigen, was sich bloß angeerbter Weise an ihnen befindet? Wahr ist es freilich, daß durch die h. Handlung der Taufe ihre Natur durchaus nicht umgeschaffen werde, daß sie auch nach wie vor reizbar, zum Bösen geneigt \usw\ bleiben: aber auch eben so wahr, daß sie nun aufgenommen sind in eine kirchliche Gesellschaft, in welcher ihnen so viele natürliche sowohl als übernatürliche Hülfsmittel zur Beherrschung ihrer Begierden, so viele Gnaden Gottes geboten werden, daß es sich schlechterdings nicht geziemen will, zu sprechen, daß sie auch jetzt noch Gegenstände des göttlichen Mißfallens wären. \par
\gliederungslinie\par
Der vierte Abschnitt: \danf{von den Verhältnissen Gottes zu uns Menschen} (\RWpar{IIIb}{197--220}), beginnt mit einer Lehre \danf{von Gottes Vatersinne,} (\RWpar{IIIb}{198--202}), \seitenw{187} die man in den gewöhnlichen Dogmatiken nicht aufgestellt findet. Dennoch besorgen wir nicht, man werde sagen, daß sie nicht echt katholisch wäre. Aber auch eben so wenig wird Jemand etwas gegen die Vernunftmäßigkeit oder den sittlichen Nutzen dieser Lehre etwas einwenden wollen. \par
Ungleich schwieriger ist, ja nach der Lehre von der Dreieinigkeit als die schwierigste wurde von jeher betrachtet die nun folgende \danf{von der Menschwerdung des Sohnes Gottes,} welche wir eben deßhalb wörtlich so anführen wollen, wie der Vf.\ sie \RWpar{IIIb}{203} vorträgt: \par
\danf{1) Auf den Willen des Vaters in Gott beschloß der Sohn, der mit ihm desselben Wesens ist, im Verfolge der Zeiten menschliche Natur an sich zu nehmen. \par
2) Zu diesem Ende ward vor etwa achtzehnhundert Jahren eine Jungfrau, mit Namen Maria, ohne die MitWirkung eines Mannes, bloß durch die Kraft des Geistes Gottes mit einem Kinde schwanger, welches den Namen Jesus (d. i. Erlöser) erhielt. \par
3) Dieß Kind war ohne Erbsünde empfangen. \par
4) Es war ein wirklicher, nicht ein bloß scheinbarer Mensch, der einen wirklichen Leib und eine wirkliche Seele hatte. \par
5) Mit diesem Menschen war die zweite göttliche Person von seines Daseyns erstem Augenblicke an (also schon in dem Schooße seiner jungfräulichen Mutter) auf's Innigste vereiniget. \par
6) Diese Vereinigung war eine Vereinigung zweier Naturen, nämlich der göttlichen (des Sohnes) und der menschlichen (des Menschen Jesu) zu einer einzigen Person, welche die Namen des Gottmenschen, des Erlösers, Messias, Christus trägt. \par
7) Diese Vereinigung war also keine Vermengung oder Vermischung beider Naturen zu einer dritten, die weder menschliche noch göttliche Eigenschaften an sich gehabt hätte. \par
8) In dem Gottmenschen befindet sich vielmehr: \seitenw{188} \par
a) ein doppelter Verstand, nämlich ein göttlicher, jener des Sohnes, und ein menschlicher, jener des Menschen Jesu; und dieser letztere erfährt durch die Erleuchtung des ersteren so viel, als nur ein menschlicher Verstand an sich zu fassen vermag, und als insonderheit Jesus zur Ausführung seiner erhabenen Zwecke, \zB\ als Lehrer der vollkommensten Religion zu wissen bedurfte. \par
b) Ein doppelter Wille, nämlich ein göttlicher, jener des Sohnes, und ein menschlicher, des Menschen Jesu. Dieser letztere stimmte aber mit jenem ersteren immer so vollkommen überein, daß Jesus der einzige Mensch ohne alle Sünde geblieben. \par
9) Auch von dem Gottmenschen kann man sagen, daß er (nämlich als Mensch, oder von Seite seiner menschlichen Natur) menschliche Pflichten und Obliegenheiten gehabt, Lust und Schmerz empfunden, selbst sich Versuchungen ausgesetzt gefühlet habe, dem Leibe nach sterblich gewesen und wirklich gestorben sey. \par
10) Von der anderen Seite kann man von diesem GottMenschen auch sagen, daß er -- nämlich als Gottes Sohn -- allmächtig, allwissend \usw\ sey. \par
11) Die Vereinigung, die zwischen dem Sohne Gottes und dem Menschen Jesu vom ersten Augenblicke der Entstehung des Letztern angefangen hat, währt seitdem immer fort und wird nie wieder aufgelöst werden. \par
12) Man kann sich der Redensarten bedienen, Gott sey Mensch geworden, Gott habe für uns gelitten \par
und sey am Kreuze gestorben \udgl\ ; nicht aber soll man sprechen: Ein Mensch sey Gott geworden; oder die Gottheit sey Mensch geworden, gestorben \udgl\ } \par 
Auch diese Lehre enthält historische Bestandtheile, die Manchem nicht erwiesen genug scheinen könnten. Allein wer nur verstehet, gehörig zu unterscheiden, aus welchen Vordersätzen eine Folgerung eigentlich fließt, und welche andere Sätze immerhin fallen können, ohne daß jene Folgerung nur im Geringsten wankend gemacht wird, der wird nicht fürchten, \seitenw{189} daß dasjenige, was er als Christ glauben soll, erschüttert werden könne durch irgend einen Einwurf, der von historischer Seite gegen die Lehre von der Menschwerdung erhoben werden möchte. Gehen wir nur die uns vorliegenden Artikel dieser Lehre mit Hinsicht auf die \RWpar{IIIb}{206} aus ihnen abgeleiteten sittlichen Folgerungen durch. Die Artikel 1, 3, 6, 10 und 12 geben sich selbst als Lehren, die etwas Bildliches enthalten, und werden unseren Lesern, wenn sie sich erst mit den übrigen einverstanden haben, gewiß keine weitere Schwierigkeit verursachen. Denn daß der Sohn Gottes auf den Willen des Vaters beschlossen habe, im Verlaufe der Zeiten menschliche Natur an sich zu nehmen, heißt ja nichts Anderes, als das Geschäft der Menschenerlösung gehöre als ein Bestandtheil zum Plane der ganzen Weltregierung. Daß keine Erbsünde an Jesu gehaftet, bedeutet (wie wir schon wissen) nur, daß sich in seiner Natur nichts Gott Mißfälliges gefunden habe. Daß die Vereinigung der göttlichen mit der menschlichen Natur in Christo die Vereinigung zu einer einzigen Person gewesen, zeigt uns nur an, daß der Wille des Menschen in Folge dieser Vereinigung nie mit dem Willen Gottes in Widerspruch getreten. Daß man von dieser Person bezüglich auf Gott sagen könne, sie sey allwissend, allmächtig \usw\, unterliegt gar keinem Bedenken. Daß man sich wohl die Redensart: \danf{Gott (eine göttliche Person) sey Mensch geworden,} nicht aber: \danf{die Gottheit (das göttliche Wesen) sey Mensch geworden,} noch weniger: \danf{ein Mensch sey Gott geworden,} habe gefallen lassen, begreift und rechtfertigt sich von selbst. -Was dem zweiten Artikel, daß Jesus von einer Jungfrau durch die Kraft Gottes empfangen worden sey, religiöse Wichtigkeit gibt, ist lediglich die \RWpar{IIIb}{206} Nr. 2. angedeutete doppelte Anwendung desselben, die Würde des jungfräulichen Standes, und die Würde Jesu uns anschaulicher zu machen. Allein was an dem jungfräulichen Stande in der That Schätzbares und Verdienstvolles ist, muß ihm das die Vernunft nicht ewig zugestehen, ob es sich mit der Entstehung der menschlichen Natur Jesu so oder anders verhalten habe? Und Jesu eigene Würde, wer begreifet es nicht, daß der wahre Werth eines Menschen nie durch die Art seiner Entstehung, sondern durch \seitenw{190} seine innere Beschaffenheit allein bestimmt wird? -- Wegen der Artikel 4, 7 und 9, daß Jesus ein wirklicher Mensch gewesen, daß die Vereinigung Gottes mit ihm keine Vermengung oder Vermischung beider Naturen gewesen, daß endlich Jesus menschliche Pflichten, Empfindungen und Versuchungen gehabt -- hierwegen findet in unseren Tagen kein Zweifel statt. Der fünfte Artikel, daß die Vereinigung des Sohnes Gottes mit Jesu schon von seines Daseyns erstem Augenblicke begonnen, soll nur an die wichtige Wahrheit erinnern, daß für die Entwicklung des menschlichen Geistes nicht gleichgültig seyen auch schon die Eindrücke, die er noch in dem Schooße der Mutter erfährt. Und könnten wir zweifeln an dieser Wahrheit, oder müßte die Lehre von der Menschwerdung auch nur aufhören, uns ein Erinnerungsmittel daran zu seyn, wenn wir die eigenthümliche Thätigkeit Gottes, welche in diesem Artikel vorausgesetzt wird, in Zweifel ziehen könnten; obgleich ich, aufrichtig gestanden, nicht begreife, wie man bei der Unbestimmtheit, in welcher die Kirche die Art dieser Thätigkeit läßt, zu einem solchen Zweifel sich auch nur versuchet fühlen könnte. Alles kommt also nur noch auf den achten Artikel an, in welchem der Einfluß, den die Vereinigung Gottes mit dem Menschen Jesu auf dessen Verstand und Willen gehabt, genauer festgesetzt wird. Dieser Artikel bewirket aber, und soll nichts Anderes bei uns bewirken, als dieses Beide: daß wir einerseits nicht zweifeln an der hohen Vervollkommnungsfähigkeit unsrer Natur, durch die es uns mit Gottes Beistande möglich seyn würde, ganz ohne Sünde zu leben; und daß wir andererseits alle diejenigen Lehren des Herrn, die uns das Evangelium vorhält, und alle diejenigen Gesinnungen, die sich aus seinem dort erzähltem Thun und Lassen uns zu erkennen geben, auch zu den unsrigen machen. Das Erste bleibt nun jedenfalls unbestreitbar; das Zweite aber könnte allerdings seine Gefahren haben, wenn nicht die Kirche ausdrücklich erklärte, daß wir nie eine solche Auslegung jener Lehren und Gesinnungen als eine echte annehmen dürfen, die sie verwirft, oder nach welchen Jesus etwas gethan oder gesagt hätte, was der Vernunft widerspricht. Halten wir uns an diese Auslegungsregel, was haben wir dann zu gefährden, die Wirk\seitenw{191}lichkeit möchte gewesen seyn, welche sie wolle? Wenn endlich der 11. Art. von uns zu glauben verlangt, daß die zwischen dem Sohne Gottes und Jesu Christo begonnene Vereinigung nie wieder aufgehört habe, noch jemals aufhören werde; so soll uns dieß zweierlei leisten: eine Bestätigung der allgemeinen Wahrheit, daß Gott von Niemand weicht, der nicht erst selbst von ihm weicht, und die Hoffnung, daß Jesus auch noch jetzt und in alle Ewigkeit für die Beseligung der Menschheit fortwirke. Das Erste betrifft eine Wahrheit, die auch, wenn diese Bestätigung derselben wegfallen könnte, unerschüttert feststeht; das Zweite begründet eine Hoffnung, von welcher selbst derjenige nicht besorgen könnte, daß sie je werde zu Schanden werden, der aus was immer für einem Grunde nicht recht daran glauben konnte, daß das unmittelbare Wirken Jesu für unser Geschlecht in alle Ewigkeit fortdauern werde. Denn was die Geschöpfe nicht unmittelbar vermögen, vermag und thut ihr Schöpfer, wenn er's versprochen hat. Und so zeigt sich denn, daß am Ende selbst der hartnäckigste Zweifler an der Geschichte, der katholischen Lehre von der Menschwerdung nichts anhaben könne, will er nicht anerkannte Vernunftwahrheiten bekriegen. \par
\gliederungslinie\par
In dem so eben Gesagten sind theilweise schon die Gesichtspunkte bezeichnet, aus welchen auch die folgende Lehre \danf{von der Erlösung} (\RWpar{IIIb}{208--212}) betrachtet seyn will, wenn jedem Einwurfe, der aus historischen Gründen vorgebracht werden mag, alle Schärfe im Voraus benommen seyn soll. Allein der Mißverstand hat gegen diese Lehre auch Einwürfe anderer Art erhoben, er hat auch etwas in sich selbst Ungereimtes und der Vernunft Widersprechendes in ihr finden wollen. Lasset uns sehen, wie viel die bloße einfache Darstellung vermag, um jene Einwürfe zu entkräften, bevor man sich noch in irgend eine gelehrte Untersuchung einläßt. Wir lesen \RWpar{IIIb}{208} wie folgt: \par
\danf{Die Menschwerdung des Sohnes Gottes hat, wie uns das Christenthum versichert, eine Menge der beseligendsten Folgen für das gesammte menschliche Geschlecht hervorgebracht. Die vornehmsten derselben sind: \seitenw{192}\par
1) Der menschgewordene Sohn Gottes wurde der erste Lehrer und Gründer des vollkommensten religiösen Lehrbegriffes, dessen das menschliche Geschlecht nur immer empfänglich ist, der auch bis an das Ende der Zeiten fortdauern und einst der allgemeine Antheil aller Menschen werden soll. \par
2) Der Gottmensch hat uns an seinem eigenen Wandel auf Erden ein Muster der menschlichen Vollkommenheit gegeben, an dem wir abnehmen können, wie auch wir handeln sollen in den verschiedensten Verhältnissen des Lebens. \par
3) Er hat uns von unseren Sünden, den eigenen sowohl als auch der Erbsünde befreit, indem er die Strafen, die wir verdient hätten, an unserer Statt getragen. Er hat dieß freiwillig gethan, bloß aus Gehorsam gegen den Vater. Er hat gelitten und ist gestorben für das ganze menschliche Geschlecht, für die Guten sowohl als für die Bösen, für Jene sowohl, die früher gelebt, als auch für Jene, die später leben. Wir sollen uns hiebei bildlicher Weise vorstellen, daß an den Leiden des Menschen auch der Sohn Gottes gleichsam Theil genommen habe, und eben in diesem Bilde die Größe der Liebe Gottes gegen uns und die Abscheulichkeit der Sünde erkennen. \par
4) Der Gottmensch hat uns die durch den Sündenfall verlorne Anwartschaft auf den Himmel und Gottes Wohlgefallen von Neuem wieder erworben. \par
5) Er hört noch jetzt nicht auf, für uns zu wirken und uns wohlzuthun; er regiert die Schicksale der auf Erden von ihm gestifteten Kirche als das unsichtbare Oberhaupt derselben. \Usw } \par
Nur der dritte Punct in dieser Darstellung ist es, der unsern Lesern noch einige Schwierigkeiten verursachen dürfte, die sie jedoch -- so glauben wir ihnen versprechen zu können -- im Buche selbst zu ihrer völligen Befriedigung behoben sehen werden. Befreien wir uns nämlich nur erst von all den \seitenw{193} falschen Vorstellungen, welche Rechtslehrer, Philosophen und Theologen über diesen Gegenstand durch ihre unglücklichen Straftheorien verbreitet; halten wir uns bloß an dasjenige, was durch den bloßen gesunden Menschenverstand schon als entschieden anzusehen ist, und was sich nach den von dem Vf.\ aufgestellten Begriffen auf das Vollkommenste begründen läßt: so wird uns klar einleuchten, von welchem Umstande allein es abhange, ob Gott zur gänzlichen Vergebung unserer Sünden durchaus nichts Anderes als Eintritt der Besserung von uns verlangen, oder noch eine fernere Bedingung beisetzen dürfe; daß es nämlich nur darauf ankomme, ob das Eine oder das Andere der Tugend förderlicher sey. Es ist nun leicht zu begreifen, daß eine Einrichtung der Art, wie sie in dieser Lehre des Christenthums vorausgesetzt wird, der Tugend bei Weitem zuträglicher sey als eine unbedingte, nämlich nur an die Besserung allein gebundene Vergebung. Oder kann man es läugnen, daß durch diese Verfügung erstlich schon (unserm Herrn) Jesu selbst Gelegenheit gegeben wurde, sich zu der höchsten dem Menschen erreichlichen Stufe der Vollkommenheit emporzuschwingen, wenn er es übernahm, die Sünden des ganzen Geschlechtes zu sühnen? Können wir läugnen, daß auch sein Beispiel nun unendlich lehrreicher für uns wird? daß nur auf diese Art möglichst gesteuert wird jenem verderblichen Leichtsinne, der durch eine späte Besserung Alles, was er früher verbrochen hatte, so gut als ungeschehen zu machen wähnet? Können wir läugnen, daß jedes noch nicht ganz verhärtete Gemüth einem eben so mächtigen als zugleich edlen Antrieb zur Vollendung seiner sittlichen Besserung in dem Gedanken finden müsse, daß Jesu kostbares Blut doch nicht vergeblich für uns möge vergossen worden seyn \umA\ Bei diesen gesunden Begriffen die Sache angesehen, verschwinden von selbst die thörichten Einwürfe, daß diese Lehre Gott als ein blutdürstiges Wesen schildere, daß Schuld und Verdienst etwas Persönliches seyen, welches nicht übertragen werden könne \usw\ Indessen wird, wer eine erschöpfende Widerlegung verlangt, diese, wie schon gesagt, im Buche nicht vergeblich suchen. \seitenw{194}\par
Die zwei noch übrigen Lehren dieses Abschnitts: \danf{vom Einflusse des h. Geistes} (\RWpar{IIIb}{213--216}) und \danf{von der Gnade} (\RWpar{IIIb}{217--220}) werden den Lesern kaum einen Anstoß verursachen; denn wenn sie absehen davon, daß Gott, wiefern er die hier in Rede stehenden Wirkungen hervorbringt, \danf{göttlicher Geist} genannt wird, und von noch einigen andern bloß bildlichen Redensarten: was ist in dieser Lehre, das nicht auch die sich selbst überlassene Vernunft als wahr einsehen müßte? Oder sollen wir etwa nicht jede Stufe der Vollkommenheit, die wir ersteigen, nicht jede einzelne gute und gottgefällige That, die wir verrichten, der Gnade Gottes zuschreiben, und sein eigenes Werk in uns nennen? Oder wollen wir etwa bezweifeln, ob Gott auch einem jeden Menschen von seiner Gnade so viel, als für ihn hinreichend ist, mittheile? oder, daß er zuweilen uns auch Gnaden gibt, welche wir nicht benützen, sogar zum Bösen mißbrauchen? \usw\ \par
\gliederungslinie\par
Der fünfte Abschnitt: \danf{von den Verhältnissen der Geschöpfe untereinander} (\RWpar{IIIb}{221--225}) enthält eine einzige Lehre, die der Vf.\ \danf{die Lehre vom Wirkungskreise des Menschen} nennt. Wir wollen auch diese den Lesern unverkürzt vorlegen: \par
\danf{Das katholische Christenthum} (sagt der Vf.\ \RWpar{IIIb}{221}) \danf{bemüht sich, uns zu zeigen, daß unser Wirkungskreis überhaupt größer sey, und viel weiter reiche, als wir es uns insgemein vorzustellen pflegen, und als wir auch wohl, ohne die Aufschlüsse einer göttlichen Offenbarung darüber benützen zu können, zu glauben berechtiget wären. Es sagt uns, daß wir \par
1) Durch eine jede sittlich gute Handlung und Willensentschließung zur Beförderung der Tugend und Glückseligkeit des Ganzen mehr beitragen, als wir mit Deutlichkeit wahrnehmen können. Es sagt, wir sollten insbesondere \par
2) versichert seyn, daß nicht eine einzige in guter Absicht begonnene Unternehmung je ganz mißlinge, sondern, wenn wir auch den Erfolg, den wir zunächst \seitenw{195} beabsichtiget hatten, nicht eintreten sehen, wenn es sogar den Anschein nimmt, als ob nur Böses aus dem, was wir wohlmeinend anfingen, hervorgehe: so sey dieß immer nur scheinbar, und unsere Unternehmung führe, uns unbemerkt, eine Menge wohlthätiger Folgen herbei, welche die schlimmen, die uns in's Auge fallen, bei Weitem überwiegen. \par
3) Wenn unsere eigenen Kräfte zur Ausführung eines gewissen guten Zweckes nicht hinreichen: so ist schon der bloße lebhafte Wunsch, daß dieser Zweck erreicht werden möge, wenn wir ihn Gott vortragen, \dh\ , wenn wir gedenken, daß Gott von unserem Wunsche wisse, und gütig und mächtig genug sey, ihn zu berücksichtigen und zu erfüllen, mit Einem Worte: unser Gebet, nie ohne allen Erfolg; sondern Gott höret und erhöret unsere Bitte immer in der Art, daß er uns Eines von Beiden, entweder eben das, um was wir bitten, oder etwas noch Besseres gewähret. \par
4) Diese Erhörung von Gott dürfen wir hoffen, nicht nur, wenn wir für uns, sondern auch, wenn wir für das Wohl Anderer bitten; Gott nimmt auch Fürbitten an. \par
5) Wir können nicht bloß für Lebende, sondern auch für bereits Verstorbene mit Erfolg fürbitten. \par
6) Und wie wir dieses für sie, so und noch mehr vermögen auch die Verstorbenen, ja alle auf einer höheren Stufe des Daseyns stehenden Wesen auch für uns Lebende mit Erfolg fürzubitten. \par
7) Wir können diese Wesen um ihre Fürbitte auch mit Nutzen anrufen; und eine solche Anrufung derselben wird auch dann nicht vergeblich seyn, wenn jene Wesen sie wirklich nicht inne werden. \par
8) Alle unsere Gebete sind um so wirksamer, je sittlich besser wir sind. \par
9) Wir können die Wirksamkeit auch jedes einzelnen unsrer Gebete dadurch gar sehr erhöhen, daß wir damit \seitenw{196} gewisse gute Werke, ober auch bloße fromme Gelübde verknüpfen, \dh\ , daß wir verschiedene sittlich gute Handlungen aus dem Beweggrunde unternehmen, um die Erfüllung unserer Bitte von Gott desto sicherer zu erhalten, oder daß wir in eben dieser Absicht den festen Vorsatz fassen, für den Fall der Gewährung unserer Bitte dieses und jenes Gute zu thun, das wir wohl nicht vollzogen hätten, wenn wir es nicht als ein Mittel, uns die Erfüllung unserer Bitte zu sichern, angesehen hätten. \par
10) Die Gebete der Christen, wenn sie gehörig eingerichtet sind, haben sich einer immer noch vollständigeren Erhörung als jene der Nichtchristen zu erfreuen. Und Gebete von Christen für Christen angestellt, sind gleichfalls wirksamer als für Nichtchristen. \par
11) Jeden Glauben dagegen, als ob wir durch Ausübung einer gewissen, wie immer gearteten Handlung, die weder einen durch unsre bloße Vernunft erkennbaren, noch einen durch eine erweisliche göttliche Offenbarung uns angezeigten Nutzen hat, dennoch uns oder Anderen Hülfe verschaffen könnten, erklärt das Christenthum für einen eben so schädlichen als sträflichen Aberglauben.} \par
Hinsichtlich der beiden ersten Puncte, die wir in den bisherigen Handbüchern der Dogmatik vergeblich suchen würden, begnügt sich der Vf., in dem \danf{historischen Beweise} zu zeigen, daß sie sich als eine Folge aus anderen Lehren des Christenthums ergeben; und es wird wohl in der That Niemand in Abrede stellen, daß beide Sätze wahr sind, und es in hohem Grade verdienen, allgemein anerkannt zu werden. -Was das Christenthum gemeinschaftlich mit jeder besseren Religion über Erhörbarkeit unserer Bittgebete behauptet, wird \RWpar{IIIb}{223} so klar erwiesen, und die dagegen erhobenen Einwürfe werden so vollständig widerlegt, daß es wohl Jedermann einleuchtend werden dürfte, so und nicht anders könne es seyn. Denn eine Fürsehung, die bei der Leitung unserer Schicksale bald unsere sittliche Beschaffenheit, \seitenw{197} bald unsere Wünsche und Bedürfnisse ganz unbeachtet ließe, wäre wohl weder weise noch gütig zu nennen. Soll aber Alles beachtet werden, so ist kein Zweifel, daß unsre Bittgebete, vollends wenn sie durch gute Handlungen und Vorsätze unterstützt werden, nicht als ein Grund wider, sondern nur als ein Grund für die Erfüllung unserer Wünsche angesehen werden müssen; und so können wir denn mit voller Zuversicht erwarten, daß Gott unsre Bitten, wofern sie anders die gehörige Einrichtung haben, nie unerhört lasse. \par
Es ist gewiß nicht am unrechten Orte, wenn der Vf.\ am Schlusse seiner Lehre vom Wirkungskreise des Menschen auch der vergeblichen und sündhaften Weisen erwähnet, auf welche Menschen nur zu oft eine Erweiterung ihres Wirkungskreises angestrebt haben. Es dürfte wohl auch ganz richtig seyn, wenn er als solche vergebliche und sündhafte Weisen alles dasjenige bezeichnet, wo wir von unserm Thun einen Erfolg erwarten, zu dessen Erwartung wir weder durch bloße Vernunftgründe (bisher gemachte Erfahrungen \udgl\ ) noch durch Verheißungen der göttlichen Offenbarung selbst berechtiget sind. Dennoch ist es gefehlt, daß er dieß so geradezu als eine Lehre des Christenthums aufstellt, und ohne Weiteres voraussetzt, daß die Benennungen: Aberglaube und -- (denn mit diesem Worte wechselt er in der Folge) -- Zauberei nur eben diesen Fehler bezeichnen; während doch in der That das erste Wort eine weitere, das zweite aber eine engere Bedeutung hat. Denn Aberglaube heißt ja nicht bloß die grundlose Erwartung einer Wirkung, sondern auch noch gar manche andere grundlose Meinung; Zauberei aber nennen wir wohl nur das Bestreben, gewisse Wirkungen hervorzubringen durch Hülfe übermenschlicher, böse gesinnter Geister. \par
\gliederungslinie\par
Der letzte Abschnitt der christkatholischen Dogmatik: \danf{die Lehre von den Belohnungen und Strafen in diesem und in jenem Leben} (\RWpar{IIIb}{226--234}) zerfällt in \danf{allgemeine Lehrsätze,} und \danf{Lehrsätze von den Belohnungen und Strafen im anderen Leben.} Die wichtigsten unter den ersteren sind: \seitenw{198}\par
\danf{1) Jede sittlich gute Handlung, die wir verrichten, bietet Gott einen Grund dar, uns eine gewisse Glückseligkeit als Belohnung zufließen zu lassen; jede sittlich böse einen Grund, uns eine gewisse Unglückseligkeit als Strafe zuzugedenken. Diese Gründe sind es, die wir Verdienst und Schuld nennen. Und es ist wohl zu merken, daß in manchen Fällen gesagt werden dürfe, die Schuld sey von uns hinweggenommen, während doch eine Strafe (eine endliche) noch zurückbleibt. \par
2) Unter den sittlich guten Handlungen gibt es solche, deren Ausübung eine Belohnung erwirbt, ohne daß ihre Unterlassung mit einer eigentlichen Strafe verbunden wäre, es sey denn, daß man das bloße Ausbleiben der Belohnung schon eine Strafe nennen wollte. Man pflegt dergleichen Handlungen verdienstliche, die übrigen strenge Pflichten zu nennen. \par
3) Unter den sittlich bösen Handlungen gibt es nicht nur solche, die eine bloß zeitliche Strafe nach sich ziehen (läßliche Sünden), sondern auch einige, die eine ewige Strafe verschulden (Todsünden). \par
4) Alle ersprießlichen Folgen, welche ein Mensch von seiner Handlung erwartet, und um derentwillen er sich zu ihrer Unternehmung auch entschließet, erhöhen die Verdienstlichkeit derselben, gleichviel ob sie in Wirklichkeit erfolgen oder nicht. Alle gemeinschädliche, die er erwartete, und die ihn gleichwohl nicht abhalten konnten, vermehren seine Schuld, sie mögen erfolgen oder nicht. \par
5) Wer eine gute Handlung aus sittlichem Grunde unternimmt, dem sollen selbst jene ersprießlichen Folgen, welche er nie vorhergesehen hatte, in einem gewissen, obgleich geringerem Grade zum Verdienste angerechnet werden. Im Gegentheile, wer Böses thut, dem sollen selbst jene schlimmen Folgen, die er nicht vorhergesehen, ja nicht einmal vorhersehen konnte, in einem gewissen, obgleich geringeren Grade zur Schuld angerechnet werden. \seitenw{199} \par
6) So sollen uns insbesondere auch alle fremden Sünden, \dh\  Sünden eines Anderen, die wir durch irgend eine eigene böse Handlung veranlasset haben, zur Schuld angerechnet werden. \par
7) Für das Schlimme dagegen, welches aus einer in guter Absicht und mit gehöriger Vorsicht von uns verrichteter Handlung ganz wider unseren Willen hervorgehet, haben wir nichts zu verantworten. Eben so dürfen wir aber auch umgekehrt von jenem Guten, das durch die weise Leitung Gottes aus unsern bösen Handlungen entspringt, keine Verminderung unserer Schuld erwarten; obgleich es, wenn wir die böse Absicht bereuet und uns gebessert haben, uns zu einigem Troste gereichen darf. \par
8) Wer tugendhaft lebt, \dh\  die herrschende Gesinnung hat, dem Sittengesetze in allen Stücken nachzukommen, der darf, auch wenn es ihm noch nicht gelingt, jede Gebrechlichkeits- oder läßliche Sünde zu meiden, doch eine überwiegende Belohnung seiner Tugend hoffen. Wer dagegen in einer lasterhaften Gesinnung lebt, \dh\  den herrschenden Willen hat, das Sittengesetz in gewissen Stücken zu übertreten, mag sich mit allen den übrigen guten Handlungen, die er daneben ausübt, nicht im Geringsten trösten. Es ist so weit entfernt, daß durch ihr Verdienst die Schuld seiner bösen Thaten könnte ausgewogen werden, daß sie im Gegentheile nicht einmal zu einem wahren Verdienste können angerechnet werden, weil sie nicht wirklich, sondern nur scheinbar gut sind. Dennoch geschieht es durch Gottes Barmherzigkeit, daß solche Handlungen die Strafe des Sünders zuweilen vermindern, ja wohl gar die Veranlassung zu seiner endlichen Besserung werden. \par
9) Wer früher tugendhaft gelebt, dann aber in eine herrschende sittlich böse Gesinnung verfällt: dem werden von nun an selbst jene wirklich guten Handlungen, die er in seinem noch tugendhaften Zustande ausgeübt hatte, der Strafe nicht entreißen, sondern sie sind nun alle gleichsam erloschen. Doch nicht auf immer und unwiederbringlich; \seitenw{200} sondern wofern er sich einst wieder bessert, leben auch jene guten Handlungen zwar wieder auf, doch nicht mehr völlig in ihrem vorigen Glanze. \par
10) Wer in einer tugendhaften Gemüthsverfassung lebt, darf jedes ihm zu Theil werdende Glück als einen Beweis des göttlichen Wohlgefallens, als eine Belohnung ansehen; jedes Unglück aber betrachten als ein Leiden, das ihm Gott nur zu seiner Prüfung und Vervollkommnung zuschickt, und das, wenn er es anders mit geduldiger Ergebung trägt, und zu seiner sittlichen Vervollkommnung benützt, sich nur in Glück und Heil für ihn auflösen wird. Der Lasterhafte dagegen, ja jeder, der eine vorsätzliche Sünde begangen, muß jedes Glück, das ihm zu Theil wird, als eine unverdiente Aufforderung Gottes zu seiner Besserung, jedes Unglück aber als einen Anfang der von Gott über ihn verhängten Strafen betrachten.} \par
So lehrt das Christenthum wirklich, wenn es gleich vielleicht in keiner Dogmatik bisher gerade so zusammengestellt worden ist; und so und nicht anders muß eine jede Religion lehren, wenn sie den Lobspruch, daß ihre (dießfällige) Lehre die größte sittliche Zuträglichkeit für uns besitze, verdienen will. Den Unterschied zwischen Handlungen, die bestimmte Pflicht und Schuldigkeit, und solchen, die bloß verdienstlich heißen, hat man zuweilen aus bloßem Mißverstande bestritten; der gemeine Menschenverstand setzet ihn überall voraus, und man muß in der That nicht wissen, was man sagt, wenn man von Gott verlangt, er solle alles dasjenige, was er nicht mit erhöheter Glückseligkeit belohnen kann, durch positive Übel und Leiden bestrafen. \par
Doch, das ist, wie gesagt. Mißverstand, dessen sich nur Gelehrte schuldig gemacht; allein der Lehrsatz, daß es auch ewige Strafen gebe, und die bekannten katholischen Lehren, die der Vf.\ \RWpar{IIIb}{231} ff. betrachtet, daß nur ein gänzlich Schuldloser gleich nach dem Tode in einen Zustand ganz reiner ungetrübter Seligkeit eingehen könne, daß Andere erst durch mehr oder minder empfindliche, länger \seitenw{201} oder kürzer andauernde Leiden geläutert werden müssen, daß endlich die Zahl der Auserwählten sogar als eine vergleichungsweise sehr kleine Zahl zu denken sey: dieses sind Lehren, welche nicht nur ein jeder lasterhaft Gesinnte aus sehr begreiflichem Grunde unerträglich findet; sondern auch selbst der Tugendhafte, der vielleicht eben nicht für seine eigene Person die Hölle fürchtet, wird doch betrübt durch den Gedanken, daß es auch nur ein einziges Wesen in Gottes Schöpfung gibt, welchem das Loos einer ewigen Verdammniß bevorstehen sollte. Allein man sage, was man will, bei diesen Lehren des Katholicismus muß es sein Verbleiben haben; nichts darf daran geändert werden: und dennoch hoffen wir es nicht nur von unseren Lesern, sondern von allen vernünftigen Menschen, daß sie die Richtigkeit und die Nothwendigkeit derselben bald einmüthig anerkennen werden. Denn bedarf es mehr als der Erwägung eines einzigen Gedankens, um diese Lehren auf das Vollkommenste gerechtfertiget zu sehen? bedarf es mehr als des Gedankens, daß nicht derjenige irrt, der das Heil seiner Seele durch stete sittliche Vervollkommnung derselben mit allem dem Eifer betreibt, den jene Lehren ihm einflößen können, sondern daß sich nur derjenige zu seinem eigenen größten und unwiederbringlichen Nachtheile täuschet, der -- weil er nicht glaubt, lässiger wird? Mag alles Übrige, was B.\ in seinem Beweise der Vernunftmäßigkeit dieser Artikel (\RWpar{IIIb}{229} u. \RWpar[233.]{IIIb}{233}) sagt, dem Zweifel unterliegen: können unsere Leser nur dieses Eine, was so hell wie das Tageslicht ist, nicht in Abrede stellen, so werden sie auch nicht in Abrede stellen, daß jene Lehren mindestens eine bildliche Wahrheit haben, und in dieser fortwährend fest gehalten zu werden verdienen. \par
Daß in dem Artikel \danf{vom jüngsten Weltgerichte} (\RWpar{IIIb}{231} Nr. 4.) bildliche Vorstellungen enthalten sind, wem müßte das nicht von selbst einleuchten? Allein ihr fraget, wie weit dieß Bildliche reiche? Wir erwiedern schnell: Ihr könnet es so weit ausdehnen, als ss euch nöthig scheint, und immer werdet ihr finden, daß diese Vorstellungen auch nur als bloße Bilder betrachtet, im höchsten Grade zweckmäßig \seitenw{202} sind, weil sie nicht besser eingerichtet werden könnten, um uns mit einer heiligen Scheu vor jeder auch der geheimsten Sünde, um uns mit Trost und Freudigkeit im Unglück jeder Art, besonders wenn uns die Welt verkennt, und mit gar manchen anderen heilsamen Empfindungen, die \RWpar{IIIb}{234} Nr.\,4. aufgezählt sind, zu erfüllen. \par
Was endlich in der letzten Nummer des \RWpar{IIIb}{231} \danf{über das Schicksal der Nichtchristen} gesagt wird, findet in den Bemerkungen des \RWpar{IIIb}{233} Nr.\,10. eine so vollständige Rechtfertigung, daß wir nicht glauben, es werde irgend einer unserer Leser sich genöthiget fühlen, bei diesem Puncte zu einer bloß bildlichen Auslegung seine Zuflucht zu nehmen. B.\ sagt: \par
\danf{Wenn die Verwerfung des christlichen Glaubens aus einer nicht unüberwindlichen Unwissenheit herrührt, sondern wenn Leidenschaft, Trägheit \udgl\  die Ursache derselben sind: so ist sie offenbar sehr sträflich, ist es um so mehr, je deutlicher das Bewußtseyn ist, daß man der Erkenntniß der Wahrheit widerstrebe. -- Daß aber eine Verwerfung des Glaubens aus unüberwindlicher Unwissenheit Niemand zur Schuld und Sünde angerechnet werden könne, ist eine nothwendige Vernunftwahrheit. -- Allein daß es auch auf der anderen Seite nicht möglich sey, die höheren Vortheile, die Jesus Christus dem menschlichen Geschlechte ausgewirkt hat, in ihrem ganzen Umfange zu genießen, wenn man ihn nicht erst kennen gelernt, und seine Lehre gläubig angenommen, laßt sich aus mehr als Einem Grunde begreifen: \par
$\alpha$) Das katholische Christenthum macht uns mit einer Menge sehr wichtiger Begriffe vom wahren Wesen der menschlichen Tugend, mit einer Menge neuer und kräftiger Mittel und Aufmunterungsgründe zur Tugend, mit einer Menge höchst tröstlicher und erfreulicher Aufschlüsse über die Zukunft, mit einer Menge der heilsamsten Wahrheiten bekannt: wer also dieses katholische Christenthum auch ohne sein Verschulden nicht kennen lernt, der muß doch aller der Vortheile, die eine bloße Wirkung dieser Erkenntnisse sind, entbehren; er schreitet langsamer fort in seiner sittlichen \seitenw{203} Vervollkommnung, und eben darum auch in seiner Glückseligkeit. \par
$\beta$) Was andere Vortheile betrifft, zu deren Genuß die Lehre Jesu vielleicht nicht eben so unumgänglich nothwendig wäre, wie das Vorhandenseyn der Ursache zur Entstehung der Wirkung: so scheint es aus einem anderen Grunde nicht schicklich zu seyn, daß sie denjenigen, die Jesum gar nicht kennen, ganz so wie seinen Kennern und Verehrern mitgetheilt würden. Vernünftige Wesen nämlich sollen nie einer Wohlthat genießen, ohne den Urheber derselben wenigstens einigermaßen zu kennen. Denn durch die Kenntniß desselben wird der Genuß erquickender und heilsamer für sie. Sie müssen sich eines Glückes, das sie nicht blindem Zufalle, sondern der wohlwollenden Gesinnung eines Andern zu danken haben, lebhafter freuen; erhalten Gelegenheit, die Pflicht der Dankbarkeit gegen denselben zu üben; fühlen endlich eine stärkere Verbindlichkeit, ein solches Glück ganz nach dem Willen seines Urhebers zu benützen. \par
$\gamma$) Nehmen wir endlich an, daß das andere Leben ein Leben geselliger Thätigkeit ist, und daß die Seligkeit desselben vornehmlich von der zweckmäßigen Verfassung, welche das Oberhaupt dieser Gesellschaft, der Gottmensch selbst, ihr gegeben, und von der genauen Beobachtung all seiner Vorschriften abhängt: so ist es ganz begreiflich, daß Niemand in den Himmel der Christen eingehen könne, der Jesum nicht erst als das rechtmäßige Oberhaupt in diesem Staate erkannt hat. \par
Anm. Von der anderen Seite gestattet eben diese Annahme die Hoffnung, daß, wenn wir uns dort in Gesellschaft mit anderen Wesen befinden, und auch dort noch an Kenntnissen und sittlicher Vollkommenheit immer zunehmen, früher oder später auch selbst denjenigen Menschen, die Jesum ohne ihr Verschulden auf Erden nicht kennen gelernt haben, dort Gelegenheit werde, mit ihm bekannt zu werden. Machen sie nun von dieser Bekanntschaft mit Jesu ganz den gehörigen Gebrauch, so läßt sich hoffen, daß Gott sie zuletzt auch eben derselben höheren Seligkeit theilhaftig machen wird, \seitenw{204} welche für uns bereitet ist, die wir des Glückes, ihn schon auf Erden zu kennen, genießen.} \par
\gliederungslinie\par
Das \RWHSfmt{dritte und letzte Hauptstück}\RWHS{Drittes Hauptstück}{255--300} des Werkes liefert unter der Überschrift: \danf{\RWHSfmt{christkatholische Moral}} (\RWpar{IIIb}{255--300}) nicht, wie es sollte, einen wenn auch gedrängten doch vollständigen Abriß dieser Moral in systematischem Zusammenhange, sondern beinahe nichts Anderes als einige dem Christenthum und insbesondere dem Katholicismus eigenthümliche, größtentheils in die Asketik gehörige Lehren. Daß es dem Vf.\ in seinem Verhältnisse als Lehrer nicht möglich gewesen, ein Mehreres zu thun, daß er den Nachtheil, den die Übergehung so vieler Lehren aus der Moral hatte, auf eine andere Weise, in seinen Erbauungsreden nämlich, theilweise zu ersetzen gesucht: das Alles mag wahr seyn, ein Mangel des vorliegenden Werkes, als eines Lehrbuches der Religionswissenschaft betrachtet, bleibt es doch immer, daß man darin eine systematische Übersicht der Moral vermisset. Indessen ist dieser Mangel keineswegs von der Art, daß er den hier zu liefernden Beweis für die Wahrheit des Christenthums, und namentlich des Katholicismus entkräftete. Denn alle hier übergangenen Lehren sind Lehren, von denen ausdrücklich eingestanden wird, daß sie mindestens nach dem, was in der Dogmatik bereits gelehrt worden ist, aus bloßen Gründen der Vernunft eingesehen werden können, und auch nur so ausgelegt werden dürfen, wie sie sich mit der Vernunft vereinigen lassen. Sind also nur alle Lehren, welche B.\ in sein Buch aufgenommen hat, mit der Vernunft vereinbar und sittlich zuträglich, so ist schon erwiesen, daß dieß von allen Lehren des Katholicismus gelte. \par
Der Vf.\ hebt nun in der ersten Abtheilung, welche als christkatholische Ethik (\RWpar{IIIb}{236--271}) überschrieben ist, zuerst die Lehre vom Daseyn, (\RWpar{IIIb}{236--238}) dann die von dem Gebiete des Sittengesetzes (\RWpar{IIIb}{239--243}) hervor, bespricht hierauf sehr kurz den im katholischen Christenthume stets fest gehaltenen Unterschied zwischen Geboten und \seitenw{205} Räthen der schon durch den in der Dogmatik aufgestellten Unterschied zwischen Schuldigkeit und Verdienst gerechtfertiget ist; wornach er unter einer ihm eigenthümlichen Bezeichnung: \danf{allgemeinste Sittengesetze,} (\RWpar{IIIb}{249--253}) mehrere im Christenthume übliche Formeln aufführt, die so beschaffen sind, daß sich aus einer jeden, wenn auch nicht alle, doch fast alle Pflichten des Menschen auf irgend eine Weise ableiten lassen. Er gibt als solche an die Sätze: \par
\danf{1) Handle immer so, wie es das allgemeine Beste (das Wohl des Ganzen) fordert. \par
2) Folge in Allem dem Willen Gottes. \par
3) Suche in Allem nur die Beförderung der Ehre Gottes. \par
4) Handle immer so, daß du Gott ähnlicher werdest. \par
5) Liebe Gott über Alles. \par
6) Ahme dem Beispiele Jesu nach. \par
7) Liebe den Nächsten, wie dich selbst. \par
8) Behandle Jeden so, wie du wünschtest, daß er auch dich behandle.} \par
Welche besondere Ansichten von der Moral und ihrem obersten Gesetze Jemand auch haben mag, mit dem, was der Vf.\ hier sagt, wird er sich einverstehen können. Weil nämlich ganz dahingestellt bleibt, ob irgend eine der hier aufgeführten Formeln das wahre oberste Gesetz sey oder nicht: so kann derjenige, der sein Princip hier antrifft, über den Ort, den man demselben angewiesen hat, nicht klagen. Derjenige aber, der es nicht antrifft, wird ohne Zweifel selbst nicht in Abrede stellen, daß es ein Satz sey, der sich bisher noch keiner allgemeinen Anerkennung zu erfreuen habe. -- Daß jedoch alle diese Sätze das Eigene haben, daß sich auf jeden Fall ein großer Theil unserer Pflichten aus ihnen ableiten läßt, dem wird kaum Jemand widersprechen. \par
Auch das \RWpar{IIIb}{254--256} behauptete Daseyn geoffenbarter Pflichten werden unsere Leser gewiß nicht anstößig finden, wenn sie erwägen, mit welcher stillschweigend immer hinzugedachten, von unserm Vf.\ aber zuerst ganz deutlich ausgesprochenen Beschränkung der Satz verstanden werde, \seitenw{206} daß nämlich eine Handlungsweise nur dann durch Offenbarung zur Pflicht erhoben werden könne, wenn sie uns auch schon durch unsre bloße Vernunft als der Sittlichkeit zuträglich erscheinet. \par
Die Lehre \danf{vom Rechte,} die der Vf.\ \RWpar{IIIb}{257--260} aufstellt, bestehet fast nur aus den wenigen Sätzen: \par
\danf{1) Es gibt rechtliche sowohl als widerrechtliche Handlungen. \par
2) Alle sittlich guten Handlungen, mindestens alle, die es nicht bloß subjectiv, sondern auch objectiv sind, sollen wir eben deßhalb auch zu den rechtlichen zählen, allein nicht umgekehrt jede rechtliche Handlung so fort für eine auch sittlich erlaubte ansehen. \par
3) Es gibt auch rechtlich erzwingbare Handlungen; und nicht immer sind sie, wenn derjenige, dem das Recht, sie zu erzwingen, zustehet, keinen Gebrauch davon macht, auch Pflichten. Eine Pflicht, die nicht erzwingbar, also frei ist, ist darum nicht immer eine geringere als die erzwingbare.} \par
Gegen diese wenigen Sätze wird nun wohl Niemand etwas einzuwenden haben, welcher Rechtstheorie er auch zugethan seyn mag. Indessen wollen wir die von dem Verf. angenommene Erklärung des Rechtsbegriffs bloß zu dem Ende beifügen, damit unsere Leser, wenn es ihnen beliebt, versuchen könnten, mit welcher Leichtigkeit sich nicht nur die obigen, sondern auch noch viele andere Lehren der Rechtswissenschaft aus dieser Erklärung ableiten lassen. B.\ sagt nämlich, eine Handlung sey rechtlich, wenn wir verpflichtet sind, sie zu dulden, \dh\  durch keine Anwendung eines Zwanges sie zu verhindern, so gut wir es etwa auch vermöchten.\footnote{% 
Bei dieser Gelegenheit erlauben wir uns zu bemerken, daß man weitere Ausführungen dieses Begriffs auch in den \danf{Ansichten eines freisinnigen Theologen über Kirche und Staat} (Sulzbach 1834.), und in den \danf{freimüthigen \seitenw{207} Blättern von Pflug} (Stuttgart 1838. Bd. 14. u. 15. in dem Aufsatze: \danf{über das Recht der Geistlichkeit, ihren Lebensunterhalt von Personen zu beziehen, die nicht ihres Glaubens sind;}) antrifft.} -- \par
\seitenw{207} \par 
Auch in der Lehre \danf{von den Obrigkeiten} (\RWpar{IIIb}{261--264}) stellt der Vf.\ nur folgende wenige Sätze auf: \par
\danf{1) Es gibt auch rechtmäßige, \dh\  solche Obrigkeiten, die man in ihrer Macht nicht stören soll, selbst wenn man es vermögte. \par
2) Es geschieht nie ohne die Leitung einer besondern göttlichen Fürsorge, daß gewisse Personen zu dem Besitze einer rechtmäßigen Obergewalt über Andere gelangen. \par
3) Ihre Gebote, so ferne sie anders nicht etwas ganz offenbar Böses betreffen, verpflichten die Untergebenen zum Gehorsam, und dieß nicht bloß wegen der Gefahr der Strafe, sondern auch selbst in dem Falle, wo keine Strafe zu fürchten ist. \par
4) Wie aber Untergebene die Pflicht des Gehorsams, so haben Vorgesetzte die Pflicht, nur Gutes und wahrhaft Gemeinnütziges zu gebieten, und werden Gott für den Gebrauch ihrer Macht Rechenschaft ablegen.} \par
Indem wir nicht besorgen, daß irgend Jemand mit diesen Puncten nicht einverstanden seyn werde, erinnern wir nur, daß der Vf.\ den dritten Artikel \RWpar{IIIb}{263} nach seinen Ansichten näher zu bestimmen versucht. Auch etwas, so an sich selbst dem Wohle des Ganzen nicht zuträglich ist, und in so fern böse genannt werden mag, kann durch Befehl einer rechtmäßigen Obrigkeit Pflicht werden; wie der Vf.\ durch mehrere Fälle beweist. Nur dann, behauptet er, erleide die Pflicht des Gehorsams eine Ausnahme, wenn wir durch Gehorchen des Übels offenbar mehr anrichten würden als durch das Ärgerniß des Ungehorsams. \par
Aus der Lehre \danf{von den Versprechungen und Gelübden} (\RWpar{IIIb}{265--267}) führen wir nur die beiden Behaup\seitenw{208}tungen an, daß die aus Versprechungen oder Vorträgen hervorgehende Verpflichtung aufhöre, wenn es etwas schlechterdings Böses betraf, wovon auch derjenige, der das Versprechen uns abnahm, wissen mußte, daß es unerlaubt sey; und daß die Gott gethanen Gelübde aufhören, sobald wir statt des Gelobten etwas Anderes thun, davon wir deutlich einsehen, daß es weit besser ist; obgleich wir in solchen Fällen, nicht unserm eigenen Urtheile allein vertrauend, die Sache, wo möglich, der Entscheidung eines vernünftigen Gewissensrathes anheimstellen sollen. \par
Die zwei noch übrigen Lehren dieser Abtheilung: \danf{von der besonderen Art, wie wir Menschen das Sittengesetz erkennen, und den Verbindlichkeiten, die wir in dieser Rücksicht haben,} (\RWpar{IIIb}{268--269}), und \danf{von der verschiedenen Gut- und Bösartigkeit des Menschen} (\RWpar{IIIb}{270} \RWpar[271.]{IIIb}{271}) müssen wir ganz übergehen, damit wir Raum behalten für mehrere dem Katholicismus eigenthümliche Lehren, welche die zweite Abtheilung aus der \danf{christlichen Asketik} (\RWpar{IIIb}{272--300}) aufstellt. \par
Unter der Überschrift: \danf{die allgemeinsten Lehren von dem Gebrauche der Tugendmittel,} (\RWpar{IIIb}{272} \RWpar[273.]{IIIb}{273}) wird nur behauptet und durch Beziehung auf das in der natürlichen Asketik schon Erwiesene dargethan, daß es Tugendmittel überhaupt gebe, daß ihre Anwendung Pflicht sey, daß dieses insbesondere auch von dem Gebrauche der so genannten Beweggründe zur Tugend gelte, und daß der zweckmäßigste aus allen Antrieben zur Tugend derjenige sey, der von dem Willen Gottes entlehnt wird. Für diese Behauptungen entscheidet zwar schon der bloße gesunde Menschenverstand; indessen hoffen wir, daß selbst diejenigen, die etwa noch der Kantsche Purismus beirrt, durch des Vfs. deutliche Auseinandersetzung einsehen lernen werden, wie unhaltbar und in sich selbst widersprechend eine Theorie sey, die einerseits unbedingt und ohne irgend einen Grund nachweisen zu können fordert, daß Tugend und Glückseligkeit in Harmonie treten sollen, daraus selbst Daseyn Gottes und \seitenw{209} Unsterblichkeit folgert, und andrerseits doch das Einzige, wozu dieß gut wäre, nämlich uns durch den Hinblick auf diese Einrichtung in unserm Streben nach Tugend zu ermuntern, als etwas Unerlaubtes uns verbietet. --\par
Der Vf.\ unterscheidet hierauf zweierlei Arten von Tugendmitteln: natürliche nämlich und übernatürliche, die man auch Sacramente oder Heiligungsmittel (gleichsam kat exochen) nennet. Was aber den Unterschied zwischen natürlichen und übernatürlichen Wirkungen anbelangt: so haben die Leser nicht nöthig, die von B.\ gegebene Erklärung (welche er eben deßhalb nicht in den Vortrag der Lehre selbst hätte verweben sollen), als richtig anzunehmen; obgleich wir vermeinen, daß seine Erklärung genau das angebe, was man bei jener Unterscheidung als das allein Wahre und Wesentliche im Sinne gehabt hat; nämlich daß die Sacramente bei einem gehörigen Gebrauche uns Gnaden mittheilen, welche so groß und außerordentlich sind, daß wir nie berechtiget wären, ihr Daseyn vorauszusetzen, wenn uns die Offenbarung selbst es nicht versicherte, während die übrigen Tugendmittel, \zB\ das Gebet, nur solche wohlthätige Wirkungen haben (\zB\ Erhörung), deren Vorhandenseyn wir schon durch unsre bloße Vernunft erkennen. Es kann, wie uns däucht, nichts leichter seyn, als nach dieser Erklärung die Vernunftmäßigkeit der katholischen Lehre von den Sacramenten zu zeigen. Denn da jedes derselben Handlungen vorschreibt, deren gehörige Vollziehung den ersprießlichsten Einfluß auf unsre Sittlichkeit ausübt: was soll darin Unbegreifliches liegen, daß Gott den würdigen Gebrauch derselben auch noch durch andere größere Gnaden und Segnungen, als wir zu erwarten berechtiget wären, zu lohnen verspricht? Genau erwogen wird ja der einzige schon zur Rechtfertigung all dieser Heiligungsmittel hinreichende Gedanke, \danf{daß wir nie einen sittlichen Schaden davon haben, nie es in irgend einer Rücksicht bedauern können, wenn wir jene heiligen Handlungen mit all der Andacht und all dem Vertrauen verrichten, welche die Vorstellung, daß Gott sie auf eine so außerordentliche Weise zu segnen versprochen hat, nur immer eingeben kann.} \seitenw{210} \par
Indem wir nun durchgehen wollen, zuerst was uns B.\ über die so genannten natürlichen Tugendmittel im Einzelnen (\RWpar{IIIb}{274--280}) vorträgt: so müssen wir erinnern, daß Alles, was die katholische Kirche hier lehrt, nach der Ansicht ihrer eigenen Theologen auch aus bloßer Vernunft erweislich seyn soll. Wenn also irgend eine Behauptung, welche in diesen §§. als eine Lehre der ganzen Kirche dargestellt wird, eine doppelte Auslegung zuließe; eine, nach welcher sie eine reine Vernunftwahrheit ist; und eine andere, nach der sich ihre Wahrheit nicht aus der bloßen Vernunft erweisen ließe: so würde die Billigkeit fordern, sie auf die erste Art zu verstehen. Was aber nicht einmal Lehre der Kirche, nämlich der allgemeinen ist, (und es ist Vieles, was B.\ hier untermengt vorträgt, nicht eine solche im strengsten Sinne des Wortes, sondern bloß eine in der Kirche herrschende Lehre; und es ist Anderes vollends nur seine eigene Meinung; wie insgemein, was er uns als den idealen Zweck, der diesen und jenen religiösen Übungen zu Grunde gelegt werden sollte, angibt): -- bei allem diesem verstehet es sich von selbst, man könne es verwerfen, und den Beweis für die Wahrheit des Katholicismus, den er in diesem Buche liefert, doch immer bündig finden. Es wird hier 1) vom Gebete; 2) von der öffentlichen Gottesverehrung; 3) von den Übungen in der Kunst der Selbstbeherrschung; 4) von den Übungen der Wohlthätigkeit gegen Arme und Leidende; 5) von der Verehrung und Nachahmung verklärter Tugendfreunde; 6) von der Benützung der schönen Künste zu sittlichen Zwecken; 7) von der zweckmäßigen Bearbeitung der eigenen Neigungen gesprochen. Wir können nur Ein und das Andere ausheben. \par
Nachdem \RWpar{IIIb}{274} von den verschiedenen Arten des Gebetes, von dem Gebrauche der Gebetsformeln \ua\  m. gesprochen worden ist, liest man Nr. 6.: \danf{Weil es jedoch bekannt ist, daß oft die nützlichste Sache unterbleibt, wenn ihr nicht eine bestimmte Zeit angewiesen wird: so ertheilt die katholische Kirche nicht nur im Allgemeinen den Rath, öfters, ja ohne Unterlaß zu beten (1 Thess. 5,17); \seitenw{211} sondern es werden uns auch bestimmte, überaus schickliche Zeiten zum Gebete vorgeschlagen; wir werden namentlich ermahnt, den Morgen und Abend eines jeden Tages, die Mittagszeit, besonders aber die Anfangszeit eines jeden wichtigen Geschäftes durch Gebet einzuweihen. Zu diesem Ende werden wir an bestimmten Stunden des Tages selbst durch den Ruf einer Glocke erinnert, einen kleinen Stillstand in unsern Geschäften zu machen, um unsern Geist vereinigt mit Tausenden, die es in eben diesem Augenblicke thun, zu unserm gemeinschaftlichen Schöpfer zu erheben.} \par
Als Zweck der öffentlichen Gottesverehrung gibt der Vf.\ \RWpar{IIIb}{275} im Widerspruche mit Andern, welche ihm diesen Zweck zu einseitig auffassen, an -- \danf{die Summe alles desjenigen Guten, was durch sie überhaupt erreichbar; also \zB\ Versinnlichung der Größe und Majestät Gottes, vor dem wir hier Alle, auch selbst die Vornehmsten aus uns, im Staube hingegossen sehen; Belebung aller religiösen Gefühle durch die Wahrnehmung, daß wir sie nicht allein, sondern so viele Andere mit uns gemeinschaftlich empfinden; sinnbildliche Darstellung der wesentlichen Gleichheit aller Menschen; brüderliche Annäherung derselben an einander \usw } -- Indem er von der Ausschmückung der Tempel spricht, äußert er den Gedanken: \danf{ob nicht Alles, was kostbar und prächtig ist, nur in den Tempeln (und andern öffentlichen Gebäuden) aufgestellt werden sollte, damit es, statt die Eitelkeit Einzelner zu nähren, nur zur Verherrlichung Gottes diene?} -- \danf{Man muß,} heißt es gleich weiter, \danf{die Lehre der Kirche absichtlich mißdeuten, wenn man ihre Verfügungen so auslegt, als ob nach ihrer Ansicht der Mensch der Geistlichen bedürfe, als gewisser schlechterdings nothwendiger Vermittler zwischen ihm und Gott; als ob er in seinen Gebeten sich durchaus nicht unmittelbar an Gott selbst wenden dürfte \usw\ Nur der guten Ordnung wegen, nur um dem öffentlichen Gottesdienste mehr Vollkommenheit zu geben, läßt ihn die Kirche durch ihre Geistlichen, als Personen, die hiezu eigends unterrichtet \seitenw{212} sind, besorgen.} -- Die öffentlichen Umgänge nimmt der Vf.\ in Schutz; und auch die Wallfahrten, meint er, verdienten bei einer gehörigen Einrichtung immer beibehalten zu werden. \danf{Denn Reisen,} (sagt er) \danf{bald kürzere, bald wieder längere, sind doch beinahe für einen jeden Menschen ersprießlich, und können bald zur Befestigung seiner Gesundheit, bald zur Erweiterung seiner Begriffe, bald zu noch an dern Zwecken dienen. Durch die Erhebung zu einer Wallfahrt, \dh\  durch die Verbindung mit einem religiösen Zwecke, mit dem festen Vorsatze, jeden zu seiner Erbauung dienenden Anlaß auf dieser Reise bestens benützen zu wollen, werden sie nur um so ersprießlicher. Und werden solche Reisen in guter Gesellschaft und unter gehöriger Aufsicht unternommen: so fallen viele Gefahren, denen Leib und Seele auf Reisen ausgesetzt seyn können, bei solchen Wallfahrten hinweg.} \par
\danf{Obgleich die Vorsteher der katholischen Kirche,} heißt es in \RWpar{IIIb}{276}, \danf{recht gut wußten, daß es der Begierden, an welchen wir die Kunst der Selbstbeherrschung zu üben Gelegenheit haben, gar mancherlei gebe; obgleich sie es auch nicht unterließen, uns auf verschiedene derselben in dieser Beziehung noch eigends aufmerksam zu machen (namentlich auf die Begierden, etwas zu hören, zu sehen, zu sagen \usw ): so erkannten sie doch, daß sich fast keine andere zum Gegenstande einer allgemein vorzuschreibenden Übung eigne, als die -- Eßbegierde. In der Beherrschung dieser sich zu üben, tritt die Gelegenheit für Jeden täglich ein, und zugleich wird durch diese Übung eine beträchtliche Menge von Nahrungsstoffen erspart, die wieder Andern zum Genusse dienen können; \dh\  es ist dieß eine Übung, die auch schon an sich selbst einen Vortheil für das Ganze gewähret.} \usw\ \par
Im \RWpar{IIIb}{278} liest man, daß es nach den Grundsätzen der katholischen Kirche als ein Verlust für die Menschheit betrachtet werden soll, wenn irgend ein Beispiel der Tugend in der Verborgenheit bleibt; es würden daher alle diejenigen, die einen Tugendhaften kennen, aufgefordert, sein Beispiel zur allgemeineren Kunde zu bringen; \seitenw{213} doch erst nach seinem Tode; den Vorstehern der Kirche liege es ob, eine möglichst getreue und glaubwürdige Geschichte seines Lebens zu Stande zu bringen; hiebei sey nicht zu fordern, daß der Mensch ganz vollkommen, sondern nur, daß so viele und so erhabene Tugenden da gewesen, daß sich erwarten läßt, ihre Darstellung werde erbaulich einwirken; vornehmlich solle man bemüht seyn, Muster von jeder Art der Tugend aufzustellen; auch solche, wie möglich, aus jedem Geschlechte, Lebensalter, Stande und Gewerbe, Volke und Zeitalter hervorzuheben \usw\ \par
\RWpar{IIIb}{279} behauptet B.: \danf{nichts sey vollkommen schön, was nicht auch sittlich gut ist; in den religiösen Ideen des Katholicismus, in den Erzählungen der Bibel sowohl des alten als des neuen Bundes, in den Lebensgeschichten der Heiligen \usw\ gebe es einen nie zu erschöpfenden Vorrath zu Darstellungen für eine jede Art der schönen Künste} \par
\usw\ \par
Bezeichnend für des Vfs. Bestreben, von jeder annoch bestehenden Einrichtung zu zeigen, wozu sie benützt werden sollte, ohne dem Fortbestande derselben darum das Wort zu reden, ist die lange Anmerkung zu \RWpar{IIIb}{280} über die evangelischen Räthe und Klostergelübde, deren Mittheilung uns jedoch der Raum verbietet. \par
Die Lehre von den Sakramenten beginnt der Vf.\ \RWpar{IIIb}{281} mit der von protestantischer Seite so mißdeuteten \danf{Lehre von ihrem gültigen und pflichtmäßigen Gebrauche im Allgemeinen.} Es zeigt sich, daß man nicht anders lehren dürfe, als die katholische Kirche hier thut, wenn einerseits der Lässige sich nicht beruhigen, sondern im Gegentheil die Strafwürdigkeit seines Verbrechens in seiner ganzen Größe erkennen soll, andrerseits aber derjenige, der Alles, was an ihm lag, gethan hat, sich soll zu trösten wissen über die Besorgniß, ob auch der Andere ganz seine Schuldigkeit erfüllet habe. Muß nicht zu diesem Zwecke ein doppelter, \seitenw{214} ein bloß gültiger sowohl als auch ein würdiger Gebrauch eines jeden Heiligungsmittels unterschieden, und von dem ersten gesagt werden, daß er eintrete, so oft der Ausspender nur gewisse von der Kirche für wesentlich erklärte Handlungen mit äußerem Ernste verrichtet, der Empfänger aber, falls er erwachsen ist, seinen Willen, das Sacrament zu empfangen, an den Tag gelegt hat. Wie furchtbar, und doch wie wahr auch zugleich, wenn nun hinzugefügt wird, daß eine solche Ausspendung, wenn sie dabei unwürdig war; deßgleichen auch ein solcher Empfang eben um jener Gültigkeit wegen für den, der dabei unwürdig verfuhr, nicht bloß gemeine, sondern übernatürliche, \dh\  solche Strafen nach sich ziehe, wie sie nur aus der Offenbarung selbst bekannt sind? Welche Beruhigung dagegen für Jeden, der sich bemühet hat, nicht bloß gültig, sondern auch würdig zu empfangen, oder nur auszuspenden, wenn ihm versichert ist, daß ihn des Andern ihm vielleicht ganz verborgene Schuld der Segnungen des Himmels nicht beraube. Doch gehen wir nun zu den einzelnen Heiligungsmitteln über. \par
In der \danf{Lehre des Katholicismus von der h. Taufe oder Aufnahmsfeier} (\RWpar[\BUparformat{282--4.}]{IIIb}{282--284}) liest man: \par
\danf{1) Wer immer nach reiflicher Prüfung des katholischen Christenthums die Wahrheit desselben erkannt hat, an den macht eben dieses Christenthum die Anforderung, daß er die ihm gewordene Überzeugung so öffentlich als möglich kund thue, und zugleich feierlich gelobe, sein Leben künftig ganz nach den Vorschriften desselben einrichten zu wollen. \par
2) Auf dieses Gelübde soll er, wenn sich vermuthen läßt, daß es ihm Ernst damit sey, und daß er der Kirche nicht zu Schanden leben werde, von einem Vorsteher der Gemeine unter die Zahl der Christen so feierlich als möglich aufgenommen werden. Das Zeichen dieser Aufnahme soll eine Reinigung des Leibes, oder auch ein diese Reinigung nur andeutendes Begießen oder Besprengen mit natürlichem Wasser seyn; wobei man den eigentlichen Zweck der Handlung durch die bestimmten Worte: Hiemit taufe (\dh\  widme) ich \seitenw{215} dich der Verehrung Gottes des Vaters, des Sohnes und des h. Geistes, ausdrücken solle} \usw \par
Was hier die Kirche als allein wesentlich und unabänderlich darstellt, ist eine so einfache Handlung und von so offenbarer Zweckmäßigkeit, daß es kaum möglich ist, etwas dagegen einzuwenden. Die Zuthaten, von denen die Kirche selbst gestehet, daß sie mangelhaft seyen und nach Zeit und Umständen einer Abänderung unterliegen, mögen dem Leser einer auch noch so großen Verbesserung bedürftig scheinen: da es nicht die Zufriedenheit mit Allem, was in der Kirche bestehet und geübt wird, sondern lediglich die Annahme dessen, was sie zu glauben vorstellet, ist, wodurch wir Katholiken sind und werden: so können Ansichten von einer solchen Art nie ein hinreichender Grund zur Verwerfung des Katholicismus werden. \par
Die Wirkungen, welche Nr. 3. der Taufe zugeschrieben werden (es sind die gewöhnlichen), kann Niemand anstößig finden, der die vorhergehenden Lehren von der Vergebung unserer Sünden, der Erbsünde, den Einwirkungen des h. Geistes schon als vernunftmäßig anerkannt hat. Die Nr. 5. erwähnte Kindertaufe ist abermal nur ein Gebrauch der Kirche, den Jemand mißbilligen dürfte, wenn er nur nicht so weit gehet, zu behaupten, daß eine solche Taufe ganz ungültig sey. Das aber wird schwerlich zu thun sich versuchet fühlen, der die \RWpar{IIIb}{284} Nr. 5. gelieferte Rechtfertigung gelesen. Denn er wird dann gewiß einsehen, daß eine solche Handlung, wenn sie auch nicht das Kind, wenigstens nicht zu der Zeit, da sie an ihm vollzogen wird, zu erbauen vermag, doch andere ersprießliche Wirkungen habe. Und wenn er nun findet, daß dieselben Theologen, welche die Kindertaufe in Schutz nehmen, eine Taufe, die wider den Willen des zu Taufenden, oder (falls er ein Kind ist) wider den Willen seiner Eltern oder Vormünder versucht wird, nicht nur für unerlaubt, sondern sogar für ungültig erklären; wenn er eben so wie bei der Taufe auch bei jedem anderen Sacramente gewahr wird, daß man nur dort allein, wo die natürlichen Folgen der Handlung aufhören, sittlich zuträglich zu seyn, auch nicht mehr \seitenw{216} zugeben will, daß übernatürliche Folgen von wohlthätiger Art Statt finden sollen: wird er da nicht sich gedrungen fühlen, in den Entscheidungen der Kirche die Leitung des Geistes Gottes um so gewisser anzuerkennen, je weniger er es den einzelnen Lehrern selbst zutrauen will, sie wären sich der Regel, nach der sie ihre Entscheidungen einrichten sollen, deutlich bewußt gewesen? \par
\danf{Die Behauptung, daß die Taufe der Seele ein unvertilgbares Merkmal eindrücke} (schreibt der Vf.\ \RWpar{IIIb}{284} Nr. 6.), \danf{hat nur den Sinn, daß ein Getaufter von nun an für immer ganz anders zu betrachten sey, als ein noch Ungetaufter. Und dieses ist sehr vernünftig. Strenge genommen hat ja eine jede Handlung, welche der Mensch entweder selbst, oder welche Andere an ihm vollziehen, Folgen, die sich in das Unendliche erstrecken, so daß er von der Zeit an für immer als ein Anderer betrachtet werden könnte, als wenn diese Handlung nicht stattgefunden hätte. Aber es wäre nicht gut, diesen Gedanken bei einer jeden Handlung zu verfolgen. Nur bei Verrichtungen, die von besonderer Wichtigkeit sind, die ferner gewisse sehr wohlthätige Folgen haben, die endlich dem Menschen eigene Pflichten und Verbindlichkeiten auflegen, wird es gerathen seyn, dieß zu Gemüthe zu fassen. Nirgends gerathener also als bei der Handlung der h. Taufe. Aus diesem Gesichtspunkte nämlich erscheint die Taufe dem Menschen in ihrer ganzen Wichtigkeit; indem er erwägt, daß sie Folgen habe, die sich bis in die Ewigkeit erstrecken. So freut er sich nun des Glückes, ein Getaufter zu seyn, um desto inniger; so fühlt er auch um so stärker die Verbindlichkeit, sein ganzes künftiges Leben bei den in ihr gemachten Gelübden zu verbleiben; so haben auch alle seine Mitchristen die Pflicht, so lange er lebt, sich für ihn eifriger zu verwenden, ihn, wenn er etwa auch vom Glauben abgefallen wäre, doch noch als Christen zu betrachten, und also an seiner Zurückbringung zu arbeiten} \usw\ \par
\danf{Die\editorischeanmerkung{Öffnende Anführungszeichen ergänzt} Lehre von der h. Firmung oder Bestätigungsfeier} (\RWpar{IIIb}{285}) fängt der Vf.\ so an: \seitenw{217} \par
\danf{1) Wer durch die Taufe als Mitglied der christlichen Gemeine ausgenommen ist, soll, wenn er die ersten in seiner Kindheit, oder doch auf keine ganz feierliche Art und nicht vom Bischofe selbst empfangen hat, in einem Alter, da er die Lehren des Katholicismus bereits vollkommen inne hat, seine Überzeugung von ihrer Wahrheit, und sein Gelübde, nach ihnen zu leben, vor der Gemeine, der er als Mitglied angehört, und vor dem Bischofe derselben kund thun. \par
2) Diese Erklärung soll, wenn sich zufolge des Urtheils eigener Personen (der Firmpathen) vermuthen läßt, daß sie ernstlich gemeint sey, von Seite des Bischofs und der übrigen Gemeine mit Freuden aufgenommen, und damit erwiedert werden, daß man für ihn bei Gott gemeinschaftlich fürbittet, auf daß ihn der h. Geist mit seinen Gaben und Gnaden ausrüsten und zur Erfüllung seines löblichen Vorsatzes stärken wolle. Zum Zeichen, daß diese Stärkung auch wirklich erfolgen werde, soll der Bischof den zu Firmenden salben, wie Kämpfer gesalbet werden, mit h. Öle} \usw\ \par
Das Übrige ist wie gewöhnlich; wir aber möchten fragen, ob wohl irgend Jemand an dieser Feier etwas auszusetzen wüßte, als daß man den Zutritt zu dieser h. Handlung häufig gestattet in einem Alter, das viel zu jugendlich ist, und daß man eben darum den Firmling insgemein dabei zu wenig selbst handeln läßt? Fehler, deren Verbesserung ganz in die Macht unsrer kirchlichen Vorsteher gelegt ist. \par
\danf{Die Lehre vom h. Abendmahle} lautet mit einigen Abkürzungen so: \par
\danf{1) Die katholische Kirche verlangt von ihren Mitgliedern, daß sie, wenn es die Umstände erlauben, öftere Zusammenkünfte halten, um Gott auch gemeinschaftlich zu verehren. Wenn nun ein Bischof oder ein Priester zugegen, soll unter Anderm auch ein gemeinschaftliches Mahl gefeiert werden, das eine Nachahmung sey von jenem Abendmahle, mit welchem Jesus von seinen Jüngern schied. Der Bischof oder Priester soll hiebei die Person des Herrn, \par
die übrigen Gläubigen aber die Apostel vorstellen. Der Er\seitenw{218}stere soll in Nachahmung dessen, was wir im Evangelio von dem Herrn lesen, Brod und Wein nehmen, und nachdem er darüber die Worte des Herrn: \danf{Das\editorischeanmerkung{Öffnende Anführungszeichen ergänzt} ist mein Leib, mein Blut,} gesprochen, davon genießen, und auch den Anwesenden zum Genusse darbieten. \par
2) Wie jene Worte über das Brod und den Wein nur ausgesprochen sind, so hat schon das Wesen des Brodes und des Weines aufgehört, da zu seyn, und unter den bloßen Gestalten, welche hier noch verweilen, ist Jesus Christus nach seiner göttlichen sowohl als menschlichen Natur wahrhaft und wesentlich zugegen; nicht etwa so, daß er den Raum dieser Gestalten ausfüllte, und in dem einen Theile derselben mit diesem, in einem andern mit einem andern seiner Theile zugegen wäre, und wenn man auf diese Gestalten verändernd einwirkt, selbst auch Veränderungen erlitte: sondern in jedem Theile derselben ist Christus ganz gegenwärtig; doch so, daß seine Gegenwart aufhört, sobald sie selbst aufhören, Gestalten von Brod und Wein (genußbare Nahrungsstoffe) zu seyn. \par
3) Es ist uns nicht nur erlaubt, sondern selbst unsere Pflicht, den unter diesen Gestalten sich uns darbietenden Gottmenschen anzubeten; diese Gestalten sonach mit der größtmöglichen Ehrerbietigkeit zu behandeln \usw\ Insbesondere sollen wir uns in ihrer Gegenwart einer so feurigen Andacht befleißen, als wir es für unsere Pflicht erachten würden, wenn unser Herr in menschlicher Gestalt vor uns erschiene. \par
4) Und wenn wir dieß thun, und -- nachdem wir uns zuerst durch alle uns zu Gebote stehende Mittel gehörig vorbereitet haben -- im demüthigen Vertrauen auf Jesu eigene Aufforderung es wagen, von jenen gesegneten Gestalten auch etwas zu genießen: so wird unsre Seele durch Jesum auf eine eben so wohlthätige Art genähret und gestärkt, wie Brod und Wein die stärkendsten Nahrungsmittel des Leibes zu seyn pflegen, und eben so innig, wie solche Nahrungsmittel sich mit dem Leibe vereinigen, wird sich auch Jesus mit unsrer Seele vereinen. \seitenw{219}\par
5) Selbst wenn wir uns im Gefühle unserer Unwürdigkeit, oder aus Besorgniß, daß uns die öftere Wiederholung gleichgültig machen dürfte, dem wirklichen Genusse des heil. Mahles entziehen, aber dabei doch mit aller uns möglichen Andacht zugegen sind, und um die Mittheilung einiger Segnungen bitten: bitten wir nicht vergebens. \par
6) Wer dagegen der h. Handlung ohne Andacht beiwohnt, macht sich nur strafwürdig; und wer es wagt, den Leib des Herrn zu empfangen, ohne sich erst in eine gottgefällige Gemüthsverfassung versetzt zu haben, der isset und trinket sein eigen Gericht. \par
7) Um der gesegneten Wirkungen des h. Mahles theilhaft zu werden, genügt es, dasselbe auch nur unter einer Gestalt zu empfangen. \par
8) Die fast bei allen Völkern bestandene Gewohnheit, Gott unter Anderm durch Opfer zu ehren, soll auch noch unter uns fortdauern; doch so, daß wir nicht ferner glauben, Gott wohlgefällig zu werden, wenn wir eines seiner Geschöpfe des Lebens berauben, oder irgend einen nutzbaren Gegenstand zerstören: sondern wir sollen wissen, daß Gott kein anderes Opfer verlange, als die Verzichtleistung auf jeden eigenen Vortheil, und selbst auf das Leben, wenn das gemeine Beste und die Gesetze der Tugend es erheischen; daß eben deßhalb unter allen Opfern, die Gott je dargebracht wurden, keines ihm wohlgefälliger gewesen, als das Opfer, das Jesus Christus durch seinen freiwilligen Versöhnungstod am Kreuze dargebracht hat. Wir sollen ferner glauben, daß Christus, so oft wir dieß h. Mahl auf Erden feiern, im Himmel Kenntniß davon erlange, daß es ihm angenehm sey, daß er beim Vater fürbitte für uns, ja bildlicher Weise zu reden diesen neuerdings an seine Verdienste um uns erinnere, und so für die Erhörung unserer Bitten geneigter mache. Wir sollen eben deßhalb diese Gelegenheit benützend ein Jeder nicht nur für uns, sondern auch für Andere, Verwandte und Freunde, Lebende und Verstorbene mit dem gewissesten Erfolge einer Erhörung beten} \usw\ 
\seitenw{220}\par
Aus dem \RWpar{IIIb}{288}, der die Vernunftmäßigkeit und den sittlichen Nutzen dieser Lehre erörtert, heben wir nur einige abgebrochene Bemerkungen aus: \danf{Wir können schon dann sagen, nicht mehr das Wesen des Brodes und Weines sey da, wenn nur die Wirkungen, welche die unter diesen Gestalten vorhandenen Stoffe auf uns ausüben, durch den Hinzutritt gewisser anderer Umstände jetzt ganz anders und unendlich wohlthätiger geworden sind, als es die Wirkungen eines gemeinen Brodes oder Weines werden können.} -- \danf{Ob die Wirksamkeit Jesu im h. Abendmahle eine ganz unvermittelte sey oder nicht, das muß uns, wenn wir vernünftig denken, gleichgültig seyn; und eben darum gehört die Entscheidung dieser Frage einer bloß müßigen Neugier gar nicht zur Religion und in das Gebiet kirchlicher Unfehlbarkeit.} -- \danf{Sagen, daß wir im h. Abendmahle nichts Mehreres als eine bloß bildliche Gegenwart Jesu annehmen dürfen, heißt sagen, daß wir uns von dem Genusse dieses Mahles durchaus keinen andern und größeren Nutzen versprechen dürfen, als den wir von irgend einem lebhaften Andenken an Jesum hätten. Behaupten, daß Jesus zwar wirklich gegenwärtig sey, doch nur im Genusse, heißt einen der wichtigsten natürlichen Vortheile, den uns der Glaube an diese Gegenwart gewähren könnte, aufheben.} -- \danf{Es ist ein merkwürdiges Beispiel, wie aufgeklärt die Vorsteher der katholischen Kirche dachten, als sie die Meinung, daß man das h. Abendmahl unter beiden Gestalten genießen müsse, um seiner Segnungen theilhaft zu werden, als einen Aberglauben verwarfen} \usw\ \par
In Bezug auf die katholische Lehre \danf{von der Buß- und Besserungsanstalt,} wie sie \RWpar{IIIb}{289--291} dargestellt und gerechtfertiget wird, besorgen wir keine Einwendungen von Seite unserer Leser, sofern es ihnen ein wahrer Ernst um ihre eigene sittliche Ausbildung ist, was wir voraussetzen müssen. Wer aber mehr auf die Eingebungen seiner Sinnlichkeit als auf die Forderungen seiner Vernunft zu achten gewohnt ist, und auch noch ferner zu achten die Freiheit behal\seitenw{221}ten will, den wird allerdings nicht überzeugen, was der Vf.\ hier sagt, denn er will nicht überzeuget werden. Für ihn wäre auch Alles, was wir hier über die Unabhängigkeit der vorliegenden Lehre von des Vfs. anderweitigen Ansichten und Behauptungen anmerken könnten, vergeblich. \par
Auch über die katholische Lehre \danf{vom Ehestande} (\RWpar{IIIb}{292--294}) haben wir nichts zu erinnern; denn es muß Jedem einleuchten, daß nur durch solche Grundsätze, wie sie hier aufgestellt sind, für die Fortpflanzung und Erziehung des menschlichen Geschlechtes am Besten fürgesorgt, und zugleich auch eine naturgemäße Befriedigung des Geschlechtstriebes bei der größten Anzahl von Individuen möglich gemacht wird. Zu beklagen ist nur, daß es dem Einflusse, den die Begriffe des Christenthums auf die allmähliche Verbesserung unserer Sitten, Gebräuche und Einrichtungen ausüben könnten und sollten, bis auf den heutigen Tag noch nicht gelungen ist, aus unsrer bürgerlichen sowohl als kirchlichen Verfassung alles dasjenige zu entfernen, was jenen wohlthätigen Zwecken geradezu widerspricht. Übrigens lehrt die katholische Kirche hier nichts, was nicht auch schon die bloße Vernunft für nöthig erklärte, und was nicht alle christliche Religionen beibehalten hätten; es sey denn der Satz, daß Gott die h. Handlung des Ehebundes, wenn wir sie auf die gehörige Weise vollziehen, übernatürlich, \dh\  so reichlich segne, als wir es ohne seine eigene Erklärung gar nicht zu hoffen berechtiget wären. Ob nun das Beste der Menschheit erheische, daß dieser Satz je eher je lieber weggestrichen werde, mag Jeder selbst beurtheilen. So viel ist offenbar, daß weder die Unauflöslichkeit der Ehe (eine der Abänderung unterworfene Disciplinarverordnung, bestehend bloß in der lateinischen Kirche), noch die Verwerflichkeit jedes von einem Katholiken mit einem anders Gläubigen eingegangenen Ehebundes aus der sacramentalischen Würde dieses Bundes mit Recht gefolgert werden könne. \par
In der nun folgenden Lehre \danf{von der Kirche und von den Vorstehern derselben oder dem geist\seitenw{222}lichen Stande} (\RWpar{IIIb}{295}) hat der Vf.\ abermal einige seiner Privatansichten, \zB\ über den Zweck der Kirche (Nr. 2.), über die Glieder derselben (Nr. 5.), über die Grenzen der geistlichen Macht (Nr. 21.) \umA\  nicht deutlich genug von den allgemein angenommenen Lehren gesondert. Auch ist bei der großen Anzahl der Punkte, die man hier aufgestellt findet, Alles zu kurz und unbestimmt gegeben; und noch weniger war in der Rechtfertigung all dieser Punkte (\RWpar{IIIb}{297}) Raum genug vorhanden, die zum Theile sehr eigenthümlichen Ansichten so auseinanderzusetzen, wie es zu ihrem völligen Verständnisse nothwendig wäre. Wohl mußte er also darauf gefaßt seyn, daß nicht ein Jeder in diesem §. mit ihm durchgängig einverstanden seyn werde. Wir wollen die wichtigsten Punkte ausheben, und bei einigen sogleich auch etwas von demjenigen, was er zu ihrer Rechtfertigung vorgebracht hat, beifügen. \par
\danf{1) Der katholische Lehrbegriff macht es Jedem, der sich von seiner Wahrheit überzeugt hat, zur Pflicht, sich der Religionsgesellschaft, die Jesus gestiftet, der er auch fortwährend als ihr oberster Leiter und Gesetzgeber vorstehet, anzuschließen. \par
Der Begriff einer Religionsgesellschaft oder Kirche ist eben der, daß sie eine Gesellschaft sey, der beizutreten es unsere Pflicht schon bloß aus dem Grunde wird, weil wir uns von der Wahrheit einer gewissen Religion überzeugt haben. Es ist vernünftig, daß eine jede Religion von ihren Bekennern das Zusammentreten in eine eigene Gesellschaft verlange. \par
2) Es besteht aber der Zweck dieser Gesellschaft überhaupt in der Erreichung alles desjenigen Guten in Zeit und Ewigkeit, was sich durch eine solche Summe von Kräften, als da Zusammenkommen, Hervorbringen läßt. \par
Immer muß von dem Zwecke, der die Entstehung einer Gesellschaft oder auch nur unsern Beitritt zu ihr bewirkte, der Zweck, den die nun einmal vorhandene Vereinigung von Menschen sich vorzusetzen hat, unter\seitenw{223}schieden, und dieser letztere in die Erreichung alles durch sie nur immer möglichen Guten gesetzt werden. \par
5) Glieder der Kirche sind überhaupt alle Jene, durch deren Beizählung zu ihr irgend ein überwiegender Vortheil erreicht wird. Daher \zB\ unter gewissen Umständen auch noch unmündige Kinder. \par
7) Man kann der Kirche bald nur innerlich (der Gesinnung nach) bald nur äußerlich, bald auf beide Arten zugleich angehören. \par
8) Wer die rechtglaübige Gesinnung hat, \dh\  bereit ist, Alles zu glauben, was Gott geoffenbart hat, oder was sich ihm überhaupt als Wahrheit darstellt, auch darnach leben will, der gehöret der Kirche bloß hiedurch schon innerlich an; gesetzt auch, daß ihn irgend ein ungerechter Spruch irdischer Richter der äußeren Zeichen beraubt, oder daß er dergleichen Zeichen der Aufnahme noch nie empfangen, oder die Lehren der katholischen Religion nicht einmal noch kennen gelernet hätte. \par
17) Diese Kirche kann in Wahrheit behaupten, daß man ihr wenigstens innerlich angehören müsse, um der ewigen Seligkeit theilhaft zu werden. \par
18) In der sichtbaren Kirche soll es fortwährend einen eigenen Stand geben, dem es obliegt, sich mit den Wahrheiten der Religion zuvörderst selbst auf das Vollkommenste vertraut zu machen, dann aber auch Andre darin zu unterrichten. \par
20) Nebst dem Geschäfte des Unterrichts soll diesem Stande auch die Leitung des Gottesdienstes und die Ausspendung der Heiligungsmittel anvertraut seyn. \par
21) Das Ansehen, das die genannten Rechte und Obliegenheiten diesem Stande geben, und die Macht, die ihm die weltlichen Obrigkeiten einräumen, soll er benützen, um manche heilsame Anordnungen zu ertheilen. \par
Es ist vernünftig, die Grenzen der geistlichen Macht auf keine andere Art zu bestimmen, als durch die \seitenw{224} Grenzen der Möglichkeit selbst; denn eine jede andere Bestimmmung würde fehlerhaft seyn oder höchstens für eine Zeit nur gelten. So ist es \zB\ ganz falsch zu sagen, nur was die Tugend und das Heil der Seele betrifft, darüber hätte die geistliche Obrigkeit zu entscheiden. Denn da nicht leicht irgend etwas ganz gleichgültig für diese Zwecke ist, so müßte die Geistlichkeit nach diesem Grundsatze über alle Dinge, ohne Ausnahme, verfügen dürfen. -- Die gesunde Vernunft sagt, daß jeder Mensch des Guten so viel thun solle, als er nur irgendwie, unter Anderm auch dadurch vermag, daß er es Andern, welche ihm zu gehorchen bereit sind, gebiete. Also versteht es sich von selbst, daß auch die geistliche Obrigkeit die Pflicht habe, über Alles Anordnungen zu ertheilen, worüber die Glaübigen ihr zu gehorchen bereitwillig sind, und woran die weltliche Obrigkeit sie nicht hindert. Da aber die Gesinnungen der letztern in verschiedenen Ländern und Zeiten verschieden sind, so läßt sich eben deßhalb keine bestimmtere Regel als die obige ertheilen. \par
22) Die Mitglieder des geistlichen Standes sollen den Übrigen zum Muster der Nachahmung dienen; also auf einer entschieden höheren Stufe der Vollkommenheit stehen. \par
24) Wenn es die weltliche Obrigkeit erlaubt, soll die Gemeine das Recht der Wahl ausüben. \par
26) Es soll in diesem Stande Beides, verschiedene Abstufungen in der einem Jeden zugestandenen Fähigkeit zur Ausspendung der Heiligungsmittel sowohl als auch in der äußern ihm wirklich eingeräumten Amtsgewalt geben. \par
28) Niemand soll zu einer höheren Stufe der Weihe zugelassen werden, ohne erst alle untergeordneten empfangen, und sich durch eine löbliche Verwaltung gewisser für diese passenden Ämter des Empfangs einer höheren Weihe würdig bewiesen zu haben. \par
30) In Betreff der Ämter soll es in der katholischen Kirche geben: \seitenw{225}\par
a) Einen Primas, d. i. Vorsteher der ganzen Christenheit, der in dieser Eigenschaft gleichsam der Stellvertreter Christi seyn soll. (Gegenwärtig der Bischof von Rom). \par
b) Jedem der übrigen Bischöfe soll das Aufseheramt über einen bestimmten nicht allzu ausgedehnten Theil der katholischen Christenheit angewiesen werden. \par
c) Kann dieser nicht alle geistlichen Bedürfnisse seiner Gemeine allein befriedigen, so soll er untergeordnete Seelsorger zur Aufsicht über kleinere Theile aufstellen \usw\ \par
Die Nothwendigkeit eines Primas ist unschwer darzuthun. Wenn auch Jesus versprochen, er selbst werde das unsichtbare Oberhaupt der Kirche durch alle Zeiten bleiben: so versteht sich doch, daß er uns hiedurch nicht der Pflicht entheben wollte, alle diejenigen Einrichtungen zu treffen, durch welche, nach menschlicher Einsicht, am ehesten Böses verhütet und Gutes herbeigeführt werden kann. Und unter diese gehört ein zweckmäßig eingerichteter Primat unwidersprechlich. In jeder Gesellschaft, die länger fortdauern soll, wird es entschiedene Vortheile haben, wenn ein Einzelner da ist, dem man es dadurch, daß man ihn für den Vorsteher der ganzen Gesellschaft erklärt, zur Obliegenheit gemacht hat, über das, was die Gesellschaft thun soll, fortwährend nachzudenken, was ihm das Beste scheint, nicht nur bekannt zu geben, sondern durch Bitten, Aufforderungen, Befehle \udgl\  dahin zu wirken, daß es zur Ausführung komme; \usw\  -- Daß aber die Rechte und Pflichten des Primas nicht von der Offenbarung selbst festgesetzt sind, hat seinen guten Grund. Es könnte sonst nicht füglich etwas an ihnen abgeändert werden; was zu verschiedenen Zeiten doch äußerst nothwendig ist. Gewiß gab es Zeiten, wo es sehr gut war, daß der Primas der Kirche nur bittend vortrug, was er von Andern ausgeführt zu sehen wünschte; und wieder andere Zeiten, wo er allmählig einen befehlenden Ton annehmen durfte und sollte. Auch gab es Zeiten, wo es recht heilsam war, daß sich der Papst herausnahm, in Dingen zu entscheiden, von denen in unsern Tagen gesagt wird, daß sie ganz außerhalb des Bereiches seiner Macht \seitenw{226} lägen. Und wieder gibt es jetzt eine Zeit, wo es nothwendig ist, daß er von solchen Forderungen abstehe, Einiges nur erbitte \usw\ \par
35. Kein Geistlicher darf die Einkünfte, die ihm sein geistliches Amt einbringt, als sein durch diese Dienste erworbenes Eigenthum betrachten; er hat sie vielmehr als ein der Kirche gehöriges Gut, d. i. als ein Gut anznsehen, das bloß zu wohlthätigen Zwecken, \zB\ namentlich zur Unterstützung der Armen, verwendet werden darf. Für sich darf er nur dann, wenn er sich seinen Lebensunterhalt auf keine andere Weise verschaffen kann, so viel nehmen, als er zu solchem Zwecke nothwendig braucht. Wer mehr genommen, der ist zur Rückstellung verpflichtet. \par
Was hier dem Geistlichen zur Pflicht gemacht wird, ist eigentlich eine auch allen übrigen Menschen, nur nicht in gleichem Grade, obliegende Verbindlichkeit. Der Unterschied zwischen demjenigen, was wir unser Eigenthum nennen, und was Gemeingut heißt, bestehet nicht in dem Zwecke, zu dem es verwendet werden soll; nicht ist es uns erlaubt, das Eigenthum bloß für uns selbst zu verbrauchen, unangesehen, was wir damit für Andere ausrichten könnten; sondern wir sollen es, wie das Gemeingut, immer nur dergestalt verwenden, daß das gemeine Beste dadurch am meisten gewinnt; also für uns, wenn wir es nöthiger als Andere, für Andere, wenn diese es nöthiger haben, als wir. Der einzige Unterschied ist, daß wir von unsrer Verwendung des Eigenthums nur unserm Gewissen, von unsrer Verwendung des Gemeinguts aber auch Andern (etwa dem Staate) Rechenschaft zu legen haben. In ihrem Gewissen also sind eigentlich alle Menschen verbunden, ihr Eigenthum wie ein Gemeingut zu betrachten, und zur Befriedigung ihrer eigenen Bedürfnisse davon nur so viel zu verbrauchen, als durchaus nothwendig ist, das Übrige aber an Arme, oder doch zu Zwecken, die gemeinnützig sind, zu verwenden. Für den geistlichen Stand konnte man dieses Verfahren zu einer um desto strengeren Obliegenheit erheben; erstlich schon darum, weil er den Übrigen mit seinem Beispiele vorleuchten \seitenw{227} soll; dann aber auch, weil die mit geistlichen Ämtern verbundenen Einkünfte insgemein nur aus den frommen Spenden der Gläubigen entstanden, die diese nur in der bestimmten Absicht gaben, damit sie für Arme oder zu andern wohlthätigen Zwecken verwendet würden, so daß die Geistlichen nur in sofern auch selbst einen theilweisen Anspruch darauf hätten, als sie den Armen beizuzählen wären, \dh\  als sie außerdem gar nicht zu leben vermöchten.} --\par
Was nun in allen diesen Behauptungen eigentliche Lehre der Kirche ist, das ist dem bloßen gesunden Menschenverstande so einleuchtend, daß es ein billig Denkender kaum wird bestreiten wollen. Denn stimmen nicht selbst die akatholischen Parteien in den meisten dieser Punkte mit uns überein? und sind nicht die Verschiedenheiten, die zwischen ihrer und unserer Lehre doch angetroffen werden, alle auf sehr erklärliche Weise entstanden? entstanden nämlich nur aus dem Verdrusse über die empörenden Mißbräuche, zu welchen jene Lehren gerade damal Veranlassung gegeben hatten? Hätte man je die Nothwendigkeit eines Primates verkannt, wenn nicht so manche Päpste eine für das Heil der Kirche wirklich zu große und schädlich wirkende Gewalt sich angemaßt hätten? Würde man je ein Bedenken getragen haben, der Weihe des geistlichen Standes eine von Gott verheißene Segenskraft zuzugestehen, hätte man nicht die Folgerungen gefürchtet, die -- obgleich sehr unberechtigter Weise -- aus einem solchen Vordersatze gezogen wurden? \par
Wer aber in den bisher besprochenen Lehren des Katholicismus keinen Grund zur Verwerfung desselben gefunden, der wird ihn gewiß auch nicht in derjenigen finden, die der Vf.\ zuletzt gestellt hat, in der Lehre \danf{von der den Kranken und Sterbenden zu leistenden Hülfe oder der letzten Ölung} (\RWpar{IIIb}{298--300}). \par
\danf{1) Wenn ein Christ} (sagt der Vf.) \danf{auf längere Zeit durch Krankheitsumstände, ja auch wohl andere Ursachen verhindert ist, die gottesdienstlichen Versammlungen zu besuchen: so soll er berechtiget seyn, die Geistlichen der Gemeine \seitenw{228} zu sich bitten zu lassen, damit sie ihm die entbehrte Gabe des Unterrichts reichen, ihn seinen Bedürfnissen nach belehren, trösten, mit ihm gemeinschaftlich beten, ihn mit den h. Sakramenten der Buße und des Altares versehen, \usw\ \par
2) Die Priester sollen sich weder durch die Beschwerlichkeiten des Weges, noch durch die Gefahren der Ansteckung, noch durch andere Rücksichten auf bloß leibliche Güter abhalten lassen, demjenigen, der sie um einen solchen der Seele zu leistenden Dienst ersucht, wer er auch sey, zu willfahren. \par
3) Wird die Gefahr des Todes größer: so soll der Geistliche abermal herbei gerufen werden, und nachdem er manch herzliches Gebet um die Vergebung der Sünden des Kranken, um seiner Leiden Linderung, ja auch (wenn sie zu seinem Seelenheile ist) um seine gänzliche Genesung zum Throne Gottes gesandt, ihn an verschiedenen Theilen des Leibes, wo es der Wohlstand erlaubt, wie etwa an den fünf Sinneswerkzeugen, mit h. Öle salben. \par
4) Um dieser Gebete und Salbungen willen wird Gott, wenn es dein Seelenheile des Kranken dienlich ist, seine Genesung bewirken, und wenn dieß nicht ist, ihm jene übernatürlichen Gnaden verleihen, die er, um eines seligen Todes zu sterben, nöthig hat.} \par
Wer sich einmal erhoben hat zu jenem geläuterten Begriffe einer Offenbarung, der gleich in dem ersten Bande des Lehrbuches aufgestellt worden ist, der wird zur Wesenheit eines Sakraments nicht verlangen, daß es unmittelbar und mit ausdrücklichen Worten durch Christum eingesetzt sey; der wird auch eben deßhalb weder an den so eben besprochenen, noch andern Heiligungsmitteln der katholischen Kirche einen Anstand nehmen, bloß darum, weil solche Einsetzungsworte für sie nicht in der Bibel sich nachweisen lassen; der wird nicht glauben, daß irgend ein Mehreres nothwendig sey, um alle diese Sakramente mit der vollkommensten Zuversicht anzunehmen, als zu erkennen, was ihm sein innerstes Gefühl bezeugt, daß es Handlungen sind, welche im höchsten Grade erbaulich und sittlich zuträglich für uns sind. \seitenw{229}\par
Und so hätten wir denn nun getreulich angegeben, wie wenige und wie leicht zuzugestehende Voraussetzungen es sind, die man zugeben muß, um den Beweis, der für die Wahrheit des katholischen Christenthums in B.'s Lehrbuche geführt wird, auch für sich selbst überzeugend zu finden. Bevor wir schließen, müssen wir uns nur noch gegen eine Mißdeutung unserer Absicht verwahren, die um so mehr zu besorgen stehet, je öfterer schon die Erfahrung gelehrt hat, daß gewisse böswillige oder auch nur im Denken ungeübte Leute auch das Unschuldigste, was B.\ oder die Freunde seiner Ansichten sagen, durch eine unrichtige Auslegung ärgerlich finden. Es ist nämlich weder unsere Meinung, noch war es je B.'s Meinung gewesen, die Leser seines Buches sollten alles dasjenige, was er bei seinem Beweise nicht eben als ein nothwendiges Zugeständniß voraussetzt, als etwas Unentschiedenes betrachten, oder wohl gar als etwas erweislich Falsches bestreiten und verwerfen. Wenn uns ein Mathematiker aufmerksam darauf macht, daß verschiedene Sätze, aus welchen wir bisher irgend ein wichtiges Theorem dieser Wissenschaft ableiteten, nicht eben nothwendig bei demselben vorausgesetzt werden müssen, indem es eine viel kürzere Reihe von Schlüssen gibt, aus welchen dasselbe gefolgert werden kann (wie dieß B.\ selbst in Hinsicht mancher mathematischer Lehren gethan hat): wem fällt es ein zu glauben, derselbe wolle die Sätze, deren er zu seinem Beweise nicht bedarf, sofort für unzuverlässig oder gar falsch erklären? Ein Gleiches ist es mit jeder anderen Wissenschaft; nur mit dem Unterschiede, daß es bei religiösen Lehren, wegen der großen Geneigtheit der Menschen, hier Alles in Zweifel zu ziehen, ein eigenes Verdienst von Wichtigkeit ist, dergleichen Wahrheiten darzuthun auf dem möglich kürzesten Wege und aus Voraussetzungen, die fast auf keine Weise bestritten werden können. Keineswegs muß derjenige, der solche einfachere Beweise liefert, der uns zu zeigen sucht, wie viele bisher gemachte Voraussetzungen recht füglich wegbleiben können, -- keineswegs muß er dieß thun, weil er bei sich selbst glaubt, daß jene von ihm entbehrlich gemachten Voraussetzungen alle falsch oder auch nur unzuverlässig wären. \seitenw{230}\par
Nur weil er weiß, daß sie von Andern bestritten worden sind, oder besorgt, daß sie, wenn nicht bisher, doch vielleicht einst noch könnten bestritten werden, erachtet er es, und gewiß nicht mit Unrecht, für ein der Mühe lohnendes Unternehmen, zu zeigen, wie wir nicht nothwendig erst sie alle zugestehen müßten, um uns zu überzeugen, daß eine gewisse religiöse Lehre ihre unwiderlegliche Richtigkeit habe und ewig behalten werde, was für Entdeckungen die Zeit auch einst noch an das Licht bringen möchte. Hiebei bleibt jenen Sätzen der Grad der Verlässigkeit, welchen sie etwa besitzen, ganz ungeschmälert, auch ihre anderweitige Brauchbarkeit, wenn sie eine solche haben, wird ihnen nicht benommen. Wir mögen sie kennen lernen, wir mögen nach wie vor ihre Wahrheit behaupten und beweisen durch alle Gründe, die sich für sie auffinden lassen. Dieß Alles, sagen wir, bleibt uns künftig nicht nur verstattet, wie zuvor, sondern es kann und muß uns das Geschäft, den Grad ihrer Verlässigkeit auf eine der Wahrheit gemäße Art zu bestimmen, dadurch noch sehr erleichtert werden, daß wir sie unbefangener betrachten können, weil unsere Religion, \dh\  derjenige Theil unserer Überzeugungen, der alle übrigen an Wichtigkeit übertrifft, von dem Ergebnisse unserer Untersuchungen ganz unabhängig fest steht. \par
Wenn es, um diese allgemeinen Betrachtungen mit einem Beispiele zu schließen, welches gewiß das wichtigste ist, -wenn es nach B.'s Ansichten nicht unumgänglich nothwendig ist, sich von der historischen Glaubwürdigkeit der Bücher des neuen Bundes zuvor überzeugt zu haben, um von der Wahrheit des katholischen Christenthums überzeugt zu werden: dürfen wir darum wohl so weit gehen, der evangelischen Geschichte gar keine Wichtigkeit mehr zuzugestehen? Gewiß nicht; zu erfahren, was wir aus ihr allein erfahren können, wie der vollkommenste aus allen Sterblichen gesinnt gewesen, und wie er sich in den verschiedensten Verhältnissen des Lebens benommen habe, das ist und muß für jeden Gebildeten eine Sache von hoher Bedeutenheit bleiben. So wahr es also auch ist, daß \zB\ Straußens bekanntes Buch vom Leben Jesu, wie manches andere ältere Werk ähnlichen Inhaltes, \seitenw{231} uns nicht zu kümmern braucht, wenn wir nur eben mit Untersuchung der Wahr - oder Falschheit des Christenthums (zumal nach katholischer Auffaffungsweise desselben) beschäftiget sind: ferne sey es von uns, daß wir dergleichen Schriften mit Gleichgültigtigkeit betrachten und aufnehmen sollten; ferne, daß wir es nicht der Mühe werth halten sollten, solche zu prüfen, und mit dem größten Fleiße zu prüfen! Und wenn wir dieß mit aller derjenigen Gemüthsruhe thun, die nun eben der Gedanke möglich macht, daß unser religiöse Glaube auf keinen Fall, welches auch das Ergebniß unserer Prüfung wäre, erschüttert werden könne: wird uns die Wahrheit dann entgehen? --\par
\par


\endinput