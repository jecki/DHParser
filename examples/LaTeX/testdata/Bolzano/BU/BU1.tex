\clearpage\ifnackt\else\linenumbers\fi%
\thispagestyle{ctplain}\noindent\seitenwohne{19}\WLsec{I}{1}{120}
Der vollständige Titel des Werkes, von dessen wichtigsten Lehren wir unsern Lesern zuerst eine beurtheilende Uebersicht vorlegen wollen, lautet: \danf{Dr. B.\ Bolzanos Wissenschaftslehre. Versuch einer ausführlichen und größtentheils neuen Darstellung der Logik mit steter Rücksicht auf deren bisherige Bearbeiter. Herausgegeben von mehren seiner Freunde. Mit einer Vorrede des Dr. Ch. A. Heinroth.} (4 Bde. Sulzbach, in der J. E. v. Seidelschen Buchh. 1837.)\par
Das Erste nun, was unsern Augen bei Eröffnung dieses Buches nach gelesenen Vorworten der Verlagshandlung und der Herausgeber und nach der schon auf dem Titel erwähnten Vorrede des Hrn. Prof. Heinroth, begegnet, ist die Erklärung, die der Vf.\ \WLpar{I}{1} von seiner hier zu behandelnden Wissenschaft, der von ihm so genannten \danf{Wissenschaftslehre,} aufstellt. Er sagt, daß sie ihm eine Anweisung sey, wie das gesammte Gebiet des menschlichen Wissens in einzelne Wissenschaften zerlegt, und eine jede derselben in zweckmäßigen Lehrbüchern dargestellt werden solle. Dieses als eine Erklärung der Logik betrachtet, -- denn mit der Logik identificirt B.\ seine Wissenschaftslehre \WLpar{I}{3}, -- mag den Worten nach neu seyn, in der Sache selbst weicht es nur wenig ab von dem, was die berühmtesten Logiker von Aristoteles bis auf die neueste Zeit, -- etwa mit Ausnahme der sich selbst so nennenden \danf{speculativen Philosophen} unserer Tage, -- unter Logik sich vorgestellt haben. Unwidersprechlich ist wenigstens, und kann, ohne das Buch selbst noch zu lesen, aus einem bloßen Durchblicke seiner Inhaltsverzeichnisse entnommen werden, daß B.\ alle Gegenstände, die man bisher in der Logik abzuhandeln pflegte -- (Hegeln\pindex{Hegel, Gottfried Wilhelm} noch immer ausgeschlossen) -- auch in seinem Lehrbuche, größtentheils nur umständlicher bespricht. Jedoch auch selbst das Meiste von dem, was in den drei Bänden der Hegelschen\pindex{Hegel, Gottfried Wilhelm} Logik zur Sprache gebracht wird, berühret B., obwohl er es durchaus in einer anderen \seitenw{20} Weise behandelt, und fast immer etwas von Hegels Resultaten sehr Abweichendes herausbringt. In Hegels\pindex{Hegel, Gottfried Wilhelm} Logik wird im ersten Bande über die Frage, womit der Anfang der Wissenschaft zu machen sey, dann von der Qualität, vom Seyn, dem Nichts, dem Werden, dem Etwas, der Beschaffenheit, Veränderung, dem Sollen, der Negation, der Endlichkeit sowohl als Unendlichkeit, von Einheit und Vielheit, Quantität, Zahl, Größe, quantitativer Unendlichkeit, den Kantschen Kategorien; im zweiten Bande vom Wesen, Wesentlichen und Unwesentlichen, dem Scheine, der Reflexion, der Idealität und dem Unterschiede, dem Gegensatze, den Sätzen der Verschiedenheit, des ausgeschlossenen Dritten, des Widerspruches, des Grundes, von Form und Materie, Bedingtem und Unbedingtem, Existenz, Verhältniß, Möglichkeit, Wirklichkeit und Zufälligkeit, vom Gesetze der Causalität; im dritten Bande endlich vom Begriffe, dem allgemeinen, besondern und einzelnen; vom Urtheile, dem positiven, negativen und unendlichen; dem singulären, particulären und universellen; dem kategorischen, hypothetischen und disjunctiven; dem assertorischen, problematischen und apodiktischen; vom Schlusse, den vier Figuren, der Induction, Analogie, dem kategorischen, hypothetischen und disjunctiven Schlusse; von Zweck und Mittel, von der Idee, dem Individuo, und der Gattung; vom Begriffe des Wahren, vom analytischen und synthetischen Erkennen, von Definition, Eintheilung und Lehrsatz gesprochen: und siehe! all diese Gegenstände finden wir auch in B.'s Lehrbuche bald mehr bald minder ausführlich als bei Hegel\pindex{Hegel, Gottfried Wilhelm} erörtert. So sind es denn am Ende nur folgende Puncte: die Repulsion und Attraction, die continuirlichen und discreten Größen, das unendlich Kleine, die directen, indirecten und Potenzverhältnisse, das Maß, die Porosität, die Sollicitation der Kräfte, und endlich der Mechanismus, der Chemismus und das Leben, welche in Hegels\pindex{Hegel, Gottfried Wilhelm} Logik aufgeführt werden, bei B.\ aber unerwähnt bleiben. \seitenw{21}\par
Wurde er aber durch seinen eigenthümlichen Begriff von der Logik, wie wir nun sehen, keineswegs verleitet, Lehren bei Seite zu stellen, die man bisher in dieser Wissenschaft vortrug: so hat er vielleicht im Gegentheile dadurch an ihr sich versündigt, daß er eine Menge von Untersuchungen, die ihr ganz fremd sind, aufnahm! Diesen Anschein gewinnt es bei dem großen Umfange des Werkes von 153 1/2 Medianbogen wirklich gar sehr! Wie verhält es sich also hiemit in Wahrheit? Dieß zu erfahren verlangen unsere Leser billig, bevor sie uns noch in das Innere des Buches folgen, damit es von ihrem Gutdünken abhange, ob und welche Lehren desselben sie näher kennen lernen. \par
Wir müssen also berichten, daß B.'s Logik nebst einer Einleitung von 68 Seiten aus folgenden fünf Theilen bestehe.\par
\begin{compactenum}[1)]
\item Fundamentallehre, enthaltend den Beweis, daß es Wahrheiten an sich gebe, und daß wir Menschen auch die Fähigkeit, sie zu erkennen, haben. \item Elementarlehre, oder die Lehre von Vorstellungen, Sätzen, wahren Sätzen und Schlüssen an sich.
\item Erkenntnißlehre, oder von den Bedingungen, denen die Erkenntniß der Wahrheit, insonderheit bei uns Menschen, unterstehet.
\item Erfindungskunst, oder die Regeln bei dem Geschäfte des Nachdenkens, wenn es Erkenntniß der Wahrheit bezweckt.
\item Eigentliche Wissenschaftslehre, oder die Regeln bei der Zerlegung des gesammten Gebietes der Wahrheit in einzelne Wissenschaften und bei der Darstellung derselben in Lehrbüchern.
\end{compactenum}
Hier nun gestehen wir offen, daß gleich die Aufgabe, die der Verf. sich in seiner Fundamentallehre (Bd. I. S.\,69--212.)\WLS{I}{69--212} setzt, in einem Lehrbuche der Logik bisher noch niemals abgehandelt wurde; und aus diesem Umstande ziehe man, wenn man will, immerhin den Schluß, daß diese Aufgabe auch in alle Ewigkeit nicht in die Logik gehöre: dennoch wird man, um billig zu seyn, bei der Berechnung, wie viel das nicht hieher Gehörige betrage, alle diejenigen §§. aus\seitenw{22}scheiden müssen, in welchen gewisse zur Logik allgemein gezählte Begriffe, wie der eines Satzes, einer Wahrheit, eines Urtheils, eines Erkenntnisses, näher bestimmt, so wie auch den §., in welchem die in den bisherigen Lehrbüchern üblichen Denkgesetze betrachtet werden, wo denn zuletzt nicht mehr als die \WLpar[§§. 30--33.]{I}{30--33} und \WLpar[40--44.]{I}{40--44}, zusammen kaum 50 Seiten betragend, als ein \BUgriech{p'arergon} zu exscindiren seyn würden. \par
Der zweite, \dh\  der weitläufigste Theil des Buches, der von S.\,213 des ersten bis an das Ende des zweiten Bandes reicht, und 58 Bogen ausfüllt, bespricht durchaus nur Lehren, die auch in jeder anderen Logik unter derselben Ueberschrift: Elementarlehre, entweder schon wirklich vorkommen, oder, falls sie nur nicht als unwahr widerlegt werden können, künftig noch aufzunehmen seyn dürften. Denn wenn es nicht unrichtig wäre, was in den \WLpar{I}{198--222} über den objectiven Zusammenhang zwischen Wahrheiten an sich vorgebracht wird; wenn es nicht widerlegt werden könnte, daß die von \WLpar{I}{225--253} aufgeführten Schlußweisen sämmtlich wahre, eigenthümliche und zum Denken unentbehrliche Schlußweisen sind: müßte dann nicht in jedem ausführlicheren Lehrbuche künftig auch über diese Gegenstände etwas gesprochen werden? Zwar hat B.\ in diesen Theil, namentlich in die Lehre von den verschiedenen Arten der Vorstellungen und Sätze ein und das Andere aufgenommen, das seinem eigenen Geständnisse nach nicht eben nothwendig aufgenommen wurde. Allein wenn es wahr ist, was er unsers Erachtens sehr deutlich zeigt, daß sich die Grenze, wie weit man in Aufzählung jener Arten zu gehen habe, keineswegs so scharf abstecken läßt, als Manche es durch das gelehrte Zauberwort: Form, zu vermögen sich einbildeten: dann verdient es wohl keinen so strengen Tadel, daß er hier lieber ein Mehreres als zu wenig thun wollte; zumal wenn die Begriffe, die er bei solcher Gelegenheit erklärte, von einer solchen Wichtigkeit für mehrere Wissenschaften sind, wie die Begriffe von Zeit und Raum \WLpar{I}{79}, von Mengen, Summen und Größen \WLpar{I}{84}, von einer Reihe \WLpar{I}{85}, vom Gegensatze \WLpar{I}{107}, und von einem sittlichen Satze \WLpar{I}{144} Diese Beispiele verbunden mit der \WLpar{I}{143} geschehenen Erwähnung von Sätzen, die eine psychische Erscheinung \seitenw{23} aussagen, dürften so ziemlich Alles seyn, was aus dem Grunde, weil es vielleicht ein für die allgemeine Logik zu spezielles Thema ist, gestrichen werden könnte. \par
Etwas Ungewöhnliches in einem Lehrbuche der Logik ist freilich auch ein Theil mit der Überschrift: \danf{Erkenntnißlehre} (Bd.\,III. S.\,1--292). So nennen Einige vielmehr eine eigene Wissenschaft, die sie der Logik nachfolgen, der Metaphysik aber unmittelbar vorhergehen lassen. Will man sich jedoch nicht an Namen, sondern an die Sache selbst halten, so sieht man, daß die Untersuchungen, welche B.\ in diesem dritten Theile vornimmt, mit Ausschluß zweier (\WLpar{III}{303--5}. und \WLpar[314.~5.]{III}{314--315}) in hundert andern Logiken gleicherweise zu treffen sind. Von der Untersuchung \WLpar[\BUparformat{303--5}]{III}{303--305} (über die Art, wie wir diejenigen unserer Erfahrungsurtheile vermitteln, die uns wie unvermittelte erscheinen) bekennt B.\ selbst, daß man sie überschlagen könne. Was aber die \WLpar[\BUparformat{314.5}]{III}{314--315} oder die Frage über die Grenzen unsers Erkenntnißvermögens anlangt: so begreifen wir wohl, daß man sie ihrer Wichtigkeit wegen seit Kant zu dem Gegenstande einer eigenen Wissenschaft erhoben; begreifen ebenfalls, daß man schon wegen der schwerern Faßlichkeit der hiezu nöthig geglaubten Untersuchungen, zum Theile vielleicht auch wegen des immer noch strittig erschienenen Ergebnisses derselben ein Bedenken getragen habe, selbe der Logik, \dh\  derjenigen philosophischen Wissenschaft einzuverleiben, welche man einmal nur für Anfänger bestimmte, welcher man überdieß den Ruhm, daß in ihr Alles nicht minder unbestritten wie in der Mathematik wäre, zu bewahren suchte. Wer aber solche Rücksichten bei Seite setzen kann -- und wir sollten meinen, daß man sie wenigstens an gewissen Orten bei Seite setzen dürfe --: wird der es wohl verkennen, daß die Frage über die Grenzen unseres Wissens recht eigentlich in die Logik gehöre, auch wenn sie noch gar nicht uns eine Wissenschaftslehre, sondern nur eine einfache Anweisung zum richtigen Denken seyn soll? Denn wie doch dürften wir uns berühmen, richtig denken zu lehren, wenn wir nicht lehren, ob es gewisse und welche Gegenstände es gebe, über die man kein richtiges Urtheil zu fällen vermag, wie sehr man auch alle sonstigen Regeln des Denkens befolge; \seitenw{24} weil sie die Grenze unsers Erkenntnißvermögens selbst ganz und gar überschreiten? --\par
Der so eben erwähnte Begriff der Logik, daß sie eine Denklehre sey, ist bekanntlich der herrschende: daraus ergibt sich aber unmittelbar, daß B.'s vierter Theil, die Erfindungskunst \Druckfehlerkorr{(Bd.}{Bd.}\,III. S.\,293 bis Ende), enthaltend die Regeln für das Geschäft des Denkens, wenn die Erkenntniß neuer Wahrheiten bezweckt wird, im Grunde nur den Gegenstand behandle, welchen man als die eigentliche von der Logik zu lösende Aufgabe ansieht. Über das also, was B.\ in diesem vierten Theile vorträgt, kann Niemand klagen, daß es nicht in die Logik gehöre; es wäre denn, er wollte einwenden, -- was höchstens bei einer Untersuchung von ein paar Seiten (über die Kennzeichen einer göttlichen Offenbarung) im \WLpar{III}{388} der Fall seyn könnte, -- daß es nicht genug Allgemeinheit habe. \par
Ganz anders verhält es sich mit dem letzten Theile, dem der Vf.\ die Überschrift: \danf{Eigentliche Wissenschaftslehre} gegeben. Denn weil bisher die wenigsten Logiker mit ausdrücklichen Worten erklären, daß der Zweck der Logik, ihr letzter nämlich, ein höherer sey als nur überhaupt richtig denken zu lehren: so muß gerade der Theil, in welchem B.\ das Ziel aller übrigen Untersuchungen sieht, die Lehre von der Zerlegung des gesammten menschlichen Wissens in einzelne Wissenschaften, und die Bearbeitung derselben, ihnen als ein Auswuchs erscheinen. Hierüber nun wollen wir mit Niemand rechten; also auch gar nicht die Frage aufwerfen, ob es nicht doch eine Wissenschaft, welche die Regeln zur Bildung und Bearbeitung aller übrigen aufstellt, in der That geben solle? nicht fragen, welcher Wissenschaft man dieß Geschäft füglicher als der Logik auftragen könne? nicht einmal untersuchen, ob nicht gar Manches, das man bisher schon in die Logik aufgenommen, nur dadurch in ihr am rechten Orte erscheine, daß man ihr jenen Zweck stillschweigend unterlegt hat? Dieß Alles lassen wir fallen, und geben Jedem, der einmal schlechterdings nicht will, daß Logik eine Wissenschaftslehre seyn dürfe, die Erlaubniß, den ganzen vierten Band ungelesen zu lassen. Auch so noch hoffen wir, daß er des Neuen und Belehrenden in dem Werke gar Manches antreffen werde. \seitenw{25}\par
Nächst dem Begriffe, den B.\ von seiner Wissenschaft aufstellt, verdienet aus der Einleitung als etwas Auffallendes hervorgehoben zu werden, auf welche Weise er \WLpar{I}{12} die Redensart, daß die Logik eine bloß formale Wissenschaft seyn solle, auslegt; nämlich nur so, daß sie keine einzelnen, sondern bloß ganze Arten von Sätzen betrachte. Auch diese Auslegung mögen die Leser verwerfen, und um ihrentwillen schon im voraus besorgen, daß der Vf.\ durch eine solche Ansicht im Verfolge zu vielen Mißgriffen werde verleitet werden. Immerhin; wir verlangen nur, daß sie erst den Erfolg selbst abwarten, und bei jeder einzelnen Lehre, die sie verwerfen wollen, den Grund, aus welchem sie dieses Verwerfungsurtheil aussprechen wollen, zu einem deutlichen Bewußtseyn sich erheben. \par
Bemerkenswerth ist noch, daß \WLpar{I}{13} die Logik für keine ganz unabhängige Wissenschaft erklärt wird; weil sie, wenn sonst aus keiner anderen Wissenschaft, doch aus der Psychologie Hülfssätze entlehne. Dieses nun kann wohl nicht anstößig seyn, wenigstens Niemanden, dem die Logik eine Denklehre seyn soll, oder der jedenfalls zugibt, daß in derselben gesprochen werden dürfe von einer Verknüpfung unserer Vorstellungen im Gemüthe, von der Nothwendigkeit ihrer Bezeichnung, von klaren und dunkeln, deutlichen und verworrenen Vorstellungen \udgl\ Denn sind nicht alles dieß psychologische Lehren? \par
Über den Plan endlich, den der Vf.\ \WLpar{I}{15} für seinen Vortrag entwirft, brauchen wir nach demjenigen, was unsere Leser hierüber bereits erfahren haben, nichts Weiteres beizufügen. Denn das Einzige, was hier noch einer besondern Rechtfertigung bedürfte, nämlich die Unterscheidung, die B.\ zwischen Sätzen und Vorstellungen an sich und gedachten Sätzen und Vorstellungen gemacht wissen will, und sein Begehren in der Elementarlehre nur die ersteren, die letzteren aber erst in der Erkenntnißlehre zu betrachten, wird sich in dem gleich Folgenden durch die Erklärung dieses Unterschieds von selbst erledigen. \seitenw{26}\par
Beginnen wir also nun mit dem ersten Theile, \dh\  mit der Fundamentallehre, wo der Vf.\ in dem ersten Hauptstücke den Satz, daß es Wahrheiten an sich gebe, und in dem zweiten den Satz, daß auch wir Menschen einige derselben erkennen, darzuthun sucht. Begreiflicher Weise fand er hier nöthig, den Lesern erst recht deutlich zu erkennen zu geben, was er unter Sätzen und Wahrheiten an sich, die er zuweilen auch objective nennt, im Gegensatze von gedachten Sätzen und Wahrheiten verstehe. Dieser Verständigung sind denn \WLpar{I}{19}, \WLpar[24--26.]{I}{24--26} gewidmet, und \WLpar{I}{21} u. \WLpar[27.]{I}{27} zeigen, daß auch schon Andere diese Begriffe gekannt und auf ähnliche Weise bezeichnet. Je gewisser dieß Letztere ist, um so mehr hoffen wir, es werde hier genug seyn, zu sagen, daß B.\ unter Sätzen und Wahrheiten an sich durchaus nichts Anderes sich denke, als was wir Alle uns bei diesen Worten denken, wenn wir sie etwa in folgenden Redensarten gebrauchen; wenn wir \zB\ die Frage aufwerfen, ob eine jede Wahrheit von irgend einem Wesen erkannt, jeder Satz, auch jeder falsche sogar, von irgend Jemand gedacht oder vorgestellt werde; oder wenn wir bemerken, daß durch den Unterricht in der Geometrie der pythagorische Satz nicht an sich selbst, wohl aber der Gedanke an ihn, die Erkenntniß desselben in unsern Gemüthern vervielfältiget werde; oder endlich auch, wenn wir sagen: \danf{Gäbe es kein einziges denkendes Wesen, so wäre der Satz, daß es kein einziges denkendes Wesen gibt, selbst eine Wahrheit.} Offenbar ist es nämlich, daß wir in allen diesen Fällen unter den Worten: Satz und Wahrheit etwas ganz Anderes als eine Art von Gedanken verstehen, daß wir vielmehr nur etwas Solches verstehen, was durch ein Denken erst aufgefaßt werden, \dh\  den Stoff eines Gedankens ausmachen kann. Eben so offenbar ist aber auch, daß wir uns unter den Sätzen und Wahrheiten hier (und wohl nirgends) die Sachen selbst, von welchen in ihnen gesprochen wird, denken; denn sonst müßte man sagen, daß der Satz: es gibt einen Gott, Gott selbst, und der Satz: es gibt kein Einhorn, etwa das Einhorn, oder das Nicht-Einhorn selbst sey. \par
Gerade nur so, wie in diesen Redensarten, will nun B.\ die \seitenw{27} Worte: Satz und Wahrheit aufgefaßt wissen. Obgleich also, wie jene Beispiele zeigen, die von ihm aufgestellten Begriffe selbst im alltäglichen Leben vorkommen: so ist doch nur zu wahr, daß man sie in den Schriften der Weltweisen noch nie einer recht umständlichen Betrachtung gewürdigt; in älteren Werken sich gewöhnlich mit einer bloßen Erwähnung derselben begnüget hat, in den neueren sie ganz mit Stillschweigen übergehet, obgleich man nicht umhin kann, sich ihrer gelegenheitlich (wahrscheinlicher Weise, ohne sich dessen selbst deutlich bewußt zu seyn) doch zu bedienen. Um ein einziges Beispiel hievon aus den Schriften eines der angesehensten Philosophen unserer Zeit anzuführen: Hr.~Prof.~I.~H.~Fichte\pindex{Fichte, Immanuel Heinrich} legt in seinem Werke: Gegensatz, Wendepunkt und Ziel heutiger Philosophie (Heidelberg, 1832. Bd. I. S.\,133) folgendes unsere ganze hierortige Untersuchung bestätigende Geständniß ab: \danf{Wenn wir dort} (schreibt er, indem er eine \DruckVariante{Äußerung}{Aüßerung} von Franz Baader beurtheilt) \danf{vernehmen, daß Glaube der einzige Anfang zur rechten Erkenntniß sey}, so kann dieß im Sinne der Wissenschaft nur die \danf{erste formale Voraussetzung alles Erkennens bedeuten, daß es überhaupt Wahrheit gebe, und daß sie uns zugänglich sey.} Was man unter \danf{Wahrheit} versteht, wenn man behauptet, daß sie dem Menschen \danf{zugänglich} sey, kann durchaus nichts Anderes seyn als die von B.\ sogenannte objective Wahrheit oder die Wahrheit an sich. \par
Es fragt sich nun, ob auch unsere Leser mit diesem Begriffe B.'s sich befreunden, und demselben Gegenständlichkeit zugestehen können, \dh\  ob sie zugestehen können, daß es dergleichen Sätze und Wahrheiten an sich gebe? Bejahen sie dieß, dann ist der Zweck seines ersten Hauptstücks erreicht; denn ob sie die von ihm \WLpar{I}{28} versuchte Zergliederung oder Erklärung dieses Begriffes richtig oder nicht richtig finden, darauf kommt gar nichts an. Und eben so wenig darauf, ob sie durch seinen Beweis, daß es der Wahrheiten an sich unendlich viele gibt (\WLpar{I}{32}), befriediget werden. Denn hievon werden sie sich ein Jeder auf manche andere Weise leicht überzeugen, sobald sie nur einmal nicht zweifeln, \seitenw{28} daß es wenigstens eine oder etliche Wahrheiten gebe. Vermögen sie aber jenen Begriff nicht anzuerkennen, dann sind sie zwar noch eben nicht bemüssiget, Alles, was in der Folge vorgetragen wird, zu verwerfen, weil gleich das Erste ihnen nicht einleuchten will: aber sie werden doch Vieles und Wichtiges nicht annehmen können. Was eigentlich? das wollen wir unseren Lesern in der Folge überall genau bemerklich machen. \par
B.'s einfacher Vorgang in seinem Beweise ist aber dieser. Daß es der Wahrheiten wenigstens einige gebe, erweiset er -- nach einem schon von Andern gebrauchten Schlusse -- daraus, weil die entgegengesetzte Behauptung, daß nämlich nichts wahr sey, sich selber aufhebt. Daß es aber der Wahrheiten mehrere, ja unendlich viele gebe, folgert er daraus, weil jede endliche Menge von Wahrheiten A, B, C, -- wenn sie in einen einzigen Satz zusammengefaßt wird, eine neue von jeder einzelnen verschiedene Wahrheit, nämlich die Wahrheit: \danf{der Inbegriff der Sätze A, B, C ist ein Inbegriff von lauter wahren Sätzen,} -- darbeut. Daß diese neue Wahrheit nicht eben von einer besonderen Merkwürdigkeit sey, wird er selbst zugestehen; aber genug, wenn Jeder anerkennen muß, daß dieser Satz ein neuer, von jedem der einzelnen A, B, C verschiedener Satz sey; denn dadurch allein wird schon die Meinung, daß die Menge aller Wahrheiten eine bloß endliche sey, widerlegt. \par
Für Philosophen, die sich an eine ganz entgegengesetzte Ansicht der Dinge gewöhnt, wird es allerdings schwer halten, sich von der Richtigkeit einer so einfachen Darstellung, und insbesondere auch davon, daß auf diese Art der Anfang des Philosophirens gemacht werden müsse, zu überzeugen. Gerade weil die Sache so einfach ist, und selbst dem gemeinen Menschenverstande so zusagt, wird man nur ungern anerkennen, daß sich auch jeder Weltweise mit dieser Darstellung befriedigen könne und müsse. Namentlich alle Jene, welche sich einbilden, es könne keinen andern, weiter ausholenden Anfang des Philosophirens als einen psychologischen geben, werden sie wohl geneigt seyn, zuzugestehen, daß B.'s Standpunkt ein höherer sey? Und doch ist es so. Denn \seitenw{29} Untersuchungen über unser Erkennen, psychologische Betrachtungen können wir nur erst anstellen, nachdem wir -- laut oder stillschweigend, gilt gleich -- vorausgesetzt haben, daß es Wahrheiten an sich gebe, und daß wir Urtheilenden im Stande sind, derselben einige zu erkennen. Es ist also vollkommen so, wie Fichte\pindex{Fichte, Immanuel Heinrich} in der oben angezogenen Stelle gesagt hat: \danf{Die erste formale Voraussetzung alles Erkennens ist, daß es überhaupt Wahrheit gebe, und daß sie uns zugänglich sey.} Diese Voraussetzung selbst aber, und zwar zuerst nur der Satz, daß es Wahrheiten überhaupt gebe, können wir recht füglich einsehen, und müssen sie einsehen können, ohne in unsrer Betrachtung vorauszusetzen, \dh\  als Prämisse zu gebrauchen den Satz, daß wir erkennende Wesen sind; obgleich freilich nicht, ohne dergleichen zu seyn. Der Unterschied zwischen diesem Beiden: daß ein gewisser Satz in unseren Betrachtungen als Prämisse gebraucht werden müsse, wenn wir zur vollen Überzeugung eines Schlußsatzes gelangen sollen, und daß dasjenige, was in diesem Satze ausgesagt wird, \Druckfehlerkorr{als}{\glqq als} Bedingung nothwendig sey, wenn wir diese Betrachtungen mit dem Erfolge der Überzeugung anstellen sollen, dieser Unterschied ist, dächte ich, nicht zu verkennen. Zu allem Philosophiren ist als Bedingung unerläßlich ein Zustand des Wachens \umA\ Wer aber wird sagen, daß wir bei allem Philosophiren von der Prämisse (dem Princip), daß wir jetzt wach seyen \usw\ ausgehen müßten? \par
Indessen müssen wir unsere Leser doch auch auf einen scheinbaren Einwurf gegen B.'s in diesem Hauptstücke zu erweisende Behauptung aufmerksam machen. Er gestehet selbst, daß seine Wahrheiten an sich, gerade so wie auch seine Sätze an sich durchaus nichts Wirkliches, nichts Existirendes seyen; nur der gedachte Satz, das Urtheil, die erkannte Wahrheit hätten als Erscheinungen in dem Gemüthe eines denkenden Wesens ihre auf eine gewisse Zeit beschränkte Wirklichkeit, nicht aber der Satz oder die Wahrheit an sich. Wenn diese nun aber nichts Wirkliches sind, wie sind sie doch überhaupt Etwas? wie ist die Redensart, es gebe Sätze und Wahrheiten an sich, zu deuten? und \seitenw{30} wie vollends ist die diesem Hauptstücke gegebene Überschrift: \danf{Vom Daseyn der Wahrheiten an sich,} zu rechtfertigen? -- Hierauf erwiedern wir nun: Nicht jegliches Etwas hat und muß Wirklichkeit haben. Denn sprechen wir nicht allgemein auch von Dingen, die in der bloßen Möglichkeit bestehend, noch keine Wirklichkeit haben, ja auch von solchen, die einmal zur Wirklichkeit gelangen? Es ist also offenbar falsch, daß das Nichtwirkliche Nichts sey. Somit kann in demselben Sinne gesagt werden, es gebe Wahrheiten an sich, obgleich diese Wahrheiten nichts Wirkliches sind, in welchem gesagt werden kann, es gebe Möglichkeiten, welche nichts Wirkliches sind. Das Erste hat nämlich nur den Sinn, daß der Begriff einer Wahrheit an sich -- und das letzte nur den Sinn, daß der Begriff einer Möglichkeit -- Gegenständlichkeit habe. Die Sprache erlaubt sich indessen auch von demjenigen, was keine Wirklichkeit hat, in Ausdrücken zu sprechen, die eigentlich nur für das Wirkliche gehören. So sagt man auch von dem, was bloße Möglichkeit hat, es sey, nämlich es sey möglich. Und nur in diesem Sinne erlaubte sich der Vf.\ in jener Überschrift den Ausdruck: Vom Daseyn der Wahrheiten an sich. Gewiß aber hätte er besser gethan, wenn er einen so leicht zu mißverstehenden Ausdruck durchaus vermieden hätte. \par
Bevor wir dieß erste Hauptstück verlassen, glauben wir unseren Lesern noch die Erklärung, die der Vf.\ von dem Begriffe der Wahrheit (an sich) versucht, vorlegen zu sollen; wäre es auch nur, damit sie um so leichter die Unrichtigkeit einer andern Erklärung dieses Begriffes, die viele Irrungen veranlaßt hat, erkennen. B.\ sagt also, daß nur Sätze, nicht aber bloße Vorstellungen in wahre und falsche eingetheilt werden können; er bemerkt ferner, daß jeder Satz einen Gegenstand, von dem er handelt, habe, und daß er diesem Gegenstande (den er in seiner Subjectvorstellung vorstellt) eine Beschaffenheit (vorgestellt in seiner Prädicatvorstellung) beilege. Wahr heißt nun dieser Satz, wenn er dem Gegenstande eine Beschaffenheit beilegt, die diesem zukommt; und falsch, wenn er dieß nicht thut. Diese Erklärung ist (\DruckVariante{däucht}{daücht} uns) wenigstens sehr verständlich; auch \seitenw{31} dürfte Jeder zugeben, daß der zu erklärende Begriff in keinem ihrer Bestandtheile noch unzerlegt stecke. Wenn man dagegen, wie sehr gewöhnlich ist, sagt, daß die Wahrheit eine Übereinstimmung unserer Vorstellungen mit ihren Gegenständen sey: begehet man da nicht einmal schon den Fehler, daß man die Wahrheit in einer Vorstellung sucht, während sie eigentlich doch nur den Sätzen zukommt? und ist man wohl im Stande, uns einen deutlichen Begriff davon zu geben, was man sich unter der hier verlangten Übereinstimmung denke? Nachdem man vergeblich sich abgemühet hat, etwas Anderes anzugeben, wird man zuletzt genöthigt seyn, zu sagen, daß eine Übereinstimmung zwischen unsrer Vorstellung und ihrem Gegenstande herrsche, wenn wir dem Gegenstande Beschaffenheiten beilegen, welche er wirklich hat; was auf B.'s Erklärung \DruckVariante{hinausläuft}{hinauslaüft}. Und so sollte man denn die Meinung, als ob eine gewisse Ähnlichkeit herrschen müsse zwischen der Vorstellung und ihrem Gegenstande, und die hieraus entsprungene Frage, wie man sich von dem Vorhandenseyn dieser Ähnlichkeit überzeugen könne? eine Frage, welche man nie sich zu beantworten vermochte, ohne in neue Irrthümer zu gerathen, \zB\ daß Vorstellung und Gegenstand mit einander identisch seyn müßten \udgl\ -- doch einmal fahren lassen. Die Vorstellung, falls sie erst eine gegenständliche ist, stellt ihren Gegenstand vor, ohne irgend eine Ähnlichkeit mit demselben zu haben, die größer wäre, als je zwei der verschiedenartigsten Dinge mit einander haben. Ob aber eine gegebene Vorstellung zu den gegenständlichen gehöre, und welche Gegenstände sie habe, das Alles erkennen wir durch Schlüsse einer ganz anderen Art, in deren Auseinandersetzung wir uns jetzt freilich nicht einlassen können. \par
Nicht mehr von Wahrheiten an sich, sondern von unsrer Erkenntniß derselben, also von Urtheilen ist in dem zweiten Hauptstücke die Rede, und es soll dargethan werden, daß wenigstens einige unserer Urtheile wahr, \dh\  Erkenntnisse sind. Wer das im voraus zugibt, mag dieses Hauptstück nur überschlagen; wer aber \seitenw{32} schon zuweilen bei sich eine Anwandlung verspürt hat, an Allem zu zweifeln, der lese, und wird vielleicht Manches für ihn Befriedigende finden. Manches, sagen wir; denn daß ihm Alles, was etwa in dem Zweigespräche \WLpar{I}{42} vorkömmt, verständlich seyn sollte, erwarten wir nicht, da hier Verschiedenes berührt wird, was erst in den späteren Theilen des Werkes sein volles Licht erhält. Indeß wird Jeder mindestens so viel einsehen lernen, daß man nur Eines von Beiden thun müsse, entweder sich durchaus alles Urtheilens enthalten, oder zugeben, daß man doch eine und die andere Wahrheit erkenne; weil man im Gegenfalle sich nur selbst widerspricht. \par
Wer auch das \WLpar{I}{43} angegebene Kriterium der wahren Urtheile liest, dem rathen wir, auf den Unterschied des hier Gesagten von der gewöhnlichen Lehre (vergl. \WLpar{I}{44} u. \WLpar{III}{347} Anm.) zu achten. Es ist hier nämlich durchaus von keiner unmittelbar zu erkennenden Nothwendigkeit eines Urtheils (die nie ein Gegenstand unsrer Wahrnehmung seyn kann) die Rede; sondern lediglich davon, was sich recht füglich beobachten läßt, ob mehrere von uns angestellte Versuche, das Urtheil abzuändern, immer erfolglos geblieben sind. Von solchen Urtheilen wird hier gesagt, daß sie Verlässigkeit hätten. Und wer kann dieß wohl in Abrede stellen? Was endlich noch anhangsweise \WLpar{I}{45} über die in andern Lehrbüchern vorkommenden obersten Denkgesetze gesagt wird, mag sehr verschiedentlich beurtheilet werden; es ist ohne allen Einfluß auf das Nachfolgende, da der Vf.\ sich nirgends wieder darauf bezieht, auch jene Gesetze nicht etwa \DruckVariante{läugnet}{laügnet}, sondern nur sagt, daß es nicht bloße subjective Gesetze unsers menschlichen Denkens, vielmehr objective Wahrheiten seyen, obgleich auf keinen Fall solche, die als die obersten Gründe von allen übrigen betrachtet werden könnten. Aus mehreren schlagenden Bemerkungen, die in gedrängtem Raume hier auf einander folgen, wollen wir unseren Lesern nur diese zwei ausheben, daß der Satz des ausgeschlossenen dritten ganz allgemein, nicht (wie Einige sagten) von bloß möglichen oder (wie Andere wollten) von logischen, sondern von allen Gegenständen gelte; und daß der Satz des Widerspruchs von Schelling \uA\ mit \seitenw{33} Unrecht bestritten worden sey, weil, was diese einen Widerspruch nennen, nur höchst uneigentlicher Weise so genannt werden kann. \par
\gliederungslinie\par
Wir wenden uns nun sofort zu demjenigen Theile des Werkes, welcher die meisten und wichtigsten Eigenthümlichkeiten in B.'s logischen Ansichten darbeut, nämlich zur Elementarlehre, die in vier Hauptstücken: von den Vorstellungen an sich (Bd. I. S.\,215--Ende), von den Sätzen an sich (Bd. II. S.\,3--326), von den wahren Sätzen (S.\,327--390) und von den Schlüssen (S.\,391--Ende) handelt. \par
Was B.\ unter Vorstellungen an sich oder auch objectiven Vorstellungen im Gegensatze zu gedachten oder subjectiven verstehe (Hauptst. I. Abschn. I.), kann Niemand schwierig finden, der einmal den Begriff von Sätzen oder Wahrheiten an sich gefaßt hat. Wie nämlich jeder gedachte Satz aus verschiedenen Theilen zusammengesetzt ist, die wir gedachte Vorstellungen nennen (einer Vorstellung von jenem einen oder von jenen mehreren Gegenständen, worüber in dem Satze etwas ausgesagt wird, einer Vorstellung von der Beschaffenheit, die jenem Gegenstande beigelegt wird \udgl ): so hat auch jeder objective Satz gewisse den erstern entsprechende Theile, die B.\ objective Vorstellungen nennet. Wie jedem gedachten Satze ein Satz an sich entspricht, der in jenem eben gedacht wird, so auch entspricht jeder gedachten Vorstellung eine Vorstellung an sich, welche in ihr unserm Gemüthe erscheint. -- Auch dieser Begriff der objectiven Vorstellungen ist nicht neu, und kommt zwar seltener, doch oft genug selbst in der Sprache des gemeinen Lebens vor, wo wir zu seiner Bezeichnung bald das Wort Vorstellung schlechtweg, bald und noch lieber das Wort Begriff gebrauchen. Ist es \zB\ wohl eine ungewöhnliche Rede, \danf{daß die Vorstellung oder der Begriff, den die Ästhetiker mit dem Worte Schönheit verbinden, ein und derselbe sey mit dem, was schon die Griechen bei ihrem \BUgriech{k'allos} sich dachten}? Hier nun verstehen wir offenbar unter den Worten: Vorstellung oder \seitenw{34} Begriff nicht eine Vorstellung in der subjectiven Bedeutung, nicht eine in dem Gemüthe eines denkenden Wesens zu einer bestimmten Zeit stattfindende Erscheinung: denn wie könnten wir da von einem Begriffe in einfacher Zahl reden? müßten wir uns nicht vielmehr etwa so ausdrücken: \danf{Die Vorstellungen, die unsre Ästhetiker haben, wenn sie das Wort Schönheit aussprechen, und die Vorstellungen, die das Wort \BUgriech{k'allos} einst bei den Griechen anregte, sind gleiche Vorstellungen}? Sprechen wir statt von mehreren von einer einzigen Vorstellung, die allen jenen denkenden Individuen vorgeschwebt habe oder noch vorschwebe: so ist klar, daß wir uns unter dieser Vorstellung nicht die Erscheinungen im Gemüthe, deren es mehrere gab, sondern nur ein diesen Erscheinungen entsprechendes, in ihnen aufgefaßtes Etwas, \dh\  die objective Vorstellung denken. Da der so eben berührte Unterschied zwischen Vorstellungen und Sätzen an sich und zwischen gedachten Sätzen und Vorstellungen (daß diese etwas Wirkliches, jene nichts Wirkliches sind) gewiß nicht der einzige ist; da \zB\ sicherlich nur gedachte Vorstellungen und Sätze sich eintheilen lassen in klare und dunkle \usw : so wird man sich bald auch damit befreunden, daß unser Vf.\ nach seinem Plane die Sätze und Vorstellungen in der objectiven Bedeutung gesondert von den gedachten Sätzen und Vorstellungen betrachtet wissen will, für jene die Elementarlehre festsetzt, diese in die Erkenntnißlehre verweiset. Man wird ihm, sagen wir, dieses verstatten, auch wenn man noch gar keine Ahnung davon hat, zu welchen wichtigen Aufschlüssen über die Natur unsers Erkennens ihn die Festhaltung dieses Unterschiedes verhelfen werde. \par
Wer aber schon den Begriff von Sätzen an sich verwarf, der freilich kann noch weniger den der Vorstellungen an sich zugeben; doch müssen wir anmerken, daß selbst ein Solcher alles dasjenige, was in dem nächstfolgenden Abschnitte (\glqq Innere Beschaffenheiten und Unterschiede der \Druckfehlerkorr{Vorstellungen\grqq}{Vorstellungen}\ S.\,237--428) und mehr noch das, was in den späteren Abschnitten gelehrt wird, auf seine Weise verstanden, nämlich von Vorstellungen, die im Gemüthe eines \seitenw{35} denkenden Wesens erscheinen, annehmen könne bis auf den einzigen Satz \WLpar{I}{54}: \danf{Vorstellungen an sich haben kein Daseyn;} indem er den Vorstellungen in seiner Bedeutung allerdings ein Daseyn zugestehen wird. Bevor wir jedoch dieß Alles näher betrachten, müssen wir, um recht vollständig aufzufassen, welche Begriffe B.\ mit den Worten: subjective und objective Vorstellung verbunden wissen will, noch zwei Bemerkungen nachtragen: erstens, daß er nur einen solchen Theil eines Satzes, der nicht schon für sich selbst ein ganzer Satz ist, und zweitens, daß er jeden dergleichen Theil, wie er sonst immer beschaffen seyn möchte, Vorstellung nenne. Die Zweckmäßigkeit dieser beiden Bestimmungen ist (\DruckVariante{däucht}{daücht} uns) für den Gebrauch der Logik (und außerhalb ihrer beobachtet B.\ sie selbst nicht) sehr einleuchtend. Andern beliebt es gleichwohl eine ganz andere Sprache zu reden; einerseits wollen sie auch vollständige Sätze als bloße Vorstellungen, wohl gar als einfache Anschauungen betrachtet wissen; andrerseits wollen sie wieder nicht eine jede Vorstellung, sondern nur eine solche, die sich auf einen so eben abwesenden Gegenstand beziehet, oder von uns (vermittelst eines Urtheiles) auf einen solchen nur bezogen wird, wohl gar nur eine Vorstellung, die von uns irriger Weise auf einen Gegenstand bezogen wird, welchen sie in der That nicht vorstellt, oder endlich nur eine Vorstellung, die mit einem gewissen sinnlichen Bilde begleitet ist, im Gegensatz mit dem Begriffe, für eine eigentliche Vorstellung erklären. Um die Verwirrung zu vollenden, geben sie diesen Vorstellungen, ja auch selbst den Begriffen Bewegung, \dh\  sie lassen sich selbe allmählich verändern, selbst in ihr Gegentheil umschlagen. Daher denn, daß sie \zB\ von Einem und eben demselben Begriffe nicht Eine, sondern mehrerlei Definitionen glauben angeben zu dürfen, deren die eine sein Wesen immer noch wahrer als die andere darstellen soll \usw\ Es ist nun freilich wahr, daß gedachte Vorstellungen, weil sie eigentlich nur gewisse Veränderungen in unserm Innern sind, auch selbst wieder der Veränderung unterliegen; und nur allzuwahr, daß wir oft, ohne uns dessen auch nur bewußt zu seyn, von der einen zur andern übergehen; daß wir selbst solche Vor\seitenw{36}stellungen, die wir als Prädicate auf denselben Gegenstand beziehen (\zB\ dieses ist etwas Löbliches, Tadelnswerthes \udgl ), allmählich abändern und zwar so abändern, daß unser zweites Urtheil das gerade Gegentheil von dem ersten aussagt: doch eben so wahr und einleuchtend ist es auch, daß wir, wenn wir uns selber Rechenschaft über unser Denken ablegen, vor Verwirrung und Irrthum uns möglichst sicher stellen, um so mehr, Andre sogar unterrichten wollen, jenes Übergehen des einen Gedankens in den andern zwar eben nicht zu verhindern, wohl aber erst dadurch ersprießlich zu machen uns bestreben müssen, daß wir es uns zum Bewußtseyn bringen, jeden dieser Gedanken durch ein bestimmtes Wort bezeichnen, und diesen Worten unwandelbare Bedeutungen geben. Diese Bedeutungen nun, sind sie etwas Anderes als B.'s objective Vorstellungen? Daß er somit den objectiven Vorstellungen keine Bewegung zugestehen will, und daß er auch von einer subjectiven Vorstellung, die einen andern Gegenstand vorstellt, oder aus anderen Theilen bestehet, oder überhaupt einer andern Vorstellung an sich entspricht, nicht sagen will, daß sie noch immer die nämliche sey, die sich nur fortbeweget habe, dürfte wohl keinen Tadel verdienen. Daß er aber das Wort Vorstellung in einer so weiten Bedeutung nimmt, daß jeder Bestandtheil eines Satzes, der noch selbst kein ganzer Satz ist, eine Vorstellung heiße, doch auch nicht weiter ausdehnet: das rechtfertigt sich wohl zur Genüge dadurch, daß diese Gattung von Dingen gewiß einer eigenen Bezeichnung werth ist, die Sprache aber kein anderes passendes Wort dafür hat. \par
Hiernächst nun werden wir ohne Mühe die meisten in dem bereits erwähnten zweiten Abschnitte behaupteten Lehren begreifen. Daß Vorstellungen weder wahr noch falsch seyen (\WLpar{I}{55}), ergibt sich unmittelbar aus ihrem Begriffe, wenn man ihn, wie B.\ oben gethan hat, bestimmt; und gilt in gleicher Weise, man mag an objective oder subjective Vorstellungen denken; denn eben weil die Vorstellung kein ganzer Satz ist, also nichts aussagt, kann sie auch weder wahr noch falsch seyn. Behaupteten Manche gleichwohl das \seitenw{37} Gegentheil: so kann dieß nur entweder, weil sie auch ganze Sätze zu den Vorstellungen zählten, oder (was \DruckVariante{häufiger}{haüfiger}) an einen bestimmten Gegenstand dachten, auf welchen die in Rede stehende Vorstellung passen soll, also sie eigentlich durch einen Satz auf ihn bezogen, oder endlich von jeder Vorstellung verlangten, daß sie einen Gegenstand habe, und dann jene, die dieser Erwartung nicht entsprach, für eine falsche Vorstellung erklärten, anstatt sie eine bloß gegenstandlose zu nennen. Übrigens ist auch diese Streitfrage von keiner Wichtigkeit. \par
Von großem Einflusse dagegen ist es, ob wir B.\ in der Behauptung \WLpar{I}{56} beipflichten können, daß es einfache sowohl als auch zusammengesetzte Vorstellungen gebe. Auch dieß gilt, wenn es gilt, gleicherweise wie von Vorstellungen an sich, so von gedachten Vorstellungen. Allein Hegel\pindex{Hegel, Gottfried Wilhelm} hat ein für allemal uns für Barbaren erklärt, wenn wir dieß lehren, und \zB\ sagen, daß der Begriff eines vernünftigen Wesens zusammengesetzt sey aus den Begriffen: Wesen, Vernünftigkeit \udgl\ Werden wir es nun trotz diesem wagen, bei der alten Lehre zu bleiben? Wahr ist es wohl, daß jedes denkende Wesen ein lebendiges ist, und zugegeben auch, daß hieraus folge, das Denken selbst müsse etwas Lebendiges oder (um Euch die Freude dieses Worts zu gönnen!) etwas Organisches seyn; zugegeben, man dürfe es deßhalb keineswegs den Gesetzen eines starren Mechanismus unterwerfen, noch weniger Vorstellungen und Gedanken wie todte Zahlen bloß addiren, subtrahiren \udgl ; es herrsche vielmehr (so hört Ihr es gern) eine geistige Wechselwirkung zwischen unsern Gedanken, deren der eine den andern, ohngefähr wie? -- wie die Knospe die Blüthe, diese die Frucht \usw\ hervortreibt: dieß Alles zugestanden, müssen wir doch darauf beharren, uns auf die klareste Aussage unsers Bewußtseyns stützend, darauf beharren, daß wir Vorstellungen haben, die wir aus der Verbindung mehrerer andrer erzeugen, die also zusammengesetzt sind, und wieder andere, deren wir uns zu solchen Verbindungen erst bedienen, die also einfach sind. \seitenw{38}\par
B.\ gehet allerdings noch weiter und behauptet \WLpar{I}{58}, daß manche Vorstellungen sogar ganze Sätze als Theile enthalten; wie die Subjectvorstellung in dem Satze: \danf{Die Erkenntniß der Wahrheit, daß Gott heilig sey \BUhaben{hat} große Wichtigkeit;} und hiedurch glaubt er noch gar nicht seiner Erklärung des Begriffes einer Vorstellung zu widersprechen; denn was einen Satz als Bestandtheil einschließt, muß darum noch nicht selbst wieder ein Satz seyn. Er behauptet sogar, daß auch schon eine solche Vorstellung wie: \danf{Vernünftiges Wesen,} einen ganzen Satz enthalte; indem sie, wenn ihre Bestandtheile klarer hervortreten sollen, eigentlich so ausgedrückt werden müsse: \danf{Ein Wesen, das Vernünftigkeit hat.} Er behauptet weiter, daß die Bestandtheile einer Vorstellung nicht wie die Theile einer Menge in beliebiger, sondern nur in bestimmter Ordnung verbunden sind; weil zwei Vorstellungen, wie \danf{ein Mensch von vielem Verstande, aber wenig Gefühl,} und \danf{ein Mensch von wenig Verstande, aber vielem Gefühl,} -- bei gleichen Bestandtheilen sich nur durch die Ordnung, in welcher sie verbunden sind, unterscheiden. Er behauptet \WLpar{I}{64}, daß die Bestandtheile einer Vorstellung und die Vorstellungen von den Beschaffenheiten ihres Gegenstandes (Merkmale desselben) zu unterscheiden wären; wie denn z.B.\ die Vorstellung einfach seyn, oder nur eine sehr mäßige Anzahl von Bestandtheilen enthalten könne, während ihr Gegenstand eine unendliche Menge Beschaffenheiten besitzt. Er behauptet, daß einem Gegenstande nicht nur Beschaffenheiten, deren in seiner Vorstellung gar nicht erwähnet wird, zukommen, sondern auch solche, deren gedacht wird, mangeln. So kommt in der Vorstellung: \danf{gleichseitiges Dreieck,} nichts von der Größe der Winkel vor, und doch ist es bekannt, daß alle gleichseitigen Dreiecke Winkel besitzen, welche einander gleich sind und 60° haben. Wieder wird in der Vorstellung: \danf{der Sohn eines tugendhaften Vaters,} die Beschaffenheit: tugendhaft erwähnt, und kann doch dem Gegenstande dieser Vorstellung mangeln. Namentlich, meint B., sey dieß der Fall bei allen verneinenden Vorstellungen, wie \danf{ein lebloses Wesen.} Denn diese Vorstellung enthält nach ihm den Begriff des \seitenw{39} Lebens, obgleich verbunden mit jenem der Verneinung; sie ist nämlich zu fassen als die Vorstellung eines Wesens, welches kein Leben hat \usw\ \par
Diese Behauptungen bestreite man, wenn man vermag; darum wird doch kaum etwas von demjenigen, was in der Folge von B.\ vorgetragen wird, wegfallen. Ein Ähnliches gilt noch von mehreren in diesem Abschnitte vorkommenden Lehren; namentlich von der Art, wie \WLpar{I}{60} der Unterschied zwischen concreten und abstracten Vorstellungen bestimmt wird (eine Bestimmung, welche durch ihre Klarheit sehr absticht von dem Nebelgebilde, in welches Hegel\pindex{Hegel, Gottfried Wilhelm} diesen Unterschied gehüllt hat); von der Behauptung (\WLpar{I}{69}), daß es auch überfüllte, \dh\  solche Vorstellungen gebe, in denen Merkmale verbunden sind, deren eines schon eine Folge der übrigen ist, \zB\ eine runde Kugel; daß es ferner auch (\WLpar{I}{70}) gegenstandslose Vorstellungen gebe; wie der Begriff des Nichts, oder der einer Billion jetzt lebender Menschen; sogar auch imaginäre, \dh\  solche, die widersprechende Merkmale einigen, \zB\ ein rundes Viereck, ein hölzernes Eisen \udgl\  Auch ob wir dem Vf.\ in den Erklärungen beistimmen, welche er \WLpar{I}{80} von dem Begriffe eines Verhältnisses, \WLpar{I}{81} von Form und Materie, \WLpar[§.\,82--\Hgkorr{87}{98}]{I}{82--87} von den Begriffen einer Menge, Summe, Reihe, Einheit, Vielheit und Allheit, Größe, Zahl und dem Unendlichen aufstellt, das Alles ist mindestens von keinen weiteren Folgen für unsre Zustimmung zu den übrigen Lehren des Buches, so wichtig auch einige dieser Begriffbestimmungen in mancher anderen Hinsicht seyn mögen. So wird von dem Unendlichen behauptet, daß es nur dort statt finde, wo eine Vielheit statt findet, und folgende sehr genaue Erklärung gegeben: \danf{Jede Vielheit von der Art $A$, welche als Glied erscheint in der Reihe, die man erhält, wenn man $A$ zum ersten Gliede macht, jedes folgende aber aus dem vorhergehenden bloß dadurch ableitet, daß man ein neues $A$ hinzuthut, -- heißt eine endliche Vielheit. Eine Vielheit dagegen, die so beschaffen ist, daß jede endliche nur als ein Theil in ihr erscheinet, heißt eine unendliche Vielheit.}\editorischeanmerkung{%
	Die zitierte Stelle aus WL I §.\,87. lautet wörtlich:\par
	\zitfn{\anf{Jede Vielheit von der Art $A$, die als ein Glied in der Reihe erscheint, die wir erhalten, wenn wir die Vielheit: Zwei $A$, zum \RWbet{ersten} Gliede machen, jedes nachfolgende aber aus dem nächstvorhergehenden dadurch ableiten, daß wir ein neues $A$ zu demselben (oder vielmehr zu einer demselben gleichen Vielheit) hinzuthun, -- heißt eine Vielheit von \RWbet{endlicher Größe}, oder auch bloß eine \RWbet{endliche Vielheit} von der Art $A$. Eine Vielheit von der Art $A$ dagegen, die so beschaffen ist, daß jede \RWbet{endliche} Vielheit von der Art $A$ nur als ein \RWbet{Theil} von ihr erscheint, \dh\ daß es zu jeder endlichen Vielheit von der Art $A$ einen dieser gleichkommenden Theil in ihr gibt, nenne ich eine Vielheit von \RWbet{unendlicher Größe}, oder auch nur eine \RWbet{unendlich große} oder \RWbet{unendliche Vielheit} von der Art $A$.}}}
Aus dieser Erklärung läßt sich, wie der Vf.\ zeigt, leicht darthun, daß auch unendliche Größen ver\seitenw{40}schieden seyn können, daß es dergleichen auch im Gebiete der Wirklichkeit gebe, und daß es insbesondere gar nichts Unmögliches sey, daß eine unendliche Reihe von Ereignissen eine vergangene sey. \par
Wenn diese Behauptungen von der größten Wichtigkeit für Physik und Metaphysik sind, so haben sie doch, wie gesagt, auf die Logik selbst keinen weiteren Einfluß. Ein Anderes ist es mit der in den \WLpar{I}{72--77} ausgeführten Unterscheidung zwischen Anschauungen, Begriffen und gemischten Vorstellungen; von welcher der Vf.\ in der Folge eine so \DruckVariante{häufige}{haüfige} Anwendung macht, daß sich der Anstände in der That viele ergäben, wenn unsere Leser ihm hier in keiner Art beipflichten könnten. Des bessern Verständnisses wegen erachten wir für nöthig, zuvor noch einen schon etlichemal gebrauchten Unterschied zu erklären, den B.\ zwischen dem Gegenstande macht, \Hgkorr{den}{der} eine Vorstellung hat oder vorstellt, und zwischen einem solchen, auf den wir sie so eben in einem Urtheile als Prädicat beziehen. Wir thun es durch ein Beispiel. Die Vorstellung: \danf{sinnlich vernünftiges Wesen,} gehöret unter diejenigen, die Gegenständlichkeit haben; sie stellt nun vor nicht nur alle auf Erden lebende Menschen, sondern höchst wahrscheinlicher Weise noch eine unendliche Menge von Geschöpfen, welche auf andern Weltkörpern leben; stellt (sagen wir) alle diese Gegenstände vor, gleichviel ob ich sie alle im Einzelnen kenne, und diese Vorstellung auf sie -- (welche ich dann noch durch eine andere Vorstellung mir vorstellen müßte) -- beziehe, \dh\  das Urtheil: \danf{Auch Dieses und Jenes ist ein sinnlich vernünftiges Wesen,} fälle oder nicht. Fälle ich daher irgend ein Urtheil, in welchem diese Vorstellung die Subjectvorstellung ist, \zB\ \danf{Jedes sinnlich vernünftige Wesen hat einen Leib}: so darf man sagen, ich habe über alle diese Geschöpfe, auch über diejenigen, von deren Daseyn ich nichts weiß, oder an die ich im Besondern jetzt eben nicht denke, geurtheilt; denn auf sie alle, mag ich's bedenken oder nicht, erstrecket sich mein Urtheil. Die Vorstellung: \danf{Leib} dagegen, welche in eben \seitenw{41} diesem Urtheile als Prädicatvorstellung auftritt, wird gegenwärtig von mir lediglich auf sinnlich vernünftige Wesen bezogen, obgleich es noch andere Gegenstände gibt, welche sie vorstellt (indem auch Wesen, die nicht vernünftig sind, Leiber haben). Ja, eine Prädicatvorstellung kann in einem unrichtigen Urtheile sogar bezogen werden auf Gegenstände, welche sie in der That gar nicht vorstellt, wie in dem Urtheile: Dieses Buch ist von Henoch. \par
Eine Vorstellung nun, die wir auf einen einzigen Gegenstand nicht etwa bloß beziehen, sondern die an sich selbst nur einen einzigen Gegenstand hat, z.B.\ der Fixstern Sirius, nennet B.\ eine Einzelvorstellung; eine Vorstellung, die nicht aus mehrern andern zusammengesetzt ist, wie etwa die Vorstellung Nicht, nennet er einfach; und untersucht nun die Frage, ob es auch Vorstellungen gebe, die diese beiden merkwürdigen Beschaffenheiten (den kleinsten Umfang mit dem kleinsten Inhalte) vereinigen, \dh\  die einfach sind und zugleich auch nur einen einzigen Gegenstand haben? Er bejahet dieß, indem er nachweiset, daß von dieser Art alle diejenigen (subjectiven) Vorstellungen sind, welche in uns zunächst und unmittelbar entstehen, so oft wir unsere Aufmerksamkeit auf die Veränderungen richten, die ein vor unsere Sinne gebrachter Körper, \zB\ eine Rose, in unsrer Seele hervorbringt. Die Vorstellungen, die hier zunächst und unmittelbar entstehen, müssen einfach seyn, weil zusammengesetzte Vorstellungen erst mittelbar, nämlich durch die Verbindung einfacher zum Vorscheine kommen. Sie müssen ferner etwas ganz Einzelnes, nämlich nur die in unserer Seele jetzt eben vor sich gehende Veränderung zu ihrem Gegenstande haben; denn Vorstellungen, welche der Gegenstände mehrere haben, entstehen wohl auch bei einer solchen Gelegenheit, \zB\ die Vorstellung: Wohlgeruch, wenn wir allmählich das Urtheil bilden: \danf{Dieß (was ich jetzt eben rieche) \BUhaben{ist} ein Wohlgeruch} \udgl\  Allein es ist offenbar, daß wir zu dieser letzteren Vorstellung erst mittelbar, nämlich nur eben durch jene einfache Einzelvorstellung, die wir in dem erwähnten Beispiele durch ein bloßes Dieß bezeichneten, und die hier als Subjectvorstellung auftrat, veranlasset wurden. Somit \seitenw{42} ist es erwiesen, daß es auch Vorstellungen, subjective Vorstellungen gebe, welche trotz aller Einfachheit dennoch nur einen einzigen Gegenstand vorstellen. Solche nun sind es, die B.\ Anschauungen genannt wissen will; subjective oder gedachte Anschauungen, wenn sie selbst subjective oder gedachte Vorstellungen sind; die ihnen entsprechende objective Vorstellung aber heißt ihm die objective Anschauung. Vorstellungen dagegen, die nicht nur selbst keine Anschauungen sind, sondern dergleichen nicht einmal als Theile enthalten, nennt er Begriffe, reine Begriffe; alle andern Vorstellungen endlich, die also weder reine Anschauungen, noch Begriffe sind, nennt er gemischte Vorstellungen. \par
Was wird nun erfordert, um ihm in diesen Begriffbestimmungen zu folgen? Wesentlich gewiß nichts Mehreres, als daß man zulasse, a) es gebe einfache sowohl als zusammengesetzte Vorstellungen; b) es gebe Einzelvorstellungen, und c) auch eine Einzelvorstellung könne eine einfache Vorstellung seyn. Über das Erste glauben wir oben genug gesagt zu haben; denn daß es nur zusammengesetzte, aber gar keine einfache Vorstellungen gebe, das wird wohl Niemand behaupten, da alles Zusammengesetzte einfache Theile, aus denen es zusammengesetzt ist, in endlicher oder unendlicher Menge voraussetzt. Was aber das Daseyn von Einzelvorstellungen belangt, so haben wir eine beträchtliche Menge derselben nicht nur unter den Vorstellungen, die man bisher schon Anschauungen genannt hat, \zB\ dieser Mensch, diese Rose \udgl , sondern selbst unter solchen Vorstellungen, die man auch schon bisher den reinen Begriffen beigezählt hatte, wie Gott, Weltall, oberstes Sittengesetz \umA\ Wie jedoch auch Beides vereinigt seyn könne; wie eine Vorstellung, die durchaus einfach ist, doch Eigenthümlichkeiten genug haben könne, um nur auf einen einzigen Gegenstand zu passen: das mag freilich demjenigen räthselhaft scheinen, der wähnet, daß jede Bestimmung eines Gegenstandes in der Vorstellung desselben, falls sie auf ihn allein passen soll, mit vorgestellt werden müsse. Diese Ansicht wird aber \WLpar{I}{64} durch viele Gründe von dem Vf.\ als irrig widerlegt. Wir führen nur dieses Eine an, wie denn, wenn es sich also \seitenw{43} verhielte, je eine Vorstellung zu Stande kommen könnte, welche nur einen einzigen Gegenstand hat, da doch, genau genommen, ein jeder der Bestimmungen unendlich viele besitzt? In der That, die Vorstellung, welche wir in dem obigen Falle durch Dieß bezeichneten, war so gewiß als eine Einzelvorstellung, auch eine einfache Vorstellung. Denn selbst die Bestimmungen, die wir dem Dieß bei einer mündlichen Mittheilung unserer Gedanken an einen Andern etwa beisetzen (Dieß, was ich jetzt eben fühle, rieche, sehe \usw ), werden von uns nur beigesetzt, um das Dieß, welches wir meinen, dem Andern kenntlicher zu machen; nicht aber werden sie beigesetzt, als ob sie zur Vorstellung selbst gehörten, und sie zur Einzelvorstellung erst erhüben. Denn daß dasjenige Dieß, was ich jetzt eben meine, etwas Solches sey, was ich jetzt fühle, rieche oder sehe, das liegt ja schon darin, daß es nur eben das jetzt gemeinte und kein anderes Dieß ist. \par
Dieses Wenige nur zugestanden, und man muß offenbar schon B.'s ganze Unterscheidung zwischen Anschauungen, Begriffen und gemischten Vorstellungen als eine in der Natur der Sache gegründete anerkennen. -- Bei Vergleichung derselben mit den bisherigen Begriffbestimmungen aber dürfte man finden, was B.\ einen reinen Begriff nennt, falle genau zusammen mit dem, was man auch schon bisher -- (etwa mit Ausnahme der Hegelschen\pindex{Hegel, Gottfried Wilhelm} Schule, welcher der Begriff eben so sehr die Sache selbst ist!) -- so genannt hat, und wohl noch ferner so nennen wird. Nur das Wort Anschauung hat man bisher insgemein in einem viel weiteren Sinne genommen, und auch jede gemischte Vorstellung, wenn sie nur eine Einzelvorstellung war, darunter begriffen. Ja hie und da rechnete man sogar ganze empirische Sätze mit zu den Anschauungen. Welche höchst wichtige Vortheile nun aus B.'s engerer Begrenzung dieses Begriffes, und vornehmlich aus jener Angabe seiner Bestandtheile hervorgehen, werden die Leser erst in der Folge immer mehr kennen lernen. Merkwürdig dürften sie indessen schon die erste, ihnen gleich jetzo mitzutheilende Anwendung finden, die der Vf.\ von seinen Erklärungen auf die genauere Bestimmung der Zeit- \seitenw{44} und Raumvorstellungen \WLpar{I}{79} macht, indem er ganz der herrschenden Ansicht zuwider erweiset, daß es Anschauungen weder von Zeit noch Raum gebe; und uns zugleich die erste genaue Erklärung dieser Begriffe liefert. \par
Wenn von irgend einigen die Zeit oder den Raum betreffenden Vorstellungen vermuthet werden könnte, daß sie Anschauungen sind, so wären es gewiß solche, die wie die Vorstellungen: diese bestimmte Zeit, dieser bestimmte Ort, nur einen einzigen Gegenstand haben. Denn die Vorstellungen: Zeit überhaupt, Ort überhaupt, Linie, Fläche \usw , welche der Gegenstände mehrere haben, können schon eben deßhalb, weil sie keine Einzelvorstellungen sind, nach der gegebenen Erklärung mindestens keine reinen (unvermischten) Anschauungen seyn. Daß aber auch die beiden Vorstellungen: die ganze unendliche Zeit und der ganze unendliche Raum, ob sie gleich wahre Einzelvorstellungen sind, doch nur ein paar reine Begriffe seyen, erweiset sich daraus, weil wir sie aus lauter reinen Begriffen zusammensetzen (definiren) können. Zeit nämlich -- (wenn wir es wagen sollen, diese Erklärungen nur zur Befriedigung der Neugier unserer Leser hieher zu setzen, ohne irgend etwas zu ihrer \DruckVariante{Erläuterung}{Erlaüterung} oder Rechtfertigung beifügen zu können) -- Zeit ist diejenige Bestimmung an einem Seyenden, unter welcher allein ihm irgend eine Beschaffenheit mit Wahrheit beigelegt werden kann. Der Inbegriff aller Zeiten aber heißet die ganze Zeit. Die Orte der Dinge sind diejenigen Bestimmungen an denselben, die uns erklären, wienach sie bei ihren Kräften gerade diese und keine andere Einwirkungen auf einander üben. Der Inbegriff aller Orte aber heißet der ganze Raum. Dieses vorausgesetzt, ist es nun leicht, zu zeigen, daß die Vorstellungen: diese Zeit, dieser Ort und andere ihnen ähnlich gebildete Einzelvorstellungen, wohl gemischte Vorstellungen sind, wohl eine Anschauung, keineswegs aber eine solche, die zugleich eine Zeit- oder Raumvorstellung wäre, enthalten. Unter dieser bestimmten Zeit verstehen wir nämlich gewiß nie etwas Anderes, als eine Zeit, die wir durch eine in ihr befindliche Anschauung, oder durch irgend ein Verhältniß derselben zu dieser Anschauung bestimmen, \zB\ die \seitenw{45} Zeit, in der ich jetzt eben urtheile \udgl\ Ähnlicher Weise verstehen wir auch unter diesem bestimmten Orte nichts Anderes als einen Ort, den wir durch sein Verhältniß zu einer unserer Anschauungen, \zB\ als den Ort eines Körpers, der diese und jene Anschauung in uns so eben veranlasset hat, bestimmen \usw\ Allein wenn wir die Meinung, daß Zeit und Raum Anschauungen wären, als einen Irrthum bezeichnen, wie wollen wir die Entstehung und weite Verbreitung dieses Irrthums erklären? Nur aus dem doppelten Umstande, daß man einerseits Vorstellungen, die eine Anschauung nur als Bestandtheil enthalten, schon zu den Anschauungen selbst zählte, und daß man andererseits noch nicht dahin gelangt war, die ersten in die Zeit- und Raumlehre gehörigen Begriffe, \zB\ von einem Ausgedehnten, von Linie, Fläche, Körper \udgl\ in ihre Bestandtheile zu zerlegen, und noch weniger gewisse erste Sätze, \zB\ daß die Zeit Eine, daß der Raum drei Dimensionen habe \usw , aus ihren objectiven Gründen abzuleiten. Die oben angedeuteten Erklärungen von Zeit und Raum haben den Weg hiezu gebahnet; aus ihnen lassen sich alle jene Begriffe erklären, und alle jene Sätze erweisen; was wir jedoch hier freilich nicht weiter aus einander setzen können. \par
In dem dritten Abschnitte: \danf{Verschiedenheiten unter den Vorstellungen nach ihrem Verhältnisse unter einander} (S.\,428--515), stößt man gleich anfangs \WLpar{I}{91} wieder auf einen Satz: \danf{Es gibt nicht zwei einander völlig gleiche Vorstellungen,} den man nur zulassen sollte, wenn man Vorstellungen an sich zuläßt. Aber der Umstand, daß fast alle Logiker eben dasselbe behaupten, mit der einzigen Ausnahme, daß sie statt des Worts Vorstellung gemeinhin das Wort Begriff gebrauchen, verräth uns, daß sie -- mögen sie dessen sich deutlich bewußt gewesen seyn oder nicht -- hier an Vorstellungen oder Begriffe in objectiver Bedeutung gedacht. Denn von subjectiven Vorstellungen hätten sie offenbar das gerade Gegentheil behaupten sollen, daß nämlich tausende derselben einander gleich seyn können. -- Alles, was sonst noch in diesem Abschnitte \seitenw{46} gelehrt wird, gilt auch von Vorstellungen in subjectiver Bedeutung, und kann sonach auch von denjenigen angenommen werden, die den Begriff einer Vorstellung an sich verwerfen. Überdieß sind alle hier aufgestellten Behauptungen von einer solchen Art, daß sie theils allgemein zugestanden werden, theils ohne allen Einfluß auf das Folgende verworfen werden können. Wir führen Einiges zur Probe an. Seine Erklärung einer Anschauung setzt den Vf.\ (\WLpar{I}{91}) in den Stand, auch den für die Mathematik so wichtigen Begriff der Ähnlichkeit in seine wahren Bestandtheile zu zerlegen. Ähnlich im mathematischen Sinne sind nämlich Dinge, die alle durch Begriffe darstellbare Beschaffenheiten gemeinschaftlich haben (somit sich entweder durch bloße Anschauungen oder Verhältnisse unterscheiden). In \WLpar{I}{93} unterscheidet der Vf.\ weitere und engere Vorstellungen (die Vorstellung: Söhne Israels, ist sechsmal so weit als die: Söhne Isaaks); in \WLpar{I}{94} verträgliche und unverträgliche; zu den erstern gehören gleichgeltende, höhere und niedere, verkettete \usw ; zu den letztern widerstreitende, widersprechende \usw\ Nach ihm gibt es auch Vorstellungen, die einander in Absicht auf Weite oder Höhe zunächst stehen. (\WLpar{I}{100}) Die wichtige Gattung der gleichgeltenden oder Wechselvorstellungen (\WLpar{I}{96}) (die bei verschiedenem Inhalt denselben Umfang haben) mögen unsere Leser nicht mit den verschiedenen Benennungen einer und eben derselben Vorstellung vermengen. Dreieck und Triangel sind nicht (wie es oft heißt) Wechselbegriffe, sondern nur verschiedene Benennungen desselben Begriffes; wohl aber sind die Begriffe: \danf{eine Figur, welche drei Seiten hat,} und \danf{eine Figur, deren sämmtliche Winkel zwei rechte betragen,} sehr deutlich unterschieden, obgleich sie dieselben Gegenstände haben. Nur diese sind Wechselbegriffe. -- Das Wichtige ist aber, daß der Vf.\ \WLpar{I}{108} erklärt, auf welche Weise man (ohne sich dessen bisher deutlich bewußt gewesen zu seyn) die Verhältnisse der Weite und Höhe selbst auf gegenstandlose Vorstellungen ausdehnen könne und ausgedehnt habe; wie dieß besonders in der Mathematik bei den Gleichungen und in der Rechnung mit imaginären Größen der Fall ist. Es geschieht dieß aber, \seitenw{47} indem man gewisse Bestandtheile in den gegebenen Vorstellungen als veränderlich ansieht und das Verhältniß beachtet, das diese Vorstellungen gegen einander befolgen, so oft an die Stelle jener veränderlichen Theile solche gesetzt werden, durch die sie Gegenständlichkeit erhalten. In der Lehre von den Sätzen werden wir einer noch viel fruchtbareren Anwendung dieses Gedankens begegnen. \par
\gliederungslinie\par
Der letzte Abschnitt, der die \danf{Verschiedenheiten unter den Vorstellungen} betrachtet, \danf{die erst aus ihrem Verhältnisse zu andern Gegenständen entspringen,} bietet der Anstöße noch weniger dar. Gleich anfangs \WLpar{I}{109} wird zur Rechtfertigung aller derjenigen Logiker, welche die Vorstellungen in richtige und unrichtige (oder wahre und falsche) eintheilen, gezeigt, in welchem Sinne dieß allerdings angehe, nämlich in sofern man eine gegebene Vorstellung als Vorstellung eines bestimmten (durch eine andere Vorstellung bereits erfaßten) Gegenstandes ansieht. So möchte \zB\ \danf{ein großer Feldherr} eine sehr richtige, \danf{ein Weiser} aber wohl eine unrichtige Vorstellung von Napoleon heißen. \par
Auch hier wieder begegnen wir \WLpar{I}{111} einer Anwendung, die der Vf.\ von seiner Unterscheidung zwischen Begriffen und Anschauungen macht, indem er die wesentlichen (oder nothwendigen) Beschaffenheiten eines Gegenstandes als diejenigen erklärt, die ihm vermöge eines reinen Begriffes, den wir uns von ihm gebildet, zukommen, während die übrigen, die nicht aus diesem Begriffe abgeleitet werden können, in sofern außerwesentlich (oder zufällig) heißen. So heißt Besitz gewisser Gliedmaßen eine wesentliche, Gesundheit oder Krankheit aber eine außerwesentliche Beschaffenheit eines organischen Wesens, \usw\ \par
In dem Anhange (über die bisherige Darstellungsart der Lehre von den Vorstellungen. S.\,537--571) dürfte das Wichtigste seyn, aber auch den meisten Anstoß erregen die Prüfung der Kantschen Tafel der Kategorien, da diese noch gegenwärtig den Systemen unserer Weltweisen von den verschiedensten Farben unverkennbar zu \seitenw{48} Grunde liegt, obgleich sie schon Manches daran geändert oder zugesetzt haben. Wenn diese Tafel die sämmtlichen Stammbegriffe des menschlichen Verstandes enthalten soll, so frägt B., wie man doch nur \zB\ folgende Begriffe: Vorstellung an sich, Satz an sich, Wahrheit, Erkenntniß, Wille, Empfindung, Pflicht \uma\ aus den dort aufgeführten vermeine ableiten zu können? Andere Ausstellungen müssen wir zur Ersparung des Raumes ganz unberührt lassen. \par
\gliederungslinie\par
\WLsec{II}{121}{268}
Indem wir nun zu dem zweiten Hauptstücke: \danf{Von den Sätzen an sich,} übergehen, und hier den ersten Abschnitt von den \danf{allgemeinen Beschaffenheiten der Sätze} vornehmen wollen, stehen wir vor einer Abtheilung in B.'s Buche, die zu den mangelhaftesten gehöret. Er wagt es, \WLpar{II}{127} die Ansicht aufzustellen, daß alle Sätze nur einer und eben derselben Form, nämlich der folgenden: \danf{$A$ hat (die Beschaffenheit) $b$,} unterständen; daß sie also immer nur drei nächste Bestandtheile hätten: eine Unterlage oder Subjectvorstellung $A$, einen Aussagetheil oder eine Prädicatvorstellung $b$, und den Begriff des Habens, der jene beiden verbindet, daher auch Copel genannt wird. B.\ getrauet sich selbst nicht, diese Ansicht als eine ganz entschiedene geltend zu machen; denn er weiß sie nicht anders als durch eine ihrer Natur nach nur unvollständige Induction zu erweisen; indem er theils gleich hier, theils in der Folge überall, wo eine scheinbar abweichende Form von Sätzen ihm vorkömmt, zu zeigen sucht, daß man auch diesen Ausdruck auf die Form: $A$ hat $b$, zurückführen könne, ja müsse, wenn recht verständlich werden soll, was man hier eigentlich sage. Ob nun die Leser mit dieser Zurückführung überall zufrieden seyn werden, wissen wir nicht zu sagen: gewiß ist nur, daß das Ganze nie als ein vollständiger Beweis angesehen werden könne. \par
Indessen wird doch auf ziemlich befriedigende Art erwiesen, daß wenigstens alle diejenigen Sätze, die der gewöhnliche Redegebrauch in die Form: $A$ ist $B$ (Gott ist allwissend) kleidet, (welche Form andere Logiker bekanntlich, ohne irgend \seitenw{49} einen Beweis dafür nur zu versuchen, für die allgemein geltende erklären) -- nicht mit diesen, sondern vielmehr mit den Worten: $A$ hat $b$ (Gott hat Allwissenheit) ausgedrückt werden müssen, sofern der ihnen zu Grunde liegende Gedanke recht deutlich ausgedrückt werden soll. Daß nämlich das Ist in solchen Sätzen nicht in seiner eigentlichen Bedeutung, wir meinen, nicht als die Aussage eines Seyns (wie in dem Satze: Gott ist) auftrete, hat man längst anerkannt. Dieß erhellet auch klärlich daraus, weil wir selbst in dem Falle, wo im Prädicate nur eben das Seyn (die Wirklichkeit) eines Gegenstandes ausgesagt werden soll, wie: \danf{Gott ist ein Seyendes (oder etwas Wirkliches,)} ja sogar auch in dem Falle, wo wir das Seyn einem Gegenstande eben absprechen wollen, \zB\ \danf{das bloß Mögliche ist nichts Seyendes,} -- jenes Wörtlein: Ist, noch immer beibehalten, obgleich es im ersten Fall einen Pleonasmus, im zweiten vollends eine Contradictio in adjecto gäbe, wenn es in seiner eigentlichen Bedeutung genommen würde. Was ist denn also die wahre Bedeutung, die dieses Ist als Copel hat? Keine andere als die des Habens. \danf{Gott ist allwissend,} heißt durchaus nichts Anderes als: \danf{Gott \BUhaben{hat} Allwissenheit;} und eben so: \danf{Gott ist ein Seyendes,} heißt nur: \danf{Gott \BUhaben{hat} Seyn (oder Wirklichkeit);} und \danf{das bloß Mögliche ist nichts Seyendes,} heißt: \danf{das bloß Mögliche \BUhaben{hat} keine Wirklichkeit.} \usw\ Daß diese Auslegungen die richtigen sind, gehet auch daraus hervor, weil die letzteren Ausdrücke, obgleich sie der Buchstaben mehrere zählen, dem Gedanken nach doch in der That einfacher sind. Denn statt des Ist in den ersteren erscheinet in den letztern das Hat, welches jedenfalls kein zusammengesetzterer Begriff als der des Ist in seiner uneigentlichen Bedeutung seyn kann; allem Anscheine nach vielmehr ein durchaus einfacher Begriff ist. Statt der concreten Vorstellung allwissend im ersten Ausdrucke dagegen erscheinet in dem zweiten die abstracte: Allwissenheit, die doch gewiß einfacher ist als jene, wie auch die sprachlichen Ausdrücke sich zu einander verhalten mögen. Ja, wenn wir die Beschaffenheit der concreten Vorstellungen, wie allwissend, genauer untersuchen: so finden wir, daß das zu dem Abstracto $b$ ge\seitenw{50}hörige Concretum $B$ keinen andern Sinn habe als: Etwas, das $b$ hat. Allwissend \zB\ heißt nichts Anderes als: Etwas, das Allwissenheit hat. Der Satz: Gott ist allwissend, hätte somit, wenn das Ist beibehalten werden müßte, eigentlich nur den Sinn: Gott ist Etwas, das Allwissenheit hat. So käme denn also in allen Sätzen, in denen man die Subjectvorstellung mit der Prädicatvorstellung durch ein Ist verbinden will, immer doch noch die Copel Hat, und zwar im Prädicate, zum Vorschein. Wer sieht nicht, daß dieses ungereimt sey? \par
Da wir die eigenthümliche Art, wie B.\ einige andere Formen von Sätzen auf sein: $A$ hat $b$, zurückführe, füglicher erst in der Folge betrachten, wo er auf diese Sätze des Mehreren zu reden kommt: so haben wir aus diesem ganzen Abschnitte nichts weiter zu erwähnen, als die Behauptung des \WLpar{II}{130}, daß der Umfang eines Satzes immer einerlei sey mit dem Umfange seiner Subjectvorstellung. Es ist dieses begreiflicher Weise nur eben in der Voraussetzung gesprochen, daß sich alle Sätze auf die Form: $A$ hat $b$, zurückführen lassen, wo dann in der That $A$ so viel als \danf{jedes $A$} bedeutet. Über diesen letzteren Ausdruck aber erinnerte B.\ schon \WLpar{I}{57}, daß man ihn keineswegs als den Ausdruck eines aus den Begriffen $A$ und Jedes zusammengesetzten Begriffes auszulegen habe, sondern daß derselbe ohngefähr eben so wie die verwandten Ausdrücke: alle $A$, $A$ überhaupt \udgl\ nur den Zweck habe, zu verhüten, daß wir zu der durch $A$ bezeichneten Vorstellung nicht irgend eine ihren Umfang beschränkende Bestimmung hinzudenken. \par
\gliederungslinie\par
Wie schon Alles, was der Vf.\ in diesem ersten Abschnitte von Sätzen an sich behauptete, -- mit bloßer Ausnahme der Behauptung des \WLpar{II}{122}, daß solche Sätze nichts Existirendes sind, -- auch von gedachten Sätzen gilt: so sind auch die in dem zweiten Abschnitte (S.\,29--91) aufgezählten \danf{Verschiedenheiten der Sätze nach ihrer inneren Beschaffenheit} alle der Art, daß sie auch zugeben kann, wer nur gedachte Sätze zugibt. \seitenw{51}\par
Wichtig ist hier besonders die Unterscheidung der Begriffs- und Anschauungssätze \WLpar{II}{133} so aufgefaßt, daß unter jenen nur Sätze, die lauter Begriffe enthalten, und unter diesen dann alle übrigen, in deren Bestandtheilen also irgend eine Anschauung vorkömmt, verstanden werden sollen. Es ist aber wohl zu beachten, daß nicht nur derjenige, der den Unterschied zwischen Begriffen und Anschauungen gerade so, wie B.\ auffaßt, sondern auch jeder, der nur zugibt, daß es ein objectiver, \dh\  an diesen Vorstellungen selbst befindlicher Unterschied sey, daß es somit nicht etwa bloß von dem Subjecte, in welchem diese Vorstellungen sich befinden, abhange, sondern aus ihrer inneren Beschaffenheit selbst zu entnehmen sey, ob sie Begriffe oder Anschauungen wären, an dem hier aufgestellten Unterschiede zwischen Begriffs- und anderen Sätzen einen Unterschied besitze, der in eben dem Sinne des Wortes objectiv ist. Der Satz: Alle Weltkörper haben eine Umdrehung um ihre eigene Achse, wie auch der ihm entgegenstehende: Nicht alle Weltkörper \usw\ sind ein paar reine Begriffssätze; dagegen der Satz: diese Erde dreht sich um ihre eigene Achse, ist ein Anschauungssatz -- gleichviel ob wir den einen oder den andern derselben für wahr oder nicht wahr halten, und wie wir zu diesen Urtheilen gelangt seyen. Denn alle Vorstellungen in den beiden ersteren Sätzen sind reine Begriffe, in dem letzteren Satze aber enthält die Vorstellung: diese Erde, eine sich durch das Wort dieses kund gebende Anschauung. B.'s Eintheilung ist also keineswegs mit der durch Kant so berühmt gewordenen Eintheilung unserer Urtheile in Sätze a priori und a posteriori zu verwechseln. Denn abgesehen davon, daß sich ohne Voraussetzung jener gar nicht erklären läßt, was eigentlich Erfahrungen, \dh\  Urtheile a posteriori und also auch Urtheile a priori seyen; abgesehen auch davon, daß diese Eintheilung sich nur auf solche Sätze, die zugleich Urtheile sind, anwenden läßt: das Wichtigste ist, daß sie etwas bloß Subjectives, nämlich die bloße Art betrifft, wie wir zu diesen Urtheilen gelangten, oder nach einer gewissen Verbesserung hätten gelangen können. Nicht jedes Urtheil aber, worüber wir nun a posteriori, \dh\ aus Erfahrungen entscheiden können, muß \seitenw{52}\par
\par
darum eben ein Anschauungsurtheil seyn, wie gleich die obigen Beispiele lehren. -- Auch was man als ein paar objective Merkmale anzugeben versuchte, Allgemeinheit und Nothwendigkeit bei den Urtheilen a priori, reicht zur Begründung eines solchen Unterschiedes, wie der zwischen Begriffsund anderen Sätzen nach B.'s Auffassung nicht hin. Denn außerdem, daß beide Merkmale nur auf wahre Sätze, nicht aber auf Sätze überhaupt anwendbar sind (weil man ja doch von einem falschen oder noch unentschiedenen Satze nicht sagen wird, daß er allgemein gelte und nothwendig sey): so kommt das Merkmal der Allgemeinheit auch vielen empirischen Wahrheiten zu, \zB\ alle Söhne Jakobs hinterließen Nachkommen; wogegen es reine Begriffswahrheiten gibt, die (mindestens nach der gewöhnlichen Auffassungsart) keine Allgemeinheit haben, \zB\ einige Quadratwurzeln sind rational, oder die Welt ist abhängig. Was aber die Nothwendigkeit belangt, so kann man in einem gewissen Verstande freilich von allen reinen Begriffswahrheiten, und zwar von ihnen ausschließlich sagen, sie hätten Nothwendigkeit. Sollen wir aber erklären, was wir da Nothwendigkeit nennen, so müssen wir (wie später sich deutlicher zeigen wird) sagen: eben nichts Anderes, als daß der Satz eine reine Begriffswahrheit oder aus einer solchen ableitbar sey. So setzt also der Begriff der Nothwendigkeit den eines reinen Begriffssatzes schon voraus. \par
\gliederungslinie\par
Bei Weitem nicht so wichtig ist B.s ohnehin nur sehr zweifelhaft vorgetragene Meinung, daß in den sogenannten verneinenden Sätzen, wie: $A$ hat nicht $b$ (\dh\  kein $A$ hat $b$), die Verneinung nicht in der Copel, sondern im Prädicate liege, dergestalt, daß der Satz eigentlich so auszudrücken sey, wenn seine Bestandtheile recht deutlich hervortreten sollen: $A$ \BUhaben{hat} Mangel an $b$. Den ihm entgegen stehenden Umstand, daß man in allen Sprachen die Verneinung so innig mit der Copel verbinde, erklärt B.\ sich bloß daraus, daß man in solchen Fällen eigentlich nichts Anderes als die Wahrheit des Satzes selbst \DruckVariante{läugnen}{laügnen} wolle, wie man denn auch, wenn man den Grad der Gewißheit eines Satzes bestimmen will, \seitenw{53} das hiezu nöthige Bestimmungswort gleichfalls der Copel beigesellt. Wie also in dem Satze: die Erde ist schwerlich lebendig, das schwerlich nicht zu der Copel, sondern zum ganzen Satze gehöret, und der Sinn eigentlich ist: die Behauptung, daß die Erde lebendig sey, hat wenig Wahrscheinlichkeit: so gehört auch in dem Satze: die Erde ist nicht lebendig, in welchen der erstere allmählich übergehet, das Nicht nur zu dem ganzen Satze, und der Sinn ist im Grunde kein anderer, als: die Behauptung, daß die Erde lebendig sey, ist falsch. \par
\gliederungslinie\par
Eine von B.\ zuerst bemerkte Gattung von Sätzen, die, wenn er recht hat, bisher unter sehr verschiedenen Ausdrücken vorgekommen wären, sind die Aussagen (oder auch Verneinungen) der Gegenständlichkeit einer Vorstellung (\WLpar[\BUparformat{137.~u.~8.}]{II}{137--138}) Hieher gehören alle die räthselhaften Sätze, die der gewöhnliche Sprachgebrauch bald durch die Form: Es gibt ein A, bald durch die Form: Ein gewisses $S$ ist $P$, bald auch wohl durch die Form: Einige $S$ sind $P$, ausdrückt, wofern (wie in der Syllogistik) der letzte Ausdruck so ausgelegt werden soll, daß er noch wahr bleibe, sowohl wenn alle $S$, $P$ wären, als auch wenn nur ein einziges $S$, $P$ wäre. Wo in der ersten Form die Subject-, wo die Prädicatvorstellung sey, läßt man ganz unbestimmt, oder erklärt, daß der Satz eigentlich nichts Anderes aussage, als daß $A$ Wirklichkeit hat. Dieß Letztere ist aber gewiß unrichtig, weil wir ja diese Form auch bei Dingen, die keine Wirklichkeit haben, anwenden; und \zB\ ohne Anstand sagen: \danf{Es gibt eine Möglichkeit, die nicht zur Wirklichkeit gelangt.} -- Eben so ungenügend ist es, wenn man in der zweiten Form: ein gewisses $S$, und in der dritten: einige $S$, für die Subjectvorstellung erklärt, ohne eine nähere Angabe der Bestandtheile, aus welchen diese wunderbaren Vorstellungen dann bestehen müßten, nur zu versuchen. Denn da man einerseits zugestehen muß, daß die Worte: einige $S$, in dem Satze: Einige $S$ sind $P$, -- (und ganz dasselbe gilt auch von den Worten: ein gewisses $S$, in dem Satze: Ein gewisses $S$ ist $P$,) -- nicht nothwendig eben dieselben \seitenw{54} Gegenstände bezeichnen, welche sie in dem Satze: Einige $S$ sind $Q$, bezeichnen; und da es andererseits offenbar ist, daß dieselbe Vorstellung nur dieselben Gegenstände vorstellen kann: hätte man nicht billig erklären sollen, worin der Unterschied der beiden Vorstellungen, welche man hier mit gleichen Worten ausdrückt, bestehe? Wie schwer dieß aber geworden wäre, das mögen unsere Leser daraus entnehmen, weil die Art der Gegenstände, welche die Worte: Einige $S$ in dem Satze: Einige $S$ sind $P$, und wieder in dem Satze: Einige $S$ sind $Q$, bezeichnen, von der Beschaffenheit der Prädicatvorstellungen $P$ und $Q$ abhängig ist; woraus denn folgt, daß man hier eine Subjectvorstellung hätte, in welcher die jeweilige Prädicatvorstellung auf irgend eine Weise schon als Bestandtheil stecken müßte! Denn ohne daß P oder Q in jener Vorstellung: Einige $S$, stecken, kann sie wohl nicht mit Änderung derselben eine andere werden! -- Alle diese Schwierigkeiten verschwinden augenblicklich, sobald wir mit B.\ annehmen, daß in solchen Sätzen nichts Anderes ausgesagt werde, als -- die Gegenständlichkeit einer Vorstellung; daß insbesondere der Satz: Es gibt ein $A$, keinen andern Sinn habe, als: die Vorstellung $A$ \BUhaben{hat} Gegenständlichkeit; und die Sätze: Ein gewisses $S$ ist $P$, oder Einige $S$ sind $P$, keinen andern Sinn als: die Vorstellung eines Etwas, das $S$ und $P$ wäre (die Beschaffenheiten $s$ und $p$ hätte) \BUhaben{hat} Gegenständlichkeit. \danf{Einige Menschen sind tugendhaft,} heißt also nichts Anderes als: die Vorstellung eines Menschen, der tugendhaft ist, hat Gegenständlichkeit; und \danf{einige Menschen sind lasterhaft,} heißt: die Vorstellung eines Menschen, der lasterhaft ist, hat Gegenständlichkeit.--\par
Obgleich nun die meisten Sätze, die nach dem sprachlichen Ausdrucke die Wirklichkeit oder das Daseyn einer Sache aussagen, im Grunde als bloße Aussagen der Gegenständlichkeit einer Vorstellung zu betrachten sind: so gibt es doch auch (meinet B.) Sätze der ersteren Art, Aussagen eines Daseyns, \zB\ dieß, (was ich jetzt eben fühle) \BUhaben{ist} etwas Wirkliches. Er behauptet also, daß Wirklichkeit oder Seyn einem Dinge auch als Prädicat beigelegt werden \seitenw{55} könne. Diese Ansicht, womit der Vf.\ Kanten und Mehreren widerspricht, kann man bezweifeln oder verwerfen, ohne daß etwas in dem Folgenden geändert werden müßte. Ein Gleiches gilt von der Ansicht, die der Vf.\ \WLpar{II}{145} über die Fragen aufstellt, daß nämlich auch sie vollständige Sätze wären. Schon nicht so gleichgültig ist die Unterscheidung der analytischen und synthetischen Sätze (\WLpar{II}{148}); deren eigenthümliche Auffassungsweise wir erst mittheilen, und im folgenden Hauptstücke zu ihrer wichtigsten Anwendung auf wahre Sätze kommen wollen. \par
\gliederungslinie\par
Der dritte Abschnitt, zu dem wir uns sonach jetzt wenden: \danf{Verschiedenheiten der Sätze nach ihrem Verhältnisse untereinander,} (S.\,92--196) ist unserm Dafürhalten nach einer der wichtigsten im ganzen Werke. Schon \WLpar{II}{147} hatte der Vf.\ erwähnt, daß wir bei der Vergleichung der Sätze -- (nach \WLpar{I}{108} geschieht es, wie wir vernommen, auch bei den Vorstellungen) -- oft ohne uns dessen deutlich bewußt zu seyn, gewisse Theile derselben uns als veränderlich denken, und demnächst das Verhalten betrachten, das alle diejenigen Sätze zur Wahrheit befolgen, die aus den vorliegenden hervorgehen, wenn an die Stelle der veränderlichen Theile was immer für andere treten. Gegenwärtig weiset er in den \WLpar{II}{154--162} nach, daß nur auf dieser stillschweigenden Voraussetzung mehrere Verhältnisse zwischen den Sätzen beruhen, die man bisher in jedem Lehrbuche der Logik abgehandelt, ja als den wichtigsten Inhalt derselben angesehen hat. Dergleichen sind das Verhältniß der Verträglichkeit (die Sätze $A, B, C,.. M, N...$ können zugleich wahr seyn), der Ableitbarkeit oder Abhängigkeit. (So oft die Sätze $A, B, C..$ wahr sind, sind es auch $M, N, O ..$ oder die $N, N, O,..$ folgen aus $A, B, C,..$) der Gleichgültigkeit (So oft die Sätze $A, B, C ..$ wahr sind, sind es auch $M, N, O,..$ und umgekehrt, so oft diese wahr sind, sind es auch wieder jene), der Unverträglichkeit oder des Widerstreites (die Sätze $A$ und $M$ sind nie zugleich wahr), des Widerspruches (So oft einer der Sätze $A$ und $M$ wahr, ist der andere falsch, und umgekehrt, so oft der \seitenw{56} eine falsch, ist der andere wahr) der Wahrscheinlichkeit (Nicht immer, wenn die Sätze $A, B, C,..$ wahr werden, wird auch wahr $M$, aber das Verhältniß der Fälle, in denen dieß geschieht, zu der ganzen Menge der Fälle ist $m : m+n$) \umA Um diesen Behauptungen des Vfs. beizustimmen, bedarf es durchaus nicht, auch nur in einer einzigen seiner früheren Lehren, die etwas Besonderes hatte, seiner Meinung gewesen zu seyn. Man widerspreche ihm, wenn man kann, in Allem: und man wird doch nicht widersprechen können, daß \zB\ das Wenn und So in dem Satze: \danf{Wenn Leipzig nördlicher liegt als Dresden, so sind die Wintertage in Leipzig kürzer als in Dresden,} wesentlich nur den Sinn hat, daß man die Vorstellungen: Leipzig und Dresden, als veränderlich betrachten solle, und daß, so oft man an ihre Stelle ein paar solche Vorstellungen setzt, dabei der Satz: \danf{L. liegt nördlicher als D.,} wahr wird, auch der Satz: \danf{In L. sind die Wintertage kürzer als in D.,} wahr werde. Gibt sich das doch einigermaßen schon dadurch zu erkennen, daß wir statt jenes Wenn auch ein Sobald oder So oft zu setzen pflegen. Denn da es ausgemacht ist, daß ein Satz, so lange sich nichts in seinen Theilen ändert, nicht bald wahr, bald wieder falsch werden könne: so verrathen ja die erwähnten Ausdrücke schon, daß wir an eine Änderung des Satzes denken. Doch spricht man die Sache nicht zuweilen noch viel deutlicher aus, wenn man sagt, daß man die Orte Leipzig und Dresden hier nur wie Beispiele zu betrachten habe, statt deren auch was immer für andere Orte genannt werden könnten? Ähnlicher Weise gibt es sich schon durch den Ausdruck, dessen man sich zur Bezeichnung der Unverträglichkeit gegebener Sätze bedient, daß nämlich solche Sätze \danf{nie zugleich wahr werden könnten,} zu erkennen, daß wir sie nicht so, wie sie vorliegen, sondern daß wir in ihnen die ganze Gattung von Sätzen betrachten, welche zum Vorschein kommt, wenn gewisse stillschweigend als veränderlich vorausgesetzte Theile darin beliebig abgeändert werden. Nur unter dieser Voraussetzung läßt sich ja reden von einer Möglichkeit, daß diese Sätze bald wahr, bald wieder falsch, nie aber beide zugleich (\dh\  bei derselben Bestimmung ihrer veränderlichen Theile) wahr werden \seitenw{57} könnten. Doch in der Lehre von den Schlüssen that man noch mehr, man bezeichnete die Theile, die als veränderlich angesehen werden sollen, durch allgemeine Zeichen (Buchstaben), indem man \zB\ sagte: \par
\begin{compactenum}[]
\item Wenn alle $A$, $B$ sind,
\item und alle $B$, $C$ sind: 
\item so sind auch alle $A$ auch $C$.
\end{compactenum}
Endlich bei Untersuchung des Verhältnisses der Wahrscheinlichkeit (womit sich freilich die Mathematiker mehr als die Logiker befaßten) erkannte man es so deutlich, man habe hier nicht die gegebenen Sätze allein, sondern eine ganze Gattung derselben zu betrachten: daß man nur eben aus dieser Betrachtung und durch eine geschickte Berechnung des Verhältnisses, in welchem die Menge der wahren Sätze zur Menge aller stehet, den Grad der Wahrscheinlichkeit bestimmen lehrte. Unglücklicher Weise wurde gerade dieses Verhältniß der Wahrscheinlichkeit in den bisherigen Lehrbüchern der Logik noch gar nicht als ein objectives Verhältniß zwischen den vorliegenden Sätzen, das ihnen zukomme, auch ganz abgesehen davon, ob irgend Jemand sie sich denke oder nicht, sondern es wurde stets als eine bloß subjective Beschaffenheit derselben, wiefern sie Urtheile sind, betrachtet; und nicht einmal ward man sich, wie es scheint, deutlich bewußt, daß diese Beschaffenheit einem Urtheile immer nur in Beziehung auf bestimmte andere Urtheile, \dh\  als ein bloßes Verhältniß zukomme. Jedoch wir glauben, Alles, was hier B.\ lehrt, dürfte zu derjenigen Art von Wahrheiten gehören, auf die man nur einmal aufmerksam gemacht zu werden braucht, um sie gleich zu erkennen und für immer beizubehalten, weil Jeder findet, daß er die Sache sich im Grunde immer so vorgestellt habe, ob er sich gleich nicht so bestimmte Rechenschaft darüber gegeben. Denn daß (um nur bei dem Begriffe der Wahrscheinlichkeit zu bleiben) daß sich der Satz: \danf{Aus dieser mit 99 weißen und einer schwarzen Kugel gefüllten Urne zieht Jemand auf's geradewohl eine heraus,} -- zu dem Satze: \danf{Er zieht eine weiße heraus,} verhalte gleich wie ein Vordersatz zu einem sich mit Wahrscheinlichkeit aus ihm ergebenden Schlußsatze, \seitenw{58} gleichviel ob irgend Jemand diese Sätze denke oder nicht; daß also dieses Verhältniß derselben objectiv sey: das ist doch in der That viel offenbarer, als daß der Grad dieser Wahrscheinlichkeit, was alle Mathematiker wissen, \ensuremath{= \frac{99}{100}}. \par
Sollten die Leser fragen, ob diese Verdeutlichung unsrer eigenen Gedanken nun auch einen Nutzen schaffe: so würden wir entgegnen, daß es in der Wissenschaft immer und überall seinen Nutzen bringe, wenn wir dasjenige, was wir uns bisher bloß dunkel vorgestellt hatten, zu einem deutlichen Bewußtseyn erheben. Für jetzt mag es genug seyn, nur dieses Einzige zu berühren: Ist B.'s hier vorgetragene Theorie richtig, so wird man in Zukunft, so oft man von irgend einem der aufgezählten Verhältnisse zwischen Sätzen sprechen wird, also \zB\ bei jedem hypothetischen Urtheile (bei jedem Wenn, so) erst dann glauben dürfen, ganz deutlich angegeben zu haben, was man jetzt eigentlich behauptet, wenn man entweder ausdrücklich sagte, oder wenn es doch aus dem Zusammenhange der Rede ohne alle Zweideutigkeit entnommen werden kann, welche Vorstellungen in diesen Sätzen man eben als die veränderlichen betrachtet wissen wolle. \par
Nach diesen allgemeinen Andeutungen halten wir es nicht mehr für nöthig, unsere Leser in das Detail dieses Abschnitts einzuführen. Wir wollen sie weder mit einer Aufzählung aller eigenthümlichen Verhältnisse, die B.\ hier aufstellt (obgleich ein jedes derselben in wissenschaftlichen Untersuchungen, \zB\ in der Mathematik seine \DruckVariante{häufige}{haüfige} Anwendung findet); noch weniger mit Anführung der verschiedenen Lehrsätze, die er über diese Verhältnisse mittheilt, aufhalten, um sie nicht zu ermüden, da es auch in den folgenden Abtheilungen des Buches noch manches Andere gibt, wofür wir ihre Aufmerksamkeit in Anspruch nehmen wollen. \par
\gliederungslinie\par
Darum werden wir auch die in dem kurzen Abschnitte, der sich nur mit Auffindung der Bestandtheile solcher Sätze beschäftigt, welche die früher betrachteten Verhältnisse aussagen, (S.\,197---211) ganz übergehen, da es wenig verschlägt, \seitenw{59} ob man in diesen Bestimmungen dem Vf.\ beipflichte oder nicht, hat man ihm einmal nur eingestanden, daß er das Wesen jener Verhältnisse selbst richtig aufgefaßt habe. Aus dem letzten Abschnitte aber: \danf{Einige Sätze, die ihres sprachlichen Ausdruckes wegen einer \DruckVariante{Erläuterung}{Erlaüterung} bedürfen,} (S.\,211--245) können wir nicht umhin, Mehreres auszuheben. \par
Eine merkwürdige Bestätigung zweier Behauptungen B.'s, nämlich daß jeder Satz wesentlich von der Form: $A$ hat $b$, sey, und daß die Subjectvorstellung $A$ in jedem wahren Satze eine gegenständliche seyn müsse, gewähren die räthselhaften Sätze, deren sprachlicher Ausdruck die scheinbare Subjectvorstellung: Nichts, hat, wie: Nichts ist besser als Vertrauen auf Gott in Leiden; oder allgemein: Nichts hat (die Beschaffenheit) $b$. Wäre die Vorstellung Nichts in der That so, wie es scheint, die Subjectvorstellung dieser Sätze, so widersprächen sie dem bekannten Grundsatze: \BUlat{Non entis nullae sunt affectiones}. Sagen wir aber den Lesern, wie solche Sätze von B.\ ausgedrückt werden, nämlich: Die Vorstellung eines Etwas, das (die Beschaffenheit) b hätte, \BUhaben{hat} keine Gegenständlichkeit: so stimmen sie wohl sogleich bei, und begreifen, daß es nun allerdings einen Gegenstand, von welchem der Satz handelt, gebe. Denn nicht die Vorstellung: Etwas, das (die Beschaffenheit) b hat, selbst, sondern erst eine Vorstellung von dieser Vorstellung ist es, welche die eigentliche Subjectvorstellung in dem so ausgedrückten Satze bildet; also allerdings eine Vorstellung, die einen Gegenstand hat; nämlich die gegenstandlose Vorstellung selbst hat sie zu ihrem Gegenstande. (\WLpar{II}{170})\WL{II}{170} \par
Aus welchen logischen Theilen die Sätze von der Form: Einige $S$ sind $P$, bestehen, wenn sie so aufgefaßt werden sollen, daß sie wahr bleiben, sowohl wenn jedes $S$, als auch wenn nur ein einziges $S$ ein $P$ ist, haben wir oben schon gesehen. Aber gar oft nehmen wir dergleichen Sätze auch in einem ganz anderen Sinne, \zB\ so, daß wir sagen wollen, die Beschaffenheit $b$ komme zwar gewissen, aber nicht allen, an der Zahl mehreren, aber nicht eben beträchtlich vielen $A$ \seitenw{60} zu. Da lautet denn also unser Satz eigentlich: \danf{Die Menge der $A$, welche $B$ sind, \BUhaben{hat} die Beschaffenheit eines Ganzen, welches aus einer geringen Anzahl von Theilen zusammengesetzt ist.} -- Es gibt aber der Bedeutungen, die wir mit diesem und ihm ähnlichen sprachlichen Ausdrücken verbinden, noch mehrere; in Betreff deren wir auf \WLpar{II}{173} verweisen. Schon das Gesagte reicht hin, zu zeigen, daß die bisherige Lehre der Logiker von den particulären Sätzen sehr mangelhaft sey, da sie alle diese Bedeutungen nicht unterscheidet, mindestens nie versucht, uns die Bestandtheile, aus welchen diese verschiedenen Sätze bestehen, anzugeben. \par
Nicht weniger Tadel dürfte es verdienen, daß man auch die mancherlei Bedeutungen der Sätze mit Entweder Oder in den bisherigen Lehrbüchern nicht unterschieden habe. Denn es kann doch gewiß Niemand in Abrede stellen, daß das Entweder Oder eine ganz andre Bedeutung hat in dem Satze: \danf{Um Mitglied jener Akademie zu werden, muß man entweder Philosoph, oder Mathematiker oder Historiker seyn;} und in dem Satze: \danf{Wir haben jetzt entweder Frühling oder Sommer oder Herbst oder Winter;} indem der erste Satz so zu verstehen ist, daß jedes aufzunehmende Mitglied wenigstens einem der Begriffe: ein Philosoph, ein Mathematiker, ein Historiker, unterstehen muß, aber auch mehreren unterstehen kann; der zweite Satz dagegen so, daß es erlaubt seyn soll, aus dem Vorhandenseyn des Einen Falles sofort auf die Abwesenheit aller übrigen zu schließen. Eben so einleuchtend ist es, daß in dem Satze: \danf{diese Blüthe ist entweder männlich oder weiblich oder ein Zwitter,} und in dem Satze: \danf{Jede Blüthe ist entweder männlich oder weiblich oder Zwitter,} -- die beiden Vorstellungen: \danf{diese Blüthe,} und \danf{jede Blüthe,} nicht auf dieselbe Weise mit den übrigen Theilen zusammenhangen, wie man dem bloßen Ausdrucke nach wohl sollte annehmen können. Durch den ersten Satz will man erklären, daß unter den drei Sätzen: Diese Blüthe ist männlich, ist weiblich, ist ein Zwitter, -- ein wahrer sey; durch den zweiten Satz aber will man gewiß nicht sagen, daß unter den drei Sätzen: jede Blüthe ist männlich, weiblich, ein Zwitter, -- ein wahrer sey. Man sieht also \seitenw{61} jedenfalls, daß auch in der Lehre von den disjunctiven Sätzen noch manche Lücke sey, und prüfe, ob sich nicht annehmen lasse, was B.\ (\WLpar{II}{181} verb. mit \WLpar{II}{165}) hierüber vorbringt. Wir heben nur Einiges aus, überlassen es aber dem Scharfsinne unserer Leser zu errathen, welche der obigen Beispiele unter die folgenden Formen gehören.  \danf{Die Vorstellung eines wahren Satzes unter den Sätzen \ensuremath{A}\Druckfehlerkorr{\ensuremath{,}}{\ensuremath{.}} \ensuremath{B}, \ensuremath{C}... hat Gegenständlichkeit.} -- \danf{Die Vorstellung eines wahren Satzes unter den Sätzen $A$\Druckfehlerkorr{\ensuremath{,}}{\ensuremath{.}} $B, C,$ \Druckfehlerkorr{...}{..} ist eine Einzelvorstellung.} -- \danf{Die Vorstellung eines wahren Satzes unter den Gruppen von Sätzen, welche zum Vorscheine kommen, wenn an die Stelle der veränderlichen Theile $i, j,...$ in den Sätzen $A, B, C,...$ beliebige andere, doch immer nur solche treten, wobei die Subjectvorstellungen Gegenständlichkeit behalten, -- ist eine Einzelvorstellung.} \par
\gliederungslinie\par
Aussagen der Nothwendigkeit, Möglichkeit und Zufälligkeit, mit deren Zerlegung sich \WLpar{II}{182} beschäftigt, sind in Aller Munde, und die wichtigsten Untersuchungen über Gott, Freiheit \umA\ hangen von einer richtigen Bestimmung dieser Begriffe ab. Gleichwohl hat man in neuester Zeit zu diesem Zwecke kaum etwas Mehreres gethan, als daß man das Nothwendige als das, was nicht nicht seyn könne, erklärte. Man wollte sonach den Begriff des Nothwendigen aus dem des Könnens, \dh\  des Möglichen ableiten. Ist aber dieser schon einfach, oder ist er nur einfacher? Wohl schwerlich; denn von jeher hat man beide Begriffe durch den des Widerspruchs zu bestimmen gesucht, und das Nothwendige als etwas, dessen angenommenes Nichtseyn auf einen Widerspruch führt, das Mögliche aber als etwas, dessen angenommenes Seyn auf keinen Widerspruch führt, erkläret. Und sollte dieß wohl so unrichtig seyn? Nur daran fehlte es noch, daß nicht bestimmt wurde, mit welcher Art von Wahrheiten das angenommene Seyn oder Nichtseyn in einen Widerspruch gerathe. Denn wollte man Alles für nothwendig erklären, dessen angenommenes Nichtseyn mit irgend einer Wahrheit in Widerspruch tritt, und das allein \seitenw{62} für möglich, dessen angenommenes Seyn gar keiner Wahrheit widerstreitet: so fiele Nothwendiges sowohl als Mögliches ganz mit dem Wirklichen zusammen; denn die Annahme, daß etwas nicht wirklich sey, das gleichwohl wirklich ist, widerspricht mindestens der einen Wahrheit, die dessen Wirklichkeit aussagt; und somit müßte ein jedes Wirkliche auch schon nothwendig heißen; woraus denn weiter folgt, daß jedes Nichtwirkliche auch schon unmöglich, das Mögliche also stets wirklich seyn müßte. Wollen wir also dennoch Mögliches, Wirkliches und Nothwendiges dem Umfange nach unterscheiden: so müssen wir die Wahrheiten, welche hier in Betracht zu ziehen sind, auf eine besondere Classe beschränken. Und was bietet sich natürlicher dar, als die Classe der reinen Begriffswahrheiten? Es läßt sich ja beinahe gar nicht bezweifeln, sobald man es einmal nur aussprechen hört, daß man sich unter dem Nothwendigen denke etwas, dessen angenommenes Nichtseyn einer reinen Begriffswahrheit widerspricht; und unter dem Möglichen etwas, dessen angenommenes Seyn keiner reinen Begriffswahrheit widerspricht. Die Frage ist nur, ob man, was eine reine Begriffswahrheit sey, erklären könne, ohne die Begriffe der Nothwendigkeit oder Möglichkeit schon vorauszusetzen? B.\ hat nun, wie wir schon aus dem Obigen wissen, gezeigt, daß wir dieß allerdings vermögen, indem er erklärte, daß eine reine Begriffswahrheit eine solche sey, in deren Bestandtheilen keine einzige Anschauung, \dh\  keine einzige einfache Vorstellung, welche nur einen einzigen Gegenstand hätte, vorkommt. Mit aller Deutlichkeit also wissen wir anzugeben, was möglich, was nothwendig (auch was zufällig) heiße, sobald wir B.'s Erklärung einer Anschauung annehmen. Allein selbst wer ihm diese Erklärung nicht zugestehen will, wenn er nur irgend eine andere Erklärung kennt, oder sich wenigstens versichert hält, daß der Begriff einer Anschauung, und folglich auch der eines Begriffes und einer Begriffswahrheit auf eine nicht die Begriffe der Nothwendigkeit oder der Möglichkeit voraussetzende Weise bestimmbar sey: der wird noch immer die hier gegebene Erklärung dieser letzteren als befriedigend ansehen müssen. \seitenw{63}\par
Eine Probe, von welcher Wichtigkeit auch manche Untersuchungen der Wissenschaftslehre sind, die auf den ersten Blick ganz unfruchtbar scheinen, liefert \WLpar{II}{183}, worin B.\ einige Sätze mit Zeitbestimmungen in ihre logischen Bestandtheile zerlegt, und die sehr einleuchtende Bemerkung macht, daß in Sätzen der Art, wie: \danf{der Gegenstand $A$ hat zu der Zeit $t$ die Beschaffenheit $b$,} die Bestimmung: zu der Zeit $t$, weder zur Copel, noch zum Prädicate, sondern zur Subjectvorstellung gehöre, dergestalt daß \zB\ der Satz: dieß Blatt ist heute grün, eigentlich so aufgefaßt werden muß: dieß Blatt am heutigen Tage \BUhaben{hat} eine grüne Farbe. Nur so vermeiden wir einmal den Übelstand, von dem bekannten Grundsatze, daß keinem Gegenstande widersprechende Beschaffenheiten zukommen, eine Ausnahme machen zu müssen bei allen veränderlichen Dingen, oder ihn für diese nur durch den Zusatz: zu derselben Zeit, retten zu können. Wenn eine und eben dieselbe Substanz in der Zeit $t$ die Beschaffenheit $b$, in einer andern $t+\Theta$ aber die Beschaffenheit \BUlat{non} $b$ hat: so ist, wenn solche Zeitbestimmungen in die Subjectvorstellung gehören, in den beiden Sätzen: $A$ in der Zeit $t$ -- hat $b$, und: $A$ in der Zeit $t+\Theta$ -- hat \BUlat{non} $b$, nur von derselben Substanz, nicht aber von demselben Gegenstande die Rede, weil die Subject-, \dh\  Gegenstandsvorstellungen der beiden Sätze: $A$ in der Zeit $t$, und $A$ in der Zeit $t+\Theta$, einander ausschließen. -- Allein wie überraschend und wichtig ist nicht die Anwendung dieser logischen Lehre, welcher wir in der zweiten Ausgabe der Athanasia S.\,292 u. 3 begegnen, wo daraus ein Beweis für die endliche Fortdauer aller Substanzen geführt wird. Könnte es nämlich für irgend eine Substanz $A$ einen Zeitpunkt $t$ geben, in welchem sie keine Wirklichkeit hat; so müßte ein Satz von der Form: Die Substanz $A$ in der Zeit $t$ \BUhaben{hat} keine Wirklichkeit, wahr seyn, und hätte doch eine Subjectvorstellung, die gegenstandlos ist. Denn weil die Substanz $A$ zu der Zeit $t$ nicht existiren soll, so wäre der Begriff: $A$ in der Zeit $t$, \dh\  der Begriff einer Substanz oder eines Wirklichen in der Zeit $t$, wo es doch keine Wirklichkeit hat, offenbar gegenstandlos. \seitenw{64} \par
Wie das Hauptstück von den Vorstellungen, so schließt B.\ auch das von den Sätzen mit einem Anhange, darin er die Darstellung Anderer beurtheilt. Das Merkwürdigste ist der hier gelieferte Beweis, daß jene Tafel der Urtheile, aus welcher Kant seine Tafel der Kategorien ableitete, jene von so viel andern Logikern, auch von Hegel\pindex{Hegel, Gottfried Wilhelm} noch unverändert beibehaltene Tafel, auf lauter Irrthümern beruhe. Schon die vier Gesichtspunkte der Quantität, Qualität, Relation und Modalität sind nicht nur an sich selbst fehlerhaft, sondern werden auch auf den vorliegenden Gegenstand fehlerhaft angewendet. So, um nur Einiges zu sagen, umfaßt die Qualität einer Sache schon ihre Quantität, wenn anders diese, wie in dem vorliegenden Fall, eine innere Beschaffenheit der Sache ist; bestehet sie aber in einem bloßen Verhältnisse, dann gehört sie eben deßhalb zur Relation, dahin auch jederzeit die Modalität gehört, wenn man sie als ein Verhältniß zu unserm Erkenntnißvermögen erkläret. Wie irrig ist es ferner, die Urtheile der Quantität nach in allgemeine, besondere und einzelne einzutheilen; da die Subjectvorstellung höchstens zu der Eintheilung in allgemeine und Einzelurtheile berechtigt, je nachdem sie der Gegenstände mehrere oder nur einen einzigen vorstellt; die sogenannten besondern Urtheile aber (Einige S sind P) eigentlich wieder nur zu den Einzelurtheilen gehören, da sie nichts Anderes als die Gegenständlichkeit einer Vorstellung (nämlich der eines Etwas, das zugleich S und P ist) aussagen! Welche verschiedene Urtheilsformen hat man nicht auf das Gewaltsamste unter dem Gesichtspunkte der Relation vereinigt, wenn man gesagt, daß alle Urtheile hiernächst entweder kategorisch oder hypothetisch oder disjunctiv wären! Denn wie man die kategorischen Urtheile beschreibt, so sind ja im Grunde alle Urtheile beschaffen. Die hypothetische Form aber setzt, wo sie wesentlich ist, veränderliche Bestandtheile in den gegebenen Sätzen voraus; doch ist ihr dieses nicht ausschließlich eigen, sondern auch in den Aussagen einer Verträglichkeit oder Unverträglichkeit \umA\ geschieht dasselbe, so wie es auch in den disjunctiven Sätzen der Fall seyn kann. Überhaupt paßt die Beschreibung, die man von diesen \seitenw{65} letztern gibt, nur auf eine einzige Art derselben. Unter dem Titel der Qualität werden bejahende, verneinende (wo die Verneinung zur Copel) und limitirende Urtheile (wo die Verneinung zum Prädicate gehören soll); unter dem Titel der Modalität endlich problematische, assertorische und apodiktische Urtheile aufgeführt (die eine Möglichkeit, Wirklichkeit und Nothwendigkeit aussagen): alle übrigen Urtheilsformen sucht man in dieser Tafel vergebens. \par
Das nun folgende \danf{dritte Hauptstück: von den wahren Sätzen} (S.\,327--390), beschäftigt sich mit Aufstellung einiger Beschaffenheiten, die allen Sätzen nur sofern sie auch wahr sind, also allen Wahrheiten zukommen. Obgleich nun die Rede noch immer von bloßen Sätzen und Wahrheiten an sich ist: so mag doch auch derjenige, der nur gedachte Wahrheiten zugibt, nachsehen, ob er das hier Gelehrte in seiner Weise verstanden nicht gleichfalls annehmen könne und müsse. Da kann nun, was der Vf.\ gleich \WLpar{II}{196} als eine allgemeine Beschaffenheit aller Wahrheiten aufstellt, daß die Subjectvorstellung derselben immer eine gegenständliche Vorstellung seyn müsse, Jeder leicht zugestehen, wenn er erwägen will, daß es in einem wahren Satze doch immer einen Gegenstand, von dem er handelt, \dh\  von dem er eine Beschaffenheit aussagt (welche demselben in der That zukommt), geben müsse; zumal da der Vf.\ gezeigt hat, wie jene wenigen Fälle, die eine scheinbare Ausnahme machen, zu erklären seyen. Wenn es aber weiter heißt, daß auch die Prädicatvorstellung in einem jeden Satze eine gegenständliche Vorstellung, näher eine Beschaffenheitsvorstellung seyn müsse: so ist dieß freilich nur unter der Voraussetzung gesprochen, daß alle Sätze der Form: A hat b, unterstehen. Übrigens ist dieß eine Behauptung von keinen weiteren Folgen. \par
Wie viel Gewicht die Anhänger der kritischen Philosophie auf die Unterscheidung zwischen analytischen und synthetischen Sätzen, und auf die Behauptung gelegt, daß es auch synthetische Wahrheiten und Erkenntnisse für uns Menschen \seitenw{66} gebe, ist bekannt. Gleichwohl gelang es ihnen nicht, die Sache zu einer vollkommnen Klarheit zu erheben, denn noch gegenwärtig gibt es gar Viele, die den ganzen Unterschied verwerfen. B.\ nun faßt \WLpar{II}{148} den Begriff der analytischen Sätze weiter, den der synthetischen aber enger. Denn statt bei der Erklärung zu bleiben, daß man nur solche Sätze, bei denen die Prädicatvorstellung in der Subjectvorstellung schon als Bestandtheil vorkömmt, also nur solche, wie: \danf{$A$ ist $A$,} oder: \danf{$A$, welches $B$ ist, ist $A$,} oder endlich: \danf{$A$, welches $B$ ist, ist $B$,} analytisch nennen wolle, legt er diese Benennung einem jeden Satze bei, darin nur irgend eine Vorstellung steckt, die nach Belieben abgeändert werden kann, ohne die Wahrheit desselben zu verletzen, so lange nur die Subjectvorstellung noch eine gegenständliche geblieben. Nur alle noch übrigen Sätze nennt er synthetisch. Nach einer solchen Erklärung sind also nicht nur die obigen drei Satzformen analytisch zu nennen, weil sie wahr bleiben, was man auch immer an die Stelle von A und B setzt: sondern auch schon der Satz: \danf{In diesem Dreiecke betragen die sämmtlichen Winkel zwei rechte,} wird analytisch heißen, weil auch in ihm ein Bestandtheil, nämlich die Vorstellung Dieß vorkommt, der nach Belieben abgeändert werden kann, ohne seine Wahrheit zu stören, so lange er nur ein Subject behält, \dh\  so lange das Dieß nur ein wirkliches Dreieck bezeichnet. Diese Erweiterung kann man genehmigen oder verwerfen, so bleiben doch B.'s Beweise dafür, daß es synthetische Wahrheiten gebe, brauchbar. Denn gibt es dergleichen in seinem engeren Sinne, so muß es deren um so gewisser geben im weiteren Sinne. Nun mögen die Leser selbst sagen, ob sie \zB\ folgenden Beweis B.'s, den wir aus mehreren \WLpar{II}{197} u. \WLpar{I}{64} ausheben wollen, überzeugend für sich finden. Es gibt gewiß Wahrheiten von der Form: \danf{Einige $A$ sind $B$,} (\dh\  die Vorstellung eines Etwas, das die Beschaffenheiten $a$ und $b$ hat, hat Gegenständlichkeit); ingleichen auch Wahrheiten von der Form: \danf{Kein $A$ ist $B$} (\dh\  die Vorstellung eines Etwas, das die Beschaffenheiten $a$ und $b$ hätte, hat keinen Gegenstand). Es gibt unter diesen Wahrheiten gewiß auch solche, worin die Vorstellungen $a$ und $b$ einfach sind; \seitenw{67} denn sicher gibt es auch einfache Beschaffenheitsvorstellungen, die sich vereinigen, und andere, welche sich nicht vereinigen lassen. Wenn aber $a$ und $b$ einfach sind, dann sind jene beiden Sätze einleuchtender Weise synthetisch; weil wir ja weder die Vorstellung $A$ noch $B$, noch sonst eine andere in ihnen beliebig abändern können, ohne ihre Wahrheit zu stören.\par
\gliederungslinie\par
Das Wichtigste in diesem ganzen Hauptstücke ist die Behauptung (\WLpar{II}{198}), es gebe unter den Wahrheiten an sich einen objectiven, \dh\  von der Art, wie wir sie etwa erkennen, ganz unabhängigen Zusammenhang, vermöge dessen wir einige derselben als Gründe und andere als deren Folgen betrachten dürfen, wozu dann noch Einiges von demjenigen kömmt, was \WLpar{II}{199--221} über die nähere Beschaffenheit dieses ganz eigenthümlichen Verhältnisses -- großentheils freilich nur als eine noch unerwiesene Ansicht -- zur ferneren Prüfung aufgestellt wird; wie daß reine Begriffswahrheiten nie in Anschauungswahrheiten gegründet seyn können; daß keine Wahrheit, von welcher eine andere objectiv abhängt (abfolgt), zusammengesetzter als diese sey; daß es auch Wahrheiten gebe, die keinen weiteren Grund ihrer Wahrheit haben (Grundwahrheiten); daß die Begriffe von Ursach und Wirkung aus jenen von Grund und Folge entspringen, so zwar, daß irgend ein Wirkliches $B$ Wirkung eines anderen $A$ heißt, wenn die Wahrheit: $B$ ist, zu der Wahrheit: $A$ ist, in dem Verhältnisse einer objectiven Abfolge stehet. \par
Ist nun wohl dieser Begriff eines Grundes und einer ihr zugehörigen Folge, als eines Verhältnisses, das nicht zwischen wirklichen Dingen, sondern nur zwischen Wahrheiten an sich obwaltet, etwas ganz Neues? Gewiß nicht; denn schon die alten griechischen Weltweisen verlangten, daß man in einem streng wissenschaftlichen Vortrage Beweise führen solle, die nicht nur zeigen, daß (\BUgriech{<'oti}) etwas sey, sondern auch, warum (\BUgriech{di'oti}) es sey. Konnten sie bei diesem Letzteren wohl an etwas Anderes denken, als was B.\ die objective Begründung einer Wahrheit nennet? Nur in unse\seitenw{68}rer Zeit scheint man auf diesen Unterschied beinahe vergessen zu haben. Aber ist es wohl recht, die Vordersätze, durch deren Betrachtung die Erkenntniß einer Wahrheit in uns vermittelt werden kann (Erkenntnißgründe), nicht zu unterscheiden von jenen Wahrheiten, in welchen ihr objectiver Grund liegt? Der ersteren kann es gar mannigfaltige geben, der objective Grund aber ist nur einer. Und wie deutlich ist er zuweilen nicht von jenen unterschieden! Daß man nicht lügen, daß man dem Darbenden von seinem Überflusse gern etwas zukommen lassen, daß man nicht einmal dem vernunftlosen Thiere ohne Noth Schmerzen verursachen solle: das Alles sind Wahrheiten, die wir auf mancherlei Weise erkennen, \zB\ auch daraus, daß alle Menschen darüber einstimmen: allein der objective Grund, auf welchem sie beruhen, ist ein ganz anderer; er bestehet (unsers Erachtens) nur in der Wahrheit, daß wir verpflichtet sind zu Allem, wodurch die Summe des allgemeinen Wohls vermehrt wird. \par
Nun erhebt sich aber die Frage, ob wohl auch Jemand, der den Begriff von Wahrheiten an sich verwirft, und nur gedachte Wahrheiten zugibt, dennoch einen Zusammenhang zwischen denselben annehmen könnte, mehr oder weniger ähnlich demjenigen, den uns B.\ hier unter dem Namen des objectiven beschreibet? Und dieß ist, glauben wir, zu bejahen. Denn auch ein Solcher könnte sich ja recht füglich ein gewisses Verhältniß zwischen den Wahrheiten denken, welches in sofern den Namen eines objectiven Zusammenhanges verdienen könnte, in wiefern es auf bestimmten, von unsrer zufälligen Erkenntnißweise derselben ganz unabhängigen Umständen beruhet. Und wenn er festsetzen würde, daß \zB\ immer nur ein solcher Inbegriff reiner Begriffswahrheiten, welcher der einfachste ist unter allen, aus denen sich eine gewisse andere Begriffswahrheit ableiten läßt, ihr objectiver Grund genannt werden solle; wenn er dergleichen ganz mit \WLpar{II}{221} übereinstimmende Kriterien für das Vorhandenseyn seines Verhältnisses aufstellen wollte: wäre dann nicht die größte Ähnlichkeit zwischen seiner und B.'s Lehre bis auf den einzigen Umstand vorhanden, daß er von wahren Gedanken redete, während B.\ von Wahr\seitenw{69}heiten an sich redet? Auch er \zB\ könnte den Satz von der Beförderung des allgemeinen Wohles eben so gut wie B.\ als den obersten Grundsatz der ganzen Sittenlehre betrachten, so fern er fände, daß die Verbindung dieses Satzes mit andern reinen Begriffswahrheiten den einfachsten Inbegriff solcher Wahrheiten darbeut, aus welcher sich alle übrigen praktischen Wahrheiten ableiten lassen. \par
Die Lehre \danf{von den Schlüssen} behandelt B.\ -- abweichend hierin von andern Logikern -- als eine bloße Abtheilung der Lehre von den Sätzen (im vierten Hauptstücke S.\,391--568), weil er die Schlüsse als eine besondere Art von Sätzen betrachtet. Denn wenn man (meint er) den Satz $M$ für einen aus den Vordersätzen $A, B, C ...$ sich ergebenden Schlußsatz erklärt: so spricht man hiemit nur folgenden Satz aus: \danf{Jeder Inbegriff von Vorstellungen, der an der Stelle gewisser in $A, B, C, ..., M$ als veränderlich gedachter, die Sätze $A, B, C ...$ wahr macht, macht auch den Satz $M$ wahr.} Mag man in diese Ansicht eingehen oder nicht (obwohl wir glauben, und es auch schon gesagt, daß man in Schlüssen das Vorhandenseyn gewisser veränderlicher Theile von jeher anerkannt habe): so hindert dieß auf keinen Fall, den Lehren dieses Hauptstückes beizutreten. Das Wichtigste aber, und wovon Jeder, der auch nur einige §§. aufmerksam durchgelesen hat, überzeugt werden dürfte, ist, daß jene wenigen Schlußweisen, welche in den gewöhnlichen Lehrbüchern der Logik vorgetragen werden, die sogenannten unmittelbaren Schlüsse, der Syllogismus, die Induction und Analogie, gar nicht die einzigen einfachen Schlußweisen sind, deren wir uns zu unserm Denken bedienen und unumgänglich bedürfen. Ob man aber die hier versuchte Aufzählung vollständig finden werde (sie macht eigentlich selbst keinen Anspruch auf Vollständigkeit); ob dem Vf.\ bei so vielen Schlüssen nicht das Menschliche begegnet sey, selbst irgendwo fehlgeschlossen zu haben:\BUfootnote{%\par
Jedenfalls wird man aber, um solche Fehlschlüsse zu finden, die Augen besser aufthun müssen, als der Rec. in der Jenaer allg. \seitenw{70} Literaturzeit. 1838. Oct. Nr. 185 u. 6., der eine Prämisse für die Conclusion angesehen hat. Er hätte sich mehr Dank verdient, wenn er die Unrichtigkeit eines gleich im I. Bd. S.\,441, Z. 3--7 \par
angeführten Beispiels verträglicher Vorstellungen aufgedeckt hätte; denn damit ist offenbar gefehlt, und es muß etwa so lauten: \par
Unter den vier Vorstellungen:
\begin{compactenum}[]
\item Wurzeln der Gleichung $(x-a) (x-b) = 0$
\item W.~d.~G.\ $(x-a) (x-c) = 0$ 
\item W.~d.~G.\ $(x-b) (x-d) = 0$ 
\item W.~d.~G.\ $(x-c) (x-d) = 0$ 
\end{compactenum}
sind immer nur drei unter einander verträglich.} 
darauf kommt es hier gar \seitenw{70} nicht an. Das Mangelnde läßt sich ergänzen, das etwa Irrige verbessern, und immer noch könnte dem Vf.\ das Verdienst, hier neue Untersuchungen veranlasset zu haben, bleiben. \par
Um aber die Leser zu überzeugen, daß die bisherige Syllogistik Schlußarten, die \DruckVariante{unläugbar}{unlaügbar} zum Denken nothwendig sind, übergangen habe, wollen wir nur an drei gewiß einfache Schlüsse, die Jeder täglich macht, erinnern: \par
1) Aus der beliebigen Anzahl von Sätzen: \\
Jedes $A$ ist ein $P$\\
Jedes $B$ ist ein $P$\\
Jedes $C$ ist ein $P$ \usw\ \\
folgt doch gewiß: Jedes der Dinge $A, B, C ...$ ist ein $P$. \par
2) Aus der beliebigen Anzahl von Sätzen: \\
Jedes $S$ ist ein $A$\\
Jedes $S$ ist ein $B$\\
Jedes $S$ ist ein $C$ \usw\ \\
folgert man ohne Anstand: Jedes $S$ ist ein $A$ u. $B$ u. $C$ \usw\ \par
3) Aus der beliebigen Anzahl von Sätzen: \\
Jedes $A$ ist ein $P$ \\
Jedes $B$ ist ein $Q$ \\
Jedes $C$ ist ein $R$ \usw\ \\
folgern wir Alle: In dem Ganzen, das aus den Theilen $A, B, C,...$ bestehet, fehlet es weder an einem $P$, noch an einem $Q$, noch an einem $R$ \usw\ \par
Zwar dürfte man einwenden, daß die Sätze, welche wir hier für Schlußsätze ausgegeben, von den Prämissen selbst \seitenw{71} dem Gedanken nach gar nicht verschieden, nicht wirklich ein einziger Satz, sondern nur eine sprachliche Zusammenziehung mehrerer Sätze wären. Aber die Unrichtigkeit dieser Behauptung muß den Lesern einleuchten, sobald sie nur erwägen, daß jene Schlußsätze ganz neue in keiner ihrer Prämissen vorkommende Begriffe enthalten, was besonders bei dem dritten ganz unverkennbar ist. \par

\WLsec{III}{269}{391}
Indem wir hier unsern Bericht über die \danf{Elementarlehre} schließen, und zu dem dritten Bande greifen, können wir unseren Lesern mindestens die Versicherung geben, daß sie fortan nicht mehr so trocknen und abstracten Untersuchungen, als die bisherigen waren, begegnen, weil wir sie nicht mehr mit Sätzen und Vorstellungen an sich, sondern mit deren Erscheinungen in dem Gemüthe, und mit der Art ihrer Bearbeitung für den Zweck der Vermehrung unsrer Kenntnisse, endlich zu dieser letzteren Darstellung in brauchbaren Lehrbüchern zu beschäftigen haben. \par
Zunächst liegt uns der dritte Theil des Ganzen, d. i. die \danf{Erkenntnißlehre} (Bd. III. S.\,1--292) vor, und hier werden die Leser hoffentlich gleich in dem ersten Hauptstücke: \danf{von den Vorstellungen} (S.\,1--107) Einiges antreffen, was ihnen eben so verständlich als anziehend seyn dürfte. Was man nur eine einzige, was mehrere, was gleiche Vorstellungen nenne (\WLpar{III}{273}), was Stärke oder Lebhaftigkeit einer Vorstellung sey (\WLpar{III}{275}), was klare und dunkle, deutliche und verworrene Vorstellungen (\WLpar{III}{280} u. \WLpar[1.]{III}{281}), wann eine Vorstellung beginne oder ende (\WLpar{III}{282}), daß jede eine gewisse Spur hinterlasse (\WLpar{III}{283}), daß zu jeder Vorstellung, die wir für gegenständlich halten, bei haüfigerem Gebrauche sich auch noch ein von ihr selbst zu unterscheidendes Bild ihres Gegenstandes hinzugeselle (\WLpar{III}{284} u. \WLpar[7.]{III}{287}): alle diese Punkte werden sie auf eine Weise, die etwas Neues hat, ausgeführt finden. Die ganze Zeit, da eine und dieselbe objective Vorstellung, wenn gleich in mancherlei Zustande (bald mehr, bald minder lebhaft \udgl ), doch ununterbrochen in unserm Gemüthe vorgestellt wird, \seitenw{72} betrachtet der Vf.\ als die Dauer einer einzigen subjectiven Vorstellung; und nur, wenn entweder dieselbe objective Vorstellung in demselben Menschen zu unterbrochenen Zeiten erscheint, oder wenn ihre Erscheinung in verschiedenen Wesen vorgeht, oder wenn schon die objective Vorstellung, welche den Stoff der subjectiven bildet, eine andre geworden, will er von mehreren subjectiven Vorstellungen gesprochen wissen. Hiegegen läßt sich wohl nichts einwenden. -- \DruckVariante{Unläugbar}{Unlaügbar} ist es ferner, daß wir das Vermögen besitzen, uns unsere eigenen Vorstellungen selbst wieder vorzustellen: wie anders könnten wir von ihnen sprechen, und sagen, daß wir sie haben? Dieß Vorstellen derselben kann aber auf zweierlei Weise geschehen: durch eine Vorstellung, die ihrer mehrere umfaßt, \zB\ wenn wir \danf{von den gesammten Vorstellungen, die wir in dieser Nacht gehabt,} sprechen; oder auch durch eine Anschauung, welche nur eine einzige derselben vorstellt; denn ohne Zweifel besitzen wir auch das Vermögen, wie manche andere in uns vorgehende Veränderung, so auch unsere eigenen Vorstellungen uns zur Anschauung zu bringen, \dh\  uns eine andere einfache nur sie allein betreffende Vorstellung von denselben zu bilden. Solche von uns selbst wieder angeschaute Vorstellungen nun sind es, welche wir -- nach des Vfs. Behauptung -- klare Vorstellungen nennen. Vorstellungen aber, von denen wir keine Anschauung haben, nennen wir dunkle Vorstellungen. Deutlich dagegen heißt nach ihm eine Vorstellung, wenn wir auch anzugeben wissen, ob sie einfach, oder, wenn sie zusammengesetzt, aus welchen Theilen sie bestehe. Aus dem Umstande, daß eine jede Vorstellung in uns als eine bestimmte Art von Veränderung nur so lange währet, als die sie hervorbringende Ursache fortwirkt, folgert B., daß der Zustand, der in dem letzten Augenblicke des Einwirkens Statt hatte, von jetzt an, wenn keine neue Einwirkung eintritt, ununterbrochen fortdauern müsse, was er die von der Vorstellung hinterlassene Spur nennt. \par 
Wenn wir von irgend einer unserer Vorstellungen wissen, oder auch nur fälschlicher Weise vermeinen, daß ihr ein Gegenstand entspreche: so pflegen wir uns auch von diesem noch eine eigene Vorstellung und zwar aus den gesammten uns von \seitenw{73} ihm bekannt gewordenen Beschaffenheiten, besonders den sinnlichen, wenn er dergleichen in unserer Meinung hat, zusammenzusetzen; und diese der ersteren sich genau anschließende, gleichwohl von ihr wesentlich verschiedene Vorstellung ist es, welche B.\ das sie begleitende Bild nennt. Was für ein Unterschied ist nicht bei vielen unserer Begriffe, \zB\ von Gott, vom Tode, von den Genüssen und Leiden, die dieser oder jener Lebensstand mit sich führt, zwischen den eben erwähnten Begriffen und den sie (zuweilen ganz unwillkührlich) begleitenden Bildern zu machen! \par 
Sollte man übrigens mit einer oder der andern dieser Erklärungen sich nicht zufrieden stellen können, so besorge man nicht, daß von dem Folgenden darum viel wegfallen werde. Noch weniger hat man dieß zu befürchten bei den Behauptungen der \WLpar[§\BUparformat{279.}]{III}{279} \WLpar[284--6.]{III}{284--286} \WLpar[289.]{III}{289} über den Unterschied zwischen sinnlichen und übersinnlichen Vorstellungen, den Begriff eines Zeichens und einer Menge damit verwandter Begriffe, den Ursprung der Sprache und die merkwürdigsten Thätigkeiten und Zustände unseres Geistes, die das Geschäft des Vorstellens betreffen; wie abweichend von der gewöhnlichen Ansicht auch manche lauten mögen. Wir heben nur Eines zur Beschäftigung des Nachdenkens unserer Leser aus. Der Vf.\ wagt es zu \DruckVariante{äußern}{aüßern}, daß die Vorstellungen: roth, süß, wohlriechend \usw , die man bisher insgemein, wenn nicht als reine, doch gemischte Anschauungen betrachtete, reine Begriffe seyn dürften, Begriffe nämlich von den Gesetzen, nach welchen die Veränderungen in unserer Seele, welche der Gegenstand unserer Anschauungen bei etwas Rothem, Süßem, Wohlriechendem \usw\ sind, vor sich gehen. \par 
Daß er in dem zweiten Hauptstücke: \danf{von den Urtheilen} (S.\,108--205) auch die Urtheile auf eine ähnliche Weise, wie in dem vorhergehenden die Vorstellungen, in klare und dunkle, deutliche und verworrene abtheilt (\WLpar[\BUparformat{295. u. 6.}]{III}{295--296}), daß er \WLpar{III}{299} auch von dem Urtheile behauptet, es hinterlasse Spuren \udgl : das werden ihm \seitenw{74} wohl auch jene Logiker, die es bisher nicht ausdrücklich gelehrt hatten, nicht zum Vergehen anrechnen. Auch daß es unvermittelte (\dh\  nicht erst aus andern abgeleitete) sowohl als auch vermittelte Urtheile gebe (\WLpar{III}{300}), ingleichen was \WLpar{III}{302} über die Entstehungsart der ersteren gesagt wird, daß wir \zB\ im Stande seyn müßten, mindestens einige Urtheile bloß darum zu fällen, weil wir die in denselben vorkommende Subject- und Prädicatvorstellung besitzen: das dürfte zugelassen werden. Einer desto schärferen Prüfung empfehlen wir die eigenthümliche Art, wie \WLpar{III}{301} die Entstehung der Urtheile aus bloßer Wahrscheinlichkeit erklärt wird; da von der Richtigkeit dieser Erklärung abhängt, ob die \WLpar{III}{309} versuchte Erklärung des Irrens das Räthsel löse oder nicht. Unstreitig ist wohl so viel, daß wir auch das bloß Wahrscheinliche mit einem entsprechenden Grade der Zuversicht annehmen, \dh\  daß wir urtheilen, eine Sache sey so, wenn eigentlich nur geurtheilt werden sollte, es sey wahrscheinlich, daß sie so sey. Ob man aber genügend erkläret habe, woher dieß komme, wenn man bloß sagt, daß es aus der Beschränktheit unsers Vorstellungsvermögens komme, welches die minder wichtigen Bestandtheile einer zusammengesetzteren Vorstellung aus dem Bewußtseyn fallen lasse: das ist es, was B.\ selbst nur als Vermuthung vorträgt, und was bestritten werden könnte. \par 
Daß die Behauptungen des langen \WLpar{III}{303} über die Entstehung derjenigen unserer Erfahrungsurtheile, die wir ohne ein deutliches Bewußtseyn vermitteln, zugestanden oder verworfen werden können, ohne Einfluß auf etwas Folgendes, verstehet sich von selbst. Da es indessen doch Mehrere unserer Leser interessiren dürfte, hier wenigstens den Gang dieser Untersuchungen zu erfahren: so bemerken wir, daß der Vf.\ zuerst angebe, wie wir die zwischen unsern eigenen Vorstellungen herrschenden Zeitverhältnisse erkennen. Denn unmittelbar vermögen wir (wie er behauptet), und dieß nur unter gewissen Umständen, hierüber nichts Mehreres zu erkennen, als ob eine Vorstellung jetzt eben gegenwärtig in unserm Inneren sey. Daß aber eine Vorstellung früher als \seitenw{75} eine andere, eine dritte wieder mit jener gleichzeitig gewesen \udgl , das Alles müssen wir erst schließen. Er gibt nun an, aus welchen Kennzeichen wir Solches entnehmen; \zB\ die Gleichzeitigkeit, wenn gewisse Vorstellungen als Theile in einer anderen auftreten; das Früherseyn, wenn Urtheile einander widersprechen, und das Eine des Andern gleichwohl als seiner Theilursache bedarf. Ferner zeigt er, auf welche Weise wir uns versichern, ob wir gewisse Vorstellungen schon mehrmal gleichzeitig gehabt; ob auch die Zwischenzeiten einander beinahe gleich gewesen (wenn nämlich gleiche Vorstellungsreihen dazwischen gefallen); wie wir das Daseyn von Dingen außer uns (dem denkenden Subject) erfahren (nämlich aus den Veränderungen, die mit uns vorgehen); wie wir die Kräfte derselben kennen lernen; wie wir allmählich inne werden, daß unser eigene Wille Ursache mannigfaltiger Erfolge sey; wie wir so nach und nach zu der Kenntniß gelangen, daß uns ein Theil der Außenwelt, als Leib, \dh\  viel unmittelbarer als andere Theile derselben zu Gebote stehe; in welchen Theilen dieses Leibes der Sitz (die Ursache) dieser und jener Empfindungen liege; wie wir die raümlichen Verhältnisse beurtheilen; daß der Sinn des Gesichtes hiezu nicht eben nothwendig sey \umA\ Besonders empfehlen wir nicht zu übersehen, was auf den Einwurf, daß ja auch Thiere und Kinder, die jene Überlegungen gewiß nicht anstellen können, \DruckVariante{räumliche}{raümliche} Verhältnisse \umA\  erkennen, Nr. 31. geantwortet wird. \par 
\gliederungslinie\par
Die Untersuchungen des dritten Hauptstückes: \danf{über das Verhältniß unserer Urtheile zur Wahrheit,} (S.\,206--262) sollen immerhin nicht in die Logik gehören: daß sie für die Philosophie selbst von höchster Wichtigkeit sind, wird Niemand in Abrede stellen. Wir möchten aber Jedem, der das hieher Gehörige im Zusammenhange zu übersehen wünscht, rathen, die diesen Gegenstand betreffenden §§. aus der Fundamentallehre noch einmal nachzulesen, besonders wenn ihm das erste Mal Einiges dunkel geblieben. Wegen der Erklärungen, die der zuweilen vielleicht allzu \seitenw{76} genaue Vf.\ \WLpar{III}{307} über die Begriffe des Erkennens, der Unwissenheit und des Irrthums vorausschickt, braucht man auf keinen Fall mit ihm zu brechen, auch wenn man sie mißbilligen sollte. Denn auch nach des Vf.\ Willen soll es doch bei den Bedeutungen bleiben, die der gemeine Sprachgebrauch mit diesen Worten verbindet; ja eben um diesen Sprachgebrauch nicht zu verletzen, glaubte er nur von den gewöhnlichen etwas abweichende Erklärungen geben zu müssen. Und was könnte man auch in der That dagegen einwenden, wenn er \zB\ die Kenntniß einer Wahrheit als denjenigen Zustand unsers Gemüthes erklärt, bei dem wir ein dieser Wahrheit entsprechendes Urtheil nicht nur schon einmal gefällt haben, sondern uns seiner unter gehörigen Umständen auch noch erinnerlich, und ihm noch fortwährend zugethan sind? \par 
Gegen die Art, wie er hierauf \WLpar{III}{308} die Möglichkeit eines bloßen Nichtwissens erklärt (nämlich aus der Endlichkeit unserer Kräfte), dürfte sich kaum etwas einwenden lassen. Ein Gleiches gilt wohl auch noch von der Behauptung \WLpar{III}{309}, die Möglichkeit des Irrens beruhe nur auf dem Umstande, daß wir auch etwas, so bloße Wahrscheinlichkeit hat, erwarten oder für wahr annehmen. Daß uns aber hiedurch allein noch nicht das ganze Räthsel gelöset scheine, haben wir schon erinnert. Inzwischen müssen wir beifügen, daß die Erklärungen, die wir bisher bei Andern antrafen (die wichtigsten werden \WLpar{III}{310} angeführt), die Sache noch weniger erschöpfen. Hier versucht es B.\ auch (\WLpar{III}{311}) von den viel besprochenen Worten: Verstand und Vernunft Erklärungen zu geben, die sich nicht allzusehr von dem gemeinen Sprachgebrauche entfernen, zugleich auch von einiger wissenschaftlichen Brauchbarkeit wären. Er nennt Verstand das Vermögen bloß solcher Erfahrungskenntnisse, die wenn sie auch der Vermittlung gewisser reiner Begriffswahrheiten bedürfen, doch nicht bedürfen, daß wir uns diese zu einem deutlichen Bewußtseyn bringen. Wahrheiten dagegen, die, ob sie gleich auf Wahrnehmungen beruhen mögen, doch nur durch deutlich gedachte Vordersätze aus ihnen abgeleitet werden können, und alle diese reinen Begriffswahrheiten selbst schreibt er unsrer Vernunft zu. Da wir im Voraus nicht erwarten, \seitenw{77} daß diese Erklärungen auch nur Einem der Philosophen unserer Tage genehm seyn werden: so erinnern wir nur, daß es ganz gleichgültig für alles Folgende sowie Vorhergehende sey, ob man sie annehme oder verwerfe; denn der Vf.\ bezog sich weder in dem Vorhergehenden, noch bezieht er sich in dem Folgenden auf diese Begriffe, indem er es überhaupt (nach Herbarts Weise) vermeidet, von Seelenkräften zu reden, und aus willkürlich angenommenen Begriffbestimmungen derselben zu folgern, was der Mensch zu erkennen oder nicht zu erkennen vermöge. \par 
Da die zwei folgenden §§. nachweisen, es sey uns möglich, gar manche Wahrheiten zu erkennen, ohne den objectiven Grund derselben zu erfahren; wie wir \zB\ wissen können, daß es heut kälter sey als gestern, ohne zu wissen, warum: so gibt dieß Anlaß zweierlei Erkenntnißgründe, subjective nämlich und objective zu unterscheiden. Nach der von uns oben gemachten Anmerkung könnte man eine Unterscheidung von ähnlicher Art, wie der Verf. sie hier im Sinne hat, ihm zugestehen, auch ohne seinen Begriff von objectiven Wahrheiten selbst zuzugestehen. \par 
Endlich kommt der Vf.\ \WLpar{III}{314} zu der berühmten Frage, ob sich bestimmte Grenzen für unser Erkenntnißvermögen angeben lassen? und zeigt auf eine Weise, die wir nicht füglich in einem Auszuge mittheilen können, zwar nicht die völlige Unmöglichkeit einer solchen Angabe, aber doch welche Schwierigkeiten sie habe, ja warum eine Grenze der Art wohl an sich selbst gar nicht vorhanden seyn dürfte. Da Kant einst mit so großer Bestimmtheit das gerade Gegentheil hievon behauptet hatte, und bei so Vielen noch immer Beifall findet: so war es nöthig, seine Gründe umständlicher zu prüfen. Dieß geschieht \WLpar{III}{315}; doch muß derjenige, der Alles beisammen haben will, was B.\ gegen diese Lehre vorbringt, noch viele andere §§., wie 77. 79. 116. 148. 187--191. 287. 305. \ua\  berathen. Daß aber diese Grenzbestimmung an sich selbst widersprechend sey, zeigt schon die einzige Bemerkung, die man a. a. O. Nr. 5. antrifft, weil nämlich das Urtheil, daß die übersinnlichen Gegenstände von uns gar nicht synthetisch beurtheilt werden könnten, selbst \seitenw{78} ein synthetisches Urtheil über sie ist. Daß nicht bloß die Mathematik und Physik, sondern auch selbst die Logik (um der Moral nicht zu gedenken) synthetische Sätze enthalte, deren Daseyn Kants Theorie unerklärt läßt, ist schon aus Nr. 2. zu ersehen; oder ist etwa der Satz, daß aus zwei Sätzen von der Form: A ist B, B ist C, ein dritter von der Form: A ist C, folge, nicht ein synthetisches Urtheil? -- Daß es überhaupt unthunlich sey, die Entstehung synthetischer reiner Begriffsurtheile aus Anschauungen zu erklären, wenn diese Urtheile mehr als bloß wahrscheinlich seyn sollen, wurde schon \WLpar{III}{305} u.~1--8.\ auf sehr einleuchtende Weise gezeigt; sind ja doch überhaupt Raum und Zeit keine Anschauungen; und ist es doch grundfalsch, daß den Lehrsätzen der Arithmetik die Vorstellung der Zeit eben so zu Grunde liege, wie denen der Geometrie die Vorstellung des Raumes. Daß Kant mit Unrecht behauptet habe, es gebe vier metaphysische Antinomien, die sich mit gleicher Strenge beweisen lassen, zeigt Nr. 7. in \WLpar{III}{315}, wo in jedem dieser Beweise Fehler nachgewiesen werden, \umA\  \par 
Weder von gleicher Wichtigkeit, noch eben so abweichend von der gewöhnlichen Lehre ist das letzte Hauptstück dieses Theiles: \danf{von der Gewißheit und Wahrscheinlichkeit, wie auch der Zuversicht in unsern Urtheilen} \par 
(S.\,263--292). Denn es ist von keinem weiteren Einflusse, sondern betrifft ein bloßes Wort, wenn die Wahrscheinlichkeit eines Satzes und die Zuversicht, mit der wir ihn annehmen, unterschieden werden. Doch \DruckVariante{däucht}{daücht} uns, daß B.\ das Recht offenbar auf seiner Seite habe. Denn wenn \zB\ die Wahrscheinlichkeit eines Satzes $= \frac{1}{2}$ ist, \dh\  wenn uns gleich starke Gründe für und wider denselben vorliegen: so entscheiden wir uns gewiß weder für noch wider ihn, wir fällen gar kein Urtheil, das heißt: unsere Zuversicht ist $= 0$. Allgemein findet der Vf., wenn der Grad der Wahrscheinlichkeit eines Satzes $= \frac{m}{m+n}$ ist, den Grad der Zuversicht, mit dem wir das Urtheil fällen $= \frac{m-n}{m+n}$. \seitenw{79}\par
Von einer größeren Brauchbarkeit, besonders für die Moral, wird man hoffentlich die \WLpar{III}{317} u. \WLpar[319.]{III}{319} aufgestellten Begriffe der Verlässigkeit und sittlichen Zuversicht oder Überzeugung finden; wie auch die \WLpar{III}{321} versuchte Bestimmung der Begriffe des Glaubens und Wissens. Ein Satz heißt verlässig, wenn solche Umstände obwalten, daß es thöricht oder gar unerlaubt wäre, die Möglichkeit seines Gegentheils noch zu beachten, und dafür Anstalten treffen zu wollen; und die Zuversicht, die wir diesem Verhältnisse gemäß zu dem Satze fassen, heißt eine sittliche Zuversicht. So ist \zB\ der Satz, daß die Decke dieses Saales heute nicht einstürzen werde, verlässig, wenn dieser Einsturz, ob ich ihn gleich nicht unmöglich finde, doch einen so niedrigen Grad der Wahrscheinlichkeit hat, daß es thöricht wäre, in meinen Handlungen darauf Rücksicht zu nehmen, und also \zB\ den Saal zu verlassen; weil die Gefahr, zu verunglücken, der ich auf diese Art zu entfliehen gedächte, nicht größer ist, als eine jede, in die ich eben durch dieß Bestreben, wenn ich \zB\ lieber die Nacht unter freiem Himmel zubringen wollte, geriethe. -- \danf{Wenn uns} (heißt es \WLpar{III}{321}) \danf{eine Wahrheit, sey es unmittelbar, oder erst nachdem wir unsere Aufmerksamkeit auf gewisse Gründe derselben gerichtet, in der Art offenbar wird, daß wir erachten, von nun an würde es uns selbst, wenn wir wollten, nicht gelingen, uns von dem Gegentheile zu überreden: so sagen wir, daß wir die Sache wissen. Wenn wir dagegen einen Satz für wahr halten, ohne daß es uns unmöglich scheint, durch die willkürliche Richtung unserer Aufmerksamkeit auf ihm entgegenstehende Gründe noch zu dem entgegengesetzten Urtheile zu übergehen, wir aber sind gesonnen, die Gründe für dessen Wahrheit im Auge zu behalten: so nennen wir dieses Verhältniß zu dem Satze ein Glauben.}\editorischeanmerkung{%
		Die zitierte Stelle aus WL I §.\,321. lautet wörtlich:\par
	\zitfn{\anf{Wenn uns die Wahrheit $M$, sey es sogleich oder erst nachdem wir unsere Aufmerksamkeit auf gewisse Gründe derselben gerichtet haben, in der Art offenbar geworden ist, daß wir erachten, von nun an würde es uns selbst, wenn wir wollten, nicht gelingen, uns von dem Gegentheile zu überreden, wenn also die Zuversicht, mit der wir dem Urtheile $M$ anhängen, uns als eine solche erscheint, die zu vernichten gegenwärtig nicht mehr in unserer Macht steht: so sage ich, die Wahrheit $M$ sey bei uns zu einem \RWbet{Wissen} erhoben. \auslass Wenn wir dagegen einen Satz $M$ für wahr halten, ohne ein Wissen desselben zu haben, wenn es uns also eben nicht unmöglich scheint, daß wir durch Richtung unserer Aufmerksamkeit auf alle demselben entgegenstehenden wahre, oder nur scheinbare, Gründe zu dem entgegengesetzten Urtheile \BUlat{Neg.} $M$ verleitet werden könnten, wir aber gesonnen sind, die Gründe für dessen Wahrheit im Auge zu behalten: so nenne ich dieses Verhältnis unsers Gemüthes zu dem Satze $M$ ein \RWbet{Glauben an diesen Satz}.}}} -- Bei diesem Letzteren verharren auch wir; die Erklärung des Wissens aber scheinet uns wenigstens nicht dem gemeinen Sprachgebrauche des Wortes zu entsprechen; weil dieser nicht sowohl einen sehr hohen Grad der Zuversicht, als vielmehr ein bestimmtes Verhältniß zur Wahrheit vom Wissen verlangt, nämlich daß es ein mit der Wahrheit übereinstimmendes Urtheilen seyn solle. Wie \seitenw{80} sonst könnte man sagen: \danf{Er weiß es wohl, aber nur nicht gewiß;} und wieder: \danf{Er bildet sich ein, es zu wissen;} während es uns nie einfällt zu sagen, daß Jemand sich einbilde, etwas zu glauben\Druckfehlerkorr{.}{.\grqq} Somit hätte die Aufstellung dieses Begriffes vielmehr in das dritte Hauptstück gehört. \par 
Der vierte Theil der Wissenschaftslehre oder die \danf{Erfindungskunst,} welche die noch übrige Hälfte des dritten Bandes (S.\,293--575) füllet, liefert im ersten Hauptstücke die allgemeinen Regeln des Nachdenkens, wenn es Erfindung der Wahrheit bezwecket, und in dem zweiten eine nähere Anweisung zu 33 besondern Aufgaben des Denkens. \par 
Unter den allgemeinen Regeln (S.\,304---389) ist, \DruckVariante{däucht}{daücht} uns, nicht eine einzige, welche man dem Vf.\ nicht zugeben könnte und müßte, wie man auch in Bezug auf alles Frühere mit ihm stehe. Überhaupt ist wohl Alles, was er hier sagt, wenn auch nicht eben ausdrücklich aufgestellt worden, doch (wie er selbst meinet) Gebrauch eines jeden geübteren Denkers gewesen. Oder wer hätte \zB\ nicht für dienlich erachtet, die Wahrheit, welche er eigentlich sucht, gleich beim Beginn seines Nachdenkens so genau als möglich zu bestimmen; wer hätte nicht für zweckmäßig erachtet, zu versuchen, ob sich die Unmöglichkeit der gewählten Aufgabe nicht vielleicht schon im Voraus einsehen lasse; wer hätte das Mittel der Beschäftigung mit gewissen Vorfragen, auf welche sich unsre Aufgabe zurückführen läßt, oder die sonst nur einiges Licht auf sie werfen können, nicht selbst schon angewendet; wer hätte nicht durch Folgerungen aus schon bekannten Wahrheiten (directes Verfahren) sowohl als auch durch versuchsweise Annahmen (indirectes Verfahren) und durch Verbindung mehrerer Verfahrungsarten untereinander die Lösung einer Aufgabe betrieben? \usw\ Von großer Wichtigkeit ist aber gewiß dasjenige, was der Vf.\ \WLpar{III}{331}, (wo er von der Berathung des Urtheils Anderer und der Erfahrung spricht) über die hohe Verlässigkeit der Aussprüche des gemeinen Menschen\seitenw{81}verstandes beibringt. Denn nur wenn man zu diesen Aussprüchen bloß zählt, was er und sie auf solche Fälle allein beschränket, wie er, kann man in Wahrheit und mit großem Nutzen in allen praktischen Wissenschaften sich auf sie berufen. \danf{So oft} -- sind des Vfs. eigene Worte -- \danf{der Gegenstand unserer Untersuchung zu der Art Dinge gehört, zu deren Beurtheilung nichts als Vernunft und gewisse allen Menschen zu Gebote stehende Erfahrungen nothwendig sind: so oft sind eigentlich alle Menschen im Stande, ein giltiges Urtheil über denselben zu fällen. Wenn wir nun finden, daß Alle, oder doch fast Alle den Gegenstand auf eine gleichlautende Weise beurtheilet haben; und wenn sich überdieß zeigt, daß ihr Urtheil \auslass\ nicht etwa den menschlichen Neigungen schmeichelt, sondern vielmehr sie beschränket: dann sage ich, daß wir in einem sehr hohen Grade der Zuversicht voraussetzen dürfen, dieß Urtheil (welches ich in einem solchen Falle einen \RWbet{Ausspruch des gemeinen Menschenverstandes} nenne) müsse der Wahrheit gemäß seyn.}\editorischeanmerkung{%
	Die zitierte Stelle aus WL III §.\,336 Nr.\,4 lautet wörtlich (mit \abweich{Kennzeichnung} der Abweichungen, wie den Auslassungsmarkierungen \anf{\auslass} oben, durch uns):\par
	\zitfn{\anf{So oft der Gegenstand unserer Untersuchung zu der Art Dinge gehört, zu deren Beurtheilung nichts als Vernunft und gewisse\abweich{,} allen Menschen zu Gebote stehende Erfahrungen nothwendig sind: so oft sind eigentlich alle Menschen im Stande, ein g\abweich{ü}ltiges Urtheil über denselben zu fällen. \abweich{Wir werden also wohl thun, zu fragen, wie sich die allgemeine Meinung hierüber ausgesprochen habe?} Wenn wir nun finden, daß \abweich{a}lle, oder doch fast \abweich{a}lle \abweich{Menschen} den Gegenstand auf eine gleichlautende Weise beurtheilet haben; und wenn sich überdieß zeigt, daß \abweich{das} Urtheil \abweich{zu dem fast alle Menschen sich wie mit Einem Munde bekennen} nicht etwa den menschlichen Neigungen schmeich\abweich{le}, sondern vielmehr sie beschränk\abweich{e}: dann sage ich, daß wir \abweich{in} einem sehr hohen Grade der Zuversicht voraussetzen dürfen, dieß Urtheil (welches ich in einem solchen Falle einen \RWbet{Ausspruch des gemeinen Menschenverstandes} nenne) müsse der Wahrheit gemäß seyn.}}}
-- Im nächsten \WLpar{III}{332} (Prüfung der eigenen bereits gefällten Urtheile) berichtiget der Vf.\ die noch immer gangbare Regel des Cartes: \danf{An Allem wenigstens einmal im Leben zu zweifeln,} indem er statt des bald etwas Unmögliches, bald etwas sehr Unrechtes fordernden Ausdruckes: Zweifeln, ein davon ganz verschiedenes Prüfen setzt, und den Begriff desselben genau erkläret; -- auch dieses Prüfen endlich nicht auf Alles ausgedehnt wissen will, sondern, nachdem er gezeigt, daß zu jeder Prüfung einige andere Sätze als wahr vorausgesetzt werden müssen, nachweist, daß wir das Prüfen dort getrost unterlassen können, wo wir es nicht anders anstellen könnten, als durch Voraussetzung von Sätzen, die einen noch geringeren Grad der Verlässigkeit haben. -- Um endlich das über Zeichen Gesagte \WLpar{III}{334--344} nicht unvollständig zu finden, vergesse man nicht, daß hier die Rede bloß sey von Zeichen, deren wir uns bei dem Geschäfte des Nachdenkens selbst, oder etwa zur Aufbewahrung des Erdachten wieder nur für uns selbst bedienen. Aus einem andern Gesichtspuncte wird dieser Gegenstand später betrachtet werden. \seitenw{82}\par 
In dem Hauptstücke von den \danf{besonderen Regeln} (S.\,390--575) kommt freilich Einiges vor, was sich auf des Vfs. eigenthümliche, im Früheren aufgestellte Ansichten gründet. Wer nicht zugeben will, daß Begriffe aus andern zusammengesetzt seyn können, oder zwischen Begriffen von gleichem Umfange keinen Unterschied sieht, der kann die Anweisung des \WLpar{III}{350} (\danf{Erklärung einer durch unser Bewußtseyn gegebenen Vorstellung}) freilich nicht brauchbar finden; wer nicht zuläßt, daß es auch gegenstandlose und imaginäre Begriffe gibt, dem muß die Aufgabe des \WLpar{III}{352} entbehrlich scheinen; wer keinen Begriff hat von dem, was B.\ eine überfüllte Vorstellung nennt, für den ist \WLpar{III}{354} sinnlos; wer keinen Unterschied zulassen will zwischen dem objectiven Zusammenhange der Wahrheiten und ihrer subjectiven Ableitbarkeit, muß \WLpar{III}{378} (\danf{Auffindung des objectiven Grundes einer gegebenen Wahrheit}) verwerfen. Allein wie aufrichtig wir dieß Alles eingestehen, eben so aufrichtig bekennen wir nicht abzusehen, was einen mit B.\ in seinen früheren Behauptungen wie immer zerfallenen Leser abhalten könnte, ihm wenigstens in den Anweisungen zu folgen, welche er in den meisten übrigen §§. ertheilt. Als Beispiel deuten wir auf \WLpar[\BUparformat{356--8.}]{III}{356--358}, wo die Auffindung einer Ähnlichkeit, eines Unterschiedes, einer ausschließlichen Beschaffenheit gelehrt wird. Besonders richtig aber und annehmbar für einen Jeden scheinen uns die Anweisungen der §\WLpar{III}{369--376} \danf{Prüfung der Wahrheit eines gegebenen Satzes;} \danf{der Überzeugungskraft eines gegebenen Beweises,} die gewöhnlichen \danf{Fehler im Beweisen,} und \danf{die Kennzeichen der Fehlerhaftigkeit eines Beweises.} Da der Raum nicht verstattet, eine der unter diesen Titeln ertheilten Anweisungen hier zu wiederholen: so wollen wir statt dessen andeuten, wie der Vf.\ \WLpar{III}{377} etliche der bekannten Trugschlüsse löse, die in den Lehrbüchern der Logik gewöhnlich angeführt werden. Der Trugschluß des Diodorus: \danf{Wenn sich ein Körper bewegen sollte, so müßte er sich entweder in dem Orte, in dem er ist, oder in einem, in dem er nicht ist, bewegen; was beides ungereimt ist} hat noch für unsere \seitenw{83} Zeit Wichtigkeit wegen der ähnlichen Lehren, die Herbarts Schule aufstellt (s. \zB\ Hartensteins Metaphysik S.\,401). B.\ sagt nun einfach, weil Bewegung nichts Anderes als eine durch eine gewisse Zeit hindurch dauernde Veränderung des Ortes ist, so sey es ungereimt vorauszusetzen, daß sie in einem einzigen Orte erfolge; sondern man muß der Orte mehrere, in denen sie vorgehen soll, annehmen. -- Bei Lösung des berühmten Trugschlusses Achilles (der eine Schildkröte nicht einholen soll, obgleich er tausendmal schneller als sie ist) zeiget B.\ -- was zu beachten auch anderwärts noth thut, und von höchster Wichtigkeit ist, -- daß auch Zeitreihen, aus unendlich vielen Gliedern vergangen seyn können; und dieß zuweilen sogar in einer endlichen Zeit; wie denn \zB\ die ganze Zeitreihe: \par 
\centerline{\ensuremath{\frac{1}{2}} Stunde + \ensuremath{\frac{1}{4}} St. + \ensuremath{\frac{1}{8}} St. + ... \RWlat{in inf.}}\par\noindent
nur eine einzige Stunde ausfüllt. Das \BUlat{Sophisma pigrum} hört man unter verschiedenen Formen fast alle Tage: \danf{Was ich durch meine Thätigkeit bestrebt seyn könnte hervorzubringen, könnte nur Eines von Beiden betreffen, einen Erfolg, der wirklich, oder einen, der nicht wirklich eintritt. Im ersten Falle ist meine Thätigkeit dabei entbehrlich, im zweiten vergeblich; also immer unweise.} Hier steckt der Irrthum nur in dem Satze, daß meine Thätigkeit entbehrlich zu nennen sey, wenn sie einen Erfolg betrifft, der wirklich eintritt. Denn sie ist eben eine der Theilursachen, deren Verbindung ihn herbeiführt \usw\ \par 
Was in den §\WLpar{III}{379--384} über die Entdeckung der Ursachen gegebener Wirkungen und umgekehrt; \WLpar{III}{387} über die Auslegung gegebener Zeichen; §.388 u. 9. über die Entdeckung vorhandener Zeugnisse und die Prüfung ihrer Glaubwürdigkeit, endlich \WLpar{III}{390} über die Bestimmung der Glaubwürdigkeit eines Satzes aus dem Ansehen Aller, die für oder wider ihn sind, gesagt wird, möchten wir beziehungsweise Physikern, Historikern, Hermeneuten und Theologen anzuempfehlen wagen, versprechend, daß sie ein Jeder, wenn sie ihre Wissenschaft mit philosophischem Geiste betreiben, hier Einiges, das ihre Prüfung in Anspruch nehmen wird, antreffen werden. \seitenw{84}\par 

\WLsec{IV}{392}{718}
Doch hiemit wären wir nun schon bei dem letzten Bande des Werkes angelangt, welcher den fünften Theil oder die \danf{eigentliche Wissenschaftslehre,} der alles Vorhergehende als Mittel dienen soll, vorträgt. Wir haben schon oben gesagt, daß wir das Studium dieses Bandes Jedem erlassen, der einmal durchaus nicht will, daß die Logik eine Wissenschaftslehre seyn solle; nur geben wir ihm, bevor wir Abschied nehmen, noch zu bedenken, daß er sich auch in einem jeden andern Lehrbuche der Logik mindestens einige der Untersuchungen, die jetzt noch kommen sollen, habe gefallen lassen; namentlich eine Bestimmung des Begriffes: Wissenschaft, dann ganze Abhandlungen über Erklärung, Eintheilung und Beweis; ja daß ihm diese Belehrungen hie und da sogar unter demselben Titel: \danf{Wissenschaftslehre,} geboten wurden, unter welchem sie hier freilich durch manche andere noch vermehret auftreten. -- Damit er nun wisse, was unter diesem Titel B.\ bespricht, was also eigentlich dieser von ihm verschmähete Theil enthält: so legen wir ihm die kurze Anzeige davon mit des Vfs. eigenen Worten (in \WLpar{IV}{392}) hier vor: \par 
\danf{1. Erst werde ich noch den Begriff wie einer Wissenschaft selbst, so auch den eines Lehrbuches etwas genauer als es schon \WLpar{I}{1} geschehen ist, zu bestimmen, ingleichen zu untersuchen haben, ob es nicht irgend einen obersten Grundsatz gebe, aus dem sich alle Regeln, welche bei Bildung der einzelnen Wissenschaften sowohl als bei Bearbeitung ihrer Lehrbücher zu beobachten sind, ableiten lassen; und im bejahenden Falle einige der unmittelbarsten Folgerungen, so fern es ersprießlich ist, sie bei Entwicklung der übrigen Regeln immer vor Augen zu haben, anknüpfen. \par 
2. Das Nächste hierauf wird seyn, die Regeln vorzutragen, nach welchen das gesammte Gebiet der Wahrheit in einzelne Wissenschaften zerlegt, und zugleich beurtheilt werden kann, ob eine Wissenschaft, deren Begriff man uns vorlegt, zweckmäßig sey.} \par 
\danf{3. Nach Aufstellung dieser Regeln wird sich die Logik zu ihrem zweiten (ungleich weitläufigeren) Geschäfte wenden, die Regeln anzugeben, die bei der Abfassung eines Lehr\seitenw{85}buches einer Wissenschaft zu beobachten sind. Da aber bei Abfassung eines jeden Buches eine bestimmte Klasse von Lesern gedacht werden muß, so müssen nun zuerst die Regeln aufgestellt werden, nach welchen zu beurtheilen, ob eine Klasse von Lesern zweckmäßig gewählt sey.} \par 
\danf{4. Hierauf erhebt sich die Frage, welches die Sätze sind, die eine Aufnahme in unser Buch verdienen. \par 
5. Wir werden aber fast immer wohl thun, den ganzen Inbegriff dieser Sätze unter verschiedene Abtheilungen zu bringen. Auch zu diesem Geschäfte ist eine eigene Anleitung nöthig. \par 
6. Da ferner jene Sätze nie alle auf einmal, sondern nur einer nach dem andern dem Bewußtseyn des Lesers vorgeführt werden können, so müssen wir auch die Folge, nach der wir sie vortragen wollen, bestimmen, \dh\  sie ordnen; wozu abermals eine Anleitung zu geben. \par 
7. Da ferner alle Bücher nur eine Art schriftlicher Darstellungen sind, und da es überdieß in einem Lehrbuche nothwendig ist, nebst jenen schriftlichen Zeichen, deren wir uns im Buche selbst bedienen, auch unsern Lesern gewisse, theils schriftliche, theils mündliche Zeichen zu ihrem eigenen Gebrauche zu empfehlen: so wird über die Art, wie auch dieß Beide zu geschehen, einige Anweisung gegeben werden. \par 
8. Da endlich in allen denjenigen Regeln, die ich so eben Nr. 4--7. angekündiget habe, nur von gewissen dem Buche zu ertheilenden Beschaffenheiten, nicht aber davon, wie der Verfasser sich dabei selbst zu verhalten habe, die Rede seyn wird: so dürfte sich wohl eine kurze Belehrung auch noch über diesen Punkt geziemen. \par 
9. Endlich gibt es Bücher, die, ob sie gleich nicht eigentliche Lehrbücher sind, doch einen wissenschaftlichen Unterricht bezwecken, und eben deßhalb fast nach denselben Regeln wie jene abgefaßt werden müssen. Es dürfte also in einem Lehrbuche der Logik nicht am unrechten Orte seyn, mit einigen Bemerkungen auch über diese zu schließen. \par 
Aus dieser Andeutung ergeben sich von selbst jene neun Hauptstücke, welche der Leser in diesem Theile antrifft.} 
\gliederungslinie\par
\seitenwohne{86}So weit B.\ Heben wir nun aus jedem dieser Hauptstücke Einiges aus für Leser, die etwas davon kennen zu lernen wünschen. \par 
Das erste Hauptstück unter der Überschrift: \danf{allgemeine Regeln} füllt nur 43 Seiten. Der Vf.\ erklärt eine Wissenschaft als einen \danf{Inbegriff von solchen Wahrheiten, deren bekannter und uns merkwürdiger Theil es verdient, in einem eigenen Buche dergestalt niedergeschrieben und nöthigen Falls auch mit so viel andern zu ihrem Verständnisse und Beweise dienlichen Sätzen verbunden zu werden, daß sie die größte Faßlichkeit und Überzeugungskraft erhalten.} Hiernächst hängt es freilich von unsrer menschlichen Empfänglichkeit für gewisse Wahrheiten und unserm Bedürfnisse ab, ob wir den Inbegriff derselben eine Wissenschaft nennen oder nicht; gleichwohl ist dem Vf.\ die Wissenschaft nach dieser Erklärung etwas Objectives und Unveränderliches, weil nicht bloß die uns bekannten und für uns merkwürdigen Wahrheiten zu ihrem Inhalte gezählt werden, sondern auch solche, die wir noch nicht kennen, oder nie werden kennen lernen; ingleichen weil auch nur die Wahrheiten selbst, die diesem Artbegriffe entsprechen, nicht aber auch alle diejenigen, die etwa zu ihrem subjectiven Beweise für uns erforderlich sind, die zu verschiedenen Zeiten und für verschiedene Personen verschieden seyn können, zur Wissenschaft selbst gehören. Diese Beweisführungen, selbst wenn es versuchte objective Begründungen wären, rechnet B.\ nur zu der Darstellung einer Wissenschaft in ihrem Lehrbuche. Daß dieser Begriff trotz allen Gründen der Rechtfertigung, die der Vf.\ ihm beigegeben hat, Widerspruch finden werde, läßt sich erwarten; besonders bei allen denjenigen, die, wenn sie eben ihre Philosophie im Sinne haben, nicht anders sich geberden, als wäre diese die einzige Wissenschaft, die es nur überhaupt gibt. Da aber eben diese Gelehrten zu anderer Zeit und an anderen Orten großmüthig genug sind, auch dem Mathematiker, dem Naturforscher, dem Arzte, dem Historiker sogar zuzugestehen, daß auch sie irgend ein, obgleich nur fragmentarisches Wissen besäßen, und daß sie auch die\seitenw{87}ses fürderhin noch wie bisher in Büchern darstellen, mit Beweisen unterstützen, und dann demselben allenfalls auch den Namen einer Wissenschaft ad honores beilegen dürften: so wird im Wesentlichen B.\ wohl nicht ganz unrecht haben, aber freilich auf eine beifällige Aufnahme zu rechnen, wie dürfte ihm das in den Sinn kommen, da er kein einziges der beliebten Modeworte anbringt, weder des \danf{gegenseitigen Durchdringens aller Theile,} noch des \danf{organischen Zusammenhanges derselben,} nicht einmal \danf{eines lebendigen Ganzen} erwähnet? \par 
Zwar wagt er es -- und dieses wäre so unrecht nicht, -- einen obersten Grundsatz der ganzen Wissenschaftslehre aufzustellen; aber, Himmel! wie lautet dieser? \danf{Bei der Zerlegung des gesammten Gebietes der Wahrheit in einzelne Wissenschaften, und bei der Darstellung derselben in eigenen Lehrbüchern so zu verfahren, daß -- die größtmögliche Summe des Guten daraus hervorgehe!} -- So etwas mögen wohl ein Aristoteles, ein Baco von Verulam und alle in der That große und preiswürdige Gelehrte, die sich entweder durch Einführung neuer Wissenschaften, oder durch die Erweiterung bereits bestehender, oder auch nur durch neue gründliche lichtvolle Darstellungen derselben einen Ruhm erwarben, bezwecket haben: allein jetzt einmal lieben es unsere Weltweisen eben so wenig als gewisse Regierungen, die Aufmerksamkeit auf das gemeine Beste zu lenken. Es ist ein verächtlicher Utilitarismus! -- Und wie der Grundsatz, so sind auch die 13 aus ihm gezogenen \danf{nächsten Folgerungen} beschaffen, \zB\ daß ein zweckmäßiges Lehrbuch den Lesern das Verstehen des darin Vorgetragenen so leicht und so sicher als möglich machen müsse (\WLpar{IV}{398}) (o hem!) daß es den Grad der Verlässigkeit jeder Lehre bemerklich machen solle (\WLpar{IV}{400}); daß es so oft als möglich den objectiven Zusammenhang nachweisen (\WLpar{IV}{401}); daß es so eingerichtet werden müsse, damit auch dessen etwaige Fehler dem Leser den mindesten Schaden verursachen (\WLpar{IV}{407}), daß es von seinen Einrichtungen auch den Grund solle wahrnehmen lassen (\WLpar{IV}{408}), \usw\ Man muß es gestehen, auch nicht eine einzige dieser Forderungen entspricht dem Geschmacke der Zeit! \seitenw{88}\par 
In Betreff der Anweisungen des zweiten Hauptstückes: \danf{von der Bestimmung des Gebietes der Wissenschaften} (S.\,44--84) ist B., nach S.\,81 \danf{gefaßt, daß man das hier Gesagte theils mit Geringschätzung, theils selbst mit Spott aufnehmen werde.} Und nicht mit Unrecht erwartet er dieß; denn das Hauptsächlichste beruhet doch nur auf 14 Lehrsätzen (\WLpar{IV}{410--423}), deren Überschriften wir unsern Lesern hier vorlegen wollen, damit sie selbst urtheilen mögen, und nicht über getäuschte Erwartungen klagen, wenn sie das Werk etwa doch einst zur Hand nehmen wollten. \par 
\danf{1) Für Wahrheiten, die sich durch Schrift nicht mittheilen lassen, braucht es auch keine Wissenschaft zu geben. 2) Jede durch Schrift mittheilbare Wahrheit aber, welche nicht bloß als Hülfssatz merkwürdig ist, soll wenigstens in Einer Wissenschaft einheimisch seyn. 3) Nicht ein zu kleiner, wohl aber ein zu großer Umfang kann ein hinreichender Grund zur Verwerfung einer Wissenschaft werden. 4) Es ist kein hinreichender Grund zur Verwerfung einer Wissenschaft, daß viele, ja alle ihre Lehren Jedem schon ohnehin bekannt sind. 5) Es ist kein hinreichender Grund, Wahrheiten zu vereinen, bloß weil sie viele Ähnlichkeit mit einander haben; 6) noch sie zu trennen, bloß weil sie einen großen Unterschied, namentlich eine andere Erkenntnißquelle haben. 7) Es darf auch Wissenschaften geben, welche gewisse Lehren gemeinschaftlich haben, oder deren die eine ganz in der andern steckt; 8) auch die von einer andern entweder nur subjectiv, oder objectiv, oder in beiden Hinsichten abhangen; 9) ja auch in dem Verhältnisse einer gegenseitigen Abhängigkeit stehen. 10) Es ist nicht zu verlangen, daß die Anwendungen einer Wahrheit immer in dieselbe Wissenschaft mit ihr gehören; 11) daß alle Wahrheiten einer Wissenschaft einen einzigen objectiven oder subjectiven Grundsatz haben. 12) Es ist sehr gut, die Wahrheiten nach einer solchen Beschaffenheit, vermittelst deren man nach ihnen fragen kann, abzutheilen. 13) Wenn irgend ein reiner Begriff, zumal ein einfacher, in gewissen Wahrheiten ausschließlich vorkommt, so ist sehr zu erwarten, daß sie die Vereinigung in eine eigene Wissenschaft verdiene. 14) Jeder Untersuchung ist ein Platz anzuweisen, in einer \seitenw{89} Wissenschaft, in der sie auf das Fruchtbarste angestellt werden 
kann.} \par 
Nun folgt in drei einzelnen §§., die keines Auszuges fähig sind, die Lösung folgender drei Aufgaben: 1) Prüfung der Zweckmäßigkeit einer gegebenen Wissenschaft. 2) Erfindung des Begriffes einer neuen zweckmäßigen Wissenschaft. 3) Eintheilung des gesammten Gebietes der Wahrheit in einzelne Wissenschaften. \par 
Am Schlusse versichert B., er \danf{wünsche von Herzen, daß man bald etwas finde, das nicht nur gelehrter aussieht, sondern auch richtiger und brauchbarer ist.} \par 
In dem sehr kurzen dritten Hauptstücke, \danf{von der Wahl der für ein Lehrbuch bestimmten Klasse der Leser} (S.\,85--91), dürfte das Wichtigste seyn der Vorschlag des \WLpar[\BUparformat{431}]{IV}{431}, für eine jede Wissenschaft von allgemeinerer Brauchbarkeit drei Arten Lehrbücher für drei Klassen von Lesern zu unterscheiden: vollständige Werke für den Gelehrten vom Fach, worin auch dasjenige, was vor der Hand noch keine Anwendung gefunden, hinterlegt wird; Lehrbücher für den Geschäftsmann, die Alles, was dieser für sein Geschäft brauchen kann, enthalten; Bücher für Jedermann, die nur das allgemein Brauchbare ausheben. \par 
Das vierte Hauptstück: \danf{von den Sätzen, welche in einem Lehrbuche vorkommen sollen,} (das ausführlichste von allen, 300 Seiten umfassend) beginnt \WLpar{IV}{434} mit einer Unterscheidung der verschiedenen Arten, wie Sätze in einem Lehrbuche überhaupt vorkommen können: man kann sie nämlich bald aufstellend, \dh\  mit der Erwartung, vortragen, daß der Leser, falls nicht schon früher, wenigstens jetzt mit einem bestimmten Grade der Zuversicht ihnen beitreten werde; bald bloß erwähnend, oder sich zu denselben bekennend, oder sie voraussetzend \usw\ Dann werden \WLpar{IV}{435} drei Arten, wie der Leser von den in einem Buche vorkommenden Sätzen Gebrauch machen soll, (Auffassung in das Gedächtniß, einmaliges Durchdenken, gelegenheitliches Nachschlagen); endlich \WLpar{IV}{436} dreierlei Verhältnisse \seitenw{90}\par 
derselben zu der Wissenschaft, die hier gelehrt werden soll, unterschieden; indem es entweder wesentliche (einheimische), oder Hülfssätze oder gelegenheitliche Sätze seyn können. Von jeder dieser drei Arten wird nun in einem eigenen Abschnitte gehandelt; weil es jedoch auch Sätze gibt, deren Eigenthümlichkeit gar nicht auf ihrem Verhältnisse zur Wissenschaft, sondern auf andern Umständen beruhet: so wird von diesen noch in einem vierten Abschnitte gesprochen. \par 
In dem ersten Abschnitte: \danf{von den wesentlichen Sätzen,} (S.\,102--130) wird untersucht, wann ein Satz merkwürdig genug sey, um die Zumuthung, daß ihn der Leser in sein Gedächtniß aufnehme, zu rechtfertigen; wann nur, um eine einmalige Betrachtung desselben zu verlangen; wann endlich zu einem bloß gelegenheitlichen Nachschlagen. Es werden die Fragen erörtert, ob eine allgemeinere Wahrheit allzeit den Vorzug vor der besondern verdiene? wann auch unmittelbare Folgerungen einer Anführung bedürfen? wann völlig gleichgeltende Sätze, ferner Sätze mit imaginären, überfüllten Vorstellungen, identische, bloße Berichtigungssätze, bloße Wahrscheinlichkeiten eine Aufnahme verdienen? \usw\ Ein Beispiel heben wir aus. \danf{In einer Anleitung zur Heilkunst} (sagt der Vf.\ S.\,119) \danf{kann man vielleicht nicht oft genug die Warnung wiederholen, daß der Arzt die Natur in ihren Verrichtungen nie stören, daß er nur ihren Diener machen, nur ihr nachhelfen müsse \udgl\ Allein nie sollte man vergessen, daß man hiemit im Grunde nichts Anderes ausspreche, als den identischen Satz, daß sich der Arzt immer vorsehen müsse, die Mittel der Kunst nicht am unrechten Orte, \dh\  dort, wo sie eigentlich nicht vorgeschrieben sind, zu gebrauchen. Wollte man aber dergleichen Regeln für Grundsätze ausgeben, aus denen sich das Verhalten des Arztes am Krankenbette objectiv herleiten läßt, dann wäre dieß wahrlich ein lächerlicher Irrthum.}\par 
In dem zweiten Abschnitte: \danf{von den Hülfssätzen} (S.\,134--142) empfehlen wir der Erwägung besonders die zwei §§.: wann man in einer Wissenschaft, die \seitenw{91} reine Begriffssätze zu ihrem Gegenstande hat, auch empirische Hülfssätze anwenden dürfe, und umgekehrt? und wo der vom Ansehen hergenommene Beweis zu brauchen? Es wäre sicher zu tadeln, wenn wir in einem Lehrbuche der Religion die wichtige Wahrheit von Gottes Daseyn und Beschaffenheiten aus bloßen Begriffen und nicht auch aus der zweckmäßigen Einrichtung des \DruckVariante{Weltgebäudes}{Weltgebaüdes} \usw\ ableiten wollten; und es ist eben so gefehlt, wenn man in den Erfahrungswissenschaften gar keine Schlüsse a priori will gelten lassen und vor aller Philosophie sich scheut. Das Wahre an der Behauptung, daß in reinen Begriffswissenschaften keine Autorität gelte, ist nur, daß wir bei dem bloßen Ansehen hier nie stehen bleiben, sondern dem objectiven Grunde nachforschen sollen, was bei empirischen Wissenschaften nicht immer möglich ist. \par 
Zu den \danf{gelegenheitlichen Sätzen} im dritten Abschnitte (S.\,142--177) zählt der Vf.\ nebst manchem Anderen: die Bestimmung und Rechtfertigung des Begriffes der vorzutragenden Wissenschaft; die Bestimmung ihres Verhältnisses zu andern Wissenschaften; geschichtliche Mittheilungen; Angabe der bei Abfassung des Buches befolgten Regeln; der Leserclasse, für die wir geschrieben; des Nutzens, den wir den Lesern von unserer Wissenschaft und dem Studio dieses Lehrbuches denselben versprechen können; Geständnisse seiner Mängel; Forderungen an den Leser; Anwendungen; Warnungen vor Mißverstand oder Mißbrauch; Abtheilungen \umA\  Wir heben auch hier nur ein paar Stellen, ihres wichtigen Inhaltes wegen, heraus. \danf{Wenn bei irgend einer Wissenschaft (sagt der Vf.\ in der Anm. zu \WLpar{IV}{470}) Anwendungen in der hier angegebenen Bedeutung nicht sollten weggelassen werden, so ist es bei der Geschichte in Darstellungen, welche nicht für den Gelehrten, sondern für das größere Publikum, oder vollends für die Jugend bestimmt sind. Das Studium dieser Wissenschaft gewährt doch eigentlich gar keinen Nutzen, wenn wir nicht über die uns hier bekannt gewordenen Ereignisse und Thaten Betrachtungen anstellen, die einer ganz anderen als historischen Natur sind, die darauf abzielen, uns immer anschaulicher zu machen, was recht und \seitenw{92} unrecht, löblich und tadelnswerth, klug oder unklug sey, welche verderbliche Folgen gewisse Sitten und Einrichtungen nach sich ziehen, wie viel der Mensch, oft auch der Einzelne, durch eine kluge und beharrliche Anstrengung seiner Kräfte vermöge, wie Gottes Fürsehung die Schicksale unsers Geschlechtes zu allen Zeiten und in allen Ländern leite \usw\ -- Was soll man aber sagen, wenn Sätze, wie: \deinfanf{ein Geschichtschreiber soll nur die nackten Thatsachen darstellen, ohne ein lobendes oder tadelndes Urtheil darüber einfließen zu lassen,} oder: \deinfanf{man soll die rechtlichen, sittlichen und religiösen Begriffe, denen er zugethan ist, einem Geschichtschreiber nicht einmal anmerken können,} -- sogar als Grundsätze aufgestellt werden? Richtig wäre es wohl, zu sagen, daß ein Geschichtschreiber sich sehr in Acht zu nehmen habe, damit die vorgefaßte Meinung über den Werth oder Unwerth einer Handlung ihn nicht zu einer untreuen Darstellung derselben, bald zur Verschönerung, bald zur Entstellung verleite. Billig wäre es auch, zu verlangen, daß ein Geschichtschreiber sein politisches oder religiöses Glaubensbekenntniß den Lesern nicht etwa dadurch kund gebe, daß er für diejenigen, die mit ihm gleich denken, eine parteiliche Vorliebe an den Tag legt. Allein muß man denn nothwendig, wenn man lobt oder tadelt, in den Fehler der Übertreibung, oder vollends in den einer ungetreuen Darstellung dessen, was man beurtheilen will, verfallen? Und das offene Geständniß, weß Glaubens man sey, ist es für edle Gemüther nicht eher noch ein eigener Abhaltungsgrund, die Anhänger der entgegengesetzten Partei in irgend einer Art ungerecht zu behandeln?} In der Anm. zu \WLpar{IV}{471} liest man: \danf{Nur einer ununterbrochenen Befolgung dieser Regel} (der Warnung vor Mißverstand und Mißbrauch) \danf{bedarf es, die Menschheit vor einem Übel zu bewahren, von welchem Viele behaupten, daß es als eine nicht zu vermeidende Folge eintreten müsse, wo immer die Gelehrten sich beikommen lassen, einen Theil des Wissens, das sie bisher als ihr ausschließliches Eigenthum besaßen, zu einem Gemeingute zu erheben, und deßhalb eine beträchtliche Menge fragmentarischer Kenntnisse aus allen Wissenschaften unter das Publikum zu verbreiten suchen. Halbwissen soll \seitenw{93} das Übel seyn, welches auf solche Weise unausbleiblich entstehe, und in seinen Wirkungen sich verderblicher als selbst die völlige Unwissenheit erprobe. In jeder Wissenschaft, sagt man, gibt es eine gewisse, ewig unverrückbare Grenze, welche das Wissen der großen Menge, ja überhaupt Aller, die diese Wissenschaft sich nicht in ihrem ganzen Umfange aneignen können, nicht überschreiten darf, soll es kein Halbwissen werden. Und von gewissen Wissenschaften, wie von der Theologie, von der Arzneikunde u. m. a. hat man sogar behauptet, daß sie in einem Unterrichte, der keine Halbwisser bilden will, wenn man sie nicht erschöpfend abhandeln kann, nicht einmal berührt werden dürften. -- Auch ich stelle, wie man das schon aus \WLpar{IV}{431} \ua\  O. weiß, gar nicht in Abrede, daß die Überschreitung gewisser Grenzen, besonders in Lehrbüchern, welche (nach der Bedeutung des \WLpar{IV}{430}) für Jedermann bestimmt sind, wichtige Nachtheile habe; besonders weil dadurch anderen Kenntnissen, deren Erlernung nöthiger gewesen wäre, Abbruch geschieht, oder weil über dem vielen Lernen die Zeit zum Handeln \DruckVariante{verabsäumet}{verabsaümet} wird; allein was ich nicht zugebe, ist, daß durch ein solches Verfahren immer dasjenige Übel, das man das Halbwissen nennt, erzeugt werden müsse. Soll das Wort Halbwissen einen schädlichen Zustand des Geistes, ja einen Zustand bezeichnen, der schlimmer als Nichtwissen ist: so dürfen wir doch nur demjenigen den Vorwurf des bloßen Halbwissens machen, der aus den einzelnen Begriffen einer Wissenschaft, welche er aufgerafft hat, Folgerungen ableitet, die mit der Wahrheit nicht bestehen, und die nur ihm selbst oder Andern Nachtheil verursachen. Nicht aus der bloßen Anzahl der Lehren, die wir aus einer Wissenschaft uns angeeignet haben, darf es bemessen werden, ob wir Halbwisser in diesem Fache zu heißen verdienen oder nicht; wie denn sonst selbst der gründlichste und umfassendste Gelehrte, weil auch sein Wissen noch immer nur Stückwerk ist, vielleicht noch keine Hälfte von dem, was man auf diesem Gebiete in einem kommenden Jahrhunderte entdeckt haben wird, beträgt, in einem gewissen Betrachte der Halbwisserei müßte beschuldigt werden können. Allein hier kommt es auf etwas Anderes, hier kommt es \seitenw{94} lediglich auf die Art an, wie wir die Lehren verstehen, und welche nächste Folgerungen wir aus ihnen ableiten. Wenn also zwei Personen ungefähr dieselben unvollständigen Begriffe von der Wirksamkeit gewisser Arzneikörper haben, etwa wie man dergleichen vom bloßen Hörensagen erhält, wenn man mit Aerzten oft umgeht; die Eine derselben aber vermeinet, daß sie genug wisse, um in einem vorkommenden Falle beurtheilen zu können, ob dieses oder jenes Mittel gebraucht werden solle, die Andere dagegen sieht ein, daß ihre Kenntniß zu einer solchen Beurtheilung lange nicht zureichend sey: so werden wir nur das Wissen der ersteren, keineswegs aber jenes der zweiten ein Halbwissen nennen dürfen. Wer mir dieß zugestehet, der begreift mich schon, wienach wir der Entstehung des Halbwissens vorbeugen können, der Unterricht, den wir in einer Wissenschaft ertheilen, sey noch so fragmentarisch, wenn wir es nur uns zum Gesetze machen, bei jeder einzelnen Lehre eigends zu untersuchen, welche etwaige Mißverständnisse oder \DruckVariante{Mißbräuche}{Mißbraüche} sie bei dem Vorhandenseyn dieser und jener irrigen Vorstellungen unsers Lehrlings veranlassen könnte, und nun umständlich nachweisen, daß solche Folgerungen aus unserm Satze nicht gezogen werden dürfen. Nicht ein fragmentarischer, sondern ein ungeschickt ertheilter Unterricht erzeuget Halbwissen.} \par 
In dem vierten Abschnitte: \danf{Bestandtheile eines Lehrbuchs, deren Eigenthümlichkeit aus andern Rücksichten hervorgehet,} (S.\,177--391), handelt der Vf.\ I. von Grundsätzen, II. Vergleichungen und Unterscheidungen, III. Bestimmungen, IV. Beschreibungen, V. Beweisen, VI. Einwürfen und Widerlegungen, VII. Beispielen, VIII. Betrachtungen bloßer Sätze und Vorstellungen. \par 
I. In dem ersten Absatze: \danf{von den Grundsätzen} (S.\,177--203) unterscheidet er objective oberste Grundsätze, subjective Erkenntnißprincipe und Hauptsätze. Ohne den Werth all dieser Arten von Grundsätzen zu verkennen, warnt er doch vor ihrer Überschätzung und vor dem Vorurtheile, das Vorhandenseyn eines obersten objectiven Grundsatzes \seitenw{95} bei einer jeden Wissenschaft vorauszusetzen. Bei dieser Gelegenheit wird dann in einer langen Anm. über das Streben nach Einheit gesprochen. Wie irrig es ist, den Satz vom Grunde so zu verstehen, als ob eine jede Wahrheit in einer anderen Wahrheit, und jedes Seyende somit in einem anderen Seyenden gegründet seyn müßte: so wenig gibt es auch eine unbedingte Forderung der Vernunft, jede Vielheit auf eine höhere Einheit zurückzuführen. Dieß Streben ist es, welches so viele unserer Weltweisen zu einer Art von Pantheismus verleitet, dessen Unhaltbarkeit der Vf.\ aufzudecken versucht. Nicht nur erreicht dieser Pantheismus auf keinen Fall jene absolute Einheit, nach welcher er strebt, weil doch immer das Seyende und die Wahrheiten an sich aus keiner Beides umfassenden Einheit sich ableiten lassen: sondern er kann auch nie genügend erklären, wie sein Absolutes sich in Gott und Welt spalte. \danf{Ist es,} frägt der Vf., \danf{nicht ein Widerspruch, zu sagen, daß etwas Absolutes, \dh\  doch wenigstens ein Wesen, dessen Seyn in keinem Anderen gegründet ist, welches somit als durchaus unabhängig, in allen seinen Kräften und Eigenschaften unendlich oder doch nur durch sich selbst begrenzt seyn sollte, gleichwohl veränderlich sey; indem alle diejenigen Veränderungen, welche die endlichen Dinge in dieser Welt erfahren, in seinem eigenen Innern vorgehen sollen? Gewiß ist es ungleich vernünftiger, oder es ist vielmehr das einzig Vernünftige, was sich hier sagen läßt, daß die Veränderungen, welche wir in der Welt wahrnehmen, nicht in der unbedingten Substanz, sondern in andern, bedingten Substanzen vorgehen, in Substanzen, welche ihr Daseyn dem Willen und der Wirksamkeit der unbedingten Substanz verdanken. Warum \DruckVariante{sträubt}{straübt} man sich doch nur vor der Annahme solcher bedingter Substanzen, deren Daseynsgrund in einer andern liegt; wenn man anders nicht von dem falschen Begriffe Spinoza's (Per substantiam intelligo id, quod in se est et per se concipi potest, h. i. id, cujus conceptus non indiget conceptu alterius rei, a quo formari debeat. Eth. def. 3.), sondern von der \WLpar{II}{142} gegebenen Erklärung ausgehet?} (Substanz ist ein Wirkliches, das keine Be\seitenw{96}schaffenheit ist.) \danf{Wenn man voraussetzen müßte, daß eine Substanz, die ihrem Daseyn nach in einer anderen gegründet ist, erst in der Zeit entstanden seyn müßte; dann dürfte man allerdings einen Widerspruch in der Annahme geschaffener Substanzen finden. Wenn man voraussetzen müßte, daß es solcher Substanzen nur eine endliche Menge gebe: dann könnte man wohl fragen, wie diese endliche Wirkung sich mit der unendlichen Kraft, die ihrem Schöpfer beigelegt wird, vergleiche? Wenn alles Unendliche als solches unbestimmt seyn müßte (wie freilich Viele glauben), dann müßte man allerdings an dem Gedanken einer aus unendlich vielen Theilen bestehenden Welt einen Anstoß nehmen. Wenn geistige und materielle Substanzen in der Art unterschieden wären, daß es keinen allmählichen Übergang von der Stufe der letztern zu jener der ersteren gäbe: dann könnte es freilich räthselhaft seyn, warum Gott zwei so entgegengesetzte Gattungen von Wesen geschaffen habe. Wenn daraus, weil alle erschaffenen Substanzen von gleicher Dauer sind, folgen müßte, daß zu demselben Zeitpunkte alle auch auf derselben Stufe der Ausbildung stehen: so müßte uns freilich die große Mannigfaltigkeit der Wesen, welche wir in der Welt antreffen, befremden. Wenn es Widersprüche, und nicht vielmehr recht wohl begreifliche und erweisliche Wahrheiten wären, daß ein Wesen, das unendliche Kräfte besitzt, auch mit denselben etwas hervorbringen müsse; daß die Substanzen, welche durch seine Wirksamkeit bestehen, als endliche Substanzen in einer wechselseitigen Einwirkung auf einander begriffen seyn müssen, und hiedurch und durch die stete Einwirkung jener unendlichen Substanz in das Unendliche fortschreiten müssen: dann hätte man Ursache, mit dem Systeme des Theismus unzufrieden zu seyn; allein auch dann noch früge es sich, was man denn wohl durch die Annahme des Pantheismus gewinne? Die Freunde des Letztern erklären die endlichen Dinge bald für Theile des Einen Absoluten (der Urmaterie), bald nur für Modificationen oder Accidenzen dieses alleinigen Wesens, bald noch bestimmter für bloße Gedanken desselben, bald wieder für Ausflüsse aus demselben, bald für die Wirkungen, die \seitenw{97} seine Selbstentfaltung oder Selbstoffenbarung erzeuge, bald für bloße in diesem Wesen vorhandene Relationen \usw\ Hegel insonderheit benützte zu dieser Erklärung die von ihm erfundene dialektische Methode, indem er sagte, daß sich das Absolute (oder Gott) zuerst \DruckVariante{entäußere}{entaüßere} und in der Gestalt seines Andersseyn als Natur erscheine, damit es sodann, durch Rückkehr zu sich selbst, zu seinem Selbstbewußtseyn gelange, d. i. zum Geiste werde. Ihm also waren alle endliche Wesen nichts Anderes als \deinfanf{vorübergehende Momente in dem unendlichen Entwickelungsprocesse des göttlichen Lebens.} -- Lasset uns diese Erklärungen doch etwas näher betrachten! Wenn man die einzelnen endlichen Dinge, die wir in dieser Gestalt gewahren, Theile des Einen Absoluten nennet: so muß man jene Dinge zu dem Innern des Absoluten rechnen, und somit nothwendig zugeben, daß dieses Absolute Veränderungen in seinem Inneren erfahre; was gewiß ungereimt ist. Allein in eben diese Ungereimtheit verfällt man auch bei all den übrigen Erklärungen, bei deren jeder man sich noch gegen irgend eine andere Wahrheit verstößt. So ist es \zB\ eine Wahrheit, welche wohl Jedem, der mit den Worten Substanz und Raum die allgemein angenommenen Begriffe verbindet, einleuchten muß, daß jeder Gegenstand, der einen besondern Theil des Raumes (auch nur den eines Punktes) einnimmt, eine eigene, von andern unterschiedene Substanz seyn müsse. Gegen diese Wahrheit verstößt man, wenn man die zahllose Menge der Körper für bloße Accidenzen oder Modificationen einer einzigen unendlichen Substanz erkläret. Wenn man die mancherlei endlichen Dinge der Welt für bloße Gedanken Gottes erklärt: so muß man, um folgerecht zu seyn, allen in Gott vorhandenen Gedanken ein Seyn von ähnlicher Art, wie es die Dinge dieser Welt haben, zugestehen; und wird dieß nicht zu den größten Ungereimtheiten führen? Gott denket alle Wahrheiten, also auch diejenigen, deren Gegenstand nichts Existirendes ist, \zB\ daß diese und jene Handlung, die nicht verrichtet wurde, hätte verrichtet werden sollen; er kennt und denkt auch alle mathematischen Begriffe, die gegenständlichen \seitenw{98} sowohl als gegenstandlosen, \zB\ auch die Begriffe $\sqrt{2}$, $\sqrt{-1}$, $0$, \usw\ Welches sind nun wohl jene wirklichen Dinge, die diesen Gedanken Gottes entsprechen? -- Die bildlichen Redensarten endlich, deren man sich bei der Darstellung des Emanationssystems nach seinen verschiedenen Modificationen in älterer sowohl als in der neuesten Zeit bedienet, können gewiß nur Solche befriedigen, die nie gewohnt sind, sich etwas klar und deutlich zu denken. Was soll man sich vorstellen unter einem Ausflusse aus Gott, was unter einer Selbstentfaltung, Entwicklung oder Selbstoffenbarung? -- Wie ungereimt ist es zu sagen, daß die endlichen Wesen nichts Anderes wären, als vorübergehende Momente in dem unendlichen Entwicklungsprocesse des göttlichen Lebens? Daß sich Gott erst \DruckVariante{entäußern}{entaüßern} und in sein Andersseyn übergehen müsse, um dann durch Rückkehr zum Selbstbewußtseyn zu gelangen? Die dialektische Methode, durch welche diese und die meisten übrigen Entdeckungen der absoluten Identitätsphilosophie erwiesen werden, wollen wir später betrachten. Hier genüge es nur, zu bemerken, daß die unzähligen Abstufungen, welchen wir in der wirklichen Welt begegnen, und die allmählichen Übergänge vom scheinbar Leblosen zu dem Lebendigen, vom Mineral zur Pflanze, von dieser zum Thiere, und von dem Thiere zum Menschen -- einem Systeme, welches überall nur eine aus Einheit entsprungene Dreiheit erklärlich findet, in der That schlecht entsprechen!} -- Sein hier versprochenes Urtheil über die dialektische Methode liefert B.\ erst im letzten Paragr. des Werkes; wir müssen es aber, um nicht zu \DruckVariante{weitläufig}{weitlaüfig} zu werden, übergehen. \par 
II. Die \danf{Vergleichungen und Unterscheidungen} (S.\,203--212) bemühet sich der Vf.\ zu ihrem beinahe vergessenen Werthe wieder zu erheben, indem er ihre mancherlei Vortheile auseinandersetzt; warnt aber hiebei vor den Fehlern der Unbestimmtheit, der falschen Spitzfindigkeit, des Schlusses von der gleichen Benennung auf die Gleichheit der Sache, wie von denjenigen geschieht, welche das so genannte moralisch Unmögliche als eine Art des Unmög\seitenw{99}lichen im eigentlichen Sinne betrachten. Auch wird \WLpar{IV}{494} ausgeführt, daß irrige Gleichsetzungen insgemein schädlicher sind als irrige Unterscheidungen. \danf{Dieß einmal schon darum, weil es insgemein schwerer hält, von einem Irrthume der ersten, als von einem der zweiten Art wieder zurückzukommen. Denn sehen wir gewisse Dinge für gleich an, so halten wir es nicht mehr für nöthig, sie jedes im Einzelnen genauer zu betrachten, und eben darum werden wir auch die zwischen ihnen obwaltenden Verschiedenheiten kaum jemals kennen lernen. Halten wir aber für ungleich, was doch in der That gleich ist: so liegt in unserm Irrthume selbst die Veranlassung zu einer näheren Betrachtung der für verschieden gehaltenen Dinge, und unser Irrthum kann also nicht lange bestehen,} \usw\ \par 
III. Unter den \danf{Bestimmungen} (S.\,212--230) verstehet B.\ hier bloß solche mit den Erklärungen bis jetzt \DruckVariante{häufig}{haüfig} verwechselte Sätze, die eine gewissen Gegenständen ausschließlich zukommende Beschaffenheit, somit Wechselvorstellungen angeben. Er \DruckVariante{räumt}{raümt} verschiedene hinsichtlich ihrer bestehende Vorurtheile hinweg, indem er \zB\ zeigt, daß solche Bestimmungen zwar von ganz vorzüglichem Werthe sind, wenn sie das eigentliche Wesen des Gegenstandes angeben (\dh\  diejenigen Beschaffenheiten desselben, in welchen alle übrigen objectiv gegründet sind), daß aber auch andere, und selbst bloß analytische im Allgemeinen keine Verwerfung verdienen; daß ferner sehr nützliche Bestimmungen auch in der Aussage eines bloßen Verhältnisses bestehen können, daß sie auch eine Verneinung, auch eine Eintheilung enthalten dürfen \usw\ Er gibt endlich an, wie sie beschaffen seyn müssen, wenn sie zu dem besonderen Zwecke als Kennzeichen dienen sollen. \par 
IV. \danf{Beschreibungen} (S.\,231--236) nennt der Vf.\ eine eigenthümliche Gattung von Sätzen, die oft nur aufgestellt werden müssen, damit das Bild, das sich die Leser von einem Gegenstande zusammensetzen, seine gehörige Beschaffenheit erhalte. Es sind nur einige Wissenschaften, wie die Moral, die Religionslehre, die Geschichte, in denen ein \seitenw{100} sorgfältigerer Gebrauch, als es bisher geschieht, von solchen Beschreibungen zu machen wäre. \par 
V. In dem wichtigen Absatze \danf{von den Beweisen,} (S.\,237--299) wird untersucht, welche Sätze eines Beweises bedürfen (wobei er das Bestreben, solche Sätze, die für sich selbst schon einleuchten, zu beweisen, wenn unter diesem Beweisen nicht vielmehr nur die Angabe des objectiven Grundes verstanden wird, mit allem Rechte tadelt), dann welche Vordersätze und Schlußarten zu gebrauchen seyen? (ob auch Erklärungen zu solchen Vordersätzen gehören?) welche Beschaffenheiten die Beweise in einem Lehrbuche nothwendig haben müssen? (\WLpar{IV}{516--521}) und welche bloß empfehlenswerth sind? (\WLpar{IV}{522--526}) Zu diesen letzteren zählt er leichte Behältlichkeit, begreiflichen Gang, Erklärung der Art, wie man den Satz gefunden haben mochte, besonders aber die Angabe des objectiven Grundes. Viel Neues und Beachtenswerthes wird in dem langen \WLpar{IV}{530} über die apagogische Beweisart beigebracht, und unter Anderm gezeigt, daß und auf welche Art jeder solche Beweis, wiefern sein Wesen in der Zurückführung auf eine Ungereimtheit besteht, vermieden werden könne. -- Ob es wahr sey, was man in neuerer Zeit \DruckVariante{häufig}{haüfig} behauptet hat, daß bei Beweisen aus reinen Begriffen (a priori) gar keine Möglichkeit des Gegentheils Platz greife; und daß eben deßhalb in der Mathematik, Metaphysik und Moral kein Meinen Statt finde, daß man (wie Hegel sagt) einem Menschen gleich den Mangel der ersten Bildung ansehe, wenn er von philosophischen Meinungen spricht, wird \WLpar{IV}{532} untersucht, und gezeigt, daß auch bei Beweisen aus reinen Begriffen ein Irrthum möglich sey, weil wir es nur auf das Zeugniß unsers Gedächtnisses oder unserer Sinne, also jedenfalls nur vermittelst eines Wahrscheinlichkeitsschlusses entnehmen, daß diese und jene in unserm Beweise gebrauchten Vordersätze von uns schon früher als wahr erkannt worden seyen. Ein Beispiel hievon gibt jeder im Rechnen begangene Fehler. Die hohe Zuversicht also, mit welcher der Mathematiker, der Moralist, und vielleicht auch der Metaphysiker einige ihrer Lehrsätze aussprechen, beruhet nicht darauf, daß sie diese Wahrheiten aus rei\seitenw{101}nen Begriffen ableiten, sondern auf andern Umständen, \zB\ daß schon so viele Andere diese Reihe von Schlüssen geprüft, und keinen Fehlschluß darin wahrgenommen haben, \udgl\ Daher denn auch, wo diese Umstände fehlen, der Mathematiker wenigstens nicht zu stolz ist, von einer bloßen Meinung zu reden; der Philosoph aber, der mit so hoher Zuversicht auftritt, nur sich selbst lächerlich macht. \par 
VI. \danf{Einwürfe und Widerlegungen.} (S.\,299--315). Wir wollen nur aufzählen, was der Vf.\ bei den letztern verlangt: a) den Sinn des Einwurfes nöthigenfalls erst noch deutlicher anzugeben; b) das Wahre darin einzugestehen; \par 
c) nachzuweisen, wie sich dasselbe mit unsern Lehren vertrage; d) nicht einmal ihre Wahrscheinlichkeit vermindre; e) das Falsche aufzudecken; f) zu zeigen, mit welchen Beschränkungen es allenfalls zugegeben werden könne; g) wie mangelhaft die dafür sowohl als h) auch für das Wahre darin geführten Beweise seyen; i) wie unsere Lehre sich vielleicht selbst aus des Gegners eigenen Behauptungen erweisen lasse; k) wie in sich selbst widersprechend diese Behauptungen seyen; endlich \par 
l) wie sie entstanden seyn mögen. -- Auch der bekannte Satz: Neganti incumbit probatio \umA\  wird hier besprochen. \par 
VII. \danf{Von den Beispielen.} (S.\,315--326). Wie diese eingerichtet seyn müssen, wenn sie das Verständniß erleichtern, den Vortrag abkürzen, die Aufmerksamkeit befördern, das Behalten sichern, zur Bestätigung oder zuweilen auch zu einem vollständigen Beweise dienen, endlich nützliche Wahrheiten gelegenheitlich verbreiten sollen. \par 
VIII. \danf{Von den Betrachtungen bloßer Vorstellungen und Sätze.} (S.\,326--391.) Unter diesem unscheinbaren Titel wird, nachdem der Vf.\ im Allgemeinen gezeigt hat, welche Vorstellungen und Sätze in einem Lehrbuche Gegenstand einer eigenen Betrachtung zu werden verdienen, und auf welche Beschaffenheiten derselben die Betrachtung sich zu erstrecken habe -- im Einzelnen A. von den Erklärungen, B.\ Vergleichungen und Unterscheidungen, C. Eintheilungen, und endlich D. von den Nachweisungen des objectiven Zusammenhanges gehandelt. \seitenw{102} \par 
A. \danf{Von den Erklärungen der Vorstellungen und Sätze.} (S.\,330--350). Ohne Zweifel eine der wichtigsten Abtheilungen. Vor Allem fasse man genau, was der Vf.\ unter Erklärung oder Definition verstehe. So nämlich nennt er nur die meistens mit einem eigenen Beweise zu begleitende Entscheidung der Frage, ob gewisse Vorstellungen oder Sätze einfach oder aus welchen Theilen sie zusammengesetzt sind. Nicht bloß Begriffe nämlich (wie es gewöhnlich geschieht), sondern auch gemischte Vorstellungen und ganze Sätze bedürfen oft einer Erklärung; welche man nach dem so eben Gesagten weder mit einem Bestimmungssatze, noch mit einer bloßen Verständigung verwechseln darf. Ich spreche einen Bestimmungssatz aus, wenn ich sage, es sey eine ausschließliche Beschaffenheit aller Dreiecke, daß sie Figuren seyen, deren sämmtliche Winkel zwei rechte betragen. Ich verständige mich über diesen Begriff, wenn ich Jemand mehrere Dreiecke vorzeichne, und hiebei sage, daß man dergleichen Figuren Dreiecke nenne. Eine Erklärung dieses Begriffes aber gebe ich nur, wenn ich sage, daß er aus den Begriffen Figur, Seite, drei \usw\ auf die Art zusammengesetzt sey, welche die folgenden Worte: Ein Dreieck ist eine Figur, welche drei Seiten hat, zu erkennen geben. Es wird nun untersucht, welche Vorstellungen und Sätze einer Erklärung, und welche Erklärungen eines Beweises bedürfen, und wie dieser zu führen, wenn man die Vorstellung für einfach, und wie, wenn man sie für zusammengesetzt erkläret. Ein und derselbe Begriff hat auch nur eine und dieselbe Erklärung; und die so genannten Namen-, Sach- und genetischen Erklärungen sind nicht Erklärungen des nämlichen, sondern verwandter Begriffe, etwa solcher, die einerlei Gegenstände haben. Wenn der zu erklärende Begriff ein Verhältnißbegriff ist, \zB\ Meilenzeiger, so muß auch seine Erklärung durch ein Verhältniß -- und wenn er verneinend ist, \zB\ schiefer Winkel, durch eine Verneinung geschehen; es ist somit ein unrichtiger Kanon der Logiker, daß eine Erklärung nie auf einem Verhältnisse beruhen, oder verneinend seyn dürfe. Eben so falsch ist es, daß jede Erklärung aus der Angabe der nächsten Gattung und des Artunterschiedes \seitenw{103} bestehen müsse. Der Begriff: Nichts, kann gewiß nicht anders erklärt werden als durch die Worte: Nicht etwas; er ist gegenstandlos, und kann somit weder eine nächste Gattung, noch einen Artunterschied besitzen. -- Ein Vorurtheil ist es auch, daß eine jede Erklärung leicht verständlich seyn müsse. Zwar \danf{daß man auf möglichst leichte Verständlichkeit, wie des ganzen wissenschaftlichen Vortrags, so insbesondere auch der Erklärungen dringe (sagt der Vf.\ \WLpar{IV}{559}), verdienet alles Lob. Allein wenn Einige verlangen, daß jede Erklärung leichter verstanden werden solle, als der zu erklärende Begriff, mit dem man doch vielleicht schon von seiner Kindheit an vertraut ist: dann gehen sie zu weit, und scheinen die Erklärungen mit den Verständigungen zu verwechseln. Wenn wir den Zweck einer bloßen Verständigung haben, dann handeln wir allerdings thöricht, wenn wir nicht das leichteste zu diesem Zwecke führende Mittel wählen; und wenn der Begriff sowohl als auch das Zeichen schon bekannt sind, bedarf es gar keiner Verständigung über dasselbe. Wenn aber die Bestandtheile aufgezählt werden sollen, aus denen wir einen, seit unsrer Kindheit uns schon \DruckVariante{geläufigen}{gelaüfigen} Begriff, wir wissen selbst nicht wie, zusammensetzen: dann ist es kein hinreichender Grund zur Verwerfung unsers Versuches, daß es einige Mühe verursacht, die von uns angegebenen Bestandtheile alle zu fassen, sie in die vorgeschriebene Verbindung zu bringen, und zu erkennen, daß der so entstehende Begriff wirklich derselbe sey mit dem, den wir mit jenem bekannten Zeichen verbinden.} -- Auch die Behauptungen, daß empirische Begriffe,ingleichen Vorstellungen von Individuen nie definirt werden könnten, erklärt der Vf.\ für irrig; denn so schwierig es auch in manchen Fällen seyn mag, darüber zu entscheiden, ob eine gewisse in unserm Bewußtseyn vorhandene Vorstellung einfach sey, oder aus welchen einfacheren wir sie zusammensetzen: können wir wohl jemals berechtiget seyn, dieß für unmöglich zu erklären? -- \Usw\ \par 
B.\ Nur kurz wird (S.\,350--352) besprochen, wie nöthig auch bloße \danf{Vergleichungen zwischen Sätzen und \seitenw{104} Vorstellungen} oft sind, und wie sie angestellt werden müssen. Desto ausführlicher ist \par 
C. \danf{von den Eintheilungen} (S.\,353--385) die Rede, wo abermals gar manche in der gewöhnlichen Lehre stehen gebliebene Mängel berichtiget werden. Zuvörderst werden mehrere bisher nicht unterschiedene Zwecke und Arten der Eintheilungen nachgewiesen, dann wird ihre besondere Einrichtung zu diesen Zwecken bestimmt, und manche wichtige Untersuchung eingeflochten, vornehmlich die von den Naturforschern so oft besprochene Frage über die Möglichkeit eines \danf{natürlichen Systemes.} Vor Allem zeigt der Vf., sehr einleuchtend, wie uns \DruckVariante{däucht}{daücht}, daß wir unter der Redensart: \danf{ein gewisses System oder ein Gesetz überhaupt werde von der Natur befolgt,} für den Zweck einer Eintheilung der Systeme in natürliche und nicht natürliche, noch etwas Mehreres verstehen müssen, als daß die natürlichen Dinge diesem Gesetze gemäß sind (daß es ein wahrer Satz sey). Wir können dieß nur in sofern, als wir uns die Natur, oder vielmehr ihren Urheber, Gott, als ein vernünftiges und freiwaltendes Wesen denken; und nur dann finden wir es schicklich, eine gewisse Regel eine von der Natur (besser von Gott) befolgte Regel zu nennen, wenn sich vernünftiger Weise behaupten läßt, daß es gewisse Wesen und Einrichtungen in der Welt nur eben darum gebe, damit diese Regel erfüllet werde; ein Fall, der wieder nur dann statt hat, wenn sich begreifen läßt, daß die Summe des Wohlseyns hiedurch erhöhet werde. \danf{Hieraus wird man nun schon entnehmen} (sagt der Vf.), \danf{daß es eine gewagte Sache sey um die Behauptung, daß es ein wahres Natursystem gebe, und noch mehr, daß es in diesen und jenen in der Welt wahrgenommenen Arten und Gattungen bestehe. Aus jenem einzigen uns bekannten Gesetze des göttlichen Waltens, das die Zustandebringung der möglich größten Glückseligkeit bei den geschaffenen Wesen bezwecket, begreift sich wohl noch leicht, warum es eine zahllose Menge von Verschiedenheiten unter den Geschöpfen geben müsse. Es ist überdieß nicht zu verkennen, daß die \seitenw{105} unzähligen Arten organischer sowohl als unorganischer Wesen, die wir auf Erden antreffen, zu dem Zwecke dienen, daß jede einzelne Substanz Gelegenheit zu einer allmählichen Entwicklung aller ihrer Kräfte finde. Daher ist denn auch sehr wahrscheinlich, daß eine gewisse Rangordnung unter diesen mannigfaltigen Arten herrsche, in dem Sinne, daß einige derselben auf einer höheren, andere auf einer niedrigeren Stufe der Vollkommenheit stehen, und daß von diesen auch ein Übergang zu jenen statt finde. So ist es \zB\ kaum zu bezweifeln, daß ein organisches Wesen (ich meine, die diesen Organismus belebende Seele) auf einer höheren Stufe stehe, als eine unorganische Materie, und daß eben so unter den organischen Wesen die Thiere überhaupt höher stehen als die Pflanzen. Die von dem geistreichen Oken gemachte Beobachtung, daß in gewissen Gattungen organischer Wesen gewisse, in andern wieder andere Organe eine ganz vorzügliche Ausbildung gewinnen, erzeugt die Vermuthung, daß diese Einrichtung wohl nur darum bestehe, damit eine und eben dieselbe Substanz, wenn sie allmählich alle jene verschiedenen Stufen des Daseyns durchgeht, auf jeder einzelnen Gelegenheit erhalte, gewisse in ihr schlummernde Anlagen und Kräfte zu entwickeln, sich mit Vorstellungen einer eigenen Art zu bereichern, und so immer vollkommener zu werden. Wenn wir nun durch solche Betrachtungen veranlaßt, den Entschluß faßten, ein System auszudenken, in welchem die sämmtlichen auf Erden anzutreffenden Naturprodukte nach ihrem vermuthlichen Range geordnet würden, so daß wir diejenigen, welche uns höher als andere zu stehen scheinen, immer in eine folgende Classe versetzten, diejenigen aber, die ihrer großen Ähnlichkeit wegen von einem gleichen Range scheinen, in Eine Classe zusammenstellten: so käme ein System zum Vorschein, welches, wenn irgend ein von Menschen anzugebendes, den Namen eines natürlichen am ehesten noch verdienen würde. Denn wenn wir irgendwo zu der Vermuthung berechtigt seyn können, daß der Schöpfer einer gewissen Gattung von Wesen nur darum oder doch auch darum Daseyn gegeben habe, damit in einer von uns entworfenen Eintheilung keine Lücke entstehe: so ist es dort, \seitenw{106} wo es den Anschein hat, daß das Vorhandenseyn einer solchen Gattung nothwendig sey, um keine Sprosse des Übergangs auf der Stufenleiter der Schöpfung zu verlieren. Hiebei muß ich jedoch erinnern, daß dieß natürliche System, so schätzbar es auch in vielen Rücksichten wäre, gewiß doch nicht für alle Zwecke, zu welchen Systeme oder Eintheilungen in der Naturwissenschaft nothwendig sind, genügen würde. Es müßte also davon immer noch andere (künstliche) geben; es müßte namentlich wenigstens Eines noch aufgestellt werden, in welchem wir die Naturkörper bloß nach gewissen \DruckVariante{äußern}{aüßern} leicht zu ermittelnden Kennzeichen ordnen, damit derjenige, der einen vorliegenden Körper genauer kennen zu lernen wünscht, ohne viel Mühe die Stelle im Systeme auffinden könne, wo seine sämmtlichen bisher bekannten Beschaffenheiten zusammengestellt sind.} \par 
D. Über die \danf{Nachweisungen des objectiven Zusammenhanges} (S.\,385--391) stellt der Vf.\ sehr gemäßigte Forderungen auf; doch sind auch diese bisher, besonders in den reinen Begriffswissenschaften, in der Philosophie (der theoretischen sowohl als praktischen) und in der Mathematik, zum größten Nachtheile für den in beiden zu erreichenden Grad der Vollkommenheit noch sehr vernachlässiget worden. Der Mathematiker (um nur von der letzteren Wissenschaft hier zu reden) liefert in seinen Beweisen, besonders in der Geometrie viel \DruckVariante{häufiger}{haüfiger}, als man es noch bedacht zu haben scheinet, statt einer objectiven Begründung bloße Gewißmachungen. So wird \zB\ gleich der erste Satz im Euklides, daß es zu je zwei Punkten $a$, $b$ einen dritten $c$ gebe, von welchem sie eben so weit als von einander selbst abstehen, bekanntlich nur daraus erwiesen, weil sich die mit dem Halbmesser $ab$ aus $a$ und $b$ (in einerlei Ebene) beschriebenen Kreislinien irgendwo (nämlich in $c$) schneiden müssen. Dieß Schneiden aber enthält sicherlich nicht den objectiven Grund des Vorhandenseyns jenes Punktes, sondern umgekehrt schneiden die Kreislinien einander eben, weil es einen solchen dritten Punkt gibt. Ein Gleiches \BUgriech{<'usteron pr'oteron} wird bei einer Menge anderer Sätze begangen, und daher kommt es dann, daß man den objectiven Grund gewisser \seitenw{107} anderer Sätze (\zB\ des Satzes von den Parallelen) nicht aufzufinden vermochte, welcher sich leicht genug darbietet, wenn man einmal die Lehren in ihre natürliche Stellung gebracht hat. \par 
Daß nun so viele der bisherigen Lehre widersprechenden Ansichten, als B.\ in diesem \DruckVariante{weitläufigen}{weitlaüfigen} Hauptstücke vorträgt, sich nur schwer Bahn brechen werden, das läßt sich wohl ohne viel Menschenkenntniß voraussetzen. Zählten wir doch in dem einzigen \WLpar{IV}{559}, der die gewöhnliche Lehre von den Erklärungen prüft, an sechszehn, und im \WLpar{IV}{575}, darin die Lehre von den Eintheilungen beurtheilt wird, an neun bis eilf Punkte, die B., freilich in aller Bescheidenheit, angreift. Zur Steuer der Wahrheit müssen wir inzwischen für diejenigen unserer Leser, die unbefangen genug sind, um nach diesem Umstande zu fragen, erinnern, daß alle hier vorkommenden Lehren B.'s aus dem Vorhergehenden keiner andern Zugeständnisse bedürfen, als dieser drei: 1) daß es einfache sowohl als auch zusammengesetzte Vorstellungen gebe; 2) daß sich Begriffe und Anschauungen objectiv unterscheiden lassen; 3) daß sich auch irgend ein Unterschied denken lasse zwischen dem objectiven Zusammenhange der Wahrheiten und zwischen der bloßen subjectiven Ableitbarkeit des Einen aus dem anderen Satze. \par 
Bei dem nun folgenden fünften Hauptstücke \danf{von den Abtheilungen eines Lehrbuchs} (S.\,392--423), besorgen wir wenig davon, daß der Vf.\ gleich anfangs -- (eigentlich wurde es schon \WLpar{IV}{472} erwähnt) -- die ungewöhnliche Bemerkung macht, daß Abtheilungen in einem Buche, selbst wenn sie keine Überschrift führen, wirkliche Sätze wären; denn dieses übt auf die nun folgenden Auseinandersetzungen gar keinen weiteren Einfluß. Überhaupt dürfte unsers Erachtens dieß Hauptstück unter allen sich noch am ehesten einer günstigen Aufnahme zu gewärtigen haben. Denn was sollte man wohl gegen die acht \WLpar{IV}{580} angedeuteten Vortheile, die durch geschickte Abtheilungen erreicht werden können, oder gegen die allgemeinen Regeln für das Geschäft des Abtheilens in \WLpar{IV}{581}, oder gegen die besondern Vorschrif\seitenw{108}ten für Abtheilungen, die 1) auf der besonderen Weise, wie die Sätze im Buche vorgebracht werden; 2) auf der innern Beschaffenheit der gebildeten Theile; 3) auf ihrem Verhältnisse unter einander; 4) auf den abgehandelten Gegenständen; 5) auf unsrer Erkenntnißart der Sätze; 6) auf ihrem Gebrauche; 7) auf ihrem Verhältnisse zu unserm Empfindungsvermögen beruhen; oder die 8) das Verstehen oder 9) das Auffinden, oder 10) das Behalten erleichtern, oder 11) auf das Verhältniß der Sätze zu unsrer Wissenschaft, oder endlich 12) auf das Verhältniß derselben zu unserm Lehrbuche sich gründen, -- einzuwenden haben? Daß er auch hier nicht ganz von seiner Gewohnheit lassen, und \zB\ (\WLpar{IV}{593}) den Begriff einer Einleitung von dem eines vorbereitenden Theiles unterscheidend, den Inhalt der ersteren genauer dahin beschränket und sehen will, \danf{daß sie nur lauter gelegenheitliche Lehren, und zwar nur diejenigen enthalte, die man dem Vortrage der wesentlichen Lehren oder auch der zu denselben benöthigten Hülfslehren noch vorausschickt,} -- kann man ihm wohl vergeben. \par 
Doch auch das sechste Hauptstück: \danf{von der Ordnung, in welcher die in ein Lehrbuch gehörigen Sätze vorgebracht werden sollen,} (S.\,424--499) enthält, so viel wir uns erinnern, nichts, was das Gemüth der Leser zum Widerspruch aufreizen könnte; denn es ist eigentlich das matteste aus allen, der Vf.\ selbst weiß es, daß er hier, vornehmlich in der zweiten Abtheilung Manches gesagt, was als von selbst sich verstehend, auch hätte wegbleiben können. Indessen sollen die Leser auch aus dieser Abtheilung einige Proben erhalten. S.\,477 spricht der Vf.\ über die Stelle, die den geschichtlichen Mittheilungen in einem Lehrbuche anzuweisen sey. \danf{Geschichtliche Mittheilungen über unsre Wissenschaft,} behauptet er, \danf{lassen sich an den verschiedensten Orten mit Nutzen anbringen. Einmal geziemet es sich schon gleich im Anfange zu bemerken, ob die Wissenschaft, die wir in unserm Buche darstellen wollen, etwa von uns zuerst bearbeitet werde, oder bereits seit einer längeren Zeit bestehe, und für den letzteren Fall in Kürze anzuzeigen, bei wie vielen Völkern und wie vielfältig sie \seitenw{109} schon bearbeitet sey, und welches die gelungensten ihrer Lehrbücher wären. Kommen wir dann zur Darstellung ihrer einzelnen Lehren, so wird es wieder bei jeder wichtigeren zweckmäßig seyn, zu erzählen, wer ihr Erfinder gewesen, wie er auf sie gekommen, mit welchen Schwierigkeiten man bei ihrer Verbreitung zu kämpfen gehabt, was für verschiedene Ansichten über diesen Gegenstand früher geherrscht oder auch jetzt noch anzutreffen sind \usw\ In den meisten Fällen werden dergleichen Bemerkungen der Aufstellung einer Lehre am Besten nachfolgen; zuweilen aber wird es doch zuträglicher seyn, die widerstreitenden Meinungen dem Leser vorzulegen, bevor wir ihm noch unsere eigene Meinung eröffnen. Das Eine, wenn die verschiedenen Meinungen leichter gefaßt und richtiger beurtheilt werden können, nachdem wir erst dasjenige, was uns als Wahrheit erscheint, vorausgeschickt haben; das Andere, wenn dadurch die Aufmerksamkeit der Leser gespannt, und ein um so unbefangneres Urtheil bei ihnen erzielt werden kann. Noch andre geschichtliche Mittheilungen versparen wir füglicher erst auf das Ende eines Abschnittes, oder wohl gar dahin, wo wir den Vortrag aller, in unsere Wissenschaft gehörigen Lehren schließen. So nämlich bei Mittheilungen, welche zu \DruckVariante{weitläufig}{weitlaüfig} sind, oder in zu geringem Zusammenhange mit unserm Vortrage stehen} \usw\ -- Nachdem in \WLpar{IV}{632} über den Ort der Erklärungen das Nöthige gesagt, heißt es in der Anmerkung: \danf{Wer dem so eben Gesagten beistimmt, wird den gewöhnlichen Kanon älterer Logiker, daß man mit den Erklärungen anfangen müsse, nicht in Schutz nehmen wollen; doch wird er denselben nicht so befremdend finden, wenn er sich erinnert, daß jene Logiker bei ihren Erklärungen insgemein nur an das gedacht, was ich Verständigungen nenne, oder daß sie doch beide Geschäfte, das Erklären und das Verständigen, immer vereinigt abthun zu müssen geglaubt. Was aber soll man dazu sagen, wenn Kant (der hierin freilich schon an Campanella einen Vorgänger gehabt) die Behauptung aufstellte, daß man in philosophischen Abhandlungen weit entfernt, mit der Erklärung anfangen zu können, im günstigsten Falle \seitenw{110} mit ihr nur endigen dürfe? Ich gestehe offen, daß mir kein Grund bekannt sey, durch den sich eine solche Übertreibung rechtfertigen ließe. Denn so schwer es auch, was ich gern zugebe, seyn mag, gewisse philosophische Begriffe zu erklären: so gibt es ja doch auch in mancher andern Wissenschaft, namentlich selbst in der Mathematik Begriffe, deren Zergliederung nicht minder schwierig ist, \zB\ die Begriffe von Linie, Fläche, Körper, Richtung, Länge einer Linie, Inhalt einer Fläche, eines Körpers, Krümmung \umA\  Dennoch lesen wir nicht, daß Kant den Mathematikern die gleiche Erlaubniß gegeben hätte, die Erklärung ihrer Begriffe erst an das Ende ihres Vortrages zu verlegen. Und wenn es überhaupt nur keine Unmöglichkeit ist, sich die Bestandtheile, aus welchen man einen Begriff zusammensetzt, durch längeres Nachdenken zum deutlichen Bewußtseyn zu bringen: warum müßte dieß nothwendig erst am Ende einer Abhandlung geschehen? Wenn überdieß die Zerlegung eines Begriffes in dieser Behauptung nicht auf das Bestimmteste von einer bloßen Verständigung über ihn unterschieden wird: zu welch' einem verworrenen Hin- und Herreden in philosophischen Dingen gibt man nicht Anlaß und Berechtigung, wenn man behauptet, hier könne es dem Leser nicht eher als erst am Ende der ganzen Abhandlung, wenn man zu reden aufgehört hat, klar werden, welche Begriffe man mit den gebrauchten Worten verbunden oder nicht verbunden habe?} \par 
Nur noch Ein Hauptstück von etwas größerem Umfange erübriget uns aus dem ganzen Werke; es ist das siebente: \danf{von den in einem Lehrbuche theils vorzuschlagenden, theils zu gebrauchenden Zeichen} (S.\,500--596), \par 
welches in zwei Abschnitten erst von den Zeichen, welche den Lesern vorzuschlagen sind, dann von denjenigen, die wir dort selbst gebrauchen sollen, handelt. In dem ersten Abschnitte (S.\,504--517) wird vor Allem bemerkt, wie viele Arten von Zeichen es gebe, mit denen ein sorgfältiger Schriftsteller seine Leser bekannt machen muß; es sind dieß nämlich nicht nur Zeichen, durch die sie in Stand \seitenw{111} gesetzt werden, über die Gegenstände unseres Unterrichts bei sich selbst nachzudenken, sondern auch solche, vermittelst deren sie ihre Gedanken Andern in mündlichem Gespräche sowohl als schriftlich mittheilen könnten. Hierauf wird untersucht, welche Beschaffenheiten diesen Zeichen gemeinschaftlich, und welche besonders den mündlichen zukommen müssen; es wird ihr Zusammenhang theils unter einander, theils mit den Zeichen, deren wir uns im Buche selbst bedienten, bemerklich gemacht; es werden die besondern Rücksichten, welche wir bei der Bestimmung dieser Zeichen zu nehmen haben; es wird endlich die Einrichtung unsrer Vorschläge und ihrer Rechtfertigung (denn oft bedürfen sie einer eigenen Rechtfertigung) besprochen. Im letzten §. dieses Abschnitts wird über den einem Buche zu gebenden Titel gesprochen; und es heißt unter Anderm, ein guter Titel müsse a) wenn nicht neu seyn, doch etwas so Eigenes haben, daß es nicht einen zweiten ihm völlig gleichen gibt, mindestens bei einem Buche, mit dem das unsrige leicht zu verwechseln wäre; er müsse b) kurz seyn; c) dem einmal herrschenden Geschmacke nicht allzusehr widerstreiten, und überhaupt keine widrigen, aber auch d) keine solchen Nebenvorstellungen wecken, die eine allzu hohe Meinung von unserm Buche verrathen. Oft thue man wohl, seinem Buche einen obgleich sehr dunkeln und unbestimmten, doch kurz und neu lautenden Titel zu geben, dem sodann eine nähere Erklärung beigefügt ist. \par 
Der zweite Abschnitt ist wie der wichtigere, so auch der größere (S.\,517--596). Er zerfällt in zwei Abtheilungen, deren erste die allgemeinen, die zweite die besonderen Regeln umfaßt. Mit Recht erklärt der Vf.\ \WLpar{IV}{614} die Reinheit (\dh\  die Abwesenheit schädlicher Nebenvorstellungen) unter die wesentlichen Beschaffenheiten eines Zeichens, und will \zB\ Ausdrücke, die einem früheren Jahrhunderte ganz unanstößig klangen, in unsern Tagen aber von gewissen widrigen Nebenvorstellungen nicht mehr zu reinigen sind, lieber bei Seite gelegt, damit nicht um des Wortes die Sache selbst verliere. -- \danf{Tadelnswerth,} sagt er \WLpar{IV}{655}, \danf{ist es nur, wenn einem Zeichen zwei oder mehre Bedeutungen beigelegt werden, die eine so große Ähnlichkeit mit einander haben, \seitenw{112} daß es schwer hält, aus dem jeweiligen Zusammenhange der Rede allein zu erkennen, welche derselben gemeint ist, oder daß es dem Unaufmerksamen selbst nach vorhergegangener Warnung begegnet, von der einen Bedeutung zur andern überzuspringen. Geschieht dieß, dann denkt sich der Leser bei unsern Worten nicht mehr, was wir doch wollten, daß er sich denken möge, \dh\  wir werden mißverstanden. Und besonders dann werden wir uns vergeblich bemühen, einem solchen Mißverstande unserer Worte zu wehren, wenn sich auf Seite der Leser Leidenschaft einmischt, und sie ihre Rechnung dabei finden, jede Redensart, welche zuweilen in einem schlimmen oder thörichten Sinne gebraucht worden ist, bei uns in dem schlimmsten zu nehmen, \dh\  wenn mit dem Mißverstande auch noch Mißdeutung sich vereinigt. Auf dem Gebiete der Theologie ist dieses, leider! gar keine seltene Erscheinung, so zwar, daß man denselben Schriftsteller oft von der einen Partei als einen verkappten Freigeist, von der andern als einen Finsterling verschreien hört, nicht weil er das Eine oder das Andere ist, sondern weil jede Partei seine Worte auf eine eigene Weise deutet.} (Ist das nicht nahebei das Schicksal des Vfs.?) -- \danf{Nur dann,} (heißt es \WLpar{IV}{660}) \danf{kann es erlaubt seyn, von einer Bezeichnung, die ein Anderer vor uns gebraucht hat, abzugehen, wenn es entweder a) nicht möglich ist, dem Beispiele des Einen zu folgen, ohne das eben so beachtungswerthe eines Anderen zu verlassen; oder wenn b) die bisher übliche Bezeichnung entschieden zweckwidrig ist in einem Grade, daß die Vortheile, die wir durch ihre Abänderung zu erreichen hoffen, die dabei unvermeidliche Unbequemlichkeit weit überwiegen.} -- \danf{Daß Kunstwörter gebildet werden sollen} (liest man \WLpar{IV}{661}), \danf{so oft es der Zweck der Wissenschaft fordert, weil uns ein Zeichen nöthig ist, für einen Begriff, für welchen die Sprache des gemeinen Lebens kein schickliches hat, unterliegt keinem Streite. Allein nur allzuoft geschieht es, daß man auch dort Kunstwörter bildet, wo es das Beste der Wissenschaft auf keine Weise erheischt. Nur ein gelehrteres Aussehen will man oft seinem Vortrage geben, nur durch die neuen Worte, welche man eingeführt hat, will man den Anschein, als ob \seitenw{113} man die Wissenschaft mit eben so vielen neuen Begriffen bereichert hätte, gewinnen \usw\ Ein solches Verfahren ist nicht nur in jedem Falle tadelnswerth, sondern in solchen Wissenschaften, von deren Lehren es zu wünschen ist, daß sie recht ausgebreitet würden, und in das gesellige Leben selbst übergehen möchten, ist es als eine wahre Versündigung an der Menschheit zu betrachten.} \par 
Die verschiedenen \WLpar{IV}{668} beschriebenen Weisen, wie wir uns über den Sinn unserer Zeichen mit unsern Lesern verständigen können (dergleichen Verständigungen also etwas ganz Anderes als die früher besprochenen Bestimmungen und Erklärungen sind), dürften durch ihre Anzahl wohl Manchen überraschen; und kaum möchte es etwas Anderes als Eigensinn seyn, wenn man in Werken, wo man sich über schwer zu fassende und noch schwerer zu erklärende Begriffe zu verständigen hat, statt seine Zuflucht zu einigen der hier vorgeschlagenen Mitteln zu nehmen, die Worte nur geradezu anwenden wollte, unbekümmert darum, ob und wann der Leser aus dem Gebrauche selbst ihre Bedeutung sich werde abzuziehen wissen. Erst wenn wir das, was der Vf.\ schon früher (\WLpar{IV}{500--509}) über Bestimmungen und (\WLpar{IV}{554--559}) über Erklärungen beigebracht hat, mit der uns jetzt vorliegenden Lehre von den Verständigungen (\WLpar{IV}{668--670}) vergleichen, also auch die zu dem letzten §. gehörige Anmerkung über die Klage, daß Jemand in seine Erklärungen lege, was er beweisen solle, nicht übersehen, endlich auch noch die \WLpar{IV}{698} gelieferte Prüfung der Ansichten Anderer hinzunehmen, wird es uns klar werden, wie groß und folgenreich der bis auf den heutigen Tag fast gänzlich übersehene Unterschied zwischen Bestimmungen eines Gegenstandes, Erklärungen eines Begriffs und Verständigungen über ein Zeichen sey; und wir werden ahnen, wie vielen Verirrungen auf dem Gebiete der Philosophie dadurch allein ein Ende gemacht werden könnte, daß man der letzteren beiden sich auf gehörige Weise bediente. \seitenw{114}\par 
\gliederungslinie
Wir können nicht umhin, bevor wir dieß Hauptstück verlassen, noch ein paar Stellen daraus ihres Zeitinteresses wegen hervorzuheben. Was der Vf.\ \WLpar{IV}{675} über die Nothwendigkeit einer gelehrten Universalsprache sagt, \DruckVariante{däucht}{daücht} uns der ernstesten Erwägung werth: \danf{Statt eine einzige gelehrte Sprache unter sich einzuführen, haben sich die Gelehrten Europa's entschlossen, ein Jeder nebst zwei, auch mehreren sogenannten alten Sprachen, noch die Volkssprachen aller übrigen wenigstens in dem Grade zu erlernen, um in denselben Geschriebenes zur Noth verstehen zu können. So erfreulich dieß eines Theils ist, weil es die rühmlichste Wißbegierde und eine gegenseitige Achtung beweiset: so kann man doch fragen, ob die Erlernung so vieler Sprachen, die wir jetzt einem jeden Gelehrten zumuthen, nicht der Erlernung wichtigerer Kenntnisse Abbruch thue, und besonders wie das enden solle, wenn allmählich auch alle übrigen Völker (nur in Europa), die ihre eigenen Sprachen reden, Werke, die einer allgemeinen Aufmerksamkeit werth sind, zu schreiben anfangen werden? Schon jetzt \DruckVariante{beläuft}{belaüft} sich ja die Anzahl der Sprachen, deren Kenntniß wir von einem jeden Gelehrten verlangen, auf 7 bis 8; fahren wir aber so fort, wird bald ein Dutzend nicht genügen.} In der That, daß ein jeder Mensch (wir sagen, ein jeder; und wollen somit auch die Mitglieder der untersten Volksclasse nicht ausgeschlossen wissen) neben seiner Muttersprache noch eine zweite Sprache (jene gelehrte nämlich) erlerne, das wäre wohl eben kein unbilliges oder schwer auszuführendes Verlangen; und wie unsäglich viele Vortheile würde es gewähren noch außer dem, daß die Gelehrten sich dieser Sprache bedienen könnten bei Abfassung aller Werke, die auf dem ganzen Erdenrunde gelesen werden sollen. Was aber wollen wir beginnen, wenn sich die Anzahl der Völker, die Lesenswerthes schreiben, wie wir doch wünschen müssen, verdoppeln, verdreifachen wird? \par 
Im \WLpar{IV}{697} spricht der Vf.\ von den gewöhnlichsten Fehlern der schriftlichen Darstellung in einem Lehrbuche unter Anderm: \danf{Ich mache den Vorwurf einer geflissentlichen Dunkelheit nur dem, der mit Bedacht \seitenw{115} seine Worte so stellt, daß man nicht deutlich abnehmen kann, was er meine; und dieses thut, weil er aus einer solchen Unbestimmtheit seines Ausdrucks unerlaubte Vortheile für sich zu ziehen hofft; also nur dem, der dunkel ist, weil er hofft, daß seine Leser gutmüthig genug seyn werden, in Allem, was sie nicht recht verstehen, tiefe und eben nur ihrer Tiefe wegen von ihnen nicht ergründete Wahrheiten zu verehren; oder dem die Vieldeutigkeit seiner Ausdrücke behülflich werden soll, wenn wir die Ungereimtheit und den inneren Widerspruch seiner Behauptungen aufdecken, mit einigem Anschein des Rechts zu entgegnen, daß wir ihn mißverstanden hätten. Wir sind nun, \DruckVariante{däucht}{daücht} mir, berechtigt, dergleichen unredliche Absichten bei dem dunkeln Vortrage eines Schriftstellers zu vermuthen, wenn wir gewahr werden, a) daß er sich eine Menge neuer Ausdrücke schafft, oder auch schon \DruckVariante{gebräuchliche}{gebraüchliche} in neuen Bedeutungen nimmt, ohne in beiden Fällen der Pflicht zu gedenken, sich erst über den Sinn dieser Zeichen mit seinen Lesern zu verständigen; wenn er b) Ausdrücke braucht, die eine mehrfache Bedeutung zulassen, ohne je zu erklären, welche derselben er meine, obgleich auch der Zusammenhang den Sinn, in welchem wir sie zu nehmen haben, unbestimmt läßt; wenn er endlich c) gerade dort, wo wir am ehesten eine bestimmte Erklärung erwarten sollten, sich hinter allgemeine bildliche Ausdrücke zurückzieht.} -- \danf{Ungern sage ich es} (heißt es sodann in der Anm.), \danf{allein die Pflicht der freimüthigen Rüge dessen, was wir als einen verderblichen Unfug erkennen, verbindet mich, nicht zu verhehlen, daß ich fast alle jetzt eben angeführten Fehler in den Schriften einiger der gepriesensten neueren Weltweisen, namentlich in den Schriften Hegels vereinigt anzutreffen glaube; ja was noch betrübender ist, daß diese Eigenheiten von einem großen Theile des deutschen Publicums gar nicht als Fehler angesehen, sondern als Tugenden bewundert werden. Aber so ist es leider! Schriftsteller sowohl als Leser finden in Deutschland gegenwärtig an einer Schreibart, welche jeden Gedanken in eine aus dunkeln Worten gewobene Wolke so einhüllt, daß er zur Hälfte nur durchblickt, ein so ausschließliches Wohlgefallen, daß Bücher aus dem \seitenw{116} Gebiete der Philosophie, deren Vf.\ einem so verdorbenen Geschmacke nicht huldigen wollen, fast in Gefahr stehen, ungelesen zu bleiben. Was klar und verständlich ist, wird eben darum gering geachtet; man schämt sich, es nachzuerzählen; denn, meint man, es klinge nicht gelehrt. In Räthseln muß sprechen, wer Aufmerksamkeit zu erregen wünscht; und wer seine Unwissenheit in einen Schwall gelehrter Modeworte so zu verhüllen versteht, daß die gemeinsten Gedanken durch das Helldunkel seines Ausdruckes wie tiefe Weisheit erscheinen, dessen Name wird gefeiert. Deutsche! wann werdet ihr von einer Verirrung, welche euch eueren Nachbarn nur ungenießbar und lächerlich macht, endlich zurückkehren?} \par 
In dem nun folgenden achten Hauptstücke (S.\,597--620) beschreibt uns der zum Schlusse eilende Vf.\ das Verhalten, das man bei Abfassung eines Lehrbuches annehmen muß, um eben die in dem Vorhergehenden geforderten Beschaffenheiten demselben geben zu können. Obwohl er nun auch hier, wie oben in der \danf{Erfindungskunst,} nichts vorbringt, was nicht von guten Schriftstellern längst schon beobachtet worden wäre: so dürfte doch die ausdrückliche Aufstellung dieser Vorschriften um so weniger etwas Entbehrliches seyn, je mehr es allgemein Noth thut, aus Regeln, die wir aus Trägheit oder Leidenschaft nicht gern beobachten, vernehmlich zuzurufen. Gewiß können wir uns gleich die erste von B.\ hier aufgestellte Wahrheit, \danf{daß sittliche Gesinnung auch bei der glücklichsten Bearbeitung einer jeden Wissenschaft förderlich sey,} -- nicht oft genug vorhalten. Sehr nothwendig ist, zumal in unserer Zeit, auch die Einschärfung der Pflicht, sich mit den Leistungen der Vorgänger genau bekannt zu machen. Ein Gleiches gilt von der Sorgfalt, die auch der sprachlichen Darstellung in einem Lehrbuche gewidmet werden soll. In Betreff der doppelten Frage endlich: wann man der Arbeit ein Ziel setzen, und wann man sein Buch dem Publico vorlegen dürfe? -- (denn B.\ bemerkt, daß dieses zweierlei sey, \seitenw{117} und daß man zuweilen das Letztere füglich thun dürfe, ohne das Erstere noch gethan zu haben) -- wird vielleicht Jeder finden, daß der Vf.\ sich weder zu strenge noch zu nachgiebig erweise. Als die gewöhnlichsten Fehler bei der Abfassung eines Lehrbuches werden \WLpar{IV}{711} bezeichnet die Selbstgefälligkeit, die Übereilung, die fehlerhafte Nachahmung berühmter Vorgänger, das zu Gefallen reden oder die einer gewissen Partei erwiesene Huldigung, das eitle Streben nach Originalität und endlich der Hang zu unfruchtbaren Speculationen oder der Mangel an praktischem Interesse. \danf{Was kann betrübender seyn für die Menschheit,} ruft der Vf.\ bei Nr. 4 aus, \danf{als wenn Schriftsteller, nicht nur solche, die für den bloßen Zweck der Unterhaltung zu sorgen beauftraget sind, sondern selbst jene, denen die Menschheit ihre heiligsten Angelegenheiten, die Bearbeitung ihrer Wissenschaften, anvertraut hat, keine redliche Liebe zur Wahrheit besitzen, sondern aus schnöder Gefallsucht, nur um den Beifall einer gewissen Partei zu gewinnen} (er hätte beisetzen können, um ein paar Thaler Geldes, eine goldene Dose, ein Titelchen \udgl\ zu erhaschen) \danf{sich selbst sowohl als ihre Leser bethören! Und dennoch wie klein mag nicht unter den Schriftstellern, welche die Fächer der Theologie, Politik, Geschichte bearbeiten, die Zahl derjenigen seyn, die einen solchen Vorwurf in keiner Hinsicht verdienen!} \par 
Wer nun das Werk -- sicher nicht ohne öftere Ermüdung, die nicht nur der große Umfang desselben, sondern noch mehr das viele Neue darin verursachen mußte -- bis hieher durchgelesen hat, der wird aus freien Stücken wohl auch noch die wenigen Blätter des neunten Hauptstückes: \danf{von solchen wissenschaftlichen Büchern, die keine eigentliche Lehrbücher sind} (S.\,621--636), nicht ungelesen lassen. Der Vf.\ spricht hier 1) von Abhandlungen, unter denen er abermal Monographien, Beiträge, Repertorien, Streitschriften und Recensionen unterscheidet; 2) von Hülfsbüchern, die entweder so eingerichtet sind, daß sie vor, oder während, oder nach einem \seitenw{118} mündlichen Unterrichte benützt werden sollen; 3) von Handbüchern, die zum gelegenheitlichen Nachschlagen tauglich seyn sollen, dergleichen die Wörterbücher; endlich 4) von wissenschaftlichen Unterhaltungsbüchern, bei denen der Zweck der Unterhaltung vorwaltet. Was nun über die Einrichtung einer jeden dieser besonderen Arten von Schriften angedeutet wird, ist so gedrängt und kurz, daß es wenn sonst keinem anderen Vorwurfe mindestens dem der Unvollständigkeit mit Recht entgegensieht. \par 
\endinput

