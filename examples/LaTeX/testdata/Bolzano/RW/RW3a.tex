\RWteil[\RWchselbeseite{Erster Band.\\ Die Lehre von den Erkenntnißquellen der göttlichen Offenbarung, von Gott, von der Welt und von den Engeln.}]{III}{Dritter [Haupt-]Theil.}{Systematische Darstellung der Lehre des Katholicismus nach ihrer inneren Vortrefflichkeit.}{\RWSeitenwohne{1}}
\clearpage\linenumbers%
\def\dieserteilseiten{IIIa}
\RWch[Einleitung.]{Einleitung.\RWSeitenwohne{3}}
\RWpar{1}{Inhalt und Wichtigkeit dieses Theiles}
\begin{aufza}
\item Nachdem wir in dem vorhergehenden Haupttheile gefunden, daß die katholisch-christliche Religion allerdings viele außerordentliche Begebenheiten aufzuweisen habe, welche in einer so innigen Verbindung mit ihr stehen, daß man, wenn anders ihre Lehre die Eigenschaft sittlicher Zuträglichkeit hat, nicht anstehen kann, diese Ereignisse für Wunder, die Gott zu ihrer Bestätigung veranstaltet hat, zu erklären: so bleibt uns nur noch zu untersuchen übrig, \RWbet{ob diese sittliche Zuträglichkeit der Lehre wirklich vorhanden sey}, und zwar in einem solchen Grade, daß keine andere Religion, die gleichfalls Wunder aufzuweisen hätte, sich eines gleichen Grades von Vollkommenheit zu rühmen vermag. Dieß soll nun eben in diesem dritten und letzten Haupttheile der Religionswissenschaft geleistet werden.
\item Hieraus erhellet von selbst die hohe Wichtigkeit dieses Haupttheiles; denn
\begin{aufzb}
\item selbst wenn es möglich wäre, sich von der Wahrheit, daß eine gewisse Religion Gottes Offenbarung an uns sey, ohne die Prüfung ihrer Lehre, bloß aus gewissen äußeren Kennzeichen derselben (den Wundern nämlich) zu überzeugen: so dürfte gleichwohl dieser dritte Theil der Religionswissenschaft nicht fehlen. Sobald man sich~\RWSeitenw{4}\ nämlich von der Wahrheit, daß die katholisch-christliche Religion eine wahre Offenbarung sey, überzeugt hätte: entstünde von nun an die Pflicht, sich \Ahat{die}{in die} \RWbet{genaueste Kenntniß der Lehren dieser Religion} und des wohlthätigen Einflußes derselben auf unsere Tugend und Glückseligkeit zu verschaffen; indem wir nur, wenn dieß geschehen ist, diese Religion zu unserem eigenen Nutzen gehörig anwenden könnten (1ster Hptth. 1stes Hptst. \RWparnr{38}). Auch jetzt müßte also eine vollständige Darstellung der Lehre des katholischen Christenthumes nach ihrer inneren Vortrefflichkeit folgen.
\item Um wie viel nothwendiger wird eine solche Darstellung seyn, wenn man sich ohne sie (wie wir oben gesehen) nicht einmal vernünftig überzeugen kann, daß diese Religion eine wahre göttliche Offenbarung an uns sey.
\item Und diese Nothwendigkeit ist um so dringender in unseren Tagen, je \Abweichung{frecher}{früher}{frecher~\Blabel} und mit je scheinbareren Gründen von Seite der Gegner des Katholicismus behauptet worden ist, daß diese Religion eine Menge unnützer, ja sogar widersinniger Lehrmeinungen enthalte.
\end{aufzb}
\end{aufza}


\RWpar{2}{Plan und Ordnung des Vortrags in diesem Theile}
\begin{aufza}
\item Da der hier zu ertheilende Unterricht in der vollkommensten Religion nicht ein gemeiner, sondern ein \RWbet{wissenschaftlicher} seyn soll: so wird es sich geziemen, auch bei dem Vortrage der einzelnen Lehren des Katholicismus nicht ohne Plan, sondern nach einer wissenschaftlichen Anordnung zu Werke zu gehen. Wie nun die Lehren jeder Religion bequem unter die zwei Hauptabtheilungen einer \RWbet{theoretischen} und einer \RWbet{praktischen} (Dogmatik und Moral) gebracht werden können: so werde ich mich auch hier an diese Eintheilung halten. Doch wird vor Allem die Lehre des Katholicismus \RWbet{von den Erkenntnißquellen seiner Lehren} vorausgeschickt werden müssen. Aus dieser nämlich erfahren wir
\begin{aufzb}
\item wofür sich der Katholicismus selbst ausgebe; ob er sich für eine göttliche Offenbarung erkläre oder nicht. Nach~\RWSeitenw{5}\ diesem Umstande aber müssen wir darum zu allererst fragen, weil eine Religion, die sich nicht selbst für geoffenbart ausgäbe, wohl eben darum auch nicht geoffenbart seyn könnte.
\item Nebstdem erhalten wir durch diese Lehre gewisse Kennzeichen, vermittelst deren wir leichter als sonst werden beurtheilen können, \RWbet{was Alles zur Lehre des Katholicismus gehöre oder nicht}.
\item Endlich werden wir durch die vorläufige Kenntniß dieser Lehre auch in den Stand gesetzt, besser zu beurtheilen, ob sich nicht irgend ein Widerspruch in den Lehren des Katholicismus finde.
\end{aufzb}
\item Weisen wir also dieser Lehre von den Erkenntnißquellen, um ihrer Wichtigkeit willen, ein eigenes Hauptstück an: so würde dieser dritte Haupttheil der Religionswissenschaft in folgende drei Hauptstücke zerfallen:
\begin{aufzb}
\item[\RWbet{Erstes Hauptstück:}] die Lehre des Katholicismus von den Erkenntnißquellen seiner Lehren.
\item[\RWbet{Zweites Hauptstück:}] Katholisch-christliche Dogmatik.
\item[\RWbet{Drittes Hauptstück:}] Katholisch-christliche Moral.
\end{aufzb}
\item Bei jeder einzelnen Lehre, welche ich aufstelle, werde ich aber drei bis vier Stücke aufzuweisen haben.
\begin{aufzb}
\item Ich werde zuerst darzuthun haben, daß diese Lehre wirklich zu den Lehren des Katholicismus gehöre. Hiezu wird Zweierlei erfordert:
\begin{aufzc}
\item daß diese Lehre eine \RWbet{Religionslehre} sey, \dh\ einen Einfluß auf Tugend und Glückseligkeit habe, und zugleich \RWbet{sittlicher} Art sey; sodann
\item daß sich zu dieser Lehre alle, oder doch fast alle (jetzt lebende) katholische Christen, für welche sie religiöse Wichtigkeit hat, bekennen.
\end{aufzc}
Da sich das Erstere aus einer der folgenden Untersuchungen immer von selbst ergibt: so werde ich es hier füglich weglassen können, und folglich nur das Zweite allein~\RWSeitenw{6}\ zu zeigen haben. Dieß erste Geschäft in Ansehung jeder Lehre nenne ich nun den historischen Beweis derselben; ein Ausdruck, worunter ich also nicht etwa den Beweis der Wahrheit dieser Lehre, sondern nur einen Beweis, daß sie eine Lehre des Katholicismus sey, verstehe. Bevor wir noch wissen, was für Erkenntnißquellen seiner Lehre der Katholicismus selbst angibt, läßt sich der oben erwähnte Beweis auf keine andere Art führen, als durch die allgemeinen Mittel, die bei jeder Volksreligion anwendbar sind, um ihren Inhalt zu erfahren. Wir müssen nämlich sehen, was in Betreff des Punctes, den wir ausmitteln wollen,
\begin{aufzc}
\item die \RWbet{öffentlichen Lehrbücher} sagen, deren man sich in den verschiedenen Ländern, die dieser Religion zugethan sind, zum Unterrichte des Volkes und seiner Geistlichkeit bedient. Denn es läßt sich im Voraus erwarten, daß der größte Theil des Volkes sowohl, als auch der Geistlichkeit diejenigen Grundsätze werde angenommen haben, die man demselben in dem öffentlichen Unterrichte beibringt. Eben so merkwürdig sind in dieser Rücksicht
\item die \RWbet{Gebetbücher}, deren sich das Volk, dessen Religion wir prüfen wollen, bedient; ingleichen
\item \RWbet{Predigten} und andere Schriften zur Erbauung, die mit Genehmigung der Obern herausgegeben, und von dem Volke mit Beifall aufgenommen worden sind; endlich
\item mündliche Aeußerungen des Volkes, \usw
\end{aufzc}
\item Ist es bewiesen, daß eine Lehre wirklich Lehre des Katholicismus sey: so ist nun ihre \RWbet{sittliche Zuträglichkeit} zu prüfen. Da aber eine Lehre, die erweislich falsch ist, schon eben darum auch keine sittliche Zuträglichkeit haben könnte, weil es unmöglich wäre, an ihre Wahrheit zu glauben; und da von mehreren Lehren des Katholicismus wirklich behauptet worden ist, daß sie erweislich falsch wären: so wird es mein zweites Geschäft seyn müssen, bei jeder aufgestellten Lehre zu zeigen, \RWbet{daß sie zum Wenigsten nicht erweislich falsch sey}. Da nun die Falschheit einer Behauptung~\RWSeitenw{7}\ nur dadurch dargethan werden kann, daß man zeigt, daß sie mit irgend einer als wahr erwiesenen, oder mit ihr zugleich als wahr vorauszusetzenden Behauptung in Widerspruch steht: so werde ich dargethan haben, daß eine Lehre nicht erweislich falsch sey, wenn ich gezeigt, daß sie
\begin{aufzc}
\item mit keiner bloßen Begriffs-Wahrheit (oder mit dem, was man Vernunft in engerer Bedeutung nennt); auch
\item mit keiner Erfahrung (oder mit dem, was man Geschichte nennt); und endlich
\item auch mit keiner anderen Lehre des Katholicismus in Widerspruch stehe.
\end{aufzc}
Doch werde ich, wie sich von selbst versteht, diese Vergleichung nur mit solchen Wahrheiten und Lehren aufzustellen brauchen, bei welchen es einigen Anschein hat, daß sie mit der zu prüfenden Lehre in einem Widerspruche stehen. Den Paragraphen, in welchen ich diese Untersuchung vornehme, will ich die Ueberschrift: \RWbet{Vernunftmäßigkeit dieser Lehre}, geben; ein Ausdruck, den also Niemand so deuten müsse, als ob ich mich in solchen Paragraphen anheischig machte, die Wahrheit der behandelten Lehren \RWbet{aus der Vernunft selbst} herzuleiten. Das habe ich keineswegs nöthig, ob ich es gleich in einzelnen Fällen, wo die Vernunft es vermag, versuchen werde.
\item Das \RWbet{dritte} Geschäft wird nun seyn, daß ich den wohl\-thä\-ti\-gen Einfluß darstelle, den die gläubige Annahme der betrachteten Lehre auf unsere Tugend und Glückseligkeit verspricht, und hiebei zeige, daß keine andere Ansicht der Dinge ersprießlicher für uns seyn könnte; oder wenigstens, daß sich in keiner von denjenigen Religionen, die Wunder aufzuweisen haben, eine ersprießlichere Ansicht finde. Diese Untersuchung werde ich unter der Ueberschrift: \RWbet{sittliche Zuträglichkeit} oder \RWbet{sittlicher Nutzen der Lehre}, vortragen.
\item Endlich wird es nicht überflüßig seyn, wenn nicht bei jeder, doch bei einigen Lehren des Katholicismus, zu untersuchen, was für einen Einfluß ihre Bekanntwerdung~\RWSeitenw{8}\ auf die Menschheit bisher \RWbet{wirklich gehabt habe}. Diese Erörterung, welche ich unter der Ueberschrift: \RWbet{wirklicher Nutzen dieser Lehre}, anstellen werde, ist nämlich zu folgenden drei Zwecken dienlich:
\begin{aufzc}
\item Wenn wir gesehen, daß eine Lehre bisher bei allen Menschen, von denen sie nur gehörig verstanden worden ist, heilsame Wirkungen hervorgebracht hat: so werden wir hiedurch um so gewisser überzeugt, daß diese Lehre auch bei uns, wenn wir sie gläubig annehmen werden, heilsame Wirkungen erzeugen werde. Mit anderen Worten, der wirkliche Nutzen einer Lehre (den sie bisher bei Anderen gehabt hat) macht uns den \RWbet{möglichen} Nutzen derselben (den sie bei uns haben soll) im Voraus anschaulich.
\item Nach jenem Nutzen oder Schaden, den eine Lehre des Katholicismus bisher hervorgebracht hat, läßt sich beurtheilen, was auch in Zukunft von ihr zu hoffen oder zu fürchten seyn werde, und was für Pflichten wir also in Ansehung ihrer Verbreitung bei unseren Nebenmenschen haben.
\item Da es von mancher Lehre des katholischen Christenthums scheint, daß sie wenigstens in gewissen Jahrhunderten und bei gewissen Völkern durch Mißbrauch oder Mißverstand mehr Schaden als Nutzen gestiftet habe: so entsteht die Frage, warum die göttliche Vorsehung gleichwohl die Entstehung und Verbreitung dieser Lehre in solchen Zeiten und bei solchen Völkern zugelassen habe?
Ob wir uns nun, auch ohne diese Frage zu beantworten, von der sittlichen Zuträglichkeit einer solchen Lehre \RWbet{für uns}, und folglich auch von der Wahrheit, daß sie für uns geoffenbaret sey, vollkommen überzeugen können: so muß uns die Beantwortung jener Frage doch in einer anderen Rücksicht überaus willkommen seyn; nämlich um uns in unserem Glauben an eine göttliche Vorsehung, die über den Schicksalen des menschlichen Geschlechtes waltet, zu befestigen. Denn untersuchen wir den Nutzen sowohl als auch den Schaden, den eine gewisse Lehre gestiftet hat, genauer: so zeigt sich fast immer recht sichtbar, daß jener diesen bei Weitem überwiege, und so wird unser Glaube an Gottes Vorsehung bestärket.
\end{aufzc}
\end{aufzb}
\item Durch das Geschäft, welches ich vorhin (3.~a.) den historischen Beweis einer Lehre nannte, wird zwar Jeder versichert werden, daß die Lehren, die ich in Zukunft vortrage, wirkliche Lehren des Katholicismus sind; aber es bliebe doch die Frage übrig: ob es nebst denjenigen Lehren, die ich hier anführe, nicht vielleicht noch manche andere Meinungen im Katholicismus gebe, die ich verschwiegen habe, und die des Gepräges der sittlichen Vollkommenheit ermangeln. Diese Bedenklichkeit zu heben, kann ich nichts Anderes thun, als Jeden, der sie hegt, auf die vorhin erwähnten allgemeinen Mittel, wie man den Inhalt einer Volksreligion erfahren kann, zu verweisen. Gewiß wird er keine von jenen verderblichen und ungereimten Behauptungen, die man dem Katholicismus aufbürden will, weil sie von Einigen seiner Bekenner, ja wäre es selbst von einem großen Theile, angenommen worden sind, \RWbet{allgemein} verbreitet finden; gewiß wird er sie nicht in jenen \RWbet{öffentlichen Lehrbüchern}, deren die Katholiken bei ihrem Unterrichte sich bedienen, aufgestellt lesen; und schon hieraus allein wird er schließen können, daß sie keineswegs als ein Bestandtheil des Katholicismus anzusehen seyen.
\end{aufza}


\RWch[Erstes Hauptstück.\\ Die Lehre des Katholicismus von den Erkenntnißquellen seiner Lehren.]{Erstes Hauptstück.\RWSeitenwohne{10}\\ Die Lehre des Katholicismus von den Erkenntnißquellen seiner Lehren.}
\RWpar{3}{Darstellung dieser Lehre}
\begin{aufza}
\item Wenn man die Katholiken fragt, nach welcher Regel sie sich in der Annahme ihrer religiösen Meinungen zu richten hätten: so ist ihre Antwort nicht durchgängig gleichlautend. Der größte Theil der Ungebildeten glaubt, er habe in diesem Stücke nichts Anderes zu thun, als auf den \RWbet{Unterricht seiner geistlichen Lehrer} zu merken, und was ihm diese sagen, anzunehmen. Unter den Lehrern selbst aber, und überhaupt unter dem gebildeteren Theile der Katholiken behaupten die Meisten, man sey zur Annahme Alles dessen verbunden, \RWbet{was das Oberhaupt der katholischen Kirche} (nämlich der Bischof zu Rom oder der Papst) zu glauben aufstellt. Andere dagegen sagen, man wäre nur das zu glauben schuldig, was in einer \RWbet{allgemeinen Versammlung der Lehrer} (\dh\ der Bischöfe in einem \RWbet{ökumenischen Concilio}) entschieden worden ist. Einige endlich meinen, daß man selbst einem allgemeinen Kirchenrathe nicht zu glauben schuldig wäre, wenn sich nicht nachweisen ließe, daß die Entscheidung desselben mit dem, was zu \RWbet{allen Zeiten}, \RWbet{an allen Orten}, und \RWbet{von Allen} geglaubt worden ist, übereinstimmt.
\item Fragt man ferner, ob der Katholik wohl eine jede seiner religiösen Meinungen, die er nach einer von den so~\RWSeitenw{11}\ eben beschriebenen Regeln annimmt, als eine ihm von Gott selbst zu Theil gewordene \RWbet{Offenbarung} ansehen dürfe: so ist die Antwort, daß er dieß freilich nicht von allen seinen Meinungen, namentlich nicht von denjenigen voraussetzen könne, in denen er nicht so glücklich ist, eine Uebereinstimmung bei seinen sämmtlichen Glaubensgenossen zu finden; wohl aber müsse er alle diejenigen religiösen Meinungen, in denen auch \RWbet{alle übrigen Katholiken}, für die sie von Wichtigkeit sind, gleichförmig mit ihm denken, als von Gott selbst geoffenbart anerkennen. Kurz \RWbet{der Gesammtglaube der Katholiken ist eine eigentliche göttliche Offenbarung}, und eben deßhalb auch \RWbet{unfehlbare Wahrheit}.
\item Fragt man weiter, ob dieser Vorzug einer Offenbarung, der dem Gesammtglauben der Katholiken zukommt, ihm erst vor Kurzem, oder schon seit geraumer Zeit, zu Theil geworden sey, und wie lange er fortdauern werde: so ist die Antwort, die christliche Kirche, \dh\ diejenige religiöse Gesellschaft, die Jesus Christus gestiftet hat, erfreue sich dieses Vorzuges \RWbet{seit ihrer Stiftung}, und werde auch \RWbet{bis an das Ende der Menschheit} in seinem Besitze bleiben. Als aber im Verlaufe der Zeit von ihr, die späterhin den Namen der katholischen annahm, einzelne Theile sich trennten, sey jener Vorzug auf diese nicht übergegangen. Die akatholischen Parteien also, \zB\ die evangelische, die reformirte, die socinianische, wenn sie gleich alle sich christliche Kirchen nennen, dürfen sich dennoch nicht rühmen, daß der Gesammtglaube ihrer Glieder eine göttliche Offenbarung sey.
\item Auf die Frage, woher die Katholiken es wüßten, daß ihr Gesammtglaube eine göttliche Offenbarung sey? wird die Antwort ertheilt, daß \RWbet{Jesus selbst} der Kirche diesen Vorzug der Unfehlbarkeit versprochen habe, indem er ihren Lehrern den \RWbet{Beistand des heiligen Geistes bis in die spätesten Zeiten} verheißen.
\item Fragen wir ferner, ob dieser Beistand der Kirche \RWbet{unbedingt} versprochen sey, so daß sie denselben \RWbet{nie} verscherze, wie wenig sich auch vielleicht einzelne, oder auch alle ihre Lehrer desselben würdig betragen: so ist die Antwort bejahend.~\RWSeitenw{12}
\item Fragt man nach dem \RWbet{Objecte} der kirchlichen Unfehlbarkeit, \dh\ worauf sich eigentlich der Beistand des göttlichen Geistes erstrecke; ob etwa \RWbet{alle Meinungen}, in denen die Katholiken übereinstimmen, als göttliche Offenbarungen zu betrachten seyen: so wird erwiedert, daß dieser Beistand Gottes sich nur auf \RWbet{religiöse} Meinungen erstrecke. Die Gelehrten schließen aus dem Gebiete der kirchlichen Unfehlbarkeit ausdrücklich aus:
\begin{aufzb}
\item \RWbet{Disciplinarvorschriften}, \dh\ Verordnungen, von welchen die Kirche selbst gesteht, daß sie nicht von dem Stifter, sondern bloß von den Vorstehern herrühren; was so viel sagen will, als daß diese Vorschriften nicht geoffenbart wären.
\item Meinungen, die bloß \RWbet{historisch} sind, und nicht etwa zum Inhalte der biblischen Geschichte selbst gehören.
\item Meinungen, die ein bloß \RWbet{wissenschaftliches} (philosophisches) Interesse haben.
\end{aufzb}
\item Fragen wir, ob die katholische Kirche zu Folge jenes ihr niemals fehlenden Beistandes des heiligen Geistes ihren Lehrbegriff auch nach den Bedürfnissen und der Empfänglichkeit der Zeiten \RWbet{ändern}, \RWbet{ausbilden} und \RWbet{vervollkommnen} könne; ob sie in späteren Jahrhunderten nicht vielleicht einige Lehrsätze aufgestellt habe, welche die früheren Jahrhunderte nicht kannten: so wird dieß von dem größten Theile der Katholiken (besonders von den Ungebildeten) geradezu \RWbet{verneint}. Wir glauben noch heute (sagen sie) nichts Anderes, als was man schon zu Jesu und der Apostel Zeiten geglaubt hat. -- Mehrere aus den Gelehrten aber geben zu, daß der Lehrbegriff der Katholischen Kirche eines gewissen Zuwachses, einer gewissen Ausbildung und Vervollkommnung fähig sey; denn ob sie gleich nicht die Macht hat, \RWbet{ganz neue} und den Aposteln unbekannte Lehrsätze aufzustellen, auch ihre Meinungen nicht ändert: so könne sie doch Dasjenige, was man in früherer Zeit \RWlat{implicite} geglaubt, \dh\ stillschweigend angenommen, ohne sich darüber deutlich auszusprechen, später, wenn das Bedürfniß dazu eintritt, \RWlat{explicite}, \dh\ deutlich und ausdrücklich lehren.~\RWSeitenw{13}
\item Fragen wir endlich, ob es nicht etwa gewisse Bücher und schriftliche Aufsätze gebe, in welchen der ganze Glaube der Katholiken vollständig aufgezeichnet wäre: so sehen wir daraus, weil uns der Eine auf dieß, der Andere auf jenes Buch verweiset, der Dritte gar keines angeben will, es sey kein Lehrsatz des Katholicismus, daß irgend ein solches Buch (eine ganz vollständige, geschriebene Erkenntnißquelle des Katholicismus) wirklich vorhanden sey. Wohl gibt es dagegen mehrere Bücher, von denen alle Katholiken mit Uebereinstimmung behaupten, daß sie \RWbet{nichts ihrer Lehre Widersprechendes} enthalten, und daß man einen großen Theil der letztern aus ihnen schöpfen könne. Als solche Bücher, welche man eben deßhalb die mittelbaren Erkenntnißquellen des Katholicismus nennen könnte, werden uns angegeben:
\begin{aufzb}
\item die \RWbet{heilige Schrift} sowohl des alten, als des neuen Bundes, von welcher der Katholik behauptet, daß sie
\begin{aufzc}
\item durch Gottes Eingebung geschrieben worden sey, und daß Gott den Verfassern derselben nicht nur die Sachen oder Gedanken, sondern auch selbst die \RWbet{Worte} eingegeben habe; so zwar, daß
\item Alles, was in Betreff der Religion in diesen Büchern gelehrt wird, \RWbet{wahre göttliche Offenbarung} sey; jedoch nur dann, wenn Alles nach dem Sinne aufgefaßt wird, in dem die \RWbet{Kirche} es auslegt.
\item Zu diesem Zwecke preiset man als besonders brauchbar die bei ihr übliche lateinische Uebersetzung dieser Bücher, welche \RWbet{Vulgata} heißt.
\end{aufzc}
\item Die Verhandlungen der \RWbet{allgemeinen Kirchenräthe}, deren die Katholiken eigentlich folgende neunzehn zählen:
\begin{leercompactenum}
\item\Anotelabel{19konz}\pause{19konz}<1. \auslass\ 19.> Den ersten Nicäischen (im Jahre 325);
\item den ersten Constantinopolitanischen (i.\,J. 381);
\item den Ephesinischen (i.\,J. 431);
\item den Chalcedonischen (i.\,J. 451);
\item den zweiten Constantinopolitanischen (i.\,J. 553);
\item den dritten Constantinopolitanischen (i.\,J. 680);
\item den zweiten Nicäischen (i.\,J. 787);
\item den vierten Constantinopolitanischen (i.\,J. 869);
\item den ersten Lateranensischen (i.\,J. 1123);~\RWSeitenw{14}
\item den zweiten Lateranensischen (i.\,J. 1139);
\item den dritten Lateranensischen (i.\,J. 1179);
\item den vierten Lateranensischen (i.\,J. 1213);
\item den ersten Lyoner (i.\,J. 1245);
\item den zweiten Lyoner (i.\,J. 1274);
\item den zu Vienne in Frankreich (i.\,J. 1311);
\item den Kostnitzer (i.\,J. 1414); 
\item den Basler (i.\,J. 1421); 
\item den Florentiner (i.\,J. 1438); 
\item den Tridentiner (i.\,J. 1545).\resume{19konz}\donote{19konz}{in \Alabel\ ohne Nummerierung}
\end{leercompactenum}
\end{aufzb}
\end{aufza}

\RWpar{4}{Historischer Beweis dieser Lehre}
\begin{aufza}
\item Daß sich das Volk an den Unterricht seiner geistlichen Lehrer zu halten habe, liest man in allen Katechismen, Predigten, \usw\ -- Die Lehrer bekennen selbst, auch heut zu Tage noch mit einer großen Allgemeinheit, daß dem Oberhaupte der Kirche, dem Papste, die Macht zustehe, in streitigen Fällen zu entscheiden, was man zu glauben habe. Sie nennen ihn deßhalb \RWlat{doctorem ecclesiae}. Ja, in älteren Zeiten gingen Verschiedene so weit, zu behaupten, daß der Papst in solchen Entscheidungen eine wirkliche \RWbet{Unfehlbarkeit} besitze. Dieses behaupteten \zB\ im 16ten Jahrhunderte gegen Luther der \RWbet{Cardinal Cajetan}, \RWbet{D.~Joh.~Eck}, \RWbet{P.~Didier}, \RWbet{Card.\ Orfius}, \RWbet{Card.\ Bellarmin}, \uA\ -- Die Unfehlbarkeit der allgemeinen Kirchenräthe dagegen wird auch noch jetzt von dem größten Theile der katholischen Schriftsteller behauptet und vertheidiget. Siehe \zB\ \RWbet{Klüpfel} (\RWlat{P.~I.~Prolegom.} \RWparnr{76}\RWlit{}{Kluepfel1}) und viele Andere. -- Einige wollen jedoch, daß man erst nachsehen müsse, ob, was die Bischöfe in dem Concilio entschieden, auch mit dem ältesten Glauben der Kirche übereinstimme; wie Ruh, Simon, Launoy, Courayer, der ungenannte Verfasser des Buches: kritische Geschichte der kirchlichen Unfehlbarkeit zur Beförderung einer \Ahat{freyern}{strengeren} Prüfung des Katholicismus. Frankf.~a.~M. 1791\RWlit{}{Blau1} \uA
\item Daß jene religiösen Meinungen der Katholiken, in welchen sie Alle gleichförmig denken, zum Inhalte der wahren~\RWSeitenw{15}\ Offenbarung gehören, findet man in allen Lehrbüchern und Katechismen, die in unseren Tagen geschrieben worden sind. Man sehe \zB\ \RWlat{Juenin[us]} (\RWlat{Instit. Theol.\ T.\,I.\ Diss.\,4.\ qu.\,2.\ c.\,5. art.\,3.}\RWlit{}{Juenin1}), Engelbert Klüpfel (\RWlat{Institutiones theologiae dogmaticae. Pars I. Proleg.} \RWparnr{73}\RWlit{}{Kluepfel1}), J.~M.~Sailer's (Grundlehren der Religion, 1ste Aufl.\ \RWparnr{79}\RWlit{}{Sailer1b}), Ildeph. Schwarz (Handbuch der christlichen Religion. 1.\,B.\ VI.\RWlit{}{Schwarz1}), \RWlat{Fleury} (\RWlat{Catechisme historique. Tom.\,2. Lec. 47.}\RWlit{}{Fleury1}), Beda Mayr's (Vertheidigung der natürlichen, christlichen und katholischen Religion, 3ter Th. 5ter Abschn.\RWlit{}{Mayr1}), \uA

\begin{RWanm} Der heil.\ \RWbet{Antonin}, Erzbischof zu Florenz, einer der vornehmsten Lehrer seiner Zeit, behauptete \RWlat{(Summa doctr. part.\,3. tit.\,23. c.\,2. \RWparnr{6}\RWlit{}{Antonius1}): Possibile est, quod tota fides remaneat in uno solo. [\textsymmdots ] Et hoc patuit post passionem Christi, ubi remansit in sola Virgine, quia omnes alii scandalizati sunt. [...] Ergo non dicitur ecclesia deficere aut errare, si remaneat fides in uno solo.} Dieß scheint zwar auf den ersten Blick der Behauptung, daß der Gesammtglaube Aller immer unfehlbar sey, zu widersprechen; weil ja schon dann gesagt werden kann, daß Alle einer gewissen Meinung zugethan sind, wenn nur fast Alle ihr zugethan sind; welches der Fall ist, wenn ihr nicht Mehrere, als nur ein Einziger entgegen ist. Allein man muß erwägen, daß in dem Falle, den der heil.\ Antonin anführt, die Anzahl der Gläubigen noch so gering war, daß der standhafte Widerspruch auch nur einer einzigen Person schon hinreichend seyn konnte, um die Meinung, in der die Uebrigen übereinstimmten, nicht für den Gesammtglauben Aller zu erklären.
\end{RWanm}
\item und 4. Man schlage was immer für einen katholischen Katechismus nach, \zB\ den römischen, oder den österreichischen, oder den französischen \uA: so wird man diese Lehre ausdrücklich aufgestellt finden.
\stepcounter{enumi}
\item Daß der Vorzug der Unfehlbarkeit den Lehrern der Kirche \RWbet{unbedingt} zukomme, weil er denselben nicht um ihretwillen, sondern der gesammten Kirche wegen zu Theil geworden ist, lehren ausdrücklich Klüpfel (\RWlat{l.\,c. P.\,II. L.\,II.} \RWparnr{118}\RWlit{}{Kluepfel1}), Ildeph. Schwarz (a. a. O. 1.\,B.\,\RWlat{VI.}\,17.\RWlit{}{Schwarz1}), u.\,A. Doch gibt es Einige, die das Gegentheil, aber nur in Betreff der Lehrer, behaupten, \zB\ \RWlat{Muratori} (\RWlat{de modera\RWSeitenw{16}mine ingeniorum l.\,1. c.\,14.}): \RWlat{Ultro fatemur, Christi opem, quam catholici conciliis tribuunt, ut doctrinam Christi in suis definitionibus servent, esse hypotheticam., c.\,19.: Tunc solum se servandos ab omni erroris periculo immunes ecclesiae pastores in proponenda coelesti doctrina sperare et credere possunt, quum debitam fidem et diligentiam ipsi adhibebunt etc.}
\item Niemand glaubt, daß sich der Beistand des göttlichen Geistes auch auf solche Meinungen erstrecke, die für den Zweck der Tugend und Glückseligkeit ganz gleichgültig sind. -- Daß die Unfehlbarkeit der kirchlichen Entscheidungen sich insbesondere nicht auf Disciplinarvorschriften, auf bloß historische und philosophische Fragen erstrecke \usw\ -- lehren ausdrücklich \RWlat{Klüpfel} (\RWlat{l.\,c.\ \RWparnr{78}\ Not.\,2.}\RWlit{}{Kluepfel1}), Beda Mayr (a.\,a.\,O. \RWparnr{60}.\ \RWparnr{66}\RWlit{}{Mayr1}), \umA
\item In allen Katechismen heißt es, daß die katholische Kirche sich eben deßhalb auch die \RWbet{apostolische} nenne, um anzudeuten, daß sie noch eben das glaube und lehre, was die Apostel einst geglaubet und gelehret haben. -- Unter den Gelehrten aber haben Mehrere ein gewisses Fortschreiten des Katholicismus angenommen. Hieher gehören \zB\ der schon mehrmals erwähnte B.~Mayr (\aaO), Sailer (\aaO\ am Schlusse der 32sten Vorlesung), Mich. Feder und Klüpfel (in ihren Herausgaben des \RWlat{Common. Vinc. Lirin.} in den Noten) \umA\ Besonders merkwürdig aber ist die so alte Erklärung des \RWbet{heil.\ Vincenz von Lirin} aus dem 5ten Jahrhunderte in seinem \RWlat{Commonitorio}\RWlit{}{Vincentius1}, einem Büchelchen, das noch heut zu Tage von den Katholiken sehr geschätzt wird. Dieser stellte zuerst (\RWlat{cap.}\,\Ahat{3}{4}) die strenge (unter den katholischen Schriftstellern neuerer Zeit so beliebt gewordene) Regel auf: \erganf{\RWlat{In [ipsa item] catholica ecclesia magnopere curandum est, ut id teneamus, quod \RWbet{ubique}, quod \RWbet{semper}, quod \RWbet{ab omnibus} creditum est. Hoc est [et]enim vere proprieque catholicum, quod ipsa vis nominis, ratioque declarat [, quae omnia fere universaliter comprehendit]. Sed hoc ita demum fiet: si sequamur \RWbet{universitatem}, \RWbet{antiquitatem}, \RWbet{consensionem}. Sequimur autem 1.~universitatem~\RWSeitenw{17}\ hoc modo, si hanc unam fidem veram esse fateamur, quam tota per orbem terrarum confitetur ecclesia; 2.~antiquitatem vero ita, si ab his sensibus nullatenus recedamus, quos sanctos majores ac patres nostros celebrasse manifestum est; (.~Consensionem quoque itidem; si in ipsa vetustate, \RWbet{omnium} [vel] certe \RWbet{pene omnium} sacerdotum pariter et magistrorum definitiones sententiasque sectemur.}} Hierauf macht er sich (\RWlat{c.}\,23) selbst den Einwurf: \erganf{\RWlat{Sed \var{forsitan}{{forsitans}{}{}} dicit aliquis: Nullusne ergo in Ecclesia Christi \RWbet{profectus}?} (Fortschreiten) -- \RWlat{Habeatur plane et maximus. Nam quis ille est \RWbet{tam invidus hominibus, tam exosus Deo}, qui [istud] prohibere conetur?} -- \RWlat{Sed ita tamen, ut \RWbet{vere profectus sit ille fidei, non permutatio}. Si quidem ad profectum pertinet, ut in semet ipsam unaquaeque res \RWbet{amplificetur}: ad permutationem vero, ut aliquid ex alio in aliud transvertatur. \RWbet{Crescat igitur oportet, et multum vehementerque proficiat, tam singulorum quam omnium, tam unius hominis, quam totius ecclesiae, aetatum, ac saeculorum gradibus intelligentia, scientia, sapientia: sed in suo duntaxat genere, in eodem scilicet dogmate, eodem sensu, eademque sententia.}}} \RWlat{(Cap. 24.)}: \erganf{\RWlat{Imitetur animarum religio rationem \RWbet{corporum}, quae, licet annorum processu numeros suos evolvant et explicent, eadem tamen, quae erant, permanent. Multum interest inter pueritiae florem et senectutis maturitatem; sed iidem tamen ipsi fiunt senes, qui fuerant adolescentes, ut, quamvis unius ejusdemque hominis status habitusque mutetur, \RWbet{una} tamen nihilominus \RWbet{eademque} natura, \RWbet{una eademque} persona sit. Parva lactantium membra, magna juvenum; eadem ipsa sunt tamen. Quot parvulorum artus tot virorum; et si quae illa sunt, quae aevi maturioris aetate pariuntur, jam in seminis ratione proserta sunt; ut nihil novum postea proferatur in senibus, quod non in pueris jam \var{[antea]}{{ante}{}{}}}~\RWSeitenw{18}\ \RWlat{latitaverit.}} -- \erganf{\RWlat{Ita etiam christianae religionis dogma sequatur has, decet, profectuum leges, ut annis scilicet consolidetur, dilatetur tempore, sublimetur aetate, incorruptum tamen illibatumque permaneat, et universis partium suarum mensuris, cunctisque quasi membris ac sensibus propriis plenum atque perfectum sit, quod nihil praeterea permutationis admittat, nulla proprietatis dispendia, nullam definitionis [sustinet definitionis] varietatem.}} -- -- \RWlat{(cap.\,25.)}:
\erganf{\RWlat{Fas est [etenim], ut prisca illa coelestis philosophiae} (der göttlichen Offenbarung) \RWlat{dogmata processu temporis \RWbet{excurentur}, \RWbet{limentur}, \RWbet{poliantur}; sed nefas est, ut commutentur; nefas, ut detruncentur, [ut] mutilentur. Accipiant licet evidentiam, lucem, distinctionem: sed retineant necesse est plenitudinem, integritatem, proprietatem.}} --
\item Daß die heil.\ Schrift, sowohl des alten als des neuen Bundes, durch \RWbet{göttliche Eingebung} geschrieben sey \usw , liest man in jedem Katechismus, hört man in jeder Predigt. -- Was aber die angeführten neunzehn ökumenischen Concilien anbelangt: so ist nicht zu läugnen, daß die Allgemeinheit einiger derselben von einzelnen Katholiken bestritten wird; weil aber auch diese die Beschlüsse jener Concilien, so ferne sie Glaubens- (Religions-) Lehren und nicht bloße Disciplinarsachen betreffen, ohne Widerspruch annehmen: so durfte ich sie hier immerhin anführen.
\end{aufza}

\RWpar{5}{Vernunftmäßigkeit dieser Lehre, und zwar des ersten Artikels}
Da dieser Artikel nichts als gewisse praktische Vorschriften aufstellt: so fällt die Untersuchung seiner Vernunftmäßigkeit mit der Betrachtung seiner sittlichen Zuträglichkeit, die ich später anstellen will, zusammen. Denn Regeln, die sittlich zuträglich sind, sind auch der Vernunft gemäß.


\RWpar{6}{Vernunftmäßigkeit des zweiten Artikels}
Ich werde die Vernunftmäßigkeit dieser Behauptung erweisen, wenn ich die Nichtigkeit folgender Einwürfe, die~\RWSeitenw{19}\ man theils in der That erhoben hat, theils doch erheben könnte, zeige.\par
\RWbet{1.~Einwurf}. Es ist schon ungereimt, daß es die Kirche versucht, ein Zeugniß der Unfehlbarkeit \RWbet{sich selbst} zu geben; beiläufig eben so, wie wenn Jemand seine eigene Wahrhaftigkeit dadurch beweisen wollte, daß er sich selbst für wahrhaft ausgibt.\par
\RWbet{Antwort.} Es wäre freilich ungereimt, wenn Jemand verlangte, wir sollen ihm glauben, \RWbet{bloß darum}, weil er uns versichert, daß er die Wahrheit spreche. Dieses thun aber die Katholiken gar nicht. Sie verlangen keineswegs, daß wir die Lehren, die sie mit allgemeiner Uebereinstimmung bekennen, für eine göttliche Offenbarung annehmen sollen, weil \RWbet{sie} dieselben dafür erklären; sondern weil wir an diesen Lehren die beiden Kennzeichen einer göttlichen Offenbarung, nämlich die innere Vortrefflichkeit und die Bestätigung durch Wunder leicht wahrnehmen können. Darum ist aber jene Erklärung doch nicht etwa überflüßig. Wie nämlich ein Mensch, an den man die Frage stellt, ob er nicht vielleicht nur zum Scherze, sondern in der ernsten Absicht, damit man ihm glaube, und im deutlichen Andenken an die Pflicht der Wahrhaftigkeit gesprochen habe, diese Frage nicht von sich abweisen darf, sondern ganz ausdrücklich bejahen muß, will er nicht alsbald unser Vertrauen verlieren: so muß auch eine Religion, die für geoffenbart angesehen werden will, den Lehrsatz, daß sie es sey, nicht bloß stillschweigend voraussetzen, sondern zur Antwort für Jene, die darnach fragen könnten, ausdrücklich aufstellen. -- Und gesetzt auch, daß man die ausdrückliche Aufstellung dieses Lehrsatzes einer geschriebenen Offenbarung erlassen könnte, in Rücksicht dessen, daß aus der ganzen Art des Vortrages, die in dem schriftlichen Aufsatze herrscht, vielleicht ersichtlich genug wäre, daß dieser Aufsatz für eine göttliche Offenbarung angesehen seyn will, ohne es ausdrücklich zu sagen: so ist es doch bei einer lebenden Religion (\dh\ bei einer solchen, die von gewissen Menschen wirklich geglaubt wird) eine unerläßliche Bedingung, daß diese Menschen es wissen, ihr Glaube sey eine göttliche Offenbarung, wenn er es wirklich seyn soll. Denn wissen sie dieses~\RWSeitenw{20}\ nicht: so glauben sie auch das, was sie glauben, um keines göttlichen Zeugnisses willen; ihr Glaube ist also schon für sie selbst keine göttliche Offenbarung. Und eben so wenig stehet wohl zu erwarten, daß Andere, die diesen Glauben erst von ihnen annehmen sollen, an ihm die Eigenschaft einer göttlichen Offenbarung erkennen werden, wenn sie bemerken, daß Jene (die Lehrer) selbst ihn nicht dafür erkennen. Um diese Nachtheile zu vermeiden, muß also Gott (so scheint es wenigstens) zu einem jeden Glauben, den er bei Menschen hat entstehen lassen, und als geoffenbart anerkannt wissen will, gleich in denjenigen Individuen, die diesem Glauben zugethan sind, die Ueberzeugung erwecken, daß er geoffenbart sey. Wenn also die Katholiken nicht selbst behaupten würden, daß Alles, was sie mit allgemeiner Uebereinstimmung lehren, göttliche Offenbarung sey: so wäre auch schwer zu glauben, daß es dieß wirklich ist.\par
\RWbet{2.~Einwurf}. Es ist schon an sich selbst unglaublich, daß Gott, wenn er sich den Menschen offenbaren will, den \RWbet{Gesammtglauben} einer gemischten Volksmenge zu jenem Kennzeichen erheben werde, woraus man abnehmen soll, was zu dem sicheren Inhalte dieser Offenbarung gehöre oder nicht. Die Meinung der größeren Menge ist meistens die thörichtere Meinung. \RWlat{Stultorum infinitus est numerus}; sagt die Schrift selbst (s.~\Ahat{Köppen}{Kopper} in seiner Philosophie des Christenthums. 1816. Leipzig, 2.\,Theil.\RWlit{}{Koeppen2} -- \uA ). Dieß haben auch mehrere Katholiken selbst anerkannt; \zB\ der berühmte \RWbet{Melchior Canus} (de locis theologicis l.\,5. c.\,5\RWlit{}{Cano1a}.: \RWlat{Nego [enim], cum de fide agitur, sequi plurimorum judicium oportere.}) Derselben Meinung waren auch \RWbet{Gerson, Picus von Mirandula,} \uA\  --\par
\RWbet{Antwort.} Es ist nichts schicklicher, als dieses Kennzeichen; da die Bemerkung, daß eine Meinung von allen Mitgliedern einer gewissen Gesellschaft angenommen werde, schon an sich selbst geeignet ist, ihr unser Vertrauen zu verschaffen. Auch kann man nicht behaupten, daß in Dingen, welche die Religion (in unserer Bedeutung des Wortes) betreffen, die Meinung der Meisten immer die thörichtere wäre, weil ihre Gegenstände meistens zu ihrer Entscheidung keiner~\RWSeitenw{21}\ Gelehrsamkeit, sondern des bloßen gesunden Menschenverstandes bedürfen, und für alle Menschen von gleicher Wichtigkeit sind. Endlich wird zu Folge dieser Lehre auch nicht die Meinung der \RWbet{Mehrzahl}, sondern die Meinung \RWbet{Aller} oder doch \RWbet{beinahe Aller} zu einem sicheren Kennzeichen erhoben; zu diesen gehören aber auch die Gelehrten; und wenn nur diese einer Meinung widersprechen: so kann man eben deßhalb nicht sagen, daß sie der Gesammtglaube der ganzen Gesellschaft sey, also das Kennzeichen einer göttlichen Offenbarung an sich trage. Canus und jene anderen Katholiken reden insgesammt von dem Falle, wo noch keine allseitige Uebereinstimmung vorhanden ist, sondern eine so eben noch strittige Frage (und dieß zwar eine solche, zu deren Beurtheilung Gelehrsamkeit erforderlich ist) erst entschieden werden soll.\par
\RWbet{3.~Einwurf.} Nach dieser Lehre geben die Katholiken zu, daß jedes einzelne Mitglied ihrer Gesellschaft für sich selbst fehlbar sey; behaupten aber gleichwohl, daß alle zusammengenommen unfehlbar wären; welches sich doch widerspricht. Denn eine Eigenschaft, die jedem einzelnen Theile eines Ganzen zukommt, muß auch dem Ganzen selbst zukommen. Wenn alle Theile der Kirche fehlbar sind: so muß es auch die ganze Kirche seyn. Wie lächerlich wäre es \zB\ von einer Versammlung, deren einzelne Glieder alle blind sind, behaupten zu wollen, die ganze Versammlung sey sehend?\par
\RWbet{Antwort.} 
\begin{aufza}
\item Daß dieser Einwurf ein Trugschluß sey, fühlt gewiß Jeder, der gesunden Menschenverstand hat; obgleich er vielleicht nicht im Stande seyn wird, deutlich zu zeigen, worin hier gefehlt sey. Um aber darzuthun, daß der katholische Glaube an die Unfehlbarkeit der Kirche nichts Ungereimtes habe, dürfte man nur ein Beispiel von ganz ähnlicher Bewandniß anführen, durch welches das im Einwurfe gegebene entkräftet wird. Jeder einzelne Mensch, dürfte man sagen, ist sterblich; und von Niemand kann man behaupten, daß er auch nur noch den heutigen Tag gewiß überleben werde. Dennoch sind wir gewiß, daß das gesammte Menschengeschlecht heute nicht aussterben werde; ja, wenn es Gott gefiele, so könnte das Menschengeschlecht bei aller Sterblichkeit seiner einzelnen Glieder doch in alle Ewigkeit fortdauern,~\RWSeitenw{22}\ so wie es bisher schon sechs Jahrtausende gedauert. Und völlig eben so kann, trotz der Fehlbarkeit jedes einzelnen Katholiken, die Gesammtheit derselben von Gottes Vorsehung so geleitet werden, daß sich niemals alle in einem und eben demselben Irrthume vereinigen.
\item Allein wo liegt der eigentliche Grund des Fehlschlusses in diesem Einwurfe? Es ist hier
\begin{aufzb}
\item erstlich schon der \RWbet{Obersatz,} dessen man sich bedient, falsch, nämlich der Satz, daß eine Eigenschaft, die allen Theilen eines Ganzen fehlt, auch dem Ganzen selbst fehlen müsse. -- Ein jedes Ganze kann und muß sogar, als Ganzes, manche Eigenschaft \RWbet{haben}, welche den einzelnen Theilen nicht zukommt; und eben so können und müssen ihm auch gewisse Eigenschaften \RWbet{abgehen}, welche die einzelnen Theile besitzen. So können \zB\ die Theile eines Ganzen die Eigenschaft der Einfachheit besitzen; dem Ganzen selbst aber muß diese Einfachheit nothwendig fehlen, und jene der Zusammengesetztheit muß ihm zukommen. So können die einzelnen Stäbe in einem Holzbündel die Eigenschaft der Zerbrechlichkeit von eines Knaben Hand besitzen, dem Ganzen kann aber diese Eigenschaft fehlen, und die entgegengesetzte der Unzerbrechlichkeit zukommen.\RWfootnote{%
Der bekannte Satz der Logiker: Was von allen Theilen gilt, gilt auch vom Ganzen, ist also falsch, wenn man hier unter dem \RWbet{Ganzen} etwas Anderes, als vorhin unter dem Ausdrucke: \RWbet{alle Theile}, versteht, \dh\ wenn man den Satz nicht so auslegt, daß er \RWbet{identisch} wird, und eigentlich so lautet: Was von allen Theilen gilt, das gilt von allen Theilen. Denn wenn man unter dem Ganzen den Inbegriff aller Theile (\RWlat{collective non distributive}) versteht, wie dieß der eigentliche Begriff des Ganzen ist: so ist es offenbar falsch, daß Alles, was von allen Theilen \RWlat{distributive} gilt, auch von allen Theilen \RWlat{collective} gelte.}
\item Der \RWbet{Untersatz}, daß jeder einzelne Katholik fehlbar sey, ist wahr oder falsch, je nachdem man ihn verstehet. Wahr, wenn man den Begriff der Möglichkeit, der als Bestandtheil in dem Begriffe der Fehlbarkeit vorkommt, von der \RWbet{problematischen} Möglichkeit auslegt, und also den Satz so versteht: die Annahme, daß dieser oder jener einzelne Katholik in einen Irrthum verfallen werde, widerspricht keiner uns bekannten Wahrheit.
Allein aus die\RWSeitenw{23}sem Satze folgt offenbar gar nichts Nachtheiliges gegen die Unfehlbarkeit der Kirche. Falsch aber wäre der Satz, wenn man unter der Möglichkeit, die der Begriff der Fehlbarkeit ausdrückt, die \RWbet{absolute} Möglichkeit verstünde, und also behaupten wollte: die Annahme, daß jeder einzelne Katholik in einen gewissen religiösen Irrthum verfalle, sey nicht nur mit keiner uns bekannten Wahrheit, sondern überhaupt mit gar nichts Vorhandenem im Widerspruche; dergestalt, daß es Gott selbst nicht möglich sey, zu verhindern, daß nicht alle Katholiken in einen und denselben Irrthum gerathen. Dieß wird Gott allerdings verhindern können.
\end{aufzb}
\end{aufza}\par
\RWbet{4.~Einwurf}. Zu Folge dieser Lehre scheint es, daß ein jeder Katholik, um völlig sicher zu seyn, daß er nicht irre, sich nur nach dem umsehen müsse, was seine übrigen Mitkatholiken glauben. Wenn aber Alle dieß thun: so findet Keiner einen Vorgänger, nach dem er sich in seiner Meinung richten könnte, und folglich bleiben Alle unentschieden; beiläufig eben so, wie in einer Rathssitzung sicherlich nichts entschieden würde, wenn Jeder nur erst die Meinung der Uebrigen abwarten wollte.\par
\RWbet{Antwort.} Man legt diese Lehre ganz falsch aus, wenn man sich vorstellt, zu Folge ihrer habe der einzelne Katholik bei Ausbildung seiner religiösen Begriffe nichts Anderes zu thun, als nur seine Mitkatholiken zu fragen, was sie hierüber glauben. Es ist vielmehr gemeint, daß jeder Katholik zuerst noch ohne Rücksicht auf das, was seine jetzt lebenden Mitkatholiken meinen, bei sich selbst auszumachen suche; was für eine Meinung von dem betreffenden Gegenstande nach inneren und äußeren Gründen (aus der Auctorität früherer Zeiten \udgl ) die allerwahrscheinlichste sey; daß sie dann diese Resultate ihres eigenen Nachdenkens, so viel es möglich ist, einander mittheilen, und dann diejenige Meinung, von der es sich zeigt, daß sich nach Anhörung aller Gründe, die Dieser oder Jener für sie vorbringt, Alle in ihr vereinigen, als eine ihnen so eben zu Theil gewordene göttliche Offenbarung ansehen mögen. Dieß ist nun ganz das Verfahren, das man bei einer Rathsversammlung befolgt.~\RWSeitenw{24}\ Jeder Einzelne trägt seine Meinung zuerst nur als Privatmeinung vor, und unterstützt sie mit Gründen; dann nach vernommener Meinung der Uebrigen ändert er wohl noch etwas an ihr, oder verläßt sie auch ganz. Zeigt sich auf diese Art zuletzt, daß Alle in einer gewissen Meinung sich vereinigen können: so sieht man diese als die Entscheidung der ganzen Versammlung, und in dieser Rücksicht \zB\ als rechtskräftig an.\par
\RWbet{5.~Einwurf.} Die Unfehlbarkeit der Kirche setzt ein immerwährendes Wunder von Seite Gottes voraus; denn nur durch beständige Wunder kann Gott die Gemüther der einzelnen Katholiken so leiten, daß sie in keinem religiösen Irrthume jemals zusammenstimmen.\par
\RWbet{Antwort.} 
\begin{aufza}
\item Gesetzt, es wäre so; doch würde daraus die Ungereimtheit dieser Lehre noch nicht folgen. Denn es ist ja nicht unmöglich, und Gott auf keine Weise unanständig, zu einem wohlthätigen Zwecke -- woferne er ihn auf keine andere Art, als durch Wunder erreichen kann -- selbst Wunder zu veranstalten. Daß aber die Unfehlbarkeit der Kirche eine sehr wohlthätige Wirkung sey, leuchtet unter der Voraussetzung, daß die Lehren des Katholicismus sittliche Zuträglichkeit haben, von selbst ein.
\item Indessen ist es nicht einmal nöthig, hier Wunder in irgend einer Bedeutung des Wortes anzunehmen. Denn versteht man unter Wundern \RWbet{unmittelbare} Wirkungen Gottes: so ist es offenbar, daß Gott, um die Gesammtheit der Katholiken nie in einen religiösen Irrthum verfallen zu lassen, keiner unmittelbaren Einwirkungen bedürfe; sondern sich in der That sehr vieler Mittelursachen bedienen könne, und nach der Lehre der Katholiken auch wirklich bediene. So braucht er \zB\ nur, wenn eine große Menge sich zu einem Irrthume hinneigt, und zu befürchten steht, daß dieser Irrthum bald allgemein werde, irgend einige Männer von großem Ansehen auftreten zu lassen, die diesen Irrthum bekämpfen, und durch die Macht ihrer Gründe und ihres Ansehens auch die übrigen umstimmen, \udgl\,m. Aus diesem Beispiele erhellet auch schon, daß Gott zu diesem Zwecke nicht einmal \RWbet{übernatürliche} Kräfte in Thätigkeit setzen müsse. Endlich~\RWSeitenw{25}\ sind auch nicht einmal \RWbet{außerordentliche} oder \RWbet{ungewöhnliche} Begebenheiten nöthig; denn Gott vermag uns zur Annahme gewisser Meinungen zu leiten, die Entstehung dieser oder jener Vorstellung in uns zu veranlassen, ohne daß wir oft selbst wissen, durch welche Begebenheiten dieß eigentlich geschehen sey, geschweige denn, daß diese Begebenheiten das Merkmal der Außerordentlichkeit immer an sich tragen müßten. Höchstens also könnte man sagen, daß die Leitung Gottes, die zu dieser Unfehlbarkeit der Kirche erfordert wird, ein Wunder in der Bedeutung sey, in der man darunter eine Verfügung versteht, die zu bewundern wir Ursache haben.
\end{aufza}


\RWpar{7}{Vernunftmäßigkeit des dritten Artikels}
\begin{aufza}
\item Dieser Artikel enthält zwei verschiedene Puncte:
\begin{aufzb}
\item Daß die Kirche, die jetzt den Namen der katholischen trägt, schon seit der Zeit ihrer Stiftung durch Jesum Christum im Besitze des großen Vorzuges sey, daß ihr Gesammtglaube allemal eine wahre göttliche Offenbarung ist;
\item daß dieses bei den akatholischen Parteien keineswegs der Fall ist.
\end{aufzb}
\item Der letzte Punct hat keine Schwierigkeiten; denn
\begin{aufzb}
\item da die Lehre der akatholischen Parteien in mehreren Stücken der Lehre der katholischen Kirche geradezu widerspricht: so ist es vermöge des Begriffes der Wahrhaftigkeit Gottes ganz folgerecht, zu behaupten, daß die akatholischen Religionen, in sofern, als sie der katholischen widersprechen, keine wahre göttliche Offenbarung enthalten. Dieß kann um desto füglicher behauptet werden, da
\item (wie wir bald sehen werden) keine von diesen Religionen, wiefern sie Volksreligion ist, sich auch nur selbst für geoffenbaret ausgibt.
\end{aufzb}
\item Wir untersuchen also nur noch den ersten Punct. Eigentlich könnte die Frage, ob der Gesammtglaube der Katholiken, die lange vor uns gelebt, eine wahre göttliche Offenbarung gewesen? uns gleichgültig seyn, wenn wir nur wissen,~\RWSeitenw{26}\ daß es der Glaube der jetzt Lebenden ist. Wird dieser Lehrsatz gleichwohl von der Kirche ausdrücklich aufgestellt: so geschieht dieß nur in der Absicht, um aus demselben alle die Folgerungen herzuleiten, die wir bei der Betrachtung seines sittlichen Nutzens angeben werden; vornehmlich, um uns desto gewisser und desto begreiflicher zu machen, daß und auf welche Art der Gesammtglaube der jetzt lebenden Katholiken eine wahre göttliche Offenbarung sey. Denn wenn wir nicht glaubten, daß schon die früheren Katholiken in dem Besitze dieses Vorrechtes gewesen: so würde uns unglaublich dünken, wie er uns später Lebenden zu Theil geworden sey. -- Aber gerade darum, weil dieser Lehrsatz nur zum Behufe solcher praktischer Folgerungen aufgestellt ist, wird auch zu seiner Rechtfertigung eigentlich nicht seine \RWbet{objective} Wahrheit, sondern nur die Richtigkeit und Zuträglichkeit dieser \RWbet{praktischen Folgerungen} erfordert. Gesetzt also auch, daß dieser Lehrsatz nicht leicht in völliger Strenge erweislich wäre: so würde doch das, was an ihm eigentlich allein religiösen Gehaltes ist, völlig gerechtfertiget seyn, sobald wir tiefer unten nur jene praktischen Folgerungen genauer kennen lernen. Indessen läßt sich doch auch das \RWbet{historische Factum}, das hier zu Grunde gelegt ist, mit großer Wahrscheinlichkeit darthun. Hiezu gehört nämlich dreierlei. Man muß außer Zweifel setzen,
\begin{aufzb}
\item daß die katholische Kirche wirklich von Jesu gestiftet worden sey; denn dieses wird in jenem Lehrsatze vorausgesetzt. Man muß erweisen,
\item daß diese Kirche von jeher geglaubt, den Vorzug zu haben, daß der Gesammtglaube ihrer Glieder eine göttliche Offenbarung ist. Denn hätte sie dieß zu irgend einer Zeit selbst nicht geglaubt: so wäre es auch gar nicht wahrscheinlich, daß sie zu dieser Zeit dieß Vorrecht wirklich gehabt habe. Man muß endlich darthun,
\item daß die katholische Kirche zu keiner Zeit mit Uebereinstimmung aller ihrer Glieder etwas gelehrt habe, was erweislicher Maßen nicht Gottes Offenbarung seyn kann.
\end{aufzb}
\end{aufza}
Diese drei Stücke werde ich also jetzt einzeln darzuthun suchen.~\RWSeitenw{27}



\RWpar{8}{I.~Die katholische Kirche ist von Jesu Christo wirklich gestiftet worden}
Wohl haben einige Feinde des Christenthums sich von ihrem Widerspruchsgeiste so weit verleiten lassen, zu behaupten, daß Jesus von Nazareth gar nicht die Absicht gehabt habe, eine neue Religion unter den Menschen einzuführen; oder wenn er dieß ja gewollt, so sey es doch bloß die natürliche Religion gewesen; jene Geheimnißlehren aber, welche die christkatholische Kirche in der Folge aufgestellt hat, seyen den Absichten, die er gehabt, völlig zuwider. Dieß Alles läßt sich nun hinlänglich widerlegen.
\begin{aufza}
\item Schon aus dem Umstande, den selbst die heftigsten Feinde des Christenthums nicht läugnen wollen, daß Jesus ein Mann von den ausgezeichnetesten Kräften des Geistes gewesen, läßt sich mit vieler Wahrscheinlichkeit schließen, er habe diese Kräfte nicht unbenützt lassen, er habe irgend eine große und wohlthätige Veränderung auf Erden zu Stande bringen wollen. Daß dieß auch wirklich der Fall gewesen sey, beweisen nicht nur die Anhänger, die er gewählt, sondern auch sein Tod am Kreuze. Daß aber die Veränderung, die zu bewirken er sich vorgenommen hatte, nicht in politischen Umwälzungen (wenigstens nicht zunächst, und nur allein in diesen) bestanden habe, folgt schon daraus, weil er sich an die Großen und Mächtigen seines Landes, die ihm in diesem Stücke gewiß gerne zu Hülfe gekommen wären, da sie sehr unzufrieden mit der bestehenden Verfassung waren, so ganz und gar nicht gewendet hatte. Hieraus folgt denn unwidersprechlich, daß die Veränderung, die Jesus bewirken wollte, in den \RWbet{Gemüthern} der Menschen vorgehen, oder hier wenigstens ihren Anfang nehmen sollte. Durch Einführung neuer Begriffe über die wichtigsten Gegenstände des menschlichen Wissens war er gesonnen, eine viel wohlthätigere, bleibendere, weiter um sich greifende Revolution zu bewirken, als es durch bloße politische Veränderungen je möglich gewesen wäre. Allein die wichtigsten Begriffe sind und bleiben jederzeit die \RWbet{religiösen}. Kein Zweifel also, daß Jesus die Religion aus dem Gebiete dessen, worüber er die Men\RWSeitenw{28}schen aufklären wollte, nicht nur nicht ausgeschlossen, sondern daß diese vielmehr den vornehmsten Gegenstand seines Unterrichtes ausgemacht habe; kein Zweifel, daß er die Absicht gehabt, eine neue und bessere Religion auf Erden einzuführen.
\item Ganz eine andere Frage ist es, ob Jesus auch mit völliger Gewißheit vorhergesehen und gewollt habe, daß aus dem Institute, das er zur Einführung seiner neuen Religion errichtet hatte, in der Folgezeit gerade diese und jene religiösen Ansichten hervorgehen, die daraus wirklich hervorgegangen sind? Gesetzt auch, daß dieß Jemand zweifelhaft fände: so könnte er gleichwohl Jesum noch immer den Stifter und Veranlasser all dieser religiösen Ansichten nennen, sobald nur dargethan ist, daß auch sie der Tugend und Glückseligkeit der Menschen beförderlich sind. Denn Jeder, der eine Unternehmung in guter Absicht anfängt, kann mit Recht Stifter und Veranlasser von allen guten Folgen derselben, auch selbst derjenigen heißen, die er nie mit Bestimmtheit vorausgesehen hatte. Wäre es also auch wahr, daß die Religion, die Jesus seinen Zeitgenossen vortrug, und deren allgemeine Einführung er wünschte, bloß die natürliche gewesen war: so könnten wir dennoch behaupten, er sey der Stifter der \RWbet{ganzen katholischen Religion}, weil auch die positiven Lehren derselben (die sogenannten Geheimnißlehren) einen überaus wohlthätigen Einfluß auf die Menschheit haben, und einen noch größeren in der Zukunft versprechen; so, daß es also nur ein bloßer Irrthum Jesu gewesen seyn müßte, wenn er sich eingebildet hätte, daß jeder Zusatz zur natürlichen Religion durchaus verwerflich sey.
\item Doch in der That haben wir gar keinen Grund, Jesum in dem Verdachte zu haben, daß er auch nur diesem einzigen Irrthume ergeben gewesen sey. Aus den Evangelien geht vielmehr Verschiedenes hervor, das diesem Verdachte laut widerspricht. Den Evangelien zu Folge verlangte unser Herr ausdrücklich,
\begin{aufzb}
\item für einen \RWbet{göttlichen Gesandten} angesehen zu werden, was er nicht nöthig gehabt haben würde, hätte er ein bloßer Lehrer der natürlichen Religion seyn wollen; den Evangelien zu Folge ließ er sich~\RWSeitenw{29}
\item bei dem Vortrage seiner Lehre nie in Vernunftbeweise derselben ein; nahm
\item so manche Meinung als gewiß an, die doch durch bloße Vernunft keineswegs eingesehen werden kann, \zB\ von dem Daseyn der Engel, guter sowohl als böser, \usw\ stellte endlich
\item einige selbst zu seiner Zeit noch neue Ansichten auf, die aus der bloßen Vernunft allein schwerlich erwiesen werden können; wie dieß \zB\ aus seiner Vorschrift, zu taufen im Namen des Vaters, des Sohnes und des heiligen Geistes erhellet; \usw\
\end{aufzb}
\end{aufza}


\RWpar{9}{II.~Die katholische Kirche befand sich zu allen Zeiten in der Meinung, daß der Gesammtglaube ihrer Glieder eine wahre göttliche Offenbarung sey}
Wenn sich die Kirche schon seit ihrem Ursprunge, den sie von Jesu herleitet, in dem Besitze des Vorrechtes befunden, daß der Gesammtglaube ihrer Glieder eine göttliche Offenbarung gewesen: so muß sie dieß auch von jeher \RWbet{gewußt} haben. Denn es ist ungereimt, anzunehmen, daß Gott ihr dieses Vorrecht geschenkt, und nicht zugleich auch das Bewußtseyn desselben gegeben habe; da der vorzüglichste Nutzen, den dieses Vorrecht gewähret, nämlich das Zutrauen, das nun ein Jeder zu den mit Allgemeinheit herrschenden Lehren haben kann, dann weggefallen wäre. Ich muß also, besonders weil es von Seite der Gegner öfters geläugnet worden ist, eigens beweisen, daß die Kirche zu aller Zeit in der Meinung gestanden sey, daß ihr Gesammtglaube eine göttliche Offenbarung wäre. Doch brauche ich diesen Beweis nur von den früheren, dem zweiten oder dritten Jahrhunderte, zu führen, weil es von Jedermann zugegeben wird, daß sich dieser Glaube in späteren Zeiten (nach dem dritten Jahrhunderte) allerdings vorgefunden habe.
\begin{aufza}
\item Ich sage nun, schon die \RWbet{Apostel} sind dieser Meinung gewesen. Denn als zu ihrer Zeit die wichtige Streitfrage entstand, ob das mosaische Gesetz auch noch für Christen eine Verbindlichkeit habe: so nahmen sie, um diese Frage~\RWSeitenw{30}\ zu entscheiden, ihre Zuflucht zu einer allgemeinen Versammlung; es wurde der erste \RWbet{Kirchenrath zu Jerusalem} gehalten; und was man hier mit allgemeiner Uebereinstimmung herausgebracht hatte, das sah man für eine göttliche Eingebung an. \erganf{Es hat dem heiligen Geiste und uns gefallen}, \usw\ (\RWbibel{Apg}{Apostelg.}{15}{28})
\item Das ist es wahrscheinlich auch, was man in jenem sogenannten \RWbet{apostolischen Glaubensbekenntnisse}, dessen Abfassung, wenn [sie] auch nicht von den Aposteln herrührt, doch gewiß sehr nahe an ihr Zeitalter reicht, unter den Worten verstanden hatte: \RWbet{Ich glaube eine heilige allgemeine Kirche}; denn wie verdiente die Kirche den Beinamen \RWbet{heilig}, wenn es nicht ihre Lehre ist? und wie könnte diese heilig genannt werden, wenn man sich nicht vorstellte, daß sie \RWbet{geoffenbar}t ist? Der Beisatz \RWbet{allgemein} aber kann in jener Zeit doch sicher nicht von der großen Ausbreitung der Kirche im Vergleich mit andern Religionen genommen worden seyn; denn im Vergleiche mit dem Juden-- und Heidenthume war ja das Christenthum damals noch gar nicht der allgemein herrschende Glaube. Also wird wohl der Ausdruck: \RWbet{allgemeine Kirche}, gerade nur so viel bedeutet haben, als der Ausdruck: \RWbet{ganze Kirche}; und somit wollte man nur dasjenige, was alle Mitglieder der Kirche glauben, für heilig, für geoffenbart anerkannt wissen.
\item \RWbet{Irenäus} widerlegt (in seinen \RWlat{libris V adversus haereses}\RWlit{}{Irenaeus1}) die Irrthümer der Ketzer seiner Zeit immer nur dadurch, daß er den Widerspruch ihrer Meinung entweder mit einer klaren Bibelstelle, oder mit dem \RWbet{allgemeinen Glauben} der Kirche zu zeigen sucht. Den Umstand also, daß eine Meinung in der Kirche \RWbet{allgemein} geglaubt wird, mußte nicht nur er selbst, sondern auch die übrige große Menge der Christen, ja sogar Mehrere der Ketzer mußten denselben für ein sicheres Kennzeichen halten, daß eine solche Meinung geoffenbart sey. Daher sagt er auch (\RWlat{l.\,5. c.\,20. nro.\,1.}) mit audrücklichen Worten: Der Kirche ist das Licht der Wahrheit anvertraut; sie redet überall die Wahrheit. Und (\RWlat{l.\,3. c.\,4.}): Da so viele Beweise in der Tradition (\dh\ mündlichen Ueberlieferung) vorhanden sind: so~\RWSeitenw{31}\ dürfen wir nicht erst bei Andern (\RWlat{inter alios}) die Wahrheit suchen, die wir so leicht \RWbet{von der Kirche} annehmen können. Denn die Apostel haben Alles, was Wahrheit ist, in der Kirche, als in einer reichen Schatzkammer niedergelegt, auf daß ein Jeder, der will, daraus den Lebenstrank empfange. -- -- -- Wie aber, wenn die Apostel gar keine Schriften zurückgelassen hätten? (fragt er diejenigen, die Alles nur aus der Schrift bewiesen haben wollten.) Müßten wir dann uns nicht an die mündliche Ueberlieferung halten? \usw\
\item \RWbet{Clemens von Alexandrien} schreibt (\RWlat{Stromat.\ l.\,7.}\RWlit{}{Clemens1}): Derjenige hat aufgehört, ein Gott getreuer Mensch zu seyn, der gegen die \RWbet{Uebergabe} (Tradition) \RWbet{der Kirche} sich sträubt, und in die Meinungen \RWbet{menschlicher Secten} sich einläßt. Der Gegensatz, der hier zwischen der Uebergabe der Kirche und den Meinungen menschlicher Secten gemacht wird, beweist deutlich, daß Clemens die erstere, \di\ die kirchliche Uebergabe, oder das, was Alle glauben, für eine göttliche Offenbarung gehalten habe.
\item \RWbet{Tertullian} schreibt (\RWlat{de praescriptione, c.\,19.}\RWlit[: \eanf{\RWlat{Ergo non ad scripturas provocandum est nec in his constituendum certamen, in quibus aut nulla aut incerta victoria est aut parum certa. Nam etsi non ita evaderet conlatio scripturarum, ut utramque partem parem sisteret, ordo rerum desiderabat illud prius proponi, quod nunc solum disputandum est: quibus competat fides ipsa, cuius sint scripturae, a quo et per quos et quando et quibus sit tradita disciplina, qua fiunt christiani.}}]{c.\,19}{Tertullian2a}) gleichfalls gegen diejenigen, die Alles aus der Schrift bewiesen haben wollen: \erganf{So muß man sich denn nicht auf die heiligen Schriften berufen, und den Streitpunct nicht auf Gründe ankommen lassen, welche entweder gar keinen, oder einen nur ungewissen, oder doch wenigstens nicht ganz gewissen Sieg gewähren können; denn wenn sich auch durch die Vergleichung mehrerer Schriftsteller zuweilen etwas ausmitteln ließe, was nicht beide Theile mit einem gleichen Scheine für sich auslegen können: so gebeut doch die gute Ordnung, daß jene Frage, worauf es im Streite allein ankommt, zuerst erörtert werde, die Frage nämlich, \RWbet{wem} eigentlich der Glaube angehöre, \RWbet{wem} die Schrift eigen sey, \RWbet{von wem} und \RWbet{durch wen, an wen} und \RWbet{wann} die Lehre, welche die Christen zu Christen macht, sey übergeben worden?} Tertullian setzt also voraus, daß die Auctorität der Kirche größer sey, als die der Bibel. Jene, meint er, sey eine ursprüngliche, diese nur eine abgeleitete. \erganf{Denn (schreibt er weiter) wo die wahre Lehre und der wahre Glaube der Christen ist (\dh\ wie Tertullian meint, bei der Kirche), da wird wohl auch die wahre~\RWSeitenw{32}\ Schrift, die wahre Auslegung derselben, und die wahre Uebergabe seyn. -- -- -- Demnach wird alle Lehre, die mit jenen Kirchen übereinstimmet, welche die apostolischen und Mutter-Kirchen, die Originalkirchen unseres Glaubens heißen, als wahr angenommen werden müssen; indem sie ohne Zweifel nur das behaupten, was die Kirche von den Aposteln, diese von Christo, Christus von Gott empfangen hat}, \usw\RWlit[: \eanf{\RWlat{Si haec ita sunt, constat proinde omnem doctrinam quae cum illis ecclesiis apostolicis matricibus et originalibus fidei conspiret, veritati deputandam, id sine dubio tenentem, quod ecclesiae ab apostolis, apostoli a Christo, Christus a deo accepit}}.]{c.\,19}{Tertullian2a} Und in dem Buche \RWlat{de corona militis}\RWlit[: \eanf{\RWlat{Harum et aliarum ejusmodi disciplinarum si legem expostules Scripturarum, nullam invenies: traditio tibi praetendetur auctrix, consuetudo confirmatrix, et fides observatrix. Rationem traditioni, et consuetudini et fidei patrocinaturam aut ipse percipies, aut ab aliquo qui perspexerit disces; interim nonnullam esse credes, cui debeatur obsequium.}}]{c.\,4}{Tertullian2b} (\RWlat{c.}\,4.): \erganf{Wenn du von diesen und anderen ähnlichen Einrichtungen ein Gebot in der heiligen Schrift suchst: so findest du keines. Die mündliche Uebergabe wird als Stifterin, die Gewohnheit als Bestätigerin, und die Glaubwilligkeit als Beobachterin angeführt werden. Daß aber diese Uebergabe, daß die Glaubwilligkeit und Gewohnheit \RWbet{durch die Vernunft gerechtfertiget} werde, wirst du entweder selbst einsehen, oder von Jemand, der es versteht, lernen können. Bis dahin glaube nur, daß allerdings ein Grund vorhanden sey, der deiner Unterwürfigkeit werth ist.} --
\item \RWbet{Origenes} sagt (\RWlat{Princip.\ l.\,1. c.\,2.}%
\RWlit[: \eanf{\RWlat{servetur vero ecclesiastica praedicatio per successionis ordinem ab Apostolis tradita, et usque ad praesens in ecclesiis permanens: illa sola credenda est veritas, qua in nullo ab ecclesiastica et apostolica discordat traditione.}}]{l.\,1, praef.}{Origenes2a}): 
\erganf{Man halte sich an jene Lehren der Kirche, die von den Aposteln her durch die ordentliche Reihe ihrer Nachfolger überliefert worden sind. Nur das ist wahr, was von der kirchlichen und apostolischen Uebergabe nirgends abweicht.}
\item Der Heilige \RWbet{Cyprian} (Bischof zu Karthago, der i.\,J.\ 261 starb) schrieb unter Anderm ein ganzes Buch \RWlat{de unitate ecclesiae}\RWlit[: \eanf{\RWlat{Deus unus est, et Christus unus, et una ecclesia ejus, et fides una, et plebs una in solidam corporis unitatem concordiae glutino copulata.}}]{}{Cyprianus1}, in dem er beweiset, daß Alles, was die Kirche mit Uebereinstimmung aller ihrer Glieder lehrt, unfehlbare Wahrheit sey. Da heißt es unter Anderm: \erganf{Ein Gott, Ein Christus, Eine Kirche Christi, Ein Glaube, Ein Volk, das durch das Band der Eintracht zu Einem Leibe geeiniget ist.}
\item Der heilige \RWbet{Augustin} schreibt (\RWlat{ad Catech.\ de Symbolo}\RWlit[: \eanf{\RWlat{Ipsa est Ecclesia sancta, Ecclesia una, Ecclesia vera, Ecclesia catholica, contra omnes haereses pugnans: pugnare potest, expugnari tamen non potest.}}]{}{Augustinus4a}): \erganf{Die Kirche ist eine heilige Kirche, eine einige Kirche, eine wahre Kirche, eine katholische (allgemeine) Kirche, die wider alle Irrlehren streitet. Streiten dawider kann sie; aber besiegt werden im Streite vermag sie nicht;} \usw\ -- Im Buche \RWlat{contra Manich\RWlit[: \eanf{\RWlat{Ego vero Evangelio non crederem, nisi me catholicae Ecclesiae commoveret auctoritas.}}]{}{Augustinus4b}}. (\RWlat{c.}\,5.) kommet der bekannte Ausdruck vor: \erganf{\RWlat{Nec evangelis crederem, nisi ecclesiae}~\RWSeitenw{33}\ \RWlat{auctoritas me commoveret.}} Nothwendig also mußte die Auctorität der Kirche, \dh\ das Ansehen, das ihr Gesammtglaube einer Meinung gibt, das Ansehen einer göttlichen Offenbarung seyn. \RWlat{Epistola II.\ ad Januarium}\RWlit{}{Augustinus4c} heißt es: \erganf{\RWlat{Omnia itaque talia, quae neque sanctarum scripturarum auctoritate continentur, nec in conciliis episcoporum statuta inveniuntur, nec consuetudine universae ecclesiae roborata sunt, sed pro diversorum locorum diversis moribus innumerabiliter variantur, ita, ut [vix] aut omnino nunquam inveniri possint causae, quas in eis instituendis homines secuti sunt, ubi facultas tribuitur, sine ulla dubitatione resecanda existimo.}} Umgekehrt also dasjenige, was sich auf \RWbet{Schrift, allgemeine Concilien,} oder \RWbet{allgemeinen Gebrauch der Kirche} gründet, will Augustin immer beobachtet wissen, er hält es also für \RWbet{unfehlbar}. \erganf{\RWlat{Quamvis enim neque hoc inveniri possit, quomodo contra fidem sint}} (wenn diese Sachen auch nicht wider den Glauben sind): \erganf{\RWlat{ipsam tamen religionem, quam paucissimis et manifestissimis celebrationum sacramentis misericordia Dei liberam esse voluit, servilibus oneribus premunt, ut tolerabilior sit conditio Judaeorum, qui, etiamsi tempus libertatis non agnoverunt, legalibus tamen sarcinis, non humanis praesumptionibus subjiciuntur. Sed ecclesia Dei inter multam paleam, multaque zizania constituta, multa tolerat, et tamen, quae sunt contra fidem vel bonam vitam, non approbat, nec tacet, nec facit.}} U.\,m.\,a.
\item Endlich ist auch die \RWbet{Geschichte aller Kirchenversammlungen} selbst ein Beweis, daß sich die allgemeine Kirche allezeit für unfehlbar gehalten habe. Denn so oft Streitigkeiten unter den einzelnen Lehrern der Kirche entstanden waren, die man für wichtig genug hielt, und durch kein leichteres Mittel beizulegen wußte: so veranstaltete man zu ihrer Beilegung eine allgemeine Kirchenversammlung, und was nun diese entschieden hatte, das nahmen alle Christen als untrügliche Glaubenslehre an, und schloßen eben deßhalb jene Wenigen, die sich noch jetzt nicht fügen wollten, aus ihrer Gemeine aus.~\RWSeitenw{34}
\end{aufza}

\begin{RWanm}
Der Wahrheit zu Liebe muß ich gleichwohl gestehen, daß es auch einige \RWbet{Einwürfe} gibt, welche die Richtigkeit der hier vorgetragenen Behauptung in Etwas zweifelhaft machen können. Bei den Schriftstellern des \RWbet{ersten} christlichen Jahrhunderts findet sich keine ganz deutliche Spur, daß auch schon sie den Gesammtglauben der Kirche für eine wahre göttliche Offenbarung angesehen hätten; und bei den Schriftstellern der folgenden zwei oder drei Jahrhunderte, selbst bei denjenigen, welche wir oben angeführt, finden sich einzelne Stellen, aus denen das Gegentheil von dem hervorzugehen scheint, was wir vorhin behaupteten. So schreibt \zB\ \RWbet{Cyprian} (\RWlat{Ep.}\,74.)\RWlit{74}{Cyprianus2}: \erganf{\RWlat{Unde est ista traditio?} (daß die Ketzer bei ihrer Bekehrung nicht wiedergetauft werden sollen) \RWlat{Utrumne de dominica et evangelica auctoritate descendens, an de Apostolorum mandatis atque epistolis veniens? Ea enim facienda [esse], quae scripta sunt, Deus testatur.}} -- \erganf{\RWlat{Si ergo aut in evangelio praecipitur, aut in Apostolorum epistolis vel actibus continetur, ut a quacumque haeresi venientes non baptizentur, sed tantum manus illis imponatur in poenitentiam, observetur divina haec et sancta traditio.} -- \RWlat{[Nam] si ad divinae traditionis caput et originem revertamur, cessat error humanus.}} \usw\ Ich erinnere denn, daß es nicht eben so viel zu bedeuten hätte, wenn es auch wirklich so wäre, daß die Kirche den Glauben an die Unfehlbarkeit ihrer mit allgemeiner Uebereinstimmung getroffenen Entscheidungen erst in den \RWbet{späteren Jahrhunderten} (etwa im vierten) angenommen hätte. Es ließe sich nämlich sehr leicht erklären, aus welchem Grunde Gott diesen Glauben erst in den späteren Jahrhunderten habe entstehen lassen. Wie nämlich Alles, was Gott veranstaltet, auch immer gerade zur \RWbet{rechten} Zeit und an dem rechten Orte von ihm herbeigeführt wird: so ist dieß auch der Fall bei allen \RWbet{Wahrheiten}, die er den Menschen offenbaret. Er theilt uns jede derselben genau zur \RWbet{rechten} Zeit, also nie früher mit, als bis wir sie eben verstehen und anwenden können. Nun könnte es allerdings seyn, daß die christliche Kirche in den drei ersten Jahrhunderten noch keine Gelegenheit gehabt, den Glauben an ihre Unfehlbarkeit zu gebrauchen. Damals hatten die einzelnen Glieder derselben noch keine Mittel in den Händen, den Glauben der Uebrigen kennen zu lernen; damals konnte man noch keine allgemeine Versammlungen halten \usw\ Es hatte also auch noch keinen sonderlichen Nutzen, zu wissen, daß nur dasjenige, worüber sich Alle vereinigen, unfehlbar sey. Erst als den Christen die Gelegenheit wurde, all\RWSeitenw{35}gemeine Zusammenkünfte und Berathschlagungen zu halten, war es an der Zeit, sie auf dieß Mittel zu verweisen, und den Glauben aufkommen zu lassen, daß sie durch den Gebrauch desselben in streitigen Fällen mit aller Sicherheit entscheiden könnten, auf welcher Seite die Wahrheit liege.
\end{RWanm}


\RWpar{10}{III.~Die katholische Kirche hat zu keiner Zeit mit Uebereinstimmung aller ihrer Glieder etwas gelehret, das erweislicher Maßen nicht Gottes Offenbarung seyn kann}
Um zu beweisen, daß die katholische Kirche schon Manches gelehrt habe, was sich doch unmöglich als eine wahre göttliche Offenbarung ansehen läßt, bringen die Gegner des Katholicismus eine sehr große Liste von Beschuldigungen zum Vorschein. Einwürfe, die ihre Widerlegung noch in der Folge (bei der weiteren Darstellung des katholischen Lehrbegriffes) finden, kann ich jetzt füglich unberührt lassen. Von den übrigen aber will ich, da die Zeit nicht hinreicht, alle umständlich anzuführen, nur die wichtigsten ausheben, und durch die Art, wie ich sie widerlege, zeigen, wie auch die übrigen widerlegt werden können. Man hat also gesagt,
\begin{aufza}
\item die Kirche habe sich in den Verfassern verschiedener Bücher der Bibel offenbar geirrt, \zB\ wenn sie den Prediger oder die Sprichwörter dem Könige Salomo zuschrieb, da diese Schriften doch sicher aus einem viel späteren Zeitalter sind;
\item sie habe unzählige Stellen der Bibel und den Sinn ganzer Bücher mißverstanden, \zB\ wenn sie das hohe Lied allgemein so auslegte, als ob hier die geistige Liebe Jesu und seiner Kirche besungen würde, oder wenn das \RWlat{Concilium Sirmiense} in der Mitte des vierten Jahrhunderts gegen Photinus den Anathematismus aufstellte: \erganf{\RWlat{Si quis: \RWbet{Pluit Dominus a Domino} (\RWbibel[Gen.\ 19.]{Gen}{}{19}{24}), non de Filio et Patre intelligat, anathema sit}}; oder wenn sie so viele Stellen des alten Bundes auf den Messias deutete, die doch nichts weniger, als von ihm handeln; \usw\
\item sie habe erweislich historische Unwahrheiten in Schutz genommen; \zB\ die Fabel vom himmlischen Ursprunge des Rosenkranzes; \udgl~\RWSeitenw{36}
\item sie habe insonderheit viele Personen für Heilige erklärt, die doch kein frommes Leben geführt, ja deren Daseyn wohl gar noch zweifelhaft ist, \zB\ die heil.\ Veronika (\RWlat{Vera} \RWgriech{e>ik'wn}), die heil.\ \RWlat{Ursula cum undecim virginum millibus (XI.\ V.\ M.}\RWfootnote{%
Diese drei Zeichen heißen bekanntlich: \RWlat{undecim virgines martyres}.}\RWlat{)}%
; den heil.\ Christophorus; die heil.\ Barbara; \usw ; \usw
\item sie habe Schriften für echt gehalten, welche doch sicher unterschoben sind; \zB\ das apostolische Glaubensbekenntniß, das \RWlat{Symbolum Athanasianum},\editorischeanmerkung{Das \RWlat{Symbolum Athanasianum} oder auch \erganf{Athanasianisches Glaubensbekenntnis} oder nach den Anfangsworten \erganf{\RWlat{Quicumque vult}} ist ein christliches Glaubensbekenntnis aus dem späten 6.\ oder frühen 7.~Jahrhundert.} die \RWlat{Canones apostolorum},\RWlit{}{CanonesApostolorum} die \RWlat{Decretales Isidori peccatoris}\RWlit{}{DecretalesIsidori1}, \uma\ 
\item die spätere Kirche habe der früheren ausdrücklich widersprochen; was ein vorhergehendes Concilium gutgeheißen, habe ein späteres für Ketzerei erklärt. Z.\,B.\ der allgemeine Kirchenrath zu Chalcedon erklärte den Theodoretus von Mopsveste und den Ibas von Edessa für unschuldig; das allgemeine Concilium zu Constantinopel (das 2te) verdammte die Schriften Beider; \udgl\,m.
\item mehr als einmal habe sich in der Kirche die schädliche Meinung verbreitet, daß das Ende der Welt (\dh\ der Untergang des menschlichen Geschlechtes, oder der jüngste Tag) schon im Anzuge sey. Die Christen des ersten Jahrhunderts und die \RWbet{Apostel selbst} scheinen dieß durchgängig geglaubt zu haben. Häufig erwartete man dann auch ein \RWbet{tausendjähriges irdisches Reich}, in welchem die Christen unter der Oberherrschaft Jesu auf Erden durch einen Zeitraum von 1000 Jahren in dem Genusse sinnlicher Vergnügungen zubringen würden;
\item der so gefährliche Glaube an Besitzungen des Teufels sey einst so herrschend in der Kirche gewesen, daß man jede nur etwas ungewöhnlichere Krankheit dem Teufel zuschrieb; daß man in jedem noch ungetauften Menschen eine Besitzung des Teufels befürchtete, und eben deßhalb eine förmliche Teufelsbeschwörung (den Exorcismus) vor jeder Taufe einführte, und noch heute beibehält.
\item Das heil.\ Abendmahl, das Jesus zu seinem Gedächtnisse eingesetzt hat, reichte die frühere Kirche den kleinsten~\RWSeitenw{37}\ Kindern, wodurch sie hinlänglich verrieth, daß sie den Zweck desselben verkannte. \Usw , \usw
\end{aufza}

\begin{center}\RWbet{Antwort auf diese Einwürfe.}\end{center}

Die Irrthümer, die man der Kirche in diesen Einwürfen vorrückt, sind
\begin{aufzb}
\item entweder gar \RWbet{nicht zur Religion} gehörig, weil sie einen an sich ganz gleichgültigen Gegenstand betreffen; oder
\item zwar zur Religion gehörig, aber \RWbet{nie allgemein herrschend} gewesen, und folglich nicht zur Religion der \RWbet{Katholiken} gehörig; oder endlich
\item auch allgemein herrschend, aber \RWbet{nicht schädlich für ihre Zeit,} sondern vielmehr so nützlich, daß sie eben um dieses Nutzens wegen von Gottes weiser Vorsehung geduldet werden konnten. Nämlich
\end{aufzb}
\begin{aufza}
\item Der Irrthum über die Verfasser einzelner Bücher der Bibel ist ein sehr gleichgültiger Irrthum, oder vielmehr, er war noch nützlich, weil es zur Hebung des Ansehens der Bibel beitrug, wenn man sich vorstellte, die Verfasser eines jeden Buches zu kennen. Und wenn man meinte, daß gewisse Bücher von einem Könige geschrieben sind, so gab dieß den Fürsten der Erde die heilsame Erinnerung, daß sie nicht bloß da wären, um Steuern vom Volke zu beziehen, sondern auch für die Aufklärung und das ewige Heil \Ahat{desselben}{derselben} Sorge zu tragen.
\item Eben so gleichgültig ist es an sich, ob die Verfasser der heil.\ Schrift bei dieser oder jener Stelle an eben das gedacht haben, woran die Kirche denkt, wenn nur das Letztere erbaulich für uns ist. Und in diesem Betrachte muß jeder Unparteiliche eingestehen, daß die Auslegung, welche die Kirche angenommen hat, immer einen viel fruchtbareren Sinn gebe, als es derjenige ist, den man in unserer Zeit für den ursprünglichen erklären will; daß dieser letztere oft sogar irrig und gefährlich ist.
\item Die historischen Irrthümer, die in der Kirche allgemein geglaubt worden sind, waren zum Wenigsten für ihre Zeit erbaulich. So waren \zB\ und sind noch jetzt für Chri\RWSeitenw{38}sten, welche auf einer niedrigen Stufe der Geistesbildung stehen, Ceremonien, die etwas Sinnliches haben, lange Gebete \udgl\ nicht immer unschicklich zu nennen.\RWfootnote{%
	An und für sich sind sie es auch für Gebildete nicht; nur ihr Inhalt, ihre Bedeutung und ihre Einförmigkeit kann machen, daß sie nicht nur unschicklich, sondern selbst ärgerlich werden.}
\item Wenn diese und jene Person, die man unter dem Namen eines Heiligen zur allgemeinen Verehrung und Nachahmung aufstellte, nicht wirklich so fromm und vollkommen gewesen ist, als man sich in der Folge sie dachte: so hatte dieser Irrthum nicht den geringsten Nachtheil; es ist vielmehr, überhaupt zu reden, nützlicher, wenn wir uns von der Vollkommenheit verstorbener Menschen zu gute, als zu schlechte Begriffe bilden.
\item Der Irrthum, daß man das apostolische Glaubensbekenntniß oder das \RWlat{Symbolum Athanasianum}\RWlit{}{SymbolumAthanasianum} für echte Schriften hielt, trug zur leichteren Aufnahme dieser an sich brauchbaren Glaubensbekenntnisse bei. Die Meinung, daß die \RWlat{Canones apostolorum}\RWlit{}{CanonesApostolorum}, die \RWlat{decretales Isidori}\RWlit{}{DecretalesIsidori1} \udgl\  echte Erzeugnisse sind, veranlaßte freilich große Veränderungen in der Disciplin der Kirche, unterstützte die Macht des Papstes \usw ; allein in Glaubenslehren ward durch diese Schriften nichts geändert. Und auch jene Veränderungen scheinen im Ganzen mehr genützt, als geschadet zu haben.
\item Nie haben zwei wirklich \RWbet{allgemeine} Concilien einander in einer \RWbet{Glaubenslehre} widersprochen; obgleich selbst dieses hätte geschehen können, wenn sich nur nachweisen ließe, daß eben dieselbe Meinung, die in der früheren Zeit zuträglich war, in der späteren es zu seyn aufgehört habe. Doch die Geschichte kennt, wie gesagt, kein erweisliches Beispiel hievon. Die Frage, ob Theodoretus und Ibas Irrlehrer seyen, war eine bloß historische Untersuchung. In dem Concilio zu Chalcedon konnten sich diese Männer, da sie zugegen waren, über den Sinn ihrer Schriften auf eine solche Art erklärt haben, daß man sie freisprechen durfte. Das \RWlat{Concilium Constantinopolitanum} verdammte späterhin, da sie bereits todt waren, ihre Schriften, weil es sich zeigte, daß~\RWSeitenw{39}\ diese, bei aller Rechtgläubigkeit ihrer Verfasser, gleichwohl beständige Veranlassung zu Irrungen gaben.
\item Die Meinung, daß der \RWbet{jüngste Tag} schon im Anzuge sey, und vollends der Lehrsatz vom \RWbet{tausendjährigen Reiche} herrschten nie \RWbet{allgemein}; und was die erstere betrifft, so hatte sie bei jener Anwendung, die man im ersten Jahrhunderte von ihr machte, eine unläugbar wohlthätige Wirkung. Oder mußten die Christen nicht um so geneigter werden, alle irdischen Güter und Ehrenstellen, und selbst das Leben zu opfern, wenn sie sich vorstellten, daß ohnedies alle irdische Glückseligkeit sehr bald am Ende seyn, dann aber jenes Reich der ewigen Vergeltung anfangen werde? Daß aber irgend ein \RWbet{rechtgläubiger} Christ im anderen Leben sinnliche Vergnügungen erwartet habe, ist falsch.
\item Der Glaube der ersten Christen, daß sie nicht bloß gegen Menschen, sondern auch gegen die ganze Macht der Hölle selbst zu kämpfen hätten, vereinigt mit der Versicherung Jesu, daß sie in diesem Kampfe gleichwohl gewiß obsiegen würden, hatte entschiedene Vortheile. Er forderte einerseits die Christen zur Anstrengung all ihrer Kräfte auf, und lohnte sie andererseits mit dem erfreulichen Gefühle, das der Gedanke geben muß, nicht bloß über Menschen, sondern selbst über höhere Geister den Sieg davon getragen zu haben. Dieser Glaube lehrte sie endlich den Haß, den die Betrachtung der menschlichen Thorheit und Boshaftigkeit in ihnen erregt hatte, gehörig mäßigen, indem sie den größten Theil der Schuld nicht sowohl diesen Menschen, als vielmehr ihrem Verführer, dem Teufel, zuschrieben. Dabei finden wir nie, daß die ersten Christen, um jene Krankheiten oder andere Uebel, die sie für Wirkungen des Teufels ansahen, zu heben, sich nebst dem Exorcismus nicht auch noch aller derjenigen Mittel, die ihnen nur bekannt waren und zu Gebote standen, bedient hätten. Der Exorcismus, dessen wir uns bei der heiligen Handlung der Taufe noch heut zu Tage bedienen, ist eine symbolische Handlung, die sehr vernünftig ausgelegt wird.\RWfootnote{Von der aber doch zu wünschen wäre, daß ihr erbaulicher Sinn \RWbet{in ihr selbst} deutlich ausgesprochen, und daß es dadurch dem Unverstande mehr erschwert würde, sie zu mißdeuten.
}~\RWSeitenw{40}
\item Was hat es geschadet, wenn man das heil.\ Abendmahl in guter Meinung auch Kindern reichte? Mußte durch diese Sitte vielleicht der Zweck des heil.\ Abendmahles nothwendig in Vergessenheit gerathen? -- Aber wir finden nicht, daß die Kirche je auf den Zweck des Gedächtnisses Jesu bei diesem Mahle vergessen habe. Nebst diesem Einen Zwecke aber hat dieses Mahl noch manche andere; es soll auch unserer Seele verschiedene Gnaden mittheilen, und ihr zur Nahrung dienen.\par
War es gefehlt, zu glauben, daß Gott solche Gnaden auch selbst den Kindern mittheilen werde, denen die Eltern das heil.\ Abendmahl in frommer Meinung reichen? -- Nur erst, als sich bei dieser Austheilung an Kinder verschiedene Ungebührlichkeiten ergaben, that die Kirche recht, die Sitte abzustellen.
\end{aufza}


\RWpar{11}{Vernunftmäßigkeit des vierten Artikels}
Nach dem so glaubwürdigen Berichte der Evangelien hat Jesus wirklich Verschiedenes gesprochen, das am Vernünftigsten ausgelegt wird, wenn wir es so verstehen, \RWbet{daß er dem Gesammtglauben der Kirche die Gabe der Unfehlbarkeit verheißen habe.}
\begin{aufza}
\item Bei Matth. (\RWbibel{Mt}{}{16}{17\,ff}) wird erzählt, daß der Apostel Petrus Jesu das feierliche Bekenntniß abgelegt habe, er halte ihn für den Messias. Hierauf erwiederte ihm der Herr: Selig bist du, Simon, Sohn des Jonas! denn nicht Fleisch und Blut hat dir das geoffenbaret; sondern mein Vater im Himmel. Wahrlich, nicht umsonst habe ich dir den Namen \RWbet{Felsenmann} (Kephas, Petrus) gegeben; sondern du \RWbet{bist} der Fels, auf den ich meine Kirche gründen will; und die Macht des Unterreichs (des Todes oder der Hölle\RWfootnote{%
	Das griechische \RWgriech{<'a|dhs} entspricht nämlich dem hebräischen \Ahat{\RWhebr{+s:'Ol}}{\RWhebr{+s:'ol}} (Scheol), welches sowohl das \RWbet{Grab}, als auch die \RWbet{Unterwelt} bedeutet.}%
) soll sie nicht überwältigen. Hier verspricht also Jesus die Stiftung einer \RWbet{religiösen Gesellschaft, die immer fortdauern soll.} Es fragt sich nun, ob diese stete Fortdauer der christlichen Kirche sich nur auf die \RWbet{gesellschaftliche Verbindung allein}, oder \RWbet{auch auf die Lehre} erstrecke,~\RWSeitenw{41}\ welche in dieser Gesellschaft die herrschende seyn wird. Das Erstere nehmen die Protestanten an; die Katholiken dagegen das Letztere. Da nun das Wesentliche einer religiösen Gesellschaft in ihrer \RWbet{Lehre} bestehet, und da die Fortdauer der bloßen gesellschaftlichen Verbindung, wenn ihre Lehren ausgeartet sind, von keinem Werthe seyn kann, vielmehr nur schädlich ist: so muß man gestehen, daß die Auslegung der Katholiken die vernünftigere sey. Nur einem eitlen Manne hätte darum zu thun seyn können, eine Gesellschaft zu stiften, die ewig fortdauern, und seinen Namen tragen soll, wie irrig übrigens auch ihre Lehren werden. Um aber behaupten zu können, daß die Lehre der Kirche stets fortdauere, muß Eins von Beiden geschehen, entweder ihre Lehre muß \RWbet{immer dieselbe} bleiben, oder, falls sie gewisse Zuwächse erhält, müssen diese doch \RWbet{immer sittliche Zuträglichkeit haben.}
\item Bei Joh. (\RWbibel{Joh}{}{10}{16}) spricht Jesus: \erganf{Ich habe auch noch andere Schafe, die nicht von dieser Herde sind; auch diese muß ich herbeiführen, und sie werden meiner Stimme folgen, und es wird Eine Herde nur, und nur Ein Hirt seyn.} Daß hier von einer religiösen Vereinigung die Rede sey, erhellet aus dem ganzen Zusammenhange. Also ging Jesu Plan dahin, einst alle Menschen zu dem Bekenntnisse Einer und eben derselben Religion zu bringen. Wenn er nun nicht darauf sieht, daß diese Religion zugleich auch eine sittlich zuträgliche sey, ja eine wahre Offenbarung: so nützt er durch dieses Vorhaben der Menschheit nichts. Sehr vernünftig also nehmen die Katholiken an, daß Christus dieser Gesellschaft seinen fortwährenden Beistand angedeihen lassen werde.
\item Bei Matth. (\RWbibel{Mt}{}{28}{20}) spricht Jesus zu seinen Jüngern: \erganf{Ich bin bei euch durch alle Tage bis an das Ende der Zeiten.} Dieses \RWbet{mit euch seyn} war gewiß von keiner \RWbet{körperlichen Gegenwart} zu verstehen; denn Jesus sprach diese Worte nach der Erzählung des Evangelisten, als er so eben im Begriffe war, seine Jünger leiblicher Weise zu verlassen (nämlich gegen Himmel aufzusteigen). Offenbar also versprach er hier einen \RWbet{unsichtbaren geistigen Beistand}; und dieser soll dauern -- bis an das Ende der Zeiten. Ein Ausdruck, der etwas dunkel ist, indem das hebräische Wort~\RWSeitenw{42}\  \RWhebr{`OlAm} (Olam), auf das sich das griechische \RWgriech{a>i`wn} hier beziehet, eine nicht ganz entschiedene, oder vermuthlich mehrerlei Bedeutungen hat. Doch ist die wahrscheinlichste derselben die einer langen Zeit. Die Katholiken nun legen die obige Redensart aus: \RWbet{bis an das Ende der Welt}, \dh\ des menschlichen Geschlechtes. Einige Gegner des Katholicismus aber, \zB\ einige Protestanten übersetzen die Stelle: \RWbet{bis an das Ende des jüdischen Staates.} Da aber die Juden sich mit dem Ende des jüdischen Staates auch schon das Ende des menschlichen Geschlechtes verbunden dachten: so kommen beide Auslegungen ziemlich auf Eines hinaus. Da ferner nicht abzusehen ist, warum unser Herr der Kirche seinen Beistand nur gerade bis zu dem Zeitpuncte des Unterganges des jüdischen Staates hätte versprechen wollen; wenn diese Kirche doch noch länger fortdauern sollte: so muß man wohl gestehen, daß die Auslegung der Katholiken wieder die vernünftigste sey.
\item In folgender Stelle (\RWbibel{Joh}{Joh.}{14}{16}) erklärte Jesus noch deutlicher, worin der Beistand, den er seiner Kirche versprach, eigentlich bestehen werde: \erganf{Ich werde den Vater bitten, und er wird euch einen andern Beistand geben, der immer bei euch bleiben wird, nämlich den \RWbet{Geist der Wahrheit}.} Da hier der Beistand, den Jesus seinen Anhängern verspricht, ein Geist der Wahrheit genannt wird: so schließen die Katholiken wohl sehr vernünftig, daß sein Geschäft seyn werde, die Erkenntniß der Wahrheit zu verbreiten.
\item Und dieß wird auch (\Ahat{\RWbibel{Joh}{Joh.}{16}{13}}{16,12.}) mit ganz ausdrücklichen Worten gesagt: \erganf{Wenn jener Geist der Wahrheit kommen wird: so wird er euch \RWbet{zu aller Wahrheit führen}.} Die Protestanten erklären diese und die vorige Stelle von einem Beistande, der \RWbet{nur den Aposteln} versprochen worden sey. Allein, wenn es Jesus nöthig gefunden, schon den Aposteln, die er doch selbst drei Jahre unterrichtet hatte, einen eigenen Beistand zu versprechen, der ihnen über Alles, was sie damals noch nicht zu fassen fähig waren, Aufschluß ertheile: wie will man glauben, daß dieser Beistand den späteren Lehrern der Kirche nicht mehr nothwendig gewesen sey? Ist es wohl wahrscheinlich, zu glauben, daß die Apostel das~\RWSeitenw{43}\ ganze Maß des Wissens, dessen der Mensch empfänglich ist, erschöpft haben werden? daß es nach ihrem Tode so ganz und gar nichts mehr gegeben habe und geben werde, worüber den Menschen ein Aufschluß, den das Zeitalter der Apostel noch nicht zu fassen vermochte, nützlich wäre? Oder selbst wenn die Apostel schon Alles erschöpft: ist es wohl wahrscheinlich, daß ihre Nachkommen im Lehramt ohne Gottes Beistand vermögend seyn sollten, auch nur das schon Empfangene getreu und unverfälscht \RWbet{aufzubewahren?} Wenn also die Worte Jesu der Deutung auf einen \RWbet{fortdauernden Beistand} nur nicht ausdrücklich widersprechen: so ist es schon vernünftig, sie in diesem Sinne zu nehmen. -- Aber freilich durfte die Kirche diesen Beistand nicht auf ein jedes einzelne Mitglied, sondern nur auf den \RWbet{Glauben der Gesammtheit} beziehen; denn eine persönliche Unfehlbarkeit im Glauben dürfte vielleicht nicht einmal den Aposteln selbst versprochen gewesen seyn. So scheint es aus \RWbibel{Lk}{Luk.}{22}{32}\ \RWbibel{Joh}{Joh.}{21}{23}\ \RWbibel{Apg}{Apostelg.}{15}{1\,ff}, besonders \RWbibel{Apg}{}{15}{19}\ \RWbibel{1\,Kor}{1\,Kor.}{7}{40}\ \uma\  Stellen hervorzugehen. -- Wenn aber die Kirche nur den Gesammtglauben ihrer Glieder für unfehlbar erklärt: so thut sie im Grunde nichts Anderes, als daß sie bei jedem entstandenen Streite diejenige Meinung, in der sich Alle vereinigen, und die sie folglich auch ohne allen verheißenen Beistand doch als die \RWbet{wahrscheinlichste} annehmen müßte, gestützt auf diese Aussprüche Jesu, für ganz gewiß erklärt. Ist es nun vernünftig, eine Meinung, die man nun auf jeden Fall annehmen muß, lieber für sicher, als für noch ungewiß zu halten? --

\begin{RWanm} Aus dieser letzteren Bemerkung ist zu ersehen, daß der Lehrsatz der katholischen Kirche von der \RWbet{Unfehlbarkeit ihres Gesammtglaubens} selbst in dem Falle \RWbet{vernunftmäßig} wäre, wenn Jesus der Kirche die Gabe der Unfehlbarkeit wirklich nicht versprochen hätte; ja wenn auch gar keine Verheißungen Jesu, die sich so auslegen lassen, zu finden wären. Nun aber, da sich dergleichen Stellen wirklich vorfinden, kann man behaupten, daß ihr Daseyn, selbst in dem Falle, wenn sie auf einem bloßen Mißverstande der ersten Zuhörer Jesu, oder worauf sonst immer beruhen sollten, uns einen hinlänglichen Beweis des Willens Gottes gebe, daß wir an die Unfehlbarkeit der Kirche glau\RWSeitenw{44}ben sollen. Denn da dieser Lehrsatz (wie wir bald sehen werden) die höchste sittliche Zuträglichkeit hat; so würde sich schon in der sonderbaren Fügung, durch die diese Stellen in unsere Evangelien gekommen sind, die Absicht Gottes aussprechen, daß wir aus ihnen den Glauben an die Unfehlbarkeit der Kirche schöpfen sollen.\end{RWanm}
\end{aufza}


\RWpar{12}{Vernunftmäßigkeit des fünften Artikels}
\begin{aufza}
\item Man hat behauptet, daß ein so \RWbet{unbedingtes} Versprechen der Erleuchtung, als es dasjenige seyn soll, dessen sich die katholische Kirche rühmt, der Weisheit Gottes zuwider sey, und zwar
\begin{aufzb}
\item weil er hiedurch die Menschen zum Leichtsinn, zur Trägheit und zur Vermessenheit verleitet; indem sie sich, stützend auf dieses Versprechen, ohne vorhergegangene Ueberlegung, nach bloßer Leidenschaft \udgl\  entscheiden, und ihre Entscheidungen dann doch für Aussprüche des Geistes Gottes ausgeben würden;
\item weil es ferner ungerecht ist, daß Menschen, die ihre Pflicht und Schuldigkeit nicht thun, noch oben darein dadurch belohnt werden sollen, daß sie der heil.\ Geist erleuchtet;
\item weil es endlich der Ehre des göttlichen Geistes zuwider ist, daß er genöthigt seyn soll, den Menschen beizustehen, wie sie auch immer sich betragen; und daß somit öfters auch die unwissendsten und lasterhaftesten Menschen ein Werkzeug (Organ), durch welches der heil.\ Geist zu uns spricht, abgeben sollen.
\end{aufzb}
\item Ich antworte hierauf:
\begin{aufzb}
\item Statt des befürchteten Leichtsinnes, der Trägheit und Vermessenheit muß dieß Versprechen vielmehr, bei jedem \RWbet{Vernünftigen} wenigstens, gerade die entgegengesetzte Wirkung erzeugen, wie wir dieß bei Betrachtung des sittlichen Nutzens dieser Lehre sehen werden.
\item Menschen, die ihre Pflicht und Schuldigkeit nicht gethan haben, werden nicht belohnt, wenn sie nichts desto weniger der göttliche Geist erleuchtet; denn diese Erleuchtung~\RWSeitenw{45}\ ist dann kein Vortheil für sie; gereicht vielmehr ihnen zu um so größerer Verdammniß.
\item Es kann der Ehre des göttlichen Geistes unmöglich Abbruch thun, wenn er
\begin{aufzc}
\item den Menschen, wie sie auch immer sich betragen, beisteht; denn er thut dieses nicht gezwungen, indem das Christenthum jenes Versprechen Gottes, zu Folge dessen er dieß thut, als ein ganz freiwilliges darstellt.
\item Nur bei sehr ungebildeten Menschen könnte die Ehrwürdigkeit des heiligen Geistes verlieren wegen der Unwürdigkeit derjenigen, durch die er zu uns spricht. \RWbet{Vernünftige} werden das Werkzeug und den, der sich desselben bedient, unterscheiden; werden die Weisheit Gottes nur um so mehr bewundern, wenn es sich zeigt, daß er sich auch der Thorheiten und der Laster der Menschen zur Ausführung seiner wohlthätigen Zwecke zu bedienen wisse.
\end{aufzc}
\end{aufzb}
\end{aufza}


\RWpar{13}{Vernunftmäßigkeit des sechsten Artikels}
\begin{aufza}
\item Daß sich der göttliche Beistand, den der Gesammtglaube der Katholiken erfährt, nur auf solche Meinungen erstrecke, die für den Zweck der Tugend und Glückseligkeit \RWbet{nicht gleichgültig} sind, ist sehr vernunftmäßig. Denn das Gegentheil, daß dieser Beistand sich auch auf Meinungen erstrecken sollte, die gleichgültig sind, oder doch wenigstens für die Gesammtheit kein Interesse haben, würde nicht nur der Weisheit Gottes widersprechen; sondern es ließe sich auch nicht mit denjenigen Begriffen von einer Offenbarung und von den Kennzeichen derselben, die wir (1. Hptthl. 4. Hptstck.) kennen gelernet haben, vereinigen. Zu Folge dieser Begriffe nämlich kann eine Lehre nur dann als Gottes Offenbarung erkannt werden, wenn sie das Merkmal der inneren Vortrefflichkeit hat, also auf unsere Tugend und Glückseligkeit einen wohlthätigen Einfluß verspricht.
\item Ganz der Vernunft gemäß ist es nun, wenn die Gelehrten aus dem Gebiete der kirchlichen Unfehlbarkeit ausschließen~\RWSeitenw{46}
\begin{aufzb}
\item alle \RWbet{Disciplinar-Vorschriften} der Kirche. Wenn die katholische Kirche selbst eine gewisse Vorschrift für eine Disciplinarvorschrift erkläret, \dh\ wenn sie gestehet, daß diese Vorschrift nicht von Gott selbst, (durch den Mund Jesu \udgl ); sondern nur von ihren Vorstehern herrühre: so will sie durch diesen Gegensatz nichts Anderes anzeigen, als daß sie in dieser Vorschrift sich keiner Unfehlbarkeit rühme. Und nun versteht es sich, daß man solche Lehren einer religiösen Gesellschaft, die von ihr selbst nicht für geoffenbart ausgegeben werden, auch nicht dafür annehmen dürfe. Dieß kann um desto weniger bei den Disciplinar-Vorschriften der katholischen Kirche geschehen, da diese größtentheils nicht durch den Gemeinwillen Aller, sondern nur durch den Beschluß Eines oder einiger Männer (\zB\ des Papstes und seiner Räthe, oder höchstens durch den Beschluß sämtlicher auf einem Concilium versammelter Bischöfe) eingeführt worden sind. Sie müssen also aus dem Gebiete der kirchlichen Unfehlbarkeit schon darum ausgeschlossen werden, weil sich in ihnen gar nicht der Sinn und Wille Aller ausspricht.\par
\RWbet{Einwurf.} Aber es wäre doch besser, wenn Gott den Vorstehern der Kirche in der Abfassung aller ihrer Vorschriften die Gabe der Unfehlbarkeit verheißen und wirklich mitgetheilt hätte. Durch diese Verheißung hätten die Vorsteher der Kirche sich um so strenger verpflichtet gefühlt, bei der Abfassung ihrer Gebote nicht auf die Stimme der Leidenschaft, des Eigennutzes \udgl , sondern nur auf das allgemeine Wohl der christlichen Gemeine zu achten, und nicht unwürdige Werkzeuge des heil.\ Geistes zu werden. Die Gläubigen aber würden diese Gebote nun um so gewissenhafter und um so freudiger befolgen, je sicherer sie jetzt von ihrer Zweckmäßigkeit überzeugt wären. Durch die Erfüllung dieser Verheißung endlich wäre es geschehen, daß alle Verordnungen, die in der Kirche Gottes jemals bestanden haben, und noch bestehen, die größte Zweckmäßigkeit erhalten hätten; wodurch das Wohl der Kirche und der ganzen Menschheit ungemein viel gewonnen haben würde.\par
\RWbet{Antwort.} Wenn dieser Einwurf etwas bewiese, so würde er \RWbet{zu viel} beweisen; denn aus ganz gleichen Grün\RWSeitenw{47}den könnte man auch verlangen, daß Gott nicht nur die geistlichen, sondern auch alle \RWbet{weltlichen} Obrigkeiten, ja \RWbet{alle Gebieter und Herren ohne Ausnahme} in der Abfassung ihrer Befehle für unfehlbar erkläre, und auch unfehlbar mache. Wer sollte nicht einsehen, daß wir zu einer solchen Forderung nicht berechtiget wären? Wir wissen nämlich so ganz und gar nicht, wie viele Anstalten von Seite Gottes, wie viele Eingriffe in den natürlichen Lauf der Dinge \usw\ dazu erforderlich wären, wenn es dahin gebracht werden sollte, daß irgend eine menschliche Obrigkeit in allen ihren Geboten und Verordnungen unfehlbar werde; \dh\ immer nur dasjenige verfüge, was in diesen Umständen wirklich das Zweckmäßigste ist. Sicher wird hiezu ungleich mehr erfordert, als dazu, um es nur dahin zu bringen, daß die ganze \RWbet{christliche Gemeine} nie in einem und eben demselben Irrthume zusammenstimme. Ist aber die Ertheilung dieser Gabe der Unfehlbarkeit nicht vereinbarlich mit Gottes Weisheit: so war es auch nicht zweckmäßig, sie nur zu \RWbet{verheißen}, zumal da die in dem Einwurfe erwähnten Vortheile jener Verheißung größtentheils auch ohne sie erreichbar sind; dagegen es gewisse selbst uns bemerkbare Nachtheile gibt, die sie hervorgebracht hätte. Denn 
\begin{aufzc}
\item die Verpflichtung der geistlichen Vorsteher unserer Kirche zur möglichsten Vorsicht bei Abfassung ihrer Gesetze besteht wohl auch schon in einem gleichen Grade bei einer nur \RWbet{bedingten} Verheißung des göttlichen Beistandes, die in der That gelehret wird.
\item Und eben, weil diese gelehrt wird, so müssen die Gläubigen fromm voraussetzen, daß dieser Beistand ihren Vorstehern wirklich zu Theil geworden sey, wenn nicht die offenbare Zweckwidrigkeit einer Verordnung sie für einen bestimmten Fall vom Gegentheile überzeugt. Schon diese fromme Voraussetzung aber wird in den meisten Fällen hinreichen, sie zu einer \RWbet{gewissenhaften,} ja auch wirklich \RWbet{freudigen} Erfüllung jener Vorschriften zu vermögen. --
\end{aufzc}
Wäre dagegen diese Verheißung \RWbet{unbedingt} ertheilt worden, so hätte dieß folgende Nachtheile:~\RWSeitenw{48}
\begin{aufzc}
\item Die Vorsteher der Kirche wären zum Stolz und Uebermuth verleitet worden.
\item Wenn der Gebildete schon jetzt zuweilen Ursache hat, zu klagen, daß sich die Vorsteher der Kirche (aus Anhänglichkeit an das Alte, aus Trägheit oder aus was immer für anderen Gründen) schwer bewegen lassen, eine von ihren Vorfahren überkommene Einrichtung nach dem Bedürfnisse der Zeiten abzuändern: was würde erst der Fall seyn, wenn man alle diese Anordnungen für unfehlbar ansehen würde! Wahr ist es freilich, daß man auch dann noch berechtiget wäre, eine Veränderung mit ihnen vorzunehmen, weil ja die Umstände, auf denen ihre Zweckmäßigkeit beruhet, sich verändert haben könnten; aber gewiß würden sich die Vorsteher der Kirche zu solchen Aenderungen nun um so schwerer entschließen. Denn nun würden sie verlangen, daß man ihnen zeige, was für eine plötzliche Veränderung sich wohl so eben zugetragen habe, durch die es geschehen sey, daß die von uns bestrittene Anstalt, die bis auf diese Stunde, weil der Geist Gottes ihre Aufhebung noch nicht veranlasset hatte, gewiß zweckmäßig seyn muß, von jetzt an es zu seyn aufhöre? Welch eine harte Aufgabe! --
\item Da es der kirchlichen Vorschriften sehr viele gibt, (und billig auch geben soll); und da, wenn diese Vorschriften alle für unfehlbar gelten, alle als Bestandtheile der göttlich geoffenbarten Lehre angesehen seyn wollten, und somit erst ihre Vernunftmäßigkeit und ihr sittlicher Nutzen untersucht werden müßte, bevor man das Christenthum als Gottes Offenbarung anerkennen dürfte: so sieht ein Jeder, wie das Geschäft dieser Untersuchung hiedurch in das Unendliche verlängert würde.
\item Vollends wenn Jemand nachweisen sollte, daß nicht nur alle \RWbet{jetzt bestehenden} kirchlichen Anordnungen \RWbet{für die Gegenwart,} sondern auch alle \RWbet{ehemals bestandenen für ihre Zeit} den Charakter der Vernunftmäßigkeit und sittlichen Zuträglichkeit gehabt hätten.~\RWSeitenw{49}
\end{aufzc}
\item Alle \RWbet{bloß historischen Behauptungen} mit Ausnahme jener, die in der Bibel vorkommen.\par
Daß man bloß historische Behauptungen aus dem Gebiete der kirchlichen Unfehlbarkeit ausschließt, wird Niemand anstößig finden. Wohl aber könnte es Manchen befremden, daß man die \RWbet{biblischen Erzählungen alle} für unfehlbar erklärt, indem sich doch schwer beweisen ließe, daß wirklich alle Facta, die uns die Bücher des alten sowohl als neuen Bundes erzählen, bis auf die kleinsten Umstände, die oft durch Unachtsamkeit der Abschreiber oder Uebersetzer eine Veränderung erfahren konnten, \zB\ Namen und Zahlen \udgl , ihre unfehlbare Richtigkeit haben. Dieser Anstoß verschwindet aber, sobald wir uns des eigentlichen Zweckes erinnern, aus welchem die Kirche auf die Annahme der historischen Richtigkeit dieser Erzählungen dringt. Sie bedient sich nämlich jener Erzählungen bloß, um eine Menge \RWbet{sittlicher} und \RWbet{religiöser} Lehren aus ihnen \RWbet{herzuleiten}, oder doch an ihnen \RWbet{anschaulich zu machen}. Weil nun die Gläubigen diese religiösen Lehren selbst verwerfen würden, sobald sie die Richtigkeit jener Erzählungen bezweifeln dürften: so dringt die Kirche auch auf die unbezweifelte Annahme der letzteren. Wer aber einsichtsvoll genug ist, um zu begreifen, daß alle religiösen Lehren der Kirche auch dann noch beibehalten werden müssen, wenn diese oder jene biblische Erzählung, auf welche sie von ihr gegründet worden sind, unrichtig seyn sollte, für den hat die so eben besprochene Behauptung der Richtigkeit dieser Erzählungen \RWbet{keine religiöse Wichtigkeit} mehr; sie gehört für ihn nicht mehr zum Inhalte der Religion; er kann es also ruhig dahin gestellt seyn lassen, ob diese Erzählungen alle wirklich ganz in der Wahrheit gegründet sind oder nicht.
\begin{RWanm} Dieser von uns hier aufgestellten Meinung tritt der gelehrtere Theil der Katholiken (die Theologen \uA ) von Jahr zu Jahr näher. Auch sie geben jetzt fast durchgängig zu, daß es gewisse Erzählungen in der Bibel gebe, die nicht zur Religion gehören, und in Betreff deren ein Irrthum obwalten könne; \zB\ in den Jahrszahlen \udgl\  Hieronymus sagte dieß schon seinen Zeitgenossen.~\RWSeitenw{50}
\end{RWanm}
\item Alle Meinungen, die ein bloß \RWbet{wissenschaftliches Interesse haben.} Die Richtigkeit dieser Ausnahme ist für sich selbst klar.
\end{aufzb}
\end{aufza}


\RWpar{14}{Vernunftmäßigkeit des siebenten Artikels}
Die Meinung, daß man in der katholischen Kirche auch heute noch nichts Mehreres lehre und glaube, als was zu der Apostel Zeiten gelehret und geglaubet wurde, ist freilich nicht ganz richtig. Doch dieser Vorstellung hängt nur die \RWbet{ungebildete} Menge der Katholiken an, für welche sie zuträglich ist, wie wir bald sehen werden. Von dem \RWbet{gebildeteren} Theile dagegen wird zugestanden, daß der katholische Lehrbegriff einer gewissen \RWbet{Ausbildung} und \RWbet{Vervollkommnung} fähig sey. Und das ist sehr vernünftig; denn das ist nicht nur
\begin{aufzb}
\item jenem Gesetze der \RWbet{stufenweisen Entwicklung,} das wir sonst allenthalben in der Natur bemerken, angemessen; sondern dieß fordert auch
\item das stete \RWbet{Fortschreiten des menschlichen Geschlechtes.} Das Geschlecht der Menschen nämlich schreitet, wie in so manchen anderen Stücken, so vornehmlich und auf das Unwidersprechlichste in seinen \RWbet{Begriffen} und \RWbet{Erkenntnissen} stets weiter fort. Hiedurch aber wird es von Zeit zu Zeit einer \RWbet{ausführlicheren Belehrung über die Gegenstände der Religion} Beides \RWbet{empfänglich} sowohl, als auch \RWbet{bedürftig}. Eine Religion also, die gar keines Zuwachses, gar keiner weiteren Ausbildung und Vervollkommnung fähig wäre, würde eben darum auch nicht für alle Zeiten die \RWbet{zuträglichste}, und folglich nicht Gottes wahre Offenbarung seyn können.
\end{aufzb}

\RWpar{15}{Vernunftmäßigkeit des achten Artikels}
\begin{aufza}
\item Der Umstand, daß uns die Katholiken \RWbet{kein einziges Buch} aufweisen können oder wollen, das nach dem Urtheile \RWbet{Aller} die Lehren ihres Glaubens vollständig enthielte,~\RWSeitenw{51}ist ein sehr wesentlicher Umstand. Wäre das Gegentheil, würden die Katholiken alle einstimmig auf Ein und dasselbe Buch hinweisen, in welchem ihr Lehrbegriff vollständig aufgezeichnet sey: so würden sie eben hiedurch gestehen, daß es \RWbet{kein Fortschreiten} in ihrem Glauben gebe, wenn anders dieses Buch ein vollständiges Verzeichniß ihrer Lehren nicht nur für diese Zeit, sondern für \RWbet{alle Zeiten} seyn sollte. Würden sie aber behaupten, daß dieses Buch nur für die Gegenwart Vollständigkeit besitze: so würden sie hiedurch die Lehre von dem Wachsthume ihres Glaubens, welche nur dem gelehrteren Theile des Volkes, nicht aber der großen Menge der Ungebildeten bekannt seyn soll, zu offen darstellen.
\item Die Behauptung, daß
\begin{aufzb}
\item die \RWbet{heilige Schrift} des alten sowohl als des neuen Bundes \RWbet{durch Gottes Eingebung} geschrieben sey, ist ganz vernunftmäßig, wenn man hierunter nichts Anderes versteht, als daß diese Bücher in einem \RWbet{ganz ausgezeichneten Grade} tauglich sind, uns zu \RWbet{belehren} und zu \RWbet{erbauen}; und daß sie diese Tauglichkeit durch den Zusammenfluß so vieler und so zufälliger Umstände erhielten, daß man nicht umhin kann, hierin die Absicht Gottes zu erkennen, daß wir uns dieser Bücher auch wirklich zu unserer Belehrung und Erbauung bedienen. Die überaus große Brauchbarkeit dieser Bücher hängt nicht nur von den \RWbet{Sachen} ab, die darin abgehandelt werden; sondern auch von der Art, wie \RWbet{über diese Sachen gesprochen wird;} so zwar, daß viele tausend Stellen bloß durch die zufällige \RWbet{Wahl des Ausdruckes}, zu welchem Gott die heil.\ Schriftsteller veranlasset hat, einen Sinn erhalten, der so belehrend und erbaulich für uns ist. Wir können also mit allem Rechte sagen, daß in der so eben erklärten Bedeutung nicht nur die Sachen, sondern auch \RWbet{selbst die Worte} den Verfassern der Bibel eingegeben worden sind. Indessen ist nicht zu läugnen, daß die meisten Katholiken mit dem Worte \RWbet{Eingebung} einen viel engeren Sinn verbinden. Sie stellen sich nämlich vor, daß die Verfasser der heil.\ Schrift, \RWbet{ohne ihr eigenes Nachdenken,} durch eine~\RWSeitenw{52}\ \RWbet{unmittelbare} Einwirkung Gottes inne geworden wären, was, und mit welchen Worten sie es zu schreiben hätten; sie stellen sich vor, daß diese Schriftsteller Alles, was sie geschrieben, \RWbet{in eben dem Sinne}, in welchem die Kirche es auslegt, auch selbst genommen hätten. Das dürften nun wohl Irrthümer seyn, aber Irrthümer, die für den Gebildeten gleichgültig sind, und also für ihn gar nicht zum Inhalte der Religion gehören.
\item Der Katholik sagt weiter, daß dieser außerordentliche Beistand, der den Verfassern der Bibel zu Theil geworden ist, \RWbet{sich nur auf das}, aber auch auf \RWbet{Alles} das erstrecke, \RWbet{was die Religion} betrifft; so zwar, daß alle religiösen Ansichten, welche in diesen Büchern aufgestellt werden, \RWbet{wahre göttliche Offenbarungen} sind; jedoch nur dann, wenn Alles auch nach dem Sinne aufgefaßt wird, in dem die \RWbet{Kirche} es auslegt. Diese Behauptung ist durch den einschränkenden Beisatz, den sie am Ende erhält, ganz folgerecht und vernünftig. Denn wenn man die Bibel überall nur nach der Auslegung versteht, die ihr die Kirche gibt: so stimmt sie durchgängig mit den religiösen Lehrsätzen der Kirche selbst überein. Wollte man aber diesen Beisatz weglassen, und von dem \RWbet{ursprünglichen} Sinne, der jeder biblischen Stelle zu Grunde liegt, behaupten, daß er ganz übereinstimmend mit den Lehren des Katholicismus sey: so würde man etwas behaupten, das wenigstens schon darum schwer zu beweisen wäre, weil es sehr schwer ist, diesen ursprünglichen Sinn einer jeden Stelle zu erkennen.
\item Sehr wahr ist auch, daß die in der Kirche übliche lateinische Uebersetzung, die unter dem Namen der \RWlat{Vulgata} bekannt ist, für den Zweck der Erbauung die allerbrauchbarste sey; denn wirklich gibt es nicht wenige Stellen, die in der \RWlat{Vulgata} einen viel lehrreicheren und erbaulicheren Sinn geben, als es im hebräischen oder griechischen Grundtexte der Fall ist.
\end{aufzb}
\item Daß die Verhandlungen jener \RWbet{neunzehn allgemeinen Kirchenräthe} als echte Erkenntnißquelle des Katholicismus anzusehen wären, ist eine Behauptung, die nichts~\RWSeitenw{53}\ Widersprechendes enthält, weil die religiösen Ansichten, welche in diesen Verhandlungen aufgestellt wurden, mit dem jetzigen Glauben der katholischen Kirche wirklich auf's Beste übereinstimmen.
\end{aufza}

\RWpar{16}{Sittlicher Nutzen des ersten Artikels}
Ich habe schon gesagt, daß ich die \RWbet{sittliche Zuträglichkeit} dieses Artikels hier in Vereinigung mit seiner \RWbet{Vernunftmäßigkeit} betrachten wolle; und darum werde ich jetzt
\begin{aufza}
\item vor Allem dem Vorwurfe vorbeugen müssen, daß die Behauptungen dieses Artikels im Widerspruche stehen mit dem, was der nächstfolgende aussagt. In diesem heißt es nämlich, daß nur Dasjenige göttlich geoffenbart sey, was \RWbet{alle Katholiken }glauben; hier aber wird befohlen, daß man auch das schon annehmen solle, was nur die \RWbet{Lehrer} der Kirche, ja wohl gar nur die \RWbet{Bischöfe} derselben, oder der \RWbet{Papst allein} zu glauben vorstellen. Sollte dieß nicht ein Widerspruch seyn? -- Bei einer näheren Betrachtung zeigt sich, daß hier nicht der geringste Widerspruch obwalte. Denn wenn im ersten Artikel gesagt wird, daß der Katholik verbunden sey, dieß oder jenes zu glauben, \zB\ das, was der Papst entschieden hat, \udgl , so wird doch keineswegs hiemit gesagt, daß er es anzusehen habe als eine \RWbet{göttlich geoffenbarte Wahrheit}. Es wird hier lediglich von der \RWbet{Pflicht}, daran zu glauben, gesprochen. Daß man aber verpflichtet seyn könne, etwas zu glauben, auch ohne noch zu wissen, ob es \RWbet{objectiv wahr} sey, um wie viel weniger, ob es zum Inhalte einer \RWbet{göttlichen Offenbarung} gehöre, wird Niemand in Abrede stellen.
\item Da übrigens keine dieser Regeln von allen Katholiken einstimmig angenommen wird: so sind sie eben deßhalb auch nicht als Lehrsätze des Katholicismus anzusehen; und folglich könnten sie immerhin etwas Vernunftwidriges enthalten, ohne daß hieraus für den Lehrbegriff der katholischen Kirche ein Nachtheil entspränge.
\item Wie wir (im 1.~Hptthl. 1.~Hptstck. \RWparnr{37}) gesehen; so lautet eine Regel des Glaubens, welche die Vernunft selbst~\RWSeitenw{54}\ aufstellt: \RWbet{Suche dich immer von derjenigen Ansicht der Dinge zu überzeugen, die unter allen, von denen es dir möglich ist, dich zu überreden, die allerzuträglichste für die Beförderung deiner Tugend und Glückseligkeit ist.} -- Es fragt sich nun, warum das Christenthum nicht lieber diese Regel ausdrücklich aufgestellt hat? -- Hierauf erwiedere ich: weil diese Regel zwar dunkel geahnet und befolget wird von jedem besseren Menschen; um aber deutlich erkannt und richtig angewandt zu werden, eine geübtere Urtheilskraft voraussetzt, als sie die größere Menge der Menschen noch heut zu Tage hat.
\item Obgleich aber die Regeln dieses Artikels nicht als eigentliche Lehrsätze des Katholicismus angesehen werden dürfen; so kann uns ihr Daseyn doch manchen sittlichen Nutzen gewähren:
\begin{aufzb}
\item Sie erinnern uns nämlich daran, worauf wir \RWbet{merken} sollen, um auf \RWbet{ersprießliche Ansichten} geleitet zu werden, nämlich nicht bloß auf das, was von der Kirche allgemein geglaubt wird, sondern auch auf dasjenige, was nur von einem Theile ihrer Gläubigen, \zB\ nur von dem Lehrkörper, oder nur von den Bischöfen, ja auch nur vom Papste allein vorgetragen wird. Und wenn wir dieß thun, so können wir in der That manche ersprießliche Ansicht erhalten, auf die wir ohne dieß Mittel nie verfallen wären.
\item Verlangt der Papst, verlangen die Bischöfe, verlangt der ganze Lehrstand der Kirche, daß wir eine gewisse Meinung annehmen sollen: so legt uns dieses Gebot (wie jedes andere, das von einer rechtmäßigen Obrigkeit kommt) die Verbindlichkeit auf, ihm zu gehorchen -- so lange der Schaden nicht \RWbet{offenbar} den Nutzen \RWbet{überwiegt,} \dh\ in gegenwärtigem Falle, so lange wir nicht \RWbet{deutlich} einsehen, daß die aufgestellte Meinung unserer Tugend und Glückseligkeit eher nachtheilig als zuträglich wäre. Obgleich es auch schon ohne dieß Gebot unsere Pflicht wäre, eine uns zuträgliche Meinung zu glauben: so wird diese Verbindlichkeit doch durch das Gebot verstärket.
Zumal, da wir im Voraus vermuthen können, daß eine~\RWSeitenw{55}\ Ansicht, auf deren Annahme so viele und so verständige Personen dringen, zuträglich seyn müsse.
\end{aufzb}
\end{aufza}

\RWpar{17}{Sittlicher Nutzen des zweiten Artikels}
Der Lehrsatz, daß der Gesammtglaube der Katholiken in \RWbet{religiösen} Gegenständen eine wahre göttliche Offenbarung sey, dient uns zu einem \RWbet{Kennzeichen}, vermittelst dessen wir entnehmen können, welche religiöse Meinungen zur Zeit die \RWbet{Gewißheit göttlicher Offenbarungen} haben, während die übrigen nur so viel Wahrscheinlichkeit besitzen, als ihre Gründe gelten.\par
Dieß Kennzeichen göttlich geoffenbarter Lehren hat nun 
\begin{aufza}
\item zuvörderst den Vorzug, \RWbet{daß es sich selbst leicht und mit Sicherheit erkennen läßt.} Ein taugliches Kennzeichen nämlich muß selbst leicht und mit Sicherheit zu erkennen seyn. Nichts läßt sich nun leichter und mit größerer Sicherheit ausmitteln, als ob eine gewisse Meinung von allen, oder doch fast allen Katholiken \RWbet{äußerlich} angenommen werde (und nur darauf allein kommt es hier an, nicht was sie innerlich glauben). Ein jedes andere Kennzeichen würde weit schwerer und unsicherer seyn. Z.\,B.\ wenn als Kennzeichen einer geoffenbarten Lehre der Umstand angegeben würde, daß sie in einem gewissen \RWbet{Buche} (etwa der Bibel) anzutreffen seyn müsse: so würde der Zweifel entstehen, ob wir bei Auslegung dieses Buches auch wohl den \RWbet{rechten Sinn} treffen. Wenn als Kennzeichen einer geoffenbarten Meinung der Umstand dienen sollte, daß sie die \RWbet{innerlich} (aus inneren Gründen) \RWbet{wahrscheinlichste} aus allen übrigen seyn soll: wie schwer wäre dieses nicht auszumitteln! \udgl\,m.
\item Dieß Kennzeichen ist ferner auch das \RWbet{beruhigendste}, das uns gegeben werden kann; so daß unsere Zweifelsucht sich gegen jedes andere weit mehr, als gegen dieses, sträuben würde. Denn die Bemerkung, daß eine gewisse religiöse Meinung von so viel tausend Menschen, als die katholische Lehre jetzt in ihrem Schoße zählt, worunter sich auch so viele Gelehrte befinden, einstimmig angenommen werde, muß ihr schon~\RWSeitenw{56}\ an sich selbst einen sehr hohen Grad von Wahrscheinlichkeit ertheilen; und wer nicht einmal glauben wollte, was doch so Viele glauben, würde gewiß um so weniger etwas Anderes glaubwürdig finden.
\item Durch die Aufstellung dieses Kennzeichens wird auch die \RWbet{Würde der menschlichen Vernunft geehret.} Hier nämlich heißt es recht eigentlich: Das Urtheil Aller ist das Urtheil Gottes! (\RWlat{Vox populi, vox Dei!}) Die menschliche Vernunft ist also, durch Gottes Leitung, vermögend, auf jede ihr nützliche Wahrheit nach und nach zu kommen! \usw\
\item Durch Aufstellung dieses Kennzeichens wird auch die \RWbet{Gleichheit aller Menschen} in einem der wesentlichsten Stücke, nämlich in dem Rechte, zu urtheilen über dasjenige, was zu seiner Beurtheilung bloßer Vernunft bedarf, anschaulich dargestellt. Zu Folge dieser Lehre nämlich sind wir gehalten, das Urtheil des Reichen und Vornehmen nicht mehr, als das des Armen und Geringen; selbst jenes des Gelehrten nicht mehr, als das des Ungelehrten (vorausgesetzt, daß es sich nicht um eine Sache handelt, die zu ihrer Beurtheilung durchaus Gelehrsamkeit erfordert) gelten zu lassen.
\item Und was insonderheit unsere \RWbet{Glaubensgenossen} belangt: so lernen wir diese nicht nur uns gleich achten, sondern wir fühlen uns auch auf das Innigste \RWbet{mit ihnen verbunden, und ihnen verpflichtet}. Denn uns gemeinschaftlich, und jedem Einzelnen aus uns nur um der Uebrigen willen, und durch sie, hat sich Gott geoffenbaret.
\item Durch Aufstellung dieses Kennzeichens erhält wenigstens jeder \RWbet{gebildetere} Katholik einen erhöhten Beweggrund, sich zu bestreben, daß die religiösen Ansichten unter den Katholiken \RWbet{immer gleichförmiger} werden. Denn wenn es ihm gelingt, diese Gleichförmigkeit zu vermehren: so hat er zum Wachsthum und zur Vervollkommnung der katholischen Religion selbst etwas beigetragen; indem er zwar nicht \RWbet{neue} Glaubenslehren (\RWlat{novos articulos fidei condidit}) aufgebracht, wohl aber den schon bekannten eine \RWbet{genauere Bestimmung} verschafft hat. (\RWlat{Fidem explicitam ampliavit.})~\RWSeitenw{57}
\item Je größer aber die Gleichförmigkeit in den religiösen Gesinnungen bei einer Gesellschaft ist, um desto mehr \RWbet{Eintracht und Liebe} wird auch in derselben herrschen, um desto mehr Vergnügen wird man am geselligen Umgange und an vernünftigen religiösen Gesprächen finden, \usw\ So kann denn also Jeder in der katholischen Kirche mehr gute Beispiele der Eintracht zu sehen und für sich selbst mehr Aufmunterung zu dieser Tugend zu finden hoffen.
\item Durch diesen Lehrsatz erscheint uns endlich auch \RWbet{Gottes Weisheit und gütige Fürsorge} für das menschliche Geschlecht in dem \RWbet{hellsten Lichte.} Der Weisheit Gottes ist es gemäß, daß er bei jedem Zwecke, zu dessen Ausführung auch wir Menschen etwas beitragen können, erst \RWbet{uns} verhalte, daß wir das Unsrige thun, dann aber dasjenige, was wir aus Schwachheit oder Trägheit unterließen, zum Besten des Ganzen durch \RWbet{seine} Dazwischenkunft ersetze. Um zur Erkenntniß der uns wohlthätigsten religiösen Ansichten zu gelangen, können wir allerdings auch selbst sehr Vieles beitragen. Wir können dasjenige, was Gott bereits Andern vor uns geoffenbaret hat, kennen zu lernen trachten; wir können darüber noch weiter nachdenken, und daraus allerlei Folgerungen abzuleiten suchen; unsere Ansichten einander mittheilen, und wenn wir recht sicher gehen wollen, zuletzt nur bei demjenigen stehen bleiben, was die Einstimmung Aller für sich hat. Will sich uns also Gott offenbaren, und zwar auf eine Art, die seine Weisheit in das helleste Licht setzt: so muß er uns zu Allem, was ich so eben gesagt habe, verhalten; dann aber versprechen, daß er den Glauben Aller immer so leiten wolle, daß wir uns nie in einem Irrthume vereinigen. Soll uns die Vorsehung Gottes in ihrer Güte recht anschaulich werden: so muß die Offenbarung, welche uns Gott ertheilt, nicht in irgend einem in einer todten Sprache geschriebenen Buche, sondern in dem lebendigen Glauben der Menschen selbst niedergelegt seyn; und das Wohlthätigste ist offenbar, wenn sich nicht bloß Einer oder Einige, sondern eine ganze Gemeine in dem Besitze der wahren göttlichen Offenbarung befindet, \dh\ wenn der Gesammtglaube der Gemeine selbst den Inhalt der göttlichen Offenbarung ausmacht.~\RWSeitenw{58}
\end{aufza}

\RWpar{18}{Sittlicher Nutzen des dritten Artikels}
Die Behauptung, daß der Vorzug einer Offenbarung, der dem Gesammtglauben der Katholiken zukommt, ihnen schon seit der Errichtung der Kirche zukomme, und bis an das Ende des menschlichen Geschlechtes beiwohnen werde, macht uns
\begin{aufza}
\item das \RWbet{Daseyn dieses Vorzuges in der gegenwärtigen Zeit begreiflicher und glaubwürdiger.} Denn wenn wir nicht hörten, daß auch schon die frühere Kirche sich dieses Vorzuges erfreut habe, und daß auch die späteste noch seiner genießen werde: so würden wir es kaum glaublich und begreiflich finden, daß und warum nur wir allein eines so hohen Vorzuges würdig erachtet worden sind. Ferner begreifen wir nun auch leichter die Möglichkeit, wie unser gegenwärtiger Glaube Gottes Offenbarung sey. Denn da wir ihn, wenigstens größtentheils so angenommen haben, wie unsere Vorfahren (Eltern und Lehrer) ihn uns überlieferten: so würden wir nicht begreifen, wie er jetzt den Charakter einer göttlichen Offenbarung und der Unfehlbarkeit haben könne, wenn er in früherer Zeit der Möglichkeit des Irrthums ausgesetzt war.
\item \RWbet{Die Fürsorge Gottes für das menschliche Geschlecht erscheint uns in einem um desto helleren Lichte}, je größer bereits der Zeitraum ist, durch den diese so wohlthätige Leitung der katholischen Kirche währet.
\item Da aber diese Leitung nur der \RWbet{katholischen} Kirche allein zu Theil wird: so erscheint uns hiedurch der \RWbet{Lehrbegriff dieser Kirche um so wichtiger}. Nur dieser Lehrbegriff allein ist es, dem die Gottheit einen so außerordentlichen Schutz angedeihen läßt, daß er bis an das Ende der Menschheit fortdauern wird, einen Schutz, dessen sich sonst keine andere Religion je zu erfreuen gehabt hat, noch jetzt erfreuet. Also muß wohl unser katholischer Lehrbegriff ungleich vortrefflicher seyn, als alle übrigen, der mosaische, die protestantischen, \uma~\RWSeitenw{59}
\end{aufza}

\RWpar{19}{Sittlicher Nutzen des vierten Artikels}
Wenn wir glauben, daß Jesus ausdrücklich versichert habe, er werde uns seinen Beistand, oder vielmehr den Beistand des heil.\ Geistes senden, so glauben wir auch
\begin{aufza}
\item um desto zuversichtlicher daran, daß der Gesammtglaube der Katholiken eine wahre göttliche Offenbarung sey.
\item Und unserer Dankbarkeit wird der Gegenstand derselben, der heil.\ Geist, genauer ausgezeichnet.
\end{aufza}

\RWpar{20}{Sittlicher Nutzen des fünften Artikels}
Wenn wir nicht wüßten, daß das Versprechen des göttlichen Beistandes der Kirche \RWbet{unbedingt,} sondern vermeinten, daß es nur unter der Bedingung eines würdigen Betragens der Lehrer, oder wohl gar der sämmtlichen Gläubigen selbst gegeben sey: so würden wir uns eigentlich nie auf dasselbe verlassen können, weil wir niemals versichert seyn könnten, daß sich die Lehrer der Kirche, und vollends, daß sich die ganze christliche Gemeinde würdig betragen habe.

\RWpar{21}{Sittlicher Nutzen des sechsten Artikels}
\begin{aufza}
\item Wenn die Kirche zuvörderst allgemein behauptet, daß sich ihre Gabe der Unfehlbarkeit auf alle \RWbet{religiöse} Gegenstände erstrecke: so wissen wir, daß wir in allen Dingen, welche die Religion betreffen, also in allen Dingen, die für den Menschen von der größten Wichtigkeit sind, nicht bloß uns Wahrscheinlichkeit verschaffen, sondern zu völliger \RWbet{Gewißheit} gelangen können; wenn wir uns nur an das, was in der Kirche hierüber mit allgemeiner Uebereinstimmung geglaubt wird, halten wollen.
\item Doch hat es auch seinen Nutzen, daß von dem Gebiete dieser Unfehlbarkeit einige Gegenstände ausdrücklich \RWbet{ausgeschlossen} werden. Und zwar
\begin{aufzb}
\item zuerst schon die große Anzahl von \RWbet{Disciplinarvorschriften}. Denn da es dieser Vorschriften so viele~\RWSeitenw{60}\ gibt: so würde im widrigen Falle ein Jeder, der sich von der Göttlichkeit des Katholicismus überzeugen wollte, in eine sehr weitläufige und beinahe nie zu beendigende Untersuchung verwickelt werden. Er müßte nämlich, wenn diese Vorschriften alle für unfehlbar erkläret, und folglich als gehörig zu dem Inhalte der Offenbarung angesehen würden, auch in Rücksicht ihrer untersuchen, ob sie eine jede den höchsten Grad sittlicher Zuträglichkeit besitzen; was sich in Ansehung vieler nie mit gehöriger Sicherheit entscheiden ließe.
\item Auf einem gleichen Grunde beruhet der Nutzen, den es hat, daß auch alle bloß \RWbet{historische} Behauptungen, und
\item endlich auch alle \RWbet{bloß wissenschaftliche} Begriffsbestimmungen und Beweise von dem Gebiete der kirchlichen Unfehlbarkeit ausgeschlossen werden.
\end{aufzb}
\item Daß aber die in der Bibel enthaltenen Erzählungen für unfehlbar angesehen werden, hat den unläugbaren Nutzen, daß wir die \RWbet{religiösen Lehren}, welche die Kirche aus diesen Erzählungen herleitet, oder an ihnen versinnlichet, nun um so williger annehmen. Was insbesondere die Evangelien betrifft, so stellen uns diese in der Person Jesu Christi ein Muster so hoher Vollkommenheit dar, daß jedem Freunde der Menschheit und jedem Liebhaber der Tugend, um wie viel mehr jedem Christen, überaus viel daran gelegen seyn muß, glauben zu können, daß einst wirklich ein so vollendeter Mann, als dieser Jesus uns hier geschildert wird, gelebt hat, \dh\ glauben zu können, daß die Evangelien wenigstens in allen denjenigen Stellen, in denen sie uns den Charakter Jesu zeichnen, alle Verlässigkeit haben.
\end{aufza}

\RWpar{22}{Sittlicher Nutzen des siebenten Artikels}
\begin{aufza}
\item Die Meinung, daß sich in den religiösen Ansichten der katholischen Kirche durch alle Jahrhunderte ihres Daseyns nicht das Geringste geändert habe, und auch in Zukunft nicht ändern werde, hat für den Ungebildeten einen sehr wesentlichen Vortheil, den ich bei der Betrachtung des wirklichen Nutzens angeben will.~\RWSeitenw{61}
\item Für \RWbet{Gebildetere} dagegen ist es von größter Wichtigkeit, zu wissen, daß der Lehrbegriff der katholischen Kirche einer gewissen \RWbet{fortschreitenden Ausbildung} nach dem Bedürfnisse der Zeiten fähig sey. Denn nun wird es
\begin{aufzb}
\item für's Erste aufhören, ein uns beirrender Einwurf wider die Unfehlbarkeit der Kirche zu seyn, wenn man mit mehr oder weniger Wahrscheinlichkeit uns darthut, daß diese oder jene Lehre, welche die Kirche jetzt vorträgt, in den frühesten Zeiten noch nicht vorhanden gewesen sey; nun werden wir uns zu bescheiden wissen, daß eine solche Lehre, welche die Kirche jetzt vorträgt, unter diejenigen gehöre, die uns der heil.\ Geist, weil das Bedürfniß für sie erst in der Folge unter uns entstanden ist, auch in der Folge erst zum deutlichen Bewußtseyn gebracht hat.
\item Wir können nun glauben, daß die katholische Religion für \RWbet{jedes Zeitalter} eine gleich hohe Vollkommenheit, und eine so große sittliche Zuträglichkeit besitze, daß keine andere zuträglicher seyn könnte. Da wir wissen, daß sich mit jedem Jahrhunderte in den Begriffen der Menschen und in ihrer Empfänglichkeit für religiöse Belehrung so Vieles ändert: so könnten wir unmöglich glauben, daß derselbe Lehrbegriff, der vor Jahrhunderten der allerzuträglichste für die Menschheit war, es ohne weitere Ausbildung auch jetzt noch sey. Wissen wir aber, daß die katholische Religion eine Ausbildung nach dem Bedürfnisse der Zeiten zuläßt: so können wir auch nicht zweifeln, daß der Geist Gottes ihr in einem jeden Zeitalter gewiß gerade diejenige Ausbildung geben wird, deren sie bedarf, um diesem Zeitalter am Allerzuträglichsten zu seyn.
\item Eben hiedurch erhalten wir auch einen neuen Beweis von Gottes weiser und gütiger Fürsorge für uns, vermöge der er, nicht zufrieden damit, uns eine Offenbarung zu geben, die so vollständig ist, als es unsere Fassungskraft in \RWbet{Einem} Zeitalter erlaubt, auch jetzt noch fortfährt, sie nach dem Maße unserer Empfänglichkeit und unserer Bedürfnisse zu erweitern.
\item Wissen wir, daß die katholische Religion einer fortschreitenden Ausbildung fähig sey, so fühlen wir uns verpflich\RWSeitenw{62}tet, zu dieser Ausbildung selbst das Unsrige \RWbet{beizutragen}; und werden, weit entfernt, ersprießliche Ansichten bloß darum zu verwerfen, weil sich kein Zeugniß aus dem apostolischen Zeitalter für sie aufbringen läßt, vielmehr uns Mühe geben, sie immer weiter auszubilden.
\item Hören wir, daß die katholische Religion einer fortschreitenden Ausbildung fähig sey: so können wir um so zuversichtlicher an die erfreuliche Wahrheit glauben, daß auch das \RWbet{ganze menschliche Geschlecht} stets weiter fortschreite in der Vollkommenheit. Denn jene fortschreitende Ausbildung der katholischen Religion beweiset ja schon, daß ein beträchtlicher Theil der Menschheit (die katholischen Christen nämlich) in einem sehr wichtigen Stücke, in ihren religiösen Begriffen, nicht auf derselben Stufe unverrückt stehen bleibe. Läßt sich hiernächst nicht hoffen, daß wir auch noch in manchen anderen Stücken, größtentheils eben durch diese fortschreitende Ausbildung unserer Religion, Fortschritte machen, und daß durch uns Christen allmählich auch die übrigen Völker der Erde auf eine höhere Stufe der Vollkommenheit werden erhoben werden?
\end{aufzb}
\item So nützlich aber der Glaube an eine fortschreitende Ausbildung des katholischen Lehrbegriffes ist, so anstößig wäre es, zu hören, daß der katholische Lehrbegriff \RWbet{geändert} werden könne. Denn dieses hätte den Sinn, daß dieselbe Lehre, welche die katholische Kirche in unseren Tagen aufstellt, in einer späteren Zeit verworfen werden könnte. Wer würde nun, wenn er dieß für möglich hielte, Lust haben, den gegenwärtigen Lehrern der Kirche sein volles Zutrauen zu schenken? --
\end{aufza}

\RWpar{23}{Sittlicher Nutzen des achten Artikels}
\begin{aufza}
\item Würden wir glauben, irgend ein Buch zu besitzen, in welchem alle Lehrsätze des Katholicismus auf das Vollständigste aufgezeichnet sind: so würden wir uns nach Durchlesung dieses Einen Buches einbilden, daß wir schon Alles gethan, was uns in Ansehung auf unsere religiösen Meinungen zu thun obliegt. Jetzt dagegen dürfen wir unseren reli\RWSeitenw{63}giösen Forschungen niemals ein Ende setzen; denn noch in unseren späteren Lebenstagen kann ja die Kirche sich über gewisse Lehrsätze vereinigen, in Betreff deren man in unserer Jugend vielleicht im Streite war. Diese würden denn also früher noch nicht zum Inhalte der göttlichen Offenbarung gehört haben; jetzt aber würden sie hinzugekommen seyn.
\item Von der anderen Seite ist es auch wieder gut, daß uns verschiedene Schriften angegeben werden, in denen wir, zwar nicht den vollständigen Inhalt, aber doch den \RWbet{größten} und \RWbet{wichtigsten Theil} des katholischen Lehrbegriffes verzeichnet finden. So erhalten wir nämlich Gelegenheit, uns die Lehren der Kirche durch wiederholte Lesung jener Schriften öfters in das Gedächtniß zurückzurufen.
\item Daß aber die \RWbet{Bibel} als das vornehmste Werk dieser Art genannt wird, hat die größten und wichtigsten Vortheile. Dieses Buch ist wirklich so vollkommen, es ist so reich und unerschöpflich reich an Belehrungen und Erbauung jeder Art, daß wir ein Jeder, der wir es auf die gehörige Art benützen, die herrlichsten Wirkungen an uns verspüren werden. Dieß gilt ganz vornehmlich von den vier heil.\ Evangelien, in welchen uns an der Person Jesu Christi das Urbild menschlicher Tugend und Vollkommenheit verwirklicht dargestellt wird.
\item Auch die Bemerkung, daß die \RWbet{lateinische Uebersetzung} der Bibel, die \RWbet{Vulgata}, für den Zweck der Erbauung sicher, ja selbst noch sicherer zu gebrauchen sey, als der hebräische oder griechische Grundtext, ist nicht ohne Nutzen; denn da nur die wenigsten Menschen im Stande sind, die Bibel in jenen Grundsprachen zu lesen: so dient es diesen zu keinem geringen Troste, zu hören, daß sie durch diesen Umstand für den Zweck der Erbauung nichts verlieren.
\end{aufza}

\RWpar{24}{Ueber den wirklichen Nutzen dieser Lehren}
Es läßt sich im Voraus vermuthen, daß die aufgezählten sittlichen Vortheile, welche die hier besprochenen Lehren \RWbet{möglicher Weise} hervorbringen \RWbet{können}, durch einen so langen Zeitraum, als sie bereits bekannt sind und geglaubt~\RWSeitenw{64}\ werden, bei einer zahllosen Menge von Menschen auch schon zur \RWbet{Wirklichkeit} theilweise wenigstens gelanget seyn werden; und schon aus diesem Grunde ist zu entnehmen, daß die katholische Lehre von den Erkenntnißquellen der göttlichen Offenbarung keinen geringen Nutzen bisher gestiftet haben müsse. Allein es verlohnt sich der Mühe, daß wir den \RWbet{Nutzen}, den einige einzelne Puncte dieser Lehre gestiftet haben, noch etwas näher erwägen; dann aber auch die wichtigsten Einwürfe, die man von dem vermeintlichen oder wirklichen \RWbet{Schaden} dieser Lehrsätze hergenommen hat, in der Kürze kennen lernen.
\begin{aufza}
\item Die Verschiedenheit der in dem ersten Artikel angegebenen Regeln machte, daß sich die meisten Katholiken in jedem einzelnen Falle gerade an diejenige hielten, welche für diesen Fall eben die \RWbet{zuträglichste} war. Entsprach der Lehrsatz, den der Papst aufgestellt hatte, den sittlichen Bedürfnissen der Zeit: so hielt man sich an die Regel, daß man dem Papste gehorchen müsse, und nahm so den Lehrsatz an. Entsprach er diesen Bedürfnissen nicht, wenigstens bei Einigen nicht: so widersetzten sich diese der Entscheidung des Papstes, und sagten, man müsse erst die Entscheidung eines allgemeinen Kirchenrathes abwarten. Entstandene Streitigkeiten hätten oft zu lange fortgedauert, oder sie wären wohl gar nie entschieden worden, wenn man nie etwas eher hätte annehmen wollen, als bis ein allgemeiner Kirchenrath dafür entschieden hätte; also war es wohl gut, daß Einige behaupteten, auch schon die bloße päpstliche Entscheidung habe Gesetzeskraft. Würde man wieder dem Papste allgemein gefolgt haben: so hätte dieß gleichfalls sehr große Nachtheile gehabt: und so war es also wohl gut, daß es Andere gab, die wieder nur die Entscheidung eines allgemeinen Kirchenrathes für verbindlich erklärten, \usw\
\item Die Meinung, daß der Gesammtglaube der Katholiken göttliche Offenbarung sey, ist
\begin{aufzb}
\item als die vornehmste Ursache von der \RWbet{allmählichen Ausbildung} zu betrachten, welche der Lehrbegriff der Katholiken im Verlaufe der Jahrhunderte erfahren. Wir müßten noch heut zu Tage auf eben dem Puncte stehen, auf dem sich unsere Vorfahren vor achtzehn Jahrhunderten befanden, wenn diese Meinung nicht gewesen wäre.~\RWSeitenw{65}
\item Eben so ist es nur diese Meinung, der wir das größtentheils wohlthätige Bestreben zu verdanken haben, die Gesinnungen der Katholiken bei entstandenen Streitigkeiten wieder zu \RWbet{vereinigen}, die Veranstaltung so vieler Kirchenversammlungen, und alle die übrigen meistens sehr zweckmäßigen Mittel, die man zur Ausrottung entstandener Irrthümer und zur Verbreitung der Glaubenseinigkeit gebraucht hat.
\end{aufzb}
\item Der Meinung, daß die Kirche in ihren Disciplinarvorschriften keine Unfehlbarkeit besitze, verdanken wir es, daß man dergleichen Vorschriften, wenn auch nicht so oft, als es vielleicht zu wünschen gewesen wäre, nach den Bedürfnissen der Zeit geändert hat.
\item Die Meinung, daß sich in den religiösen Ansichten der Kirche durch alle Jahrhunderte ihres Daseyns nicht das Geringste geändert habe, und auch in Zukunft nicht ändern dürfe, hat bei der größeren Menge der Christen, worunter die Menge der Ungebildeten gehört, wichtige Vortheile erzeugt.
\begin{aufzb}
\item Ohne diese Meinung hätten sie auch den Glauben an die \RWbet{Vollkommenheit des katholischen Lehrbegriffes} niemals mit Ueberzeugung annehmen können; denn Menschen, die noch auf einer sehr niedrigen Stufe der Bildung stehen, vermögen sich kaum von einem anderen Zustande, als eben der ihrige ist, einen Begriff zu machen. Sie können sich nicht vorstellen, daß und wienach die Menschen vor Jahrhunderten gewisse Begriffe, die wir jetzt allgemein kennen, noch gar nicht gehabt haben sollten, und eben deßhalb auch gewisser Aufschlüsse der Offenbarung, die uns jetzt mitgetheilt worden sind, noch nicht empfänglich und bedürftig waren. Sie sind noch weniger fähig, sich vorzustellen, daß erst in späteren Jahrhunderten noch neue Begriffe, von denen wir jetzt nichts ahnen, entstehen, und das Bedürfniß neuer Aufschlüsse von Seite Gottes herbeiführen könnten. Der Umstand also, daß die katholische Religion eines gewissen Zuwachses nach den Bedürfnissen der Zeiten fähig sey, der in der That einen ihrer größten Vorzüge ausmacht, durch~\RWSeitenw{66}\ den sie allein eine für alle Zeiten taugliche Religion wird, dieser Umstand erscheint in den Augen der Ungebildeten eher als eine Unvollkommenheit. Sie meinen, wenn wir jetzt mehr wissen, als die Christen früherer Zeiten: so hatten diese Ursache, mit ihrer Religion unzufrieden zu seyn; und wenn in späteren Zeiten noch mehr bekannt werden soll, als jetzt: so gilt dasselbe von uns. Sie meinen, wenn sich das Christenthum stets weiter ausbildet, so kann es nie ganz vollendet und vollkommen heißen, welches doch gar nicht folgt; weil eine Religion jederzeit vollkommen ist, wenn sie nur alles das enthält, dessen die Menschen für diese Zeit bedürfen und empfänglich sind.
\item Durch diese Meinung wurde der eitlen \RWbet{Neuerungssucht}, die in so vielen Fächern des menschlichen Wissens, und, wo man die Kirche nicht gehört hat, auch im Gebiete der Religion, Unheil gestiftet hat, ein wohlthätiger Zaum angelegt, und bewirkt, daß man in Aufstellung neuer religiöser Ansichten in der katholischen Kirche nie allzuschnell und ohne hinlängliche Prüfung vorgehen konnte. Bevor man eine gewisse Ansicht als Glaubenslehre aufstellen und bewirken konnte, daß sie von Allen angenommen wurde, mußte man erst erwiesen haben, daß diese Ansicht den Begriffen der früheren Jahrhunderte nicht widerspreche, daß sich wohl eher Zeugnisse für sie ausfindig machen ließen. Die Liebe, welche sich diese Ansicht durch ihre innere Vortrefflichkeit erworben hatte, mußte so groß seyn, daß man, falls sich keine ganz unzweideutige Zeugnisse vorfanden, von ihrem Daseyn wenigstens sich überreden konnte; \usw
\end{aufzb}
\item Welche ersprießliche Folgen hat nicht die Aufstellung der Bibel, als eines Wortes Gottes für die gesammte Christenheit gehabt! Wie viele \RWbet{Erbauung} hat nicht dieß heilige Buch durch achtzehn Jahrhunderte unter uns gestiftet! Und vollends, was wäre aus \RWbet{denjenigen Kirchen}, die sich von der katholischen \RWbet{getrennt}, geworden, wenn sie von der katholischen Kirche nicht gelernt hätten, die Bibel als ein Wort Gottes zu verehren?~\RWSeitenw{67}
\end{aufza}

\RWpar{25}{Einwürfe}
\RWbet{1.~Einwurf.} Der Glaube an ihre Unfehlbarkeit war es, der die Mitglieder der katholischen Kirche von jeher so \RWbet{unduldsam} machte; denn eben, weil sie fälschlicher Weise vermeinten, daß ihr Gesammtglaube eine göttliche Offenbarung sey, erlaubten sie sich in der Vertheidigung desselben, und in der Verfolgung der Andersdenkenden Hitze und Ungestüm, Gewalt und Grausamkeit, und glaubten hiebei noch etwas Gott Gefälliges zu thun.\par
\RWbet{Antwort.} Ich läugne nicht, daß die zuletzt erwähnte, gewiß sehr irrige Vorstellung, daß die Verfolgung der Andersdenkenden etwas Gott Wohlgefälliges sey, in der katholischen Kirche wirklich sehr häufig Statt gefunden habe; allein ich glaube, daß eine genauere Untersuchung der Sache am Ende zeigen würde, aller der Schaden, den diese Unduldsamkeit angerichtet hat, sey durch die Vortheile, welche die Weisheit Gottes auch aus ihr herzuleiten wußte, bei Weitem aufgewogen worden. Diese Unduldsamkeit war freilich
\begin{aufzb}
\item schon an sich selbst eine Sünde, indem diejenigen, die sich derselben schuldig machten, meistens ein dunkles Gefühl davon, daß sie hier Unrecht thun, hatten; indem sich auch so manche andere Leidenschaften, Haß, Mißgunst, Bereicherungssucht \udgl\  mit ihr verbanden, oder sich hinter sie versteckten. Sie veranlaßte auch
\item so manches unredliche, aus bloßer Furcht vor Verfolgung abgelegte, also erheuchelte Glaubensbekenntniß; sie verursachte endlich
\item so vielen Tausend Menschen unsägliche Leiden, und oft sogar den Tod.
\end{aufzb}
Allein wir müssen, wenn wir unparteilich seyn wollen, auch erwägen, daß
\begin{aufzb}
\item durch die Verfolgungen, mit denen man jeden Andersdenkenden bedrohte, unzählig viele Menschen vor den gefährlichsten Irrthümern bewahrt geblieben sind; daß
\item unzählig viele Menschen als Kinder schon erzogen wurden in einer wahrhaft beseligenden Religion, zu deren~\RWSeitenw{68}\ Annahme sie, bei einem anderen Verfahren, höchstens in ihren späteren Jahren, oder wohl gar nicht, sondern erst ihre Nachkommen gelanget wären; daß endlich
\item die Leiden und Verfolgungen, denen so Viele sich lieber unterzogen, als daß sie ein Bekenntniß wider den Ausspruch ihres Gewissens abgelegt hätten, zu ihrer sittlichen Veredlung beitrugen, und selbst Andern ein lehrreiches Beispiel gaben.
\end{aufzb}
Behaupte ich aber, daß auch die Unduldsamkeit manches Gute gehabt: so bin ich darum gar nicht gesonnen, sie zu rechtfertigen; denn noch ungleich mehr Gutes würde man durch ein bescheidenes Bestreben, die Andersdenkenden auf dem Wege des Unterrichtes zu gewinnen, ausgerichtet haben.\par
\RWbet{2.~Einwurf.} Die Lehre der Katholiken von den Erkenntnißquellen ihrer Religion befördert einen höchst unvernünftigen Auctoritätsglauben; ein recht Pythagoräisches: \RWbet{Er hat's gesagt.} Statt nach den Gründen einer Lehre zu fragen, fragt man nur nach, ob sie der Papst, die Bischöfe, oder wohl gar die gesammten Mitglieder der Kirche vortragen. Findet man dieses, so glaubt man. Daher denn auch, daß nirgends so vielfältiger und so grober Aberglaube herrscht, als bei den Katholiken, die durch den Auctoritätsglauben, den ihre Religion fordert, gewöhnt werden, ohne Gründe zu glauben, was sie hören.\par
\RWbet{Antwort.} Es ist sehr wahr, daß sich der Glaube der Katholiken auf Auctorität, nämlich auf die Bemerkung gründe, daß bald der Papst allein, bald auch noch andere Bischöfe mit ihm, bald wohl die ganze Gemeinde einer gewissen Lehre zugethan sey. Nur kann man diesen Glauben nicht einen \RWbet{unvernünftigen} Auctoritätsglauben nennen; denn wohl gibt es sehr vernünftige Gründe, aus denen der Katholik das glauben kann, was die erwähnte Auctorität für sich hat. Entschließt er sich nämlich, dasjenige zu glauben, was der Papst oder gewisse Bischöfe zu glauben aufgestellt haben: so kann er es thun, weil er von Einer Seite die Nützlichkeit oder doch wenigstens die Unschädlichkeit dieses Glaubens eingesehen hat, und von der andern weiß, daß es Pflicht sey, der Obrigkeit in unschädlichen Dingen zu gehor\RWSeitenw{69}chen. Entschließt er sich etwas zu glauben, wovon er bemerkt, daß es die ganze Kirche glaubt: so kann dieß geschehen, weil er aus Gründen (nämlich aus Wahrnehmung der beiden Kennzeichen einer Offenbarung) überzeugt worden ist, daß dieser Gesammtglaube der Katholiken eine wahre göttliche Offenbarung sey. -- Daß aber Viele, ja vielleicht die meisten Katholiken, ohne solche vernünftige Gründe glauben, was immer ihnen zu glauben von ihren nächsten Lehrern vorgetragen wird, ist freilich nur zu wahr. Aber in welcher andern Religion ist dieses besser? -- Auf jeden Fall ist dieß ein Fehler, den die hier beschuldigten Lehren nicht erst verursachen, sondern schon vorfinden.\par
\RWbet{3.~Einwurf.} So macht diese Lehre die Katholiken wenigstens träge im Denken. Denn statt den Gründen der Wahrheit nachzuforschen, fragen sie nur: was hat die Kirche hierüber entschieden? -- Daher denn, daß die Wissenschaften bei den Katholiken niemals so sehr im Flor gestanden sind, als bei den akatholischen Parteien.\par
\RWbet{Antwort.} Es ist wahr, daß sich die protestantischen Länder durch einen größeren Fleiß und Eifer in der Bearbeitung einiger Wissenschaften schon lange auszeichnen; aber dieß rührt sicher wohl \RWbet{nicht ganz} von jener Lehre des Katholicismus her, ob ich gleich zugeben will, daß auch sie einigen Antheil daran habe. Gewiß ist es aber, daß der Nutzen, den die Lehre von den Erkenntnißquellen der Offenbarung den Katholiken theils schon bisher geleistet hat, theils in der Folge noch leisten wird, den Schaden überwiege, den jene zeitweilige Vernachlässigung einiger Wissenschaften, \zB\ der Auslegungskunde, der orientalischen Sprachen \ua , gehabt hat. Es läßt sich nämlich mit aller Zuversicht erwarten, daß auch die Katholiken einst den Nutzen einsehen werden, den ein recht gründliches Verstehen der Bibel, eine Erforschung des ursprünglichen Sinnes, den die Verfasser selbst mit jeder Stelle verbanden, haben könne. Dann wird das Studium der Bibel bei ihnen auf eine weit gedeihlichere Art betrieben werden können, als bei den Protestanten. Denn diese vermeinen, bei jeder neuen Entdeckung, welche sie in der Bibel machen, an ihrem Glauben ändern zu müssen; und eben~\RWSeitenw{70}\ darum wird es ihnen sehr schwer, die Bibel mit ganz unbefangenem Gemüthe zu studiren. Die Katholiken aber (bis der gelehrtere Theil derselben sich zu den hier angedeuteten Begriffen wird erhoben haben) werden mit Ruhe und Gelassenheit abwarten können, welche Entdeckungen ein freieres Forschen über den ursprünglichen Sinn, über die Aechtheit und Glaubwürdigkeit einzelner Bücher der heil.\ Schrift an den Tag bringen mag; ihr Glaube, werden sie wissen, kann durch Entdeckungen dieser Art niemals erschüttert werden.\par
\RWbet{4.~Einwurf.} Die katholische Lehre von den Erkenntnißquellen der göttlichen Offenbarung hat die Katholiken, oder doch ihre Lehrer stolz und hochmüthig gemacht; denn was diese aussprachen, mußten die Uebrigen glauben, und das ward also, wenn nicht schon in der Gegenwart, doch in der Folge durch die allgemeine Annahme, zu göttlicher Offenbarung erhoben.\par
\RWbet{Antwort.} Daß die Katholiken im Ganzen genommen stolz auf das Vorrecht einer Offenbarung, das ihr Gesammtglaube hat, gewesen wären, \RWbet{kann schlechterdings nicht aus der Geschichte nachgewiesen werden.} Auch hat es nicht die mindeste innere Wahrscheinlichkeit; indem bei der so großen Ausbreitung, die diese Kirche hat, der Antheil, welchen ein Einzelner an der Constituirung einer Glaubenslehre nimmt, äußerst klein ausfällt, und ihm mit allen Uebrigen gemein ist. Nicht einmal kann man beweisen, daß sich die Bischöfe und der Papst \RWbet{aus diesem Grunde} vom Geiste des Hochmuthes hätten beherrschen lassen. Der Hochmuth, den diese Personen zuweilen an den Tag gelegt, entsprang vielmehr aus ganz anderen Quellen, vornehmlich wurde er durch den hohen Rang, die prächtigen Titel, den Reichthum und die bürgerliche Macht, welche die weltlichen Fürsten ihnen vielleicht allzu freigebig eingeräumt hatten, erzeugt und unterhalten.\par
\RWbet{5.~Einwurf.} Die katholische Lehre von den Erkenntnißquellen der Offenbarung hinderte mächtig die Freiheit im Denken. Von Jahrzehend zu Jahrzehend vermehrte sich und kann sich noch jetzt vermehren die Anzahl der Entscheidungen der Kirche, welche bestimmen, was man von diesem oder~\RWSeitenw{71}\ jenem Gegenstande zu denken oder nicht zu denken habe. Fast keine Stelle der Bibel ist übrig, deren Sinn und Auslegung die Kirche nicht schon festgesetzt hätte. Dem ächten Katholiken also, der sich in allen diesen Stücken nach dem Ausspruche der Kirche richten will, ist beinahe gar kein Spielraum für seine Denkfreiheit gelassen. Daher kommt es auch, daß man sogar keine Lust zum Denken bei den Katholiken antrifft.\par
\RWbet{Antwort.}
\begin{aufza}
\item Wenn anders die Entscheidungen der Kirche göttliche Offenbarungen sind: so kann man nicht sagen, daß sie die Freiheit im Denken auf eine \RWbet{schädliche} Art beschränken; denn was sie dann beschränken, ist eigentlich nur eine \RWbet{Freiheit zu irren}. Ueber eine solche Einschränkung aber kann sich wohl kein Vernünftiger beklagen.
\item Uebrigens muß man nicht glauben, als ob dem Katholiken bei allen kirchlichen Entscheidungen nicht immer noch ein unendlich großer Spielraum zur freien Uebung seiner Denkkraft übrig gelassen wäre. Der Gegenstände nämlich, auf die sich das menschliche Denken erstrecken kann, gibt es im strengsten Sinne des Wortes \RWbet{unendlich viele}, die Entscheidungen der Kirche aber, so zahlreich sie immer seyn, oder werden mögen, sind doch nur von endlicher Menge.
\item Auch ist es so wenig wahr, daß die Entscheidungen der Kirche dem Katholiken die Lust zum Denken nehmen, daß sie ihm vielmehr einen eigenen Stoff und eine nähere Aufforderung dazu geben. Da er nämlich (wie schon die bloße Vernunft sagt) die Entscheidungen nicht eher annehmen soll, als bis er ihre Vernunftmäßigkeit und ihre sittliche Zuträglichkeit erkannt hat: so gibt ihm jede neue Entscheidung der Kirche auch einen neuen Stoff und eine neue Aufforderung zum Nachdenken darüber, ob diese Entscheidung auch vernunftmäßig und sittlich zuträglich sey.
\item Endlich ist zu bemerken, daß selbst bei denjenigen Dingen, worüber die Kirche bereits entschieden hat, die Freiheit zu denken noch immer einen sehr großen Spielraum übrig behalte. Die Kirche entscheidet nämlich nur, daß wir an eine gewisse Ansicht von diesen Dingen uns halten sollen; ob aber diese Ansicht \RWbet{objectiv wahr} sey, oder nur als die~\RWSeitenw{72}\ \RWbet{sittlich zuträglichste} angesehen werden solle, läßt sie und muß es unentschieden lassen, so oft es gleichgültig ist. Hierüber also kann jeder Katholik nach seinen Einsichten so oder anders urtheilen. Daher wird denn auch ganz falsch im Einwurfe gesagt, die Kirche habe beinahe von jeder Stelle der Bibel Sinn und Auslegung schon für immer festgesetzt. Sie hat bloß vorgeschrieben, daß man sich dieser und jener Bibelstellen zur Erinnerung an diese und jene religiöse Wahrheiten bediene; keineswegs aber hat sie den \RWbet{ursprünglichen} Sinn, den die Verfasser mit diesen Worten verbanden, entschieden und entscheiden wollen.
\item Sind also die Katholiken zuweilen in der That träge im Denken gewesen: so rührte dieß wahrlich nicht von ihrem Glaubenssysteme her, sondern die Ursache muß irgendwo anders liegen. So viel ist wenigstens gewiß, wenn sich die Katholiken erst selbst recht verstehen werden, wird Niemand mehr Stoff und Aufforderung zum nützlichen Denken haben, als eben sie in ihrer eigenen Religion.
\end{aufza}\par
\RWbet{6.~Einwurf.} Der Nutzen, den die Lesung der Bibel gestiftet, kann nicht auf Rechnung der katholischen Kirche kommen; denn sie verbot ja vielmehr die Lesung dieses Buches.\par
\RWbet{Antwort.} Gesetzt auch, die Lesung der Bibel wäre den Laien verboten gewesen: so war doch nie verboten, sondern vielmehr ausdrücklich \RWbet{befohlen,} sie mit dem wichtigsten Inhalte derselben in Predigten, im Schulunterrichte, durch allerlei Bücher, die zur Erbauung geschrieben waren, und auf so manche andere Weise bekannt zu machen. Doch auch die häusliche Lectüre dieses Buches wurde den Laien nie \RWbet{unbedingt} verboten, sondern nur wollte man zu gewisser Zeit, daß nicht ein jeder Laie die Lesung dieses Buches sich erlaube, wenn er nicht erst seinen Gewissensfreund darüber berathen, und seine Gutheißung erhalten. Auch dieses geschah nur in Zeiten einer Gährung, wo zu befürchten war, daß man die Bibel in \RWbet{protestantischem Geiste}, \dh\ mit dem Vorsatze lesen werde, alle diejenigen Lehren und Anordnungen der Kirche, die man in ihr nicht ausdrücklich vorgetragen fände, zu verwerfen. In solchen Fällen aber hätte die Lesung dieses Buches wirklich mehr Schaden als Nutzen gestiftet.~\RWSeitenw{73}

\RWpar{26}{Ein Blick auf andere Religionen, welche sich für geoffenbart ausgeben}
Da es nicht hinlänglich ist, zu erweisen, daß die Lehre der katholischen Religion nur \RWbet{an sich selbst} sittliche Zuträglichkeit habe; sondern erwiesen werden muß, daß ihre Lehre \RWbet{zuträglicher} sey, als die Lehre jeder anderen Religion, welche sich gleichfalls für geoffenbart ausgibt, und zu ihrer Bestätigung gewisse Wunder aufweisen kann: so wird es nöthig seyn, daß wir auch diese erste Lehre des Katholicismus, nämlich die Lehre von den Erkenntnißquellen seines Glaubens mit demjenigen vergleichen, was andere Religionen über eben diesen Gegenstand lehren. Da zeigt sich nun der äußerst merkwürdige Umstand, \RWbet{daß die katholische Religion wirklich die einzige aus allen sich für geoffenbart ausgebenden Religionen ist, die eine ähnliche Erkenntnißquelle ihrer Lehren aufstellt}, \dh\ die nur alle solche Lehren für den eigentlichen Inhalt der göttlichen Offenbarung ausgibt, welche von allen Mitgliedern einer gewissen, noch jetzt vorhandenen Gesellschaft angenommen werden. Alle anderen Religionen auf Erden, welche sich für geoffenbart ausgeben, wenigstens alle, die etwas vollkommener sind, verweisen uns bei der Frage nach der Erkenntnißquelle ihrer Lehren auf eine gewisse \RWbet{schriftliche Urkunde}, also auf einen Erkenntnißgrund, der durchaus unveränderlich ist. Aus diesem Umstande ergeben sich zwei sehr wichtige Folgen:
\begin{aufza}
\item Alle Gesellschaftsreligionen haben, mit Ausnahme der katholischen, nach ihrem eigenen Geständnisse wenigstens Einen Fehler, jenen der \RWbet{Unvollständigkeit}. Unter der Religion einer gewissen Gesellschaft kann man dem Sprachgebrauche zu Folge nichts Anderes verstehen, als den Inbegriff aller derjenigen religiösen Meinungen, die von allen oder fast allen Mitgliedern dieser Gesellschaft angenommen werden. (1.~Hptthl.\ 1.~Hptst.\ \RWparnr{22}) Wenn also alle religiöse Gesellschaften, die im Besitze einer göttlichen Offenbarung zu seyn behaupten, nicht auf dieß Merkmal ihres Gesammtglaubens hinweisen, sondern auf irgend eine andere Erkenntnißquelle ihrer Lehren:~\RWSeitenw{74}\ so liegt am Tage, daß sie dieß nicht thun würden, wenn sie nicht Alle besorgten, daß es noch manchen wichtigen Lehrsatz der göttlichen Offenbarung gebe, den man nicht aus der übereinstimmenden Meinung aller ihrer Glieder, sondern nur bloß aus jener anderen Quelle mit Sicherheit schöpfen könne. Dadurch gestehen sie aber, daß der Gesammtglaube ihrer Glieder, wenn sonst keinen andern Fehler, doch wenigstens den der Unvollständigkeit habe. So ist also \zB\ die Religion der Protestanten, \dh\ der Inbegriff aller derjenigen Lehren, welche von allen oder fast allen Protestanten angenommen werden, eine, nach ihrem eigenen Geständnisse, noch unvollständige Religion, eine Religion, welche nicht alle Lehrsätze der wahren göttlichen Offenbarung enthält; und die vollständige Offenbarung befindet sich nur verzeichnet in einem Buche. Ob aber irgend Jemand vorhanden sey, der sie aus diesem Buche in ihrer ganzen Vollständigkeit aufgefaßt habe, ist unter ihnen selbst noch eine Frage.
\item Alle Religionen auf Erden, welche sich für geoffenbart ausgeben, können dieß höchstens nur \RWbet{für ein gewisses Zeitalter} seyn; für alle Zeiten aber und für die ganze Menschheit taugt keine aus ihnen, als die katholische. Denn weil alle übrigen Religionen einen Erkenntnißgrund ihrer Lehren aufstellen, der durchaus unveränderlich ist; so sind sie auch keiner Vermehrung und weiteren Ausbildung fähig, unmöglich können sie also bei jenem Fortschreiten, welches der Menschheit eigen ist, für die Bedürfnisse aller Zeiten vollkommen angemessen seyn.
\end{aufza}

\RWpar{27}{Die Lehre des Protestantismus von den Erkenntnißquellen seines Glaubens}
\begin{aufza}
\item Obgleich das eben Gesagte schon hinreicht, zu zeigen, daß in der Lehre von den Erkenntnißquellen ihres Glaubens die katholische Religion jede andere weit übertrifft: so will ich doch die Lehre derjenigen Religion, die mit der katholischen übrigens die größte Aehnlichkeit, und eben deßhalb auch den größten Anspruch auf den Namen einer wahren göttlichen Offenbarung hat, nämlich der \RWbet{protestantischen}, noch etwas näher in das Auge fassen.~\RWSeitenw{75}
\item Es war im Anfange des sechszehnten Jahrhunderts der christlichen Zeitrechnung, als die Vorurtheile und Mißbräuche in der katholischen Kirche einen so hohen Grad erstiegen hatten, daß die zu gleicher Zeit entstandene Aufklärung (durch die aus Konstantinopel entflohenen griechischen Gelehrten, durch die Erfindung der Buchdruckerkunst, durch die Errichtung so vieler Schulen und Universitäten herbeigeführt) diesem Unfuge durchaus nicht länger geduldig zusehen konnte. \RWbet{Wiklef} in England, \RWbet{Joh. Huß} in Böhmen, \RWbet{Martin Luther} in Deutschland, \RWbet{Zwingli} und \RWbet{Calvin} in der Schweiz und in Frankreich, \RWbet{Olof} (oder \RWbet{Olaus}) in Schweden \umA , zum Theile sehr edel denkende Männer, die man die \RWbet{Reformatoren} genannt hat, wagten es allenthalben, das Thörichte und Verdammliche der herrschenden Vorurtheile und Mißbräuche in ein helles Licht zu setzen, und auf ihre Abstellung und Beseitigung zu dringen. Aber diesen Männern begegnete, leider! was so gewöhnlich geschieht, daß sie mit manchem Irrthume auch eine sehr heilsame Wahrheit, mit manchem Mißbrauche auch eine Anstalt angriffen, welche um ihrer inneren Vortrefflichkeit willen nie hätte angetastet werden sollen. Als nun die Vorsteher der katholischen Kirche und so manche andere, gemäßigter denkende Katholiken ihnen zu zeigen suchten, daß sie in ihrem Eifer \RWbet{zu weit} gingen, und daß die von ihnen verworfenen Lehren und Anstalten schon in der ältesten Kirche vorhanden gewesen, und die Entscheidungen allgemeiner Kirchenräthe für sich hätten: da fingen sie an, auch selbst den Lehrsatz von der Unfehlbarkeit der Kirche, welchen sie früherhin noch alle zugegeben hatten, in Zweifel zu ziehen und zu bestreiten. Als man sie fragte, was sie denn nun noch als eine sichere Erkenntnißquelle der göttlichen Offenbarung wollten gelten lassen: erklärten sie sich, die \RWbet{Bibel} sey die einzige sichere, aber auch völlig hinreichende Erkenntnißquelle aller Offenbarung. Diese Bibel enthalte die sämmtlichen Glaubens- und Sittenlehren, die Jesus vorgetragen, und deren Kenntniß uns jemals nützlich seyn könnte; was die katholische Kirche noch \RWbet{über} diese Bibel, und wohl gar (also vermeinten sie) \RWbet{gegen} dieselbe Lehre, das wäre nichts als --
 \RWbet{Menschensatzung}, \dh\ eine von Menschen herrührende Erdichtung, welche sich unbefugter Weise anmaßt,~\RWSeitenw{76}\ göttliche Offenbarung zu seyn. Als man endlich umständlicher zu wissen verlangte, wie denn die eigentliche Lehre der göttlichen Offenbarung nach ihren Ansichten laute, weil sie doch in der Bibel nicht geordnet vorgetragen würde; so schritten sie zur Ausarbeitung gewisser Lehrbücher, Glaubensbekenntnisse, Katechismen \udgl , in denen die ganze Lehre der göttlichen Offenbarung vollständig und geordnet, und auf eine für Jedermann faßliche Art dargestellt seyn sollte. Da aber zeigte sich auch, daß sie in ihren Meinungen sich nicht vereinigen konnten; und es entstanden \zB\ in Deutschland gleich Anfangs zwei Parteien, die Partei \RWbet{Luther's} oder die \RWbet{evangelische}, und die Partei \RWbet{Calvin's} oder die \RWbet{reformirte.} Beide begriff man unter dem gemeinschaftlichen Namen der \RWbet{Protestanten.} Später entstanden in eben diesem Deutschlande noch mehrere Secten, als die \RWbet{Arminianer}, die \RWbet{Anabaptisten} und \RWbet{Mennoniten}, die \RWbet{Socinianer}, die \RWbet{Quäker}, die \RWbet{Herrnhuter}, \uam\  Eben so blieben auch in England und Frankreich Jene, die an die Bibel allein sich halten wollten (die Puritaner, die Hugenotten), nicht völlig einig unter einander. Und ohne Zweifel wären der Secten noch ungleich mehrere entstanden, wenn man nicht gleich beim Anfange der Reformation der Freiheit, die Bibel nach seinem eigenen Gutdünken auszulegen, und dem zu Folge sich auch ein eigenes Glaubenssystem zu erbauen, verschiedene Schranken gesetzt, und nicht von Seite der weltlichen Obrigkeiten selbst festgesetzt hätte, daß \zB\ in Deutschland nur die drei Religionen: die katholische, evangelische und reformirte geduldet werden sollten. In unserer neuesten Zeit, wo man sich wegen verschiedener Meinungen in der Religion (wenn sie nicht allzusehr das praktische Leben berühren) nicht mehr verfolgt, haben sich die religiösen Begriffe unter den Protestanten abermals sehr beträchtlich geändert; und ein großer, ja vielleicht der größte Theil der protestantischen Theologen behauptet, daß eine jede göttliche Offenbarung, und somit auch das Christenthum, uns nichts Anderes lehren könne, als was wir auch schon durch unsere \RWbet{bloße Vernunft} als wahr zu erkennen vermögen. Auch diese sogenannten \RWbet{Rationalisten} nehmen die Bibel zwar an, aber sie deuten sie überall so, daß sie nach ihrer Auslegung durchgängig nur Wahrheiten der \RWbet{natürlichen Religion} enthält.~\RWSeitenw{77}
\end{aufza}

\RWpar{28}{Beurtheilung dieser Lehre}
Der Behauptung des Protestantismus, daß die heilige Schrift die einzige und völlig hinreichende Erkenntnißquelle der göttlichen Offenbarung sey, stehen, meiner Ansicht nach, folgende Gründe entgegen:
\begin{aufza}
\item Für's Erste glaube ich schon, \RWbet{daß ein Buch überhaupt nicht das zweckmäßigste Mittel sey, dessen die Vorsehung sich zur Erkenntnißquelle der Lehren einer Offenbarung, die einst der Antheil aller Menschen werden, und durch Jahrtausende fortdauern soll, bedienen könne.}
\begin{aufzb}
\item Da es wenigstens bisher noch keine allgemeine Sprache gibt, die von allen Menschen auf Erden verstanden und gesprochen wird: so müßte ein solches Buch, in welcher Sprache es auch ursprünglich abgefaßt wäre, doch erst in mehrere andere Sprachen übertragen werden, um nicht für den größten Theil des menschlichen Geschlechtes unbrauchbar zu bleiben. Dann aber müßten sich diejenigen, die es in Uebersetzungen lesen, auf die Geschicklichkeit und Treue der Uebersetzer verlassen, und könnten billig zweifeln, ob diese auch überall den rechten Sinn ausgedrückt haben.
\item Selbst für Jene, welche die Grundsprache des Buches verstehen, würde dasselbe im Verlaufe der Zeiten manche Dunkelheiten erhalten; indem auch der Schriftsteller, der für sein Land und Zeitalter vollkommen deutlich geschrieben, für andere Leute und Zeiten hie und da dunkel zu werden pflegt.
\item Das menschliche Geschlecht ist (wie ich glaube) in einem steten Fortschreiten begriffen; es kommen von Zeit zu Zeit wenigstens neue, vorhin noch unbekannte Begriffe zum Vorschein, aus welchen sich neue, vorhin noch ungeahnete Fragen und Zweifel entwickeln, deren Beantwortung und Lösung für dieses aufgeklärtere Zeitalter, wenn nicht ein unumgängliches Bedürfniß, doch etwas sehr Ersprießliches ist. Allein in jenem Buche könnten dergleichen Aufschlüsse wohl nicht zu finden seyn, weil es~\RWSeitenw{78}\ sonst eben hiedurch für alle früheren Jahrhunderte, wo diese Begriffe noch nicht vorhanden waren, unverständlich gewesen wäre.
\item Dergleichen neue Fragen und Zweifel würden um so unausbleiblicher eintreten, da eine Offenbarung größtentheils nur bildliche Lehren aufstellt, Lehren, die ihrer Natur nach immer noch Vieles unbestimmt lassen; indem bei ihnen nur auf \RWbet{verneinende} Art bestimmt werden kann, worin die Aehnlichkeit des Bildes mit der verglichenen Sache \RWbet{nicht} bestehe, wie weit dieselbe \RWbet{nicht} auszudehnen sey.
\item Bisher gibt es noch immer eine sehr große Menge von Menschen, welche die Kunst des Lesens gar nicht erlernet haben. Diese also müßten sich immer noch an ein anderes Mittel, nämlich an jenen Unterricht halten, den ihnen Andere ertheilen, welche versichern, ihn aus dem Buche gezogen zu haben. Wenn nun diese mündlichen Lehrer, wie aus den bisherigen Gründen sehr zu besorgen steht, in ihren Aeußerungen nicht ganz übereinstimmen würden; wenn sie, beiläufig in gleichzählige Parteien getheilt, einander widersprächen: an wen sollte und könnte sich die große Menge da halten? --
\item Endlich scheint dieses Mittel einer geschriebenen Erkenntnißquelle der Weisheit Gottes auch schon darum nicht angemessen zu seyn, weil es nicht jener Regel entspricht, deren ich oben schon erwähnte, der Regel nämlich, vermöge deren Gott bei einem jeden sittlich guten Zwecke, zu dessen Herbeiführung die Menschen selbst etwas beitragen können, diese zuerst verhalten muß, alles das Ihrige zu thun; und nur erst dann dasjenige, was sie aus Unvermögenheit oder auch Trägheit nicht geleistet haben, zum Besten des Ganzen durch seine eigene Dazwischenkunft vollenden darf. -- Um zur Erkenntniß der vollkommensten Religion zu gelangen, können die Menschen nun gewiß mehr thun, als bloß ein Buch, in welchem die Lehren derselben vollständig aufgezeichnet sind, durchlesen.
Sie können auch selbst nachdenken; sie können die Resultate ihres Nachdenkens der Eine dem Andern mittheilen, sie können Zusammenkünfte veranstalten, die Urtheile Aller,~\RWSeitenw{79}\ die sich zu einer und eben derselben Gesellschaft zählen, einholen, \usw\ Es wäre also nicht weise von Gott, wenn er, so wie die Protestanten behaupten, nur Lesung jenes Buches allein verlangte, und dann schon Jeden so zu leiten verspräche, daß er die wahre Offenbarung erkennen werde. Nach den Begriffen des Katholicismus verlangt Gott in der That mehr, verlangt er Alles, was oben angezeigt wurde. Wir sollen bei jedem entstandenen Zweifel nicht nur die Bibel, sondern auch den Glauben Anderer befragen; wir sollen auch auf Vernunftgründe achten; wir sollen unsere Meinungen einander mittheilen, und endlich nur das als Gottes Offenbarung ansehen, worüber wir uns Alle vereinigen konnten. Mehr als dieß können wir offenbar nicht thun, um zur Gewißheit zu gelangen; und darum verlangt Gott auch nicht mehr nach der katholischen Lehre. Nach eben dieser Lehre aber mögen wir Einzelne das, was Gott von uns verlangt, vollständig oder nicht vollständig leisten: so leitet er doch, um des gemeinen Besten willen, die Meinung Aller stets so, daß wir auf keinen Fall Alle in einen Irrthum gerathen, sondern daß dasjenige, worin wir Alle übereinstimmen, immer als seine eigene Offenbarung angesehen werden kann. Wer muß, wenn er aufrichtig seyn will, nicht eingestehen, daß dieses Verfahren der Weisheit Gottes viel angemessener sey, als dasjenige, von welchem die Protestanten annehmen, daß Gott es beobachte?
\end{aufzb}
\item Betrachten wir aber erst die \RWbet{innere Beschaffenheit des Buches selbst}, das nach der Annahme der Protestanten die vollständige Erkenntnißquelle der göttlichen Offenbarung seyn soll: so finden wir, daß es nicht einmal diejenigen Beschaffenheiten habe, die ein Buch doch noch besitzen könnte und müßte, um diesem Zwecke möglichst zu entsprechen. Ich behaupte nicht, daß ein solches Buch in einer systematischen Form geschrieben seyn müßte. Auch, wenn man heut zu Tage ein Buch zum Selbstunterrichte in der Religion, tauglich für alle Menschen, schreiben wollte: so müßte man demselben nicht eine systematische, sondern vielmehr eine historische Form ertheilen; also beiläufig eine solche, wie es die~\RWSeitenw{80}\ meisten Theile der Bibel wirklich haben. Allein sollte die Bibel eine vollständige und völlig hinreichende Erkenntnißquelle der göttlichen Offenbarung werden: so hätte
\begin{aufzb}
\item der Zweck derselben billig \RWbet{in ihr selbst} irgendwo angezeigt werden sollen. Es müßte sich in ihr selbst irgendwo die ausdrückliche Erklärung vorfinden, daß in dem Inbegriffe dieser Bücher (welche hier namentlich aufgezählt seyn müßten) alle Lehren der göttlichen Offenbarung vollständig vorkommen. Diese Erklärung nämlich wäre zu Jedermanns Beruhigung nöthig; \Ahat{sie}{so} hätte auch die Entstehung so vieler Irrthümer, die Erfindung so vieler Menschensatzungen, \usw  , deren die Protestanten uns beschuldigen, verhindert. Aber bekanntlich wird in der Bibel nirgends gesagt, daß sie die einzige Erkenntnißquelle der göttlichen Offenbarung sey. Man könnte daher behaupten, daß die Protestanten im Grunde doch eine mündliche (nicht in der Schrift enthaltene) Ueberlieferungslehre annehmen müssen, die nämlich, daß es sonst keine anderen Ueberlieferungslehren gebe, oder, was eben so viel heißt, daß die Bibel hinreiche, um alle Offenbarungslehren aus ihr allein kennen zu lernen.
\item Sollte die Bibel die einzig hinreichende Erkenntnißquelle der göttlichen Offenbarung werden: so hätte sie schicklicher Weise \RWbet{von Jesu selbst} entweder verfaßt, oder doch wenigstens aus schon vorhandenen Schriften von ihm zusammengesetzt werden sollen. Denn sollte ein Buch die einzig hinreichende Erkenntnißquelle der göttlichen Offenbarung für alle Zeiten werden: so mußte diese zur Zeit der Abfassung des Buches wohl schon vollendet und beschlossen seyn, und in der Folge gar keine Zusätze mehr erhalten. Da nun die Bibel bald nach den Zeiten Jesu zusammengetragen wurde: so muß man annehmen, daß schon Jesus selbst, der Stifter des Christenthums, alle Lehren, die zu dem Inhalte der göttlichen Offenbarung gehören, vollkommen inne gehabt habe, ja daß die Christen (die Apostel \uA ) Alles, was sie in dieser Hinsicht gewußt, erst von ihm selbst erfahren haben; und so hätte er billig, statt diese Lehren nur mündlich vorzutragen, sie schriftlich abfassen, oder die schriftlichen Aufsätze Anderer~\RWSeitenw{81}\ wenigstens redigiren sollen; weil die Erklärung, daß diese Bücher die ganze göttliche Offenbarung umfassen, doch wohl aus keinem Munde so viel Glaubwürdigkeit erhalten konnte, als aus dem Munde dessen, der diese Offenbarung uns gegeben. Statt dessen sind die einzelnen Bücher, aus denen die Bibel besteht, von mehreren, und in Betreff des alten Bundes, uns größtentheils unbekannten Verfassern geschrieben, und nur allmählich, wir wissen abermals nicht von wem gesammelt, und diese Sammlung derselben ist ohngefähr erst gegen das Ende des zweiten Jahrhunderts geschlossen worden. Wem kann man es unter solchen Umständen verargen, wenn er den Zweifel hegt, ob diese Sammlung auch vollständig sey, ob sie uns Alles liefere, was in den Inhalt der göttlichen Offenbarung gehörte?
\item Sollte die Bibel die einzige Erkenntnißquelle der göttlichen Offenbarung werden: so hätte billig so Manches, was in ihr äußerst kurz berührt, oder ganz übergangen wird, und was um seiner Wichtigkeit willen in das Gebiet der Religion doch gehört, \RWbet{ausführlicher} und \RWbet{deutlicher} behandelt werden sollen. Hieher gehört \zB\ die Lehre vom heil.\ \RWbet{Abendmahle,} von welchem die Bibel zwar oft spricht, aber nie mit der nöthigen Ausführlichkeit, sondern nur so, wie man für Leser schreibt, bei denen ein mündlicher Unterricht bereits voraus gegangen ist. Hieher gehört auch die Lehre von der \RWbet{Taufe,} von welcher nicht einmal angegeben wird, wie sie verrichtet werden soll, ob sie auch Kindern ertheilt werden dürfe, ob die von Ketzern ertheilte Taufe auch ihre Gültigkeit habe, \usw\  Eben so wird in der heil.\ Schrift mehrmals von einer gewissen Auflegung der Hände und einer hiedurch mitzutheilenden \RWbet{Kraft und Gabe des heil.\ Geistes} gesprochen, ohne die Sache irgendwo näher aufzuklären. Am andern Orte wird von einer gewissen, durch den Herrn Jesum ertheilten Macht, \RWbet{Sünden zu vergeben oder vorzuenthalten}, gesprochen, aber die Frage, wie und unter welchen Bedingungen von dieser Macht Gebrauch zu machen sey, wird nicht beantwortet. Die Handlung des \RWbet{Fußwaschens} am letzten Abend\RWSeitenw{82}mahle wird (\RWbibel{Joh}{Joh.}{13}{4\,ff}) auf eine solche Art erzählt, daß Jemand, der keine anderen Belehrungen darüber hätte, berechtiget wäre, zu behaupten, diese Handlung sey eben so wesentlich nöthig, als \zB\ die heil.\ Handlung der Taufe, oder das heil.\ Abendmahl \udgl , wie dieß auch \RWbet{Lavater} in vollem Ernste behauptet, \udgl\,m.
\item Endlich wird auch Niemand läugnen können, daß besonders gewisse Bücher der Bibel, \zB\ die Briefe des heil.\ Paulus, mehrere Bücher des alten Bundes, wirklich dunkler und unverständlicher sind, als es je eine zum allgemeinen Gebrauche bestimmte Volksschrift seyn soll, und als es nothwendig gewesen wäre. Denn schon zu den Zeiten der Apostel waren viele dieser Bücher den Lesern dunkel, und führten Mehrere irre, wie dieses \RWbet{Petrus} selbst beklaget (\RWbibel{2\,Petr}{2\,Petr.}{3}{15}): \erganf{Erwäget, daß die Langmuth unsers Herrn zur Beseligung diene; wie auch unser geliebter Bruder \RWbet{Paulus}, nach der ihm verliehenen Weisheit, euch, wie in allen Briefen, in denen er davon redet, geschrieben hat; wobei freilich Manches \RWbet{schwer zu verstehen} ist, was Unwissende und Unbefugte, gleich seinen übrigen Schriften, \RWbet{zu ihrem Verderben verdrehen}.}
\end{aufzb}
\end{aufza}
\begin{RWanm}  
Merkwürdig ist, was \RWbet{Wieland} in dieser Hinsicht (im deutschen Merkur, Gedanken von der Freiheit in Glaubenssachen zu philosophiren, Jul.\ 1788.\ S.\,12.)\RWlit{}{Wieland3} gestand: \erganf{Ob ein solches Buch (welches alle Glaubenslehren enthielte) möglich sey, ist eine Frage, die ich mir so wenig zu beantworten anmaße, als sie zu meinem Zwecke gehört: aber dieß wird doch wohl niemand zu läugnen begehren, daß die Bibel dieses Buch nicht ist; -- daß man sehr viel Hebräisch und Griechisch wissen, sehr viele andere Bücher gelesen haben, und eine unendliche Menge historischer, philosophischer, kritischer, antiquarischer, chronologischer, geographischer, physikalischer, und anderer wissenschaftlicher Kenntnisse besitzen, um es mit Verstande zu lesen; und dem ungeachtet für Leser, die mit allen diesen Kenntnissen in dem erforderlichen Grade versehen sind, beynahe auf allen Blättern Stellen, die von verschiedenen Personen verschieden verstanden und ausgelegt werden. Nichts von Stellen zu sagen, die mit einer so \RWbet{unerklärlichen Unbegreiflichkeit} behaftet sind, daß alle~\RWSeitenw{83}\ Bemühungen, die man angewandt hat, den Glaubenspuncten, die demungeachtet daraus gezogen werden sollen, nur so viel Licht, als zu einem nicht ganz vernunftwidrigen Glauben nöthig ist, zu geben, bis auf diesen Tag fruchtlos gewesen sind.} -- Daher konnte denn auch \RWbet{Sam. Werenfels} (einer der gelehrtesten Gottesgelehrten von der reformirten Kirche zu Schweiz) in seine Bibel schreiben:\par
\RWlat{\erganf{Hic liber est, in quo quaerit sua dogmata quisque\\
Invenit et pariter dogmata quisque sua.}\\
(Sam.~Werenfelsii Opuscula theolog.\ philos.\ et philolog.\ Laus.\ 1739.\ T.\,2.\ p.\,509.)}\RWlit{}{Werenfels1}
Bemerkenswerth ist auch, daß der so unparteiliche \RWbet{Lessing} (in seiner nöthigen Antwort auf eine sehr unnöthige Frage des Herrn Hauptpastor \RWbet{Göze} in Hamburg. Wolfenb.\ 1788.) selbst aus den Schriften der Kirchenväter bewies, daß man die Bibel von jeher nicht für die einzige Erkenntnißquelle der Offenbarung gehalten habe. 
\end{RWanm}

\RWpar{29}{Prüfung der angegebenen Vertheidigungsgründe}
Ihre Behauptung von der Zulänglichkeit der Bibel suchen die Protestanten durch allerlei Gründe zu rechtfertigen, die ich in Kürze noch anführen und beurtheilen will.
\begin{aufza}
\item \RWbet{Grund.} Die Bibel ist das Werk Gottes; nennen ja doch selbst die Katholiken sie Gottes Wort. Als solches muß sie denn vollkommen seyn. Vollkommen aber ist nur dasjenige, was seinem Zwecke entspricht. Also muß auch die Bibel ihrem Zwecke entsprechen, \dh\ eine hinreichende Erkenntnißquelle der göttlichen Offenbarung seyn; und die entgegengesetzte Behauptung der Katholiken setzt das göttliche Ansehen der Bibel herab.
\item[\RWbet{Widerlegung.}] Hier ist nur \RWlat{vitio subreptionis} angenommen, daß der Zweck der Bibel sey, eine \RWbet{hinreichende Erkenntnißquelle der Offenbarung abzugeben.} Gerade dieß wäre erst zu beweisen. Oben ist aber, wie ich glaube, das Gegentheil erwiesen und gezeigt worden, daß ein \RWbet{Buch überhaupt} zu diesem Zwecke nie recht taugen könne; daher es das Ansehen der Bibel nicht~\RWSeitenw{84}\ herabsetzt, wenn wir behaupten, daß auch sie diesem Zwecke nicht ganz entspreche.
\item \RWbet{Grund.} Gleichwohl erklärt sich die Bibel selbst deutlich genug für die allgemein hinreichende Erkenntnißquelle der Offenbarung, indem sie alle mündliche Ueberlieferungen als verderbliche Menschensatzungen verwirft, und es den Christen zur Pflicht macht, fleißig in ihr zu lesen. So sagt \RWbet{Jesus selbst} (\RWbibel{Joh}{Joh.}{5}{39}): \erganf{\RWbet{Forschet in der Schrift;} in ihr meinet ihr doch das ewige Leben zu haben.} \RWbet{Lukas} belobt diejenigen, die diese Pflicht erfüllen, indem er (\RWbibel{Apg}{Apostelg.}{17}{11}) von den Beroensern schreibt: \erganf{Sie dachten viel besser, als die Thessalonicenser; sie nahmen das Wort mit Bereitwilligkeit an, und \RWbet{forschten täglich in den Schriften}, ob sich Alles so verhielte.} \RWbet{Petrus} erklärt die heil.\ Schrift für ein Licht, das im Dunkeln leuchtet (\RWbibel{2\,Petr}{2\,Petr.}{1}{19}): \erganf{Wir haben ein noch festeres prophetisches Wort, und ihr thuet wohl, wenn ihr darauf achtet, als auf ein \RWbet{Licht, das im Dunkeln leuchtet,} bis der Tag anbricht, und der Morgenstern in euerem Herzen aufgehet.} \RWbet{Paulus} erklärt die heil.\ Schrift für göttliche Eingebung, und nützlich zu aller Art von Belehrung (\RWbibel{2\,Tim}{2\,Tim.}{3}{16}): \erganf{\RWbet{Die ganze Schrift ist von Gott eingegeben} und nützlich zur Belehrung, Zurechtweisung, Besserung und Bildung in der Gerechtigkeit.} Die Schrift verbietet, (\RWbibel{Dtn}{5\,Mos.}{4}{2}) etwas zu ihrem Inhalte hinzuzufügen, oder davon wegzunehmen: \erganf{Ihr sollt dem Worte, das ich spreche, weder etwas hinzufügen, noch davon wegnehmen.} Und (\RWbibel{Offb}{Offenb.}{22}{18}): \erganf{Wenn Jemand etwas hinzufügen wollte, dem wird Gott zulegen die Plagen, die in diesem Buche beschrieben sind; und wenn Jemand etwas von den Worten der Weissagung dieses Buches hinwegnehmen wollte: so würde ihm Gott nehmen seinen Antheil an dem Baume des Lebens und an der heiligen Stadt, die in diesem Buche beschrieben sind.} -- Schon \RWbet{Jesus} eiferte auf das Nachdrücklichste gegen alle mündlichen Ueberlieferungen, die er nur \RWbet{Menschensatzungen} nannte, im klaren Gegensatze von \RWbet{Gottes} Offenbarungen: \erganf{Sie (die Pharisäer und Schriftgelehrten) lehren Satzungen (sagte er \RWbibel{Mk}{Mark.}{7}{7}), die nur Menschengebote sind. Die Gebote Gottes setzen sie bei Seite, und halten auf Menschensatzungen.} Endlich ver\RWSeitenw{85}bietet es \RWbet{Paulu}s mit ausdrücklichen Worten (\RWbibel{1\,Kor}{1\,Kor.}{4}{6}), \RWbet{daß man nicht solle mehr wissen wollen, als geschrieben steht.}
\item[\RWbet{Widerlegung.}] Auch wir Katholiken geben es zu und lehren es ausdrücklich, daß es recht nützlich sey, die Schrift fleißig zu lesen; besonders zu dem Zwecke, von welchem Jesus in der angeführten Stelle spricht, um von der Wahrheit seiner göttlichen Sendung überzeugt zu werden. Und so verdienten auch die \RWbet{Beroenser} alles Lob, daß sie mit Wahrheitsliebe prüften, und, den Predigern des Christenthums nicht auf das bloße Wort glaubend, erst in den Büchern des alten Bundes nachschlugen, um sich zu überzeugen, ob sich dort Alles auch so verhalte, wie jene Prediger angaben. Für ein \RWbet{Licht}, das im Dunkeln leuchtet, erklären auch wir die Schrift; selbst wenn es zweifelhaft seyn sollte, ob \RWbet{Petrus} in der angeführten Stelle von der Bibel rede. Für \RWbet{tauglich zum Unterrichte und zur Erbauung} erklären wir die Bibel gleichfalls; daraus folgt aber nicht, daß sie ganz hinreichend seyn müsse, den Inhalt der göttlichen Offenbarung kennen zu lernen. Würde dieß aus jenen Worten folgen: so müßte man behaupten, daß schon das \RWbet{alte Testament allein} hinreichend sey; denn nur von diesem kann Paulus in jener Stelle gesprochen haben. Zu den Worten der Bibel etwas hinzuzusetzen oder von denselben etwas wegzunehmen (um dieselbe zu verfälschen), erklären auch wir für ein Verbrechen; aber das geschieht ja nicht, wenn wir zum Inhalte der \RWbet{Religion} noch Manches zählen, das die Bibel nicht enthält. Widrigenfalls hätte schon \RWbet{Jesus} nichts zum \RWbet{alten Bunde} hinzusetzen dürfen; denn die Stelle \RWbibel{Dtn}{5~Mos.}{4}{2}\ ist ja doch aus dem alten Bunde! Wir halten es endlich auch für ein Verbrechen, die Lehren der göttlichen Offenbarung mit menschlichen Meinungen zu vermengen, oder auch etwas von ihnen wegzulassen: aber wir glauben nicht, daß wir das Erstere thun, wenn wir noch nebst den Lehren, die in der Bibel stehen, gewisse andere für eine göttliche Offenbarung erklären, weil erst erwiesen seyn müßte, daß die Bibel die einzige Erkenntnißquelle der Offenbarung sey; wir fürchten vielmehr, daß wir uns, wenn wir dieß nicht thun wollten, des zweiten Fehlers schuldig machen~\RWSeitenw{86}\ würden. Wohl konnte Jesus gegen die \RWbet{Zusätze und Auslegungen} mit Recht eifern, welche die Pharisäer zur Schrift hinzuthaten, weil dieses \RWbet{unvernünftige und sittlich verderbliche Zusätze} waren. Könnte man eben dieß von den Lehren, welche die katholische Kirche mit Allgemeinheit vorträgt, beweisen: dann wäre es erlaubt, ja Pflicht, sie zu verwerfen. Wenn in der Stelle \RWbibel{1\,Kor}{1\,Kor.}{4}{6}\  von der Schrift die Rede wäre, so wäre es abermals nur die Schrift des alten Bundes. In der That aber ist hier gar nicht von der heil.\ Schrift die Rede; sondern der Sinn dieser Stelle ist bloß: Ihr sollt euch nicht höher erheben, als \RWbet{vorgeschrieben}, \dh\ geziemend ist.
\item \RWbet{Grund.} Obgleich es wahr ist, daß die Schrift einige Dunkelheit hat: so folgt hieraus doch nicht, daß sie nicht auch von dem Ungelehrtesten, der sie mit einem wahrheitsliebenden Gemüthe und mit Gebet um göttliche Erleuchtung zur Hand nimmt, verstanden werden könne. Der Allmächtige kann und der Höchstheilige muß wohl auch selbst den Ungelehrtesten erleuchten, daß er sie recht verstehe, zumal da der Getreueste es auch (\Ahat{\RWbibel{Joh}{Joh.}{14}{26}}{24,6.}) verheißen hat: \erganf{Der heilige Geist wird euch Alles lehren.}
\item[\RWbet{Widerlegung.}] Der allmächtige und höchst heilige Gott muß jeden Menschen zur Erkenntniß der ihm nothwendigen Heilswahrheiten leiten, wenn der Mensch erst von seiner Seite Alles dasjenige thut, was er vermag, um zur Erkenntniß derselben zu gelangen. Nun vermögen wir aber, wie schon gezeigt worden ist, mehr zu thun, als bloß die Bibel für uns selbst zu lesen; wir können auch Andere zu Rathe ziehen, auch ihre Meinungen vernehmen, \usw\ Als Katholiken wissen wir sogar, daß dieses unsere bestimmteste Pflicht ist; und folglich würden wir vergeblich auf die Erleuchtung des göttlichen Geistes hoffen, wenn wir uns, mit Verschmähung der übrigen Mittel, bloß an die Schrift allein halten wollten. Den Beistand aber, den Jesus in der angezogenen Stelle versprochen, hat er nicht jedem Einzelnen, sondern nur der Gesammtheit Aller versprochen; und wir sahen schon oben, wie die Apostel selbst dieses Versprechen ausgelegt, daß sie bei einer entstandenen Streitigkeit keines\RWSeitenw{87}wegs auf gut protestantische Art Jeder nur seine Bibel zur Hand genommen, sondern vielmehr Berathschlagungen gehalten, und erst dasjenige, worüber sich die Meinung Aller vereinigte, als eine Eingebung des göttlichen Geistes angesehen haben. -- Aus dem Beweisgrunde der Protestanten würde, wofern er richtig wäre, die Ungereimtheit folgen, daß auch derjenige, der nicht einmal lesen kann, wenn er die Bibel zur Hand nimmt, sie durch ein wahres Wunderwerk verstehe! --
\item \RWbet{Grund.} Wahr ist es, daß die Schrift auch selbst in einigen, den \RWbet{Glauben} betreffenden Gegenständen dunkel und undeutlich ist, besonders für den ganz Ungelehrten. Allein man muß nur einen Unterschied zwischen den Glaubenslehren machen. Es gibt gewisse \RWbet{Fundamentalartikel,} die jedem Menschen zu wissen nöthig sind; und diese sind in der heil.\ Schrift so deutlich ausgesprochen, daß sie auch von dem Ungelehrtesten verstanden werden können. Nebst diesen aber gibt es auch andere \RWbet{Nichtfundamentalartikel,} \dh\ Lehren, welche zwar auch die Religion betreffen, aber entweder überhaupt, oder doch für den Ungelehrten von keiner Nothwendigkeit sind. Diese sind freilich nicht alle mit der vollkommensten Deutlichkeit in der Schrift enthalten; aber hieraus erwächst kein wesentlicher Schaden.
\item[\RWbet{Widerlegung.}] Der Unterschied, den man hier zwischen Fundamental- und Nichtfundamentalartikeln macht, ist nicht ganz deutlich. Versteht man ihn so, daß erstere Lehren seyn sollen, die von größerer, letztere solche, die von geringerer Wichtigkeit sind: so erinnere ich, daß schon aus den oben gegebenen Beispielen zu ersehen sey, es gebe manche Frage von großer Wichtigkeit, darüber sich die Schrift nicht deutlich erkläret. Meint man dagegen, die Nichtfundamentalartikel sollen Erklärungen seyn, welche für Niemand von einem Nutzen, geschweige denn nothwendig wären: so sage ich, daß es dergleichen gar nicht in einer Offenbarung geben könne, daß aber die Bibel uns in der That über Manches in Ungewißheit lasse, was uns zu wissen von Wichtigkeit ist, und was der Gesammtglaube der Katholiken auf das Bestimmteste entscheidet. -- Versteht man endlich unter den Fundamentalartikeln Lehren, welche für Alle, unter den Nichtfundamental\RWSeitenw{88}artikeln aber solche, die bloß für den Gelehrten von Wichtigkeit sind: so gebe ich das Daseyn eines solchen Unterschiedes allerdings zu; erinnere aber zugleich, daß man hier unter einem \RWbet{Gelehrten} jeden im Denken geübten, vielseitig ausgebildeten Mann verstehen müsse, gleichviel, ob er sich eben auf Schriftstudium verlegt hat, oder nicht. Jeder im Denken Geübte hat nämlich das Bedürfniß einer vollständigeren religiösen Belehrung als der Ungeübte. Allein wofern es wahr ist, daß die Bibel in jenen Lehren, welche nur dem Gelehrten wichtig sind, dunkel und undeutlich ist: so wird nicht jeder Gelehrte, sondern höchstens nur derjenige, der sich mit Schriftgelehrsamkeit beschäftiget, seine Bedürfnisse durch die Lesung der Bibel befriedigen können. Die Protestanten scheinen sich irriger Weise vorzustellen, als ob jene höheren Bedürfnisse nur für den \RWbet{Theologen} vorhanden wären. Aber so ist es nicht. Wohl können gewisse \RWbet{historische }Fragen (\zB\ über die Aechtheit dieses oder jenes Buches \udgl ) eine ausschließliche Wichtigkeit nur für den Schriftforscher als solchen haben; aber dieß sind nicht -- \RWbet{Glaubenslehren.} Für \RWbet{religiöse Lehrsätze} dagegen hat jeder Andere, der sich mit dem gelehrten Schriftforscher auf einer gleichen Stufe der Geistesbildung befindet, auch gleiches Bedürfniß; und dennoch der Nichttheolog kein hinlängliches Mittel, es zu befriedigen, wenn die Bibel die einzige Erkenntnißquelle der göttlichen Offenbarung ist.
\item \RWbet{Grund.} Es ist ein verkehrter Weg, wenn sich der Mensch anmaßen will, aus \RWbet{Gründen} \RWlat{a priori} zu bestimmen, was Gott für die Verbreitung seiner Offenbarung unter den Menschen habe veranstalten oder nicht veranstalten müssen, und daß er kein Buch zu diesem Zwecke habe erwählen dürfen. Was Gott thun oder nicht thun solle, darüber vermag der menschliche Verstand nichts zu entscheiden. Genug, wir finden, Gott habe wirklich für die Erhaltung, Ausbreitung und Vervollkommnung seiner Offenbarung nichts Anderes gethan, als daß er die Bibel hat schreiben, und bis auf unsere Zeiten gelangen lassen. Außerhalb dieser Bibel finden wir nämlich sonst keine andere sichere Quelle, aus der wir die Lehre Jesu zu schöpfen vermöchten.
Die Bibel muß~\RWSeitenw{89}\ uns also die völlig hinreichende Erkenntnißquelle aller Offenbarung seyn, weil sie die \RWbet{einzige} ist.
\item[\RWbet{Widerlegung.}] 
\begin{aufzb}[1.] \item Es gibt Fälle, in denen der menschliche Verstand, ohne \RWbet{anmaßend} zu seyn, bald mit Wahrscheinlichkeit, bald auch mit völliger Gewißheit behaupten kann, daß Gott dieß thun, und jenes nicht thun dürfe. So glauben wir Alle ganz ohne Anmaßung behaupten zu dürfen, daß Gott jedes Gute belohnen und jedes Böse bestrafen müsse; \udgl\  So können wir auch mit einem hohen Grade von Wahrscheinlichkeit behaupten, daß Gott, wenn es ihm darum zu thun ist, eine geoffenbarte Religion, die für alle Zeitalter bestimmt seyn soll, in die Welt einzuführen, mehr thun werde, als bloß ein Buch schreiben zu lassen, in dem die vornehmsten Lehren dieser Offenbarung verzeichnet sind. Wir können im Voraus vermuthen, daß er die Menschen veranlassen werde, nicht unverrückt stehen zu bleiben bei dem, was ihre Vorfahren einmal geglaubt, daß er den Inhalt seiner Offenbarung von Zeit zu Zeit nach den Bedürfnissen der Menschen erweitern werde, \usw\ Aus der besonderen Beschaffenheit der Bibel überzeugen wir uns nun immer stärker, daß sie Gott unmöglich zu dem Zwecke bestimmt haben könne, die einzig hinreichende Erkenntnißquelle der Offenbarung für uns abzugeben. Endlich gerathen wir in dieser Bibel sogar auf Stellen, aus denen wir ersehen, daß Jesus der religiösen Gesellschaft, die er gestiftet, den Geist der Wahrheit auf ewige Zeiten versprochen habe, was könnte nun weniger anmaßend seyn, als wenn wir hieraus schließen, daß nicht irgend ein einzelner Christ, und wäre er auch ein Apostel, wohl aber die ganze Gesellschaft nur in demjenigen, was sie mit Allgemeinheit glaubt, unfehlbar sey?
\item Uebrigens ist es eine sehr willkürliche Behauptung, in diesem Einwurfe, daß es \RWbet{außer der Bibel} gar keine andere sichere Quelle gebe, aus der wir erfahren könnten, was Jesus vorgetragen habe. Wie? sollten wir nicht, auch wenn die Bibel uns verloren ginge, bloß aus dem einstimmigen Berichte der Schriftsteller der ersten christlichen Jahrhunderte mit ziemlicher Sicherheit Manches von dem, was Jesus gelehret oder nicht gelehret habe, entnehmen können?~\RWSeitenw{90}\ Ist dieß nicht eben derselbe Weg, auf dem wir erfahren, was der oder jener Weltweise des Alterthums, von dem wir keine eigene Schriften haben, \zB\ Thales, Anaxagoras, Pythagoras, Sokrates \uA\ gelehret haben?
\item Will man aber nur sagen, daß diese anderen Quellen, die neben der Bibel fließen, uns eben \RWbet{nicht mehr} mittheilen, als wir schon aus den Evangelien wissen; so ist auch dieses falsch. Denn Vieles in den Büchern des neuen Bundes würde uns dunkel und unverständlich seyn, wenn wir nicht aus den Schriftstellern der ersten christlichen Jahrhunderte die nöthigsten Aufschlüsse darüber erhielten, wie dieß die oben angeführten Beispiele vom heil.\ Abendmahle, von der Taufe, der Händeauflegung \usw\ hinlänglich darthun.
\item Doch gesetzt, wir könnten wirklich nicht mit Gewißheit nachweisen, daß diese oder jene Lehre der Kirche, die nicht in der Bibel steht, von \RWbet{Jesu} herrühre: was verschlüge dieß denn, sobald wir wissen, daß der Gesammtglaube der Katholiken eine \RWbet{wahre göttliche Offenbarung} sey? Erst müßte man darthun, daß der katholischen Kirche dieser Vorzug nicht wirklich zukomme; dann möchten jene Gründe von einiger Wichtigkeit seyn.
\end{aufzb}
\item \RWbet{Grund.} Es ist freilich wahr, daß die Bibel die Lehren der göttlichen Offenbarung weder \RWbet{vollständig,} noch ganz \RWbet{deutlich} vorträgt. Aber sie liefert uns doch den \RWbet{obersten Grundsatz} der ganzen Offenbarung, \dh\ den Satz, aus dem alle übrigen Lehren derselben durch Schlüsse herleitbar sind, den Satz, der allen Nebensätzen nicht nur ihre gehörige Deutung gibt, sondern auch jede Mißdeutung und jeden fremden Einschub willkürlicher Satzungen verhütet. Dieser oberste Grundsatz lautet: \erganf{\RWbet{Liebe Gott, und deinen Nächsten, wie dich selbst}}; oder in philosophischer Sprache: \erganf{\RWbet{Denke und handle nach dem praktischen Gesetze der Freiheit, und erkenne in diesem den Willen Gottes.}} (\RWbet{J.\,H.\,Tieftrunk}: Censur des christlich-protestantischen Lehrbegriffes. Berlin 1796. 1.\,B. S.\,361\,ff.\RWlit{}{Tieftrunk1}).
\item[\RWbet{Widerlegung.}] Gesetzt, es wäre wahr, daß es nur einen einzigen gemeinschaftlichen Ableitungsgrund aller Lehren der göttlichen Offenbarung gebe, und der von Tieftrunk auf\RWSeitenw{91}gestellte Satz wäre dieß einzige Princip: so entstände doch erst die Frage, ob auch wir Menschen im Stande sind, diese Herleitung vorzunehmen? Bekanntlich werden zu jeder Herleitung einer Folge aus einem Obersatze noch ein oder mehrere Untersätze erfordert. Denn aus einem einzelnen Satze allein (dem obersten Princip) folgt nichts. Je nachdem man aber bald diese, bald jene (richtigen oder auch unrichtigen) Untersätze hinzunimmt, ergeben sich bald diese, bald jene (richtigen oder unrichtigen) Folgesätze. Die Kenntniß des obersten Princips der göttlichen Offenbarung allein würde uns also noch wenig nützen, wenn wir nicht auch die Untersätze alle erführen, die wir unter dasselbe zu subsumiren haben, um alle Lehren der göttlichen Offenbarung vollständig zu erhalten. Tieftrunk hätte denn also beweisen sollen, daß auch diese Untersätze alle, entweder in der Bibel deutlich enthalten, oder uns aus der bloßen Vernunft bekannt sind. Weder das Eine, noch das Andere hat er gethan. -- Aber selbst wenn dieß wäre, so käme es noch darauf an, ob wir diese Herleitung aus den gegebenen Vordersätzen wirklich vornehmen und dabei richtig verfahren? Wem es bekannt ist, wie träge und ungeschickt die Menschen sich oft bei Herleitung auch selbst der einleuchtendsten Folgerungen aus bekannten Vordersätzen bezeigen; der wird sich kaum versprechen, daß durch die Aufstellung eines solchen obersten Ableitungsgrundes aller Offenbarungslehren den Irrthümern und Uneinigkeiten unter den Menschen abgeholfen seyn würde.
\end{aufza}

\RWpar{30}{Geständnisse über die Vernunftmäßigkeit der katholischen Lehre von den Erkenntnißquellen der göttlichen Offenbarung}
Die katholische Lehre von den Erkenntnißquellen der göttlichen Offenbarung ist, wie wir jetzt gesehen haben, so vernunftmäßig, so folgerecht und so zusagend den Bedürfnissen des Menschen, daß man sich eben nicht zu wundern braucht, wenn unparteiische Gelehrte dieß zuweilen anerkannt haben, ob sie gleich einer andern Kirche zugethan waren. Einige der merkwürdigsten Geständnisse dieser Art sollen zum Schlusse noch folgen.~\RWSeitenw{92}
\begin{aufza}
\item \RWbet{Tieftrunk} in seiner \RWbet{Censur des christlich-pro\-tes\-tan\-ti\-schen Lehrbegriffes} behauptet (1.\,Bd.\ S.\,10\,ff.)\RWlit{}{Tieftrunk1}, daß man nach protestantischen Begriffen (\dh\ wenn man die Unfehlbarkeit der Kirche in ihrem Gesammtglauben läugnen will) stets uneinig im Christenthume bleiben müsse, es sey denn, daß man irgend ein einziges Vernunftprincip annehme, aus welchem sich alle Lehren des Christenthums durch bloße menschliche Vernunftschlüsse herleiten lassen. -- Allein wir haben so eben die Untauglichkeit eines solchen Mittels erkannt. Auf jeden Fall ist doch so viel gewiß, wofern ein solches Princip (wie Tieftrunk stillschweigend vorauszusetzen scheint) durch bloße Vernunft erkennbar seyn soll, und wenn eben so auch die dazu nöthigen Untersätze durch die sich selbst überlassene Vernunft sollen aufgefunden werden können: so wird die ganze Offenbarung in eine bloß \RWbet{natürliche} Religion verwandelt. Tieftrunk will gleichwohl, daß man so etwas, wie \RWbet{Symbole} (allgemeine Glaubensbekenntnisse) allezeit beibehalte, \RWbet{weil diese doch immer besser wären, als gänzliche Anarchie.} Allein dieß gilt nur dann, wenn wir voraussetzen dürfen, daß diese mit Uebereinstimmung Aller entworfenen Symbole göttliche Offenbarungen sind, und somit nähert sich Tieftrunk gar sehr der katholischen Ansicht. Daher sagt er denn auch (S.\,178.): \erganf{Ich will den sehen, welcher mir das Papstthum mit aller seiner hierarchischen Politik (er versteht darunter hier vornehmlich die Macht des Papstes, Glaubenslehren festzusetzen) wegvernünfteln soll, wenn er mir einmal zugestanden hat, daß die Theologie das Principium der ganzen Religion sey} (\dh\ daß die Lehren des Christenthums nicht aus der \RWbet{Vernunft allein} hergeleitet werden können, sondern auf \RWbet{Gottes Zeugniß} angenommen werden sollen, \dh\ Offenbarungen sind).
\item Eben so schließt \RWbet{Reinhold} in seinen \RWbet{Bemerkungen über die kritische Philosophie} (1.\,Thl.\ S.\,197.)\RWlit{}{Reinhold1}: Wenn eine Religion Geheimnisse enthält, wenn sie ihren Glauben durch Wunder erlangt (er will sagen, wenn sie materielle Offenbarung seyn will): so ist das System der Unfehlbarkeit das einzig mögliche. -- Das System der Unfehlbarkeit ist \erganf{das einzige unter allen auf historischem Grunde entstandenen~\RWSeitenw{93}\ Lehrgebäuden (\dh\ deren Lehren nicht \RWlat{a priori} eingesehen werden können, sondern auf Zeugnissen beruhen), welches durch den Zusammenhang und die Gleichartigkeit seiner Theile den Namen eines Systems verdient}.
\item \RWbet{Wieland} (im deutschen Merkur 1788.\ Jul. Gedanken von der Freiheit, in Glaubenssachen zu philosophiren. S.\,13.)\RWlit{}{Wieland3}: \erganf{Bey dieser unläugbaren und weltbekannten Beschaffenheit der Sache bleibt also -- so viel ich wenigstens begreifen kann -- in Ansehung Alles dessen, was in der heil.\ Schrift dunkel, vieldeutig und geheimnißvoll, im Widerspruch mit allgemeinen Vernunft- und Erfahrungswahrheiten, oder mit andern Stellen der Bibel selbst, mit Einem Wort, was nicht allgemein faßlich und verständlich ist -- nichts übrig, als diese Alternative, entweder sich einem unfehlbaren Richter in Glaubenssachen, der allein über den Sinn zweifelhafter Worte und Sätze zu entscheiden berechtigt ist, zu unterwerfen; oder Allen [, die darin mit uns übereinstimmen, daß sie sich zur Religion Christi halten, und keinen unfehlbaren Richter in Sachen des Glaubens über sich erkennen, das Recht, nach ihrer eigenen Überzeugung zu glauben, oder (welches einerley ist,)] das Recht, sich über alles dunkle und unbegreifliche der Religion diejenige Vorstellungsart zu machen, die ihnen die richtigste scheint, [(wie verschieden sie auch von der unsrigen seyn mag) einzugestehen, sie dieser Verschiedenheit ungeachtet für unsere Brüder zu erkennen, und, durch diese dem Geist Christi höchstgemäße Sinnesart, allen gehässigen Zänkereyen, Verketzerungen, und Verfolgungen, sammt allem in der bürgerlichen und christlichen Gesellschaft daraus entstehenden Unheil, auf einmal und auf ewig ein Ende zu machen.] Wollen wir die erste Parthey ergreifen, so sehe ich dann keine neue Alternative mehr. Dann bleibt uns nichts übrig, als geraden Weges uns zu den Füßen des dreymal gesegneten Vaters in dem dreyfach gekrönten Heiligthum zu werfen (warum gerade zu des Papstes Füßen?), uns mit unsrer guten alten Mutter \RWlat{Sainte Eglise} aussöhnen zu lassen, und zu glauben, was sie uns zu glauben befiehlt. -- Wer darf so dreiste seyn, seinen Verstand, seine Einsichten nicht nur zum Maaßstabe, sondern sogar zur Regel und zum Gesetz aller übrigen zu machen?}
\item \RWbet{Friedrich Köppen} (Philosophie des Christenthums. Leipzig, 1816. 2.\,Thl.\ S.\,91\,ff.)\RWlit{}{Koeppen2}: \erganf{Darum lehrt die katholische Dogmatik ganz folgerecht, daß der heil.\ Geist sich in der christlichen Kirche erhalte, und fortwährend wirksam sey; von ihm geleitet redeten und schrieben die Apostel, von ihm durchdrungen reden, schreiben und handeln die Glieder des Christenvereins. -- --
Warum wäre eine mündliche Ueberlieferung nicht eben so belehrend und erweckend, als eine geschriebene? Warum darf nicht bei dem Vorhandenseyn von apostolischen Schriften eine Kirchentradition als zweite Quelle~\RWSeitenw{94}\ der Erkenntniß gebraucht werden, wenn etwa in der ersten noch nicht genug Entscheidung vorhanden ist? Erwägend ferner die Wandelbarkeit menschlicher Gesinnungen, die Fehlgriffe eines irre geleiteten Nachdenkens, die Leidenschaften einzelner Menschen, Secten und ihren Einfluß: so muß eine Auctorität, welche die wahre Bedeutung und Auslegung des göttlichen Wortes aufrecht erhält, für die Erleuchtung der Erwählten sehr wünschenswerth scheinen. Christus und seine Apostel waren eine solche Auctorität für ihr Zeitalter; sollte sie hernach ganz gefehlt haben, da sie aus vielen Rücksichten immer nothwendiger und unentbehrlicher wurde?} --~\RWSeitenw{95}
\end{aufza}


\RWch{Zweites Hauptstück.\\ Christkatholische Dogmatik.}
\RWpar{31}{Inhalt und Unterabtheilungen dieses Hauptstückes}
\begin{aufza}
\item Wie die Dogmatik einer jeden Religion nur die theoretischen Lehren derselben enthält, so zähle ich auch zur christkatholischen Dogmatik alle \RWbet{bloß theoretischen Lehren} des Katholicismus mit Ausnahme der bereits vorgetragenen Lehre von den Erkenntnißquellen seiner gesammten Lehren, und mit Ausnahme noch einiger anderer, die mit gewissen practischen so innig zusammenhängen, daß ich sie füglicher mit diesen zugleich, also erst in dem dritten Hauptstücke vortragen werde.
\item Auch nach dieser Absonderung sind die Lehren, die in dem gegenwärtigen Hauptstücke abgehandelt werden sollen, noch so zahlreich, daß es zu ihrer leichtern Uebersicht dient, sie in gewisse Unterabtheilungen zu bringen. Ich habe folgende sechs gewählt:
\begin{aufzb}
\item \RWbet{Lehren von Gott,} (Katholische Theologie.)
\item \RWbet{Lehren von der Welt und den Geschöpfen}, (Katholische Kosmologie und Dämonologie.)
\item \RWbet{Lehren vom Menschen insbesondere,} (Katholische Anthropologie.)
\item \RWbet{Lehren von den Verhältnissen Gottes zu uns Menschen}, oder von der Art, wie Gott zur Beglückung der Menschen wirke, (Katholische Soterologie.)
\item \RWbet{Lehren von den Verhältnissen der Geschöpfe unter einander}, oder \RWbet{von der Gemeinschaft}.~\RWSeitenw{96}
\item \RWbet{Lehren von den Belohnungen und Strafen} in diesem und jenem Leben, (Katholische Eschatologie.)
\end{aufzb}
\item Gerne bescheide ich mich, daß bei diesen sechs Unterabtheilungen, so wie auch bei der von mir getroffenen Sonderung zwischen den Lehren, die ich in dieses oder ein anderes Hauptstück beziehe, keine so scharfe Begrenzung Statt finde, daß sich nicht über manche Lehre der Streit erheben ließe, daß man sie eben so bequem an einem andern Orte hätte abhandeln können. So sind \zB\ die Lehren von Gott, die in dem ersten Abschnitte vorkommen sollen, im Grunde doch auch nur Verhältnisse Gottes zu uns; denn auch in diesem Abschnitte wird nicht von Gott, wie er an sich ist, sondern wie wir Menschen uns ihn vorstellen sollen, gehandelt; daher man sagen könnte, der erste und vierte Abschnitt wären im Grunde nicht unterschieden. Inzwischen liegt doch der Unterschied darin, daß in dem \RWbet{ersten} Abschnitte die Lehren abgehandelt werden, welche bestimmen, was für \RWbet{bleibende} Beschaffenheiten wir in Beziehung auf uns in Gott annehmen sollen; im \RWbet{vierten} aber, welche \RWbet{zeitweilige} Einwirkungen Gottes auf uns wir uns zu Folge jener bleibenden Beschaffenheiten vorzustellen haben. Ueberhaupt kommt ja nicht so viel darauf an, an welchem Orte eine Lehre vorgetragen wird, als nur darauf, daß keine übergangen, jede in ihr gehöriges Licht gesetzt werde, und nicht durch eine einseitige Darstellung verliere.
\end{aufza}

\RWpar{32}{Begriff der Geheimnißlehren}
\begin{aufza}
\item Ein großer Theil der Lehren, die wir von jetzt an vorzutragen haben, wird von den Lehrern des katholischen Christenthums mit dem Namen der \RWbet{Geheimnißlehren} bezeichnet. Da nun dieser Theil der Lehren, ja schon ihr bloßer Name in neuerer Zeit häufig angefochten wurde: so halte ich es für nöthig, erst eine genauere \RWbet{Erklärung des Begriffes,} den man mit solchen Geheimnißlehren verbindet, anzugeben; dann aber Einiges, ihre \RWbet{Vernunftmäßigkeit} und ihren \RWbet{sittlichen Gebrauch} betreffend, vorläufig zu erinnern.~\RWSeitenw{97}
\item Wenn man von irgend einer Behauptung oder Wahrheit weiß, \RWbet{auf welchem Grunde} sie beruht, \dh\ \RWbet{warum} die Sache sich so und nicht anders verhalte, als es in dieser Behauptung ausgesagt wird: so haben wir sie begriffen. -- \RWbet{Begreifen} ist also in dieser Bedeutung vom bloßen \RWbet{Verstehen} sowohl, als vom \RWbet{Erkennen} (einer Wahrheit) wohl zu unterscheiden. Man \RWbet{versteht} einen Satz, wenn man den \RWbet{Sinn} desselben faßt; und man \RWbet{erkennt} seine Wahrheit, wenn man von seiner Richtigkeit \RWbet{überzeugt} ist. Daß dieses Letztere geschehen könne, ohne den eigentlichen \RWbet{Grund der Wahrheit} zu erfahren, ist gleich in der Einleitung (\RWparnr{3}) bemerkt worden. Erfährt man aber auch diesen Grund der Wahrheit selbst: so hat man sie \RWbet{begriffen.} Eine Wahrheit nun, deren Grund man erkennt, heißt eine begriffene, eine Wahrheit, deren Grund man nicht erkennt, eine unbegriffene; und eine Wahrheit, deren Grund entweder an sich, oder doch für uns Menschen nicht erkennbar ist, heißt eine entweder an sich, oder doch für uns Menschen \RWbet{unbegreifliche} Wahrheit.
\item Aus diesen Erklärungen ergibt sich, daß der Zweck einer \RWbet{Wissenschaft} (in dieses Wortes strengstem Sinne, Einleitung \RWparnr{2}) kein anderer sey, als daß wir die Wahrheiten, welche sie aufstellt, begreifen. Allein nicht jede Wahrheit kann begriffen werden. So sind \zB\ alle \RWbet{Grundwahrheiten} auch eben darum unbegreifliche Wahrheiten, und dieß zwar nicht bloß für uns, sondern für jedes denkende Wesen, oder an sich; indem sie gar keinen Grund ihrer Wahrheit haben. Sie können also zwar wohl erkannt, aber nicht begriffen werden, weil nichts an ihnen zu begreifen ist.
\item Wenn wir von einer Wahrheit zwar nicht den vollständigen Grund, auf welchem sie beruht, aber doch Einiges, das zu ihrem Grunde gehört, erkennen: so sagt man von ihr, daß wir sie einiger Maßen begreifen. So sage ich \zB\ von der mathematischen Wahrheit: Das Quadrat der Hypotenuse gleicht der Summe der Quadrate der beiden Katheten, daß ich sie einiger Maßen begreife, wenn ich zum Wenigsten erkenne, daß das Quadrat der Hypotenuse allezeit größer seyn müsse, als das Quadrat jedes einzelnen Katheten, weil die~\RWSeitenw{98}\ Hypotenuse die größte Seite im Dreiecke ist. Eben so sage ich von der empirischen Wahrheit: Sempronius hat sich ermordet, ich begreife dieß einiger Maßen, wenn ich erfahre, daß Sempronius kurz vorher einen großen Geldverlust erlitten habe, \udgl\ 
\item Wenn wir den Grund einer Wahrheit, die gleichwohl einen hat (die also keine Grundwahrheit ist), ganz und gar nicht einsehen können, oder doch nur äußerst Weniges von ihm erkennen: so nennen wir sie ein \RWbet{Geheimniß,} auch wohl (in Dingen von minderer Wichtigkeit) ein \RWbet{Räthsel}. So sagen wir von einem Selbstmorde, er sey uns ein Geheimniß oder ein Räthsel, wenn wir den Grund, der ihn veranlaßt hat, so gar nicht einsehen können. So sagen wir auch, daß die Kraft, wodurch der Magnet das Eisen an sich zieht, uns ein Geheimniß oder ein Räthsel sey, weil wir noch äußerst wenig wissen, wie und wodurch der Magnet diese Wirkung hervorbringt, \usw\
\item Und so verstehen wir unter einer \RWbet{Geheimnißlehre der Offenbarung} gleichfalls nichts Anderes, als eine Lehre, deren inneren Grund wir Menschen nicht zu erreichen vermögen; von der wir zwar aus Gottes Zeugnisse wissen, \RWbet{daß} es so sey, wie sie angibt, aber nicht einzusehen vermögen, \RWbet{warum}, oder \RWbet{aus welchem Grunde} es so sey.
\end{aufza}

\RWpar{33}{Einige allgemeine Bemerkungen über die Vernunftmäßigkeit der Geheimnißlehren}
Die Geheimnißlehren des Christenthums sind es, an welchen die Gegner desselben, besonders diejenigen, die sich das Ansehen von Weltweisen gegeben, von jeher das meiste Aergerniß genommen haben; und doch sind, wie mir däucht, folgende Bemerkungen schon hinreichend, um für einen Jeden, der sie wohl aufgefaßt hat, allen Anstoß, den ihm solche Lehren verursacht haben möchten, für immer zu beheben.
\begin{aufza}
\item Bloß aus dem Umstande, daß eine gewisse Lehre des Christenthums ein \RWbet{Geheimniß} für uns ist, \RWbet{folgt gegen die Wahrheit und Vernunftmäßigkeit derselben}~\RWSeitenw{99}\ \RWbet{gar nichts;} denn eine ganz unzertrennliche Folge von der Eingeschränktheit unserer Vernunft ist es ja, daß wir sehr viele Wahrheiten gar nicht, und auch von vielen, welche wir einsehen, doch nicht den eigentlichen Grund derselben einsehen können, \dh\ daß es in allen Fächern des menschlichen Wissens eine beträchtliche Anzahl \RWbet{Geheimnisse} für uns gebe, und geben müsse. Dieses bestätiget auch die Erfahrung, indem es in der Naturlehre, in der Naturgeschichte, in der Arzneikunde, und in allen empirischen Wissenschaften viele hundert Wahrheiten gibt, die wir mit aller Gewißheit erkennen, und deren Grund wir gleichwohl nicht einzusehen vermögen, die also Geheimnisse sind. Daß es insonderheit auch in der \RWbet{Offenbarung} Geheimnisse geben werde, läßt sich im Voraus erwarten. Schon unter den Lehren der natürlichen Religion, von deren Wahrheit wir uns doch ohne ein göttliches Zeugniß überzeugen können, gibt es verschiedene, deren inneren Grund wir nicht einsehen, die also Geheimnisse für uns sind. Eine wahre Offenbarung aber muß nicht nur alle Lehren, welche die natürliche Religion enthält, aufnehmen, sondern (wofern sie eine nicht bloß formelle, sondern auch eine materielle Offenbarung seyn will) noch manche andere Lehren hinzuthun. Diese nun müssen eben deßhalb, weil wir sie ohne ein göttliches Zeugniß gar nicht erkennen würden, ihrem inneren Grunde nach uns völlig unbekannt, folglich Geheimnisse für uns seyn.
\item Selbst wenn sich zeigen ließe, daß eine Lehre des Christenthums ihrem \RWbet{buchstäblichen} Sinne nach auf einen \RWbet{Widerspruch} führt, würde hieraus nicht das Geringste gegen ihre Vernunftmäßigkeit folgen, so ferne sie nur sittliche Zuträglichkeit hat. Aus diesem Widerspruche nämlich würde höchstens folgen, daß jene Lehre \RWbet{nicht buchstäblich} zu verstehen sey, sondern zur Classe der \RWbet{bildlichen,} \dh\ derjenigen Lehren gehöre, in denen uns ein Gegenstand nicht, wie er an sich ist, sondern wie seine Vorstellung die wohlthätigsten Wirkungen bei uns hervorbringt, und eben um dieser Wirkungen wegen dargestellt wird. Solcher bildlichen Lehren gibt es und muß es in jeder guten Religion sehr viele geben; denn schickliche Bilder sind doch das wirksammste Mittel, um uns zu rühren und recht lebhafte Gefühle und Entschließungen~\RWSeitenw{100}\ in uns hervorzubringen. Eine Religion also, die dieses vortrefflichen Mittels sich nicht bedienen, und unsere Einbildungskraft ganz unbenützt lassen wollte, würde ihrem Zwecke sehr schlecht entsprechen. Der Fehler wäre um desto größer, je mehr es gerade die \RWbet{abstracten} Gegenstände der Religion bedürfen, daß sie durch bildliche Vorstellungen belebt und wirksam gemacht werden. Sollen die Wahrheiten von Gott, vom andern Leben \usw\ ohne alle bildliche Zusätze aufgefaßt werden: so wirken sie nur sehr wenig; durch Verbindung mit schicklichen Bildern dagegen können sie die lebhaftesten und die wohlthätigsten Rührungen in uns hervorbringen. Dem katholischen Christenthume kann man nun nachsagen, daß es den zweckmäßigsten Gebrauch von dieser Wirksamkeit der Bilder mache. Daß es aber nicht immer ausdrücklich anzeigt, ob eine gewisse Lehre eine bloß bildliche sey, darüber kann man demselben schon darum keinen Vorwurf machen, weil eine solche Anzeige für den Gebildeten entbehrlich, und für den Ungebildeten öfters anstößig wäre.
\end{aufza}

\RWpar{34}{Sittliche Vortheile, die allen Lehrsätzen einer materiellen Offenbarung gemeinschaftlich zukommen}
Außer den \RWbet{eigenthümlichen} sittlichen Vortheilen, die jede einzelne der uns geoffenbarten Lehren nach ihrer besondern Beschaffenheit hat, gibt es gewisse Vortheile, die \RWbet{allen} Lehrsätzen einer materiellen Offenbarung schon darum, weil es Lehrsätze einer materiellen Offenbarung, \dh\ Lehren sind, von deren Wahrheit wir uns nur durch das Zeugniß Gottes überzeugen können, \RWbet{gemeinschaftlich} zukommen. Dieser erwähne ich denn gleich hier, um sie in der Folge nicht bei jeder einzelnen Lehre wiederholen zu müssen.
\begin{aufza}
\item Alle Lehrsätze einer materiellen Offenbarung geben uns Gelegenheit zu einer pflichtmäßigen und uns zum Verdienste gereichenden \RWbet{Richtung unserer Aufmerksamkeit und unseres Nachdenkens auf sie.} Wenn wir uns nämlich erst überzeugt haben, daß eine gewisse Lehre wirklich geoffenbart sey, daß Gott sie von uns geglaubt wissen wolle: so entsteht eben hieraus für uns die Pflicht, diese Lehre jetzt~\RWSeitenw{101}\ durch öftere Wiederholung, und durch Verbindung mit unseren übrigen Begriffen in unser Eigenthum zu verwandeln, und sie zur Richtschnur unseres Lebens anzunehmen. Thun wir dieses nun wirklich: so machen wir einen pflichtmäßigen Gebrauch von unserer Aufmerksamkeit und unserem Nachdenken, der uns zu einem Verdienste gereicht, und die Gelegenheit zu diesem Verdienste gab jene Offenbarungslehre.

\begin{RWanm}  Man hat diesen pflichtmäßigen Gebrauch unserer Aufmerksamkeit und unseres Nachdenkens zuweilen auch als einen \RWbet{Dienst} betrachtet, den wir Gott selbst durch unsere Denkkraft leisten; und Einige haben hieraus sogar die Nothwendigkeit einer Offenbarung herleiten wollen. Weil nämlich Gott der oberste Herr aller Geschöpfe ist, und alle unsere Kräfte nur sein Geschenk sind: so wäre es (meinten sie) nothwendig, daß wir ihm auch mit \RWbet{allen} unseren Kräften dienen. Damit wir ihm auch mit unserem \RWbet{Verstande} dienen, oder die Oberherrlichkeit Gottes auch über unsere Denkkraft zu erkennen geben könnten: wäre es nothwendig, daß uns Gott in einer eigenen Offenbarung gewisse Wahrheiten zu glauben auflege, die wir durch unsere bloße Vernunft nicht zu erkennen vermögen. -- Hiegegen ist nun meines Erachtens zu bemerken, daß der Begriff eines \RWbet{Dienstes,} den wir Gott leisten, ein Anthropomorphismus sey. Nicht für Gott, sondern für uns erwächst ein Vortheil daraus, wenn wir an seine Offenbarung glauben. Nichts desto weniger kann man diesen Glauben eine Pflicht gegen Gott nennen, wenn wir unter einer Pflicht gegen diesen oder jenen Gegenstand eine Pflicht verstehen, die uns ein diesen Gegenstand betreffendes, oder auf ihn sich beziehendes Verhalten vorschreibt, in welcher Bedeutung es auch Pflichten gegen leblose Gegenstände, \zB\ gegen das Eigenthum, gegen die Ehre \udgl\  gibt.\end{RWanm}
\item Alle Lehren einer materiellen Offenbarung \RWbet{erhöhen und beleben unsere Ueberzeugung von Gottes Vollkommenheit, und von der überschwenglichen Erhabenheit des göttlichen Verstandes über den menschlichen.} Jede dieser Lehren gibt uns nämlich ein neues Beispiel von einer Wahrheit, die unser menschliche Verstand nicht zu begreifen vermag, der göttliche aber durchschaut. Eine vermehrte Ueberzeugung von Gottes Erhabenheit über uns ist aber unserer Tugend und Glückseligkeit in~\RWSeitenw{102}\ mehr als einer Hinsicht beförderlich. Sie stärkt \zB\ unser Vertrauen auf Gott auch in denjenigen Fällen, wo wir die Absichten seiner Führungen mit uns nicht zu begreifen vermögen.
\item Alle Lehren einer materiellen Offenbarung \RWbet{erhöhen in dem vernünftigen Menschen die Begierde nach einer vollkommeneren Erkenntniß, und nach dem andern Leben,} wo sie ihm werden soll. Das Erstere, weil sie Lehrsätze sind, von deren Wahrheit wir zwar durch Gottes Zeugniß versichert sind, deren innere Gründe wir aber jetzt noch nicht einsehen können. Das Zweite, weil uns verheißen ist, daß wir im anderen Leben zu dieser Einsicht gelangen sollen. Eine bescheidene Begierde nach vollkommener Einsicht aber, und ein gemäßigtes Verlangen nach dem anderen Leben sind unserer Tugend und Glückseligkeit entschieden vortheilhaft. Beide ziehen uns von sinnlichen Begierden ab; denn je mehrere Wünsche der Mensch kennt, um so weniger hängt er an Einem; und wer des andern Lebens öfters gedenkt, wird die Freuden der Sinne gewiß nicht mit Unmäßigkeit lieben. Das Letztere dient noch insbesondere auch dazu, unsere Furcht vor dem Tode, welche das Leben so sehr verbittern kann, in Etwas zu vermindern.
\item Haben wir uns durch Uebung erst eine gewisse Geläufigkeit in den Ansichten der geoffenbarten Religion erworben: so freuen wir uns dieses Vorzuges, und eben \RWbet{diese Freude wird ein neuer Antrieb, recht oft Gebrauch von denselben zu machen.} Daß wir uns jener Geläufigkeit, die wir uns in den Ansichten der geoffenbarten Religion erworben haben, freuen, beruht auf dem natürlichen Grunde, weil wir sie als ein Werk unserer Thätigkeit, und einer pflichtmäßigen und verdienstlichen Thätigkeit erkennen; und weil sie uns das Gefühl eines gewissen Vorzuges vor anderen Menschen gibt, die diese Geläufigkeit sich nicht erworben haben. Eben so natürlich ist aber auch, daß wir, um dieser Freude recht oft zu genießen, öfteren Gebrauch von diesen Ansichten machen. Dieß muß aber die glücklichsten Folgen haben. Wenn sich \zB\ irgend eine Versuchung zum Bösen bei uns einstellt, zu deren Beseitigung wir diese~\RWSeitenw{103}\ oder jene Lehre der göttlichen Offenbarung in unser Gemüth zurückrufen sollten: so thun wir das gerne, obgleich wir voraussehen, daß das Vergnügen, welches uns die Sünde gewähren könnte, wird weichen müssen; denn der Mensch ist so geartet, daß er ein zwar nur kleines, aber doch völlig erlaubtes Vergnügen leicht einem andern vorzieht, welches zwar größer (lebhafter) ist, welches er aber nicht mit Zustimmung seines Gewissens genießen könnte. Nur dann entschließt er sich schwer, auf eine sündhafte Lust Verzicht zu leisten, wenn ihm dafür nicht einmal die geringste Belohnung gleich in der Gegenwart zu Theil werden soll.

\begin{RWanm}  Man könnte noch zusetzen, daß durch Geheimnißlehren auch unser Nachdenken über den eigentlichen Grund derselben, über ihre Vereinbarkeit mit anderen Wahrheiten \udgl\  angeregt wird, welches denn \RWbet{unsere Geistesbildung und hiedurch auch mittelbar unsere Tugend und Glückseligkeit befördert.} Die Geschichte der christlichen Religion zeigt wirklich, daß sie durch ihre Geheimnißlehren, besonders diejenigen, die am heftigsten angefochten worden sind, \zB\ durch ihre Lehre von Gottes dreifacher Persönlichkeit, von dem heil.\ Abendmahle, von der Menschwerdung des Sohnes Gottes, von der Erlösung, von der Erbsünde \uma\  Veranlassung zu den scharfsinnigsten Untersuchungen gegeben habe.\end{RWanm}
\end{aufza}


\RWabs{Erster Abschnitt}{Katholische Theologie oder Lehre von Gott}
\RWpar{35}{Inhalt dieses Abschnittes}
\begin{aufza}
\item Auch bei dem weitesten Begriffe, den man dem Worte Religion geben mag, bleibt es doch immer wahr, daß Gott den wichtigsten Gegenstand der theoretischen Lehrsätze derselben ausmacht. Billig also fange ich die Darstellung der theoretischen Lehren des Katholicismus (nachdem ich die Lehre von~\RWSeitenw{104}\ den Erkenntnißquellen desselben bereits vorausgeschickt habe) mit der katholischen Lehre von Gott an.
\item Es macht sich aber die katholische Religion keineswegs anheischig, uns das Wesen Gottes, \RWbet{wie es an sich ist,} aufzuschließen, oder uns Gott nach solchen Beschaffenheiten, welche in gar keiner Beziehung mit uns Menschen stehen, bekannt zu machen; sondern sie verspricht uns bloß, Gott so, wie er von uns Menschen begriffen werden kann, und nach denjenigen Beschaffenheiten desselben zu schildern, welche uns Menschen besonders angehen, und deren Kenntniß von einem eigenen Nutzen für uns ist.
\item Daher sagen denn alle Lehren, welche in diesem Abschnitte vorkommen, im Grunde nichts Anderes, als gewisse \RWbet{Verhältnisse Gottes zu uns Menschen} aus; sie unterscheiden sich nichts desto weniger von den Lehren des vierten Abschnittes darin, daß die Verhältnisse, welche jetzt aufgezählt werden sollen, als in Gott unabänderlich vorhanden vorgestellt werden, während diejenigen, welche der vierte Abschnitt bekannt machen wird, in gewissen Einwirkungen Gottes auf uns bestehen, die sich nach Zeit und Ort verändern.
\item Ich werde also in diesem gegenwärtigen Abschnitte die Lehren von Gottes Daseyn und einigem Wesen, von seiner Allvollkommenheit und Unbegreiflichkeit, von seinem [un]endlichen Verstande und allvollkommenen Willen, von seiner [un]endlichen Seligkeit, reingeistiger Natur, Ewigkeit und Allgegenwart, von seiner dreifachen Persönlichkeit, und von seinen Rathschlüssen abhandeln.
\end{aufza}

\RWpar{36}{Die katholische Lehre vom Daseyn eines in seinem Wesen nur einzigen Gottes}
Die katholische Religion fängt ihren Unterricht von Gott mit der Behauptung an, daß es einen Gott, \dh\ ein Wesen, das keinen weitern Grund seines Daseyns hat, in der That gebe; und setzet bei, daß es auch nur ein einziges Wesen dieser Art gebe.~\RWSeitenw{105}

\RWpar{37}{Historischer Beweis dieser Lehre}
\begin{aufza}
\item Nichts ist bekannter, als diese Lehre des Katholicismus. In allen Volkslehrbüchern (Katechismen) heißt es ja ausdrücklich, daß es einen Gott, (\dh\ wie man es dort erklärt) ein von sich selbst bestehendes Wesen gebe; und daß dieser Gott seinem Wesen nach auch nur ein Einziger sey.
\item In den Büchern der heil.\ Schrift, besonders in jenen des neuen Bundes, und in den späteren des alten, wird diese Lehre auch ganz ausdrücklich vorgetragen. So heißt es \zB\ \RWbibel{Jes}{Isai.}{44}{24}: \erganf{Ich bin der Herr, der Alles schafft, ich spanne die Himmel aus, \RWbet{ich allein}, und ich befestige die Erde. Ich und sonst Niemand mit mir.} Bei \RWbibel{Joh}{Joh.}{17}{3}\ spricht Jesus: \erganf{Das ist das ewige Leben, daß sie dich erkennen, den Einen wahren Gott, und den du gesandt hast, Jesum Christum.} Und Paulus schreibt \RWbibel{1\,Kor}{1\,Kor.}{8}{4}\ (wo er von dem Genusse des den Götzen geopferten Fleisches spricht): \erganf{Wir wissen, daß ein Götze nichts ist, und daß es außer dem Einen Gott keinen anderen gibt.} So auch \RWbibel{Apg}{Apostelg.}{14}{7}\ \RWbibel{Apg}{}{17}{22}\ \RWbibel{Eph}{Ephes.}{4}{3}\ \uma\ 
\item Daß die Rücksicht, in welcher Gott nur Einer ist, eigentlich das \RWbet{Wesen}, oder die \RWbet{Substanz} Gottes betreffe, diese schon mehr wissenschaftliche Bestimmung kann man in so populär geschriebenen Büchern, als es die meisten biblischen sind, nicht suchen. In den Schriften der Kirchenväter aber, und in andern von der Kirche allgemein angenommenen Aufsätzen kommt diese Bestimmung ausdrücklich genug vor. So heißt es \zB\ in dem \RWlat{Symbolo Athanasiano,}\RWlit{}{SymbolumAthanasianum} welches im \RWlat{Breviario Romano} steht, und etwa im fünften Jahrhunderte verfaßt seyn dürfte: \erganf{\RWlat{Fides autem catholica haec est, ut unum Deum in Trinitate, et Trinitatem in Unitate veneremur, neque confundentes personas, neque substantiam separantes.}} Also ist Gott in Rücksicht auf seine Substanz nur einfach.
\end{aufza}

\RWpar{38}{Vernunftmäßigkeit}
Daß es einen Gott, \dh\ ein Wesen, das keinen weitern Grund seines Daseyns hat, sondern unbedingt wirklich~\RWSeitenw{106}\ ist, in der That gebe, wurde bereits (1.\,Hptthl.\ 2.\,Hptstck.) aus bloßen Gründen der Vernunft erwiesen.

\RWpar{39}{Auflösung eines Einwurfes}
In verschiedenen Stellen der Bücher des alten Bundes, besonders in den fünf Büchern Mosis, wird nicht undeutlich das Daseyn mehrerer Götter vorausgesetzt. So ist dort öfters die Rede von Göttern anderer Völker, und es wird behauptet, daß der Gott Israels, oder der Gott Abrahams, Isaaks und Jakobs: \RWhebr{y:hOAh} (Jehova) stärker als alle übrigen sey. So wird von Hausgöttern gesprochen, deren Einen Rachel ihrem Vater Laban gestohlen. Der Name des Gottes, dem (\RWbibel{Gen}{1\,Mos.}{1}{1}) die Schöpfung der Welt zugeschrieben wird, nämlich \Ahat{\RWhebr{'E:lOhiym}}{\RWhebr{'E:lo.hiym}} (Elohim), schließt die vielfache Zahl in sich, und bedeutet eigentlich: die Starken, Mächtigen. Dieser Gott spricht im Plural von sich: Lasset uns den Menschen erschaffen nach unserem Ebenbilde, \usw\ Und ein andermal spricht er: Sieh! Adam ist nun geworden wie Einer aus uns. -- Die Vorfahren des israelitischen Volkes also, und Moses selbst müssen noch an das Daseyn mehrerer Götter geglaubt haben. Gleichwohl behaupten die Katholiken insgemein, daß Moses und Patriarchen in dem Besitze einer wahren göttlichen Offenbarung, und folglich auch in dem Besitze des Glaubens an einen einzigen Gott gewesen seyen.\par
\RWbet{Antwort.}
\begin{aufza}
\item Wenn es auch wahr wäre, daß Moses und die Vorfahren des israelitischen Volkes sich zu dem reinen Begriffe der Einheit Gottes noch nicht emporgeschwungen hätten, und wenn man auch um dessentwillen annehmen müßte, daß ihre Religion keine wahre göttliche Offenbarung gewesen sey: so würde hieraus noch immer nichts Nachtheiliges gegen den Katholicismus folgen; denn die Meinung, daß Moses und alle Erzväter des israelitischen Volkes in dem Besitze einer wahren göttlichen Offenbarung gewesen sind, erzeugt, so angewandt, wie sie in dem katholischen Lehrbegriffe angewandt wird, keine nachtheiligen Folgen, und könnte also, auch wenn sie ein Irrthum wäre, die Göttlichkeit des Katholicismus nicht widerlegen.~\RWSeitenw{107}
\item In der That aber ist es gar keine Folge, daß der Glaube Mosis und der Vorfahren des israelitischen Volkes nicht eine wahre göttliche Offenbarung gewesen seyn könne, wenn sie die Einheit Gottes nicht deutlich anerkannt haben; denn es könnte ja seyn, daß die Menschen jenes Zeitalters noch nicht im Stande waren, die Einheit Gottes ganz deutlich anzuerkennen und zu glauben; und dann wäre es der Gottheit eben nicht unanständig, wenn sie sich herablassend zu der damaligen Empfänglichkeit der Menschen, ihnen nur so viel geoffenbart hätte, als für sie damals verständlich und zuträglich war.
\item Endlich ist nicht einmal erwiesen, daß Moses und die Vorfahren des israelitischen Volkes, denen die katholische Kirche einen wahren göttlich geoffenbarten Glauben beilegt, das Daseyn mehrerer Götter angenommen hätten. Denn von Laban und Rachel behauptet die Kirche keineswegs, daß sie den wahren göttlich geoffenbarten Glauben gehabt, oder ihm lebenslänglich zugethan gewesen wären. Wenn aber Moses von den Göttern anderer Völker redet, so muß man nicht vergessen, daß er den Gott der Israeliten für stärker, als alle übrigen erklärt, und nur von ihm rühmt, er sey derjenige, der Himmel und Erde geschaffen. Also ist eigentlich nur dieser Jehova das unbedingt nothwendige Wesen, das, was wir Gott nennen; und die übrigen tragen den Namen Gottes in einer anderen Bedeutung, sie sind im Grunde nur endliche, geschaffene, abhängige Wesen, von einer vielleicht sehr großen, aber doch immer nur endlichen Macht; wobei es noch überdieß unentschieden bleibt, ob Moses wirklich geglaubt, daß es dergleichen Götter (wir würden sie Dämonen nennen), die sich das Schicksal einzelner Völker angelegen seyn lassen, gebe; oder ob dieses nur ein herrschender Volksglaube war, dem er nicht widersprechen konnte, dem er aber durch seine Behauptung von der Uebermacht Jehova's über alle diese übrigen Götter die Schädlichkeit benehmen wollte. -- Die vielfache Zahl in dem Namen Elohim scheint bloß ein sogenannter \RWlat{pluralis majestaticus} zu seyn, so wie auch der Plural in den beiden Redensarten: \erganf{Lasset uns den Menschen schaffen}, und: \erganf{Siehe! Adam ist geworden wie Einer aus uns.} Denn die Zeitwörter, die mit Elohim~\RWSeitenw{108}\ übereinstimmen, stehen in der einfachen Zahl. Endlich heißt es (\RWbibel{Dtn}{5\,B.~Mos.}{4}{39}) ausdrücklich genug: \erganf{Wisse es (Israel!) von heute an, und bewahre es in deinem Herzen; Jehova, der ist der Elohim (\dh\ Gott) im Himmel und auf Erden, und sonst ist keiner mehr.}
\end{aufza}

\RWpar{40}{Beantwortung einer gelegenheitlichen Frage}
Daß die katholische Religion das Daseyn Gottes behaupte, findet man ganz in der Ordnung. Bei dieser Gelegenheit erhebt sich aber die Frage, ob auch derjenige, der das Daseyn Gottes noch nicht glaubt, es aus der christlichen Religion, als einer Offenbarung, erst annehmen könne? Auf diese Frage erwiedere ich:
\begin{aufza}
\item Wer an das Daseyn Gottes nicht glaubt, kann diese Lehre auch nicht aus einer Offenbarung \RWbet{als solcher} annehmen. Denn um eine Religion als eine Offenbarung annehmen zu können, müßte er ja schon voraussetzen, daß ein Gott sey.
\item Allein wenn eine Offenbarung auch nicht als solche den Glauben an Gott uns erst beibringen kann: so kann sie doch auf einige andere Arten zur ersten Entstehung sowohl, als auch zur mehreren Befestigung dieses Glaubens viel beitragen; und zwar
\begin{aufzb}
\item \RWbet{als das Zeugniß von irgend einem für endlich, aber doch für wahrhaft gehaltenen Wesen.} Gesetzt, Gott wäre beiläufig so, wie uns das erste Buch Mosis erzählt, den ersten Menschen in einer sichtbaren (\zB\ menschlichen) Gestalt erschienen, gesetzt, er hätte sie da über gar mancherlei Dinge belehrt, und am Ende, nachdem er ihr volles Vertrauen erworben, ihnen auch eröffnet, es gebe ein Wesen, das alle diese Dinge, die sie vor sich sehen, erschaffen hat, selbst aber unerschaffen, allmächtig, allwissend und höchst heilig ist, \usw : so würden die ersten Menschen dieser Belehrung gewiß geglaubt, und also das Daseyn Gottes aus seinem eigenen Unterrichte, und auf sein Zeugniß angenommen haben. Aber genauer betrachtet sieht man wohl ein, daß dieses~\RWSeitenw{109}\ eigentlich keine göttliche Offenbarung im strengsten Sinne des Wortes gewesen wäre; denn die Menschen hätten jenem Unterrichte nicht um des göttlichen Zeugnisses willen, \RWbet{als eines göttlichen}, sondern nur darum geglaubt, weil dieser Lehrer schon sonst durch sein vorheriges Betragen sich ihr volles Vertrauen erworben hatte, oder weil sie sich auch nicht einmal einen Begriff von der Möglichkeit einer Lüge machen konnten.
\item Die göttliche Offenbarung kann uns das Daseyn Gottes auch lehren \RWbet{als eine außerordentliche Begebenheit, die sich nicht anders, als durch die Voraussetzung des Daseyns Gottes erklären läßt.} Gesetzt, es wäre Jemand durch Schlüsse seiner Vernunft nur erst dahin gekommen, daß er die problematische Möglichkeit des Daseyns Gottes, aber noch nicht die Wirklichkeit desselben anerkennete; \dh\ daß er zwar keine entscheidende Gründe für, aber noch weniger einige wider das Daseyn Gottes hätte; nun aber würde er das Daseyn einer Religion auf Erden entdecken, welche die größte sittliche Vollkommenheit hat, und ihre Entstehung, Erhaltung und Ausbreitung einem Zusammenflusse der ungewöhnlichsten Ereignisse verdankt: so behaupte ich, daß gerade die Wahrnehmung dieser für die Beglaubigung jener Religion so passenden Ereignisse ihm den Schluß abzwingen würde, daß ein Gott, den er bisher sich immer nur als möglich vorstellte, wirklich vorhanden sey, und eben hier sich geoffenbart habe. Bei einer näheren Betrachtung sieht man jedoch, daß auch ein so entstandener Glaube an Gott nicht ein auf Gottes Zeugniß angenommener Glaube, also nicht eine Offenbarung, sondern ein Glaube sey, der aus Bemerkung einer so höchst zweckmäßigen Welteinrichtung (jenes so passenden Zusammenflusses der ungewöhnlichsten Begebenheiten, durch welche die erwähnte Religion entstanden, erhalten und verbreitet wurde) geschlossen worden ist.

\begin{RWanm}  Dieser letztere Fall ist in der That von einer sehr großen Wichtigkeit für alle jene Menschen, die an dem Daseyn Gottes noch zweifeln. Sie können auf diese Art, wenn sie nur~\RWSeitenw{110}\ anders wollen, wenn sie die innere Vortrefflichkeit des Christenthums, und alle die außerordentlichen Begebenheiten, welchen es seine Entstehung, Erhaltung und Ausbreitung verdanket, in Erwägung ziehen, zu ihrer völligen Beruhigung gelangen. Das Daseyn des Christenthums selbst (sage ich) kann ihnen der sicherste Beweis vom Daseyn Gottes werden.\end{RWanm}
\end{aufzb}
\end{aufza}

\RWpar{41}{Sittlicher Nutzen}
\begin{aufza}
\item Es ist von größter Wichtigkeit die Rücksicht, in welcher Alles, was unbedingt wirklich ist, nur Eins ausmacht, nie aus den Augen zu lassen; denn wenn in der Folge noch weiter gelehrt wird, daß man diesem Einen Verstand und Willen beilegen könne: so wird eben hiedurch mit einem Male festgesetzt, daß Alles, was außerhalb des unbedingten Wesens, nämlich als seine freie Wirkung besteht, zu einem sittlich guten Zwecke eingerichtet sey; und gerade nur diese Folgerung ist es, die uns den Glauben an Gott erst fruchtbar macht.
\item Ferner hat unsere Vernunft ein unvertilgbares Streben nach Einheit und Vereinfachung. Würde nun gelehrt, daß es der Götter mehrere gebe: so würde die Vernunft zu Folge jenes Strebens gewiß nicht ermangeln, auch diese mehreren Götter abermals aus irgend einer gemeinschaftlichen Quelle herzuleiten, und so würden sie eben darum Geschöpfe und nicht Götter, und folglich endlich und unvollkommen seyn.
\item Glaubten wir an mehrere Götter, so würden wir jedem derselben eine eigene Thätigkeit anweisen, so würden dann nicht mehr alle Geschöpfe der Welt Kinder Eines und eben desselben Vaters seyn, und eben darum auch nicht unter einander Brüder.
\item So würden wir fürchten, daß diese Götter selbst vielleicht unter einander uneins sind, einander entgegen arbeiten, daß der Eine den Plan des Andern zerstören könne; Irrthümer, denen bekanntlich die weisesten Männer des Alterthums huldigten.
\item Wir würden eben darum den Einen Gott zu beleidigen fürchten, indem wir andere verehren. \Usw~\RWSeitenw{111}
\end{aufza}

\RWpar{42}{Wirklicher Nutzen}
Es ist bekannt, daß der Polytheismus nur durch das Christenthum verdrängt worden ist; wo er aber geherrscht hat, da hat er alle die vorhin angeführten Nachtheile im vollsten Maße hervorgebracht. Dem Christenthume also haben wir es zu danken, daß diesem Uebel gesteuert worden ist.

\RWpar{43}{Die Lehre von Gottes Allvollkommenheit}
Obgleich Gott nur einfach in seinem Wesen ist, so vereiniget er nichts desto weniger (wie das Christenthum lehrt) alle möglichen Vollkommenheiten in dem möglich höchsten Grade in sich; oder was immer wahre Vollkommenheit ist, das ist in Gott zu finden; er ist das allvollkommene Wesen.

\RWpar{44}{Historischer Beweis dieser Lehre}
Man findet diese Lehre in jedem Handbuche der katholischen Religion ausdrücklich vorgetragen. In der heil.\ Schrift, welche weder für Gelehrte, noch von Gelehrten geschrieben ist, dürfen wir freilich den etwas schwer zu fassenden Begriff der Allvollkommenheit nicht mit einem ihn bestimmt ausdrückenden Worte bezeichnet anzutreffen hoffen. Daß aber dieser Begriff den Verfassern vieler Stellen der heil.\ Schrift gar nicht unbekannt gewesen sey, läßt sich aus so vielen herrlichen Beschreibungen von Gottes Allvollkommenheit in den Psalmen, in den Propheten \usw\ deutlich genug entnehmen. So heißt es \Ahat{\RWbibel{Jes}{Isai.}{40}{12}}{44,12.}: \erganf{Wer mißt mit hohler Hand das Meer? Wer mißt den Himmel mit einer Spanne? Wer faßt der Erde Staub in einen Dreiling? Wer hält die Erde im Gleichgewicht? Wer wiegt in einer Wage Berg' und Hügel? Wer lenkt Jehova's Geist? Wer theilt ihm einen Rathschlag mit? Wen fragt er, daß er ihm ertheile Unterricht, daß er den Pfad des Rechts ihn lehre, ihn Einsicht lehre, und der Weisheit Weg' ihm zeige? Sieh! Völker sind vor ihm dem Tröpfchen am Eimer, dem Stäubchen an der Wage gleich zu achten! Der Libanon reicht nicht zum Feuer, sein Wild reicht nicht zum Opfer zu. Vor ihm sind alle Völker wie ein~\RWSeitenw{112}\ Nichts, und weniger als nichts, für eitel Nichts sind sie zu achten. Wem also wollt ihr ihn vergleichen? was für ein Bildniß für ihn wählen? Er breitet wie ein Tuch die Himmel, und spannet sie wie ein Zelt zur Wohnung aus. Hebt eure Augen in die Höhe und sehet, wer dieses schuf; dieß Heer der Sterne hat er gezählt, hervor ruft er sie alle mit Namen; vor seiner großen Macht bleibt keines aus. Jehova ist ein ewiger Gott, er schuf der Erde Grenzen, er wird nicht müde und nicht matt, und seine Weisheit ist unerforschlich!} -- Auf diese Allvollkommenheit Gottes scheint auch Jesus (\RWbibel{Lk}{Luk. }{18}{19}) hingedeutet zu haben, wenn er dem Jünglinge, der ihm den Ehrentitel eines vollkommenen Lehrers (\greek{Didask'alos >'agajos}) ertheilen wollte, zur Antwort gab: \erganf{Warum nennst du mich \greek{>'agajos}? Niemand ist \greek{>'agajos}, als nur Einer, Gott}; denn \erganf{\greek{>'agajos}} scheint hier durch \RWbet{vollkommen} übersetzt werden zu müssen. In dem Briefe Jakobi liest man (\RWbibel{Jak}{}{1}{17}): \erganf{Von oben herab kommt jede gute Gabe, und Alles, was vollkommen ist, vom Vater des Lichtes, bei dem kein Wechsel des Lichtes und kein Schatten ist}; eine Stelle, die den Begriff der Allvollkommenheit Gottes so deutlich, als es in einer populären Schreibart nur immer möglich ist, ausdrückt.

\RWpar{45}{Vernunftmäßigkeit}
Die Wahrheit von der Allvollkommenheit Gottes leuchtet, wie dieses schon in der Darstellung der natürlichen Religion erinnert worden ist, der sich selbst überlassenen Vernunft des Menschen so sehr ein, daß viele Weltweise den Begriff des allvollkommensten Wesens mit jenem der Gottheit sogar verwechselten, dergestalt, daß sie Gott als das Wesen erklärten, das alle Vollkommenheiten (oder Realitäten) in sich vereiniget.

\RWpar{46}{Sittlicher Nutzen dieser Lehre}
Die Lehre von Gottes Allvollkommenheit gewährt uns
\begin{aufzb}
\item den Nutzen, daß wir nun alle übrigen Eigenschaften Gottes leichter herleiten, und eine richtigere Vorstellung von denselben bekommen. Sie dienet ferner,~\RWSeitenw{113}
\item die Gefühle der Bewunderung und der Ehrfurcht gegen Gott zu erhöhen.
\end{aufzb}

\RWpar{47}{Lehre von Gottes Unbegreiflichkeit}
Nachdem uns das Christenthum gesagt, daß es ein von sich selbst bestehendes und allvollkommenes Wesen gebe, fängt es den ferneren Unterricht von dessen Eigenschaften damit an, daß es uns auf die Unbegreiflichkeit dieses Wesens hinweiset. Alles, was ihr von Gott noch ferner hören werdet, spricht es, wird ihn noch nicht so, wie er an sich ist, sondern nur wie in einem Bilde darstellen. Wie er an sich ist, ist er euch unbegreiflich, ihr könnt nur einige Züge von ihm in einem Schattenbilde erfassen, ganz aber und gleichsam von Angesicht zu Angesicht kann ihn kein Sterblicher schauen.

\RWpar{48}{Historischer Beweis dieser Lehre}
Eine der merkwürdigsten Stellen hierüber ist wohl jene \RWbibel{Ex}{2\,Mos.}{33}{18}: \erganf{Da sagte Moses zu Gott: Laß mich deine Herrlichkeit schauen. Ich will, erwiederte Gott, meine ganze Güte vor dir hergehen lassen und ausrufen: Jehova ist vor dir. Aber mich selbst wirst du nicht sehen. Auf diesem Berge ist ein Ort, den du betreten sollst. Wenn meine Herrlichkeit nun vorüberziehen wird, dann will ich dich in die Kluft des Felsens stellen, und dich mit meiner Hand bedecken, bis ich vorüber seyn werde. Wenn ich dann meine Hand werde zurückgezogen haben, dann wirst du mich vom Rücken (im Widerscheine oder im Schattenbilde) sehen; mein Angesicht aber kann Niemand schauen.}\par
Im Buche \RWbibel{Hiob}{Hiob}{38}{2--41} \ua ~Stellen wird aus der Eingeschränktheit des menschlichen Verstandes, wenn es sich darum handelt, nur die gemeinsten Naturerscheinungen aus ihren Ursachen zu erklären, der Schluß gezogen, daß Gott und Gottes Rathschlüsse den Menschen unbegreiflich seyen.\par
\RWbibel{Joh}{Joh.}{1}{18}\ spricht Johannes der Täufer: \erganf{Niemand hat Gott gesehen, sondern der eingeborne Sohn Gottes, der in dem Schooße des Vaters ist, hat ihn uns zu erkennen~\RWSeitenw{114}\ gegeben.} \RWbibel{1\,Kor}{1\,Kor.}{2}{11}\ schreibt Paulus: \erganf{Niemand weiß, was in Gott ist, als nur der Geist Gottes}; und \Ahat{\RWbibel{1\,Tim}{1 \,Tim.}{6}{16}}{2,16.}\ sagt er von Gott: \erganf{Der allein Unsterblichkeit hat, und in einem unzugänglichen Lichte wohnt, den kein Sterblicher gesehen hat, noch sehen kann}; und \RWbibel{Röm}{Röm.}{11}{33}: \erganf{O Tiefe der Reichthümer, der Weisheit und der Erkenntniß Gottes! Wie unbegreiflich sind seine Gerichte, wie unerforschlich seine Wege! Wer hat den Sinn des Herrn erkannt? oder wer ist sein Rathgeber gewesen?}

\RWpar{49}{Vernunftmäßigkeit}
Kein vernünftiger Mensch wird sich an die Behauptung stoßen, daß Gott viel Unbegreifliches für uns an sich habe. Auch ist es ein Leichtes, den wahren Grund hievon anzugeben. Bekanntlich hat das unbedingt wirkliche Wesen oder die Gottheit mehrere ganz unendliche Kräfte, \zB\ eine Erkenntnißkraft, die alle Wahrheiten umfaßt. Wer also die Gottheit ganz begreifen, \dh\ Alles, was in ihr ist, erkennen wollte, der müßte auch, wie sie selbst, allwissend seyn. Dieses ist aber kein Mensch, und überhaupt kein endliches Wesen. Hieraus ersieht man also, daß Gott nicht nur für uns Menschen, sondern für alle endlichen Wesen (obwohl in ungleichem Grade) unbegreiflich sey.

\RWpar{50}{Sittlicher Nutzen}
Durch diese Lehre wird uns
\begin{aufza}
\item die \RWbet{Gottheit wichtiger,} und zieht unsere Aufmerksamkeit stärker auf sich; denn wir Menschen sind einmal so geartet, daß jeder Gegenstand, der etwas Räthselhaftes, etwas Geheimnißvolles und Unbegreifliches an sich hat, unsere Aufmerksamkeit um desto stärker fesselt.
\item Auch \RWbet{ehrwürdiger wird uns die Gottheit}, denn wenn wir glauben könnten, daß wir die Gottheit ganz begreifen: könnten wir eben darum ihr keine unendliche Vollkommenheiten, keine Allwissenheit, keine unendliche Macht \usw\ beilegen. Nothwendig würde also auch unsere Ehrfurcht vor Gott, als einem bloß endlichen Wesen, viel verlieren.~\RWSeitenw{115}
\item Unsere Begierde, über Gott und göttliche Anstalten zu urtheilen, wird durch diese Lehre \RWbet{in die gehörigen Schranken der Bescheidenheit} zurückgewiesen. Auch wenn wir nicht einsehen können, zu welchem Zwecke diese oder jene Einrichtung in der Welt, dieses oder jenes Schicksal dienen soll: werden wir es jetzt nicht wagen, Gott darüber zu tadeln.
\item Und eben deßhalb werden wir auch \RWbet{zufrieden mit allen Welteinrichtungen} seyn.
\item Endlich eröffnet sich uns eine um \RWbet{so frohere Aussicht in's andere Leben}, wo wir in unserer Erkenntniß Gottes immer mehr fortschreiten, und die Seligkeit genießen sollen, dieses vollkommenste der Wesen (wie sich die Bibel ausdrückt) zu schauen gleichsam von Angesicht zu Angesicht.
\end{aufza}

\RWpar{51}{Wirklicher Nutzen}
Die Lehre von der Unbegreiflichkeit Gottes hat die im vorigen Paragraph erwähnten möglichen Vortheile bei vielen tausend Menschen wirklich hervorgebracht, und ist dabei gar keinem Mißbrauche ausgesetzt gewesen; indem man diese Beschaffenheit Gottes niemals so weit ausdehnte, wie etwa \RWbet{Robinet,} wo er behauptet, es wäre dem Menschen gar kein richtiger Begriff von Gott möglich. Bei aller Unbegreiflichkeit Gottes glaubte man doch, von ihm zu wissen, daß er ein höchst weises, allmächtiges, allgütiges Wesen sey, \usw\

\RWpar{52}{Die Lehre von Gottes unendlichem Verstande}
Das Christenthum stellt uns die Gottheit, oder das von sich selbst bestehende Wesen als eine \RWbet{denkende} Substanz vor, und lehret, daß Gottes Denkkraft eine unendliche sey, \dh\ daß sie \RWbet{Alles, was immer erkennbar ist,} nämlich alle Wahrheiten an sich umfasse, namentlich also das Mögliche sowohl als das Unmögliche, das Wirkliche so wie das Nichtwirkliche, das Nothwendige so wie das Nichtnothwendige, das Vergangene, das Gegenwärtige und das Zukünftige, alle Gedanken der Menschen, auch die geheimsten~\RWSeitenw{116}\ Wünsche ihres Herzens, deren sie sich kaum selbst deutlich bewußt werden, \usw\ -- Diese Unendlichkeit der göttlichen Denkkraft oder des göttlichen Verstandes nennt das Christenthum auch mit einem eigenen Worte die \RWbet{göttliche Allwissenheit}.\par
In wiefern Gott vermöge dieser Allwissenheit auch unter andern immer die schicklichsten Mittel zu seinen Endzwecken kennet, legt ihm das Christenthum insbesondere eine \RWbet{unendliche Weisheit} bei.

\RWpar{53}{Historischer Beweis dieser Lehre}
A.~Der \RWbet{Allwissenheit.}
\begin{aufza}
\item Schon in den Büchern Mosis wird die Allwissenheit Gottes vorausgesetzt. Gott weiß es alsogleich, als das erste Menschenpaar im Paradiese sein ihm gegebenes Gebot übertreten hatte. Gott sagt dem Abraham seine und seiner Nachkommen Schicksale auf viele Jahrhunderte vorher.
\item Noch weit deutlicher wird diese Lehre in den späteren Büchern des alten Bundes vorgetragen; \zB\ bei \RWbibel{Jer}{Jerem.}{17}{10}: \erganf{Ich bin der Herr, der Herzen und Nieren (den Sitz der Gedanken und Leidenschaften) prüfet, um einem Jeden zu vergelten nach seinem Wege und nach seiner That.}\par
\RWbibel{Ps}{Ps.}{139}{1\,ff}\par
\erganf{Du erforschest und kennest mich, Herr! -- Ich sitze \par
Oder stehe nun auf: Du siehst es Alles; \par
\RWbet{Selbst der Seele Gedanken}\par
\RWbet{Sind Dir von ferne bekannt.}\par
Lager missest Du aus und Gang mir, führest \par
Meine Wege mich stets. \RWbet{Bevor ein Wort noch} \par
\RWbet{Meinen Lippen entschwebet,} \par
\RWbet{Weißt Du es, Ewiger! schon ganz. --}\par

Dickste Finsterniß ist vor Dir nicht finster, \par
Nacht Dir leuchtend wie Tag, Dir Hell und Dunkel \par
Beides gleich; denn gebaut hast \par
Alles mein Innerstes Du! -- --}~\RWSeitenw{117}\par

\RWbet{Deine Augen, sie sahen mich schon im Keime,}\par
\RWbet{Alle standen bestimmt in Deinem Buche}\par
\RWbet{Meine künftigen Tage,}\par
\RWbet{Ehe der erste noch war.}\par
Wie sind Deine Begriffe, o Gott! so schwer mir, \par
Wie unendlich an Zahl! Den Sand am Meere\par
Zählt' ich eh'; -- ich erwache --\par
Immer noch bin ich bei Dir.
\item Aus den Büchern des neuen Bundes will ich nur eine einzige Stelle anführen, \RWbibel{1\,Joh}{1 Joh.}{3}{20}: \erganf{Wenn uns unser eigenes Gewissen verdammt, o! so vergessen wir nicht, daß Gott noch strenger richtet; denn \RWbet{er durchschaut Alles}.}\par

\vabst B.~Der \RWbet{Weisheit.}\par
Man lese \zB\ den 103ten Psalm, wo es unter Anderm heißt:\par
\erganf{Wie sind so groß und herrlich Deine Werke,\par
Alle geordnet von Dir mit Weisheit!}\footnoteC{\RWbibel{Ps}{Ps}{104}{24}}\par
Hieher gehören auch die schon so oft angeführten Worte Pauli \RWbibel{Röm}{Röm.}{11}{33}: \erganf{O Tiefe der Reichthümer, der Weisheit} \usw\ Die Theologen unterscheiden eine dreifache Kenntniß in Gott:
\begin{aufzb}
\item \RWlat{Scientiam necessariam}, \dh\ die Kenntniß dessen, was \RWbet{nothwendig} ist, \zB\ die Kenntniß Gottes von seinem eigenen Wesen, ingleichen von den allgemeinen apriorischen (Begriffs-) Wahrheiten.
\item \RWlat{Scientiam liberam,} \dh\ die Kenntniß dessen, was \RWbet{zufällig} ist, der wirklichen Dinge in der Welt, \zB\ der freien Handlungen, welche wieder abgetheilt wurde in: \RWlat{reminiscentiam, scientiam visionis} und \RWlat{praescientiam,} je nachdem sie sich auf vergangene, gegenwärtige oder künftige Dinge bezieht.
\item \RWlat{Scientiam mediam,} oder \RWlat{simplicis intelligentiae,} \dh\ die Kenntniß dessen, was \RWbet{bloß möglich} ist, ohne je wirklich zu werden. Ueber die letztere Art der göttlichen Erkenntniß entstand in der Mitte des 16ten Jahrhunderts ein Streit zwischen den Jesuiten~\RWSeitenw{118}\ und Dominikanern, indem die Letzteren sie läugneten. Die Mehrzahl der Christen war aber immer der Meinung, daß Gott auch diese Kenntniß habe, und berief sich auf: \RWbibel{Ps}{Psalm}{139}{2--4}\ \RWbibel{Jer}{Jerem.}{38}{17--20}\ \RWbibel{Ez}{Ezech.}{3}{6}\ \RWbibel{Mt}{Matth.}{11}{21}\ \RWbibel{1\,Sam}{1\,Sam.}{23}{10--13}
\end{aufzb}
\end{aufza}

\RWpar{54}{Vernunftmäßigkeit}
Auch diese Lehre ist der Vernunft so wenig widersprechend, daß vielmehr alle Weltweisen (mit Ausnahme der wenigen Atheisten) sie auch aus bloßen Vernunftgründen glaubten herleiten zu können. Die Art der Herleitung, die mir die richtigste scheint, kennt man aus dem 1sten Haupttheile.

\begin{RWanm} 
\RWbet{Lambert} in seiner \RWbet{Architektonik} (2\,B.\ \RWparnr{903})\RWlit{Bd.\,2, \RWparnr{903}}{Lambert1} und einige Andere versuchten die Allwissenheit Gottes auf folgende Art aus dem Begriffe der Wahrheit zu erweisen. Sie gingen von der Behauptung aus, daß jede Wahrheit an sich selbst denkbar sey. Denkbar wäre sie aber (wie sie behaupteten) nicht, wenn nicht irgend ein Verstand vorhanden wäre, welcher sie denken kann. Also muß ein Verstand vorhanden seyn, der alle Wahrheiten zu umfassen vermag. Da aber die Menge der Wahrheiten unendlich ist, so muß der Verstand, der sie erkennt, ein unendlicher seyn. Aber kein Mensch und überhaupt kein endliches Wesen kann einen unendlichen Verstand besitzen. Also muß es nebst den Menschen und nebst allen endlichen Wesen irgend ein unendliches, nämlich die Gottheit, geben, die einen unendlichen Verstand besitzt.\par
In diesem Beweise werden meinem Dafürhalten nach zwei Fehler begangen.
\begin{aufza}
\item Daraus, daß jede Wahrheit an sich denkbar seyn muß, wird übereilter Weise auf das Daseyn eines Verstandes, für den sie denkbar sey, geschlossen. Denn zur bloßen Möglichkeit einer Sache, hier zur bloßen Gedenkbarkeit einer Wahrheit, wird nie das \RWbet{wirkliche Daseyn} ihrer Ursache, sondern immer nur die \RWbet{bloße Möglichkeit} derselben erfordert. So ist \zB\ zur Möglichkeit eines Hauses nicht nöthig, das wirkliche Daseyn, sondern nur die Möglichkeit eines Baumeisters.
\item Daraus, daß jede Wahrheit das Daseyn eines Verstandes, der sie zu denken vermag, voraussetzt, wird übereilter Weise~\RWSeitenw{119}\ auf das Vorhandenseyn eines unendlichen Verstandes geschlossen. Wenn man auch zugeben wollte, daß jede Wahrheit das Daseyn eines Verstandes, der sie zu denken vermag, voraussetze: so würde man noch nicht annehmen müssen, daß diese Wahrheiten alle von \RWbet{Einem} und eben demselben Verstande gedacht werden müssen, sondern es ist genug, wenn nur jede Wahrheit von irgend Einem Verstande gedacht wird. Nun ist es freilich wahr, daß jeder endliche Verstand nur eine endliche Menge von Wahrheiten denken kann, und wenn die Menge der endlichen Geister nur endlich wäre: so würden sie auch alle zusammen nur eine endliche Menge von Wahrheiten, mithin nicht alle, umfassen können. Allein es läßt sich annehmen, und muß sogar nothwendig angenommen werden, daß die Menge endlicher Geister in der Welt unendlich ist. Diese unendliche Menge endlicher Wesen zusammengenommen könnte also wohl allerdings eine unendliche Menge, kurz, alle Wahrheiten, welche es gibt, umfassen, und sonach ist es aus diesem Grunde noch gar nicht nöthig, das Daseyn eines unendlichen Verstandes anzunehmen.
\item Wollte man aber den Satz, daß jede Wahrheit das Daseyn eines Verstandes voraussetze, für den sie denkbar ist, dadurch rechtfertigen, daß man denselben für eine bloß analytische Folgerung aus dem recht aufgefaßten Begriffe der Wahrheit ausgäbe, und wollte man deßhalb die Wahrheit als etwas, das von irgend einem Verstande gedacht werden kann, erklären, so würde man dadurch den Fehler nur noch ärger machen; denn jetzt ist der Satz: Jede Wahrheit setzt das Daseyn eines Verstandes voraus, ein bloß identischer Satz, aus welchem nie etwas gefolgert werden kann. Jetzt könnte man, ohne \RWlat{petitionem principii} zu begehen, nicht mehr behaupten, daß die Menge der Wahrheiten unendlich sey.
\end{aufza}
\end{RWanm}

\RWpar{55}{Sittlicher Nutzen}
\begin{aufza}
\item Der Hauptnutzen ist, daß wir nun erst, wenn wir uns Gott als allwissend und höchst weise denken, die Vorstellung ergreifen können, \RWbet{daß diese Welt höchst weise eingerichtet sey}, und daß wir Ursache haben, mit allen Einrichtungen derselben, und mit allen Schicksalen, die uns begegnen, zufrieden zu seyn. Zwar bleibt freilich, wir mögen uns diese oder jene Vorstellung von dem unbedingten Wesen bilden, oder wir mögen auch glauben, daß gar keines vor\RWSeitenw{120}handen sey, in einem gewissen Sinne doch immer den Satz wahr, daß die Welt so vollkommen (\dh\ dem Zwecke der Tugend und Glückseligkeit so entsprechend) eingerichtet sey, als es nur immer möglich ist; dieses Mögliche aber ist immer ein Anderes, je nachdem wir uns bald diese bald jene Vorstellung von den Eigenschaften des unbedingten Wesens, und von dem Grade der Vollkommenheit der freien und vernünftigen Wesen, die es in dieser Welt gibt, machen. Nämlich je mehrere freie und vernünftige Wesen wir in der Welt annehmen, welche mit einer großen Macht auch einen hohen Grad von Weisheit und sittlicher Güte verbinden, um desto begreiflicher wird uns, daß durch die Wirksamkeit dieser Wesen allerlei dem Zwecke der Tugend und Glückseligkeit entsprechende Einrichtungen und Ereignisse der Welt möglich und wirklich werden. Sagt uns also das Christenthum, daß jenes unbedingte Wesen, von welchem alles Uebrige, das nicht unbedingt ist, entspringt, Allwissenheit habe: so wird uns hiedurch (wenn wir mit dieser Allwissenheit auch Allmacht und Heiligkeit verbinden) mit einem Male begreiflich, wie eine Menge vollkommener Einrichtungen möglich und wirklich geworden sey, die ohne die Voraussetzung eines solchen Gottes nicht möglich gewesen wäre.
\item Gottes Allwissenheit ist eine von jenen Eigenschaften, welche uns dieses Wesen besonders \RWbet{bewunderungswürdig} machen. Wenn wir versuchen, uns von dem kurzen Satze: Gott weiß Alles, durch eine auch noch so unvollständige Aufzählung dieses \RWbet{Alles} einen anschaulicheren Begriff zu machen; wenn wir uns nur zu Gemüthe führen, welch eine unendliche Menge von Wahrheiten mathematischen, physikalischen, philosophischen, historischen und anderen Inhaltes es gebe, die dem Verstande Gottes fortwährend gegenwärtig sind; \zB\ wie viele \RWbet{Zahlen} nur in der natürlichen Reihe $1, 2, 3, 4, 5, 6, 7, 8, 9, 10, \ldots$, wie viele Zusammensetzungen unter diesen Zahlen zu je zweien, dreien, vieren \usw\ \auslass\ als: $1 + 2, 1 + 3, 1 + 4, 1 + 5, 1 + 6, 1 + 7, 1 + 8, 1 + 9$ \usw \auslass , wie viele Producte aus diesen Zahlen $2 \times 3, 2 \times 4, 2 \times 5, \ldots 2 \times 3 \times 4, 2 \times 4 \times 5, 2 \times 5 \times 7$  \auslass , wie viele Quotienten $\frac{1}{2}, \frac{1}{3}, \frac{1}{4}, \frac{1}{5} \ldots \frac{2}{3}, \frac{2}{4}, \frac{2}{5}$ \auslass \usw ; -- in der \RWbet{Geometrie}, wie viele, und wie~\RWSeitenw{121}\ verschiedentlich gelegene gerade Linien, wie viele Winkel, wie viele Dreiecke, Vierecke, \usw , wie viele Ordnungen der krummen Linien, wie viele Flächen, körperliche Räume; -- in der \RWbet{Mechanik}, wie viele Zusammensetzungen von 2, 3, 4, 5, Kräften an 1, 2, 3, 4, und mehreren Puncten \usw \auslass ; -- in der \RWbet{wirklichen Welt}, wie viele Fixsterne, Planeten, Kometen \usw \auslass ; -- \RWbet{auf dieser Erde nur}, wie viele belebte, leblose, organische und unorganische Wesen, wie viele Atome nur in einer einzigen von jenen tausend Mücken, die wir in einem Raume von ein Paar Kubikschuhen im Glanze der Sonne sich ergötzen sehen; wenn wir erwägen, daß jeder von diesen Atomen mit jedem Atom in dem entferntesten aus allen Fixsternen in wechselseitiger Anziehung stehe, die Gott berechnet haben muß, wenn diese Welt nicht in ein Chaos zusammenstürzen soll \usw\ \usw : dann sinken wir nieder auf unsere Kniee vor dem Unendlichen, denn lebhaft fühlen wir da, was es auf sich habe, \RWbet{unendlich} zu seyn.
\item Wenn wir glauben, daß Gott nichts unbekannt sey, daß er auch unsere verborgensten Handlungen, ja unsere geheimsten Gesinnungen und leisesten Wünsche kenne; so hoffen wir
\begin{aufzb}
\item \RWbet{bei unseren guten Handlungen} auch dann noch eine Belohnung zu finden, wenn sie von Menschen nicht gesehen werden; beruhigen uns also darüber, daß dieses nicht geschehen ist, und entschließen uns leichter, Gutes auch im Verborgenen zu thun.
\item Wir sehen ferner ein, daß es die größte Thorheit wäre, wenn wir uns bei der \RWbet{Verübung einer bösen That} mit dem Gedanken trösten wollten, daß uns die Menschen nicht sehen. Genug, daß Er uns sieht, Er, der allein nach Würde strafen und belohnen kann. Daher bemühen wir uns
\item selbst bei dem Guten, welches wir thun, daß wir es
auch \RWbet{aus dem rechten Grunde} thun, damit wir nicht nur den Menschen, sondern auch Ihm, dem Herzenskundigen, der unsere Absichten kennt, gefallen mögen.~\RWSeitenw{122}
\end{aufzb}
\item Wir haben die überaus tröstliche Vorstellung, \RWbet{daß Gott uns nie vergesse}, daß er stets für uns sorge, daß er unser Rufen höre, wo und wie oft wir uns in unseren Nöthen zu ihm wenden, \usw\
\end{aufza}

\RWpar{56}{Wirklicher Nutzen}
Wenn man den Nutzen, den das Christenthum durch diese Lehre gestiftet hat, gehörig würdigen will: so muß man erwägen, wie häufig diese Eigenschaft Gottes außerhalb des Christenthums, sogar von Weltweisen, verkannt worden sey.

\RWpar{57}{Die Lehre von Gottes unendlich vollkommenem Willen, und zwar a)~von seiner Allmacht}
Nebst einem Verstande legt das Christenthum Gott auch einen \RWbet{Willen} bei, und von diesem Willen lehrt es, daß auch er allvollkommen, und also zuvörderst \RWbet{allmächtig} sey. Unter dieser Allmacht wird aber verstanden, daß es Gott möglich sey, Allem, was an sich Möglichkeit hat, die Wirklichkeit zu geben, wieferne es mit seinen übrigen Vollkommenheiten übereinstimmt. Von dieser Allmacht sagt das Christenthum, daß sie eine \RWbet{unendliche Macht} sey, und unter Anderm auch die Kraft zu \RWbet{schaffen}, oder Substanzen das Daseyn zu geben, in sich schließe.

\RWpar{58}{Historischer Beweis dieser Lehre}
\begin{aufza}
\item Schon in den Büchern Mosis erhält Gott öfters den Namen des \RWbet{Allmächtigen}, \zB\ \Ahat{\RWbibel{Gen}{1\,Mos.}{35}{11}}{35,10.}, wo er zu Jakob spricht:  \erganf{Ich bin Gott, der Allmächtige.}
\item \RWbibel{Ps}{Psalm}{33}{6\,ff}:\par
\erganf{Durch des Ew'gen Wort \par
Entstanden die Himmel, all' ihr\par 
Heer durch des Schaffenden Hauch.\par 
Wogen des Meeres fasset er auf, wie in Schläuche,\par 
Häufet einen Schatz sich zusammen wie Oceane! --~\RWSeitenw{123}\par
Erde! fürchte den Herrn; erbebt ihr,\par
Alle Bewohner der Welt!\par 
Denn was er will, ruft er: es ist; er gebeut nur: \par
Da vor ihm steht's! Von Jehova nur ein Wink macht\par 
Alle Entwürfe der Heiden, alle \par
Plane der Völker zu nichts.}
\item \RWbibel{Ps}{Psalm}{135}{5}: \erganf{Groß ist, ich weiß es, der Ewige; unser Gott ist erhaben, über alle Wesen der Götter! Was ihm gefällt, im Himmel, auf Erden, im Meere und in allen Tiefen, er schafft es.} --
\item \RWbibel{Lk}{Luk.}{1}{37}\ spricht der Engel zu Maria: \erganf{Bei Gott ist nichts unmöglich.} 
\item Jesus selbst \RWbibel{Mk}{Mark.}{14}{36}: \erganf{Vater! dir ist Alles möglich; laß diesen Kelch an mir vorübergehen.}
\end{aufza}

\RWpar{59}{Vernunftmäßigkeit}
Auch diese Eigenschaft Gottes wurde bereits im 1sten Haupttheile aus bloßen Gründen der Vernunft erwiesen.

\begin{RWanm}[\RWbet{1.~Anmerkung.}]
Auf ähnliche Art, wie die Allwissenheit, glaubten Lambert und einige Andere auch die Allmacht Gottes erweisen zu können. Wenn eine Sache möglich seyn soll, so muß ein Wesen da seyn, welches die Kraft, ihr das Daseyn zu geben, besitzt. Nun ist kein endliches Wesen im Stande, Allem, was an sich möglich ist, das Daseyn zu geben; also muß es noch irgend ein unendliches Wesen, nämlich die Gottheit, geben, die Allem, was möglich ist, das Daseyn zu geben vermag; also ist Gott allmächtig. -- Dieser Beweis däucht mir eben die Fehler zu haben, welche ich in dem ähnlichen \RWparnr{54}\ gerügt.
\begin{aufzb}
\item Die bloße Möglichkeit einer Sache setzt keineswegs das wirkliche Daseyn eines Wesens voraus, welches die Kraft, sie hervorzubringen, besitzt; sondern nur die Möglichkeit eines solchen Wesens.
\item Wenn gleich kein endliches Wesen für sich allein allem an sich Möglichen das Daseyn zu geben vermag: so könnte dieß doch die unendliche Menge aller endlichen Wesen zusammengenommen.
\item Auch wäre es ein vergeblicher Versuch, den Mängeln dieses Beweises dadurch abzuhelfen, daß man eine eigene Erklärung der Möglichkeit aufstelle, sagend, möglich heiße nur dasjenige, was~\RWSeitenw{124}\ durch ein schon vorhandenes Wesen wirklich gemacht werden kann; denn nun würde zwar der erste Satz, als ein bloß analytischer, gerettet; aber es ließe sich nicht ohne Zirkel beweisen, daß es des Möglichen unendlich Vieles gebe.
\end{aufzb}
\end{RWanm}
\begin{RWanm}[\RWbet{2.~Anmerkung.}]
Eben so fehlerhaft ist auch der Beweis für Gottes Allmacht, der \RWbet{aus dem Begriffe einer schöpferischen Kraft} geführt worden ist. Nebst der unbedingten (absolut nothwendigen) Substanz gibt es noch allerlei bedingte (hypothetisch nothwendige, oder zufällige) Substanzen, dergleichen \zB\ wir Menschen sind. Diese zufällige Substanzen haben den Grund ihres Daseyns nicht in sich selbst; also in Gott. Folglich besitzt Gott die Kraft, bloß möglichen Substanzen die Wirklichkeit zu geben, welches man Schöpfen heißt. Eine schöpferische Macht aber muß ihrer Natur nach unendlich seyn, und alles Mögliche hervorbringen können; denn eine Kraft, die eine einzige Substanz aus ihrer bloßen Möglichkeit in die Wirklichkeit hervorzurufen vermag, muß eben so auch jede andere mögliche Substanz wirklich zu machen im Stande seyn; denn Eine zur Wirklichkeit zu bringen, muß eben so schwer seyn, wie die andere, weil keine, so lange sie noch nicht ist, etwas zu ihrem Werden beitragen kann.\par
In diesem Beweise ist die Behauptung, daß eine Kraft, welche Ein Wesen wirklich zu machen vermag, auch alle übrigen wirklich zu machen im Stande seyn müsse, nicht richtig erwiesen. Daraus, weil kein Wesen, so lange es noch nicht vorhanden ist, etwas zu seinem Werden beitragen, es gleichsam erleichtern kann, folgt gar nicht, daß es einerlei Kraftäußerung fordere, das Eine wie das Andere zu schaffen. Im Gegentheil läßt sich leicht einsehen, daß zur Hervorbringung jedes eigenen Wesens auch eine eigene, immer anders beschaffene Kraftäußerung erforderlich sey. Gleiche Kraftäußerungen würden als gleiche Ursachen auch nur gleiche Wirkungen, \dh\ gleiche und nicht verschiedene Wesen zum Vorschein bringen.
\end{RWanm}
\begin{RWanm}[\RWbet{3.~Anmerkung.}] 
Der Beweis \RWbet{aus der Größe des Weltalls,} so wirksam er übrigens ist, um uns von Gottes Allmacht einen anschaulichen Begriff zu geben, beweiset doch begreiflicher Weise nicht -- Allmacht, sondern nur große Macht.
\end{RWanm}


\RWpar{60}{Sittlicher Nutzen}
\begin{aufza}
\item Nur unter der Voraussetzung, daß Gott mit seiner Allwissenheit auch Allmacht vereinige, können wir glauben,~\RWSeitenw{125}\ daß diese Welt die \RWbet{allervollkommensten Einrichtungen} habe, daß sie \zB\ eine unendliche Menge lebendiger Wesen umfasse, \usw\
\item Insonderheit ist in \RWbet{Noth und Unglücksfällen}, dort, wo alle menschliche Hülfe versagt, und unser banges Herz vergebens nach Rettung sich umsieht, \zB\ bei großen Lebensgefahren, bei schweren Krankheiten \udgl\  nichts Tröstlicheres, als der Gedanke an Gottes Allmacht, der uns retten kann und will! Wenn auch nicht wirklich geschieht, was wir bei diesem Gedanken zu hoffen anfangen, so ist doch das schon ein großer Vortheil, daß dieser Gedanke unseren Schmerz, wenn er am heftigsten ist, lindert.
\item Wenn wir bei unserm redlichen Streben nach sittlicher Verbesserung \RWbet{unser eigenes Unvermögen manchmal auf eine höchst niederschlagende Weise} empfinden: so ist es abermals der Gedanke an Gottes Allmacht, der uns in solchen Fällen wieder aufrichten kann.
\item Wenn wir \RWbet{der Tugend ein schweres Opfer bringen}, wenn wir \zB\ Verfolgungen, vielleicht selbst einen martervollen Tod um unserer Pflichten wegen erleiden sollen: so stärkt uns abermals der Gedanke an Gottes Allmacht: Er kann dir überschwenglich wieder ersetzen, was du hier leiden mußt! Man lese die rührende Geschichte \RWbibel{2\,Makk}{2\,Makk.}{7}{1\,ff}
\item Endlich trägt diese Eigenschaft auch sehr viel bei, uns \RWbet{Gott selbst ehrwürdiger} zu machen, beiläufig eben so, wie dieß von der Allwissenheit \RWparnr{55}\,2.\ gezeigt worden.
\end{aufza}

\RWpar{61}{Wirklicher Nutzen}
Wenn man den Nutzen einsehen will, den die christliche Lehre von Gottes Allmacht gestiftet hat: so braucht man nur zu bemerken, wie häufig diese göttliche Eigenschaft außerhalb des Christenthums verkannt worden ist, und welche Nachtheile dieß gehabt hat. Man hat sich 
\begin{aufza}
\item von der hohen Vollkommenheit dieser Welt keinen richtigen Begriff bilden können. Man sagte, die Materie hätte durch ihre Trägheit Gott widerstanden, und ihn gehin\RWSeitenw{126}dert, daß er dem Weltall nicht diejenige Vollkommenheit geben konnte, die er ihm geben wollte; beiläufig eben so, wie man aus einem groben Sandsteine nicht die beliebige Figur formen kann. Man hatte
\item in Unglücksfällen keinen Trost;
\item bei schweren Opfern, die man der Tugend bringen sollte, keine hinreichende Aufmunterungsgründe dazu gehabt; man hatte sich wohl gar
\item gefürchtet, daß irgend ein böser Gott den guten überwältigen werde, \udgl\ -- Alle diese Nachtheile nun hat das Christenthum beseitiget bei so vielen Millionen Menschen!
\end{aufza}

\RWpar{62}{b)~Die Lehre von der Freiheit des göttlichen Willens}

Der Wille Gottes ist, sagt das Christenthum, nicht nur allmächtig, sondern auch \RWbet{vollkommen frei,} \dh\ nichts ist vorhanden, das ihn zwänge, gerade so zu verfahren, wie er in der Wirklichkeit verfährt; er könnte, abgesehen von seinen übrigen Vollkommenheiten, von seiner höchsten Heiligkeit und Weisheit, auch etwas ganz Anderes wollen und hervorbringen, als er wirklich thut.

\RWpar{63}{Historischer Beweis dieser Lehre}
Es versteht sich von selbst, daß man in der heil.\ Schrift keine logisch genaue Erklärung über die Freiheit des göttlichen Willens suchen und antreffen könne. Allein wenn
\begin{aufza}
\item Moses bei seiner Darstellung der Schöpfung oder Ausbildung der Erde (\RWbibel{Gen}{1\,Mos.}{1}{2\,ff}) immer die Worte gebraucht: Gott sprach: \erganf{Es werde; -- und es ward}: so kann man hieraus wohl nichts Anderes schließen, als daß er Gott einen freien Willen beigelegt habe.
\item Eben so, wenn \RWbibel{Ps}{Ps.}{115}{3\,ff}\ der wahre Gott, Jehova, mit den Götzen der Heiden, die Füße haben, und nicht gehen können, \usw\ verglichen, und von dem Ersteren gesagt wird: \erganf{Aber Jehova, unser Gott, thront im Himmel, und schafft, \RWbet{was er will}}; und \Ahat{\RWbibel{Ps}{Ps.}{135}{6}}{134,5.}\ \erganf{Alles, \RWbet{was er gewollt hat}, das hat er vollbracht, im Himmel, auf~\RWSeitenw{127}\ Erden, im Meere und allenthalben} --: wer könnte da zweifeln, daß der Psalmist den Willen Gottes sich als einen freien dachte?
\item Der heil.\ Paulus schreibt \RWbibel{Eph}{Ephes.}{1}{11}: \erganf{Wir sind nach dem Rathschlusse dessen, \RWbet{der Alles nach dem Rathschluß seines Willens wirket}, erwählt.}
\end{aufza}

\RWpar{64}{Vernunftmäßigkeit}
Schon im 1sten Haupttheile wurde gezeigt, daß die Lehre von der Freiheit Gottes zu jenen Wahrheiten gehöre, welche auch die sich selbst überlassene Vernunft erkennet.

\RWpar{65}{Sittlicher Nutzen}
\begin{aufza}
\item Denken wir uns den Willen Gottes als frei: so gewinnt unsere Vorstellung von seiner Unabhängigkeit und Vollkommenheit, und \RWbet{unsere Ehrfurcht vor ihm;} denn wir sind gewohnt, die freien Wesen für unabhängiger und edler zu halten, als die nicht freien. Dürften wir uns also Gott in keiner Bedeutung des Wortes als frei denken: so würden wir glauben, es fehle ihm eine Vollkommenheit, die wir doch selbst haben.
\item Erinnern wir uns, daß Gott bei seiner höchsten Freiheit dennoch nie von demjenigen abweicht, was die Vernunft als recht erkennt: so wird dieß uns zu einem Aufmunterungsgrunde mehr, \RWbet{auch unsere Freiheit nicht zu mißbrauchen}, und dem, was die Vernunft als recht erkennt, so treu, als ob wir gar nicht anders könnten, zu gehorchen.
\end{aufza}

\begin{RWanm} 
Unrichtig wäre es, wenn wir sagten, daß mit Aufhebung der Lehre von Gottes Freiheit auch aller Nutzen der Lehre von Gottes Allmacht und Weisheit aufgehoben würde; indem ein nicht freier, also der blinden Naturnothwendigkeit, dem Fato unterworfener Gott gar nicht thun könnte, was er für gut befindet, und will. -- Wer so urtheilt, würde in der That nicht bloß die Freiheit, sondern auch die Allmacht Gottes läugnen. Wer aber~\RWSeitenw{128}\ bloß die Freiheit läugnen, und doch die Allmacht und Weisheit zugeben würde, der könnte noch immer glauben, daß diese Welt die besten Einrichtungen habe. -\end{RWanm}

\RWpar{66}{c)~Die Lehre von der Heiligkeit des göttlichen Willens}
Der allmächtige, und in sich völlig freie Wille Gottes stimmt gleichwohl auf das Vollkommenste mit dem Sittengesetze, \dh\ mit allem demjenigen überein, was der allwissende Verstand Gottes als gut und recht erkennet, oder mit anderen Worten: Gott will nur immer das, aber auch alles das, was die Tugend und Glückseligkeit der geschaffenen Wesen bestmöglich befördert. Diese Eigenschaft Gottes nennt man mit Einem Worte die \RWbet{Heiligkeit}.

\RWpar{67}{Historischer Beweis dieser Lehre}
\begin{aufza}
\item Schon in den Büchern Mosis findet sich der Begriff von Gottes Heiligkeit deutlich genug. Alle Vorschriften, welche der israelitische Gesetzgeber seinem Volke gibt, die offenbar nur die Beförderung der Tugend und Glückseligkeit des Volkes zur Absicht haben, also sittlich gute, heilige Vorschriften sind, trägt er als Gottes Gebote vor. Und in jenem Liede, welches er kurz vor seinem Tode abfaßte, spricht er (\RWbibel{Dtn}{5\,Mos.}{32}{3}): \erganf{Den Namen Jehova's will ich besingen; gebt unserem Gott die Ehre! Er ist der Urgrund aller Dinge, vollkommen ist sein Werk; alle seine Wege sind \RWbet{gerecht}. Gott ist die Wahrheit und \RWbet{keine Ungerechtigkeit} ist in ihm. \RWbet{Gerecht und heilig ist er!}} -- \uam\ 

\begin{RWanm} Aus der Stelle \RWbibel{Lev}{3 Mos.}{19}{2}: Seyd heilig, denn auch ich, euer Gott, bin heilig; aus der man gewöhnlich die Heiligkeit Gottes beweiset, läßt sich nichts schließen; da \RWhebr{qAdO+s} (Kadosch) hier mehr die Bedeutung des Abgesonderten und Reinen hat.
\end{RWanm}

\item \RWbibel{Ps}{Psalm}{5}{5}:\par
\erganf{Nicht ein Gott, der Missethaten liebet\par
Und Verbrechen schützet, bist Du! \par
Ungerechte dürfen Dir nicht nahen,\par
Und die Sünder hassest Du.~\RWSeitenw{129}\par
Du vertilgst verläumderische Lügner;\par
Der Erfinder schlauen Trug's \par
Und der Mann des Blutes ist ein Gräuel \par
Vor Jehova's Angesicht.}\par

\RWbibel{Ps}{Psalm}{15}{1\,ff}:\par
\erganf{Gott! wer darf in Deinem heil'gen Zelte wohnen,\par
Wer auf Deinem heil'gen Berge ruh'n? --\par
Der unsträflich wandelt, Tugend übet, \par
Aus dem Herzen Wahrheit spricht; \par
Der mit seiner Zunge nie verläumdet,\par
Arges nie an seinem Bruder thut, \par
Wider seinen Nächsten nie verbreitet\par
Gifterfüllte Lästerung; \par
Der das Laster, wo es ist, verachtet,\par
Wo sie ist, die Frömmigkeit verehrt, \par
Schwört, und was sein Mund einmal beschworen,\par
Selbst zu seinem Schaden hält; \par
Geld nicht leiht auf ungerechten Wucher,\par
Wider Unschuld nicht Bestechung nimmt: \par
Dieser, dieser wird im Glückgenusse \par
Bleiben bis in Ewigkeit.}\par

\item Daß auch das neue Testament hinter dem alten nicht zurückbleibe, versteht sich von selbst. Man lese \zB\ \RWbibel{1\,Petr}{1 Petr.}{1}{15}\ \RWbibel{1\,Petr}{}{3}{12} \uam\ 
\end{aufza}

\RWpar{68}{Auflösung eines Einwurfes}
Es gibt allerlei Stellen der heil.\ Schrift, und manche Lehrsätze der katholischen Kirche, die sich mit Gottes Heiligkeit nicht vertragen; so daß man glauben muß, die Kirche habe keineswegs, wenn sie von Gottes Heiligkeit gesprochen, den richtigen Begriff damit verbunden, nach dem wir sagen, daß Heiligkeit die gänzliche Uebereinstimmung des Willens mit dem Sittengesetze bezeichne.\par
Schriftstellen dieser Art sind:
\begin{aufzb}
\item Nachdem Gott den Menschen geschaffen, und sieht, daß alles Fleisch verderbten Weges wandle, \RWbet{gereuet} es ihn,~\RWSeitenw{130}\ den Menschen geschaffen zu haben; und nachdem er durch jene Sündfluth beinahe das ganze menschliche Geschlecht ausgerottet hat, \RWbet{gereuet} es ihn wieder, so strenge gewesen zu seyn. (\RWbibel{Gen}{1\,Mos.}{6}{6} und \RWbibel{Gen}{}{8}{21}) Reue ist mit ächter Heiligkeit unvereinbar.
\item Dem Abraham \RWbet{befiehlt} Gott, ihm seinen Sohn Isaak zu \RWbet{opfern}. (\RWbibel{Gen}{1\,Mos.}{22}{1\,ff})
\item Gott \RWbet{verhärtet} das Herz des Pharao, daß dieser König die Israeliten zum größten Schaden des Landes nicht ausziehen läßt. (\RWbibel{Ex}{2\,Mos.}{10}{28})
\item Er läßt den Israeliten durch Moses auftragen, von den Aegyptern vor ihrem Abzuge allerlei goldene und silberne Geschirre \RWbet{auszuborgen,} um sie \RWbet{nie wieder} zurückzustellen. (\RWbibel{Ex}{2\,Mos.}{11}{3.--12,36})
\item Er befiehlt den Israeliten, daß sie die Amalekiter aus Rache der erfahrnen Beleidigungen (\RWbibel{Ex}{2\,Mos.}{17}{8}) \RWbet{alle vertilgen} sollen. (\RWbibel{Dtn}{5\,Mos.}{25}{17}) Ein Gleiches wird in Rücksicht der Hethäer, Gergezäer, Kananiten \ua\ Einwohner des Landes Kanaan befohlen. (\RWbibel{Dtn}{5\,Mos.}{7}{1\,ff})
\item David und andere Männer, welche die Kirche insgemein heilige und Gott gefällige Personen nennt, lassen sich \RWbet{große Verbrechen} zur Schuld kommen.
\item In den Psalmen, die nach der Vorstellung der Kirche von Gott selbst eingegeben sind, kommen die schrecklichsten \RWbet{Verwünschungen gegen die Feinde} David's und des israelitischen Volkes vor.
\item In den Sprichwörtern (\RWbibel{Spr}{}{1}{26}) wird die Weisheit Gottes folgender Maßen redend eingeführt: \erganf{Ihr verwerfet alle meine Rathschläge, wollt meine Zurechtweisung nicht ergreifen? Wohlan! so will auch ich \RWbet{zu eurem Untergange lachen}; will euer \RWbet{spotten}, wenn Angst euch überfällt.}
\item Gott straft die Kinder für die Sünden ihrer Eltern bis in's dritte und vierte Geschlecht. (\RWbibel{Ex}{2\,Mos.}{20}{5})
\item \erganf{Er erbarmt sich, \RWbet{wessen er will}, und macht \RWbet{verstockt}, wen er will.} (\RWbibel{Röm}{Röm.}{9}{18})
\end{aufzb}
\RWbet{Antwort.} Was diese Schriftstellen anlangt, so hätten wir zwar eben nicht nöthig zu beweisen, daß auch das\RWSeitenw{131}jenige, was ihre Verfasser sich dabei dachten, ganz richtig sey; sondern es wäre genug zu zeigen, daß nur die Kirche diese Stellen nie so ausgelegt habe, daß ihre Auslegung mit der Lehre von Gottes Heiligkeit in einem Widerspruche stehe. Indessen läßt sich auch selbst das Erstere zeigen.
\begin{aufzb}
\item Die Redensart, daß es Gott \RWbet{gereuet} habe, nahm gewiß schon der Verfasser jener Aufsätze (\RWbibel{Gen}{1\,Mos.}{6}{}\ und \RWbibel{Gen}{}{8}{}) nur bildlich; denn aus der ganzen Beschaffenheit dieser Aufsätze, und aus dem tiefen Sinne, der in ihnen liegt, ist zu ersehen, daß ihr Verfasser ein sehr denkender Mann gewesen, der gewiß nicht im Ernste geglaubt, daß der unendliche Urheber aller Dinge des Gefühls der Reue fähig sey. Wenn demnach Er sowohl als mehrere andere biblische Schriftsteller der Gottheit gewisse, nur uns Menschen zukommende Gemüthszustände beilegten, \zB\ Freude, Wohlgefallen, Zorn, Reue \udgl : so thaten sie dieß erstlich schon darum, weil die damalige Armuth der Sprache sie weit mehr als uns nöthigte, bildlich von Gott zu reden; dann aber auch, weil das Verfahren, welches Gott in gewissen Fällen gegen uns beobachtet, wirklich nicht kürzer und gemeinverständlicher bezeichnet werden kann, als wenn man sagt, er handle hier so, wie wir Menschen, wenn dieser oder jener Gemüthszustand in uns vorhanden ist; endlich weil solche Bilder auch viel schöner und kräftiger sind, als abstracte Ausdrücke. Wenn also Moses sagt: es habe Gott gereuet, so will er hiemit im Grunde nichts Anderes sagen, als: Gott habe sich durch das geänderte Betragen der Menschen bewogen gesehen, ein gleichfalls geändertes Betragen gegen sie anzunehmen, ein Betragen, das von dem vorigen so sehr verschieden war, wie es nur immer bei uns, wenn wir etwas gethan zu haben bereuen, der Fall zu seyn pflegt.
\item Es ist der Heiligkeit Gottes nicht im Geringsten zuwider, jenen lebhaften Traum, oder was für ein ungewöhnliches Ereigniß es sonst gewesen war, zuzulassen, durch welches Abraham auf den Gedanken kam, daß Gott \RWbet{die Opferung seines Sohnes} verlange; denn hieraus entstand, so viel wir sehen, nicht Uebles, wohl aber manches Gute.~\RWSeitenw{132}
\begin{aufzc}
\item Abraham und Alle, die von dieser Geschichte hörten oder noch hören, erfuhren, daß Gott an Menschenopfern \RWbet{kein Wohlgefallen habe}, weil er auch hier ein solches Opfer nicht vollziehen, sondern vielmehr durch einen Engel davon abhalten ließ.
\item Abraham und alle Menschen lernten daraus, daß man auch selbst das Theuerste Gott zu Liebe aufzuopfern bereit seyn müsse.
\item Abraham selbst erhielt Gelegenheit, diese so heldenmüthige Liebe zu Gott wirklich in Ausübung zu bringen, und eben hiedurch die edelste That seines Lebens zu verrichten, eine That, durch welche nicht nur sein sittlicher Charakter überaus viel gewinnen mußte, sondern durch die er auch
\item jener erhabenen Verheißung theilhaftig wurde, welche ihm Gott zur Belohnung seines Gehorsams gab.
\end{aufzc}
\item Jener Ausdruck Mosis bedeutet nur so viel: \erganf{Es war recht sonderbar, daß Pharao bei so viel Gründen, dem israelitischen Volke den verlangten Abzug zu gestatten, dennoch von seinem Eigensinne nicht abstand; und Gott ließ dieß absichtlich zu, um so Gelegenheit zu erhalten, den Menschen seine Macht zu zeigen.} Wenn nämlich Gott etwas aus einer für uns Menschen bemerkbaren Absicht zuläßt, so bedienet sich die Schrift des Ausdruckes: Er thut, er bringt es selbst hervor. Was aber den Nutzen betrifft, den jene Weigerung des Pharao und die durch sie veranlaßten Wunder und Zeichen hatten: so läßt sich leicht erachten, daß die Furcht vor Gott, die hiedurch nicht nur den Aegyptern, sondern selbst allen benachbarten Völkern eingeflößt wurde, den Schaden derselben überwogen haben dürfte.
\item[d)~u.~e)] Die Kirchenväter, welche der Vorstellung lebten, daß Alles, was Moses that und befahl, auf Gottes unmittelbaren Auftrag geschehen sey, erklärten diese Stellen auf mancherlei Art, um die Heiligkeit Gottes zu retten. Es mögen nun ihre Erklärungen gezwungen oder ungezwungen seyn: so geht doch aus ihnen hervor, daß diese Kirchenväter einen sehr richtigen Begriff von Gottes Heiligkeit hatten.~\RWSeitenw{133}
\item[f)~u.~g)] Eben dieselbe Kirche, die jene Personen für heilig, und jene Psalmen für göttliche Eingebungen erklärt, suchte auch jene Fehler und diese Verwünschungen immer aus irgend einem Gesichtspuncte zu betrachten, dabei die Heiligkeit Gottes gerettet würde. Von jenen Fehlern \zB\ sagte sie, daß die Personen sie hinterher bereuet und abgebüßt hatten; jene Verwünschungen aber erklärte sie als prophetische Weissagungen der traurigen Schicksale, welche dem Feinde des Christenthums drohen, \udgl\ 
\item[h)] Ist zu beantworten wie a. Die ganze Stelle hat nur den wohlthätigen Zweck, einen Menschen, der auf Gottes Barmherzigkeit freventlich fortzusündigen gedenkt, zur Besinnung zu bringen, durch die Erinnerung an die so furchtbare Wahrheit, daß, wenn seine Lasterhaftigkeit erst einen gewissen Grad erstiegen haben wird, dann kaum mehr eine Rettung für ihn möglich seyn werde, indem dann so schreckliche Uebel über ihn einbrechen werden, daß es ihm scheinen wird, als hätte Gott selbst an seinem Unglücke eine Freude! -
\item[i)] Es ist nicht gegen die Gerechtigkeit Gottes, daß er die schädlichen Folgen (oder Strafen) gewisser Verbrechen auch auf die Kinder und Kindeskinder übergehen lasse, wie dieß täglich geschieht. Dieß ist nämlich eines Theils eine nothwendige Folge von dem Zusammenhange der Dinge in der Welt, und dienet anderen Theils zu einer desto nachdrucksvolleren Warnung vor gewissen Lastern.
\item[k)] Der Ausdruck: Gott macht verstockt, ist wie der unter e) betrachtete zu erklären. Wenn übrigens der heil.\ Paulus sagt: Gott führe zur Tugend und lasse in Laster versinken, wen er will: so ist es nicht seine Meinung, auch von der Kirche nie so ausgelegt worden, als ob Gott willkürlich, \dh\ ohne vernünftige Gründe so handle. Der Apostel will nur sagen, daß es aus Gründen geschehe, die, weil sie in dem Zusammenhange des Ganzen liegen, von uns nicht begriffen und nachgewiesen werden können. Weil wir über solche Rathschlüsse Gottes nicht zu \Ahat{urtheilen}{ertheilen} vermögen: so ist es nicht unschicklich, von ihnen zu sagen, Gott thue hier, was er will, \dh\ was~\RWSeitenw{134}\ Er, und nicht, was wir für recht erkennen. Ob endlich Lehren der Kirche, die man in einem Widerspruche mit der Heiligkeit Gottes finden will, ihr in der That widersprechen, werden wir sehen, bis wir zur Darstellung dieser Lehren kommen.
\end{aufzb}

\RWpar{69}{Vernunftmäßigkeit}
Auch die Heiligkeit des göttlichen Willens läßt sich aus bloßen Gründen der Vernunft (1.\,Hptthl.) erweisen.
\begin{RWanm}[\RWbet{Anmerkung~1.}] 
Man hat versucht, die Heiligkeit Gottes aus seiner \RWbet{Einheit} darzuthun. In dem allerrealsten Wesen, sagte man, muß die vollkommenste Einheit herrschen. Wenn aber sein Wille nicht mit dem Sittengesetze übereinstimmte: so wäre ein Widerspruch, also nicht Einheit in Gott. -- Das Fehlerhafte dieses Beweises liegt in der Zweideutigkeit der Worte Einheit und Widerspruch. Es läßt sich in Gott keine andere Einheit beweisen, als:
\begin{aufzb}[1.]
\item jene, die jedes Wesen haben muß, die sogenannte \RWbet{logische,} die in der Abwesenheit eines eigentlichen Widerspruches (welcher die Setzung und Aufhebung eines und eben desselben Prädicats von einerlei Subjecte ist) besteht; dann auch
\item die Einheit des Wesens oder die \RWbet{physische,} vermöge der es nur eine einzige Substanz gibt, welche Gott ist. Weder aus der Einen, noch aus der andern dieser beiden Einheiten folgt, daß Gott heilig seyn müsse. Nicht aus der logischen Einheit; denn die Nichtübereinstimmung des Willens mit dem Sittengesetze ist kein eigentlicher Widerspruch; sonst müßten alle Wesen heilig seyn. Nicht aus der Einheit des Wesens, indem die Nichtübereinstimmung des Willens mit dem Sittengesetze keine Vielfachheit des göttlichen Wesens voraussetzt. -- Ueberhaupt erhält man in keiner Rücksicht, nicht einmal
\item in Rücksicht \RWbet{der Art des Wirkens} mehr Einheit in Gott, wenn man annimmt, daß der Wille Gottes mit dem Sittengesetze übereinstimmt, als wenn man das Gegentheil annimmt; denn es wäre im Grunde eine eben so einfache Regel des Wirkens, wenn man annehmen dürfte, daß Gott immer das thue, was das Wohl des Ganzen am Meisten stört, als wenn wir annehmen, daß er das Wohl des Ganzen am Meisten befördere.
\end{aufzb}
\end{RWanm}
\begin{RWanm}[\RWbet{Anmerkung~2.}] 
Eben so unrichtig war es, wenn man die Heiligkeit Gottes aus dem Vorhandenseyn \RWbet{einer praktischen Ver}\RWSeitenw{135}\RWbet{nunft in uns} beweisen wollte. Gott gab uns, sagte man, eine Vernunft, durch die wir erkennen, was recht ist; ohne Zweifel will er also, daß wir es thun, und ist mithin selbst heilig. -- Hiegegen ließe sich einwenden, daß Gott uns nicht nur eine praktische Vernunft, sondern auch einen Glückseligkeitstrieb gegeben, und daß wir durch diese beiden nicht nur der Tugend, sondern auch des Lasters fähig sind. Inzwischen will ich nicht läugnen, daß die Betrachtung der moralischen Anlagen in der menschlichen Natur, wenn auch kein ganz genügender Beweis, doch eine sehr wichtige Bestätigung für Gottes Heiligkeit gewähre. Es findet sich nämlich, daß alle Triebe und Neigungen, die von sich selbst und auf natürliche Weise im Menschen entspringen, eine gewisse Richtung (Tendenz) zur Tugend haben, dergestalt, daß der Mensch durch ihre Gesammtwirkung weit mehr zur Tugend als zum Laster hingezogen wird. Und diese Einrichtung könnte, soviel wir sehen, auch ganz anders seyn. Daß sie also nicht anders ist, darin liegt allerdings eine neue Bestätigung der Wahrheit, daß ein höchst heiliges Wesen die Einrichtungen in dieser Welt getroffen habe.
\end{RWanm}

\RWpar{70}{Sittlicher Nutzen}
\begin{aufza}
\item Erst Gottes Heiligkeit ist es, die den Glauben an ihn für uns wohlthätig macht, indem sie die Ueberzeugung vollendet, daß es in der Welt \RWbet{unendlich viele der Tugend und Glückseligkeit entsprechende Einrichtungen} gebe. Im Gegentheile, ein allmächtiger und allwissender Gott, der doch nicht heilig wäre, würde uns eher fürchterlich als erfreulich seyn müssen.
\item Der Gedanke, daß auch Gott selbst, das allervollkommenste, das allerfreieste und zugleich auch das allerseligste Wesen dennoch das Sittengesetz beobachte, vermehret
\begin{aufzb}
\item \RWbet{unsere Achtung gegen dieß heilige Gesetz;}
\item fügt zu den Beweggründen, die uns bestimmen sollen, demselben zu gehorchen, noch jenen der \RWbet{Ehre} hinzu, dadurch Gott ähnlich zu werden; ingleichen
\item jenen der \RWbet{Hoffnung,} eine der göttlichen ähnliche Seligkeit zu erlangen.~\RWSeitenw{136}
\end{aufzb}
\end{aufza}

\RWpar{71}{Wirklicher Nutzen}
Die wichtige Lehre von Gottes Heiligkeit ist eigentlich nur im Juden- und Christenthume vollständig anerkannt worden; in allen übrigen Religionen, \zB\ selbst in der mahomedanischen, ist mehr nur der Name, als die Sache selbst zu finden. Der Gott \RWbet{Mahomed's} will, daß man seine Religion durch Feuer und Schwert verbreite, erlaubt den Menschen die Vielweiberei, seinem Gesandten Mahomed für seine eigene Person Ehebruch, \usw\ -- Nicht besser steht es mit dem Gotte der \RWbet{Chinesen}, und jenem der \RWbet{Indier}. Der höchste Gott der \RWbet{Perser} schuf zwei Geister, einen guten und einen bösen, und überläßt dem Ersteren, der in seiner Wirksamkeit häufig durch den Letzteren gehindert wird, die Leitung aller Dinge. Von den \RWbet{heidnischen Völkern}, die an mehrere Gottheiten glaubten, ist es bekannt, daß selbst die weisesten aus ihnen, die Aegyptier, Griechen und Römer beinahe jeder Gottheit gewisse Lieblingssünden beigelegt; daß sie selbst für die schändlichsten Laster eigene Schutzgottheiten erdachten (so hatte \zB\ selbst die Schmach, die Schamlosigkeit, die Frechheit ihre Tempel, \RWlat{Cic.\ de leg.\ II.\ c.\,11 et 17.}); daß sie diese Gottheiten durch die schändlichsten Ausschweifungen (selbst durch Menschenopfer) verehrten, und sich in ihre Gunst einschmeicheln zu können glaubten. Welch' einen nachtheiligen Einfluß mußte dieß auf die Tugend der Menschen haben! Selbst die Gelehrten hatten keine ganz richtige Begriffe von Gottes Heiligkeit, denn sie glaubten ja, daß sich Gott um die Angelegenheiten der Welt nicht bekümmere, \udgl\  Das Christenthum hat sich also ein überaus wichtiges Verdienst um die Menschheit erworben, indem es den Glauben an einen höchst heiligen Gott eingeführt und so weit ausgebreitet hat.

\RWpar{72}{Die Lehre von den verschiedenen Arten der göttlichen Heiligkeit, und zwar \protect\RWgriech{a})~der Gerechtigkeit}
Das Christenthum begnüget sich nicht, uns bloß im Allgemeinen zu sagen, daß Gottes Wille stets mit dem Sitten\RWSeitenw{137}gesetze übereinstimmt, \dh\ höchst heilig ist; sondern um diesem wichtigen Begriffe desto mehr Deutlichkeit zu geben, macht es uns noch mit einer ganzen Reihe untergeordneter Begriffe, die gleichsam Theile der göttlichen Heiligkeit bezeichnen, bekannt. Hieher gehört zuvörderst Gottes \RWbet{Gerechtigkeit}, die wieder in einer weitern und engern Bedeutung genommen wird. Unter der göttlichen Gerechtigkeit in dieses Wortes \RWbet{weitester Bedeutung} versteht das Christenthum, daß Gott bei seiner Weltregierung nie etwas thut, wovon wir Menschen schon selbst einsehen könnten, daß es nicht gut wäre. Aus diesem Grunde beobachtet er \zB\ unter Anderm die Regel, jede gute That zu belohnen, und jede böse zu bestrafen. Und die Befolgung dieser Regel ist es, die man unter der göttlichen Gerechtigkeit im \RWbet{engern Sinne}, oder unter der \RWbet{vergeltenden} Gerechtigkeit verstehet.

\RWpar{73}{Historischer Beweis dieser Lehre}
\begin{aufza}
\item Um seine Gerechtigkeit den Menschen anschaulich zu machen, bediente sich Gott, nach der Erzählung der Bücher des Alten Bundes, schon in den frühesten Zeiten des Mittels, Belohnung und Strafe oft gleich in dieser Welt auf das Gute oder Böse, welches die Menschen gethan hatten, folgen zu lassen. So ward gleich jene erste Sünde, welche das erste Menschenpaar im Paradiese beging, von Gott noch an demselben Tage geahndet; so ward der erste Mörder sichtbar und abschreckend für Gegenwart und Nachwelt bestraft; \usw\ In der Folge fand es die Gottheit zwar immer seltener nöthig, Belohnungen und Strafen gleich auf der Stelle eintreten zu lassen; aber sie beobachtete doch noch die Regel, daß sie die ganze Familie oder das ganze Volk glücklich oder unglücklich werden ließ, je nachdem die Mehrzahl der Mitglieder dieser Familie oder dieses Volkes tugendhaft oder lasterhaft geworden war. So ging es besonders dem Volke Israel immer wohl, so lange es den Geboten seines Gottes, oder, was eben so viel heißt, der Tugend getreu blieb; fiel es dagegen von Gott ab, und befleckte es sich mit allerhand groben Verbrechen und Ausschweifungen: so ließ es Gott~\RWSeitenw{138}\ unglücklich werden, es in die Hände seiner Feinde gerathen, \usw\
\item \RWbibel{Weish}{Weish.}{3}{4}: \erganf{Wenn gleich die Frommen vor Menschenaugen gestraft zu werden scheinen: so ist doch ihre Hoffnung voll der Unsterblichkeit. Nach kurzer Qual erwartet sie Belohnung. Gott prüfte sie, und fand sie seiner würdig. Wie Gold im Läuterofen reinigte er sie. Wenn die Vergeltungsstunde kommt, werden sie glänzend erscheinen und richten die Völker, und herrschen über die Nationen, und Gott wird ewiglich ihr König seyn. -- Die Gottlosen aber werden nach ihrer Denkart gestraft werden}; \usw 
\item \RWbibel{2\,Kor}{2\,Kor.}{5}{10}: \erganf{Wir alle müssen vor dem Richterstuhle Christi erscheinen, damit Jeder empfange, was er durch seine Thaten im Leben verdienet hat, Gutes oder Uebles.}
\end{aufza}

\RWpar{74}{Vernunftmäßigkeit}
\begin{aufza}
\item Vermöge seiner unendlichen Weisheit und Heiligkeit muß Gott ohne Zweifel überall solche Verfügungen treffen, daß nicht nur kein menschlicher und überhaupt kein endlicher Verstand bessere anzugeben vermag, sondern daß sich auch an sich selbst keine bessere angeben lassen. Etwas, wovon wir selbst, nicht etwa bloß irriger Weise uns einbilden, sondern mit Wahrheit einsehen könnten, daß es nicht gut sey, wird also Gott nie thun; und somit ist kein Zweifel, daß er die Eigenschaft der Gerechtigkeit in jener \RWbet{weitern Bedeutung} habe.
\item Vermöge dieses Begriffes der Gerechtigkeit Gottes sind wir berechtigt, zu behaupten, Gott habe eine gewisse Einrichtung in der Welt getroffen, so oft wir mit Deutlichkeit einsehen, daß diese Einrichtung
\begin{aufzb}
\item an sich selbst \RWbet{nicht unmöglich,} und
\item dem Wohle des Ganzen \RWbet{gewiß zuträglich} und zuträglicher als jede andere ist.
\end{aufzb}
Eine solche Einrichtung ist aber unter Anderm auch diese, daß jede gute That eines vernünftigen und freien Wesens irgend einmal ihre Belohnung, jede böse ihre Bestrafung~\RWSeitenw{139}\ finde. Daß es dem allmächtigen und allwissenden Gott möglich sey, die Schicksale eines jeden vernünftigen und freien Wesens auf eine solche Art zu leiten, daß es für jede gute oder böse That irgend einmal belohnet oder bestrafet werde, ist außer allem Zweifel. Eben so offenbar ist es aber auch, daß eine solche Einrichtung dem Wohle des Ganzen zuträglich sey, und zuträglicher, als jede andere wäre; denn nur wenn diese Einrichtung besteht, und wenn zugleich alle vernünftigen und freien Wesen wissen, daß sie bestehe, erhalten sie alle den stärksten Aufmunterungsgrund zur Tugend, und den stärksten Abhaltungsgrund vom Laster; und es wird also die Tugend in der Welt möglichst befördert, dem Laster möglichst gesteuert. Da aber Tugend nichts Anderes ist, als die Gesinnung, dem Sittengesetze gemäß zu handeln, und das Sittengesetz, selbst nach den unvollkommenen Begriffen, die Viele davon haben, fast durchgängig nur solche Handlungsweisen fordert, welche dem Wohle des Ganzen zuträglich sind: so ist begreiflich, daß durch Beförderung der Tugend auch die Glückseligkeit des Ganzen befördert werde. Also muß Gott diese Regel ohne Zweifel auch beobachten; und er hat somit auch diejenige Beschaffenheit, die wir Gerechtigkeit in dieses Wortes \RWbet{engerem Sinne} nennen.
\end{aufza}

\RWpar{75}{Sittlicher Nutzen}
\begin{aufza}
\item Wenn wir uns vorstellen, daß Gott nach seiner Gerechtigkeit in der \RWbet{weiteren Bedeutung} immer auf eine solche Art verfährt, daß kein vernünftiges Wesen, wenn es nach seiner besten Einsicht schließt, ihn eines Fehlers beschuldigen könne: so erinnert uns dieß, daß auch wir bei unserem Verfahren nicht bloß darauf sehen sollen, daß es uns selbst, sondern, daß es auch \RWbet{Anderen,} die es nicht leichtsinnig beurtheilen, \RWbet{untadelhaft} erscheine.
\item Der Glaube an die \RWbet{vergeltende} Gerechtigkeit Gottes ist das wirksamste Mittel, uns zu jeder, auch noch so beschwerlichen Uebung der Tugend aufzumuntern, weil sie gewiß belohnt; von jedem auch noch so reizenden Verbrechen abzuhalten, weil es unausbleiblich bestrafet werden wird.~\RWSeitenw{140}
\item Wenn wir den \RWbet{Tugendhaften auf Erden im Unglücke} antreffen, wenn wir ihn von der großen Menge der Menschen verkannt, verfolgt sehen, \udgl : so ist es nur der Glaube an die vergeltende Gerechtigkeit Gottes, der uns hierüber beruhigen kann.
\item Auch diese vergeltende Gerechtigkeit Gottes kann uns \RWbet{zur Nachahmung} dienen, in wiefern auch wir dahinwirken sollen, daß jede gute Handlung der Menschen ihre Belohnung und jede böse ihre Strafe finde.
\end{aufza}

\RWpar{76}{Wirklicher Nutzen}
Zur Ehre der menschlichen Vernunft und mit Dank gegen Gottes Vorsehung müssen wir es bekennen, daß unter allen Eigenschaften des unendlichen Wesens die Gerechtigkeit desselben, besonders die vergeltende, am Seltensten verkannt worden sey; \dh\ daß man fast allenthalben an eine gewisse Belohnung der Tugend sowohl, als auch an eine gewisse Bestrafung des Lasters geglaubet habe. Selbst jene Völker, welche der Gottheiten mehrere, und unter ihnen auch manche böse annahmen, glaubten an eine gewisse Wiedervergeltung, -- etwa durch jenes Fatum, das sie noch über ihre Gottheiten setzten, und das also eigentlich, wenn auch nicht eben dem Namen, doch der Sache nach, der wahre Gott war. Daß aber das Christenthum gleichwohl zu einer deutlicheren und völligeren Anerkennung dieser Lehre viel beigetragen habe, ist außer Zweifel, besonders dadurch, daß Jesus die Zeit der vollständigen Wiedervergeltung erst \RWbet{in das andere Leben} versetzte, während daß Juden und Heiden sie hier schon erwarteten, und wenn sie nicht eintrat, sich beirret fühlten.

\RWpar{77}{\protect\RWgriech{b})~Die Lehre von Gottes Wahrhaftigkeit und Treue}
\begin{aufza}
\item Das Christenthum lehrt, daß Gott auch höchst \RWbet{wahrhaftig} sey, \dh\ daß er nicht irren und nicht lügen könne, oder (wenn man dieß weiter erklärt) daß eine jede Meinung, die ein vernünftiges Wesen auf Gottes Zeugniß annimmt,~\RWSeitenw{141}\ vorausgesetzt, daß es bei dieser Annahme jeden vermeidlichen Irrthum vermieden habe, vollkommene Wahrheit sey.
\item Wie fern sich diese Wahrhaftigkeit auf künftig zu erfüllende \RWbet{Verheißungen} bezieht, nennt man sie insbesondere die \RWbet{Treue}. Gott ist denn also auch höchst getreu, \dh\ er erfüllt Alles wirklich, was ein vernünftiges Wesen durch die Dazwischenkunft seines Zeugnisses erwartet, vorausgesetzt, daß es bei Annahme dieser Erwartung abermals jeden vermeidlichen Irrthum vermieden habe.
\end{aufza}

\RWpar{78}{Historischer Beweis dieser Lehre}
\begin{aufza}
\item \RWbibel{Num}{4\,Mos.}{23}{19}: \erganf{Gott ist nicht, wie ein Mensch, daß er lüge, nicht wie ein Sterblicher, daß er sich ändere. Er sollte geredet haben, und es nicht thun? Er sollte gesprochen haben, und es nicht erfüllen?}
\item \RWbibel{Tit}{Tit.}{1}{2}: \erganf{Der Glaube der Auserwählten Gottes führt zur Hoffnung des ewigen Lebens, welches Gott, \RWbet{der nicht täuschet}, schon vor ewigen Zeiten verheißen hat.}
\item \RWbibel{1\,Thess}{1\,Thess.}{5}{24}: \erganf{Der euch berufen hat, \RWbet{ist treu}, und wird es (was er verhieß) auch erfüllen.}
\end{aufza}

\RWpar{79}{Vernunftmäßigkeit}
\begin{aufza}
\item Daß Gott nicht irren könne, ist eine nothwendige Folge seiner Allwissenheit, er kann daher nicht, ohne es zu wissen und zu wollen, täuschen. Daß er aber auch nicht \RWbet{wissentlich} durch die Dazwischenkunft seines Zeugnisses täusche, \dh\ nicht \RWbet{lüge}, ist der Vernunft ganz gemäß; denn eben derselbe Grund, aus dem die Vernunft das Lügen unter den Menschen verbietet, läßt sich, mit weniger Abänderung, auch auf Gott anwenden. Wir würden nämlich kein Vertrauen zu seinen Zeugnissen fassen können, wenn wir nicht voraussetzen dürften, daß sie wahrhaft sind.
\item Ist nun die Wahrhaftigkeit Gottes außer Zweifel: so folgt die Treue, als eine besondere Art, schon von selbst.~\RWSeitenw{142}
\end{aufza}

\RWpar{80}{Sittlicher Nutzen}
\begin{aufza}
\item Durch diese Lehre wird unser \RWbet{Vertrauen zu Gottes Aussagen} begründet; insbesondere läßt uns die Vorstellung der Treue Gottes die unausbleibliche Erfüllung aller uns gemachten Verheißungen, der versprochenen Belohnungen sowohl, als auch der angedrohten Strafen, erwarten; und nur bei einer sicheren Erwartung derselben können sie nachdrücklich auf uns wirken.
\item Auch diese Eigenschaft Gottes soll uns zur \erganf{Nachahmung} dienen.
\item Ist Gott höchst wahrhaft: so verabscheuet er jede Lüge und strafet sie (\RWbibel{Ps}{Ps.}{5}{5}--\RWbibel{Ps}{}{15}{1}).
\end{aufza}

\RWpar{81}{\protect\RWgriech{g})~Die Lehre von Gottes Unparteilichkeit}
Gottes Heiligkeit äußert sich unter Anderm, wie uns das Christenthum lehret, auch in einer völligen \RWbet{Unparteilichkeit} in der Behandlung seiner lebendigen (der Glückseligkeit empfänglichen) Geschöpfe. Hierunter versteht man, daß Gott nie irgend eines seiner Geschöpfe aus einem bloßen Belieben an demselben, oder aus einer bloßen Abneigung vor demselben -- aus bloßer Willkür -- günstiger oder härter behandle; kurz daß ihn bei seiner Behandlung der lebendigen Wesen gar keine Willkür, sondern lediglich nur die möglich größte Beförderung des allgemeinen Wohles bestimme.

\RWpar{82}{Historischer Beweis dieser Lehre}
\begin{aufza}
\item \RWbibel{Dtn}{5\,Mos.}{10}{14\,ff}: \erganf{Seht, dem Jehova, eurem Gotte, gehört der Himmel, und der Himmeln Himmel, die Erde und Alles, was auf ihr ist. Dennoch hat Er euere Väter mit so viel Liebe gepflegt, hat euch, die Nachkommen derselben, aus allen Völkern erwählt bis auf den heutigen Tag. -- Jehova, euer Gott, ist der Götter Gott, der Herren Herr, der große, mächtige, furchtbare Gott ist Er, \RWbet{der die Person nicht ansieht,} und keine Geschenke annimmt.}~\RWSeitenw{143}
\item \RWbibel{2\,Chr}{2\,Chronik.}{[19]}{6\,ff}\ spricht der König von Juda, \RWbet{Josaphat}, zu den von ihm bestimmten Richtern: \erganf{Bedenket, was ihr zu thun habt. Nicht im Namen eines Menschen haltet ihr Gericht, sondern im Namen Gottes, der bei euch ist, wenn ihr auf euern Richterstühlen sitzet. Fürchtet euch also vor Gott, und bedenket, was ihr thut; denn bei Jehova, unserem Gott, \RWbet{gilt kein Ansehen der Person und keine Bestechung durch Geschenke}.}
\item \RWbibel{Hiob}{Hiob}{34}{18--19}:\par
\erganf{Wie wagest du, den höchst Gerechten, Mächtigen\par
Des Unrechts zu beschuldigen?\par
Ihn, der zum König spricht: Du Niederträchtiger!\par
Zu Fürsten: Ihr seyd ungerecht!\par
Ihn, der auf Edle \RWbet{nicht parteilich} sieht,\par
Und Reiche nicht vor Armen achtet,\par
Weil seiner Hände Werk sie alle sind.}
\item \RWbibel{Weish}{Weish.}{6}{7}: \erganf{Der Herr über Alles \RWbet{sieht nicht auf die Person}, und achtet nicht auf die Größe. Er hat den Kleinen und Großen geschaffen, und \RWbet{sorgt auf gleiche Art für Alle}.}
\item \Ahat{\RWbibel{Sir}{Sirach}{35}{17}}{35,14.}: \erganf{Gott verachtet nicht das Klagen der Waisen, noch die Klagen der Wittwe}, \usw\
\item \RWbibel{Apg}{Apostelg.}{10}{34}\ spricht Petrus zu dem römischen Hauptmanne Cornelius: \erganf{Jetzt erkenne ich in Wahrheit, daß bei ihm (Gott) \RWbet{kein Ansehen der Person} gilt, sondern in jeglichem Volke ist ihm derjenige, der ihn fürchtet und Gerechtigkeit übet, angenehm.}
\item \RWbibel{Röm}{Röm.}{2}{10}: \erganf{Preis, Ehre und Heil Jedem, der das Gute thut, besonders dem Juden, aber auch dem Heiden; denn \RWbet{bei Gott gilt kein Ansehen der Person}.} s.\ \RWbibel{Eph}{Ephes.}{6}{9}; \RWbibel{1\,Tim}{1\,Tim.}{2}{4}; \RWbibel{1\,Petr}{1\,Petr.}{1}{17}
\end{aufza}

\RWpar{83}{Vernunftmäßigkeit}
Die Vernunftmäßigkeit dieser Lehre ist für sich selbst klar. Ob aber diese Lehre nicht etwa mit einer andern, der zu Folge Gott den Menschen, ohne Rücksicht auf ihre Ver\RWSeitenw{144}dienste, dem Einen viel, dem andern wenig gibt \usw , im Widerspruche stehe, werden wir in der Folge untersuchen, bis wir zur Betrachtung dieser Lehre kommen.

\RWpar{84}{Sittlicher Nutzen}
\begin{aufza}
\item Halten wir Gott für unparteilich: so sehen wir deutlich ein, daß uns unsere besonderen Talente, unsere Abstammung von diesem oder jenem Vorfahren und alle andern Aeußerlichkeiten dieser Art vor seinem Angesichte nichts nützen werden. Wir müssen uns \RWbet{also der Tugend allein} befleißen, so wie der heil.\ \RWbet{Petrus} (\RWbibel{1\,Petr}{I.}{1}{17}) diese Anwendung macht: Da ihr Denjenigen als Vater anrufet, der ohne Ansehen der Person Jeden nach seinen Werken richten wird: so führet einen gottesfürchtigen Wandel in der Zeit euerer Pilgerschaft.
\item Eben so deutlich wird es uns nun, daß auch der geringste aus unsern Mitmenschen, ja das verachtetste Geschöpf auf dem Erdboden, von Gott geliebet werde; \RWbet{auch wir müssen also diese Geschöpfe lieben}, und ihre Glückseligkeit möglichst befördern, oder wir ziehen uns das Mißfallen Gottes zu.
\item Diese Unparteilichkeit Gottes dient uns ebenfalls zur \RWbet{Nachahmung}.
\end{aufza}

\RWpar{85}{Wirklicher Nutzen}
Daß diese Lehre des Christenthums wirklich sehr viel genützt habe, kann man entnehmen, wenn man erwägt, wie so viele andere Nationen, und selbst die Israeliten, sich äußerst unvollständige Begriffe von Gottes Unparteilichkeit gemacht. Trotz dem, daß den Letztern in ihren heil.\ Büchern das Gegentheil so oft und so nachdrücklich eingeschärft worden war, pochten sie doch stets auf den Vorzug, Nachkommen Abrahams, Gottes auserwähltes Volk, \udgl\  zu seyn. Ja selbst die Christen (wir müssen es eingestehen) verachteten zuweilen die nicht christlichen Völker. Was würde erst geschehen seyn, wenn diese Lehre nicht vorhanden gewesen wäre?~\RWSeitenw{145}

\RWpar{86}{\protect\RWgriech{d})~Die Lehre von Gottes Güte und Liebe}
Das Christenthum lehrt uns, daß Gott in der Welt so viel Glückseligkeit wirklich zu machen suche, als nur der Tugend unbeschadet geschehen kann, oder daß er ein jedes lebendige Wesen so glücklich zu machen trachte, als es nur immer ohne Verletzung Anderer möglich ist. Diese Eigenschaft Gottes nennt nun das Christenthum die \RWbet{Güte} oder \RWbet{Liebe,} und will, daß wir uns bildlicher Weise in Gott auch eine gewisse \RWbet{Geneigtheit des Willens}, allen seinen Geschöpfen, vornehmlich aber den Menschen, wohlzuthun, vorstellen sollen. Es nennt ihn eben deßhalb auch wohl Güte, die Liebe selbst, ingleichen auch den Menschenfreundlichen \usw\

\RWpar{87}{Historischer Beweis dieser Lehre}
\begin{aufza}
\item \RWbibel{Ps}{Psalm}{145}{7}:\par
\erganf{Vom Denkmal Deiner großen Huld, von Deiner \par
Gerechtigkeit fleußt über jede Zunge. \par
Barmherzig, gnädig ist der Herr, voll Langmuth\par
Und großer Güte.\par
Wohlthätig Allen ist der Herr; erbarmt \par
Sich aller seiner Werke. -- --\par
Wer fällt, dem reicht der Herr die Hand, und richtet \par
Die Unterdrückten auf. -- Die Augen Aller \par
Sehen hin nach Dir, und Allen gibst Du Speise,\par
Da sie's bedürfen. \par
Du öffnest Deine Hand: was lebt, ersättigest \par
Mit Wohlthat Du!}
\item \Ahat{\RWbibel{Weish}{Weish.}{11}{24}}{11,25.}: \erganf{Du liebst Alles, was da ist, und hassest nichts von dem, so Du geschaffen hast.}
\item \RWbibel{Jes}{Isai.}{49}{15}: \erganf{Kann auch ein Weib ihres Säuglings vergessen, daß sie der Frucht ihres Leibes sich nicht erbarmete? Vergäße sie seiner: so will doch ich deiner nicht vergessen!} --
\item \RWbibel{1\,Joh}{1.\,Joh.}{4}{8}: \erganf{Wer keine Liebe hat, der kennet Gott nicht; denn Gott ist die Liebe.}~\RWSeitenw{146}
\item \RWbibel{Tit}{Tit.}{3}{4}: \erganf{Als die Güte und Menschenfreundlichkeit Gottes erschien, -- -- hat er uns erlöset.}
\end{aufza}

\RWpar{88}{Vernunftmäßigkeit}
Da das oberste Sittengesetz die möglichste Beförderung der Tugend und Glückseligkeit verlangt: so muß Gott allerdings, vermöge seiner Heiligkeit, jedem Geschöpfe so viel Glückseligkeit zutheilen, als nur ohne größere Verletzung des Wohles Anderer geschehen kann. Die bildliche Vorstellung aber, die uns das Christenthum hiebei empfiehlt, hat
\begin{aufzb}
\item \RWbet{Aehnlichkeit;} Gott handelt wirklich so, wie Menschen, oder andere endliche Wesen zu handeln pflegen, wenn sie eine Neigung (Liebe oder Freundschaft genannt) für Jemand empfinden; und dabei enthält dieses Bild
\item \RWbet{nichts Unanständiges}; denn was kann edler seyn, als eine solche Liebe! --
\end{aufzb}

\begin{RWanm} 
Auch muß man nicht glauben, daß das Christenthum mit sich selbst in Widerspruch trete, wenn es bei anderen Gelegenheiten Gott Zorn oder Rache beilegt. Beides geschieht nämlich bildlich, und beide Bilder sollen wir doch jedes zu andern Zeiten und in andern Verhältnissen gebrauchen.
\end{RWanm}

\RWpar{89}{Sittlicher Nutzen}
\begin{aufza}
\item Stellen wir uns Gott als die Liebe vor: so kann es nicht anders kommen, wir werden \RWbet{Gegenliebe zu ihm} empfinden. (\erganf{Lasset uns Gott lieben, weil er uns zuerst geliebet hat.} \RWbibel{1\,Joh}{1\,Joh.}{4}{19})
\item Alle Gebote dieses Gottes werden uns um so leichter dünken. (\erganf{Nur in Deinen Geboten finde ich Wonne, denn ich liebe sie innigst.} \Ahat{\RWbibel{Ps}{Psalm}{119}{47}}{113,6.})
\item Wir fühlen dann auch um so lebhafter die Pflicht, selbst gütig und liebevoll zu seyn, wie Gott der Heiligste es ist.
\item Und sehen um so deutlicher ein, wie gröblich wir uns an ihm selbst versündigen, wenn wir irgend eines seiner Geschöpfe, die er alle liebt, und zur Glückseligkeit schuf, unnöthiger Weise beschädigen oder quälen, vor Allem die Menschen.~\RWSeitenw{147}
\end{aufza}

\RWpar{90}{\protect\RWgriech{e})~Die Lehre von Gottes Gnade, Barmherzigkeit, Milde und Langmuth}
\begin{aufza}
\item Vermöge der Güte, die Gott besitzt, ist er auch gnädig gegen uns, \dh\ er läßt uns so manches Gute zukommen, das wir auf keine Weise von ihm hätten fordern können und dürfen, und das wir eben deßhalb eine bloße \RWbet{Gnade}, ein freies \RWbet{Gnadengeschenk} nennen. -- Bildlicher Weise sollen wir dem Willen Gottes auch eine eigene Geneigtheit beilegen, dergleichen Gnaden auszutheilen. Sie heißt die \RWbet{Gnädigkeit}, oder die \RWbet{Gnade} Gottes.
\item Diese Gnädigkeit Gottes äußert sich auf verschiedene Weise; vornehmlich sind folgende drei Arten derselben, die eben deßhalb ihre besondern Namen erhalten haben, zu merken:
\begin{aufzb}
\item Die \RWbet{Barmherzigkeit,} oder diejenige Eigenschaft Gottes, vermöge der er die in einer gewissen uns bekannten Rücksicht wohlverdiente Strafe eines freien Wesens aus einer andern uns unbekannten Rücksicht doch gleichwohl aufhebt; verbunden mit dem bildlichen Nebenbegriffe einer Geneigtheit des göttlichen Willens, so zu verfahren.
\item Die \RWbet{Milde}, oder diejenige Eigenschaft Gottes, vermöge der er die in einer gewissen Rücksicht verdiente Strafe aus einer andern Rücksicht, wo nicht ganz aufhebt, doch vermindert.
\item Die \RWbet{Langmuth}, oder diejenige Eigenschaft Gottes, vermöge der er in gewisser Rücksicht schon längst verdiente Strafen aus anderen Rücksichten verschiebt, gleichsam wie Einer, dem es schwer ankommt, zu strafen, und der gern einen Grund, die Strafe völlig aufzuheben, abwarten möchte.
\end{aufzb}
\end{aufza}

\RWpar{91}{Historischer Beweis dieser Lehre}
\begin{aufza}
\item \RWbibel{Röm}{Röm.}{9}{20}: \erganf{Mensch! wer bist du, daß du mit Gott rechtest? Spricht wohl das Werk zum Meister: Warum hast du mich so gebildet?} (Also gibt es Gaben, über deren Entziehung wir mit Gott nicht rechten dürften, \dh\ Gnaden).~\RWSeitenw{148}\par
\RWbibel{Röm}{Röm.}{11}{6}: \erganf{Geschah dieß (daß Gott das israelitische Volk nicht ganz vom wahren Glauben abfallen ließ) aus Gnade: so war es nicht Verdienst; sonst wäre die Gnade keine Gnade.} Diese und viele ähnliche Stellen bestimmen den Begriff der \RWbet{Gnade} ganz so, wie wir ihn oben erklärten.
\item Was insbesondere
\begin{aufzb}
\item Gottes \RWbet{Barmherzigkeit} betrifft, wie schön ist nicht die Stelle \RWbibel{Joel}{Joel}{2}{13}: \erganf{Das Herz zerreißet, nicht das Gewand, und kehret zu dem Ewigen zurück, zum Ewigen, euerem Gotte. Allgnädig, \RWbet{allbarmherzig} ist Er, ja, langmüthig und von unendlicher Huld; \RWbet{gibt seinen Rathschluß gerne auf}, wenn er uns Strafe zugedacht hatte.} Und \RWbibel{Jona}{Jon.}{3}{11}\ heißt es: \erganf{Gott sah ihre Werke, und daß sie von ihren bösen Wegen zurückgekehrt wären, und \RWbet{es gereuete ihn der harten Strafe,} die er über sie ausgesprochen hatte, und \RWbet{er vollzog sie nicht}.} Und \RWbibel{Jon}{}{4}{10}\ wird Gott redend eingeführt: \erganf{Dich dauert diese Pflanze, zu deren Entstehung und Wachsthum du doch nichts beigetragen hast, die über eine Nacht emporgewachsen, und über eine Nacht dahingewelkt ist: und mich sollte Ninive nicht dauern, die große Stadt, in welcher mehr als zwölfmal zehn Tausend Menschen, die Rechts und Links noch nicht zu unterscheiden wissen, und so viele Thiere leben?}
\item Die \RWbet{Milde}. Gott hatte dem David nach seinem Ehebruche eine härtere Strafe bestimmt, als er nachher, da er Buße gethan, wirklich an ihm vollzog, \usw\
\item Die \RWbet{Langmuth.} \RWbibel{Sir}{Sir.}{5}{4}: \erganf{Sprich nicht: Ich habe gesündiget, und was ist mir widerfahren? Denn der Herr ist \RWbet{langmüthig}.} \RWbibel{2\,Petr}{2\,Petr.}{3}{9}: \erganf{Der Herr verabsäumt seine Verheißung (von dem Weltgerichte) nicht, wie Einige meinen; er ist nur langmüthig, und will nicht, daß Jemand verloren gehe, sondern daß Jeder sich bekehre.}
\end{aufzb}
\end{aufza}

\RWpar{92}{Vernunftmäßigkeit}
\begin{aufza}
\item Wenn wir die Lehre von Gottes Gerechtigkeit mit dieser von Gottes Gnade vergleichen: so sehen wir, daß das~\RWSeitenw{149}\ Christenthum eine doppelte Art von Wirkungen Gottes unterscheide:
\begin{aufzb}
\item Wirkungen, die wir mit aller Sicherheit von ihm erwarten, und gleichsam fordern können, \RWbet{Werke der Gerechtigkeit}, und
\item Wirkungen, die wir, ob sie uns gleich sehr wohlthätig scheinen, doch nicht mit Bestimmtheit von ihm erwarten, und wenn sie ausbleiben, ihn auch nicht darüber anklagen können,\RWbet{ Gnaden.}
\end{aufzb}
\item Dieser Unterschied ist nun in der That gegründet; denn es gibt allerdings Wirkungen Gottes, die wir von ihm mit aller Sicherheit erwarten, und gleichsam fordern dürfen, nämlich solche, von denen wir deutlich einsehen, daß sie
\begin{aufzb}
\item an sich nicht unmöglich sind, und
\item zur Beförderung der Tugend und Glückseligkeit des Ganzen sicher dienen. Hieher gehört nun, wie wir bisher gesehen, 
\begin{aufzc}
\item daß Gott jede gute That belohne und jede böse bestrafe, was das Christenthum unter der Gerechtigkeit Gottes im engeren Sinne versteht;
\item daß er nie lüge, was das Christenthum Gottes Wahrhaftigkeit und Treue nennt;
\item daß er keines seiner Geschöpfe aus bloßer Willkür so oder anders behandle; oder die Unparteilichkeit Gottes.
\end{aufzc}
\end{aufzb}
\item Dagegen gibt es auch wieder Anstalten Gottes, die wir, obgleich sie für uns oder Andere höchst erwünschlich scheinen, doch nicht mit Sicherheit erwarten, und gleichsam fordern können, weil es leicht möglich ist, daß der Zusammenhang des Ganzen oder gewisse, uns nicht erkennbare Nachtheile derselben sie verbieten. So können wir \zB\ keineswegs fordern, daß uns Gott von dieser oder jener Krankheit genesen lasse, daß er uns diese oder jene Hülfsmittel, günstige Gelegenheiten \udgl\  herbeiführe, uns eine Offenbarung schenke, \usw\ Freilich, wenn man die Gnaden so erklärte, wie sie von Einigen wirklich erklärt worden sind, nämlich, daß es Verfügungen Gottes wären, welche er nach Belieben~\RWSeitenw{150}\ treffen oder auch nicht treffen könne: so wäre der Begriff einer Gnade mit Gottes Heiligkeit im Widerspruche. Vermöge der Heiligkeit darf Gott nie nach Belieben handeln, ja es sind auch keine Neigungen und kein Belieben in ihm vorhanden; sondern ob er etwas thue oder nicht thue, hängt lediglich davon ab, ob es das allgemeine Wohl befördere oder nicht befördere. Uebrigens ist leicht begreiflich, wie jene falsche Erklärung Einiger habe entstehen können. Man hat nämlich die beiden zwar verschiedenen, aber doch immer ähnlichen Begriffe verwechselt; den Einen: Wenn es ein Werk der \RWbet{Gnade} betrifft: so mag Gott so oder anders handeln, wir Menschen werden ihn nie tadeln dürfen; und den anderen: Er wäre auch \RWbet{an und für sich} nicht zu tadeln, er hat hier an und für sich betrachtet eine freie Wahl. Das Erstere ist ganz richtig; das Letztere falsch.
\item Was insbesondere die Begriffe der \RWbet{Barmherzigkeit, Milde} und \RWbet{Langmuth} betrifft: so enthalten sie durchaus nichts Widersprechendes. Es kann allerdings Gründe, und zwar Gründe, die Gott allein weiß, geben, um derentwillen es dem Wohle des Ganzen zusagt, wenn er uns eine Strafe, die wir in anderen, uns bekannten Rücksichten gar wohl verdient hätten, bald ganz, bald doch zum Theil nachläßt, bald auch nur verschiebt. So kann das Letztere \zB\ oft dazu dienen, um uns noch Zeit zur Besserung zu gewähren. Was aber den \RWbet{bildlichen} Theil dieser Lehre, oder die Vorschrift betrifft, daß wir uns an dem göttlichen Willen eine gewisse \RWbet{Geneigtheit zu solchen Wirkungsarten} vorstellen sollen: so ist dieses eben so zu rechtfertigen, wie oben der bildliche Bestandtheil in dem Begriffe der Liebe Gottes.
\end{aufza}

\RWpar{93}{Sittlicher Nutzen}
\begin{aufza}
\item Es kann nicht anders, als erfreulich für uns seyn, wenn wir uns in dem Willen Gottes bildlicher Weise eine gewisse Geneigtheit vorstellen dürfen, uns auch so manche Wohlthaten, die wir nicht zu fordern berechtiget wären, mitzutheilen. Dadurch wird uns seine Güte und Menschenfreundlichkeit nur um so anschaulicher gemacht.~\RWSeitenw{151}
\item Was insbesondere die \RWbet{Barmherzigkeit} und \RWbet{Milde} Gottes betrifft: so ist der Glaube an sie von der größten Wichtigkeit für uns, wenn wir uns eigener Fehler und Sünden bewußt sind; denn je gewissenhafter und je weniger dem Fehler des Leichtsinnes ergeben wir sind, um desto stärker werden wir des Trostes bedürfen, daß Gott, der Heilige, so manche Rücksichten kenne, um derentwegen er uns nicht ganz so strafen wolle und werde, als wir nach anderen, uns wohl bekannten Rücksichten es nur allzu sehr verdienten. Die Vorstellung von Gottes \RWbet{Langmuth} aber wird uns, wofern wir anders nicht äußerst leichtsinnige Geschöpfe sind, zur Beschleunigung unserer Besserung antreiben, um so die wohlthätige Absicht, aus der Gott unsere Bestrafung bisher verschoben hat, zu unserem eigenen Nachtheile nicht zu vereiteln.

\begin{RWanm}
Aber vielleicht, könnte man einwenden, wird die Vorstellung von diesen drei Eigenschaften Gottes auch verderblich auf uns wirken, vielleicht wird der Gedanke, daß Gott immer geneigt und bereitwillig sey, uns unsere Sünden zu vergeben, daß es ihm gleichsam selbst wehe thue, uns zu strafen, uns vermessentlich auf Gottes Barmherzigkeit fortsündigen machen? -- Es steht an uns, diese verderbliche Wirkung stets zu verhindern, sobald wir uns nur an alle die übrigen Lehren des Christenthums halten, vornehmlich an die Lehre, daß bei aller Barmherzigkeit Gottes die Sünde doch nie völlig ungestraft bleibe, daß sie doch immer eine ihre Lust weit überwiegende Strafe erfahre, und daß vollends derjenige, der im vermessentlichen Vertrauen auf Gottes Barmherzigkeit sündiget, nicht diese gemißbrauchte Barmherzigkeit, sondern die härteste Strafgerechtigkeit erfahre, und eine Sünde begehe, die ihrer Natur nach zu denjenigen gehört, durch die sich der Mensch den Weg zur Besserung und somit auch zur Vergebung selbst abschneidet.\end{RWanm}
\item Alle diese Eigenschaften Gottes, seine unendliche Gnade, Barmherzigkeit, Milde und Langmuth können und sollen uns selbst zu einem Vorbilde dienen, damit auch wir mit unseren Nebenmenschen gnädig, barmherzig, milde und langmüthig verfahren; zumal da eine andere Lehre des Christenthums sagt, wir würden nur eben so bei Gott Barmherzigkeit finden, wie wir Barmherzigkeit geübt an unseren Mitgeschöpfen. (Man denke \zB\ an die Parabel vom unbarmherzigen Knechte, \ua )~\RWSeitenw{152}
\end{aufza}

\RWpar{94}{Wirklicher Nutzen}
Groß ist der wirkliche Nutzen, den diese Lehre des Christenthums gestiftet hat.
\begin{aufza}
\item Zahllos die Menge der Menschen, die bei dem Bewußtseyn ihrer in früheren Jahren begangenen Fehltritte durch diese Lehre beruhigt wurden, da sie im Gegentheile unter den bängsten Besorgnissen erlegen wären, und aller Kraft, alles Muthes, an ihrer weiteren Vervollkommnung zu arbeiten ermangelt haben würden. Hieher gehören besonders melancholische und zur Scrupulosität geneigte Menschen.
\item Eben so unläugbar ist es, daß die christliche Lehre von Gottes Barmherzigkeit, Milde und Langmuth viele erbitterte Feinde versöhnt, viele beleidigte und äußerst aufgebrachte Menschen in ihrer Wuth zurückgehalten, und sie zur großmüthigen Vergebung der erlittenen Unbilden, oder zur Milderung der festgesetzten Strafe, oder zu einer vorsichtigen Verschiebung derselben \udgl\  bewogen habe. Läßt es sich also auch nicht läugnen, daß diese Lehre gemißbraucht worden ist: so ist doch gewiß der Nutzen derselben bei Weitem überwiegend.
\end{aufza}

\RWpar{95}{Die Lehre von Gottes unendlicher Seligkeit}
So wie wir Menschen die Kraft oder Fähigkeit haben, unsern Zustand zu empfinden, so hat auch Gott, wie das Christenthum lehrt, ein Empfindungsvermögen und zwar ist die Empfindung, die er von seinem Zustande hat, eine \RWbet{ganz reine, ununterbrochene} und ihrem Grade nach \RWbet{unendlich hohe Seligkeit; Gott ist höchst selig.} Und diese höchste Seligkeit Gottes ist keine Lust der Sinne, Gott bedarf auch gar keines äußeren Gegenstandes, um ihrer zu genießen; sondern er findet sie in dem \RWbet{Bewußtseyn seiner unendlichen Vollkommenheiten} selbst.

\RWpar{96}{Historischer Beweis dieser Lehre}
\begin{aufza}
\item \RWbibel{Hiob}{Hiob}{35}{6}:\par
\erganf{Wofern du sündigest, was schadest du Ihm dadurch?\par
Wenn immer anwächst deine Missethat, wird Er dadurch gekränkt?~\RWSeitenw{153}\par
Bist du gerecht: gewinnet Er etwas dabei?\par
Empfängt Er ein Geschenk aus deiner Hand?\par
Geschöpfe, wie du bist, allein, geht deine Bosheit,\par
Und Menschen deine Tugend an.}
\item \RWbibel{Apg}{Apostelg.}{17}{24}: \erganf{Gott, der Herr des Himmels und der Erde wohnt nicht in Tempeln, von Menschenhand gebaut; auch wird er nicht von Menschen bedient, als ob er etwas bedürfte; er selbst ist es, der Allen Leben und Odem und Alles gibt. So würden sich diese Schriftsteller gewiß nicht ausgedrückt haben, hätten sie nicht geglaubt, daß Gott ein der Empfindung fähiges, und zwar in dem Besitze der seligsten Empfindung befindliches Wesen sey.}
\item Und dieß wird auch mit ausdrücklichen Worten gesagt \RWbibel{1\,Tim}{1\,Tim.}{6}{15}: \erganf{Ich beschwöre dich vor Gott, dem Allbeleber, der allein selig ist, und König der Könige, und Herr der Herren.}
\end{aufza}\par
\RWbet{Einwurf}. Es heißt in der Bibel hie und da, daß es Gott \RWbet{gereuet} habe, daß er \RWbet{erzürnt} worden sey, daß er dieß oder jenes \RWbet{wünsche} \udgl\  Dieses verträgt sich nicht mit dem Begriffe einer ganz reinen Seligkeit.\par
\RWbet{Antwort}. Aus der angeführten Stelle bei Job kann man schon sehen, daß jene Ausdrücke nur \RWbet{bildlich} zu verstehen seyen, und daß die Aehnlichkeit nicht in der Empfindung, die Gott hat, sondern in jener Handlungsweise, welche er gegen uns Menschen beobachtet, bestehe.

\RWpar{97}{Vernunftmäßigkeit}
Wie diese Eigenschaft Gottes aus bloßen Vernunftgründen folge, wurde im ersten Haupttheile gezeigt.

\begin{RWanm} 
Die Einwendungen, die man gegen diese Lehre zum Vorschein gebracht hat, sind ohne alle Bedeutung. So meinte \zB\ \RWbet{Epikur,} daß Gott, weil er doch keinen Körper hat, und somit keiner sinnlichen Vergnügungen fähig ist, nebstdem auch durch die Bemühungen, die ihm die Leitung des Weltalls verursacht, unaufhörlich belästiget wird, sich nicht selig fühlen könne. Und \RWbet{Peter Bayle} behauptete, daß Gott durch die Bemerkung,~\RWSeitenw{154}\ es werde nie besser in der Welt, beunruhiget werden müsse, \udgl\  -- Lauter anthropomorphistische Vorstellungen, die in unseren Tagen keiner ernstlichen Widerlegung bedürfen, und hier nur als Beispiele angeführt werden mögen, wie unbedeutend mitunter dasjenige sey, das die Gegner unseres Glaubens gegen ihn vorzubringen sich nicht entblödet haben.
\end{RWanm}

\RWpar{98}{Sittlicher Nutzen}
\begin{aufza}
\item Die Lehre von Gottes höchster Seligkeit ermuntert uns in unserem Streben, Gott immer ähnlicher zu werden, weil wir dann eine ähnliche Seligkeit zu erlangen hoffen dürfen.
\item Da uns die Offenbarung versichert, daß Gott diese Seligkeit \RWbet{in dem Bewußtseyn seiner Vollkommenheiten} finde: so ergibt sich hieraus die Folgerung, daß auch wir, wenn wir uns glücklich fühlen wollen, nach wahrer Vollkommenheit streben müssen.
\item Da es weiter heißt, Gott finde seine Seligkeit \RWbet{bloß in sich selbst:} so begreifen wir leichter, daß er bei allen Welteinrichtungen, bei allen Geboten, die er uns gab, \usw\, nicht seinen Vortheil, sondern einzig nur unser eigenes Wohl beabsichtiget habe, wir achten ihn nun um so höher, und fühlen uns um so strenger verpflichtet, und um so aufgelegter, seine Gebote zu befolgen.
\item Da wir einerseits hören, daß Gott die höchste Seligkeit genieße, und andererseits, daß er doch keinen Körper habe, und keine \RWbet{sinnliche Vergnügungen} kenne: so lernen wir einsehen, daß sinnliche Freuden gar nicht die höchsten seyn können. Eine Wahrheit, die unsere ohnehin allzuheftige Begierde nach sinnlichen Vergnügungen zum größten Vortheil für unsere Tugend und Glückseligkeit herabstimmt.
\end{aufza}

\RWpar{99}{Wirklicher Nutzen}
Es ist bekannt, daß unter den heidnischen Völkern die Vorstellung herrschte, daß Gott, oder vielmehr die Götter~\RWSeitenw{155}\ allerlei sinnliche Freuden genößen, mitunter auch sehr lasterhafte; daß sie ferner an den Opfern der Menschen, ja sogar, was das Schlimmste ist, an ihren Lastern und Ausschweifungen ein Vergnügen fänden, und dadurch verehrt und uns geneigt gemacht würden. Welche verderbliche Folgen mußte dieß für die Sittlichkeit haben!

\RWpar{100}{Die Lehre von Gottes reingeistiger Natur oder Körperlosigkeit}
Das Christenthum gehet, wie wir bisher gesehen, in seiner Darstellung des Bildes Gottes von der Vergleichung mit dem Menschen aus. Wie wir in uns eine Erkenntnißkraft, einen Willen, ein Empfindungsvermögen finden, so nimmt es diese Kräfte auch in Gott an, nur mit den gehörigen Nebenbestimmungen. Das Wesen, das diese Kräfte in uns vereiniget, heißt Seele oder Geist. Das Christenthum also nennt zu Folge dieser Aehnlichkeit auch Gott einen Geist. Die Seele des Menschen ist aber auch noch mit einem gewissen Körper oder Leibe verbunden, worunter wir eine Materie verstehen, welche, vermöge ihrer besondern Einrichtung, geeignet ist, unserer Seele gewisse Vorstellungen unmittelbar zuzuführen, ingleichen Alles, was mit dieser Materie organisch zusammenhängt.\par		
Solch einen Körper nun sollen wir uns, zu Folge des Christenthums, bei Gott nicht vorstellen; sondern ihn sollen wir als einen \RWbet{reinen} oder \RWbet{körperlosen Geist} betrachten.

\RWpar{101}{Historischer Beweis dieser Lehre}
\begin{aufza}
\item Schon in den BB.\ d.\ a.\,B.\ finden wir die Lehre von Gottes Körperlosigkeit vorgetragen. \RWbibel{Dtn}{5\,Mos.}{4}{12}: \erganf{Jehova redete zu euch aus der Mitte der Flammen, die Stimme hörtet ihr, doch eine Gestalt konntet ihr nicht gewahren. Darum hütet euch, daß ihr euch nicht versündiget, und euch kein Bildniß von Gott machet, es sey nun männlicher oder weiblicher Gestalt; es habe die Gestalt von einem Thiere,~\RWSeitenw{156}\ das auf der Erde kriecht, oder von einem Vogel, der in den Lüften fliegt, von einer Schlange im Staube, oder von einem Fische im Wasser. Und wenn ihr euere Augen gegen den Himmel erhebt, und da die Sonne, den Mond, die Sterne erblickt: laßt euch nicht beikommen, sie anzubeten}, \usw 
\item \RWbibel{Jes}{Isai.}{40}{18}: \erganf{Wem wollt ihr Gott vergleichen? was für ein Bildniß wollt ihr ihm setzen?}
\item Deutlicher noch kommt diese Lehre in den Büchern des neuen Bundes vor.
\begin{aufzb}
\item \RWbibel{Joh}{Joh.}{4}{24}:  \erganf{Gott ist ein Geist; daher müssen ihn seine Verehrer im Geiste und in der Wahrheit anbeten.}
\item \RWbibel{1\,Tim}{1\,Tim.}{1}{17}: \erganf{Ihm, dem ewigen Könige, dem Unvergänglichen, dem \RWbet{Unsichtbaren,} dem einigen Gott sey Ehre und Preis in Ewigkeit.}
\end{aufzb}
\item[\RWbet{Einwurf.}] Aber es gibt auch viele Stellen, besonders in den BB.\ d.\ a.\,B., in denen der Gottheit Augen, Ohren, Hände und Füße \udgl\ beigelegt werden. Auch haben verschiedene christliche Schriftsteller, \zB\ Melito von Sardes, Tertullian, Epiphanius, \uAm\  Gott einen Körper beigelegt.
\item[\RWbet{Antwort.}] Jene Ausdrücke des alten Bundes sind von ihren Verfassern alle nur bildlich verstanden worden. Sie bedienten sich derselben theils aus Armuth der Sprache, theils auch ihrer dichterischen Schönheit wegen. -- Wenn es auch wahr ist, daß Melito, Tertullian und einige andere christliche Schriftsteller Anthropomorphiten waren: so beweiset dieß doch nichts gegen den allgemeinen Glauben. Theodoretus nennt den Anthropomorphismus \RWlat{(lib.\,4.\ c.\,10.) stultam sententiam.} Hieronymus \RWlat{(epist.~61.\ ad Comachium) stultissimam haeresin.} Eben so auch der heil.\ Augustin \uA\ Indessen ist nicht zu läugnen, daß selbst diejenigen christlichen Schriftsteller älterer Zeit, welche die geistige Natur Gottes vertheidigten, keinen ganz richtigen Begriff von ihr gehabt haben mögen. Der völlig richtige Begriff eines Geistes nämlich, als eines Wesens, das gar keine Ausdehnung hat, ist zu schwer zu fassen, als daß er der älteren Philosophie nur überhaupt bekannt gewesen wäre. Was man Geist nannte, dachte man sich doch immer als irgend eine sehr feine Materie, die alle übrigen durchdringt, \udgl\  Diese Vorstellun\RWSeitenw{157}gen waren aber ohne nachtheilige Folgen; daher die Offenbarung sie immerhin auf sich beruhen lassen konnte. Erst in neuerer Zeit, da man den richtigen Begriff eines Geistes (etwa seit \RWbet{des Cartes}) aufgefunden hatte, wurde es ein Bedürfniß, die Lehre von Gottes rein geistiger Natur bestimmter auszudrücken.
\end{aufza}

\RWpar{102}{Vernunftmäßigkeit}
\begin{aufza}
\item[{[1.]}]\stepcounter{enumi} Wer immer zugibt, daß Gott Erkenntniß-, Empfindungs- und Wollkraft habe, der kann auch keinen Anstand nehmen, Gott einen \RWbet{Geist} zu nennen, da es gewöhnlich ist, jede Substanz, die eine der eben gesagten Kräfte besitzt, eine geistige zu nennen.
\item Was aber das zweite Stück dieser Lehre, oder die \RWbet{Körperlosigkeit} Gottes betrifft: so ist freilich gewiß, daß Gott auf die Materie der Welt nicht nur mittelbar, sondern unmittelbar einwirken könne; und das Christenthum selbst lehret dieses in der gleichfolgenden Lehre von Gottes Allgegenwart. Wollte man also, wie dieses Einige gethan, jede Materie, auf welche ein geistiges Wesen unmittelbar einwirket, den Leib desselben nennen: so ließe sich immerhin behaupten, Gott habe einen Leib, nämlich die Materie der ganzen Welt wäre in dieser Hinsicht sein Leib. Das ist schon in der That von mehreren alten Weltweisen behauptet worden, \zB\ von Thales, Aristoteles, \uA\ Allein man sieht von selbst, daß jenes Verhältniß, in dem die Materie dieser Welt zu Gott steht, nur eine sehr geringe Aehnlichkeit mit dem Verhältnisse habe, in dem der Leib des Menschen oder sonst eines andern endlichen Wesens zu demselben steht.
\begin{aufzb}
\item Ein Leib nämlich ist eine Materie, die einer \RWbet{eigenen Einrichtung} bedarf, um zu dem Zwecke jener unmittelbaren Einwirkung auf sie dienen zu können. Gott aber kann auf jeden Theil der Materie, wie er immer beschaffen sey, unmittelbar einwirken.
\item Ein Leib ist ferner eine Materie, auf die nicht nur der Geist, sondern die wechselseitig auch \RWbet{auf den Geist} selbst einwirkt; und dieß zwar so, daß sie Vorstellungen in ihm unmittelbar hervorbringt; der Leib ist die Ursache von so~\RWSeitenw{158}\ viel Lust und Schmerz, die wir empfinden, \usw\ Das Alles ist aber bei dem Verhältnisse, das die Materie der Welt zu Gott hat, nicht der Fall. Die Materie der Welt wirket auf Gott nicht ein, bringt keine Vorstellung, um so viel weniger Gefühle der Lust und des Schmerzens in ihm hervor. Um dieser vielen Unterschiede wegen konnte und mußte das Christenthum mit allem Rechte behaupten, daß Gott keinen Körper habe.
\end{aufzb}
\end{aufza}

\RWpar{103}{Sittlicher Nutzen}
\begin{aufza}
\item Durch diese Lehre erscheint uns Gott um so erhabener über uns, und um so unbegreiflicher, je unvermögender wir uns insgemein fühlen, uns einen ganz körperlosen Geist auch nur recht deutlich vorzustellen. Und so gewinnt unsere Ehrfurcht vor Gott (weil alles Unbegreifliche, sobald es mit dem sittlich Guten vereiniget ist, unsere Achtung gegen dasselbe erhöhet), und unsere Begierde, ihn einst vollkommen kennen zu lernen; besonders da uns die Offenbarung verspricht, daß wir selbst diesen unkörperlichen Gott einst, doch von Angesicht zu Angesicht anschauen sollen.
\item Durch diese Lehre können wir leichter begreifen, wie Gott das ganze Weltall übersehen und beherrschen könne; denn wenn er einen Körper, etwa dem unsrigen ähnlich, hätte: so könnte er nur einen Theil der Welt auf einmal übersehen, \usw\
\item Obgleich Gott keinen Leib hat, so ist er doch das allervollkommenste Wesen; dieses erinnert uns, daß die Vollkommenheiten des Geistes die wichtigsten, ja in gewisser Rücksicht die allein wahren sind; eine Erinnerung, die um so nothwendiger ist, je gewöhnlicher wir über der Sorge für unseren Leib und über dem Erwerbe so mancher kleinlicher Leibesgeschicklichkeiten die Ausbildung unseres unsterblichen Geistes selbst vernachlässigen.
\item Die Wahrheit, daß Gottes Seligkeit in keinen sinnlichen Vergnügungen bestehe, findet nur unter Voraussetzung der gegenwärtigen Lehre statt.~\RWSeitenw{159}
\item Wenn wir Gott irgend einen Leib, etwa ähnlich dem unsrigen, beilegen würden: so könnten wir auch seine Ewigkeit und Unzerstörbarkeit weniger begreifen.
\item Wir würden ihm ferner eben die Unvollkommenheiten beilegen, die wir an jenen Gegenständen gewahren, mit denen sein Körper die meiste Aehnlichkeit hätte, \zB\ menschliche oder thierische Schwächen, \udgl\ 
\item Unsere Gottesverehrung würde sehr bald in einen bloß sinnlichen Ceremoniendienst ausarten; wir würden der Gottheit durch unsere Opfer \udgl\  eine eigentliche Freude zu machen glauben. Am Ende könnten wir wohl gar in den Irrthum gerathen, ihr ein Wohlgefallen an gewissen sinnlichen Ausschweifungen beizulegen, und sie daher durch solche Ausschweifungen zu verehren suchen. Dieses wichtigen Einflusses der Lehre von Gottes reingeistiger Natur auf unsere Verehrung desselben erwähnte auch Jesus in der angeführten Stelle \RWbibel{Joh}{Joh.}{4}{24}, und der Apostel (\Ahat{\RWbibel{Röm}{Röm.}{1}{23\,f}}{Röm. 1,13.}): \erganf{Die Majestät des unverweslichen Gottes haben sie verwandelt in verweslicher Menschen-, Thiere-, Vögel- und Schlangen-Gestalt, und darum hat sie Gott dahingegeben in ihre unlautere Begierden und Wünsche,} \usw\
\end{aufza}

\RWpar{104}{Wirklicher Nutzen}
Die so eben berührten Nachtheile, die aus Verkennung der Lehre von Gottes Körperlosigkeit hervorgehen, bestätiget nur allzusehr die Geschichte der heidnischen Völker.

\RWpar{105}{Die Lehre von Gottes Allgegenwart und Ewigkeit}
\begin{aufza}
\item Obgleich das Christenthum einerseits behauptet, daß Gott ein geistiges Wesen sey, und also nicht einen einzigen Theil des Raumes auf Art der Materie ausfülle: so lehrt es doch andrerseits, daß Gott \RWbet{allgegenwärtig} sey, \dh\ daß er in \RWbet{allen Theilen des Raumes}, und zwar \RWbet{in jedem ganz} zugegen sey, obgleich auf eine andere Art, als die Materie, die den Raum ausfüllt.~\RWSeitenw{160}
\item Und wie in allen Theilen des Raumes, so ist Gott auch \RWbet{in allen Theilen der Zeit} durch seine Wirksamkeit vorhanden, er war, er ist und wird immer seyn, welches die \RWbet{Ewigkeit} genannt wird.
\end{aufza}

\RWpar{106}{Historischer Beweis dieser Lehre}
A.~\RWbet{Allgegenwart.}
\begin{aufza}
\item \RWbibel{Hiob}{Hiob}{11}{7}: \erganf{Willst du die Einsicht deines Gottes erschöpfen, und ihn, den Allmächtigen, ergründen? Höher ist er, als des Himmels Höhe, tiefer als des Abgrundes Tiefe, länger als der Erde Maß, und breiter als die Meere.} (Doch ist es nicht sicher, daß diese Uebersetzung den Sinn des Originals richtig treffe.)
\item \Ahat{\RWbibel{Ps}{Psalm}{139}{7--10}}{138,7--10.}:\par
\erganf{Ach wohin soll ich fliehen vor Deinem Geiste!\par
Wo verbergen mich hin vor Deinem Blick?\par
Steig' ich auf in den Himmel: da bist Du!\par
Machte mein Bett\par
Tief im Abgrund ich mir: so bist auch dort Du!\par
Schwäng auf Flügeln ich mich der Morgenröthe,\par
Ruh'te dann an der letzten\par
Gränze des Meeres aus: --\par
Dort auch würde mich Deine Hand noch leiten,\par
Noch erfassen mich dort Dein Arm.}
\item \RWbibel{2\,Chr}{2\,Chron.}{6}{18}\ spricht Salomo bei Einweihung des neu erbauten Tempels: \erganf{Doch wie? wohnt denn in Wahrheit Jehova bei uns Menschenkindern auf Erden? O nein! der Himmel und der Himmel Himmel fassen Dich nicht, Jehova! um wie viel weniger dieß kleine Haus, das ich Dir baute.}
\item \RWbibel{Jes}{Isai.}{66}{1}: \erganf{So spricht der Herr: Der Himmel ist mein Thron, die Erde meiner Füße Schemel. Wo könnte stehen das Haus, das ihr mir bauen wolltet? und wo die Stätte, da ich ruhen soll?}
\item \Ahat{\RWbibel{Jer}{Jer.}{23}{23}}{Ezech.~23,23.}: \erganf{Bin ich nur Gott in der Nähe, spricht Jehova, und nicht auch in der Ferne Gott? Kann~\RWSeitenw{161}\ Jemand sich so heimlich verbergen, daß ich ihn nicht gewahre? spricht Jehova. Erfüll' ich nicht Himmel und Erde?}
\item \RWbibel{Apg}{Apostelg.}{17}{26}: \erganf{Er hat aus Einem Blute alle Völker der Erde hervorgehen lassen; und setzte ihnen bestimmte Zeit ihres Vorhandenseyns auf Erden; gab ihnen Zeit, den Herrn zu suchen, ob sie wohl so vernünftig seyn würden, ihn zu erkennen, da er doch in der That nicht fern ist von einem Jeglichen aus uns; denn in ihm leben und weben wir ja.}
\end{aufza}

\vabst B.~\RWbet{Ewigkeit.}
\begin{aufza}
\item Schon Abraham nennt (\Ahat{\RWbibel{Gen}{1\,Mos.}{21}{33}}{1\,Mos.~21,23}) Gott den Ewigen.
\item Den Israeliten ließ sich Gott durch Mosen (\RWbibel{Ex}{2\,Mos.}{3}{14}) als den, \RWbet{der war, ist und seyn wird,} ankündigen; worauf sich auch jener eigene Name, den er sich gab, \RWbet{Jehova}, bezieht.
\item \RWbibel{Ps}{Psalm}{102}{25}:\par
\erganf{Fort währen Deine Jahr' in Ewigkeit! \par
Der Erdball, gegründet ehedem von Dir, \par
auch selbst die Himmel, Deiner Finger Werk, \par
vergehen; Du bleibst; \par
sie altern wie ein Kleid,\par
wie ein Gewand legst Du sie ab.\par
Du aber bleibst Derselbe \par
und Deine Zeit ist ohne Ende.}
\item \RWbibel{Röm}{Röm.}{16}{26}\ \RWbibel{Offb}{Offenb.}{6}{1}\ \RWbibel{Offb}{}{8}{}
\end{aufza}

\RWpar{107}{Vernunftmäßigkeit}
A.~\RWbet{Allgegenwart.}
\begin{aufza}
\item Das Wort Gegenwart wird in einer andern Bedeutung von der Materie, in einer andern von geistigen Wesen genommen. Von der Materie sagt man, sie sey an einem Orte gegenwärtig, wenn sie denselben ausfüllt; bei \RWbet{geistigen} Wesen dagegen versteht man unter Gegenwart eine Art von \RWbet{Wirksamkeit}, besonders wenn solche \RWbet{unmittelbar} ist. So sagen wir von unserer Seele, sie sey in unserem Leibe gegenwärtig, in wiefern wir uns vorstellen, daß sie im ganzen Leibe gewisse Wirkungen, und dieß zwar unmittelbar~\RWSeitenw{162}\ hervorbringe oder doch hervorbringen könne. Können wir also darthun, daß Gott in allen Theilen des Raumes wirke, und dieß zwar unmittelbar: so wird man uns wohl zugeben müssen, daß er allgegenwärtig sey.
\item Zuvörderst ist zu merken, daß die Gegenwart geistiger Wesen überhaupt an ganz andere Gesetze gebunden sey, als jene der Materie. Materie kann nur an einem Orte, und an diesem kann, wie es scheint, zu derselben Zeit sonst keine andere gegenwärtig seyn; welches man eben meint, wenn man sagt, daß sie diesen Ort ausfülle. Nicht also ist es bei Geistern. Ein geistiges Wesen kann zu derselben Zeit an mehreren Orten gegenwärtig seyn, und an demselben Orte, wo Ein Geist gegenwärtig ist, kann auch ein zweiter und dritter zugegen seyn. So wissen wir es \zB\ von unserem eigenen Geiste, daß er an mehreren Orten, ja in der That an unendlich vielen Orten zugleich gegenwärtig sey, wenigstens wenn dieß so viel heißen soll, daß er an allen diesen Orten unmittelbare Wirkungen erzeuge. Dieses ist nämlich der Fall, so oft irgend eine noch so geringe Veränderung in unserem Leibe, \zB\ eine Bewegung seiner Gliedmaßen durch den Willen der Seele hervorgebracht wird. Anatomie und Physiologie lehren zwar, daß die größeren Theile unseres Körpers \zB\ Hände, Füße, nicht unmittelbar durch den Willen der Seele, sondern erst mittelbarer Weise durch Muskeln, diese durch Nerven \usw\ in Bewegung gesetzt werden; aber gewiß muß es doch immer Einen Theil der Materie unseres Leibes geben, den wir in einem solchen Falle unmittelbar (nicht erst durch einen Stoß vermittelst eines andern Theiles) in Bewegung setzen. Allenthalben, wo eine mittelbare Wirkung Statt findet, muß ja doch auch eine unmittelbare vorhanden seyn. Dieser Theil der Materie nun, den wir unmittelbarer Weise in Bewegung setzen, wenn wir \zB\ unsere Hand bewegen (er ist ein Theil des sogenannten Seelenorgans), er sey noch so klein, so muß seine Masse doch in irgend einem endlichen Verhältnisse mit der Maße des bewegten Gliedes stehen, weil nach mechanischen Gesetzen unter was immer für einer Verbindung von Kräften die sogenannten Momente der Kraft und Last doch immer gleich~\RWSeitenw{163}\ bleiben. Ein endlicher Theil der Materie aber erfüllt auch einen endlichen Theil des Raumes; in einem solchen gibt es unendlich viele Orte; und unendlich viele einzelne Atome. In alle diese wirkt also unsere Seele in einem und demselben Augenblicke (in demjenigen, da sie die erste Bewegung in unserem Seelenorgane anfängt) unmittelbar ein: sie ist also an unendlich vielen Orten zu gleicher Zeit gegenwärtig. Ist dieß einem endlichen Geiste möglich: so muß es um so mehr dem unendlichen Geiste, Gott, möglich seyn, an allen Orten des unendlichen Raumes zugleich unmittelbar zu wirken.
\item Und dieß thut er auch wirklich. Wenn keine andere Wirksamkeit Gottes unmittelbar ist: so ist es doch gewiß diejenige, durch welche er die in der Welt befindlichen Substanzen in ihrem Daseyn erhält, denn wie die Erhaltung einer Substanz mittelbar, also durch eine andere endliche Substanz bewirkt werden könnte, begreifen wir durchaus nicht, da alle Wirksamkeit endlicher Substanzen auf einander, so viel wir wissen, nur in Veränderung ihres Zustandes besteht. -- Nun gibt es aber in allen Theilen des Raumes Substanzen, wenigstens materielle; also wirkt Gott in allen Theilen des Raumes schon dadurch unmittelbar, daß er die in demselben befindlichen Substanzen in ihrem Daseyn erhält, \dh\ Gott ist in allen Theilen des Raumes gegewärtig.
\end{aufza}

\vabst B.~\RWbet{Ewigkeit.}
\begin{aufza}
\item Wie bei dem Raume gesagt wird, ein Geist sey in ihm gegenwärtig, wenn er nur unmittelbar in ihm wirkt: so pflegt man auch zu sagen, ein geistiges Wesen sey vorhanden in jeder Zeit, in der es Wirkungen, unmittelbare Wirkungen hervorbringt. Die Behauptung, Gott ist ewig, hat also keinen andern Sinn, als, er bringt zu allen Zeiten gewisse unmittelbare Wirkungen hervor.
\item Dieß läßt sich nun eben so, wie die Allgegenwart, beweisen. Wenn sonst keine andere Einwirkung Gottes, welche die in der Zeit befindlichen Substanzen erfahren, unmittelbar erfolgt: so ist doch gewiß die Erhaltung derselben eine unmittelbare Wirkung Gottes, die zu allen Zeiten Statt findet; und also ist Gott zu allen Zeiten unmittelbar wirksam, und folglich ewig.~\RWSeitenw{164}
\end{aufza}

\RWpar{108}{Sittlicher Nutzen}
A.~\RWbet{Allgegenwart.}
\begin{aufza}
\item Durch die Lehre von Gottes Allgegenwart wird uns für's Erste begreiflicher und gewisser seine Allwissenheit. Wenn wir uns Gott nur gegenwärtig an einem gewissen Orte dächten: so würde sich allmählich bei uns der Zweifel einschleichen, ob er auch Alles wisse und erfahre, was sich in anderen Theilen der Welt, in solchen, die von seinem vermeintlichen Aufenthalte sehr weit entfernt sind, zuträgt. Im Gegentheile aber ist Gott an allen Orten zugegen, auch in dem Innersten unseres Herzens (leben und weben wir in ihm, wie der Apostel sich ausdrückt): so muß er freilich \RWbet{Alles wissen.} Daher werden wir denn, wenn wir \zB\ zu Gott beten wollen, nicht zweifeln, daß er uns höre, wo wir auch immer uns befinden mögen.
\item Ein Gleiches, wie von der Allwissenheit, gilt auch von Gottes Allregierung. Wenn er überall zugegen ist, wenn er die letzte Ursache von allen Dingen ist: so wird er auch \RWbet{Alles zu seinem Zwecke zu leiten wissen.}
\item Die Lehre von Gottes Allgegenwart flößt uns auch eine heilige Furcht ein, die irdischen Gegenstände, in denen Gott zugegen ist, die nur durch seine Wirksamkeit bestehen, \RWbet{gegen seine heiligen Absichten zu mißbrauchen.} Das Sträfliche eines solchen Verbrechens stellt sich uns um so deutlicher vor Augen.
\item Was diese letztere Wirkung besonders begünstiget, ist noch der Umstand, daß sich bei dem Gedanken an Gottes Gegenwart, vermöge der Aehnlichkeit, immer das Bild von der \RWbet{leiblichen Gegenwart irgend eines großen, und von uns geachteten Mannes} einstellt. Wie wir uns scheuen würden, in Gegenwart eines solchen Mannes irgend etwas Unanständiges oder ihm Mißfälliges auch nur zu denken, geschweige denn wirklich zu thun: so und weit mehr noch (begreifen wir jetzt) müssen wir uns scheuen, in Gottes Gegenwart auch nur die geringste Sünde, selbst in Gedanken, zu begehen.~\RWSeitenw{165}
\end{aufza}

\vabst B.~\RWbet{Ewigkeit.}
\begin{aufza}
\item Glaubten wir nicht, daß Gott von Ewigkeit her sey: so könnten wir uns auch nicht überreden, daß alle Einrichtungen im Weltall sich \RWbet{nur von ihm} herschreiben.
\item Glaubten wir nicht, daß er in Ewigkeit fortdauern werde, so könnten wir auch an keine \RWbet{in Ewigkeit fortdauernde sittliche Weltregierung,} ingleichen an keine ewige Belohnungen und Strafen glauben; Wahrheiten, deren wohlthätiger Einfluß in der Folge gezeigt werden soll.
\item Gottes ewiges Daseyn, besonders der Umstand, daß er gar keinen Anfang hat, trägt wegen der Mühe, die uns die bloße Vorstellung davon verursacht, viel dazu bei, daß wir \RWbet{mehr Ehrfurcht vor Gott} haben. (s. Haller's Gedicht an die Ewigkeit.)\RWlit{}{Haller1}
\end{aufza}

\RWpar{109}{Wirklicher Nutzen}
Von der einen Seite ist kein Schade abzusehen, den diese Lehren durch Mißverstand oder Mißbrauch jemals verursacht hätten; von der andern Seite aber ist durch die Erfahrung unzähliger Menschen gewiß, daß die Lehre von Gottes Allgegenwart eines der wirksamsten Abhaltungsmittel von Sünden, besonders von geheimen Sünden, gewesen sey. Darum bedient sich die heil.\ Schrift von frommen Personen gewöhnlich der Redensart: sie wären gewandelt vor Gott, \dh\ sie hätten den Gedanken an Gottes Allgegenwart beständig bei sich zu erhalten gesucht. (s.~auch \RWlat{Senec.\ epist.~41.})\RWlit{}{Seneca4}
\begin{RWanm} Die Lehre von Gottes Allgegenwart gehört überdieß zur Classe derjenigen Lehren, welche auch einen bedeutenden wissenschaftlichen Nutzen gehabt. Eben indem man die widersprechend scheinenden Aeußerungen der christlichen Offenbarung von einem Vorhandenseyn Gottes in allen Theilen der Welt, und zwar in jedem ganz, und dann von seiner rein geistigen Natur mit einander zu vereinigen suchte, entwickelten sich nach und nach die richtigen Begriffe von Geist, Materie, Raum, Zeit \usw\ 
\end{RWanm}

\RWpar{110}{Die Lehre von Gottes dreifacher Persönlichkeit}
Die bisher vorgetragenen Lehren von Gott hat die christliche Religion beinahe mit allen bessern monotheistischen Reli\RWSeitenw{166}gionen auf Erden gemein. Die israelitische oder \RWbet{mosaische Religion} nimmt diese Lehren, wenigstens gegenwärtig, alle ausdrücklich in ihren Inhalt auf. Ja es finden sich, wie wir es oben bei den historischen Beweisen dieser Lehren gesehen, in den Büchern des alten Bundes, welche das israelitische Volk bekanntlich als die Erkenntnißquelle seines Glaubens ansieht, Stellen genug, die beweisen, daß schon die Verfasser dieser Bücher die bisher abgehandelten Eigenschaften Gottes alle sehr wohl anerkannt haben. Die \RWbet{Religion der Mahomedaner} oder der \RWbet{Islamismus} nimmt gleichfalls alle bisher erwähnten Eigenschaften Gottes an. Endlich haben auch die Weltweisen versucht, diese göttlichen Eigenschaften aus der bloßen Vernunft herzuleiten, und sie auf diese Art zu Lehrsätzen der sogenannten \RWbet{natürlichen Religion} zu machen, was ihnen zum Theil auch gelungen ist. Nun aber kommen wir zu einer Lehre, welche sich außerhalb des Christenthums bei keiner andern Religion auf Erden, auch bei der israelitischen nicht, so deutlich ausgesprochen findet, die auch seit ihrer Entstehung häufig bestritten und des Widerspruches mit der gesunden Vernunft beschuldiget worden ist. Um so nothwendiger ist es, sie etwas ausführlicher als die bisherigen abzuhandeln. --\par
Die Lehre von Gottes \RWbet{dreifacher Persönlichkeit}, die man auch wohl die Lehre von Gottes \RWbet{dreieiniger Natur} oder \RWbet{Dreieinigkeit} (Trinität) genannt hat, lautet mit allen Nebenbestimmungen, die sie von dem gelehrteren Theile der Katholiken bis auf den heutigen Tag erhalten hat, wie folgt:
\begin{aufza}
\item Obgleich Gott einfach und untheilbar in seinem Wesen (in seiner Natur, Substanz, \RWlat{essentia}) ist: so gibt es doch auch in mancher anderen Rücksicht etwas Vielfaches in ihm, wie denn \zB\ schon in Ansehung seiner Kräfte mehrere bei ihm angenommen werden. Besonders wichtig aber ist für uns Menschen die \RWbet{dreifache Persönlichkeit}, die sich in Gott befindet, kennen zu lernen; indem es eben diese drei göttlichen \RWbet{Personen} sind, die auf verschiedene Art wohlthätig auf uns einwirken.
\item Wir sollen denn also wissen, daß es \RWbet{drei von einander verschiedene} (\RWlat{distinctae}) \RWbet{Personen} (\RWlat{personae},~\RWSeitenw{167}\ \RWgriech[pros'wpa]{pr'oswpa}, \RWgriech{<upost'aseis}) in Gott gebe, die wir mit den drei bildlichen Namen, der \RWbet{Vater}, der \RWbet{Sohn}, und der \RWbet{heilige Geist} zu bezeichnen haben. Den Vater könnten wir auch den \RWbet{Urgrund} (\RWlat{principium}), \RWbet{Urheber} (\Ahat{\RWlat{auctor}}{\RWlat{author}}), \RWbet{Urquell} (\RWlat{fons}); den Sohn das \RWbet{Wort} (\RWlat{verbum}), die \RWbet{Weisheit} (\RWlat{sapientia}), den \RWbet{Abglanz} oder das \RWbet{Abbild} (\RWlat{imago}) \RWbet{des Vaters}; den heil.\ Geist aber die \RWbet{Liebe} (\RWlat{amor, charitas}), die \RWbet{Gabe} (\RWlat{donum}) nennen.
\item Fragen wir aber, was unter dem Ausdrucke einer \RWbet{Person in Gott} zu verstehen wäre: so wird uns von dem gelehrten Theile der Katholiken erinnert, eine Person in Gott oder eine göttliche Person sey ein \RWbet{reelles} (ein außerhalb unserer Vorstellung vorhandenes) \RWbet{Subject} (Gegenstand), dem alle sogenannten natürlichen Prädicate oder Attribute der Gottheit, \dh\ alle bisher vorgetragenen Eigenschaften (unendlicher Verstand, unendlich vollkommener Wille, unendliche Seligkeit, Allgegenwart, Ewigkeit) beigelegt werden können; das aber übrigens mit keinem der uns bekannten Gegenstände eine so große Aehnlichkeit hat, daß es von uns völlig begriffen werden könnte.
\item Es wäre falsch, wenn wir die drei Personen in Gott für drei \RWbet{bloße Namen}, oder für drei \RWbet{bloße Verhältnisse Gottes zur Welt}, oder für drei \RWbet{besondere Wirkungsarten} ansehen wollten, da sie vielmehr drei besondere \RWbet{Gründe} (\RWlat{subjecta}) in Gott sind, aus welchen drei verschiedene Wirkungsarten entspringen, mit diesen selbst aber nicht zu vermengen sind.
\item Eben so falsch wäre es dagegen auch, wenn wir einer jeden von diesen drei Personen ein eigenes, von den andern verschiedenes \RWbet{Wesen} (eine eigene Substanz oder Natur) beilegen wollten; ingleichen wenn wir uns vorstellten, daß die Erkenntnißkraft, oder die Wollkraft, oder die Seligkeit, die einer jeden beigelegt wird, eine \RWbet{eigene} sey, so daß es also drei Erkenntnißkräfte, \usw\ in Gott geben müßte.
\item Obwohl die erwähnten natürlichen Prädicate Gottes den drei Personen gemeinschaftlich zukommen, und diese daher in dieser Rücksicht \RWbet{einander gleich} zu nennen sind: so gibt~\RWSeitenw{168}\ es doch auch gewisse \RWbet{wirkliche} (reale) \RWbet{Unterschiede} unter denselben, und zwar theils \RWbet{innere}, theils \RWbet{äußere}.
\item Die \RWbet{inneren Unterschiede} (\RWlat{characteres interni}) bestehen darin, daß
\begin{aufzb}
\item der Vater \RWbet{von sich selbst von Ewigkeit} ist;
\item der Sohn \RWbet{vom Vater gezeugt ist, und zwar von Ewigkeit} und \RWbet{aus seinem eigenen Wesen};
\item der heil.\ Geist \RWbet{vom Vater und vom Sohne zugleich von Ewigkeit ausgeht}, oder ihr gemeinschaftlicher Aushauch ist.
\end{aufzb}
\item Gerne bescheidet sich die Kirche, daß die Redensarten des \RWbet{Zeugens} und \RWbet{Ausgehens} hier nur \RWbet{bildlich} zu verstehen wären. Sie sollen eine gewisse, dem menschlichen Verstande nicht ganz begreifliche Weise bezeichnen, auf welche die zweite Person in Gott durch die erste, und eine andere Weise, auf welche die dritte Person in Gott durch die zwei ersteren ihr ewiges Daseyn erhält. Weder die eine noch die andere Art dieses Mittheilens des Daseyns ist, sagt die Kirche, ein \RWbet{Schaffen} zu nennen; der Sohn und der heil.\ Geist sind also keine Geschöpfe, und nicht von \RWbet{ähnlicher} oder auch \RWbet{gleicher}, sondern von \RWbet{derselben Natur} (\RWlat{non aequalis, sed ejusdem naturae}) mit dem Vater. Ihr Verhältniß zum Vater soll auch nicht einmal als ein Verhältniß der \RWbet{Abhängigkeit} betrachtet werden (\RWlat{non pendent a Patre}). Der Sohn und der heil.\ Geist haben jeder ihr Seyn vom Vater, er ist der Grund (\RWlat{principium}, Quelle) ihres Daseyns; aber sie sind gleichwohl \RWbet{nicht geringer}, als er.
\item Die \RWbet{äußeren Unterschiede} (\RWlat{characteres externi}) bestehen in gewissen Handlungen oder Verhältnissen der drei göttlichen Personen, welche sich auf die Welt beziehen. Sie sind von doppelter Art:
\begin{aufzb}
\item solche, an welchen nur \RWbet{eine einzige Person} Theil genommen hat, oder noch nimmt;
\item solche, an denen wohl \RWbet{alle drei Personen} gemeinschaftlichen Antheil genommen haben, oder noch nehmen, die aber doch der Einen \RWbet{vorzugsweise zugeschrieben} werden.~\RWSeitenw{169}
\end{aufzb}
\item Zu den ersteren, welche \RWlat{opera oecumenica} genannt werden, gehört
\begin{aufzb}
\item beim Vater der Rathschluß, \RWbet{die Menschen durch den Sohn zu erlösen;}
\item beim Sohne die \RWbet{wirkliche Ausführung dieses Rathschlusses} durch Menschwerdung, ertheilten Unterricht auf Erden, erlittenen Tod, \usw\
\item beim heil.\ Geiste die \RWbet{Bildung der menschlichen Natur des Erlösers}, die Ausschmückung derselben mit den herrlichsten Gaben, die Wirksamkeit der Sacramente \usw\
\end{aufzb}
\item Zu den zweiten, welche \RWlat{opera attributiva} oder \RWlat{communia} heißen, gehören folgende:
\begin{aufzb}
\item Dem Vater wird die \RWbet{Schöpfung, Erhaltung} und \RWbet{Regierung} des Weltalls vorzugsweise zugeschrieben.
\item Vom Sohne wird gesagt, der Vater habe \RWbet{durch ihn,} doch nicht als Werkzeug, die Welt erschaffen.
\item Dem heil.\ Geiste wird die \RWbet{Erleuchtung} und \RWbet{Heiligung} der Menschen \usw\ vorzugsweise zugeschrieben.
\end{aufzb}
\item Gemäß der Eigenschaften, welche bei diesen Werken (\RWlat{num.\,10 u.\ 11.}) sich vorzugsweise äußern, werden auch selbst gewisse \RWbet{natürliche Attribute der Gottheit} vorzugsweise der Einen oder der Andern göttlichen Person beigelegt, (welches \RWlat{appropriatio} heißt). So wird \zB\ dem \RWbet{Vater} vorzugsweise die \RWbet{Allmacht,} dem \RWbet{Sohne} die \RWbet{Weisheit}, dem \RWbet{heil.\ Geiste} die \RWbet{Liebe} und \RWbet{Güte} beigelegt; doch nicht in dem Sinne, als ob diese Eigenschaften den drei Personen auch innerlich in einem ungleichen Maße zukämen; sondern nur in sofern, als diese Eigenschaften bei den genannten Wirkungsarten sich \RWbet{vorzugsweise geoffenbaret} haben.
\item Endlich ist noch zu merken, daß die drei göttlichen Personen \RWbet{nicht außerhalb einander} existiren, sondern in und durch einander; sie \RWbet{durchdringen}, wie sich die Kirche ausdrückt, einander, und bilden nur eine einzige Substanz. (\RWlat{Circumincedunt sibi. Circumincessio.})
\end{aufza}

\begin{RWanm} 
Noch haben die Theologen verschiedene Redensarten in Betreff der Lehre von Gottes dreifacher Persönlichkeit theils~\RWSeitenw{170}\ angenommen, theils verworfen. So lehren sie, man dürfe allerdings sagen: \RWlat{Pater est \RWbet{alius}, Filius est \RWbet{alius}, Spiritus sanctus est \RWbet{alius}}; nicht aber \RWlat{\RWbet{aliud}}, weil Jenes sich auf die Person, dieses auf das Wesen bezöge. \RWlat{Pater est Deus, Filius est Deus, Spiritus sanctus est Deus, et tamen non sunt tres Dii, sed totae tres personae unus sunt Deus}. Eben so könne man sagen: \RWlat{Pater est increatus, immensus, omnipotens, Dominus}; und eben so \RWlat{Filius est increatus, immensus} \usw ; und endlich \RWlat{Spiritus sanctus est increatus, immensus} \usw ; aber gleichwohl nicht: \RWlat{Sunt tres increati, tres immensi, tres omnipotentes, tres Domini}. Es ist nicht richtig gesagt: \RWlat{Deus generat Deum}; sondern: \RWlat{Deus Pater generat Deum Filium}; oder \RWlat{Deus spirat Deum}, sondern: \RWlat{Deus Pater et Filius spirant Spiritum sanctum}. -- \RWlat{Deus est \RWbet{communis} tribus personis}. \RWlat{Deus est \RWbet{trinus}}. \RWlat{Una trinitas, trina unitas}. \RWlat{Personae sanctissimae trinitatis sunt \RWbet{distinctae, non vero differentes}, sed \RWbet{aequales}; neque vero sunt separatae, circumincedunt enim sibi. Pater est \RWbet{principium, non vero causa} Filii; causa enim proprie sic dicta involvit dependentiam causati. Non datur subordinatio in divinis, ideoque nec Filius nec Spiritus sanctus est minor Patre.} -- Unrichtig sey es auch gesagt: \RWlat{Pater et Filius et Spiritus sanctus sunt partes divinae naturae vel Deitatis}, \usw
\end{RWanm}

\RWpar{111}{Historischer Beweis dieser Lehre. Plan desselben}
Da die Lehre von Gottes dreifacher Persönlichkeit in jeder Rücksicht so wichtig ist: so dürfte es sich wohl der Mühe lohnen, hier eine kurze Geschichte ihrer Entstehung und allmählichen Ausbildung zu liefern.
Daß sie gegenwärtig wirklich nicht anders vorgetragen werde, als ich sie eben jetzt dargestellt habe, braucht nicht eigens erwiesen zu werden, indem man sich hievon durch einige Blicke in die gangbarsten Lehrbücher der katholischen Dogmatik überzeugen kann. Ich will hier also:
\begin{aufzb}
\item zeigen, was schon vor Einführung dieser Lehre durch Jesum Christum vorhanden war, und eine leise Hindeutung auf sie enthielt, oder doch Einiges zu ihrer Ausbildung beigetragen haben dürfte; dann will ich~\RWSeitenw{171}\ 
\item Jesu eigene Aeußerungen über diese Lehre mittheilen; und zwar
\begin{aufzc}
\item erst solche, die wir in den drei ersteren oder harmonischen Evangelien finden, dann
\item jene, welche noch überdieß das Evangelium Johannis liefert. Hierauf will ich
\end{aufzc}
\item den Glauben der ersten Schüler Jesu, eines Paulus, Petrus, Johannes \uA\ darstellen; dann
\item die wichtigsten Bibelstellen, die dieser Lehre zu widersprechen scheinen, betrachten; hierauf noch
\item einige Proben von dem Glauben der Kirchenväter aus den drei ersten christlichen Jahrhunderten liefern, und endlich
\item die fernere Geschichte dieser Lehre bis auf die gegenwärtige Zeit in gedrängter Kürze entwerfen.
\end{aufzb}

\RWpar{112}{a.~Leise Hindeutungen auf diese Lehre, die schon vor ihrer Einführung durch Jesum Christum vorhanden waren, und Einiges zu ihrer Entstehung und Ausbildung beigetragen zu haben scheinen}
Die Vorsehung fand es, wie bereits angemerkt wurde, nicht für gut, die Lehre von Gottes dreifacher Persönlichkeit schon dem Volke der Juden bekannt werden zu lassen; vermuthlich -- wegen der allzugroßen Versuchung zum Tritheismus und zu gewissen andern sehr grobsinnlichen Vorstellungen von Gott, die dieses noch rohe Volk in einer solchen Lehre gefunden haben würde. Als aber die Zeit herankam, wo nach dem Plane der göttlichen Vorsehung diese Lehre den Menschen gepredigt werden sollte, wurde, wenn nicht schon ihre Entstehung in dem Gemüthe Jesu (nach seiner menschlichen Natur), so doch gewiß ihre weitere Ausbildung in der christlichen Kirche nicht durch ganz unmittelbare Eingebungen, sondern sehr mittelbar und durch Benützung der verschiedensten Gegenstände von Gott veranlaßt.\par

\vabst I.~So dürften nämlich die BB.\ d.\ a.\ B.\ selbst schon an der Entstehung oder Ausbildung dieser Lehre, oder wenigstens~\RWSeitenw{172}\ an dem leichteren Eingange, den sie bei Vielen in der Folge fand, einen nicht unbedeutenden Antheil haben; besonders
\begin{aufza}
\item durch mehrere Stellen, in welchen \RWbet{Gott} entweder \RWbet{in der Mehrzahl von sich spricht,} wie \RWbibel{Gen}{1 Mos.}{1}{26}; \RWbibel{Gen}{}{3}{22}; \RWbibel{Gen}{}{11}{7}, oder in welchen er mit einem \RWbet{Namen} belegt wird, der eine \RWbet{Mehrzahl in sich schließt}, dergleichen die Namen: \RWhebr{'E:lOhiym}, \RWhebr{'a:donAy} (Elohim, Adonai) sind.
\item Durch Stellen, welche die Zahl \RWbet{drei als eine heilige Zahl und in Verbindung mit Gott darstellen;} \zB\ wenn der Name Gottes auf eine geheimnißvolle Weise dreimal wiederholt wird, wie bei der Segensformel \RWbibel{Num}{4 Mos.}{6}{24--26}, oder wenn die Engel am Throne das Dreimal-heilig singen. \RWbibel{Jes}{Jes.}{6}{3}
\item Durch Stellen, in welchen von einem \RWbet{Engel} (Abgesandten Gottes) die Rede ist, dem manchmal der \RWbet{Name Jehova} beigelegt wird; wie \RWbibel{Gen}{1 Mos.}{6}{10--12}
\item Durch Stellen, in welchen dem Messias ein \RWbet{mehr als menschliches Ansehen} und ein göttlicher Name beigelegt wird. Solche Stellen sind \zB\
\begin{aufzb}
\item der \RWbibel{Ps}{Psalm}{110}{}, der die Ueberschrift: \RWbet{Ein Psalm Davids} führt und mit den Worten anfängt: \erganf{\RWbet{Jehova spricht zu meinem Herrn:} Setze dich zu meiner Rechten, bis ich dir deine Feinde zum Schemel deiner Füße lege.} Auf diesen Vers beruft sich Jesus (bei \RWbibel{Mt}{Matth.}{22}{43}) ausdrücklich, um zu beweisen, daß der Messias, auf den man den Psalm damals allgemein bezog, mehr als ein bloßer Nachkomme Davids seyn müsse, weil dieser ihn sonst nicht \RWbet{seinen Herrn} genannt haben würde. Und also scheint es, daß Jesus selbst (nach seiner menschlichen Natur) unter mehreren andern Gründen auch durch diesen Psalm zu der Ueberzeugung gelangt sey, daß er Gott nicht lästere, wenn er sich den \RWbet{Sohn Gottes} nenne.
\item \RWbibel{Jer}{Jerem.}{23}{5--6}: \erganf{Siehe! es kommt die Zeit, spricht Jehova, da ich von Davids Stamme einen echten Sprößling aufschießen lasse, der Recht und Gerechtigkeit auf Erden handhaben wird. -- Und der Name, den man ihm geben wird, ist \RWbet{Jehova, unsere Gerechtigkeit.}}~\RWSeitenw{173}\ (Vergl.\ \RWbibel{Jer}{Jerem.}{32}{16}) Eben so heißt es \RWbibel{Jes}{Isai.}{7}{14}: \erganf{Eine Jungfrau wird empfangen und einen Sohn gebären, deß Name seyn wird \RWbet{Emmanuel} (\dh\ Gott mit uns). Ein Knabe wird uns geboren, ein Kind geschenkt werden, deß Name seyn wird: \RWbet{Wundervoller, Rathgeber, Gott, der Mächtige, der Vater künftiger Zeiten, der Fürst des Friedens.}} -- \RWbibel{Dan}{Daniel}{9}{24}\ wird der Messias der \RWbet{Heilige aller Heiligen} genannt.
\end{aufzb}
\item Durch Stellen, in welchen Gott der Name eines \RWbet{Vaters} beigelegt wird, und andere, in welchen von einem \RWbet{Sohne} desselben auf eine geheimnißvolle Art die Rede ist.
\begin{aufzb}
\item \Ahat{\RWbibel{Ps}{Psalm}{89}{27}}{88,27.}: \erganf{Er wird mich \RWbet{Vater} nennen; ich aber werde ihn zu meinem \RWbet{Erstgebornen} erheben.} Eben so \RWbibel{Weish}{Weish.}{2}{16--18}\ \RWbibel{Sir}{Sir.}{51}{10}
\item \Ahat{\RWbibel{Spr}{Sprichw.}{30}{4}}{30,6.}: \erganf{Wer hält den Wind in seinen Händen? Wie heißt Er, oder wie heißt sein \RWbet{Sohn}? Weißt du es?} --
\end{aufzb}
\item Durch Stellen, in welchen die Weisheit Gottes personifizirt wird; \zB\ \RWbibel{Spr}{Sprichw.}{8}{22\,ff}, wo die Weisheit folgender Maßen redend eingeführt wird: \erganf{Jehova besaß (auch zeugete) mich \RWbet{vor Anfang der Schöpfung} von jeher. Ehe die Meere noch waren, war ich schon da; ehe die Berge eingesenkt wurden, da er den Umkreis der Erde bestimmte, den Himmel bildete, den Luftkreis feststellte, die Quellen des Meeres gründete, \RWbet{war ich bei ihm als seine Gehülfin}, und als sein tägliches Ergötzen. Ich hatte Vergnügen an seiner Erde, und bei den Menschen zu wohnen, war meine Lust.} -- \RWbibel{Weish}{Weish.}{7}{25}: \erganf{Die Weisheit ist ein \RWbet{Hauch} (\RWgriech{pne~uma}) der göttlichen Kraft, ein reiner \RWbet{Ausfluß des Glanzes des Allmächtigen}. Von Zeit zu Zeit steigt sie herab in heilige Seelen, und bildet Freunde Gottes und Propheten.} \RWbibel{Weish}{}{8}{3}: \erganf{Ihr \RWbet{trauter Umgang mit Gott} erhöhet ihren Adel; \RWbet{sie ist eingeweiht in die Rathschlüsse Gottes} und die Leiterin seiner Werke.} Vergl.\ \RWbibel{Weish}{}{9}{4}; \RWbibel{Weish}{}{10}{11\,ff}
\item Durch Stellen, in welchen von einem \RWbet{Geiste in Gott} die Rede ist; \zB\ \Ahat{\RWbibel{2\,Sam}{2\,Sam.}{23}{2}}{2\,Kön.~23,2.}: \erganf{\RWbet{Der Geist Jehova's} spricht durch mich, auf meiner Zunge liegt sein Wort.}~\RWSeitenw{174}\ \RWbibel{Jes}{Jes.}{11}{2}: \erganf{Auf ihm wird ruhen \RWbet{Jehova's Geist}, der Geist der Weisheit und der Einsicht, der Geist des Rathes und der Stärke, der Geist der Kenntniß und der Gottesfurcht.} \RWbibel{Jes}{Isai.}{63}{10}: \erganf{Doch sie empörten sich und reizten \RWbet{seinen heiligen Geist}. -- Wo ist, der ihm seinen heiligen Geist verliehen?}
\item Endlich gibt es noch viele andere Stellen in den Büchern des A.\,B., die man von Gottes dreifacher Persönlichkeit auslegte, in Betreff deren es aber zweifelhaft ist, ob sie auf die Ausbildung dieser Lehre einen Einfluß gehabt, oder nicht, vielmehr nur auf sie gedeutet wurden, nachdem man diese Lehre bereits ausgebildet hatte. Von dieser Art ist \zB\ \RWbibel{Gen}{1 \,Mos.}{19}{24}\ \RWlat{Dominus Deus pluit a Domino Deo}. Von dieser Stelle wird erzählt, daß das \RWlat{Concilium Syrmiense} das Anathema über diejenigen aussprach, die hier nicht eine Anspielung auf Gott den Vater und den Sohn anerkennen wollten. Ingleichen \RWbibel{Ps}{Ps.}{33}{6}; \RWbibel{Ps}{Ps.}{45}{8}; \RWbibel{Ps}{Ps.}{2}{7}; \RWbibel{Mi}{Mich.}{8}{1} \uam\ 
\end{aufza}\par

\vabst II.~Auch einige \RWbet{heidnische Schriftsteller} dürften zur Entstehung und Ausbildung der christlichen Dreieinigkeitslehre etwas beigetragen haben. Besonders \RWbet{Pythagoras} und \RWbet{Plato}, die eine Art von Dreiheit (\RWgriech{tr'ias}) in Gott ausdrücklich annahmen. Plato unterschied \RWgriech{tre`is >arqik`as <upost'aseis} (drei uranfängliche Selbstheiten oder Personen), deren die erste er \RWgriech{m'onas} oder \RWgriech{<`en}, auch \RWgriech{pat`hr}, die zweite \RWgriech{no~us}, auch \RWgriech{l'ogos}, die dritte \RWgriech[y'uqh]{yuq'h}, auch \RWgriech{pne~uma} nannte. Nun ist es bekannt, daß viele Kirchenväter des zweiten und dritten Jahrhundertes Freunde der Neuplatonischen Philosophie gewesen.\par

\vabst III.~Ja selbst unter dem Volke Israel waren zur Zeit unseres Herrn verschiedene Begriffe im Umlauf, die der Entstehung, Ausbildung und Verbreitung dieser Lehre zu Statten kommen mußten. Die \RWbet{kabbalistischen} Juden nämlich glaubten, wie wir aus Philo und den Apokryphen des A.\,B.\ ersehen, verschiedene aus Gott selbst hervorgegangene \RWbet{Aeonen} oder Ausflüsse, wozu sie durch die chaldäische und pythagoräisch-platonische Philosophie veranlaßt worden sind. Man findet bei ihnen sogar den Ausdruck: \RWbet{Gott Vater, Gott}~\RWSeitenw{175}\ \RWbet{Sohn, Gott heiliger Geist: Drei in Einem und Eines in Dreien.} (\RWlat{S. Hugo Grot. Comm. in Joan. 1, 2.})

\RWpar{113}{b.~Jesu eigene Aeußerungen über Gottes dreifache Persönlichkeit, und zwar \greek{a}) nach den drei ersten Evangelien.}
Die katholische Kirche bescheidet sich (wie wir im ersten Hauptstücke gesehen) dahin, daß sie alle ihre übervernünftigen Lehren eigentlich \RWbet{aus dem Munde Jesu} erhalten habe, so, daß die Apostel und die späteren christlichen Lehrer kein anderes Geschäft gehabt hätten, als die von Jesu empfangenen Lehren mehr auszubilden, und wie neuaufkommende Begriffe genauere Bestimmungen derselben nothwendig oder nützlich machten, diese hinzuzufügen. Wenn also die Lehre von Gottes dreifacher Persönlichkeit eine echte Lehre der göttlichen Offenbarung ist; so muß sie (meint die Kirche) schon Jesus vorgetragen haben; und wenn er sie vorgetragen hat, so steht zu vermuthen, daß sich auch in den Evangelien, welche die wichtigsten Reden und Thaten Jesu erzählen, einige Spuren derselben finden werden. Das ist nun wirklich der Fall, doch mit dem Unterschiede, daß in den drei ersteren (\Ahat{den}{der} sogenannten \RWbet{harmonischen}) Evangelien des \RWbet{Matthäus, Markus} und \RWbet{Lukas} nicht eben so häufige Stellen, besonders die Gottheit des Sohnes betreffend, vorkommen, als in dem Evangelio \RWbet{Johannis}. Da nun behauptet worden ist, daß in den ersteren die \RWbet{Gottheit des Sohnes gar nicht zu finden ist:} so wollen wir ihr Zeugniß abgesondert von jenem des Johannischen Evangeliums vernehmen.
\begin{aufza}
\item Bei \RWbibel{Mt}{Matth.}{28}{19}\ spricht Jesus zu seinen Jüngern: \erganf{Gehet hin, und lehret alle Völker, und taufet sie im (auf den) \RWbet{Namen des Vaters, des Sohnes und des heiligen Geistes.}} Aus dieser Stelle läßt sich erstlich schließen, daß Jesus seinen Anhängern von einem gewissen Vater, Sohne und heil.\ Geiste ein \RWbet{Mehreres} beigebracht haben müsse, weil seine gegenwärtigen Worte sonst völlig unverständlich für sie gewesen, und die hier vorgeschriebene Taufe eine sinnlose Ceremonie geworden wäre. Ferner ergibt sich~\RWSeitenw{176}\ aus dieser Stelle, daß Jesus die Lehre vom Vater, Sohne und dem heil.\ Geiste für eine der \RWbet{wichtigsten} in seinem ganzen Lehrbegriffe angesehen haben müsse, weil er sonst nicht verordnet hätte, daß bei der Aufnahme jedes neuen Mitgliedes in die christliche Gesellschaft gerade dieser Lehre ausdrücklich erwähnt werde. Weiter ist es bekannt, daß die Redensart: Taufen --, \RWbet{auf oder in Jemandes Namen oder auf Jemand}, den Sinn habe, sich durch die Handlung der Taufe Jemandem verpflichten. (Vergl.\ \RWbibel{1\,Kor}{1\,Kor.}{10}{2}: \erganf{Ich möchte nicht, daß es bei euch in Vergessenheit käme, wie unsere Vorfahren alle durch die Wolke und das Meer \RWbet{auf Moses getauft} wurden.} Also werden die Christen durch die Taufe dem Dienste des Vaters, Sohnes und heil.\ Geistes verpflichtet. Erwägen wir nun, daß Jesus bekanntlich es als den Hauptzweck seines Daseyns angesehen habe, die Menschen vom Dienste dessen, was nicht Gott ist, zum Dienste und zur Verehrung des einigen und wahren Gottes zu bringen: so können wir nicht umhin, zu schließen, daß jener Vater, Sohn und Geist, auf dessen Namen die Christen getauft, \dh\ dessen Dienste sie gewidmet werden sollen, Niemand Anderer als \RWbet{Gott selbst} seyn müsse. Und so viel wissen wir auch aus andern Stellen, \zB\ aus der Gebetformel unseres Herrn (\RWbibel{Mt}{Matth.}{6}{9}) mit völliger Gewißheit, daß unter dem Vater allerdings nur Gott verstanden werde. Es früge sich also nur noch, wer Sohn und Geist seyen? -- Da Vater, Sohn und Geist hier \RWbet{in ganz gleichen Verhältnissen} angeführt werden, da der zu Taufende auf den Namen des Einen gleicher Weise, wie auf den Namen des Andern getauft, \dh\ seinem Dienste gewidmet werden soll: so geht hieraus mit vieler Wahrscheinlichkeit hervor, daß diese drei von \RWbet{gleicher Würde}, daß sie \RWbet{dem Wesen gleich,} oder gar \RWbet{einerlei} sind. Wären Sohn und Geist nicht Gott, sondern nur endliche Wesen: so wäre es doch wirklich unanständig, daß sie so ohne allen Unterschied hier neben dem unendlichen Vater gestellt, und daß wir Christen, wie dem Dienste des Vaters, so auch dem Dienste des Sohnes und Geistes geweihet werden sollen. Nun wissen wir aber hinlänglich, daß Jesus nur einen einzigen Gott gelehret habe, also müssen Vater, Sohn und Geist~\RWSeitenw{177}\ \RWbet{dem Wesen nach Eines,} und nur in einer andern Rücksicht etwas Mehrfaches und von einander Verschiedenes seyn. In welcher Rücksicht nun? Schon die Benennungen: Vater, Sohn und Geist, wenn sie nicht äußerst ungeschickt gewählt sind, deuten auf gewisse mit Verstand und Willen begabte, \dh\ \RWbet{persönliche Subjecte.} Dasselbe folgt auch aus der Redensart: \RWbet{Taufet auf ihren Namen;} denn ein Subject, \RWbet{auf dessen Namen} etwas geschehen soll, muß mit Verstand und Willen begabt, oder \RWbet{persönlich} seyn. So geht denn also aus dieser einzigen Stelle schon hervor, daß Vater, Sohn und Geist \RWbet{drei göttliche Personen in Einer Wesenheit} sind.
\item Da die Gottheit des Vaters niemals bezweifelt worden ist: so will ich nur noch einige Stellen anführen, welche die Gottheit des Sohnes und die des heil.\ Geistes noch etwas strenger beweisen. 
\begin{aufzb}
\item \RWbet{Des Sohnes.}
\begin{aufzc}
\item Bei \RWbibel{Mt}{Matth.}{22}{41\,ff}\ beweiset Jesus aus \RWbibel{Ps}{Ps.}{110}{1}, daß der Messias \RWbet{mehr, als ein bloßer Sohn Davids} sey. Nun hielt sich Jesus ohne Zweifel für einen wahren Menschen. Glaubte er also zugleich mehr als ein bloßer Mensch zu seyn: so geschah dieß nur, in wiefern er sich in der innigsten Verbindung mit dem Sohne dachte; der also etwas Göttliches seyn muß.
\item Bei \RWbibel{Mt}{Matth.}{11}{25\,ff}\ spricht Jesus: \erganf{Alles ist mir von meinem Vater übergeben worden. (Einem bloßen Menschen, ja auch nur sonst einem endlichen Wesen hätte Gott wohl nicht Alles übergeben dürfen); und Niemand kennt den Sohn, als der Vater, so wie Niemand den Vater kennt, als der Sohn, und wem der Sohn es offenbaren will.} Hier legt sich Jesus eine Kenntniß Gottes (des Vaters) bei, wie sie kein Anderer haben soll. Eine solche Kenntniß kann er nur durch eine besondere Unterstützung von Seite Gottes erlangt haben; und das in Gott, was sie ihm mitgetheilt hat, wird der Sohn genannt, der also einer höheren Natur als alle endlichen Wesen seyn muß.~\RWSeitenw{178}
\item Bei \RWbibel{Mt}{Matth.}{13}{24}\ erzählt er das Gleichniß vom Unkraut, und sagt in der Auslegung, der Acker sey die Welt, der Säemann oder Besitzer des Ackers der \RWbet{Sohn des Menschen,} die Schnitter aber die Engel. Er nennt sich also den \RWbet{Herrn der Welt}, und \RWbet{Engel seine Diener}.
\item Bei \RWbibel{Mt}{Matth.}{12}{8}\ sagt er: \erganf{Der Sohn des Menschen ist auch \RWbet{Herr über den Sabbath}}; und doch hielten die Juden den Sabbath für eine Einsetzung Gottes, die folglich nur Gott selbst wieder aufheben könne.
\item Eben so sagt er bei \RWbibel{Mt}{Matth.}{17}{24}, daß er nicht schuldig sey, die Tempelsteuer zu entrichten, weil er als \RWbet{Sohn vom Hause} steuerfrei wäre.
\item Bei \RWbibel{Mt}{Matth.}{21}{15}\ wendet er die Worte (\RWbibel{Ps}{Ps.}{8}{3}): \erganf{Aus dem Munde der Kinder und Säuglinge hast du dir Lob bereitet}, die sich daselbst auf Gott beziehen, auf jenes Hosianna an, womit man ihn selbst gepriesen hatte.
\item Bei \RWbibel{Mt}{Matth.}{21}{22}\ vergleicht er die Propheten mit Dienern, sich selbst mit dem \RWbet{Sohne des Königs.} (Vergl.\ \RWbibel{Lk}{Luk.}{13}{22})
\item Bei \RWbibel{Mt}{Matth.}{26}{63}\ erklärt er sich vor dem Hohenpriester für den \RWbet{Sohn Gottes}, der sich zur \RWbet{Rechten des Vaters} setzen, und einst auf den Wolken des Himmels wieder kommen werde. Diese Aeußerung erklärt der Hohepriester für eine Gotteslästerung; und fühlte also richtig, daß Jesus sich hier eine göttliche Würde beilege. Allein hat sich Jesus irgendwo eine göttliche Würde beigelegt: so kann er sich dieß nur in sofern erlaubt haben, als er sich mit Gott selbst innig verbunden fühlte. Er muß daher auch in Gott selbst Etwas, das ihn zu dieser innigen Verbindung mit Gott erhob, das ihn zu einem \RWbet{Sohne Gottes} machte, vorausgesetzt haben. Dieß Etwas nun, mag Jesus selbst es so, oder nicht so genannt haben, wird immer sehr schicklich von der Wirkung, die es hervorbrachte, der Sohn in Gott genannt.~\RWSeitenw{179}
\end{aufzc}
\item \RWbet{Des heiligen Geistes.}
\begin{aufzc}
\item Bei \RWbibel{Mt}{Matth.}{12}{31}\ sagt Jesus, daß jede Lästerung wider den Sohn des Menschen (worunter er sich selbst versteht) vergeben werden könne; kaum oder gar nicht aber eine Lästerung \RWbet{des heil.\ Geistes}. Hieraus erhellet, daß der Geist etwas Persönliches seyn müsse, wie der Sohn, weil man sich wider ihn durch Lästerungen versündigen kann; daß er aber an Würde gewiß dem Sohne nicht nachstehe, weil eine Lästerung desselben noch härter, als eine des Sohnes gestraft werden soll.
\item Bei \RWbibel{Mk}{Mark.}{12}{36}\ sagt Jesus, daß David durch Eingebung des \RWbet{heil.\ Geistes} gesprochen habe. Bekanntlich aber glaubten die Israeliten zu der Zeit Jesu, daß David und andere Propheten durch Gottes Eingebung gesprochen hätten. Jesus bestimmte also nur genauer dasjenige Subject in Gott, von welchem diese Eingebungen herrührten. \Usw\
\end{aufzc}
\end{aufzb}
\end{aufza}

\RWpar{114}{\greek{b})~Nach dem Evangelio Johannis}
Noch deutlichere Erklärungen Jesu über die Lehre von Gottes dreifacher Persönlichkeit dürfte das \RWbet{Evangelium Johannis} liefern.\par

\vabst A.~Wir wollen zuerst die Aeußerungen Jesu \RWbet{über seine eigene göttliche Würde} vernehmen, die dieses Evangelium in einer so zahlreichen Menge enthält, aus welchen aber freilich das Daseyn einer göttlichen Person, der wir den Namen eines Sohnes beizulegen hätten, nur mittelbar und durch den Schluß, der schon in dem vorhergehenden Paragraph gemacht worden ist, gefolgert werden kann. Daher ist es zugleich noch eine andere Lehre des Christenthums, nämlich die von der Menschwerdung des Sohnes, die sich aus diesen Stellen erweiset, wenn die von Gottes dreifacher Persönlichkeit einmal vorausgesetzt wird. Ich führe sie also jetzt auch aus dem Grunde an, um mich auf sie künftig berufen zu können.~\RWSeitenw{180}
\begin{aufza}
\item Bei \RWbibel{Joh}{Joh.}{3}{13}\ sagt Jesus zu dem Mitgliede des hohen Rathes Nikodemus: \erganf{Keiner war im Himmel, als der vom \RWbet{Himmel kam, der Sohn des Menschen}.} Hier legt sich Jesus ausschließlich vor allen andern Sterblichen einen \RWbet{Ursprung aus dem Himmel} bei. Dasselbe thut er auch \RWbibel{Joh}{}{6}{38}\ öffentlich in der Synagoge. Von seiner menschlichen Natur konnte er dieß begreiflich nicht behaupten.
\item In diesem Gespräche mit Nikodemus fährt Jesus (\RWbibel{Joh}{Joh.}{3}{18}) weiter fort: \erganf{Wer an ihn glaubt, der wird nicht verurtheilt; wer aber an ihn nicht glaubt, der ist schon verurtheilt, weil er nicht an den eingebornen Sohn Gottes (an denjenigen Sohn Gottes, der nicht seines Gleichen hat) glaubet.} Nun wird zwar nach orientalischem Sprachgebrauche der Name \RWbet{Sohn Gottes} zuweilen angewendet, um bloß einen sehr frommen und sehr mächtigen Mann zu bezeichnen; wie denn in den BB.~d.\ a.\,B.\ selbst \RWbet{Könige} und \RWbet{Obrigkeiten} zuweilen mit diesem Namen bezeichnet werden. (\RWbet{David}, auch sogar \RWbet{Cyrus} erhält ihn). Hier aber legt sich Jesus diesen Namen offenbar in einem viel höhern Sinne bei, in einem Sinne, in welchem er sonst keinem andern Menschen beigelegt worden ist. Dieß zeigt er durch das Beiwort \RWgriech{monogen`hs} an, welches entweder \RWbet{einzig in seiner Art}, oder \RWbet{eingeboren} (\RWgriech{m'onos gen'omenos} oder \RWgriech{m'onos genn'wmenos}) bedeutet; in jedem Falle aber anzeigt, daß Jesus Sohn Gottes in seinem Sinne sey, wie es keinen zweiten gibt.
\item Bei \RWbibel{Joh}{Joh.}{5}{17}\ entschuldigt sich Jesus darüber, daß er am Sabbathe gearbeitet, mit der Vergleichung: \erganf{Mein Vater wirket bis auf diese Stunde fort; und so wirke auch ich.} Welch eine kühne Vergleichung, wenn die Kraft, durch die Jesus wirkte, nicht Gottes eigene Kraft gewesen! Daher heißt es denn auch weiter, die Juden hätten ihn über diese Aeußerung tödten wollen, weil er so nicht nur den Sabbath aufgehoben, sondern sich Gott selbst gleichgesetzt habe. Jesus, der so bescheidene Jesus, läßt sich durch dieses Aergerniß, das man an seinen Worten nimmt, nicht stören, sondern fährt weiter fort, sich mit dem Vater zu vergleichen, und sogar gleiche Ehre für sich, wie für den Vater, zu verlangen: \erganf{Der~\RWSeitenw{181}\ Sohn kann nichts von sich selbst thun, wenn er es nicht den Vater thun sieht. Was dieser thut, das thut auf gleiche Weise der Sohn. Der Vater liebt den Sohn, und zeigt ihm Alles, was er thut, und wird ihm noch größere Werke zeigen, als diese, daß ihr darüber erstaunen werdet. Wie der Vater Todte erweckt, so belebt auch der Sohn, die er beleben will. Auch richtet der Vater Keinen; sondern er hat das Gericht dem Sohne übergeben, \RWbet{damit alle den Sohn ehren wie den Vater. Wer den Sohn nicht ehrt, der ehret auch den Vater nicht.}}
\item Bei einer andern Gelegenheit (\RWbibel{Joh}{Joh.}{8}{58}) sagte Jesus, daß er vor Abraham gewesen sey: \erganf{Ehe Abraham geboren wurde, war ich.}
\item Am Feste der Lauberhütten zu Jerusalem, als ihn das Volk umzingelte, um eine bestimmte Erklärung über seine Person zu erhalten, erwiederte Jesus, er habe es ihnen schon mehrmal gesagt, und schließt (\RWbibel{Joh}{Joh.}{10}{30}) mit dem Ausspruche: \erganf{\RWbet{Ich und der Vater sind Eins.}} Darüber wollten ihn die Juden steinigen, mit dem ausdrücklich angegebenen Grunde: \erganf{Weil du, ein Mensch, dich selbst zum Gotte machest.} Statt nun zu widerrufen, oder anzuzeigen, daß er sich keineswegs für Gott ausgebe, sagt Jesus: \erganf{Stehet in euerem Gesetze nicht geschrieben: Ich habe erklärt, daß ihr Götter seyet? (Dieß wird \RWbibel{Ps}{Psalm}{82}{6}\ zu den Richtern gesagt.) Wenn es nun Diejenigen Götter nennt, an welche der Ausspruch Gottes erging, und wenn die Schrift nicht verworfen werden kann: wie könnet ihr denn zu Dem, den der Vater geweihet und in die Welt gesandt hat, sagen: Du lästerst! weil ich sagte: \RWbet{Ich bin Gottes Sohn?} Wenn ich nicht die Werke meines Vaters thue: so glaubet mir nicht; thue ich sie aber, und ihr wollet mir nicht glauben: so glaubet doch den Werken, und überzeuget euch, \RWbet{daß der Vater in mir ist, und ich in dem Vater bin.}} Diese letztere Redensart, welche als eine Umschreibung von V.~30. anzusehen ist, sollte sie nicht beweisen, daß jene Einheit, von welcher dort die Rede war, nicht bloß (wie die Socinianer wollen) in der \RWbet{Einigkeit und Uebereinstimmung der Gesinnungen des Sohnes mit dem Willen des}~\RWSeitenw{182}\ \RWbet{Vaters}, sondern in der Einheit des Wesens bestehe? Sollte sie nicht eine Anspielung auf die \RWlat{circumincessio} der Kirche seyn?
\end{aufza}

\begin{RWanm} 
Wer etwa wünschen sollte, daß Jesus sich über seine göttliche Würde noch deutlicher ausgedrückt haben möchte, erwäge, daß Jesus seiner eigenen Bescheidenheit hätte zu nahe treten müssen, wenn er sich noch deutlicher hätte erklären sollen. Als ihm (\RWbibel{Mt}{Matth.}{19}{16\,ff}) ein Jüngling, der von seiner erhabenen Natur noch keinen Begriff hatte, also aus einer Art von Schmeichelei den Titel: vollkommener Meister (\RWgriech{did'askale >agaj`e})\editorischeanmerkung{%
	Das Attribut \RWgriech{>agaj`e} tritt nur in der Parallelstelle \RWbibel{Mk}{Mk}{10}{17\,ff}\ auf.} 
beilegte, verwies er ihm dieses. So würde er also gewiß auch bei den Gelegenheiten, welche wir jetzt betrachteten, nicht in so erhabenen Ausdrücken von sich gesprochen haben, hätte er sich nicht selbst überzeugt gehalten, daß er auf's Innigste mit Gott verbunden sey, und keinen Raub begehe, so fern er sich Gott gleich stellt. 
\end{RWanm}\par

\vabst B.~Auch über die Puncte in der Dreieinigkeitslehre liefern die in dem Evangelio Johannis aufbewahrten Reden des Herrn die deutlichsten Aufschlüsse. Ich führe nur die vorzüglichsten an.
\begin{aufza}
\item Der bloße Name der \RWbet{eingeborne Sohn Gottes} berechtiget uns schon zu der Redensart, daß der Sohn vom Vater \RWbet{gezeugt} sey, und weil Jesus bei andern Gelegenheiten in eben diesem Evangelio sagt, daß er vor Abraham's Geburt schon gewesen, und zwar im Himmel gewesen, und vom Vater geliebt worden sey, ehe noch die Welt gegründet wurde (\RWbibel{Joh}{Joh.}{17}{24}): so folgt hieraus die \RWbet{Ewigkeit des Sohnes,} oder wir können sagen, der Vater habe den Sohn \RWbet{von Ewigkeit her} gezeuget.
\item \RWbibel{Joh}{Joh.}{15}{26}\ verspricht Jesus seinen Jüngern den heil Geist, und sagt: \erganf{Wenn jener Tröster, den ich euch vom Vater senden werde, jener Geist der Wahrheit, der vom Vater ausgeht, senden werde: so wird er von mir Zeugniß geben; und auch ihr werdet von mir zeugen, weil ihr vom Anfange bei mir waret.} Der Name \RWgriech[parakl'htos]{par'aklhtos} (Tröster, Vormund), welchen der heil.\ Geist hier erhält, die Vergleichung des Zeugnisses, das er (durch seine Erleuchtung der Jünger und durch die übrigen Gaben) für die Wahrheit der Lehre Jesu eben so, wie die Jünger selbst, ablegen soll, lassen~\RWSeitenw{183}\ nichts Anderes, als ein \RWbet{persönliches Subject} in ihm vermuthen. -- Daß dieser Geist vom Vater \RWbet{ausgehe,} sagt diese Stelle ausdrücklich. Die Worte: \RWbet{Den ich euch senden werde}, berechtigen aber durch eine leichte Folgerung, auch einen \RWbet{Ausgang des heil.\ Geistes vom Sohne} anzunehmen, zumal da es \RWbibel{Joh}{Joh.}{16}{14}\ von diesem Geiste heißt: \erganf{Mich wird er verherrlichen; denn er wird von dem Meinigen nehmen, und es euch verkündigen.}
\item Daß dem \RWbet{Vater} vorzugsweise die \RWbet{Schöpfung}, dem \RWbet{Sohne die Erlösung} zuzuschreiben sey, beweiset die Stelle \RWbibel{Joh}{Joh.}{3}{16}: \erganf{So sehr hat Gott die Welt geliebt, daß er seinen eingebornen Sohn hingab}, -- wo unter \RWbet{Gott} eigentlich der \RWbet{Vater} zu verstehen ist, weil es im Gegensatze vom \RWbet{eingebornen Sohne} gebraucht wird. Und daß dem heil.\ Geiste die \RWbet{Erleuchtung, die Wirksamkeit der Sacramente} \usw\ zugeschrieben werden müsse, beweiset \zB\ \RWbibel{Joh}{Joh.}{16}{12}: \erganf{Noch Vieles hätte ich euch zu sagen; aber ihr könnet es jetzt nicht tragen. Wenn aber jener Geist der Wahrheit kommen wird: so wird er euch alle Wahrheit lehren.} Ingleichen \RWbibel{Joh}{Joh.}{3}{5}: \erganf{Wer nicht aus dem Wasser und aus dem heil.\ Geiste geboren ist, kann nicht in das Reich Gottes kommen. Was vom Fleische geboren wird, ist Fleisch; was aber vom Geiste geboren wird, ist Geist.}
\end{aufza}

\RWpar{115}{c.~Aeußerungen der ersten Schüler Jesu in den Büchern des n.\,B.\ über Gottes dreifache Persönlichkeit}
Deutlicher als Jesus selbst konnten sich seine Schüler über seine göttliche Würde sowohl, als auch über die ganze Lehre von Gottes dreifacher Persönlichkeit erklären. Sie nämlich mußten aus dem dreijährigen und so genauen Umgange, den sie mit ihrem Herrn gepflogen hatten, hinlänglich wissen, für wen er eigentlich zu halten sey und in welchem Sinne er diese oder jene Aeußerung wollte verstanden wissen; und von der andern Seite fiel auch so manche Rücksicht, welche ihn selbst gehindert haben konnte, sich gegen das Volk deutlicher zu erklären, bei ihnen weg. Er hielt alle seine Vorträge nur \RWbet{mündlich}, und sprach fast nur immer zu einer~\RWSeitenw{184}\ großen und \RWbet{gemischten Volksmenge.} Da hätte es sich gar nicht geziemt, die schwer zu verstehende Lehre von Gottes dreifacher Persönlichkeit oft zu berühren. Sie aber \RWbet{schrieben}, und konnten zum Theile auch auf \RWbet{gebildete} Leser rechnen. Und da Jesus bereits gegen Himmel aufgestiegen war: so war auch nicht mehr zu besorgen, daß die deutliche Anerkennung seiner göttlichen Würde gewisse allzu sinnliche Vorstellungen und thörichte Anwendungen erzeuge. Gesetzt also, daß Jesus sich auch selbst gegen sie über die Lehre von Gottes dreifacher Persönlichkeit und seiner eigenen göttlichen Würde nicht ganz ausführlich erklärt hätte: so würde sie doch der Geist der Wahrheit, den Jesus ihnen versprach, erst in der Folge auf ein Mehreres geleitet haben.\par
Wir wollen also die Begriffe, die sie ein Jeder gehabt zu haben scheinen, so viel wir es jetzt aus ihren Schriften zu beurtheilen vermögen, kennen lernen.\par
A.~Den Anfang mache \RWbet{Johannes der Täufer}, von dem wir zwar keine eigenen Aufsätze haben; dessen einige hieher gehörige Aeußerungen aber gleichwohl in den Evangelien aufbewahrt sind. Nach \RWbibel{Mt}{Matth.}{3}{11}\ sagte derselbe: \erganf{Ich taufe nur mit Wasser auf Buße; der aber nach mir auftreten wird, vermag mehr als ich. Ich bin nicht einmal würdig, seine Schuhriemen ihm aufzulösen (also hielt er Jesum für ungleich erhabener als sich). Dieser wird uns (also auch ihn selbst) \RWbet{mit heiligem Geiste} und Feuer taufen.} Und bei \RWbibel{Joh}{Joh.}{1}{15}: \erganf{Dieser ist es, von dem ich sagte: Der nach mir auftreten wird, ist vor mir gewesen; denn dieser \RWbet{war eher, als ich}.} Dieß Früherseyn Jesu kann sich nicht auf die menschliche Natur desselben beziehen, denn dieser nach war er jünger, als Johannes; also kannte der Täufer nebst der menschlichen Natur in Jesu noch eine andere, höhere.\par

\vabst B.~Bekannt ist der Ausruf des Apostels \RWbet{Thomas,} den uns das Evangelium (\RWbibel{Joh}{Joh.}{20}{28}) aufbewahrt hat: Mein Herr und mein Gott! Aller Wahrscheinlichkeit nach sind diese Ausdrücke Prädicate, die Thomas Jesu beilegt. Hier einen bloßen Ausdruck der Verwunderung, wie in der deutschen Formel: \RWbet{Gott, wer hätte dieß gedacht!} finden wollen, ist ganz gegen den hebräischen Sprachgebrauch. Be\RWSeitenw{185}ziehet sich aber: \RWbet{Mein Gott} auf Jesum; so hat Thomas eine göttliche Natur in Jesu anerkannt.\par

\vabst C.~\RWbet{Petrus}, der Vornehmste unter den Aposteln nennt Jesum (\RWbibel{Apg}{Apostelg.}{3}{15}) den Urheber des Lebens; er ist ihm (\RWbibel{Apg}{Apostelg.}{10}{36}\ und \RWbibel[42.]{Apg}{}{10}{42}) der Herr über Alles, der Richter der Lebendigen und der Todten. \RWbibel{Apg}{Apostelg.}{13}{33}\ wendet er die Worte \RWbibel{Ps}{Ps.}{2}{7}: \erganf{Du bist mein Sohn, heute habe ich dich gezeuget}, auf Jesum an, wobei es sich denn von selbst versteht, daß er das \RWbet{Heute} in der Bedeutung von \RWbet{Ewigkeit} nimmt. \RWbibel{Apg}{Apostelg.}{5}{3}\ spricht eben dieser Apostel zu Ananias: \erganf{Warum ließest du dein Herz vom Satan verleiten, den \RWbet{heil.\ Geist} zu belügen? Du hast nicht Menschen, sondern \RWbet{Gott} belogen.} Also war ihm der heil.\ Geist offenbar Gott. -- In seinen Briefen, und zwar \RWbibel{1\,Petr}{1\,Petr.}{1}{2}, schreibt Petrus: \erganf{Nach der Vorherwissenschaft Gottes, des \Ahat{\RWbet{Vaters},}{\RWbet{Vaters}} durch die Heiligung des \RWbet{Geistes}, zum Gehorsam und zur Reinigung im Blute \RWbet{Jesu Christi}.} Hier kommen also die drei göttlichen Personen in innigster Verbindung vor; und dem Vater werden die ewigen Rathschlüsse, dem heil.\ Geiste wird die Heiligung, dem Sohne die Erlösung zugeschrieben. Damit wir aber nicht glauben, daß Petrus den Sohn etwa für geringer gehalten; so lesen wir \RWbibel{2\,Petr}{2\,Pet.}{3}{18}: Wachset in der Gnade und Erkenntniß unsers Herrn und Heilandes Jesu Christi! Ihm sey Ehre jetzt und zu ewigen Zeiten. Mit einem solchen auf Gott gerichteten Lobspruche pflegte man in jener Zeit religiöse Vorträge und Briefe zu schließen. Da Petrus hier diesen Lobspruch auf \RWbet{Christum} anwendet: so folgt, daß er ihm \RWbet{göttliche} Würde beigelegt haben müsse. Vergl.\ auch \RWbibel{2\,Petr}{2\,Petr.}{1}{1}\par

\vabst D.~\RWbet{Johannes} der Lieblingsjünger Jesu erklärt sich über die Gottheit Jesu aus allen biblischen Schriftstellern am Deutlichsten.
\begin{aufzb}
\item Der Eingang seines Evangeliums, an dessen Echtheit nicht wohl gezweifelt werden kann, lautet folgender Maßen: \erganf{Im Anfange war das Wort, und dieses Wort war bei Gott, und es war selbst Gott dieses Wort.} Unter diesem \RWbet{Worte} (\RWlat{verbum}, \RWgriech{l'ogos}) kann nicht, wie Einige wollten, der \RWbet{heil.\ Geist} verstanden werden; denn dieser wird \RWbibel{Joh}{Joh.}{1}{32}\ \RWbibel{Joh}{Joh.~\RWSeitenw{186}\ }{14}{16}\ \RWbibel{Joh}{Joh.}{16}{14}\ \uam\ Orten vom \RWbet{Worte} (oder wie es Johannes in der Folge nennt) vom \RWbet{Sohne Gottes} ausdrücklich unterschieden. Auch hat \RWgriech{l'ogos} im griechisch-palästinischen Dialecte nie die Bedeutung: \RWbet{Vernunft, Weisheit}, weder in der alexandrinischen Uebersetzung, noch in den Apokryphen des a.\,B.; sondern immer nur die Bedeutung des \RWbet{Wortes}. Dieß Wort muß ein \RWlat{subjectum intelligens}, eine \RWbet{Person} seyn, weil es weiter das Licht genannt wird, Mensch wird \udgl\ Das: \RWbet{Im Anfange} kann nicht zu Anfang dieses Zeitraumes heißen; sondern: \RWbet{Vom Anbeginn her}, weil sonst Johannes sich selbst widerspräche, indem er gleich darauf sagt, daß Alles durch den Logos geschaffen sey, was immer geschaffen ist, und auch Jesum in der Folge sagen läßt, daß er \RWbet{vor der Welt-Schöpfung} schon gewesen sey. Der letzte Satz: \RWgriech{Je`os >~hn <o l'ogos} ist wohl kaum anders zu übersetzen, als: Dieser Logos war Gott; so daß \RWgriech{l'ogos} das Subject, und \RWgriech{je`os} das Prädicat ist. Im entgegengesetzten Falle müßte der Artikel \RWgriech{<o} vor \RWgriech{l'ogos} fehlen, und vor \RWgriech{Je`os} stehen. Wegen der großen Deutlichkeit dieser Stelle hat man sich viele Mühe gegeben, ihre Beweiskraft zu schwächen. \RWbet{Bahrdt} wollte diese drei ersten Verse für unterschoben erklären; \RWbet{Crell} wollte \RWgriech{Jeo~u} statt \RWgriech{Je`os} gelesen wissen; Andere einen Punct hinter \RWgriech{>~hn} gesetzt haben, und das \RWgriech{<o l'ogos} zum folgenden Verse ziehen. Alles vergebliche Bemühungen! -- Johannes fährt weiter fort: \erganf{Es (das Wort) war im Anfange bei Gott. Alles ist durch dasselbe erschaffen worden; und ohne dasselbe wurde kein Wesen, das geschaffen ist, erschaffen. In ihm war das \RWbet{Leben} und das Leben war das \RWbet{Licht} der Menschen. (Der Logos oder Sohn Gottes ist also der Urheber des \RWbet{Lebens}, \dh\ der Seligkeit der Menschen, er ist das \RWbet{Licht}, \dh\ der Urheber der Offenbarung.) Und das Wort ist Fleisch geworden (\dh\ wurde Mensch, nahm menschliche Natur an sich), und unter uns erschienen. Wir sahen seine Herrlichkeit, eine Herrlichkeit des \RWbet{Eingebornen des Vaters,} voll Gnade und Wahrheit.} Der Logos ist also dasselbe Subject, welches bei Johannes in der Folge und auch bei den übrigen Evangelisten \RWbet{Sohn Gottes} heißt. Dieser Logos muß \RWbet{einerlei Wesen mit Gott} haben, sonst gäbe es, da auch~\RWSeitenw{187}\ \RWbet{Er} Gott heißt, wenn er nicht derselbe Gott mit dem Vater wäre, zwei Götter, was doch Johannes gewiß nicht lehren wollte. --
\item In seinen Briefen, \zB\ \RWbibel{1\,Joh}{1\,Joh.}{1}{1}, schreibt er: \erganf{Was vom Anfange her war, was wir gehört, mit unseren Augen gesehen, genau beobachtet und mit unseren Händen berührt haben in Beziehung auf das \RWbet{Wort} (\RWgriech{l'ogos}) des Lebens (denn das Leben ist erschienen, wir haben es gesehen und sind seine Zeugen, und verkündigen euch das Leben, das ewige, das beim Vater war, und uns erschien), was wir gesehen und gehört haben, das verkündigen wir euch.} Die Gleichheit dieser Ausdrücke und Gedanken mit jenen im Anfange des Evangeliums sind ein Beweis, daß beide einerlei Verfasser haben. Und wenn auch der Zusammenhang in der gegenwärtigen Stelle etwas zerrissen ist; so sieht man doch deutlich, daß jener \RWgriech{l'ogos} nicht die bloße Weisheit, sondern eine eigene Person in Gott ist. Vergl.\ \RWbibel{1\,Joh}{1\,Joh.}{5}{20}; \RWbibel{1\,Joh}{}{5}{7}
\item In der diesem Apostel zugeschriebenen \Ahat{\RWbibel{Offb}{Offenbarung}{1}{17}}{1,18.}\ wird Jesus redend angeführt: \erganf{Ich bin der Erste und der Letzte}, ein Ausdruck, der nur von Gott allein gebraucht wird. Später, \RWbibel{Offb}{}{2}{23} wird von ihm gesagt, daß er Herzen und Nieren prüfe, was sonst von Gott gesagt wird. Und \RWbibel{Offb}{}{5}{12} singen die Engel dem Lamme, das der \RWgriech{l'ogos Jeo~u} ist, und nach \RWbibel{Offb}{}{3}{21} einerlei Thron mit Gott hat, Folgendes zu: \erganf{Würdig ist das Lamm (Jesus Christus), das geopfert wurde, zu empfangen Macht und Reichthum, Weisheit und Kraft, Ehre, Preis und Lob. Und \RWbet{alle Geschaffenen}, die im Himmel und auf Erden und unter der Erde und auf dem Meere und in demselben sind, hörte ich sagen: Dem, der auf dem Throne sitzt, dem Lamme, sey Lob, Ehre, Preis und Macht in alle Ewigkeit!} --
\end{aufzb}\par

\vabst E.~\RWbet{Paulus,} derjenige Apostel, dessen religiöse Begriffe uns am bekanntesten sind, weil wir von seiner Hand auch die meisten Aufsätze haben, glaubte:
\begin{aufza}
\item Daß es in Gott einen \RWbet{Vater }gibt, der \RWbet{von sich selbst von Ewigkeit} ist und dem die \RWbet{Schöpfung, Erhaltung und Regierung des Weltalls} vorzugsweise zuzuschreiben ist. Denn \RWbibel{1\,Kor}{1\,Kor.}{8}{6} schreibt er: \erganf{Wir haben~\RWSeitenw{188}\ Einen Gott, den \RWbet{Vater}, von dem alle Dinge, und in dem wir selbst sind.} Und \RWbibel{Hebr}{Hebr.}{3}{4}: \erganf{Jedes Haus hat einen Regierer; der \RWbet{Regierer aller Dinge} aber ist \RWbet{Gott}. Auch Moses stand seinem Hause treu vor; aber nur als Diener zum Zeugniß dessen, was vorgetragen werden sollte; Christus aber ist als Sohn seinem Hause vorgesetzt.}
\item Und einen \RWbet{Sohn, der vom Vater gezeugt ist von Ewigkeit}, und dem die \RWbet{Erlösung der Menschen} vorzugsweise zuzuschreiben ist, in dem sich auch die \RWbet{Weisheit Gottes besonders verherrlichet} hat. Denn
\begin{aufzc}
\item \RWbibel{Röm}{Röm.}{8}{3} schreibt er: \erganf{Gott sandte \RWbet{seinen Sohn} in der Gestalt des sündigen Fleisches (\dh\ der sündigen Menschheit), und wegen der Sünde, und strafte die im Fleische wohnende Sünde an ihm.} \RWbet{Der Sohn Gottes} muß also nicht ein bloßer Mensch, er muß etwas Höheres, \RWbet{Gott selbst}, seyn, weil er nur die Gestalt der Menschheit an sich genommen hat (nämlich indem er sich mit dem Menschen Jesu verband). \RWbibel{Röm}{Röm.}{9}{5} zählt es der Apostel unter die besondern Vorzüge der Israeliten, daß der Messias von ihnen abstammt, und nennt ihn bei dieser Gelegenheit mit ausdrücklichen Worten \RWbet{den über Alles hochgelobten Gott in Ewigkeit}: \erganf{Ihnen gehören die Väter, und von ihnen stammt dem Fleische nach Christus, welcher Gott über Alles ist, hochgelobet in Ewigkeit.} Zwar hat man diese Stelle verschiedentlich gelesen, und interpunctirt; allein die meisten Lesearten geben doch immer denselben Hauptgedanken, daß Christus hier für Gott erklärt wird, und schon der Beisatz, \RWbet{dem Fleische nach}, beweiset, daß Christus kein bloßer Mensch gewesen seyn müsse.
\item \RWbibel{Phil}{Philipp.}{2}{5}: \erganf{Nehmet eben die bescheidene Gesinnung an, welche auch Jesus Christus hatte, der, ob er gleich \RWbet{der Gottheit Abbild} war (oder göttliche Gestalt und Würde hatte) und ob er es gleich für keinen Raub hätte ansehen dürfen, sich \RWbet{Gott selbst gleich zu stellen,} nichts desto weniger sich erniedrigte, die Gestalt eines Knechtes annahm, den Menschen gleich ward, und \RWbet{äußerlich sich ganz als einen Menschen finden ließ,}}~\RWSeitenw{189}\ \usw\ So hätte Paulus nimmermehr sprechen können, würde er Jesum für einen bloßen Menschen oder den in ihm wohnenden Sohn Gottes nur für irgend ein endliches Wesen angesehen haben.
\item \RWbibel{Kol}{Koloss.}{1}{13}: \erganf{Er (der Vater) hat uns der Herrschaft der Finsterniß entrissen, und uns in das Reich \RWbet{seines geliebten Sohnes} versetzt, in welchem wir durch sein Blut die Erlösung erlangen, die Vergebung der Sünden. Dieser ist \RWbet{das Abbild des unsichtbaren Gottes, der Erstgeborne der ganzen Schöpfung}; denn durch ihn ist Alles erschaffen im Himmel und auf Erden, das Sichtbare und das Unsichtbare, selbst die Thronen, Herrschaften, Mächte, Gewalten; Alles ist durch ihn und in Beziehung auf ihn. Er war vor Allem, und Alles bestehet durch ihn.} Sehr gezwungen ist die Erklärung Einiger, die unter der Schöpfung, von welcher hier die Rede ist, eine \RWbet{moralische Umstaltung der Menschheit} durch die Lehre Jesu verstehen wollen; aber selbst dann noch beweiset diese Stelle, daß der Sohn Gottes mehr als ein bloßer Mensch, ja mehr als irgend ein endliches Wesen sey, und die Erlösung des menschlichen Geschlechtes bewirkt habe. In eben diesem Briefe \RWbibel{Kol}{}{2}{9}\ sagt der Apostel von Christo: \erganf{In ihm wohnt die Fülle der Gottheit körperlicher (\dh\ wirklicher) Weise.}
\item \RWbibel{1\,Thess}{1\,Thess.}{1}{1}: \erganf{Gnade euch und Friede von Gott, unserem Vater, und dem Herrn Jesu Christo.} Nur unter der Voraussetzung, daß Christus göttliche Würde besitzt, war es schicklich, von ihm eben so, wie von dem Vater Gutes zu wünschen.
\item \RWbibel{1\,Tim}{1\,Tim.}{3}{16}: \erganf{Gott, erschienen im Fleische, ward gerechtfertiget im Geiste, geschaut von den Engeln, verkündiget unter den Heiden, geglaubt in der Welt und aufgenommen zur Herrlichkeit.} Die Leseart dieser Stelle ist zwar sehr streitig, indem Einige statt \RWgriech{Je`os} (Gott) \RWgriech{<'os} (welcher) lesen. Eine der ältesten Handschriften, nämlich der \RWlat{Codex Alexandrinus}\RWlit{}{CodexAlexandrinus1}, hat nach Woide's genauerer Untersuchung wirklich \RWgriech[Jeo~s]{Je`os}; auch läßt sich leichter begreifen, wie bei der abgekürzten Schreibart $\overline{\textrm{\RWgriech{J}}}$\RWgriech{s} in eini\RWSeitenw{190}gen Handschriften aus \RWgriech{Je`os}, \RWgriech{<'os}, als umgekehrt aus \RWgriech{<'os}, \RWgriech{Je`os} habe werden können; auch ist der Ausdruck viel natürlicher, wenn man \RWgriech{Je`os} liest; denn unter dem \RWgriech{<'os} müßte doch immer Jesus Christus verstanden werden. Endlich paßt schon die Redensart: \erganf{Erschienen im Fleische}, auf keinen bloßen Menschen.
\item \RWbibel{Tit}{Tit.}{2}{10} empfiehlt der Apostel allen Ständen unter den Christen einen musterhaften Lebenswandel, damit sie der Lehre \RWbet{Gottes, unsers Erlösers}, allenthalben Ehre machen. Dieser Gott Erlöser ist wohl die zweite göttliche Person, oder derjenige, von dem Paulus an anderen Orten sagt, daß er im Fleische erschienen sey. Denn gleich weiter sagt er: \erganf{Die Gnade Gottes, des Erlösers, ist allen Menschen erschienen}, und hält uns ernstlich an, daß wir, der Gottlosigkeit und den Lüsten der Welt entsagend, sittsam, gerecht und gottesfürchtig in der Welt leben, harrend in seliger Hoffnung der \RWbet{Erscheinung der Herrlichkeit des großen Gottes und unseres Erlösers Jesu Christi.} Wohl müssen die Ausdrücke: \RWbet{Gottes}, und: \RWbet{Erlösers}, hier zusammengehören, Benennungen eines und eben desselben Subjectes seyn, weil das: \RWbet{Allen Menschen erschienen}, und: \RWbet{Erscheinung der Herrlichkeit}, nur auf Jesum bezogen werden kann, nämlich das Letztere auf seine zu erwartende \RWbet{Wiederkunft}.
\item Auch der Brief an die Hebräer, der Paulo zugeschrieben wird, enthält verschiedene Beweise dieses Glaubens. Er fängt mit den Worten an: \erganf{Nachdem Gott ehemals zu verschiedenen Zeiten und auf mannigfache Art zu den Vätern durch die Propheten geredet hatte, hat er in diesen letzten Tagen durch den \RWbet{Sohn} zu uns geredet, den er zum Erbherrn über Alles gemacht, \RWbet{durch den er die Welten erschaffen hat, und der, als Abglanz seiner Herrlichkeit und als Abbild seines Wesens, Alles durch sein mächtig Wort erhält.}} Hier erklärt Paulus den Sohn für den Urheber der Welt, und für denjenigen, der durch sein Machtwort alle Dinge erhalte. In der Folge vergleicht er ihn~\RWSeitenw{191}\ mit Moses und mit den Engeln, und erhebt ihn über Alle, so daß man deutlich sieht, er halte ihn für \RWbet{Gott selbst}, indem er Stellen des a.\,B., die von Jehova reden, geradezu auf den Messias anwendet. Man sehe vornehmlich \RWbibel{Hebr}{}{5}{15} und \RWbibel{Hebr}{}{2}{14}, wo der Messias den Menschen entgegengesetzt wird, als der vom Himmel Gekommene; ferner \RWbibel{Hebr}{}{4}{14}\ \RWbibel{Hebr}{}{1}{8--13}\ \RWbibel{Hebr}{}{7}{3}\ \RWbibel[26.]{Hebr}{}{7}{26} \uam\ 
\end{aufzc}
\item Und einen \RWbet{heiligen Geist, der vom Vater und Sohne ausgeht von Ewigkeit, göttliche Persönlichkeit besitzt, und seine Wirksamkeit auf Erden vornehmlich durch Erleuchtung und Besserung der Menschen äußert.} -- \Ahat{\RWbibel{Gal}{Galat.}{4}{6}}{7,6.}: \erganf{Gott hat den \RWbet{Geist seines Sohnes} in euere Herzen gesandt, welcher ruft: Abba, Vater!} Wenn also der Geist ein \RWbet{Geist des Sohnes} heißen darf: so wird man mit Recht behaupten können, daß er nicht nur vom Vater, sondern \RWbet{auch vom Sohne} ausgehe. \RWbibel{1\,Kor}{1.\,Kor.}{6}{19}: Wisset ihr nicht, daß euer Leib ein \RWbet{Tempel des heil.\ Geistes} ist, der in euch wohnet? Und \RWbibel{1\,Kor}{1\,Kor.}{3}{16}: \erganf{Wisset ihr nicht, daß ihr \RWbet{Tempel Gottes} seyd, und daß der \RWbet{Geist Gottes in euch wohne?}} Der heil.\ Geist ist also der Urheber der innern Regungen zum Guten, die wir fühlen, er hat einen Tempel in unserm Herzen, und Tempel des heil.\ Geistes oder Tempel Gottes werden als gleichgeltende Ausdrücke gebraucht. Muß also der heil.\ Geist nicht Gott seyn? \RWbibel{1\,Kor}{1\,Kor.}{2}{11}: \erganf{Wer weiß es, was in dem Menschen ist, als nur der Geist des Menschen, der in ihm ist? Eben so weiß auch Niemand, was in Gott ist, als nur der \RWbet{Geist Gottes.}} Eine sehr merkwürdige Vergleichung, die auf das Deutlichste beweiset, daß der Apostel dem Geiste Gottes Persönlichkeit beigelegt habe. Uebrigens muß man diese Vergleichung nicht allzuweit ausdehnen. \RWbibel{1\,Kor}{1\,Kor.}{12}{11}: \erganf{Dieses Alles wirket Ein und der nämliche Geist, der Jedem \RWbet{nach seinem Willen} Dieses oder Jenes mittheilt.} Also hat der heil.\ Geist auch einen Willen. \RWbibel{Eph}{Ephes.}{4}{30}: \erganf{Betrübet nicht Gottes heiligen Geist!} Also hat der heil.\ Geist auch ein \RWbet{Empfindungsvermögen}, und kann, bildlicher Weise zu reden, auch \RWbet{betrübet} werden \usw\ \usw~\RWSeitenw{192}
\end{aufza}

\RWpar{116}{d.~Beleuchtung der wichtigsten Bibelstellen, welche der Lehre von Gottes dreifacher Persönlichkeit zu widersprechen scheinen}

Aus dem Bisherigen erhellet, daß sich der wesentlichste Inhalt der Lehre von Gottes dreifacher Persönlichkeit allerdings schon in der Bibel vorfinde. Aber in eben dieser Bibel sollen sich auch, wie man uns einwendet, verschiedene Stellen finden, die dieser Lehre geradezu widersprechen. Wir wollen dieß untersuchen, theils
\begin{aufzb}
\item um desto gewisser zu werden, daß wir den Schriftstellern des n.\,B.\ in den obigen Stellen nicht vielleicht einen ihnen fremden Sinn unterschoben haben, wenn wir jetzt sehen werden, daß selbst diejenigen Stellen ihrer Schriften, in welchen sie etwas Anderes zu sagen scheinen, unserer Auslegung nicht widersprechen; theils auch
\item um uns zu überzeugen, daß die verschiedenen Theile der Bibel nicht der Eine mit dem Andern im Widerspruche stehen, welches dem Ansehen zuwider wäre, das man der Bibel in der christlichen Kirche einräumt.
\end{aufzb}\par
\RWbet{1.~Einwurf.} Es kommen Stellen in den Büchern des a.\ und n.\,B.\ vor, die jede Mehrheit in Gott als eine Irrlehre verwerfen; \zB\ \RWbibel{Dtn}{5\,Mos.}{4}{35}: \erganf{Das Alles habt ihr gesehen, auf daß ihr erkennet, \RWbet{Jehova nur ist Gott}, und sonst ist \RWbet{keiner} mehr.} Ingleichen \RWbibel{Joh}{Joh.}{17}{3}: \erganf{Dieß aber ist das ewige Leben, daß sie dich, den \RWbet{einigen wahren Gott}, und den du gesandt hast, Jesum Christum, erkennen.}\par
\RWbet{Antwort}. Diese Schriftstellen beweisen nichts Anderes, als daß Gott in einer gewissen Rücksicht seines Wesens nur Einer ist. Daß es aber in diesem einzigen Gott nicht in irgend einer andern Rücksicht, \zB\ in Rücksicht der Person etwas Mehrfaches geben könne, wird in denselben gar nicht geläugnet.\par
\RWbet{2.~Einwurf.} Aber es finden sich Stellen, wo Gott der \RWbet{Vater} im Gegensatz des Sohnes für den \RWbet{einzigen Gott} erklärt wird. Von der Art ist der zuletzt angeführte~\RWSeitenw{193}\ Ausspruch Jesu Christi (\RWbibel{Joh}{Joh.}{17}{3}): \erganf{\RWbet{Den einigen wahren Gott,} und den du \RWbet{gesandt hast, Jesum Christum.}} Wäre Christus eine Person in Gott: so könnte er unmöglich dem einigen wahren Gott so entgegenstellt werden. Einen noch stärkeren Gegensatz macht Paulus \RWbibel{1\,Tim}{1\,Tim.}{2}{5}: \erganf{Es ist Ein Gott und Ein Mittler zwischen Gott und den Menschen, der Mensch Christus Jesus.} Hier wird Christus im Gegensatze mit Gott ausdrücklich Mensch genannt. -- Eben so \RWbibel{1\,Kor}{1\,Kor.}{8}{5}: \erganf{Obgleich es dem Namen nach Götter gibt sowohl im Himmel als auf Erden: so haben wir doch nur \RWbet{Einen Gott, den Vater}, von welchem alle Dinge, in welchem auch wir sind; und \RWbet{Einen Herrn,} Jesum Christum.}\par
\RWbet{Antwort.} Nach der Lehre der katholischen Kirche war Jesus Christus ein \RWbet{wahrer Mensch,} mit dem der Sohn Gottes auf das Genaueste (nämlich zu einer einzigen Person) verbunden war. Als Mensch konnte er mit allem Rechte Gott dem Vater so entgegengesetzt werden, wie es in den obigen Stellen geschieht. Die letzte enthält vielmehr noch einen klaren Beweis für die Gottheit des Sohnes; denn es heißt weiter: \erganf{\RWbet{durch welchen alle Dinge, durch den auch wir selbst sind}.} So wie also vorhin vom Vater gesagt wurde, daß Alles \RWbet{aus} ihm sey, so wird jetzt vom Sohne gesagt, daß Alles \RWbet{durch} ihn sey, welches ungereimt wäre, wenn er nicht zu dem Wesen Gottes selbst gehörte.\par
\RWbet{3.~Einwurf.} Es kommen Stellen vor, in welchen von Jesu Christo so Manches gesagt wird, was sich mit einer göttlichen Person auf keine Weise vereinigen läßt. So unterliegt Jesus
\begin{aufzb}
\item verschiedenen Gemüthsbewegungen, \zB\ dem Zorne, der Furcht, er zittert vor seinem Tode, er ruft am Kreuze: \erganf{Mein Gott! mein Gott! warum hast du mich verlassen!} --
\item Er betet. Wie kann wohl Gott zu Gott beten? --
\item Er gesteht selbst ein, daß er nicht allwissend sey; \zB\ \RWbibel{Mk}{Mark.}{13}{32}: \erganf{Jenen Tag und jene Stunde weiß Niemand, nicht die Engel im Himmel, \RWbet{auch nicht der Sohn,} sondern nur der Vater.}
\item Er kann nicht frei und unabhängig thun, was er will. \RWbibel{Joh}{Joh.}{5}{19}: \erganf{Der Sohn kann \RWbet{nichts von sich selbst}~\RWSeitenw{194}\ \RWbet{thun}, wenn er es nicht den Vater thun sieht.} Und \RWbibel{Mt}{Matth.}{20}{23}: \erganf{Das Sitzen zu meiner Rechten und zu meiner Linken zu verleihen, \RWbet{steht nicht bei mir;} außer wem es von meinem Vater bestimmt ist.}
\item Er wird sich dereinst wieder dem Vater unterwerfen. \RWbibel{1\,Kor}{1\,Kor.}{15}{28}: \erganf{Bis ihm (dem Sohne) Alles wird unterworfen seyn, dann wird auch selbst der Sohn sich dem unterwerfen, der ihm Alles unterworfen hat, so daß Gott Alles in Allem seyn wird.} Nach der Lehre des katholischen Christenthums ist der Sohn allwissend, allmächtig, unabhängig, und bleibt dieß Alles in Ewigkeit.
\end{aufzb}\par
\RWbet{Antwort.}
\begin{aufzb}
\item Da Jesus, wie schon gesagt, ein wahrer Mensch gewesen, so konnte und sollte er allerdings auch \RWbet{Gemüthsbewegungen} haben; nur dürfte er sich von denselben nicht auf eine solche Art beherrschen lassen, die mit der sittlichen Vollkommenheit eines Menschen streitet. Dieses aber ist in jener Schilderung, die uns die Evangelien von Jesu geben, niemals der Fall. Nach dieser zürnt Jesus wohl, aber er läßt sich durch seinen Zorn zu keiner Unbesonnenheit verleiten; er ängstiget sich vor seinem Tode, aber er spricht und thut nicht das Geringste, was eines vollkommenen Menschen unwürdig wäre.
\item Wenn Jesus nie \RWbet{gebetet} hätte, so wäre er kein guter Mensch gewesen, und man könnte mit Recht seine innige Verbindung mit Gott bestreiten.
\item In späteren Zeiten ward freilich der Sprachgebrauch allgemein festgesetzt, die Worte \RWlat{Filius} und \RWlat{Verbum} nur von der höhern Natur in Jesu Christo, nur von der zweiten göttlichen Person zu gebrauchen; dagegen, so oft man die mit der göttlichen vereinigte menschliche Natur bezeichnen wollte, lieber die Ausdrücke Christus, Messias, Sohn des Menschen, und wenn man insbesondere die menschliche Natur allein bezeichnen wollte, den Namen Jesus zu wählen. Dieser Sprachgebrauch war aber zu den Zeiten, als die Bücher des n.\,B.\ geschrieben wurden, noch nicht so bestimmt; sondern damals bediente man sich des Wortes: \RWbet{Sohn Gottes} zuweilen auch, wenn~\RWSeitenw{195}\ man den mit der Gottheit vereinigten Menschen bezeichnen wollte. Von Jesu selbst ist es vollends zu erwarten, daß er dieses Wort meistentheils nur in dieser concreten Bedeutung genommen, \dh\, daß er darunter nur sich selbst, den Menschen, den seine Zuhörer vor sich sahen, diesen mit Gott (dem Sohne) vereinigten Menschen verstanden haben werde. Und unter dieser Voraussetzung verschwindet alles Bedenkliche in jener Stelle. Wenn nämlich Jesus unter dem Sohne sich selbst, den Menschen verstanden hat; so konnte und durfte er allerdings in Rücksicht auf seine menschliche Natur gestehen, daß er dieß oder jenes nicht wisse; und bleibt es nicht immer merkwürdig, daß er sich selbst in dieser Stelle noch über die Engel erhebt?
\item Eine gleiche Auslegung haben auch die Stellen, in denen Jesus seine Abhängigkeit vom Vater bekennt. \erganf{Ich (der Mensch) muß mich in allen Stücken nach dem Willen meines Vaters fügen; ich darf in keinem Stücke, am Allerwenigsten, wo es sich um die Besetzung der wichtigsten Aemter im Messiasreiche handelt, nach meiner bloßen Neigung, sondern nach dem muß ich verfahren, was meines Vaters Rathschluß ist.}
\item Auch die Stelle, in welcher der Apostel die einstige Unterwerfung des Sohnes unter den Vater behauptet, ist mit der Lehre von Gottes dreifacher Persönlichkeit sehr wohl vereinbar. Die drei göttlichen Personen stehen nämlich, wie die katholische Kirche lehret, ihrer Göttlichkeit unbeschadet, in einem solchen Verhältnisse untereinander, daß eine Bestimmbarkeit der Einen durch die Andere allerdings Statt finden kann. In der erwähnten Stelle will Paulus nun nichts Anderes, als den Gedanken ausdrücken: Gott werde alle Ereignisse in dieser Welt so lange zu Gunsten des Christenthums leiten, bis dieses endlich die allgemeine Religion auf Erden geworden seyn wird, dann erst werde jene Auferstehung der Todten, die Einige schon zu seiner Zeit erwarteten, eintreten. Da nun die Leitung der Schicksale in dieser Welt nach christlichen Begriffen dem Vater zugeschrieben wird; die Ausbreitung~\RWSeitenw{196}\ des Christenthumes aber eine Verherrlichung des Sohnes ist: so sagt der heil.\ Paulus, der Vater werde dem Sohne alle seine Feinde, und zuletzt den Tod unterwerfen. Dann werde vollendet seyn das Geschäft des Sohnes als des Erlösers und Mittlers, und eben deßhalb werde es von jetzo an nicht mehr scheinen, als ob der Vater Alles nur zu Gunsten und zur Verherrlichung des Sohnes wirke; sondern es werde nun Alles unmittelbar zur Verherrlichung des Vaters selbst geschehen; es werde sich jetzt erst zeigen, wie die Verbreitung des Christenthums selbst keinen andern Zweck gehabt habe, als die Verherrlichung des Vaters, \dh\ der Sohn werde sich dem Vater unterwerfen, und so werde von nun an der Vater seyn Alles in Allem.
\end{aufzb}\par
\RWbet{4.~Einwurf.} Bei \RWbibel{Joh}{Joh.}{14}{28}\ sagt Jesus mit trockenen Worten: \erganf{Der Vater ist größer als ich.} Wenn diese Aeußerung nicht die Gottheit des Sohnes aufhebt: so widerspricht sie wenigstens der Lehre, daß die drei göttlichen Personen von gleichem Range seyen.\par
\RWbet{Antwort.} Die Gottheit des Sohnes wird durch diese Stelle in keinem Falle aufgehoben; indem er immerhin eine göttliche Person, ein unendliches Subject bleiben, und doch geringer als der Vater seyn könnte, weil es nicht ungereimt ist, daß Ein Unendliches das Andere übertreffe. Da aber Jesus hier unter dem \RWbet{Ich} offenbar sich \RWlat{in concreto} versteht: so widerspricht diese Stelle auch nicht der Lehre vom gleichen Range der göttlichen Personen, denn der Sohn Gottes in seiner Erniedrigung als Mensch war allerdings geringer, denn der Vater.\par
\RWbet{5.~Einwurf.} Es kommen Stellen vor, in welchen dem heil.\ Geiste die Ewigkeit abgesprochen wird. So heißt es \RWbibel{Joh}{Joh.}{7}{39}: \erganf{Denn damals war noch kein heiliger Geist, weil Jesus noch nicht zur Herrlichkeit erhoben war.}\par
\RWbet{Antwort.} Offenbar ist dieser Ausdruck des Evangelisten eine den Schriftstellern des n.\,B.\ sehr gewöhnliche Abkürzung. Der heilige Geist steht hier statt: \RWbet{Gaben des heil.\ Geistes}, wie bei \Ahat{\RWbibel{Joh}{Joh.}{20}{22}}{20,23.}: \erganf{Empfanget den heiligen Geist}, \uam\  -- Sollte der heil.\ Geist sein Daseyn~\RWSeitenw{197}\ erst durch die Himmelfahrt Jesu erhalten haben: wie könnte gesagt werden, daß er schon bei der Geburt, Taufe Jesu \usw\ zugegen und wirksam gewesen sey, daß er schon durch den Mund der Propheten gesprochen habe? \usw\


\RWpar{117}{e.~Aeußerungen der Kirchenväter aus den drei ersten Jahrhunderten über die Lehre von Gottes dreifacher Persönlichkeit}
Man hat behauptet, daß die Lehre von Gottes dreifacher Persönlichkeit erst im vierten und fünften Jahrhunderte, vornehmlich durch das Bestreben, die Lehrsätze der neuplatonischen Philosophie mit jenen des Christenthums zu verbinden, in den christlichen Lehrbegriff hineingetragen worden sey. Während der drei ersten Jahrhunderte dagegen habe gar keine Uebereinstimmung über diesen Gegenstand unter den Christen geherrscht, und wohl noch Niemand habe sich damals diejenigen Begriffe von Gottes dreifacher Persönlichkeit gebildet, die später zu dem allein richtigen und seligmachenden Glauben erhoben worden sind. Die aus dem Judenthume bekehrten Christen, sagt man, dachten wohl bei dem Ausdrucke \RWbet{Sohn Gottes} nach ihrem Sprachgebrauche an nichts, als einen \RWbet{Liebling Gottes}, den lang erwarteten Messias; die Heidenchristen aber werden sich, nach der Analogie ihrer frühern Begriffe von Götterzeugungen, unter dem Sohne Gottes irgend einen erst in der Zeit entstandenen Gott vorgestellt haben. Die zahlreichen Secten der Gnostiker dachten, vermöge ihrer Emanationstheorie, bei den Worten: \RWbet{Sohn Gottes}, und: \RWbet{heiliger Geist}, gewiß nur an Aeonen, \dh\ an gewisse höhere Geister, die (noch vor der übrigen Schöpfung) aus der göttlichen Substanz (dem Urlichte) ausgeflossen seyen. -- Die platonisirenden Kirchenväter, \zB\ \RWbet{Justin der Martyrer, Athenagoras, Tatian, Origenes}, dachten sich unter dem \RWgriech{l'ogos} die göttliche Vernunft, die von Ewigkeit in Gott vorhanden, bei der Schöpfung aber (namentlich durch das Sprechen) aus Gott hervorgegangen und selbstständig geworden sey, ohne daß jedoch der Vater vernunftlos geworden wäre, so wenig, als ein Mensch durch Sprechen oder Mittheilen seiner Gedanken selbst vernunftlos wird. Andere,~\RWSeitenw{198}\ die weder dem Gnosticismus noch Platonismus anhingen, \zB\ \RWbet{Noetus, Praxeas, Sabellius aus Pentapolis, Paul von Samosata,} Bischof zu Antiochien,\RWbet{ Beryllus} zu Bostra in Arabien (alle aus dem zweiten Jahrhunderte), sahen Vater, Sohn und Geist nur als drei verschiedene Namen, Verhältnisse oder Wirkungsarten eines und eben desselben göttlichen Wesens an. Auch diejenigen, die sich der später aufgekommenen orthodoxen Lehre noch am Meisten genähert, \zB\ \RWbet{Tertullian}, dachten sich Sohn und Geist als abhängig vom Vater.\par
Hiegegen bemerke ich nun:
\begin{aufza}
\item Wenn sich nur zeigen läßt, daß die katholische Lehre von Gottes dreifacher Persönlichkeit, wie sie jetzt vorgetragen wird, vernünftig und zuträglich sey, und zwar zuträglicher, als jede andere Ansicht: so kommen ihr auch schon die beiden Kennzeichen\RWfootnote{%
Ihre Bestätigung durch Wunder erweiset sich dann sehr leicht; denn schon ihre Entstehung ist ein eigentliches Wunder.}
einer göttlichen Offenbarung zu, und sie verdient, die einzig richtige und einzig seligmachende zu heißen, gleichviel, ob sie im ersten oder im vierten Jahrhunderte durch den Geist Gottes allgemein bekannt geworden. Ist dieses Letztere wirklich der Fall gewesen: so geschah es ohne Zweifel nur, weil eine allgemeine Bekanntschaft mit dieser Lehre in jenen früheren Jahrhunderten noch keinen besonderen Nutzen gehabt haben würde, und weil der Geist Gottes gewollt hat, daß die Menschen nicht ohne ihr eigenes Nachdenken und ohne einen durch viele Jahrhunderte auf diesen Gegenstand verwandten Fleiß zur Kenntniß der Wahrheit gelangen, und eben deßhalb auch ihren Werth einst um so höher schätzen lernen sollten.
\item Doch es ist falsch, daß die Begriffe, welche sich die Christen der drei ersten Jahrhunderte von Gottes dreifacher Persönlichkeit gemacht, von den jetzt herrschenden so beträchtlich abweichen, als man uns glauben machen will. Man findet in den Schriften der Kirchenväter aus diesen drei Jahrhunderten verschiedene einzelne Stellen, welche mit unserm jetzigen Glauben genau genug zusammenstimmen. Z.\,B.~\RWSeitenw{199}
\end{aufza}\par

\vabst A.~\RWbet{Aus dem ersten Jahrhunderte.}
\begin{aufza}
\item \RWbet{Ignatius} (Bischof von Antiochien) nennt in den sieben echten Briefen, die wir noch von ihm übrig haben, \RWbet{Christum} mehrmals mit ausdrücklichen Worten \RWbet{Gott}. Z.\,B.\ im Briefe an die Römer: \erganf{Christus unser Gott ist im Vater (\RWlat{circumincessio})}. In seinem Briefe an die Epheser: \erganf{\RWlat{Unus est medicus, carnalis et spiritualis, factus et non factus, in carne Deus, in morte vita vera, et ex Maria et ex Deo}}. Im Briefe an die Magnesier: \erganf{Sorget, daß ihr in den Lehren des Herrn und seiner Apostel immer mehr befestiget werdet, auf daß euch Alles gelinge, was ihr immer unternehmet, in dem \RWbet{Sohne}, dem \RWbet{Vater} und dem \RWbet{heil.\ Geiste.}}
\item \RWbet{Clemens} (Bischof von Rom) schreibt an die Korinther\RWlit{}{ClemensvonRom1}: \erganf{Brüder! ihr müsset \RWbet{Jesum Christum als Gott} und Richter der Lebendigen und der Todten betrachten.}
\item \RWbet{Polycarpus} (Bischof zu Smyrna) ermahnt in seinem echten Briefe an die Philippenser die Gemeine, den Aeltesten, wie \RWbet{Gott und Christo} unterthan zu seyn.
\end{aufza}

\vabst B.~\RWbet{Aus dem zweiten Jahrhunderte.}
\begin{aufza}
\item \RWbet{Justin der Märtyrer} beantwortet in seiner ersten Apologie (\RWlat{ad Antoninum})\RWlit{}{Justinus3} den Einwurf, daß die Christen Atheisten wären: \RWlat{Fatemur quidem, nos talium, qui habentur, deorum esse expertes et atheos; sed non verissimi illius Dei, \RWbet{Patris} videlicet et \RWbet{Filii} et \RWbet{Spiritus sancti}.} In seiner zweiten Apologie schreibt er: \erganf{Auch den \RWbet{Sohn} und den \RWbet{heiligen prophetischen Geist} ehren und \RWbet{beten wir an}, und beten doch nur \RWbet{Einen Gott} an.} U.\,m.\,a.\ Stellen.
\item \RWbet{Athenagoras}: \erganf{\RWlat{Deum asserimus, tria quidem secundum potentiam, \RWbet{Patrem}, \RWbet{Filium} et \RWbet{Spiritum sanctum}; actu vero et essentia unum}.} Und \RWlat{in legat. pro christianis\RWlit{}{Athenagoras1} p.\,10.}: \erganf{\RWlat{Cum sit unum \RWbet{Pater} et \RWbet{Filius}, et sit in \RWbet{Patre Filius} et \RWbet{Pater in Filio} unitate et virtute \RWbet{Spiritus sancti}, mens et verbum Dei Filius est}.}~\RWSeitenw{200}
\item \RWbet{Irenäus} \RWlat{adv.\ haeres.\ lib.\,3.\ c.\,23.}:\RWlit{}{Irenaeus1} \erganf{Der \RWbet{Sohn Gottes, welcher Gott ist.}} \RWlat{Lib.\,4.\ c.\,20.}: \RWbet{Bei ihm (dem \RWbet{Vater}) ist das \RWbet{Wort} und die Weisheit der \RWbet{Sohn} und der \RWbet{Geist}, durch welche und in welchen er Alles frei und ungezwungen geschaffen hat.}
\item \RWbet{Clemens} (Bischof von Alexandrien) bedient sich zuerst unter den Griechen des Wortes Dreieinigkeit. (\RWlat{Strom., lib.\,5 pag. \,598.})\RWlit{}{Clemens2} In der \RWlat{admonitio ad gentes} (im Anfange) heißt es: \erganf{\RWlat{\RWbet{Unus} quidem est universorum \RWbet{pater}, unum etiam \RWbet{verbum} universorum, et \RWbet{Spiritus sanctus} unus, qui et ipse est et ubique.}}
\item \RWbet{Tertullian} ist unter den lateinischen Kirchenvätern der Erste, der das Wort \RWlat{Trinitas} braucht; er erklärt und vertheidiget dieses Geheimniß in einem eigenen Buche \RWlat{advers.\ Praxeam.}\RWlit{}{Tertullian3} Da heißt es unter Anderem: \erganf{\RWlat{\RWbet{Trinitatem unius Divinitatis, Patrem, Filium et Spiritum sanctum.}}} -- \RWlat{c.\,2.: \erganf{Tres autem sunt, non statu, sed gradu; nec substantia, sed forma; nec potestate, sed specie: unius autem substantiae et unius status et unius potestatis; quia unus Deus, ex quo gradus isti, et formae et species in nomine \RWbet{Patris} et \RWbet{Filii} et \RWbet{Spiritus sancti} deputantur.}} -- \erganf{\RWlat{\RWbet{Filium} non aliunde deduco, sed de substantia \RWbet{Patris}, \RWbet{Spiritum} non aliunde puto, quam \RWbet{a Patre per Filium}.}} -- \erganf{\RWlat{\RWbet{Inseparati tamen ab alterutro}, etsi dicatur, \RWbet{alium} esse Patrem, \RWbet{alium} Filium, \RWbet{alium} Spiritum sanctum;}} \usw\
\end{aufza}\par

\vabst C.~\RWbet{Aus dem dritten Jahrhunderte}.
\begin{aufza}
\item \RWbet{Origines}, den man in so vielen Stücken in dem Verdachte der Heterodoxie hat, schreibt doch in seinen acht Büchern wider den Celsus, die echt und unverfälscht auf uns gekommen sind: \RWlat{\erganf{Sciant illi criminatores, hunc \RWbet{Deum}, quem \RWbet{ab initio Deum}, \RWbet{Deique Filium} esse credimus, ipsumque esse Verbum, ipsam veritatem, ipsam sapientiam} (lib.\,3.)}. Ueberhaupt lehrt er in diesem Werke, daß Sohn und Geist gleich ewig mit dem Vater wären, und daß dieser nicht ohne jene gedacht werden könne, wie Licht nicht ohne Glanz und Wärme. Nur darin fehlt er, daß er den~\RWSeitenw{201}\ Sohn und Geist \RWbet{Ausflüsse, Prädicate}, ja selbst \RWbet{Geschöpfe} nennt. Aber weil er gleich selbst hinzusetzt, daß sie Geschöpfe \RWbet{in einem ganz anderen Sinne}, als alle übrigen, wären: so ist im Grunde hier nur der \RWbet{Ausdruck} verfehlt. Weniger dagegen ist er zu entschuldigen, wenn er den Sohn und Geist für \RWbet{geringer} hält als den Vater; wozu ihn, wie auch einige Andere (\zB\ Tertullian, und selbst den Bischof Alexander von Alexandrien) die Stelle \RWbibel{Joh}{Joh.}{14}{29}\ verleitete.
\item Der \RWbet{heil.\ Cyprian} (Bischof von Karthago) schreibt in seinem 73sten Briefe\RWlit{}{Cyprianus2}: \erganf{Da jene \RWbet{Drei Eines} sind, wie kann der \RWbet{heil.\ Geist} versöhnet seyn mit dem, der des \RWbet{Vaters} und des \RWbet{Sohnes} Feind ist?}
\item \RWbet{Dionysius} von Rom (Bischof daselbst) schreibt in einem Briefe, den wir bei Athanasius lesen: \RWbet{\RWlat{Audio quosdam apud vos esse catechistas et doctores divini verbi, qui illius sunt opinionis auctores, quae e diametro, ut ita dicam, pugnat cum Sabellii placitis. Hic enim blasphemat, Filium ipsum asserens esse Patrem, et e converso Patrem esse Filium. At isti tres quodammodo Deos praedicant, dum sanctam Monadem dividunt in tres hypostases peregrinas a se invicem plane separatas. Etenim necesse est, \RWbet{unire} omnino \RWbet{Deo Dei verbum}, et in Deo manere et habitare \RWbet{Spiritum sanctum}.}}
\end{aufza}

\begin{RWanm} 
Selbst aus den nicht christlichen Schriftstellern ließen sich verschiedene Beweise anführen, daß die Lehre von Gottes dreifacher Persönlichkeit schon in den ersten drei Jahrhunderten unter den Christen geherrscht habe. So kommt \zB\ in dem Gespräche Philopatris, welches dem \RWbet{Lucian} zugeschrieben wird, und dann aus dem ersten Jahrhunderte seyn müßte, in der That aber ungefähr erst unter dem Kaiser Julian, \dh\ im vierten Jahrhunderte, erschien, folgende Stelle vor: \erganf{\RWlat{Ethnicus. Quemnam igitur tibi jurabo?} (Bei welchem Gotte soll ich dir also schwören?) \RWlat{Tryphon. Deum alte regnantem, magnum, immortalem, coelestem, Filium Patris, Spiritum ex Patre procedentem, unum ex tribus, et ex uno tria,} (\RWgriech{<`en >ek tri~wn, ka`i >ex <en`os tr'ia}).}\RWlit{331}{Lukian1} \RWlat{Ethn. Non intelligo, quid dicas: unum tria, tria unum. (cap.\,12.)}~\RWSeitenw{202}
\end{RWanm} 

\RWpar{118}{f.~Kurze Geschichte der Lehre von Gottes dreifacher Persönlichkeit vom vierten Jahrhunderte bis auf unsere Zeiten}

\begin{aufza}
\item Die ersten Christen, Lehrer sowohl als Schüler, waren, wo nicht ganz ungelehrte Leute, doch wenigstens nicht speculative Philosophen; und schon aus diesem Grunde geschah es, daß man die Lehrsätze des Christenthums anfangs nur in den einfachsten, aus der Sprache des gemeinen Lebens entlehnten Ausdrücken, ohne ängstliche Nebenbestimmungen, und ohne systematische Verbindung vortrug. In der Folge der Zeiten, als auch unter den Christen mehr Gelehrsamkeit, und insbesondere speculative Philosophie (vornehmlich neuplatonische) emporkam, die Christen auch schon etwas mehr Ruhe genoßen, und ihren Unterricht in ordentlichen Schulen ertheilen konnten, war es nicht, wie so viele Gelehrte neuerer Zeit behaupten, ein Unglück, sondern vielmehr etwas sehr Löbliches, daß man diese Gelehrsamkeit auch auf den Vortrag der Religion anwandte. Da wurde es denn Bedürfniß, so Manches, was man sich bisher nur dunkel gedacht hatte, zu einem deutlichen Bewußtseyn zu erheben, so Manches, was bisher noch unbenannt geblieben war, mit einem eigenen Worte zu bezeichnen, so Manches, was bisher noch schwankend und unbestimmt gewesen war, genauer zu bestimmen. Nichts war natürlicher, als daß man sich bei diesem Geschäfte nicht über Alles gleich ganz vereinigen konnte, daß oft der Einen Partei dieses, einer andern jenes Wort passender zur Bezeichnung eines gewissen noch unbenannten Begriffes schien, daß sich der Eine durch diese, ein Anderer durch jene Nebenbestimmung der bisher noch unbestimmt gebliebenen Wahrheit mehr zu nähern glaubte. So mußte es denn auch in der Lehre von Gottes dreifacher Persönlichkeit geschehen, zumal da diese Lehre eine der schwersten ist. Daher ist es nicht zu wundern, daß die Verschiedenheit der Begriffe, die sich die einzelnen Lehrer der Kirche in Betreff dieses Gegenstandes machten, in den drei ersten Jahrhunderten, wo man noch keine allgemeine Zusammenkünfte und Berathschlagungen anstellen konnte, so weit ging, als wir es in einem der vorhergehenden §§.\ gesehen.~\RWSeitenw{203}
\item Vom Anfange des vierten Jahrhundertes, seitdem die christliche Religion durch den Kaiser Constantin das Recht der öffentlichen Ausübung und den Schutz des Staates erhalten hatte, seitdem man mehrere Zusammentretungen der christlichen Lehrer, sogar aus allen Gegenden des Reiches, veranstalten konnte, erfuhr auch die Lehre von Gottes dreifacher Persönlichkeit ein günstigeres Schicksal, und man vereinigte sich, je länger je mehr über die Art und Weise, wie man sich dieselbe vorzustellen habe.
\item Daß es nur einen einzigen Gott, in diesem aber doch etwas Dreifaches gebe, nämlich den Vater, Sohn und Geist, das hatte man freilich schon von Anbeginn des Christenthums gelehrt. Nun aber, da man diese Lehre wissenschaftlich vortragen wollte, ward es Bedürfniß, jene besondere Rücksicht, in welcher Gott nur einfach, und jene andere, in welcher er dreifach ist, nämlich Vater, Sohn und Geist, mit eigenen Namen zu bezeichnen. In der lateinischen Kirche wählte man zur Bezeichnung dessen, was in Gott einfach ist, die Worte: \RWlat{natura, essentia, divinitas, substantia;} dasjenige aber, in Betreff dessen Gott dreifach ist, nannte man \RWlat{persona}, zuweilen auch \RWlat{suppositum}; ja Einige nannten auch die Personen \RWlat{substantias, subsistentias;} dann aber setzten sie diese Worte der \RWlat{essentia} entgegen. Die Lehrer der griechischen Kirche bezeichneten, was in Gott einfach ist, mit den Worten \RWgriech[o>~usia]{o>us'ia}, \RWgriech{f'usis}, zuweilen auch \RWgriech{<up'ostasis}; und zur Bezeichnung des Dreifachen in Gott wählten sie die Worte \RWgriech{pr'oswpon}, \RWgriech{<up'ostasis}, manche auch wohl \RWgriech{f'usis}. An die beiden Benennungen \RWgriech{<up'ostasis} und \RWgriech{f'usis} als Zeichen des Dreifachen in Gott, stießen sich nun die Lateiner, weil \RWgriech{<up'ostasis}, etymologisch übersetzt, \RWlat{substantia} heißt, und ließen sich erst spät das Wort \RWgriech{<up'ostasis} gefallen, wenn es dem Worte \RWgriech[o>~usia]{o>us'ia} (Wesen) entgegengesetzt wird.
\item Als man die Frage aufwarf, was man denn eigentlich unter einer göttlichen Person zu verstehen habe: wurde hierüber (schon im zweiten und dritten Jahrhunderte) mit ziemlicher Allgemeinheit gegen \RWbet{Praxeas, Sabellius} \uA\ entschieden, daß eine göttliche Person nicht ein bloßer Name, auch nicht ein bloßes Verhältniß oder eine Wirkungsart Gottes sey.~\RWSeitenw{204}
\item Im Anfange des vierten Jahrhundertes behauptete \RWbet{Arius}, Presbyter zu Alexandrien, der Sohn sey nicht von Ewigkeit her gezeugt aus dem Wesen des Vaters, sondern nur früher als alle übrigen Dinge (\RWgriech{proqr'onwn}, \RWgriech[a>i'wniwn]{a>iwn'iwn}) erschaffen aus Nichts (\RWgriech{>ex o>uk >'ontwn}) und folglich nicht Gott, sondern ein Geschöpf, obgleich das vollkommenste aus allen, durch welches Gott auch alle übrigen geschaffen hat; der heil.\ Geist sey aber aus dem Sohne gezeugt. Ihm pflichteten Mehrere, unter Anderen selbst Bischöfe, \zB\ \RWbet{Eusebius von Nikomedien} bei. Nachdem sich der Bischof von Alexandrien, \RWbet{Alexander}, vergeblich bemüht, den Arius und seine Anhänger zurecht zu bringen, wurde vom Kaiser Constantin ein allgemeiner Kirchenrath zu Nicäa in Bithynien ausgeschrieben. Auf diesem zeigte sich nun sehr deutlich, wie überwiegend die Anzahl derjenigen Christen gewesen sey, die weit erhabenere Begriffe vom Sohne Gottes hatten. Aus dreihundert und achtzehn (freilich meistens nur aus dem Morgenlande versammelten) Bischöfen nahmen nur siebzehn einen Anstand, das von Hosius oder Athanasius verfaßte Glaubensbekenntniß (das sogenannte nicenische) zu unterschreiben, und dieß nur, weil es ihrer Meinung nach einer andern Irrlehre, nämlich jener des Sabellius nicht deutlich genug zu widersprechen schien. Zuletzt bequemten sich von diesen noch fünfzehn zur Unterschrift, so daß nur zwei nicht unterschrieben. In diesem Glaubensbekenntnisse heißt es nun, der Sohn sey \RWbet{gezeugt} und \RWbet{nicht geschaffen} (\Ahat{\RWgriech{gennhj'enta}}{\RWgriech{gen'hjeis}}, \Ahat{\RWgriech{o>u poihj'enta}}{\RWgriech{o>u poi'hjeis}}) und zwar gezeugt \RWbet{aus dem Wesen des Vaters} (\RWgriech{>ek t~hs o>us'ias to~u patr'os}), er sey Gott aus Gott, Licht vom Lichte, wahrer Gott vom wahren Gotte (\RWgriech{Je`os >ek Jeo~u, f~ws >ek fwt'os, Je`os >alhjin`os >ek Jeo~u >alhjino~u}) und \RWbet{einerlei Wesens mit dem Vater} (\RWgriech[<omoo'usios t~w p'atri]{<omoo'usios t~w| patr'i}). Dieß Wort \RWgriech{<omoo'usios}, dessen Begriff die lateinische Kirche noch deutlicher durch das Wort \RWlat{consubstantialis} ausdrückte, ward von nun an als das Kennzeichen der Rechtgläubigkeit angesehen, indem die Arianer behaupteten, der Sohn sey \RWgriech{<eteroo'usios}, \dh\ von \RWbet{anderem Wesen}, oder höchstens (wie dieß die Semiarianer thaten) er sey \RWgriech{<omoio'usios}, \dh\ \RWbet{ähnlichen Wesens} mit dem Vater. Ob aber Jeder, der das Wort~\RWSeitenw{205}\ \RWgriech{<omoo'usios} annahm, auch dabei dachte, daß der Sohn einerlei Wesens mit dem Vater sey; oder ob nicht vielmehr Einige sich vorgestellt, daß der Sohn ein von dem Vater (numerisch) verschiedenes, aber doch gleiches Wesen habe, darüber ließe sich freilich noch streiten. -- Die irrigen Begriffe, die Arius auch über den heil.\ Geist hegte, ließ das Concilium ungerügt; vermuthlich, weil jener erste Irrthum über den Sohn der vorherrschende war.
\item Unter dem Kaiser \RWbet{Constantius}, Constantin's Sohn und Nachfolger, der selbst ein Arianer war, ingleichen unter \RWbet{Julian} dem Abtrünnigen wurde die Partei der Arianer abermals verstärkt. Sie hielten verschiedene (Provincial-) Concilien, in welchen sie ihre Meinung geltend zu machen suchten. Wie wenig Einigkeit aber unter ihnen geherrscht habe, beweiset schon der Umstand, daß sie in einem Zeitraume von zwanzig Jahren nichts weniger als eilf Glaubensbekenntnisse entwarfen.
\item So wurden nur zu Syrmium in Illyrien allein drei Kirchenversammlungen gehalten, deren Beschlüsse mehr oder weniger zu Gunsten des Arius ausfielen; und die Beschlüsse der Einen (man weiß nicht mehr recht, welcher) wurden selbst von dem römischen Bischofe Liborius unterschrieben.
\item Und als man, um diese Streitigkeiten einmal zu endigen, auch in Italien, in der Stadt \RWbet{Rimini}, einen Kirchenrath von 400 Bischöfen versammelte, unterschrieben auch diese ein arianisch lautendes Glaubensbekenntniß, so daß selbst \RWbet{Hieronymus} von diesem Zeitpuncte ausruft: \RWlat{Totus orbis ingemuit, et se Arianum esse miratus est!} Allein diese Erscheinung verliert ihr Auffallendes, wenn man die näheren Umstände derselben kennen lernt. Im Anfange waren die zu Rimini versammelten Bischöfe in zwei Parteien getheilt. Die Eine wollte kein neues Glaubensbekenntnis abgefaßt wissen, weil das Nicäische genüge, verdammte die Arianer, und schickte mit diesem Beschlusse Abgeordnete an den Kaiser Constantius. Die Arianer aber thaten das Nämliche, und ihre Abgeordneten erreichten das kaiserliche Hoflager früher, und nahmen den Kaiser für ihre Sache so ein, daß er die Abgeordneten der andern Partei gar nicht vor sich~\RWSeitenw{206}\ kommen ließ, sondern sie nach Mycene in Thracien zu bringen befahl, wo man ihnen eine jener zu Rimini bereits entworfenen ganz ähnliche Glaubensformel vorlegte, die sie aus Menschenfurcht unterschrieben. Hierauf kehrten sie nach Rimini zurück, und durch ihr Beispiel und auf Befehl des kaiserlichen Ministers bequemten sich am Ende auch die Meisten der hier versammelten Bischöfe zu unterschreiben, zumal, da ihnen, welche nicht griechisch verstanden, eine gelindere Auslegung von der Bedeutung des \RWgriech{<omoio'usios} gemacht wurde. Hieraus sieht man denn deutlich, daß es theils Zwang, theils auch Bethörung war, was diese Bischöfe zur Unterschrift vermochte; wie sie denn auch, sobald sie wieder in Freiheit gesetzt waren, und den Betrug erkannten, insgesammt widerriefen.
\item Eine neue Bestimmung erhielt die Lehre von Gottes dreifacher Persönlichkeit aus Veranlassung der Ketzerei des \RWbet{Photinus}. Dieser gelehrte Bischof von Syrmium trat in der Mitte des vierten Jahrhundertes mit der Behauptung auf, daß der Sohn (\RWgriech{l'ogos}) nichts als der Verstand oder die Weisheit, der heil.\ Geist nichts als eine gewisse Kraft Gottes sey. Zu gleicher Zeit mit ihm bestritten auch \RWbet{Macedonius}, Bischof von Constantinopel, und seine Anhänger (die man Pneumatomachen nennt) die Gottheit des heil.\ Geistes; indem ihn Einige für ein Geschöpf (\RWgriech{kt'isma}) und einen Diener (\RWgriech[di'akonon ka`i <up'hrethn]{di'akonon ka`i <uphr'ethn}) Gottes, Andere für eine bloße Kraft in Gott erklärten. Zur Widerlegung dieser Irrlehren wurde im Jahre 381 der allgemeine Kirchenrath zu \RWbet{Constantinopel} gehalten, in welchem das nicenische Glaubensbekenntniß in dem Artikel vom Sohne den Zusatz: Gezeugt von Ewigkeit; und im Artikel vom heil.\ Geiste den Zusatz: Und (ich glaube) an den heil.\ Geist, den Herrn, den belebenden (\RWgriech{zwopoi`on}), der vom Vater ausgehet, mit dem Vater und dem Sohne zugleich angebetet und verherrlichet wird -- erhielt.
\item Gleichwohl dauerte die Partei der Arianer sowohl, als jene der Pneumatomachen noch immer fort. Nebst den im ganzen römischen Reiche zerstreuten Arianern gab es noch ganze Völker, als die Sueven, Burgunder, Gothen, Vandalen und Longobarden, welche dem Arianismus huldigten;
so wie~\RWSeitenw{207}\ die Pneumatomachen sich durch ganz Thracien, Bithynien und in den Provinzen des Hellespontus verbreiteten.
\item Da die lateinischen Kirchenväter ein Ausgehen des heil.\ Geistes auch von dem Sohne lehrten, welches in dem nicenischen Glaubensbekenntnisse nicht ausdrücklich bemerkt war, so setzte man zunächst (wie es scheint) in Spanien und Italien zu der Formel: \RWlat{qui ex patre procedit}, das Wörtchen \RWlat{filioque} hinzu. Hierüber entstand seit 660 ein noch jetzt nicht beendigter Streit zwischen der lateinischen und griechischen Kirche, indem die letztere das Ausgehen des heil.\ Geistes vom Sohne nicht zugeben will.
\item Den größten Antheil an dem Siege, welchen die orthodoxe Lehre über die Ketzerei des Arius davon trug, hatte der alle Leiden standhaft ertragende Bischof von Alexandrien \RWbet{Athanasius}; dessen mit großer Bestimmtheit vorgetragene Begriffe auch zur Ausbildung der übrigen Puncte in der Lehre von Gottes dreifacher Persönlichkeit sehr Vieles beitrugen. Dazu kam noch das ihm zwar unterschobene, aber doch in seinem Geiste verfaßte \RWlat{Symbolum Athanasianum}\RWlit{}{SymbolumAthanasianum}, das etwa im fünften Jahrhunderte erschien. In diesem \RWlat{Symbolo}, welches bei der katholischen Kirche einen solchen Beifall fand, daß sie dasselbe in die für alle Geistlichen vorgeschriebenen Gebete und Betrachtungen, oder das sogenannte \RWlat{Breviarium}\RWlit{}{Breviarium}, aufnahm, wird ausdrücklich gelehrt, daß die drei göttlichen Personen nicht nur gleiches, sondern \RWbet{einerlei} Wesens sind. Hier heißt es: \RWlat{Pater est Deus, Filius est Deus, Spiritus sanctus est Deus; et tamen non sunt tres Dii, sed unus est Deus}. \Usw\
\item Nun hörten allmählich die Streitigkeiten über die Lehre von der Dreieinigkeit auf, und es erregte keine große Bewegung, als im siebenten Jahrhunderte der Syrer \RWbet{Askunages} und sein Schüler \RWbet{Philiponus} sich des \RWbet{Tritheismus} schuldig machten, indem sie jeder Person eine eigene Substanz und Gottheit beilegten (\RWgriech{merik`h o>us'ia}, \RWgriech[>id'ia je'oths]{>'idia je'oths}); eben so wenig, als im eilften Jahrhunderte der Nominalist \RWbet{Roscellinus} behauptete, man müsse entweder sagen, auch Vater und Geist seyen mit dem Sohne Mensch geworden, oder man müsse die drei Personen für drei verschie\RWSeitenw{208}dene Substanzen, oder nur für drei verschiedene Namen halten. Im zwölften Jahrhunderte wurden\RWbet{ Abälard, Gilbert} (\RWlat{Porretanus}) und \RWbet{Joachim von Flora} irriger Begriffe in dieser Lehre beschuldigt, ohne jedoch Anhänger zu finden.
\item Erst um die Zeit der Reformation (im sechszehnten Jahrhunderte) traten eine Menge Feinde der kirchlichen Trinitätslehre auf: \RWbet{Ludwig Hetzer, Johann Denk, Johann Campanus} aus Jülich, \RWbet{Claudius} von Savoyen, \RWbet{Michael Servetus} ein Arzt aus Villanova in Arragonien, \RWbet{Velesinus, Gentilis} ein Neapolitaner, \RWbet{Matthäus Gribaldus} ein Rechtsgelehrter von Pavia, \umA\  Man begreift sie unter dem Namen der \RWbet{Antitrinitarier} oder \RWbet{Unitarier}. Da sie von Katholiken sowohl als auch von Protestanten verfolgt, die meisten sogar durch das Schwert hingerichtet wurden: so flüchteten sie sich nach Polen, wo damals die meiste Denkfreiheit herrschte, und errichteten daselbst in mehreren Gegenden Gemeinden. Zu Krakau erschien ihr erster Katechismus. Noch weiter ging die von den Italienern \RWbet{Lälius} und \RWbet{Faustus Sozzini} gestiftete und gleichfalls größtentheils in Polen und Siebenbürgen verbreitete Partei der \RWbet{Socinianer}, die alle Geheimnißlehren des Christenthums auf eine Art zu erklären suchten, bei der ihre Wahrheit durch die Vernunft selbst eingesehen werden könnte (Krakauer Katechismus). Dieses System der Socinianer wurde unter den Deutschen von \RWbet{Soner}, \RWbet{Daniel}, \RWbet{Bahrdt}, unter den Engländern von \RWbet{Biddle, Emlyn, Priestley} verfochten; wie es denn auch von mehreren Gelehrten der neuesten Zeit, besonders in protestantischen Ländern, mit mehr oder weniger Modificationen angenommen wird.
\end{aufza}

\RWpar{119}{Vernunftmäßigkeit dieser Lehre}
Um die Lehre von Gottes dreifacher Persönlichkeit von allem Verdachte der Vernunftwidrigkeit zu retten, wird nichts Anderes nothwendig seyn, als daß wir dasjenige in ihr, was die Kirche in \RWbet{buchstäblicher Bedeutung} nimmt, von demjenigen unterscheiden, wovon sie selbst lehret, daß es nur \RWbet{bildlich} zu verstehen wäre.~\RWSeitenw{209}\par

\vabst A.~Nach seiner buchstäblichen Bedeutung will die Kirche in dieser Lehre Folgendes verstanden wissen:
\begin{aufza}
\item Daß sich in dem einigen Wesen Gottes von \RWbet{Ewigkeit} und \RWbet{nothwendiger Weise etwas Dreifaches} befinde.
\item Daß dieses Dreifache in Gott nicht bloß ein dreifaches Verhältniß zu der Welt, eine dreifache Wirkungsart, und überhaupt \RWbet{nichts uns völlig Begreifliches} sey.
\item Daß von jenem Dreifachen in Gott das \RWbet{Zweite} auf eine gewisse, uns \RWbet{unerklärbare Art von Ewigkeit her durch das Erste}, -- das \RWbet{Dritte} aber auf eine uns gleichfalls \RWbet{unbegreifliche}, aber von jener wieder \RWbet{unterschiedene} Art durch das Erste und Zweite in seinem Daseyn bestimmt werde.
\item Daß dieses Dreifache in Gott, \RWbet{jedes auf eine eigenthümliche Art sich wirksam in der Welt beweise}; daß sich das \RWbet{Erste} in Gott vornehmlich bei der \RWbet{Welt-Schöpfung} und \RWbet{Regierung}; das \RWbet{Zweite} vornehmlich \RWbet{in der Person Jesu Christi}; das \RWbet{Dritte} vornehmlich bei der \RWbet{Erleuchtung und Heiligung der Menschen} wirksam bewiesen habe und noch beweise.
\item Daß \RWbet{Jedem} aus diesen Dreien in Gott \RWbet{Allmacht, Weisheit, Heiligkeit}, und \RWbet{alle} die sogenannten \RWbet{natürlichen Prädicate der Gottheit beigelegt werden} dürfen.
\item Daß endlich diese Drei \RWbet{kein getrenntes Daseyn außerhalb einander} haben, sondern nur \RWbet{in und durch einander} bestehen.
\end{aufza}\par

\vabst B.~Dagegen ist, selbst nach dem Geständnisse der Kirche, Folgendes in dieser Lehre nur \RWbet{bildlich} zu verstehen:
\begin{aufza}
\item Die Rücksicht selbst, in welcher es ein Dreifaches in Gott gibt, wird mit dem bildlichen Worte \RWbet{Persönlichkeit} (\RWgriech{<up'ostasis}, \RWgriech{pr'oswpon} \ua ) bezeichnet.
\item Die erste dieser drei Personen in Gott erhält den bildlichen Namen \RWbet{Vater}; die zweite den Namen \RWbet{Sohn}, auch \RWbet{Wort}, \RWbet{Vernunft} und \RWbet{Weisheit}; die dritte den Namen \RWbet{heil.\ Geist,} auch \RWbet{Liebe}.~\RWSeitenw{210}
\item Die Art, wie der \RWbet{Sohn durch den Vater} in seinem Daseyn bestimmt wird, vergleicht man bildlicher Weise mit einem \RWbet{Erzeugen aus seinem Wesen}, welches man dem Schaffen aus Nichts entgegensetzt.
\item Die Art, wie der \RWbet{heil.\ Geist durch den Vater und Sohn} in seinem Daseyn bestimmt wird, heißt bildlicher Weise ein \RWbet{Ausgehen} oder \RWbet{Gesendetwerden}.
\end{aufza}\par
Lasset uns nun die Vernunftmäßigkeit aller dieser Puncte im Einzelnen betrachten.

\RWpar{120}{1.~Daß in Gott etwas Dreifaches sey}
Der erste Punct in dieser Lehre, daß in dem einigen Wesen Gottes \RWbet{von Ewigkeit her und nothwendiger Weise etwas Dreifaches} vorhanden sey, enthält nichts Widersprechendes.\par
Zwar hat man zweierlei eingewendet.\par
\RWbet{1.~Einwurf.} Es ist der aufgelegteste Widerspruch, den man nur lehren kann, daß Gott einfach und dreifach zugleich sey. Dieß hat schon jener Heide in dem Gespräche, Philopatris betitelt, gerüget: \RWlat{Non intelligo, quid dicas: unum tria, tria unum.}\par
\RWbet{Antwort.} Wir lehren, daß Gott in Rücksicht seines \RWbet{Wesens} einfach; dreifach aber in irgend einer \RWbet{andern Rücksicht,} nämlich in Rücksicht der Person sey. Das ist nun aber gar nicht widersprechend, daß ein Gegenstand in einer gewissen Rücksicht einfach, in einer andern mehrfach sey. Ein Widerspruch wäre es nur, wenn es hieße, daß Gott in eben der Rücksicht, in der er einfach ist, auch wieder mehrfach sey.\par
\RWbet{2.~Einwurf.} Die Vernunft erkennet, daß Gott in jeder Rücksicht das allereinfachste Wesen seyn müsse, sie läßt nicht zu, daß man in Gott irgend eine Zusammensetzung annehme, sondern man soll sich ihn als eine absolute Einheit (\RWgriech{mon`as}) denken. Die christliche Kirche widerspricht nun dieser Vernunftwahrheit, indem sie das Wesen Gottes aus drei Personen zusammensetzt.~\RWSeitenw{211}\par
\RWbet{Antwort.} Die Vernunft erkennt, daß Gott in Rücksicht seines \RWbet{Wesens} absolut einfach seyn müsse, \dh\, daß man ihn schlechterdings nicht als zusammengesetzt aus mehreren Substanzen denken dürfe. Allein daß man auch sonst in keiner andern Rücksicht eine Vielfachheit in Gott annehmen dürfe, lehrt die Vernunft uns nicht. Im Gegentheile, sie selbst nimmt in der natürlichen Religion mehrerlei Kräfte in Gott an. Unter andern vornehmlich diese drei: Denkkraft, Gefühl und Wollkraft, und diese drei Kräfte sind ebenfalls von Ewigkeit her und nothwendiger Weise in Gott vorhanden.

\RWpar{121}{2.~Was dieses Dreifache in Gott sey}
\begin{aufza}
\item Wenn die christliche Kirche von jenem Dreifachen in Gott behauptet, daß es \RWbet{von unserem endlichen Verstande nicht ganz begriffen} werden könne, daß wir uns keine erschöpfende Vorstellung von diesem Dreifachen, \RWbet{wie es an sich ist}, bilden können: so behauptet sie etwas, das der Vernunft nicht nur nicht widerspricht, sondern das diese vielmehr als eine nothwendige Folge von ihrer eigenen Endlichkeit eingestehen muß. Gott ist ein unendliches Wesen; wir können uns daher von keiner seiner Eigenschaften, also auch nicht von den Eigenschaften desjenigen, was in ihm dreifach ist, eine erschöpfende Vorstellung bilden.
\item Wenn aber insbesondere behauptet wird, daß dieses Dreifache in Gott auf keinen bloßen Verhältnissen Gottes zur Welt beruhe, auch keine bloßen verschiedenen Wirkungsarten Gottes in dieser Welt bezeichne; so ist dieß eine Folge von der Behauptung, daß dieses Dreifache ewig und nothwendig ist. (Nr.\,1. \RWparnr{119}) Das Gegentheil wäre ein Widerspruch. Wenn das Dreifache in Gott nichts Anderes ist, als eine dreifache Beziehung Gottes zur Welt: so kann dieß Dreifache nicht mit Nothwendigkeit in Gott vorhanden heißen, sondern es setzt zu seinem Daseyn das Daseyn dieser Welt voraus. So wie nun diese zufällig heißt: so müßte auch dieses Dreifache zufällig heißen.
\item So kann das Zweite in Gott auch nicht, wie Einige gemeint, Gottes Verstand seyn; denn, wie wir in der Folge~\RWSeitenw{212}\ sehen werden, so hat das Zweite in Gott wohl allerdings Verstand; aber es ist nicht der bloße Verstand, denn es hat auch Willen.
\end{aufza}

\RWpar{122}{3.~Die zweite und dritte Person in Gott werden in ihrem Daseyn durch eine andere bestimmt}
Da wir uns keine erschöpfende Vorstellung von Gott, wie er an sich ist, bilden können, von jenem Dreifachen in ihm vollends nur so viel wissen, als uns die Offenbarung davon bekannt gemacht hat: so können wir auch nichts dawider einwenden, wenn sie uns sagt: Das Zweite von jenen Dreien werde in seinem Daseyn durch das Erste auf eine gewisse Art bestimmt, das Dritte auf eine andere Art durch das Erste und Zweite.\par
\RWbet{Einwurf}. Diese Behauptung widerspricht der Nothwendigkeit sowohl, als auch der Ewigkeit des Daseyns, welche das Zweite und Dritte gerade so, wie das Erste, haben sollen. (\RWparnr{119}) Was von einem Andern bestimmt wird in seinem Daseyn, das ist als Folge von jenem, und dieses als Grund desselben anzusehen; so kann es denn keine Nothwendigkeit besitzen, auch nicht von Ewigkeit da seyn, weil jede Folge später ist, als ihr Grund.\par
\RWbet{Antwort.}
\begin{aufza}
\item Wenn irgend ein Grund Nothwendigkeit hat: so hat sie auch seine Folge. Wenn \zB\ ein Dreieck mit Nothwendigkeit existirt: so existirt auch ein Verhältniß unter seinen drei Seiten, \usw\ mit Nothwendigkeit. Also ist schon der erste Theil des Einwurfes widerlegt.
\item Ferner ist es ein falscher Satz, daß die Folge der Zeit nach allezeit später seyn müsse, als ihr Grund. Nicht alle Gründe und Folgen befinden sich in einer Zeit; auf diese ist also jener Satz schon gar nicht anwendbar, \zB\ das Dreieck und der Inhalt desselben. Und gerade dieses ist der Fall bei Gott. -- Aber selbst dort, wo Grund und Folge in der Zeit vorhanden sind, ist nie die Folge später, als ihr Grund, sondern beide sind vielmehr allemal gleichzeitig.
\end{aufza}\par
\RWbet{2.~Einwurf.} Wenn daraus, weil das Zweite und Dritte in Gott eine Folge von dem Ersten in Gott, \dh~\RWSeitenw{213}\ von etwas Nothwendigem ist, folgen sollte, daß auch sie nothwendig sind: so würde auch aus gleichem Grunde die Nothwendigkeit des Weltalls folgen.\par
\RWbet{Antwort.} Das Zweite und Dritte in Gott gehen aus dem Ersten nach dem Gesetze der Nothwendigkeit, \dh\ ohne Freiheit hervor; die Welt aber wird von Gott durch Freiheit hervorgebracht; sie ist also nicht absolut nothwendig, sondern nur bedingt nothwendig, nämlich nur unter Voraussetzung des freien Willensentschlusses Gottes.\par
\RWbet{3.~Einwurf.} Die katholische Kirche behauptet einerseits, daß der Sohn vom Vater in seinem Daseyn bestimmt werde, und andererseits will sie doch nicht, daß man den Sohn abhängig von dem Vater nenne. Nun ist kein wirklicher Unterschied zwischen Bestimmbarkeit und Abhängigkeit vorhanden; die Kirche widerspricht sich hier also selbst.\par
\RWbet{Antwort.} Obgleich man das Wort Abhängigkeit (\RWlat{dependentia}) zuweilen in einem solchen Sinne nimmt, daß es nichts Anderes, als die Bestimmbarkeit der Folge durch ihren Grund bezeichnet: so nimmt es die Kirche doch in einem andern Sinne, so zwar, daß beide Begriffe sich wirklich unterscheiden. Abhängig nennt sie nur eine solche Folge, die nicht von einerlei Natur und Würde mit ihrem Grunde, sondern geringer ist. So ist \zB\ eine Veränderung, die wir durch unsern freien Willen in der Sinnenwelt hervorbringen, von anderer und geringerer Natur, als ihr intelligibler Grund, der freie Willensentschluß; jene ist etwas Sinnliches, dieser etwas Uebersinnliches. So heißt die Welt abhängig von Gott, weil sie eine solche Wirkung Gottes ist, welche ganz anderer und geringerer Natur, als ihre Ursache ist. Der Sohn dagegen und der heil.\ Geist ist von einerlei Natur und Würde mit dem Vater, allmächtig, allweise, heilig wie er, \usw\ Also kann die katholische Kirche mit Recht verlangen, daß er nicht abhängig vom Vater heiße. 

\RWpar{123}{4.~Verschiedene Wirksamkeit der drei Personen in Gott}
\begin{aufza}
\item Da jenes Dreifache in Gott, mit welchem uns die christliche Offenbarung bekannt macht, gewisse innere Ver\-schie\-\RWSeitenw{214}den\-hei\-ten hat (Nr.\,3. \RWparnr{119}); so ist nicht zu wundern, wenn es sich auch in seiner äußeren Wirksamkeit auf diese Welt so unterscheidet, daß man dem Einen in Gott vorzugsweise die eine oder die andere Wirkungsart zuschreiben darf. Verschiedene Gründe können auch verschiedene Folgen haben.
\item Daß aber gerade dem \RWbet{Ersten} in Gott vorzugsweise die \RWbet{Schöpfung und Regierung der Welt} zugeschrieben werde, \usw : das können wir nicht für falsch und ungereimt erklären, weil wir von diesen Dreien nichts Anderes wissen, als was uns die Offenbarung berichtet. Nun hat sie uns von ihrem Unterschiede nichts Anderes gesagt, als was Nr.\,3.\ \RWparnr{119}\ angeführt worden ist, welches mit dem Gegenwärtigen gar nicht im Widerspruche steht. Im Gegentheile können wir sogar einen gewissen Zusammenhang zwischen diesen beiden Lehren bemerken.
\begin{aufzb}
\item Die Wirksamkeit, welche Gott bei der Schöpfung und Regierung der Welt beweiset, wird vorzugsweise dem Ersten, \dh\ eben demjenigen in Gott zugeschrieben, das schlechterdings keinen Grund seines Daseyns hat. Hierin bemerkt die Vernunft gewiß vielen Zusammenhang, es ist ihr weit begreiflicher, daß dieses Erste -- als es ihr wäre, wenn man sagte, daß das Zweite oder Dritte in Gott als vornehmster und letzter Grund der Schöpfung dieser Welt zu denken wären, sie, die den Grund ihres Daseyns selbst nicht in sich haben. Daß aber die Regierung dieser Welt eben demselben Grunde in Gott zugeschrieben werde, welchem die Schöpfung zugeschrieben wurde, das findet die Vernunft sehr übereinstimmend mit ihren eigenen Entdeckungen. Es zeigt sich nämlich bei einem genaueren Nachdenken, daß Schöpfung und Regierung dieser Welt nicht so, wie man gemeinhin glaubt, zwei der Zeit nach getrennte Handlungen (\RWlat{actus}) der Gottheit sind, sondern daß vielmehr beide einen und ebendenselben untheilbaren Actus der Gottheit bezeichnen, denjenigen nämlich, durch welchen sie der letzte Grund vom Daseyn dieser Welt, und allen Einrichtungen und Schicksalen in derselben wird. Bloß in unserer Vorstellung, je nachdem wir an diesem einfachen Actus jetzt diese,~\RWSeitenw{215}\ jetzt jene Eigenschaft betrachten, löset er sich in die beiden Begriffe der Schöpfung und Regierung auf. In wiefern wir nämlich Gott schlechthin als letzten Grund des Daseyns dieser Welt betrachten, ohne noch zu bestimmen, ob er auch ein verständiger Grund derselben sey, heißt er uns Schöpfer; in wiefern wir aber Gott als einen mit Verstand wirkenden Grund des Daseyns sowohl, als auch der mancherlei Veränderungen der Welt betrachten, heißen wir ihn Regierer. Da also Schöpfung und Regierung so enge verbunden sind: wäre es nicht äußerst gezwungen, wenn die Offenbarung die Regierung etwa einem Andern in Gott zuschreiben wollte, als demjenigen, dem sie die Schöpfung zuschreibt? Das hieße ja mit andern Worten, Gott habe die Welt erst planlos hervorgebracht, dann sich bemüht, dem Chaos Ordnung und Zweckmäßigkeit zu geben.
\item Die Art der Wirksamkeit zum Wohle des menschlichen Geschlechtes, die Gott durch \RWbet{Sendung, Erleuchtung, \usw\ des Menschen Jesu} an den Tag gelegt hat, wird jenem \RWbet{Zweiten} in Gott, also Einem von denjenigen zugeschrieben, welche ihr Daseyn von dem Ersten haben. -- Auch hier vermag die Vernunft einen gewissen Zusammenhang zu bemerken. Die Erlösung des menschlichen Geschlechtes setzt erst das Daseyn dieser Welt voraus; und eben so setzt auch derjenige in Gott, welcher das menschliche Geschlecht erlöset, zuerst das Daseyn desjenigen voraus, welcher die Welt geschaffen hat.
\item Die Art der Wirksamkeit, die Gott noch immer beweiset, wenn er \RWbet{auf unsichtbaren Wegen gute Gesinnungen in uns weckt, und gute Handlungen bewirkt,} schreibt die christliche Offenbarung dem \RWbet{Dritten} in Gott, \dh\ demjenigen zu, welcher sein Daseyn vom Ersten und Zweiten zugleich hat. Auch hier vermag die Vernunft etwas zu fassen. Die Anstalten, welche Gott zur Erleuchtung und Heiligung einzelner Menschen trifft, gehören zwar freilich im weitesten Verstande auch zur Regierung Gottes; aber es gibt unter denselben gewiß viele, welche die Lehre des Christenthums, und~\RWSeitenw{216}\ überhaupt Alles dasjenige, was Gott durch Jesum ausgeführt hat, wie ihren Grund voraussetzen. Anders hätte Gott zur Erleuchtung und Heiligung der Menschen beitragen müssen, wenn es kein Christenthum gäbe, anders, nachdem es ein solches gibt. -- Also werden diese Anstalten zur Heiligung der Menschen von demjenigen in Gott gewirkt, der sein Daseyn von Beiden, dem Ersten sowohl, als auch dem Zweiten, hat.
\end{aufzb}
\end{aufza}

\RWpar{124}{5.~Den Dreien in Gott kommen alle sogenannten natürlichen Attribute der Gottheit gemeinschaftlich zu}
Daß jene Drei in Gott, mit welchen uns die christliche Religion bekannt macht, allmächtig, allwissend, heilig, höchst selig, allgegenwärtig und ewig sind, mit Einem Worte, daß ihnen alle die natürlichen Attribute der Gottheit beigelegt werden können, dieß widerspricht nicht nur keiner uns bekannten Wahrheit, sondern es ist sogar eine gewissermaßen nothwendige Folge von dem, was wir bereits vernommen haben. Nach Nr.\,4.\ sind nämlich Jedem der Drei in Gott gewisse Wirkungen vorzugsweise zuzuschreiben, und zwar dem Ersten die Schöpfung und Erhaltung dieser Welt, dem Zweiten die Erlösung des menschlichen Geschlechtes, dem Dritten die Heiligung desselben. Diese Wirkungen sind nun offenbar von einer solchen Art, daß ihren wirkenden Gründen Allmacht, Allwissenheit, Heiligkeit, \usw\ beigelegt werden muß. So muß \zB\ dasjenige in Gott, wodurch er diese Welt erschaffen hat und regiert, gewiß Allmacht, Allwissenheit, Heiligkeit haben. Dieß lehrt selbst die natürliche Religion. Aber auch dasjenige in Gott, was die Erlösung des menschlichen Geschlechtes durch Jesum Christum bewirkt hat, muß eine mit Weisheit und Heiligkeit wirkende Kraft seyn, \usw\par
\RWbet{1.~Einwurf.} Der Sohn und der heil.\ Geist, oder das Zweite und Dritte in Gott, haben, wie die katholische Kirche lehrt, ihr Daseyn nicht aus sich selbst, sondern erhalten es erst durch den Vater, oder das Erste in Gott. Sie sind daher das, was wir abhängig oder endlich nennen, folglich können sie nicht allmächtig, allwissend \udgl\  seyn; denn~\RWSeitenw{217}\ dieß sind Eigenschaften, welche nur einem unendlichen Wesen zukommen.\par
\RWbet{Antwort.} In diesem Einwurfe geht man von der sehr wahren Voraussetzung aus, daß ein \RWbet{Wesen}, welches den Grund seines Daseyns nicht in sich selbst, sondern in irgend einem andern Wesen hat, eben darum auch ein abhängiges oder endliches Wesen sey. Wahr ist es auch, daß abhängige oder endliche Wesen keineswegs allmächtig, allwissend, \usw\ seyn und heißen können. Aber nur dieß vergißt man, daß jene Voraussetzung bei der Dreieinigkeitslehre gar keine Anwendung findet; denn \RWbet{nicht drei verschiedene Wesen} sind jene Drei in Gott, mit welchen das Christenthum uns bekannt macht. Der Sohn und der heil.\ Geist haben den Grund ihres Daseyns nicht in einem andern Wesen, sondern in Etwas, welches sich in eben dem Wesen befindet, darin sie selbst sich befinden. -- Daß aber dieser Unterschied entscheide, und daß es allerdings in Gott Etwas geben könne, welches, obwohl es in seinem Daseyn durch etwas Anderes in Gott bestimmt wird, dennoch unendlich ist, wissen wir selbst aus der natürlichen Religion. Oder nehmen wir nicht in Gott einen Verstand und einen Willen an? setzen wir nicht voraus, daß der letztere in seinen Aeußerungen durch den ersteren bestimmt werde? und behaupten wir nicht demungeachtet mit allem Rechte, daß Gottes Wille ein unendlicher sey?\par
\RWbet{2.~Einwurf.} Wenn der Vater allmächtig, allwissend, allweise, \usw ; der Sohn allmächtig, allwissend, allweise, \usw ; der heil.\ Geist allmächtig, allwissend, allweise, \usw\ ist: so folget nothwendig, daß es in Gott dreierlei Verstand, dreierlei Willen, \usw\ gebe. Diese Behauptung ist aber den Begriffen, welche sich die natürliche Religion von Gott bildet, so wie denjenigen, welche das Christenthum selbst aufstellt, gänzlich zuwider.\par
\RWbet{Antwort.} Es ist gar keine nothwendige Folge, daß, wenn der Vater sowohl, als auch der Sohn und der heil.\ Geist allmächtig, allwissend, \usw\  soll genannt werden können, ein dreifacher Verstand, ein dreifacher Wille, \usw\ in Gott vorhanden seyn müsse. Könnte es denn nicht ein ein\RWSeitenw{218}ziger Vestand, ein einziger Wille seyn, der allen Dreien gemeinschaftlich zukommt? Die christliche Offenbarung sagt uns nichts, was uns berechtigte, die Drei in Gott dergestalt von einander zu trennen, daß der Verstand, der Wille, \usw , welchen wir einem Jeden beilegen, auch immer ein eigener seyn mußte. Kann nicht derselbe Verstand, durch welchen Gott bei der Regierung dieser Welt die zweckmäßigsten Einrichtungen erkennt, auch bei den Anstalten, welche er zur Erlösung des menschlichen Geschlechtes getroffen hat, die zweckmäßigsten Einrichtungen erkannt haben? \usw\par
\RWbet{3.~Einwurf.} Die Kirche sagt allerdings Einiges, was uns berechtiget, jene Drei in Gott dergestalt zu trennen, daß der Verstand, der Wille eines Jeden ein eigener sey. Sie redet nämlich von einer Liebe, die zwischen dem Vater und dem Sohne Statt finde, von einem Gehorsame, den jener diesem bewiesen habe, \usw\par
\RWbet{Antwort.} Diese Redensarten zwingen uns gar nicht, einen eigenen Willen im Vater und Sohne anzunehmen. Der Vater liebt den Sohn, heißt ja nur so viel: dasjenige in Gott, was die Welt erschaffen hat, sie erhält und regiert, thut Manches, was auch dem Zwecke desjenigen in Gott entspricht, was für die Beglückung des menschlichen Geschlechtes sorgt. Der Sohn ist dem Vater gehorsam, heißt: die Anstalten, die Gott zur Beglückung der Menschheit trifft, dürfen den Anstalten, die zum Besten aller Geschöpfe überhaupt nothwendig sind, nicht widersprechen.

\RWpar{125}{6.~Die Drei in Gott sind nicht getrennt, und außerhalb einander, sondern in und durch einander vorhanden}
Gegen diese Behauptung kann die Vernunft nichts Gegründetes einwenden. Gott ist nicht ausgedehnt; also kann auch das, was in ihm vorhanden ist, nicht etwa außerhalb einander bestehen.\par
\RWbet{Einwurf.} Gleichwohl, wenn in der christlichen Religion behauptet wird, der Sohn Gottes, \dh\ das Zweite in Gott habe sich mit dem Menschen Jesu vereiniget, \usw : so widerspricht sie selbst der Behauptung, daß der Sohn Gottes nirgends, als im Vater, bestehe.~\RWSeitenw{219}\par
\RWbet{Antwort.} Wenn die christliche Religion lehrt, daß der Sohn Gottes sich mit dem Menschen Jesu vereiniget habe, \usw : so ist dieses nicht von einer Vereinigung im Raume zu verstehen, denn der Sohn Gottes, so wie Gott selbst, befindet sich nicht im Raume; sondern es heißt, die zweite Person habe in dem Menschen Jesu auf eine bestimmte Art gewirket.

\RWpar{126}{7.~Bildliche Vorstellungen in der Dreieinigkeitslehre: a)~Die Rücksicht, in welcher es in Gott ein Dreifaches gibt, heißt Persönlichkeit}
Mit Unrecht haben sich selbst einige Katholiken erlaubt, die Kirche wegen der Wahl des Wortes \RWbet{Person} auf eine gewisse Art zu tadeln. Es ist kein unbequemer Ausdruck, wie diese meinen, sondern nach dem zu urtheilen, was wir von jenen Dreien in Gott bisher gehört haben, ist es der schicklichste, den unsere Sprache hat. Wir verstehen nämlich, zu Folge eines fast allgemein unter uns eingeführten Sprachgebrauches, unter einer Person \RWbet{eine mit Verstand und Willen wirkende Ursache}, gleichviel, ob diese Ursache als eine eigene Substanz, oder als ein Inbegriff mehrerer Substanzen, oder wie sonst, bestehe. Nun haben wir bisher gehöret, daß jene Drei in Gott allerdings mit Verstand und Willen wirken. Also kommt ihnen der Name Person mit allem Rechte zu. Weil aber der Verstand und Wille, den diese besitzen, ein göttlicher Verstand und Wille ist, so müssen wir diese drei Personen auch göttliche Personen nennen. -- Gleichwohl weil dieser Name von sinnlichen Gegenständen entlehnt ist, weil ihn auch sinnliche Nebenvorstellungen begleiten, weil er uns endlich die innere Natur und Beschaffenheit jenes Dreifachen in Gott gar nicht erschöpfend angibt: ist es ein bildlicher Ausdruck, und diese Lehre gehört sonach zu den bildlichen.\par
\RWbet{Einwurf.} Wofern es drei Personen in Gott gibt, so gibt es auch drei Naturen oder Wesen in Gott, \dh\ drei Götter. Denn jede einzelne Person muß doch ein eigenes Wesen haben.\par
\RWbet{Antwort.} Daß jede Person ihr eigenes Wesen haben müsse, läßt sich auf keine Art beweisen. Im Gegen\RWSeitenw{220}theile, wenn man das Wort Person in jenem weitesten Verstande nimmt, in dem es nur überhaupt eine mit Verstand und Willen wirkende Ursache bezeichnet: so läßt sich deutlich zeigen, daß ein und dasselbe Wesen mehrere Personen haben könne. Man denke sich irgend ein mit Verstand und Willen begabtes Wesen, und lasse dasselbe unter verschiedenen Umständen mancherlei Wirkungen, die alle Verstand und Willen verrathen, hervorbringen: so werden wir sagen dürfen, dieß Wesen stelle mehrere Personen vor. So stellt ein und derselbe vernünftige Mensch wirklich auch mehrere Personen, \zB\ die eines Bürgers, die eines Familienoberhauptes, die eines Freundes vor. Nun ist es freilich nicht zu läugnen, daß die Kirche das Wort Person in der Dreieinigkeitslehre in einem viel engern und ganz eigenthümlichen Sinne nimmt; wenn sie damit jenen von unserem Verstande niemals ganz zu erreichenden dreifachen Grund in Gott versteht, von welchem die Schöpfung der Welt, die Erlösung des Menschengeschlechtes und dessen Heiligung drei einzelne Wirkungen sind, die doch nicht Alles erschöpfen, was jene Drei sind und wirken. Aber eben, weil die Kirche von diesen Personen nur so viel und nicht mehr behauptet, nöthiget sie uns nirgends, jeder von ihnen ein eigenes Wesen beizulegen.

\RWpar{127}{b)~Die Namen: Vater, Sohn und Geist}
Daß jene Namen, welche die drei göttlichen Personen in der christlichen Offenbarung tragen, nur \RWbet{bildliche} Namen sind, hat man von Seite der Kirche von jeher zugegeben. Es fragt sich also nur, ob diese Bilder auch eine vernünftige Auslegung zulassen.
\begin{aufza}
\item Die erste göttliche Person enthält den Grund des Daseyns der Zweiten und Dritten in sich; schon um dieses einzigen Umstandes willen kann sie sehr schicklicher Weise der \RWbet{Vater} heißen. Hiezu kommt noch, daß eben dieser Person auch die Schöpfung und Regierung der Welt vorzugsweise zugeschrieben wird. Auch in dieser Beziehung trägt sie den Namen Vater überaus schicklich.
\item Gibt es aber in Gott einen Vater: so muß es der Beziehung wegen auch einen \RWbet{Sohn} in Gott geben; und~\RWSeitenw{221}\ schon um dieses Grundes willen ist es schicklich, daß die zweite göttliche Person der Sohn, der eingeborne Sohn des Vaters heiße. Diese zweite göttliche Person ist ferner \RWbet{in dem Menschen Jesu wirksam} gewesen, hat sich mit ihm vereiniget zu einer einzigen Person. Die Menschen werden \RWbet{Gottes Kinder} genannt; und der Mensch Jesus verdiente vorzugsweise vor Allen \RWbet{Gottes Sohn} zu heißen; also ein zweiter Grund, der uns die Schicklichkeit des Namens Sohn beweiset. -- Nicht minder schicklich sind aber auch die Namen \RWbet{Vernunft, Wort} (\RWgriech{l'ogos}) oder \RWbet{Weisheit}, die diese zweite Person gleichfalls zuweilen erhält; denn Gottes Weisheit hat sich doch in der That durch die Sendung des Sohnes vorzugsweise geoffenbaret.
\item Die dritte göttliche Person ist als der Grund von allen den unsichtbaren Einwirkungen, die Gott auf unsern \RWbet{Geist} hervorbringt, zu denken; sehr schicklich also kommt ihm der Name \RWbet{Geist} zu.
\end{aufza}

\RWpar{128}{c)~Der Sohn ist gezeugt aus des Vaters Wesen}
\begin{aufza}
\item Daß auch der Ausdruck \RWbet{zeugen} bildlich zu verstehen sey, ergibt sich schon daraus, weil man die Namen Vater und Sohn allezeit nur bildlich genommen hat.
\item Seine Vernunftmäßigkeit läßt sich im Uebrigen schon aus der Vernunftmäßigkeit der bildlichen Namen Vater und Sohn ableiten; denn in der That ist der bildliche Ausdruck des \RWbet{Erzeugtseyns} mit den bildlichen Ausdrücken: Vater und Sohn, so enge verbunden, daß er, wo diese sind, sich gleichsam von selbst einstellt. Und daß er etwas Unanständiges enthalte, daß er mit niedrigen Nebenvorstellungen in unzertrennlicher Verbindung stehe, könnte wohl nur derjenige behaupten, dessen Einbildungskraft durchaus verdorben wäre. Brauchen wir doch das Wort Erzeugen auch sonst von den ehrwürdigsten Gegenständen, so wie denn auch das, was es an sich bezeichnet, etwas sehr Ehrwürdiges ist, und Jedem seyn soll.
\end{aufza}\par
\RWbet{Einwurf.} Aber die christliche Kirche macht einen Gegensatz zwischen den beiden Redensarten: \RWbet{Erzeugen aus}~\RWSeitenw{222}\ \RWbet{seinem eigenen Wesen}, und \RWbet{Schaffen aus Nichts.} Der Sohn ist, spricht sie, erzeugt aus dem Wesen des Vaters, geschaffen aber, und zwar aus Nichts geschaffen, ist die Welt. -- Zwischen diesen Redensarten ist aber kein Unterschied vorhanden, der sich auf deutliche Begriffe zurückführen ließe.\par
\RWbet{Antwort.} Es ist ein wahrer Unterschied vorhanden. Wie wir oben schon sehen, so hat der Sohn sein Daseyn vom Vater \RWbet{nothwendiger}, die Welt nur \RWbet{zufälliger} Weise, nämlich durch den freien Entschluß des Willens Gottes. Jene nothwendige Abkunft des Sohnes von dem Vater drückt die christliche Kirche sehr passend durch ein Gezeugtseyn aus des Vaters Wesen; dagegen das zufällige Daseyn der Welt durch ein Geschaffenseyn aus Nichts aus.

\RWpar{129}{d)~Der heil.\ Geist geht vom Vater und vom Sohne aus}
\begin{aufza}
\item Auch die Redensarten des \RWbet{Ausgehens} und \RWbet{Gesendetwerdens} nimmt die christliche Kirche, wie sich von selbst versteht, nur bildlich.
\item Da der heil.\ Geist vom Vater und vom Sohne zugleich auf eine andere Art bestimmt wird, als der Sohn vom Vater, so war es nothwendig, zur Bezeichnung dieser andern Art der Bestimmung auch einen andern Ausdruck zu wählen. Der gewählte bezieht sich auf den Namen Geist (\dh\ Hauch, Athem).
\end{aufza}

\RWpar{130}{Noch ein Einwurf gegen die Dreieinigkeitslehre}
Die christliche Kirche lehrt mit ausdrücklichen Worten, der Vater sey Gott, der Sohn sey Gott, der heil.\ Geist sey Gott; \dh\ mit andern Worten: der Vater ist gleich Gott, der Sohn ist gleich Gott, der heilige Geist ist gleich Gott. Nun ist es ein sicherer und unumstößlicher Grundsatz, der Grundsatz, dessen sich die Mathematiker bedienen, daß Dinge, die einem Dritten gleich sind, auch unter einander gleich sind. Aus diesem Grundsatze ergibt sich also, daß Vater, Sohn und heil.\ Geist, weil alle Drei Gott gleich sind, auch unter einander völlig gleich seyn müssen. Dieses will aber die katholische Kirche auf keine Weise zugeben, sondern dringt~\RWSeitenw{223}\ darauf, daß ein reeller Unterschied zwischen dem Vater, dem Sohne und dem heil.\ Geiste obwalte. Sie widerspricht also hier dem ewig wahren Grundsatze: \RWlat{quae sunt aequalia uni tertio, sunt aequalia inter se.} -- Ein anderer eben so wahrer und unumstößlicher Grundsatz lautet: Gleiches zu Gleichem addirt gibt Gleiches. Wendet man diesen Grundsatz auf die drei Gleichungen: Vater gleich Gott, Sohn gleich Gott, heiliger Geist gleich Gott an: so erhält man nach dem regelmäßigsten Verfahren die Folgerung: der Vater, der Sohn und der heil.\ Geist machen zusammen drei Götter aus. Eine Folgerung, vor welcher die gesunde Vernunft eben so sehr erschrickt, als die katholische Kirche sich dieselbe verbeten hat. Sie muß denn also den Grundsatz, daß Gleiches zu Gleichem addirt, gleiche Summen gibt, läugnen.\par
\RWbet{Antwort.} Die katholische Kirche läugnet keinen von jenen beiden Grundsätzen. Die scheinbaren Widersprüche aber, die man aus diesen Grundsätzen hier ableiten will, entstehen bloß daraus, daß man die kirchlichen Redensarten: der Vater ist Gott, der Sohn ist Gott, der heil.\ Geist ist Gott, in einem ganz andern Sinne auslegt, als sie die Kirche nimmt, und auch der Sprachgebrauch erheischt. Der Satz: der Vater ist Gott, soll nämlich gar nicht eine völlige Gleichheit zwischen den beiden Begriffen Vater und Gott anzeigen; sondern nur, daß die Eigenschaft der Gottheit dem Vater beigelegt werden könne. Gerade so, wie der Satz: Cajus ist Mensch, keineswegs eine völlige Gleichheit zwischen den beiden Vorstellungen Cajus und Mensch aussagen will. So wie es daher sehr falsch geschlossen wäre, wenn Jemand aus den drei Sätzen: Cajus ist Mensch, Sempronius ist Mensch, Titus ist Mensch, die Folge herleiten wollte: Cajus, Sempronius und Titus sind einander in jeder Rücksicht gleich: so ist auch der ganz ähnliche Schluß, welchen man sich in dem obigen Einwurfe erlaubt, falsch.
Der Fehler besteht darin, daß man ein Urtheil wie eine Gleichung schreibt und ansieht. So lassen sich höchstens identische Urtheile, \zB\ Definitionen, und allenfalls noch Urtheile, deren Subject und Prädicat \RWbet{Wechselbegriffe} (\dh\ Begriffe von gleichem Umfange) sind, betrachten. So kann man das Urtheil: Ein gleichschenkliges Dreieck ist ein Dreieck, dessen zwei Seiten~\RWSeitenw{224}\ einander gleich sind, in Form einer Gleichung schreiben, weil es eine \RWbet{Definition} ist; und eben so auch das Urtheil: ein gleichseitiges Dreieck ist ein gleichwinkliges, weil die Begriffe gleichseitiges Dreieck und gleichwinkliges Dreieck \RWbet{Wechselbegriffe} sind. -- Alle andern Urtheile dagegen, in welchen Subject und Prädicat nicht gleichen Umfang haben, kann man auch eben deßhalb nicht als Gleichungen schreiben. -- Wollte man die Sätze: der Vater ist Gott, der Sohn ist Gott, der heil.\ Geist ist Gott, dennoch in Gleichungen verwandelt sehen: so müßte man das Prädicat derselben genauer bestimmen, und sie auf folgende Art ausdrücken: der Vater ist gleich der ersten Person in Gott; der Sohn ist gleich der zweiten Person in Gott; der heil.\ Geist ist gleich der dritten Person in Gott. Aus diesen drei Gleichungen aber würde sich nimmermehr die Folgerung: der Vater, der Sohn und der heil.\ Geist sind einander gleich, ergeben. -- Und jetzt verschwindet von selbst auch schon der zweite Widerspruch. Denn wenn wir diese recht ausgedrückten Gleichungen addiren, so geben sie ein sehr wahres Resultat, nämlich: der Vater, der Sohn und der heil.\ Geist zusammengenommen sind die erste, die zweite und die dritte Person in Gott.

\RWpar{131}{Ist es befremdend, daß die Vernunftmäßigkeit dieser Lehre so oft bestritten worden ist?}
Aus dem Bisherigen erhellet deutlich genug, daß die katholische Lehre von Gottes dreifacher Persönlichkeit nichts der Vernunft Widersprechendes enthalte. Daß man ihr dieses gleichwohl so oft Schuld gegeben hat, läßt sich aus der Beschaffenheit der wider sie vorgebrachten Einwürfe und einigen andern Umständen sehr leicht erklären. Hieran war nämlich Schuld
\begin{aufzb}
\item das Vorurtheil, daß in dem unbedingten Wesen in durchaus keiner Rücksicht eine Art von Vielheit angenommen werden dürfe;
\item der Nebenbegriff von einer eigenen Substanz, der sich mit dem Begriffe von einer eigenen Person so gern ver\RWSeitenw{225}bindet, weil es gewöhnlich der Fall ist, daß eine eigene Person auch eine eigene Substanz hat;
\item das Vorurtheil, daß die Folge oder Wirkung allemal später als der Grund oder die Ursache sey;
\item das Vorurtheil, daß Etwas, das unendlich heißen soll, in keiner Rücksicht eine Beschränkung oder Bestimmung haben dürfe;
\item der Anschein, daß, wenn dem Vater, dem Sohne und dem heil.\ Geiste ein Verstand oder Wille zugeschrieben wird, es dreierlei Verstandes- oder  Wollkräfte in Gott geben müsse;
\item Endlich ist nur zu gewiß, daß man aus Abneigung gegen das Christenthum das Auge des Verstandes zuschloß, um die vernünftige Auslegung dieser wichtigen Lehre und die Verträglichkeit derselben mit jeder andern Wahrheit nicht anzuerkennen, sondern vielmehr eine Veranlassung zu haben, die christliche Religion um dieser Lehre wegen zu verwerfen.
\end{aufzb}

\RWpar{132}{Eine Vermuthung, die zu noch besserem Verständnisse der Lehre von Gottes dreifacher Persönlichkeit etwas beitragen könnte}
\begin{aufza}
\item Auch schon in der Gestalt, in der wir die Lehre von Gottes dreifacher Persönlichkeit vorgetragen haben, in der sie auch von der Kirche allgemein gelehrt wird, kann sie gewiß sehr wohl verstanden werden. Aber obgleich man etwas versteht, kann man doch wünschen, darüber noch mehr zu erfahren. Und da gibt es denn besonders Einen Punct in dieser Lehre, in Betreff dessen der gebildetere Christ vornehmlich wünschen dürfte, noch etwas Mehreres zu wissen, als in der Kirche Gottes bisher ausdrücklich gelehrt worden ist. Es ist dieß der Artikel \RWbet{von den besondern Wirkungen, die jeder der drei göttlichen Personen zugeschrieben werden sollen.} In der bisherigen Darstellung dieser Lehre werden uns nämlich \RWbet{nur Einige} dieser Wirkungen, gleichsam als Beispiele, angeführt, ohne bestimmt zu sagen, ob es nicht noch andere gibt, ja vielmehr so, daß die Vermuthung, es dürfte noch andere geben, von selbst entsteht.~\RWSeitenw{226}\ Da wünschten wir denn, nebst jenen einzelnen Beispielen, auch noch die \RWbet{allgemeine Regel} oder das allgemeine Kennzeichen zu erfahren, woraus beurtheilt werden kann, daß und warum gewisse Wirkungen der Einen, gewisse der andern göttlichen Person vorzugsweise zuzuschreiben sind. Da es dem göttlichen Geiste bis jetzt noch nicht gefallen hat, eine gleichförmige Meinung hierüber unter uns herrschend werden zu lassen: so kann in diesem Stücke nichts mit Gewißheit entschieden werden. Eine \RWbet{Vermuthung} aber, die man für nichts weiter, als für eine Vermuthung ausgibt, vorzutragen, kann nicht unerlaubt seyn; da man durch eine solche, auch wenn sie irrig seyn sollte, keinen Schaden anrichtet, indem es Jedem, dem sie nicht behagt, frei steht, sie zu verwerfen.
\item Meine Vermuthung nun wäre, daß dem \RWbet{Vater} in Gott alles dasjenige zuzuschreiben sey, \RWbet{was in Beziehung auf das Wohl aller geschaffenen Wesen}; dem \RWbet{Sohne} alles dasjenige, \RWbet{was in Beziehung auf das Wohl einer ganzen Wesenart, namentlich des ganzen menschlichen Geschlechtes} nothwendig ist; dem heil.\ \RWbet{Geiste} endlich alles dasjenige, \RWbet{was nur zur Beförderung des Wohles eines einzelnen Geschöpfes, namentlich \zB\ eines einzelnen Menschen,} dienet. Oder mit andern Worten: der \RWbet{Vater} ist derjenige Grund in Gott, der die zum Wohle des \RWbet{ganzen Weltalls} nothwendigen Verfügungen trifft; der \RWbet{Sohn} derjenige Grund in Gott, der die Verfügungen, die zur \RWbet{Beseligung des ganzen menschlichen Geschlechtes} nothwendig sind, einleitet; der \RWbet{heilige Geist} endlich dasjenige in Gott, was die \RWbet{Beseligung einzelner Menschen} besorgt.
\item Diese Vermuthung enthält 
\begin{aufzb}
\item keinen innern Widerspruch, indem sich die Anstalten Gottes wirklich so eintheilen lassen, wie es hier geschieht. Es gibt in der That Veranstaltungen Gottes (Ereignisse, Leitung unserer Schicksale, \udgl ), die (nicht ihren einzigen, aber doch) ihren vornehmsten Grund in dem wohlthätigen Einflusse haben, den sie auf irgend ein einzelnes Wesen, \zB\ auf einen einzelnen Menschen, äußern. Dagegen gibt es auch Veranstaltungen Gottes, die nicht so\RWSeitenw{227}wohl um der Beglückung dieses oder jenes einzelnen Wesens, als vielmehr nur um der Beglückung einer ganzen Gattung, \zB\ des ganzen menschlichen Geschlechtes, willen getroffen werden müssen; \zB\ die Einführung einer bessern Religion auf Erden, \udgl\  Endlich gibt es auch Verfügungen Gottes, die nicht um des Wohles eines einzelnen Geschöpfes, auch nicht um des Wohles einer ganzen Gattung, sondern nur um des Wohles aller lebendigen Wesen zusammengenommen nothwendig sind. Von dieser Art ist \zB\ die Regel, daß alles Gute belohnt, und alles Böse bestraft werden müsse.
\item Bei dieser Annahme läßt sich erklären, warum die Offenbarung jeder der drei göttlichen Personen gerade diese und jene Wirkungen beilegt.
\begin{aufzc}
\item Wenn der Vater dasjenige in Gott ist, was die auf das ganze Weltall sich beziehenden Verfügungen trifft: so ist wohl begreiflich, wie ihm die Schöpfung und Erhaltung der Welt, ingleichen auch ihre Regierung überhaupt zugeschrieben werde; denn es ist leicht zu erachten, daß \zB\ das Daseyn dieser Himmelskörper und alle jene durch's ganze Weltall verbreitete physische und moralische Gesetze, nach denen, wie man sich insgemein vorstellt, die meisten Ereignisse in der Welt herbeigeführt werden, nicht bloß aus Rücksicht auf ein einzelnes Geschöpf, oder auch nur auf eine einzelne Gattung von Geschöpfen, sondern aus Rücksicht auf alle lebendigen Wesen festgesetzt werden mußten. Die Sonne \zB\ ist doch gewiß nicht nur der Menschen, sondern auch anderer Geschöpfe wegen geschaffen, und gerade so, wie sie ist, eingerichtet worden. Daher wir ihr Daseyn dem Vater zuschreiben.
\item Ist der Sohn dasjenige in Gott, was die zum Wohle des ganzen Menschengeschlechtes benöthigten Anstalten herbeiführt: so ist begreiflich, wie ihm die Erlösung unseres Geschlechtes, das künftige Gericht desselben, die Einführung der mosaischen sowohl, als christlichen Religion \ua\,dgl.\ Anstalten beigelegt werden können, die für das Beste unseres Geschlechtes überhaupt noth\RWSeitenw{228}wendig sind, obgleich man zugeben muß, daß sie für manchen einzelnen Menschen (durch seine eigene Schuld) statt nützlich nur verderblich werden.
\end{aufzc}

\begin{RWanm} 
Man könnte einwerfen, daß es doch viele, das Heil eines Einzelnen betreffende Wohlthaten gebe, \zB\ die Genesung von einer Krankheit, die Niemand dem heil.\ Geiste, sondern dem Vater zuschreibt. Eine nähere Betrachtung zeigt aber, daß dieß insgemein nur solche Wohlthaten sind, die wir uns als herbeigeführt denken durch gewisse, sich auf das ganze Weltall erstreckende, und also dem Vater zuzuschreibende Naturgesetze; unsere Genesung, \zB\ durch die Kraft der Arzneimittel \udgl\  Auf den Einwurf, daß die Erleuchtung der Kirche dem göttlichen Geiste beigelegt wird, während dieß doch eine Wirkung ist, deren Wohlthätigkeit sich auf das ganze menschliche Geschlecht bezieht, ließe sich erwiedern, daß man dieß thue, weil man hiebei auch auf die Gnade hinsieht, welche den \RWbet{einzelnen Lehrern} der Kirche zu Theil wird. Endlich erklärt sich aus dieser Annahme auch sehr wohl, wie jene in ihrer Art ganz eigene Einwirkung Gottes, welche die menschliche Natur Jesu erfuhr, bald dem Sohne, bald auch dem Geiste, zuweilen auch selbst dem Vater zugeschrieben werde. So oft nämlich auf das gesehen wird, was dem ganzen Menschengeschlechte durch diesen Jesum geleistet wurde, so wird von dem durch ihn wirkenden Sohne geredet. Wird aber auf das gesehen, was er selbst durch diese außerordentliche Einwirkung Gottes gewann (auf seine hohen Tugenden, Einsichten, \usw ): so wird dieß dem göttlichen Geiste zugeschrieben. Betrachtet man endlich, wie seine Sendung auf Erden nicht nur auf das gesammte menschliche Geschlecht, sondern auch auf das ganze Weltall einen gewissen wohlthätigen Einfluß gehabt habe: so wird von einem Rathschlusse des Vaters, der den Sohn gesendet habe, gesprochen.
\end{RWanm}
\item Mit dieser Annahme stimmen auch sehr wohl die übrigen Puncte in der Dreieinigkeitslehre überein; \zB\ daß der Sohn sein Daseyn vom Vater, der heil.\ Geist vom Vater und vom Sohne habe. Die Anstalten nämlich, welche zum Wohle des ganzen menschlichen Geschlechtes getroffen werden sollen, gründen sich auf die Anstalten, welche das Wohl des ganzen Weltalls betreffen; die Anstalten aber, welche zur Rettung und Beseligung eines~\RWSeitenw{229}\ jeden einzelnen Menschen dienen, gründen sich auf jene beiden. Nichts desto weniger kann man den Sohn und den Geist nicht als dem Vater untergeordnet betrachten, weil jeder einzelne Mensch, um wie viel mehr das ganze menschliche Geschlecht als Zweck an sich betrachtet, und folglich nicht bloß als ein Mittel zur Beförderung des allgemeinen Wohles gebraucht, und diesem aufgeopfert 
wird, \usw\
\item Endlich empfiehlt sich diese Vermuthung auch durch ihre Einfachheit und leichte Verständlichkeit; denn nicht nur sie selbst läßt sich von Jedermann leicht verstehen, sondern auch die ganze Lehre von Gottes dreifacher Persönlichkeit wird faßlicher gemacht, wenn man auf diese Art die verschiedenen Wirkungen, welche den einzelnen Personen in Gott zugeschrieben werden sollen, unter drei allgemeine Gesichtspuncte zusammenfassen kann. Noch wichtiger aber ist es, daß sich bei dieser Annahme am Allerdeutlichsten zeigt, wie angemessen die Lehre von Gottes dreifacher Persönlichkeit gerade für uns sey. Wenn der Vater, der Sohn, und der heil.\ Geist in ihren äußern Wirkungen in der That so unterschieden sind, wie es in dieser Vermuthung vorausgesetzt wird: so geht uns ihr Unterschied viel näher an, als in irgend einem andern Falle. Oder was könnte uns näher angehen, als zu wissen, daß es in Gott nebst einem Grunde, der für das Wohl des gesammten Weltalls sorgt, noch einen zweiten Grund gebe, der für das Wohl unseres Geschlechtes, und einen dritten, der für das Wohl jedes Einzelnen aus uns sorget?
\end{aufzb}
\end{aufza}

\RWpar{133}{8.~Die Lehre von Gottes dreifacher Persönlichkeit ist eine Geheimnißlehre}
\begin{aufza}
\item So entschieden es auch durch alles Bisherige ist, daß die Lehre von Gottes dreifacher Persönlichkeit der Vernunft nicht widerspricht: so würde man doch auch von der andern Seite wieder zu weit gehen, wenn man behaupten wollte, daß sie sich \RWbet{aus der Vernunft selbst darthun lasse.} Die katholische Kirche hat dieß von jeher geläugnet~\RWSeitenw{230}\ und will diese Lehre immer als eine von jenen \RWbet{Geheimnißlehren} angesehen wissen, \RWbet{zu deren Aufstellung die menschliche Vernunft niemals berechtigt gewesen wäre,} wenn uns die Offenbarung nicht mit ihr bekannt gemacht hätte. Und das ist auch ganz richtig. Wie sollte die menschliche Vernunft, die das unendliche Wesen so wenig zu beurtheilen vermag, ohne es zuerst aus einer Offenbarung vernommen zu haben, berechtigt seyn, zu behaupten, daß unter allen Vorstellungsarten von Gott, die für den Menschen möglich wären, gerade diejenige, die in der christlichen Lehre von Gottes Dreieinigkeit vorkommt, der Wahrheit am nächsten komme?
\item Indessen hat man gleichwohl verschiedene Versuche angestellt, diese Lehre entweder ganz zu erweisen, oder nur einiger Maßen zu begründen und einleuchtender zu machen. Wenn man das Letztere allein beabsichtiget, und folglich nicht behauptet, daß die Lehre von Gottes dreifacher Persönlichkeit durch bloße Vernunft erweislich sey, auch keine Auslegung annimmt, die der Lehre der Kirche widerspricht: so duldet diese dergleichen Versuche als etwas, das wenigstens unschädlich ist. Aber leider ist es bei den meisten Versuchen, die man bisher gemacht hat, der Fall, daß sie entweder an sich selbst unverständlich, oder mit irgend einer Vernunftwahrheit, oder mit Lehren des Christenthums im Widerspruche stehen. Ich will nur einige der wichtigsten in Kürze anführen.
\begin{aufzb}
\item Der Erste unter den christlichen Schriftstellern, von dem wir einen obgleich sehr unvollkommenen Versuch einer Deduction der Trinität haben, ist \RWbet{Tertullian}. \RWbet{Gott} ist ein denkendes Wesen, sagte er, als solches hat er Vorstellungen, diese sind das \RWbet{Wort} (\RWlat{sermo}) oder der Sohn. Aber jedes Wort, jede Rede hat auch einen \RWbet{Sinn}, dieser ist der Geist.
\item Schon etwas deutlicher erklärte sich \RWbet{Basilius} (\RWlat{homilia in initium evangelii Johannis}), \RWbet{Gregor von Nazianz} (\RWlat{orat.\,36.})\RWlit{}{GregorvonNazianz1}, \RWbet{Augustinus} (\RWlat{in libris de Trinitate})\RWlit{}{Augustinus5}, \uA , die Alle das Daseyn des \RWbet{Sohnes} als einen \RWbet{Actus der göttlichen Denkkraft,} das Daseyn des \RWbet{heil.\ Geistes} aber als einen \RWbet{Actus der}~\RWSeitenw{231}\ \RWbet{göttlichen Wollkraft}, nämlich der zwischen dem Vater und Sohne obwaltenden Liebe ansahen. \erganf{\RWlat{Cur verbum dicitur?}} fragt Basilius; \erganf{\RWlat{ut ostendatur, ex mente processisse.}} -- \erganf{\RWlat{Verbum,}} sagt Gregor von Nazianz, \erganf{\RWlat{appellatur, quoniam sic se habet ad patrem, ut verbum ad mentem}}; \udgl\ 
\item Noch deutlicher hat diese Idee der \RWbet{heil.\ Thomas von Aquin} entwickelt; wie auch
\item unter den Protestanten \RWbet{Melanchthon,} der in seinen \RWlat{locis theologicis}\RWlit{}{Melanchthon1} und in seinem \RWlat{corpus doctrinae christianae}\RWlit{}{Melanchthon2} sich auf folgende Weise ausdrückt: Das menschliche Vorstellungsvermögen schafft sich durch's Vorstellen ein Bild vom Gegenstande, aber in dieses Bild übertragen wir nicht unsere Wesenheit, es ist auch vorübergehend. \erganf{\RWlat{At pater aeternus, sese intuens, gignit cogitationem sui, quae est imago ipsius, non evanescens, sed subsistens, et communicata ipsi essentia.}} Das sey nun der Sohn, der eben darum \RWgriech{l'ogos} heiße, \erganf{\RWlat{quia cogitatione gignitur. Ut autem filius nascitur cogitatione, ita Spiritus sanctus procedit a voluntate patris et filii; voluntatis enim est agitare, diligere, sicut et cor humanum non imagines, sed spiritus seu halitus gignit.}}
\item Eine ähnliche Deduction nahm auch der Abt \RWbet{Nonnotte} in seinem philosophischen Wörterbuch (Art. Dreieinigkeit)\RWlit{}{Nonnotte1} auf; ingleichen \RWbet{Beda Meyr} in seiner Vertheidigung der natürlichen, geoffenbarten und katholischen Religion (2.\,Thl., 2.\,Abtheilung)\RWlit{}{Mayr1}, \RWbet{Michael Sailer} in seiner Theorie des weisen Spottes 1781\RWlit{}{Sailer3}, \RWbet{Jak.\ Frint} in seinem Handbuche der Religionswissenschaft (4.\,B.)\RWlit{}{Frint1} \uA
\item Etwas anders hat diese Idee G.~ E.~\RWbet{Lessing} (in seinem theologischen Nachlaß, Berlin, 1784, S.\,221\,ff.) dargestellt. Das vollkommenste Wesen mußte sich von jeher mit der Betrachtung seiner eigenen Vollkommenheiten beschäftigen. Vorstellen, Wollen und Schaffen ist bei ihm Eins. Aber er konnte sich selbst auf zweierlei Art denken: einmal als Inbegriff aller Vollkommenheiten, dann~\RWSeitenw{232}\ jede dieser Vollkommenheiten einzeln. Durch den ersten Gedanken schuf er von Ewigkeit ein Wesen (?), das mit ihm selbst gleich vollkommen war, den Sohn. Die Harmonie zwischen dem Vater und dem Sohne ist der heil.\ Geist, \usw\
\item Einen andern Versuch machte \RWbet{Johann Matzek} in seinem Beweise für das Daseyn Gottes\RWlit{}{Maczek1}, den \RWbet{Chrysostomus Pfrogner} in seinem Buche: über den Begriff der Selbstbeurtheilung\RWlit{}{Pfronger1}, und in andern Schriften, \zB\ über die menschliche Bildung, S.\,164\,ff., noch weiter ausbildete. In jedem Selbstbewußtseyn (Selbstbeurtheilung nennt es Pfrogner) eines denkenden Wesens liegen drei von einander verschiedene Vorstellungen, die Kraft des Vorstellens, der vorgestellte Inhalt, und die wirkliche Vereinigung der Beiden in das Selbstbewußtseyn. -- In Gott, der ebenfalls ein sich selbst beurtheilendes Wesen ist, muß jeder von diesen drei Vorstellungen ein reales Subject zu Grunde liegen, weil es in Gott keine bloße Fähigkeiten, sondern lauter Wirklichkeiten geben kann. Also ist in Gott 1.~ein \RWbet{Urheber}, der das ist, was die Kraft oder Fähigkeit des Erkennens bei einem endlichen Wesen ist; 2.~ein \RWbet{Gesetzgeber} oder das letzte Ziel zweckmäßiger Wirksamkeit; Gott erkennt nämlich in sich einen reellen Inhalt von unendlichem Werthe; 3.~ein \RWbet{Beseliger}, oder der Grund unendlicher Seligkeit; diese entspringt nämlich aus dem Bewußtwerden des unendlichen Werthes.
\item Nach \RWbet{G. Schlegel} (in seiner erneuerten Erwägung der Lehre von der Dreieinigkeit (2.\,Thl., Riga, 1791)\RWlit{}{Schlegel1} gibt es in Gott drei Urkräfte, die \RWbet{schaffende, erhaltende} und \RWbet{regierende} oder den Vater; die \RWbet{welterleuchtende} oder den Sohn; die \RWbet{weltverbessernde} oder den heil.\ Geist.
\item \RWbet{D.\ Mark} (über die Vernunftwidrigkeit einiger Lehren des gewöhnlichen Kirchensystems, Halle, 1792)\RWlit{}{Mark1} trägt folgende Erklärung vor, auf welche ihn die Schriften des \RWbet{J.~J.~Heß} geleitet haben. Der Vater ist der\RWbet{ durch die Natur} sich offenbarende Gott; der Sohn der \RWbet{durch}~\RWSeitenw{233}\ \RWbet{wunderbare Begebenheiten in der sichtbaren Welt;} der heil.\ Geist der \RWbet{durch wunderbare Einwirkungen in menschliche Seelen} sich offenbarende Gott. (S.\,23.)
\item \RWbet{Kant} (Religion innerhalb der Grenzen der bloßen Vernunft, 2.\,Aufl.\ S.\,212.)\RWlit{}{Kant4} meinte, die Idee der Dreieinigkeit sey aus der dreifachen Qualität des höchsten moralischen Oberhauptes \RWgriech{a}) als des Weltschöpfers und obersten Gesetzgebers des Menschengeschlechtes, \RWgriech{b}) als des gütigen Versorgers desselben, \RWgriech{g}) als des gerechten Richters entstanden; und liege in dem Begriffe eines Volkes, als eines gemeinen Wesens, worin eine solche dreifache Gewalt nothwendig als unter drei verschiedenen Subjecten vertheilt gedacht werden müsse.
\item Andere Freunde der kritischen Philosophie, \zB\ \RWbet{Tieftrunk} (Censur des christlich-protestantischen Lehrbegriffes, 2.\,Thl.\ S.\,31\,ff.)\RWlit{}{Tieftrunk1}, meinen, in der Dreieinigkeitslehre werde Gott \RWgriech{a}) als Vater zu seinen Kindern, und folglich durch das Prädicat der Liebe; \RWgriech{b}) als Weisheit, Gesetzgeber; \RWgriech{g}) als Heiligkeit oder Heiligmacher vorgestellt.
\item Der katholische Theolog \RWbet{Peutinger} (Religion, Offenbarung und Kirche. Salzburg, 1795.\ S.\,378.)\RWlit{}{Peutinger1} glaubte, die Trinitätslehre als nothwendige Vernunftwahrheit aufstellen zu können; denn die Vernunft müsse sich Gott \RWgriech{a}) als das realeste Wesen von jeher hervorbringend (Vater) denken, \RWgriech{b}) das Hervorgebrachte müsse abermals als das Allerrealeste, als Gott, gedacht werden (Sohn), \RWgriech{g}) und daher neuerdings etwas hervorbringen, und zwar die Einigung zwischen dem Vater und sich selbst (den heil.\ Geist).
\item \RWbet{Friedr.\ Köppen} (Philosophie des Christenthums. 2.~Thl. Leipzig, 1815. S.\,72.)\RWlit{}{Koeppen2} findet in der Dreieinigkeitslehre folgende Symbolisirung der Idee Gottes: Gott ist dem Christen der ewige Vater, der im Wechsel der Zeiten für seine Kinder sorgt. Die vollkommenste Ausführung des göttlichen Rathschlusses, den Menschen zu helfen, erkennet der Christ in der Erscheinung Jesu Christi, des Sohnes Gottes. Die Erhaltung dieser Anstalt bewirkt der heil.~\RWSeitenw{234}\ Geist. Ohne den Vater überall keine Anstalt, ohne den Sohn keine Gründung derselben in der Zeit, ohne den Geist keine Erhaltung derselben, \usw\
\end{aufzb}
\end{aufza}

\begin{RWanm} 
Zu bemerken ist noch, daß wir bei mehreren andern Religionen gleichfalls eine Art von Dreieinigkeitslehre antreffen, die jedoch freilich nirgends mit unserer christlichen ganz zu vergleichen ist. So sollen nach Jamblichus schon die alten \RWbet{Aegyptier} eine Art Dreiheit in Gott angenommen haben: einen Vater (Eiklon), ein Wort (Phthah), und einen Geist (Cneph). Auch bei den \RWbet{Persern} ließe sich eine Dreiheit nachweisen, indem sie über den Ormuzd und Ahriman (das gute und böse Princip) noch den Zeruane-akherene setzten. Die \RWbet{Indianer} unterscheiden gleichfalls ein Dreifaches in der Gottheit: den Schöpfer (Brahma), den Erhalter (Wischnu) und den Zerstörer (Schiwen). Die Religion des Foh bei den \RWbet{Chinesen} enthält nebst andern aus dem Christenthume entlehnten Lehren auch die Dreieinigkeitslehre. Daß unter den \RWbet{griechischen Philosophen} auch Pythagoras und Plato eine Dreiheit gelehrt, haben wir schon oben bemerkt. Dieß Alles mag denn hier nur zu einem Beweise dienen, daß die Annahme einer gewissen Dreiheit in Gott dem menschlichen Verstande eben nicht Gewalt anthue, wie Manche vorgegeben haben. \end{RWanm}

\RWpar{134}{Sittlicher Nutzen der Lehre von Gottes dreifacher Persönlichkeit}
\begin{aufza}
\item Schon dadurch, daß uns die christliche Lehre von Gottes dreifacher Persönlichkeit ein \RWbet{Mehreres} beibringt, als uns die bloße Vernunft über Gott zu sagen weiß, leistet sie uns einen doppelten Vortheil:
\begin{aufzb}
\item Sie macht, daß \RWbet{der Gedanke an Gott uns geläufiger} wird, \dh\ daß wir uns seiner öfterer und leichter erinnern. Je Mehreres wir nämlich von einem Gegenstande wissen, um desto mehrere Berührungspuncte hat die Erinnerung an ihn, um desto öfter und leichter denken wir an ihn zurück; indem ein jeder Gegenstand, dessen Vorstellung durch Aehnlichkeit oder auf irgend eine andere Art mit den Begriffen, die man uns von dem ersteren beigebracht hat, verwandt ist, durch das Gesetz der Ideenverknüpfung den Gedanken an ihn in uns zu~\RWSeitenw{235}\ erwecken vermag. Indem also die Lehre von Gottes dreifacher Persönlichkeit unsere Kenntniß von Gott vermehrt, so wird eben hiedurch auch bewirkt, daß wir an Gott öfter erinnert werden. Alle die zahlreichen Gegenstände, die irgend eine Aehnlichkeit oder Verwandtschaft mit den Begriffen haben, die in der christlichen Dreieinigkeitslehre vorkommen, \zB\ die Begriffe Vater, Sohn, Zeugung, Geist, Hauch, \usw\, sind fähig, uns an Gott zu erinnern.
\item Auch \RWbet{wichtiger} wird uns Gott durch diese Lehre. Einmal schon darum, weil sie die Erinnerung an ihn befördert; denn insgemein kommt uns ein Gegenstand um so wichtiger vor, je öfter wir uns an ihn zu denken veranlaßt fühlen. Ferner auch darum, weil wir die Wichtigkeit der Dinge gewöhnlich nach der Menge der Kenntnisse, die wir von ihnen haben, zu schätzen pflegen. Die meisten Menschen legen wohl nur aus dem Grunde so wenig Werth auf Gott, und halten es für geringe, ob sie ihm wohlgefällig oder mißfällig sind, weil es nur so äußerst Weniges ist, was sie von ihm gelernt haben und zu sagen wissen. Würden die Kenntnisse, die sie von Gott besitzen, keinen so unbeträchtlichen Theil von dem gesammten Umfange ihres Wissens ausmachen, würde die Zeit und Mühe, die sie ihr Leben hindurch auf die Ausbildung ihrer Begriffe von Gott verwandt haben, in einem größeren Verhältnisse mit dem Fleiße stehen, welchen sie der Erlernung so mancher anderer Wissenschaft schenkten: schon daraus würden sie schließen, daß Gott doch wahrlich kein so unwichtiger Gegenstand für sie seyn müsse. Die christliche Lehre von Gottes dreifacher Persönlichkeit bringt also dadurch, daß sie uns über Gott weit mehr sagt, als uns die bloße Vernunft zu sagen weiß, daß sie den Unterricht über Gott erweitert, die Wirkung hervor, daß uns Gott wichtiger wird. -- Nun ist es aber offenbar, je geläufiger und zugleich wichtiger uns der Gedanke an Gott gemacht wird, desto wirksamer kann er und muß er sich auch bei uns bezeigen.
\end{aufzb}
\item Da ferner der Unterricht, den uns die christliche Dreieinigkeitslehre über Gott ertheilt, von einer solchen Art~\RWSeitenw{236}\ ist, daß er uns Einiges eröffnet, Anderes aber verschweigt, \dh\ \RWbet{geheimnißvoll} ist, und verschiedene neue Fragen und Zweifel veranlaßt: so hat dieß den weiteren Nutzen, daß \RWbet{unsere Wißbegierde gespannt} wird, indem wir auch das uns noch Unbekannte zu erfahren wünschen. Wißbegierde aber, besonders aber eine auf Gott gerichtete Wißbegierde, ist in mehr als einer Rücksicht vortheilhaft für den Menschen. Sie zieht ihn von anderen gefährlicheren Begierden ab, sie lehrt ihn den Werth der übersinnlichen Wahrheiten gehörig schätzen, sie macht ihm Gott wichtiger, \usw\
\item Die Lehre von der Dreieinigkeit sagt uns, daß es in Gott drei Gründe zu drei verschiedenen Arten der Wirksamkeit gebe, und daß er in jeder dieser drei Arten zu wirken, also auf jede Art, wie er nur überhaupt zu wirken vermag, thätig gewesen sey zum Besten des ganzen Weltalls, zum Besten unseres Geschlechtes, ja auch zum Besten eines jeden Einzelnen aus uns. \RWbet{Wem sollte durch diese Vorstellung Gott nicht theurer werden?} Wer sollte nicht mit Erhebung den großen Werth fühlen, welchen das menschliche Geschlecht, ja jeder Einzelne aus uns durch diese Vorstellung erhält!
\item Daß diesen drei Gründen in Gott ein \RWbet{ewiges Daseyn} beigelegt wird, ist gleichfalls nicht ohne Nutzen für uns; denn dadurch werden uns die Wirkungen, die diese Gründe hervorbringen, wichtiger. Das Gegentheil, oder die Behauptung, daß der Sohn und der heil.\ Geist ihr Daseyn erst in der Zeit erhalten haben, hieße: nicht schon von Ewigkeit her, und vor der Weltschöpfung, sondern erst in der Zeit fing Gott an zu denken an das Wohl der Menschheit, und jedes Einzelnen aus uns, und eigene Anstalten dafür zu treffen.
\item Noch nothwendiger ist es, daß diesen drei Gründen in Gott \RWbet{Allwissenheit, Allmacht} und \RWbet{Heiligkeit} beigelegt wird. Das Gegentheil, oder die Behauptung, daß \zB\ der Sohn oder der heil.\ Geist nur eine endliche Macht oder Weisheit besitze, hieße, daß Gott für die Beförderung des Wohles der Menschheit und jedes Einzelnen aus uns nicht mit unendlicher, sondern mit einer endlichen Macht, Weisheit und Güte wirke.~\RWSeitenw{237}
\item Sind aber diese drei Gründe mit Weisheit und Güte wirkende Ursachen, so war es auch schicklich, sie \RWbet{Personen}, und weil die Weisheit, die Macht und die Güte, die sie besitzen, eine \RWbet{göttliche} ist, sie \RWbet{göttliche} Personen zu nennen. Auch wird durch diese Personificirung unsere Vorstellung derselben \RWbet{lebhafter}.
\item Die Behauptung, daß die drei Personen, obwohl die zweite und dritte ihr Daseyn von der ersten haben, einander doch \RWbet{am Range gleich} seyen, und daß man die letzteren Beiden nicht als untergeordnet betrachten dürfe, lehrt uns, daß das Geschäft der Beseligung eines jeden Einzelnen aus uns bei Gott sehr hoch geachtet werde, daß er nicht einen Einzigen aus uns, um wie viel weniger unser ganzes Geschlecht, der Beförderung des Wohles der gesammten Welt aufopfere.
\item Auch die Lehre: \RWbet{daß der Sohn sein Daseyn vom Vater, der heil.\ Geist aber sein Daseyn vom Vater und vom Sohne zugleich habe}, ist keine unfruchtbare Wahrheit. Sie erinnert uns, daß dasjenige, was Gott bestimmt, etwas zum Besten des menschlichen Geschlechtes zu veranstalten, seinen letzten Grund in dem habe, was ihn bestimmt, das Wohl aller lebendigen Geschöpfe überhaupt zu befördern; dasjenige aber, was ihn bestimmt, etwas zum Wohle eines Einzelnen aus uns zu thun, in seiner Fürsorge für das ganze Menschengeschlecht sowohl, als auch in seiner Fürsorge für alle geschaffenen Wesen überhaupt gegründet sey. Und daraus folgt, daß wir erst dann recht zuversichtlich hoffen können, Gott werde viele Anstalten für unsere eigene Rettung und Beseligung treffen, wenn wir gesonnen sind, unser eigenes Wohl als ein Mittel zur Beförderung des Wohles der ganzen Menschheit, ja aller lebendigen Wesen überhaupt, bestmöglich anzuwenden.
\item Der Name \RWbet{Vater}, den die christliche Dreieinigkeitslehre dem Ersten in Gott, oder demjenigen beilegt, was diese Welt erschaffen hat, erhält und regieret, (oder was überhaupt alle Anstalten trifft, die zur Beförderung des Wohles aller lebendigen Wesen erforderlich sind) erzeugt in uns die Vorstellung, daß Gott gegen alle seine Geschöpfe, also auch~\RWSeitenw{238}\ gegen uns, väterliche Gesinnungen hege. Und weil dasjenige in Gott, welches der \RWbet{Sohn} genannt wird, mit dem Menschen Jesu, also mit einem aus unsern Brüdern eine persönliche Vereinigung eingegangen ist: wie nahe werden da nicht auch wir zu Gott hinangehoben! Hier heißt es wahrlich: \erganf{Wir sind verwandter Natur mit Gott, sind göttlichen Geschlechtes.} \RWbibel{Apg}{Apostelg.}{17}{28}
\item Die Dreieinigkeitslehre sagt, es gebe ein inneres Verhältniß zweier Subjecte in Gott, das sich nicht wahrer ausdrücken läßt, als durch das \RWbet{bildliche Verhältniß eines Vaters zu seinem Sohne,} und dieß zwar eines solchen Vaters, der seinen Sohn auf das Innigste liebt, und eines Sohnes, der seinen Vater nicht nur liebt auf das Innigste, sondern ihm auch gehorsam ist; \usw\ Wie ganz geeignet ist nicht dieses schön menschliche Bild von Gott, uns herzliches Vertrauen zu diesem Gotte einzuflößen, und jede unkindliche Furcht von dem Gedanken an ihn zu entfernen! Wie ehrwürdig wird uns durch diese Lehre nicht der Vaterstand, da Gott selbst den Namen eines Vaters anzunehmen sich würdiget! Und welch ein lehrreiches Bild der Eintracht und Liebe, die zwischen Vater und Sohn herrschen soll! Der göttliche Sohn ist dem Vater gehorsam bis zum Tode!
\item Wie ehrwürdig, und wir dürfen wohl sagen, in welch ein heiliges Gewand gehüllt, erscheint der \RWbet{Begriff der Zeugung}, wenn Gott die Anwendung dieses Begriffes auf sein eigenes Wesen erlaubt! Wie werden nicht alle leichtsinnigen Nebenbegriffe von diesem an sich so ehrwürdigen Gegenstande durch eine solche Heiligung desselben verscheucht!
\item Eben so anständig und erhaben ist auch das Bild, \RWbet{daß der heil.\ Geist der Aushauch jener gemeinschaftlichen Liebe ist, welche den Vater und den Sohn vereint}.
\item Indem die christliche Dreieinigkeitslehre uns mit diesen drei Personen in Gott, die uns so viele Wohlthaten erweisen, und von denen wir noch so viele andere in alle Ewigkeit hin zu erwarten haben, bekannt macht, legt sie uns auch die \RWbet{Pflicht der dankbaren Verehrung derselben} und ihrer Anrufung um neue Gnade auf. Erfüllen wir~\RWSeitenw{239}\ nun diese Pflicht: so gewinnt abermals unsere Tugend, und selbst unsere Glückseligkeit. Auch ist zu bemerken, daß das Geschäft des Gebetes, dieses gewiß so wichtigen Beförderungsmittels der Tugend, durch die Lehre von Gottes dreifacher Persönlichkeit mehr Mannigfaltigkeit und Abwechslung erhält, indem wir uns bald an die Eine, bald an die andere Person in Gott vornehmlich wenden, und eben hiedurch uns die Aufmerksamkeit und Sammlung des Gemüthes erleichtern.
\item Der hauptsächlichste Nutzen der Lehre von Gottes dreifacher Persönlichkeit aber ist ohne Zweifel der, \RWbet{daß sie uns zum Vereinigungspuncte und zum deutlicheren Verständnisse so vieler anderer, ja beinahe aller Lehren des Christenthums dient;} und dieses ist eben der Grund, um dessentwillen Jesus verordnete, daß ihrer bei der Taufe erwähnt werde. Schon die bloße Aussprache des Namens Vater erinnert uns an die wichtige Lehre von Gottes Vatersinne, und die damit verwandte von aller Menschen Brüderschaft, und wesentlicher Gleichheit. Die bloße Aussprache des Namens Sohn erinnert uns an die durch Jesum (dem dieser Name gleichfalls beigelegt worden ist) bewirkte und noch ferner zu bewirkende Erlösung und Beseligung des menschlichen Geschlechtes; und die bloße Aussprache des Namens Geist erinnert uns an die Erleuchtung und Heiligung und alle andern Wohlthaten, die jedem Einzelnen aus uns durch diesen Geist zu Theil geworden sind, und noch immer werden sollen.
\end{aufza}

\begin{RWanm} 
Wohl dürfte man jeden Weltweisen auffordern, daß er, wofern er es vermag, uns eine andere Vorstellungsart von Gott angebe, welche sich fruchtbarer beweisen könnte, als diese katholische Lehre von der Dreieinigkeit. Gewiß wird es keiner vermögen. 
\end{RWanm}

\RWpar{135}{Wirklicher Nutzen dieser Lehre}
Jeder Vernünftige wird voraussetzen, daß von so viel möglichen Vortheilen, die ich im vorigen Paragraph in der Kürze angedeutet habe, gewiß sehr viele in Wirklichkeit Statt gefunden haben. Aber diese Lehre hat auch manchen Schaden angerichtet, und da dürfte es Einigen scheinen, daß die\RWSeitenw{240}ser weit größer, als ihr Nutzen gewesen. Man möchte nämlich den \par
\RWbet{Einwurf} machen: Die Lehre von der Dreieinigkeit habe 
\begin{aufzb}
\item zu \RWbet{allerlei grobsinnlichen Begriffen von Gott}, und zu einer Art von Tritheismus, der nicht viel besser als heidnischer Polytheismus sey, Anlaß gegeben. Das Wort Person verleitete nämlich unzählige Menschen, sich in Gott drei von einander getrennte, abgesondert bestehende Wesen (wie etwa drei Menschen) vorzustellen.
\item Noch wichtiger ist es, daß diese Lehre die unglückliche \RWbet{Veranlassung zu einer Menge von Streitigkeiten} geworden ist, die mit der größten Erbitterung und mit Hintansetzung aller christlichen Liebe durch viele Jahrhunderte hindurch geführet worden sind.
\end{aufzb}\par
\RWbet{Antwort.} Es läßt sich nicht Alles, was man in diesem Einwurfe behauptet, läugnen; wohl aber läßt sich zweierlei entgegnen.\par
\RWbet{Erstlich}, daß man das Unheil, das diese Lehre angerichtet haben soll, in der Vorstellung leicht übertreibe, indem man nicht weiß, ob es nicht ohne sie noch weit schlimmer gewesen wäre. -Grobsinnliche Begriffe sollen sich die Christen von Gott gebildet haben? -- Zugegeben; aber was wäre der Fall gewesen, wenn diese Christen die Lehre von der Dreieinigkeit nicht gehabt hätten? -- Wie drei für sich bestehende Wesen soll sich der große Haufen die drei Personen vorgestellt haben? -- Sey es; ist aber dieser Irrthum so schädlich, als man ihn darzustellen sucht? -- Viele Streitigkeiten sollen durch die Dreieinigkeitslehre verursacht worden seyn? -- Wahr; aber wären diese Streitigkeiten nicht vielleicht auch entstanden, wenn es nie eine Dreieinigkeitslehre gegeben hätte? -- Viele Verfolgungen und Kriege hat sie veranlaßt? -- Nicht eigentlich sie, sondern die menschlichen Leidenschaften.\par
\RWbet{Zweitens} vergißt man das Gute, welches aus diesen Uebeln hervorgegangen ist. Die Untersuchungen, welche die christliche Dreieinigkeitslehre verursacht hat, haben sicher viel beigetragen zur Ausbildung des menschlichen Verstandes. Die Begriffe Natur und Wesen, Grund und Folge, Kraft, Per\RWSeitenw{241}son, \usw\ sind durch die Lehre von der Dreieinigkeit entwickelt worden. Wer mag die Wichtigkeit dieses Vortheils nach seinem wahren Werthe schätzen können? -- So haben auch die Verfolgungen, wozu die Lehre von der Dreieinigkeit Veranlassung gab, viel sittlich Gutes erzeugt. Oder gaben sie nicht Gelegenheit zu der Heldenthat, für seinen (gleichviel ob richtiger oder irriger Weise) für wahr gehaltenen Glauben Blut und Leben zu wagen? Wurde der großen Menge der Menschen, wenn Einige wegen verschiedener Begriffe über Gott einander auf Tod und Leben verfolgten, nicht um so anschaulicher, wie wichtig die Lehre von Gott sey? --

\RWpar{136}{Die Lehre des Christenthums von Gottes Rathschlüssen}
\begin{aufza}
\item In wiefern das göttliche Willensvermögen in ein wirkliches Wollen übergeht, heißt es ein \RWbet{Rathschluß} (\RWlat{decretum, consilium divinum}).
\item Alles, was immer war, ist und werden wird in der Welt, geschieht nach Gottes Rathschlüssen.
\item In \RWbet{relativer} Hinsicht also, in wiefern wir der geschehenen Dinge mehrere in der Welt sehen, unterscheiden wir auch mehrere Rathschlüsse Gottes.
\item Von diesen Rathschlüssen Gottes gelten im Allgemeinen folgende Eigenschaften. Sie sind
\begin{aufzb}
\item frei,
\item höchst weise und heilig,
\item für den Menschen (und jedes endliche Wesen) nie ganz erforschlich,
\item ewig und unabänderlich,
\item gehen allezeit in Erfüllung.
\end{aufzb}
\item Die wichtigsten Eintheilungen derselben sind:
\begin{aufzb}
\item Einige Rathschlüsse Gottes sind \RWbet{unbedingt}, andere \RWbet{bedingt. Bedingte} Rathschlüsse heißen diejenigen, vermöge derer ein Wesen gewisse Schicksale erfährt, deren alleiniger oder doch hauptsächlichster Grund \RWbet{in seinen freien Handlungen} selbst liegt. Die übrigen heißen \RWbet{unbedingt}. Auch diese haben sonach einen Grund;~\RWSeitenw{242}\ aber er liegt entweder in Gott allein, oder wohl auch in den Handlungen gewisser anderer Wesen, nur nicht desjenigen, das diese Rathschlüsse eben betreffen.
\item Rathschlüsse Gottes, vermöge deren etwas, das \RWbet{an sich gut} ist, erfolgt, werden \RWbet{wirkende}; solche, vermöge deren etwas, das \RWbet{an sich böse} ist, erfolgt, werden nur \RWbet{zulassende} genannt.
\end{aufzb}
\item Jener Rathschluß Gottes, zu Folge dessen gewissen Geschöpfen die ewige Seligkeit ertheilt wird, heißt die \RWbet{Vorherbestimmung} (Erwählung, Berufung, \RWlat{praedestinatio}, \RWlat{electio}, \RWlat{vocatio}). Es ist keine eigentliche Glaubenslehre, wird aber doch von den Meisten behauptet, daß dieser Rathschluß ein \RWbet{unbedingter}, und aus bloßer freier Gnade, nicht aus Vorhersehung der Verdienste entsprungener Rathschluß sey.
\item Jener Rathschluß Gottes, zu Folge dessen er einige Geschöpfe von der Seligkeit ausschließt, und sie zur ewigen Strafe verurtheilt, heißt die \RWbet{Verwerfung} (\RWlat{reprobatio}). Von diesem Rathschlusse, besonders in wiefern er die Zuerkennung der ewigen Strafe einschließt (\RWlat{reprobatio positiva}), ist es Glaubenssache, daß er nur ein\RWbet{ bedingter}, erst aus Vorhersehung der Sünde entsprungener Rathschluß sey.
\end{aufza}

\RWpar{137}{Historischer Beweis dieser Lehre}
\begin{aufza}
\item[1.~u.~2.]\stepcounter{enumi}\stepcounter{enumi} Der heil.\ Paulus schreibt an die \RWbibel{Eph}{Ephes.}{1}{11}: \erganf{Durch ihn (Jesum) ist uns das herrliche Loos des Christenthums zu Theil geworden, nachdem wir dazu vorherbestimmt waren durch den Rathschluß desjenigen (Gottes), der \RWbet{Alles nach dem Rathschlusse seines Willens ausführt}.} Also geschieht Alles, was geschieht, nur nach dem Rathschlusse Gottes, und dieser Rathschluß ist das wirkliche Wollen des göttlichen Willensvermögens.
\item Daß man in der christlichen Kirche in relativer Hinsicht mehrere Rathschlüsse in Gott annehme, wird aus dem Folgenden von selbst erhellen.
\item Gottes Rathschlüsse sind
\begin{aufzb}
\item \RWbet{frei}. In dem Briefe an die \RWbibel{Eph}{Ephes.}{1}{5}\ kommt der Ausdruck vor: \RWbet{nach dem Belieben des Willens.}~\RWSeitenw{243}\ Hieher gehören auch alle die übrigen Stellen, aus welchen die Freiheit des göttlichen Willens oben erwiesen wurde;
\item \RWbet{höchst weise und heilig}. Im B.~d.~\Ahat{\RWbibel{Weish}{Weish.}{11}{20}}{11,22.}\ heißt es: \erganf{Du ordnest Alles nach Maß, Zahl und Gewicht; die Allgewalt steht dir immer zu Gebote; wer kann der Macht deines Armes widerstehen? Aber du erbarmest dich Aller und hast Nachsicht mit den Sünden der Menschen, damit sie sich bessern}, \usw ;
\item \RWbet{unerforschlich}. \RWbibel{Röm}{Röm.}{11}{33}: \erganf{O Tiefe der Weisheit und der Erkenntniß Gottes! wie \RWbet{unerforschlich} sind seine Rathschlüsse, und wie \RWbet{unergründlich} seine Wege! Wer hat den Sinn des Herrn erkannt? oder wer ist sein Rathgeber gewesen?}
\item \RWbet{ewig und unveränderlich}. \RWbibel{Eph}{Ephes.}{1}{4}: \erganf{Er hat uns zu Christen erwählt vor der Weltschöpfung.} \RWbibel{Apg}{Apstg.}{15}{18}: \erganf{Gott wußte von Ewigkeit her, was er thun wollte.} \RWbibel{Ps}{Psalm}{33}{10}: \erganf{Der Herr zerstört der Heiden Rathschluß; vereitelt der Völker Entwürfe. Aber ewig besteht des Ewigen Rathschluß; der Entschluß seines Herzens für und für};
\item \RWbet{gehen allezeit in Erfüllung}. \RWbibel{Jes}{Isai.}{14}{24}: \erganf{Jehova, der Gott des Weltalls, schwur und sprach: Fürwahr, es soll geschehen, wie ich beschloß; wie ich mir vornahm, bleibt es. Dieß ist der Rath, beschlossen über alle Welt. Der Allmächtige ist's, der es beschloß; wer wird es vereiteln? Seine Hand ist ausgestreckt, wer treibt sie wohl zurück?}
\end{aufzb}
\item Die Eintheilungen der Rathschlüsse Gottes in \RWbet{bedingte} und \RWbet{unbedingte, wirkende} und \RWbet{zulassende} sind viel zu wissenschaftlich, als daß man sie in der Bibel nachweisen könnte. In den Schriften der Kirchenväter und der scholastischen Theologen aber kommen sie häufig vor. Daß die Eintheilung der göttlichen Rathschlüsse wirklich in dem Sinne genommen wurde, welchen ich oben aufstellte, gehet nicht sowohl aus den gegebenen Erklärungen, als vielmehr aus dem Gebrauche, den die Theologen von diesen Begriffen machen, deutlich genug hervor. In Rücksicht der~\RWSeitenw{244}\ zweiten Eintheilung aber ist die Redensart, daß Gott das Gute wirke, das Böse nur zulasse, allgemein bekannt.
\item Daß Jene, welche zur ewigen Seligkeit gelangen, dazu bestimmt und auserwählt sind, beweisen \zB\ folgende Stellen. Jesus spricht bei \RWbibel{Lk}{Luk.}{10}{20}: \erganf{Freuet euch; denn eure Namen sind im Himmel angeschrieben.} \RWbibel{Röm}{Röm.}{8}{30}: \erganf{Die er vorherbestimmt hat, diese beruft er auch, macht sie gerecht, und verherrlichet sie.} Diejenigen, welche behaupten, daß die Erwählung zur ewigen Seligkeit ein unbedingter, nicht aus Vorhersehung der Verdienste entsprungener Rathschluß sey, führen besonders folgende Gründe an:
\begin{aufzb}
\item Der heil.\ Apostel Paulus beruft sich in seinem Briefe an die \RWbibel[Römer]{Röm}{}{9}{}\ im 9ten Kapitel auf das Beispiel Jakob's und Esau's, deren den Einen Gott ohne alle Verdienste dem Andern vorgezogen, und auf andere dergleichen Beispiele, um zu beweisen, daß die Judenchristen kein Recht hätten, sich vor den Heidenchristen einen Vorzug beizulegen; er sagt unter Anderem \RWbibel{Röm}{}{9}{15}, Gott habe schon zu Moses gesagt: \erganf{Ich beweise Gnade, wem ich will, und Erbarmen, wem ich will.} (\RWbibel{Ex}{2\,Mos.}{33}{19})
\item Kinder, die gleich nach empfangener Taufe sterben, gelangen, nach der christlichen Lehre, zum Genusse der außerordentlichen, für die Christen bestimmten Seligkeit, und haben doch nicht das mindeste Verdienst. Also ist jener Rathschluß Gottes, durch welchen sie zu jener Seligkeit gelangen, nicht aus dem vorhergesehenen Verdienste entsprungen.
\item Die Rathschlüsse Gottes, in Hinsicht auf Beseligung der Menschen heißen unerforschlich. Das könnten sie aber nicht heißen, wenn sich die Beseligung der Menschen auf die Vorhersehung ihrer Verdienste gründete.
\item Der heil.\ Augustin behauptete ausdrücklich eine \RWlat{praedestinationem ad gloriam mere gratuitam.}
\end{aufzb}
\item Daß aber der göttliche Rathschluß der Verwerfung, wenigstens der positive, kein unbedingter sey, sondern auf der Vorhersehung der Unbußfertigkeit beruhe, ist eine Lehre, der für's Erste die Bibel in jener Stelle (\RWbibel{Röm}{Röm.}{9}{}), wenigstens nicht ausdrücklich widerspricht. Im Gegentheile können so~\RWSeitenw{245}\ manche andere Stellen der Bibel angeführt werden, aus welchen zu ersehen ist, daß Gott nur darum verdamme, weil der Mensch lasterhaft ist; \zB\ \RWbibel{Ez}{Ezech.}{33}{11}: \erganf{Ich will nicht den Tod des Sünders, sondern seine Besserung.} \RWbibel{Weish}{Weish.}{11}{[24]}: \erganf{Du hassest nichts von dem, so du geschaffen hast.} Der heil.\ Augustin war der Meinung, daß selbst der negative Rathschluß der Verwerfung ein bloß bedingter, von der Vorhersehung der Sünde abhängiger Rathschluß sey. Der zweite Kirchenrath zu Orange (\RWlat{Concil. Arausicanum 2.} im J.\ 529) schreibt \RWlat{canone\ 25.: \erganf{Aliquos ad malum divina potestate praedestinatos non solum non credimus, sed etiam si sint, qui tantum malum credere velint, cum omni detestatione in illos anathema dicimus.}} Im neunten Jahrhunderte haben mehrere gegen Gottschalk gehaltene Provincialconcilien dasselbe entschieden.
\end{aufza}

\RWpar{138}{Vernunftmäßigkeit}
\begin{aufza}
\item Daß das göttliche Willensvermögen nicht durchaus ein Vermögen bleibe, sondern in Rücksicht auf einzelne Gegenstände in ein wirkliches Wollen übergehe, versteht sich von selbst. Dieses Wollen kann man nun immer einen Rathschluß nennen (\RWlat{decretum, consilium}), da es ja ein vernünftiges (aus vernünftigen Vorstellungen hervorgehendes) Wollen ist.
\item Alles, was immer war, ist und werden wird in dieser Welt, geschieht nach diesem Rathschlusse Gottes; denn da wir Gott Verstand und Willen beilegen, so sagen wir wohl mit Recht, er wirke Alles, was er wirkt, mit Bewußtseyn und nicht gezwungen, sondern nach seinem Willen, mithin (nach obiger Redensart) nach seinem Rathschlusse. Da wir uns unter diesem Gott ferner den Inbegriff Alles dessen, was eine unbedingte Wirklichkeit hat, denken: so müssen wir nothwendig behaupten, daß Alles, was immer war, ist und werden wird in der Welt (\dh\ im Reiche des Zufälligen), durch seine Wirkung geschehe, mithin nach seinem Rathschlusse.
\item Hat keine Schwierigkeit.
\item Eben so wenig; denn diese Eigenschaften, welche wir oben dem göttlichen Willen beilegten, gelten eigentlich nicht~\RWSeitenw{246}\ von dem bloßen Vermögen zu wollen, sondern von seinem eigentlichen Willen, also von Gottes Rathschlüssen.
\item Die Eintheilung der göttlichen Rathschlüsse
\begin{aufzb}
\item in unbedingte und bedingte hat, so verstanden, wie ich sie oben darstellte, einen ganz richtigen Eintheilungsgrund. Gott nimmt bei den Rathschlüssen, welche er in Betreff auf ein bestimmtes Wesen faßt, \dh\ bei den Schicksalen, die er demselben zudenkt, allerdings Rücksicht auf alle Eigenschaften, folglich auch auf das sittliche Verhalten des Wesens; aber doch muß der Grund seines Beschlusses nicht immer ganz in dem Wesen, sondern er kann nach Umständen bald mehr, bald weniger auch in anderen Wesen liegen. Also gibt es sowohl bedingte, als unbedingte Rathschlüsse Gottes.
\item Auch die Eintheilung in wirkende und zulassende Rathschlüsse läßt sich rechtfertigen, obgleich die Namen uneigentlich sind.
\end{aufzb}
\item Daß es auch eine Vorherbestimmung zur ewigen Seligkeit gebe, erhellet aus Nr. 1. und 2., wenn hier unterdessen als bekannt vorausgesetzt wird, daß nach der Lehre des Christenthums einige Menschen nach dem Tode zum Genusse einer endlosen Seligkeit gelangen. Das Fernere in diesem Puncte ist zwar nicht Glaubenssache, mir aber scheint es doch richtig, und eine bloße Folgerung aus der katholischen Lehre von der Nothwendigkeit der Gnade zu jedem guten Werke zu seyn; denn wenn zu jedem guten Werke erst Gottes Gnade nöthig ist: so muß Gott Jedem, der zum Genusse der ewigen Seligkeit gelangen soll, selbst in dem Falle, wenn er sie seiner Werke wegen gewisser Maßen verdienen soll, erst seine Gnade geben. Es fragt sich also nur, ob der letzte Beweggrund, weßhalb Gott einem Menschen jene Gnade mittheilt, der sey, daß er tugendhaft, oder, daß er durch Tugend am Ende selig werde? Da nun das oberste Sittengesetz kein anderes Verhalten von Gott verlangt, als nur ein solches, dadurch die möglichst größte Summe der Glückseligkeit in der Welt zu Stande komme: so ist offenbar, daß auch der letzte Grund, weßhalb Gott einem Menschen gewisse Gnaden mittheilt, nicht sowohl darin liege, damit \Ahat{er}{es}~\RWSeitenw{247}\ durch diese Gnaden tugendhaft, als vielmehr darin, damit \Ahat{er}{es} durch diese Tugend glückselig werde, und glückselig mache, also in der Beförderung der Glückseligkeit Aller.
\end{aufza}\par
\RWbet{Einwurf.} Wenn es eine Vorherbestimmung zur Tugend und Glückseligkeit gibt, so ist der Mensch nicht frei.\par
\RWbet{Antwort.} Dieser Einwurf rührt von der Mehrdeutigkeit des Wortes Bestimmung her.
\begin{aufzb}[1.]
\item Sehr häufig, ja am gewöhnlichsten versteht man unter der Bestimmung einer Sache die Festsetzung ihrer Eigenschaften durch Festsetzung ihres Grundes. So sagt man \zB , ein Dreieck ist bestimmt durch zwei Seiten und den eingeschlossenen Winkel, \dh\ alle Eigenschaften desselben sind festgesetzt, wenn man zwei Seiten und den eingeschlossenen Winkel festsetzt.
\item Bei freien vernünftigen Wesen versteht man unter der Bestimmung derselben oft eine Wirkung (Zweck), deren Realisirung sie sich nach Möglichkeit vorsetzen sollen, oder deren möglichste Realisirung ihre Pflicht ist. In dieser Bedeutung sagt man, es sey die Bestimmung des Arztes, dem Kranken aufzuhelfen; in dieser Bedeutung sagt man auch, alle Menschen seyen zur Tugend und Glückseligkeit bestimmt.
\item In keiner von diesen beiden Bedeutungen wird das Wort Bestimmung hier genommen. Hier ist die Rede von der \RWbet{kosmischen} Bestimmung einer Sache, \dh\ von dem, was eine Sache ist und wird, wiefern man sich vorstellt, daß sie mit Gottes Wissen und Willen das sey und werde. Dieser Wille Gottes ist aber nicht überall die nöthigende Ursache von dem Seyn und Werden des Dinges; sondern nur die Bedingung (\RWlat{conditio sine qua non}). Wenn man \zB\ sagt, diesem Menschen ist es bestimmt, morgen zu sterben: so will dieses nur so viel sagen, es werde morgen geschehen, und zwar mit Gottes Wissen und (wenn nicht wirkendem, so doch zulassendem) Willen; keineswegs aber will man behaupten, daß Gottes Wille die \RWbet{nöthigende} Ursache seines Todes seyn werde; vielmehr kann dieser Tod vielleicht durch einen freien Willensentschluß dieses Menschen herbeigeführt werden, \zB\ durch einen Zweikampf, den Gott nur nicht verhindert, weil die Verhinderung desselben ein noch größeres Uebel wäre. Diese Bedeutung nun ist es, in welcher die katholische Kirche~\RWSeitenw{248}\ in ihrer Lehre von Gottes Rathschlüssen sagt, daß jede gute Handlung und auch das Seligwerden der Menschen durch Gottes Vorherbestimmung erfolge. Also wird durch diese Lehre die Freiheit des Menschen gar nicht geläugnet.
\end{aufzb}
\begin{aufza}[7.]
\item Daß es auch einen \RWbet{Rathschluß zur ewigen Verwerfung} gebe, erhellet eben so. Dieser Rathschluß kann aber nur ein \RWbet{bedingter,} \dh\ nur aus Vorhersehung der Sünde entsprungener Rathschluß seyn; denn glücklich machen darf Gott auch ohne Verdienst, darf zwar nicht solche, die Strafe verdienen, für immer glücklich machen; aber doch muß der Grund, weßhalb er ihnen diesen oder jenen Grad der Glückseligkeit zuführt, nicht immer in ihrem Verdienste liegen. Anders aber verhält es sich mit dem Unglücke. Unglück und ewiges Unglück darf Gott, vermöge seiner Gerechtigkeit, über kein freies Wesen verhängen, als wenn und weil es lasterhaft ist.
\end{aufza}\par
\RWbet{Einwurf.} Wenn der göttliche Rathschluß der Erwählung unbedingt ist: so ist auch jener der Verwerfung unbedingt. Und wenn im Gegentheile dieser bedingt ist, so muß es auch jener seyn; denn jene, die Gott zur ewigen Seligkeit nicht bestimmt, bestimmt er ja schon eben darum zur ewigen Verdammniß.\par
\RWbet{Antwort.} 
\begin{aufzb}[1.]
\item Gesetzt, dieß wäre wahr; so würde hieraus nichts Anderes folgen, als daß man, um diese Lehre von jedem Widerspruche zu retten, nur das aufgeben müsse, was in Nr.\,6. nicht Glaubenslehre ist.
\item In der That aber ist es ein übereilter Schluß, daß, wenn der Rathschluß der Erwählung unbedingt ist, auch jener der Verwerfung unbedingt seyn müsse, weil ja diejenigen, die nicht erwählet werden, schon eben darum verworfen würden. Die Sache verhält sich nämlich so. Diejenigen, welche verworfen werden, werden nur darum verworfen, weil sie es um ihrer Werke willen verdienen, und somit ist der Rathschluß Gottes, der sie verwirft, ein bedingter. Nun gibt es aber auch Andere, welche die ihnen angetragenen Gnaden, etwa weil es besonders große waren, besser gebrauchen, und also am Ende selig werden. Daß ihnen diese besonders großen Gnaden angetragen wurden, geschah zum Wenigsten \RWbet{nicht bloß}~\RWSeitenw{249}\ wegen des vorhergesehenen bessern Gebrauches, den sie von ihnen machen würden; sondern um anderer Gründe willen, und also aus einem unbedingten Rathschlusse Gottes. Die Ersten haben darüber nicht zu klagen, daß ihnen nicht eben so große Gnaden zu Theil geworden sind; denn sie haben diejenigen Gnaden, welche sie wirklich empfingen, nicht gehörig benützt. Und auch Gott ist darüber nicht anzuklagen, weil es der Zusammenhang des Ganzen nicht gestattet, daß er Allen gleich große Gnaden ertheile. Ein Gleichniß wird diese Sache noch deutlicher machen. Man denke sich einen Erzieher, der allen seinen Zöglingen einen sehr guten Unterricht ertheilt, etlichen Wenigen aber, nicht gerade denjenigen, von welchen er besondere Fortschritte vorhersieht, sondern die etwa der Zufall in seine Nachbarschaft geführt hat, ertheilt er noch einen besonderen Unterricht in seinen Nebenstunden. Die Ersteren gerathen ohne des Lehrers Verschulden in ihren freien Stunden auf schädliche Beschäftigungen, und gehen zu Grunde. Diejenigen aber, denen er den besondern Unterricht ertheilt hatte, widerstehen glücklich allen Verführungen zum Bösen, und werden wackere Männer. Können die Ersten ihren Untergang dem Lehrer zur Last legen? Oder verdient er Tadel, daß er nur Einige, nicht Alle besonders unterrichtet hat, da ihm das Letztere nicht möglich war? -- Die Anwendung auf Gott ist leicht.
\end{aufzb}


\RWpar{139}{Sittlicher Nutzen}
\begin{aufza}
\item[1.~und~2.]\stepcounter{enumi}\stepcounter{enumi} Daß Alles, was immer war, ist und werden wird in der Welt, nach Gottes Rathschlusse geschieht, muß man sich nothwendig denken, wenn der Gedanke an Gott den Nutzen gewähren soll, den er seiner Natur nach gewähren kann. Nämlich nur dann erst, wenn wir glauben, daß nichts in der Welt wider und ohne den Willen Gottes da sey, werden wir \RWbet{mit allen Einrichtungen der Welt zufrieden,} weil wir jetzt wissen, daß Alles zum Besten der lebendigen Wesen sey und erfolge.
\item Zur leichteren Faßlichkeit für uns dient es, uns Gottes Rathschlüsse so vielfältig zu denken, als vielfältig die Wirkungen derselben in der Erscheinung sind.~\RWSeitenw{250}
\item Aber freilich wird hiezu noch erfordert, daß wir den göttlichen Rathschlüssen die Eigenschaften der Freiheit, Heiligkeit \usw\ beilegen.
\item Der Nutzen der Eintheilung der göttlichen Rathschlüsse in \RWbet{bedingte} und \RWbet{unbedingte} zeigt sich in dem gleich Folgenden. Die Eintheilung in \RWbet{wirkende} und bloß \RWbet{zulassende} Rathschlüsse aber ist deßhalb nothwendig, weil es höchst anstößig wäre, wenn wir aus Unbekanntschaft mit diesem Unterschiede meinen und sagen würden, daß Gott auf eben die Art der letzte Grund des Bösen in der Welt sey, wie er der letzte Grund des Guten in ihr ist.
\item Die Meinung, daß Gottes \RWbet{Rathschluß der Erwählung unbedingt} sey, \dh\ nicht von dem eigenen Verdienste des Erwählten abhänge, scheint, wenn sie wohl verstanden wird, gleichfalls nicht ohne Nutzen zu seyn. Es demüthiget nämlich den Menschen, und verwahrt ihn vor allem Stolze, wenn er erwägt, daß es nicht die Vorhersehung seiner Verdienste sey, weßhalb ihm Gott die Gnaden mittheile, durch deren Annehmung er zur ewigen Seligkeit gelangt.
\begin{RWanm} 
Durch Mißverstand aber kann diese Lehre manchmal auch Schaden hervorbringen:
\begin{aufzb}
\item Glaubt man, der Rathschluß der Erwählung sey darum unbedingt, weil er nicht von dem Verdienste des zu Erwählenden abhängt: so macht man sich den unwürdigen Begriff von Gott, daß er \RWbet{aus bloßem Belieben} handle.
\item Schließt man daraus, weil Gottes Rathschluß der Erwählung nicht von dem Verdienste des zu Erwählenden abhängt, man könne \RWbet{auch ohne Verdienste selig} werden: so kann dieß zur sträflichsten Gleichgültigkeit verleiten.
\item Glaubt man, die Lehre von Gottes unbedingter Erwählung habe zur nothwendigen Folge, daß auch der göttliche Rathschluß der Verdammung unbedingt seyn müsse: so macht man sich einen Begriff von Gott, der völlig geeignet ist, Entsetzen und Abscheu zu erregen.
\item Befürchtet man hiebei noch vollends, selbst unter der Zahl dieser Verworfenen zu seyn: so muß Schwermuth und Verzweiflung entstehen.~\RWSeitenw{251}
\end{aufzb}\par
Es läßt sich nun denken, daß, wenn jene Meinung zur Glaubenslehre erhoben worden wäre, die jetzt erwähnten Mißverständnisse derselben so oft eingetreten, und der aus ihnen entsprungene Schaden so groß geworden wäre, daß er den Nutzen derselben überwogen hätte. Daher kann man die Kirche nicht tadeln, daß sie hierüber nicht entscheiden wollen. 
\end{RWanm}
\item Im Gegentheile aber, daß Gottes \RWbet{Rathschluß der Verwerfung nur ein bedingter} sey, ist eine Entscheidung von offenbar wohlthätigem Einflusse. Der entgegengesetzte Gedanke, daß Gottes Rathschluß der Verwerfung unbedingt wäre, müßte Entsetzen und Abscheu erregen. Ein Gott, der ein Geschöpf, ein freies vernünftiges Wesen, wenn gleich nicht aus willkürlichen, doch aus Gründen, die nicht in der sittlichen Bösartigkeit desselben liegen, ewig verwerfen kann -- wer schaudert nicht vor einem solchen Gott? -- Einem Gott dagegen, der Niemanden aufopfert, um das Wohl Anderer zu befördern, als den, der es um seiner eigenen Lasterhaftigkeit willen verdienet, bestraft und verworfen zu werden, muß nicht der Lasterhafte selbst in seinem Untergange einem solchen Gotte das Zeugniß geben, daß er gerecht und ohne Tadel sey?
\end{aufza}

\RWpar{140}{Wirklicher Nutzen}
\begin{aufza}
\item Es hat gewiß seinen Nutzen gehabt, daß man allenthalben, wohin das Christenthum vordrang, die Versicherung desselben annahm, daß Alles in der Welt nach Gottes heiligem und allweisem Rathschlusse erfolge. Der einzige Schaden, den diese Meinung vielleicht bei einigen Wenigen hervorgebracht hat, war, daß sie ihre eigenen bösen Handlungen entschuldigten, indem ja auch diese in Gottes Rathschluß enthalten wären. Doch diese Täuschung hat nur sehr wenige Menschen, und diese gewiß nur auf eine durch eigene Schuld zugezogene Art verblendet, zumal, da eben dasselbe Christenthum, welches uns lehrt, daß auch der Lasterhafte nur den Rathschluß Gottes ausführe, zugleich versichert, daß es dem Lasterhaften nichts nütze, Gottes Rathschlüsse befördert zu haben, sondern daß er der Strafe seiner Lasterhaftigkeit darum auf keine Weise entgehe.~\RWSeitenw{252}
\item Hiezu kommt noch der Unterschied, den das Christenthum zwischen den wirkenden und zulassenden Rathschlüssen Gottes eingeführt hat; denn dieser machte es um so deutlicher, daß sich der Lasterhafte nicht im Mindesten trösten könne durch die Redensart, daß er Gottes Rathschluß ausführe. Gott läßt nur zu, was er thut.
\end{aufza}

Die Artikel 6.\ und 7.\ können nur Nutzen, und keinen Schaden gestiftet haben.

\RWabs{Zweiter Abschnitt}{Katholische Kosmologie}
\RWpar{141}{Inhalt dieses Abschnittes}
Nachdem wir die Ansichten, die uns das Christenthum von Gott, gleichsam wie er an sich ist, mittheilt, kennen gelernt haben, so kommen wir zu den Begriffen, welche es uns \RWbet{von Gottes Werken} gibt, und zwar zuerst von Gottes Werken \RWbet{im Allgemeinen} und von der Welt überhaupt. Wir tragen hier die Lehren von der \RWbet{Schöpfung}, von der \RWbet{Vorsehung}, von der \RWbet{Vollkommenheit der Welt}, und von dem \RWbet{letzten Zwecke derselben}, endlich von den \RWbet{vernünftigen Geschöpfen, die es noch außerhalb des Menschen gibt}, vor.

\RWpar{142}{Die christkatholische Lehre von Gottes Schöpfung und Vorsehung}
\begin{aufza}
\item Die Welt, \dh\ (im Sinne des Christenthums) Himmel und Erde, sammt Allem, was darauf ist, \RWbet{hat ihr Daseyn nicht durch sich selbst, sondern nur durch den allmächtigen Willen Gottes.}
\item Gott ist die letzte Ursache nicht nur \RWbet{von der Gestalt}, die diese Welt hat, sondern auch \RWbet{von den Sub}\RWSeitenw{253}\RWbet{stanzen,} die ihr zum Grunde liegen, \dh\ von der Materie der Welt. Gott hat die Welt nicht etwa aus einem vorhandenen Stoffe gebildet, sondern \RWbet{aus Nichts hervorgebracht}, was man mit Einem Worte die \RWbet{Schöpfung} nennt.
\item Aber nicht nur, daß alle Dinge in der Welt ihr erstes Entstehen nicht aus sich selbst haben, auch ihre \RWbet{Fortdauer}, sagt das Christenthum, verdanken sie der Wirkung Gottes, dergestalt, daß es zu dieser Fortdauer nicht genug ist, daß sich Gott dabei nur negativ verhalte, \dh\ \RWbet{dieselbe nur zulasse}, nur nicht entgegenwirke; sondern er muß zu ihrer Erhaltung selbst positiv durch alle Augenblicke ihres Daseyns einwirken, oder sie fallen gleich in ihr voriges Nichts zurück. Dieß nennt man Gottes \RWbet{Erhaltung.}
\item Noch nicht genug, daß Gott alle Dinge erschaffen hat und erhält, \RWbet{er führt sie auch alle zu einem vorgesetzten Zwecke}. Dieß nennt man göttliche Regierung. Erhaltung und Regierung zusammen nennet man \RWbet{Vorsehung.}
\item Die göttliche Vorsehung erstreckt sich, wie schon gesagt, \RWbet{auf alle Dinge, auch die geringscheinendsten;} nichts desto weniger läßt sie auch in gewisser Rücksicht doch ihre \RWbet{Grade} zu, und man kann sagen, daß sich vor allen Dingen, die auf Erden sind, und vor vielen Tausenden noch außerhalb der Erde, \RWbet{das menschliche Geschlecht einer ganz vorzüglichen Vorsehung} von Seite Gottes zu erfreuen habe.
\item Fragt man, welcher aus den drei göttlichen Personen man das Geschäft der Schöpfung und Vorsehung am eigentlichsten zuzuschreiben habe: so gibt das Christenthum zur Antwort: \RWbet{Der Vater sey der eigentliche Schöpfer, Erhalter und Regierer der Welt;} doch aber habe auch der Sohn einen Antheil von der Art, daß man sagen könne, der Vater hat die Welt erschaffen \RWbet{durch den Sohn}. Dieß soll jedoch nicht so viel sagen, als ob sich der Vater des Sohnes als eines \RWbet{Werkzeuges} bedient hätte; sondern nur, daß auch der Sohn bei der Weltschöpfung mitgewirkt habe, und zwar unmittelbarer, als der Vater.~\RWSeitenw{254}
\end{aufza}

\RWpar{143}{Historischer Beweis dieser Lehre}
\begin{aufza}
\item Schon im \RWbibel[1.~Buche Mosis]{Gen}{}{1}{1}\ heißt es: \erganf{Im Anfange schuf Gott Himmel und Erde.}
\item Daß Gott \RWbet{nicht bloßer Weltbildner}, sondern eigentlicher Schöpfer sey, kann man mit einiger Wahrscheinlichkeit schon selbst aus dieser Stelle schließen. Nicht aus dem Worte Schaffen; denn das hebräische \RWhebr{b*ArA'} (\RWlat{bara}) heißt eigentlich \RWbet{zeugen}, und kann auch von einer Hervorbringung aus schon vorhandenem Stoffe gebraucht werden, wie es \RWbibel{Gen}{}{1}{27}\ bei der Bildung des Menschen wirklich in diesem Sinne gebraucht wird; aber wohl daraus, weil, wenn Moses eine Bildung der Welt aus schon vorhandenem Stoffe geglaubt hätte, er dieses Stoffes gewiß erwähnt haben würde. Vielleicht auch daraus, weil in der Folge vom 2.\,V.\ anzufangen, erst von einer gewissen Ausbildung der Welt die Rede ist, so daß man auf den Gedanken verfallen muß, der 1.\,V.\ spreche von einer bloßen Schöpfung der Materie. Im \RWbibel{2\,Makk}{2\,B.\,Mach.}{7}{28}\ spricht jene Mutter ganz ausdrücklich: \erganf{Sohn! ich beschwöre dich, daß du betrachtest den Himmel und die Erde, und Alles, was darauf ist, um einzusehen, (dich zu erinnern) daß Gott dieß Alles \RWbet{aus Nichts} hervorgebracht habe.}\par
In den BB.\ d.\ n.\,B.\ schreibt Paulus \RWbibel{Hebr}{Hebr.}{11}{3}: \erganf{Durch den Glauben halten wir uns für überzeugt, daß die Welt \RWbet{durch Gottes Wort erschaffen}, und das Sichtbare aus dem Unsichtbaren entstanden sey.} \RWbibel{Kol}{Koloss.}{1}{15--17}\ schreibt er von \RWbet{Jesu Christo}: \erganf{Durch ihn ist \RWbet{Alles erschaffen im Himmel und auf Erden, es sey sichtbar oder unsichtbar}. Selbst Thronen, Herrschaften, Mächte, Gewalten, Alles ist durch ihn und in Beziehung auf ihn erschaffen. Er war vor Allem, und \RWbet{Alles bestehet durch ihn}.} Hätte der Apostel geglaubt, daß Gott die Welt durch den Sohn aus einer Materie gebildet habe, so hätten sie es erwähnt. Uebrigens ist diese Meinung von einer Schöpfung der Welt aus Nichts in der christlichen Kirche allezeit herrschend gewesen. Tertullian, Clemens von Alexandrien, Irenäus, Lactantius, Augustinus \uA\ lehren einstimmig so.~\RWSeitenw{255}
\item \RWbet{Erhaltung Gottes}.\par
\RWbibel{Ps}{Psalm}{104}{27\,ff}:\par
\erganf{Alles hofft auf Dich, erwartet,\par
Daß Du ihm Speise gibst in der Zeit.\par
Du gibst, sie sammeln, Du öffnest Deine Hand,\par
Da werden sie mit Gut gesättiget.\par
Du wendest ab Dein Angesicht; sie schwinden,\par
Nimmst ihren Odem weg, und sie vergeh'n\par
In ihren vorigen Staub zurück.\par
Du sendest Deinen Odem aus, da sind sie wieder,\par
Und die Gestalt der Erde ist verjünget.}\par
Noch beweisender ist \RWbibel{Weish}{Weish.}{11}{24}: \erganf{Du liebst Alles, was ist, und hassest nichts von dem, was du geschaffen hast. Wie könnte auch etwas fortdauern, wenn du nicht wolltest, wie noch bestehen, wenn du es nicht hießest?}
\item \RWbet{Regierung}. Der ganze \RWbibel[104te Psalm]{Ps}{}{104}{}\ handelt von den Zwecken Gottes bei den mannigfaltigen Einrichtungen auf dieser Erde. Eben so der \Ahat{\RWbibel[95ste.]{Ps}{}{95}{}}{94ste.}\ \Ahat{\RWbibel{Weish}{Weish.}{14}{2}}{10,2.}: \erganf{Das Schiff denkt an Gewinn, aber deine Vorsehung, o Vater! regieret es; du bahnst ihm den Weg, in Meeresfluthen einen sichern Pfad.} -- \RWbibel{Mt}{Matth.}{10}{29}\ spricht Jesus: \erganf{Kauft man nicht zwei Sperlinge um einen Heller? Dennoch fällt keiner derselben auf die Erde ohne eueren Vater}, -- woraus zugleich erhellet, daß die Vorsehung Gottes sich auf die \RWbet{kleinsten und geringscheinendsten Dinge} erstreckt.
\item Daß aber der Mensch sich einer \RWbet{vorzüglichen Fürsorge} vor allen andern Geschöpfen zu erfreuen habe, lehrt Jesus in der oben angeführten Stelle, indem er weiter spricht: \erganf{Fürchtet euch nicht; denn ihr seyd mehr werth, als alle Sperlinge.} So auch \RWbibel{Lk}{Luk.}{12}{28}: \erganf{Wenn Gott das Gras auf dem Felde, das heute da ist, und morgen in den Ofen geworfen wird, so herrlich bekleidet: um wie viel mehr wird er euch kleiden, ihr Kleingläubigen!} So wird auch, wenn im \RWbibel{Gen}{1\,B.\ Mos.}{1}{}\ von der Schöpfung des Menschen die Rede ist, gesagt, daß Gott ihn zum Herrn über alles Uebrige auf Erden erhoben habe. Es wird uns auch der erhabenste Begriff von allen den natürlichen und übernatürlichen Anstal\RWSeitenw{256}ten beigebracht, die Gott getroffen hat, das menschliche Geschlecht zu seinem Ziele hinzuleiten. Die ganze Natur muß sich als Mittel dazu brauchen lassen, Sterne sogar müssen bei der Geburt Jesu erscheinen, Engel zur Bedienung, zum Schutze der Menschen sich beschäftigen, \usw\
\item \RWbibel{Joh}{Joh.}{1}{3}: \erganf{Alles ist durch das Wort erschaffen und ohne dasselbe ist nichts, von Allem, was ist, erschaffen.} \RWbibel{Hebr}{Hebr.}{1}{2}: \erganf{Durch den Sohn hat er (Gott) die Welt erschaffen.} Hieher gehört auch die schon vorhin angeführte Stelle \RWbibel{Kol}{Koloss.}{1}{16}
\end{aufza}

\RWpar{144}{Vernunftmäßigkeit}
\begin{aufza}
\item Wenn man von dem Begriffe Gottes ausgeht, daß er das unbedingte Wesen, und zwar der Inbegriff alles desjenigen, was unbedingt ist, sey: so versteht man dagegen unter der Welt, \dh\ unter dem Wirklichen, was nicht Gott ist, alles bloß Bedingte. Hieraus ergibt sich denn unmittelbar, daß sie ihr Daseyn nicht aus sich selbst haben könne, sondern von dem unbedingten Wesen, von Gott, herrühren müsse. Sodann ist nur zu beweisen, daß Alles dasjenige, was die christliche Offenbarung Welt nennet, also Himmel, Erde und Alles, was darauf ist, nicht zu dem unbedingten Wesen gehöre, sondern bedingt sey. Diese Wahrheit ist durch das Urtheil des gesunden Menschenverstandes gewiß. Unter den Philosophen hat es aber mehrere gegeben, welche sie läugneten. \RWbet{Benedict Spinoza} und seine Anhänger, \zB\ der Graf \RWbet{Boulainvilliers, Toland} \uA\ zogen aus einer irrigen Erklärung, die sie vom Begriffe der Substanz gaben (daß sie nämlich etwas für sich Bestehendes sey), den Schluß, daß jede Substanz absolut nothwendig und also auch unbedingt seyn müsse, und daß es eben daher nur eine einzige wahre Substanz, nämlich die Gottheit gebe. Alles, was wir um uns sehen, die sogenannte Welt, auch uns selbst mitgerechnet, hat kein für sich bestehendes Wesen, sondern ist eine bloße Modification der Einen unendlichen Substanz Gottes. Um dieser sonderbaren Behauptung willen nannte man diese Philosophen Pantheisten. Die neueste (\RWbet{Schelling'sche}) Philosophie lehrt etwas Aehnliches. -- Die ganze Schwierigkeit~\RWSeitenw{257}\ beruht auf dem noch nicht hinlänglich erörterten Begriffe der Substanz. Behält man anders jenen, obgleich schwer zu erklärenden, doch ganz bestimmten Begriff bei, den der Sprachgebrauch mit dem Worte Substanz im Gegensatze des Wortes Adhärenz verbindet, so wird man folgende zwei Sätze ohne Widerrede als wahr zugeben können, ob man gleich ihre Beweise (die wissenschaftlichen) noch nicht kennt.
\begin{aufzb}
\item Jedes Subject, welches ein eigenes, von anderen abgesondertes Bewußtseyn hat, ist auch eine eigene, von andern abgesonderte Substanz, \dh\ so viele mit Bewußtseyn begabte Subjecte es gibt, so viele Substanzen gibt es. Und dann
\item jede Materie, \dh\ Alles, was einen Raum erfüllt, und wäre es auch nur ein Punct im Raume, hat eine eigene Substanz. Wird dieß Beides zugegeben, und das Letztere behauptet man in jeder Physik, so hat es wohl keine Schwierigkeit mehr, zu beweisen, daß Erde, Himmel, \usw\ bedingte Substanzen sind. Die Eigenschaften, die eine unbedingte Substanz haben muß, haben wir bereits kennen gelernt. Sie muß unendlich seyn in ihren Kräften, in ihrem Verstande und Willen, \usw\ Dergleichen finden wir aber nicht an den Substanzen dieser Erde.
\end{aufzb}
\item Daß Gott nicht bloßer Weltbildner (wie die ersten alten Weltweisen geglaubt, die ihn eben deßhalb auch nur \RWgriech{dhmiourg`os} genannt), sondern im eigentlichen Sinne \RWbet{Schöpfer}, \dh\ die letzte Ursache von dem Vorhandenseyn der Welt sey, folgt unmittelbar aus dem Begriffe Gottes, zu Folge dessen wir Alles, was unbedingt wirklich ist, in sein Wesen ziehen.
\item Der Begriff der \RWbet{Erhaltung} ist von jenem der Schöpfung nur dadurch unterschieden, daß er noch den Begriff der Zeit hinzufügt. \erganf{Gott ist Schöpfer} heißt, er ist Ursache, und zwar \RWbet{alleinige Ursache} von dem Vorhandenseyn aller Substanzen, die es noch außer ihm gibt. \erganf{Er ist Erhalter} heißt, er ist \RWbet{fortwährende} Ursache von dem Vorhandenseyn dieser Substanzen; oder diese Substanzen können nicht nur überhaupt nicht seyn ohne Gott, -- sondern sie kön\RWSeitenw{258}nen auch durch keinen einzigen Augenblick seyn ohne Gott. Das Letztere ist nun eine bloße Folge aus dem Ersteren; und somit ist die Lehre von der Erhaltung eben so richtig, als die von der Schöpfung Gottes. Sehr richtig wird auch bemerkt, daß wir uns diese Erhaltung nicht als etwas bloß Negatives, sondern als eine \RWbet{positive Wirkung} Gottes vorstellen müssen; denn auf Seite Gottes ist ja der Act der Erhaltung völlig derselbe mit jenem der Schöpfung.
\item Die Lehre von der \RWbet{Regierung} Gottes, oder die Lehre, daß Gott alle geschaffenen Dinge zu vorgesetzten Zwecken leite, setzt voraus,
\begin{aufzb}
\item daß Gott \RWbet{zu allen Zeiten in die Welt} einwirke, \dh\ verschiedene Veränderungen und Modificationen in den Substanzen der Welt beiläufig eben so hervorbringe, wie unser Geist in der Maschine unseres Körpers, und in andern uns umgebenden Körpern Bewegungen hervorbringt; widrigenfalls könnte es nicht heißen, daß Gott die Welt noch jetzt regiere, sondern nur, daß er sie einst regieret habe.
\item Daß er diese Einwirkungen nicht absichtslos, sondern \RWbet{nach einem vernünftigen Zwecke} vornehme. Das Letztere hat keine Schwierigkeit, sondern ist eine leichte Folge aus der Lehre von Gottes Allwissenheit und Weisheit, oder von Gottes Rathschlüssen, sobald man uns nur das Erste zugibt. Aber gegen dieses hat man allerlei Einwendungen gemacht.
\end{aufzb}\end{aufza}\par
\RWbet{1.~Einwurf.} So wäre Gott beständig mit der Welt beschäftiget, eine Geschäftigkeit, die nothwendig \RWbet{ermüden} müßte, und sich auf keine Weise mit Gottes höchster Seligkeit verträgt (Epikur, Peter Bayle).\par
\RWbet{Antwort.} Einen so groben Anthropomorphismus zu widerlegen, ist kaum der Mühe werth.\par
\RWbet{2.~Einwurf.} Wenn Gott noch jetzt in die Welt einwirkt: so wirkt er successiv, und ist mithin selbst an die Zeitbedingung gebunden, selbst in der Zeit.\par
\RWbet{Antwort.} Noch jetzt, sagt man in diesem Einwurfe, und gibt sonach stillschweigend zu, Gott habe einmal (etwa~\RWSeitenw{259}\ vor Millionen Jahren) allerdings in die Welt einwirken können, ohne an die Bedingung der Zeit gebunden zu seyn. Wie widersprechend ist dieß! wenn Gott in einem Augenblicke der Zeit, seiner Erhabenheit über die Zeit unbeschadet, wirksam seyn konnte, so kann er auch wohl jeden Augenblick der Zeit wirksam seyn, ohne sich darum selbst in der Zeit zu befinden. Es läßt sich nämlich sehr wohl denken, daß eine Ursache Wirkungen in der Zeit hervorbringe, die sich gleichwohl selbst nicht in der Zeit befindet. Solch eine Ursache ist nämlich Gott, der sich in sofern in keiner Zeit befindet, in wiefern er unveränderlich ist.\par
\RWbet{3.~Einwurf.} Vollkommener wäre doch immer der Gott, der eine Welt zu Stande gebracht hat, die, nachdem er sie hergestellt, keiner Nachhülfe mehr bedarf, aus sich selbst fortdauert, und ihren vorgesetzten Zweck erreicht. Kann doch ein Künstler unter den Menschen eine Maschine, ein Uhrwerk \udgl\  herstellen, das viele Jahre hindurch ohne weitere Nachhülfe besteht und seinem Zwecke entspricht. Sollte der Allmächtige weniger vermögen? --\par
\RWbet{Antwort.} 
\begin{aufzb}
\item Daß die erschaffenen Wesen ihr Daseyn aus sich selbst und ohne Gottes Wirkung auch nur einen Augenblick fortsetzen könnten, ist an sich selbst unmöglich; denn Wesen, die aus sich selbst und ohne irgend eines Andern Wirkung existiren, sind eben darum unabhängige und also auch unerschaffene Wesen. Was nun an sich selbst unmöglich ist, davon kann man auch nicht sagen, daß Gott vollkommener wäre, wenn er es vermöchte. Auch ist es eine arge Täuschung, wenn man annimmt, der Künstler unter den Menschen vermöge so etwas. Die Maschine, die er hervorgebracht hat, dauert zwar ohne seine, ja vielleicht auch ohne aller Menschen Mitwirkung fort, allein nicht ohne Mitwirkung Gottes, der die Materie derselben und ihre Kräfte erhält.
\item Was aber die übrige Einwirkung Gottes in die Welt betrifft, nämlich diejenige, durch die er nicht nur Alles erhält, sondern auch durch continuirliche Einwirkungen \RWbet{zu bestimmten Zwecken hinführt}: so ließe sich freilich eine Welt denken, in welcher das nicht geschieht, eine Welt, in welcher Jahrhunderte lang Alles seinem~\RWSeitenw{260}\ geraden Gange überlassen bleibt; aber eine solche Welt müßte bei weitem unvollkommener, als die gegenwärtige seyn. Tugend und Glückseligkeit könnten in ihr nicht in einem so hohen Grade befördert werden, als es in derjenigen geschieht, in welche Gott vielfältig, ja ununterbrochen einwirkt.
\end{aufzb}\par
\RWbet{4.~Einwurf.} So wird sich Gottes Vorsehung wenigstens nur auf sehr wichtige Dinge erstrecken; er wird \zB\ nur für die Erhaltung ganzer Geschlechter, der Thiere, der Menschen \usw\ sorgen; nicht aber für jedes einzelne Geschöpf, das viel zu geringfügig ist, als daß es sich für den Unendlichen auch nur geziemte, sich damit zu beschäftigen.\par
\RWbet{Antwort.} 
\begin{aufzb}
\item Auch das Geringfügigste ist näherer oder entfernterer Weise durch Gottes Wirkung da. Nun wissen wir bereits, daß Gott nur mit Bewußtseyn und mit Absicht wirken könne, also hat er auch bei dem Daseyn des Geringfügigsten seine Absicht.
\item Ist dieser Gegenstand ferner noch ein lebendiger: so muß Gott, vermöge seiner Heiligkeit, auch auf den Grad der Glückseligkeit, dessen dieses Wesen empfänglich ist, und wirklich genießt, bei seiner Leitung der Welt Rücksicht nehmen, weil auch dieser Grad der Glückseligkeit ein (obgleich nur geringes) Element von jener Glückseligkeit des Ganzen ausmacht, die Gott zum Maximum erheben muß.
\item Und wenn das Wesen überdieß frei und der Tugend fähig ist: so muß er auch auf den Grad der Tugend in diesem Wesen Rücksicht nehmen, und ihn, so viel es der Zusammenhang des Ganzen erlaubt, zu befördern suchen.
\item Daß dieses der Würde Gottes zuwider wäre, ist ein sehr thörichter Anthropomorphismus. Menschen, die wegen ihrer Eingeschränktheit nicht das Ganze leiten, und doch zugleich jeden einzelnen geringfügigen Theil beachten können, ist es freilich nicht nur erlaubt, sondern sogar zur Pflicht zu machen, daß sie auf Kleinigkeiten nicht achten; und eben dieses hat Gelegenheit gegeben zu der Redensart: Es ist unter deiner Würde, dich mit dieser~\RWSeitenw{261}\ Kleinigkeit zu befassen; nämlich, weil du bei deiner Endlichkeit sonst die wichtigeren Geschäfte, die dir obliegen, nicht bestreiten könntest. Da aber dieser Grund bei Gott wegfällt: so sieht man, daß es auch nicht unter Gottes Würde sey, auf die geringfügigsten Gegenstände in dieser Welt seine Aufmerksamkeit zu richten.
\end{aufzb}\par
\RWbet{5.~Einwurf.} Aber durch dieses Einwirken werden wir in der Beobachtung der Naturgesetze irre gemacht, indem wir für eine Wirkung der sich selbst überlassenen Naturkräfte ansehen, was vielleicht nur durch Gottes Einwirkung erfolgt ist.\par
\RWbet{Antwort.} Dieses ist nicht zu befürchten, weil Gott, nach seiner unendlichen Weisheit, seine Einwirkungen nie dort anbringen wird, wo sie uns in der Beobachtung der Naturgesetze auf eine für uns nachtheilige Weise irre führen müßten.
\begin{aufza}\stepcounter{enumi}\stepcounter{enumi}\stepcounter{enumi}\stepcounter{enumi}
\item Daß Gottes Vorsehung in gewisser Rücksicht auch \RWbet{ihre Grade} habe, läßt sich sehr wohl vertheidigen. Nämlich nicht so, als ob die Aufmerksamkeit, die Gott auf ein gewisses Wesen verwendet, oder die Mühe, welche er sich für dasselbe gibt, hier größer, dort geringer wäre; denn Gott berücksichtiget bei jedem Wesen alle Umstände, in welchen es sich befindet; und Mühe verursacht ihm keines derselben; aber wohl findet er ein Wesen wichtiger für das Wohl des Ganzen, als irgend ein anderes, und opfert eben deßhalb der Leitung dieses Wesens mehr, als der Leitung anderer auf. So muß ihm \zB\ der Mensch, weil er eines höheren Grades der Glückseligkeit, und nebstdem auch der Tugend fähig ist, viel wichtiger seyn, als alle übrigen Erdengeschöpfe, er darf und muß dem Zwecke, den Menschen zur Tugend und Glückseligkeit zu leiten, manchen Vortheil der thierischen Welt im Falle einer Collision zum Opfer bringen. Und dieß ist es, was das Christenthum mit den Worten ausdrückt: Der Mensch habe sich der Vorsehung Gottes in einem vorzüglichen Grade zu erfreuen.
\item Daß wir die Schöpfung, Erhaltung und Regierung der Welt vorzugsweise \RWbet{dem Vater zuschreiben} sollen, enthält nichts Ungereimtes, wie wir dieß schon oben gezeigt haben; und eben so wenig, daß sich der Vater bei dieser Wirkung auf eine gewisse Art \RWbet{des Sohnes bedienet}.~\RWSeitenw{262}\ Wenn aber die Kirche verbietet, den Sohn als ein \RWbet{Werkzeug} zu betrachten, durch welches der Vater die Welt erschaffen hat: so will sie damit nur so viel sagen, der Sohn habe bei diesem Geschäfte auf eine edlere und selbstständigere Art mitgewirkt, als es durch die figürliche Redensart: als Werkzeug angedeutet werde. Werkzeug heißen wir doch nur etwas Lebloses, oder, wofern es ja belebt ist, etwas, das nicht aus eigenem Willen, sondern nur nothgedrungen wirkt, oder das wenigstens in einem niedrigeren Range steht, als derjenige, der es als Werkzeug braucht. Dieß Alles soll nun bei dem Mitwirken des Sohnes weggedacht werden.
\end{aufza}

\RWpar{145}{Sittlicher Nutzen}
\begin{aufza}
\item Die Lehre, \RWbet{daß Himmel und Erde, sammt Allem, was darauf ist, durch Gottes freien Willen da ist}, gibt erst dem Glauben an Gott Nutzen und Fruchtbarkeit. Wer sie nicht annimmt, muß entweder glauben,
\begin{aufzb}
\item daß alle diese Dinge, oder doch ein beträchtlicher Theil derselben, \zB\ die großen Weltkörper, die verschiedenen Geschlechter der Thiere und Pflanzen, unbedingt sind, oder
\item durch die Wirkung irgend eines zwar freien, aber doch endlichen Wesens da sind.
\end{aufzb}
Beides hat seine Nachtheile.
\begin{aufzb}
\item Hielten wir jene Dinge, oder doch einen großen Theil derselben für unbedingt, so zählten wir sie eben deßhalb mit zu dem Wesen Gottes, dem wir eine unbedingte Wirklichkeit beilegen. Wir würden zwar auch jetzt noch sagen, daß diese Dinge die höchst mögliche Vollkommenheit besitzen; aber von der Beschaffenheit dieser höchsten Vollkommenheit selbst würden wir uns jetzt keinen so erhabenen Begriff zu bilden vermögen, als dann, wenn wir diese Dinge nicht mehr für Theile Gottes selbst, sondern nur für Wirkungen Gottes ansehen, die er durch seinen Willen, also dem obersten Sittengesetze, der möglichsten Beförderung des allgemeinen Wohles gemäß, hervorbringt.~\RWSeitenw{263}
\item Hielten wir diese Dinge für Wirkungen eines oder mehrerer freier, aber nur endlicher Wesen, so könnten wir ihnen eine zwar hohe, aber doch nicht jene höchste Vollkommenheit beilegen, die sie nur dann haben können, wenn sie unmittelbar von dem Unendlichen herrühren.
\item Hiezu kommt ferner: wenn wir Himmel und Erde und Alles, was darauf ist, als Gottes Werk ansehen, so erhalten wir einen anschaulichen Begriff von Gottes Allmacht und Größe.
\item Aus der Vollkommenheit, die diese Welt besitzt, aus den zweckmäßigen Einrichtungen, die wir in ihr bemerken, wird uns die Weisheit Gottes, und
\item aus der Güte der Zwecke, die wir hier allenthalben wahrnehmen, aus dem vielfältigen Guten, das in der Welt herrscht, wird uns die Güte und Heiligkeit Gottes anschaulicher. Endlich
\item erkennen wir auch weit deutlicher die Verbindlichkeit, Gott dankbar zu seyn und ihm zu gehorchen; denn von ihm haben wir Alles, in seinen Händen steht Alles.
\end{aufzb}
\item Einen ähnlichen Nutzen hat auch der Glaube, daß Gott die Welt \RWbet{aus Nichts} erschaffen habe.
\begin{aufzb}
\item Die entgegengesetzte Vorstellung, daß er die Welt aus einer ewigen und von ihm selbst unabhängigen Materie bloß gebildet habe, daß also nur die Gestalt dieser Welt, nur die Verbindung ihrer Theile, nicht aber sie selbst durch die Wirkung eines höchst weisen Wesens da ist, würde unwillkürlich auf den Gedanken leiten, daß Gott, gehindert durch den Widerstand, den jene Materie durch ihre Trägheit und andere ihr wesentliche Unvollkommenheiten seiner bildenden Hand entgegengesetzt hätte, der Welt bei Weitem nicht jene vollkommene Gestalt gegeben habe, die er ihr hätte geben können, wenn er sie aus einem selbst geschaffenen Stoffe gebildet hätte. Beiläufig eben so, wie selbst der geschickteste Künstler aus einem groben Stoffe nur eine sehr unvollkommene Bildsäule herstellen kann.
\item Diese Lehre dient, uns eine richtigere, erhabenere und eben darum auch wohlthätigere Vorstellung von Gottes~\RWSeitenw{264}\ Allmacht beizubringen. Wenn Gott die Welt aus einem bereits vorhandenen Stoffe nur bloß gebildet hat: so ist diese Verrichtung desselben eine ganz menschenähnliche gewesen; denn auch wir Menschen vermögen einem vorhandenen Stoffe mannigfaltige Umstaltungen zu geben. Denken wir uns dagegen, daß Gott selbst diesem Stoffe der Welt das Daseyn gegeben habe: so that er etwas, so wir Menschen nicht vermögen. Wir bilden uns also einen weit erhabeneren Begriff von Gottes Allmacht, der in der That viel richtiger ist. Allein je erhabener unsere Begriffe von Gottes Allmacht sind, um desto größer ist unser Vertrauen zu ihm, um desto zufriedener werden wir mit allen Welteinrichtungen seyn.
\end{aufzb}
\item Obgleich der Unterschied zwischen \RWbet{Erhaltung} und Schöpfung in philosophischer Betrachtung verschwindet, so kann er uns doch gar nicht beunruhigen, weil wir (wie es sich unten zeigen wird) deutlich einsehen, daß ihn das Christenthum uns gar nicht aufdringt, sondern nur zum Behufe derjenigen, die sich zu keiner so hohen Abstraction erhoben haben, daß sie die Abhängigkeit der Welt von Ewigkeit her fassen könnten, eingeführt hat. Wohlthätig ist aber auch für uns die Bemerkung des Christenthums, \RWbet{daß die Substanzen dieser Welt ihr Daseyn auch eine noch so kurze Zeit hindurch nicht fortsetzen können ohne die positive Mitwirkung Gottes}; denn
\begin{aufzb}
\item hiedurch erkennen wir wahrer und lebhafter, wie unsere eigene, so aller übrigen geschaffenen Dinge Schwäche und Ohnmacht. Nicht einmal eine auch noch so kurze Zeit hindurch können wir ohne Mitwirkung Gottes bestehen, immer bedürfen wir seiner positiven Mitwirkung, und wir wollten gleichwohl stolz auf unsere Größe und Macht seyn?
\item Wir wollten Gott vergessen, und nicht in jedem Augenblicke unseres Daseyns ihm für die Erhaltung desselben danken?
\item Wir wollten uns erfrechen, das Daseyn, das nur er uns erhält, die Kräfte, die nur durch seine Mitwirkung in uns vorhanden sind, in irgend einem Falle gegen seine heiligen Absichten zu mißbrauchen? --~\RWSeitenw{265}
\item Wir wollten unser Vertrauen auf irdische Dinge mit Hintansetzung Gottes gründen? Und er ist's doch, der sie und uns erhält?
\item Hiedurch begreifen wir auch leichter, wie Gott allgegenwärtig ist. Er wirket in allen Dingen, \dh\ ist gegenwärtig in denselben, denn alle Dinge bestehen ja nur durch seine Wirkung.
\end{aufzb}
\item Die Lehre von \RWbet{Gottes Weltregierung}, oder daß Gott \RWbet{zu allen Zeiten} auf die Welt einwirke, und ihre Theile zu bestimmten Zwecken leite, hat folgenden Nutzen:
\begin{aufzb}
\item Wenn wir uns vorstellen, daß Gott noch fortwährend in der Welt einwirke, nicht nur dadurch, daß er ihre Substanz erhält, sondern auch, daß er diese Substanzen und ihre Kräfte in allerlei neue Verbindungen bringt, und dieselben modificirt \udgl : so können wir uns von der Vollkommenheit, die diese Welt erhalten muß, von der Summe des Guten, das in ihr zu Stande kommen wird, einen weit höheren Begriff machen, als wenn wir glaubten, daß er es nur bei der Schöpfung und Erhaltung allein bewenden lasse.\par
\RWbet{Einwurf.} Wir könnten uns aber auch vorstellen, daß er jene Veränderungen, neue Verbindungen \usw , die wir ihn hier in der Zeit bewirken lassen, durch solche Anordnungen, die er schon von Ewigkeit her getroffen hat, mittelbar bewirke, und diese Vorstellung würde die Weisheit Gottes in ein noch helleres Licht setzen.\par
\RWbet{Antwort.} Die christliche Offenbarung verbietet uns gar nicht, zu glauben, daß Gott so manches Ereigniß durch Anordnungen, die er schon von undenkbaren Zeiten, auch wohl von Ewigkeit her getroffen hat, mittelbar bewirke. Aber wenn alle Ereignisse in der Welt von Gott nur so bewirkt werden sollten: so müßte man annehmen, daß der ganze unmittelbare Einfluß auf die Welt eigentlich nur in der Erhaltung der Substanzen bestehe, und von Ewigkeit her nur darin bestanden habe. Ereignisse also, die sich nach unserer Vorstellung sehr leicht durch eine bloße Einwirkung in den Zusammenhang der Substanzen ausführen ließen, müßte er nach~\RWSeitenw{266}\ dieser Vorstellung entweder ungeschehen lassen, oder, um sie zu bewirken, von Ewigkeit her gewissen Substanzen das Daseyn gegeben haben, nur damit sie zur rechten Zeit durch ihre Einwirkung jene Veränderung hervorbringen, die er durch eine unmittelbare Einwirkung leicht selbst hätte hervorbringen können. Das Erste würde der Vollkommenheit der Welt Abbruch thun, das Zweite scheint mit der Weisheit Gottes zu streiten, die nicht durch Umwege wirkt. So viele Substanzen Gott immer geschaffen haben mag: so kann er dadurch, daß er auf diese Substanzen noch neuerdings in der Zeit einwirkt, mehr ausrichten, als dadurch, daß er es unterläßt. Es ist also der Vorstellung eines vollkommenen Gottes gemäß, daß er das Erstere thue.
\item Aber je wirksamer sich Gott in seiner Welt bezeugt, eine je vollkommenere Gestalt er ihr ertheilt, um desto mehr Dankbarkeit müssen wir gegen ihn empfinden.
\item Die Versicherung, daß Gottes Weltregierung sich \RWbet{bis auf die geringscheinendsten Gegenstände} erstrecke, dient uns zu unserer größten Beruhigung; denn im entgegengesetzten Falle müßten wir billig bezweifeln, ob er auch für uns selbst sorge, da wir, so wichtig wir uns auch scheinen mögen, wenn wir uns etwa mit Wesen von noch geringerer Art vergleichen, doch in ein Nichts verschwinden, wenn wir auf die unendliche Menge der Weltkörper und Geschöpfe Gottes, die es noch außerhalb unser gibt, hinsehen. Nun werden wir ferner jedes, auch das geringste Geschöpf als einen Gegenstand der göttlichen Vorsorge ehren und schätzen, und uns wohl in Acht nehmen, es ohne Noth zu zerstören.
\end{aufzb}
\item Die Lehre, \RWbet{daß unser menschliches Geschlecht ganz vorzüglich vor allen andern Geschlechtern der Erde die Vorsorge Gottes genieße,} hat einen offenbar wohlthätigen Einfluß. Sie dient sowohl
\begin{aufzb}
\item uns eine vernünftige Schätzung des Werthes, den wir als Menschen behaupten, einzuflößen, als auch
\item uns desto stärker zur Dankbarkeit und Liebe Gottes zu verpflichten.~\RWSeitenw{267}
\end{aufzb}
\item Dadurch, daß uns das Christenthum sagt, daß es \RWbet{eigentlich der Vater} sey, dem wir die Schöpfung, Erhaltung und Regierung des Weltalls als letztem Grunde zuschreiben sollen, bestimmt es uns genauer das Object unserer Verehrung und Dankbarkeit. Dadurch aber, daß es hinzusetzt, der Vater habe die Welt \RWbet{durch den Sohn} geschaffen, weiset es uns zu einer gleichen Verehrung auch dieses Sohnes an, und zeigt die Größe desjenigen, von dem es gleichwohl (nach einer andern Lehre) heißt, daß er die menschliche Natur an sich genommen habe, und unser Bruder geworden sey.
\end{aufza}

\RWpar{146}{Wirklicher Nutzen}
\begin{aufza}
\item Durch die Lehre, daß Himmel und Erde und Alles, was darauf ist, durch Gottes Willen da sey, hat das Christenthum nicht nur den (\RWparnr{145}) angeführten Nutzen bei unzähligen Menschen wirklich hervorgebracht; sondern überdieß manche schädliche Irrthümer verdrängt, auf welche die menschliche Vernunft in dieser Rücksicht verfallen war; nämlich
\begin{aufzb}
\item daß diese Welt das Werk des blinden Ohngefährs sey, daß sie durch den zufälligen Zusammenstoß der von Ewigkeit her in Bewegung gewesenen Atome entstanden sey, wie dieses Demokritus und Epikur mit ihrer ganzen Schule gelehrt haben.
\item Andere meinten, die Welt habe ihr Daseyn eben so wie ihre Fortdauer einem nothwendigen Schicksale (\RWlat{fato}) zu danken, dem selbst die Gottheit unterworfen wäre. Daß weder diese, noch jene an eine Vollkommenheit der Welt glauben konnten, und daß ihr Glaube an Gott ihnen im Grunde nichts nützen konnte, fällt in die Augen.
\item Noch Andere meinten, der höchste Gott habe die Schöpfung der Welt, zum wenigsten der Erde und der Geschöpfe, die sie bewohnen, gewissen untergeordneten endlichen Geistern überlassen (dem \RWgriech{dhmiourg`os}). Sie mußten eben darum der Welt verschiedene Unvollkommenheiten beilegen.
\item Andere glaubten sogar, daß mehrere Geister, die an der Einrichtung der Welt Antheil gehabt, böse neidische Mächte~\RWSeitenw{268}\ gewesen wären, welche die Welt absichtlich so schlecht eingerichtet hätten (\RWbet{Manichäer}).
\end{aufzb}
\item Der Irrthum, daß Gott die Welt aus einem schon vorhandenen Stoffe, einer von Ewigkeit her vorhandenen, von ihm unabhängigen Materie, bloß gebildet habe, mit seinen nachtheiligen Folgen, herrschte vor der Einführung des Christenthums bei den vernünftigsten Philosophen, \zB\ bei den Stoikern. Der Wahn, daß die Materie an sich böse sey, und daß man, um vollkommen zu werden, sich möglichst entsinnlichen müsse, war eine dieser Folgen.
\item Da das Christenthum aus einem Grunde, den wir später anführen wollen, die Anfangslosigkeit der Welt nicht lehren konnte: so mußte man einen etwas unrichtigen Begriff der Schöpfung aufstellen, nämlich, daß sie eine Hervorbringung aus Nichts (gleichsam in der Zeit) sey. Dieser Begriff konnte die nachtheilige Folge haben, daß man die einmal geschaffene Welt für unabhängig von Gott annehmen konnte. Um diesem Irrthume vorzubeugen, mußte das Christenthum zur Lehre von der Schöpfung noch die von der Erhaltung hinzuthun, welche von einem offenbaren Nutzen und keinem Mißbrauche ausgesetzt ist.
\item Wie herzerhebend und tröstlich ist nicht die Lehre von Gottes Vorsehung für Millionen Menschen geworden! Wie vielen Tausenden ist sie die einzige Stütze, der einzige Trost in ihrem Unglücke gewesen! Wie trostlos war dagegen, was der Glaube der Heiden mit sich brachte, und was auch wir ohne Zweifel, wenn uns das Christenthum nicht eines Bessern belehrt hätte, mit ihnen glauben würden, daß Gott sich gar nicht um die Welt bekümmere, oder nur für die Erhaltung ganzer Geschlechter, nur für die Schicksale besonders wichtiger Menschen sorge.\par
\RWbet{Einwurf.} Die Lehre von Gottes ganz besonderer Fürsorge hat Manche träge gemacht. Sorget nicht, was ihr essen und trinken, oder womit ihr euch bekleiden wollet; euer himmlischer Vater weiß, daß ihr Alles dessen bedürfet. In diesen Worten Jesu glaubte mancher leichtsinnige und träge Mensch eine Entschuldigung für seine Lebensart zu finden.~\RWSeitenw{269}\par
\RWbet{Antwort.} Das wollen wir nicht ganz läugnen; aber wer sollte glauben, daß der Schade, den dieser Mißbrauch angerichtet hat, größer gewesen sey, als der Nutzen, den diese Lehre gestiftet?
\item Eben so offenbar ist auch der Nutzen des fünften Punctes.\par
\RWbet{Einwurf.} Aber diese Lehre ist öfters mißverstanden worden, und hat den stolzen Wahn erzeugt, als ob der Mensch des Schöpfers einziger Zweck, wo nicht im ganzen Weltall, doch auf der ganzen Erde wäre. Daher erlaubte man sich dann die grausamste Behandlung der thierischen Welt; denn, sagte man, sie ist nur unsertwegen da.\par
\RWbet{Antwort.} Ganz ist dieser Mißbrauch nicht zu läugnen; aber wir finden, daß die Christen des stolzen Wahnes, von welchem hier geredet wird, und jener Grausamkeit gegen die Thiere sich eben nicht in einem höhern Grade schuldig gemacht haben, als andere Menschen. Dieses verhinderten nämlich so viele andere Lehren des Christenthums, unter andern der Umstand, daß uns das Christenthum Barmherzigkeit gegen die Thiere ausdrücklich vorschreibt.
\item Der Glaube des sechsten Punctes ist keinem Mißbrauche ausgesetzt.
\item Wenn man den Nutzen, welchen das Christenthum durch diese Lehre bisher gestiftet hat, gehörig schätzen will, so muß man nicht vergessen, daß es demselben gelungen ist, diese Lehre nicht nur unter seinen eigenen Bekennern allgemein herrschend zu machen, sondern ihr auch sogar bei manchen andern Völkern, die übrigens das Christenthum nicht ganz angenommen, oder das angenommene wieder verlassen haben, Eingang zu verschaffen. Z.\,B.\ bei den Mahomedanern.
\end{aufza}

\RWpar{147}{Warum das Christenthum die Lehre von der Unendlichkeit der Welt nach Raum und Zeit nicht aufgestellt habe?}
Ich habe bereits in der natürlichen Religion bemerkt, daß und warum ich dafür halte, daß die Welt, wenn man~\RWSeitenw{270}\ darunter den Inbegriff aller geschaffenen Wesen versteht, weder dem Raume noch der Zeit nach bestimmte Grenzen habe. Daß nun gewisse geschaffene Wesen, \zB\ namentlich unsere eigene Seele, in alle Ewigkeit fortdauern werde, lehret auch das Christenthum ausdrücklich; daß aber die Welt auch ohne Anfang bestehe, und daß sie eben so auch dem Raume nach gar keine Grenzen habe, das wird im Christenthume nicht ausdrücklich vorgetragen; im Gegentheile kommen in den Büchern der heil.\ Schrift so manche Stellen vor, aus welchen man sogar einen Beweis vom Gegentheile glaubte ableiten zu können; und es ist wirklich die entgegengesetzte Meinung, fehlt es ihr gleich nicht an einzelnen Vertheidigern, worunter sich zum Theile auch einige sehr angesehene Kirchenschriftsteller befinden, von der großen Menge der Christen von jeher für irrig angesehen worden. Es fragt sich also, warum das Christenthum diesen Irrthum, wofern es einer ist, zugelassen habe, und warum uns insbesondere die heil.\ Schrift nicht eines Besseren belehre? Auf diese Frage glaube ich nun erwiedern zu können:
\begin{aufza}
\item Die Lehre von der unendlichen Ausdehnung der Welt dem Raume sowohl als der Zeit nach scheint nur für denjenigen faßlich genug zu seyn, dessen Verstand eine gewisse \RWbet{wissenschaftliche Bildung} erreicht hat; denn nur ein solcher wird keinen Widerspruch in der Behauptung finden, daß diese Welt einerseits eine Wirkung, und andererseits gleichwohl ewig sey. Für einen im Denken ganz Ungeübten erscheint dieß widersprechend; weil er sich vorstellt, daß eine jede Ursache früher als ihre Wirkung seyn müsse. Soll er sich also die Welt als abhängig oder gewirkt von Gott denken: so muß er sich dieselbe auch als entstanden in der Zeit denken; und soll er im Gegentheile sie als ewig denken: so, glaubt er, könne sie nicht eine Wirkung seyn. Beinahe eben so schwer oder unmöglich wird es ihm auch, die Welt dem Raume nach unbegrenzt zu denken. Offenbar ist es aber ein weit geringeres Uebel, die Ewigkeit der Welt zu verkennen, als die Abhängigkeit derselben von Gott in Zweifel zu ziehen. Das Christenthum hat also recht gethan, daß es das Erstere verschwieg, zumal da es ein Mittel wußte, den wichtigsten Nachtheil, den der unrichtige Begriff von Gottes Schöpfung~\RWSeitenw{271}\ etwa hervorbringen könnte (nämlich die Meinung, daß die zu einer Zeit einmal geschaffene Welt von nun an der Beihülfe Gottes minder bedürfe, ja, auch wohl ganz ohne ihn fortdauern könne), durch die Lehre von der Erhaltung zu verhindern.
\item Aber durfte die christliche Offenbarung die Lehre von der Unendlichkeit der Welt aus dem erwähnten Grunde auch nicht ausdrücklich vortragen: so widerspricht sie ihr doch nicht geradezu. Die Stellen der heiligen Schrift, aus welchen man etwas der Art beweisen wollte, lassen sich bequem auch so auslegen, daß der scheinbare Widerspruch verschwindet. Zuvörderst ist zu bemerken, daß die heil.\ Schrift überhaupt weder \RWbet{für} Gelehrte, noch, ihrem größeren Theile nach, \RWbet{von} Gelehrten geschrieben sey; und dieß ist schon Grundes genug, zu begreifen, weßhalb sie die Lehre von der Unendlichkeit der Welt nirgends ausdrücklich aufstellen konnte. Gleichwohl sind ihre Ausdrücke überall so, daß sie ihr nicht widersprechen.
\begin{aufzb}
\item \RWbibel{Gen}{1\,Mos.}{1}{1}: \erganf{\RWbet{Im Anfange} schuf Gott Himmel und Erde. Die Erde aber war wüste und leer, und Finsterniß deckte den Abgrund}, \usw\ In dieser Stelle ist von der \RWbet{Schöpfung der Welt} nur in dem 1.\,V.\ die Rede; in den folgenden wird von der \RWbet{Ausbildung unserer Erde} (die erst vor beiläufig 6000 Jahren Statt fand) gesprochen. Daß also die ganze Welt nicht älter sey, als 6000 Jahre, das wird in dieser Stelle mit keiner Sylbe gesagt. Der Ausdruck: \RWbet{Im Anfange} \RWhebr{b*:re'+siyt} (\RWlat{Bereschit}) zwingt nicht einmal, an einen Anfang in der Zeit zu denken; sondern er \RWbet{könnte} eben so gut: \RWbet{Von Ewigkeit her} übersetzt werden, wie jener ihm entsprechende griechische Ausdruck (\RWbibel{Joh}{Joh.}{1}{1}) \RWgriech{>en >arq~h|}, \erganf{von Ewigkeit her} übersetzt werden muß.
\item Die Redensart: \RWbet{Vor der Weltschöpfung} (\RWgriech{pr`o katabol~hs k'osmou}), welche in den Büchern des n.\,B.\ so häufig vorkommt, läßt sich in allen Stellen als ein Vorherseyn nicht der Zeit, sondern \RWbet{dem Range}, oder \RWbet{dem Grunde} nach erklären. Z.\,B.\ \RWbibel{Joh}{Joh.}{17}{5} wörtlich: \erganf{Und nun, Vater! verherrliche mich mit jener Herrlichkeit, welche ich bei dir hatte vor der Welt-Schö\RWSeitenw{272}pfung}; was man auch so deuten könnte: \erganf{Verherrliche mich mit jener Herrlichkeit, die du mir zugedacht hattest, bevor du den Rathschluß der Schöpfung einer Welt gefaßt}, \dh\ mit jener Herrlichkeit, deren Beförderung so manche schon in der Welt getroffene Einrichtungen bezwecken. -- Auf ähnliche Art läßt sich auch die Stelle \RWbibel{Joh}{Joh.}{17}{24} erklären, wo noch zu bemerken ist, daß gleich darauf \RWgriech{k'osmos} vom Menschengeschlechte gebraucht wird. Vergleiche auch \RWbibel{Eph}{Ephes.}{1}{4}, \RWbibel{1\,Petr}{1\,Petr.}{1}{20} Der Sohn ist Ursache von der Welt, er kann also \erganf{\RWbet{vor} der Welt} heißen. Die Verherrlichung des Sohnes, die Erwählung dieser oder jener Menschen zur Seligkeit wird als Zweck Gottes betrachtet, weßhalb er der Welt gerade diese und keine andere Einrichtungen gegeben, also kann man sagen: Der Sohn besaß seine Herrlichkeit beim Vater, Gott habe diese und jene Menschen erwählt \usw\ vor der Welt-Schöpfung.
\item \Ahat{\RWbibel{Ps}{Ps.}{90}{2}}{89,2.} \erganf{Ehe die Berge entstanden, und die Erde und ihre Feste gegründet waren, von Ewigkeit zu Ewigkeit warst du, o Gott!} -- Hier ist nur von der Erde, nicht aber von der Welt die Rede.
\item \RWbibel{Spr}{Sprichw.}{8}{22} wird von der Weisheit erzählt, wie sie bei der Welt-Schöpfung zugegen gewesen, und gleichsam Gott geholfen habe. Die Weisheit war mitwirkende Ursache bei der Schöpfung, also konnte man allerdings sagen, daß sie früher gewesen sey, nicht der Zeit, aber dem Grunde nach. Denn daß man den Grund nicht nur in der gemeinen, sondern auch in der wissenschaftlichen Sprache früher als die Folge nennt, unterliegt keinem Zweifel. So sagt man \zB , weil zwei Seiten und der eingeschlossene Winkel die dritte Seite und den Inhalt des Dreieckes bestimmen -- zwei Seiten und der eingeschlossene Winkel müßten früher da seyn, als die dritte Seite und der Inhalt.
\end{aufzb}
\item Uebrigens ist die Lehre von der endlosen Ausdehnung der Welt dem Raume sowohl als der Zeit nach für denjenigen, der sie gehörig zu fassen vermag, nicht ohne Nutzen.
\begin{aufzb}
\item Sie hebt einen sehr wichtigen Einwurf, welchen man gegen die Allmacht, Weisheit und Heiligkeit Gottes daher~\RWSeitenw{273}\ nimmt, daß er die Welt nicht früher geschaffen, oder ihr keinen größeren Umfang gegeben habe.
\item Durch sie wird eine andere erhabene Lehre, nämlich die von der höchsten Vollkommenheit der Welt begreiflich, und gegen einen sonst unauflöslichen Einwurf gerettet.
\item Erst durch diese Lehre zeigt sich uns Gottes Allmacht und Größe in ihrem ganzen Lichte; denn wenn wir die Welt ihrem Umfange und ihrer Dauer nach auch noch so groß und alt annehmen: so ist sie doch immer nichts gegen diejenige Welt, mit der uns der Glaube an die Endlosigkeit der Welt bekannt macht.
\item Also gewinnt auch unsere Bewunderung, Liebe und Dankbarkeit gegen Gott, und wir fühlen, daß er eine wirklich unendliche Bewunderung und Dankbarkeit von uns verdiente, wenn wir derselben fähig wären.
\end{aufzb}
\end{aufza}

\RWpar{148}{Die christliche Lehre vom letzten Zwecke der göttlichen Schöpfung und Weltregierung}
Es ist gesagt worden, daß Gott die Welt zu einem gewissen Zwecke geschaffen habe, und wirklich hinleite. Fragen wir nun, welches eigentlich dieser Zweck sey: so gibt uns das Christenthum die Antwort: \erganf{Es ist die Ehre Gottes, oder die Verherrlichung der göttlichen Eigenschaften.} Fragen wir weiter, aus welchem Grunde Gott diese Verherrlichung gesucht habe: so sagt uns das Christenthum: \erganf{Nicht aus eitler Ehrliebe, sondern aus einem andern Grunde, der in seinen eigenen Vollkommenheiten liegt.} Fragen wir endlich, ob dieser Zweck die möglichst größte Beförderung der Tugend und Glückseligkeit der geschaffenen Wesen etwa ausschließe: so erwiedert das Christenthum: \erganf{Nicht im Geringsten; sondern eben dadurch, daß Gott Tugend und Glückseligkeit befördert, befördert er auch seine Ehre.}

\RWpar{149}{Historischer Beweis dieser Lehre}
\begin{aufza}
\item \Ahat{\RWbibel{Ps}{Ps.}{19}{2}}{18,1.}\ heißt es: \erganf{Die Himmel verkündigen die Ehre Gottes; das Firmament die Werke seiner Hände.} Aus~\RWSeitenw{274}\ dieser Stelle läßt sich wenigstens schließen, daß die Welt zur Verherrlichung Gottes diene; und da Gott jeden guten Erfolg beabsichtiget, diese Verherrlichung Gottes aber ohne Zweifel ein guter Erfolg ist: so kann man sagen, daß Gott die Welt unter Anderem \RWbet{auch} um seiner Verherrlichung willen geschaffen habe, aber noch nicht, daß dieses der einzige Zweck sey.
\item Nur eben so viel beweiset auch die Stelle Pauli \RWbibel{Röm}{Röm.}{1}{20}: \erganf{Was an ihm unsichtbar ist, seine ewige Macht und Göttlichkeit, das ist seit der Weltschöpfung an den Geschöpfen sichtbar.} Doch kann man aus dieser Stelle vermuthen, daß dieser Zweck wenigstens ein \RWbet{Hauptzweck} gewesen sey.
\item Zu eben dieser Vermuthung berechtigen noch mehr gewisse Reden Jesu, in welchen er es als den ganzen Zweck seiner Sendung betrachtet: den Vater zu verherrlichen. -- Er hätte in allen diesen Stellen eben so gut sagen können: Tugend und Glückseligkeit unter den Menschen zu verbreiten; da er aber die erstere Redensart der letzteren vorzog: so gab er uns zu erkennen, daß wir deßgleichen thun sollen. Die christliche Kirche folgte diesem Winke, und drückte den letzten Zweck Gottes bei der gesammten Weltschöpfung und Regierung lieber durch die Redensart: \erganf{Gottes Verherrlichung} als durch die andere: \erganf{Tugend und Glückseligkeit} aus. So \RWbibel{Joh}{Joh.}{17}{1}\ Vergl.\ auch \RWbibel{Mt}{Matth.}{5}{16}\ \RWbibel{1\,Kor}{1\,Kor.}{10}{31}\ \RWbibel{2\,Thess}{2\,Thessal.}{1}{16}\ \ua
\item Daß eigentlich der Grund, weßhalb Gott die Welt schuf, nur in ihm selbst, und in seinen eigenen Vollkommenheiten lag, schließen die Theologen aus \RWbibel{Spr}{Sprichw.}{16}{4}, welches die Vulgata übersetzt: \RWlat{Universa \RWbet{propter semet ipsum} operatus est Dominus, impium quoque ad diem malum.}
\item Daß aber diese Verherrlichung von Gott nicht darum gesucht werde, weil er der Ehre \RWbet{zu seiner eigenen Glückseligkeit} bedarf, darüber ist in der Kirche nie ein Zweifel gewesen; denn daß Gott sich selbst genug sey, und daß seine Glückseligkeit nicht erhöht werden könne, gab man allenthalben zu.~\RWSeitenw{275}
\item Daß Tugend und Glückseligkeit durch die Angabe dieses Zweckes nicht ausgeschlossen werde, ist gleichfalls allgemein zugegeben worden. Wir haben auch Schriftstellen, die dafür sprechen; \zB\ \RWbibel{1\,Tim}{1\,Tim.}{2}{4}: \erganf{Gott will, daß alle Menschen zur Erkenntniß der Wahrheit kommen, und dadurch selig werden.}
\end{aufza}

\RWpar{150}{Vernunftmäßigkeit}
\begin{aufza}
\item Diese Lehre des Christenthums ist eine von denjenigen, welche den beißendsten Spott der Philosophen erfahren mußten, namentlich des \RWbet{Bayle, Derniers, David Hume}, \uA\ Sie sagten, es sey der gröbste Anthropomorphismus, den man sich denken könne, zu behaupten, daß Gott die Welt seiner Ehre wegen geschaffen habe; denn weil derjenige, der nur der Ehre wegen handelt, ehrsüchtig ist: so wäre Gott, der Alles -- Alles nur um seiner Ehre willen gethan hätte, das allerehrsüchtigste Wesen. Welche Gotteslästerung! --\par
\RWbet{Antwort.} Wir antworten, daß man hier in der Hitze des Streites vergessen habe, es sey nur derjenige ehrsüchtig zu nennen, der seine Ehre \RWbet{um des Vergnügens willen, das sie ihm gewähret}, sucht; wer aber die Ehre sucht um irgend eines sittlichen Grundes willen, der kann nicht ehrsüchtig genannt werden. Die katholische Kirche sagt nun keineswegs, daß Gott seine Verherrlichung um des Vergnügens willen suche, das er an ihr finde, sondern sie gibt gerne zu, daß Gott irgend einen vernünftigen Grund habe, um dessentwillen er selbst diese Ehre sucht.
\item \RWbet{Einwurf.} Aber so ist diese Verherrlichung nicht mehr der absolut letzte Zweck, sondern ein untergeordneter.\par
\RWbet{Antwort.} Dieß ist freilich wahr; aber es ist zu bemerken, daß der Ausdruck: Letzter Zweck, von der Kirche nicht in streng wissenschaftlicher Bedeutung genommen werde; wie sie denn überhaupt niemals ihre Ausdrücke in dieser Bedeutung nimmt. Sie redet nie davon, wie etwas an sich selbst beschaffen seyn dürfte, sondern wie wir Menschen es uns vorstellen sollen. Der letzte Zweck Gottes bei der Weltschöpfung ist seine Ehre -- heißt also: Derjenige Zweck, welchen wir~\RWSeitenw{276}\ Menschen uns bei der Weltschöpfung vorstellen sollen, und zwar als den höchsten, ist Gottes Ehre. Aber auch hiemit will die Kirche noch eben nicht sagen, daß wir nicht zuweilen in der Stunde der wissenschaftlichen Untersuchungen noch weiter zu gehen versuchen dürften; sondern nur, daß es für einen Jeden aus uns, auch für den Gelehrten, gut sey, sich außer der Zeit der Speculation, im wirklichen Leben, keinen andern Zweck, als den der Verherrlichung Gottes zu denken. Und hierin hat sie, wie wir bald sehen werden, sehr recht.
\item Das Uebrige in dieser Lehre, daß nämlich Gott auch die Beförderung der Tugend und Glückseligkeit berücksichtige, bedarf keiner Rechtfertigung; die Vernunft findet es nothwendig.
\end{aufza}

\RWpar{151}{Sittlicher Nutzen}
Unter allen Zwecken, die wir der Gottheit bei der Erschaffung, Erhaltung und Regierung dieser Welt beilegen könnten, finden wir keinen, der, wenn auch nicht eben in der Stunde der philosophischen Betrachtung uns als der richtigste erschiene, doch im wirklichen Leben wohlthätiger auf uns einwirken könnte, als -- die Verherrlichung Gottes.
\begin{aufza}
\item Wenn wir (was ohne Zweifel noch das Vernünftigste wäre) die möglichst größte Beförderung der Tugend und Glückseligkeit bei den geschaffenen Wesen uns als den letzten Zweck der Schöpfung zu denken angewöhnten: so könnten wir kaum vermeiden, daß wir in mancher leichtsinnigen Stunde nicht die falsche Folgerung daraus herleiteten: Weil denn doch nur die Glückseligkeit der lebenden Wesen den letzten Zweck der ganzen Schöpfung ausmacht, was stehst du an, dir diese oder jene Handlung zu erlauben, die dich so glücklich macht? Daß diese Handlung uns nicht wahrhaft beglücke, oder daß sie Andere unglücklich mache, \udgl , das würde uns nicht deutlich genug einleuchten; wir würden uns zu ihrer Ausübung entschließen. Dagegen, haben wir uns einmal die Vorstellung, daß Gottes Ehre der oberste und letzte Zweck der ganzen Schöpfung sey, daß also auch wir (wie jedes andere freie Wesen) die Ehre Gottes zum letzten Zwecke aller unserer Handlungen zu machen haben, geläufig gemacht: so werden wir in dieser Versuchung weit glücklicher~\RWSeitenw{277}\ seyn; denn (werden wir sprechen) es ist Gottes Befehl, daß du dieß lassen sollst, seine Verherrlichung fordert, daß du ihm hierin gehorchest.
\item Da wir wissen, daß Gott nur dadurch verherrlichet wird, wenn seine vernünftigen Geschöpfe ihn nach seinen erhabenen Vollkommenheiten immer völliger erkennen, und ihm immer bereitwilliger gehorchen: so fühlen wir uns durch diese Lehre gerade zu dem wichtigsten Geschäfte, nämlich zur immer völligeren Erkenntniß Gottes, und zu einem immer pünctlicheren Gehorsam gegen seinen Willen aufgefordert; und sehen zugleich ein, daß wir auch in Bezug auf unsere Nebenmenschen nichts Wichtigeres zu thun haben, als die Erkenntniß Gottes und die Befolgung seines Willens bei ihnen auszubreiten. Wie oft vergessen wir nicht auf dieses Wichtigste! wie gut also, daß uns das Christenthum daran erinnert!
\item Erwägen wir, daß Gott seine Ehre nur in der Beglückung seiner Geschöpfe sucht: so lernen wir daraus, worin auch wir unsere wahre Ehre zu suchen haben.
\item Der Gedanke, daß wir durch unsere Handlungen auch etwas zur Beförderung der Ehre Gottes beitragen können, welch eine edle Richtung ertheilt er nicht unserer Ehrbegierde! darin nur unsere Ehre zu suchen, daß wir die Ehre Gottes befördern!
\item Noch mehr, das Christenthum gestattet uns sogar den Gedanken: Wenn wir Gott verherrlichen: so wird auch er uns nicht zu Schanden werden lassen; belohnen wird er uns, als ob es ein Dienst wäre, den wir ihm selbst erwiesen hätten!
\item Und wenn es uns ja zuweilen einfallen sollte, daß Gott doch sehr ehrsüchtig sey, weil er nur seine Ehre überall befördert wissen will: so berichtiget uns das Christenthum hierüber sogleich durch die Erinnerung, daß er ja diese Ehre nicht um des Vergnügens willen, sondern weil seine innere Vollkommenheit es also fordert, befördert wissen will.
\end{aufza}

\RWpar{152}{Wirklicher Nutzen}
Da die so eben erwähnten Vortheile so ganz natürlich sind: so kann man nicht zweifeln, daß sie wirklich in unzäh\RWSeitenw{278}ligen Fällen Statt gefunden haben, und daß sehr vieles Gute, was die Christen ausgeübt haben, mitunter auf Rechnung dieser Lehre komme. Für rohe, sinnliche Menschen ist diese Lehre doppeltes Bedürfniß. Sie können es sich nämlich nicht einmal vorstellen, daß der Unendliche so ganz ohne irgend einen eigenen Vortheil sollte handeln können. Da ist denn unter allen Vortheilen der edelste offenbar jener der Ehre. Gott hat, so ist der Inhalt dieser Lehre für sie, Gott hat gar keinen Vortheil von der Welt, von ihrer Schöpfung, Erhaltung und Regierung, von allen Geschöpfen in ihr, und allen Handlungen derselben, sie mögen Gutes oder Böses thun, als etwa den einzigen, daß seine Ehre befördert, daß er verherrlichet wird. Bloß seiner Ehre wegen hat er dieß Alles geschaffen, sonst braucht er es in keinem Stücke für sich. --\par
\RWbet{Einwurf.} Aber wie sehr hat man diese Lehre nicht gemißbraucht! Wie viele Schandthaten haben die Menschen nicht verübt unter dem heiligen Vorwande: Zur Ehre Gottes!\par
\RWbet{Antwort.} Wahr; aber ob diese Schandthaten nicht auch verübt worden wären, wenn diese Lehre nicht gewesen wäre, ob sie dann nicht unverdeckter geübt und desto ärgerlicher geworden wären?

\RWpar{153}{Nähere Bestimmung der Art, wie Gott zur Ausführung seines letzten Zweckes in der Welt wirke}
Nachdem uns das Christenthum den Zweck eröffnet hat, den Gott bei seiner Schöpfung, Erhaltung und Regierung des Weltalls hat, bestimmt es noch etwas näher die Art, wie er eigentlich zur Ausführung dieses Zweckes wirke.
\begin{aufza}
\item Zu diesem Ende macht es zuvörderst aufmerksam darauf, daß Alles, was in dieser Welt ist und geschieht, sich unter \RWbet{zwei Classen} bringen lasse, deren die Eine Alles dasjenige begreift, von dessen Daseyn Gott der \RWbet{einzige und vollkommen zureichende Grund} ist, \zB\ die Himmelskörper \udgl ; die andere enthält Alles dasjenige, was den zureichenden Grund seines Daseyns \RWbet{nicht bloß in Gott allein, sondern in Gott und in den freien Wil}\RWSeitenw{279}\RWbet{lensentschließungen eines geschaffenen freien Wesens} hat. Von dieser Art sind \zB\ die freien Handlungen der Menschen und was von ihnen abhängt, \zB\ Veränderungen, welche die Menschen durch ihre freie Thätigkeit auf Erden hervorbringen.
\item Die Dinge der ersten Classe, oder diejenigen, welche ihr Daseyn nur Gott allein zu verdanken haben, besitzen die \RWbet{höchste Vollkommenheit,} die nur gedenkbar ist, \dh\ sie entsprechen dem letzten Zwecke, welchen sich Gott bei seiner Schöpfung vorgesetzt hat, so gut, daß ihm nichts Anderes besser entsprechen könnte.
\item Nicht so die Dinge der zweiten Classe, \dh\ die freien Handlungen der frei geschaffenen Wesen, und Alles, was mittelbar von ihnen abhängt. Der Einfluß, welchen Gott auf diese freien Handlungen hat, ist seiner Natur nach \RWbet{kein ganz bestimmender}, sondern besteht in Folgendem:
\begin{aufzb}
\item daß er die freien Wesen selbst geschaffen hat und erhält;
\item daß er ihnen Kräfte und Fähigkeiten, also zu handeln, gibt, nämlich die Vorstellung der Handlung selbst (mittelbar oder unmittelbar), das mehr oder weniger deutliche Gefühl des Rechtes und Unrechtes, allerlei mehr oder weniger starke Aufmunterungsgründe, Reize und Lockungen zum Gegentheile, \usw ;
\item daß er die wirkliche Ausführung des schon gefaßten Entschlusses bald \RWbet{zuläßt}, bald auch durch äußere Umstände \RWbet{verhindert.}
\end{aufzb}
\item Diese freien Handlungen entsprechen eben deßhalb dem Zwecke der Welt nicht schlechterdings, sondern dieß thun nur
\begin{aufzb}
\item die guten, und
\item manche böse, die Gott durch seine Weisheit so leitet, daß sie das Wohl des Ganzen befördern, oder demselben doch wenigstens nicht zuwider sind.
\end{aufzb}
\item Eben um diese freien Handlungen der freien Wesen zum letzten Zwecke der Welt gehörig zu benützen, um die Guten zu belohnen und die Bösen zu bestrafen \usw , richtet sich Gott bei seiner Anordnung der Dinge der ersten Classe nach diesen freien Handlungen.~\RWSeitenw{280}
\item Und kann dieses um so mehr, da er dieselben vorhersieht, bevor sie noch verübt worden sind; diese Fähigkeit Gottes nennt man die göttliche \RWbet{Vorherwissenheit} (\RWlat{praescientia divina}).
\end{aufza}

\RWpar{154}{Historischer Beweis dieser Lehre}
\begin{aufza}
\item \RWbibel{Sir}{Sir.}{15}{11}: \erganf{Sprich nicht, mein Abfall kommt von Gott; denn was er hasset (dir verboten hat), das hättest du nicht thun sollen. Sprich nicht, er hat mich irre geführt; denn er bedarf keines Sünders. -- Er schuf im Anfange den Menschen, und überließ ihn seinem freien Willen. Wenn du willst, kannst du die Gebote halten, und wohlgefällige Rechtschaffenheit üben. Feuer und Wasser hat er dir vorgelegt; strecke deine Hand aus, wornach du willst}, \usw\ \RWbibel{Sir}{Sir.}{10}{18}: \erganf{Der Stolz ist dem Menschen nicht anerschaffen, weder die Zornwuth den Kindern der Weiber.} -- Es gibt also zweierlei Classen von Dingen in der Welt, solche, die von Gott allein, und solche, die zum Theile von der Freiheit der erschaffenen Wesen herrühren.
\item \RWbibel{Gen}{1\,Mos.}{1}{31}: \erganf{Und Gott sah Alles an, was er gemacht hatte, und siehe! es war vollkommen gut.} \RWbibel{Sir}{Sir.}{18}{1}: \erganf{Der ewig lebt, schuf Alles ohne Ausnahme; der Herr allein ist tadellos. Niemanden gab er das Vermögen, seine Werke zu verkündigen. Wer kann seine Wunder erforschen? wer messen seine Macht? wer seine Güte nach Würde rühmen? Sie kann weder ab- noch zunehmen, und Gottes wunderbare Thaten sind unerforschlich.} -- \RWbibel{Ps}{Psalm}{111}{2}: \erganf{Erhaben sind des Ewigen Thaten, allen ihren Zwecken angemessen.}
\item[3.~und 4.]\setcounter{enumi}{4} Bedarf keines eigenen Beweises, weil es sich theils von selbst versteht, theils auch aus anderen Lehren des Christenthums klar ist. Siehe \RWbibel{Gen}{1\,Mos.}{45}{8}\ \RWbibel[19,10.]{Gen}{1\,Mos}{19}{10} \uam\ 
\item Hieher gehört \zB\ die Sendung des Sohnes Gottes, \uam 
\item \RWbibel{Ps}{Ps.}{139}{1}\par
\erganf{Herr, du erforschest mich, und wenn ich sitze oder stehe:\par
Dir ist's bekannt; du prüfst von fernher, was ich denke,~\RWSeitenw{281}\par
Du hast mir Gang und Lager angemessen,\par
Und meine Wege alle angeführt.\par
Bevor ein Wort auf meiner Zunge schwebte,\par
Hast du es, Herr! schon ganz gewußt.\par
Ja aufgezeichnet sind in deinem heiligen Buche\par
Die Tage, die mir werden sollten,\par
Als keiner noch derselben war.}
\end{aufza}
Dasselbe beweisen auch die vielen Weissagungen, die in den Büchern des a.\ und n.~Bundes vorkommen, über Begebenheiten, deren Erfüllung von freien Menschenhandlungen abhing.

\RWpar{155}{Vernunftmäßigkeit}
Die Puncte 1.\ 2.\ 3.\ sind für sich klar.
\begin{aufza}\stepcounter{enumi}\stepcounter{enumi}\stepcounter{enumi}
\item Es ist zwar in philosophischer Bedeutung nicht ganz richtig gesagt, daß irgend etwas, das Gott zuläßt (auch selbst die bösen Handlungen der freien Wesen), seinem Zwecke nicht ganz entsprechend wäre, nämlich, wenn man das Wort \RWbet{Zweck} in seiner \RWbet{eigentlichen} Bedeutung nimmt, und einen Actus des göttlichen Willensvermögens darunter versteht. In dieser Bedeutung gehen alle Zwecke Gottes in Erfüllung, und die katholische Kirche lehrt dieß selbst, wenn sie sagt, daß Gottes Rathschlüsse nie vereitelt werden könnten. Hier also nimmt sie dieß Wort in einer andern Bedeutung, und nennt einen Zweck Gottes jede an sich selbst wohlthätige Wirkung, die Gott so sehr befördert, als sie nur immer von seiner Seite befördert werden kann. Der Sinn des gegenwärtigen Artikels ist also kein anderer, als: Die freien Handlungen der frei geschaffenen Wesen befördern das Wohl des Ganzen oder was sonst jene gewiß wohlthätige Wirkung ist, die Gott in seinem Weltall, so viel es bei der freien Handlungsweise der Wesen möglich ist, bezwecket, nicht immer in dem Maße, wie sie es könnten, wenn sie anders wären. Wenn die freien Wesen immer gut handeln würden: so würde es in der Welt weit mehr Glückseligkeit geben. Und ist dieses etwa nicht wahr?
\item versteht sich von selbst.
\item Daß Gott \RWbet{Vorherwissenheit} habe, \dh\ daß er auch das Zukünftige wisse, ist eine nothwendige Folge seiner~\RWSeitenw{282}\ Allwissenheit. Die Schwierigkeit aber, welche in der Vereinigung dieser Vorherwissenheit Gottes mit der Freiheit des Menschen und anderer freigeschaffener Wesen liegt, möchte auch noch so groß seyn: so beweiset sie nichts, als die Schwäche unserer Vernunft. So viel ist gewiß, daß durch Gottes Vorherwissenheit unsere Freiheit nicht aufgehoben werden könne; denn nicht darum müssen wir etwas beschließen, weil Gott vorhergesehen hat, daß wir es beschließen; sondern umgekehrt, weil wir es beschließen, so sieht es Gott vorher.
\end{aufza}

\RWpar{156}{Sittlicher Nutzen}
\begin{aufza}
\item Wie nothwendig die Unterscheidung Nr.\,1.\ sey, erhellet aus mehr als einem Grunde. Ohne sie würden wir an keine Freiheit glauben können; der Lasterhafte würde sich entschuldigen, daß es Gott selbst sey, der ihn zwinge. Hiezu kommt, daß wir auch die gleichfolgenden Puncte ohne sie nicht verstehen könnten.
\item Daß Alles, was den völlig zureichenden Grund seines Daseyns (mittelbarer oder unmittelbarer Weise) \RWbet{nur in Gott selbst hat}, seinem Zwecke auf das Vollkommenste entspreche, \dh\ die Tugend und Glückseligkeit so sehr befördere, als es nur an sich möglich ist, müssen wir nothwendig wissen, wenn wir an seine Weisheit und Heiligkeit glauben sollen, und wenn uns dieser Glaube etwas nützen soll.
\item Durch diese Lehre erhalten wir die stärkste Aufmunterung zur Tugend, wenn wir hören, daß Gott Alles, was er nur thun kann, ohne uns zu zwingen, anwendet, um uns zur Tugend zu leiten. Wir können daher auch hoffen, wenn wir mit seiner Unterstützung mitwirken, über jede Versuchung zu siegen, und alle Schwierigkeiten auf dem Pfade der Tugend zu überwinden.
\item Durch diese Bemerkung werden uns
\begin{aufzb}
\item für's Erste die Unvollkommenheiten in dieser Welt erklärlicher; wir rechnen sie nicht mehr Gott an, sondern den fehlerhaften Handlungen der freien Wesen.
\item Die entgegengesetzte Vorstellung würde alle Verbindlichkeit zur Tugend, ja ihr inneres Wesen selbst aufheben.~\RWSeitenw{283}\ Denn wenn das Wohl des Ganzen selbst durch die bösesten Handlungen unfehlbar befördert würde, und zwar in eben dem Maße, wie durch die guten: so würde es eben darum gar keine in Wahrheit böse Handlung mehr geben.
\item Im Gegentheile, wenn uns gesagt wird, daß es die bösen Handlungen allein sind, welche den höchsten Grad der Vollkommenheit der Welt verhindern: so können wir nicht einen Augenblick zweifeln, daß Gott sie hart bestrafen werde, weil sie das Einzige sind, was ihm sein schönes und sonst so vollkommenes Werk, die Welt, verdirbt.
\item Und dieses bleibt wahr, auch wenn uns das Christenthum weiter, um unsere Begriffe von Gottes Weisheit zu erhöhen, sagt, daß Gott manche böse Handlungen gleichwohl sehr zur Beförderung des allgemeinen Wohles zu benützen wisse; denn weil der Sünder nie vorauswissen kann, ob dieß bei \RWbet{seiner} Handlung der Fall seyn werde, so kann er sich damit auf keine Weise entschuldigen. Die Welt mag Nutzen von seinen Handlungen haben; er aber wird immer Strafe verdienen und erfahren.
\item Für den Guten dagegen hat diese Nachricht den wichtigen Vortheil, daß er sich nun auch bei dem Anblicke der größten Lasterthaten beruhigen kann; denn er tröstet sich nun, daß Gott weiß, warum er dieß zuläßt, er versteht, gute Folgen auch aus dem Bösen zu ziehen.
\end{aufzb}
\item Wir müßten uns die Welt viel unvollkommener denken, wenn wir nicht glauben sollten, daß Gott bei seiner Leitung derselben auf die freien Handlungen der freien Wesen Rücksicht nehme; wir könnten dann nicht einmal wohl begreifen, daß diese allemal ihre Belohnung und Strafe finden werden.
\item Stellen wir uns vor, daß Gott die Handlungen der freien Wesen erst dann erkennt, wenn sie schon wirklich ausgeübt sind: so können wir nicht glauben, daß er schon früher schickliche Anstalten, um die schädlichen Folgen dieser Handlungen zu hintertreiben, die guten Folgen auf's Beste zu benützen, getroffen habe. Durch die entgegengesetzte Vorstellung aber gewinnt unser Begriff von der Vollkommenheit der Welt, wie auch von der Zusammengesetztheit des Geschäftes, welches die göttliche Weltregierung heißt. Gott erhält jetzt um~\RWSeitenw{284}\ so viel mehrere Mittel, die Tugend zu belohnen, das Laster zu bestrafen, die bösen Handlungen der Menschen zum Guten hinzuwenden, und aus den guten Handlungen derselben den größten Vortheil für das Ganze zu ziehen; denn weil er sie von Ewigkeit schon vorhersah: so konnte er schon in den Anstalten, die er von Ewigkeit her getroffen hat, Rücksicht auf alle diese Handlungen nehmen, und Alles so einrichten, wie es am Vortheilhaftesten wäre. Ueber die Weisheit Gottes aber, die dieß voraussetzt, erstaunen wir um so mehr.\par
\end{aufza}
\RWbet{Einwurf.} Die Lehre von Gottes Vorhersehung einer freien Handlung, und Alles desjenigen, was von ihr abhängt, scheint aber gleichwohl den Nachtheil zu haben, daß wir uns oft ganz unnöthiger Weise mit dem Gedanken beschäftigen werden: Was mag doch Gott von mir vorherwissen? werde ich denn selig werden oder nicht?\par
\RWbet{Antwort.} Auf diese Frage können wir uns, so oft sie uns einfällt, die Antwort geben: Was Gott von dir vorhersehe, liegt an dir selbst. Mache also, daß er nur Gutes von dir vorhersehe. So wird denn diese Frage, statt für uns unnütz zu seyn, vielmehr uns zum Guten aufmuntern. Siehe \RWlat{Thomae a Kempis de imitatione Jesu Christi. Lib.\,1. cap.\,25. Nr.\,2}.

\RWpar{157}{Wirklicher Nutzen}
Gegen den wirklichen Nutzen, den diese Lehrsätze gestiftet haben, könnte man einwerfen, daß Nr.\,4. und 6.\ (zwar nur durch Mißverstand) hie und da einen doppelten Schaden angerichtet haben mögen:
\begin{aufzb}
\item Die Lehre, daß Gott selbst böse Handlungen zum Guten zu benützen wisse, mag manchen Sünder verleitet haben, nur um so getroster fortzusündigen.
\item Die Lehre, daß Gott alle unsere Handlungen schon von Ewigkeit vorhergesehen habe, mag manchen Aengstlichen beunruhigt, oder ihm die Freiheit unserer Handlungen verdächtig gemacht haben. Ich antworte hierauf:
\item Daß der erste Mißbrauch gewiß nur selten Statt gefunden habe, weil eben das Christenthum, welches bekennt,~\RWSeitenw{285}\ daß Gott selbst die bösen Handlungen zu seinem Zwecke zu benützen wisse, versichert, daß dieser Umstand dem Sünder zu keiner Entschuldigung diene, daß er für seine Person darum keine geringere Strafe erfahren werde.
\item Ein Gleiches gilt von dem zweiten Mißverstande, zumal da die Freiheit des Menschen eine von jenen Wahrheiten ist, die uns ein gesunder Menschenverstand am Allerwenigsten bezweifeln läßt. Das Christenthum versichert uns auch ausdrücklich, daß es seine vollkommene Richtigkeit mit dieser Freiheit habe. Also war es höchstens ein schwer zu lösendes Räthsel für die Gelehrten, wie die Vorherwissenheit Gottes mit unserer Freiheit zusammen bestehen könne. Und sehr wahrscheinlich, daß das Bestreben, dieses uns von der christlichen Offenbarung gegebene Räthsel befriedigend zu lösen, Antheil an mancher gar nicht unwichtigen Entdeckung des menschlichen Verstandes hat.
\end{aufzb}

\RWpar{158}{Warum das Christenthum die Lehre von der besten Welt nicht aufgestellt habe?}
\begin{aufza}
\item Aus dem Begriffe von Gottes höchster Vollkommenheit, von seiner grenzenlosen Allmacht, Weisheit und Heiligkeit folgt, daß diese Welt, sein Werk, \RWbet{die beste und vollkommenste aus allen andern gedenkbaren Welten seyn}, \dh\ daß die Summe der Tugend und Glückseligkeit in ihr die möglichst höchste (ein \RWlat{Maximum}) seyn müsse.
\item Der wichtigste Einwurf, welchen man dieser Lehre von der besten Welt entgegengesetzt hat, ist dieser: Die Welt, so vollkommen sie auch immer seyn mag, ist und muß doch, als ein geschaffenes und von Gott abhängiges Ding, beschränkt und endlich seyn. Alles Beschränkte und Endliche kann aber vollkommener gedacht werden; also kann man nicht sagen, daß diese Welt die vollkommenste sey, die Gott möglich war, ohne die Allmacht Gottes zu beschränken.\par
Auf diesen Einwurf glaube ich erwiedern zu können, daß die Welt, ob sie gleich abhängig von Gott und in ihren einzelnen Theilen beschränkt und endlich ist, dennoch im Ganzen unbeschränkt und unendlich sey. Obwohl man also aus~\RWSeitenw{286}\ dem angeführten Grunde von jedem einzelnen Wesen sagen kann, daß Gott ein vollkommeneres erschaffen könne: so läßt sich dieses doch nicht von der \RWbet{ganzen Welt}, in wiefern sie ein Ganzes ausmacht, behaupten.
\item Aber obgleich ich die Lehre von der vollkommensten Welt aus diesen Gründen für wahr und richtig halte: so muß ich doch gestehen, daß sie von der Kirche nicht nur nicht aufgestellt, sondern von der Mehrzahl ihrer Mitglieder sogar bestritten und verketzert worden sey. Der heil.\ \RWbet{Augustin} zwar war dieser Meinung zugethan. \RWlat{Lib.\,4. de Genesi\ 16.} schreibt er: \erganf{\RWlat{Si bona facere posset Deus, nec faceret, magna esset invidentia.}} Und in \RWlat{lib.\ quaestionum, quaest.\,50.} beweiset er die gleiche Wesenheit des Sohnes mit dem Vater aus folgendem Grunde: \erganf{\RWlat{Deus, quem genuit, quoniam se majorem gignere non potuit (nihil enim Deo melius) generare debuit aequalem; si enim voluit, et non potuit, infirmus est; si potuit et non voluit, invidus est.}} So behauptet er auch in seinen Schriften öfters, \erganf{\RWlat{id quod Deus vult, semper melius esse}} (\zB\ \RWlat{contra Faustum lib.\,19. cap.\,1.}) -- Auch \RWbet{Johannes Damascenus} (im 7ten und 8ten Jahrhunderte) war dieser Meinung. \RWlat{De fide orthodoxa lib.\,2.\ cap.\,19.}: \erganf{\RWlat{Necesse est, omnia, quae Dei providentia fiunt, secundum rectam rationem, et optima et Deo decentissima fieri, ita, ut nequeant fieri meliora.}} -- Endlich auch\RWbet{ Abälard} (\RWlat{introduct.\ theolog.\ lib.\,3. cap.\,5.}) war dieser Meinung zugethan, der aber bekanntlich wegen seiner eigenthümlichen Ansichten übel berufen war. Unter den Akatholischen vertheidigte \RWbet{Wiklef} (im 14ten Jahrhunderte) die Behauptung: \erganf{\RWlat{Deum non posse mundum majorare vel minorare.}} -- \RWbet{Leibniz} trug diese Lehre unter dem Namen des \RWlat{Optimismus} in seiner Theodicee vor, und \Ahat{\RWbet{Wolff}}{\RWbet{Wolf}} entlehnte sie von ihm und breitete sie durch seine Schriften weiter aus. Unter den katholischen Theologen des vorigen Jahrhunderts fand aber diese Meinung fast allgemeinen Widerspruch, und man stellte den Satz auf: \erganf{\RWlat{Deus potest meliora facere iis, quae fecit}}, und zwar aus dem schon angeführten Grunde (Nr.\,2.).~\RWSeitenw{287}
\begin{RWanm}
Unter den heidnischen Philosophen war auch \RWbet{Plato} unserer Meinung, und Abälard hatte sie eben aus seinen Schriften geschöpft.
\end{RWanm}
\item Nun entsteht die Frage: Wenn dieser \RWlat{Optimismus} eine richtige Lehre ist, warum hat ihn die christliche Offenbarung nicht aufgestellt, sondern vielmehr zugelassen, daß er von der Mehrzahl der christlichen Theologen verkannt und bestritten worden ist? -- Ich antworte:
\begin{aufzb}
\item Wenn die Welt endlich ist: so ist die Behauptung, daß Gott keine vollkommenere Welt als diese gegenwärtige habe erschaffen können, offenbar, statt der Ehre Gottes günstig zu seyn, ihr vielmehr nachtheilig und durchaus ungereimt. Dann nämlich würde man durch diese Behauptung der Allmacht und Weisheit Gottes bestimmte Schranken setzen; man würde behaupten, daß Gott nur einige Tausend, oder nur einige Millionen, und nicht mehr Geschöpfe habe erschaffen können, welches entehrend für ihn und ungereimt ist. -- Da nun, wie wir schon oben bemerkten, die Lehre von der Unendlichkeit der Welt ihrer schweren Begreiflichkeit wegen nicht zu einer Offenbarungslehre in einer Volksreligion geeignet ist: so ist es auch nicht die Lehre von der Vollkommenheit der Welt, da diese jene voraussetzt, wenn sie nicht schädlich werden soll.
\item Um aber den Nutzen des \RWlat{Optimismus} nicht zu verlieren, stellt das Christenthum den ausdrücklichen Lehrsatz auf, daß diese Welt, wofern sie nicht die vollkommenste ist, welche Gott hätte schaffen können, doch sicher diejenige ist, die seinen heiligen Absichten auf's Vollkommenste entspricht. Und dieser Lehrsatz ist der Sache nach mit dem \RWlat{Optimismus} ganz einerlei, und nur im Ausdrucke unterschieden.\par
Der Nutzen, den der \RWlat{Optimismus} haben kann, wird auch durch diesen Lehrsatz erreicht, nämlich
\begin{aufzc}
\item die Weisheit Gottes wird in ihr gehöriges Licht gestellt, und
\item wir werden zufrieden gemacht mit allen Einrichtungen, die wir in der Welt antreffen.~\RWSeitenw{288}
\end{aufzc}
\end{aufzb}
\end{aufza}

\RWpar{159}{Die Lehre des Christenthums von den Engeln}
Das Christenthum begnügt sich nicht bloß damit, uns von der Welt im Ganzen genommen sehr richtige Begriffe beizubringen, nämlich daß sie ihr Daseyn nur durch den Willen Gottes habe, von Gott fortwährend erhalten und regiert werde, und eben deßhalb die größte Vollkommenheit besitze, wenn man darunter versteht, daß sie dem heiligen Zwecke entspreche, zu dem sie Gott geschaffen hat. Ueber dieß Alles macht uns die christliche Religion mit den \RWbet{einzelnen Theilen dieser Welt}, \dh\ mit den Geschöpfen, welche sich in ihr befinden, so viel es nöthig oder dienlich für uns ist, bekannter. -- Von den leblosen Werken Gottes zwar, etwa von der Beschaffenheit, Größe und Anzahl der Himmelskörper, von ihren Umlaufszeiten und Verbindungen unter einander, und was dergleichen mehr ist, schweigt die göttliche Offenbarung gänzlich. Diese Kenntnisse sind einmal schon nicht so nothwendig für uns, sind ferner für gewisse Menschenclassen nicht einmal faßlich genug, würden Einigen sogar so unglaublich vorkommen, daß ihnen um dieser Sätze willen auch andere Offenbarungslehren verdächtig würden, und endlich, könnten wir uns ja diese Kenntnisse, so viel es dienlich für uns ist, durch unsere eigene Beobachtung und eigenes Nachdenken erwerben.\par
Die christliche Offenbarung also läßt sich bloß aus über das Daseyn und die Eigenschaften gewisser \RWbet{lebender und für uns unsichtbarer Geschöpfe} in diesem Weltgebäude, mit denen wir noch dazu in einer näheren Verbindung stehen. Sie nennt sie \RWbet{Geister} oder \RWbet{Engel} (\RWgriech{>'aggeloi}, \RWgriech{da'imones}). Ihre Lehre von diesen Engeln (christliche Dämonologie oder Geisterlehre) ist kürzlich folgende:
\begin{aufza}
\item In Gottes Schöpfung gibt es noch außerhalb des menschlichen Geschlechtes
\begin{aufzb}
\item eine \RWbet{zahllose Menge vernünftiger Wesen}, mitunter auch
\item von \RWbet{weit vollkommenerer Natur}, als es wir Menschen sind, und~\RWSeitenw{289}
\item von \RWbet{verschiedenem Range}. Die Leiber, welche sie besitzen, wofern sie solche besitzen, sind
\item \RWbet{nicht von sinnlicher Natur}, daher sie auch
\item \RWbet{keiner sinnlichen Vergnügungen fähig} sind.
\end{aufzb}
\item Mehrere aus diesen Geisterwesen \RWbet{stehen auch mit unserer Sinnenwelt in Verbindung}, und zwar in einer uns Menschen nicht völlig zu bestimmenden Verbindung.
\begin{aufzb}
\item Sie nehmen \RWbet{Antheil an unseren sittlichen Angelegenheiten};
\item können durch Gottes Zulassung \RWbet{manche Veränderungen in der uns umgebenden irdischen Welt,}
\item und dadurch mittelbar \RWbet{auch in unsern Gemüthern} hervorbringen.
\item Sie benützen den ihnen gestatteten Einfluß \RWbet{größtentheils} gut, und \RWbet{wirken wohlthätig auf uns}. Manche gute Gedanken, welche in uns erwachen, sind nur durch ihre Vermittlung in uns hervorgebracht.
\item \RWbet{Wünsche und Fürbitten}, die sie bei Gott für uns einlegen, \RWbet{haben eine besondere Wirksamkeit.}
\item Ist es auch nicht gewiß, daß es für jeden einzelnen Menschen einen eigenen Engel (Schutzengel) gebe, welchem die Sorgfalt für diesen Menschen von Gott besonders anvertraut wäre: so ist es doch gewiß, daß viele einzelne Personen von besonderer Wichtigkeit solche eigene Schutzengel haben, und daß ein \RWbet{jeder Mensch irgend Einem} (mehrere etwa demselben) \RWbet{höheren Wesen zur Fürsorge zugewiesen} sey.
\item In außerordentlichen Fällen hat sich Gott solcher Engel bedient, um durch sie besonders wohlthätige Zwecke auf Erden auszuführen, wobei er ihnen die Macht einräumt, selbst manche \RWbet{außerordentliche Erscheinungen} in unserer Sinnenwelt (sogenannte Wunder) zu bewirken. So hatte er insbesondere schon öfters Engel in menschlicher Gestalt erscheinen lassen, welche den Menschen gewisse, besonders wichtige Nachrichten ertheilen mußten, \usw~\RWSeitenw{290}
\end{aufzb}
\item Obgleich die Engel weit vollkommener sind, als wir Menschen, so sind sie als endliche Geister doch immer \RWbet{fehlbar}.
\begin{aufzb}
\item Und wirklich \RWbet{einige derselben sind gänzlich vom Guten abgefallen}, sie haben sich durch \RWbet{Stolz} und \RWbet{Ungehorsam} gegen Gott versündiget, und sind am Ende höchst unmoralische und bösgesinnte Wesen geworden.
\item Aber Gott hat sie auch dafür sehr hart gestraft, und sie \RWbet{ewig von seinem Angesichte verworfen.} Wir nennen sie \RWbet{böse Engel} oder \RWbet{Teufel} (\RWgriech{di'aboloi}, Versucher, \RWgriech{da'imones}).
\end{aufzb}
\item Auch diesen Engeln gestattet Gott noch \RWbet{einigen Einfluß auf unserer Erde}.
\begin{aufzb}
\item Sie haben \RWbet{Wohlgefallen an dem Bösen}, und weiden sich an dem Unglücke der Menschen.
\item Daher suchen sie denn die Menschen auf allerlei Weise \RWbet{zur Sünde zu verleiten}, und dadurch unglücklich zu machen.
\item \RWbet{Ihrer Verführung ist wirklich das meiste Uebel, die meisten Sünden der Menschen zuzuschreiben}, insonderheit die vielen falschen und abgöttischen Religionen, die es auf Erden gab, oder noch gibt, sind nur durch ihren Einfluß eingeführt worden; zumal da ihnen Gott zuweilen, wie es \zB\ besonders zu den Zeiten Jesu geschah, die Macht einräumte,
\item \RWbet{verschiedene sichtbare Veränderungen}, als: ungewöhnliche, schmerzhafte Krankheiten (Besitzungen genannt) und allerlei scheinbare Wunderwerke hervorzubringen.
\end{aufzb}
\end{aufza}

\RWpar{160}{Historischer Beweis. Plan desselben}
Nicht leicht ist eine Lehre des Christenthums in neueren Zeiten häufiger bestritten und selbst von Seite der katholischen Theologen zweideutiger vertheidigt worden, als die katholische Lehre von den Engeln. Allerlei thörichte Vorurtheile, welche der gemeine Volkshaufe mit dieser Lehre in Verbindung brachte, machte sie in den Augen der Aufgeklärteren verächtlich; ein~\RWSeitenw{291}\ und der andere Mißbrauch, den man sich mit ihr erlaubte, erregte Unwillen; gewisse scheinbare Einwürfe wider die Vernunftmäßigkeit derselben, die man nicht gleich zu heben wußte, brachten den Wunsch hervor, sie aus dem Verzeichnisse der katholischen Glaubenslehren lieber ganz wegzustreichen, und wohl nur diesem Wunsche hat man die Behauptung zuzuschreiben, welche in neueren Zeiten zuerst unter den Protestanten zum Vorschein gekommen, dann selbst von einigen katholischen Theologen nachgesprochen worden ist: daß der Glaube an Engel, besonders an böse Engel, und an Besitzungen derselben, nicht einmal der Religion des alten Bundes wesentlich eigen sey; daß das jüdische Volk diesen Glauben erst um die Zeit der assyrisch-babylonischen Gefangenschaft von den Chaldäern angenommen habe; daß Jesus denselben nur darum nicht bestritten habe, weil er dieß Vorurtheil schon zu tief eingewurzelt bei seinen Zeitgenossen fand, den schädlichen Folgen desselben aber durch gewisse andere Lehren hinlänglich vorgebeugt hatte, auch wohl vorhersehen konnte, daß die Aufklärung späterer Zeiten den Wahn, wenn er erst reif seyn würde, von selbst darniederreißen werde.\par
Um den Grund oder Ungrund dieser Behauptung gehörig würdigen zu können, wollen wir hier, so viel es die uns gesteckten Grenzen erlauben, untersuchen, was
\begin{aufzb}
\item jene Bücher des alten Bundes, die aller Wahrscheinlichkeit nach noch vor dem babylonischen Exile geschrieben sind, von der Geisterwelt lehren; dann hören wir,
\item was in den übrigen Büchern des a.\,B.\ noch Mehreres gelehrt wird; und endlich prüfen wir,
\item was Jesus und die Bücher des n.\,B.\ darüber lehren.
\end{aufzb}

\RWpar{161}{a)~Die Lehre der ältesten Bücher des alten Bundes von den Engeln}
\begin{aufza}
\item Schon in den Büchern Mosis (\RWbibel[1\,Mos.\ 18.\ u.\ 19.\ Cap.]{Gen}{1\,Mos.}{18--19}{}) wird erzählt, daß dem Abraham drei Wesen in menschenähnlicher Gestalt, drei Männer im Thale Mamre erschienen wären, welche er Anfangs für drei Reisende (den~\RWSeitenw{292}\ Einen darunter für den Vornehmsten) gehalten, und sie aus Gastfreundschaft bei sich bewirthet habe. (Der Erzähler selbst nennt den Einen aus ihnen Jehova, die beiden andern Engel.) Der Eine aus ihnen sagt dem Abraham vorher, daß er in seinem Alter noch einen Sohn von Sara erhalten werde; dann eröffnete er ihm, er sey eben jetzt gekommen, um Sodoma zu bestrafen; die beiden andern verfügen sich zu Loth, melden ihm, sie wären von Gott gesandt, um Sodoma zu bestrafen. Hieraus erhellet, daß man wenigstens schon zu den Zeiten, wo diese Offenbarung niedergeschrieben wurde, an das Daseyn gewisser Engel geglaubt habe, die man für höhere Geister hielt, die Gott zuweilen zu den Menschen sende, und die gewisse Zwecke auszuführen hätten. So auch \RWbibel{Gen}{1\,Mos.}{21}{17}\ \RWbibel{Gen}{}{22}{11}\ \RWbibel{Gen}{}{24}{7}\ \RWbibel{Gen}{}{28}{12}\ \RWbibel{Gen}{}{31}{11} Aus \RWbibel{Gen}{1\,Mos.}{48}{16}: \erganf{Der Engel, der mich von allem Bösen errettet hat} -- möchte man wohl gar schließen, daß man schon damals eigene Schutzengel geglaubt habe.
\item Auch den Glauben an \RWbet{böse Geister} finden wir in den 5 BB.\ Mos. -- Wahrscheinlich gehört hieher auch schon \RWbet{jene Schlange}, welche die ersten Menschen im Paradiese verführt hatte, denn obgleich (\RWbibel{Gen}{1\,Mos.}{3}{1}) gar nicht gesagt wird, daß diese Schlange von einem bösen Geiste besessen gewesen, sondern bloß der Ausdruck gebraucht wird: die Schlange aber war das listigste unter allen Thieren, die Gott geschaffen hatte: so scheint doch aus der ganzen Erzählung, in welcher die Schlange nachher als redend eingeführt wird, hervorzugehen, daß der Erzähler sie für besessen von einem bösen Geiste gehalten habe, da er doch nicht so thöricht gewesen seyn konnte, zu glauben, daß Schlangen von sich selbst reden könnten. Noch entscheidender ist es, wenn \RWbibel{Ex}{2\,Mos.}{15}{11} und in mehreren andern Stellen von falschen Gottheiten die Rede ist, über welche der wahre Gott Jehova erhoben wird. Wer unter den Göttern ist dir gleich, Jehova? wer glänzt wie du an Heiligkeit? und wer ist so ruhmwürdig wie du? wer so wunderthätig? Diese falschen Gottheiten mußten also doch \RWbet{etwas Wirkliches} seyn; für endliche geschaffene Wesen, für \RWbet{böse Geister} mußte sie Moses halten.~\RWSeitenw{293}
\item Das Buch \RWbibel[\RWbet{Hiob}]{Hiob}{}{}{}, welches doch so alt ist (nach Einigen Mosen zum Verfasser hat), enthält auch schon den Glauben an Geister. Gott wird hier vorgestellt als ein Regent, vor dessen Thron zur bestimmten Zeit die \RWbet{Kinder Gottes} (Engel) kommen, und ihm Bericht von ihren Thaten erstatten. Es kommt auch Satan, der ihm von Hiob erzählt, und sich die Freiheit ausbittet, diesen auf einige Zeit zu quälen. Ueberdieß heißt es\par
\RWbibel{Hiob}{Hiob}{38}{6}: \erganf{Wer setzte ihren Eckstein (der Erde) ein beim Jubel aller Morgensterne, \RWbet{als jauchzete der Kinder Gottes Schaar?}} Also sind die Engel noch vor Ausbildung der Erde zugegen gewesen.\par
\RWbibel{Hiob}{Hiob}{4}{17}: \erganf{Der Mensch, ist er wohl rein vor seinem Schöpfer? Sieh! seinen Dienern traut er nicht, auf seine Engel legt er Fehler.} Daselbst \RWbibel{Hiob}{}{10}{18}\ Also haben die Engel auch ihre \RWbet{Fehler}.
\item \RWbibel{Ps}{Psalm}{8}{6}: \erganf{Ein wenig nur geringer hast du ihn (den Menschen) gemacht, als deine Engel.}\par
\RWbibel{Ps}{Psalm}{68}{18}: \erganf{Myriaden sind des göttlichen Gefolges. Myriaden himmlischer Mächte, in ihrer Mitte der Herr; -- so Sinai im Heiligthume.}
\end{aufza}

\RWpar{162}{b)~Die Lehre der spätern Bücher des alten Bundes von den Engeln}
Aus \RWbibel{2\,Sam}{2\,Sam.}{24}{1} verglichen mit \RWbibel{1\,Chr}{1\,Chron.}{21}{1}, wo in der ersten Stelle der Einfall des Königs David, sein Volk zu zählen, der Einflüsterung irgend eines Menschen, in der zweiten dem \RWbet{Satan} zugeschrieben wird, kann man allerdings schließen, daß die jüdische Nation in späteren Zeiten eine Einwirkung böser Geister häufiger, als in früheren angenommen habe. Aber folgt hieraus wohl, daß der spätere Glaube der minder begründete seyn müsse?
\begin{aufza}
\item \RWbet{Isaias} redet (\RWbibel{Jes}{}{6}{2}) von Seraphim (Edle, Brennende), die den Thron Gottes umgeben, und das dreimal Heilig singen.
\item \RWbet{Daniel} (\RWbibel{Dan}{}{7}{10}) sieht in seinem nächtlichen Gesichte tausendmal Tausende, die Gott bedienen, und hundert\RWSeitenw{294}mal Tausende, die vor ihm stehen. -- Das sind nun freilich dichterische Bilder; aber es mußte ihnen, nach der Vorstellung des Dichters, doch etwas Wahres zu Grunde liegen.
\item Im Buche der \RWbibel{Weish}{Weish.}{2}{24} heißt es, \RWbet{durch den Neid des Teufels} sey die Sünde in die Welt gekommen.
\item Im Buche \RWbibel[\RWbet{Tobias}]{Tob}{}{}{}\ erscheint ein Engel mit Namen Raphael (\dh\ Arzt Gottes) in menschenähnlicher Gestalt, der sich dem jungen Tobias zum Leitgefährten anbietet, \usw\ Immerhin Dichtung; sie setzt doch den Glauben voraus, daß es Geister gebe, und daß sie sich um menschliche Angelegenheiten bekümmern.
\end{aufza}

\RWpar{163}{c)~Die Lehre der Bücher des neuen Bundes von den Engeln}
\begin{aufza}
\item Es gibt
\begin{aufzb}
\item noch außerhalb des Menschen eine \RWbet{zahllose Menge} vernünftiger Wesen unter dem Namen Engel. \RWbibel{Hebr}{Hebr.}{12}{22}: \erganf{Ihr seyd gekommen zu der Menge von \RWbet{tausendmal tausend Engeln}.} -- Und Jesus spricht bei \RWbibel{Mt}{Matth.}{26}{53} von \RWbet{Legionen Engeln}.
\item Sie sind \RWbet{vollkommenerer Natur} als wir. \RWbibel{Lk}{Luk.}{20}{36} verspricht uns Jesus, daß wir nach der Auferstehung \RWbet{den Engeln gleichen} sollen. \RWbibel{Mt}{Matth.}{18}{10} sagt er von ihnen, daß sie \RWbet{das Angesicht Gottes unaufhörlich anschauen}. \RWbibel{Mt}{Matth.}{6}{10} empfiehlt er uns den Wunsch, daß Gottes Willen auf Erden eben so von uns Menschen erfüllt werde, wie von den Engeln im Himmel. \Ahat{\RWbibel{1\,Petr}{1\,Petr.}{1}{12}}{\RWbibel{1\,Petr}{Petr.}{1}{2}} sagt von den Anstalten Gottes zur Erlösung des menschlichen Geschlechtes, daß sie ein Geheimniß seyen, das selbst Engel zu durchschauen wünschen.
\item Von \RWbet{verschiedenem Range.} Die Schriften des alten Bundes melden von Cherubim und Seraphim, \udgl\  Paulus schreibt \RWbibel{Kol}{Koloss.}{1}{16}: \erganf{Durch ihn (den Sohn Gottes) ist Alles erschaffen im Himmel und auf Erden, das Sichtbare und das Unsichtbare, selbst Thronen, Herrschaften, Mächte, Gewalten.} Die letzten Worte bezeichnen, wie man sehr deutlich sieht, gewisse Rangordnungen~\RWSeitenw{295}\ unter den Engeln, die der Apostel, nach der damaligen Vorstellungsart, aufzählt. Dasselbe thut er \RWbibel{Eph}{Ephes.}{1}{21} auch \RWbibel{Eph}{}{3}{10}
\item Haben \RWbet{keine sinnliche Leiber und keine sinnliche Vergnügungsart}. Diese Vorstellung hatte schon der Verfasser des Buches Tobias, wenn er den Engel Raphael folgender Maßen redend anführt: Es schien euch zwar, als ob ich gegessen oder getrunken hätte; ich aber bediene mich nur \RWbet{unsichtbarer Speisen}, und eines Trankes, der nie den Menschen werden kann. Auch Jesus scheint (\RWbibel{Lk}{Luk.}{24}{39}) diese Vorstellung vorauszusetzen, wenn er nach seiner Auferstehung seine Jünger zu überzeugen sucht, daß er es selbst, und kein bloßer Geist sey, der sich ihnen darstellt; und sie aufmerksam macht, daß Geister ja \RWbet{nicht Fleisch und Bein haben}. So auch \RWbibel{Mt}{Matth.}{22}{30} Auch Paulus zählt 
(\RWbibel{Kol}{Koloss.}{1}{16}) die Engel zu den \RWbet{unsichtbaren} Wesen. Und \RWbibel{Eph}{Ephes.}{6}{12} sagt er: \erganf{Wir haben \RWbet{nicht mit Fleisch und Blut} zu kämpfen; sondern gegen die Mächte und Gewalten, gegen die Beherrscher der Finsterniß, die bösen Geister unter dem Himmel.}
\end{aufzb}
\item Mehrere aus diesen Engeln \RWbet{stehen mit uns in Verbindung}.
\begin{aufzb}
\item Sie kümmern sich um unsere sittlichen Angelegenheiten. Jesus sagt bei \RWbibel{Lk}{Luk.}{15}{10} sehr schön: \erganf{Ich sage euch, \RWbet{bei den Engeln Gottes ist Freude} über einen Sünder, der Buße thut.} Paulus schreibt an die \RWbibel{Hebr}{Hebr.}{1}{14} von den Engeln: \erganf{Sind sie nicht alle dienstbare Geister, zum Dienste derer ausgesandt, welche die Seligkeit erben sollen?}
\item Sie können \RWbet{einwirken auf uns.} Schon \RWbibel{Ps}{Psalm}{91}{11} heißt es: \erganf{Denn er befiehlt den Himmlischen, auf allen Wegen dich zu schützen. Sie müssen dich auf den Händen tragen, daß deinen Fuß nicht etwa ein Stein verletze.} \RWbibel{Apg}{Apostelg.}{12}{7} wird erzählt, wie ein Engel Petro erscheint, und ihn aus dem Kerker befreit.
\item Sie können \RWbet{mancherlei gute Gedanken} in uns erwecken. Zwar findet sich zufälliger Weise in der heil.~\RWSeitenw{296}\ Schrift keine Stelle, in welcher ein \RWbet{guter} Gedanke der Eingebung eines Engels ausdrücklich zugeschrieben würde, es sey denn in einem solchen Falle, wo dieser Engel den Menschen sichtbar erscheint. Allein da einigemal \RWbet{böse} Gedanken der Eingebung eines bösen Geistes zugeschrieben werden, ohne daß er auch sichtbar erscheint: so können wir nicht zweifeln, daß eben die Wirksamkeit, die bösen Engeln eingeräumt ist, um so mehr guten eingeräumt seyn werde. Dieß war auch in der That von jeher die Meinung der Kirche.
\item Ihre \RWbet{Fürbitte bei Gott hat eine besondere Wirksamkeit}. Was anders als dieß konnte es bedeuten, wenn Raphael \RWbibel{Tob}{Tob.}{12}{12} es anrühmt, daß er die Gebete dieses Tobias vor Gottes Thron dargebracht habe? Eben so bringt (\RWbibel{Offb}{Offenb.}{8}{4}) ein Engel die Gebete der Gläubigen unter Weihrauchdampf vor Gottes Thron.
\item Einzelne \RWbet{Menschen sind einzelnen Engeln zur besonderen Sorgfalt} empfohlen. Jesus spricht (\RWbibel{Mt}{Matth.}{18}{10}): \erganf{Nehmet euch in Acht, daß ihr Keines von diesen Kleinen gering achtet; denn ich sage euch, ihre Engel im Himmel schauen allezeit das Angesicht meines Vaters.} Der Ausdruck: \RWbet{ihre} Engel, zeigt an, daß diesen Menschen einzelne Engel zugetheilt sind. Das muß man auch in dem Hause des Markus geglaubt haben, wo man den Engel des Petrus zu sehen glaubte. (\RWbibel{Apg}{Apstg.}{12}{1--2}) Ob übrigens jeder einzelne Mensch seinen eigenen Schutzengel habe, darüber sind die Meinungen der Kirchenväter getheilt.
\item In außerordentlichen Fällen können die Engel auch \RWbet{außerordentliche Erscheinungen auf Erden} hervorbringen, sich selbst in sichtbaren Gestalten darstellen, \udgl\  Dieses setzt der Verfasser des Buches Tobias ausdrücklich voraus. Und schon bei Moses (in den oben angezeigten Stellen) wird dieses vorausgesetzt. Auch im neuen Bunde kommen mehrere Erscheinungen von Engeln vor, die meistens gewisse Botschaften auszurichten haben; und es scheint, daß sie eben deßhalb den griechischen Namen \RWgriech{>'aggeloi} (Boten) erhalten haben.~\RWSeitenw{297}
\end{aufzb}
\item Es gibt auch
\begin{aufzb}
\item \RWbet{böse Engel}, ob sie gleich anfangs gut geschaffen waren. \RWbibel{Joh}{Joh.}{8}{44} sagt Jesus zu den Juden: \erganf{Der Teufel ist euer Vater; und die Wünsche dieses eures Vaters möchtet ihr gerne vollziehen. Vom Anfange war er ein Menschenmörder, und bestand nicht in der Wahrheit; denn in ihm ist keine Wahrheit. Wenn er lügt: so spricht er, was ihm recht eigen ist; denn er ist ein Lügner und ein Vater des Lügners.} Also ist es durch lange Gewohnheit gleichsam die zweite Natur des Teufels (\RWgriech{di'abolos}), Unwahrheit zu sprechen. Daß er aber im Anfange gut geschaffen war, scheint der Ausdruck: \RWbet{er bestand nicht in der Wahrheit}, gänzlich anzudeuten. Deutlicher aber beweiset dieses noch, was bei \RWbibel{Jud}{Judas}{6}{}\ vorkommt: \erganf{Jene Engel, die ihren ursprünglichen Zustand nicht behaupteten, sondern ihren Wohnsitz verließen, hat Gott bis zu dem großen Gerichtstag mit ewigen Banden in der Finsterniß aufbewahret.} -- Die entgegengesetzte Meinung, daß einige Engel schon \RWbet{böse geschaffen} wären; hat man in der Kirche von jeher als Ketzerei verworfen. \RWbet{Manichäer, Priscillianisten} glaubten das.
\item Sie werden für ihre Bosheit von Gott \RWbet{ewig bestraft}. \RWbibel{2\,Petr}{2\,Petr.}{2}{4}: \erganf{Gott hat nicht einmal der Engel geschont; sondern sie in die Hölle gestürzt, und an die Ketten der Finsterniß gefesselt, und zum Gerichtstage aufbewahrt.} Jesus sagt \RWbibel{Mt}{Matth.}{25}{41}, daß er einst zu den bösen Menschen sprechen werde: \erganf{Gehet hin, ihr Vermaledeiten, in das ewige Feuer, welches dem Teufel und seinen Engeln bereitet ist.}
\item Auch den bösen Engeln ist ein \RWbet{gewisser Einfluß auf die Menschen} gestattet. Dieses beweiset schon die oben angeführte Stelle \RWbibel{Weish}{Weish.}{2}{24}, welche sich allem Anscheine nach auf \RWbibel{Gen}{1\,Mos.}{3}{} bezieht. \RWbibel{Apg}{Apostelg.}{26}{18} spricht Jesus zu dem Saulus: \erganf{Ich will dich zu den Heiden senden, damit sie von der Finsterniß zum Licht, von der Gewalt des Satans zu Gott sich bekehren.} -- Also sind die Nichtchristen, die Heiden vornehmlich, unter der Herrschaft des Satans. Derselbe Ge\RWSeitenw{298}danke kommt auch vor \RWbibel{Kol}{Koloss.}{1}{13} Daher wird auch gesagt, daß Jesus Christus gekommen sey, die Werke des Teufels zu zerstören (\RWbibel{1\,Joh}{1\,Joh.}{3}{8}). Und er sagt von sich selbst (\RWbibel{Joh}{Joh.}{12}{31}): \erganf{Jetzt ergehet das Gericht über die Welt, und der Fürst dieser Welt wird vom Throne gestoßen.}
\item Die bösen Geister können auch \RWbet{sichtbare Veränderungen auf Erden}, scheinbare Wunderwerke hervorbringen. Diesen Glauben setzen alle jenen Stellen der heil.\ Schrift (alten und neuen Bundes) voraus, wo einzelne Erscheinungen auf Erden gewissen bösen Geistern zugeschrieben werden, wobei es uns gleichviel ist, ob die heiligen Schriftsteller in diesen besondern Fällen Recht oder Unrecht haben. Hieher gehört \zB\ schon die Versuchungsgeschichte Jesu, welche von den Evangelisten so erzählt wird, daß man deutlich sieht, sie haben ein wirkliches Factum zu erzählen geglaubt. Z.\,B.\ \RWbibel{Mt}{Matth.}{4}{1}. Ferner gehören hieher die sogenannten Besessenen (\RWgriech{daimoniz'omenoi}), von welchen uns in den Evangelien so viel erzählt wird; denn unläugbar haben die Erzähler, und allem Anscheine nach hat auch Jesus selbst die sonderbaren Krankheiten und Leiden, mit welchen jene Unglücklichen behaftet waren, für Wirkungen böser Geister gehalten. Den Glauben, daß böse Geister auch Wunder wirken können, setzen die Worte Jesu voraus; \RWbibel{Mt}{Matth.}{24}{24}: \erganf{Es werden falsche Propheten und falsche Messiase auftreten, und große Zeichen und Wunder thun.} Noch deutlicher sagt dieß der heil.\ Paulus \Ahat{\RWbibel{2\,Thess}{2\,Thess.}{2}{8}}{\RWbibel[2\,Thess.\ 2,9.]{2\,Thess}{2\,Thess.}{2}{8}}: \erganf{Um jene Zeit wird jener Bösewicht (Simon Magus vielleicht) erscheinen, den aber der Herr bei seiner sichtbaren Erscheinung mit einem Hauche seines Mundes zu Nichts machen wird. Wenn dieser auftreten wird: so wird er durch Satans Kraft Zeichen und Wunder aller Art, und mannigfaltige verführerische Werke unter den Verlornen ausführen.} -- Derselbe Glaube findet sich auch bei allen Kirchenvätern.\par
\RWbet{Einwurf.} Aber der heil.\ \RWbet{Hieronymus} schreibt in \RWbibel{Mt}{Matth.}{15}{19}: \erganf{Böse Gedanken kommen aus dem Herzen.}~\RWSeitenw{299}\ Folglich sind, nach diesem Ausspruche, diejenigen zu tadeln, die wähnen, daß dergleichen böse Gedanken vom Teufel eingeflüstert würden. Das \RWbet{\RWlat{Concilium Braccarense anno 500}} stellte den Anathematismus auf: \erganf{Wenn Jemand glauben würde, daß der Teufel auf dieser Welt irgend ein Geschöpf hervorbringe, oder Gewitter und Donner, Stürme und Trockene aus eigener Macht errege, wie Priscillianus behauptet, der soll verbannt seyn.} -- \RWbet{\RWlat{Burchardus Wormstatiens.}} schreibt: \erganf{Hast du geglaubt, was gewisse Leute vorgeben, sie können Ungewitter erregen? -- Wenn du es geglaubt: so sollst Du ein Jahr Buße thun. So fällt man in die Irrthümer der Heiden, da sie etwas Göttliches außerhalb des wahren Gottes glauben.} Noch zu Ende des 9ten Jahrhunderts schrieb \RWbet{Agobardus} ein ganzes Buch \RWlat{contra insulsam vulgi opinionem de grandine et tonitru.}\par
\RWbet{Antwort}: Der heil.\ \RWbet{Hieronymus} läugnet gar nicht, daß Versuchungen \RWbet{auch} vom Teufel herkommen könnten; denn er schreibt weiter: \erganf{Der Teufel kann zwar zu dergleichen Gedanken \RWbet{etwas beitragen}, und sie \RWbet{noch mehr entzünden}; aber sie erzeugen, das kann er nicht.} Das \RWbet{\RWlat{Concilium Braccarense}} eifert nur gegen diejenigen, welche das Daseyn gewisser Thiere (\zB\ der Fliegen, Mäuse \udgl ) und anderer Natureinrichtungen, \zB\ des Regens, Ungewitters \usw\, dem Teufel zuschreiben, und dieses ist allerdings eine sehr große und nachtheilige Thorheit. -- Dieselbe Antwort gilt auch von \RWbet{Burchard's} und \RWbet{Agobard's} Aeußerungen.
\end{aufzb}
\end{aufza}

\RWpar{164}{Vernunftmäßigkeit}
\begin{aufza}
\item Nichts kann vernünftiger seyn, als die Behauptung, daß es
\begin{aufzb}
\item noch außerhalb des Menschen eine zahllose Menge vernünftiger Wesen in Gottes Schöpfung gebe. Schon die bloße Vernunft kann es aus Gottes Heiligkeit und Allmacht herleiten, daß es eine, im strengsten Sinne, \RWbet{unendliche Menge} von Geschöpfen, und zwar von \RWbet{lebenden, der Glückseligkeit fähigen} Geschöpfen im Weltall geben müsse. Da ferner allem Anscheine nach~\RWSeitenw{300}\ die vernünftigen und freien Wesen eines weit höhern Grades der Glückseligkeit empfänglich sind, als die vernunftlosen: so können wir hieraus auch schließen, daß Gott die Menge der freien vernünftigen Wesen in seiner Schöpfung so groß als möglich angesetzt habe. Da wir endlich eine so zahllose Menge von Himmelskörpern, die unserem Erdballe mehr oder weniger ähnlich sind, antreffen: so können wir nicht zweifeln, daß es Gott möglich gewesen sey, auf diesen Himmelskörpern eben so gut, wie auf Erden lebendige, und insbesondere auch vernünftige und freie Wesen zu erschaffen. -- Ja aus dem Anblicke dieser Himmelskörper, und aus Erwägung der vielen Aehnlichkeiten, die sie mit unserer Erde haben, könnten wir, selbst ohne noch das Daseyn und die Güte Gottes vorauszusetzen, das Daseyn einer zahllosen Menge uns ähnlicher, \dh\ vernünftiger Geschöpfe mit größter Wahrscheinlichkeit vermuthen.\par
Jene Aehnlichkeiten sind:
\begin{aufzc}
\item die Himmelskörper haben eben so wie unser Erdball eine kugelförmige Gestalt;
\item sie sind von fester Masse, und
\item die meisten, wie es scheint, mit einem eigenen Dunstkreise umgeben.
\item Einige sind kleiner, andere größer.
\item Sie drehen sich eben so, wie unsere Erde, um eine Achse,
\item einige schneller, andere langsamer. 
\item Die Planeten, deren wir schon eilf kennen gelernt, bewegen sich noch überdieß in mehr oder weniger kreisförmigen Bahnen um die Sonne, wie unsere Erde;
\item in Zeiträumen, die bald kürzer, bald länger sind, als der Zeitraum, den unsere Erde dazu braucht.
\item Einige dieser Planeten sind der Sonne näher als unsere Erde, andere entfernter von ihr.
\item Einige haben auch Monde oder Nebenplaneten, wie unsere Erde einen hat; nämlich der Jupiter hat deren vier, der Saturn sieben und nebstdem zwei Ringe, der Uranus sechs.~\RWSeitenw{301}
\item Sie erhalten ihr Licht von der Sonne, wie unsere Erde.
\item Wir bemerken auf ihrer Oberfläche Berge und Thäler, wie wir sie auf unserer Erde finden, \usw
\end{aufzc}
\item Eben so begreiflich ist auch, daß ein Theil dieser vernünftigen Wesen (überhaupt diejenigen, welche mit uns in einer gewissen Gemeinschaft stehen) \RWbet{vollkommenerer Natur}, als wir, sey; denn da wir schon auf Erden so vielerlei Abstufungen unter den Geschöpfen antreffen: so schließen wir mit Recht, Gott werde auch in den übrigen Theilen seiner Schöpfung Mannigfaltigkeit beobachtet haben. Es werden also wir Menschen, wie nicht die einzige Art vernünftiger Wesen, so auch nicht die vollkommenste derselben seyn; eine Vermuthung, die bis zum höchsten Grade der Wahrscheinlichkeit steigt, wenn wir bemerken, daß auch der Planet, den wir bewohnen, in keiner Hinsicht ein Aeußerstes ist, daß er weder der kleinste, noch der größte, weder der nächste, noch der entfernteste von der Sonne sey, \usw\ Auch die Geschöpfe also, die wir auf Erden antreffen, werden weder die unvollkommensten, noch die vollkommensten im ganzen Weltall seyn. Ist es nun überhaupt, wie wir noch in der Folge untersuchen werden, nichts Ungereimtes, einen Einfluß vernünftiger Wesen von anderer Art auf Erden anzunehmen: so ist es gewiß auch nichts Ungereimtes, vielmehr sehr wahrscheinlich, daß diese Wesen von einer höhern Art seyen. Sie müssen, wie es scheint, schon darum mächtiger und vollkommener seyn, als wir, weil sie diesen Einfluß auf eine nicht in die Sinne fallende Weise äußern. Dieß scheint ferner auch die Weisheit und Heiligkeit Gottes zu fordern, weil er uns Wesen, die noch unvollkommener sind als wir, unter denen es also doch der Bösen vergleichungsweise noch mehr, als unter uns, geben wird, nicht überlassen soll.
\item \RWbet{Von verschiedenem Range}. Wenn es überhaupt Wesen von höherer Art frei steht, auf uns einzuwirken: warum sollte es nicht Wesen von verschiedenem Range frei stehen können?~\RWSeitenw{302}
\item \RWbet{Haben keine sinnliche Leiber und keine sinnliche Vergnügungen}. Die Redensart: Ein Leib ist nicht sinnlich, kann eine doppelte Bedeutung haben:
\begin{aufzc}
\item Dieser Leib hat nicht solche Sinne, dergleichen wir Menschen an unserem Leibe besitzen;
\item er kann durch unsere Sinne gar nicht wahrgenommen werden.
\end{aufzc}
Die zweite Bedeutung setzt die erste voraus; denn ein Leib, der in der zweiten Bedeutung nicht sinnlich ist, ist es eben darum auch in der ersten nicht. In beiderlei Sinne ist es nicht ungereimt, von den Engeln zu behaupten, daß ihre Leiber nicht sinnlich seyen. Es scheint, daß jeder endliche Geist mit einer gewissen Art von Leib verbunden seyn müsse, wenn er mit andern Geistern durch das Mittel der Materie in Verbindung stehen soll. Aber daraus folgt gar nicht, daß dieser Leib allemal gerade solche Sinne, wie unsere menschlichen, besitzen müsse. Im Gegentheile, die Engel, da sie Wesen von höherer, vollkommenerer Art, als wir Menschen, seyn sollen, müssen eben darum auch einen vollkommeneren Leib, als wir, besitzen. Und da die Offenbarung ferner behauptet, daß diese Engel oft unsichtbarer Weise auf unsere Sinnenwelt einwirken: so müssen sie in der That Leiber besitzen, welche ganz anders, als die unsrigen, beschaffen sind, ja, welche von unseren Sinnen gar nicht wahrgenommen werden können; und dieß ist die zweite und eigentlichste Bedeutung, in welcher man die Redensart: die Engel haben keine sinnliche Leiber, nehmen kann. Haben die Engel nun keine sinnliche Leiber auch in der ersten Bedeutung des Wortes: so können sie auch keine sinnliche Vergnügungsarten haben, \dh\ die sinnlichen Gegenstände, an welchen wir Vergnügen finden, sind nicht auch für Engel reizend und vergnügend; denn jenes Vergnügen entspringt aus der besondern Einrichtung, welche gerade unsere Leiber haben. Da also die Leiber der Engel anders beschaffen sind: so werden sie auch an andern Gegenständen ihr Vergnügen finden.
\end{aufzb}
\item \RWbet{Sie stehen mit uns in einer gewissen Ver\-bin\-dung}.~\RWSeitenw{303}
\begin{aufzb}
\item[a)\ b)\ c)]\setcounter{enumii}{3} In der Natur der Sache liegt es, daß Wesen vollkommenerer Art, \dh\ Wesen von \RWbet{größeren Kräften,} auch einen größeren (sich weiter ausdehnenden) Wirkungskreis besitzen; denn je größer eine Kraft ist, um desto weiter reichen auch ihre Wirkungen. Sind es mit Vernunft und Freiheit begabte Wesen: so ist unter dem Wirkungskreise, von dem hier geredet wird, nicht bloß die Sphäre, in welcher sie nur Wirkungen überhaupt (gleichviel ob bewußte oder unbewußte) hervorbringen, sondern es ist darunter die Sphäre zu verstehen, in welcher sie mit Bewußtseyn und Freiheit wirken, \dh\ je vollkommener ein solches vernünftiges Wesen ist, um desto größer und vielfältiger müssen die Wirkungen seyn, welche es mit Bewußtseyn und Freiheit hervorbringen kann. (So kann \zB\ ein weiser Mann weit wichtigere Veränderungen bei seinen Zeitgenossen und bei der Nachwelt hervorbringen, als irgend ein Thor, selbst wenn des Letzteren Körperkräfte viel größer wären). Doch dieses folgt nicht nur zum Theile schon von selbst, sondern auch aus der Weisheit Gottes ergibt sich, daß er den Wirkungskreis seiner vollkommeneren (kraftvolleren) Geschöpfe nach Möglichkeit vergrößern und erweitern müsse; denn wenn er irgend einer Classe seiner Geschöpfe besonders große Kräfte gegeben hat: so kann er sie ihnen nicht umsonst gegeben haben; sondern er muß ihnen auch einen angemessenen Wirkungskreis verschaffen, zur Kraft auch die Gelegenheit, sie zu brauchen, geben. Wer nun dieses Alles erwägt, wird gerne zugeben, daß Wesen vollkommenerer Art, als wir, dergleichen die Engel sind, auch einen größeren Wirkungskreis haben. So wie nun wir Menschen bereits im Stande sind, uns um die Angelegenheiten der thierischen Welt zu bekümmern, ihren Zustand verschiedentlich abzuändern, ihr Wohlseyn zu befördern, und dieses Alles oft auf eine solche Art zu thun, daß unsere Gegenwart von den Thieren gar nicht bemerkt wird, eben so könnte es wohl auch eine Geisterwelt geben, welche zu uns beiläufig in eben dem Verhältnisse stünde, wie wir zu der thierischen Welt, \dh\ die sich um unsere Angelegenheiten bekümmern, auf uns einwirken könnte, \usw , ohne daß~\RWSeitenw{304}\ wir es jedesmal zu bemerken vermögen. Da hieraus vielerlei Gutes hervorgehen kann; da jene höhern Wesen hiedurch Gelegenheit zur Uebung ihrer sittlichen Kräfte erhalten, \usw : so ist es schon der bloßen Vernunft sehr wahrscheinlich, daß es Gott in der That so eingerichtet habe.\par
\RWbet{1.~Einwurf}. Aber wenn jene höheren Geister Kenntniß von unseren irdischen Angelegenheiten haben und auf uns einwirken sollen: so müßten sie ihre erhabenen Wohnsitze verlassen und diese Erde besuchen können. Nun scheint es aber ein Naturgesetz zu seyn, daß kein Geschöpf den Himmelskörper, der ihm zur Wohnung angewiesen ist, verlassen könne. Wir Menschen \zB\ können uns schlechterdings nicht von dieser Erde zu irgend einem andern Weltkörper hinaufschwingen. Ferner wenn jene Engel zuweilen die Erde besuchen, hier gewisse Veränderungen \udgl\  vornehmen sollten, so müßten sie ja von uns gesehen werden.\par
\RWbet{Antwort.}
\begin{aufzc}
\item Es ist nicht nothwendig, vorauszusetzen, daß der eigenthümliche Wohnort der Engel auf andern Weltkörpern sey. Vielleicht bewohnen sie die Erde und den Luftraum, \udgl\ 
\item Eben so unerwiesen ist die Behauptung, daß diese Wesen, wofern sie auf irgend einem andern Weltkörper einheimisch sind, diesen nothwendig erst verlassen müßten, um in die Kenntniß unserer Angelegenheiten auf Erden zu kommen. Vermögen doch schon wir Menschen uns von manchen Einrichtungen und Ereignissen auf fremden Weltkörpern Kenntniß zu verschaffen, ohne unsern Wohnsitz zu verlassen, \zB\ Mondesfinsternisse, Berge und Schatten, den diese Berge werfen, Thäler, vulkanische Ausbrüche, \usw\ zu bemerken. Das Licht ist ein Mittel, das die entferntesten Weltkörper mit einander verbindet, und dazu dienen kann, um den Bewohnern des Einen Nachricht von den Veränderungen auf dem andern zu geben.
\item Nicht minder unerweislich ist es, daß die Bewohner fremder Weltkörper auf uns nicht einwirken könnten. Im Grunde wirken auch schon wir Menschen auf die Bewohner des entferntesten Himmelskörpers ein, ob wir gleich die~\RWSeitenw{305}\ bestimmte Art und Weise dieser Einwirkung nicht wissen. Da nämlich alle Materie im ganzen Weltraume in wechselseitiger Anziehung und Verbindung steht: so bringen die Veränderungen, die wir in einem auch noch so geringen Theile der Materie, \zB\ in den Gliedmaßen unseres Leibes, hervorbringen, gewisse entsprechende Veränderungen in der Materie des ganzen Weltraumes hervor; mittelbar also auch in den Gestirnen, welche mit dieser Materie in einer gewissen Verbindung stehen. Wesen vollkommenerer Art, als wir, müssen größere Veränderungen hervorbringen, und die Folgen derselben genauer beurtheilen können.
\item Falsch ist es auch, daß kein Geschöpf von Einem Weltkörper zum andern wandern könne. Aus dem Gesetze der Schwere läßt sich die Unmöglichkeit einer solchen Wanderschaft nicht beweisen; denn selbst das Licht, wenn es anders in einem materiellen Ausfluße der leuchtenden Körper bestehet, gibt uns ein Beispiel vom Gegentheile.
\item Endlich wird auch ohne Grund behauptet, daß wir die Engel, wenn sie auf Erden vorhanden seyn, und Veränderungen hervorbringen sollten, nothwendig sehen müßten. Die Materie hat ohne Zweifel noch viel mehrere Kräfte und Eigenschaften, als es diejenigen sind, die wir durch unsere fünf Sinneswerkzeuge kennen. -- Wenn wir \zB\ kein Gehör oder kein Gesicht hätten, würden wir von dem Schalle der Körper, oder von ihren Farben nichts wissen. Aus diesem Umstande erhellet, daß Geschöpfe neben uns vorhanden seyn und gewisse Wirkungen hervorbringen können, ohne daß wir sie wahrnehmen. Hiezu wird nämlich nichts Anderes erfordert, als daß diese Geschöpfe Leiber besitzen, deren Wirksamkeit nicht auf den uns wahrnehmbaren, sondern auf anderen Eigenschaften der Materie beruht. So wie man \zB\ vor einem Tauben sprechen, vor einem Blinden zeichnen kann, ohne daß beide etwas davon merken: so könnten auch Engel Manches in unserer Gegenwart verrichten, ohne daß wir es immer wahrnehmen würden; und wie es uns ein Leichtes ist, den Tauben und Blinden zu täuschen, und etwas zu thun, das er für eine bloße Wirkung der Natur hält: so könnte es wohl auch uns ergehen, daß wir manche Veränderung für eine bloße Wirkung lebloser Kräfte halten, die es im Grunde nicht ist.~\RWSeitenw{306}
\end{aufzc}\par
\RWbet{2.~Einwurf.} Aber wie sollten Engel auf unsern Geist wirken, und in ihm Vorstellungen anregen können?\par
\RWbet{Antwort.} Das \RWbet{Wie} ist uns freilich nicht bekannt; aber hieraus folgt nicht, daß die Sache unmöglich sey. So könnte es etwa mittelbarer Weise durch jene Veränderungen, welche die Engel in der Sinnenwelt bewirken, geschehen.
\item Wenn man die \RWbet{Wirksamkeit der Fürbitten überhaupt} zugibt (wir werden sie später rechtfertigen): so wird man schwerlich etwas dagegen einzuwenden haben, daß Gott der Fürbitte solcher Wesen, die einen höhern Grad von sittlicher Vollkommenheit besitzen, auch eine größere Wirksamkeit einräume.\par
\item \RWbet{Einwurf.} Wenn jedem einzelnen Menschen ein \RWbet{eigener Schutzengel} zugetheilt seyn sollte: so müßten die Engel Wesen von minderer Wichtigkeit seyn, als wir selbst.\par
\RWbet{Antwort.} Gesetzt, es folgte dieß: so hat die Kirche doch nirgends gelehrt, daß jedem einzelnen Menschen ein anderer Engel zugetheilt sey; sondern höchstens war es die Meinung der Kirchenväter, daß jedem Menschen ein bestimmter (mehreren etwa derselbe) zugewiesen sey. Und dieses kann allerdings sehr weise von Gott seyn, damit nicht einige Menschen gänzlich vernachlässiget würden. So wird auch \zB\ ein weiser Regent eine gewisse Ordnung unter seinen Beamten treffen, und jedem einen bestimmten Theil seiner Unterthanen zur Aufsicht anweisen.
\item \RWbet{1.~Einwurf.} Nur Gott allein kann Wunder wirken.\par
\RWbet{Antwort.} Nach dem Begriffe, den wir, in Uebereinstimmung mit dem gemeinen Menschenverstande, von einem Wunder aufgestellt haben, enthält es gar nichts Unmögliches, daß Engel dergleichen sollten hervorbringen können; denn Wunder sind ja nicht übernatürliche oder unmittelbare Wirkungen Gottes.\par
\RWbet{2.~Einwurf.} Wie können aber Engel, da sie doch keine menschliche Leiber haben, in menschlicher Gestalt erscheinen?\par
\RWbet{Antwort.} Sie \RWbet{scheinen} diese menschliche Gestalt nur zu haben, ohne sie wirklich zu besitzen. Wie sie diesen Schein~\RWSeitenw{307}\ hervorbringen, können wir freilich nicht mit Bestimmtheit angeben. Eine Art aber, wie dieß geschehen könnte, wäre \zB\ schon diese: Wenn sich ein Mensch, der eine etwas lebhafte Einbildungskraft besitzt, öfters mit dem Gedanken an Engel beschäftiget; wenn er der Meinung ist, daß Engel den Menschen zuweilen, und zwar in einer menschenähnlichen Gestalt, erscheinen können: so ist leicht zu begreifen, wie ihm einmal, etwa im Traume, vorkommen kann, daß ihm ein Engel erscheine, und dieß oder jenes zu ihm spreche, \usw , ohne daß dieses Alles wirklich so vorzugehen braucht. Dieses Gesicht nun, oder dieser Traum, kann in uneigentlicher Bedeutung die Wirkung eines Engels schon darum heißen, weil er doch eine Wirkung des Glaubens an solche Engel ist; er kann aber auch in einer noch eigentlicheren Bedeutung die Einwirkung eines Engels, eine Erscheinung desselben heißen, wenn der Engel durch gewisse Einwirkungen in der Sinnenwelt etwas dazu beigetragen hat, daß jener Traum entstand, und diese oder jene besondere Richtung genommen hat, \usw\
\end{aufzb}
\item Es gibt auch \RWbet{böse Engel}, die jedoch
\begin{aufzb}
\item anfangs gut waren. Freie endliche Wesen, wenn sie auch noch so vollkommen sind, sind immer fehlbar. Im Voraus also ließe sich erwarten, daß einige derselben sündigen würden. Und eben so begreiflich ist es, daß manche in ihren Sünden beharren, und je länger je bösartiger wurden. Daß jene bösartigen Fertigkeiten aber ihnen nicht schon angeschaffen seyen, ist eine nothwendige Folge aus Gottes Heiligkeit;
\item \RWbet{für ihre Bosheit von Gott ewig gestraft werden}. Die Schwierigkeiten, die man dagegen vorbringen könnte, werden bei der Lehre von der Ewigkeit der Höllenstrafen erwogen werden.
\end{aufzb}
\item Auch diesen \RWbet{bösen Engeln ist ein gewisser Einfluß auf diese Erde} gestattet.\par
\RWbet{Einwurf.} Dieß hätte die Heiligkeit Gottes nicht zulassen sollen.~\RWSeitenw{308}\par
\RWbet{Antwort.} Aber vielleicht war es eine unvermeidliche Folge davon, daß Gott den Engeln überhaupt einen Einfluß auf die Erde gestattete; er konnte ihn vielleicht den guten nicht gestatten, ohne ihn zugleich einigen bösen zu gestatten; wie Gott nicht nur den guten, sondern auch den grausamen Menschen, den Thierquälern, einen Einfluß auf die thierische Welt gestatten muß. Sah er nun vorher, daß das Gute, welches die guten Engel stiften würden, den Schaden der bösen überwiegen werde: so war es seiner Heiligkeit nicht zuwider, es zuzulassen.
\begin{aufzb}
\item Daß insbesondere die bösen Engel ein \RWbet{Wohlgefallen an der Lasterhaftigkeit und an dem Unglücke der Menschen} finden, ist eben nichts Ungereimtes. Haben doch böse Menschen gleichfalls ein solches Wohlgefallen. Können dergleichen Engel nicht gewisse Vortheile, eine gewisse Erleichterung ihres Schicksals darin finden oder doch suchen, daß sie die Menschen zum Bösen verleiten?
\item[b)\ c)] sind keiner Schwierigkeit unterworfen.
\end{aufzb}
\end{aufza}

\RWpar{165}{Sittlicher Nutzen}
\begin{aufza}
\item 
\begin{aufzb} \item Die Wahrheit, daß es auch außerhalb des Menschen und aller Erdgeschöpfe \RWbet{eine zahllose Menge lebendiger und vernünftiger Wesen} in Gottes Schöpfung gebe, muß unsere Bewunderung der Allmacht und Güte Gottes vermehren.
\item Die Nachricht, daß Millionen dieser Wesen \RWbet{vollkommenerer Natur} sind, als wir, beugt unserem Stolze vor, daß wir nicht etwa wähnen, die obersten in der Stufenleiter der vernünftigen Wesen zu seyn, die nächsten an dem Unendlichen zu stehen. Wir werden auch aufgemuntert, nach einer ähnlichen Vollkommenheit zu streben. Jesus will, daß wir uns in der Erfüllung der göttlichen Gebote täglich durch die Erinnerung an jene Bereitwilligkeit, mit welcher die Engel Gottes Befehle ausführen, ermuntern sollen: \erganf{Dein Wille geschehe, wie im Himmel, so auf Erden.} Dieser Aufmunterung wächst um so mehr Kraft zu, da uns das Christenthum ver\RWSeitenw{309}sichert, daß es uns möglich sey, und einst wirklich gelingen werde, den Engeln an Vollkommenheit zu gleichen. Nach der Auferstehung werdet ihr den Engeln Gottes gleichen (\RWbibel{Mt}{Matth.}{22}{30}).
\item \RWbet{Verschiedene Rangordnungen.} Auch diese Nachricht ist uns nicht überflüssig. Wir erhalten so einen deutlicheren Begriff von der Menge dieser Engel, von dem großen Abstande zwischen uns und Gott, \usw\
\item Sie haben \RWbet{keine sinnlichen Leiber und keine sinnlichen Vergnügungen}. Das Erste ist uns schon darum zu wissen nöthig, damit wir begreifen, wie sie auf uns einwirken und bei uns zugegen seyn können, ohne doch von uns gesehen zu werden. Das Zweite lehrt uns, daß sinnliche Vergnügungen nicht die höchsten und wünschenswerthesten seyn müssen.
\end{aufzb}
\item Mehrere dieser Engel \RWbet{stehen mit uns in Verbindung}. Wenn es einmal wahr ist, daß höhere Geister mit uns in Verbindung stehen: so ist es uns sehr dienlich, davon benachrichtiget zu seyn: denn erst wenn wir von dieser Verbindung wissen, können wir sie durch ein zweckmäßiges Betragen recht benützen. Und auch wenn es nicht wahr wäre, hätte es mancherlei Nutzen, es zu glauben, wie an dem gleich folgenden gezeigt wird. Insbesondere 
\begin{aufzb}
\item \RWbet{sie kümmern sich um unsere sittlichen Angelegenheiten.} Der Gedanke, daß wir der Aufmerksamkeit höherer Geister nicht unwürdig sind, muß uns eine erhöhte Vorstellung von unserer eigenen Wichtigkeit beibringen. Wenn wir uns vorstellen, daß wir von ihnen beobachtet werden, daß sie um unsere geheimsten Gedanken wissen: wie müssen wir uns nicht so mancher niedriger Gedanken, so mancher Lüste, schämen, die uns, mit Engeln verwandte Geschöpfe, tief unter das Thier herabziehen würden! Wenn wir zuweilen (und in der That soll es recht oft geschehen) etwas Gutes thun, das kein sterbliches Auge gewahret, ja das die Menschen sogar verkennen und lästern: wie muß uns nicht der Gedanke trösten, daß es doch Wesen, die weit vollkommener, als diese Wesen sind, bemerken, und mit ihrem Beifalle be\RWSeitenw{310}lohnen! Der Gedanke, daß es auch diesen Wesen Freude macht, wenn wir über unsere Versuchungen zum Laster siegen, und immer vollkommener werden, wird uns ein neuer, sehr edler Beweggrund, dem Bösen zu widerstehen, und an sittlicher Güte zuzunehmen.
\item[b)\ c)\ d)]\stepcounter{enumii}\stepcounter{enumii}\stepcounter{enumii} \RWbet{Sie können auf uns einwirken, und manche gute Gedanken in uns wecken}. Wissen wir dieses, so öffnen wir unser Herz ihren Eingebungen, \dh\ wir sind auf jeden guten Gedanken, der uns in den Sinn kommt, um desto aufmerksamer, achten ihn nicht gering, sondern prüfen ihn um desto sorgfältiger, und befolgen ihn um desto gewissenhafter, wenn wir ihn gut befunden haben, weil wir gedenken, vielleicht, daß dieser gute Einfall dir von deinem guten Engel kommt! du willst ihn nicht verachten! Jeden Sieg, den wir nun über das Böse davon tragen, jede Stufe sittlicher Vollkommenheit, die wir ersteigen, danken wir nun, nächst Gott, auch seinen heiligen Engeln, und dadurch wird die Tugend der Dankbarkeit geübt.\par
\RWbet{Einwurf.} Wenn uns etwas Gutes zu Theil wird, so glauben wir, ein Engel habe es gethan, und vergessen darüber, den Wohlthäter unter den Menschen aufzusuchen, der vielleicht gerechtere Ansprüche auf unsere Dankbarkeit hätte. (Lessing's Nathan.)\RWlit{}{Lessing1d}\par
\RWbet{Antwort.} Dieser Mißbrauch ist gar nicht nothwendig. Wir können und sollen immerhin, wenn uns etwas Gutes zu Theil wird, untersuchen, ob und von welchen guten Menschen es uns komme. Finden wir welche: so danken wir ihnen dafür. In jedem Falle aber, wir mögen einen irdischen Wohlthäter haben oder nicht, ergötzen wir uns mit dem Gedanken, daß vielleicht auch ein höherer Geist etwas dazu beigetragen habe, und danken eben deßhalb auch diesem dafür.
\item Wissen wir, daß die \RWbet{Fürbitten der Engel} bei Gott eine besondere Wirksamkeit haben: so ist dieß ein neuer Beweis seiner Güte und Heiligkeit, zu Folge der es eben geschieht, daß die Fürbitten der vollkommeneren Wesen mehr, als jene der unvollkommeneren bei ihm gelten; wir sehen hierin einen neuen Aufmunterungsgrund, stets wei\RWSeitenw{311}ser und besser zu werden, weil auch unser Gebet dann bei Gott in eben dem Maße wirksamer seyn wird; wir flehen die Engel um ihre Fürbitte bei Gott an, und glauben, so manche Wohlthat, die wir von Gott erhalten, zum Theil auch ihrer Fürbitte verdanken zu müssen.
\item \RWbet{Schutzengel}. Alle bisher angeführten wohlthätigen Folgen werden durch den Gedanken, daß jeder einzelne Mensch einem gewissen Engel zu seiner besondern Sorgfalt anempfohlen sey, vermehret. Welche Größe der Liebe Gottes zu uns! Welche Verpflichtung, diesen für uns wachenden Engel zu lieben, zu ehren, ihm zu gehorchen, ihn nicht zu betrüben, ihm dankbar zu seyn, ihn öfters um seinen Beistand, um seine Fürbitte bei Gott anzuflehen! \usw\ Wie enge wird das Band nicht geknüpft, das jene Geisterwelt mit uns verbindet! Welch ein genauer Zusammenhang erscheint uns jetzo nicht zwischen allen Theilen der Schöpfung!
\item \RWbet{In außerordentlichen Fällen hat sich Gott solcher Engel zu wohlthätigen Zwecken bedient.} Dieses ist
\begin{aufzc}
\item eine fast nothwendige Bedingung dazu, wenn unser Glaube an Engel und ihre Verbindung mit uns recht lebhaft werden sollte; denn was wir Menschen recht lebhaft glauben sollen, das müssen wir gleichsam mit Augen gesehen haben. Wenn also Niemand aus uns je eine Erscheinung von Engeln gehabt zu haben geglaubt hätte: so würde auch unser Glaube an ihr Daseyn und an ihre Verbindung mit uns niemals recht lebhaft und fest geworden seyn; und eben darum könnte er die bisher gerühmte Wirkung nicht hervorbringen.
\item Es dient ferner auch zur Erhöhung unserer Vorstellung von Gottes Würde, daß er gewisse Zwecke nicht selbst, sondern durch seine Engel ausführen läßt; \zB\ Offenbarungen minderer Art, wie die Flucht der Eltern Jesu nach Aegypten, \udgl\ 
\item Dadurch, daß diese Engel uns in menschenähnlicher Gestalt erscheinen, wird unsere Natur geehrt und geadelt.~\RWSeitenw{312}
\end{aufzc}
\end{aufzb}
\item 
\begin{aufzb} 
\item Es gibt auch \RWbet{böse Engel}, die aber doch anfangs gut geschaffen waren. Der Fall der bösen Engel gibt uns die wichtige Lehre, daß auch der weiseste und tugendhafteste Mensch noch der Gefahr zu sündigen ausgesetzt sey; daß er sich vornehmlich von dem Geiste des Hochmuths zu hüten habe; und daß er zuweilen nur um so tiefer falle, je mehr Kräfte und Vollkommenheiten er hat.
\item Die \RWbet{ewige Strafe}, die diese gefallenen Engel erfahren, wirkt auf uns als ein Beispiel von Gottes Gerechtigkeit und von dem unausbleiblichen Elende, das jeder Böse, so mächtig er auch sey, sich zuzieht.
\end{aufzb}
\item Auch den \RWbet{bösen Engeln ist ein gewisser Einfluß auf diese Erde gestattet}; sie haben
\begin{aufzb}
\item \RWbet{Wohlgefallen am Bösen} und am Unglücke der Menschen, und suchen es zu vermehren. Durch diese Lehre werden wir angeleitet, jeden bösen Gedanken, der uns einfällt, als eine (unmittelbare oder mittelbare) Eingebung Satans, oder eines andern bösartigen Geistes anzusehen, nicht zwar, als ob wir es mit Bestimmtheit behaupten könnten, der böse Einfall komme uns durch die Mitwirkung jenes Geistes; sondern wir sollen uns dieß nur so vorstellen. Diese Vorstellung hat nun den Nutzen, daß wir einen weit größern Abscheu gegen den bösen Einfall verspüren, und eben deßhalb leichter im Stande sind, denselben zu vertreiben; denn wenn wir einen Gedanken, der in uns aufsteigt, als unsere eigene Erfindung ansehen: so hindert uns schon die Eigenliebe, ihn mit dem verdienten Abscheu zu behandeln; glauben wir aber, daß er uns von jemand Anderm, wohl gar von einem böse gesinnten Geiste zugeflüstert werde: so hört nicht nur jene Täuschung der Eigenliebe auf, sondern nunmehr begreifen wir auch leichter, wie die Befolgung dieses Gedankens nur zu unserem eigenen Schaden seyn dürfte, und all der Abscheu und Widerwille, den wir gegen jenen verruchten Geist empfinden, geht auch auf den Gedanken, den wir für seine Einflüsterung halten, über.\par
\RWbet{1.~Einwurf.} Aber so thun wir dem Teufel Unrecht, und gewöhnen uns hiedurch, auch Menschen Unrecht zu thun.~\RWSeitenw{313}\par
\RWbet{Antwort.} Wenn es wahr ist, was das Christenthum sagt, daß jene bösen Geister sich über das Böse freuen, und es zu befördern suchen: so thun wir ihnen im Wesentlichen kein Unrecht, wenn wir sie als die Urheber unserer bösen Einfälle ansehen; denn wenn sie es auch wirklich nicht sind, so wünschten sie es doch zu seyn. Ferner kann dieser üble Argwohn auch Niemanden schaden, und nur dann ist Argwohn Sünde, wenn er Jemanden Nachtheil bringt; wie \zB\ wenn wir von Menschen Uebles argwöhnen. Auch ist es falsch, daß wir uns hiedurch unvermerkt gewöhnen, den Menschen Uebles anzudichten; denn vor diesem Letzteren warnt uns ja das Christenthum auf das Nachdrücklichste.\par
\RWbet{2.~Einwurf.} Indem wir die Ursache der bösen Gedanken, die in uns aufsteigen, im Teufel suchen, vergessen wir, die ganz natürliche Veranlassung derselben in unserem Innern und in den äußern Umgebungen wegzuschaffen, \zB\ Müßiggang, Ueberfüllung des Magens, gefährliche Bücher \udgl\par
\RWbet{Antwort}. Ist eben so, wie der ähnliche Einwurf bei 2.\,b.\ zu beantworten. Das Christenthum sagt, wir sollen das Eine thun, und das Andere nicht unterlassen.
\item  Die Bemerkung, daß der Teufel zuweilen auch sichtbare Veränderungen, Wunder und Zeichen, hervorbringen könne, dient, die Menschen bei Prüfung göttlicher Offenbarungen behutsamer zu machen. Sie dürfen sich jetzt an die Wunder nicht allein halten, sondern müssen auch den Inhalt prüfen.\par
\RWbet{Einwurf.} Aber diese große Gewalt, welche das Christenthum dem Teufel einräumt, muß uns mit Furcht und Zittern erfüllen. Wie furchtbar klingt nicht schon, was der Apostel Petrus sagt (\RWbibel{1\,Petr}{1\,Petr.}{5}{8}): \erganf{Der Teufel geht herum wie ein brüllender Löwe, und sucht, wen er verschlinge.}\par
\RWbet{Antwort.} Dasselbe Christenthum sagt aber auch, daß dieser Teufel uns nichts anhaben könne, sobald wir tugendhaft sind, und uns der Gnaden, die uns Gott auf unsere Bitten mittheilen will, bedienen. In eben dieser Stelle setzt der Apostel bei: \erganf{Widerstehet ihm standhaft im Glauben.} Also wird durch die Beschreibung, die uns das Christenthum von der Stärke des Feindes, mit dem wir kämpfen, gibt, nur~\RWSeitenw{314}\ unser Muth gestählt, nicht aber Furchtsamkeit erzeugt. Um desto muthvoller treten wir dann im Kampfe gegen böse Menschen auf. Vermag die ganze Hölle nichts gegen uns, um wie viel weniger, sprechen wir, wird dieser elende Menschenclubb über uns obsiegen können!
\end{aufzb}
\end{aufza}

\RWpar{166}{Wirklicher Nutzen}
Der Glaube an Geister findet sich bei allen Völkern, aber fast überall mit gewissen schädlichen Irrthümern verbunden. So dachte man sich \zB , daß diese Geister eine freie und ungebundene Wirksamkeit besitzen, und daß die Veränderungen, die sie hervorbringen, nicht unter der Leitung der göttlichen Vorsehung stehen. Kein Wunder, wenn man sich vor solchen Wesen fürchtete, und wenn eine abgöttische Verehrung derselben als gewisser Untergottheiten aufkam, um sich hiedurch ihre Geneigtheit zu erwerben, \udgl\  Ein anderer sehr gewöhnlicher Irrthum war es, daß die bösen Geister schon von Natur aus böse wären. Sehr häufig legte man ihnen sinnliche Leiber und Vergnügungen bei, welches die Folge nach sich zog, daß man dieselben durch sinnliche Ausschweifungen zu verehren glaubte, \usw\ Alle diese Irrthümer hat das Christenthum verdrängt. Dennoch ist der Glaube an Geister auch unter den Christen häufig gemißbraucht worden; der Glaube an den Einfluß guter Geister hat Viele zur Schwärmerei verleitet, und durch den Glauben an den Einfluß böser Geister sind Manche unnöthiger Weise beunruhigt und gequält worden. Aber wer möchte beweisen können, daß dieser Mißbrauch verhindert worden wäre, wenn das Christenthum von allem Einflusse der Geister auf uns geschwiegen hätte; und dann, daß dieser Mißbrauch den Nutzen überwiege, welchen der Glaube an Geister jedem Christen von seiner frühesten Kindheit an bis in das Greisenalter bringt?~\RWSeitenw{315}

\endinput
