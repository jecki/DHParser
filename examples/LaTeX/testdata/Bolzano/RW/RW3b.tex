\clearpage\linenumbers%
\def\dieserteilseiten{IIIb}

\RWabs{Dritter Abschnitt}{Christkatholische Anthropologie oder Lehre vom Menschen}
\RWpar{167}{Die Lehre des Christenthums von den Bestandtheilen und Kräften des Menschen}
Auch über uns selbst, so nahe wir uns sind, und so genau wir uns deßhalb auch kennen sollten, weiß uns das Christenthum manche nicht überflüssige Belehrung beizubringen. Es sagt uns, daß wir aus \RWbet{zwei Bestandtheilen, Leib} und \RWbet{Seele}, bestünden. Leib nennt es dasjenige, was an uns sichtbar ist und in die Sinne fällt, und macht die Bemerkung, daß dieser Leib \RWbet{von Gott überaus weise und künstlich gebaut} sey. Seele nennt es dasjenige, was in uns denkt, empfindet, will und handelt, oder was unser eigentliches Ich ausmacht. Diese Seele ist, wie das Christenthum weiter sagt, eine von der Materie unsers Leibes durchaus verschiedene Substanz, die vornehmlich folgende Kräfte besitzt:
\begin{aufzb}
\item Einen \RWbet{Verstand}, \dh\ eine Fähigkeit, gewisse Wahrheiten zu erkennen. Von diesem Verstande erinnert das Christenthum, daß er seine Schranken habe, dergestalt, daß wir sehr Vieles nicht wissen, und Manches zu wissen glauben, was wir in der That doch nicht wissen, \dh\ daß wir uns irren.
\item Ein \RWbet{Empfindungsvermögen}, durch welches wir für Lust und Schmerz, Glückseligkeit und Unglückseligkeit empfänglich werden. Die Erste wünschen, die Andere verabscheuen wir (\RWbet{Begehrungsvermögen.})~\RWSeitenw{4}
\item Einen \RWbet{Willen}, der \RWbet{Freiheit} besitzt, \dh\ der zwischen demjenigen, was unser Verstand als recht erkennt, und was unser Begehrungsvermögen wünscht, zu wählen vermag; im Uebrigen aber nicht immer das wirklich ausführen kann, was er sich vorsetzt.
\end{aufzb}
Diese Seele des Menschen \RWbet{dauert auch nach dem Tode des Leibes mit Bewußtseyn und mit Rückerinnerung an ihren ehemaligen Zustand auf dieser Erde fort}, oder sie ist \RWbet{unsterblich.}


\RWpar{168}{Historischer Beweis dieser Lehre}
Daß alle diese Lehrsätze wirklich von allen Katholiken angenommen werden, und folglich zum Inhalte des Katholicismus gehören, ist so bekannt, daß es nicht erst eigens erwiesen zu werden braucht. Wir wollen nur zeigen, daß auch die heil.\ Schrift so lehre.
\begin{aufza}
\item \RWbet{Leib und Seele}. \RWbibel{Mt}{Matth.}{10}{28}: \erganf{Fürchtet euch nicht vor denen, die nur den Leib, nicht aber die Seele tödten können; fürchtet vielmehr den, welcher Leib und Seele in die Hölle stürzen kann.}
\item \RWbet{Künstliche Bildung des Leibes.} \RWbibel{Ps}{Ps.}{139}{}: \erganf{Mich gebildet hast Du im Mutterschooße! Dank dir, daß ich so ward, ein hohes Wunder! Unbegreiflich sind deine Werke, dieß fühlet mein Herz!}
\item Daß die heil.\ Schrift die Seele für eine \RWbet{eigene Substanz} erkläre, erhellet schon daraus, weil es dieselbe vom Körper unterscheidet, ihr Kräfte beilegt, ihr auch nach dem Tode des Körpers ein Bestehen verspricht, \usw\
\item Daß unser Verstand wirklich die \RWbet{Fähigkeit habe, gewisse Wahrheiten zu erkennen}, setzt die Bibel an allen Orten voraus. Der heil.\ Paulus sagt: \erganf{Prüfet Alles und das Gute behaltet.} Wir müssen also die Fähigkeit haben, das Gute, somit auch das Wahre zu erkennen; aber wir irren uns hierin zuweilen auch, weil sonst nicht Prüfung nöthig wäre.~\RWSeitenw{5}
\item Daß die heil.\ Schrift auch das \RWbet{Empfindungsvermögen} des Menschen als eine wirkliche Kraft und Vollkommenheit der Seele ansehe, erhellet daraus, weil sie uns auch nach dem Tode, \dh\ nach der Trennung der Seele von dem Leibe, Empfindung und Gefühl für Lust und Schmerz beilegt.
\item \RWbet{Freiheit des Willens}. Schon zu Kain wird (\RWbibel{Gen}{1\,Mos.}{4}{7})\ gesagt: \erganf{Die Sünde wird dich reizen, dir aber wird es möglich seyn, über ihren Reiz zu siegen.} -- \RWbibel{Sir}{Sirach}{31}{10}\ heißt es: \erganf{Selig derjenige, der das Gesetz übertreten konnte, und doch nicht übertrat, Böses thun konnte, und doch nicht that.} Und \RWbibel{Sir}{}{15}{17}: \erganf{Gott schuf den Menschen, und überließ ihn seinem Willen. -- Willst du, so kannst du die Gebote halten, und Gott gefällige Rechtschaffenheit üben. Feuer und Wasser hat er dir vorgelegt; strecke deine Hand aus, wornach du willst. Leben und Tod liegt vor dem Menschen, was er wählet, das wird ihm zu Theil.}
\item Daß die \RWbet{Unsterblichkeit der Seele} in den Büchern des neuen Bundes, und in den späteren des alten ausdrücklich vorgetragen werde, ist eine bekannte Sache. Nur in den fünf Büchern Mosis, im Buche \RWbibel[Hiob]{Hiob}{}{}{}, im Prediger, und in einigen anderen kleineren Aufsätzen des A.\,B.\ haben Einige den Glauben an Unsterblichkeit vermißt. Wenn es auch wahr wäre, daß die Verfasser dieser Bücher den Glauben an Unsterblichkeit noch nicht gehabt haben: so würde hieraus doch nicht folgen, daß ihre Religion keine wahre göttliche Offenbarung gewesen. Denn wer kann darthun, daß eine jede göttlich geoffenbarte Religion den Lehrsatz von der Unsterblichkeit der Seele ausdrücklich aufstellen müsse? -- Indessen läßt sich mit hinlänglicher Sicherheit erweisen, daß der Glaube an Unsterblichkeit zu den Zeiten Mosis, ja noch viel früher bei dem israelitischen Volke anzutreffen gewesen sey. Wenn auch die Worte des sterbenden Jakobs (\RWbibel{Gen}{1\,Mos.}{49}{18}) eine andere Auslegung zuließen; so ist doch der Ausruf Bileams (\RWbibel{Num}{4\,Mos.}{23}{10}): \erganf{O, stürb' ich des Todes dieser Gerechten! O wäre mein Ende dem ihrigen gleich!} kaum anders, als durch den Glauben an Unsterblichkeit zu erklären. Noch entschiedener ist, daß Moses (\RWbibel{Dtn}{5\,Mos.}{18}{11})~\RWSeitenw{6}\ das Befragen der Todten verbietet. Wie hätte diese Sitte aufkommen können, wenn man nicht geglaubt hätte, daß die Verstorbenen noch leben? \RWbibel{Num}{4.\,Mos.}{20}{26}\ heißt es von Aaron, als er stirbt, er reihe sich an sein Geschlecht an; und doch ward er nicht in der Familiengruft begraben. -- Im Buche des Predigers heißt es (\RWbibel{Koh}{}{12}{7}), daß der Staub zur Erde zurückkehrt, der Geist aber zu Gott, der ihn gegeben hat. Dieses ist wenigstens nicht die Sprache derjenigen, welche die geistige Natur der Seele und eben darum auch ihre Unsterblichkeit läugnen. Im Buche Hiob kommen mehrere Stellen vor, besonders \RWbibel{Hiob}{}{19}{25}, die nur durch eine sehr gezwungene Auslegung anders, als von dem Glauben an Unsterblichkeit erklärt werden können.
\end{aufza}

\RWpar{169}{Vernunftmäßigkeit}
Alle diese Lehren (mit Ausnahme jener von der Unsterblichkeit der Seele) haben vor dem Urtheile des gemeinen Menschenverstandes vollkommene Gewißheit. Weil aber nichts so thöricht ist, das nicht (wie Cicero sagt) von irgend einem Weltweisen wäre behauptet worden: so sind auch keine jener Lehren frei von dem Widerspruche gewisser Weltweisen geblieben, und um so weniger, da einige dieser Wahrheiten wirklich bei aller Gewißheit, doch auf einem sehr verborgenen Grunde beruhen.
\begin{aufza}
\item So gab es Weltweise, welche das Daseyn der Materie, und also auch der menschlichen Leiber geläugnet. Nach ihrer Meinung ist die ganze Körperwelt nur \RWbet{Schein}, daher man diese Philosophen \RWbet{Idealisten} nennt, \zB\ \RWbet{Berk[e]ley}. Wirklich ist es nicht leicht, diesen Idealismus auf eine wissenschaftliche Art zu widerlegen, \dh\ die Wirklichkeit der Materie, als einer Substanz, aus objectiven Gründen herzuleiten, weil der Begriff der Substanz noch nicht gehörig erörtert ist.
\item Noch mehrere Weltweise gab es, die der Seele eine von der Materie des Leibes verschiedene Substanzialität absprachen, die also das Denken, Empfinden, Wollen und Handeln für eine bloße Wirkung der Materie erklärten, und eben~\RWSeitenw{7}\ deßhalb \RWbet{Materialisten} heißen. \RWbet{Hobbes, Spinoza, Voltaire, Helvetius,} der Verfasser des berüchtigten Buches \RWlat{Système de la nature}\RWlit{}{Dholbach1}, \RWbet{la Mettrie}, \uA\ Diesen Philosophen hat man schon öfters entgegnet, daß ein Subject, das nur ein einziges Bewußtseyn hat, auch nur eine einzige Substanz seyn müsse. Wenn nun die Materie Substanz ist: so besteht unser Körper nicht aus einer einzigen, sondern unendlich vielen materiellen Substanzen. Nur eine einzige also, oder vielmehr eine von ihnen ganz verschiedene Substanz kann es seyn, der das Bewußtseyn zukommt, die unser eigentliches Ich ausmacht, oder unsere Seele ist.
\item Bekanntlich hat es auch nicht an Weltweisen gemangelt, die unserem Verstande die Fähigkeit, objective Wahrheiten zu erkennen, absprachen, und also die sich selbst aufhebende Behauptung aufstellten: daß alle Urtheile des Menschen falsch wären (\RWbet{Pyrrhon} und andere Skeptiker).
\item Am größten aber war die Anzahl derjenigen, welche die Freiheit des Menschen entweder ausdrücklich läugneten, oder sie doch auf eine solche Art erklärten, daß nur der Name derselben übrig blieb. Zu den Ersteren gehören \zB\ \RWbet{Zeno}, der Stifter der stoischen Philosophie, die \RWbet{Skeptiker, Spinoza, la Mettrie} (\RWlat{l'homme machine})\RWlit{}{LaMettrie1}, der Verfasser des Buches: \RWlat{Système de la nature,}\RWlit{}{Dholbach1} \RWbet{Bayle, Collins}, der Verfasser des Buches: Alexander von Joch, über Belohnungen und Strafen nach türkischen Gesetzen\RWlit{}{Hommel1} (\RWbet{Humel}) \uA\ Zu den Letzteren gehören auch die sogenannten \RWbet{Deterministen, \zB\ Leibnitz, Wolf[f]} und ihre Anhänger, \RWbet{Basedow, Tetens, Feder} und fast alle deutschen Philosophen vor Kant. Der zu weit ausgedehnte Satz vom Grunde (\RWlat{principium causalitatis}) war die Ursache dieser Verirrungen. Man drückte ihn meistens aus: Alles, was ist, muß einen Grund haben (\RWlat{causam determinantem}. Wolf[f]: \RWlat{rationem sufficientem, i.\,e.\ talem, ex qua intelligi possit, cur potius sit, quam non sit}). Wenn dieser Satz, so ausgedrückt, richtig wäre, dann könnte es freilich keine freien Handlungen geben; denn jeder Willensentschluß müßte dann einen Grund haben, warum er so und nicht anders erfolgt, dieser Grund müßte wieder einen Grund haben,~\RWSeitenw{8}\ \usf\ Was aber einen Grund hat, das ist, bei Setzung dieses Grundes, nothwendig, und ohne ihn unmöglich. Aber so ausgedrückt ist der Satz offenbar falsch; denn da gäbe es überall keine erste Gründe, \dh\ keine solche Gründe, die nicht wieder Folgen wären; folglich auch keinen Gott, ja auch nicht einmal Grundwahrheiten oder Grundsätze. Andere sagten daher: Nur Alles, was geschieht, oder wird, hat einen Grund. Aber auch dieses ist noch zu viel gesagt, wenn man unter dem Ausdrucke: \RWbet{geschehen, werden}, nicht ganz eben dasselbe versteht, was man sonst unter dem Ausdrucke: \RWbet{wahrnehmbar seyn}, versteht. Denn auch die freien Willensentschließungen geschehen (erfolgen oder werden), und haben doch keinen Grund. -- Wie nun der richtige Ausdruck dieses Satzes laute, ist schwer zu bestimmen. Meiner Meinung nach ist er kein \RWbet{constitutives}, sondern ein \RWbet{regulatives Princip}, \dh\ ein Satz, der eine bloße \RWbet{Regel} aussagt, nach der sich der Verstand in seinem Nachdenken richten soll; und sollte daher nicht anders ausgedrückt werden, als so: Suche vor Allem einen Grund, forsche nach, ob einer vorhanden ist. Die Behauptung, es gebe eine Freiheit, will meiner Meinung nach nichts anderes sagen, als: es gibt manches Wirkliche, das nicht nothwendig ist. Wer also alle Freiheit läugnet, der muß zugleich behaupten, daß der Begriff des Wirklichen und der des Nothwendigen zwei Begriffe von gleichem Umfange sind; eine Behauptung, welche doch wirklich sehr ungereimt klingt, und schon allein im Stande seyn sollte, den Determinismus verdächtig zu machen. Ein Mehreres hierüber haben wir bereits an einem anderen Orte gesagt.
\item Die \RWbet{Unsterblichkeit der Seele}, die das Christenthum lehrt, findet auch die sich selbst überlassene Vernunft mehr als wahrscheinlich, wie schon im ersten Haupttheile gezeigt worden ist, wo auch die wichtigsten Einwürfe, welche die Weltweisen auch wider diese Wahrheit vorgebracht haben, berührt worden sind.
\end{aufza}

\RWpar{170}{Sittlicher Nutzen}
\begin{aufza}
\item Die Unterscheidung zwischen \RWbet{Leib} und \RWbet{Seele} ist von größter Wichtigkeit für uns, weil wir im entgegengesetz\RWSeitenw{9}ten Falle bei dem Hinsterben unseres Leibes kaum eine Fortdauer unseres Bewußtseyns hoffen könnten.
\item Eben so dankenswerth ist es, daß uns das Christenthum auf den \RWbet{kunstvollen Bau unseres Leibes} aufmerksam macht; denn wenn wir die weisen und gütigen Einrichtungen Gottes in der Maschine unseres Leibes kennen lernen: so erhalten wir eine Menge neuer Beweise von Gottes Weisheit und Güte, die uns um desto stärker rühren müssen, je näher sie uns selbst angehen. Wir bekommen auch vor uns selbst mehr Achtung. Wir fühlen lebhafter die Pflicht, dieses herrliche Meisterstück der Schöpfung vor jeder Gefahr einer Verletzung zu bewahren, und die in diesem Leibe schlummernden Kräfte und Fähigkeiten zu entwickeln; im Gegentheile aber das Verbrechen, das wir begehen würden, wenn wir durch Unachtsamkeit oder durch Ausschweifungen diesen kunstvollen Leib zerstörten, erscheint uns jetzt um so furchtbarer, \usw\
\item Noch nothwendiger aber ist es, daß uns das Christenthum die Erinnerung gibt, nicht dieser Leib, sondern \RWbet{die Seele wäre es, die unser eigentliches Ich ausmacht.} Wenn wir dieß immer bedächten, würden wir über der Sorge für unsern Leib nicht jene für das Heil unserer unsterblichen Seele versäumen.
\item Obgleich wir unserem Verstande \RWbet{die Fähigkeit, objective Wahrheiten zu erkennen,} schon eingeräumt haben müssen, bevor wir den Glauben der Katholiken als eine wirkliche göttliche Offenbarung annehmen können: so ist doch die Versicherung, die wir hier neuerdings über diesen Punct erhalten, nicht überflüssig, sondern dient vielmehr zu unserer desto größeren Beruhigung.
\item Noch wichtiger ist die Versicherung, die uns über die \RWbet{Freiheit des Willens} ertheilt wird. Diese Versicherung ist um so nothwendiger, je mehr uns böse Begierden und Leidenschaften und verjährte Gewohnheiten oft überreden wollen, daß wir nicht frei wären, und folglich auch nicht zur Verantwortung gezogen werden könnten, wenn wir das Böse thun.~\RWSeitenw{10}
\item Die wichtigste Lehre des Christenthums aber, nicht nur unter denjenigen, von denen wir eben sprechen, sondern wohl unter allen, die eine Offenbarung nur immer mittheilen kann, ist die Lehre von der \RWbet{endlosen Fortdauer unseres Geistes}, von welcher wir uns durch bloße Gründe der Vernunft nicht bis zu dem Grade versichern können, daß uns eine uns von Gott selbst gegebene Bestätigung derselben gleichgültig seyn könnte. Die Lehre gewährt uns vornehmlich folgende Vortheile.
\begin{aufzb}
\item Durch den Glauben an Unsterblichkeit gewinnen wir, und nicht nur wir allein, sondern auch alle unsere Mitmenschen in unsern eigenen Augen an Wichtigkeit. Nun ist uns der Mensch nicht mehr eine vorübergehende Gestalt, ein Schattenbild, das heute noch war, und morgen verschwunden seyn wird; sondern er dauert in alle Ewigkeit fort, und er vermag eben deßhalb, obwohl seine Kräfte nur eingeschränkt sind, durch die unendliche Dauer seines Daseyns hindurch, unendlich viel Gutes sowohl, als auch Böses zu wirken.
\item Auch jede einzelne unserer Handlungen erscheint uns unendlich wichtiger. Jede bringt nicht nur in uns selbst, sondern auch in unsern Mitmenschen Veränderungen hervor, die sich, -- da wir in Ewigkeit fortdauern, -- auch in Ewigkeit erstrecken werden. Muß dieser Gedanke nicht unserem Leichtsinne steuern? muß er uns nicht Behutsamkeit und Vorsicht lehren in Allem, was wir thun?
\item Nun fühlen wir auch ungleich mehr Lust und Eifer, an unserer eigenen, und sofern wir es vermögen, auch an der geistigen Vervollkommnung unserer Mitmenschen zu arbeiten. Jedes irdische Werk, durch dessen Hervorbringung wir uns ein Denkmal zu stiften bemüht sind, und wären es auch ägyptische Pyramiden, wird durch den Zahn der Zeit zerstöret. Nur Vollkommenheiten, die wir an unserem eigenen Geiste entwickeln, oder zu deren Entwickelung wir Andern behülflich werden, nur diese dauern in Ewigkeit fort, wenn die Seele unsterblich ist. Wenn wir dagegen im Tode vernichtet zu werden glauben: so scheint uns jede Mühe, die wir auf die Ausbildung un\RWSeitenw{11}seres Geistes verwenden sollen, vergeblich, weil dieser Geist heut oder morgen in Nichts zurückkehren wird.
\item Wenn wir nicht unsterblich sind: so ist der Lohn der Tugend und die Strafe des Lasters sehr beschränkt und endlich; so finden verschiedene edle Thaten, besonders solche, die mit Aufopferung unseres irdischen Wohles, unserer Gesundheit, ja unseres Lebens selbst verbunden sind, keine hinreichende Vergeltung; und manche schändliche Verbrechen bleiben oder hoffen straflos zu bleiben; so kann es der Lasterhafte durch einen schnellen Selbstmord gleichsam Gott selbst unmöglich machen, ihn zu bestrafen. Glauben wir also nicht an Unsterblichkeit: so fühlen wir auch keine so starken Antriebe zur Tugend und keine so mächtigen Abhaltungsgründe vom Laster, als es der Fall ist, sobald uns der Glaube an Unsterblichkeit lehrt, daß wir durch Tugend und durch Laster uns eine, in ihrer Dauer ewige, in ihrem Grade aber, wegen der immer wachsenden Empfänglichkeit unseres Geistes, alle jetzigen Vorstellungen übersteigende Seligkeit oder Unseligkeit bereiten.
\item Durch den Glauben an Unsterblichkeit gewinnt auch unser Glaube an Gottes Gerechtigkeit. Wenn es kein anderes Leben gibt: so ist das Mißverhältniß, das hier auf Erden so oft zwischen der Tugend und Glückseligkeit wahrgenommen wird, einer der stärksten Einwürfe gegen die göttliche Gerechtigkeit, der wegfällt, sobald wir nach dem Tode fortdauern.
\item Durch den Glauben an Unsterblichkeit hört der Tod auf, ein trauriges Ereigniß zu seyn für alle Tugendhaften. Sehen wir im Tode unserer Vernichtung entgegen: so können wir, -- obgleich es ein Irrthum seyn mag, -- doch nicht umhin, vor ihm zurückzuschaudern, als vor dem größten Uebel.\RWfootnote{%
	Freundliche Gewohnheit des Daseyns und Wirkens! von dir soll ich scheiden? Göthe.}
Wissen wir aber, daß der Tod gute Menschen nur in ein besseres Leben einführt: so ist er ja eben deßhalb als ein wahres Glück für sie zu erachten; so können und müssen wir uns leichter beruhigen, sowohl wenn er uns selbst sich naht, als auch, wenn er~\RWSeitenw{12}\ uns Einen aus unseren Lieben entreißt; besonders wenn wir uns jenseits wieder zu finden hoffen.
\item Durch diese Ansichten wird uns denn auch unser ganzes jetzige Leben erfreulicher, und der Gedanke an den Tod, der sich uns oft ganz unwillkürlich aufdringt, und den wir billig auch nicht verscheuchen sollen, hört auf, ein Störer unserer Freuden zu seyn, wenn sie nur unschuldig sind.
\item Besonders wohlthätig aber erscheint uns dieser Glaube in trüben, leidensvollen Tagen; denn diese werden uns nur durch den Gedanken erträglich, daß uns ein besseres Leben bevorsteht, ein Leben, in dem kein Schmerz und keine Traurigkeit mehr seyn wird, sondern in dem die reinsten Freuden ununterbrochen sollen genossen werden, zu deren Genusse wir aber uns nur durch die geduldige Ertragung der Leiden dieser Welt fähig und würdig machen.
\end{aufzb}
\end{aufza}\par
\RWbet{Einwurf.} Der Glaube an Unsterblichkeit ist der Tugend und Glückseligkeit des Menschen eher nachtheilig als vortheilhaft. Erwarteten wir kein ewiges Leben: so würden wir die Zeit des gegenwärtigen besser zu Rathe halten; jetzt dagegen denkt Mancher, wie Lessing (in seiner Erziehung des Menschengeschlechtes) geschrieben: Was habe ich denn zu verlieren? Ist nicht \anf{die ganze Ewigkeit mein?} -- In jenen Verbindungen, die Freundschaft oder Liebe knüpft, würden wir einander mit viel mehr Zärtlichkeit behandeln, und statt uns so leichtsinnig zu betrüben, viel emsiger uns bestreben, einander Freude zu machen, wenn wir bedächten, daß wir einander nur auf die kurze Zeit dieses Lebens besitzen, und was wir den Geliebten jetzt Gutes zu thun versäumen, nie wieder einbringen können. -- Die Hoffnung, jenseits der Gräber eine Seligkeit, mit der nichts Irdisches verglichen werden kann, zu finden, verleitet uns zur Verachtung der Genüsse, die uns dieß gegenwärtige Leben darbeut, besonders wenn wir glauben, daß die Verzichtleistung auf die Genüsse dieses Lebens uns in jenem andern zum Verdienste werde angerechnet werden. -- Ja, es ist nicht zu wundern, wenn wir bei einer lebhafteren Einbildungskraft sogar die Versuchung empfinden, uns durch eine freiwillige Abkürzung dieses Lebens den Ein\RWSeitenw{13}gang in jene ewige Seligkeit zu eröffnen; denn wenn nach dem Tode wirklich ein unendlich seliges Leben anhebt: wie will man uns die Unerlaubtheit des Selbstmordes gründlich erweisen? und was will man dem entgegnen, der, wenn er Tausende hinmorden läßt, sich rühmt, daß er ihr größter Wohlthäter sey, weil er ihnen so zum Genusse ewiger Seligkeit verhilft? -- Wie beängstigend ist endlich nicht der Gedanke an eine Ewigkeit, und an unser Schicksal in ihr, selbst für die Besseren aus uns! Wie furchtbar finden wir deßhalb nicht Alle den Tod! während er bei der entgegengesetzten Vorstellung, nach der er ein festes Entschlafen wäre, für Gute und Böse alles Schreckbare verlöre. (\RWbet{Wieland} in seiner \RWbet{Euthanasia}\RWlit{}{Wieland1} \uA )\par
\RWbet{Antwort.} 
\begin{aufza}
\item Es ist wohl nicht zu glauben, daß wir die Zeit unseres Aufenthaltes auf Erden besser (\dh\ gemeinnütziger) anwenden würden, wenn wir uns vorstellten, im Tode aufzuhören, als es geschehen muß, wenn wir erwägen, was uns das Christenthum sagt, daß von der guten oder schlechten Anwendung dieses kurzen Zeitraumes unser seliges oder unseliges Schicksal in einer ganzen Ewigkeit abhängt.
\item Nicht der Gedanke, daß wir einander auch in der anderen Welt noch besitzen werden, erzeugt die Trägheit, die wir uns zuweilen in unseren Beweisen der Liebe zu Schulden kommen lassen, sondern diese Trägheit entsteht, wenn wir vergessen, wie schnell uns der Tod unsere Geliebten entreißen, und uns dadurch der Gelegenheit berauben könne, ihnen Genüsse von solcher Art, wie sie gerade nur auf Erden möglich sind, zu verschaffen. Der wohlangewandte Glaube an Unsterblichkeit trägt vielmehr bei zur Veredlung aller unserer Verbindungen auf Erden und zur Verbesserung unseres ganzen Benehmens gegen unsere Mitmenschen. Wir hüten uns, Verbindungen einzugehen, die unsere Ehre selbst noch im andern Leben brandmarken müßten. Wir scheuen uns, gegen unsere Geliebten die geringste Untreue auch nur im Herzen zu begehen, weil sie dieß noch in jenem andern Leben erfahren und dadurch gekränkt werden könnten. Wir nehmen uns in Acht, Jeden, wer er auch immer sey, zu ärgern und zum Bösen zu verführen, weil sich die Folgen dieser Verführung~\RWSeitenw{14}\ durch die ganze Ewigkeit erstrecken, und also uns auch beunruhigen könnten, \usw\
\item Wenn wir vom Werthe der irdischen Freuden, besonders der sinnlichen, etwas gemäßigter denken: so ist dieß für unsere Tugend sowohl als Glückseligkeit vortheilhaft; daß wir sie aber völlig verachten sollten, ist keine nothwendige Folge, die der Glaube an Unsterblichkeit hervorbringt. So wird auch kein wohl unterrichteter Christ glauben, daß nur willkürliche Verzichtleistung auf die Freuden dieser Erde etwas Verdienstliches sey. Erwartet er aber nur für solche Opfer, die er der Tugend gebracht hat, Ersatz: so ist sein Glaube der Tugend beförderlich.
\item Die lebhafteste Vorstellung der Freuden des andern Lebens kann den vernünftigen Christen zu keinem Selbstmorde verleiten, da er weiß, daß er sich durch ein solches Verbrechen den Eingang in diese Freuden nicht öffnen, sondern vielmehr auf ewig versperren würde.
\item Ueber die Unerlaubtheit des Selbstmordes sowohl, als eines jeden andern Mordes entscheidet das Christenthum ganz ausdrücklich. Da es nun die göttliche Offenbarung allein ist, durch die wir die Nachricht erhalten, daß es ein anderes Leben gibt: so versteht es sich von selbst, daß wir aus dieser Nachricht auch keine Folgerungen, welche die Offenbarung selbst verbietet, ziehen dürfen.
\item Daß der Gedanke an den Tod für wirklich gute Menschen durch die christliche Lehre von der Unsterblichkeit furchtbarer werde, als er es bei der Vorstellung, daß er uns vernichte, seyn würde, widerspricht aller Erfahrung. Daß er aber für Menschen, die nur gut scheinen, und es nicht wirklich sind, die vieles Böse gethan, und es zu thun noch immer fortfahren wollen, überaus furchtbar werde, ist allerdings wahr, und beweiset, wie wirksam dieser Glaube für die Beförderung der Tugend sey.
\end{aufza}

\RWpar{171}{Wirklicher Nutzen}
Wenn wir erwägen, wie oft die wichtigen Lehren von der Immaterialität der Seele, von ihrer Fähigkeit, auch ob\RWSeitenw{15}jective Wahrheit zu erkennen, von der Freiheit des menschlichen Willens, zwar nicht von der großen Menge der Menschen, aber doch von Gelehrten bezweifelt und geläugnet worden sind: so sehen wir ein, daß sich die christliche Religion durch die ausdrückliche Aufstellung dieser Lehren wirklich ein sehr bedeutendes Verdienst um diese Classe der Menschen, die so viel schaden kann, wenn sie irrt, und mittelbar also auch um die ganze übrige Menschheit erworben habe.\par
Wichtiger aber als alles Andere sind die vielen sittlichen Vortheile, die aus der ausdrücklichen Versicherung von der endlosen Fortdauer unserer Seele für unzählige Menschen entsprangen. Alle Gebildete nämlich, Alle, die Gründe verlangen, bevor sie glauben sollen, hätten den so beseligenden Glauben an die Unsterblichkeit ihrer Seele nie festhalten können, wenn die christliche Offenbarung hierüber schwiege; wenn sie nicht ausdrücklich versicherte, daß wir in Ewigkeit fortleben werden. Wer möchte nun zählen die vielen Sünden und Verbrechen, die dieser Glaube verhindert, die vielen guten Werke, die er hervorgebracht, die vielen Thränen, die er getrocknet, \usw ! Welche Verwilderung würde auf Erden eingerissen seyn, und noch jetzt einreißen, wenn dieser wohlthätige Glaube nicht gewesen wäre, oder noch jetzt verdrängt werden könnte! -- Es ist zwar nicht zu läugnen, daß dieser Glaube zuweilen auch unrichtig angewandt wurde; namentlich, daß er einige Menschen veranlaßte, den Freuden dieser Welt freiwillig zu entsagen, um desto mehr Verdienste für den Himmel zu sammeln. Aber was ist dieser kleine Nachtheil gegen so viele und so große Vortheile! Und war nicht selbst diese Verirrung lehrreich für Millionen anderer Menschen, welche aus dem, was Jene für die Ewigkeit thaten, entnahmen, wie sehr es doch der Mühe lohne, seine Pflicht eifrig zu erfüllen, um nur der ewigen Seligkeit einst nicht verlustig zu werden!

\RWpar{172}{Die Lehre des Christenthums von aller Menschen wesentlicher Gleichheit}
Das ganze menschliche Geschlecht, \dh\ alle auf Erden befindlichen Geschöpfe, denen die in der vorhergehenden Lehre,~\RWSeitenw{16}\ erwähnten Eigenschaften zukommen, \RWbet{stammen von einem und eben demselben ersten Elternpaare ab.} Sie sind daher \RWbet{einander in allen wesentlichen Stücken gleich}, so gleich, wie es nur immer Geschöpfe von \RWbet{einerlei Gattung} seyn können.

\RWpar{173}{Historischer Beweis dieser Lehre}
\begin{aufza}
\item \RWbibel{Gen}{1\,Mos.}{1}{2\,ff}\ wird die Geschichte der Ausbildung unserer Erde (ihre Versetzung in diesen gegenwärtigen Zustand) erzählt. Nachdem bemerkt worden ist, daß die Erde vorher ganz wüste und leer gewesen war, wird beschrieben, wie durch die mittel- oder unmittelbare Einwirkung Gottes erst allerlei andere Geschöpfe, Pflanzen und Thiere auf Erden erschienen wären; zuletzt heißt es im \RWbibel[V.\,27.]{Gen}{}{1}{27}: \erganf{Gott schuf die Menschen nach seinem Ebenbilde, als Mann und Weib schuf er sie.} Von diesem ersten Menschenpaare, welchem die Namen \RWbet{Adam} (Erdenmann) und \RWbet{Eva} (Mutter der Lebendigen) beigelegt werden, wird in der Folge die Bevölkerung des ganzen Erdballs hergeleitet.
\item Dasselbe wird auch im Buche der Weisheit (\RWbibel{Weish}{}{10}{1}) angenommen, wo es heißt: \erganf{Die Weisheit schützte den zuerst gebildeten, allein erschaffenen Vater des Menschengeschlechtes, und half ihm auf von seinem Falle.}
\item Und eben diese Lehre wird auch in den Büchern des neuen Bundes vorausgesetzt; \zB\ \RWbibel{Apg}{Apostelg.}{17}{26}: \erganf{Gott ließ von Einem her das ganze Menschengeschlecht die Erde bewohnen.} Hieher gehören auch alle Stellen, wo von der Gleichheit aller Menschen, von der Nichtigkeit des Unterschiedes zwischen Juden und Heiden, von der Unparteilichkeit Gottes, und von der Pflicht, alle Menschen ohne Unterschied ihres Ranges auf eine gleiche Art zu behandeln, die Rede ist; \zB\ \RWbibel{Jak}{Jak.}{2}{1--10}\ \RWbibel{Hiob}{Job.}{34}{19}\ \RWbibel{Apg}{Apostelg.}{10}{34} \uam\ 
\end{aufza}

\RWpar{174}{Vernunftmäßigkeit}
\begin{aufza}
\item In dieser, so wie in allen historischen Lehren muß man dasjenige, was einen Einfluß auf Tugend und Glück\RWSeitenw{17}seligkeit hat, und folglich in der That zur christlichen Religionslehre gehört, von demjenigen trennen, was im Grunde gleichgültig für diese beiden Zwecke ist. Von Wichtigkeit in dieser Lehre ist eigentlich nur die Behauptung, \RWbet{daß alle Menschen einander wesentlich gleich}, und zwar so gleich sind, als ob sie von einem und demselben Elternpaare abstammten. Der Umstand aber, ob sie auch wirklich nur von einem einzigen Paare abstammen, ist, wie man sieht, sehr gleichgültig, wenn man nur nie das Erstere läugnet. Wir hätten also im Grunde nur nöthig, hier die Vernunftmäßigkeit des Ersteren zu zeigen. Und was kann einleuchtender seyn, als diese? Alle Menschen, die wir bisher auf Erden angetroffen haben, sind einander in allen den wesentlichen, \dh\ in allen denjenigen Stücken, welche auf unsere Behandlung derselben, auf die ihnen einzuräumenden Rechte und auf unsere Pflichten gegen sie, einen Einfluß haben, einander so gleich, wie es nur immer der Fall seyn könnte, wenn sie von einem und demselben Elternpaare abstammen würden. Die Unterschiede in der Farbe der Haut, in der Structur gewisser Glieder, \usw\ sind in dieser Hinsicht durchaus von keiner Bedeutung; sie sind viel unbedeutender, als manche andere in den Verstandeskräften, \udgl , die man oft unter Menschen antrifft, welche ganz erweislicher Maßen von eben denselben Stammeltern herrühren (\zB\ bei Kindern derselben Familie). Die Naturforscher pflegen Individuen von Pflanzen oder von Thieren zu einer und derselben Art zu zählen, wenn unter denselben (falls sie verschiedenen Geschlechtes sind) eine Vermischung Statt findet, aus welcher neue, zur Fortpflanzung geeignete Individuen hervorgehen. Daß dieses bei Menschen aus allen Gegenden der Fall sey, und daß die Menschen also auch in dieser Rücksicht nur Eine Art ausmachen, ist allgemein bekannt.
\item Was nun das Zweite betrifft, oder die Behauptung, daß die Menschen in der That nur von einem einzigen Paare abstammen: so ist auch dieses, ob es gleich nicht zum Wesen der Religion gehört, bisher wenigstens noch nicht widerlegt worden. Es hat zwar einige Schwierigkeit, zu erklären, auf welche Art \zB\ Amerika bevölkert worden sey; doch läßt sich annehmen, daß dieses mittelst der vielen dazwischen lie\RWSeitenw{18}genden Inseln aus Nordostasien geschehen sey, etwa durch die Tungusen, deren Lebensweise wirklich mit jener der wilden Amerikaner in mehreren Stücken übereinkommt. Und selbst die Sprache gibt einige Vermuthung dafür. (Siehe Untersuchungen über Amerika's Bevölkerung aus dem alten Continente. Von Sev.\ Vater.\ Leipzig, 1810.)\RWlit{}{Vater1}
\end{aufza}

\RWpar{175}{Sittlicher Nutzen}
\begin{aufza}
\item Auf die Lehre von der wesentlichen Gleichheit aller Menschen gründet sich die richtige Erkenntniß der Pflichten gegen unsere Mitmenschen, nämlich, daß wir ihnen Allen wesentlich gleiche Rechte und Ansprüche auf irdische Glückseligkeit einräumen, und für die Beförderung ihres Wohles mit eben dem Eifer, wie für die des unsrigen, besorgt seyn sollen. Wenn nicht alle Menschen wesentlich gleich wären, wenn wir uns vorstellten, daß einige (\zB\ wir selbst) einer wesentlich vollkommeneren Art seyen: so würden wir auch nicht die Pflicht der gleichen Nächstenliebe erkennen, sie wäre auch in der That gar nicht für uns vorhanden.
\item Obwohl der Umstand, daß alle Menschen von einem und eben demselben Stammpaare abstammen, nicht wesentlich ist: so ist er doch als Bild sehr vortheilhaft, um die Gesinnungen der Bruderliebe unter uns zu erhalten. Denn wenn wir uns vorstellen, daß wir in der That alle einen und eben denselben Stammvater haben: so leuchtet uns deutlicher ein die Pflicht, jeden unserer Mitmenschen gleich unserem Bruder zu behandeln; ja diese Vorstellung erweckt sogar jenes natürliche Gefühl der Liebe, welches wir gegen Brüder zu fühlen pflegen, in unseren Herzen, und macht, daß wir es auch auf jeden andern Menschen übertragen.
\item Der gemeine Menschenverstand, dem es kaum möglich ist, sich vorzustellen, daß das menschliche Geschlecht ohne Anfang sey (eine Vorstellung, der übrigens die Geschichte widerspricht) erhält auf diese Art einen bestimmteren Begriff, wie es mit der Entstehung des Menschengeschlechtes hergegangen sey, und zwar einen Begriff, welcher der Weisheit Gottes vollkommen gemäß ist. Hätte \zB\ das menschliche Ge\RWSeitenw{19}schlecht mehrere Stammeltern, so würde es scheinen, daß Gott hier etwas Ueberflüssiges gethan, und das Gesetz der Sparsamkeit verletzt habe; \usw\
\end{aufza}

\RWpar{176}{Wirklicher Nutzen}
Kein Vernünftiger wird den unberechenbar großen Nutzen verkennen, den die herrliche Lehre des Christenthums von aller Menschen wesentlicher Gleichheit gebracht hat. Wer möchte sie zählen, die Millionen Gutthaten, Hülfeleistungen, großmüthige Vergebungen, \usw\, die der Gedanke: Es ist mein Bruder, den ich vor mir habe! bewirkt hat? -- Und wenn auch die Christen noch sehr oft grausam mit ihren Nebenmenschen umgegangen sind: was hätten sie gethan, wenn diese Lehre nicht gewesen wäre? -- In vielen heidnischen Religionen findet sich in der That das Gegentheil.

\RWpar{177}{Die Lehre von der Aehnlichkeit des Menschen mit Gott}
\begin{aufza}
\item Die christliche Religion ertheilt dem Menschen die Erlaubniß, sich, wenigstens dann, wenn er im Stande der Unschuld sich befindet, als das \RWbet{Ebenbild Gottes auf Erden} zu betrachten. Dieß soll den Sinn haben, daß der Mensch vorzugsweise vor allen übrigen Geschöpfen dieser Erde die \RWbet{meiste Aehnlichkeit} mit Gott, die meisten Eigenschaften und Vorzüge mit ihm gemein habe.
\item Fragen wir, worin diese Aehnlichkeit des Menschen mit Gott bestehe: so antwortet uns die christliche Religion, es sey nicht etwa die Gestalt des Leibes, welche der Mensch mit Gott gemein hat; sondern nur darin sey der Mensch Gott ähnlich, \RWbet{daß er Verstand und Willen wie Gott habe}, daß er \RWbet{ein endloses Daseyn habe,} daß ihm ein \RWbet{freier Antheil an Gottes Weltregierung} eingeräumt sey, und daß er über die mancherlei Dinge auf Erden, \zB\ über die thierische Welt nach ähnlichen Grundsätzen \RWbet{herrschen könne und solle,} nach welchen er Gott im ganzen Weltall herrschen sieht. Daher verliert er denn auch diese Aehnlichkeit mit Gott, sobald er lasterhaft wird.~\RWSeitenw{20}
\end{aufza}

\RWpar{178}{Historischer Beweis dieser Lehre}
\begin{aufza}
\item Schon \RWbibel{Gen}{1\,Mos.}{1}{26}, wo die Geschichte der Schöpfung des Menschen erzählt wird, heißt es: \erganf{Und Gott (\RWhebr{'E:lOhiym}, Elohim) sprach: Lasset uns jetzo den Menschen erschaffen \RWbet{nach unserem Ebenbilde,} ein Gleichniß von uns selbst lasset uns ihn erschaffen. Er herrsche über die Fische des Meeres und über die Vögel des Himmels, über die Thiere des Waldes und über den ganzen Erdkreis. Und hierauf schuf Gott den Menschen \RWbet{nach seinem Ebenbilde}, nach \RWbet{Gottes Ebenbilde} erschuf er ihn.}
\item Eben so heißt es \RWbibel{Weish}{Weish.}{2}{23}: \erganf{\RWbet{Unsterblich} schuf Gott den Menschen, und machte ihn zu einem \RWbet{Ebenbilde seiner selbst.}}
\item Und der heil.\ Paulus schreibt \RWbibel{Eph}{Ephes.}{4}{24}: \erganf{Ziehet an den neuen Menschen, den, der nach \RWbet{Gottes Ebenbilde} erschaffen ist, der in der \RWbet{Gerechtigkeit}, in der \RWbet{Wahrheit} und \RWbet{Heiligkeit} wandelt.} -- Woraus zu ersehen ist, daß der Mensch die Aehnlichkeit mit Gott verlieren könne, wenn er lasterhaft lebt. So auch \RWbibel{Kol}{Koloss.}{3}{10}
\item Was übrigens den Umstand belangt, worin die Aehnlichkeit des Menschen mit Gott bestehe: so sieht man schon aus den bisherigen Stellen, daß das Christenthum diese Aehnlichkeit nicht etwa in die Gestalt des Leibes setze, zumal, da eben dasselbe Christenthum lehrt, daß Gott keinen Leib, keine Gestalt besitze. Aus der oben angeführten Stelle \RWbibel{Gen}{1\,Mos.}{1}{26}\ geht deutlich genug hervor, daß der Verfasser diese Aehnlichkeit vornehmlich in die \RWbet{Herrschaft} des Menschen, in seinen \RWbet{freien Antheil an Gottes Weltregierung} gesetzt habe; und aus der Nr.\,2.\ angeführten Stelle ersieht man, daß auch die Unsterblichkeit zu dieser Aehnlichkeit gehöre, \usw\ Die Kirchenväter haben diese Aehnlichkeit, Einige besonders in diesen, Andere in jenen Umstand gesetzt, \zB\ \RWbet{Epiphanius} und \RWbet{Chrysostomus} in die Herrschaft des Menschen, \RWbet{Augustinus} in die Erkenntnißkraft, \usw\ Darin waren sie aber alle einig, daß der Mensch dann, wenn er durch Laster sich entehret, das Recht verliere, sich Gottes Ebenbild zu nennen.~\RWSeitenw{21}
\end{aufza}

\RWpar{179}{Vernunftmäßigkeit}
\begin{aufza}
\item Ohne Zweifel hat der Mensch vorzugsweise vor allen andern Geschöpfen auf Erden die meiste Aehnlichkeit mit Gott, so daß, wenn irgend ein Geschöpf auf dieser Erde das Bild der Gottheit vorstellen soll, der Mensch am Ehesten dazu berechtiget ist.
\item Einleuchtend ist es aber, daß diese Aehnlichkeit nicht in der Gestalt des Leibes bestehen könne; sondern daß die vornehmsten Eigenschaften, welche der Mensch mit Gott gemein hat, wirklich die oben angegebenen sind.
\item Eben so richtig ist es, daß der Mensch nicht mehr verdiene, sich als ein Ebenbild Gottes zu betrachten, sobald er lasterhaft wird; denn dann überwiegt die eine Unähnlichkeit, welche aus seinem bösen, mit dem göttlichen im Widerspruche stehenden Willen hervorgeht, alle übrigen Aehnlichkeiten so sehr, daß sie nicht mehr beachtet werden können.
\end{aufza}

\RWpar{180}{Sittlicher Nutzen}
\begin{aufza}
\item Die Nachricht, daß uns Gott für sein Ebenbild auf dieser Erde erkläre, muß uns doch nothwendig eine erfreuliche, erhebende, eine unserer natürlichen Ehrliebe überaus willkommene Nachricht seyn, die uns doch keineswegs zu einem schädlichen Stolze verleiten kann, weil nicht nur wir, sondern auch jeder unserer Nebenmenschen für Gottes Ebenbild erklärt wird. Dieß Letztere muß uns also
\item nur ein Beweggrund mehr werden, unsern Nebenmenschen mit aller Achtung und Schätzung zu behandeln; denn sollten wir ihn etwa verhöhnen, verspotten, mißhandeln wollen: so müßten wir ja in der That Gott selbst in der Mißhandlung seines Ebenbildes zu beleidigen fürchten.
\item Da aber das Christenthum die Aehnlichkeit des Menschen mit Gott unter Anderm auch darein setzt, daß er nach eben den Grundsätzen, nach welchen er Gott im ganzen Weltall herrschen sieht, in dem ihm angewiesenen Gebiete regieren kann und soll: so muß uns dieser Gedanke antreiben, jene~\RWSeitenw{22}\ Grundsätze Gottes immer vollkommener kennen zu lernen, und sie in den Verhältnissen, in denen wir uns befinden, immer getreuer nachzuahmen.
\item Wie müssen wir uns endlich in Acht nehmen, daß wir durch Handlungen einer niedrigen Sinnlichkeit, oder durch eine grausame Behandlung unserer Mitmenschen, und andere dergleichen Verbrechen nicht Gottes Ebenbild in uns entweihen!
\end{aufza}

\RWpar{181}{Wirklicher Nutzen}
Um den wirklichen Nutzen, den das Christenthum durch diese erhabene Lehre gestiftet hat, gehörig zu würdigen, muß man zuvor erwägen, welche entgegengesetzte, äußerst entehrende Begriffe von dem Range der menschlichen Natur in andern Religionen, und selbst in den Systemen der Weltweisen, Statt gefunden hatten. \RWbet{Hesiodus} lehrte (\RWlat{Theogon.\ v.\,15, 4.})\RWlit{}{Hesiod}, der Mensch wäre durch die Wärme der Sonnenstrahlen aus der Erde ausgebrütet worden. Derselben Meinung waren die alten Deutschen (\RWlat{Tacitus de mor.\ German.\ cap.\,1. cap.\,40.}). \RWbet{Anaximander} ließ die ersten Menschen aus dem Bauche der Fische hervorgehen. \RWbet{Empedokles} (\RWlat{Luc.\ 1.5.}) lehrte, die Erde habe die Gliedmaßen des menschlichen Leibes erst einzeln hervorgebracht, und der Zufall habe sie dann vereiniget. Die \RWbet{Manichäer} behaupteten, der erste Mensch wäre von dem bösen Grundwesen erschaffen worden. \RWbet{Rousseau} hielt dafür, die Menschen wären nur ein verfeinertes Affengeschlecht; \usw\ Die \RWbet{Socinianer} und \RWbet{Arminianer} in unserer Zeit, und die \RWbet{Enkratiten} und \RWbet{Severianer} schon im 2ten und 3ten Jahrhunderte, behaupteten (aus \RWbibel{1\,Kor}{1\,Kor.}{11}{7}), daß nur der Mann, nicht aber das Weib das göttliche Ebenbild an sich trage; \usw\

\RWpar{182}{Geständniß der christlichen Religion über die Größe des Uebels auf Erden}
Wir haben es bereits in dem Hauptstücke von der Nothwendigkeit einer Offenbarung bemerkt, daß eine der größten Schwierigkeiten in der natürlichen Religion aus der Betrach\RWSeitenw{23}tung der vielen Uebel auf Erden, besonders derjenigen, die auf uns Menschen selbst lasten, entspringe. Das Christenthum läugnet nun das Vorhandenseyn dieser Uebel gar nicht; es würde, wenn es dieß thäte, den Knoten, der sich hier findet, nicht lösen, sondern zerhauen; denn daß es seine Richtigkeit mit dem Vorhandenseyn dieser Uebel habe, fühlen wir ja nur zu lebhaft. Aber das Christenthum ist weit entfernt, uns die Richtigkeit dieses Gefühles streitig zu machen; es schildert selbst das \RWbet{sittliche} sowohl, als auch das \RWbet{physische} Uebel auf Erden mit den lebhaftesten Farben, ohne doch darum in Uebertreibungen zu verfallen.
\begin{aufza}
\item \RWbet{Das sittliche Uebel}. Schon \RWbibel{Gen}{1\,Mos.}{8}{21} heißt es: \erganf{Das Dichten und Trachten des menschlichen Herzens ist zum Bösen geneigt von Jugend auf.} Und Johannes schreibt \RWbibel{1\,Joh}{1\,Joh.}{1}{8}: \erganf{Wenn wir sagen, daß wir nicht gesündiget haben: so täuschen wir uns selbst, und die Wahrheit ist nicht in uns.} Der heil.\ Paulus spricht (\RWbibel{Röm}{Röm.}{7}{18}) wahrscheinlich nicht sowohl in seiner eigenen, als in der Person eines jeden durch das Christenthum noch nicht geheiligten Menschen: \erganf{Ich weiß, daß in mir, das heißt in meinem Fleische, das Gute nicht wohnt. Zwar das Wollen liegt mir nahe; aber das Vollbringen des Guten finde ich nicht; denn nicht das Gute, das ich will, thue ich; sondern ich thue das Böse, das ich nicht will.} Heißt dieß nicht nachdrücklich genug über den starken Trieb der Sinnlichkeit klagen? Das Christenthum verhehlt uns also fürwahr die Beschwerden, welche das Streben nach Tugend mit sich bringt, gar nicht.
\item \RWbet{Physisches Uebel.} \RWbibel{Sir}{Sirach}{40}{1}: \erganf{Zu vieler Unruhe ist jeder Mensch bestimmt. Ein schweres Joch drückt Adams Söhne von dem Tage an, da sie aus ihrer Mutter Schooß hervortreten, bis an den Tag, da sie in jene Erde, die unser Aller Mutter ist, wieder begraben werden. Sorge und Furcht erzeugt die Erwartung der Zukunft, und jener Tag des Todes. Von dem an, der auf einem Throne sitzet, bis zu demjenigen, der sich im Staube vor ihm erniedrigt; von dem an, der in ein Purpurkleid sich hüllt und eine Krone trägt, bis zu demjenigen, der seine Blöße kaum mit grober Leinwand deckt, herrscht überall nur Zorn und Hader, nur~\RWSeitenw{24}\ Neid und Eifersucht, Verwirrung, Ungewißheit! Selbst wann er ruhen will, der Mensch, auf seinem Lager, sind es noch fürchterliche Träume, die ihn stören. Von kurzer Dauer ist seine Ruhe, sein Sinnen währet auch selbst im Schlafe fort, als müßte er Wache halten. Gesichte schrecken ihn aus seinem Schlummer auf.}
\begin{RWanm} 
Die Vernunftmäßigkeit und der sittliche Nutzen dieser Lehre bedarf keiner weitläufigen Erörterung. Wenn das Christenthum die Größe dieser Uebel nicht eingestünde: so gliche es einem Arzte, der, statt dem Kranken zu helfen, ihn überreden will, daß er nicht krank sey. Wie der Kranke zu solch einem Arzte, so könnten auch wir zu einer solchen Religion kein Zutrauen fassen.
\end{RWanm}
\end{aufza}

\RWpar{183}{Die Lehre des Christenthums von der Vollkommenheit des ursprünglichen Zustandes der Menschen}
Um nun die Frage, \RWbet{woher dieß große Uebel rühre}, und \RWbet{wie es sich mit Gottes Vollkommenheit vereinigen lasse}, zu unserer Befriedigung beantworten zu können, fängt das Christenthum mit der Behauptung an, dieß Uebel wäre \RWbet{nicht von jeher} so gewesen.
 Das menschliche Geschlecht, sagt es, befand sich ursprünglich, \dh\ zu jener Zeit, als es zunächst aus der Hand seines Schöpfers hervorgegangen war, in dem Zustande einer \RWbet{weit größeren Vollkommenheit}. Von diesem Zustande gibt es uns folgende Beschreibung.
\begin{aufza}
\item Der \RWbet{Leib} der ersten Menschen war \RWbet{durchaus gesund und wohlgebildet,} alle zum Leben nöthige Verrichtungen desselben gingen mit großer Leichtigkeit von Statten. So hätte \zB , wären die Menschen in diesem Zustande der Vollkommenheit immer geblieben, das Weib mit Leichtigkeit, und nicht mit so unsäglichen Schmerzen, als es jetzt häufig der Fall ist, Kinder gebären können.
\item Ihr \RWbet{Trieb zur Sinnlichkeit} war bei Weitem noch \RWbet{nicht so heftig}, als er sich etwa jetzt bei uns zu äußern pflegt.
\item So lange sie sich von jeder Sünde enthalten haben würden, so lange hätten sie \RWbet{auch von keinem Tode} eine~\RWSeitenw{25}\ Unterbrechung ihrer Glückseligkeit zu befürchten gehabt; sie wären auch \RWbet{ihrem Leibe nach unsterblich} geblieben, hätten sie niemals gesündiget.
\item \RWbet{Die ganze Erde hatte eine viel freundlichere Gestalt;} ohne viele Mühe brachte sie die Pflanzen und Kräuter hervor, deren die Menschen zu ihrer Nahrung bedurften; \usw\
\item Vornehmlich damals galt es von dem Menschen, daß er das \RWbet{reine, unbefleckte Ebenbild der Gottheit} an sich trage.
\item Und daher kam es denn auch, daß er \RWbet{von Seite Gottes eines ganz vorzüglichen Wohlgefallens} gewürdiget wurde, und der Mittheilung göttlicher Gnaden im reichsten Maße genoß.
\end{aufza}

\RWpar{184}{Historischer Beweis dieser Lehre}
Man mag die bekannte Geschichte, welche \RWbibel{Gen}{1\,Mos.}{2}{1--3}\ \RWbibel[24.]{Gen}{}{2}{24}\ erzählt wird, eigentlich oder uneigentlich verstehen: so hat sie doch immer den offenbaren Zweck, zu lehren, daß sich die Menschen unmittelbar nach der Schöpfung in einem weit vollkommeneren Zustande befunden hätten, als es ihr nachmaliger wurde. So ist diese Geschichte auch immer von den Kirchenvätern verstanden worden. Insonderheit aber hat man immer geglaubt,
\begin{aufza}
\item daß der \RWbet{Körper} der ersten Menschen \RWbet{eine ganz vorzügliche Vollkommenheit} erhalten habe, und eben hieraus \zB\ das \RWbet{lange Lebensalter} der Menschen noch nach dem Sündenfalle erklärt. Und daß der Mensch alle Lebensverrichtungen mit einer gewissen Leichtigkeit habe verrichten können, schloß man daraus, weil erst nach dem Sündenfalle dem Weibe, als eine Strafe, angekündigt wird, daß sie mit Schmerzen Kinder gebären werde. (\RWbibel{Gen}{1\,Mos.}{3}{16})
\item Daß sich der \RWbet{Trieb zur Sinnlichkeit} damals noch nicht so mächtig geäußert habe, als er sich etwa jetzt äußert, schloß man daraus, weil \RWbibel{Gen}{1\,Mos.}{2}{25}\ angemerkt wird, daß Adam und Eva \RWbet{sich ihrer Nacktheit nicht}~\RWSeitenw{26}\ \RWbet{geschämt} hätten, wovon das Gegentheil nach dem Sündenfalle eintrat (\RWbibel{Gen}{1\,Mos.}{3}{10}).
\item Daß sie \RWbet{unsterblich} gewesen wären, wofern sie nie gesündigt hätten, schloß man aus dem Verbote (\RWbibel{Gen}{1\,Mos.}{2}{17}): \RWbet{Vom Baume der Erkenntniß des Guten und Bösen aber sollst du nicht essen, denn an dem Tage, an dem du von ihm issest, sollst du des Todes sterben}; und aus der Strafe (\RWbibel{Gen}{1\,Mos.}{3}{19}):
\RWbet{Im Schweiße deines Angesichtes sollst du dein Brod essen, bis du zur Erde zurückkehrst, von der du genommen bist; denn du bist Staub, und sollst wieder in Staub verwandelt werden.} Auch heißt es \RWbibel{Gen}{1\,Mos.}{3}{22}, daß Gott den Menschen aus dem Paradiese habe vertreiben müssen, damit er nicht etwa esse vom Baume des Lebens, und lebe ewiglich. -- Eben so sagt auch der heil.\ Paulus \RWbibel{Röm}{Röm.}{5}{12}: \RWbet{Durch Einen Menschen ist die Sünde in die Welt gekommen, und durch die Sünde der Tod.}
\item Daß die ganze \RWbet{Erde damals eine weit freundlichere Gestalt gehabt}, schloß man aus jener reizenden Beschreibung, die von dem ersten Aufenthaltsorte der noch unschuldigen Menschen gegeben wird, wie er denn auch ein \RWbet{Garten der Lust} genannt wird; dann aus dem Fluche \RWbibel{Gen}{1\,Mos.}{3}{17}: \RWbet{Verflucht sey die Erde um deines Werkes willen! Dornen und Disteln bringe sie hervor, mit vieler Mühe sollst du sie bebauen, und -- nur im Schweiße deines Angesichtes dein Brod verzehren.}
\item Daß es vornehmlich damals von dem Menschen gegolten, daß er das \RWbet{reine Ebenbild Gottes} an sich getragen habe, setzt der heil.\ Paulus voraus, wenn er uns (\RWbibel{Eph}{Ephes.}{4}{24}) ermahnt, zu jenem Ebenbilde Gottes, welches der erste Mensch an sich getragen, wieder zurückzukehren.
\item Daß die Menschen eines ganz \RWbet{vorzüglichen Wohlgefallens Gottes} gewürdiget worden seyen, ersieht man aus den schon vorhin erwähnten Wohlthaten, die ihnen Gott erwiesen. Uebrigens werden wir weiter unten sehen, daß die christliche Religion den Menschen nach dem Sündenfalle als einen Gegenstand des göttlichen Mißfallens darstellt; woraus denn umgekehrt folgt, er müsse früher in einem vorzüglichen Wohlgefallen bei Gott gestanden seyn.~\RWSeitenw{27}
\end{aufza}

\RWpar{185}{Vernunftmäßigkeit}
In wiefern diese Lehren ein wirkliches Factum betreffen, gehört zu ihrer Vernunftmäßigkeit, daß sie\par
\textbf{A.}~\RWbet{mit keinen apriorischen oder Vernunftwahrheiten}, und\par
\textbf{B.}~auch \RWbet{mit der Geschichte in keinem Widerspruche} stehen.\par

\vabst \textbf{A.}~\RWbet{Vereinbarkeit dieser Lehre mit der Vernunft.}\\
Der Gedanke, daß das ganze menschliche Geschlecht \RWbet{einen Anfang in der Zeit}\RWfootnote{%
	Dieser Gedanke kommt zwar schon in der Lehre von der wesentlichen Gleichheit aller Menschen vor, dort aber noch nicht als religiöse Lehre, weil sich noch kein sittlicher Nutzen oder Nachtheil desselben zeigt. Erst hier, wo von den ersten Menschen behauptet wird, daß sie vollkommener gewesen wären, \usw\ erhält der Gedanke eines Anfanges des menschlichen Geschlechtes sittliche Wichtigkeit, zum Wenigsten für alle Jene, welche die eben aufgestellte Lehre nicht fassen könnten, wenn sie dem menschlichen Geschlechte in ihrer Vorstellung nicht einen wirklichen Anfang ertheilten.}
genommen habe, enthält nichts Ungereimtes. Wenn es auch Geschöpfe überhaupt von Ewigkeit her gegeben haben muß: so folgt doch keineswegs, daß jede einzelne Gattung von Wesen von jeher da seyn, und immer fortdauern müsse. So wie nun gewisse Geschlechter der Thiere vor unsern Augen entstehen und wieder vergehen, so kann auch das menschliche Geschlecht einen Anfang genommen haben, und wieder ein Ende nehmen (so entsteht \zB\ die Schmetterlingswelt im Frühling und stirbt im Herbste wieder aus; oder die Infusionsthierchen in einer gährenden Flüßigkeit). Die Art, wie dieses menschliche Geschlecht einen Anfang genommen habe, ob durch unmittel- oder eine mittelbare Einwirkung Gottes, gehört auf keine Weise zu den Glaubenslehren; sondern hieher gehört nur höchstens der Satz, daß es \RWbet{nicht durch Vermittlung eines freien endlichen Wesens} (\zB\ eines Engels) geschehen sey. Und dieses ist der Weisheit und Güte Gottes in der That sehr gemäß. Hätte er die Bildung des menschlichen Geschlechtes irgend einem endlichen Geiste überlassen: so wäre eben darum Gefahr vorhanden gewesen, daß die Natur des Menschen nicht~\RWSeitenw{28}\ so vollkommen ausgebildet wurde, als es an sich genommen möglich war. Daß aber das erste Menschenpaar, welches, es sey nun auf diese oder jene Art, immer nur durch Gott allein entstanden ist, gewisse höhere Vollkommenheiten, besonders in Hinsicht des Leibes erhalten habe, als wir Uebrigen, die wir auf eine entferntere Weise, nämlich erst durch die freie Thätigkeit unserer Vorältern, von Gott abstammen, ist der Vernunft völlig gemäß. Sie findet es beinahe nothwendig, daß das, was ohne Dazwischenkunft endlicher Kräfte von dem Unendlichen selbst herrührt, vollkommener werden müsse, als dasjenige, auf dessen Entstehung und Ausbildung endliche Wesen einen mehr oder weniger wichtigen Einfluß gehabt haben. Insonderheit
\begin{aufza}
\item wird es ein Jeder begreiflich finden, daß \RWbet{der Leib der ersten Menschen vollkommen gesund und wohlgebaut} gewesen sey; und daß sonach alle zum Leben nöthigen Verrichtungen desselben mit Leichtigkeit von Statten gingen. Das Beispiel einiger wilder Völker, deren Weiber noch jetzt beinahe ohne alle Schmerzen gebären, zeigt uns, daß es allerdings nichts Unmögliches sey, was die christliche Religion in dieser Hinsicht von dem Zustande der ersten Menschen rühmt. Nur die verkehrte Lebensart, welche die meisten cultivirten Völker führen, nur die daraus entstehende Schwächlichkeit ihres Leibes, nur die zweckwidrige Behandlung, welche insonderheit der weibliche Körper schon von der frühesten Kindheit an erfährt (\zB\ das Einzwängen in enge Schnürbrüste), sind Ursache, daß die meisten Weiber mit solchen Schmerzen und mit Gefahr ihres Lebens gebären.
\item Eben so begreiflich ist auch, daß \RWbet{der Trieb zur Sinnlichkeit bei den ersten Menschen noch nicht so heftig} gewesen seyn soll, als er es bei uns zu seyn pflegt. Dieser so mächtige Trieb ist eine Folge der allzu großen Reizbarkeit unserer Nerven, und anderer dergleichen körperlicher Fehler, von welchen der Körper der ersten Menschen frei gewesen seyn konnte.
\item Auch die Behauptung, daß der \RWbet{Leib der ersten Menschen unsterblich} bleiben sollte, so lange sie nicht gesündiget haben würden, enthält nichts Ungereimtes. Man kann doch nicht beweisen, daß eine solche Unsterblichkeit~\RWSeitenw{29}
\begin{aufzb}
\item mit der Natur des menschlichen Leibes streite. Wer kann die Grenze bestimmen, wie weit der Mensch sein Leben, wenn er es gehörig schont, verlängern könne? Wer kann behaupten, daß es keine Pflanze gebe oder gegeben habe, durch deren Genuß die bei dem menschlichen Körper allmählich eintretende Steifheit der Fasern, welche den Tod vor Alter eigentlich herbeiführt, gehindert worden wäre? (\RWbibel{Gen}{1\,Mos.}{3}{22}\ wird so etwas von dem Baume des Lebens gerühmt.) Aber eben so wenig kann man behaupten, daß diese Unsterblichkeit sich
\item mit der höhern Bestimmung des Menschen nicht vertrage. Wer kann beweisen, daß ein langer Aufenthalt des Menschen auf Erden, wenn er ihn durch keine Sünde entweiht, zweckwidrig für ihn sey? daß er nicht fortwährend auf Erden lernen könne? \usw\ Ohnehin war bei der Freiheit des Menschen nicht zu erwarten, daß er nicht über kurz oder lang jene Bedingung, unter welcher allein ihm Gott die Unsterblichkeit versprach, verletzen werde. Und hiedurch ist auch schon
\item der Einwurf beseitigt, daß die Menschen nicht einmal alle Platz auf Erden hätten, wenn keiner abtreten sollte. Gott wußte ja vorher, daß sich die Menschen bald um dieß Vorrecht der Unsterblichkeit bringen würden.
\end{aufzb}
\item \RWbet{Die ganze Erde hatte eine noch weit freundlichere Gestalt.} -- Wir haben nicht nöthig, anzunehmen, daß Gott die schön geschaffene Erde nach dem Sündenfalle etwa verunstaltet und verschlimmert habe. Nein, die Veränderung ging im Menschen selbst vor. Nachdem er sich durch den Sündenfall um seine Gewissensruhe und um seine vollkommene Gesundheit gebracht hatte: so hörten die schönsten Umgebungen auf, für ihn reizend zu seyn; die Bebauung der Erde mußte ihm jetzt, bei seinen geschwächten Körperkräften, allerdings beschwerlicher fallen, als es vorhin der Fall gewesen wäre. Endlich war es auch nothwendig, daß Gott die ersten Menschen anfangs in einen solchen Aufenthaltsort versetzte, in welchem sie Alles, dessen sie zu ihrer Nahrung \usw\ bedurften, schon zubereitet fanden; denn sie besaßen noch nicht die Kunst, sich diese Bedürfnisse selbst zu bereiten. Später~\RWSeitenw{30}\ fiel diese Nothwendigkeit hinweg, und es wäre dem sinnlichen Menschen sogar schädlich gewesen, wenn ihn Gott immer noch im Garten der Lust gelassen hätte. Arbeit ward eine Wohlthat für ihn.
\item[5.\ und 6.] unterliegen gar keinen Schwierigkeiten.
\end{aufza}\par

\vabst \textbf{B.}~Ob diese Lehre nicht etwa \RWbet{mit der Geschichte in einem Widerspruche} stehe, werden wir bei der nächstfolgenden Lehre vom Sündenfalle untersuchen.

\RWpar{186}{Sittlicher Nutzen}
Der Glaube, daß sich die Menschen, als sie zunächst aus der Hand des Schöpfers hervorgegangen waren, in einem weit vollkommeneren Zustande befanden, als es der jetzige ist, muß
\begin{aufza}
\item unsere Ueberzeugung von Gottes Allmacht, Güte und Weisheit erhöhen.
\item Macht er uns begreiflicher, wie der Mensch mit einem so gebrechlichen, so vielen Krankheiten und Leiden ausgesetzten Körper, mit so vielen Unvollkommenheiten der Seele, mit einem so heftigen Triebe zum Bösen, gleichwohl das Geschöpf eines allmächtigen, höchst weisen und gütigen Wesens seyn könne;
\item nährt und befestiget er die Hoffnung, daß es uns durch eine genaue Befolgung der Vorschriften unseres Erlösers und durch Benützung aller der Mittel, die uns auch unsere eigene Vernunft angibt, möglich seyn werde, unsere ursprüngliche Vollkommenheit in den wichtigsten Stücken wieder zu erlangen. Durch diese Lehre wird uns insbesondere die Hoffnung gemacht, daß wir durch eine zweckmäßige Lebensweise die Kräfte und Vorzüge unseres Leibes um ein Beträchtliches erhöhen, unsere Lebensdauer verlängern, die Heftigkeit unserer Begierden dämpfen können, \usw\
\end{aufza}

\RWpar{187}{Die Lehre des Christenthums von dem Sündenfalle}
Daß nun die Menschen sich heut zu Tage nicht mehr in jenem vollkommenen Zustande befinden; daß es im Gegen\RWSeitenw{31}theile des Elends so viel auf Erden gibt, das kommt, sagt uns das Christenthum, nur \RWbet{von der Thorheit und der Sünde}.\par
\textbf{A.}~Und zwar die erste und vorzüglichste Ursache all dieses Elendes ist -- \RWbet{die erste Sünde}, welche die noch vollkommenen und unschuldigen Menschen im \RWbet{Paradiese} durch die mitwirkende \RWbet{Verführung eines bösen Geistes} begingen. Diese Sünde, die man nachdrücklicher den \RWbet{Sündenfall des menschlichen Geschlechtes} heißen kann, bestand in einem Ungehorsame der ersten Eltern gegen ein ausdrückliches Gebot, das ihnen Gott gegeben hatte.\par
Durch diese erste Sünde wurde
\begin{aufza}
\item der Körper nicht nur der ersten Menschen, sondern auch aller ihrer Nachkommen mehr oder weniger geschwächt. Die Weiber \zB\ gebären jetzt nur mit Schmerzen \usw ;
\item die sinnlichen Triebe des Menschen wurden zur Ungebühr erhöht und verstärkt;
\item der Leib des Menschen verlor das Vorrecht der Unsterblichkeit;
\item die Bebauung der Erde und die Herbeischaffung aller Bedürfnisse des Lebens verursachen den Menschen seit dieser Zeit nicht wenig Mühe und Beschwerlichkeiten;
\item in dieser ersten Sünde lag endlich auch schon die Veranlassung zu allen übrigen, deren das menschliche Geschlecht sich in der Folge schuldig gemacht hat.
\end{aufza}\par

\vabst \textbf{B.}~Aber nicht bloß diese erste Sünde Adams ist Ursache von den vielen und großen Uebeln auf Erden; sondern jene Sünden, welche das menschliche Geschlecht bis auf den heutigen Tag begangen hat, und fortfährt zu begehen, tragen mit bei, das Elend der Menschen zu vermehren, so daß wir überhaupt die vornehmste Ursache von allen Uebeln, die uns auf Erden drücken, in \RWbet{unseren Sünden} suchen und finden sollen.

\RWpar{188}{Historischer Beweis dieser Lehre}
Die Geschichte der ersten Sünde wird \RWbibel{Gen}{1\,Mos.}{3}{1\,ff}\ umständlich erzählt. In dieser Erzählung wird eine \RWbet{redende}~\RWSeitenw{32}\ \RWbet{Schlange als Verführer} angegeben. Daß aber ein böser Geist mittel- oder unmittelbarer Weise hier einen Einfluß gehabt, glaubte man in der katholischen Kirche von jeher. Ja, dieses Glaubens war schon der Verfasser des Buches der \RWbibel{Weish}{Weish.}{2}{24}: \RWbet{Durch den \RWbet{Neid des Teufels} ist der Tod in die Welt gekommen.} Und Jesus selbst sagt (\Ahat{\RWbibel{Joh}{Joh.}{8}{44}}{5,44.}) vom Teufel, wahrscheinlich nur in dieser Hinsicht, daß er ein \RWbet{Menschenmörder und Lügner vom Anbeginn} gewesen. In der Offenb. (\RWbibel{Offb}{}{20}{2}) wird der Teufel die \RWbet{alte Schlange} genannt; ohne Zweifel nur in Beziehung auf jene mosaische Geschichte.\par

\vabst \textbf{A.}~Daß die Sünde Adams als die vornehmste Ursache alles Uebels auf Erden angesehen werde, ergibt sich von selbst, sobald wir zeigen, daß ihr die oben angeführten Folgen zugeschrieben werden.
\begin{aufza}
\item Die \RWbet{Schwächung des Körpers}, die Schmerzen der Geburt \usw\ werden schon \RWbibel{Gen}{1\,Mos.}{3}{16}\ als Folge angegeben. Und das \RWlat{Concilium Tridentinum}\RWlit{}{Tridentinum1} sagt: \RWlat{Si quis non confiteatur, totum Adam per illam praevaricationis offensam \RWbet{secundum corpus et animam in deterius commutatum} fuisse, anathema sit. Sess.\,5. can.\,1.}
\item \RWbet{Erhöhter Trieb zur Sinnlichkeit}. \RWbibel{Gen}{1\,Mos.}{3}{7}\ wird erzählt, daß sich die ersten Menschen nach verübter Sünde ihrer Nacktheit geschämt hätten, eine Veränderung, welche sehr deutlich beweist, daß sie von nun an einen weit stärkeren Trieb zur Sinnlichkeit empfinden mußten, als vordem.
\item Die \RWbet{Sterblichkeit des Leibes} wird \RWbibel{Gen}{1\,Mos.}{3}{19}\ ausdrücklich als Folge angegeben. So spricht auch Paulus \RWbibel{Röm}{Röm.}{5}{12}: \erganf{Durch einen Menschen ist die Sünde in die Welt gekommen, und \RWbet{durch die Sünde der Tod}.} Und im Buche der \RWbibel{Weish}{Weish.}{2}{23}\ heißt es: \erganf{Gott hat den Menschen \RWbet{unsterblich erschaffen}, zu seinem Ebenbilde ihn gemacht; aber \RWbet{durch den Neid des Teufels kam der Tod in die Welt.}}
\item \RWbet{Mühsamere Bebauung der Erde}, \usw\ \RWbibel{Gen}{1\,Mos.}{3}{17}\ \erganf{Zu Adam sprach Gott: Weil du die Stimme~\RWSeitenw{33}\ deines Weibes gehört, und von der Frucht, welche ich dir verboten hatte, gegessen hast: so sey die Erde nun verflucht; in Arbeit und mit Mühe sollst du dein Brod von ihr gewinnen; Disteln und Dornen soll sie dir tragen, und im Schweiße deines Angesichtes sollst du dein Brod essen.}
\item Daß in jener Sünde der ersten Menschen schon die \RWbet{Veranlassung zu allen folgenden Sünden} gelegen sey, scheint der heil.\ Paulus andeuten zu wollen, wenn er (\RWbibel{Röm}{Röm.}{5}{12}) sagt, daß durch einen Menschen (nämlich Adam) die Sünde in die Welt eingegangen sey; denn aus dem Zusammenhange des Ganzen ist zu ersehen, daß er hier unter der Sünde nicht nur die sogenannte Erbsünde, sondern alle Sünden des menschlichen Geschlechtes überhaupt verstehe. Gewiß ist es, daß die Kirche diese Lehre von jeher vorgetragen hat.
\end{aufza}\par

\vabst \textbf{B.}~Daß aber auch \RWbet{alle übrigen Sünden der Menschen} das Ihrige mit beitragen zur Vermehrung des menschlichen Elendes, und daß sonach der Mensch die vornehmste Ursache all seines Elendes in \RWbet{seinen Sünden} suchen solle, das lehrt die Bibel an unzähligen Orten. Z.\,B.: \erganf{So wie die Sünden der Menschen sich vermehrten, so wurden auch ihre Lebenstage auf Erden kürzer.} (\RWbibel{Gen}{1\,Mos.}{6}{3}) \erganf{Als ihre Verbrechen noch immer größer wurden, vertilgte sie Gott sogar durch eine Sündfluth bis auf einen kleinen Ueberrest.} (\RWbibel{Gen}{1\,Mos.}{6}{6\,ff}) Im Buche der \RWbibel{Weish}{Weish.}{1}{12}\ heißt es: \erganf{Suchet doch nicht den Tod geflissentlich durch eure Vergehungen, und reißt das Verderben durch eure Thaten nicht mit Gewalt herbei; denn nicht Gott schuf den Tod, und er hat keine Freude am Untergange der Lebendigen. Er erschuf Alles zum Daseyn. Gesund ging das Menschengeschlecht aus seiner Hand hervor, kein zerstörendes Gift lag noch in ihnen, kein Reich der Todten sollte die Erde werden; denn die Tugend ist unsterblich. Aber die Gottlosen rufen den Tod herbei durch Wort und That; schmachtend halten sie ihn für ihren Freund, und machen mit ihm ein Bündniß, und sie verdienen es, seine Bundesgenossen zu werden.} \RWbibel{Sir}{Sirach}{40}{8}\ \erganf{Tod, Blutvergießen, Streit, Schwert, Unglück, Hunger, Schmerz und Krankheit treffen zwar alle lebendigen Geschöpfe vom Menschen bis zum Vieh, die Sünder aber siebenfach; denn \RWbet{nur um ihretwillen} sind diese Uebel eigentlich auf Erden}, \usw~\RWSeitenw{34}

\RWpar{189}{Vernunftmäßigkeit}
Wir untersuchen erst die Verträglichkeit aller hier aufgestellten Lehren mit den Grundsätzen der Vernunft, sodann mit der Geschichte.\par

\vabst \RWbet{\textbf{I.}~Verträglichkeit dieser Lehre mit der Vernunft.}\par\noindent
\vabst \textbf{A.}~Es liegt nichts Ungereimtes in der Behauptung, daß die erste und vorzüglichste Ursache von allem Uebel auf Erden in jener \RWbet{ersten Sünde} zu suchen sey, welche die noch unschuldigen Menschen im Paradiese begingen. Es ist für's Erste nichts Ungereimtes, daß ihnen Gott ein gewisses Gebot gegeben habe: denn dazu konnten ihn die vernünftigsten Beweggründe bestimmen. Die Handlung, welche er ihnen verboten, konnte an sich gewisse Nachtheile haben, \zB\ ihrer Gesundheit schädlich seyn. Gott konnte auch nebstdem noch manche sittliche Vortheile für das menschliche Geschlecht bei der Aufstellung eines solchen Gebotes beabsichtigen, und dadurch wirklich erreichen; \zB\ Uebung ihres Gehorsames, die Ueberzeugung von Gottes Wahrhaftigkeit, \usw\ Es liegt ferner nichts Unbegreifliches darin, daß die ersten Menschen dieses Gebot, nachdem sie es erst eine Zeit lang beobachtet hatten, am Ende übertraten; denn bei aller ihrer bisherigen Unschuld, bei aller noch so genauen Unterordnung ihrer Sinnlichkeit unter den Verstand, waren sie doch immer endliche, freie, und also fehlbare Geschöpfe. -- Und ihr Vergehen wird um so begreiflicher, wenn noch irgend eine Verführung von Außen durch Vermittlung eines bösen Geistes hinzukam; die Möglichkeit einer solchen Verführung aber haben wir oben gezeigt. Wenn endlich Alles, was die christliche Religion in den gleich folgenden fünf Puncten von dieser Sünde behauptet, seine Richtigkeit hat: so ergibt sich hieraus auch von selbst, daß man sie als die erste und vornehmste Ursache alles Uebels anzusehen habe.
\begin{aufza}
\item Durch diese erste Sünde soll nämlich der Körper nicht nur der ersten Menschen, sondern auch aller ihrer Nachkommen \RWbet{bedeutend geschwächt} worden seyn. -- Das Christenthum sagt nicht, ob dieses unmittel- oder mittelbar~\RWSeitenw{35}\ geschehen sey. Man könnte sich also vorstellen, daß die verbotene Handlung, welche nach dem buchstäblichen Sinne der mosaischen Urkunde in dem Genusse einer gewissen Baumfrucht bestand, jene Schwächung des Körpers der ersten Menschen, und dadurch auch ihrer Nachkommen, zu ihrer natürlichen Folge gehabt habe. So gibt es ja auch heut zu Tage verschiedene giftige Kräuter, deren Genuß schwächend und zerstörend auf den menschlichen Körper einwirket; und daß Eltern, die einen schwächlichen, mit gewissen Krankheiten behafteten Körper haben, diesen auch auf ihre Kinder fortpflanzen, erleben wir gleichfalls alle Tage. Freilich werden die meisten Krankheiten, welche wir kennen, von den Eltern auf ihre Kinder nur im verminderten Grade fortgepflanzt, dergestalt, daß nach einigen Geschlechtern (Generationen) die Krankheit beinahe schon verschwunden ist (\RWlat{natura tendit in rectum}); man muß also entweder annehmen, daß es mit jener Störung, die der Genuß dieser Baumfrucht in dem Leibe unserer ersten Eltern hervorgebracht, eine ganz eigenthümliche Beschaffenheit gehabt habe, was eben nichts Unmögliches wäre; oder man kann auch sagen, daß die vielen Thorheiten und Sünden, welche das menschliche Geschlecht noch in der Folge beging, und immer fortfährt zu begehen, Ursache davon waren, daß der menschliche Leib nie wieder bis zu seiner ursprünglichen Gesundheit und Vollkommenheit habe zurückkehren können. Da die Offenbarung lehrt, daß diese späteren Sünden, welche das menschliche Geschlecht noch in der Folge beging, alle mehr oder weniger schon in jener ersten gegründet, durch sie veranlaßt worden wären: so kann man überhaupt sagen, daß die Schwäche und Unvollkommenheit, die wir bis auf den heutigen Tag am menschlichen Leibe gewahren, eine (wenigstens mittelbare) Folge von jener ersten Sünde sey.
\item Und die \RWbet{sinnlichen Triebe des Menschen wurden zur Ungebühr verstärkt.} Da die Lebhaftigkeit und Stärke unserer sinnlichen Triebe vornehmlich von der Beschaffenheit unseres Leibes, von der Reizbarkeit unserer Nerven \usw\ abhängt: so ist dieß eben so, wie Nr.\,1.\ zu beurtheilen.
\item Wir wurden dem \RWbet{Leibe nach sterblich.} Dieß kann einerseits eine natürliche Folge von jener verbotenen~\RWSeitenw{36}\ That gewesen seyn, und von der andern Seite auch eine weise Verfügung Gottes; denn bei der Beschaffenheit, welche der menschliche Körper wenigstens jetzt hat, scheint es der Vervollkommnung des Menschen eher hinderlich als beförderlich zu seyn, wenn er auf dieser Erde sehr lange wohnen sollte.
\item Daß die Bebauung der Erde \usw\ dem Menschen von jetzt an \RWbet{beschwerlicher} geworden, war eine natürliche Folge von der Schwächung seines Leibes, und von der andern Seite auch wieder eine weise Verfügung Gottes zu unserem Besten; denn eben die Arbeiten und Beschwerlichkeiten, welche uns die Bebauung der Erde und die Herbeischaffung unserer Bedürfnisse verursacht, entwickeln unsere sämmtlichen Kräfte, veranlassen uns zu den verschiedenartigsten Erfindungen, \usw\
\item Die erste Sünde enthielt schon die Veranlassung zu allen folgenden in sich. Dieß ist aus einem doppelten Grunde wahr:
\begin{aufzb}
\item aus jenem allgemeinen, aus welchem jedes erste Verbrechen einer gewissen Art leicht mehrere nach sich zieht, \zB\ der erste Ungehorsam des Zöglings gegen seine Lehrer, die erste Wollustsünde, \usw\ Die Ursache liegt
\begin{aufzc}
\item darin, weil die erste Sünde uns das frohe Bewußtseyn unserer Unschuld raubt, dessen Besitz einer der stärksten Abhaltungsgründe vom Bösen ist, der für die Zukunft wegfällt.
\item weil die Verübung der ersten Sünde uns meistens mit gewissen (freilich nur sehr vergänglichen, aber für unsere Sinnlichkeit doch immer reizenden) Süßigkeiten bekannt macht, die wir vorher nicht kannten; \usw\
\end{aufzc}
\item Hier waltete noch der besondere Umstand ob, daß jene erste Sünde dem Körper der ersten Menschen für eine gewisse Zeit wenigstens eine erhöhte Reizbarkeit ertheilte. Bei dieser waren sie nun einer stärkeren Versuchung zu Sünden jeder Art ausgesetzt, als es vorhin der Fall gewesen; sie sündigten daher auch wirklich öfter. Aber durch eben diese Sünden vergrößerten sie die Reizbarkeit ihres Körpers noch immer mehr, oder verhinderten wenig\RWSeitenw{37}stens, daß er nicht wieder zu seiner vorigen Gesundheit gelangen konnte. Und da es eben so auch ihre Nachkommen machten: so kann man mit allem Rechte von jener ersten Sünde sagen, sie habe die erste Veranlassung zu allen folgenden gegeben. (\RWlat{Causa causae est etiam causa causati.})\par
\end{aufzb}
\end{aufza}
\RWbet{Einwurf.} Die Weisheit Gottes, welche die üblen Folgen jener verbotenen Handlung vorhersah, hätte sie nicht bloß verbieten, sondern ihre Ausführung dem menschlichen Geschlechte ganz unmöglich machen sollen; den giftigen Baum \zB\ hätte Gott gar nicht erschaffen, oder doch wenigstens ihn nicht im Paradiese sollen wachsen lassen.\par
\RWbet{Antwort.} 
\begin{aufza}
\item Um behaupten zu können, daß Gott ein gewisses Ereigniß nicht hätte zulassen sollen, müßte man zeigen, daß aus Verhinderung desselben nicht nur für den gegenwärtigen Fall, sondern auch für das Ganze mehr Gutes hervorgegangen wäre. Da nun alle Theile der Welt mit einander in der innigsten Verbindung stehen, da eine jede, uns auch noch so unbedeutend scheinende Veränderung, Folgen, welche sich in's Unendliche erstrecken, nach sich zieht: so ist es immer mißlich, so etwas behaupten zu wollen.
\item Doch je genauer wir dieß Ereigniß von allen Seiten betrachten, um desto deutlicher leuchten uns selbst die überwiegenden Vortheile ein, die seine Zulassung herbeigeführt hat. Wenn nicht die ganze Natur des Menschen auf eine uns wirklich undenkbare Weise anders eingerichtet seyn sollte: so war es unvermeidlich, daß das mit Freiheit ausgerüstete Menschengeschlecht über kurz oder lang in allerlei Sünden und Thorheiten verfalle; daß durch dergleichen Sünden, besonders durch Unmäßigkeit in Speise und Trank, durch Wollust, Zorn, Neid \udgl\  der Körper der Menschen mehr oder weniger zerrüttet werde; daß sich die körperliche Schwäche der Eltern auf ihre Kinder fortpflanze, kurz, daß am Ende Alles beiläufig in eben denselben Zustand, in dem wir uns gegenwärtig befinden, wo nicht in einen noch viel ärgeren gerathe. So würde sich also das menschliche Geschlecht in einem jeden Falle (wenn auch kein Baum der Erkenntniß je im Paradiese gestanden wäre) um seine ursprüngliche Voll\RWSeitenw{38}kommenheit gebracht haben; der menschliche Leib würde in Zukunft immer zu jener Schwäche, welche wir gegenwärtig fühlen, herabgesunken seyn; die Triebe der Sinnlichkeit wären immer so lebhaft, als sie es jetzt sind, geworden; Gott hätte es einst immer nöthig gefunden, den Menschen das Vorrecht leiblicher Unsterblichkeit zu nehmen, \usw\ Wenn nun Gott, statt diese üblen Folgen ganz unvermerkt eintreten zu lassen, dasjenige Verfahren wählte, welches das Christenthum beschreibt: so wurden doch wenigstens folgende sehr wesentliche Vortheile erzielet:
\begin{aufzb}
\item Die ersten Menschen erfuhren an einem Beispiele, daß Gott wahrhaftig sey;
\item daß seine Gebote zu ihrem eigenen Besten abzwecken, daß sie keines derselben übertreten könnten, ohne sich selbst zu schaden, und sich
\item noch obendrein die schwerste Strafe zuzuziehen.
\item Sie lernten die Sünde als die eigentliche Ursache aller Ungemächlichkeiten und Leiden, welche sie in der Folge erfuhren, ansehen, da im entgegengesetzten Falle ihr Leichtsinn vielleicht gar nicht beobachtet haben würde, woher es komme, daß sie nun mühseliger leben.
\item Selbst auf die Nachkommen der ersten Menschen bis in die spätesten Jahrhunderte wirkte Gott wohlthätig durch dieses Ereigniß ein; denn wirklich finden wir ja die warnende Geschichte des ersten Sündenfalles bei allen Völkern ausgebreitet. Allenthalben also dient sie, den Abscheu gegen die Sünde zu vermehren, und alle die Vortheile zu erzeugen, welche wir von dem Glauben an diese Geschichte in dem gleich folgenden Paragraph anführen werden.
\end{aufzb}
\end{aufza}\par

\vabst \textbf{B.}~Daß aber auch \RWbet{alle übrigen Sünden,} welche das menschliche Geschlecht begeht, zur Vermehrung des Uebels auf Erden beitragen, und daß die vornehmste Ursache von allem Elende der Menschen sonach in \RWbet{ihren} Sünden zu suchen sey, das findet die Vernunft vollkommen richtig. Alle Gebote, die uns Vernuft und Offenbarung aufstellen, haben zuletzt keinen anderen Zweck, als die Beförderung des allge\RWSeitenw{39}meinen Wohles (vornehmlich also des Wohlseyns unter uns Menschen); jede Uebertretung dieser Gebote also, oder jede Sünde, thut, der Regel nach, dem allgemeinen Wohle Abbruch, und vermehrt das gemeinschaftliche Elend. Die meisten Sünden schwächen noch überdieß die Gesundheit des Leibes, erhöhen die Reizbarkeit der Nerven, \usw , wodurch sie mittelbarer Weise die Heftigkeit unserer Begierden, die Stärke unserer Versuchungen, und folglich auch die Anzahl unserer Vergehungen selbst vermehren. Eben dieses geschieht auch bei allen Sünden gewissermaßen schon dadurch, daß sie die Gewohnheit, vom Sittengesetze abzuweichen, in uns herrschender machen; je herrschender aber diese wird, um desto leichter entschließen wir uns auch in der Folge wieder zur Uebertretung unserer Pflichten. Wenn man nach diesen Ansichten zu berechnen fortfährt, wie viele Uebel auf Erden eine bald nähere, bald entferntere Folge der menschlichen Sünden seyen: so muß man in der That gestehen, daß es die meisten sind, und daß selbst jene wenigen, welche das menschliche Geschlecht ohne sein eigenes Verschulden treffen, durch die menschlichen Laster doch sehr vergrößert werden.\par

\vabst \RWbet{\textbf{II.}~Verträglichkeit dieser Lehre mit der Geschichte.}\par\noindent%
Die Facta, die in dieser Lehre vorausgesetzt werden, und auch fast nothwendig vorausgesetzt werden müssen, wenn jene Lehre für uns begreiflich bleiben soll, sind wesentlich folgende drei:
\begin{aufza}
\item Das menschliche Geschlecht hat einen \RWbet{Anfang in der Zeit} genommen.
\item Es war im Anfange in gewissen Hinsichten \RWbet{vollkommener}, als es jetzt ist.
\item Es hat sich dieses vollkommeneren Zustandes \RWbet{durch Sünde} beraubt.
\end{aufza}
Diese drei Facta nun sind mit der Geschichte sehr wohl verträglich.
\begin{aufza}
\item Daß unser menschliches Geschlecht einen \RWbet{Anfang in der Zeit} genommen habe, ist eine Behauptung von der Art, daß ihr die Geschichte nicht einmal widersprechen kann.~\RWSeitenw{40}\ Daß das menschliche Geschlecht von Ewigkeit her da sey, könnte, selbst wenn es wahr wäre, auf keinen Fall einen Gegenstand der Erfahrung ausmachen, weil nichts Unendliches weder unmittelbar wahrgenommen, noch zur Erklärung des unmittelbar Wahrgenommenen als dessen Ursache vorausgesetzt werden kann. Im Gegentheile aber melden uns alle Geschichtschreiber, welche von diesem Gegenstande reden, daß das menschliche Geschlecht einen bestimmten Anfang in der Zeit genommen habe, ob sie gleich in der Angabe dieses merkwürdigen Zeitpunctes sehr abweichen.
\begin{RWanm}
Daß dieser Anfang des menschlichen Geschlechtes beiläufig sechs tausend Jahre von uns entfernt sey, ist eine Behauptung, die (wie man leicht sieht) gar nicht zur Religion gehört; denn welchen Einfluß sollte es wohl auf unsere Tugend und Glückseligkeit haben, ob dieser Zeitraum größer oder kleiner sey? -- Indessen läßt sich doch mit einer ziemlichen Gewißheit darthun, daß dieser Zeitraum nicht eben größer sey, als ihn die Bibel angibt. Die Fabeln nämlich, welche uns die \RWbet{Chaldäer} (bei Berosus), die \RWbet{Aegyptier} (bei Manethon), die \RWbet{Chineser} (in ihren Geschichtsbüchern) und andere Völker von ihrem weit höheren Alter erzählen, verdienen gar keine ernste Widerlegung. Sie sind offenbar nur aus Eitelkeit oder aus Mißverstand (Verwechslung astronomischer Perioden mit chronologischen Angaben) entstanden. Dagegen finden wir, wenn wir nur etwa vier tausend Jahre zurückgehen, das menschliche Geschlecht an allen Orten noch in einem solchen Zustande, als ob es nur erst seit Kurzem vorhanden wäre; wir finden es allenthalben noch in der gröbsten Unwissenheit, und erst so eben im Begriffe, die unentbehrlichsten und doch zugleich auch leichtesten Erfindungen zu machen. Das Einzige, was uns hier auffallen könnte, sind die hohen Kenntnisse, die einige sehr alte Völker, \zB\ die Chaldäer, schon so frühzeitig in der \RWbet{Astronomie} besaßen. (S.~Bailly: \RWlat{Histoire de l'Astronomie}.)\RWlit{}{Bailly1} Aber auch diese Erscheinung läßt sich erklären, ohne ein höheres Alter des Menschengeschlechtes anzunehmen. Der immer heitere Himmel in Chaldäa, die viele Muße, die jenen Völkern ihr Nomadenleben gewährte, \ua\,dgl. Umstände waren den astronomischen Entdeckungen sehr günstig.
\end{RWanm}
\item Das menschliche Geschlecht war in gewissen Hinsichten anfangs \RWbet{vollkommener}, als jetzt.~\RWSeitenw{41}\par 
Alle Geschichtschreiber, die von dem Anfange des menschlichen Geschlechtes reden, bestätigen dieses. Alle erzählen uns von einem gewissen \RWbet{goldenen Zeitalter}, von einem Zustande der Unschuld und der Glückseligkeit.
\item Beraubte sich aber dieser Vollkommenheiten \RWbet{durch Sünde}. Auch dieses melden uns alle alten Volkssagen und Geschichtschreiber. Die \RWbet{Indier} erzählen, daß die ersten Menschen, welche Riesen waren, sich dadurch unglücklich gemacht hätten, daß sie nach der \RWbet{Speise der Unsterblichkeit} gegraben. Auch die \RWbet{Tibetaner} lassen die ersten Menschen, die sie Lahen nennen, durch eine \RWbet{Missethat} fallen. Die \RWbet{Kalmücken} erzählen, in den ersten Zeiten, da die Menschen noch vollkommen heilig waren, habe die Erde ein honigsüßes \RWbet{Gewächs} hervorgebracht, das hätte ein gefräßiger Kalmücke gegessen, und darüber wäre alle Seligkeit zum Himmel geflogen. Im \RWbet{Zendavesta} (dem lebenden Worte) der \RWbet{Perser} heißt es gleichfalls, daß die Menschen Anfangs gut und unsterblich gewesen wären, bis sie von Ahriman sich hätten verführen lassen. Die \RWbet{Türken} haben dieselbe Erzählung vom Sündenfalle, wie wir. \usw\
\end{aufza}
\begin{RWanm} 
Worin eigentlich jene erste Sünde bestanden habe, ob in dem Genusse einer verbotenen Frucht, wie der buchstäbliche Sinn der Bibel angibt, oder sonst in einem oder mehreren Verbrechen, darüber ist man in der katholischen Kirche nicht ganz einig; indem mehrere Theologen älterer und neuerer Zeit jene Erzählung verschieden auslegen. Dieses ist auch von keiner so großen Wichtigkeit; denn selbst, wenn Jemand annehmen wollte, daß es mehrere Stammältern gegeben habe, könnte er das Wesentliche jener Glaubenslehre noch immer beibehalten, sobald er nur voraussetzen wollte, daß alle diese ersten Eltern früher oder später gesündiget, und sich dadurch unvollkommener, als sie es Anfangs waren, gemacht hätten. In der That aber hat man meines Erachtens gar keine Ursache, von der buchstäblichen Auslegung der Erzählung, welche uns Moses mittheilt, abzugehen; denn sie ist bis auf die kleinsten Umstände, die sie enthält, so durchaus vernunftmäßig und gotteswürdig, daß wir, wenn uns die Aufgabe gegeben würde, eine erdichtete Schöpfung des ersten Menschen und seines Sündenfalles zu erzählen, keine Dichtung ausdenken könnten, die eine wohlthätigere Wirkung auf das Ge\RWSeitenw{42}müth des Menschen zu äußern fähig wäre, als die mosaische. Wir wollen sie, um dieses einiger Maßen zu zeigen, in Kürze durchgehen.
\begin{aufza}
\item Nach dieser alten Urkunde schuf Gott den Menschen zuletzt, nachdem er die Erde bereits ganz für ihn vorbereitet hatte. Wer sollte dieß nicht zweckmäßig finden?
\item Gott schuf zuerst den Mann. -- Dadurch wird dem Manne eine gewisse höhere Würde eingeräumt, welches zur Erhaltung der Ruhe und Eintracht in der ehelichen Gesellschaft nothwendig ist.
\item Er bildete ihn aus Erde, und hauchte ihm etwas von seinem Geiste ein. -- Unmöglich kann man sich über die Entstehung des ersten Menschen erhabener ausdrücken. Der Mensch besteht also aus zwei Bestandtheilen, aus Einem, der Staub ist, und eben deßhalb auch zum Staube zurückkehrt; und Einem, der die innigste Verwandtschaft mit Gott selbst hat. Durch welche Mittelursachen hier Gott gewirkt habe, darüber schweigt die Urkunde billig, denn dieses könnten wir theils nicht fassen, theils dürfte es kaum geeignet seyn, uns einen so erhabenen Begriff von uns selbst beizubringen, als wenn wir uns bildlicher Weise vorstellen, daß Gott gleichsam mit eigener Hand an unserem Leibe gebaut habe!
\item Er gab dem Menschen einen schon ausgewachsenen, mannbaren Körper. Das Gegentheil wäre offenbar zweckwidrig gewesen, da es noch keine Wärter gab, welche den kindlich geschaffenen Menschen hätten warten und pflegen können.
\item Er setzte ihn in einen sehr anmuthigen Garten. Wie nothwendig für den ersten Menschen, der an den Früchten der Bäume hier schon genießbare und seiner Natur am ersten zusagende Nahrungsmittel finden konnte. Wäre es schon so frühzeitig gelehrt worden, sich von dem Fleische der Thiere zu nähren: so würde ihn dieß nur verwildert haben.
\item Gott selbst erschien dem ersten Menschen in einer edlen, menschenähnlichen Gestalt, und vertrat bei ihm die Stelle des Erziehers. -- War gleich der erste Mensch dem Leibe nach schon Mann, so war er am Geiste doch noch ein Kind. Wie nun alle Kinder, wenn sie nicht Schaden nehmen, und nicht verwildern sollen, eines Erziehers bedürfen: so bedurfte es auch einer besondern Leitung und Vorsorge von Seite Gottes, wenn nicht der erste Mensch, und mit ihm das ganze menschliche Geschlecht auf diese oder jene Art verunglücken sollte. Am Allerbesten also, wenn~\RWSeitenw{43}\ ihm Gott selbst in einer sichtbaren Gestalt erschien, ihn unterrichtete, \usw\
\item Der göttliche Erzieher machte den ersten Menschen mit den verschiedenen Pflanzen und Thieren, die ihn umgaben, bekannt, und verhalf ihm dazu, für jedes einen eigenen Namen zu erdenken. -- Der erste Unterricht eines Kindes soll immer darin bestehen, daß man dasselbe anleitet, seine Sinne zu gebrauchen, aufmerksam zu seyn auf die dasselbe zunächst umgebenden Gegenstände, ihre Eigenschaften, Kräfte und Wirkungen zu beobachten, nach ihren Namen zu fragen, oder wenn sie noch keinen haben, selbst zu versuchen, sie unter einen schicklichen Namen, \dh\ Begriff zu bringen.
\item Als der erste Mensch bemerkte, daß alle Thiere paarweise geschaffen wären, und daß nur er allein nicht seines Gleichen finde, ließ ihn Gott, wie es heißt, in einen tiefen Schlaf verfallen. In diesem Schlafe kam es ihm vor, daß Gott aus einer seiner Rippen eine ihm ähnliche Gestalt erschaffe; er wachte auf, und Eva stand vor ihm. -- Durch diesen Ursprung des Weibes von dem Manne ward von der Einen Seite die Abhängigkeit der Ersteren von dem Letzteren begründet, und von der andern Seite dem Manne die Pflicht, das Weib so wie sich selbst zu lieben, angedeutet. (Das ist nun Gebein von meinem Gebein, und Fleisch von meinem Fleische.)
\item Gott fuhr noch immer fort, diesem ersten Menschenpaare von Zeit zu Zeit zu erscheinen, und ihnen Maßregeln ihres Verhaltens zu geben. Und da sich unter den mancherlei Pflanzen des Paradieses auch eine giftige befand, welche dem äußeren Anscheine nach gleichwohl sehr schön und reizend war: so warnte Gott die ersten Menschen vor dem Genusse dieses Baumes. -- Durch dieses Verbot wurde das Nachdenken der ersten Menschen geweckt; sie mußten es sonderbar finden, daß gerade dieser so anmuthige Baum schädlich seyn sollte; sie lernten den äußern Schein von der Sache selbst unterscheiden. Zugleich erhielten sie einen Gegenstand zur Uebung ihrer sittlichen Kräfte.
\item Eine Schlange verführte die ersten Menschen, daß sie demungeachtet aßen. -- Die Schlange (darf man sich vorstellen) aß mit großer Begierde von den Früchten des Baumes, ohne daß ihr doch etwas Schlimmes widerfuhr. Eva, die dieß sah, ließ sich in eine Art von Gespräch mit der Schlange ein, wobei sie sich die Antworten selbst gab, wie Kinder dieß oft zu thun pflegen.~\RWSeitenw{44}
\item Eva aß zuerst, und gab dann auch ihrem Manne. Das weibliche Geschlecht (an diese Wahrheit soll dieser Vorfall erinnern) ist gewöhnlich das schwächere; und dem Manne liegt die Pflicht ob, mit Vernunft zu überlegen, ob er den Wünschen und Einfällen des Weibes willfahren dürfe.
\item Die genossene Frucht äußerte eine betäubende Wirkung auf die ersten Menschen, die sie umgebenden Gegenstände erschienen ihnen auf eine Zeit, wie in ganz neuer und veränderter Gestalt (ihre Augen wurden aufgethan).
\item Unter Anderm war ihnen auch die Blöße ihres Körpers jetzt so auffallend geworden, daß sie es nöthig fanden, sich zu bedecken. -- Das Bedürfniß der Kleidung, wird uns hier zu erwägen gegeben, ist nicht bloß ein physisches, sondern ein sittliches Bedürfniß, obgleich dieß letztere nur durch die Schuld unseres sittlichen Verderbens.
\item Noch vor dem Abende desselben Tages vernahmen Adam und Eva die Stimme ihres göttliche Erziehers, der den Garten heraufkam. Die Strafe folgt schnell auf das Verbrechen! --
\item Zum ersten Male fürchteten die Menschen sich vor seinem Angesichte, und verbargen sich hinter den Bäumen. -- So straft sich das böse Gewissen! --
\item Doch der Erzieher rief nun mit ernster Stimme: Adam! wo bist du? Hierauf erwiederte dieser: Herr! ich hörte wohl schon lange deine Stimme, aber ich scheute mich hervorzutreten, weil ich nackt bin! --\par
Der \RWbet{Erzieher}. Nackt? Wer sagte dir, daß du nackt wärest? Hast du etwa gegessen von dem Baume, welchen ich dir verboten hatte? -- Der Erzieher muß dem Zöglinge das Verdienst des offenen Geständnisses nicht rauben, ihm Anlaß geben, die Wahrheit selbst zu gestehen, aber auf eine solche Art, daß er ihm von der andern Seite auch keine Versuchung zur Lüge gibt.\par
\RWbet{Adam.} Das Weib, welches du mir zugesellt hast, gab mir, und ich aß. --\par
Der \RWbet{Erzieher} (zum Weibe). Warum hast du das gethan?\par
\RWbet{Eva.} Die Schlange verführte mich dazu.\par
Der göttliche Erzieher straft nun zuerst die Schlange, um dadurch den ersten Menschen zu beweisen, daß ihm ihr Unglück wirklich nahe gehe. Hierauf kündigt er sowohl dem Weibe, als auch dem Manne gewisse Strafen an, \zB\ die schmerzliche Geburt, die Mühseligkeiten der Arbeit, \usw\ -- Die Ankündigung~\RWSeitenw{45}\ dieser Strafen war deßhalb nothwendig, damit die ersten Menschen diese Uebel auch in der That als Folgen ihres Verbrechens ansehen möchten. --
\item Endlich vertrieb er sie auch sogar aus ihrem schönen Aufenthalte in eine rauhere Gegend. Dieses war vortheilhaft, um einen desto stärkeren Eindruck hervorzubringen, und den Unterschied zwischen ihrem vorigen und jetzigen Zustande desto auffallender zu machen. Auch wäre ihnen der Aufenthalt in dem Orte, wo sie gesündiget hatten, wegen der Rückerinnerung nur peinlicher geworden.
\end{aufza} 
\end{RWanm}

\RWpar{190}{Sittlicher Nutzen}
Durch diese Lehre erhalten wir
\begin{aufza}
\item für's Erste schon eine bestimmte und der Vernunft nicht widersprechende \RWbet{Auflösung jener Frage vom Ursprunge des Uebels}, die nicht nur für unsern Verstand, sondern auch für unser Herz äußerst beunruhigend war. Wir sind nun vor so manchen Verirrungen, in welche die Vernunft bei dem Versuche ihrer Auflösung fiel, gesichert.
\item Diese Lehre stellt uns die \RWbet{üblen Folgen der Sünde im stärksten Lichte} dar. So viele Uebel, spricht sie gleichsam zu uns, die euch auf Erden drücken, die überaus große Gebrechlichkeit und Hinfälligkeit eures Leibes, die vielen Beschwerden, Krankheiten und Leiden, denen ihr ausgesetzt seyd, ja selbst der Tod, der euch so fürchterlich ist -- alle diese Uebel sind nur die Folge von eueren und euerer Voreltern Sünden. Wäret ihr tugendhafter, oder würdet ihr auch nur von jetzt an euch befleißen, sittlich besser zu werden: so würden euch alle diese Uebel in eben dem Maße weniger drücken. Welch ein mächtiger Antrieb für alle Menschen, an ihrer sittlichen Verbesserung zu arbeiten! -- Alle Ungeduld, die uns bei unseren Leiden zuweilen anwandeln mag, aller Unwille: welch eine heilsame Richtung wird ihm durch diese Lehre ertheilt! Die Sünde ist es, über die wir unwillig seyn sollen; nur sie ist die Ursache von all diesem Uebel! --
\item Daß es insonderheit die \RWbet{erste Sünde} war, welche die größte Schuld an diesen Uebeln trägt, auch alle folgen\RWSeitenw{46}den Sünden gleichsam nach sich zieht, dieß lehrt uns in einem warnenden Beispiele, wie äußerst wichtig der erste Schritt zum Bösen sey, und wie mit ihm beinahe schon alle übrigen gethan sind. -- Und wenn eine einzige Sünde schon so viel Unheil nach sich ziehen kann: was für ein Uebel (müssen wir denken) ist doch die Sünde! --\par
\RWbet{Einwurf.} Diese Lehre kann aber auch den Nachtheil haben, daß wir unwillig über unsere Stammeltern werden, sie als die Ursache all unseres Elendes verfluchen! --\par
\RWbet{Antwort}. Bei einem wohl unterrichteten Christen kann dieser Fehler nicht eintreten; denn er weiß,
\begin{aufzb}
\item daß die ersten Eltern nicht ohne Verführung von Außen gefallen;
\item daß sie von Gott für ihre eigene Person strenge genug bestraft worden sind;
\item daß dieses jetzt herrschende Uebel nicht durch ihre Sünde allein, sondern auch durch die Sünden aller anderen Menschen so groß und drückend geworden sey;
\item daß alle diese Leiden, wenn wir sie standhaft tragen, zu unserer eigenen, nur um so größeren Vervollkommnung und Beglückung dienen, \usw\
\end{aufzb}
\item Der Umstand, daß jene erste Sünde, die so beweinenswürdige Folgen nach sich gezogen hat, in der \RWbet{Uebertretung eines ausdrücklichen Gebotes Gottes} bestanden habe, zeigt uns, wie gut es Gott bei der Aufstellung seiner Gebote mit uns meine, und welche verderbliche Folgen es allezeit nach sich ziehen werde, wenn wir den göttlichen Geboten ungehorsam sind.
\item Der Umstand, daß auch bei dieser ersten Sünde eine \RWbet{Verführung von Außen} mitgewirkt habe, mildert nicht nur, wie gesagt, unsern Unwillen gegen unsere guten Stammeltern, sondern vermehret auch andererseits unsern Abscheu gegen alle Sünden, die wir auf diese Art alle, näherer oder entfernterer Weise, als Satans Wirkung betrachten können.
\item Durch diese Lehre, verbunden mit jener von Gottes Erlösung, wird \RWbet{Gottes Güte und Barmherzigkeit gegen das menschliche Geschlecht in das helleste Licht}~\RWSeitenw{47}\ gestellt. Das menschliche Geschlecht hat sich durch seine Sünden, durch seinen Ungehorsam gegen Gott, so unglücklich gemacht; Er aber hilft demselben doch wieder auf.
\end{aufza}

\RWpar{191}{Wirklicher Nutzen}
Es ist schon oben angemerkt worden, in welche verderbliche Irrthümer die Frage vom Ursprunge des Uebels so viele Weltweise, und mittelbar durch sie auch ganze Länder und Völker gestürzt; diese Irrthümer hat nun das Christenthum nicht nur bei seinen eigenen Bekennern, sondern auch selbst bei Vielen, die sonst nicht alle Lehren desselben angenommen haben, verdränget.

\RWpar{192}{Die Lehre des Christenthums von der Erbsünde}
Nebst dem in der vorhergehenden Lehre über den Sündenfall des menschlichen Geschlechtes aufgestellten Sätzen, welche \RWbet{ganz eigentlich} zu verstehen waren, gibt es noch einige andere, die mehr nur eine \RWbet{bildliche} Bedeutung haben.
\begin{aufza}
\item Von jenem Augenblicke an, da die Stammeltern des menschlichen Geschlechtes in ihre erste Sünde einwilligten, \RWbet{verloren} nicht nur sie selbst, sondern auch alle ihre Nachkommen, so viele auf dem natürlichen Wege der Zeugung von ihnen abstammen, \RWbet{das Recht, sich Gottes reines und unbeflecktes Ebenbild zu nennen.}
\item Auf gleiche Weise verloren sie auch die \RWbet{Anwartschaft auf gewisse höhere Seligkeiten,} die ihnen zugedacht gewesen waren, wenn sie nie gesündiget hätten.
\item Ja, was noch mehr ist, \RWbet{das ganze menschliche Geschlecht wurde von nun an ein Gegenstand des göttlichen Mißfallens;} an jedem Menschen, der auf dem gewöhnlichen Wege der Zeugung von Adam abstammt, befindet sich schon von Natur etwas, welches das Mißfallen der Heiligkeit Gottes erregt, und bildlicher Weise die von Adam \RWbet{ererbte Sünde} oder \RWbet{Erbsünde} genannt wird (\RWlat{peccatum originale}). Diese Erbsünde ist nun zwar \RWbet{keine wirkliche Sünde,} sondern sie trägt diesen Namen nur gewisser Aehnlichkeiten wegen:~\RWSeitenw{48}
\begin{aufzb}
\item in sofern sie etwas ist, das seinen Ursprung in der Sünde hat (nämlich in Adam's Sünde);
\item in sofern sie etwas ist, das Gottes Mißfallen anregt. Uebrigens wird nicht bestimmt gesagt, wie hoch dieses Mißfallen Gottes steige; sondern nur diese beiden Grenzen desselben werden uns angegeben.
\begin{aufzc}
\item Es ist \RWbet{größer}, als daß Gott dem Menschen, der die Erbsünde noch an sich hat, jene höheren übernatürlichen Seligkeiten angedeihen lassen könnte, die er dem menschlichen Geschlechte, da es noch ohne Sünde war, zugedacht hatte;
\item es ist \RWbet{nicht so groß}, daß es den heil.\ Gott in der Behandlung der Menschen von jener Regel \Ahat{der}{die} Gerechtigkeit abbringen könnte, zufolge welcher er nur den Bösen bestraft, und jeden Tugendhaften glücklich macht.
\end{aufzc}
\end{aufzb}
\end{aufza}

\RWpar{193}{Historischer Beweis dieser Lehre}
\begin{aufza}
\item \RWbibel{Eph}{Ephes.}{4}{24}\ ermahnt uns der Apostel, das Ebenbild Gottes, nach welchem der erste Mensch erschaffen war, wieder an uns zu erneuern. Im B.~\RWbibel{Hiob}{Hiob}{14}{4}\ ruft Hiob aus: \erganf{Wer schafft vom Unreinen einen Reinen? Fürwahr nicht Einer!} Welches so viel sagen will: Da alle Menschen Sünder (Unreine) sind: so sind es auch alle ihre Nachkommen. Die 70 Dolmetscher übersetzen diese Stelle: \erganf{Denn wer wird rein seyn von der Unreinigkeit? Keiner, wenn er auch einen Tag nur auf der Erde lebte.} -- Wer nur einen Tag auf Erden gelebt, kann sich noch nicht durch eigene Sünden verunreiniget haben; was ihn verunreiniget, ist also die Sünde seiner Vorfahren, die Erbsünde ist es.
\item Daß die Menschen durch Adams Sündenfall die Anwartschaft auf jene höhern Seligkeiten verloren, \usw\ Die Kirche lehrt, wie wir es in der Folge sehen werden, daß die Erlösung durch Jesum Christum schlechterdings \RWbet{für alle Menschen nöthig} sey, um die Anwartschaft auf jene höhere Seligkeiten wieder zu erlangen; sie müssen dieselbe also verloren haben. \RWbibel{Joh}{Joh.}{3}{5}\ sagt Jesus selbst: \erganf{Wahrlich, ich sage dir: Wer nicht von Neuem geboren wird, kann das göttliche~\RWSeitenw{49}\ Reich nicht sehen.} Wenn also der Mensch ohne die Wiedergeburt durch das Bad der Taufe zur Seligkeit unfähig ist: so muß es wahr seyn, daß er die Anwartschaft auf dieselbe verloren hatte.
\item Das \RWlat{Concilium Tridentinum}\RWlit{}{Tridentinum1} stellt (\RWlat{sess. 5. de peccato originali}) unter andern folgenden Artikel auf: \RWlat{Si quis Adae praevaricationem sibi soli et \RWbet{non ejus propagini} asserit \RWbet{nocuisse}, et acceptam a Deo sanctitatem et justitiam, quam perdidit, sibi soli, et \RWbet{non nobis} etiam perdidisse, aut inquinatum illum per inobedientiae peccatum, mortem et poenas corporis \RWbet{in omne genus humanum} transfudisse, non autem et peccatum, quod mors est animae, anathema sit, cum contradicat Apostolo dicenti: Per unum hominem peccatum intravit in mundum, et per peccatum mors, et ita in omnes homines mors pertransiit, in quo omnes peccaverunt.}\par
Es ließe sich darüber streiten, ob die Stelle des Apostels (\RWbibel{Röm}{Röm.}{5}{12}), welche der Kirchenrath hier zum biblischen Beweise dieser Lehre anführt, das, was er aus ihr beweisen will, wirklich enthalte; allein, wie wir schon erinnert, die Unfehlbarkeit der Kirche erstreckt sich nur auf eigentliche Glaubenslehren, nicht aber auf so durchaus gleichgültige Dinge, als es \zB\ die Frage ist, ob der Apostel Paulus bei Niederschreibung jener Worte die Lehre von der Erbsünde ausdrücklich oder nicht ausdrücklich habe berühren wollen.\par
\RWbet{Einwurf.} Wie aber, wenn sich zeigen ließe, daß der Apostel Paulus und überhaupt die ersten Christen von keiner Erbsünde etwas gewußt? daß diese Lehre erst im fünften Jahrhunderte durch den heil.\ Augustinus aus Gelegenheit der Streitigkeiten, die zwischen ihm und Pelagius (einem Mönche in England) entstanden waren, eingeführt worden sey?\par
\RWbet{Antwort}. Wenn sich nicht zeigen ließe, daß in der früheren Kirche \RWbet{das gerade Gegentheil} behauptet worden sey: so hätte auch dieses noch nichts zu bedeuten, indem es allerdings geschehen kann (und sogar muß), daß einzelne Lehrsätze der Kirche erst in nachfolgenden Jahrhunderten deutlicher entwickelt werden. In der That aber läßt sich mit~\RWSeitenw{50}\ hinlänglicher Gewißheit zeigen, daß die Lehre von der Erbsünde schon in der ersten Kirche bekannt gewesen sey. Wir finden sie schon in der jüdischen Kirche; wenn also die Schriftsteller des neuen Bundes, geborne Juden, die noch dazu größtentheils an Juden-Christen schrieben, dieser Lehre nur nicht ausdrücklich widersprechen: so geben sie eben hiedurch zu verstehen, daß es mit ihr beim Alten bleiben solle. -- \RWbet{Justinus}, der doch vom Heidenthume zur christlichen Religion übergetreten war, schreibt (\RWlat{in dialog. cum Tryphone})\RWlit{}{Justinus5}, daß Christus geboren und gekreuziget worden sey, um des menschlichen Geschlechtes willen, welches durch Adam und durch den Betrug der Schlange dem Tode unterworfen worden, ohne an das zu denken, was Jeder aus eigener Schuld verübt. -- Er setzt also der Sünde Adams die \RWbet{persönliche} Sünde entgegen, und meint, daß schon bloß um dieser (Erb-) Sünde willen dem ganzen menschlichen Geschlechte eine Erlösung vom (ewigen) Tode nothwendig gewesen sey. -- \RWbet{Irenäus} (\RWlat{adv. haeres.\ 1.\ 3.\ cap.\,20.})\RWlit{}{Irenaeus1} behauptet, daß wir in Adam das Ebenbild Gottes verloren, daß wir durch Ungehorsam überwunden und der Sünde unterworfen worden wären. Auch \RWbet{Origenes} (\RWlat{contra Celsum lib.\,4.})\RWlit{}{Origenes1} behauptet, daß selbst die kleinen Kinder der Taufe bedürfen, weil sie gewisse Flecken hätten, die abgewaschen werden müßten. Nur verleitete ihn seine Liebe zur Speculation, die Ursache hievon beiläufig eben so, wie Plato, in einer Sünde zu suchen, welche die präexistirenden Seelen vor diesem Erdenleben schon begangen hätten. (\RWlat{Philocal. de Aleg. Leg.})\RWlit{}{Platon10} Präexistentianer waren überhaupt sehr viele Kirchenväter, und zwar am meisten die griechischen.
\item Daß aber die Kirche unter der Erbsünde \RWbet{nie eine wirkliche Sünde} verstanden, erhellt aus dem Gegensatze, den sie allgemein zwischen dem \RWlat{peccato originali} und \RWlat{personali} macht. -- Das Uebrige ergibt sich aus dem bereits schon Gesagten. Der letzte Punct aber, daß Gottes Mißfallen an der Erbsünde ihn zu keiner Abweichung von den allgemeinen Regeln der Gerechtigkeit verleite, wird sich weiter unten (bei der Lehre von der Seligkeit der Nichtchristen) noch deutlicher einsehen lassen.~\RWSeitenw{51}
\end{aufza}

\RWpar{194}{Vernunftmäßigkeit}
\begin{aufza}
\item Zwar könnten wir Gott nicht tadeln, wenn er uns das Recht, uns sein Ebenbild zu nennen, nie eingeräumt hätte. Nichts desto weniger, nachdem er für gut gefunden, uns dieses Recht einmal zuzugestehen: so darf er es ohne besondern Grund uns nicht wieder nehmen; denn ohne Grund darf Gott nie handeln. Aber hiebei ist es keineswegs nothwendig, daß wir diesen Grund immer einzusehen vermögen. So würde denn also die gegenwärtige Lehre mit der Vernunft in keinem Widerspruche stehen, auch wenn wir im Mindesten nicht absehen könnten, aus welchem Grunde Gott nach dem Sündenfalle dem Menschen das Recht, sich Gottes Ebenbild zu nennen, genommen habe. In der That aber können wir hier einen Grund erkennen; denn eben diese Zurücknahme eines uns vorhin zugestandenen Rechtes macht einen sehr wohlthätigen Eindruck auf unser Gemüth, den wir im folgenden Paragraph, bei der Betrachtung des sittlichen Nutzens dieser Lehre, angeben werden.
\item Eben dasselbe gilt auch von dem zweiten Punct, daß uns die Anwartschaft auf gewisse höhere Seligkeiten genommen worden sey. Auch hierüber können wir Gott nicht tadeln; es hat vielmehr auch dieses einen für uns einleuchtenden Nutzen.
\item Eben so wenig Anstößiges hat auch der dritte Punct. Ein Mißfallen in eigentlicher Bedeutung läßt sich bei Gott freilich nicht annehmen, so wenig, als ein eigentliches Wohlgefallen, wenn man darunter eine Affection (oder Veränderung) seines Empfindungsvermögens versteht; denn dieses wird durch die Dinge in der Welt nicht afficirt, sondern Gott befindet sich in dem Genusse einer ununterbrochenen von den Veränderungen in dieser Welt ganz unabhängigen Seligkeit. Aber so wie man gar keinen Anstand nimmt, bildlicher Weise, Gott ein Wohlgefallen an Etwas beizulegen: so kann man ihm zuweilen auch ein Mißfallen an Etwas beilegen. Aber, wendet man ein, ein Mißfallen am ganzen Menschengeschlechte wegen einer Sünde, die nur ein Einziger begangen, ist ein sehr unanständiges Bild, in welchem Gott als ein höchst~\RWSeitenw{52}\ leidenschaftliches und ungerechtes Wesen erscheint. Soll ja nicht einmal ein sittlich guter Mensch das Mißfallen, das ein Verbrechen der Eltern verdient, auf ihre Kinder übertragen! -- Ich antworte: Menschen dürfen dieß freilich nicht thun, weil sie nie völlig sicher seyn könnten, daß jene Gefühle des Mißfallens sie nicht etwa zu einer ungerechten That hinreißen würden. Aber bei Gott haben wir etwas von dieser Art nicht zu befürchten, indem dieses Mißfallen ein bloß bildliches ist. Ueberdieß versichert uns das Christenthum ausdrücklich, daß jenes Mißfallen Gottes keineswegs so weit geht, um ihn zu einer Abweichung von der allgemeinen Regel der Gerechtigkeit zu verleiten. -- Wenn ferner als Ursache dieses Mißfallens angegeben wird, daß wir uns Alle schon von Geburt aus mit einer gewissen Sünde (Erbsünde) befleckt fänden: so wäre dieß freilich sehr ungereimt, wenn unter dieser Sünde eine eigentliche Sünde (\dh\ eine persönliche mit Bewußtseyn und Freiheit unternommene Uebertretung des Sittengesetzes) verstanden würde; denn eine solche kann nicht vererbt werden. -- Und wollten wir, wie dieß einige christliche Schriftsteller, \zB\ Origenes und mehrere Präexistentianer wirklich gethan, annehmen, daß diese Erbsünde in einer gewissen noch vor unserer Geburt begangenen Sünde bestehe: so würden wir zwei unstatthafte Voraussetzungen machen;
\begin{aufzb}
\item daß wir vor unserer Geburt bereits nicht nur als lebendige Wesen, was allerdings angenommen werden dürfte, sondern auch schon als freie und der Sittlichkeit fähige Wesen, existirt haben. Wenn aber dieß gewesen wäre: so hätten wir billig die Rückerinnerung an unsern vormaligen Zustand nicht ganz verlieren sollen;
\item daß wir für ein Verbrechen bestraft werden, dessen wir uns nicht mehr erinnern.
\end{aufzb}
Allein wenn wir das Wort Sünde nur bildlich nehmen, und darunter nichts Anderes verstehen, als ein gewisses Etwas, das alle diejenigen, die auf natürlichem Wege der Zeugung von Adam abstammen, an sich haben, ein Etwas, das Gottes Mißfallen erregt, und wenn nicht selbst Sünde, doch eine Wirkung der Sünde ist: so enthält diese Behauptung nicht das mindeste Ungereimte. Wir können sogar eini\RWSeitenw{53}germaßen den letzten Grund des Daseyns dieser Erbsünde selbst begreifen. Die Sünde Adams nämlich hat theils an sich, theils durch die übrigen Sünden der Menschen, die man als eine fernere Wirkung von ihr betrachten kann, eine gewisse Zerrüttung in unserem Leibe hervorgebracht, welche die Ursache von vielen Sünden wird. Diese Zerrüttung in unserem Leibe, und die daraus entstehende stärkere Reizbarkeit unserer Nerven, sammt jenen wirklichen Sünden, zu denen wir uns durch sie verleiten lassen, dieß Alles, sage ich, macht eine gewisse \RWbet{strengere Behandlungsart von Seite Gottes} für uns höchst nöthig; Vergnügen und ununterbrochenes Wohlseyn sind jetzt gefährlich für uns; Unglück und Leiden sind uns jetzt eine Wohlthat; die Sünden, welche wir begehen, müssen jetzt härter bestraft werden; \usw\ Diese im Ganzen strengere Behandlungsart Gottes erfolgt gerade so, \RWbet{als ob er ein gewisses Mißfallen} an uns hätte. Die Offenbarung also legt ihm ein solches Mißfallen bildlicher Weise bei, und gibt als Grund desselben, gleichfalls nur bildlicher Weise, an, die Sünde Adams klebe noch an uns, jeder Gedanke Gottes an uns sey mit der Rückerinnerung an jene Sünde unserer Stammeltern verbunden.\par
\RWbet{Einwurf}. Allein wenn auf diese Art der letzte Grund der Erbsünde nur in der Zerrüttung unseres Körpers liegt: wie kann die Kirche lehren, daß die Getauften von der Erbsünde befreit werden, da sich im Organismus ihres Körpers doch nichts ändert? (\RWlat{Manere autem in baptizatis concupiscentiam vel fomitem, haec sancta synodus fatetur et sentit. Trident. sess. 5. de peccato originali.}) Wie kann die Wirkung aufgehoben werden, wo die Ursache bleibt?\par
\RWbet{Antwort}. Ohne daß die Ursache aufgehoben wird, kann oft die Wirkung aufgehoben werden, wenn eine Gegenkraft hinzukommt. Im gegenwärtigen Falle verhält sich die Sache so. Die Zerrüttung unseres Körpers macht nur im Allgemeinen, und nicht für jeden einzelnen Menschen eine strengere Behandlung nothwendig. Diejenigen, welche zur Kenntniß der Lehren des Christenthums gelangen, finden in eben diesen Lehren, in diesen besseren Begriffen von ihrer~\RWSeitenw{54}\ Bestimmung, von ihren Pflichten und von den Belohnungen, die ihrer harren, in jenen eigenthümlichen Hülfsmitteln zur Tugend, die das Christenthum darbeut, ein so kräftiges Gegengewicht gegen die Reize der Sinnlichkeit, daß sie auch ohne eine so strenge Behandlung von Seite Gottes zu erfahren, gerettet werden können. -- Das bloße Ereigniß, daß Jemand getauft wird, \dh\ als Mitglied in die christliche Gesellschaft aufgenommen wird, ist eine so große von Gott ihm erwiesene Wohlthat, daß man bei einem solchen von einem Mißfallen Gottes an ihm nicht ferner sprechen kann.
\end{aufza}

\RWpar{195}{Sittlicher Nutzen}
\begin{aufza}
\item Die Lehre, daß Adams Sünde nicht nur ihm selbst, sondern auch allen seinen Nachkommen das Recht, sich Gottes reines Ebenbild zu nennen, benommen habe, macht uns
\begin{aufzb}
\item für's Erste recht anschaulich die Häßlichkeit der Sünde; denn wie häßlich, sprechen wir nun zu uns selbst, muß doch die Sünde seyn, da nicht nur Adam, sondern das ganze menschliche Geschlecht von jenem Augenblicke an aufgehört hat, Gottes reines Ebenbild zu seyn, seitdem es sich mit einer einzigen Sünde befleckt hat! Wenn uns schon Adams Sünde, welche wir doch selbst nicht begangen haben, um unsern Adel, \dh\ um unsere Aehnlichkeit mit Gott, gebracht hat: wie sehr muß Gottes Ebenbild an uns nicht erst durch eigene Sünden entstellt werden!
\item Durch diese Lehre wird uns auch recht fühlbar gemacht, wie viel es auf sich habe, ein Ebenbild Gottes zu heißen und zu seyn! -- Unsere Ehrfurcht vor Gott, unser Begriff von seiner unendlichen Heiligkeit gewinnt durch den Gedanken, daß nicht nur Wesen, die selbst gesündiget haben, sondern auch solche, die nur aus sündigen Wesen entsprungen sind, kein Recht haben sollen, sich Ebenbilder jenes unendlich Heiligen zu nennen. --
\item Da uns nun gleichwohl gesagt wird, daß wir alle, die wir das Christenthum angenommen, uns durch das Bad der Taufe gereiniget, und keiner eigenen Sünde schuldig~\RWSeitenw{55}\ gemacht haben, in den Besitz dieses Rechtes wieder zurückgestellt worden sind: so lernen wir dieß Geschenk jetzt erst nach seinem ganzen Werthe schätzen, und uns desselben freuen.
\end{aufzb}
\item Ganz ähnliche Vortheile hat auch die zweite Lehre, daß auf uns allen, so Viele wir von Adam auf dem gewöhnlichen Wege der Zeugung abstammen, ein gewisses Mißfallen Gottes ruhe.
\item Der dritte Artikel der Lehre von der Erbsünde soll uns vornehmlich dazu dienen, daß wir nicht ungeduldig werden, und nicht wider Gott zu murren anfangen, wenn er auch uns noch jene üblen Folgen, welche sich Adam durch seine Sünde zugezogen hat, schmerzlich empfinden läßt! Haben wir ja, sprechen wir jetzt zu uns selbst, die Sünde Adams gleichsam geerbt; was Wunder also, wenn wir auch seine Strafe erben? --
\begin{RWanm} 
Vielleicht scheint es Jemand, daß die hier erwähnten Vortheile in einem noch höhern Grade Statt finden würden, wenn das Christenthum eben das, was es von Adams erster Sünde lehrt, auch von allen übrigen Sünden der Menschen lehren würde, \dh\ daß alle fortgeerbt werden. -- Aber bei einer näheren Betrachtung sieht man, daß eine solche Lehre überaus niederschlagend wäre. Denn welch eine ungeheure Last müßte uns nicht drücken, wenn wir die Sünden aller unserer Vorfahren zu tragen hätten! Es war nur nöthig, dieses von Einer, und am schicklichsten, es von der ersten Sünde zu lehren, um jene vorhin angeführten Vortheile zu erreichen. 
\end{RWanm}
\end{aufza}

\RWpar{196}{Wirklicher Nutzen}
Nebst jenen sittlichen Vortheilen, die wir im vorhergehenden Paragraph als möglich angegeben, und die in unzähligen Fällen gewiß in der Wirklichkeit Statt gefunden haben, hat diese Lehre auch einen bedeutenden \RWbet{scientifischen} Nutzen gehabt. Die Begriffe von Sünde, Freiheit, Schuld \usw\ sind durch ihre Veranlassung genauer entwickelt worden. -- Nur möchte man der Lehre von der Erbsünde vorwerfen, daß sie das Vorurtheil von der Vererblichkeit der Verdienste be\RWSeitenw{56}stärkt, sehr viele Streitigkeiten verursacht, und Manche vom Christenthume ganz abwendig gemacht habe, weil diese Lehre ihnen widersinnig vorkam. Ohne Zweifel aber wird der Nutzen dieser Lehren den Schaden, wo nicht schon längst überwogen haben, sicher in Zukunft noch überwiegen; denn des Nutzens wird sie in Zukunft immer mehr, des Schadens immer weniger bringen.

\RWabs{Vierter Abschnitt}{Von den Verhältnissen Gottes zu uns Menschen}
\RWpar{197}{Inhalt dieses Abschnittes}
Nachdem uns das Christenthum von Gottes Daseyn und seinen Eigenschaften; von Gottes Werken und von dem Zwecke, zu dem, und von der Art, wie er sie zu demselben leite; von uns selbst, und von der wahren Ursache, weßhalb unsere Glückseligkeit keine vollkommenere ist, unterrichtet hat, übergeht es dazu, uns mit den \RWbet{Anstalten, die Gott zur Vervollkommnung unserer Glückseligkeit getroffen} hat, bekannt zu machen. -- Es fängt mit der Versicherung an, daß Gott die \RWbet{väterlichsten Gesinnungen gegen uns} hege; erzählt uns dann weiter, daß \RWbet{Gottes zweite Person sich mit dem Menschen Jesu auf's Innigste vereinigt} habe, um ihn fähig zu machen, uns die beseligendste Religion zu lehren, und durch das Leiden und den Tod, welchen er unverschuldeter Weise litt, von unserer eigenen Schuld zu befreien. Es eröffnet uns endlich, daß \RWbet{Gottes dritte Person unaufhörlich zu unserer Heiligung und Beseligung wirke}.

\RWpar{198}{Die Lehre des Christenthums von Gottes Vatersinne}
Das Christenthum will, daß wir uns Gottes Verhältniß zu uns Menschen bildlicher Weise als das Verhältniß eines~\RWSeitenw{57}\ \RWbet{Vaters zu seinen Kindern} vorstellen sollen. Gott ist euer Vater, sagt es, und zwar euer Vater in einem weit höheren Sinne, als ihr es begreifen möget; denn jene göttliche Person in Gott, welche der Vater heißt, hat ihren gleichgöttlichen Sohn gesandt, ihn menschliche Natur an sich nehmen, und folglich euren Bruder werden lassen. Und eben darum seyd auch ihr selbst, als Brüder dessen, der Gottes Sohn in einem ganz ausnehmenden Verstande heißet, Kinder Gottes in sehr erhabener Bedeutung geworden.

\RWpar{199}{Historischer Beweis dieser Lehre}
\begin{aufza}
\item Erfreulich ist es, die erhabene Vorstellung von Gottes väterlichen Verhältnissen zu uns hie und da schon in den Büchern des alten Bundes, besonders in den spätern, anzutreffen. So wird \zB\ \RWbibel{2\,Sam}{2\,Sam.}{7}{14} dem Salomo versprochen, daß Gott ihn wie einen Sohn behandeln werde: \erganf{Ich werde sein Vater, und er wird mein Sohn seyn.} An diese Verheißung wird Gott \Ahat{\RWbibel{Ps}{Ps.}{89}{27}}{88,27.} erinnert: \erganf{Er wird mich anrufen: Du bist mein Vater! Ich aber werde ihn zu meinem Erstgebornen erheben.} -- Die Propheten erhoben sich einigemal zu dieser Vorstellung; \zB\ \RWbibel{Jes}{Isai.}{63}{16}: \erganf{Jehova, du bist unser Vater, unser Retter! Dein Name ist von Ewigkeit.} Eben so \RWbibel{Jes}{}{64}{8} -- \RWbibel{Jer}{Jerem.}{3}{19} -- \RWbibel{Mal}{Malach.}{1}{6} \erganf{Der Sohn ehret seinen Vater, seinen Herrn der Knecht; bin ich Vater: wo ist die Ehre? bin ich Herr: wo ist der Gehorsam? spricht der Ewige zu euch, ihr Priester, die ihr meinen Namen entweihet!} -- Und \RWbibel{Mal}{}{2}{10}: \erganf{Haben wir nicht Alle Einen Vater? schuf uns nicht Ein Gott Alle?} -- Besonders aber war diese Vorstellung dem Verfasser des Buches der Weisheit (der zur Zeit der \Ahat{Makkabäer}{Machabäer} gelebt haben mag) geläufig. \RWbibel{Weish}{Weish.}{2}{16} läßt er den Gottlosen von dem Gerechten sagen: \erganf{Er rühmt sich, Gott zum Vater zu haben: lasset uns sehen, ob sein Vorgeben wahr sey; erfahren, was für ein Ende es mit ihm nehmen werde. Ist der Gerechte Gottes Sohn: so wird er ihm helfen, und von der Gewalt seiner Feinde ihn befreien.} -- \RWbibel{Weish}{Weish.}{11}{10} \erganf{Diese (die Israeliten) prüftest du wie ein warnender Vater; jene (die~\RWSeitenw{58}\ Aegypter) foltertest und verurtheiltest du wie ein strenger König.} \RWbibel{Weish}{Weish.}{14}{3} \erganf{Deine Vorsehung, o Vater! regiert das Schiff in den Fluthen.} -- Jesus Sirach (der unter Antiochus Epiphanes gelebt) gebraucht ähnliche Ausdrücke \RWbibel{Sir}{}{23}{1}: \erganf{Gott, Vater und Herr des Lebens! gib mich ihren Rathschlägen nicht Preis!} \RWbibel{Sir}{}{23}{4}: \erganf{Gott, Vater! Gott meines Lebens! laß mich auf Andere nicht stolz herabschauen.} -- Zu den Zeiten unseres Herrn Jesu selbst war es \Ahat{nichts}{nicht} Ungewöhnliches, Gott den Vater des israelitischen Volkes zu nennen; \RWbibel{Joh}{Joh.}{8}{41} sagen die Juden selbst zu Jesu: \erganf{Wir haben Gott zum Vater.}
\item Aber erst Jesus war es, der die Menschen belehrte, daß nicht nur einzelne Menschen und Völker, sondern daß alle Menschen ohne Ausnahme Gott als ihren Vater zu betrachten die Erlaubniß hätten. Bekanntlich pflegte er Gott beinahe nie anders, als den Vater im Himmel zu nennen. Z.\,B.\ \RWbibel{Mt}{Matth.}{5}{44}: \erganf{Liebet eure Feinde, thuet Gutes denen, die euch hassen, und betet für eure Verfolger und Beleidiger, damit ihr Kinder eures Vaters im Himmel seyd, der seine Sonne aufgehen läßt über Gute und Böse, und regnen über Fromme und Lasterhafte.} -- Er findet es sogar unanständig, wenn man gewissen Menschen den Namen Vater als einen Ehrennamen beilegt, da wir nur Einen Vater im Himmel oben hätten (\RWbibel{Mt}{Matth.}{23}{9}): \erganf{Ihr sollet Keinen von euch auf Erden Vater nennen; denn Einer ist euer Vater, der im Himmel ist.} So auch \RWbibel{Mt}{Matth.}{5}{16}\ \RWbibel{Mt}{}{6}{1}\ \RWbibel{Mt}{}{6}{4}\ \RWbibel{Mt}{}{6}{18}\ \RWbibel{Mt}{}{10}{20}\ \RWbibel{Mt}{}{10}{29}\ \RWbibel{Lk}{Luk.}{6}{36}\ \RWbibel{Lk}{}{12}{30} \uam\ 
\item Aber noch nicht genug, Jesus gab zu verstehen, und die katholische Kirche hat es von jeher gelehrt, daß selbst jene höhere Verwandtschaft, in welcher Jesus mit Gott gestanden, zum Theile auch auf uns, als seine Brüder über[ge]gangen sey. Denn \RWbibel{Joh}{Joh.}{20}{17} spricht er nach seiner Auferstehung zu Maria: \erganf{Sage meinen Brüdern: Ich fahre auf zu meinem Vater und zu eurem Vater.} Worte, durch die er uns Menschen bildlicher Weise zu eben dem Verhältnisse mit Gott hinan hebet, in welchem er selbst sich befindet. Dieß erlaubt sich denn auch der heil.\ Paulus \RWbibel{Röm}{Röm.}{8}{14}: \erganf{Alle, die durch den Geist Gottes sich leiten lassen, sind Gottes Kinder. Ihr~\RWSeitenw{59}\ habet nicht wieder einen knechtischen Geist der Furchtsamkeit empfangen; sondern einen kindlichen Geist, in welchem wir Abba, Vater!\ rufen. Dieser Geist gibt unserem Geiste die Ueberzeugung, daß wir Kinder Gottes sind. Sind wir aber Kinder: so sind wir auch Erben, Erben Gottes und Miterben Jesu Christi.} Also beiläufig eben so, wie Jesus Gottes Sohn, und darum Erbe Gottes ist, sind auch wir Gottes Kinder, und darum Erben Gottes. So auch \RWbibel{Röm}{Röm.}{1}{7}\ \RWbibel{2\,Kor}{2\,Kor.}{6}{18}\ \RWbibel{Eph}{Ephes.}{4}{6}\ \RWbibel{Gal}{Gal.}{4}{4}\ Eben so \RWbibel{Jak}{Jak.}{1}{27}\ \RWbibel{1\,Petr}{1\,Petr.}{1}{17}\ \RWbibel{1\,Joh}{1\,Joh.}{3}{1} \usw\
\end{aufza}

\RWpar{200}{Vernunftmäßigkeit}
\begin{aufza}
\item Da diese Lehre bloß bildlich zu verstehen ist: so ist ihre Vernunftmäßigkeit erwiesen, sobald man nur zeigen kann, daß
\begin{aufzb}
\item eine gewisse \RWbet{Aehnlichkeit} zwischen dem Bilde und der Sache selbst Statt finde; und
\item daß es \RWbet{kein schicklicheres} Bild, als dieses, gebe.
\end{aufzb}
\item Das letztere wird im folgenden Paragraph bei der Betrachtung des sittlichen Nutzens dieser Lehre erwiesen werden. Also nur von dem Ersteren. -- Man kann allerdings das Verhältniß Gottes zu uns Menschen mit dem Verhältnisse eines Vaters zu seinen Kindern vergleichen. Die Aehnlichkeit liegt nämlich hier freilich nicht in gewissen sinnlichen Neigungen und Wünschen, in einer gewissen sinnlichen Liebe zu uns, \udgl ; denn es gibt keine Neigungen und Wünsche bei Gott. Die Aehnlichkeit liegt vielmehr in dem Betragen Gottes zu uns. Gott beträgt sich auf eine solche Art gegen uns, daß unter allen uns bekannten Betragungsarten diejenige, welche ein Vater, ein zärtlicher Vater, gegen seine Kinder beobachtet, die meiste Aehnlichkeit mit dem Betragen Gottes gegen uns hat. Die Wahrheit dieser Behauptung ist sehr einleuchtend. Gott nämlich sucht unsere Tugend und Glückseligkeit, so viel er nur immer vermag, zu befördern; und was er vermag, ist überaus viel, ist eigentlich Alles, was hier nur überhaupt von einem Andern geschehen kann. Eben so suchen nun auch gute Väter die Tugend und Glück\RWSeitenw{60}seligkeit ihrer Kinder, so viel sie nur immer können, zu befördern, und das, was sie vermögen, ist der Regel nach weit mehr, als jeder andere Mensch vermag. Sie sind diejenigen, welche dem Kinde das Daseyn gegeben haben, welche ihm Nahrung und Kleidung reichen, seine Erziehung besorgen, \usw\
\end{aufza}

\RWpar{201}{Sittlicher Nutzen}
Wenn der Gedanke an Gott recht brauchbar für uns werden soll: so ist es nothwendig, daß wir entweder Gott selbst, oder doch sein Verhalten gegen uns unter irgend einem Bilde uns vorstellen; denn es ist ausgemacht, daß nur sinnliche Bilder lebhafte und mit Empfindung begleitete Vorstellungen in uns erzeugen. Nun ist es offenbar, daß unter allen sinnlichen Gegenständen, womit wir Gott und seine Verhältnisse zu uns vergleichen könnten, keines schicklicher ist, als das Bild eines Vaters. Die Vorstellung Vater ist doch bei allen Menschen mit den lebhaftesten und für Gott schicklichsten Gefühlen und Empfindungen begleitet. Alle, oder doch fast alle Menschen empfinden bei dem Gedanken Vater Gefühle der Liebe, der Ehrfurcht und des Vertrauens. Und gerade diese Gefühle sind die geziemendsten, die wir in Rücksicht auf Gott in uns unterhalten sollen. Wir sollen Liebe zu ihm in unserem Herzen erwecken, weil uns durch diese Liebe die Erfüllung seiner Gebote beträchtlich erleichtert wird; weil wir dann mit den Schicksalen, die er uns zuschickt, zufriedener werden; und uns alles des Guten, das er uns schenkt, dann um so inniger freuen. Wir sollen Ehrfurcht vor ihm hegen, um dort, wo es schwer wird, ihm zu gehorchen, dieses zum Wenigsten aus einer Art heiliger Furcht vor ihm zu thun. Wir sollen endlich Vertrauen zu ihm fassen, damit wir nie zweifeln, daß er in Allem, was er von uns verlangt, und was er uns erfahren läßt, nur unser Bestes beabsichtige.
\begin{RWanm} 
Man wird die Schicklichkeit dieses vom Christenthume gewählten Bildes Gottes als eines Vaters noch deutlicher einsehen, wenn man die Versuche, Gott unter einem andern sinnlichen Bilde vorzustellen, vergleicht. Wie unvollkommen waren~\RWSeitenw{61}\ \zB\ die Bilder, unter welchen einige heidnische Völker, und selbst ihre Weltweisen, Gott vorstellig machten, wenn sie bald die \RWbet{Sonne}, bald das \RWbet{Feuer,} bald die \RWbet{Luft}, bald irgend ein \RWbet{Thier} als Sinnbild der Gottheit angaben! Alle diese Gegenstände haben ja viel zu wenig Würde und Aehnlichkeit, um mit Gott verglichen zu werden. Das Bild eines \RWbet{Herrn} und \RWbet{Königs}, dessen sich selbst die Offenbarung des alten Bundes bediente, ist zwar geschickt, Ehrfurcht, aber nicht Zutrauen und Liebe zu erwecken. Für jenen unvollkommenen Zustand der Sittlichkeit, in welchem sich das israelitische Volk zu den Zeiten Mosis befand, für eine Zeit, wo dieses Volk durch Furcht allein geleitet werden konnte, mag jenes Bild sehr zweckmäßig gewesen seyn; und selbst das Christenthum verschmäht es nicht, davon Gebrauch zu machen, so lange der noch unbußfertige Sinn eines Christen es nothwendig macht, daß er an Gottes Strafgerechtigkeit mehr, als an seine Güte denke. In dem Gemüthe des schon Gebesserten aber soll nicht der traurige Gedanke an die Strafgerechtigkeit, sondern der freudige an die Güte Gottes vorwalten. Das Bild eines \RWbet{Freundes} kann zwar Liebe, aber nicht genug Ehrfurcht und Zutrauen begründen, \usw\ Nur zu bewundern ist es daher, wie neuerlich \RWbet{Link} (in seiner: Natur und Philosophie, Leipzig, 1811.)\RWlit{}{Link1} die Vorstellung Gottes als eines \RWbet{väterlich gütigen Regierers}, der mit Sanftmuth herrscht, zur Schwärmerei habe zählen können. 
\end{RWanm}

\RWpar{202}{Wirklicher Nutzen}
Die herrliche Lehre von Gottes Vatersinne hat ohne Zweifel zur Beseligung von vielen Tausenden gedient. Außerhalb des Christenthums war diese Lehre so gut als unbekannt; denn wenn gleich die schon oben angeführten Stellen in den Büchern des a.\,B.\ die Israeliten auf diesen Begriff von Gott hätten hinleiten können; wenn auch selbst unter den Heiden zuweilen ein und der Andere sich zu diesem Begriffe erhoben, wie denn \zB\ \RWbet{Plato} Gott einmal ausdrücklich den Vater des Weltalls nennt; wenn wir auch finden, daß die alten Deutschen Gott den schönen Namen des Allvaters gaben: so waren dieß doch bloße Worte und Namen, deren Sinn höchstens irgend ein Einziger begriff, ohne daß von der großen Menge der Menschen je hätte gesagt werden können,~\RWSeitenw{62}\ sie denke sich Gott wirklich als ihren Vater so, wie wir Christen dieß thun.

\RWpar{203}{Die Lehre des Christenthums von der Menschwerdung des Sohnes Gottes}
Um das menschliche Geschlecht aus jenem Zustande des Verderbens, in den es sich durch seine eigene Schuld gestürzt hatte, wieder herauszureißen, und es zu jenem möglichst höchsten Grade der Glückseligkeit, den es theils hier auf Erden, theils noch in anderen Aufenthaltsorten erreichen kann, zu erheben, traf der allgütige Gott, wie uns das Christenthum versichert, mehrere Anstalten; deren wichtigste die der \RWbet{Menschwerdung des Sohnes Gottes} ist.\par
Die christkatholische Lehre von dieser Menschwerdung lautet nun mit allen ihren Nebenbestimmungen so:
\begin{aufza} 
\item \RWbet{Auf den Willen des Vaters in Gott beschloß der Sohn}, der mit ihm gleiches Wesens ist, im Verfolge der Zeiten, \RWbet{menschliche Natur an sich zu nehmen.}
\item Zu diesem Ende ward vor etwa achtzehnhundert Jahren eine \RWbet{Jungfrau,} mit Namen \RWbet{Maria, ohne die Mitwirkung eines Mannes}, bloß durch die Kraft des Geistes Gottes mit einem Kinde schwanger, welches den Namen \RWbet{Jesus} (\dh\ Erlöser) erhielt.
\item Dieß Kind war \RWbet{ohne Erbsünde} empfangen.
\item Es war \RWbet{ein wirklicher, nicht ein bloß scheinbarer Mensch,} der einen \RWbet{wirklichen Leib} und eine \RWbet{wirkliche Seele} hatte.
\item Mit diesem Menschen war \RWbet{die zweite göttliche Person} von seines Daseyns erstem Augenblicke an (also schon in dem Schooße seiner jungfräulichen Mutter) auf's Innigste \RWbet{vereiniget.}
\item Diese Vereinigung war eine \RWbet{Vereinigung zweier Naturen}, nämlich der göttlichen (des Sohnes) und der menschlichen (des Menschen Jesu) \RWbet{zu einer einzigen Person}, welche die Namen des \RWbet{Gottmenschen,} des \RWbet{Erlösers, Messias, Christus} trägt.~\RWSeitenw{63}
\item Diese Vereinigung war also \RWbet{keine Vermengung oder Vermischung beider Naturen} zu einer dritten, die weder menschliche noch göttliche Eigenschaften an sich gehabt hätte.
\item In dem Gottmenschen befindet sich vielmehr:
\begin{aufzb}
\item ein \RWbet{doppelter Verstand}, nämlich ein göttlicher, jener des Sohnes, und ein menschlicher, jener des Menschen Jesu; und dieser letztere erfährt durch die Erleuchtung des ersteren so viel, als nur ein menschlicher Verstand an sich zu fassen vermag, und als insonderheit Jesus zur Ausführung seiner erhabenen Zwecke, \zB\ als Lehrer der vollkommensten Religion, zu wissen bedurfte.
\item Ein \RWbet{doppelter Wille,} nämlich ein göttlicher, jener des Sohnes, und ein menschlicher, des Menschen Jesu. Dieser letztere stimmte aber mit jenem ersteren immer so vollkommen überein, daß Jesus der einzige Mensch ohne alle Sünde geblieben.
\end{aufzb}
\item Auch von dem Gottmenschen kann man sagen, daß er (nämlich als Mensch, oder von Seite seiner menschlichen Natur) \RWbet{menschliche Pflichten} und Obliegenheiten gehabt, \RWbet{Lust und Schmerz} empfunden, selbst sich \RWbet{Versuchungen ausgesetzt} gefühlt habe, dem Leibe nach \RWbet{sterblich} gewesen und wirklich \RWbet{gestorben} sey.
\item Von der andern Seite kann man von diesem Gottmenschen auch sagen, daß er (nämlich als Gottes Sohn) \RWbet{allmächtig, allwissend,} \usw\ sey.
\item Die Vereinigung, die zwischen dem Sohne Gottes und dem Menschen Jesu vom ersten Augenblicke der Entstehung des Letzteren angefangen hat, \RWbet{währt seitdem immer fort, und wird nie wieder aufgelöst werden.}
\item Man kann sich der Redensarten bedienen, \RWbet{Gott sey Mensch geworden, Gott habe für uns gelitten und sey am Kreuze gestorben,} \udgl\  Nicht aber soll man sprechen: Ein Mensch sey Gott geworden; oder: die Gottheit sey Mensch geworden, gestorben \udgl~\RWSeitenw{64}
\end{aufza}

\RWpar{204}{Historischer Beweis dieser Lehre}
\begin{aufza}
\item Bei \RWbibel{Joh}{Joh.}{3}{16} sagt Jesus selbst von Gott: \erganf{So sehr \RWbet{hat Gott die Welt geliebt}, daß er seinen eingebornen Sohn \RWbet{hingab}, damit Jeder, der an ihn glaubt, das ewige Leben habe.} -- Und bei \RWbibel{Joh}{Joh.}{17}{4} spricht er in jenem letzten feierlichen Gebete: \erganf{Ich habe auf Erden dich verherrlichet, und das Geschäft vollendet, \RWbet{das du mir zu vollziehen aufgetragen} hast.}
\item Die übernatürliche Geburt Jesu aus einer Jungfrau wird \RWbibel{Mt}{Matth.}{1}{18} und \RWbibel{Lk}{Luk.}{1}{26} erzählt, und wurde überdieß von der Kirche allgemein geglaubt. -- Bei \RWbibel{Lk}{Luk.}{1}{35} spricht der Engel: \erganf{Der heilige Geist wird über dich kommen, und die Allmacht Gottes wird dich überschatten; daher wird auch das Heilige, das aus dir wird geboren werden, Sohn Gottes heißen.}
\item Schon aus dem Ausdrucke, \RWbet{das Heilige}, dessen der Engel sich bediente, wurde gefolgert, daß Jesus keine Erbsünde an sich gehabt habe, weil es sich widerspricht, was \RWbet{Etwas Heiliges} genannt wird, zugleich auch ein Gegenstand von Gottes Mißfallen zu nennen. Eben so sagt auch Paulus von Jesu (\RWbibel{Hebr}{Hebr.}{7}{26}), daß er heilig, schuldlos, ohne Fehler, nicht aus der Zahl der Sünder, und höher als der Himmel war.
\item Die Kirche hat diejenigen, welche behauptet, daß Christus nur einen \RWbet{scheinbaren} Leib gehabt habe, die sogenannten \RWbet{Doketen} oder \RWbet{Phantasiasten} (welcher thörichten Meinung schon \RWbet{Simon Magus}, später \RWbet{Menander, Basilides, Marcion}, im dritten Jahrhunderte die \RWbet{Manichäer,} dann die \RWbet{Priscillianisten} \umA\  zugethan waren) als Irrlehrer verdammt. Eben so auch diejenigen, welche Christo zwar einen menschlichen Leib beilegten, aber die menschliche Seele ihm absprachen, weil, wie sie glaubten, die Stelle dieser der \RWbet{Logos} vertreten habe. (Dergleichen \RWgriech{>'ayuqoi} waren \RWbet{Arius}, in neuerer Zeit der Engländer \RWbet{Whiston}, \uA ) Daß die Verfasser der heil.\ Schriften von diesem Irrwahne entfernt gewesen, kann man aus vielen Stellen erweisen. Der Evangelist Johannes sagt offenbar vor\RWSeitenw{65}aus, daß der Herr Jesus ein wirklicher Mensch gewesen sey, wenn er \RWbibel{Joh}{}{1}{14} schreibt: \erganf{Das \RWbet{Wort wurde Mensch} und \RWbet{wohnte} unter uns.} Und \RWbibel{2\,Joh}{2.\,Joh.}{7}{}.\ zählt er die Meinung, daß Jesus keinen Leib gehabt habe, unter die Irrlehren des Antichrist's: Viele Verführer sind in die Welt ausgegangen, welche läugnen, daß Jesus Christus \RWbet{im Fleische} erschienen sey. Der heil.\ Paulus unterscheidet die Abstammung Jesu dem Fleische nach von seiner Abstammung als Gottes Sohn. \RWbibel{Röm}{Röm.}{9}{5}: \erganf{Ihnen (den Israeliten) gehören die Väter; ja von ihnen stammt, dem Fleische nach, Christus, welcher Gott über Alles ist.} -- Daß Jesus insonderheit auch eine \RWbet{menschliche Seele} gehabt, glaubte der Evangelist Lukas, wenn er von Jesu sagt (\RWbibel{Lk}{}{2}{52}), daß er an Weisheit und Gnade zugenommen habe. Von dem Verstande des \RWbet{Sohnes Gottes} konnte man das nicht sagen. Und Jesus selbst unterscheidet seinen Willen von dem Willen des Vaters. \RWbibel{Mt}{Matth.}{26}{39}: \erganf{Nicht wie ich will, sondern wie du willst.} Wie könnte er auch sonst gesprochen haben: \erganf{Bange bis zur Todesangst ist meiner Seele.}
\item Für die Behauptung, daß der Sohn Gottes mit dem Menschen Jesu \RWbet{gleich von dem ersten Augenblicke seiner Entstehung} an vereinigt gewesen sey, haben wir zwar keine ausdrücklichen Schriftstellen. Aber die Redensart \RWbibel{Joh}{Joh.}{1}{14}: \erganf{Und das Wort wurde Fleisch,} -- deutet doch hierauf hin. Im entgegengesetzten Falle, wenn der Sohn Gottes erst später, \zB\ erst bei dem Antritte seines öffentlichen Lehramtes, mit Jesu sich vereiniget hätte (wie dieses einige Irrlehrer angenommen): hätten die Evangelisten dieß ausdrücklich anmerken müssen, und Lukas hätte das, was aus Marien geboren werden sollte, nicht das Heilige nennen können. (\RWbibel{Lk}{}{1}{35})
\item Daß die Rücksicht, in welcher das Wort und der Mensch Jesus nur Eines ausmachen, die \RWbet{Person} sey, findet sich freilich nicht in der Schrift. Diese mehr \RWbet{wissenschaftliche} Bestimmung ist von der Kirche erst später, namentlich bei Veranlassung der Ketzerei des \RWbet{Nestorius} aufgestellt worden. Dieser Patriarch von Konstantinopel nahm nicht nur zwei Naturen, sondern auch zwei Personen in Christo an, eine menschliche und eine göttliche, und wollte eben darum~\RWSeitenw{66}\ nicht dulden, daß man Mariam eine Gottesgebärerin (\RWgriech{Jeot'okos}) nenne. (\RWgriech{Qristot'okos} wollte er zugeben.) Seine Irrlehre ward im dritten allgemeinen Kirchenrathe zu Ephesus (J.~431) verdammt, indem man festsetzte, daß jene beiden Naturen in Christo zu einer einzigen Person vereinigt gewesen seyen. (\RWlat{Unio hypostatica personalis, non vero moralis tantum.})
\item Der konstantinopolitanische Archimandrit \RWbet{Eutyches} verfiel, indem er die Irrlehre des Nestorius zu widerlegen suchte, in einen entgegengesetzten Irrthum, zu behaupten, in Christo hätten sich nicht einmal zwei Naturen befunden; sondern die Gottheit und die Menschheit hätten sich in ihm zu einer einzigen Natur vermischt. Diese Irrlehre der \RWbet{Monophysiten} wurde im vierten allgemeinen Kirchenrathe zu \RWbet{Chalcedon} (J.~451) verdammt.
\item Daß sich in Christo ein \RWbet{doppelter Verstand und Wille} befunden habe, ersieht man aus den schon vorhin aufgeführten Schriftstellen (Nr.\,4.). Daß Jesu menschlicher Verstand, obgleich er nicht allwissend gewesen, doch Alles das gewußt habe, was Jesus zur Erreichung seiner Bestimmung zu wissen nöthig hatte, setzten die evangelischen Geschichtschreiber allenthalben voraus. -- Im siebenten Jahrhunderte behaupteten die \RWbet{Monotheliten} (Theodorus Sergius, \uA , meist konstantinopolitanische Bischöfe, selbst \RWbet{Papst Honorius I.}), daß es nur \RWbet{einen einzigen Willen} in Jesu Christo gegeben habe. -- Ihr Irrthum wurde im sechsten allgemeinen Kirchenrathe, dem \RWbet{dritten zu Konstantinopel,} (J.~661) verdammt. -- Daß aber der menschliche Wille in Jesu mit dem göttlichen allezeit zusammengestimmt habe, und daß sonach Jesus ganz \RWbet{ohne alle Sünde} geblieben sey, wurde aus Jesu eigenen Worten \RWbibel{Joh}{Joh.}{8}{46}: \erganf{Wer von euch kann mich einer Sünde überführen?} -- und noch deutlicher aus den oben (Nr.\,3.) angefangenen Worten Pauli \RWbibel{Hebr}{Hebr.}{7}{26}\ gefolgert, wo es weiter heißt: \erganf{Der nicht, wie jene Oberpriester, nöthig hatte, zuerst für seine eigenen und dann für des Volkes Sünden zu opfern.}
\item[9.\ 10.\ 11.\ 12.] Die übrigen Artikel dieser Lehre sind theils eine Folge von den vorhergehenden, theils sind sie in der Kirche \RWlat{consensu extraconciliari} entschieden.~\RWSeitenw{67}
\end{aufza}

\RWpar{205}{Vernunftmäßigkeit}
\begin{aufza}
\item Gegen die Vernunftmäßigkeit des ersten Punctes möchte man vielleicht folgende Einwendungen erheben.\par
\RWbet{1.~Einwurf.} Indem das Christenthum sagt, \RWbet{der Sohn habe auf den Willen des Vaters beschlossen}, menschliche Natur an sich zu nehmen, setzt es voraus, daß Vater und Sohn jeder eine eigene Wollkraft besitzen; eine Voraussetzung, der es doch in der Lehre von der Dreieinigkeit selbst widersprochen hat.\par
\RWbet{Antwort.} Die bloß \RWbet{figürliche} Redensart: der Sohn habe auf den Willen des Vaters etwas beschlossen, zwingt uns noch gar nicht, eine eigene Wollkraft in beiden Personen vorauszusetzen. Wenn Göttliches mit Menschlichem verglichen werden darf: so haben wir ja bei der Dreieinigkeitslehre bereits gesagt, daß auch ein und derselbe Mensch gewissermaßen verschiedene Personen vorstellen, \zB\ Bürger und Gelehrter zugleich seyn könne. Könnte man nun nicht füglich sagen, der Gelehrte N.\,N. habe in Rücksicht dessen, daß er auch Bürger sey (\dh\ auf den Willen des Bürgers in ihm) beschlossen, ein nützliches Volksbuch zu schreiben? Eben so können wir auch sagen, daß Gott, als Sohn, auf den Willen des Vaters in Gott beschlossen habe, die menschliche Natur an sich zu nehmen, ohne daß darum zweierlei Wollkräfte in Gott vorausgesetzt werden müßten.\par
\RWbet{2.~Einwurf}. Die Unveränderlichkeit Gottes, welche das Christenthum lehrt, widerspricht einer erst in der Zeit erfolgten Annahme der menschlichen Natur; denn da wäre ja eine Veränderung in Gott vorgegangen.\par
\RWbet{Antwort}. Die Unveränderlichkeit Gottes, welche das Christenthum lehrt und die Vernunft beweisen kann, bezieht sich bloß auf seine \RWbet{inneren Beschaffenheiten.} In diesen inneren Beschaffenheiten oder Eigenschaften Gottes (\dh\ in seinen Kräften und Verhältnissen zwischen diesen Kräften) kann sich nie etwas ändern. In seinen äußern Beschaffenheiten aber, \dh\ in seinen Verhältnissen zur Welt, und in seinen Einwirkungen auf dieselbe kann und muß sich sehr~\RWSeitenw{68}\ Vieles verändern, weil auch die Dinge in der Welt sich ändern. Die Menschwerdung des Sohnes Gottes ist aber offenbar ein bloßes Verhältniß Gottes zur Welt; also konnte sie allerdings erst in der Zeit erfolgen, ohne daß Gottes Unveränderlichkeit dadurch aufgehoben würde.
\item Daß eine \RWbet{Jungfrau gebäre}, zwar ist ein Wunder, aber eben nicht unbegreiflicher, als viele andere, \zB\ die Wiederauflebung eines Todten.
\item Dieß Kind war \RWbet{ohne Erbsünde.} Die entgegengesetzte Behauptung würde sich ziemlich schlecht vertragen mit der Lehre, daß dieses Kind vom ersten Augenblicke seines Daseyns an mit dem Sohne Gottes auf's Innigste vereinigt gewesen sey. Diese Vereinigung war ja doch in der That eine Probe, nicht des göttlichen Mißfallens, sondern des höchsten Wohlgefallens Gottes an Jesu. -- Und wenn der letzte Grund der Erbsünde in jener körperlichen Unvollkommenheit (in jener größern Reizbarkeit zum Bösen) liegt, die wir von Adam erben: so ist auch begreiflich, wie Jesus ohne Erbsünde seyn konnte, da er nicht auf dem natürlichen Wege der Zeugung von Adam abstammte, indem sein Leib unmittelbar durch Gottes Einwirkung war gebildet worden, und eben darum jene Unvollkommenheit bei ihm vermieden werden konnte und mußte.
\item Er war ein \RWbet{wirklicher Mensch}. -- Das wird Jeder zugeben.
\item Mit diesem Menschen war vom ersten Augenblicke seines Daseyns \RWbet{die zweite göttliche Person vereiniget}. Bei dem Worte vereiniget muß man an keine Vereinigung im Raume denken, so wie sich etwa zwei Körper in einem Raume vereinigen, \zB\ Gold und Kupfer. Wenn man das Wort Vereinigung von einem geistigen Wesen, dergleichen Gott ist, gebraucht: so versteht man darunter nichts Anderes, als ein Mitwirken dieses Wesens mit jenem andern, mit dem es sich vereiniget hat, zu einem gemeinschaftlichen Zwecke. In dieser Bedeutung sagt man \zB , der Astronom N.\ habe sich mit dem Astronomen M.\ vereiniget, um diese oder jene Beobachtung zu machen, obgleich sie dem Raume nach vielleicht sehr weit von einander entfernt seyn können.~\RWSeitenw{69}\ Die Kirche nun, obwohl sie die bildlichen Redensarten gebraucht: der Sohn sey aus dem Schooße des ewigen Vaters zu uns herabgestiegen, er habe mit einer menschlichen Natur sich bekleidet, \udgl , will durch dieß Alles nichts Anderes sagen, als daß die zweite göttliche Person auf den Menschen Jesu auf eine eigene Art eingewirkt habe. Ist also nur die Art dieser Einwirkung selbst, die in den folgenden Puncten bestimmt wird, nicht ungereimt: so enthält auch diese Lehre von der Vereinigung Gottes mit einer menschlichen Natur nichts der Vernunft Widersprechendes. Die Vernunftmäßigkeit des Umstandes, daß der Sohn Gottes mit dem Menschen \RWbet{Jesu schon von dem ersten Augenblicke} des Daseyns dieses Letztern vereint gewesen sey, wollen wir Nr.\,11.\ betrachten.
\item Um nun die Art der Vereinigung, die zwischen Gott und dem Menschen Jesu Statt fand, näher zu bestimmen, fängt die Kirche damit an, sie eine \RWbet{Vereinigung zu einer einzigen Person} zu nennen. Unter Person versteht man insgemein eine mit Wissen und Willen wirkende Ursache, und wenn insonderheit zwei oder mehrere mit Verstand und Willen begabte Wesen, \zB\ zwei Menschen, sich zur Hervorbringung eines gemeinschaftlichen Zweckes vereinigen, so daß sie in Zukunft ein Jeder nur das zu thun beschließen, was die gemeinschaftliche Ueberlegung und der Wille beider entscheidet: so pflegt man diese Vereinigung mit dem Namen einer persönlichen zu bezeichnen. So pflegt man \zB\ von den Mitgliedern eines ganzen Staates zu sagen, daß sie zusammen nur eine einzige Person ausmachen. Diese Bedeutung auch hier zum Grunde gelegt: sagt dieser Artikel nichts Anderes aus, als der bald folgende achte. -- Daß übrigens die Person, welche aus dieser Vereinigung hervorging, den Namen einer \RWbet{göttlichen} erhielt, darf Niemand anstößig finden; da diese Benennung durchaus nur der bekannten Regel der Semiotik entspricht: \RWlat{denominatio fiat a parte potiori.}
\item Jene Vereinigung zwischen Gott und den Menschen war also, sagt die Kirche weiter, \RWbet{keine Vermengung} oder Vermischung der beiden Naturen zu einer dritten, die weder Gott noch Mensch gewesen wäre. -- Eine solche Vermengung~\RWSeitenw{70}\ oder Vermischung wäre fürwahr sehr ungereimt; denn dieses würde heißen, daß der Sohn Gottes mit dem Menschen Jesu dergestalt mitgewirkt habe, daß jener aufhörte, Gott, und dieser, Mensch zu seyn. Es hätte also eine innere Veränderung in Gott selbst vorgehen müssen.
\item So ungereimt dieß ist, so nothwendig war es, der Person, welche aus jener Vereinigung hervorging, einen \RWbet{doppelten Verstand und einen doppelten Willen} beizulegen. Wenn aber die Kirche ferner behauptet, daß der Verstand des Menschen Jesu durch die Mitwirkung des Sohnes Gottes zwar nicht allwissend (im strengsten Sinne) geworden sey (wie dieß die Protestanten behauptet hatten); wohl aber Alles das erkannt habe, was er zur Erreichung seiner Bestimmung zu wissen nöthig hatte, daß er \zB\ sein ganzes Leben niemals geirrt habe, \udgl : so enthält dieß gewiß nicht die geringste Ungereimtheit. Warum sollte es Gott nicht möglich seyn, einen Menschen von seiner frühesten Jugend an so zu leiten, daß er vor jedem Irrthume bewahret bliebe, daß er zu jener Zeit, da er sein öffentliches Lehramt antreten soll, in dem vollkommensten Besitze aller derjenigen Kenntnisse sey, die Gott durch ihn den Menschen mittheilen will, \usw ?\par
Noch außerordentlicher aber und wirklich einzig in ihrer Art war jene Unterstützung, welche der \RWbet{Wille Jesu} durch den Sohn Gottes erfuhr; denn er blieb immer ohne Sünde. Also kann die Kirche mit Recht sagen, die Unterstützung, die Jesus erfuhr, sey nicht bloß dem Grade nach, sondern \RWbet{der Art nach} von derjenigen unterschieden, die jeder andere Mensch erfährt. \par
\RWbet{Einwurf}. Diese Behauptung widerspricht der Freiheit des Willens Jesu.\par 
\RWbet{Antwort}. Die Kirche würde nur dann der Freiheit des Willens Jesu widersprechen, wenn sie behauptete, daß er nie habe sündigen \RWbet{können;} allein dieß sagt sie nicht, sondern behauptet vielmehr, daß Jesus zum Bösen sogar versucht worden sey; aber nur das sagt sie, daß er \RWlat{de facto} nie wirklich gesündiget \RWbet{habe.} Dieß Letztere wäre nun, strenge genommen, auch bei jedem anderen Menschen möglich, ob es~\RWSeitenw{71}\ gleich, der Erfahrung zu Folge, niemals in Wirklichkeit erfolgt. Und daß sich bei Jesu dieses Außerordentliche zutrug, daran lag der Grund (wenn auch nicht der bestimmende, doch der veranlassende Grund) in dem ganz außerordentlichen Beistande, der ihm von Seite Gottes zu Theil ward.
\item Man kann von dem Gottmenschen in Beziehung auf seine Menschheit sagen, daß er \RWbet{menschliche Pflichten} und Obliegenheiten gehabt, für \RWbet{Lust und Schmerz} empfänglich, sogar \RWbet{Versuchungen zum Bösen} ausgesetzt gewesen sey, \usw\ Eine nothwendige Folge davon, daß die Vereinigung des Sohnes Gottes mit dem Menschen Jesu in nichts Anderem bestanden, als in demjenigen, was wir so eben beschrieben haben; denn zu Folge des bisher Gesagten wurde dem Menschen Jesu die sonst bei Andern anzutreffende Fehlerhaftigkeit des Verstandes und Willens genommen; in allen übrigen Stücken aber ward nicht das Geringste geändert. Also mußte er allerdings Pflichten und Obliegenheiten behalten, \usw\
\item Nicht minder richtig ist aber auch, daß man dem Gottmenschen \RWbet{in Beziehung auf seine göttliche Natur Allmacht, Allwissenheit,} und alle die übrigen Prädicate des Sohnes Gottes beilegen könne; denn Gott ist unveränderlich; hat also der Sohn Gottes mit Jesu sich zu einer und eben derselben Person vereinigt: so muß man von dieser Person in Hinsicht auf den Sohn Gottes sagen können, daß sie allwissend, allmächtig sey, \usw\
\item Aus dem Bisherigen kann nun auch beurtheilt werden, ob die Behauptung der Kirche: daß der Sohn Gottes mit dem Menschen Jesu \RWbet{vom ersten Augenblicke des Daseyns} dieses Letzteren vereinigt gewesen sey, und seitdem immer in Verbindung bleibe, vernunftmäßig sey. Da diese Vereinigung nichts Anderes, als ein gewisses Mitwirken des Sohnes Gottes mit dem Menschen Jesu ist: so kann es immer seyn, daß sie schon von dem ersten Augenblicke des Daseyns dieses Letzteren nothwendig war. War er auch in den ersten Tagen seines Lebens, und vollends damals, als er noch eingeschlossen in dem Schooße seiner jungfräulichen Mutter lag, keiner deutlichen Vorstellungen, und eben darum auch~\RWSeitenw{72}\ keiner ganz freien Willensentschließungen fähig; konnte er also auch von seiner Seite bei dieser Vereinigung noch nicht so kräftig mitwirken, als es in Zukunft geschah, da seine Kräfte wuchsen: so läßt sich gleichwohl begreifen, daß eine Einwirkung des Sohnes Gottes auf ihn schon damals nöthig gewesen sey; denn wer weiß es nicht, daß jene Begriffe und Meinungen, jene Begierden und Willensentschließungen, die sich in unserem Jünglings- und Mannesalter bei uns entwickeln, einem sehr großen Theile nach von jenen Eindrücken abhängen, die wir in unseren ersten Lebenstagen, ja schon im Mutterschooße empfingen? Sollte also der Mensch Jesus werden, was er wirklich ward: so ist es begreiflich, daß Gottes Vorsehung ihn schon von dem ersten Augenblicke seines Daseyns an auf ganz besonderen Wegen habe leiten müssen; daß der Sohn Gottes seine Einwirkungen auf ihn schon damals habe anfangen müssen. -- Eben so wenig Anstößiges liegt in der zweiten Behauptung, \RWbet{daß diese Verbindung seitdem nie wieder aufgehört habe, und nie aufhören werde.} Eher wäre noch das Gegentheil der Güte Gottes zuwider, einen Menschen, den Gott einmal gewürdiget hat, so außerordentlich zu unterstützen, und der diese Unterstützung nicht gemißbraucht hat, der so getreu mit ihr mitgewirkt, so große Thaten ausgeführt, und keiner Sünde je sich schuldig gemacht hat, doch wieder zu verlassen! --\par
\RWbet{Einwurf.} Aber jene außerordentliche Unterstützung war nur für Zwecke nothwendig, die der Mensch Jesus auf Erden auszuführen hatte. Waren diese ausgeführt: so fiel die Nothwendigkeit derselben weg.\par
\RWbet{Antwort.} Als ob der Mensch Jesus in jenen höheren Gegenden, in die er sich von dieser Erde aufgeschwungen hat, nicht auch noch fortfahren könnte zu wirken, und selbst für uns zu wirken? als ob der Beistand Gottes ihm nicht auch dort nützlich und nothwendig wäre?\par
\item Die Redensart: \RWbet{Gott ist Mensch geworden,} enthält nichts Unanständiges, weil es dem Sprachgebrauche völlig entsprechend ist, mit ihr den Sinn zu verbinden, daß der Sohn Gottes menschliche Natur an sich genommen, mit einem Menschen sich vereiniget habe. Ungereimt wäre sie~\RWSeitenw{73}\ nur, wenn man darunter verstünde, daß Gott in einen Menschen sich verwandelt, \dh\ Gott zu seyn aufgehört habe. Die Redensart dagegen: \RWbet{ein Mensch sey Gott geworden,} klingt immer anstößig; denn wollte man sie noch auf's Glimpflichste und nach dem Beispiele der vorigen etwa so auslegen, daß sich ein Mensch mit Gott vereiniget habe: so wäre doch das hieran zu tadeln, daß diese Vereinigung als etwas mehr von der Willkür des Menschen, als von Gott Abhängiges dargestellt würde. Man soll nicht sagen, daß sich der Mensch mit Gott, sondern daß Gott sich mit dem Menschen vereiniget habe, denn nicht der Mensch, sondern Gott macht die vornehmste Ursache bei einer solchen Vereinigung aus. Es ist zwar freilich wahr, daß wenn eine solche Vereinigung zu Stande kommen sollte, nicht nur Gott, sondern auch der Mensch sie wollen und dabei habe mitwirken müssen; aber nur bleibt der Wille und die Mitwirkung Gottes hier bei Weitem das Wichtigste.
\end{aufza}
\begin{RWanm}[\RWbet{Schlußbemerkung.}] Man hat sonst noch verschiedene andere Einwürfe gegen die Lehre von der Menschwerdung des Sohnes Gottes vorgebracht, die aber durch die bisherige Darstellung derselben beinahe von selbst gehoben werden. So gab \zB\ \RWbet{Peter Bayle} folgenden Einwurf für unwiderleglich aus: Da der Sohn Gottes schon eine Person ausmacht, da der Mensch Jesus, wenn er ein Mensch, wie wir, gewesen, gleichfalls Persönlichkeit gehabt; wie können beide zusammen nur eine einzige Person ausgemacht haben? -- So hielten es auch selbst einige Katholiken für eine unbeantwortliche Frage: wie, da die drei göttlichen Personen so unzertrennlich vereiniget sind, der Sohn mit dem Menschen Jesu eine Verbindung habe eingehen können, die nicht zugleich der Vater und der heil.\ Geist eingegangen sind? -- Aus unseren oben aufgestellten Begriffen ergibt sich die Antwort auf diese Fragen von selbst. Dürfen wir Menschliches mit Göttlichem vergleichen; so dürfen wir zur Erläuterung der ersten Schwierigkeit das Gleichniß anführen, daß ja auch jeder einzelne Bürger des Staates für sich selbst eine Person vorstelle, und daß gleichwohl auch alle Bürger zusammen wieder nur Eine gewisse Person (nämlich die eines einzelnen Staates) vorstellen. Auf die zweite Schwierigkeit ist zu erwiedern, daß jene Verbindung des Sohnes Gottes mit dem Menschen Jesu in einem Einwirken auf den letztern bestanden habe; und je nachdem diese Einwirkungen bald nur das~\RWSeitenw{74}\ Beste der menschlichen Natur unseres Herrn Jesu selbst, bald wenigstens mittelbar die Beförderung des Wohles der ganzen Menschheit, ja aller geschaffenen Wesen überhaupt bezweckten, können sie bald der Einen, bald der Andern der drei göttlichen Personen zugeschrieben werden. Noch fanden Einige einen scheinbaren Widerspruch zwischen dieser Lehre und einigen Stellen des alten Bundes, in denen es heißt, daß Gott seinen Namen auf kein Geschöpf übertragen werde; \zB\ \RWbibel{Jes}{Isai.}{42}{8}: \erganf{Ich bin Jehovah, dieß ist mein Name, nie werde ich meine Ehre an einen Andern überlassen, nie meinen Ruhm den Götzenbildern.} In der Lehre von der Menschwerdung, sagt man, geschieht dieß aber dennoch. Christus erhält den Namen eines Gottmenschen, und ihm wird göttliche Ehre erwiesen. Der neue Bund also hebt das Wort des alten auf. -- Ich antworte: Die göttliche Ehre, die Christo erwiesen wird, wird nicht dem Menschen, sondern dem Sohne Gottes erwiesen. Uebrigens heißt es bei eben dem \RWbibel{Jes}{Isai.}{9}{5}: \erganf{Ein Kind ist uns geboren, ein Sohn ist uns geschenkt, auf dessen Schultern Herrscherwürde ruhet: \RWbet{sein Name} ist der Wunderbare, der weise Rathgeber, \RWbet{der starke Gott}, der Vater der Ewigkeiten, der Fürst des Friedens.}
\end{RWanm}

\RWpar{206}{Sittlicher Nutzen}
\begin{aufza}
\item Nur \RWbet{auf den Willen} des Vaters hat sich der Sohn entschlossen, menschliche Natur an sich zu nehmen, \usw\ Ein schönes Bild der Einigkeit, die zwischen Vätern und Söhnen herrschen, und des Gehorsams, welchen die Letzteren den Ersteren leisten soll[en]. Selbst der Sohn Gottes war gehorsam, und bis zum Tode, ja bis zum Tode des Kreuzes gehorsam!
\item Eine \RWbet{reine Jungfrau empfing durch die Kraft Gottes}. Wer sieht nicht ein, wie sehr durch diese Lehre
\begin{aufzb}
\item die Würde des jungfräulichen Standes erhoben wird! Nun ist zwar nicht zu läugnen, daß der eheliche Stand den Menschen im Allgemeinen natürlicher, mithin auch meistentheils zuträglicher für ihre Tugend und Glückseligkeit ist, als der ehelose; es steht auch zu hoffen, je weiter das menschliche Geschlecht in seiner Vollkommenheit fortrücken wird, um desto geringer werde die Anzahl jener Menschen werden, denen es ihre Verhältnisse bald durch\RWSeitenw{75}aus nothwendig, bald doch sehr rathsam machen, unverehelicht zu leben; aber nie wird doch die Classe dieser Menschen ganz verschwinden, und wohl für Niemand wird es je rathsam seyn, sich in den Ehestand gleich dann zu begeben, wenn er die ersten Regungen dazu fühlt. Für diese Zeit des Aufschubes also wird es für einen jeden Menschen dienlich seyn, einen hohen Begriff von der Würde und Verdienstlichkeit des jungfräulichen Standes zu haben; denn solche Vorstellungen werden ihm seine Lage überaus erleichtern, ihn für die Vergnügungen, deren er entbehrt, schadlos halten, ihm eine sichere Schutzwehr seyn vor jeder Versuchung und vor jedem Angriffe auf seine Tugend. Daher finden wir auch, daß man bei allen weiseren Völkern den jungfräulichen Stand in besondern Ehren gehalten, ihm eigene Vorrechte eingeräumt, \udgl\  Das Christenthum nun erhebt die Würde dieses ehelosen Standes, ohne der Heiligkeit des ehelichen den mindesten Abbruch zu thun, wenn es behauptet, daß Maria ihren Erstgebornen ohne Mitwirkung eines Mannes, mit dem sie doch in der That verehelicht war, zur Welt gebracht habe, und daß sie eben um ihrer Reinigkeit willen von Gott für würdig erachtet worden sey, Mutter des Herrn zu werden.
\item Ein zweiter Nutzen dieser Lehre besteht darin, daß sie unsere Vorstellung von der Würde und Erhabenheit Jesu erhöht; denn nicht durch den Willen des Mannes, nicht nach dem Willen des Fleisches, sondern aus Gott war er erzeugt. Gibt dieser außerordentliche Ursprung uns die Erhabenheit, die Jesus auch schon als Mensch über uns Alle gehabt, nicht deutlich zu erkennen?
\end{aufzb}\par
\RWbet{Einwurf.} Aber je höher die menschliche Natur des Erlösers über die unsrige erhoben wird, um desto unwirksamer wird sein Beispiel für uns. Wir denken dann: Er, der von keinem Weibe geboren war, Er, dessen Leib Gott selbst gebildet hatte, er konnte freilich Dieß und Jenes thun; aber für uns ist es nicht möglich, ihm das nachzuthun.\par
\RWbet{Antwort.} Diesem Nachtheile beugt die bald folgende Lehre des Christenthums vor, daß Jesus gleichfalls Versuchungen ausgesetzt war.~\RWSeitenw{76}\par
\item Jesus war \RWbet{ohne Erbsünde} empfangen. Diese Nachricht hat den einleuchtenden Nutzen, unsere Verehrung der Person Jesu zu vermehren, und um so begreiflicher zu finden, wie ihm so große Gnaden schon von dem ersten Augenblicke seines Daseyns an von Gott dem Heiligsten mitgetheilt werden konnten. In Rücksicht jenes Nachtheils aber, den diese Lehre gemeinschaftlich mit der vorhergehenden haben könnte, gilt die vorige Antwort.
\item Jesus war ein \RWbet{wirklicher Mensch}, der einen wirklichen Leib und eine wirkliche Seele hatte. Diese Behauptung hat folgende Vortheile von größter Wichtigkeit:
\begin{aufzb}
\item Nur wenn wir Jesum für einen wirklichen, nicht bloß scheinbaren Menschen halten, wurde die menschliche Natur durch die Vereinigung des Sohnes Gottes mit ihm auf's Höchste geehrt und geadelt. Die Gottheit selbst hat sich herabgelassen, Mensch zu heißen, und nicht nur den äußeren Anschein, sondern die wirkliche Natur eines Menschen an sich genommen. Wäre der Leib Jesu nur ein scheinbarer gewesen: so würde gerade durch diese Lehre die menschliche Natur eher herabgewürdiget, als erhoben. Nun würde sie folgenden Sinn geben: Wenn es Gott nöthig findet, den Menschen einen Lehrer höherer Wahrheiten zu senden: so findet er zu diesem Zwecke unter allen Menschensöhnen nicht einmal Einen Tauglichen; derjenige, den Er auf Erden erscheinen läßt, muß zwar die Gestalt eines Menschen nachahmen, weil er sonst von ihnen nicht verstanden werden könnte; aber er muß sich hüten, menschliche Natur an sich zu nehmen; sonst würde er eben dadurch sich selbst zu einem so großen Zwecke untauglich machen. Wie niederschlagend für uns!
\item Wir sehen in einem erstaunungswürdigen Beispiele die große Vollkommenheit, deren die menschliche Natur empfänglich ist. Einer Vereinigung mit Gott selbst ist sie fähig, und dieß zwar einer so innigen, daß der Wille des Menschen auch nicht ein einziges Mal von dem göttlichen abweicht!
\item Wenn Jesus ein wahrer Mensch gewesen ist, und dabei ohne Sünde: so ist sein Lebenswandel uns das voll\-\RWSeitenw{77}kom\-men\-ste Muster der Nachahmung, das wir nur wünschen können. Er lebte in ähnlichen Verhältnissen wie wir, er hatte Pflichten und Obliegenheiten wie wir, ähnliche Versuchungen, denselben untreu zu werden; hat aber diese Versuchungen alle auf's Glücklichste besiegt: so lernen wir denn an ihm Beides, wie der Mensch seyn soll, und wie er das, was er seyn soll, werde.
\item Nur wenn Jesus ein wahrer Mensch gewesen ist, wird es uns leicht, eine recht herzliche Liebe zu ihm zu fassen; denn nur zu Wesen, welche uns ähnlich sind, und nur, in wiefern sie uns ähnlich sind, fühlen wir uns von Natur hingezogen, und es wird uns leicht, sie zu lieben. Ist Jesus ein Engel oder ein höherer Geist gewesen, der nur die äußere Gestalt eines Menschen an sich genommen hatte: so ist er uns im Grunde fremd; wir bleiben kalt und ungerührt bei allen seinen Leiden; denn es sind ja nur scheinbare Leiden gewesen.
\item Ist Jesus ein wahrer Mensch gewesen: so können wir, wenn er einst kommen wird, uns zu richten, einen gerechten und billigen Richter an ihm zu finden hoffen; denn er hat unsere Schwachheiten kennen gelernt, hat die Versuchungen, die uns zuweilen dahinreißen, selbst gefühlt; \usw\
\end{aufzb}
\item Mit der menschlichen Natur Jesu war der Sohn Gottes \RWbet{vom ersten Augenblicke ihres Daseyns} an vereiniget. Durch diese Behauptung macht uns das Christenthum auf die wichtige Wahrheit aufmerksam, daß die Bildung des menschlichen Geistes schon in der frühesten Kindheit, ja schon im Mutterleibe ihren Anfang nehme.
\item Der Name \RWbet{Gottmensch} ist überaus schicklich, um uns die beiden Naturen, die hier zu einer und ebenderselben Person vereinigt sind, zugleich erinnerlich zu machen; in diesem bloßen Namen liegt gewissermaßen schon Alles angedeutet. Eben so schicklich ist auch der Name einer \RWbet{Person}, und das zwar einer \RWbet{göttlichen} Person; das Uebrige dieses Artikels folgt Nr.\,8.
\item Es wäre offenbar entehrend für Gott, den Unveränderlichen, zu sagen, daß Gott und Mensch in ein gewisses Drittes, das weder Gott noch Mensch ist, verschmolzen sey.~\RWSeitenw{78}
\item Daß der \RWbet{Verstand des Menschen} Jesu dergestalt von Gott erleuchtet worden sey, daß \usw\ dient zu unserer größten Beruhigung, wenn wir die religiösen Ansichten, die Jesus uns vorgetragen hat, annehmen sollen; denn jetzt wissen wir, daß Alles, was aus seinem heiligen Munde hervorging, sichere und unfehlbare Wahrheit sey, und verlangen nicht mehr, daß er die Wahrheit jeder einzelnen Behauptung durch Wunder und Zeichen bekräftige. -- Daß aber sein \RWbet{Wille} durch die Mitwirkung des Sohnes Gottes völlig unfehlbar geworden sey, hat die Nr.\,4. erwähnten Vortheile.
\item Der Nutzen dieses Punctes hat sich bereits Nr.\,4.\ \RWlat{lit.\,c.}\ gezeigt.
\item Obgleich die Eigenschaften der \RWbet{Allmacht, Weisheit}, \usw\ dem Gottmenschen nur in Beziehung auf seine göttliche Natur beigelegt werden: so ehren sie doch auch den Menschen in ihm, und also uns Alle.
\item Wenn der Mensch Jesus \RWbet{immerwährend} in der Vereinigung mit dem Sohne Gottes bleibt, die einmal angefangen hatte: so ist uns dieses 
\begin{aufzb}
\item ein neuer Beweis der allgemeinen Wahrheit, daß Gott von Niemand weicht, der nicht erst selbst von ihm gewichen ist; und gewährt uns
\item die erfreuliche Hoffnung, daß Jesus auch jetzt noch und in alle Ewigkeit wirksam für unsere Beseligung bleiben werde.
\end{aufzb}
\item Das Ehrenvollste, was man von unserer menschlichen Natur nur immer sagen kann, ist ohne Zweifel in drei Worten enthalten: Gott ist Mensch geworden. Unschicklich aber und Gott zu nahe tretend wäre es, zu sagen, ein Mensch sey Gott geworden.
\end{aufza}

\RWpar{207}{Wirklicher Nutzen}
Die Lehre von der Menschwerdung des Sohnes Gottes ist schon in den ältesten Zeiten sehr Vielen anstößig gewesen, und hat besonders viele Juden von der Annahme des Christenthums abgehalten, da sie in ihr durch Mißverstand eine Art Abgötterei, also den ihnen so verhaßten Polytheismus~\RWSeitenw{79}\ der Heiden, zu finden glaubten. Diese Lehre hat auch in der Folge zu vielen Streitigkeiten Anlaß gegeben, deren einiger wir (\RWparnr{204}) erwähnten. Nichts desto weniger wird jeder Unparteiliche bemerken, daß auch von dieser Lehre gelte, was wir schon oftmals gesagt, daß ihr bei Tausenden gestifteter Nutzen den Schaden bei Einzelnen weit überwogen habe. Auch ist nicht zu vergessen, daß diese Lehre unter diejenigen gehört, die einen bedeutenden scientifischen Nutzen gehabt, und zur Verdeutlichung der Begriffe Gott, Natur, Person, Vereinigung, \usw\ viel beigetragen haben.

\RWpar{208}{Die Lehre von der Erlösung durch Iesum Christum}
Die Menschwerdung des Sohnes Gottes hat, wie uns das Christenthum versichert, eine Menge der \RWbet{beseligendsten Folgen} für das gesammte menschliche Geschlecht hervorgebracht. Die vornehmsten derselben sind:
\begin{aufza}
\item Der menschgewordene Sohn Gottes wurde der \RWbet{erste Lehrer und Gründer des vollkommensten religiösen Lehrbegriffes}, dessen das menschliche Geschlecht nur immer empfänglich ist, der auch bis an das Ende der Zeiten fortdauern, und einst der allgemeine Antheil aller Menschen werden soll.
\item Der Gottmensch hat uns an seinem eigenen Wandel auf Erden \RWbet{ein Muster der menschlichen Vollkommenheit} gegeben, an dem wir abnehmen können, wie auch wir handeln sollen in den verschiedensten Verhältnissen des Lebens.
\item Er hat uns \RWbet{die Vergebung unserer Sünden, der eigenen sowohl als auch der Erbsünde erwirkt}, indem er die Strafen, die wir verdient hätten, an unserer Statt getragen. Er hat dieß \RWbet{freiwillig} gethan, bloß aus Gehorsam gegen den Vater. Er hat gelitten und ist gestorben \RWbet{für das ganze menschliche Geschlecht}, für die Guten sowohl als für die Bösen, für Jene sowohl, die früher gelebt, als auch für Jene, die später leben. Wir sollen uns (auch bildlicher Weise) vorstellen, daß an den Leiden des Menschen auch der Sohn Gottes gleichsam Theil genommen habe,~\RWSeitenw{80}\ und eben in diesem Bilde die Größe der Liebe Gottes gegen uns und die Abscheulichkeit der Sünde erkennen.
\item Der Gottmensch hat uns die durch den Sündenfall \RWbet{verlorene Anwartschaft auf den Himmel und Gottes Wohlgefallen von Neuem wieder erworben.}
\item Er hört noch jetzt nicht auf, für uns zu wirken und uns wohlzuthun; er \RWbet{regiert die Schicksale der auf der Erde von ihm gestifteten Kirche} als das unsichtbare Oberhaupt derselben, \usw\
\end{aufza}

\RWpar{209}{Historischer Beweis dieser Lehre}
Daß durch die Menschwerdung des Sohnes Gottes überhaupt die beseligendsten Folgen für das ganze menschliche Geschlecht beabsichtiget worden seyen, erkennen wir schon aus dem bloßen Namen \RWbet{Jesus}, den man dem Kinde geben mußte, und der einen Beseliger bedeutet. So legen auch die Apostel Jesu häufig den Namen \RWgriech{swt`hr <hm~wn} (unser Retter) bei; \zB\ \RWbibel{Tit}{Tit.}{1}{4}\ \RWbibel{2\,Petr}{2\,Petr.}{2}{20}\ \RWbibel{1\,Joh}{1\,Joh.}{4}{14}\ \uma\  Und Jesus selbst sagt \RWbibel{Joh}{Joh.}{3}{17}: \erganf{Gott sandte seinen Sohn nicht in die Welt, damit er die Welt richte (\dh\ verdamme); sondern damit sie durch ihn gerettet würde.}
\begin{aufza}
\item Daß die katholische Religion, welche wir gegenwärtig auf Erden antreffen, Jesum zu ihrem ersten Gründer und Stifter habe, lehren die Katholiken mit größter Einmüthigkeit. Auch halten sie diese Religion unstreitig für die vollkommenste, deren das menschliche Geschlecht nur immer fähig ist, und nennen sie eben deßhalb, wie wir in der Folge sehen werden, die alleinseligmachende. Auch behaupten sie, daß diese Religion bis an das Ende des menschlichen Geschlechtes fortdauern werde, zu dessen Versicherung sie unter andern die Worte Jesu anführen \RWbibel{Mt}{Matth.}{16}{18}: \erganf{Die Pforten der Hölle werden sie nicht überwältigen.} Endlich glauben sie auch, daß diese Religion einmal der Antheil aller Menschen werden solle, und führen die Worte an \RWbibel{Joh}{Joh.}{10}{16}: \erganf{Ich habe noch andere Schafe, die nicht von dieser Heerde sind. Auch diese muß ich herbeiführen, und sie werden mei\RWSeitenw{81}ner Stimme folgen, und es wird Ein Hirt nur seyn und Eine Heerde.}
\item Daß sein Lebenswandel für uns ein Muster der Nachahmung seyn soll, hat Jesus selbst bei mehr als einer Gelegenheit gesagt; \zB\ \RWbibel{Joh}{Joh.}{13}{15}\ nach jener Handlung des Fußwaschens: \erganf{Ich habe euch ein Beispiel gegeben, damit ihr, was ich euch gethan habe, einander auch thuet.} -- \RWbibel{Joh}{Joh.}{15}{10}: \erganf{Wenn ihr meine Gebote befolget: so beharret ihr in meiner Liebe; so wie ich die Gebote meines Vaters befolge, und in seiner Liebe beharre.} Der Apostel Petrus (\RWbibel{1\,Petr}{1\,Petr.}{2}{21}) ermuntert die Christen zur Geduld und Standhaftigkeit durch das Beispiel Jesu, indem er beisetzt: \erganf{Denn dazu seyd ihr berufen, weil auch Christus gelitten, und uns ein Beispiel hinterlassen, auf daß auch wir in seine Fußstapfen treten.} Der heil.\ Paulus ermuntert durch eben dieß Beispiel zur Demuth, \Ahat{\RWbibel{Phil}{Philipp.}{2}{5}}{2,6.}: \erganf{Ihr sollet gesinnet seyn, wie Jesus Christus gesinnet war}, \RWbibel{Eph}{Ephes.}{5}{1}\ zur Liebe, \usw\
\item Daß die Leiden und der Tod Jesu
\begin{aufzb}
\item zur Vergebung der Sünden für uns gedienet haben, lehren unzählige Stellen der heil.\ Schrift. Schon in den Büchern des alten Bundes wurde dieß deutlich genug durch den Propheten Isaias vorhergesagt in der bekannten Stelle \RWbibel{Jes}{}{50}{4}\ \erganf{Fürwahr, die Leiden duldet er, die wir verschuldet, nur unsere Schmerzen ladet er auf sich. Er ist durchbohrt um unserer Sünden willen, zerschlagen wegen unserer Missethat; durch seine Wunden wurden wir geheilt.} -- Jesus selbst hat dieß bei mehreren Gelegenheiten ausdrücklich gesagt, \zB\ in jenem Gespräche mit Nikodemus \RWbibel{Joh}{Joh.}{3}{14}\ \erganf{Wie Moses in der Wüste die Schlange aufrichtete: so muß auch der Sohn des Menschen erhöhet werden, damit Alle, die an ihn glauben, nicht verloren gehen, sondern das ewige Leben erhalten.} Im \RWbibel{Num}{4\,Mos.}{21}{1\,ff}\ wird erzählt, daß die Israeliten in der Wüste einmal durch eine gewisse Art Schlangen, als Strafe ihres Murrens wider Moses, sehr beunruhiget worden wären; und daß darum Moses auf Gottes Befehl eine eherne Schlange aufgerichtet habe, worauf dann ein Jeder, der sie anblickte (mit gläubigem Sinne),~\RWSeitenw{82}\ gesund worden sey. Mit dieser ehernen Schlange Mosis vergleicht sich hier Jesus, indem er eben so, wie jene, am Kreuze erhöht wurde, und zur Genesung aller derjenigen, die an ihn glauben, dienen sollte. \Ahat{\RWbibel{Mt}{Matth.}{20}{28}}{16,28.}\ sagt Jesus: \erganf{Des Menschen Sohn kam in die Welt, nicht damit ihm gedienet werde, sondern daß er selbst diene und seine Seele hingebe zur Erlösung für Viele.} Und \RWbibel{Mt}{Matth.}{26}{28}\ beim letzten Abendmahle: \erganf{Dieses ist mein Blut, welches für Viele zur Vergebung der Sünden wird vergossen werden.} Ein Gleiches lehrten auch die Apostel; \zB\ Paulus \RWbibel{2\,Kor}{2\,Kor.}{5}{18--21}: \erganf{Gott hat uns mit sich selbst durch Jesum Christum versöhnt; denn Gott versöhnte durch Christum die Welt mit sich selbst. Er ließ ihn, der von keiner Sünde wußte, für uns ein Schuldopfer werden, damit wir durch ihn vor Gott gerecht würden.} Und \RWbibel{Röm}{Röm.}{3}{25}: \erganf{Jesum Christum hat Gott zu einem Sühnopfer verordnet für Alle, die auf sein Blut vertrauen.} Eben so \RWbibel{1\,Petr}{1\,Petr.}{2}{22}\ \usw\
\item Daß unter die Sünden, von welchen uns Jesus Christus befreit, auch die Erbsünde gehöre, hat die Kirche von jeher gelehrt. Man schloß es unter Anderem aus dem Gespräche Jesu mit Nikodemus, ingleichen aus der nur vorhin angeführten Stelle Pauli \RWbibel{2\,Kor}{2\,Kor.}{5}{21}
\item Daß aber Jesus für die ganze Welt gelitten habe, erhellet \zB\ schon aus den oben angeführten Worten Jesu (\RWbibel{Joh}{Joh.}{3}{16}): \erganf{Also hat Gott die Welt geliebt}, \usw\ den Ausdruck \RWbet{für Viele} (\RWgriech{per`i poll~wn}) beim heiligen Abendmahle (\RWbibel{Mt}{Matth.}{26}{28}) \uaO\ erklärt man entweder für einen Hebraismus, oder man sagt, hier sey nur von dem \RWbet{wirklichen Erfolge} der Beseligung die Rede, der freilich bei Manchen durch ihre eigene Schuld ausbleibt. -- \RWbibel{Röm}{Röm.}{14}{15}\ warnet der Apostel vor der Sünde des Aergernisses mit folgenden Worten: \erganf{Laß doch um einer Speise willen nicht denjenigen zu Grunde gehen, für den Christus gestorben ist.} -- Hieraus ist zu ersehen, daß diejenigen, für welche Christus gestorben ist, auch zu Grunde gehen können, mithin daß Christus auch für die Sünder gestorben ist. Und der heil.\ Johannes schreibt~\RWSeitenw{83}\ mit ausdrücklichen Worten \RWbibel{1\,Joh}{1 \,Joh.}{2}{2}: \erganf{Er ist das Sühnopfer für unsere Sünden; doch nicht nur für die unsrigen, sondern auch für die Sünden der ganzen Welt.}
\item Er hat dieß freiwillig gethan, bloß aus Gehorsam gegen den Vater; denn bei \RWbibel{Joh}{Joh.}{10}{18}\ sagt Jesus: \erganf{Niemand entreißt mir das Leben, sondern ich lasse es freiwillig. Ich habe die Macht, es hinzugeben, und es wieder zu nehmen. Aber diesen Auftrag habe ich von meinem Vater erhalten.} So auch \RWbibel{Phil}{Philipp.}{2}{8}
\item Wir sollen uns auch bildlicher Weise vorstellen, daß der Sohn Gottes selbst an den Leiden des Menschen Jesu gleichsam Theil genommen habe, und in diesem Bilde eben die Größe der Liebe Gottes gegen uns erkennen. -- Zu dieser Vorstellung weiset uns Jesus selbst an, wenn er spricht \RWbibel{Joh}{Joh.}{3}{16}: \erganf{Daß er (Gott) seinen eingebornen Sohn hingab.} Und Paulus schreibt \RWbibel{Röm}{Röm.}{8}{32}: \erganf{Er hat seinen eigenen Sohn nicht verschont, sondern ihn für uns Alle hingegeben; wie sollte er uns in ihm nicht Alles schenken?}
\end{aufzb}
\item Christus hat uns die verlorene Anwartschaft auf den Himmel und Gottes Wohlgefallen wieder erworben. \RWbibel{Eph}{Ephes.}{2}{4}\ \erganf{Da Gott so reich an Erbarmungen ist, so hat er uns nach jener großen Liebe, mit der er uns geliebt hat, uns, die wir todt durch unsere Sünde waren, wieder lebendig gemacht in Christo, durch dessen Gnade ihr selig geworden seyd. Er hat uns mit ihm erweckt, und einen Platz im Himmel angewiesen.}
\item Er hört noch jetzt nicht auf, für uns zu wirken, \usw\ Jesus sagt selbst \RWbibel{Mt}{Matth.}{28}{18}\ kurz vor seiner Himmelfahrt: \erganf{Mir ist alle Gewalt gegeben im Himmel und auf Erden; siehe, ich bleibe bei euch alle Tage bis an das Ende der Zeiten.} -- Und bei \RWbibel{Joh}{Joh.}{14}{2}: \erganf{Ich werde nun zum Vater gehen, und euch dort einen Ort bereiten, damit auch ihr seyd, wo ich bin.} Und der heil.\ Paulus schreibt \RWbibel{Hebr}{Hebr.}{7}{25}: \erganf{Er (Jesus) kann diejenigen, welche durch ihn zu Gott sich wenden, vollständig beseligen, weil er ewig lebt, um sich ihrer anzunehmen.}~\RWSeitenw{84}
\end{aufza}

\RWpar{210}{Vernunftmäßigkeit}
\begin{aufza}
\item Daß der menschgewordene Sohn Gottes uns \RWbet{eine neue Religion, als eine göttliche Offenbarung, habe mittheilen wollen}, kann Niemand vernunftwidrig finden, der die Gründe, womit wir die Möglichkeit und Nothwendigkeit einer Offenbarung erwiesen, befriedigend gefunden hat. Daß aber diese Religion einst aller Menschen Antheil werden, und bis an das Ende des menschlichen Geschlechtes fortdauern solle, könnte demjenigen, der an ein stetes Fortschreiten auf Erden glaubt, anstößig seyn, wenn er nicht bedächte, daß eben dieselbe katholische Kirche, welche die stete Fortdauer ihres Glaubens lehrt, auch eine stete Ausbildung und Entwicklung desselben zugibt. Uebrigens enthält diese Lehre auch einen historischen Bestandtheil, daß nämlich Jesus von Nazareth wirklich der Stifter derjenigen Kirche sey, welche sich heut zu Tage die katholische nennt. Dieses Factum wird durch die Geschichte auf's Glaubwürdigste bestätigt, so daß man sich in der That wundern muß, wie einige Feinde des Christenthums ihren Widerspruchsgeist so weit treiben konnten, daß sie behaupteten: Jesus von Nazareth habe gar nie die Absicht gehabt, eine neue Religionsgesellschaft zu errichten. Die \RWbet{Apostel}, deren bloßer Name das schon widerlegen könnte, mußten doch nothwendig wissen, ob ihr Herr mit diesem Zwecke umgegangen sey oder nicht. Wären sie nicht wirklich so, wie sie es an unzähligen Stellen ihrer Schriften sagen, beauftragt gewesen, die neue Lehre Jesu zu verkündigen, wie hätten sie sich zu diesem mühe- und gefahrvollen Geschäfte entschließen können? Man sehe hierüber auch \RWbibel{Mk}{Mark.}{16}{25}\ \RWbibel{Mt}{Matth.}{16}{8}\ \RWbibel{Joh}{Joh.}{15}{1} \uam\ 
\item Daß uns das \RWbet{Beispiel Jesu als Muster} aufgestellt wird, enthält nichts Widersprechendes; da wir in diesem Beispiele nicht den geringsten sittlichen Fehler gewahren.
\item \mbox{}\\
\textbf{A.}~Die Lehre von der \RWbet{Vergebung der Sünden durch den Tod Jesu} ist aber eine derjenigen, welche am Häufigsten bestritten worden ist. Es ist hier zweierlei zu rechtfertigen:~\RWSeitenw{85}
\begin{aufzb}
\item daß eine Vergebung der Sünden überhaupt möglich sey;
\item daß die Leiden oder der Tod Christi ein Mittel zu ihrer Herbeiführung werden konnten.
\end{aufzb}
\begin{center}*\ *\ *\end{center}
\begin{aufzb}
\item Was die sich selbst überlassene Vernunft, ohne die Grenzen der Bescheidenheit zu verletzen, über den ersten Punct entscheiden könne, haben wir bereits in dem Hauptstücke von der Nothwendigkeit einer Offenbarung gezeigt. Sie kann nur sagen, daß eine Vergebung der Sünden ohne Besserung unmöglich sey; weil diese der Beförderung der Tugend und somit auch der Glückseligkeit des Ganzen äußerst nachtheilig wäre. Ob aber, wenn die Besserung erfolgt, eine Vergebung eintreten könne; darüber, zeigten wir, könne die menschliche Vernunft nichts mit Sicherheit entscheiden, und dieß zwar aus folgendem Grunde. Gott ist, vermöge seiner Heiligkeit, an keine andere Regel des Verhaltens gebunden, als überall das zu thun, was die Tugend und Glückseligkeit des Ganzen am meisten befördert. Wenn es sich also fragt: ob er die Sünde auch nach erfolgter Besserung noch strafen oder nicht strafen solle: so kommt dieß bloß darauf an, ob das Erstere oder das Letztere die Tugend und Glückseligkeit des Ganzen mehr befördern werde. Nun sehen wir deutlich ein, daß sowohl Dieses als Jenes gewisse Vortheile und gewisse Nachtheile habe; wo aber das Uebergewicht sey, läßt sich, nach der Natur dieser Folgen, von unserem menschlichen, ja überhaupt von jedem endlichen Verstande nicht bestimmen. Wird nämlich festgesetzt, daß auch erfolgte Besserung die Strafe der vorhin begangenen Sünden nicht aufheben solle: so hat dieß einerseits den Vortheil, daß der Abschreckungsgrund vor der Sünde verstärkt, und dem leichtsinnigen Gedanken: Du kannst dich ja bessern, und dann ist Alles wieder gut! -- aller Zugang abgeschnitten wird. Andererseits hat es aber den Nachtheil, daß sich der Muth und die Lust zur Besserung bei demjenigen, der einmal gesündiget hat, vermindert; daß sich leicht Verzweiflung einfinden kann; daß endlich die Menge der Strafen, \dh\ die Menge der Schmerzen und des Elends in der Welt, um so viel größer ausfällt.~\RWSeitenw{86}\ Wer wollte nun entscheiden, ob jene Vortheile, oder ob diese Nachtheile größer seyn werden? Müßte man nicht, um dieß beurtheilen zu können, wissen, bei welchen Menschen die vortheilhaften, bei welchen dagegen die nachtheiligen Wirkungen, und in welchem Maße sie bei jedem eintreten werden? Kann dieß ein Anderer als der Allwissende? -- Wenn aber kein endliches Wesen hierüber zu entscheiden vermag: so dürfen wir eben deßhalb nicht klagen, daß die Behauptung des Christenthums, Vergebung sey möglich, der Vernunft widerspreche.
\item Wird aber die Möglichkeit einer Vergebung überhaupt zugegeben: so kann man die zweite Behauptung, daß die Leiden Christi ein Mittel zur Herbeiführung dieser Vergebung seyen, einmal schon darum nicht für ungereimt erklären, weil wir gewiß nicht alle Veränderungen, die eine nähere oder entferntere Folge der Leiden Christi sind, kennen. Gesetzt also, wir würden auch nicht im Geringsten begreifen, wienach die Leiden Christi zur Herbeiführung dieser Vergebung etwas beitragen konnten: so würde uns dieses doch nicht berechtigen, jene Behauptung als eine Ungereimtheit zu verwerfen. In der That aber zeigen sich uns bei einer näheren Betrachtung mehrere Gründe, die Gott bestimmen konnte[n], uns eine Vergebung der Sünden um der Leiden Christi willen angedeihen zu lassen. Ein genaueres Nachdenken zeigt, daß eine solche Sündenvergebung, die nebst der Besserung des Sünders noch die Leiden und den Tod Christi für nothwendig erklärt, vor einer andern, die der bloßen Besserung allein versprochen wird, verschiedene Vortheile voraus hat. Diese Vortheile, können wir also denken, vermehren die Summe der guten Folgen, welche die Sündenvergebung hat, so sehr, daß nun die Nachtheile derselben von ihnen überwogen werden. Wir wollen hier nur die wichtigsten berühren:
\begin{aufzc}
\item Gerade dadurch, daß Gott die Besserung nicht allein für hinreichend erklärte, sondern noch die Leiden und den Tod Christi forderte, erhielt die menschliche Natur Jesu Gelegenheit, das größte und verdienstvollste Werk, das je ein Sterblicher vollzogen hat, in Ausführung~\RWSeitenw{87}\ zu bringen, und eben hiedurch sich auf den höchsten Gipfel sittlicher Vollkommenheit zu schwingen, und sich der erhabensten Krone der Seligkeit, die einem Sterblichen je als Belohnung zukommen kann, würdig und theilhaftig zu machen. Er bot sich an, die Leiden, welche das ganze menschliche Geschlecht um seiner Sünden willen verdiente, auf sein unschuldiges Haupt zu laden; und leerte ihn auch wirklich aus, diesen so bitteren Leidenskelch, obgleich ihm der bloße Gedanke daran einen Schweiß, gleich Blutstropfen, erpreßte! -- 
\item Das Beispiel der Tugend, welches Jesus uns aufstellen sollte, erhielt die größte Brauchbarkeit für uns. Sollte uns das Beispiel Jesu recht lehrreich werden: so war es nicht genug, daß Jesus nur die höchste Stufe sittlicher Vollkommenheit ersteige; sondern er mußte auch in solche Verhältnisse versetzt werden, welche die schwierigsten für alle Menschen sind, in denen wir uns am Allerwenigsten zu rathen wissen, und am Geneigtesten zur Verzweiflung sind. Dergleichen Verhältnisse sind nun Leiden, besonders solche, die uns von bösen Menschen zugefügt werden. Also mußte auch Jesus Leiden von aller Art, Verfolgungen und selbst die schimpflichste Todesart am Kreuze bestehen, wenn uns sein Beispiel Beides, zur Lehre und zum Troste in jeder Lage des Lebens dienen sollte.
\item Durch jene Leiden, die Jesus erduldete, werden wir auf's Gewisseste überzeugt, daß er kein Betrüger, sondern ein wahrhaft göttlicher Gesandte gewesen sey. Wenn er nie gelitten hätte, wenn er in einem ununterbrochenen Wohlseyn bis an sein Ende geblieben wäre: so könnten wir glauben, er habe vielleicht aus bloßem Eigennutz, oder aus Eitelkeit und Ruhmsucht die Rolle eines göttlichen Gesandten übernommen, und seltene Klugheit habe ihn in den Stand gesetzt, sie mit so gutem Glücke zu spielen. Allein da er nichts als Verfolgungen erfuhr, und gleich beim Anfange seines öffentlichen Lehramtes voraussagte, daß er des schmerzlichsten und schmählichsten Todes sterben werde; da er~\RWSeitenw{88}\ seine Standhaftigkeit bis auf den letzten Augenblick behielt, und mit den Worten: \erganf{Vater! in deine Hände empfehle ich meinen Geist!} verschied: wer kann noch zweifeln, daß er die Wahrheit geredet, daß er fest überzeugt gewesen sey von Allem, was er lehrte, und daß sich außerordentliche Dinge mit ihm zugetragen haben. Wenn diese beiden Vortheile (\RWgriech{b} und \RWgriech{g}) zunächst nur unsere Tugend vollkommener machen; so tragen sie hiedurch schon mittelbar bei, die Lehre von der Vergebung der Sünden unschädlicher zu machen.
\item Wenn die Vergebung der Sünden an die Verdienste Christi gebunden wird: so haben eigentlich nur Christen allein, und zwar nur solche, die sich nach allen Grundsätzen des Christenthumes richten, Hoffnung auf diese Vergebung. Nun ist aber leicht zu begreifen, daß jene Mißbräuche, denen die Hoffnung der Vergebung ausgesetzt ist, bei Christen, im Ganzen genommen, seltener Statt finden müssen, als bei Nichtchristen. Die christlichen Völker stehen doch bekanntlich auf einer höheren Stufe der Bildung; sie werden also deutlicher, als andere, einsehen, daß es nicht nur die Pflicht, sondern auch selbst der eigene Vortheil erheische, die Sünde zu meiden, obwohl sie wieder vergeben werden kann; sie haben in ihrer eigenen Religion verschiedene kräftige Abhaltungsgründe vom Bösen, \zB\ den Glauben, daß eine einzige, mit Vorsatz begangene und nicht wieder abgebüßte Sünde eine ewige Verdammniß zuziehe; sie haben viel mehrere Mittel zur Tugend, \zB\ in ihrer vortrefflichen Buß- und Besserungsanstalt, \usw\ Aus allen diesen Umständen läßt sich erwarten, daß die Christen die Lehre von der Vergebung weit weniger mißbrauchen werden, als andere Menschen; und daraus wird begreiflich, wie hier die Vortheile derselben die \Ahat{Nachtheile}{Vortheile} überwiegen können.
\item Endlich gibt es noch mehrere andere Vortheile dieser Anstalt, die wir bei der Betrachtung des sittlichen Nutzens dieser Lehre anführen werden, weil sie zugleich dahin gehören.~\RWSeitenw{89}
\end{aufzc}
\end{aufzb}
\RWbet{1.~Einwurf.} Entweder Besserung ist zur Vergebung der Sünden allein hinlänglich oder nicht. Im ersten Falle waren die Leiden Christi überflüssig; im zweiten unzulänglich. Die Lehre von der Erlösung durch Christum ist also in jedem Falle ungereimt.\par
\RWbet{Antwort.} In diesem Einwurfe vergißt man, daß der Umstand, ob Christus gelitten oder nicht gelitten habe, selbst etwas an der Alternative, ob Besserung zur Vergebung der Sünden hinlänglich oder nicht hinlänglich seyn soll, ändern könne. Die Besserung nämlich ist zur Vergebung der Sünden nur dann unzulänglich, wenn Christus nicht für uns gelitten hat; sie wird es aber, wenn er gelitten hat. Und also sind seine Leiden nicht überflüssig.\par
\RWbet{2.~Einwurf.} Verdienst und Schuld sind etwas Persönliches, das von dem Einen schlechterdings nicht auf den Andern übertragen werden kann; aus gleichem Grunde können auch Belohnungen und Strafen nicht übertragen werden. Nach der Lehre von der Erlösung aber hätte uns Jesus von den Strafen, die wir verdient haben, dadurch befreit, daß er sie selbst getragen hat; es müßte also eine Uebertragung der Strafen Statt finden, was ungereimt ist.\par
\RWbet{Antwort.} Verdienst und Schuld sind allerdings etwas Persönliches, \dh\ sie können nicht in dem Sinne des Wortes übertragen werden, daß derjenige, von dem man bildlicher Weise sagt, daß ein gewisses Verdienst, oder eine gewisse Schuld auf ihn übergehe, im ersten Falle eine Belohnung als eine eigentliche Belohnung genieße (\dh\ als ein Glück, das ihm nur um seines in ihm allein liegenden Grundes willen zu Theil wird), im zweiten Falle die Strafe als eine eigentliche Strafe erleide (\dh\ als ein Uebel, das ihm nur um eines in ihm allein liegenden Grundes willen zugefügt wird); sondern jene Belohnung genießt er als ein Glück, das ihm um eines Andern willen zugedacht wurde, und diese Strafe erfährt er als ein Leiden, das ihm vergolten werden soll. Wenn also der Apostel (\Ahat{\RWbibel{2\,Kor}{2\,Kor.}{5}{21}}{5,19.}) sich ausdrückt: Jesus sey für uns zum Sünder geworden: so will er hiemit nur sagen: Jesus habe Leiden, ähnlich denjenigen, welche ein Sünder verdient, gelitten; doch nicht als Strafe, sondern als Lei\RWSeitenw{90}den, die ihm vielmehr sehr überschwenglich vergolten werden sollten; diese That Jesu aber hatte die Gottheit bestimmt, uns übrigen Menschen gewisse Leiden, die wir in einer Rücksicht sehr wohl verdient hätten, nachzulassen. Hierin liegt nun nichts Ungereimtes, indem wir schon vorhin gezeigt, daß in den Leiden Jesu wirklich mancher vernünftige Bestimmungsgrund zu dieser Anordnung Gottes liegen konnte.\par
\RWbet{3.~Einwurf}. Die Leiden Christi können unmöglich ein taugliches Mittel seyn, um uns Vergebung auszuwirken, da wir bei ihnen gar nicht selbstthätig mitwirken, sondern bloß müßige Zuschauer machen. Ein taugliches Versöhnungsmittel kann nur ein solches seyn, das unsere eigene Thätigkeit in Anspruch nimmt.\par
\RWbet{Antwort.} Und dieses thut auch das Versöhnungsmittel, welches das Christenthum uns lehrt; denn überdieß, daß es die eigene Besserung von uns fordert, verlangt es auch noch, daß wir mit doppelter Wachsamkeit uns vor dem Rückfalle hüten, dessen Bestrafung dann auch um so härter seyn würde; daß wir auf's Eifrigste bemüht seyn sollen, jeden durch die Sünde angerichteten Schaden wieder gut zu machen; daß wir die Leiden Jesu oft mit gerührter Dankbarkeit betrachten; daß wir uns allen denjenigen Handlungen unterziehen, die in den beiden Sacramenten der Taufe und Buße vorgeschrieben werden; daß wir mit Einem Worte alles dasjenige thun, was wir nur immer vermögen, um uns der Wohlthat der Vergebung würdiger zu machen.\par
\RWbet{4.~Einwurf.} Die Lehre von der Versöhnung durch Christum stellt uns Gott als ein sehr ungerechtes, grausames und in seinem Zorne blind strafendes Wesen vor. Als ein sehr ungerechtes Wesen; denn er verursacht demjenigen, der nichts verbrochen hat, unsägliche Leiden und Qualen. Als ein grausames Wesen; denn er läßt sich nicht anders, als durch Blut versöhnen, und zwar nicht etwa durch das Blut der Rinder und Böcke, nein, nur durch das Blut seines vielgeliebten einzigen Sohnes. Hieraus ersieht man zugleich, daß er blind straft in seiner Wuth, er hebt den Arm zur Strafe auf, er fällt, gleichviel auf wen, -- er sieht Blut fließen, und ist versöhnt.~\RWSeitenw{91}\par
\RWbet{Antwort.}
\begin{aufzc}
\item Es war nicht ungerecht von Seite Gottes, wenn er dem Menschen Jesu, der sich freiwillig anbot, für die Verbrechen der Menschen zu leiden, die edle Bitte gestattete, und ihm sein Leiden dann durch eine unendlich höhere und ewig fortdauernde Seligkeit vergalt.
\item Grausam würde Gott nur erscheinen, wenn er das Blut Jesu Christi ohne Noth und Nutzen, bloß zu seinem eigenen Vergnügen hätte fließen lassen; aber so ist es nicht, sondern, wie wir schon oben gesehen, so hatten die Leiden Jesu die größten Vortheile für uns.
\item Und eben darum läßt sich auch nicht sagen, daß Gott blind strafe in seiner Wuth; nicht Hitze und Uebereilung, sondern höchst weise und von Ewigkeit her gefaßte Rathschlüsse, die größte Liebe und Barmherzigkeit, sprechen sich in der Lehre von der Versöhnung durch Jesum Christum aus.
\end{aufzc}
\RWbet{5.~Einwurf.} Die Lehre von Christi stellvertretender Genugthuung widerspricht der Schrift selbst; denn \RWbibel{Ez}{Ezechiel}{18}{20}\ heißt es: \erganf{Die Seele, welche sündiget, soll sterben, der Gerechte selbst soll den Lohn seiner Gerechtigkeit, und der Ruchlose selbst den Lohn seiner Ruchlosigkeit empfangen.} Diese vernünftige Lehre des alten Bundes hebt nun der neue auf.\par
\RWbet{Antwort.} Er hebt sie nicht auf, sondern er sagt nur, daß ein Tugendhafter auf eine kurze Zeit leiden, ein Sünder aber, der sich gebessert hat, um jener Leiden willen die Nachlassung der Strafen erhalten könne. Wo ist wohl der geringste Widerspruch zwischen dieser und der Behauptung bei Ezechiel?\par
\RWbet{6.~Einwurf.} Aber Ezechiel und überhaupt der ganze alte Bund weiß nichts von einer stellvertretenden Genugthuung, sondern hält eigene Besserung für hinlänglich.\par
\RWbet{Antwort.} Isaias (\RWbibel{Jes}{}{53}{}) sagte die Genugthuung, die der Messias leisten werde, sehr deutlich und bestimmt vorher. Wenn übrigens andere Bücher des alten Bundes nichts hievon wissen sollten, oder wenn in denjenigen Stellen, welche die Kirche auf diese stellvertretende Genugthuung bezogen hat, nach ihrem buchstäblichen Sinne wirklich von etwas Anderem die Rede wäre: so wäre dieß doch kein Einwurf gegen diese Lehre; indem ja nicht nothwendig ist, daß eine jede Lehre~\RWSeitenw{92}\ des Christenthums schon in der alttestamentalischen Offenbarung enthalten sey; indem auch ferner es zur Unfehlbarkeit der Kirche gar nicht gehört, zu wissen, welchen Sinn diese oder jene Stelle der Bücher des a.\,B., an und für sich betrachtet, habe, sondern nur, in welchem Sinne sie genommen werden müsse, wenn das Lesen jener Bücher für uns erbaulich werden soll.\par
\RWbet{7.~Einwurf.} Jesus hatte nicht einmal das Recht, sein Leben zu einem Sühnopfer für die Sünden der Welt Gott anzubieten; denn Niemand hat das Recht, mit seinem Leben nach Belieben zu schalten und zu walten. Es ist eine Art von Selbstmord, den er beging, wenn er sich freiwillig in den Tod begab.\par
\RWbet{Antwort.} Wenn durch die Aufopferung des Lebens ein Vortheil für das Ganze erreicht werden kann, der dieses Opfers werth ist, und wenn es kein anderes Mittel gibt, durch welches dieser Vortheil eben so vollkommen erreicht werden kann: so ist der Entschluß, sein Leben aufzuopfern, nicht nur kein Selbstmord zu nennen, sondern die edelste und großmüthigste Heldenthat. Und gerade dieses ist der Fall bei Jesu. Wenn er gesagt, daß er sein Leben freiwillig hingebe, daß es ihm Niemand mit Gewalt entreiße: so wollte er hiedurch nur anzeigen, daß es sein freier Entschluß sey, sich für das Wohl der Menschheit aufzuopfern, und daß es ihm, nicht nur wenn er die Wunderkraft, die er besitze, gebrauchen, sondern selbst durch natürliche Mittel, wenn er von seinem Zwecke nur abstehen wollte, ein Leichtes seyn würde, sein Leben zu erhalten, und glückliche Tage zu sehen.\par
\RWbet{8.~Einwurf.} Wie kann man aber sagen, daß Christus für uns genug gethan hat, da er doch gar nicht die nämlichen Strafen, die wir verdient haben, erlitten hat? Er hat nur endliche irdische Leiden erfahren, wir aber hätten, wie das Christenthum behauptet, die endlosen Strafen der Hölle verdient.\par
\RWbet{Antwort.} Man macht sich einen sehr unrichtigen Begriff von der Genugthuung, wenn man vermeint, daß sie nur dort Statt finde, wo ein Anderer eben dasselbe gethan oder erlitten hat, was Jemand hätte thun oder leiden sollen.~\RWSeitenw{93}\ Die Nothwendigkeit einer so verstandenen Genugthuung läßt sich auf keine Weise darthun, und ist auch von der Kirche nie behauptet worden. Wenn diese sagt, daß Christus für uns genug gethan habe; so will sie hiemit nicht etwa sagen, er habe Gleiches, sondern er habe Gleichgeltendes gethan und erlitten, \dh\ er habe so viel gethan und erlitten, als eben nöthig war, um uns Vergebung auszuwirken.\par

\vabst \textbf{B.}~Daß Christus uns nicht nur von unseren eigenen Sünden, sondern \RWbet{auch von der Erbsünde befreit} habe, wird Niemand anstößig finden, der den Begriff der Erbsünde gehörig aufgefaßt hat. Sie ist ein Grund in uns, weßhalb uns Gott, ohne ungerecht zu seyn, strenger behandeln darf, als er zu thun für gut befunden hätte, wenn Adam nie gesündiget haben würde. Nun ist es, wie wir schon oben gezeigt, nicht eben so unbegreiflich, daß Gott bei jenen Menschen, welche die Lehre des Christenthums kennen lernen, von jener strengeren Behandlung in etwas abgehen könne.\par

\vabst \textbf{C.}~Er hat sich \RWbet{freiwillig geopfert.} Allerdings; er war ja ein freier Mensch, und wenn er zu dieser Aufopferung gezwungen gewesen wäre, so fiele alles Verdienstliche derselben von seiner Seite weg.\par

\vabst \textbf{D.}~Er litt nicht bloß für einige, sondern \RWbet{für alle Menschen}. Diese Behauptung will nichts Anderes sagen, als daß von Seite des Sohnes Gottes und der menschlichen Natur Jesu kein Grund obwalte, weßhalb ein oder der andere Mensch der Früchte der Erlösung nicht theilhaftig wird, oder noch deutlicher: Der Sohn Gottes that Alles, was an sich möglich war, um einen jeden Menschen in den Genuß der Früchte der Erlösung zu versetzen, und der Mensch Jesus wirkte nicht nur getreulich zu diesem Zwecke mit, sondern es war auch der heißeste Wunsch seines Herzens, daß, wo möglich, kein einziger Mensch verloren gehe. Wer sollte dieß Alles nicht vernunftmäßig finden?\par

\vabst \textbf{E.}~Wir sollen uns bildlicher Weise vorstellen \usw\ Eben weil dieß nur bildlicher Weise geschehen soll, so enthält diese Forderung nichts Ungereimtes.\par
\item Christus hat uns die \RWbet{Anwartschaft auf den Himmel} und Gottes Wohlgefallen wieder verschafft. Wenn~\RWSeitenw{94}\ uns die Menschwerdung des Sohnes Gottes, sein Unterricht, \usw\ zu besseren Menschen gemacht hat: so ist es auch begreiflich, wie wir die Anwartschaft auf den Himmel wieder erhalten können.\par
\item Er wirkt auch noch gegenwärtig wohlthätig für uns fort. Sobald wir ein anderes Leben und eine Verbindung derjenigen, die uns vorangegangen sind, mit uns, die wir noch übrig sind, glauben: so ist dieß sehr begreiflich.
\end{aufza}

\RWpar{211}{Sittlicher Nutzen}
\begin{aufza}
\item Der Umstand, daß \RWbet{die katholische Religion durch den menschgewordenen Sohn Gottes selbst gegründet} worden ist, muß ihre Wichtigkeit in unseren Augen vermehren. \erganf{Zu verschiedenen Zeiten und auf mannigfache Art hat Gott ehemals zu den Vätern durch die Propheten geredet; in den letzten Tagen hat er zu uns geredet durch seinen Sohn} (\RWbibel{Hebr}{Hebr.}{1}{1}). Wer muß hieraus nicht schließen, daß die Lehre des Sohnes unendlich wichtiger sey, als jene des Propheten? Hiezu kommt noch, daß wir um so gewisser werden, überall die reine Wahrheit zu vernehmen. Alle andern Menschen, die mit dem Sohne Gottes in keiner so innigen Verbindung, als Jesus, gestanden, waren nicht schlechterdings unfehlbar in ihrem Unterrichte; sondern nur in denjenigen Stücken, welche den Menschen beizubringen Gott sie beauftragt hatte, deren Wahrheit sie durch Wunder und Zeichen bewiesen, verdienten sie Glauben. Oft aber ist die Grenzlinie, wie weit die Zeugenschaft eines Wunders reiche, schwer zu bestimmen. Wie wohlthätig also für uns, wenn wir bei Jesu keine solche Grenzlinie bedürfen! -- Die Behauptung, daß die katholische Kirche bis an das Ende des menschlichen Geschlechtes fortdauern, und einst der Antheil aller Menschen werden solle, ist gleichfalls geeignet, uns einen höheren Begriff von ihrer Wichtigkeit und von ihren Vorzügen beizubringen.
\item Daß wir an dem \RWbet{Beispiele Jesu ein Muster zur Nachahmung} erhielten, ist eine der größten Wohlthaten, die eine Offenbarung uns nur immer leisten kann.~\RWSeitenw{95}
\begin{aufzb}
\item Wie sehr bedürfen wir nicht eines solchen Musters für's Erste schon darum, um verwickelte Fragen, die es auch in der Sittenlehre gibt, richtig zu entscheiden.
\item Und wenn wir bereits wissen, was unsere Pflicht in einem gegebenen Falle sey; wie sehr erleichtert uns nicht das Andenken an das Vorbild Jesu die wirkliche Erfüllung! Hier nämlich können wir auf die Möglichkeit der Tugend aus ihrer Wirklichkeit schließen, und alle Einwürfe und Zweifel fallen mit einem Male weg.
\item Und welchen Antrieb haben wir nicht, diesem erhabenen Muster nachzuahmen! Von einer Seite wird diese Nachahmung uns als eine bestimmte Pflicht vorgestellt, ohne deren Befolgung wir uns nicht schmeicheln dürfen, Christen zu seyn. Von der andern Seite, wie aufmunternd, erhebend und ehrenvoll ist nicht der Gedanke, Jesu ähnlich zu werden! Und welch ein Preis ist nicht auf diese Aehnlichkeit gesetzt! Denn dieses wissen wir (sagt der Apostel \RWbibel{Röm}{Röm.}{6}{5}), sofern wir Christo in seiner Erniedrigung (auf Erden) ähnlich werden, so werden wir ihm auch ähnlich in seiner Auferstehung.
\item Dieß Muster ist endlich auch ein solches, das wir nie ganz erreichen werden, das eben darum unseren Eifer nie wird erkalten lassen, bei dem wir nie stolz und selbstgefällig werden können, weil wir bei jeder Vergleichung fühlen, wie sehr wir noch unserem Vorbilde nachstehen.
\end{aufzb}
\item \mbox{}\\
\textbf{A.}~Christus hat uns durch seine Leiden und seinen Tod die \RWbet{Vergebung unserer Sünden} ausgewirkt. Diese Lehre gewährt folgende Vortheile von größter Wichtigkeit:
\begin{aufzb}
\item Die Frage von der Vergebung der Sünden, deren Unentschiedenheit in der natürlichen Religion uns sehr beunruhigen könnte, erhält nun eine bestimmte und für uns überaus tröstliche Entscheidung. Ja, ist die Antwort Gottes, eure Sünden werden euch vergeben, wenn ihr euch aufrichtig bessert, \usw\
\item Die christliche Religion gewinnt an Werth und Wichtigkeit in unseren Augen, indem es heißt, daß nicht allen Menschen, sondern den Christen nur die wirkliche Vergebung ihrer Sünden durch den Tod Jesu zu Theil wird.~\RWSeitenw{96}
\item Die Größe der Sünde zeigt sich uns in dem grellesten Lichte, indem es heißt, daß Besserung allein nicht hinreichend ist, die Strafen, welche wir für sie verdient haben, aufzuheben.
\item Aus gleichem Grunde werden wir auch von Gottes Gerechtigkeit und Heiligkeit lebhafter überzeugt. Bei der entgegengesetzten Vorstellung würde uns Gott als ein Wesen erscheinen, welches nicht strenge genug für die Befolgung des Sittengesetzes wacht, sondern durch seine Barmherzigkeit selbst Ursache ist, daß es so häufig übertreten wird.
\item Wenn Besserung allein hinreichend wäre zur Vergebung der Sünden, so würde sich unvermerkt der Irrthum bei uns einschleichen, daß wir mehr thun können, als wir sollen.
\item Von einer Seite soll der Gedanke: Du sollst alles Gute, was du zu thun vermagst, auch wirklich thun, lebendig erhalten, und von der andern Seite doch auch die Beunruhigung gehoben werden: Du kannst das Geschehene nie wieder ungeschehen machen. Dieses bewirkt nun die Lehre von der Erlösung, indem sie uns zuruft: Was bei den Menschen unmöglich ist, das vermag Gott; der Sohn Gottes hat in der That für euch genug gethan.
\item Auf diese Art werden wir auch von Gottes unendlicher Liebe und Barmherzigkeit für uns auf's Innigste überzeugt.
\item Die Leiden Jesu schon an sich selbst, besonders aber, wenn wir erwägen, daß er sie nur eben um unsertwillen erfahren hat, erfüllen uns mit einer gewissen dankbaren Liebe zu ihm. Diese Liebe zu ihm aber ist auch zugleich Liebe zur Tugend und Liebe zu Gott, weil ja in Christo Beides, die Tugend sowohl, als auch die Gottheit selbst in sichtbarer Gestalt erschienen ist. Wer also Jesum lieb gewinnt, der wird gewiß auch bald dem Laster abgeneigt, liebet die Tugend, und ehret Gott.
\item Besonders aber wird sich ein Solcher hüten, irgend etwas zu thun, wodurch er die Früchte des Todes Jesu an sich vereiteln könnte. Nein, nicht umsonst (wird er mit zärtlich bewegtem Herzen sprechen), nicht umsonst soll der~\RWSeitenw{97}\ Beste aller Menschen gelitten haben; nicht umsonst soll sein kostbares Blut am Stamme des Kreuze geflossen seyn! Ich will mich seiner würdig betragen, er hat es sehr um mich verdient, hat mich um einen theuren Preis erkauft!
\item Aus dieser Lehre ersehen wir endlich auch in einem Beispiele, welche erstaunungswürdige Kräfte der Mensch besitze, und welch einen hohen Werth das Leiden und die Aufopferungen, die man zum Besten Anderer auf sich nimmt, in Gottes Augen haben. Ein einziger Mensch hat das ganze menschliche Geschlecht durch seinen Tod errettet. Die Leiden, welche er freiwillig übernahm, hatten in den Augen Gottes einen so hohen Werth, daß er dem ganzen menschlichen Geschlechte um dieses Einen willen die sonst verdiente Strafe nachließ.
\end{aufzb}\par
\RWbet{Einwurf.} Aber vielleicht wird diese Lehre uns nur leichtsinnig machen?\par
\RWbet{Antwort.} Auf diesen Einwurf gilt dieselbe Antwort, die wir schon bei der Lehre von Gottes Barmherzigkeit gaben. Wir selbst können uns, wenn wir nur wollen, wenn wir die Lehren des Christenthums nur in ihrem Zusammenhange kennen lernen, vor einer so nachtheiligen Wirkung derselben sicher stellen.\par

\vabst \textbf{B.}~Zu wissen, daß wir durch Christum \RWbet{auch von der Erbsünde befreit} sind, richtet uns auf, und bewirkt den heilsamen Vorsatz in uns, da sich nichts mehr Sträfliches, nichts mehr Mißfälliges, nichts mehr, das unrein heißen könnte, an uns befindet, auch mit doppelter Wachsamkeit dafür zu sorgen, daß wir die wieder erhaltene Unschuld und Reinigkeit des Herzens durch keine neue Sünde beflecken.\par

\vabst \textbf{C.}~Die Lehre, daß Jesus Christus \RWbet{für alle Menschen gestorben}, macht uns die Größe seiner Liebe anschaulich, erfüllt uns mit einem heiligen Eifer, daran zu arbeiten, daß doch auch Niemand verloren gehe. Wie müssen wir uns nicht hüten, Jemand zu ärgern, und die Ursache seines Unterganges zu werden, da Jesus Christus für ihn am Kreuze gestorben ist.~\RWSeitenw{98}\par

\vabst \textbf{D.}~Er hat es \RWbet{freiwillig, aus bloßem Gehorsam gegen den Vater}, gethan. Wenn wir nicht glaubten, daß er es freiwillig gethan: so würde eben darum alles Verdienst bei seiner That wegfallen. Entschloß er sich aber aus freien Stücken dazu: so war wieder kein Beweggrund lehrreicher, und nachahmungswürdiger für uns, als der: Es ist der Wille des Vaters! Selbst der Beweggrund: Aus Liebe zur Menschheit! ob dieser gleich durch jenen nicht ausgeschlossen wird, wäre doch weniger erbaulich, als der aus Liebe zu Gott.\par

\vabst \textbf{E.}~Wir sollen uns bildlicher Weise vorstellen, \usw\ Wir bedürfen, wie schon erwähnt worden ist, eines sinnlichen Bildes von Gott, und da ist in der That keines zweckmäßiger, keines geeigneter, die Empfindungen der Liebe, der Ehrfurcht und des Vertrauens in uns zu wecken und zu erhalten, als eben dasjenige, das uns das Christenthum hier aufstellt: Denket euch Gott als euren liebevollen Vater! Denket euch, ihr habt ihn beleidiget durch euere Sünden; und weil es kein anderes Mittel gibt, um euch zu retten: so sendet er seinen eingebornen Sohn, der eure menschliche Natur an sich nimmt, um euch durch seinen Tod erlösen zu können! Denket euch, ein König würde sein Volk so zärtlich lieben, daß er selbst seinen Sohn dahin gäbe, um es zu retten. Wie viele Gegenliebe, wie viele Dankbarkeit, wie viel Vertrauen und Gehorsam ein solcher König verdiente, so viele Gegenliebe, so viele Dankbarkeit, so viel Vertrauen und Gehorsam verdient auch Gott!
\item Christus hat uns die Anwartschaft auf den Himmel und Gottes Wohlgefallen wieder verschafft. Wir begreifen leicht, daß Jesus uns auch so manche irdische Güter hätte auswirken können. Da er dieß nicht gethan, sondern nur jenes Ueberirdische, den Himmel, für uns erbeten: so lernen wir hieraus, wie wichtig dieses Gut sey, und wie sehr es alles Irdische übertreffen müsse.
\item Er \RWbet{wirkt noch gegewärtig für uns}. Wie Vieles also können wir nicht von seinem Beistande hoffen, nachdem er selbst gesagt hat: \erganf{Nun ist mir alle Gewalt im Himmel und auf Erden übergeben!}~\RWSeitenw{99}
\end{aufza}

\RWpar{212}{Wirklicher Nutzen}
Es gibt nur zweierlei Schaden, den diese Lehre verursacht hat.
\begin{aufza}
\item Gab sie, besonders im eilften Jahrhunderte, zu manchen Streitigkeiten Anlaß. Schon in den früheren Jahrhunderten hatten sich mehrere Kirchenväter (\zB\ Gregor von Nazianz, Augustinus, Ambrosius, Chrysostomus, \usw ) zuweilen des Ausdruckes einer stellvertretenden Genugthuung, welche Christus für uns geleistet habe, bedienet; aber erst \RWbet{Anselmus} (Erzbischof von Kanterbury im 11ten Jahrhunderte) bildete diesen Begriff vollständiger aus, fand jedoch an seinem eigenen Schüler, dem berühmten \RWbet{Abälard,} einen sehr heftigen Gegner. Im 16ten Jahrhunderte trat \RWbet{Faustus Socinus} auf, und verwarf, nebst allen übrigen Geheimnißlehren des Christenthums, auch die Lehre von der Erlösung durch Jesum Christum. Die einzige Bedeutung, in der er sie zugeben wollte, war, daß Jesus uns durch seine Lehre gebessert, und durch die Besserung uns der Vergebung würdiger gemacht habe. Alle Stellen der heil.\ Schrift, in welchen von der Vergebung der Sünden gesprochen wird, suchten er und seine zahlreichen Anhänger auf diese Art zu deuten. Der Nachtheil nun, den die Verketzerung dieser Personen gehabt hat, ist, um wenig zu sagen, gewiß schon durch den Vortheil aufgewogen worden, den die wissenschaftliche Zerlegung der Begriffe von Schuld, Genugthuung, \usw\ für die Bildung des Geistes gehabt hat.
\item Weit wichtiger ist der Schaden, den diese Lehre bei Vielen durch Mißverstand und Leichtsinn hervorgebracht hat, indem sie um so sorgloser fortsündigen zu können glaubten, je leichter sie mittelst der Verdienste Jesu Vergebung ihrer Sünden zu finden hofften. Dieß ging so weit, daß Einige unter den protestantischen Theologen allen Ernstes behaupteten, die Sünder seyen Gott noch angenehmer, als die Gerechten, weil er an jenen mehr Gelegenheit habe, seine unendliche Huld und Barmherzigkeit zu zeigen, als an diesen. Aber so groß auch dieser Schaden war (er ist leicht der größte, den irgend eine Lehre des Christenthums hervorgebracht hat):~\RWSeitenw{100}\ so wird doch Niemand beweisen, ja auch nur wahrscheinlich machen können, daß er den Nutzen überwogen habe, den eben diese Lehre bei einer unzähligen Menge von Menschen hervorgebracht hat, die nur in ihr allein Lust, Muth und Kraft zu ihrer Besserung und Gewissensruhe in ihrem gebesserten Zustande fanden.
\end{aufza}

\RWpar{213}{Die Lehre des Christenthums von dem Einflusse des heiligen Geistes}
Auch die \RWbet{dritte Person in Gott oder der heil.\ Geist} wirkt, und hat von Anbeginn schon auf unser menschliches Geschlecht gewirkt.
\begin{aufza}
\item Er wirket auf \RWbet{unser Erkenntnißvermögen} sowohl, als auch auf \RWbet{unsern Willen}; und dieß zwar oft übernatürlicher Weise, \dh\ auf Wegen, von deren Vorhandenseyn wir nur durch die Offenbarung wissen; doch ohne hiebei je unsere Freiheit zu verletzen.
\item Er ist es, dem so viele gemeinnützige Entdeckungen in allen Wissenschaften, besonders aber in religiösen Gegenständen, so viele heilsame Gedanken, edle Entschließungen, erhabene Tugenden, wodurch sich die Menschen aller Jahrhunderte ausgezeichnet haben, zuzuschreiben sind. Er war es vornehmlich, der in den Schriftstellern des a.\,B.\ solche Vorstellungsreihen erweckte, daß sie schrieben, was noch heut zu Tage zu unserer Belehrung, Erbauung und zum Beweise der göttlichen Sendung Jesu Christi dient. Er war es, der auch die menschliche Natur unseres Erlösers bildete, der sie erleuchtete, und mit Tugenden schmückte.
\item Besonders wohlthätig aber wirkt der Geist Gottes seit jener Zeit auf uns Menschen ein, \RWbet{seit der Erlöser ihn nach seiner Himmelfahrt als unseren Vormund oder Sachwalter in Uebereinstimmung mit dem Vater gesendet.}
\item Er ist es, der die \RWbet{Meinungen aller Mitglieder der katholischen Kirche} so leitet und lenket, daß ihr Gesammtglaube nie in einem Irrthume sich vereiniget.
\item Wir Christen dürfen uns \RWbet{bildlicher} Weise sogar vorstellen, daß der \RWbet{Geist Gottes seine Wohnung in}~\RWSeitenw{101}\ \RWbet{unseren Herzen aufgeschlagen} habe, daß unser Leib sein Tempel, und unsere Glieder des Geistes Gottes Glieder sind; dieß Alles nämlich so lange, als wir ihn nicht durch eine vorsätzliche Sünde aus diesem Wohnsitze vertreiben.
\end{aufza}

\RWpar{214}{Historischer Beweis dieser Lehre}
\begin{aufza}
\item Daß der heil.\ Geist auf unser \RWbet{Erkenntnißvermögen} einwirkt, ersehen wir \zB\ daraus, daß der heil.\ Paulus (\RWbibel{1\,Kor}{1\,Kor.}{12}{8--11}) verschiedene nützliche Kenntnisse und Geschicklichkeiten als eben so viele Gaben des heil.\ Geistes aufzählt. Wirket er aber auf unser Erkenntnißvermögen: so folgt schon, daß er mittelbar auch auf unsern \RWbet{Willen} wirke. Dieß erhellet auch daraus, weil die Schrift dem heil.\ Geiste gleicher Weise die Tugenden des Menschen zuschreibt, und von dem Lasterhaften sagt, daß er dem Geiste Gottes widerstehe. Von den verschiedenen Tugenden, die der heil.\ Geist in Menschen hervorbringt, gibt ihm die heil.\ Schrift verschiedene Namen; und nennt ihn \zB\ bald einen Geist der Stärke, bald einen Geist der Liebe, \udgl\ 
\item Der Apostelfürst sagt (\RWbibel{2\,Petr}{2\,Petr.}{1}{21}): \erganf{Keine Weissagung ist je durch bloße menschliche Bemühung vorgebracht worden; sondern die heil.\ Männer Gottes sprachen getrieben vom heil.\ Geiste.}
\item Bei \RWbibel{Joh}{Joh.}{14}{16}\ spricht Jesus: \erganf{Ich werde den Vater bitten, und er wird euch einen andern Sachwalter senden; damit er bei euch bleibe in alle Zeiten, den Geist der Wahrheit.} So auch \RWbibel{Joh}{Joh.}{16}{7}\ \ua\,St.
\item Die historische Richtigkeit dieses Artikels ist bereits gezeigt worden.
\item Der heil.\ Paulus schreibt (\RWbibel{Gal}{Gal.}{4}{6}): \erganf{Weil ihr Kinder Gottes seyd: so hat Gott den Geist seines Sohnes in eure Herzen gesendet.} -- Und (\RWbibel{1\,Kor}{1\,Kor.}{3}{16}): \erganf{Wisset ihr nicht, daß ihr ein Tempel Gottes seyd, und daß der Geist Gottes in euch wohne?} -- Und (\RWbibel{1\,Kor}{1\,Kor.}{6}{19}): \erganf{Wisset ihr nicht, daß euer Leib ein Tempel des in euch wohnenden Geistes ist?}~\RWSeitenw{102}
\end{aufza}

\RWpar{215}{Vernunftmäßigkeit}
\begin{aufza}
\item Gott wirket
\begin{aufzb}
\item auf unser \RWbet{Erkenntnißvermögen}. Denn alle Vorstellungen, die wir, auf was immer für Wegen, erhalten, \zB\ durch den Eindruck äußerer Gegenstände, \usw\ rühren am Ende von Gottes Veranstaltung her. Gott wirket ferner auch
\item auf unser \RWbet{Willensvermögen;}
\begin{aufzc}
\item einmal schon mittelbar dadurch, daß er auf unser Erkenntnißvermögen wirkt; denn unser Wille bestimmt sich nach unseren Vorstellungen und Begriffen. In diesem Sinne wirken auch Menschen, ja auch selbst leblose Gegenstände, auf unsern Willen ein; wie wir denn auch zu sagen pflegen: Dieser Mensch hat mich bewogen, \udgl\  Allein Gott wirket
\item in einem noch ungleich höheren Sinne auf unsere Entschließungen ein; denn er ist es, von dem wir unser ganzes Daseyn, und alle unsere Kräfte, auch unsere Wollkraft selbst haben.
\end{aufzc}
\item Gott wirkt auf unser Erkenntnißvermögen sowohl als auch auf unseren Willen zuweilen auch \RWbet{übernatürlicher Weise.} Da dieser Ausdruck, im Sinne des Christenthums, nichts Anderes sagen will, als daß diese Einwirkungen auf Wegen erfolgen, von deren Vorhandenseyn wir nur durch die Offenbarung wissen: so wird dieß Jeder vernunftgemäß finden. Sicher müssen Gott, dem Allwissenden, auch Mittel und Wege zu einer wohlthätigen Einwirkung auf uns bekannt seyn, deren Vorhandenseyn wir durch unsere bloße Vernunft nicht zu erkennen vermöchten; und die verschiedenen Mittel zu unserer Heiligung, von denen die christliche Offenbarung uns meldet, sind durchaus von der Art, daß sich die Möglichkeit einer wohlthätigen Einwirkung durch dieselben durch bloße Vernunft nicht begreifen läßt; wie dieses tiefer unten bei der Lehre von den christlichen Heiligungsmitteln oder Sacramenten gezeigt werden soll.~\RWSeitenw{103}
\item Daß endlich diese Einwirkungen Gottes auf das Gemüth einzelner Menschen der \RWbet{dritten göttlichen Person} oder dem \RWbet{Geiste} zugeschrieben werden, ist gleichfalls nichts Widersprechendes. Dieß scheinet vielmehr seinen Grund darin zu haben, daß der Geist Gottes (wie wir schon oben bemerkten) überhaupt dasjenige in Gott ist, dem alle solche Wirkungen beigelegt werden, die sich vornehmlich nur auf die Beglückung einzelner Geschöpfe in der Welt beziehen. (S.\ die Dreieinigkeitslehre.)
\end{aufzb}
\item Nach diesen Erklärungen ist es nun sehr begreiflich, daß wir dem heil.\ Geiste alle gemeinnützigen Entdeckungen, gute Entschließungen, Tugenden \usw\ zuschreiben sollen; wie dieses in der gleich folgenden Lehre von der Gnade noch umständlicher gerechtfertiget werden soll.
\item Konnte die Menschwerdung des Sohnes Gottes dem menschlichen Geschlechte überhaupt eine Quelle neuer Wohlthaten werden: so liegt auch darin nichts Ungereimtes, daß die Einwirkungen des heil.\ Geistes seit dieser Zeit, und insbesondere \RWbet{bei den Christen, in einem viel höheren Grade} erfolgen, als bei den Nichtchristen. Die christliche Religion selbst, und die durch sie eingeführten Anstalten, vornehmlich die Heiligungsmittel, können als Mittel und Veranlassung zu diesen häufigeren und wohlthätigeren Einwirkungen des Geistes Gottes dienen.
\item Die Vernunftmäßigkeit dieses Artikels ist bereits an einem andern Orte gezeigt worden.
\item Wenn der Geist Gottes in der That ununterbrochen wohlthätig auf den Geist desjenigen einwirket, der seinen Einwirkungen nicht widersteht: so hat die Redensart: daß er in unserem Geiste wohne, allerdings Wahrheit.
\end{aufza}

\RWpar{216}{Sittlicher Nutzen}
Den Vortheil, den es uns gewährt, zu wissen, daß im Allgemeinen auch die dritte göttliche Person zu unserer Beseligung beitrage, haben wir bereits bei der Dreieinigkeitslehre angeführt.~\RWSeitenw{104}
\begin{aufza}
\item Zu wissen, daß der Geist Gottes insbesondere unseren Verstand erleuchte, um das Gute zu erkennen, und unsern Willen stärke, um das erkannte Gute wirklich auszuführen, hat einen doppelten Nutzen:
\begin{aufzb}
\item Nur wenn wir dieß wissen, öffnen wir unser Herz den Einflüssen des Geistes Gottes, und benützen sie; sind also aufmerksam auf jede neue und wichtige Ansicht, die sich uns darbeut, prüfen dieselbe, und befolgen sie, wenn wir sie richtig finden, mit einer um so größeren Gewissenhaftigkeit, weil wir sie, als bewirkt durch den Geist Gottes, ansehen. Eben so achten wir auch auf jede Regung zum Guten, und hüten uns sorgfältig, sie uns aus dem Sinne zu schlagen.
\item Unsere Hoffnung, es in der Tugend weiter zu bringen, und eben darum auch unsere Lust und unser Eifer gewinnen, wenn wir uns vorstellen, daß der allmächtige Geist Gottes selbst uns beisteht.
\end{aufzb}
\item Dieser Glaube vermehrt unsere Dankbarkeit gegen Gott, unsere Verbindlichkeit, alles Gute und Vortreffliche, das uns die Vorwelt hinterlassen hat, gehörig zu benützen; unser Vertrauen, daß Gott auch uns selbst in allen unsern gemeinnützigen Unternehmungen unterstützen werde.
\item Ein neuer Beweis, wie wohlthätig die Menschwerdung des Sohnes Gottes für uns gewesen. Die Wohlthaten, die wir vom heil.\ Geiste empfangen, danken wir nun zugleich auch dem Sohne, und so wird dieser uns um so unschätzbarer.
\item Der sittliche Nutzen dieses Artikels ist bereits betrachtet worden.
\item Die bildliche Vorstellung, daß der Geist Gottes in uns seinen Tempel aufschlage, erhebt uns, und hält uns mächtig ab von jedem unheiligen Gebrauche dieses Leibes und aller Gliedmaßen desselben, weil wir das Sündliche einer solchen That nun um so lebhafter fühlen. S.\ \RWbibel{1\,Kor}{1\,Kor.}{6}{15--19}
\end{aufza}

\RWpar{217}{Die Lehre des Christenthums von der Gnade}
\begin{aufza}
\item Jede Stufe der Tugend und Glückseligkeit, welche der Mensch ersteigt, auch jede einzelne gute und gottgefällige~\RWSeitenw{105}\ That, die er verrichtet, soll er dem Beistande der \RWbet{Gnade Gottes} verdanken, und Gottes \RWbet{unverdientes} Werk an sich nennen. Beraubt von dieser Gnade vermag er nichts Gutes.
\item Um aber insonderheit auch \RWbet{solche gute Werke} zu Stande zu bringen, \RWbet{die ihn der höheren durch Christum uns erworbenen Seligkeit würdig machen}, bedarf der Mensch einer höheren \RWbet{übernatürlichen} Gnade.
\item Das \RWbet{Maß}, nach welchem der Geist Gottes den Menschen seine Gnade mittheilt, richtet sich \RWbet{nicht etwa nach ihrem bisherigen Tugendgrade}, oder \RWbet{nach dem Gebrauche, den sie von ihren bisherigen Gnaden gemacht}, überhaupt \RWbet{nach keiner uns bekannten Regel,} es sey denn nach jener allgemeinen, welche die möglichst größte Beförderung des allgemeinen Wohles bezwecket. Auch manchem Lasterhaften also werden von Gott sehr große Gnaden zu Theil.
\item Jedoch erhält ein Jeder, der nur Gebrauch von Gottes Gnade machen will, und darum bittet, \RWbet{gewiß so viel, als ihm zu seiner fortschreitenden Besserung und Beglückung hinreicht} (\RWlat{gratiam sufficientem.})
\item Es gibt Einwirkungen der Gnade, die den Erfolg der Besserung \RWbet{erreichen}, und andere, welche ihn \RWbet{nicht erreichen.} (\RWlat{Gratia Dei efficax et inefficax.}) -- Worin der Grund des Unterschiedes zwischen jenen und diesen liege, ist nicht entschieden worden; doch ist es die herrschende Meinung, daß dieser Grund in der \RWbet{Beschaffenheit dieser Gnaden an sich}, somit nicht einzig im Menschen selbst liege.
\end{aufza}

\RWpar{218}{Historischer Beweis dieser Lehre}
\begin{aufza}
\item \RWbibel{1\,Kor}{1\,Kor.}{15}{10}\ schreibt der Apostel von sich: \erganf{Aber durch Gottes Gnade bin ich, was ich bin, und seine Gnade ist in mir nicht fruchtlos gewesen; sondern ich habe mehr gearbeitet als sie Alle, aber nicht sowohl ich, als die Gnade Gottes durch mich.} -- Daß dieses aber auch von allen Menschen gelte,~\RWSeitenw{106}\ zeigt \Ahat{\RWbibel{1\,Kor}{1\,Kor.}{4}{7}}{3,7.}: \erganf{Was hast du, das du nicht empfangen hast? so du es aber empfangen, was rühmst du dich, als ob du es nicht empfangen hättest?} Und \RWbibel{Röm}{Röm.}{9}{16}: \erganf{Es kommt nicht auf das Rennen und Wollen, sondern auf Gottes Gnade an.} \RWbibel{Phil}{Philipp.}{2}{13}\ \RWbibel{Joh}{Joh.}{15}{5}
\item Dieser Punct findet sich zwar nicht ausdrücklich in der heil.\ Schrift vorgetragen; aber die Kirche hat, wenigstens seit den Zeiten des heil.\ Augustinus, ausdrücklich so gelehrt. Auch stimmet hiemit ganz überein die Antwort, welche der Herr (\RWbibel{Mt}{Matth.}{19}{26}) seinen Jüngern ertheilte, als sie die Frage: \erganf{Wer wird denn selig werden können?} an ihn gestellt hatten. \erganf{Bei Menschen}, sprach er (\dh\ durch bloß natürliche Kräfte) \erganf{ist dieses unmöglich; aber bei Gott ist Alles möglich.}
\item Daß sich das Maß der göttlichen Gnadenmittheilungen \RWbet{nicht nach unserem bisherigen Verhalten} richte, zeigt die Stelle \RWbibel{Röm}{Röm.}{11}{6}: \erganf{Ist es aber eine Gnade, so ist es nicht aus den Werken, sonst würde die Gnade nicht mehr Gnade seyn.} \RWbibel{Röm}{Röm.}{9}{18}: \erganf{Er erbarmt sich, wessen er will, und lässet verstockt werden, wen er will.} -- Daß diese Gnadenmittheilungen sich überhaupt nach keiner uns bekannten Regel richten, setzt der Apostel voraus, wenn er eben bei Betrachtung dieser sonderbaren Mittheilung ausruft (\RWbibel{Röm}{Röm.}{11}{33}): \erganf{O Tiefe des Reichthums, der Weisheit und der Erkenntniß Gottes! Wie unbegreiflich sind seine Rathschläge, wie unerforschlich seine Wege! Wer hat den Sinn des Herrn erkannt, oder wer ist sein Rathgeber gewesen?}
\item Daß aber Jeder, der darum bittet, eine \RWbet{hinreichende Gnade} erhalte, ward in der Kirche allezeit geglaubt, und die entgegengesetzte Behauptung des \RWbet{Cornelius Jansenius} (Bischofs zu Ypern) ward im Jahre 1653 vom Papste Innocenz X.\ mit Einstimmung aller Richter verdammt. Der Kirchenrath von Trident hatte schon früher erklärt \RWlat{(sess.\,2.\ cap.\,2.)\RWlit{}{Tridentinum1}: Deus sua gratia semel justificatos non deserit (quo sobrie, juste et pie vivendo proficere possint), nisi ab ipsis prius deseratur.} -- Eben dieses setzen auch so viele Schriftsteller voraus, wo von der Besserung gesprochen, und diese nicht nur als möglich dargestellt,~\RWSeitenw{107}\ sondern denjenigen, welche sie ernstlich wollen, als etwas, das unbezweifelbar eintreten werde, verheißen wird. Ingleichen \RWbibel{Joh}{Joh.}{16}{23}: \erganf{Um was ihr den Vater in meinem Namen bitten werdet, das wird er euch geben.}
\item Daß einige Gnadenwirkungen Gottes \RWbet{ohne Erfolg} bleiben, oder daß es möglich sey, der Gnade Gottes zu widerstehen, zeigt \RWbibel{Mt}{Matth.}{11}{21}, wo Jesus spricht: \erganf{Weh dir, Chorozaim, weh dir, Bethsaida! denn wären solche Wunder zu Sidon und Tyrus geschehen, die bei euch geschehen sind: so hätten sie schon längst in Sack und Asche Buße gethan.} -- Der heilige Stephan sagt von den Juden geradezu (\RWbibel{Apg}{Apostelg.}{7}{51}): \erganf{Ihr Hartnäckigen und Unbeschnittenen am Herzen und an den Ohren! ihr widersetzt euch allezeit dem heil.\ Geiste, wie eure Väter, so auch ihr.} -- \RWbet{Ludwig Molina}, ein Jesuit, im Jahre 1588, behauptete, daß die wirksame Gnade nur dadurch wirksam werde, daß der Wille des Menschen ihr beistimmt. Die \RWbet{Thomisten} und \RWbet{Augustinianer} dagegen behaupteten, der Unterschied zwischen der wirksamen und unwirksamen Gnade liege in der Beschaffenheit der Gnade selbst, die wirksame sey nämlich stärker, oder wohl gar von durchaus anderer Natur.
\end{aufza}

\RWpar{219}{Vernunftmäßigkeit}
\begin{aufza}
\item Da wir alle unsere Kräfte und Fähigkeiten, alle Gelegenheiten, die uns zum Guten nöthig sind, selbst unser Daseyn, Gott zu verdanken haben: so können wir allerdings bildlicher Weise sagen, daß jede Stufe der Vollkommenheit, die wir ersteigen, Gottes Werk in uns sey. Daß aber diese Unterstützung Gottes, die uns in den Stand setzt, dergleichen gute Werke zu üben, eine \RWbet{Gnade} sey, erhellet aus dem, was wir oben von dem Begriffe einer Gnade sagten. Gewiß, wir könnten Gott nicht tadeln, wenn er Dem oder Jenem aus uns nicht einmal das Daseyn gegeben hätte; um wie viel weniger können wir fordern, daß er uns gerade in diese und jene Verhältnisse setze, diese und jene Gelegenheiten und Antriebe zum Guten schenke, \usw\
\begin{aufzb}
\item \RWbet{Einwurf.} Wenn unsere Tugend Gottes Werk wäre: so wäre sie eben darum gar kein Werk unserer Freiheit,~\RWSeitenw{108}\ also könnte sie uns auch zu keinem Verdienste angerechnet werden. Schon Cicero hat die entgegengesetzte Wahrheit erkannt, wenn er \RWlat{(de natura deorum III. 63.)}\RWlit{}{Cicero1a} sagt: \RWlat{Caeteras res omnes diis acceptas referendas esse, praeter virtutem, hanc habere quemque a se ipso, non a Deo.}\par
\RWbet{Antwort.} Wenn die Kirche will, daß wir unsere Tugend Gottes Werk nennen sollen: so will sie hiemit keineswegs sagen, daß Gott die völlig bestimmende Ursache (\RWlat{causa determinans}) derselben wäre; denn freie Willensentschließungen haben gar keinen völlig bestimmenden Grund, weder in noch außerhalb unser. Aber es ist dem Sprachgebrauche gemäß, einen gewissen Gegenstand die Ursache (oder den Grund) von einem andern zu nennen, wenn er auch nur eine \RWlat{conditio sine qua non,} eine mitwirkende Ursache von seinem Daseyn ist. So nennen wir \zB\ die Tugenden, welche der Zögling eines würdigen Lehrers annimmt, das Werk des Letzteren. Wenn wir uns diese Redensart erlauben, ohne eine Verletzung der Lehre von der Freiheit zu befürchten: mit einem um wie viel größeren Rechte werden wir nicht von unserem göttlichen Erzieher sagen dürfen: Alle Tugenden, die wir an uns haben, sind dein Werk!
\item \RWbet{Einwurf.} Aber in eben dem Sinne, in welchem man die Tugend das Werk Gottes nennt, könnte man auch das Laster so nennen; denn Gott trägt zu dem Einen so viel wie zu dem Andern bei, auch zu dem Bösen gibt er uns die Kräfte, Gelegenheiten \usw\par
\RWbet{Antwort.} Wem ist es noch eingefallen, die Unarten, welche der Zögling ganz gegen die Bemühungen seines Lehrers annimmt, das Werk des Lehrers zu nennen? Der Grund des Unterschiedes ist hier nämlich der, die Tugenden, welche der Zögling annimmt, hat der Lehrer, so viel es nur immer in seinen Kräften stand, zu bewirken getrachtet; den Unarten aber suchte er, so viel er konnte, zu steuern. Gerade so verhält es sich aber auch mit Gott. Tugendhafte Gesinnungen sucht er, so viel es nur der Zusammenhang des Ganzen gestattet, zu befördern, dem~\RWSeitenw{109}\ Laster aber, so viel es nur möglich ist, zu steuern. Die Ersteren will er um ihrer selbst willen, die letzteren will und duldet er nur, wo ihre Verhinderung ein noch größeres Uebel wäre. 
\end{aufzb}
\item Daß jene höheren Tugenden, welche des Himmels würdig machen, auch einer höheren Gnade von Seite Gottes bedürfen, ist sehr begreiflich; denn alle Belohnungen müssen in einer gewissen Angemessenheit mit den Verdiensten stehen. Um also der Seligkeiten, die nur erst Jesus Christus dem menschlichen Geschlechte in der andern Welt bereitet hat, empfänglich und würdig zu werden, müssen wir erst diejenige Stufe der Vollkommenheit ersteigen, zu der nur er uns den Zugang eröffnet hat, also durch Gnaden erleuchtet und geheiliget werden, welche recht füglich den Namen höherer führen. Auch die Benennung übernatürlich, welche man diesen Gnaden zuweilen gibt, hat nichts Anstößiges; denn obgleich der Begriff, den wir mit diesem Ausdrucke verbinden sollen, von verschiedenen Gottesgelehrten verschiedentlich, und von Einigen auch wohl so angegeben wird, daß es schwer halten würde, die Nothwendigkeit übernatürlicher Gnade in diesem Sinne des Wortes durch bloße Vernunft zu erkennen: so ist doch eben wegen des Mangels an Uebereinstimmung keine jener Erklärungen als eine eigentliche Glaubenslehre anzusehen, und wenn wir uns an den Begriff halten sollen, auf welchen nicht sowohl die Definitionen der Schule, als vielmehr der Gebrauch, welchen die Kirche von dem Worte macht, deuten: so scheint es mir, daß man alle Gnaden Gottes übernatürliche nenne, welche wir eben nur erst dem Werke unserer Erlösung durch den Sohn Gottes verdanken; oder was eben so viel ist, von deren Vorhandenseyn wir nur aus der Offenbarung wissen. In dieser Bedeutung aber wäre es aus dem schon vorhin Gesagten begreiflich, daß wir der übernatürlichen Gnade Gottes bedürfen, um des Himmelreichs würdig zu werden.
\item Daß sich das Maß der göttlichen Gnade nicht durchaus nach unserer bisherigen Benützung, und überhaupt nach keiner uns bekannten Regel, es sey denn nach jener allgemeinen, die das oberste Sittengesetz ausspricht, richte, ist gar nichts Widersprechendes; denn~\RWSeitenw{110}
\begin{aufzb}
\item schon zu der ersten tugendhaften Handlung, die wir verrichten, bedürfen wir ja der Gnade; die erste Gnade also, die uns zu Theil wird, diese wenigstens muß sich nach irgend einem andern Grunde, als nach unserer bisherigen Verwendung richten.
\item Aber auch die folgenden Gnadenmittheilungen müssen sich nicht nothwendig nach unserer Verwendung richten; denn es läßt sich auf keine Weise darthun, daß die Gerechtigkeit Gottes eine solche Vertheilung fordere, \dh\ daß die Beförderung des allgemeinen Wohles dieselbe nothwendig mache. Es ist vielmehr sehr vortheilhaft, wenn auch der Lasterhafte zuweilen durch große Gnaden Gottes unterstützt wird; durch solche allein kann er gebessert werden.
\item Ueberhaupt aber läßt sich leicht einsehen, daß wir Menschen gar keine andere Regel, nach welcher Gott bei der Vertheilung seiner Gnaden verfahren müsse, anzugeben vermögen, als die schon in dem obersten Sittengesetze enthaltene, so zu verfahren, wie es das allgemeine Wohl am Meisten befördert. Um eine bestimmtere Regel angeben zu können, müßten wir den ganzen Zusammenhang der Dinge übersehen, und diese Gnaden würden dann aufhören, Gnaden zu seyn, und Werke der Gerechtigkeit werden.
\end{aufzb}
\item Daß aber Jedem, der nur Gebrauch von Gottes Gnaden machen will, diejenigen, die er zu seiner fortschreitenden Besserung und Beglückung nöthig hat, nicht versagt werden, findet die Vernunft der Heiligkeit und Güte Gottes vollkommen angemessen.
\item Daß es endlich Gnadeneinwirkungen gebe, die den Erfolg der Besserung nicht hervorbringen, ist eine sehr begreifliche Folge aus der Freiheit des Menschen. Der Grund aber, warum gewisse Gnaden Gottes wirksam, und andere unwirksam sind, dürfte wohl in der Vereinigung beider Umstände, welche die Molinisten und Augustinianer angeben, liegen.
\end{aufza}

\RWpar{220}{Sittliche Brauchbarkeit}
\begin{aufza}
\item Die Vorschrift des Christenthums, daß wir eine jede Stufe der sittlichen Vollkommenheit, die wir ersteigen, ja jede~\RWSeitenw{111}\ einzelne gute That, die wir verrichten, als ein Werk Gottes betrachten sollen, dient ganz vortrefflich, uns in der geziemenden Bescheidenheit und Demuth zu erhalten, ohne uns doch durch Absprechung alles eigenen Verdienstes muthlos zu machen. Dieser Vortheil ist um so wichtiger, je größer, der Erfahrung zu Folge, für jeden Tugendhaften die Gefahr ist, durch das Bewußtseyn seiner Verdienste allmählich zum Stolze und zur Geringschätzung Anderer verleitet zu werden. Die Geschichte der heidnischen Weisen, besonders der Stoiker, zeigt uns dieß nur zu deutlich. In welchen Uebermuth artete nicht ihr bischen Tugend aus! mit welcher Verachtung sahen sie nicht herab auf ihre schwächern Nebenmenschen, und wie viel Antheil an ihren glänzendsten Thaten hatte nicht der geheime Grund, dafür gerühmt und bewundert zu werden!\par
\RWbet{Einwurf.} Lieber ein wenig stolz, als muthlos und niederträchtig. Dieß muß man aber werden, wenn man, nach Vorschrift der katholischen Kirche, sich jenes Einzigen, das des Menschen eigenes Werk ist, entäußern soll. Welch eine \Ahat{unnatürliche}{übernatürliche} Selbstverläugnung, wogegen sich das unverdorbene Gefühl empört! Daher denn auch jene kriechenden Heiligen in der katholischen Kirche, welche sich selbst für die größten Sünder erklären, sich ganz unwürdige Gefäße der göttlichen Barmherzigkeit nennen, \usw\par
\RWbet{Antwort.} Es kann seyn, daß einige angebliche, oder auch wahre Heilige in der katholischen Kirche es in der Demuth übertrieben, oder daß ihre Lebensbeschreiber ihnen Gesinnungen, Reden und Thaten andichteten, welche der Vorwurf einer solchen Uebertreibung trifft. Allein wenn die Lehre der katholischen Kirche gehörig verstanden wird: so bringt sie diesen Erfolg sicher nicht hervor; denn die Kirche will keineswegs, daß wir die sittlichen Vorzüge, die wir besitzen, an uns nicht anerkennen sollen; sondern sie will nur, damit wir dieß um so sicherer thun könnten, daß wir an jenen großen Antheil denken, den Gott daran hat, und sie eben deßhalb (obgleich nur bildlicher Weise) als Gottes Werk an uns betrachten.
\item Die Bemerkung, daß uns zur Erreichung eines so hohen Tugendgrades, als für den Himmel erfordert wird,~\RWSeitenw{112}\ übernatürliche Gnaden Gottes nöthig seyen, bringt uns einen erhabeneren Begriff von jener Seligkeit sowohl als auch von jenem Tugendgrade bei. Wir werden nun um so gewissenhafter seyn in der Benützung jeder Gnade und jedes Mittels, uns göttliche Gnaden zu verschaffen.
\item Die Erinnerung, daß sich das Maß der göttlichen Gnadenmittheilungen nicht nach unserer bisherigen Verwendung, und überhaupt nach keiner uns bekannten Regel richte, dient
\begin{aufzb}
\item für den Tugendhaften, ihn um so gewisser vor Stolz und Uebermuth zu bewahren, je weniger er sich verlassen kann, daß Gottes Gnade stets so stark wie bisher in ihm wirksam seyn werde. Auch über dich -- muß er befürchten -- können einst noch sehr mächtige Versuchungen kommen, und wer weiß, wie schwach du dich dann fühlen wirst, obwohl es dir niemals unmöglich seyn wird, obzusiegen!
\item Für den Sünder, der sich zur Besserung entschließt. Sollte sich Gottes Gnade nur nach der bisherigen Benützung derselben richten: so hätte der Sünder, der Gottes Gnade so oft von sich gewiesen, nun keine zu erwarten; jetzt aber bleibt ihm die Hoffnung unbenommen, daß Gott ihm seine Gnade vielleicht gerade dießmal in einem ganz vorzüglichen Grade mittheilen werde, und durch Gebet kann er diese Hoffnung verstärken.
\item Für den Sünder, der seinen Lasterweg noch immer fortzusetzen gedenkt. Wenn er sich vorstellt, daß ihm Gott, ob er auch seine bisherigen Gnaden sehr schlecht benützt hat, gleichwohl noch immer neue anbiete, mitunter vielleicht sehr hohe Gnaden: so muß es ihm um so einleuchtender werden, wie schwer er sich durch die Verschmähung so vieler Gnaden versündige, und welch ein hartes Gericht einst über ihn ergehen werde.
\end{aufzb}
\item Daß aber Jedem, der um die Gnade Gottes bittet, und sie gebrauchen will, gegeben werde, was für ihn hinlänglich ist, ist eine ungemein tröstliche Versicherung, wir mögen nun auf dem Pfade der Tugend bereits weit, oder nicht~\RWSeitenw{113}\ weit vorgerückt seyn; besonders aber dann, wenn wir sehr schwere Versuchungen herannahen sehen.
\item Daß endlich nicht alle Einwirkungen der göttlichen Gnade ihren Zweck erreichen, dient uns aufzumuntern, daß wir mit Gottes Gnade mitwirken.
\end{aufza}
\begin{RWanm} Daß es übrigens unentschieden geblieben, ob die Ursache der Wirksamkeit gewisser Gnadenmittheilungen Gottes in ihrer inneren Beschaffenheit, oder in unserem guten Willen liege, hat gleichfalls seinen Nutzen. Das Erstere nämlich stünde in einem wenigstens scheinbaren Widerspruche mit unserer Freiheit; das Letztere scheint unserer eigenen Willkür einen zu großen Antheil einzuräumen, und der so wohlthätigen Bescheidenheit, mit welcher der Mensch von seinen Verdiensten reden soll, Abbruch zu thun. 
\end{RWanm}

\RWabs{Fünfter Abschnitt}{Von den Verhältnissen der Geschöpfe unter einander}
\RWpar*{Vorerinnerung}
Wir können die wichtigsten in diesem Abschnitte vorkommenden Lehren bequem unter Einem Begriff zusammenfassen, wenn wir sie als die \RWbet{Lehren von unserem Wirkungskreise} betrachten; denn eine Belehrung darüber, wie weit unser Wirkungskreis reiche, ist in der That fast Alles, was uns die christliche Religion über unsere Verhältnisse zu unseren Mitgeschöpfen mittheilt. Wir werden also auch nur diese einzige Lehre in diesem Abschnitte abhandeln.

\RWpar{221}{Die Lehre vom Wirkungskreise des Menschen}
Das katholische Christenthum bemüht sich, uns zu zeigen, daß unser Wirkungskreis überhaupt \RWbet{größer sey, und viel weiter reiche, als wir es uns insgemein vorzustellen pflegen,} und als wir auch wohl, ohne die Aufschlüsse~\RWSeitenw{114}\ einer göttlichen Offenbarung darüber benützen zu können, zu glauben berechtiget wären. Es sagt uns zuvörderst, daß wir
\begin{aufza}
\item durch eine jede sittlich gute Handlung und Willensentschließung zur \RWbet{Beförderung der Tugend und Glückseligkeit des Ganzen} mehr beitragen, als wir mit Deutlichkeit wahrnehmen können. Es sagt, wir sollten insbesondere
\item versichert seyn, daß \RWbet{nicht eine einzige in guter Absicht begonnene Unternehmung je ganz mißlinge;} sondern, wenn wir auch den Erfolg, den wir zunächst beabsichtiget hatten, nicht eintreten sehen, wenn es sogar den Anschein nimmt, als ob nur Böses aus dem, was wir wohlmeinend anfingen, hervorgehe: so sey dieß immer nur scheinbar, und unsere Unternehmung führe, uns unbemerkt, eine Menge wohlthätiger Folgen herbei, welche die schlimmen, die uns in's Auge fallen, bei Weitem überwiegen.
\item Wenn unsere eigenen Kräfte zur Ausführung eines gewissen guten Zweckes nicht hinreichen; so ist \RWbet{schon der bloße lebhafte Wunsch,} daß dieser Zweck erreicht werden möge, wenn wir ihn Gott vortragen, \dh\ wenn wir gedenken, daß Gott von unserem Wunsche wisse, und gütig und mächtig genug sey, ihn zu berücksichtigen und zu erfüllen, mit Einem Worte: unser Gebet \RWbet{nie ohne allen Erfolg;} sondern Gott höret und erhöret unsere Bitte immer in der Art, daß er uns Eines von Beiden, entweder eben das, um was wir bitten, oder etwas noch Besseres gewähret.
\item Diese Erhörung von Gott dürfen wir hoffen, nicht nur wenn wir für uns, sondern auch, wenn wir \RWbet{für das Wohl Anderer} bitten; Gott nimmt auch \RWbet{Fürbitten} an.
\item Wir können nicht bloß für Lebende, sondern auch \RWbet{für bereits Verstorbene} mit Erfolg fürbitten.
\item Und wie wir dieses für sie, so und noch mehr vermögen \RWbet{auch die Verstorbenen, ja alle auf einer höhern Stufe des Daseyns stehende Wesen auch für uns Lebende} mit Erfolg fürzubitten.
\item Wir können diese Wesen um ihre Fürbitte auch mit Nutzen \RWbet{anrufen;} und eine solche Anrufung derselben wird~\RWSeitenw{115}\ auch dann nicht vergeblich seyn, wenn jene Wesen sie auch nicht wirklich inne werden.
\item Alle unsere Gebete sind \RWbet{um so wirksamer, je sittlich besser} wir sind.
\item Wir können die Wirksamkeit auch jedes einzelnen unserer Gebete dadurch gar sehr erhöhen, \RWbet{daß wir damit gewisse gute Werke, oder auch bloße fromme Gelübde verknüpfen,} \dh\ daß wir verschiedene sittlich gute Handlungen aus dem Beweggrunde unternehmen, um die Erfüllung unserer Bitte von Gott desto sicherer zu erhalten, oder daß wir in eben dieser Absicht den festen Vorsatz fassen, für den Fall der Gewährung unserer Bitte dieses und jenes Gute zu thun, das wir wohl nicht vollzogen hätten, wenn wir es nicht als ein Mittel, uns die Erfüllung unserer Bitte zu sichern, angesehen hätten.
\item Die \RWbet{Gebete der Christen,} wenn sie gehörig eingerichtet sind, haben sich einer immer \RWbet{noch vollständigeren Erhörung als jene der Nichtchriste}n zu erfreuen. Und Gebete von Christen für Christen angestellt, sind gleichfalls wirksamer als für Nichtchristen.
\item Jeden Glauben dagegen, als ob wir durch Ausübung einer gewissen wie immer gearteten Handlung, die weder einen durch unsere bloße Vernunft erkennbaren, noch einen durch eine erweisliche göttliche Offenbarung uns angezeigten Nutzen hat, dennoch uns oder Andern Hülfe verschaffen könnten, erkläret das Christenthum für einen eben so schädlichen als sträflichen \RWbet{Aberglauben.}
\end{aufza}

\RWpar{222}{Historischer Beweis dieser Lehre}
\begin{aufza}
\item Daß wir durch unsere sittlich guten Handlungen und Willensentschließungen des Guten mehr bewirken, als wir mit Deutlichkeit wahrnehmen können, lehrt uns die heilige Schrift in vielen Beispielen. So wird (\RWbibel{Gen}{1\,Mos.}{18}{23}) dem Abraham von Gott versprochen, er wolle ganz Sodoma verschonen, wenn er nur fünfzig, nur fünf und vierzig, dreißig, zwanzig, ja selbst nur zehn Fromme darin fände. Ferner wird (\RWbibel{Gen}{1\,Mos.}{39}{2}) erzählt, daß Gott das Haus des~\RWSeitenw{116}\ Potiphar um des rechtschaffenen Josephs willen, den er zu sich genommen hatte, auf alle Art gesegnet habe. Der heil.\ Paulus befiehlt (\RWbibel{1\,Kor}{1\,Kor.}{7}{12}) im Namen des Herrn, daß Eheleute, deren der eine Theil gläubig, der andere ungläubig ist, einander nicht verlassen möchten, wenn der ungläubige Theil die Verbindung selbst nicht getrennt wissen will; und fügt den Grund bei: \erganf{denn der ungläubige Mann wird geheiligt durch das gläubige Weib, und das ungläubige Weib wird geheiligt durch den gläubigen Mann, sonst wären eure Kinder unrein, nun aber sind sie heilig.} Eben so heißt es (\Ahat{\RWbibel{1\,Petr}{1\,Petr.}{3}{1}}{5,1.}): \erganf{Deßgleichen sollen die Weiber ihren Männern unterthan seyn, auf daß auch die, so nicht das Wort glauben, durch der Weiber Wandel ohne Wort gewonnen werden.} -- So herrscht auch unter uns Christen bekanntlich das Sprichwort: \RWlat{Omne bonum est communicativum.} Und im römischen Breviere liest man (\RWlat{lect.\,7.\ in Nativ.\ S.~Joannis Bapt.})\RWlit{}{RoemischesBrevier} folgende Worte aus einer Homilie des heil.\ Ambrosius: \RWlat{Habet sanctorum editio laetitiam plurimorum, quia \RWbet{commune} est bonum. Justitia enim \RWbet{communis} est virtus.}
\item Daß aus dem Guten nie Böses hervorgehen könne, liegt gewissermaßen schon in dem bekannten Ausspruche Jesu (\RWbibel{Mt}{Matth.}{7}{16}): \erganf{Aus ihren Früchten werdet ihr sie erkennen; denn jeder gute Baum trägt gute Früchte, und schlimme kann ein guter gar nicht tragen.} Uebrigens ist auch bekannt, daß man allgemein glaube, zu allem Guten gebe Gott selbst seinen Segen: wie könnte es also mißlingen in dem Sinne, daß daraus Böses und überwiegendes Böses hervorgehen sollte?
\item Ueber die Wirksamkeit \RWbet{zweckmäßig eingerichteter Bittgebete} erklärt sich die heil.\ Schrift in den stärksten Ausdrücken. Z.\,B.\ \RWbibel{Mt}{Matth.}{7}{7}: \erganf{Bittet, und es wird euch gegeben werden; suchet, und ihr werdet finden; klopfet an, und es wird euch aufgethan werden; denn jeder, der bittet, wird empfangen, und jeder, der suchet, wird finden, und jedem, der anklopfet, wird aufgethan werden. Wenn ihr, die ihr doch böse seyd, gute Gaben euren Kindern zu geben wisset: um wie viel mehr wird euer Vater im Himmel Gutes geben denen, die ihn darum bitten?} \Ahat{\RWbibel{Mt}{Matth.}{17}{20}}{17,19.}:~\RWSeitenw{117}\ \erganf{Wahrlich, sage ich euch, wenn ihr nur Glauben habt, gleich einem Senfkörnlein, und ihr sprecht dann zu jenem Berge: Hebe dich weg von hier: so wird er sich heben, und nichts wird euch unmöglich seyn.}
\item Daß insbesondere auch \RWbet{Fürbitten} bei Gott angenommen werden, beweisen die vielen Aufforderungen zu solchen Fürbitten, die in den Büchern nicht nur des neuen, sondern auch schon des alten Bundes vorkommen, und widersinnig wären, wenn dergleichen Gebete nichts nützen könnten. Doch auch an ausdrücklichen Versicherungen ihrer Wirksamkeit fehlt es nicht. Moses bittet öfters für sein Volk vor, und Gott verzeiht dem Volke um seinetwillen; \zB\ \RWbibel{Num}{4\,Mos.}{11}{2}\ \RWbibel{Gen}{1\,Mos.}{20}{7}\ \RWbibel{Lev}{3\,Mos.}{5}{10} \udgl\  -- Der Apostel Paulus schreibt \RWbibel{Röm}{Röm.}{15}{30}: \erganf{Ich bitte euch, Brüder! bei unserem Herrn Jesu Christo und bei der Liebe des heiligen Geistes, \RWbet{helfet mir durch Fürbitte zu Gott für mich!}}
\item Daß wir \RWbet{selbst für Verstorbene} noch wirksam fürbitten können, wurde in der katholischen Kirche zu allen Zeiten geglaubt. Ja dieser Glaube war schon in der jüdischen Kirche vorhanden, wie die Bücher der Makkabäer beweisen. Denn \RWbibel{2\,Makk}{2\,Makkab.}{12}{43}\ heißt es: \erganf{Judas der Makkabäer veranstaltete eine Sammlung von zwölf tausend Drachmen Silber, und schickte dieß nach Jerusalem zum Opfer für die Sünden der Verstorbenen. Fromm und heilsam ist der Gedanke, für die Verstorbenen zu beten, damit sie von ihren Sünden befreiet werden.} Eben dieß scheint auch die Stelle \RWbibel{1\,Kor}{1.\,Kor.}{15}{29}\ von dem apostolischen Zeitalter zu beweisen: \erganf{Was machen die, welche \RWbet{um der Verstorbenen willen sich taufen lassen}, wenn die Todten überhaupt nicht auferstehen? Warum lassen sie sich taufen?} -- Der Gebrauch, sich für Verstorbene taufen zu lassen, von welchem hier gesprochen wird, hätte nicht aufkommen können, würde man nicht geglaubt haben, daß man den schon Verstorbenen dadurch, und also ohne Zweifel auch durch andere fromme Wünsche und Gebete helfen könne. Daß auch sehr frühzeitig schon der Gebrauch aufgekommen sey, selbst in den öffentlichen gottesdienstlichen Versammlungen für die Verstorbenen zu beten, beweiset unter Andern die Stelle \RWbet{Tertullian's} \RWlat{(de}~\RWSeitenw{118}\ \RWlat{corona milit.\ cap.\,3.)\RWlit{c.\,3}{Tertullian2b} Oblationes pro defunctis annua die facimus.}
\item Daß die Verstorbenen, die fromm gelebt, und überhaupt alle Wesen, die sich auf höheren Stufen des Daseyns befinden, für uns auf Erden noch Wallende wirksam fürbitten können, ist ein Glaube, von dem wir Spuren schon in den Büchern des alten Bundes antreffen. So heißt es (\RWbibel{2\,Makk}{2\,Makk.}{15}{14}) von dem damals schon längst verstorbenen Propheten Jeremias: Er betete viel für das Volk und für die ganze heilige Stadt, der Seher Gottes, der Freund des Vaterlandes und des Volkes Israel. -- Und bei \RWbibel{Jer}{Jerem.}{15}{1}\ spricht Gott in seinem Unwillen: Auch wenn Moses oder Samuel für dieses Volk fürbitten möchten: so könnte ich demselben (jetzt) doch nicht gut werden. Woraus nicht nur erhellet, daß Verstorbene im Allgemeinen fürbitten können; sondern daß die Fürbitten der Tugendhaften einen besonderen Werth in Gottes Augen haben. -- Auch in den Büchern des n.\,B.\ dürfte es nicht ganz an solchen Stellen fehlen. In der Stelle \Ahat{\RWbibel{Offb}{Offenb.}{5}{8}}{\RWbibel{Offb}{Offenb.}{4}{8}}\ bedeuten die vier Lebendigen (\RWgriech{z~w|a}) und die vier und zwanzig Aeltesten (\RWgriech{presb'uteroi}), die vor dem Lamme niederfallen und in goldenen Weihrauchschaalen die Gebete der Heiligen (\RWgriech{t~wn <ag'iwn}), \dh\ der Gläubigen, darbringen, doch sicher nur Wesen, die für uns Menschen fürbitten; die Frage ist nur, ob diese Wesen als solche gedacht werden sollen, die einst auf Erden gelebt, oder nicht. Doch möchte sich auch in den Büchern des n.\,B.\ kein ganz ausdrückliches Zeugniß für diese Lehre nachweisen lassen; und möchte man auch in den drei ersten christlichen Jahrhunderten aus Besorgniß eines möglichen Mißbrauches etwas zurückhaltend gewesen seyn: später, als diese Besorgniß je mehr und mehr wegfiel, sprach man sie immer deutlicher aus. So lesen wir schon bei \RWbet{Origenes} (\RWlat{hom.\,3.\ in Cant.})\RWlit{}{Origenes4}: \erganf{Wer sagt, daß die Heiligen, welche aus diesem Leben ausgetreten sind, noch immer Sorge tragen für die Zurückgebliebenen, und ihnen durch ihre \RWbet{Fürbitte} ersprießlich werden, weil ja gewiß ihre Liebe gegen sie noch nicht aufgehört hat, der lehrt nichts Ungereimtes. Steht es doch in den Büchern der Makkabäer so.} -- Und der heil.\ \RWbet{Cyprian} schreibt (\RWlat{epist.\,59.})\RWlit{}{Cyprianus2}: \erganf{Wenn Jemand von uns durch die Beschleunigung der göttlichen~\RWSeitenw{119}\ Huld früher aus diesem Leben austritt: so bleibe auch dort noch bei dem Herrn unser Liebesbund; er höre nicht auf, für seine Brüder und Schwestern die Barmherzigkeit des Vaters anzurufen.} -- Und \RWbet{Eusebius,} Bischof zu Cäsarea, schreibt: \erganf{Wir gestehen, daß wir auch aus den Fürbitten der Heiligen bei Gott nicht geringe Vortheile zu ziehen hoffen.}
\item Daß der Gebrauch, Engel oder auch tugendhafte Menschen, die ihre irdische Laufbahn bereits vollendet, um ihre Fürbitte bei Gott anzurufen, in der christlichen Kirche erst etwas später aufgekommen sey, könnte wahr seyn, ohne daß daraus sich folgern ließe, daß die Behauptung von der Nützlichkeit solcher Anrufungen nicht zu dem Lehrbegriffe der christlichen Offenbarung gehöre; denn wir wissen ja schon, daß dieser Lehrbegriff einer zeitgemäßen Entwickelung und Ausbildung fähig sey. Es gibt ja der Gründe gar manche, um derentwillen es nicht eben sehr zweckmäßig gewesen wäre, wenn der Gebrauch der Anrufung der Engel und Heiligen gleich bei der Entstehung des Christenthumes eingeführt worden wäre. Wie hätte man doch zu einer Zeit, da es der Lehrer und Aufseher in den Gemeinden noch so wenige gab; da viele dieser Gemeinden noch keine Mittel hatten, mit den übrigen einen sie vor der Ausartung ihrer Begriffe verwahrenden Verkehr zu pflegen; da der größte Theil der Bekenner des Christenthums aus geborenen, an das Vergöttern gewöhnter Heiden bestand, verhüten können, daß aus der Engel- und Heiligenverehrung nicht die ärgsten Mißbräuche entsprungen wären? -- Inzwischen finden wir doch seit dem vierten christlichen Jahrhunderte dergleichen Anrufungen schon in den liturgischen Büchern; und im \RWbet{Concilio zu Chalcedon} (im 5ten Jahrh.) riefen die versammelten Väter: \RWlat{Martyr (sc. Flavianus) pro nobis oret!} Der heil.\ \RWbet{Ambrosius} \RWlat{(de Viduis)} schreibt: \RWlat{obsecrandi sunt angeli, qui nobis in praesidium dati sunt; Martyres obsecrandi,} \usw\
\item Daß unsere Gebete allgemein um so wirksamer seyen, je tugendhafter wir sind, beweiset \zB\ gleich folgende Stelle \RWbibel{Jak}{Jak.}{5}{16}: Viel vermag des Gerechten eifriges Gebet. Und aus \RWbibel{Joh}{Joh.}{9}{31}\ ersehen wir, daß dieß ein allgemeiner Glaube auch selbst der Juden schon gewesen.~\RWSeitenw{120}
\item Daß wir die Wirksamkeit unserer Bittgebete durch \RWbet{gute Werke} und \RWbet{Gelübde} noch erhöhen können, glaubte man schon in dem alten Bunde; daher so viele Beispiele von Gelübden, wie \RWbibel{Gen}{1\,Mos.}{28}{20}\ \RWbibel{1\,Sam}{1\,Sam.}{1}{21}\ \uma\  Dergleichen Gelübde kommen aber auch im neuen Bunde vor. So macht \RWbibel{Apg}{Apostelg.}{18}{18}\ selbst Paulus ein Gelübde. Auch heißt es \RWbibel{Apg}{Apostelg.}{10}{4}: \erganf{Dein (des Hauptmanns Cornelius) Gebet und dein Almosen sind hinaufgestiegen vor Gott.}
\item Daß die Gebete der Christen unter übrigens gleichen Umständen wirksamer sind, als die der Nichtchristen, beweisen die Worte Jesu \RWbibel{Joh}{Joh.}{14}{14}: \erganf{Wenn ihr etwas \RWbet{in meinem Namen} bitten werdet: so werde ich es thun.} Daher denn auch seit den ältesten Zeiten in der christlichen Kirche der Gebrauch bestand, alle Gebete mit den bekannten Worten, durch Jesum Christum \usw\ zu schließen. Auch alle Wunderthaten, welche von den Aposteln und ersten Christen verrichtet worden sind, wurden nur im Namen Jesu gewirket, \zB\ \RWbibel{Apg}{Apostelg.}{3}{6}\ \uma\,O. -- Daß Gebete von Christen für Christen wirksamer als für Nichtchristen sind, erhellet unter Anderem aus allen denjenigen Stellen, in welchen die Gemeinschaft, die zwischen den Christen obwaltet, mit der Gemeinschaft der Glieder eines und eben desselben Leibes verglichen wird; denn dieß will nichts Anderes sagen, als daß die Gemeinschaft unter den Christen viel inniger sey, als unter den Christen und Nichtchristen; \zB\ \RWbibel{1\,Kor}{1\,Kor.}{12}{12}\ \RWbibel{Röm}{Röm.}{12}{4}\ \uma\  Hieher gehört auch die Vergleichung Jesu mit einem Weinstocke, an welchem seine Jünger die Zweige waren (\RWbibel{Joh}{Joh.}{15}{1}). Dieses meinte man auch, wenn man im apostolischen Glaubensbekenntnisse die zwischen den Heiligen (\dh\ Christen) bestehende Gemeinschaft zu einem eigenen Artikel erhob.
\item Verordnungen gegen den Aberglauben gab schon der vortreffliche Gesetzgeber des israelitischen Volkes. \Ahat{\RWbibel{Lev}{3\,Mos.}{19}{31}}{19,3.}\ heißt es: \erganf{Kehret euch nicht zu den Wahrsagern und fragt die Zeichendeuter nicht um Rath.} Und \RWbibel{Num}{4\,Mos.}{23}{23}\ \erganf{Zauberkünste vermögen nichts wider Jakob, Beschwörungen nichts wider Israel.} -- Und \RWbibel{Dtn}{5\,Mos.}{18}{10}\ \erganf{Laßt keinen unter euch gefunden werden, der seinen Sohn oder seine~\RWSeitenw{121}\ Tochter verbrenne, keinen Wahrsager, keinen Sterndeuter, keinen, der aus dem Vogelfluge weissagt, keinen Zauberer} \usw\ Daß aber auch die Apostel und ersten Verkündiger des Christenthums dem Glauben an Zauberei \udgl\  kräftig entgegenwirkten, zeigt das Ereigniß, das \RWbibel{Apg}{Apostelg.}{19}{19}\ erzählt wird, daß eine beträchtliche Anzahl von Menschen zu Ephesus, die sich bisher mit Zauberkünsten beschäftiget hatten (\RWgriech[t`a per'ierga prax'antes]{t`a per'ierga pr'axantes}), ihre Bücher öffentlich verbrannten. So erzählt uns auch noch \RWbet{Lukian}, daß Zauberer in Gegenwart der Christen ihre betrügerischen Künste zu üben sich gescheuet hätten; gewiß nur, weil sie besorgten, daß ihr Betrug von den Christen aufgedeckt würde. Ausdrückliche Warnungen vor Leichtgläubigkeit und Aberglauben kommen \RWbibel{1\,Tim}{1\,Tim.}{4}{7}\ \RWbibel{Tit}{Tit.}{1}{14}\ \RWbibel{2\,Tim}{2\,Tim.}{4}{4}\ \uamO\  vor.
\end{aufza}

\RWpar{223}{Vernunftmäßigkeit}
\begin{aufza}
\item Daß wir durch eine jede sittlich gute Handlung und Willensentschließung zur Beförderung der Tugend und Glückseligkeit des Ganzen mehr beitragen, als wir mit Deutlichkeit wahrnehmen können, ist eine Behauptung, der Niemand widersprechen wird; denn aus der Beschränktheit unseres Wissens ergibt sich unmittelbar, daß wir von keiner unserer Handlungen (sie seyen gut oder böse) die sämmtlichen Folgen, die sie nach sich ziehen werden, zu überschauen vermögen. Da nun gewiß der Theil der Folgen, die für uns unsichtbar sind, in keinem Falle von durchaus schlimmer Art seyn wird: so ließe sich von einer jeden unserer Handlungen in einer gewissen Bedeutung sagen, sie bringe der Folgen, die für das Ganze wohlthätig sind, viel mehrere hervor, als wir eben wahrnehmen können. Hiezu kommt noch, daß Gott, vermöge seiner unendlichen Macht und Weisheit, aus einer jeden unserer Handlungen unzählig viel Gutes, woran wir gar nicht denken, abzuleiten im Stande seyn muß, und seiner unendlichen Güte und Heiligkeit wegen auch gewiß ableitet.
\begin{RWanm} 
Wahr ist es, und ich gebe es in dem so eben Gesagten selbst zu verstehen, daß dieses Alles auch von den \RWbet{bösen} Handlungen gelte; auch aus diesen muß Gott gar manches Gute, wovon wir nichts wissen, abzuleiten verstehen und wirklich ableiten.~\RWSeitenw{122}\ Aber hievon schweiget das Christenthum billig; denn nur das Gute, das Gott aus unseren guten Handlungen ableitet, kann uns zum Troste und zur Ermunterung dienen; das Gute aber, das er aus unseren bösen Handlungen ableitet, dürfen wir durchaus zu keiner Entschuldigung derselben mißbrauchen; zumal da es gewiß ist, daß die Summe des Guten, das Gott aus den guten Handlungen ableiten kann, der Regel nach immer viel größer ist, als die Summe des Guten, das aus den bösen sich ableiten läßt. 
\end{RWanm}
\item Daß Gott uns auch nicht eine einzige in guter Absicht begonnene Unternehmung so ganz mißlingen lasse, daß nicht eine überwiegende Menge seliger Folgen aus ihr hervorgehen sollte: stimmt auf das Beste überein mit den Begriffen, die wir uns von der Vollkommenheit der göttlichen Weltregierung bilden. Wenn nämlich das Gegentheil Statt fände; wenn Gott es zuließe, daß wir durch Irrthum etwas als zuträglich für das gemeine Beste, und also eben deßhalb als unsere Pflicht ansehen, und in dieser Absicht auch unternehmen würden, was doch in Wahrheit nur überwiegend nachtheilig in seinen Folgen ist: so wäre nur er selbst durch die Zulassung dieses Irrthumes Schuld, daß das gemeine Beste weniger befördert wird, als es bei unserem guten Willen hätte geschehen können. Es stimmt also mit der unendlichen Macht, Weisheit und Heiligkeit Gottes vollkommen überein, daß er einen solchen Irrthum nie zulasse, \dh\ daß Alles, was wir in guter Absicht beginnen, durch seine Leitung in der That Gutes und überwiegendes Gute zur Folge habe.
\item Auch von demjenigen, was uns das Christenthum über die Art sagt, wie unsere Bittgebete von Gott aufgenommen werden, muß die Vernunft gestehen, daß es ihr vollkommen als Gottes würdig erscheine, ob sie gleich, bloß sich selbst überlassen, kaum den Muth gehabt haben würde, mit aller Zuversicht zu behaupten, daß Gott nicht anders, als nur gerade so verfahren dürfe. Erst nachdem es das Christenthum ausdrücklich ausgesprochen hat (\RWbibel{Mt}{Matth.}{10}{29}), wagt es auch die Vernunft zu behaupten, daß Gottes Weltregierung nicht vollkommen heißen könnte, wäre bei jedem einzelnen Ereignisse, das in der Welt zu Stande kommt, nicht die genaueste Rücksicht genommen auf die Veränderungen, die es in dem Zustande eines jeden noch so unbedeutenden Geschö\RWSeitenw{123}pfes hervorbringen wird. Jetzt erst erkennt die Vernunft, daß Gott, nach seiner unendlichen Güte, auch nicht einen einzigen Wunsch, um wie viel weniger einen solchen, der mit den Gesetzen der Sittlichkeit nicht übereinstimmt, und den ein vernünftiges Wesen mit dem Vertrauen heget, daß die Huld Gottes ihn gewähren werde, \dh\ der zum Gebete geworden ist, bei seiner Weltregierung unbeachtet lassen dürfe; daß vielmehr jeder Wunsch dieser Art seinen Platz finde unter den Gründen, nach deren Ausschlage der Allmächtige ein Ereigniß herbeizuführen oder nicht herbeizuführen beschließt. Nun erst leuchtet es der Vernunft ein, daß die Erhörung eines jeden zweckmäßig eingerichteten Gebetes den göttlichen Vollkommenheiten um so angemessener sey, je gewisser es ist, daß
\begin{aufzb}
\item jedes solche Gebet eine Art Tugendübung ist, die eben deßhalb von Seite Gottes möglichst befördert und belohnt werden müsse; daß ferner
\item durch die Gewährung eines Wunsches, den wir Gott im Gebete vorgetragen haben, des Guten mehr bewirkt wird, als durch die Gewährung eines übrigens gleichen Wunsches, den wir nicht zum Gebete erhoben; denn nicht nur muß uns ein Gut, welches wir uns glauben erbeten zu haben, ungleich mehr freuen; sondern durch eine solche Erhörung muß auch unser Vertrauen zu Gott bei uns selbst sowohl, als auch bei allen Andern, die davon Kenntniß erhalten, gar sehr erhöht werden; und wir werden in Zukunft selbst in den Fällen, wo uns dasjenige, um was wir eigentlich gebeten, nicht zu Theil wird, schon in der Hoffnung, daß wir es erhalten, und in der Ueberzeugung, daß wir doch jedesmal Etwas empfangen, viel Trost und Beruhigung finden.
\end{aufzb}\par
\RWbet{1.~Einwurf.} Die Lehre von der Erhörbarkeit unserer Bittgebete streitet mit der Unveränderlichkeit der göttlichen Rathschlüsse; denn wenn Gott um unseres Bittgebetes wegen etwas veranstalten soll, was er außerdem nicht veranstaltet hätte, so muß er seine von Ewigkeit her gefaßten Rathschlüsse ändern.\par
\RWbet{Antwort.} Nicht im Geringsten; denn auch schon von Ewigkeit her wußte Gott unsere Gebete, und beschloß ihnen gemäß, was in der Zeit erfolget.~\RWSeitenw{124}\par
\RWbet{2.~Einwurf.} Aber so streitet dieß doch mit Gottes unendlicher Güte, vermöge der er uns Alles, was wahrhaft gut für uns ist, geben muß, auch wenn wir ihn nicht darum bitten.\par
\RWbet{Antwort.} Allerdings gibt uns Gott Alles, was wahrhaft gut für uns ist, auch wenn wir ihn nicht darum bitten; daraus folgt aber keineswegs, daß unsere Bitten überflüssig seyen, und daß nicht erst wegen derselben uns Manches gegeben werden könnte, was wir ohne sie nie empfangen hätten. Der Umstand, ob wir einen gewissen Wunsch, den wir hegen, Gott bittend vortragen oder nicht, kann sehr viel daran ändern, ob die Gewährung desselben etwas für uns und für unsere Mitgeschöpfe wahrhaft Ersprießliches ist oder nicht. Wenn wir das Erstere thun: so wird (wie wir gesehen haben) die Summe der ersprießlichen Folgen, die aus der Gewährung unseres Wunsches theils für uns, theils für Andere hervorgehen, bedeutend größer als im entgegengesetzten Falle; und so kann es sich fügen, daß sie nun groß genug sey, um die Summe der Nachtheile zu überwiegen, und somit wird sich Gott, nach seiner unendlichen Güte, jetzt für die Gewährung entscheiden, für die er sich, wenn wir nicht gebetet hätten, nach eben dieser unendlichen Güte nicht hätte entscheiden können.
\item Wer einmal eingesehen hat, daß eine Erhörung unserer Bittgebete in der Art, wie sie das Christenthum lehrt, Statt finden könne, ja müsse, der wird auch die Erhörbarkeit der Fürbitten bald begreifen. Offenbar wird dadurch, daß Gott die Wirksamkeit des Bittgebetes auch auf unsere Mitgeschöpfe ausdehnt, \dh\ daß er sich als bereitwillig erklärt, uns auch selbst dann zu erhören, wenn es nicht eben etwas nur uns selbst Betreffendes, sondern, wenn es das Heil unserer Mitgeschöpfe ist, um das wir ihn bitten, die Summe des Guten, das die Erhörung der bloß für uns angestellten Gebete mit sich führt, gar sehr vermehrt, und es entspringen noch allerlei neue Vortheile. Die wichtigsten derselben wollen wir, weil sie schon aus dem bloßen Glauben an diese Erhörbarkeit hervorgehen, und also auch zu dem sittlichen Nutzen dieser Lehre gehören, in dem gleich folgenden Paragraph anführen.~\RWSeitenw{125}\par
\RWbet{Einwurf.} Aber es widerspricht der göttlichen Vollkommenheit, irgend einem Geschöpfe eine gewisse Wohlthat nicht darum, weil es dieselbe verdient, sondern bloß, weil ein Anderer für dasselbe fürbittet, zu verleihen. Ob ein Anderer für mich fürbittet oder nicht, das ändert ja nichts an mir selbst, an meinen Eigenschaften, Bedürfnissen, Verdiensten, \usw\ Folglich soll es auch nichts an jenen Schicksalen, die mir Gott zugedacht hat, ändern; oder man kann nicht sagen, daß die Schicksale, welche Gott seinen Geschöpfen zuführet, zweckmäßig für sie eingerichtet seyen.\par
\RWbet{Antwort.} Gott leitet die Schicksale seiner Geschöpfe auch dann noch zweckmäßig, wenn der Beweggrund, der ihn bestimmt, diesem oder jenem Geschöpfe dieß oder jenes Schicksal zuzuführen, nicht immer ganz in dem Geschöpfe selbst, sondern zuweilen und zum Theile auch in einigen seiner Mitgeschöpfe liegt. Dadurch, daß mehrere Andere vereinigt wünschen, daß diesem oder jenem ihrer Mitgeschöpfe eine gewisse Wohlthat von Gott erwiesen werden möchte, wird die Ertheilung dieser Wohlthat ein größeres Gut; und dieß kann Gott bestimmen, sie dem Geschöpfe wirklich zu ertheilen. Z.\,B.\ ein Kranker wünscht sich Genesung, aber der Vortheil, der aus seiner Genesung für das Ganze entsprünge, würde, an und für sich betrachtet, noch nicht die Vortheile, die sein Tod hätte, aufwiegen. Gott dürfte ihn also auch nicht genesen lassen. Allein nun trifft es sich, daß dieser Kranke Kinder und Freunde hat, welche die inbrünstigsten Fürbitten um seine Genesung zu Gottes Throne senden; durch diesen Umstand wird die Summe des Guten, welches aus der Genesung des Kranken entspringen würde, beträchtlich erhöhet; denn nicht nur, daß die sehnlichsten Wünsche so vieler Menschen auf einmal befriediget werden, es wird auch ihr Zutrauen, ihre Liebe und Dankbarkeit zu Gott vermehrt. Einige aus ihnen haben vielleicht gewisse gute Werke gelobt, \usw\ -- Die Summe des Guten kann jetzt allerdings so groß seyn, daß sie die vorhin erwähnten Nachtheile überwiegt, und daß mithin Gott die Genesung des Kranken beschließen kann und muß.
\item Beruhet die Erhörbarkeit der Fürbitten auf den hier angegebenen Gründen: so läßt sich durchaus nicht absehen,~\RWSeitenw{126}\ warum wir unsere Fürbitten nur eben auf Menschen, die noch am Leben sind, und nicht auf schon Verstorbene ausdehnen dürften. Da uns sowohl Vernunft als Offenbarung lehren, daß der Zustand, der für uns unmittelbar nach dem Tode anhebt, nur in den seltensten Fällen ein Zustand einer so unvermischten und hohen Seligkeit seyn könne, daß es Thorheit seyn müßte, sich einen besseren zu wünschen: so gelten dieselben Gründe, die für die Zulässigkeit der Fürbitten für noch Lebende gelten, auch bei Verstorbenen.
\item Ist es der Güte und Heiligkeit Gottes geziemend, auf die Gebete zu merken, die wir, Bewohner der Erde, also Geschöpfe verrichten, deren Einsicht und Tugend noch so mangelhaft ist; um wie viel mehr müssen die Bitten derjenigen einer Erhörung werth seyn, die schon viel höhere Stufen des Daseyns erstiegen haben!
\item Die katholische Lehre von der Anrufung der Engel und Heiligen bleibt der Vernunft vollkommen angemessen, obgleich wir zugeben müssen, daß die katholische Kirche selbst nicht einig ist über die Frage, ob diese Engel und Heiligen alle unsere an sie gerichteten Gebete oder Anrufungen im Einzelnen gleich auf der Stelle erfahren. Dieß nämlich brauchen wir eben nicht mit Entschiedenheit voraus zu wissen, um uns vernünftiger Weise entschließen zu können, die Engel oder Heiligen um ihre Fürbitte bei Gott anzurufen, und um auch einzusehen, daß eine solche Anrufung nicht ohne Nutzen seyn werde. Zu dem Ersteren, oder um eine Anrufung vernünftiger Weise unternehmen zu können, bedarf es nichts mehr, als daß wir es in diesem Augenblicke wenigstens nicht für etwas ganz Unmögliches halten, daß jene Engel oder Heilige Kunde von unserer Anrufung auf irgend eine Art erhalten; denn, auch damit wir Menschen vernünftiger Weise um Hülfe anrufen könnten, ist keineswegs nöthig, zuvor vollkommen versichert zu seyn, daß wir gehört werden; sondern wir werden, wenn wir in dringender Noth sind, auch dort um Hülfe rufen, wo es nur eine uns nicht in die Augen fallende Unmöglichkeit ist, daß Jemand unser Rufen höre, und dadurch bewogen werde, uns beizuspringen. Ist aber diese Anrufung der Engel oder Heiligen einmal geschehen: so ist auch schon~\RWSeitenw{127}\ gewiß, daß sie nicht ohne Nutzen seyn werde, auch wenn sie in der That von keinem der Wesen, an welche sie unmittelbar gerichtet war, sollte gehört und erhört worden seyn; denn
\begin{aufzb}
\item gewährt uns ja schon die bloße Hoffnung, daß wir vielleicht doch gehört worden sind, einen nicht unbedeutenden Trost; und dann ist es auch
\item kein Zweifel, daß auf jeden Fall Gott den frommen Beter höre, und um so geneigter seyn werde, ihm zu gewähren, um was er bittet, je größer der Eifer ist, den er dadurch an den Tag legt, daß er, nicht zufrieden damit, Gott nur selbst anzuflehen, auch noch so viele andere Mitgeschöpfe, sogar Wesen von höherer Art, zu bewegen sucht, daß sie mit ihm sich vereinigen mögen; je löblicher ferner die Demuth ist, nach welcher der Betende seinen eigenen Bitten den Werth, daß sie erhört werden müßten, nicht zutrauen will; je mehrere wichtige Vortheile endlich die Gewährung gerade dießmal herbeiführen wird, weil sie dem Betenden eine Bestätigung seyn wird, daß die von ihm bisher fromm angenommene Gemeinschaft zwischen Wesen von höherer Art und uns Bewohnern des Staubes in Wahrheit Statt habe.
\end{aufzb}
\item Daß unsere Gebete durchgängig \RWbet{um so wirksamer seyn müssen, je sittlich besser wir sind;} ergibt sich aus den Gründen, auf welchen ihre Wirksamkeit beruhet, von selbst. Je sittlich besser wir sind, um desto vollkommener stimmt der Gegenstand unserer Bitte meistens mit dem überein, was zur Beförderung des allgemeinen Wohles dient; je sittlich besser wir sind, um desto mehr sind wir es werth, Erhörung zu finden, weil wir von der empfangenen Wohlthat guten Gebrauch machen werden, weil sich auch Andere durch unser Beispiel ermuntert fühlen werden, nach gleicher sittlicher Vollkommenheit zu streben, \usw\
\item Hieraus ergibt sich aber von selbst, daß auch \RWbet{gute Werke und Gelübde} die Wirksamkeit unserer Gebete gar sehr erhöhen müssen; denn sie machen uns sittlich besser, und vermehren die Summe des Guten, welches aus der Gewährung unserer Bitte hervorgehen wird.
\item Aus eben diesen Gründen begreifet sich auch, wienach gesagt werden könne, daß die \RWbet{Gebete der Christen}~\RWSeitenw{128}\ \RWbet{wirksamer} seyen, als die der Nichtchristen, und daß \RWbet{für Christen wirksamer als für Nichtchristen} könne fürgebeten werden. Schon aus den reinen Begriffen, welche wir Christen von Gott, von unseren Pflichten und vom Gebete insbesondere haben, folgt, daß wir zweckmäßiger als Andere zu beten verstehen, und daß die Gewährung unserer Bitten des Guten mehr hervorbringen werde, als anderwärts. Dann fordern es auch die Verdienste Jesu und die Beförderung seines uns so nothwendigen Ansehens, daß Gott die Gebete, die im Vertrauen auf seine Fürbitte zu seinem Throne gesendet werden, ingleichen Gebete, die das Wohl seiner Bekenner betreffen, einer vorzugsweisen Erhörung gewürdiget werden.
\item Endlich ist nichts gewisser, als daß der Glaube, \RWbet{wir könnten uns oder Anderen Hülfe verschaffen, durch die Verrichtung gewisser Handlungen, die gleichwohl weder einen durch unsere bloße Vernunft erkennbaren, noch einen aus einer erweislich göttlichen Offenbarung uns angezeigten Nutzen haben,} ein \RWbet{Aberglaube} sey, der nicht nur allemal schädlich, sondern meistentheils auch mehr oder weniger verantwortlich ist. Schädlich, weil wir im eitlen Vertrauen auf die Wirksamkeit solcher Handlungen die Mittel, die uns zu unserem Zwecke geführt haben würden, versäumen. Verantwortlich, weil schon die Trägheit des Geistes, welche ein solcher Glaube voraussetzt, tadelnswerth ist; weil es auch insgemein nur gewisse sinnliche Neigungen sind, die sich durch diesen Glauben geschmeichelt fühlen, und einen bald größeren bald geringeren Antheil an seiner Entstehung oder Fortdauer haben.
\end{aufza}

\RWpar{224}{Sittlicher Nutzen}
Die Bemühung des Christenthums in dieser Lehre, uns zu zeigen, \RWbet{daß unser Wirkungskreis größer sey, als wir uns insgemein vorstellen,} ist schon an sich höchst dankenswerth. Denn es ist nichts ersprießlicher für unsere Tugend und Glückseligkeit, als daß wir die verschiedenen Arten des Wirkens, welche uns zu Gebote stehen, gehörig kennen~\RWSeitenw{129}\ lernen, und von der Wirksamkeit derselben nicht zu geringschätzig denken. Was kann insonderheit
\begin{aufza}
\item ermunternder seyn für den Tugendhaften als zu vernehmen, daß er \RWbet{durch jede einzelne sittlich gute Handlung und Willensentschließung zur Beförderung der Tugend und Glückseligkeit des Ganzen} mehr beitrage, als er mit Deutlichkeit wahrnehmen kann. Von der andern Seite liegt in eben dieser Lehre ein neuer Abschreckungsgrund vor dem Laster; denn wenn wir durch jede gute That mehr Nutzen stiften, als sich bemerken läßt: so müssen wir ja im Gegentheil besorgen, durch jede böse That mehr Schaden anzurichten, als wir wahrnehmen können.
\item Wie nöthig wird es besonders dann, wenn wir ein Werk anfangen sollen, das nur mit vieler Anstrengung vollendet werden kann, und dessen Erfolg, so gut auch unsere Absichten seyn mögen, doch immer ungewiß bleibt; ingleichen wenn uns eine Unternehmung, bei der wir die besten Absichten hatten, scheinbar mißlungen ist, daß wir des Glaubens leben, \RWbet{nicht eine einzige dergleichen Unternehmung könne in Wahrheit mißlingen;} was gut beabsichtiget sey, das müsse jedesmal auch gute Folgen haben. Wer dieß nicht weiß, wird öfters, wenn es sich um die gemeinnützigsten Unternehmungen handelt, unschlüssig da stehen, oder nur muthlos und mit halber Kraft wirken; und wenn die nächsten Folgen, diejenigen, die er mit Augen wahrnehmen kann, von der gehofften Güte nicht sind, der bittersten Betrübniß sich hingeben, daß er statt Gutes nur Böses gestiftet, sich hiedurch abschrecken lassen von fernerem Wirken, ja am Ende wohl gar den Lenker aller Schicksale, Gott selbst, anklagen, daß er die Welt nicht besser eingerichtet habe.
\item Daß dort, wo unsere Kräfte an sich nicht zureichen, \RWbet{auch unser bloßer zum Gebete erhobene Wunsch} eine Kraft habe, den Willen Gottes zu bestimmen, daß er uns entweder das, um was wir bitten, oder etwas noch Besseres gewähre, ist eine der trostreichsten und erbaulichsten Lehren, die eine Offenbarung uns mittheilen kann.
\begin{aufzb}
\item In den betrübtesten Verhältnissen, und wenn unsere Kräfte auch noch so ohnmächtig sind, verzagen wir jetzt~\RWSeitenw{130}\ nicht mehr; sondern wir rufen zu Gott, und fassen Muth, und hoffen das Beste.
\item Wird uns zuletzt geholfen, wird uns zu Theil, um was wir Gott gebeten: so erfreuen wir uns der Wohlthat um so mehr, da uns erlaubt wird, zu denken, daß wir sie uns durch unsere Bitten ausgewirkt hätten.
\item So fühlen wir uns um so stärker verpflichtet, den besten Gebrauch von dieser Wohlthat zu machen.
\item Auch wenn uns am Ende das nicht zu Theil wird, was wir durch unser Gebet zu erlangen hofften: so ist doch schon der Trost, welchen wir mittlerweile in dieser Hoffnung fanden, ein Vortheil von nicht zu berechnendem Werthe. Ohne diese Hoffnung zu haben, wären wir vielleicht ein Raub der Verzweiflung geworden. Nun aber, nachdem die Heftigkeit unserer Empfindungen durch die Länge der Zeit gemildert worden ist; nachdem auch so manche veränderte Verhältnisse eingetreten sind; nachdem wohl auch unsere eigenen Ansichten manche Berichtigung erfahren haben; nachdem wir auf jeden Fall mit dem Gedanken, daß uns vielleicht nicht beschieden sey, unsere Wünsche erfüllt zu sehen, allmählich vertrauter gemacht worden sind: nun sind wir stark genug, ihre Verweigerung ertragen zu können; nun fangen wir an, zu begreifen, daß es doch allerdings noch etwas Besseres geben könne, das die unendliche Weisheit und Güte Gottes für uns erkoren hat, und werden ruhig.
\item In welchem Glanze erscheint uns die Güte Gottes und die Vollkommenheit seiner heiligen Weltregierung, wenn auch nicht eine einzige unserer Bitten vor seinem Throne unerhört bleibt! --
\end{aufzb}\par
\RWbet{Einwurf.} Werden wir aber durch diese Lehre von der Wirksamkeit unserer Bittgebete nicht träge gemacht, und statt zu versuchen, ob wir den Gegenstand unserer Wünsche nicht etwa durch Anstrengung unserer eigenen Kräfte herbeiführen könnten, zu müßigem Gebete unsere Zuflucht nehmen?\par
\RWbet{Antwort.} Dieß steht nicht zu befürchten, wenn wir die Lehre des Christenthums gehörig auffassen; denn zu Folge dieser kann nur ein solches Gebet Ansprüche auf Erhörung~\RWSeitenw{131}\ machen, das auf die rechte Weise verrichtet wird; und dazu gehört, daß wir Alles, was wir zur Erreichung unseres gewünschten und sittlich guten Zweckes selbst beitragen können, mit allem Eifer leisten.
\item Der Glaube, \RWbet{daß auch Fürbitten bei Gott angenommen werden,} gewähret uns folgende Vortheile von größter Wichtigkeit:
\begin{aufzb}
\item Durch diesen Glauben wird uns in unzählbaren Fällen die Gelegenheit gegeben, einen Drang unseres Herzens zu stillen, dessen Nichtbefriedigung für unsere Sittlichkeit mehr oder weniger nachtheilig werden müßte; \zB\ den Drang, das zugefügte Unrecht wieder gut zu machen, wenn wir es werkthätig auf keine Weise vermögen; oder den Drang der Dankbarkeit gegen einen Wohlthäter, der zu entfernt von uns ist, als daß wir ihm unsere Gefühle ausdrücken könnten; oder den Drang, denjenigen zu helfen, die durch die innigsten Bande der Liebe mit uns verknüpft sind, und von einem Uebel bedroht werden, dessen Abwendung gar nicht in unserer Macht steht; \udgl\  Wenn wir in Fällen von solcher Art unsere Zuflucht nicht einmal zu Fürbitten nehmen könnten; wenn wir für solche Personen, die wir durch unsere Unvorsichtigkeit geärgert, oder zum Laster verführt, oder in schädliche und gefährliche Irrthümer gestürzt, oder an ihrer Gesundheit verletzt, oder an die wir eine durch Thaten nie abzutragende Schuld der Dankbarkeit haben, wenn wir für alle diese Personen nicht einmal beten könnten: so würde nur Eines von Beidem eintreten, entweder wir müßten uns beständige Vorwürfe machen, oder wir würden uns die Mahnungen unseres Gewissens auf eine für unsere Sittlichkeit höchst nachtheilige Weise ganz aus dem Sinne schlagen.
\item Betend für Andere können wir uns in den Gefühlen des Wohlwollens gegen Alles, was Mensch heißt, üben; indem wir wissen, daß wir auf diese Weise nichts Unnützes thun, sobald wir nur eben nichts Nützlicheres, \dh\ nichts solches zu thun Gelegenheit haben, dadurch wir Andern werkthätig helfen könnten.~\RWSeitenw{132}\par
\RWbet{Einwurf.} Werden wir aber auf diese Art nicht zu Empfindlern und Schwärmern gebildet?\par
\RWbet{Antwort.} Empfindelei wäre nur dann vorhanden, wenn wir uns in unseren wohlwollenden Gefühlen, als solchen, gefielen, und es bei ihnen bewenden ließen, und über denselben die Gelegenheit zu werkthätigen Beweisen der Liebe verabsäumten. Schwärmerei wäre es nur, wenn wir unserer Einbildungskraft einen zu freien Lauf gestatteten in der Vorstellung der guten Wirkungen, die wir durch unsere Gebete herbeiführen können, \udgl\  Es liegt nun wohl an uns, daß wir dergleichen Verirrungen meiden.
\item Der Glaube an die Wirksamkeit der Fürbitten kann uns vor Stolz und Hochmuth, vor Härte und Gleichgültigkeit bewahren. Denn nun scheine es, daß wir auch noch so unabhängig von Andern da stehen; nun mögen wir so ganz und gar keine in die Sinne fallende Wohlthat von Anderen angenommen haben: so können wir doch durch die Gebete, die sie für unser Wohl angestellt haben, an sie verpflichtet seyn; und wir werden uns also, wenn wir dieses stets vor Augen haben, wohl kaum versucht fühlen, uns über irgend Einen unserer Mitmenschen stolz zu erheben, oder ihn hart, ja auch nur gleichgültig zu behandeln.
\end{aufzb}
\item Indem uns das katholische Christenthum die Versicherung gibt, daß wir \RWbet{auch für Verstorbene noch wirksam fürbitten} können, befestiget es
\begin{aufzb}
\item unseren Glauben an die Unsterblichkeit, und macht, daß uns der Gedanke an das andere Leben und an den Zustand, in dem wir uns dort befinden werden, auf eine sehr heilsame Art geläufiger wird. Besorgen wir nämlich, daß Manche von denjenigen, die uns vorangegangen sind, dort sich in einem Zustande befinden, in welchem unsere Fürbitte für sie so gar nichts Ueberflüßiges seyn dürfte: so muß uns dieß ein beständiger Antrieb seyn, so zu leben, daß wir nicht selbst auch der Fürbitte Anderer einst bedürfen.
\item Durch diesen Glauben wird das Schmerzliche der Trennung, welche der Tod zwischen uns und denjenigen~\RWSeitenw{133}\ unserer Lieben verursacht, die früher, als wir, von diesem Schauplatze der Erde abtreten müssen, um ein Bedeutendes vermindert; denn wir bleiben ja noch mit ihnen verbunden, wir können noch fortwährend auf sie wohlthätig einwirken, wir können noch beten für sie.
\end{aufzb}
\item Doch das katholische Christenthum lehrt, daß nicht nur wir für die Verstorbenen, sondern \RWbet{auch diese und überhaupt alle auf höheren Stufen des Daseyns befindliche Wesen für uns mit Erfolg fürbitten können, und oft wirklich fürbitten.} Durch diese Lehre wird uns
\begin{aufzb}
\item die Lehre von der Erhörbarkeit unserer eigenen Fürbitten glaubwürdiger; denn wenn wir nicht wüßten, daß die Bitten der Engel und der Heiligen Erhörung finden: so könnten wir um so weniger an die Erhörung unserer eigenen Bitten glauben.
\item Das Beispiel dieser höheren Wesen ermuntert uns zu ihrer Nachahmung.
\item Wir fühlen uns nun um so inniger mit ihnen verbunden und ihnen um so mehr verpflichtet.
\end{aufzb}
\item Das katholische Christenthum setzt endlich noch hinzu, \RWbet{daß auch die Anrufung dieser höheren Wesen um ihre Fürbitte bei Gott nie ohne Wirkung verbleibe.} Durch diese Aeußerung wird
\begin{aufzb}
\item veranlaßt, daß wir uns die Verbindung, die zwischen diesen höheren Wesen und uns bestehet, als eine so innige und für uns so wohlthätige Verbindung vorstellen, als es der Wahrheit unbeschadet nur immer geschehen kann. Die Engel und Heiligen, so dürfen wir uns jetzt wenigstens vorstellen, bekümmern sich um jeden unserer Wünsche und Seufzer, und beeilen sich, uns auf die Art, die ihnen ihr höherer Standpunct erlaubt, zu Hülfe zu kommen.
\item Die Hoffnung, daß wir in unseren Nöthen bei Gott Erhörung finden werden, nimmt zu, wenn wir nicht mehr allein zu ihm rufen, sondern wenn auch Wesen, die so viel vollkommener und bei ihm so werth geachtet sind, ihre Bitten mit uns vereinigen.~\RWSeitenw{134}
\item Der Gedanke, daß wir nicht werth seyen, diese und jene Wohlthat von Gott durch unser eigenes Gebet uns zu erwirken, daß er es aber vielleicht doch thun werde, nur weil auch Andere, mitunter selbst höhere Wesen, verklärte Tugendfreunde, ihn um dieselbe anrufen, ist eine vortreffliche und in gewissen Fällen, \zB\ wenn wir so eben erst uns schwer versündiget haben, recht nöthige Uebung der Demuth.
\item Wird uns die Wohlthat, um die wir Gott gebeten, gewähret: so ist es, wenn wir auch höhere Wesen um ihre Fürbitte angerufen haben, billig, uns vorzustellen, daß wir jene Gewährung zum Theile vielleicht nur dieser Fürbitte verdanken; und wenn wir dieß thun: so üben wir uns in den Gesinnungen einer echten Bescheidenheit und eines dankbaren Herzens; denn ein wahrhaft bescheidener und dankbarer Mensch muß bei allem Guten, das er empfängt, mit Fleiß nachforschen, wem er es, wenigstens theilweise, zuzuschreiben habe; und muß aufgelegt seyn, sich selbst denjenigen, gegen die er eine, und wäre es auch nur wahrscheinliche, Verpflichtung hat, erkenntlich zu bezeugen.
\end{aufzb}
\item Die Erklärung, daß Gott unsere Bittgebete um so mehr bei sich gelten lassen wolle, \RWbet{je höher der Grad unserer sittlichen Vollkommenheit ist,} dient uns
\begin{aufzb}
\item zu einem neuen Beweise von Gottes Heiligkeit; zeigt uns
\item von einer neuen Seite den hohen Werth der Tugend; und legt uns
\item eine ganz eigene Verbindlichkeit auf zu einem tadellosen Wandel, wenn wir durch unsere Verhältnisse verpflichtet sind, für Andere zu bitten. Personen, bei denen es, wie bei den Geistlichen, eine eigene Standespflicht ist, für Andere fürzubitten, die wohl gar einen Theil ihres Unterhaltes nur eben dafür beziehen, müssen erkennen, daß sie sich eigens verantwortlich machen, wenn sie durch ihren schlechten sittlichen Charakter selbst daran Schuld sind, daß ihre Fürbitten bei Gott nicht angenehm sind. -- In dieser Lehre liegt also ein eigener Beweggrund für sie, sich zu bessern.~\RWSeitenw{135}
\end{aufzb}
\item Wissen wir ferner, daß wir die Wirksamkeit unserer Bitten am Meisten dadurch erhöhen können, \RWbet{daß wir mit ihnen gewisse gute Werke und Gelübde verbinden:} so ist dieß ein neuer Antrieb, Gutes zu thun, und nie darin müde zu werden.
\item Daß die \RWbet{Gebete der Christen bei Gott wirksamer sind, als die der Nichtchristen}, muß ein Beweggrund mehr für uns seyn:
\begin{aufzb}
\item Christum zu ehren und zu lieben;
\item unsere Gebete ganz so zu verrichten, wie er uns vorgeschrieben; und
\item die Erhörung derselben auch nur um seiner Verdienste wegen zu erwarten.
\item Wissen wir ferner, daß die \RWbet{Gebete der Christen für andere Christen wirksamer sind, als die für Nichtchristen;} so muß uns dieses ein Beweggrund mehr seyn, für unsere christlichen Brüder recht fleißig fürzubitten; auch
\item ein Grund mehr, sie zu lieben und uns ihnen verpflichtet zu halten; ingleichen
\item in ihrer Gemeinde zu verbleiben, oder wenn wir durch eigene Verschuldung oder durch Mißverstand ausgeschlossen seyn sollten, dahin zu wirken, damit wir wieder aufgenommen, und so der Fürbitten Aller wieder theilhaftig werden mögen.
\end{aufzb}
\item Die Warnung vor derjenigen Art von \RWbet{Aberglauben}, der heilsame Wirkungen von Dingen erwartet, in Betreff deren gar kein vernünftiger Grund vorhanden ist, um dergleichen Wirkungen zu erwarten, ist selbst für Gebildete nicht überflüssig; denn die Erfahrung lehrt, daß auch diese sich oft nur allzu leichtsinnig bezeugen, wenn sie der Anschein des Geheimnißvollen, oder, was noch viel schlimmer ist, irgend eine Leidenschaft verblendet.
\end{aufza}

\RWpar{225}{Wirklicher Nutzen}
Wie sehr ist doch jede Religionsgesellschaft zu bedauern, die diese vortrefflichen Lehren nicht kennt, oder sie nur zum~\RWSeitenw{136}\ Theile annimmt. Dieß Letztere ist ein Vorwurf, den wir den protestantischen Gemeinden machen können; denn diese wollen uns nicht verstatten, daß wir in unseren Gebeten der bereits Verstorbenen eingedenk seyen; sie wollen nicht zugeben, daß die Verstorbenen, wenn sie auch noch so tugendhaft auf dieser Erde gelebt, oder was immer für andere Wesen, welche auf einer höheren Stufe des Daseyns stehen, wirksam für uns fürbitten können; sie wollen noch weniger erlauben, daß wir sie um diese Fürbitten anrufen. Nichts als zeitweiliger Mißbrauch dieser Lehrpuncte war es, der die Protestanten, sie zu verwerfen, bestimmte. Aber so groß auch dieser Mißbrauch in gewissen Zeiten gewesen seyn mochte: so hat er doch sicher nie die wohlthätigen Wirkungen derselben überwogen, und es war auf jeden Fall Unrecht, wegen des Mißbrauches auch selbst den guten Gebrauch zu untersagen. Um so weniger läßt sich der Eigensinn rechtfertigen, mit dem man sich der Annahme dieser an sich so heilsamen Lehren auch noch jetzt widersetzt, wo man der Mittel genug hat, um einem solchen Mißbrauche derselben vorzubeugen.

\RWabs{Sechster Abschnitt}{Die Lehre von der Zukunft, oder von den Belohnungen und Strafen in diesem und in jenem Leben}
\RWpar{226}{Inhalt dieses Abschnittes}
Nachdem uns das Christenthum über Gott und seine Eigenschaften, über seine Werke und über die Art, wie er sie leitet, über unser eigenes Wesen, und über die Ursache der vielen Leiden, die uns jetzt noch immer drücken, dann auch über die Anstalten, welche Gott selbst zu unserer Besserung und Beglückung getroffen hat, endlich auch über den Einfluß, den wir der Eine wechselseitig auf den Anderen haben, zur Genüge belehrt hat, beschließt es den theoretischen Theil seiner~\RWSeitenw{137}\ Lehren mit einem \RWbet{Unterrichte über die Zukunft.} Da aber Alles, oder doch beinahe Alles, was uns in Zukunft bevorsteht, sich nach dem Gebrauche unserer Freiheit richtet, und also entweder als Belohnung oder als Strafe sich ansehen läßt: so ist die katholische Lehre von der Zukunft nur eine \RWbet{Lehre von den Belohnungen und Strafen}, die uns bevorstehen. Die zahlreichen und wichtigen Sätze, welche in diesen Abschnitt gehören, wollen wir der Kürze wegen nur unter zwei Titel bringen:
\begin{aufza}
\item \RWbet{Allgemeine Lehrsätze von Lohn und Strafe,} \dh\ solche Lehrsätze, die von Belohnungen und Strafen sowohl in diesem als in jenem Leben gelten.
\item \RWbet{Besondere Lehrsätze von den Belohnungen und Strafen im andern Leben.}
\end{aufza}

\RWpar{227}{Allgemeine Lehrsätze des katholischen Christenthums von Lohn und Strafe}
\begin{aufza}
\item Das katholische Christenthum erkläret, daß sich die Schicksale, die wir ein Jeder, theils schon in diesem Leben erfahren, theils noch im anderen zu erwarten haben, wenn nicht durchgängig, doch größtentheils, nur \RWbet{nach unserem eigenen Verhalten} richten; Gott selbst nämlich, unter dessen Leitung sie alle stehen, nimmt bei der Anordnung derselben seine vornehmste Rücksicht auf unser sittliches Betragen; so zwar, daß
\item eine \RWbet{jede sittlich gute Handlung}, die wir verrichten, bei Gott einen Grund \Ahat{abgibt}{angibt}, uns \RWbet{eine gewisse Glückseligkeit, als Belohnung} derselben zufließen zu lassen, während umgekehrt eine \RWbet{jede sittlich böse Handlung} einen Grund enthält, der Gottes Heiligkeit bestimmt, uns \RWbet{eine gewisse Unglückseligkeit als Strafe} zuzugedenken. Jene Beschaffenheit einer sittlich guten Handlung, vermöge deren sie einen Grund zur Belohnung enthält, heißt die \RWbet{Verdienstlichkeit} derselben, während die Beschaffenheit einer sittlich bösen Handlung, vermöge deren sie einen Grund zur Bestrafung enthält, eine \RWbet{Schuld} genannt wird.~\RWSeitenw{138}
\item Der \RWbet{Unterschied, der zwischen Schuld und Strafe besteht}, ist von der Art, daß in manchen Fällen gesagt werden darf, die \RWbet{Schuld sey von uns hinweggenommen worden, während noch eine Strafe}, eine endliche wenigstens, bleibt.
\item \RWbet{Nicht alle sittlich guten Handlungen haben ein gleich großes Verdienst, nicht alle sittlich bösen eine gleich große Schuld.}
\item Es lassen sich überhaupt \RWbet{zwei Arten guter sowohl als böser Handlungen} unterscheiden.
\begin{aufzb}
\item Es gibt nämlich unter den \RWbet{sittlich guten} Handlungen
\begin{aufzc}
\item solche, deren \RWbet{Ausübung eine Belohnung erwirkt, ohne daß ihre Unterlassung mit einer eigentlichen Strafe verbunden wäre}; es sey denn, daß man das bloße Ausbleiben der Belohnung schon eine Strafe nennen wollte. Dergleichen Handlungen werden \RWbet{verdienstliche} genannt. Es gibt aber auch
\item sittlich gute Handlungen, auf deren \RWbet{Ausübung nicht nur eine Belohnung, sondern auf deren Unterlassung überdieß noch eine Strafe gesetzt ist.} Man nennt sie \RWbet{Schuldigkeit, Pflichten} im engsten Sinne, oder \RWbet{strenge Pflichten.}
\end{aufzc}
\item Unter den \RWbet{sittlich bösen} Handlungen, die man auch \RWbet{Sünden} nennt, unterscheidet das katholische Christenthum
\begin{aufzc}
\item \RWbet{läßliche}, \dh\ solche, die eine nur \RWbet{zeitliche Strafe} nach sich ziehen; und 
\item \RWbet{schwere} oder \RWbet{Todsünden}, durch die wir eine \RWbet{ewige Strafe} verschulden.
\end{aufzc}
\end{aufzb}
\item Obgleich sich nicht immer mit Bestimmtheit ausmachen läßt, ob eine gewisse sittlich gute Handlung zu den verdienstlichen oder den bloßen Schuldigkeiten, eine sittlich böse dagegen zu den bloß läßlichen oder den schweren Sünden gehöre: so gelten doch folgende Regeln, die dieß wenigstens in den meisten Fällen mit hinreichender Sicherheit entscheiden.
\begin{aufzb}
\item Eine sittlich gute Handlung ist keine Schuldigkeit, sondern \RWbet{bloß verdienstlich}, wenn durch ihre Ausübung~\RWSeitenw{139}\ \RWbet{das Wohl des Ganzen zwar befördert}, durch ihre Unterlassung aber doch \RWbet{Niemandem ein wirklicher Schaden zugefügt} wird; sie überdieß auch so beschaffen ist, daß ihre Ausübung \RWbet{manche Opfer und Beschwerlichkeiten} erfordert, von denen es nicht gleich auf den ersten Blick einleuchtet, \RWbet{daß sie geringer sind als der Vortheil}, der für das Ganze hervorgeht.
\item Eine sittlich böse Handlung ist \RWbet{keine Todsünde}, wenn der Schaden, den sie verursacht, \RWbet{so gering ist, daß eine weitläufige Ueberlegung, ob man sie ausüben solle oder nicht, mit der Gefahr verknüpft ist, Dinge, die wichtiger sind, zu versäumen;} ingleichen wenn sich der Handelnde \RWbet{nicht deutlich bewußt war, daß er hier dem Gesetze zuwider gehandelt habe.}
\end{aufzb}
\item \RWbet{Alle ersprießlichen} (\dh\ der Tugend und Glückseligkeit des Ganzen zuträglichen) \RWbet{Folgen,} welche ein Handelnder von seiner Handlung \RWbet{erwartet}, und um derentwillen er sich zu ihrer Unternehmung auch entschließt, \RWbet{erhöhen die Verdienstlichkeit seiner Handlungen} (\dh\ sie geben einen Grund ab, der Gottes Heiligkeit zur Vermehrung seiner Belohnungen bestimmt,) und dieß zwar \RWbet{gleichviel, ob sie in Wirklichkeit erfolgen oder nicht.} Im Gegentheile aber, \RWbet{alle gemeinschädlichen Folgen}, die er \RWbet{erwartete}, und die ihn gleichwohl von ihrer Unternehmung nicht abhalten konnten, \RWbet{vermehren seine Schuld,} und dieß zwar abermals, \RWbet{sie mögen erfolgen oder nicht erfolgen.}
\item Wer eine \RWbet{gute Handlung} (und zwar aus sittlichem Grunde) unternimmt, dem sollen \RWbet{selbst jene ersprießlichen Folgen derselben, welche er nie vorhergesehen hatte}, die aber doch durch Zufall eintraten, in einem gewissen, obgleich \RWbet{geringeren Grade}, als die vorigen, \RWbet{zum Verdienste angerechnet} werden. Im Gegentheile, wer eine \RWbet{böse Handlung} (und zwar aus bösem Grunde) unternimmt, dem sollen \RWbet{selbst jene gemeinschädlichen Folgen derselben, die er nicht vorher}\RWSeitenw{140}\RWbet{gesehen hatte}, auch nicht vorhersehen \RWbet{konnte}, die aber doch wirklich eintraten, in einem gewissen, obgleich \RWbet{geringeren Grade}, als wenn er sie vorhergesehen hätte, \RWbet{zur Schuld angerechnet} werden.
\item So sollen uns insbesondere auch alle \RWbet{fremden Sünden,} \dh\ alle Sünden eines Andern, die wir durch irgend eine eigene böse Handlung veranlaßt haben, \RWbet{zur Schuld angerechnet} werden, und dieß zwar, selbst dann, \RWbet{wenn wir sie nicht vorhergesehen}, auch nicht einmal hatten voraussehen \RWbet{können}.
\item Für das \RWbet{Böse} dagegen, welches aus einer in \RWbet{guter Absicht} und mit \RWbet{gehöriger Klugheit} von uns verrichteten Handlung ganz \RWbet{wider unseren Willen} hervorgehet, haben wir \RWbet{nichts zu verantworten}. Eben so dürfen wir aber auch umgekehrt von jenem \RWbet{Guten}, das durch die weise Leitung Gottes aus unseren\RWbet{ bösen Handlungen} entspringt, \RWbet{keine Verminderung unserer Schuld} erwarten; obgleich es uns hintenher, wenn wir die böse Absicht bereuet und uns gebessert haben, zu einigem Troste gereichen kann.
\item Wer \RWbet{tugendhaft} lebt, \dh\ die herrschende Gesinnung hat, dem Sittengesetze in allen Stücken nachzukommen, der darf, auch wenn es ihm noch nicht gelingt, \RWbet{jede Gebrechlichkeit oder läßliche Sünde} zu meiden, doch eine \RWbet{überwiegende Belohnung }seiner Tugend hoffen.
\item Wer dagegen in einer \RWbet{lasterhaften Gesinnung} lebt, \dh\ den herrschenden Willen hat, das Sittengesetz in gewissen Stücken zu übertreten, mag sich \RWbet{mit allen den übrigen guten Handlungen,} die er daneben ausübt,\RWbet{ nicht im Geringsten trösten.} Es ist so weit entfernt, daß durch ihr Verdienst die Schuld seiner bösen Thaten könnte aufgewogen werden, daß sie im Gegentheile \RWbet{nicht einmal zu einem wahren Verdienste} können angerechnet werden, weil sie nicht wirklich, sondern nur scheinbar gut sind. Nicht aus der Gerechtigkeit Gottes folgt es mit Nothwendigkeit, sondern aus seiner bloßen Gnade geschieht es, wenn solche Handlungen die Strafe des Sünders vermindern, oder wohl gar die Veranlassung zu seiner endlichen Besserung werden.~\RWSeitenw{141}
\item \RWbet{Wer früher tugendhaft gelebt, dann aber in eine herrschende sittlich böse Gesinnung verfällt}; \Ahat{den}{dem} werden von nun an \RWbet{selbst jene wirklich guten Handlungen, die er in seinem noch tugendhaften Zustande ausgeübt hatte, der Strafe nicht entreißen,} sondern sie sind nun alle gleichsam \RWbet{erloschen.} Doch \RWbet{nicht auf immer} und unwiederbringlich; sondern, wofern er sich einst wieder bessert: so \RWbet{leben auch jene guten Handlungen, die er in früherer Zeit verrichtet hatte, zwar wieder auf, doch nicht mehr völlig in ihrem vorigen Glanze,} \dh\ sie werden ihm jetzt zwar wieder zum Verdienste angerechnet, doch haben sie etwas von ihrem Werthe verloren.
\item Wer in einer \RWbet{tugendhaften Gemüthsverfassung} lebt, darf \RWbet{jedes Glück}, das ihm zu Theil wird, \RWbet{als einen Beweis des göttlichen Wohlgefallens an ihm,} als eine \RWbet{Belohnung} ansehen; jedes \RWbet{Unglück} aber darf er als ein Leiden betrachten, \RWbet{das ihm Gott nur zu seiner Prüfung und Vervollkommnung zuschickt}, und das, wenn er es anders mit geduldiger Ergebung in Gottes Willen erträgt, und zu seiner sittlichen Vervollkommnung anwendet, sich nur in Glück und Heil für ihn auflösen wird. Der \RWbet{Lasterhafte} dagegen, und Jeder, der eine vorsätzliche Sünde begangen, muß \RWbet{jedes Glück}, das ihm zu Theil wird, als eine \RWbet{unverdiente Aufforderung Gottes zu seiner Besserung; jedes Unglück} aber als einen \RWbet{Anfang der von Gott über ihn verhängten Strafen} betrachten.
\end{aufza}

\RWpar{228}{Historischer Beweis dieser Lehrsätze}
\begin{aufza}
\item Daß sich unsere Schicksale vornehmlich nur \RWbet{nach unserem eigenen Verhalten} richten, wird unter uns Christen so allgemein geglaubt, daß eben darum auch unter uns das Sprichwort: \RWlat{Quilibet fortunae suae faber est}\RWlit{}{Walther1}, bestehet. So lesen wir auch in den Büchern des alten Bundes, daß Gott dem Volke Israel ausdrücklich vorhergesagt habe, es werde demselben wohl oder schlimm ergehen, je nach\RWSeitenw{142}dem es das ihm gegebene Gesetz getreu befolgen, oder es außer Acht lassen werde.
\item Daß insonderheit eine \RWbet{jede sittlich gute Handlung von Gott belohnt} und eine \RWbet{jede sittlich böse von Gott gestrafet} werde, und daß somit der ersteren ein \RWbet{Verdienst,} der letzteren eine \RWbet{Schuld} beizulegen sey: erhellet aus den Worten Jesu (\RWbibel{Mt}{Matth.}{10}{41}): \erganf{Wer einen Propheten aufnimmt, im Namen eines Propheten (\dh\ eben weil er Prophet ist), der wird den Lohn eines Propheten (den für die Aufnahme eines Propheten bestimmten Lohn) empfangen; und wer einen Gerechten aufnimmt im Namen des Gerechten, der wird den Lohn eines Gerechten empfangen. Und wer nur irgend einem dieser Geringen einen \RWbet{Trunk frischen Wassers} reichet, im Namen eines Schülers: in Wahrheit sage ich euch, er wird nicht seines Lohnes verlustig werden.} Und \Ahat{\RWbibel{Mt}{Matth.}{12}{36}}{12,10.}: \erganf{Ich aber sage euch, daß die Menschen von einem \RWbet{jeden unnützen Worte}, das sie geredet haben, am Tage des Gerichtes Rechenschaft ablegen werden.}
\item Zwischen \RWbet{Schuld und Strafe} (\RWlat{culpa et poena}) haben die katholischen Theologen von jeher unterschieden; und insbesondere gelehrt, daß, wer sich bessert, durch die Verdienste Jesu Christi wohl die Verzeihung seiner \RWbet{Schuld}, nicht aber alsbald auch eine gänzliche Nachlassung seiner \RWbet{Strafe} erhalte.
\item Daß es \RWbet{Grade der Verdienstlichkeit sowohl als auch der Schuld} gebe, lehren die Worte Jesu (\RWbibel{Joh}{Joh.}{19}{11}): Der mich an dich überlieferte, hat eine größere Sünde (\RWgriech{me'izona <amart'ian}) begangen. Ingleichen (\RWbibel{Mt}{Matth.}{5}{22}): \erganf{Ich aber sage euch, wer auch nur ohne Grund (\RWgriech{e>ik~h|}\editorischeanmerkung{\RWgriech{e>ik~h|} steht nur bei einem Teil der Textzeugen.}) auf seinen Bruder zürnet, der ist schon schuldig des Gerichtes; wer aber überdieß ihn Raka (\RWgriech{<rak`a}) schilt, verdient vor das Synedrium gerufen zu werden; wer ihn vollends lächerlich macht (\RWgriech{<'os d'>`an e>'iph| mwr`e}) verdient im Thale Gehinnon verbrannt zu werden.}
\item Daß es
\begin{aufzb}
\item \RWbet{sittlich gute Handlungen gebe, deren Ausübung nur belohnet wird, ohne daß ihrer Unterlassung eine Strafe droht}, erhellet aus \RWbibel{1\,Kor}{1\,Kor.}{7}{28} \RWbibel[37.]{1\,Kor}{}{7}{37},~\RWSeitenw{143}\ wo Paulus von einem gewissen Rathe, den er daselbst ertheilet, ausdrücklich anmerkt, daß man sich nicht versündige (\RWgriech{o>uq <'hmartes}), wenn man ihn nicht befolget. Dieses Nichtversündigen kann hier nichts Anderes bedeuten, als daß man keine Strafe dafür zu befürchten habe. Daß es aber auch \RWbet{sittlich gute Handlungen} gebe, \RWbet{durch deren Unterlassung wir uns strafwürdig machen}, beweiset der Ausspruch \RWbibel{Jak}{Jak.}{4}{17}: \erganf{Wer Gutes zu thun weiß, und es nicht thut; dem ist es Sünde.}  Dieses kann nur heißen: Wer mit Bestimmtheit merkt, daß etwas sittlich gut ist, und daß er es somit thun sollte, und er unterlässet es gleichwohl, aus bloßer Gleichgültigkeit gegen das Sittengesetz, der wird nicht ungestraft bleiben.
\item Daß es auch unter den \RWbet{sittlich bösen Handlungen} einen solchen Unterschied gebe, wie ihn die Kirche zwischen den \RWbet{läßlichen} oder \RWbet{Gebrechlichkeits-} und \RWbet{schweren} oder \RWbet{Todsünden} annimmt, erhellet aus \RWbibel{1\,Joh}{1\,Joh.}{5}{16}: \erganf{Wenn Jemand bemerkt, daß sein Bruder eine Sünde begehe, die jedoch \RWbet{nicht eine Todsünde} ist (\RWgriech{<amart'ian m`h pr`os j'anaton}): so bete er für ihn, und er wird ihm hiedurch das Leben geben. Ist's aber eine \RWbet{Sünde zum Tode}: so verlange ich nicht, daß er für einen solchen bete.} -- Es gibt also Todsünden, und es gibt andere, die geringer sind, so zwar, daß sie uns durch die bloße Fürbitte eines Andern erlassen werden können. Der heil.\ Paulus (\RWbibel{1\,Kor}{1\,Kor.}{6}{9}) zählt mehrere Sünden auf, von denen er ausdrücklich beisetzt, daß Menschen, die dergleichen Sünden begehen, nicht in das Himmelreich eingehen werden (\RWgriech{basile'ian Jeo~u o>u klhronom'hsousin}). Es gibt also Sünden, die ewig unglücklich machen.
\end{aufzb}
\item Die oben angegebenen zwei Bestimmungen darüber, in welchen Fällen eine sittlich gute Handlung \RWbet{bloß verdienstlich,} eine sittlich böse aber noch \RWbet{keine Todsünde} ist, beruhen auf Begriffen, welche viel zu zusammengesetzt sind, als daß man sie in einem Buche, das, wie die heil.\ Schrift, nicht für Gelehrte geschrieben ist, erwarten könnte. In den~\RWSeitenw{144}\ Schriften der katholischen Sittenlehrer aber sind diese Bestimmungen allerdings anzutreffen.
\item Daß die \RWbet{ersprießlichen} oder die \RWbet{nachtheiligen Folgen}, die wir von unseren Handlungen als \RWbet{möglich vorstellen}, ihre \RWbet{Verdienstlichkeit} sowohl als ihre \RWbet{Strafwürdigkeit erhöhen}, wenn sie im ersten Falle \RWbet{mit zu den Bestimmungsgründen unseres Entschlusses} gehörten, im letztern uns wenigstens davon \RWbet{nicht abgehalten haben}: wird unter allen gebildeten Völkern allgemein angenommen.
\item Daß aber auch selbst solche ersprießliche Folgen unserer guten, und solche schädliche unserer bösen Handlungen, \RWbet{welche wir nicht vorhergesehen, nicht einmal vorhersehen konnten,} uns zum Verdienste sowohl als auch zur Schuld, obgleich nur in einem geringeren Grade, angerechnet werden können, ist wenigstens unter Christen immer vorausgesetzt worden. Nur daher kommt es \zB , daß wir demjenigen, der eine Erziehungsanstalt gegründet hat, alles das Gute zum Verdienste anrechnen, was die aus derselben hervortretenden Zöglinge stiften, obgleich er diese Folgen im Einzelnen gewiß nicht vorhergesehen hatte.
\item Die Lehre von der Zurechnung \RWbet{fremder Sünden} erweiset die Stelle \RWbibel{Ez}{Ezech.}{33}{7}: \erganf{Dich, Menschenkind! habe ich zum Wächter über Israel gesetzt, aus meinem Munde sollst du das Wort hören, und sie in meinem Namen warnen. Wenn ich zum Gottlosen spreche: Du Gottloser wirst des Todes sterben, und du redest ihm nicht zu, und er bessert sich nicht auf seinem Wege: so wird der Gottlose zwar um seiner Missethat willen sterben, aber \RWbet{sein Blut werde ich von deiner Hand fordern.} Wenn du ihn aber ermahnst, und er bekehrt sich doch nicht: so wird er zwar um seiner Missethat willen sterben, \RWbet{du aber hast deine Seele gerettet}.} Wer also dadurch, daß er zu warnen unterläßt, \dh\ durch eine sittlich böse Handlung, Ursache daran ist, daß sich der Gottlose nicht bessert, dem wird die Sünde desselben angerechnet.
\item Aus eben dieser Stelle erhellet zugleich, daß wir dasjenige \RWbet{Böse,} was durchaus \RWbet{wider unseren Willen}~\RWSeitenw{145}\ aus unseren guten Handlungen hervorgeht, nicht zu verantworten haben. Hatte Ezechiel die Gottlosen gewarnt, und verachteten diese die Warnung: so hatte der Prophet doch seine eigene Seele gerettet. Daß wir dagegen diejenigen guten Folgen, die durch die weise Leitung Gottes oft selbst aus unseren bösen Handlungen hervorgehen, uns keineswegs zu einem Verdienste anrechnen können, setzet die heil.\ Schrift allenthalben stillschweigend voraus, wenn sie die Größe des Verbrechens, welches \zB\ die Juden begingen, als sie den Urheber des Lebens an das Kreuz geschlagen, nirgends durch die Bemerkung entschuldigt, daß hieraus die gesegnetesten Folgen für das ganze menschliche Geschlecht hervorgingen. Daß wir uns jedoch hintenher, wenn wir uns einmal gebessert, mit diesem Gedanken immerhin trösten dürfen, setzet die Kirche voraus, wenn sie \zB\ von Adam's Sündenfalle ausruft: \RWlat{O felix culpa, quae talem meruit habere Redemptorem!} -
\item Daß der Tugendhafte bei aller seiner Gebrechlichkeit doch auf \RWbet{überwiegende Belohnung} hoffen könne, beweiset \zB\ gleich die Stelle \RWbibel{Spr}{Sprichw.}{24}{16}, wo es heißt: \erganf{Auch der Gerechte sündiget des Tages siebenmal;} -- denn es ist doch bekannt, daß die heil.\ Schrift dem Gerechten die Aussicht auf eine glückliche Zukunft verstattet. So heißt es auch \Ahat{\RWbibel{Apg}{Apostelg.}{10}{35}}{10,15.}: \erganf{Wer Gott fürchtet und recht thut, der ist Gott angenehm, aus welchem Volke er auch abstammen mag.} Wer nun Gott angenehm ist, der hat doch sicher die Aussicht auf eine glückliche Zukunft.
\item Daß sich der \RWbet{Lasterhafte des Guten,} das er doch nebenbei thut, \RWbet{nicht trösten} dürfe, erhellet aus \RWbibel{Jak}{Jak.}{2}{10}: \erganf{Wer auch das ganze Gesetz hält, aber in Einem Puncte dasselbe übertritt, macht sich der Uebertretung aller Puncte schuldig}, welches offenbar nur den Sinn haben kann, daß die Beobachtung dieser übrigen Puncte wenigstens nichts wahrhaft Verdienstliches sey, die Strafe, die wir durch die Uebertretung des Einen Punctes verschuldeten, nicht aufheben könne. Daher sagt denn auch Paulus \RWbibel{1\,Kor}{1\,Kor.}{13}{3}: \erganf{Wenn ich auch alle meine Habe dahin gäbe, und meinen Leib verbrennen ließe, die Liebe aber nicht hätte: so würde es mir~\RWSeitenw{146}\ nichts nützen.} Das Uebrige in diesem Puncte ist nicht eine eigentliche Glaubenslehre, sondern nur Meinung der besten Theologen. So sagt der heil.\ Thomas: \RWlat{opera bona extra charitatem facta ex condigno} (\dh\ so, daß wir es fordern könnten) \RWlat{nullius boni meritoria sunt; ex congruo vero} (\dh\ nach dem, was durch bloße Gnade Gottes geschieht) \RWlat{meritoria dici possunt.}\RWlit{}{ThomasAquinas1}
\item Daß das Verdienst unserer Tugend erlösche, wenn wir in Sünden verfallen, lehret \RWbibel{Ez}{Ezech.}{33}{12}: \erganf{Die Tugend des Gerechten wird ihn nicht retten am Tage, da er abfällt. Wenn ich dem Gerechten verheiße, er soll leben; er aber verläßt sich auf seine Tugend und thut jetzt Unrecht: so soll all seiner Tugend nicht mehr gedacht werden um seiner Missethat wegen, und er soll sterben.} Das Uebrige in diesem Absatze ist ebenfalls keine eigentliche Glaubenslehre; aber so lehren der heil.\ Thomas, \uA
\item Daß der Tugendhafte das Glück als Belohnung, das Unglück nicht als Strafe, sondern als Prüfung ansehen dürfe; der Lasterhafte dagegen im Unglücke die Strafe Gottes erblicken solle -- ist allgemeiner Glaube der Christen, und in der heil.\ Schrift wird es auch allenthalben so dargestellt.
\end{aufza}

\RWpar{229}{Vernunftmäßigkeit}
\begin{aufza}
\item Daß sich die Schicksale, die uns bevorstehen, \RWbet{vornehmlich nur nach unserem eigenen Verhalten }richten, ist eine Wahrheit, die wir schon durch die bloße Vernunft erkennen.
\item Daß aber insonderheit \RWbet{jede sittlich gute That uns ein Verdienst vor Gott erwerbe, und jede böse uns vor ihm schuldig mache,} ist eine einleuchtende Folge aus Gottes Heiligkeit, und zwar aus demjenigen Theile derselben, welchen man die \RWbet{Gerechtigkeit} nennt. Wir können nämlich sehr deutlich einsehen, daß Gott auf diese Art verfahren müsse, wenn er die Tugend möglichst befördern, und dem Laster steuern soll.
\item Die \RWbet{Schuld} ist allerdings von der \RWbet{Strafe} gerade so, wie das \RWbet{Verdienst} von der \RWbet{Belohnung} zu unterschei\RWSeitenw{147}den, jene ist nämlich der Grund und diese die Folge. Aber eben weil der Unterschied von einer solchen Art ist: so ist es, möchte man sagen, ungereimt, daß die Schuld aufgehoben werden, und doch die Strafe bleiben könne. Wenn der Grund behoben ist, muß auch die Folge ausbleiben. Eher ließe sich wohl noch begreifen, daß ein Grund ohne seine Folge (nämlich wenn diese durch irgend einen andern hinzukommenden Grund behoben wird), als daß die Folge ohne ihren Grund bestehe. Hierauf erwiedere ich aber, daß zwischen den beiden Behauptungen, die Schuld ist aufgehoben und die Strafe dauert noch, nur dann ein Widerspruch wäre, wenn beide nicht nur ganz eigentlich, sondern auch in Beziehung auf eine und dieselbe Strafe ausgelegt werden müßten. Ungereimt wäre es nur, zu behaupten, daß dieselbe Strafe, deren Schuld von uns genommen worden ist, noch immer fortdauere. Dieß aber lehret kein katholischer Theolog; sondern die Meisten verstehen unter der Schuld, die durch die Besserung aufgehoben wurde, die Schuld der ewigen Strafe, unter der Strafe aber, die noch fortdauern könne, eine bloß zeitliche Strafe. Hiedurch verschwindet denn aller Widerspruch. -- Hiezu kommt noch, daß wir unter der Redensart: Die Schuld ist aufgehoben, füglich auch nur so viel verstehen könnten, daß uns Gott die Erlaubniß gebe, uns vorzustellen, daß wir von nun an nicht mehr ein Gegenstand seines Mißfallens sind; vorausgesetzt, daß wir bereitwillig sind, die einmal über uns verhängte Strafe (eine bloß endliche nämlich) geduldig zu ertragen.
\item Die Stoiker hatten behauptet, daß alle Sünden einander gleich wären; auch in des Cicero Buche \RWlat{de paradoxis}\RWlit{}{Cicero3a} lautet das dritte Paradoxon: \RWlat{Omnia peccata esse aequalia.} -- Auch einige neueren Gelehrten haben diese seltsame Behauptung nachgesprochen. Wahr ist es freilich, daß die Sünde an sich, als eine bloße Uebertretung des Sittengesetzes, keinen Grad oder keine Größe habe; denn nur dasjenige, was sich durch eine Einheit und durch gewisse Zahlen bestimmen läßt, hat einen Grad oder eine Größe. Das Wollen desjenigen, was das Sittengesetz verlangt, oder das Nichtwollen desselben ist aber keineswegs eine Sache, zu deren Bestimmung man einer Einheit und gewisser Zahlen bedürfte. --~\RWSeitenw{148}\ Allein wenn auch die gute Handlung und die Sünde an sich keinen Grad haben: so hat doch die Verdienstlichkeit einer guten, und die Schuld einer bösen Handlung, oder ihre Strafwürdigkeit einen Grad. Dieses erhellet daraus, weil Verdienstlichkeit und Schuld die Gründe von Belohnungen und Strafen sind. Belohnung und Strafe aber, oder Glückseligkeit \Ahat{und}{oder} Unglückseligkeit haben allerdings einen Grad. Wenn nun die Folge einen Grad hat: so muß ihn jederzeit auch der Grund haben. Das Christenthum fehlt also gar nicht, wenn es den guten sowohl als auch den bösen Handlungen einen Grad beilegt; denn es versteht dieß eigentlich nur von ihrer Verdienstlichkeit oder Schuld. Aber die guten und bösen Handlungen haben nicht nur einen Grad, sondern auch einen verschiedenen Grad; denn obwohl sie einander alle in der Rücksicht gleich sind: so gibt es doch auch wieder andere Rücksichten, in welchen sie von einander verschieden sind, und um dieser anderer Rücksichten wegen kann allerdings auch der Grad ihrer Verdienstlichkeit oder Schuld sehr ungleich seyn.
\item Die Lehre, daß es
\begin{aufzb}
\item unter den sittlich guten Handlungen zwei Arten gebe, die sich so unterscheiden, daß Gott zur Ausübung der Einen \RWbet{nur durch Belohnung} aufmuntern will, \RWbet{ohne die Unterlassung derselben mit Strafen zu bedrohen,} während er bei der andern Art auch dieß Letztere nothwendig findet, enthält gewiß nichts der Vernunft Widersprechendes. Warum sollte es nicht gewisse Handlungen geben, in Ansehung deren es genug ist, wenn Gott zu ihrer Ausübung nur durch Belohnungen ermuntert, ohne zugleich durch Strafen anzutreiben? Und warum soll es nicht auch wieder andere Handlungen geben, in Betreff deren es bei Weitem nicht genug gethan wäre, wenn Gott zu ihrer Ausübung nur auf die eine Art, nämlich nur durch Belohnungen und nicht auch durch Strafen antreiben wollte?
\item Daß es unter den sittlich bösen Handlungen zuvörderst \RWbet{läßliche} oder bloße \RWbet{Gebrechlichkeitssünden} gebe, die eine bloß zeitliche Strafe erfahren, wird Niemand anstößig finden. Wohl aber hat man es anstößig finden~\RWSeitenw{149}\ wollen, daß es auch Sünden geben solle, die eine \RWbet{ewige} Strafe erfahren. Bei einer näheren Betrachtung wird inzwischen auch hier jede Bedenklichkeit verschwinden; denn
\end{aufzb}
\begin{aufzb}
\item je größer die Strafe ist, die Gott auf ein gewisses Verbrechen setzt, um desto kräftiger ist auch der Abhaltungsgrund von demselben, um desto seltener wird es verübt, um desto mehr also wird die Glückseligkeit des Ganzen gewinnen. Diejenigen, welche nichts desto weniger so leichtsinnig oder boshaft sind, das Verbrechen zu begehen, verlieren freilich an ihrer eigenen Glückseligkeit immer um so mehr, je größer die für dasselbe bestimmte Strafe ist; aber es läßt sich, besonders bei Sünden gewisser Art, sehr wohl als möglich denken, daß selbst das ewige Leiden, welches nun einige Verbrecher tragen müssen, noch ein geringeres Uebel ist, als jene Summe von Leiden, welche die häufigere Begehung der Sünde angerichtet hätte, wenn sie durch jene Strafe nicht vermindert worden wäre.
\item Es ist offenbar, daß die Strafe wenigstens immer größer seyn müsse, als der Vortheil, den uns die Sünde verspricht. Nun gibt es viele Sünden, bei welchen der Vortheil (so gering er auch in der Wirklichkeit ist) in den Augen des verblendeten Sünders an das Unendliche grenzt. Nur die Vorstellung einer unendlichen Strafe also kann ihn hier abhalten. Dergleichen Sünden sind die Sünden der Wollust und andere, die eine augenblickliche Sinnenlust versprechen, die Sünde der Rache, ein Diebstahl oder Raub, wodurch man sich für sein ganzes Leben lang glücklich zu machen wähnt, \udgl\ 
\item Endlich ist auch nicht zu vergessen, daß Gott den Sünder nur dann die ewige Strafe, die er verschuldet hat, wirklich erleiden läßt, wenn er sich nicht mehr bessert. Ist es nun nicht begreiflich, daß derjenige, der sich \RWbet{nie} bessert, auch \RWbet{nie} glückselig werden könne, also in Ewigkeit Strafe erfahre? --
\begin{aufzc} 
\item \RWbet{Einwurf}. Eine Veränderung des Lebenswandels, welche durch die Betrachtung der ewigen Strafen der Sünde bewirkt wird, ist keine wahre sittliche Besserung, denn diese findet nur Statt, wo man das Böse, weil es nicht recht ist, unterläßt.~\RWSeitenw{150}\par
\RWbet{Antwort}. Dieser Einwurf, wenn er etwas bewiese, würde alle Belohnungen und Strafen überhaupt, wenigstens ihre Ankündigung, verbieten. -- Es ist freilich wahr, daß derjenige, der das Böse erst nur aus bloßer Furcht vor der Strafe zu unterlassen anfängt, noch kein sittlich guter Mensch sey. Aber sobald er durch einige Entwöhnung von der Sünde erst etwas mehr Herrschaft über sich selbst erhalten hat, und bei der größeren Ruhe des Geistes, die sich jetzt bei ihm einfindet, zu einer deutlicheren Anerkennung des Sittengesetzes gelangt: so wird er die bloß aus Furcht angefangene Aenderung seines Lebenswandels jetzo aus einem reineren, aus dem ächt sittlichen Beweggrunde fortsetzen, weil es so recht ist, und weil es Gott so will. Die Lehre von den Belohnungen und Strafen war es, die ihm zu dieser Vollkommenheit verhalf.
\item \RWbet{Einwurf.} Zwischen der Lust der Sünde und dem Schmerz der Strafe muß immer ein gewisses Verhältniß herrschen. Wenn aber die Strafe unendlich seyn soll: so ist, da alle Lust der Sünde nur endlich ist, gar kein Verhältniß zwischen beiden.\par
\RWbet{Antwort.} Es ist eine falsche, oder doch wenigstens durch nichts erweisliche Behauptung, daß zwischen der Lust der Sünde und dem Schmerz der Strafe allezeit ein endliches Verhältniß herrschen müsse; die Vernunft kann nur beweisen, daß der Schmerz der Strafe größer seyn müsse, als die Lust der Sünde in den Augen des Menschen erscheint. Wie vielmal aber der erste größer als die letztere seyn müsse, wer möchte das bestimmen? Und wie wir schon oben (b) erinnert, so grenzt ja bei manchen Sünden die Lust, die scheinbare nämlich, wirklich an das Unendliche.
\item \RWbet{Einwurf}. Wenn gewisse Sünden ewige Strafen erfahren, so hätte dieß Gott billig allen Menschen auf Erden bekannt machen sollen, damit sie doch wüßten, nach welchen Grundsätzen sie einst gerichtet werden; und damit die Kenntniß dieser Strafe sie um desto mehr von den Sünden abhalte. Ist es nicht ungerecht, nach Gesetzen, die man nicht angekündigt hat, zu richten?\par
\RWbet{Antwort.} Wohl dürfte es nicht immer ungerecht seyn, nach Gesetzen, die man nicht angekündigt hat, zu richten.~\RWSeitenw{151}\ Indessen sagt das Christenthum keineswegs, daß Gott dieß thue, und ewige Strafen für vorsätzliche Sünden drohen ja nur uns Christen. Wie Gott diejenigen, welche das Christenthum nicht kennen, richten werde, darüber macht uns das Christenthum nichts bekannt.
\item \RWbet{Einwurf.} Die Ewigkeit ist ein viel zu abstracter Begriff, als daß die Vorstellung der ewigen Strafen der Sünde im Augenblicke der Versuchung von großer Wirksamkeit seyn könnte. Der Mensch vergißt darauf in diesem Augenblicke, wie schon der Umstand, daß er sich versucht fühlen kann, beweiset. Wenn also Gott mit ewigen Strafen droht, um von der Sünde abzuschrecken, und die gedrohten Strafen dann in Erfüllung bringt, um wahrhaft zu seyn: so opfert er vom Wohle des Ganzen mehr auf, als er gewinnt; er handelt eben so, wie ein Erzieher handeln würde, der seine Zöglinge des geringsten Versehens wegen gleich zu verstümmeln drohen, und dann auch wirklich verstümmeln würde.\par
\RWbet{Antwort.} Die Ewigkeit ist freilich ein sehr abstracter und schwer zu fassender Begriff; aber je schwerer er zu fassen ist, desto wirksamer bezeigt er sich; denn eben aus der Mühe, die es dem Menschen kostet, sich die Ewigkeit auch nur zu denken, schließt er, wie viel es auf sich habe, sich ewig unglücklich zu machen. Auch ist es nicht wahr, daß derjenige, der sich versucht fühlen kann, auf die ewigen Strafen vergessen haben müsse; er kann wohl an sie denken, aber entweder nicht lebhaft genug, oder nicht überzeugt genug. Durch die Bedrohung mit ewigen Strafen opfert Gott keineswegs mehr auf, als er gewinnt, handelt auch keineswegs, wie jener Erzieher handeln würde. Erwachsene Menschen haben doch mehr Ueberlegung als Kinder, Gebote Gottes sind unendlich wichtiger, als Gebote der Menschen, \usw\
\end{aufzc}
\end{aufzb}
\item Daß sich die katholische Kirche nicht anmaße, in jedem einzelnen Falle bestimmen zu wollen, \RWbet{wann eine sittlich gute Handlung bloß verdienstlich, oder schon Schuldigkeit, und eine sittlich böse eine nur läßliche oder schon schwere Sünde sey}, wird kein Vernünftiger tadeln. Aber auch die zwei Bestimmungen, die wir darüber antreffen, wird man bei näherer Betrachtung billigen müssen.~\RWSeitenw{152}
\begin{aufzb}
\item Nichts ist begreiflicher, als daß für solche gute Handlungen, welche die oben angegebenen drei Beschaffenheiten haben, nur die Ermunterung durch angebotene Belohnung, nicht aber auch die Androhung von Strafen für den Fall der Unterlassung zweckmäßig sey. Denn
\begin{aufzc}
\item wenn auch solche Handlungen, durch deren Unterlassung das Wohl des Ganzen nicht gestört wird, sondern nur nicht befördert, unter der Bedingung der Strafe geboten würden: so würde, wie es scheint, durch diese Anordnung die Glückseligkeit des Ganzen mehr verlieren als gewinnen; denn weil sie doch sehr oft unterlassen würden: so könnte die Menge der Uebel, die die Bestraften leiden, bald den Gewinn, welchen das Ganze durch ihre Ausübung hat, überwiegen. Hiezu kommt
\item daß solche Handlungen ihrer Natur nach nur von vernünftigeren und besseren Menschen ausgeübt werden können. Für solche Menschen ist die erfreuliche Aussicht auf eine künftige Belohnung wohl ein hinlänglich starker Antrieb, und es bedarf nicht erst der Drohung einer Strafe.
\end{aufzc}
\item Niemand wird es anstößig finden, daß eine sittlich böse Handlung, die einen nur sehr geringen Schaden verursacht, ingleichen die mit keinem ganz deutlichen Bewußtseyn ihrer Gesetzwidrigkeit verübt worden ist, noch keine Todsünde sey. Wird aber hiedurch zugleich zu verstehen gegeben, daß wir im Gegentheile Sünden, durch die wir das Wohl des Ganzen positiv stören, und die wir überdieß mit einem deutlichen Bewußtseyn ihrer Gesetzwidrigkeit verüben, als Todsünden ansehen sollen: so ist wohl dieses auch keine zu harte Entscheidung zu nennen. Soll es irgend einige Sünden, die eine ewige Strafe verschulden, geben (und dieses haben wir bereits gerechtfertiget): so müssen es wohl die Sünden der eben beschriebenen Art seyn; denn wer in irgend einem Falle mit deutlichem Bewußtseyn Gottes Gebot übertritt, der wägt bei dieser Gemüthsstimmung nicht erst die Größe des Schadens, den er durch seine Handlung anrichten werde. Wenn er dieß thäte und sich nur darum zur Handlung entschlöße,~\RWSeitenw{153}\ weil der Schaden nur so geringe ist, daß sein sinnlicher Vortheil ihn überwiegt: so würde er eben darum noch an der Unerlaubtheit der Handlung zweifeln; er hätte mithin kein deutliches Bewußtseyn des Verbrechens. Wer also Böses thut mit deutlichem Bewußtseyn; der würde, selbst wenn der Schaden seiner That noch so bedeutend wäre, sie dennoch nicht unterlassen. Nicht unbillig also, daß er auch eine Strafe gleich demjenigen erleidet, der die verderblichste That unternommen. Gibt es also nur einige vorsätzliche Sünden, die einer ewigen Strafe werth sind: so sind es im Grunde alle, weil sie im Wesentlichen gleich sind.
\begin{RWanm} 
Hiemit wird nur gesagt, daß alle solche Sünden eine \RWbet{unendliche}, nicht aber eine in jeder anderen Hinsicht \RWbet{gleiche} Strafe erfahren dürften. Auch unter den ewigen Strafen gibt es noch Unterschiede und Grade. 
\end{RWanm}
\end{aufzb}
\item Die Vernunftmäßigkeit dieses Punctes ist keiner Schwierigkeit unterworfen.
\item Gegen den achten Punct möchte man einwenden, daß sonach die Größe der Belohnung oder Strafe von einem Grunde, der außerhalb des Handelnden liegt, vom bloßen Zufalle, abhängig gemacht werde; denn Folgen, welche wir \RWbet{nicht vorhersahen, noch vorhersehen konnten}, wurden von uns auch nicht beabsichtiget; sie treten also auf eine Weise ein, welche in Hinsicht auf uns ein bloßer Zufall genannt werden darf. -- Wahr; aber hierin liegt gar nichts Ungereimtes, weil die Gerechtigkeit Gottes eben nicht fordert, daß der Grund, warum Gott Jemanden gerade in diesem oder in jenem Maße belohnt oder bestraft, ganz in ihm selbst liegen müsse. Die Verfügung, von welcher hier gesprochen wird, ist also nicht ungerecht, hat aber den Nutzen, daß sie den Antrieb zur Uebung des Guten und den Abhaltungsgrund von allem Bösen vermehrt.
\item Die Lehre von der Zurechnung \RWbet{fremder Sünden} ist eine bloße Folge aus dem vorhergehenden Satze.
\item Die Vernunftmäßigkeit des zehnten Artikels fällt mit dem sittlichen Nutzen desselben zusammen.~\RWSeitenw{154}
\item Ein Gleiches gilt von dem eilften Artikel; denn wenn jene Strafen, welche die bloßen Gebrechlichkeitssünden verschulden, in ihrer Summe so groß wären, daß sie die Summe der Belohnungen, welche der Tugendhafte durch seine guten Handlungen verdienet, überwiegen: so könnte ja nicht einmal der Tugendhafte hoffen, daß er einst werde glücklich werden.
\item Es ist sehr richtig bemerkt, \RWbet{daß jene guten Handlungen, welche der Lasterhafte im Zeitraume seiner Lasterhaftigkeit ausübt, nur scheinbar gut sind;} denn da er den herrschenden Willen hat, das Sittengesetz in gewissen Stücken zu übertreten: so ist es offenbar, daß er auch in denjenigen Fällen, wo er etwas äußerlich Gutes thut, es nicht aus dem Grunde, weil es gut ist, thue, sonst würde er Alles thun, was er als recht erkennt.
\begin{RWanm} 
Als Sünden aber (wie dieses Einige wollten) können ihm solche äußerlich gute Handlungen keineswegs angerechnet werden; denn sie sind ja dem Gesetze gemäß, das Wohl des Ganzen wird also durch sie befördert. Würden sie ihm aber als Sünden angerechnet, folglich auch zur Vermehrung seiner Strafwürdigkeit dienen: so würde er von ihrer Ausübung abgehalten werden; das Wohl des Ganzen würde daher nur verlieren. 
\end{RWanm}
\item Daß aber auch die \RWbet{wirklich guten Handlungen, die wir einst verrichtet hatten, uns, wenn wir lasterhaft geworden sind, nicht weiter zum Verdienste angerechnet werden} sollen, ist durchaus nothwendig, damit wir uns in unserer Lasterhaftigkeit nicht auf die Verdienste unserer früheren Jahre verlassen. Daß aber diese Verdienste bei unserer Besserung wieder aufleben sollen, ist überaus aufmunternd für den Gebesserten; das Gegentheil müßte ihn niederschlagen.
\item Da Gott allwissend ist, also den sittlichen Zustand eines jeden seiner Geschöpfe kennt; da er ferner bei seiner Leitung unserer Schicksale nicht nur auf einige, sondern auf alle unsere Beschaffenheiten und Verhältnisse Rücksicht nehmen muß; da endlich jede gute Handlung einen Grund, sie zu belohnen, und eine jede böse einen Grund, sie zu bestrafen, enthält: so ist es unläugbar, daß, wenn uns \RWbet{nach Aus}\RWSeitenw{155}\RWbet{übung einer guten That ein Glück} zu Theil wird, unter den Gründen, die Gott bestimmten, uns dieß Glück zuzuführen, auch der vorhanden war, weil wir vor Kurzem gut gehandelt hatten; daß eben so umgekehrt, wenn uns \RWbet{nach Ausübung einer bösen That ein Unglück} zu Theil wird, unter den Gründen, die Gott bestimmten, dieß Unglück über uns kommen zu lassen, auch der gewesen sey, weil wir nämlich böse gehandelt. Glück oder Unglück aber, das uns (unter Anderem auch) aus dem Grunde zu Theil wird, weil wir vorher gut oder böse gehandelt, ist Belohnung oder Strafe. Also behauptet die katholische Kirche mit Recht, daß wir das Glück, das uns in einer tugendhaften Verfassung zu Theil wird, als eine göttliche Belohnung, das Unglück aber, das uns in einer lasterhaften Verfassung zu Theil wird, als eine göttliche Bestrafung anzusehen hätten. Wenn aber im Gegentheile auf Gutes Unglück erfolgt: so ist offenbar, daß wir dieß nicht als Strafe ansehen können; wir müssen also denken, daß es ein Uebel sey, das uns vergolten werden soll. Und daß es noch ein Mittel zu unserer eigenen Vervollkommnung werden könne, liegt am Tage; denn Leiden können uns bekanntlich in allerlei sehr wichtigen Tugenden üben. Der Ausdruck aber, dessen sich das Christenthum bedient, daß uns dieß Unglück \RWbet{zur Prüfung} geschickt worden sey, kann freilich nur bildlich ausgelegt werden. Gott prüft uns nämlich nicht, damit \RWbet{er selbst} erfahre, was wir vermögen, sondern damit wir und unsere Nebenmenschen inne werden, was sich (nach Lessing's Ausdrucke) der Gott ergebene Mensch für Thaten abgewinnen könne.
\end{aufza}

\RWpar{230}{Sittlicher Nutzen}
\begin{aufza}
\item Die Lehre, daß Gott bei der Leitung unserer Schicksale die \RWbet{vornehmste Rücksicht auf unser sittliches Verhalten} nehme, zeigt uns die Wichtigkeit dieses Verhaltens in einem ganz eigenen Lichte. Bei all unserer Ohnmacht also hängt es im Grunde doch nur von uns selbst ab, ob wir ein seliges oder unseliges Loos erfahren sollen.
\item Wenn wir insonderheit glauben, daß eine \RWbet{jede sittlich gute Willensentschließung} einen Grund ent\RWSeitenw{156}halte, der Gottes Heiligkeit bestimmt, uns eine gewisse \RWbet{Belohnung}, und \RWbet{jede sittlich böse} einen Grund, der Gottes Heiligkeit bestimmt, uns eine gewisse \RWbet{Bestrafung} zuzugedenken: so muß uns dieses ein fortwährender Antrieb zu allem Guten, und ein fortwährender Abhaltungsgrund von allem Bösen werden.
\item Es muß die Lust und den Muth zur Besserung ohne Zweifel sehr erhöhen, wenn man dem Sünder sagt, daß jenes Verhältniß, in dem er zu Gott steht, gleich von dem Augenblicke an, da er den festen Entschluß der Besserung faßt, und die von dem Christenthume ihm vorgeschriebenen Heiligungsmittel gebraucht, sich auf das Vortheilhafteste verändere, daß er \RWbet{von diesem Augenblicke an vor Gott von aller Schuld frei} sey. Wollte man aber noch weiter gehen, und mit dieser Befreiung von der Schuld auch die Erklärung, daß alle Strafe von nun an aufgehoben sey, verbinden: so würde man den Menschen nur leichtsinnig machen, und jene natürlichen Strafen, die er durch seine Sünden sich zugezogen hat, \zB\ Krankheiten \udgl , müßten nun entweder durch ein Wunder aufgehoben werden, oder Gott müßte erlauben, daß der Mensch von nun an diese Uebel als durchaus unverdiente Leiden betrachte, als Leiden, für welche er einst noch Entgeltung zu erwarten habe. Das Erstere wäre gegen die Ordnung der Natur; das Zweite würde alle wohlthätige Wirkung, die eine geduldige Ertragung selbstverschuldeter Leiden nach sich zieht, (die Warnung für die Zukunft, die Demüthigung, den Abscheu vor der Sünde, \usw ) hindern, und allen Unterschied, der zwischen selbstverschuldeten und unverdienten Leiden billig Statt finden soll, vernichten. Der eitle Sünder würde mit jenen Leiden, welche die Folgen seiner Thorheiten sind, wie mit den Leiden eines Martyrers, prahlen.
\item Die Lehre von der \RWbet{ungleichen Größe Beides der Verdienstlichkeit unserer guten sowohl, als auch der Schuld unserer bösen Handlungen} hat einen offenbaren Nutzen; denn glaubten wir, daß alle Sünden eine gleiche Strafe erfahren, so würden wir
\begin{aufzb}
\item in jenen Fällen, wo wir mit deutlichem Bewußtseyn sündigen, aus allen uns möglichen Handlungsweisen die\RWSeitenw{157}jenige herausheben, die, wenn auch die verderblichste für das Ganze, doch für uns selbst die genußreichste wäre. Du erfährst nun schon einerlei Strafe, würden wir glauben, wähle also wenigstens das, was dich am Meisten ergötzt.
\item Eben dieser Glaube würde uns bei geringen Sünden muthlos machen, weil wir, nur die Zahl derselben erwägend, erschrecken, und eben die Strafe befürchten würden, welche der größere Verbrecher, der eben so oft, aber viel gröber gesündiget hat, erfährt.
\end{aufzb}
\item Die Unterscheidung, die das katholische Christenthum
\begin{aufzb}
\item zwischen den oben beschriebenen \RWbet{zwei Arten sittlich guter Handlungen} macht, ist äußerst nothwendig; denn wenn wir
\begin{aufzc}
\item glauben müßten, was einige Weltweise uns hatten einreden wollen, daß die Unterlassung einer jeden sittlich guten Handlung ein positives Uebel als Strafe nach sich ziehen werde: so müßte uns wahrlich das Leben verdrießen; denn für die geringste Zerstreuung, für die geringste Unachtsamkeit, durch die es geschieht, daß wir in einem gewissen Falle nicht vollkommen richtig beurtheilen, welche unter allen hier möglichen Handlungen das Wohl des Ganzen am Meisten befördert haben würde, müßten wir Strafe besorgen. Müßte da nicht ein Jeder, der nur nicht im höchsten Grade leichtsinnig ist, in steter Furcht und Bangigkeit leben, und statt mit Freudigkeit Gutes zu wirken, es nur mit Zittern üben? Wenn wir dagegen 
\item uns vorstellen würden, daß Gott auf keine Unterlassung einer guten Handlung eine Strafe gesetzt habe: so würde uns dieses zu vielen guten Handlungen, besonders zu solchen, die mit einem größeren Opfer verbunden sind, wogegen die Sinnlichkeit sich stärker sträubt, nicht mächtig genug antreiben.
\end{aufzc}
\item \RWbet{Daß es auch unter den sittlich bösen Handlungen zwei Arten} gebe, deren eine nur eine \RWbet{zeitliche}, die andere eine ewig dauernde Strafe verschuldet, ist eine der wichtigsten Lehren des Christenthums.~\RWSeitenw{158}
\begin{aufzc}
\item Wie aufmerksam müssen wir nicht auf unsere Handlungsweise werden, wenn wir wissen, daß es auch einige gebe, durch die wir uns ewig unglücklich machen können.
\item Müßten wir aber besorgen, daß eine jede sittlich böse Willensentschließung eine ewige Strafe nach sich ziehe; so müßte uns dieß in Verzweiflung stürzen.
\end{aufzc}
\end{aufzb}
\item Auch die Erklärung, daß es sich nicht in jedem einzelnen Falle mit Bestimmtheit ausmachen lasse, ob eine gewisse sittlich gute Handlung zu den verdienstlichen oder zu den Schuldigkeiten, und eine sittlich böse zu bloß läßlichen oder Todsünden gehöre, dürfte ihren Nutzen haben. Um so eifriger müssen wir uns bestreben, die gute Handlung, von der wir nicht wissen, ob wir durch ihre Unterlassung nicht etwa straffällig würden, zu vollenden; die böse Handlung aber, die uns vielleicht selbst eine ewige Strafe zuziehen könnte, zu vermeiden. Daß aber diese Ungewißheit nicht überall Statt hat, daß wir insonderheit
\begin{aufzb}
\item bei solchen guten Handlungen, welche die oben angegebene Beschaffenheit haben, sicher seyn können, daß wir durch ihre Unterlassung noch eben nicht straffällig werden, ist zu unserer Beruhigung höchst nöthig; denn wenn das Gegentheil wäre, so könnten wir auch bei der besten Absicht und bei dem eifrigsten Bestreben, immer zu thun, was recht ist, nie gewiß seyn, ob wir statt Lohnes nicht vielmehr Strafe verschuldet haben.
\item Eben so nothwendig ist es zu wissen, daß wir in solchen Fällen, wo wir mit keinem deutlichen Bewußtseyn der Gesetzwidrigkeit gehandelt hatten, wenigstens keine Strafe, die ewig dauern wird, verwirkten. In solche Gebrechlichkeitssünden verfallen wir ja täglich. Daß wir aber dort, wo wir mit einem deutlichen Bewußtseyn etwas zu thun beschlossen, was wir für unrecht erkennen, eine ewige Strafe verschulden, das ist der mächtigste Abhaltungsgrund vom Bösen, den es nur immer geben kann. Wären die Strafen auch noch so groß, aber doch gleichwohl endlich: wir würden im Augenblicke der Versuchung oft thöricht genug seyn, uns in eine Vergleichung der~\RWSeitenw{159}\ nahen sinnlichen Lust der Sünde und der Schmerzen der Strafe einzulassen, und unsere Einbildungskraft würde uns jene vielleicht größer als diese vormahlen. Im Gegentheile aber jetzt, da es heißt, daß wir uns eine ewige Strafe bereiten: wen sollte nicht das fürchterliche Wort der Ewigkeit zur Besinnung aufschrecken? wer sollte jetzt noch zaudern, vergleichen, nachrechnen wollen, ob auch die Lust der Sünde um diesen Preis zu theuer erkauft werde? In der That, wer sich die Vorstellung von der Ewigkeit der Strafen, die eine jede vorsätzliche Sünde bedrohen, recht geläufig macht, der muß es bald dahin bringen, daß er gar keine vorsätzliche Sünde mehr begeht.
\end{aufzb}
\item Wenn wir versichert sind, daß uns eine gute Handlung zu einem um desto größeren Verdienste werde angerechnet werden, \RWbet{je mehrere ersprießliche Folgen von ihr wir erwarten und bei unserem Entschlusse eben beabsichtigen}: so läßt sich hoffen, daß wir gerade diejenigen guten Handlungen, die am Gedeihlichsten für das gemeine Beste sind, auch mit dem größten Eifer vollziehen werden; und hiedurch wird das Wohl des Ganzen offenbar gewinnen. Und eben so, wenn wir es einmal wissen, daß eine jede böse That eine um desto härtere Bestrafung finden werde, \RWbet{je mehrere schlimme Folgen derselben wir von ihr vorhersahen,} ohne uns gleichwohl durch diese Vorhersehung von ihr abschrecken zu lassen: so stehet zu erwarten, daß wir gerade diejenigen bösen Handlungen, die für das Ganze die verderblichsten geworden wären, am Oeftesten vermeiden, und das ihnen entgegengesetzte Verhalten annehmen werden.
\item Wenn wir wissen, daß uns selbst jene ersprießlichen Folgen einer sittlich guten Handlung, \RWbet{welche wir nicht vorhergesehen hatten,} zwar nicht in eben dem Maße, wie die vorhergesehenen und beabsichtigten, aber doch einiger Maßen, zum Verdienste angerechnet werden sollen: so werden wir uns bei dem Eintritte solcher Folgen innigst erfreuen, und den stärksten Antrieb fühlen, nie müde zu werden im Guten, weil sich der Lohn desselben in das Unendliche erstreckt. Wissen wir ferner, daß uns im entgegengesetzten Falle, wo~\RWSeitenw{160}\ wir etwas in böser Absicht thaten, die schlimmen Folgen, auch selbst diejenigen, \RWbet{die wir auf keine Weise vorhergesehen und gewollt hatten,} einiger Maßen zur Last fallen sollen: so wird jede dieser verderblichen Folgen, die im Verlaufe der Zeit zum Vorschein kommt, unsere Gewissensbisse vermehren, und eben darum auch den Vorsatz befestigen helfen, nie wieder Böses zu thun, da es so furchtbar sich rächet.
\item Daß wir insonderheit auch alle \RWbet{bösen Handlungen Anderer,} die wir durch eine eigene böse That veranlasset, zu verantworten haben, muß uns um desto vorsichtiger in unserem Umgange machen, und ist ein beständiger Antrieb, auf ihre sittliche Vervollkommnung vortheilhaft einzuwirken.
\item Daß wir im Gegentheile für jene schlimmen Folgen, welche aus einer Handlung hervorgehen, die wir \RWbet{in guter Absicht und mit der gehörigen Vorsicht} unternommen, nichts zu verantworten haben, ist eine Erklärung, die uns überaus nothwendig ist, wenn wir das Gute mit wahrer Freudigkeit anfangen sollen. Müßten wir auch für den guten Ausgang desselben stehen; müßten wir es büßen, wenn daraus einige schlimme Folgen hervorgehen: wie könnten wir das jemals mit Muth angreifen? Eben so nothwendig ist es aber auch von der anderen Seite, dem Bösen zu erklären, daß er von jenem Guten, welches durch Gottes weise Leitung aus seinen bösen Handlungen hervorgeht, durchaus nicht die geringste Verminderung seiner Strafwürdigkeit zu erwarten habe. Widrigenfalls würde ein starker Abhaltungsgrund vom Bösen wegfallen; da wir aus Gottes Weisheit wohl schließen können, daß er auch selbst aus unseren bösesten Handlungen noch manches Gute werde herzuleiten wissen. Daß wir aber doch hintenher, wenn wir uns einmal gebessert haben, aus der Betrachtung des Guten, das Gott aus unseren bösen Handlungen hervorgehen ließ, einigen Trost schöpfen dürfen, ist gewiß billig. So hören allmählich unsere nun schon entbehrlich gewordenen Gewissensbisse auf, und unsere Liebe zu Gott muß um so höher steigen.
\item Durch die Erklärung, daß die Belohnungen, die der Tugendhafte für seine guten Handlungen zu erwarten hat, die Strafen, die er durch seine Gebrechlichkeitssünden ver\-schul\-\RWSeitenw{161}det, \RWbet{überwiegen}, wird uns einerseits eine so heitere Aussicht in die Zukunft eröffnet, als eben nöthig ist, um den Vorsatz der Tugend in uns zu befestigen; und doch werden wir anderseits nicht gleichgültig gemacht gegen die vielen Gebrechlichkeitssünden, in die wir von Zeit zu Zeit gerathen.
\item Wenn nicht gelehrt würde, daß die guten Handlungen, welche der Lasterhafte in dem Zustande seiner Lasterhaftigkeit ausübt, \RWbet{alles wahren Werthes} ermangeln: so stünde sehr zu befürchten, daß wir ein Jeder nur die für uns leichtesten Puncte in den Geboten Gottes beobachten, die übrigen aber gewissenlos übertreten, indem wir uns des Gedankens trösten würden, daß wir durch das Verdienst unserer guten Werke die Schuld unserer bösen aufwiegen werden.
\item Aus gleichem Grunde ist nöthig, uns zu erklären, daß von dem Augenblicke an, da wir in eine lasterhafte Gesinnung verfallen, auch \RWbet{das Verdienst aller unserer früheren in Wahrheit sittlicher Werke erlösche.} Dieß glaubend werden wir nie auf die thörichte Meinung gerathen, daß wir des Guten bereits genug gethan, und uns somit auf einige Jahre auch wohl dem Laster hingeben könnten. Sollten wir inzwischen dennoch so unglücklich gewesen seyn, uns auf dem Abwege des Lasters verirrt zu haben; sind wir aber jetzt wieder zurückgekehrt: so wäre es doch in der That überaus traurig für uns, wenn wir alles das Gute, das wir in früherer Zeit verrichtet, auch jetzt noch und für immer als völlig ungeschehen ansehen müßten, und also demjenigen gleichgestellt seyn sollten, der durch sein ganzes früheres Leben nie etwas Gutes gethan hat.
\item Wenn jeder Tugendhafte das Glück, das ihm zu Theil wird, \RWbet{als eine Belohnung seiner Tugend} ansehen darf: so erfreut er sich desselben inniger, und erhält eine stärkere Aufmunterung, in der Tugend fortzufahren. -- Wenn er das Unglück, das ihn trifft, \RWbet{als eine Prüfung, und als ein Mittel zu seiner Vervollkommnung} ansehen darf: so wird er nicht ermangeln, es zu diesem Zwecke zu benützen; und wenn er weiß, daß es ihm dann überschwenglich entgolten werden soll: so wird er es auch mit Geduld, oft selbst mit Freudigkeit ertragen. Der Laster\RWSeitenw{162}hafte dagegen, wenn er das über ihn einbrechende Unglück \RWbet{als einen Anfang der göttlichen Strafe} betrachtet, fühlt sich erschüttert, und kommt zur Besinnung. Wird aber irgend ein Glück ihm zu Theil, und sieht er es nach diesem Grundsatze\RWbet{ als eine Aufforderung der Langmuth Gottes zu seiner Besserung} an: so wird ihn dasselbe wenigstens nicht übermüthig machen, sondern er wird, wenn er klug ist, dem Rufe Gottes noch folgen, weil es Zeit ist.
\end{aufza}

\RWpar{231}{Die Lehre des Christenthums von den Belohnungen und Strafen im andern Leben}
\begin{aufza}
\item Nicht immer tritt die verdiente Belohnung oder Strafe schon in dem gegenwärtigen Leben ein; sondern erst \RWbet{jenseits der Gräber ist das eigentliche Land der Vergeltung}.
\item Und zwar \RWbet{gleich nach der Trennung der Seele vom Leibe fängt die Vergeltung an}, kein langer Seelenschlaf also trennt dieß gegenwärtige Leben von jenem Zustande der Vergeltung; sondern so wie der Mensch stirbt, tritt seine Seele \RWbet{vor Gottes Richterstuhl} hin, um alsbald Lohn oder Strafe von ihm zu empfangen (\RWbet{besonderes Gericht}). Die Stunde des Todes ist völlig \RWbet{ungewiß,} wird aber \RWbet{durch kein blind wirkendes Schicksal}, sondern \RWbet{durch Gottes besondere Leitung} herbeigeführet.
\item Es gibt aber überhaupt einen \RWbet{dreifachen, sehr wohl zu unterscheidenden Zustand,} in deren Einen wir Alle, so viele wir Christen sind, gleich nach dem Tode eintreten werden.
\begin{aufzb}
\item Einen Zustand \RWbet{reiner und immerwährender Seligkeiten im Himmel;} der jedoch nur für diejenigen aus uns bestimmt ist, die \RWbet{frei von aller Schuld}, selbst von jener der Erbsünde, aus dieser Welt austreten.
\item Einen Zustand der \RWbet{Reinigung durch Leiden,} \dh\ einen Zustand, in welchem wir uns schmerzlichen, aber ein Ende nehmenden, und unsere Besserung bewirkenden Leiden ausgesetzt sehen; bestimmt für diejenigen aus uns, die zwar \RWbet{mit keiner Todsünde} behaftet aus dieser~\RWSeitenw{163}\ Welt austreten, aber doch noch \RWbet{eine oder die andere läßliche Sünde begangen, oder gewisse endliche Strafen für schon bereute Todsünden zu ertragen haben}. Durch diese Leiden werden sie geläutert und vervollkommnet, und eben deßhalb zuletzt fähig und würdig, selbst in den Himmel aufgenommen zu werden.
\item Einen Zustand \RWbet{ewiger Verdammniß} oder Hölle, \dh\ einen höchst unglückseligen und kein Ende nehmenden Zustand aller derjenigen, welche, \RWbet{behaftet mit einer Todsünde,} aus diesem Leben treten.
\end{aufzb}
\item Nebst jenem besonderen Gerichte, das jeder einzelne Mensch gleich nach dem Tode erfährt, steht uns auch noch ein \RWbet{allgemeines} oder \RWbet{öffentliches} bevor. An einem Tage nämlich, den wir den \RWbet{jüngsten} oder \RWbet{letzten} nennen, von dem wir übrigens nicht wissen, \RWbet{wie nahe oder ferne} er sey, wird \RWbet{Jesus Christus selbst} in göttlicher Majestät auf dieser Erde erscheinen, um über alle Menschen, die damals noch lebenden sowohl, als die schon längst Verstorbenen Gericht zu halten. Die bereits Verstorbenen werden, \RWbet{bekleidet mit ihren ehemaligen, doch jetzt umwandelten Leibern,} aus ihren Gräbern hervorgehen, die Leiber der noch Lebenden werden gleichfalls \RWbet{umwandelt} werden. Wir werden uns Alle fühlen, als wären wir in denselben Leibern, die wir als Menschen getragen hatten, ob sie gleich jetzt eine ganz andere, für unser künftiges Leben berechnete Beschaffenheit haben werden. Bei dem Gerichte, welches nun Jesus über uns halten wird, \RWbet{werden alle unsere guten Handlungen}, auch selbst diejenigen, die bisher Niemand erkannt hat, sammt ihren uns bisher selbst nicht ganz bekannt gewordenen \RWbet{Folgen, allgemein kund werden.} Aber auch \RWbet{alles Böse,} das wir verübt, und noch nicht abgebüßt haben, \RWbet{sammt seinen verderblichen Folgen, wird nun an's Tageslicht kommen}. Nach dem endlichen Ausspruche dieses Gerichtes werden alle Menschen nur in \RWbet{zwei Classen} abgesondert werden. Die \RWbet{Guten} werden in den Ort der \RWbet{ewigen Seligkeit} eingehen; die \RWbet{Bösen} aber in den Ort der \RWbet{ewigen Verdammniß} verstoßen werden. Jene und diese aber werden erkennen,~\RWSeitenw{164}\ \RWbet{daß unser göttlicher Richter durchaus gerecht} geurtheilet habe.
\item Wie groß die Anzahl derjenigen sey, die nach dem Ausspruche dieses letzten Gerichtes in den Himmel aufgenommen werden sollen, wird uns nicht angegeben. Doch will das katholische Christenthum, daß wir uns diese Zahl lieber \RWbet{sehr klein}, als groß vorstellen sollen.
\item Von den \RWbet{Belohnungen, die wir im Himmel erfahren,} sagt das katholische Christenthum bestimmt, daß sie in \RWbet{keinen sinnlichen Freuden} bestehen werden; von den \RWbet{Strafen} dagegen, die unserer im Reinigungszustande, oder wohl gar in der Hölle erwarten, läßt es \RWbet{unentschieden}, ob sie auch sinnlicher Art seyn werden.
\item Die \RWbet{Seligkeiten, die uns im Himmel bereitet sind, werden Alles übertreffen, was wir uns vorstellen können.} Sie werden unter Anderm auch \RWbet{in dem Umgange mit vielen anderen Seligen und mit den vollkommensten Geistern} bestehen. Was sich hier liebte mit einer reinen Liebe, wird sich dort \RWbet{wieder finden und lieben. }Doch unsere höchste Seligkeit wird in dem \RWbet{Umgange mit Jesu Christo,} und in \RWbet{dem Anschauen}, \dh\ in einer überaus deutlichen und lebhaften Erkenntniß \RWbet{der unendlichen Vollkommenheiten Gottes} bestehen.
\begin{RWanm}
Hieraus ergibt sich von selbst, daß unser Leben im Himmel ein \RWbet{thätiges Leben} seyn werde, und daß eben deßhalb auch ein \RWbet{Fortschreiten zu immer höherer Vollkommenheit} darin Statt finden werde. Doch hat das Christenthum Keines von Beiden ausdrücklich entschieden.
\end{RWanm}
\item Nach der \RWbet{Verschiedenheit des Grades unserer Verdienste} wird auch unsere \RWbet{Seligkeit im Himmel ihre verschiedenen Stufen} haben. Eben so haben auch die \RWbet{Strafen in der Hölle oder des Reinigungszustandes} nach der Verschiedenheit unserer Schuld \RWbet{verschiedene Grade}.
\item Niemand, wie groß die Fehltritte seines bisherigen Lebens auch immer gewesen seyn mögen, hat zu \RWbet{verzweifeln an der Möglichkeit, selig zu werden.} Er fange~\RWSeitenw{165}\ nur an, sich zu bessern, er büße nur seine Fehler, brauche alle Mittel zu seiner sittlichen Vervollkommnung, die ihm Gott an die Hand gibt: so kann er noch immer die frohe Hoffnung fassen, daß er einst werde selig werden. \RWbet{Je größere Fortschritte er auf dem Pfade der Tugend bereits gemacht hat, um desto zuversichtlicher kann seine Hoffnung werden, daß er in dieser gottgefälligen Gesinnung bis an sein Ende ausdauern werde; in eine völlige Gewißheit aber kann sie nie übergehen}; es müßte denn seyn, daß ihm Gott selbst hierüber eine eigene Offenbarung gegeben hätte.
\item Endlich gibt uns das katholische Christenthum auch \RWbet{über das Schicksal, welches die Nichtchristen im anderen Leben erwartet}, einige Aufschlüsse. Es will, daß wir in dieser Hinsicht Jene, die von der Wahrheit des katholischen Lehrbegriffs \RWbet{ganz ohne ihr Verschulden} nicht überzeugt worden sind, von jenen unterscheiden, welche \RWbet{an ihrem Unglauben schuldig} sind. Die Letzteren zählt es nur des Verbrechens ihrer Ungläubigkeit wegen den ewig Verworfenen bei. Die Ersteren aber läßt es, um ihres unverschuldeten Unglaubens wegen, zwar keine Strafe erleiden; sagt aber doch, daß sie \RWbet{derjenigen höheren Seligkeit, welche nur Jesus Christus unserem Geschlechte ausgewirkt hat}, in ihrem ganzen Umfange nicht anders theilhaftig werden können, als wenn sie durch die unendliche Barmherzigkeit Gottes erst noch im andern Leben Gelegenheit erhalten, mit seiner Lehre bekannt zu werden, und an ihn zu glauben.
\end{aufza}

\RWpar{232}{Historischer Beweis dieser Lehre}
\begin{aufza}
\item Daß erst \RWbet{jenseits der Gräber das Land der eigentlichen Vergeltung} liege, zeiget die Stelle \RWbibel{Mt}{Matth.}{13}{24}, wo Jesus das Menschengeschlecht, so lang es noch auf dieser Erde ist, mit einem Ackerfelde, auf welchem guter Weizen und Unkraut unter einander aufwachsen, das jüngste Gericht aber mit jener Ernte vergleicht, nach welcher der Weizen in den Scheuern aufbewahret, das Unkraut aber in~\RWSeitenw{166}\ Büschlein gebunden, und in den Feuerofen geworfen wird. Vergl.\ auch \RWbibel{Weish}{Weish.}{4}{7\,ff}
\item Zu den Zeiten Jesu glaubten die Juden insgemein, daß der verdiente Zustand des Lohnes oder der Strafe \RWbet{gleich nach dem Tode anfange}, und unser Heiland hat diesen Glauben nie bestritten, sondern vielmehr ihn noch bekräftiget. So sagte er \zB\ in der Parabel vom reichen Prasser (\RWbibel{Lk}{Luk.}{16}{22}), daß der arme Lazarus nach seinem Tode von den Engeln in den Schooß Abrahams getragen worden sey; welches nicht erst nach Jahrtausenden geschehen seyn konnte, weil der Reiche noch lebte. Zu dem Schächer am Kreuze sprach Jesus \RWbibel{Lk}{Luk.}{23}{43}: \erganf{\RWbet{Heute} noch wirst du mit mir im Paradiese seyn.} So schreibt auch der heil.\ Paulus \RWbibel{Hebr}{Hebr.}{9}{27}: \erganf{Es ist der Menschen Loos, einmal zu sterben, worauf das Gericht folgt}, -- wo von dem besondern Gerichte die Rede seyn muß. Ueber die Ungewißheit der Todesstunde drückt sich sehr schön das Buch des Predigers (\RWbibel{Koh}{}{9}{12}) aus: \erganf{Auch weiß der Mensch seine Zeit (nämlich des Todes) nicht; sondern wie die Fische im Netz und die Vögel in der Schlinge, ehe sie sich's versehen, gefangen werden: so wird auch der Mensch bestrickt in der Unglückszeit, ehe er sich's vermuthet.} -- Daß aber nicht ein blindes Schicksal, sondern Gott selbst den Tod herbeiführe, beweisen so viele Stellen der Bücher des alten und neuen Bundes, wo bald ein plötzlicher Tod als eine besondere Strafe Gottes, bald die Verlängerung der Lebensfrist als eine Gnade Gottes dargestellt wird. \RWbibel{Hiob}{Hiob}{14}{5}\ heißt es mit ausdrücklichen Worten: \erganf{Kurz sind die Tage des Menschen; die Zahl seiner Monde steht bei dir: du hast ihm die Grenze bestimmt, die er nicht überschreiten wird.}
\item Daß es einen dreifachen Zustand nach dem Tode, und zwar
\begin{aufzb}
\item einen Zustand der reinsten und ewigen Seligkeit im Himmel gebe, beweiset die Stelle \RWbibel{2\,Kor}{2\,Kor.}{4}{17}: \erganf{Unser jetziges vorübergehendes und erträgliches Leiden bringt uns eine Alles überwiegende ewige Herrlichkeit.}
\item Das Daseyn eines Reinigungszustandes glaubte dem Wesen nach schon die jüdische Kirche, wie aus der oben angeführten Stelle (\RWbibel{2\,Makk}{2\,Makk.}{12}{43}) zu ersehen ist;~\RWSeitenw{167}\ denn hätten jene Juden nicht geglaubt, daß manche Verstorbene noch gewisse endliche Leiden, deren Dauer durch Fürbitte verkürzt werden könne, zu ertragen hätten: so würden sie nicht die dort beschriebenen Opfer für sie in den Tempel gesendet haben. Wäre nun diese Meinung irrig: so hätte sie Jesus und die Apostel billig bestreiten sollen, welches sie nirgends gethan. Im Gegentheile, was Jesus bei \RWbibel{Mt}{Matth.}{12}{32}\ sagt, setzet vielmehr das Daseyn eines Reinigungszustandes voraus: \erganf{Wer etwas wider den Sohn sagt, dem mag es nachgelassen werden; wer aber selbst dem heiligen Geiste sich widersetzt, dem wird es weder in diesem, noch in jenem Leben erlassen (\RWgriech{o>'ute >en to'utw| t~w| a>i~wni, o>'ute >en t~w| m'ellonti}).} Da es hier Jesus von der Sünde wider den heil.\ Geist als eine Besonderheit anmerkt, daß sie weder in diesem, noch in jenem Leben nachgelassen werden könne: so muß es wohl umgekehrt andere Sünden geben, die ihre Nachlassung im andern Leben finden. So sagt man \zB\ nicht: Ich werde nicht heirathen weder in diesem noch im andern Leben; weil es vom andern Leben sich schon von selbst verstehet, da dort Niemand heirathen kann. Auch die Stelle \RWbibel{1\,Kor}{1\,Kor.}{3}{9\,ff}\ beweiset das Daseyn eines Reinigungszustandes nach dem Tode. Von sich und seinen Mitarbeitern sagt der Apostel hier: \erganf{Wir sind Gottes bestellte Arbeiter; ihr seyd der Acker Gottes, ihr sein Gebäude. Nach jener Gnade nun, welche mir Gott verliehen hat, habe ich mich bemüht, als ein verständiger Baumeister einen Grund zu legen, auf diesen sollen nun Andere fortbauen; ein Jeder aber sehe zu, auf welche Art er fortbaut. Was den Grund anlangt, so kann wohl Niemand einen anderen legen, als der schon gelegt ist, Jesum Christum. Je nachdem aber das, was auf diesen Grund weiter gebaut wird, Gold oder Silber, Edelgestein, Holz, Stroh oder Stoppelwerk ist: so wird einst eines Jeden Werk offenbar werden. Die Zeit wird es lehren (man könnte auch übersetzen: der Tag, nämlich des Gerichts, wird es lehren); im Feuer wird sich's offenbaren, da wird sich's zeigen, von welcher Haltbarkeit das Werk eines Jeden sey. Wessen Werk besteht, der wird~\RWSeitenw{168}\ den Lohn empfangen (nämlich im andern Leben); wessen Werk aber verbrennt, der wird den Schaden davon haben. Er selbst wird (wollen wir hoffen) wohl noch davon kommen, jedoch wie Einer, der dem Brande entronnen ist.} -- Wenn anders der Apostel, wie es höchst wahrscheinlich ist, unter dem Lohne jenes Predigers des Evangeliums, der ein haltbares Werk aufbaut, einen Lohn jenseits des Grabes versteht: so versteht er auch unter dem Schaden, den der unweise Prediger erfährt, einen Schaden jenseits des Grabes, und einen solchen, bei dem er doch noch das Leben rettet, \dh\ der nicht ewig währt. -- \Ahat{\RWbibel{1\,Petr}{1\,Petr.}{3}{18}}{3,19.}\ wird von Jesu Christo gesagt: \erganf{Am Leibe ward er zwar getödtet, am Geiste aber ward er lebendig gemacht (\dh\ in ein noch höheres wirksameres Leben versetzt) und eben in diesem Geiste (\dh\ getrennt vom Leibe) verfügte er sich zu den Seelen der Abgestorbenen, die in einem Gefängnisse (\dh\ in einem nicht glücklichen Zustande) schmachteten, und verkündigte die Lehre des Heils auch ihnen, die einst ungläubig waren.} -- Wenn Jesus den Seelen der Verstorbenen erschienen ist, um ihnen die Lehre des Heils zu verkündigen: so mußte ihr Zustand noch einer Verbesserung fähig seyn; und so hätten wir denn auch hier wieder das Wesentliche der katholischen Kirche von einem Reinigungszustande. -- Daß die Kirche das Daseyn eines Reinigungszustandes in der That immer geglaubt habe, läßt sich gegen diejenigen, welche behaupten, daß diese Meinung aus der platonischen Philosophie in das Christenthum aufgenommen, und erst im 15ten Jahrhunderte von dem \RWbet{Concilium zu Florenz} als Glaubenslehre aufgestellt worden sey, aus vielen Stellen der Kirchenväter beweisen. Hieher gehört die schon oben angeführte Stelle \RWbet{Tertullian's}: \RWlat{Oblationes pro defunctis annua die facimus}\RWlit{}{Tertullian4}; (aus welcher zugleich erhellt, daß dieses nicht nur seine, sondern die Meinung \RWbet{seiner ganzen Kirche} gewesen seyn mußte). \RWbet{Clemens von Alexandrien} (\RWlat{Stromat.\ lib.\,5.})\RWlit{}{Clemens2} gibt die Lehre vom Reinigungszustande sogar als eine solche an, welche \RWbet{die heidnischen Philosophen aus den heiligen Büchern der Juden entlehnt} hätten.~\RWSeitenw{169}\ \RWbet{Augustinus} hat ein ganzes Buch \RWlat{de cura pro mortuis}\RWlit{}{Augustinus7} geschrieben, \uam\ 
\item Daß es auch eine Hölle, \dh\ auch einen Zustand endloser Strafen gebe, beweisen die Worte Jesu \RWbibel{Mt}{Matth.}{25}{46}\ \erganf{Und es werden die Einen zur ewigen Strafe, die Anderen aber zum ewigen Leben eingehen}; denn wie das Wort \RWgriech{a>i'wnios} in der letzten Hälfte dieser Stelle eine im strengsten Sinne \RWbet{ewige} Dauer bedeutet: so muß es eben diese Bedeutung auch in der ersten Hälfte haben. Also muß auch die Strafe ewig im strengsten Sinne des Wortes seyn. Dasselbe beweiset auch die Stelle \RWbibel{Mk}{Mark.}{9}{44}: \erganf{Wenn dich dein Fuß ärgert: so haue ihn ab; denn es ist dir besser, lahm in das ewige Leben einzugehen, als mit beiden Füßen in die Hölle, in das \RWbet{unauslöschliche Feuer}, geworfen zu werden.}
\end{aufzb}
\item Von jenem künftigen allgemeinen Weltgerichte hat uns derjenige, der es einst halten soll, selbst folgende Beschreibung hinterlassen, \RWbibel{Joh}{Joh.}{5}{28}: \erganf{Es kommt die Zeit, wo Alle, die in den Gräbern sind, die Stimme des Sohnes Gottes hören, und die Gutes gethan haben, \RWbet{zur Auferstehung des Lebens}, die aber \RWbet{Böses} gethan haben, \RWbet{zur Auferstehung des Gerichtes} (\dh\ der Verdammniß) \RWbet{hervorgehen} werden.} \RWbibel{Mt}{Matth.}{25}{31\,ff}: \erganf{Wenn der Sohn des Menschen \RWbet{in seiner Herrlichkeit}, begleitet von allen Engeln, kommen wird: dann wird er auf dem Throne der Herrlichkeit sitzen, vor ihm werden sich \RWbet{alle Völker} versammeln, er wird sie von einander sondern, wie ein Hirt die Schafe von den Böcken sondert, und wird die Schafe an seine rechte, die Böcke aber an seine linke Seite stellen. Dann wird der König zu denen an seiner Rechten sagen: \RWbet{Kommet, ihr Gesegneten meines Vaters! und nehmet das Reich in Besitz, welches euch vom Anbeginn der Welt bereitet ist; denn ich war hungrig, und ihr gabt mir zu essen;} \usw\ Dann wird er auch denen zur Linken sagen:\RWbet{ Hinweg von mir, ihr Verfluchten! in das ewige Feuer, welches dem Teufel und seinem Anhange bereitet ist; denn ich war hungrig, und ihr gabt mir nicht zu essen;} \usw~\RWSeitenw{170}\ So werden diese zur ewigen Strafe, die Gerechten aber in das ewige Leben eingehen.} -- Wann dieser Tag des allgemeinen Gerichtes erscheinen werde, ist nicht geoffenbaret; es heißt nur immer, wie \RWbibel{Apg}{Apostelg.}{17}{31}: \erganf{Er hat einen Tag festgesetzt, an welchem er über die ganze Welt ein gerechtes Gericht halten wird}, -- ohne zu bestimmen, ob dieß früh oder spät geschehen werde. \RWbibel{2\,Petr}{2\,Petr.}{3}{8}\ redet der Apostel von denjenigen, welche des jüngsten Tages spotteten, weil er noch nicht da ist: \erganf{Nur dieß Eine entgehe euch nicht, Geliebte! daß bei dem Herrn Ein Tag wie tausend Jahre, und tausend Jahre wie Ein Tag sind. Der Herr \RWbet{verabsäumet} seine Verheißung nicht, wie es Einige für Verabsäumung halten; sondern er ist nur \RWbet{langmüthig} gegen uns, und will nicht, daß Jemand verloren gehe; sondern daß Jeder sich zur Buße bekehre. \RWbet{Aber kommen wird der Tag des Herrn, wie ein Dieb in der Nacht; krachend werden die Himmel vergehen, die Elemente werden im Feuer aufgelöset werden, und die Erde mit Allem, was darauf ist, wird verbrennen.}} Ueber die Auferstehung der Leiber spricht der heil.\ \RWbet{Paulus} sehr deutlich \RWbibel{1\,Kor}{1\,Kor.}{15}{35\,ff}: \erganf{Aber es möchte Jemand fragen: Wie werden die Todten auferstehen? in welchen Körpern werden sie erscheinen? -- Du Thor! Was du säest, das lebt nicht auf, wenn es nicht zuvor erstorben ist. Und wenn du säest: so säest du noch nicht den Körper, der erst werden soll; sondern ein bloßes Samenkorn, zum Beispiele Weizen oder sonst ein anderes; Gott aber gibt ihm einen Körper, wie es ihm gefällt, und jeder Samenart ihren besonderen Körper. So ist auch nicht alles Fleisch einerlei Fleisch; sondern anders ist das Fleisch der Menschen, anders das der Thiere, anders das der Fische, anders das der Vögel. Auch gibt es himmlische und irdische Körper; aber ein anderes Ansehen haben die himmlischen, ein anderes die irdischen. Einen anderen Glanz hat die Sonne, einen anderen der Mond, einen anderen die Sterne; ja ein Stern ist von dem anderen an Glanz verschieden. So verhält es sich auch mit der Auferstehung der Todten. Verwesliches wird gesäet, Unverwesliches wird auferstehen. Unansehnliches wird gesäet, Herrliches wird auferstehen. Gebrechliches wird gesäet, Kraft\RWSeitenw{171}volles wird auferstehen. Ein sinnlicher Körper wird gesäet, ein geistiger wird auferstehen; denn es gibt einen sinnlichen Körper und einen geistigen. -- Der erste Mensch (Adam), aus Erde gebildet, war irdisch; der andere Mensch (Christus), dem Himmel entstammend, war himmlisch. -- \RWbet{Wie wir das Bild des Irdischen getragen haben: so werden wir auch das Bild des Himmlischen tragen.} Dieß aber sage ich, Brüder! \RWbet{daß Fleisch und Blut das Reich Gottes nicht erben können, und daß das Verwesliche der Unverweslichkeit nicht theilhaftig wird.} Seht, ich sage euch ein Geheimniß: \RWbet{Wir werden nicht Alle entschlafen, aber Alle verwandelt werden,} plötzlich, in einem Augenblick, auf den Schall der letzten Posaune; denn erschallen wird die Posaune, und unsterblich werden die Todten auferstehen, und wir werden verwandelt werden.} -- Und \Ahat{\RWbibel{1\,Thess}{1\,Thess.}{4}{16}}{4,17.}: \erganf{Der Herr selbst wird, wenn der Befehl ergeht, der Erzengel ruft, und die Posaune Gottes erschallt, vom Himmel kommen; \RWbet{dann werden die in Christo Verstorbenen zuerst auferstehen}; sodann werden wir, Lebenden, zugleich mit ihnen dem Herrn entgegengeführet werden in die Luft, und so allezeit bei dem Herrn seyn.} Aus dieser Stelle erhellet, daß Alle, auch die noch Lebenden, umwandelt werden sollen; daß der Stoff des neuen Leibes aus dem des gegenwärtigen beiläufig eben so genommen seyn wird, wie die Substanz der Pflanze aus der Substanz des Samenkorns; \usw\
\item Daß wir die Anzahl derer, die selig werden, uns eher klein als groß vorstellen sollen, lehrt unser Herr ausdrücklich bei \RWbibel{Mt}{Matth.}{7}{13}: \erganf{Gehet ein durch die enge Pforte; denn weit ist die Pforte, und breit die Straße, die zum Verderben führet, und Viele gehen darauf; aber eng ist die Pforte, und schmal der Weg, der zum Leben führet, und \RWbet{Wenige} finden ihn.}
\item Daß die Belohnungen des \RWbet{Himmels} gewiß \RWbet{nicht sinnlicher} Art seyn werden, sagt Jesus \RWbibel{Mt}{Matth.}{22}{30}: \erganf{Bei der Auferstehung werden sie weder zur Ehe nehmen noch genommen werden; sondern sie werden seyn wie die Engel Gottes im Himmel.} Ob aber die Leiden des Reinigungs\RWSeitenw{172}zustandes und der Hölle sinnlicher oder nicht sinnlicher Art wären, darüber hat die Kirche nichts entscheiden wollen. Ein großer Theil der Gläubigen war von jeher der ersteren Meinung. Der heil.\ \RWbet{Augustinus} aber schreibt \RWlat{(lib.\,20.\ de civit.\ Dei cap.\,16.)\RWlit{}{Augustinus1}: Qui ignis cujusmodi et in qua parte mundi vel rerum futurus sit, \RWbet{hominem scire arbitror neminem}.} -- Und der gelehrte \RWbet{Petavius} (\RWlat{de angelis lib.\,3.\ cap.\,5.})\RWlit{}{Petavius1} drückt sich über diesen Gegenstand folgender Maßen aus: \RWlat{Nullo ecclesiae decreto adhuc obsignatum videtur, neque enim ulla in synodo sancitum illud est.}
\item Daß die Seligkeiten des Himmels alle unsere Vorstellungen übertreffen werden, scheint der heil.\  \RWbet{Paulus} \RWbibel{1\,Kor}{1\,Kor.}{2}{9}\ zu sagen: Kein Auge hat es gesehen, kein Ohr gehört, und keines Menschen Herz kann es begreifen, was Gott denen bereitet hat, die ihn lieben. Oder wenn diese Stelle vielleicht nicht von den Seligkeiten des Himmels handelt: so erhellet es hinlänglich aus anderen, die wir noch anführen werden; \zB\ \RWbibel{1\,Joh}{1\,Joh.}{3}{2}: \erganf{Noch ist es nicht enthüllt, was wir seyn werden; doch sind wir gewiß, daß wir, wenn es sich enthüllen wird, ihm (Gott) ähnlich seyn werden; denn wir werden sehen, wie er ist.} Diese Stelle lehret zugleich, daß unsere vornehmste Seligkeit in einem Anschauen, \dh\ in einer lebhaften Erkenntniß Gottes bestehen werde. -- \RWbibel{1\,Kor}{1\,Kor.}{13}{12}\ sagt Paulus: \erganf{Jetzt sehen wir noch dunkel und räthselhaft, wie in einem Spiegel; einst aber (werden wir sehen) von Angesicht zu Angesicht.} Unsere Erkenntniß wird also zunehmen. -- Daß sich Liebende dort wieder finden würden, glaubten die Juden zu den Zeiten Jesu, wie wir \zB\ aus \RWbibel{2\,Makk}{2\,Makkab.}{7}{9}\ ersehen, wo jene heldenmüthige Mutter zu ihrem jüngsten Sohne spricht: \erganf{Erdulde den Tod auf eine deiner Brüder würdige Weise, damit ich dort in jenem Reiche der Erbarmung auch dich vereiniget mit deinen Brüdern antreffe.} -- Und Jesus bestärkte uns in diesem Glauben; man denke \zB\ an die Parabel vom reichen Prasser (\RWbibel{Lk}{Luk.}{16}{20}). Ja er hat ausdrücklich versprochen, daß er alle Guten um sich herum versammeln wolle. \RWbibel{Joh}{Joh.}{14}{2}: \erganf{In meines Vaters Hause sind viele Wohnungen; ich gehe hin, um euch~\RWSeitenw{173}\ einen Platz zu bereiten, und will wieder kommen, und euch zu mir nehmen, damit auch ihr seyet, wo ich bin.}
\begin{RWanm} 
Aus diesen Stellen läßt sich nun sehr leicht schließen, daß unser Leben im Himmel ein \RWbet{thätiges} seyn werde; denn wozu jenes Versammeln, \usw ? Wie würde auch ferner die Vergleichung des Lebens der Seligen mit dem Leben der Engel passen, da Letztere durchaus als thätig beschrieben werden? -- Dieß folgt auch daraus, daß von den bereits Verstorbenen gerühmt wird, sie wirkten noch immer fort für die Zurückgebliebenen; \zB\ \RWbibel{2\,Makk}{2\,Makkab.}{15}{14}\ (von Jeremias). Nach \RWbibel{Offb}{Offenb.}{5}{8}\ haben die Heiligen auch noch im Himmel Beschäftigung; sie loben Gott, sie bringen ihm die Bitten der Gläubigen dar; \usw\ Wenn aber unser Leben jenseits des Grabes ein thätiges seyn wird: so folgt hieraus auch schon, daß wir \RWbet{fortschreiten} werden in unserer Vollkommenheit; denn jede Uebung im Guten vermehrt die Vollkommenheit. Bei den Verstorbenen, die in den Reinigungszustand kommen, ist dieses Fortschreiten sogar Glaubenssache. Warum sollten sie aber, sobald sie in den Himmel aufgenommen werden, aufhören fortzuschreiten? \end{RWanm}
\item Daß es verschiedene Stufen des Lohnes und der Strafe gebe, erhellet aus \RWbibel{Joh}{Joh.}{14}{2}, wo unser Herr von vielen Wohnungen spricht, die er im Hause des Vaters uns zu bereiten gehe. Nicht alle diese Wohnungen dürften in jeder Rücksicht gleich seyn. Der heil.\ Paulus in der schon oben angeführten Stelle (\Ahat{\RWbibel{1\,Kor}{1\,Kor.}{15}{41\,ff}}{13,51.}) redet von einem Unterschiede der Leiber selbst bei den Verklärten: \erganf{Wie es Verschiedenheiten unter den Sternen gibt, so auch bei der Auferstehung.} -- Und \RWbibel{1\,Kor}{1\,Kor.}{3}{8}\ heißt es ausdrücklich: \erganf{Jeder wird seinen eigenen Lohn empfangen nach seiner eigenen Bemühung.} Und \RWbibel{2\,Kor}{2\,Kor.}{9}{6}: \erganf{Wer kärglich säet, wird auch kärglich ernten; wer aber reichlich säet, wird auch reichlich ernten.} Ein Gleiches gilt von den Strafen. \Ahat{\RWbibel{Mt}{Matth.}{11}{22}}{11,20.}: \erganf{Tyrus und Sidon wird am Gerichtstage \RWbet{mehr Nachsicht} erfahren, als ihr.}
\item Daß auch selbst der größte Sünder noch keine Ursache zu verzweifeln habe, lehrt \RWbibel{Ez}{Ezechiel}{33}{14}: \erganf{\RWbet{Wenn ich zum Gottlosen spreche: du wirst des Todes sterben; und er bekehret sich von seiner Sünde und übt Gerechtigkeit; gibt fremdes Eigenthum}~\RWSeitenw{174}\ \RWbet{zurück, wandelt die Wege des Rechts, und thut nichts Böses weiter: so soll er nicht sterben, sondern leben; und alle seine früheren Sünden sollen ihm nicht mehr angerechnet werden.}} Noch deutlicher heißt es \RWbibel{Jes}{Isai.}{1}{18}: \erganf{\RWbet{Wenn eure Sünden roth wie Scharlach wären: doch sollen sie weiß wie Schnee werden, wenn ihr gehorchen wollt. Des Landes beste Frucht sollt ihr genießen.}} -- Daß die Hoffnung des Gerechten oder Gebesserten noch immer zuversichtlicher werden könne; beweiset \RWbibel{Röm}{Röm.}{5}{1}: \erganf{Da wir durch den Glauben gerecht wurden: \RWbet{so haben wir Frieden mit Gott} durch unseren Herrn Jesum Christum, durch den wir auch im Glauben \RWbet{Zutritt zu der Gnade erhielten,} in deren Besitz wir uns befinden, und (durch den wir) \RWbet{uns der Hoffnung zur Herrlichkeit Gottes rühmen.}} Daß wir gleichwohl nie völlig gewiß seyn können, zeigt \RWbibel{Phil}{Philipp.}{2}{12}: \erganf{Mit Furcht und Zittern arbeitet an eurem Seelenheile.} \RWbibel{1\,Kor}{1\,Kor.}{10}{12}: \erganf{Wer zu stehen glaubt, der sehe zu, daß er nicht falle.}
\begin{RWanm} 
Die Protestanten haben aus der Stelle \RWbibel{2\,Petr}{2\,Petr.}{1}{10}: \erganf{Beeifert euch, Brüder! \RWbet{eueren Beruf und euere Erwählung zu befestigen;} denn wenn ihr dieses thuet: \RWbet{so werdet ihr nie sündigen}}, -- gefolgert, daß der Gläubige seiner Erwählung völlig gewiß werden könne, auch ohne eine eigene göttliche Offenbarung darüber erhalten zu haben. Sie beweiset dieß, wie man sieht, gar nicht. \end{RWanm} 
\item Daß jene Nichtchristen, welche die Wahrheit des katholischen Christenthums \RWbet{durch ihre eigene Schuld} nicht anerkannten, oder die schon erkannte Wahrheit verwarfen, dieß mit den härtesten Strafen, ja mit einer ewigen Verdammniß büßen werden: sagt Jesus selbst ausdrücklich \RWbibel{Mk}{Mark.}{16}{16}: \erganf{Wer glaubet und sich taufen läßt, wird selig werden; wer aber nicht glaubt, wird verurtheilt werden.} Daß aber ein Unglaube, der aus \RWbet{unüberwindlicher Unwissenheit} entsprang, Niemandem zur Schuld und Sünde angerechnet werden könne, wurde in der katholischen Kirche von jeher gelehrt. Unser Herr selbst sagt von den Pharisäern: \erganf{Wenn ihr blind wäret: so hättet ihr keine Sünde; nun aber saget ihr: Wir sehen; daher bleibt euere Sünde.} Die ent\RWSeitenw{175}gegengesetzte Behauptung des \RWbet{Michael Bajus} (eines Professors zu Löwen in der Mitte des 16ten Jahrhundertes) verdammte Papst Pius V. Nicht minder einstimmig sind auch die Meinungen der Katholiken darüber, \RWbet{daß es nicht möglich sey, die höheren Seligkeiten, die Jesus Christus dem menschlichen Geschlechte ausgewirket hat, zu genießen, ohne ihn selbst und seine Lehre kennen zu lernen und an ihn zu glauben.} Dieß zu beweisen berief man sich vornehmlich auf die Worte Pauli \RWbibel{Hebr}{Hebr.}{11}{6}: \erganf{Ohne Glauben ist es unmöglich, Gott zu gefallen; denn wer zu Gott kommen will, muß glauben, daß er ist, und daß er denen, die ihn suchen, ein Vergelter seyn werde.} Obgleich hier nur von dem Glauben an Gott die Rede ist: so gilt doch derselbe Grund, aus welchem Paulus schließt, \RWbet{daß man die Belohnung Gottes, nicht ohne ihn zu kennen, empfangen könne,} auch dafür, \RWbet{daß man die Vortheile der Erlösung nicht ohne Jesum zu kennen genießen könne.} Eben so berief man sich auf \Ahat{\RWbibel{Apg}{Apostelg.}{4}{12}}{4,2.}: \erganf{\RWbet{Auch ist in keinem Andern das Heil; denn es ist auch kein anderer Name unter der Sonne dem Menschen gegeben, in welchem wir Heil finden können.}} Ein Ausdruck, der so gewählt ist, daß selbst die unverschuldete Unwissenheit nicht ausgeschlossen wird: \RWbet{Jesus ist der einzige Retter der Menschen! wer ihn also nicht kennt, wird aus Mangel dieser Kenntniß auch nicht gerettet.} -- Daher denn auch die Redensart der Kirche, vermöge der sie ihren Glauben den \RWbet{allein selig machenden} nennt; wodurch sie anzeigen will, daß -- sollte es auch vielleicht möglich seyn, in einem andern Glauben selig zu werden -- doch wenigstens nur dieser einzige das uns \RWbet{bekannte und ordentliche Mittel} sey, welches den Menschen selig macht. Das \RWlat{Symbolum Athanasianum}\RWlit{}{SymbolumAthanasianum} fängt mit den Worten an: \erganf{\RWlat{Quicumque vult salvus esse, ante omnia opus est, ut teneat catholicam fidem, quam nisi quisque integram inviolatamque servaverit, absque dubio in aeternum peribit.}} \udgl\  \RWbet{Wie aber das Schicksal dieser Nichtchristen, welche es ohne ihr Verschulden sind, eigentlich beschaffen seyn werde, darüber denken die Katho}\RWSeitenw{176}\RWbet{liken nicht ganz übereinstimmig.} Viele, ja wohl die Meisten schließen diese Nichtchristen von dem Heile der Christen in alle Ewigkeit aus. Andere dagegen sind der Meinung, daß die unendliche Barmherzigkeit Gottes ihnen Gelegenheit verschaffe, wo nicht noch in der letzten Stunde des gegenwärtigen Lebens, doch in dem künftigen, unseren Herrn Jesum kennen zu lernen, und durch den Glauben an ihn zur Seligkeit zu gelangen. Dieses erwarten Viele besonders dann mit aller Bestimmtheit, wenn diese Nichtchristen wenigstens den sehnlichen Wunsch genährt, die Anstalten, welche Gott zu unserer Beseligung getroffen hat, kennen zu lernen und zu benützen. In einem solchen Wunsche erkennen die ersten katholischen Theologen die \RWbet{echte gläubige Gesinnung} (\RWlat{fidem implicitam}), welche den Mangel des \RWbet{ausdrücklichen Glaubens} (\RWlat{fides explicita}) ersetzen könne. Wir wollen die Aeußerungen Einiger von diesen milder denkenden christlichen Schriftstellern anführen, und es wird sich zeigen, daß sich darunter auch Männer befinden, die in der katholischen Kirche im größten Ansehen gestanden sind.
\begin{aufzc}
\item \RWbet{Justinus} (\RWlat{apologia 2.})\RWlit{}{Justinus1}: \erganf{Diejenigen, so ehemals nach der Vernunft und dem Worte gelebt, \RWbet{sind Christen gewesen, wenn man sie gleich für Gottesläugner gehalten.} Solche sind unter den Griechen Sokrates, Heraklitus und Andere, die ihnen gleich gewesen; unter den Barbaren Ananias, Abraham, Azarias und Andere. Und noch jetzt sind Alle, \RWbet{die der Vernunft und dem Worte gemäß leben, als Christen anzusehen, und haben sich vor nichts zu fürchten}.} Durch den Ausdruck \RWbet{Christen} will hier Justinus gewiß nichts Anderes andeuten, als daß die genannten Personen die \RWbet{echte gläubige Gesinnung} eines Christen gehabt.
\item Der \RWbet{heil.\ Gregor von Nazianz} (im 4ten Jahrhunderte \RWlat{sermon.\ in sacr.\ lavacr.})\RWlit{}{GregorvonNazianz2} scheint nicht zu glauben, daß Jemand der bloßen Erbsünde wegen verdammt werde. Und eben dasselbe sagt auch 
\item der \RWbet{heil.\ Gregor von Nyssa} (\RWlat{oratio de infantibus})\RWlit{}{GregorvonNyssa1}.
\item Der \RWbet{heil.\ Augustin} (\RWlat{contra Julianum l.\,5.\ c.\,8.})\RWlit{}{Augustinus6} schreibt: \erganf{\RWlat{Non dico, \RWbet{parvulos, sine Christi}}~\RWSeitenw{177}\ \RWlat{\RWbet{baptismo morientes}, tanta poena plectendos esse, ut eis non nasci profuisset.} (Lib.\,18.\ de civitate Dei c.\,47.)\RWlit{}{Augustinus1}: \erganf{Non incongrue creditur, fuisse et in \RWbet{aliis gentibus} homines, quibus hoc mysterium (de incarnatione) revelatum est. -- -- \RWbet{Multi inter gentes} pertinuerunt ad civitatem spiritualem Jerusalem.}} --
\item \RWbet{Salvianus} (im 5ten Jahrhunderte Bischof zu Marseille) schreibt \RWlat{(de gubernatione Dei lib.\,5.)\RWlit{}{Salvianus1}: \erganf{\RWbet{Qualiter} pro hoc falsae opinionis errore in die judicii puniendi sunt, \RWbet{nullus potest scire, nisi judex}.}}
\item \RWbet{Chrysostomus} schreibt (\RWlat{in commentario in Matthaeum 11.}\RWlit{}{JohannesChrysostomos2}, wenn dieser Commentar ja echt ist, und \RWlat{ad Romanos 2, 10.})\RWlit{}{JohannesChrysostomos1}: \erganf{Diejenigen, welche \RWbet{vor Christi Zeit gestorben} sind, und ihn daher auch nicht erkannt haben, wenn sie sich nur vom Götzendienste entfernt gehalten, und dagegen den einzigen wahren Gott angebetet und tugendhaft gelebt haben, \RWbet{werden die ewige Seligkeit erlangen.} Höre, was Paulus sagt: Ehre und Heil Jedem, der Gutes thut, er sey ein Jude oder Heide.}
\item \RWbet{Paulus Orosius} (ein spanischer Priester im Anfange des 5ten Jahrhundertes) schreibt (in seiner \RWlat{apologia pro libertate arbitrii})\RWlit{}{Orosius1}: \erganf{Das war immer meine aufrichtige und unbezweifelte Gesinnung, daß Gott seinen Beistand nicht bloß denjenigen, die in der Kirche leben, sondern \RWbet{allen Völkern auf Erden} nach seinem gütigen und ewigen Rathschlusse angedeihen lasse. Allen und jedem insbesondere kommt er täglich zu Hilfe; Niemand ist von seiner gnädigen Unterstützung ausgeschlossen.} (Ohne Zweifel glaubte er also auch, daß diese Menschen, wenn sie die Unterstützung Gottes benützen, selig werden können).
\item \RWbet{Thomas von Aquino} (dieser eben so fromme als gelehrte und hochgerühmte Dominikaner des 13\hoch{ten} Jahrhunderts, den man \RWlat{doctorem angelicum} nannte) äußert (\RWlat{Quaest.\,14.\ de veritate, art.\,11.} \uamO )\RWlit{}{ThomasAquinas2} die Meinung: \erganf{\RWlat{Hoc pertinet ad divinam providentiam, \RWbet{ut cuilibet provideat de mediis ad}~\RWSeitenw{178}\ \RWbet{salutem}, dummodo ex parte ejus non impediatur. Unde si aliquis nutritus in silvis inter lupos, ductum rationis naturalis sequeretur in appetitu boni et fuga mali, certissime est tenendum, quod ei Deus vel per internam inspirationem revelaret, quae sunt ad credendum necessaria, vel aliquem fidei praedicatorem ad eum dirigeret, sicut Petrum ad Cornelium.}}
\item Papst \RWbet{Alexander VIII.}\ verdammte den Satz: \erganf{\RWlat{Pagani, Judaei, haeretici etc. nullum omnino accipiunt a Jesu Christo influxum.}}
\item Der kürzlich verstorbene Fürst Primas \RWbet{Dalberg}, und die meisten neueren Theologen bekennen sich zu der milderen Meinung.
\item Von jeher lehrte man endlich in der katholischen Kirche, daß auch die sogenannte \RWlat{fides implicita}, die bloße gläubige Gesinnung, zur Seligkeit genüge, wenn Jemand (und zwar ohne sein Verschulden) keine Gelegenheit hat, sich eine deutliche Erkenntniß, \dh\ die sogenannte \RWlat{fidem explicitam} zu verschaffen. Die \RWlat{fides implicita} nun, oder die echte gläubige Gesinnung, \dh\ der ernste Wille, Alles zu glauben, was Gott geoffenbaret hat, wenn es uns nur bekannt wird, muß sich bei jedem Tugendhaften finden; denn jeder Tugendhafte muß, wenn auch nicht eben mit ausdrücklichen Worten sprechen: Ich bin bereit, Alles zu glauben, \usw ; diese Gesinnung doch in der That haben. Durch diese Lehre von der \RWlat{fides implicita} also gibt das katholische Christenthum jedem Tugendhaften die Hoffnung, zur ewigen Seligkeit einst zu gelangen.
\end{aufzc}
\end{aufza}

\RWpar{233}{Vernunftmäßigkeit}
\begin{aufza}
\item Daß die Vergeltung \RWbet{nicht immer schon in diesem gegenwärtigen Leben} eintritt, lehrt uns selbst die Erfahrung, und eben deßhalb findet es die Vernunft nothwendig, daß sie in jenem anderen Leben eintrete.
\item Daß \RWbet{die Stunde des Todes völlig ungewiß} sey, und durch \RWbet{kein blindes Schicksal}, sondern \RWbet{durch}~\RWSeitenw{179}\ \RWbet{Gottes besondere Vorsehung} herbeigeführt werde, ist der Vernunft vollkommen angemessen. Allerdings gibt es Grade der göttlichen Vorsehung, und die Stunde des Todes verdient es um ihrer Wichtigkeit willen, daß Gott sehr viele andere (nämlich minder wichtige) Vortheile aufopfere, um nur den Augenblick des Todes für jeden Menschen genau zu der Zeit herbeizuführen, welche die zuträglichste für ihn und Andere ist. Die Behauptung, daß die \RWbet{Vergeltung gleich nach dem Tode} anfange, enthält nichts Unmögliches. Noch eher möchte man die entgegengesetzte Vorstellung von einem Seelenschlafe, der durch Jahrtausende währet, mit der Weisheit Gottes im Widerspruche finden, weil so ein großer Zeitraum der menschlichen Existenz unbenützt für Tugend und Glückseligkeit verloren ginge.
\item \RWbet{Dreifacher Zustand nach dem Tode.}
\begin{aufzb}
\item \RWbet{Himmel.} Es ist der göttlichen Vollkommenheit ganz angemessen, daß der Tugendhafte, der frei von aller Sünde und Schuld ist, belohnet, und zwar mit einer ewigen Seligkeit belohnet werde, wenn nur auch er noch tugendhaft zu bleiben fortfährt.
\begin{aufzc}
\item \RWbet{Einwurf.} Aber eben in der Annahme einer ununterbrochen fortwährend moralischen Gesinnung bei einem endlichen Wesen liegt ein Widerspruch mit dessen Freiheit. Was immerwährend ist, das muß auch nothwendig seyn.\par
\RWbet{Antwort.} Was nothwendig ist, muß allerdings auch immerwährend seyn; aber nicht umgekehrt muß alles Immerwährende auch nothwendig seyn. -- Es steht mit unserer Freiheit auf Erden in keinem Widerspruche, daß wir durch eine längere Zeit hindurch, \zB\ einen oder mehrere Tage, dem Sittengesetze so getreu nachleben, daß wir uns keine einzige Verletzung desselben zu Schulden kommen lassen. Was nun durch Eine gewisse Zeit geschehen kann, das kann wohl auch durch eine längere Zeit hindurch geschehen; wir werden also besonders im andern Leben, wo die Versuchungen viel seltener und viel schwächer seyn werden, auch mehrere Jahre, Jahrzehende hindurch \usw\ dem Sittengesetze getreu leben können.~\RWSeitenw{180}
\item \RWbet{Einwurf.} Die Voraussetzung, daß die Seligkeit des Himmels ganz ungetrübt seyn, die Tugend uns gar keinen Kampf mehr kosten werde, hebt unsere Freiheit selbst auf.\par
\RWbet{Antwort.} a)~Es widerspräche der göttlichen Offenbarung nicht schlechterdings, wenn wir zur Lösung dieses Einwurfes annähmen, daß es auch selbst im anderen Leben einen gewissen Streit zwischen Vernunft und Glückseligkeitstrieb, also gewisse Kämpfe, gebe, welche jedoch so leicht zu besiegen sind, daß wir uns bei denselben niemals unglücklich fühlen, sondern vielmehr unseren Zustand als einen höchst seligen preisen. Wahr ist es freilich, daß das katholische Christenthum hievon nichts sagt; aber es ist auch einleuchtend, daß diese Bemerkung, selbst wenn sie wahr wäre, sonst keinen anderen Nutzen hätte, als uns zur Lösung des gegenwärtigen Einwurfes zu dienen; wohl aber, durch Mißverstand, unsere Vorstellung von der hohen Seligkeit des Himmels leicht auf eine nachtheilige Art verdunkeln könnte. Billig also mußte die göttliche Offenbarung von diesem Umstande schweigen.\par
b)~Doch eben so gut läßt sich zur Lösung jenes Einwurfes auch sagen, es komme im andern Leben mit uns dahin, daß wir gar keine Versuchungen zum Bösen mehr fühlen. Dieses muß nämlich geschehen, sobald wir es im anderen Leben endlich dahin bringen, daß uns die Wahrheit, die wir schon hier auf Erden erkennen, daß nur die Tugend allein beglücke, beständig gegenwärtig bleibt. Dann werden wir eben deßhalb gar keine eigentliche Versuchung zum Bösen mehr fühlen. Wahr ist es, daß wir dann auch keine eigentliche Freiheit, nämlich diejenige, die in der Möglichkeit, Böses zu thun, besteht, besitzen werden; aber wir werden doch immer den Namen sittlicher Wesen verdienen, und der Belohnung werth seyn. Denn eben nur durch den sehr guten Gebrauch, den wir von unserer Freiheit auf Erden gemacht, kann es dahin kommen, daß wir dereinst in diesen glücklichen Zustand, in dem wir nicht einmal sündigen können, versetzet werden.
\end{aufzc}
\item \RWbet{Reinigungszustand.} Daß Menschen, die behaftet mit gewissen läßlichen Sünden aus dieser Welt austreten, oder die noch gewisse endliche Strafen für schon bereute Todsünden zu ertragen haben, in einen Zustand~\RWSeitenw{181}\ endlicher Leiden gerathen, ist der Vernunft ganz angemessen. Nicht nur daß solche Leiden wegen der Wahrhaftigkeit Gottes, so wie er sie angedrohet hatte, eintreten müssen, sondern sie können selbst ein Mittel zur Vervollkommnung seyn, so wie schon Leiden auf Erden öfters zu diesem Zwecke dienen.
\begin{RWanm} 
Die katholische Lehre vom Reinigungszustande ist in der That der Vernunft so angemessen, daß auch schon \RWbet{Plato} und mehrere andere heidnische Weltweisen aus bloßen Gründen der Vernunft die Nothwendigkeit eines solchen Zustandes erkannten; und es wäre wirklich nicht zu begreifen, wie Luther und seine Anhänger diese Lehre einst verwerfen konnten, wenn wir nicht wüßten, mit welcher Leidenschaftlichkeit man in jener Zeit zu Werke gegangen, und wenn der mit ihm verbundene Mißbrauch und Aberglaube nicht dazu beigetragen hätte, die Gemüther wider sie einzunehmen, und sie verhaßt zu machen.
\end{RWanm}
\item \RWbet{Hölle.} Um desto häufiger bestritten wurde die Behauptung des Christenthums, daß alle diejenigen Menschen, die mit einer Todsünde behaftet, \dh\ in einer unmoralischen Gesinnung, aus dieser Welt austreten, in einen ewig unseligen Zustand gerathen.\par
Zur Rechtfertigung dieser Lehre werden folgende Betrachtungen dienen: 
\begin{aufzc}
\item Könnte es nicht seyn, daß Menschen, die mit einer lasterhaften Gesinnung aus dieser Welt austreten, diese Gesinnung auch jenseits des Grabes beibehalten? Niemand kennt den Zusammenhang zwischen dem jetzigen und künftigen Leben, den Einfluß, den unsere hier erworbenen Gesinnungen und Fertigkeiten im Guten sowohl als im Bösen auf unsere Beschaffenheit in der zukünftigen Welt haben, genau genug, um dieses läugnen zu können.
\item Könnte es nicht ferner seyn, daß diejenigen, die mit einer Todsünde behaftet aus dieser Welt austreten, selbst in dem Falle, wenn sie in jenem andern Leben Gelegenheit finden, sich einiger Maßen zu bessern, und eben deßhalb auch zu dem Besitze eines in Etwas erträglicheren Schicksals gelangen, doch immer noch weit hinter denjenigen zurückbleiben, die schon auf Erden der Tugend nach\RWSeitenw{182}gestrebt haben? Könnten wir, wenn dieß ist, nicht immer sagen, daß sie ein unglückseliges, ein wenigstens im Vergleiche mit Andern ein sehr unglückliches Loos erfahren?
\item Endlich gesetzt, es wäre wahr, daß eine Besserung noch nach dem Tode möglich ist: doch würde das Christenthum nicht Unrecht daran thun, daß es auch schon demjenigen, der nur ein einziges Mal mit deutlichem Bewußtseyn sündiget, die ewigen Strafen der Hölle androht. Denn diese Drohung würde, ob sie nun gleich etwas Bildliches enthielte, doch ihrem Geiste nach wahr seyn, \dh\ gerade diejenigen Wirkungen in uns hervorbringen, welche die wahre Ansicht der Dinge, wie sie an sich sind, in uns hervorbringen müßte, wenn wir derselben fähig wären. Daß nämlich jede Sünde uns unglücklich mache, und (sonach schon um unseres eigenen Vortheiles willen) aus allen Kräften von uns gemieden zu werden verdiene, ist außer allem Zweifel. Gerade diese Wirkung aber hat oder befördert wenigstens die katholische Lehre von der Hölle, und zwar in einem Maße, wie keine andere Ansicht der Sache es vermöchte. Diese Ansicht hat mithin Wahrheit, unwidersprechliche Wahrheit für uns. Und nicht derjenige, der durch den Gedanken an die ewigen Strafen der Hölle sich von der Sünde abschrecken läßt, sondern im Gegentheile nur derjenige, der, weil er nicht glaubt an die Ewigkeit der Hölle, eben darum weniger aufmerksam auf die Vermeidung jeder Sünde ist, wird einst mit Schmerzen inne werden, daß er sich selbst getäuscht und betrogen habe.
\end{aufzc}
\begin{RWanm} 
Daher kommt es denn auch, daß selbst \RWbet{heidnische} Weltweisen häufig die Ewigkeit der Höllenstrafen lehrten. \RWbet{Plato} \zB\ schreibt im Phädon: Alle Verstorbenen werden von ihrem Geiste dem höchsten Richter vorgeführt. Jene, die auf dem Pfade der Gerechtigkeit, Weisheit und Tugend gewandelt, und jene, die einen anderen Weg betreten haben, vernehmen auf gleiche Art das für sie bereitete Schicksal vor diesem höchsten Richterstuhle. Die Letzteren, \dh\ diejenigen, die sich mit gräulichen Lastern, mit Gottesraub, Meuchelmord, Verachtung heiliger Gesetze beflecket haben, werden in die Hölle gestürzt, \RWbet{aus der sie nie wieder entkommen}. -- Und im Gorgias läßt er den Sokrates über~\RWSeitenw{183}\ diese Lehre folgende Aeußerung thun: Man wird vielleicht das, was ich hier sage, wenig beachten; aber ich muß gestehen, daß ich nach reiflicher Ueberlegung der ganzen Sache nichts gefunden habe, was der Vernunft, der Weisheit und der Wahrheit entsprechender wäre, als eben dieß. \end{RWanm}
\begin{aufzc}
\item \RWbet{Einwurf.} Der \RWbet{Zweck} aller Strafen muß doch Besserung seyn. Bei den Strafen der ewig Verdammten findet aber die letztere nicht Statt. Also sind solche Strafen zweckwidrig.\par
\RWbet{Antwort.} Es ist nicht wahr, daß Besserung der \RWbet{einzige} Zweck aller Strafen sey, sondern wie man von einer jeden Anstalt, die ein vernünftiges Wesen trifft, verlangen kann, daß es sich bei derselben die Erreichung alles desjenigen Guten, so durch eine Anstalt dieser Art nur immer erreicht werden kann, zum Zwecke setze: so gilt dasselbe auch von den Strafen, sowohl denjenigen, die von uns Menschen, als auch von denen, die von Gott selbst über Jeden, der Böses thut, verhängt werden. Durch solche Strafen können im Allgemeinen mehrere gute Wirkungen, als nur die Besserung des Bestraften allein, bewirket werden; und somit wäre es einseitig von Gott geurtheilet, wenn wir glaubten, daß er sich nur die Besserung des Bestraften allein vorsetze, und also dort, wo diese Eine gute Wirkung nicht Statt finden kann, schon keine Strafe verhänge, als ob er vergäße, daß doch noch gewisse andere Vortheile durch sie erreicht werden können. Dergleichen Vortheile sind: die Warnung Anderer, gewisse ganz eigenthümliche Vortheile, die aus den Leiden des Bestraften für Andere hervorgehen können. Könnte es \zB\ nicht gewisse minder angenehme Aufenthaltsorte, Verrichtungen \udgl\  in Gottes Schöpfung geben, in und zu welchen die Bösen zum Nutzen für die Uebrigen verurtheilet würden? Ja, könnte es endlich nicht seyn, daß gewisse Sünden nur eben durch eine Strafe, die ewig dauert, gebessert würden?
\item \RWbet{Einwurf.} Es widerspricht der Freiheit des Menschen, daß er sich ewig nicht bessere.\par
\RWbet{Antwort.} Eben so wenig, als daß er ewig tugendhaft bleibe.
\item \RWbet{Einwurf.} Aber die Allmacht Gottes muß doch im Stande seyn, jeden Lasterhaften zu bessern.~\RWSeitenw{184}\par
\RWbet{Antwort.} Auch die Allmacht kann den frei geschaffenen Willen nicht zwingen, weil sich dies widerspricht.
\item \RWbet{Einwurf.} So hätte Gott ein Geschöpf, von dem er vorhersah, daß es sich niemals bessern, und also ewig unglücklich machen würde, gar nicht erschaffen, oder das schon geschaffene wieder vernichten sollen; denn er ist nicht berechtigt, irgend eines seiner Geschöpfe ewig unglücklich zu machen, auch wenn hiedurch das Wohl der Uebrigen noch so bedeutend gewinnen sollte.\par
\RWbet{Antwort.} Es ist sehr falsch, daß Gott auf keine Weise berechtiget wäre, eines seiner Geschöpfe ewig unglücklich zu machen. Ein Tugendhaftes allerdings nicht, weil eine ausnahmslose Regel (eine Regel, die zur Beförderung des allgemeinen Wohles so ausnahmslos seyn muß) es eben fordert, daß der Tugendhafte glücklich gemacht werde. Aber ein lasterhaftes Geschöpf kann Gott allerdings ewig unglücklich machen, wenn andere Geschöpfe überwiegende Vortheile davon haben, \zB\ abgeschreckt werden, und durch diese Abschreckung sehr vieles Böse unterlassen. Läßt Gott den Lasterhaften ewig fortdauern: so wirkt auch das Beispiel der Abschreckung ewig fort, und kann daher unendlich vielen Geschöpfen zum Besten dienen, unendlich viel Gutes in der Welt stiften. --
\end{aufzc}
\end{aufzb}
\item Daß es nebst dem besonderen Gerichte, das Jeder gleich nach dem Tode erfährt, auch noch ein \RWbet{allgemeines und öffentliches} gebe, ist eine Lehre, die, wenn wir dasjenige, was in ihr bildlich ist, gehörig absondern, durchaus nichts Ungereimtes enthält. Auch unsere bloße Vernunft vermag mehrere sittliche Vortheile, welche die Einführung eines solchen Weltgerichtes hat, zu begreifen.
\begin{aufzb}
\item Durch eine solche Einrichtung wird Gottes Gerechtigkeit für alle künftigen Wesen in das hellste Licht gesetzt, weil nun ein jedes selbst einsehen soll, aus welchem Grunde Gott das Eine belohne, das Andere bestrafe, und daß hiemit jedem genau nur das zu Theil werde, was er verdienet oder verschludet hat.
\item Wenn alles Gute, auch selbst dasjenige, was im Verborgenen geschieht, oder auf Erden verkannt wird, einst noch an's Tageslicht gebracht werden soll: so wird es~\RWSeitenw{185}\ eben darum auch zur Erbauung aller in der Tugend noch unbefestigter Wesen dienen, und vieles Aergerniß wird gehoben werden; dem Tugendhaften selbst aber wird die verdienteste Rechtfertigung und die erfreulichste Belohnung zu Theil, wenn er erfahren wird, wie viele gute Früchte der Same, den er unter Thränen ausgestreut hat, getragen habe, \usw\ Noch mehrere Vortheile, die schon \RWbet{der bloße Glaube} an diese Lehre hervorbringt, werden wir unter der Betrachtung des sittlichen Nutzens desselben berühren.
\end{aufzb}
In dieser Lehre wird stillschweigend vorausgesetzt, daß die gegenwärtige Verfassung des menschlichen Geschlechtes einmal ein Ende nehmen soll. Auch dieses enthält nichts Widersprechendes, so wenig, als die Behauptung, daß sie einen Anfang genommen habe. Vielmehr läßt sich Beides aus bloßen Gründen der Vernunft als nothwendig erkennen.\par
Eben so wenig Anstößiges enthält die Behauptung, \RWbet{daß Jesus Christus einst wieder auf Erden erscheinen soll,} und zwar in \RWbet{göttlicher Majestät} \usw ; obgleich sich von selbst versteht, daß in der Ausmalung dieses Bildes nicht Alles eigentlich zu nehmen ist (\zB\ Posaunenschall \usw ). Dieß Bild soll nur den Zweck haben, eine recht lebhafte Vorstellung in uns zu bewirken.\par
Die \RWbet{Auferstehung der Leiber} könnte nur demjenigen ungereimt dünken, der sich die Vorstellung machte, daß wir \RWbet{die gegenwärtigen Leiber mit allen ihren Bestandtheilen}, auch selbst denjenigen, die seitdem Bestandtheile anderer Leiber geworden sind, wieder annehmen sollen. Allein die Offenbarung lehrt, wir würden \RWbet{neue, verklärte} Leiber erhalten, die nur in soferne unsere vorigen werden genannt werden, in wiefern wir uns \RWbet{in ihnen als in den vorigen fühlen werden.} Zu dieser Wirkung ist vielleicht nicht einmal nöthig, daß nur ein einziger Atom derselbe sey, und wenn ja nothwendig ist, daß einige Theile der Materie die nämlichen seyen: so sind dieß sicher nur einige sehr feine Theile, etwa jene des Seelenorgans. Andere gröbere Theile des Körpers verlieren wir ja auch schon in diesem Leben, ohne daß wir das Bewußtseyn der Identität verlieren. Jene~\RWSeitenw{186}\ feineren Bestandtheile aber dürften eben deßhalb so flüchtig und unter einander zugleich so innig verbunden seyn, daß sie gleich nach dem Tode sich in Begleitung der Seele selbst entfernen, oder daß sie doch wenigstens nie in die wesentlichen Bestandtheile des Körpers eines anderen Menschen verarbeitet und aufgenommen werden.
\begin{RWanm} 
Was man noch ferner eingewendet hat, daß, wenn alle Menschen, die je auf Erden gelebt, auf ein Mal auferstehen sollten, die Erde nicht einmal \RWbet{Masse} genug haben würde, sie alle mit Leibern zu versehen; daß diese Menschen nicht einmal \RWbet{Platz} genug haben würden, um neben einander zu stehen und den Spruch ihres Richters zu vernehmen, \usw\ -- das Alles sind Einwürfe, die gar keiner Widerlegung werth sind, indem sie auf grobem Mißverstande beruhen, und zum Theile selbst die frechsten Unwahrheiten enthalten. So hat \zB\ \RWbet{Jacques Massé} (eigentlich \RWbet{Simon Tyssot de Patot}) behauptet, daß die Erde nicht Masse genug haben würde, um alle die Menschen, die je auf ihr gelebt, mit Leibern zu versehen; und doch läßt sich darthun, daß, wenn auch zu jeder Zeit 1000 Millionen Menschen auf Erden gelebt haben würden, und wenn man verlangen sollte, daß sie bei ihrer Auferstehung Leiber von einer solchen Größe, wie unsere jetzigen, erhalten sollen, Alle, die bis auf diesen Tag gelebt, kaum einer Masse bedürften, die 19 Meilen in's Gevierte beträgt, und 6 Schuh tief ist; was nur ein 2000 Milliontheil der ganzen Erdmasse ausmacht. Der k.~Ingenieur und Geograph \RWbet{Joulain} trieb die Unverschämtheit sogar so weit, zu behaupten, daß alle Menschen nicht einmal Platz auf Erden hätten: und dennoch, wenn wir Jedem zwei volle Quadratschuhe einräumen wollten: so würden Alle, die bis auf den heutigen Tag auf Erden gelebt, kaum einen Platz von 28 Meilen in's Gevierte einnehmen, so daß also noch 72 Millionen Jahre verfließen können, bis sie die ganze Erde bedecken würden. Dieß Alles führe ich hier bloß zum Beweise an, wie unredlich die Gegner der göttlichen Offenbarung in ihren Einwürfen sind. 
\end{RWanm}
\item Da das katholische Christenthum über die \RWbet{Zahl der Seligen} nichts Ausdrückliches festsetzt, und nur, um eine heilsame Furcht und einen immer regen Eifer zum Guten in uns zu unterhalten, verlangt, daß wir uns diese Zahl lieber \RWbet{sehr klein vorstellen} sollen: so kann hierin Niemand etwas Anstößiges finden.~\RWSeitenw{187}
\item Daß die \RWbet{Belohnungen des Himmels nicht von sinnlicher Art} seyn werden, ist eigentlich schon eine Folge davon, daß unsere neuen verklärten Leiber nicht mehr die vorigen Organe und Sinneswerkzeuge besitzen werden.
\item Hieraus ergibt sich aber auch, daß wir uns von der Beschaffenheit unseres Glückes in jenem andern Leben \RWbet{keinen ganz deutlichen Begriff machen können.} Der Weisheit und Güte Gottes ist es nun allerdings gemäß, uns eine Seligkeit bereitet zu haben, die Alles übersteigt, was wir uns vorstellen können. Was insbesondere die \RWbet{Zusammenkunft mit anderen seligen Geistern, mit unseren Freunden und Bekannten,} welche gleich uns der Tugend nachgestrebt haben, endlich \RWbet{mit Jesu Christo} selbst betrifft: so sieht die Vernunft hier durchaus nichts Unmögliches. Man hat zwar eingewendet, daß Jene, die uns schon vor Jahrhunderten, \udgl\  vorangegangen sind, auf einer viel höheren Stufe der Vollkommenheit stehen, wenn wir erst anlangen, daß es ferner auch kein Mittel, sich zu erkennen geben werde; daß endlich mit Ablegung unseres irdischen Leibes und unserer sinnlichen Leidenschaften und Neigungen auch die Liebe und Freundschaft, die wir der Eine gegen den Andern hier empfunden haben, verschwunden seyn werde; \udgl\  -- Aber diese Einwürfe entstehen größtentheils nur daraus, daß man zu dieser Lehre sich Bestimmungen hinzudenkt, welche in ihr nicht wirklich enthalten sind. Wer heißt uns doch, die Verschiedenheit der Stufen, auf denen die Seligen im andern Leben stehen, als eine \RWbet{örtliche} Verschiedenheit betrachten, und daraus folgern, daß sie nun gar keinen Umgang mit einander pflegen könnten? Wer heißt uns annehmen, daß alle Wesen in Gottes Schöpfung auf dem Wege der Vervollkommnung mit einer gleichen Geschwindigkeit fortschreiten, dergestalt, daß die Abstände, welche zwischen dem Einen und dem Anderen jetzt Statt finden, für immer bleiben müßten? -- Was berechtigt uns zu der Behauptung, daß mit der Ablegung des Leibes auch alle Mittel, einander zu erkennen, wegfallen werden? Stehen uns denn nicht selbst auf Erden gar mancherlei andere Mittel, einander zu erkennen, wenn sich auch die Gesichtszüge bis zum Unkenntlichen verändert~\RWSeitenw{188}\ haben, zu Gebote? -- Wie mag man endlich behaupten, daß durch die Auflösung oder die Umschaffung unseres Leibes auch alle auf Erden gefaßten Neigungen und freundschaftlichen Gefühle aufgehoben werden? Sind denn dergleichen Neigungen und Gefühle durchaus auf sinnliche Verhältnisse gegründet? Gibt es nicht auch eine Liebe der Seelen? \usw\par
Das \RWbet{Anschauen Gottes}, wenn hier ein sinnliches Anschauen (\RWlat{oculo corporeo}) verstanden werden müßte, wäre wohl allerdings eine Ungereimtheit. Allein der eigentliche Sinn dieses kirchlichen Ausdruckes ist nur, daß wir zu einer \RWbet{weit lebhafteren und vollständigeren Erkenntniß Gottes} gelangen werden, als unsere jetzige ist; daß wir Gott dann so lebhaft erkennen werden, wie wir jetzt jene Gegenstände erkennen, die wir mit Augen schauen. Dieß ist nun allerdings möglich; gewiß aber ist es, daß eine so lebhafte und so vollständige Erkenntniß Gottes für jedes tugendhafte und höherer Vergnügungen fähige Wesen die größte Seligkeit seyn müsse.
\item \RWbet{Verschiedene Stufen des Lohnes sowohl als auch der Strafe} muß es nothwendig geben, \Ahat{damit}{wenn} für den Tugendhaften nie der Antrieb, noch besser zu werden, und für den Lasterhaften nie der Abhaltungsgrund vor noch größeren Verbrechen wegfalle.
\item Daß Niemand Ursache habe, an der Möglichkeit seiner Besserung und somit auch \RWbet{an seiner endlichen Seligkeit zu verzweifeln,} folgt aus der Lehre von der Freiheit unseres Willens und von dem Beistande der Gnade Gottes, die einem Jeden, der nur von ihr Gebrauch machen will, in einem hinreichenden Maße zu Theil werden soll. Aus eben dieser Freiheit aber, und aus Betrachtung der vielen traurigen Erfahrungen, die wir in dieser Hinsicht öfters, theils an Andern gemacht, folgt, daß wir nie völlig sicher seyn können, daß wir, wenn wir jetzt stehen, nicht einst wieder fallen werden.
\item Wenn die \RWbet{Verwerfung des christlichen Glaubens} aus einer nicht unüberwindlichen Unwissenheit herrührt, sondern wenn Leidenschaft, Trägheit \usw\ die Ursachen derselben sind: so ist sie offenbar sehr sträflich, und um so mehr,~\RWSeitenw{189}\ je deutlicher das Bewußtseyn ist, daß man der Erkenntniß der Wahrheit widerstrebe. Daß aber eine Verwerfung des Glaubens aus unüberwindlicher Unwissenheit, Niemand zur Schuld und Sünde angerechnet werden könne, ist eine nothwendige Vernunftwahrheit. -- Allein, daß es auch auf der anderen Seite nicht möglich sey, die höheren Vortheile, die Jesus Christus dem menschlichen Geschlechte ausgewirkt hat, in ihrem ganzen Umfange zu genießen, wenn man ihn nicht erst kennen gelernt und seine Lehre gläubig angenommen hat, läßt sich aus mehr als Einem Grunde begreifen.
\begin{aufzc}
\item Das katholische Christenthum macht uns mit einer Menge sehr wichtiger Begriffe vom wahren Wesen der menschlichen Tugend, mit einer Menge neuer und kräftiger Mittel und Aufmunterungsgründe zur Tugend, mit einer Menge höchst tröstlicher und erfreulicher Aufschlüsse über die Zukunft, mit einer Menge heilsamer Wahrheiten bekannt: wer also dieses katholische Christenthum auch ohne sein Verschulden nicht kennen lernt, der muß doch aller der Vortheile, die eine bloße Wirkung dieser Erkenntnisse sind, entbehren; er schreitet langsamer fort in seiner sittlichen Vervollkommnung, und eben darum auch in seiner Glückseligkeit.
\item Was andere Vortheile betrifft, zu deren Genuß die Lehre Jesu vielleicht nicht eben so unumgänglich nothwendig wäre, wie das Vorhandenseyn der Ursache zur Entstehung der Wirkung: so scheint es aus einem anderen Grunde doch nicht schicklich zu seyn, daß sie denjenigen, die Jesum gar nicht kennen, ganz so wie seinen Kennern und Verehrern mitgetheilt würden. Vernünftige Wesen nämlich sollen nie eine Wohlthat genießen, ohne den Urheber derselben wenigstens einiger Maßen zu kennen; denn durch die Kenntniß desselben wird der Genuß erquickender und heilsamer für sie. Sie müssen sich eines Glückes, das sie nicht blindem Zufalle, sondern der wohlwollenden Gesinnung eines Andern zu danken haben, lebhafter freuen; sie erhalten Gelegenheit, die Pflicht der Dankbarkeit gegen dieß Wesen zu üben; sie fühlen endlich eine stärkere Verbindlichkeit, ein solches Glück ganz nach dem Willen seines Urhebers anzuwenden.~\RWSeitenw{190}
\item Nehmen wir endlich an, daß das andere Leben ein Leben geselliger Thätigkeit ist, und daß die Seligkeit desselben vornehmlich von der zweckmäßigen Verfassung, welche das Oberhaupt dieser Gesellschaft, der Gottmensch selbst, ihr gegeben, und von der genauen Beobachtung aller seiner Vorschriften abhängt: so ist es ganz begreiflich, daß Niemand in den Himmel der Christen eingehen könne, der Jesum nicht erst als das rechtmäßige Oberhaupt in diesem Stande erkannt hat.
\begin{RWanm} 
Von der anderen Seite gestattet eben diese Annahme die Hoffnung, daß, wenn wir uns dort in Gesellschaft mit anderen Wesen befinden, und auch dort noch an Kenntnissen und sittlicher Vollkommenheit immer zunehmen, früher oder später auch selbst denjenigen Menschen, die Jesum ohne ihr Verschulden auf Erden nicht kennen gelernt haben, dort erst Gelegenheit werde, ihm bekannt zu werden. Machen sie nun von dieser Bekanntschaft mit Jesu ganz den gehörigen Gebrauch: so läßt sich hoffen, daß Gott sie zuletzt auch ebenderselben höheren Seligkeit theilhaftig machen wird, welche für uns bereitet ist, die wir des Glückes, ihn schon auf Erden zu kennen, genießen.
\end{RWanm}
\end{aufzc}
\end{aufza}

\RWpar{234}{Sittlicher Nutzen}
\begin{aufza}
\item Die Lehre, daß die \RWbet{Vergeltung nicht immer schon in diesem Leben, wohl aber in dem zukünftigen eintrete}, gewährt uns folgende sittliche Vortheile:
\begin{aufzb}
\item Der Gute, dem es auf Erden nicht immer nach Wunsche geht, tröstet sich mit der Aussicht in das zukünftige Leben; aus eben diesem Grunde erscheint ihm
\item der Tod nicht mehr so fürchterlich.
\item Der Böse dagegen kann sich durch jenen Wohlstand, dessen er vielleicht auf Erden genießt, zu keiner falschen Sicherheit verleiten lassen. \anf{Es wird nicht immer so dauern!}, ruft ihm das Christenthum in dieser Lehre zu; sprich nicht, ich habe Böses gethan, und was ist mir Schlimmes widerfahren?
\item Wir Alle werden durch diese Lehre verpflichtet, unser Urtheil über den sittlichen Werth oder Unwerth der Menschen zurückzuhalten, und denjenigen, der etwa unglücklich~\RWSeitenw{191}\ ist, nicht gleich für einen geheimen Sünder, oder umgekehrt, zu halten. Befolgen wir diese Pflicht: so üben wir uns von einer Seite in der Bescheidenheit, und von der andern gewinnt unsere Menschenkenntniß, weil wir uns jetzt bemühen müssen, den sittlichen Werth der Menschen (so viel seine Kenntniß uns nothwendig ist) aus andern richtigeren Gründen, als aus seinem bloßen Schicksale, zu beurtheilen.
\end{aufzb}
\item Die Versicherung, daß der verdiente glückliche oder unglückliche Zustand \RWbet{gleich nach dem Tode} anfangen werde, verstärket ungemein die wohlthätigen Wirkungen, welche die Verheißung einer künftigen Vergeltung überhaupt hat. Jede Vorstellung von einer künftigen Belohnung oder Strafe ist um so wirksamer, je näher wir ihre Erfüllung glauben. Wie viel kräftiger war doch das: \RWbet{Heute noch} wirst du mit mir im Paradiese seyn! als wenn der Herr gesprochen hätte: Leide doch nur geduldig, denn einst -- zwar sind Jahrtausende noch dahin -- wirst du zu einer ewigen Seligkeit erwachen. -- Soll aber diese Lehre ihre vollkommene Wirkung hervorbringen: so muß uns wohl zu Gemüthe geführet werden, daß wir \RWbet{keinen Augenblick vor dem Tode sicher} sind. Dann nämlich haben wir Beides, unsere Belohnung sowohl als unsere Strafe, jeden Tag, ja jede Stunde zu erwarten. Könnten wir aber glauben, daß dieser Augenblick des Todes nicht \RWbet{von Gottes weiser und gütiger Vorsehung}, sondern vom Schicksale oder von einem blinden Ohngefähr herbeigeführt werde: dann würde der Gedanke an den Tod aufhören, wohlthätig auf uns zu wirken, und nur Furcht und Bangigkeit erzeugen.
\item \RWbet{Dreifacher Zustand nach dem Tode.}
\begin{aufzb}
\item \RWbet{Himmel.} Je größer die Belohnung ist, die uns versprochen wird, um desto stärker der Aufmunterungsgrund zur Tugend. Und in dieser Hinsicht hätte die göttliche Offenbarung gewiß nicht mehr thun können, als daß sie uns eine Seligkeit verheißt, die ihrer Dauer nach unendlich seyn, und ihrem Grade nach Alles, was wir uns vorstellen können, übertreffen werde. Dadurch aber, daß sie erinnert: nur solche Menschen würden dieser Seligkeit~\RWSeitenw{192}\ theilhaftig werden, die völlig sünd- und schuldlos aus dieser Welt austreten, erreicht sie Zweierlei:
\begin{aufzc}
\item unsere Vorstellung von der Erhabenheit jener himmlischen Freuden wird um so mehr erhöht, je größer die Vollkommenheit geschildert wird, die man besitzen muß, um ihrer fähig und würdig zu seyn;
\item wir lassen uns jetzt um so angelegener seyn, unser Gewissen vor jeder auch selbst der kleinsten Verschuldung zu bewahren, und diejenigen, die wir uns etwa zugezogen haben, sobald als möglich wieder gut zu machen.
\end{aufzc}
\item \RWbet{Reinigungszustand.} Wenn wir das Daseyn nur eines Himmels und einer Hölle glaubten, sonst aber von keinem Mittelzustande wüßten: so würden wir unser ganzes Leben hindurch zwischen zwei Aeußersten schwanken. Bald würden wir glauben, zur Hölle zu gut zu seyn, und deßhalb, ohne uns Mühe zu geben, daß wir noch erst gebessert würden, unsere einstige Aufnahme in den Himmel erwarten; bald würden wir wieder zum Himmel zu schlecht zu seyn fürchten, und nun mit düsterer Verzweiflung dem Zustande einer ewigen Unseligkeit entgegensehen.
\begin{RWanm} 
Schon \RWbet{Kant} hat diese nachtheilige Folge des protestantischen Glaubens gerügt. (S.\ Religion innerhalb der Grenzen der bloßen Vernunft.\ S.\,83.)\RWlit{}{Kant4}
\end{RWanm}
\item \RWbet{Hölle.} Je größer die Strafe ist, die Gott auf eine Sünde gesetzt hat, um desto stärker ist der Abhaltungsgrund von derselben. Offenbar hätte das Christenthum nicht mehr thun können, als wenn es Jedem, der mit einer sittlich bösen Gesinnung aus diesem Leben austritt, die ewige Verdammniß androht. Wünschen wir also in Wahrheit, stets besser zu werden, und eine möglichst starke und eben deßhalb auch die heilsamste Furcht vor allem Bösen in uns zu erzeugen: so müssen wir die Lehre des Christenthums von diesem Zustande einer ewigen Unseligkeit, der jeden Bösen bedroht, mit allem Eifer ergreifen und sie uns anzueignen suchen. Wer dieses unterließe, besonders da er an sich selbst erfährt, daß er nur allzu\RWSeitenw{193}oft noch falle, dem müßte es kein wahrer Ernst um seine sittliche Vervollkommnung seyn.
\end{aufzb}
\item Die Erwartung eines \RWbet{allgemeinen und öffentlichen Gerichtes} gewähret uns manche sittliche Vortheile:
\begin{aufzb}
\item Der Tugendhafte erhält einen Beweggrund mehr, sich wegen so mancher guten That, die hier nicht anerkannt, wohl gar für böse ausgelegt wird, zu trösten; denn es kommt ja ein Tag, an welchem seine gekränkte Ehre gerechtfertiget werden soll.
\item Der Böse dagegen erfährt mit Schrecken, daß er vergeblich sich bemühe, seine Schandthaten den Augen der Welt zu entziehen; an jenem großen Tage wird er entlarvt, und alle seine Schandthaten werden vor allen Menschen, ja vor allen vernünftigen Wesen in Gottes Schöpfung, dargelegt werden. Durch diese Lehre also wird jener Unterschied, welchen wir zwischen heimlichen und offenbaren Verbrechen annehmen, eigentlich aufgehoben oder höchstens nur dahin bestimmt, daß jene nur etwas später als diese offenbar werden.
\item Der Tugendhafte kann sich im Voraus freuen, daß er so manche gute Wirkung seiner Bemühungen, die er auf Erden nicht wahrnehmen konnte, in jener andern Welt erst werde kennen lernen.
\item Der Lasterhafte dagegen hat sich im Voraus darauf zu fürchten, daß er so manche böse verderbliche Folgen, die seine bösen Handlungen, sein böses Beispiel \usw\ nach sich gezogen haben, erst an dem Tage des Gerichtes erfahren und dafür büßen werde.
\item Durch den Umstand, daß \RWbet{alle Menschen zu einerlei Zeit und von demselben Richter} gerichtet werden sollen, wird uns die wesentliche Gleichheit Aller anschaulich gemacht. Wir haben sonach Keiner nur das Geringste voraus vor dem Andern. Mag es doch hier auf Erden so manche größtentheils nur willkürlich eingeführte Unterschiede unter uns geben: dort hören sie alle auf, dort stehen wir Alle vor Einem und ebendemselben Richter, und werden, der Reiche und der Arme, der Vornehme und der Gemeine, der Gelehrte und Ungelehrte nach denselben Grundsätzen gerichtet.~\RWSeitenw{194} 
\item In der Erwartung dieses allgemeinen Gerichtes wächst unser Vertrauen zu Gottes Gerechtigkeit. An jenem Tage will er ja seine Gerichte und Wege allen Menschen, gleichsam zur Prüfung, vorlegen, und Alle, verspricht er, werden gestehen müssen, daß er Alles wohl gethan habe.
\item Der Umstand, daß unser künftiger \RWbet{Richter der Gottmensch} seyn soll, ist
\begin{aufzc}
\item erhebend für uns. Einem aus unserem Geschlechte ist alles Gericht übergeben! -- Um so erhebender, da das Christenthum beisetzt, daß die Guten dieß Richteramt mit Jesu theilen werden.
\item Wie erschütternd dagegen für jeden Lasterhaften, wenn er bedenkt, daß ihm von ebendemselben Herrn Jesu, dessen Gebote er hier so oft übertreten, in Uebereinstimmung mit seinen eigenen Brüdern, vielleicht denjenigen, die er hier keines Blickes gewürdiget hatte, das Urtheil der Verdammniß gesprochen werden solle!
\item Endlich zeigt dieser Umstand auch sinnbildlich die große Güte und Gerechtigkeitsliebe unseres Gottes an, der unser Schicksal auf den Ausspruch eines Richters will ankommen lassen, der einerseits Mitleid mit unseren Schwachheiten zu tragen wissen wird, weil er sie selbst getragen hat und unser Bruder ist; und der doch andererseits mit der größten Unparteilichkeit und ohne sich nur die geringste Beugung des Rechtes zu erlauben, richten wird, da sein Wille mit dem des Vaters ununterbrochen übereinstimmt.
\end{aufzc}
\item Weislich ist uns \RWbet{die Zeit, wann dieser große Tag des Gerichts erscheinen wird, verschwiegen.} Wüßten wir, daß er sehr ferne sey: so würde die wohlthätige Wirksamkeit seiner Vorstellung in eben dem Maße an Kraft verlieren. Glaubten wir aber, er sey sehr nahe, ja schon da: so würde dieser Glaube die größten Unordnungen in den Geschäften des geselligen Lebens nach sich ziehen.
\item Was uns das Christenthum von einer \RWbet{Auferstehung unserer Leiber} sagt, dient zur Bildung einer obgleich nicht ganz bestimmten und deutlichen, doch um so lebhaf\RWSeitenw{195}teren Vorstellung von dem zukünftigen Leben. Je lebhafter aber das Bild ist, das wir von jenem Leben uns entwerfen, um desto lebhafter werden auch die Beweggründe, die wir von den Belohnungen und Strafen desselben hernehmen. -- Die Versicherung, daß wir mit eben denselben Leibern, die wir in diesem Leben getragen, auferstehen sollen, gewähret unserer Einbildungskraft einen verstärkten Beweggrund, die Leiden willig zu ertragen, die wir der Tugend wegen zuweilen auch selbst dem Körper zufügen müssen, alle sündhaften Lüste dagegen uns um so williger zu versagen, weil ja an eben dem Leibe uns soll vergolten werden, was wir jetzt leiden; an eben dem Leibe aber auch gebüßet werden muß, für jede sündhafte Lust, die wir ihm jetzt gestatten. Nothwendig aber muß erinnert werden, daß unser Leib bei seiner Auferstehung zum Leben -- oder zur Seligkeit, von allen den Mängeln und Schwachheiten, von der Gebrechlichkeit, der er in diesem gegenwärtigen Leben unterworfen war, befreit seyn werde. Wie tröstlich ist diese Versicherung nicht für die Siechen und Krüppelhaften, und wie anders als so könnten wir nach unserer Auferstehung ein ewig seliges Leben zu führen hoffen?
\item Was endlich jene \RWbet{bildlichen} Vorstellungen von einem \RWbet{himmlischen Posaunenschalle}, von einer \RWbet{Versammlung an einem und ebendemselben Gerichtsplatze}, von einer \RWbet{Scheidung zur Rechten und Linken} \usw\ betrifft: so dienen diese Bilder (bei denen wir übrigens nie, daß es nur Bilder sind, vergessen dürfen), uns eine möglichst lebhafte Vorstellung vom Ganzen beizubringen. (Sieh das schöne Gedicht: \RWlat{Dies irae, dies illa,} \usw )
\end{aufzb}
\item Wenn uns die \RWbet{Zahl der Seligen} bestimmt angezeigt wäre: so würde das Selig- oder Nichtseligwerden den Anschein einer Nothwendigkeit erhalten. Wäre diese Zahl überdieß groß (und Unwissenden könnte sie, auch wenn sie im Vergleiche mit dem ganzen Menschengeschlechte sehr unbeträchtlich wäre, groß zu seyn scheinen): so würde uns dieß sorglos und leichtsinnig machen. Ein Jeder würde nun glau\RWSeitenw{196}ben, daß er mit unter dieser Zahl begriffen seyn müsse. Ja vielleicht würden wir sogar auf den thörichten Einfall gerathen, uns unter einander zu einer gemeinschaftlichen Lasterhaftigkeit zu bereden; denn da Gott festgesetzt hat, daß er auf jeden Fall so Viele selig machen wolle: so wird er uns (würden wir denken) bei aller unserer Bösartigkeit doch in den Himmel aufnehmen müssen. (Man erinnere sich \zB\ nur an die Denkart der auf ihre Abstammung von Abraham pochenden Juden zu den Zeiten Jesu \RWbibel{Mt}{Matth.}{3}{9}) -- Umgekehrt wieder, wenn diese Zahl sehr klein wäre, in welche bange Zweifel würden wir nicht gerathen! Je älter das Menschengeschlecht würde, je größer die Anzahl der Menschen, die schon gelebt haben, wäre; je mehrere derselben ein offenbar tugendhaftes Leben geführt: um desto weniger Hoffnung bliebe für die noch Uebrigen! Und wer auch für seine eigene Person selig zu werden hoffte: wie schmerzlich müßte doch der Gedanke: so klein nur ist die Zahl der Seligen! -- so ungeheuer jene der ewig Verdammten! -- jedes gute Herz ergreifen! Aber dieß Alles fällt weg, sobald uns hierüber gar nichts Bestimmtes geoffenbaret, sondern nur im Allgemeinen die Weisung ertheilt worden ist, zu unserer eigenen nur desto größeren Aneiferung uns jene Zahl lieber viel kleiner als größer vorzustellen.
\item Die Erinnerung, daß die \RWbet{Belohnungen des Himmels nicht von sinnlicher Art} seyn werden, hat einen mehrmals schon erwähnten Nutzen, daß nämlich unsere Einbildung von dem Werthe der sinnlichen Freuden hiedurch gemäßigt wird, daß wir die Nothwendigkeit erkennen, unseren Geschmack für die höheren Freuden des Geistes je mehr und mehr zu entwickeln, weil wir sonst billig befürchten müßten, im Himmel selbst den Himmel nicht zu finden. Im Gegentheile aber, ob die \RWbet{Leiden der Hölle,} ingleichen jene \RWbet{des Reinigungszustandes} mitunter \RWbet{nicht auch sinnlicher Art} seyn werden, kann immer unentschieden bleiben; zumal da viele Menschen sich keine heftigen Schmerzen und Leiden vorstellen können, als sinnliche.
\item Sollen wir uns die Seligkeiten des andern Lebens nach Würde vorstellen: so müssen wir uns die Vorstellung~\RWSeitenw{197}\ von ihnen machen, daß sie erhabener sind, als wir es uns jetzt nur einbilden können. Die Freuden der Geselligkeit aber gehören schon hier auf Erden unter die reinsten, edelsten und lebhaftesten Freuden, welche wir kennen. Vortrefflich also, daß uns das Christenthum diese Freuden auch in dem andern Leben verspricht. Für denjenigen, der theure Freunde und Anverwandte hat, ist es ein wahres Bedürfniß, die \RWbet{Wiedervereinigung} mit denselben im andern Leben zu hoffen. Alle Seligkeit jenes Lebens würde an Werth für ihn verlieren, er könnte sich nicht recht an ihr freuen, wenn er nicht hoffen dürfte, sie in Gesellschaft dieser Geliebten zu genießen. Hier auf Erden ist es zuweilen sogar unsere Pflicht, uns ohne die Theilnahme gewisser Personen nicht einmal vergnügen zu wollen. So sollen wir, \zB\ der Gatte nicht ohne die Gattin, Eltern nicht ohne ihre Kinder, Geschwister nicht ohne die übrigen wünschen, vergnügt und glücklich zu leben. Das zarte Freundschaftsgefühl, das zur Erfüllung dieser Pflicht nothwendig ist, wie sehr wird es nicht durch die Lehre des Christenthums, von der wir hier sprechen, befördert und unterstützt! Laßt uns schon hier auf Erden, sprechen wir jetzt, Alles gemeinschaftlich haben; denn dort werden wir ewig mit einander vereiniget leben, und an jedem Vergnügen des Einen wird der Andere ungehinderten Antheil nehmen! -- Wie erhebend für uns, daß wir in jenem anderen Leben in die genaueste Verbindung und Freundschaft treten sollen mit allen edlen und großen Männern, die je auf Erden gelebt. Am Allererhebendsten aber, daß wir gewürdiget werden sollen, mit Christo selbst dort in Verbindung zu kommen! Wie muß uns dieß ermuntern, uns schon hier auf Erden mit seinen Grundsätzen recht vertraut zu machen! -- Die Nachricht endlich, daß unsere vornehmste Seligkeit im Himmel im Anschauen Gottes bestehen werde, ist eben so lehrreich als demüthigend für Jeden, der wähnet, schon einen hohen Grad von sittlicher Vollkommenheit erstiegen zu haben, und doch an der Erhebung des Geistes zu Gott noch keinen wahren Geschmack findet! Er lernt nun einsehen, daß ihm noch Vieles fehle, um des Genusses der himmlischen Seligkeit fähig zu werden! -- Wie erhebend aber auch von der andern Seite die Zusicherung, daß wir es einst, so Viele wir ernstlich bestrebt sind,~\RWSeitenw{198}\ immer vollkommener zu werden, Alle dahin bringen sollen, daß wir nur in der Betrachtung der Werke Gottes und seiner Vollkommenheiten unsere höchste Lust und Seligkeit finden werden!
\item Gäbe es keine \RWbet{Stufen der Seligkeit}: so würde derjenige, der einmal glaubt, vollkommen genug zu seyn, um in den Himmel aufgenommen zu werden, keinen ferneren Antrieb, noch immer vollkommener zu werden, fühlen. Da wir aber gewohnt sind, uns Dinge, die man uns unter einerlei Namen bekannt macht, so lange auch als völlig gleich zu denken, so lange man uns nicht ausdrücklich sagt, daß sie verschieden wären: so würden wir, wenn nicht das Gegentheil ausdrücklich angemerkt würde, glauben, Alle, die in den Himmel kommen, hätten da einerlei Schicksal. Einen ähnlichen Nutzen hat auch der Lehrsatz von der Verschiedenheit der Strafen im Reinigungszustande und in der Hölle.
\item So groß die Menge der Sünden eines Menschen auch immer seyn mag, so darf man ihm, wenn er den Vorsatz der Besserung zu fassen Willens ist, die Hoffnung der Vergebung nicht nehmen, oder er wird fortfahren, lasterhaft zu seyn. Andererseits wieder, so lange auch Jemand schon auf dem Pfade der Tugend wandelt; und so viele Fortschritte er auch schon gemacht haben mag: so darf er doch niemals seine Beharrlichkeit auf dieser Bahn für etwas völlig Sicheres halten, weil eine solche Sicherheit im Grunde nichts Anderes, als ein stolzes Vertrauen auf sich selbst seyn würde. In diesem stolzen Vertrauen würde ein Solcher bald träge in dem Gebrauche der Mittel zu seiner weiteren Vervollkommnung, sorglos in der Vermeidung der Gefahren, die seiner Tugend drohen, und würde eben deßhalb, ehe er sich's versieht, in Sünden und Laster verfallen.
\item Alle bisherigen Lehren des Christenthums gingen unmittelbar diejenigen an, welche selbst Anhänger dieses Glaubens sind: was für ein \RWbet{Schicksal der Nichtchristen} gewarte -- ist uns im Grunde zu wissen nicht nöthig. Dennoch sagt uns das Christenthum hierüber so viel, als wir mit Nutzen hören können.
\begin{aufzb}
\item Diejenigen Menschen, welche das Christenthum aus einer nicht unüberwindlichen Unwissenheit verwarfen, ziehen sich~\RWSeitenw{199}\ selbst die härtesten Strafen, ja eine ewige Verdammniß zu. Dieser Lehrsatz flößt
\begin{aufzc}
\item uns, die wir das Christenthum kennen, und bis jetzt gläubig angenommen haben, eine heilige Furcht ein, es zu verlassen;
\item Diejenigen aber, die es bisher verwarfen, die aber doch einige Kenntniß davon erhielten, oder die wenigstens nur diesen einen Lehrsatz desselben kennen lernten, wird dieser Satz antreiben, es genauer zu prüfen.
\end{aufzc}
\item Eine Verwerfung des Christenthumes, welche aus unüberwindlicher Unwissenheit herrührt, wird Niemand zur Schuld und Sünde angerechnet. Auch diese Vernunftwahrheit verdiente ausdrücklich erwähnt zu werden, um das Christenthum von dem Verdachte zu retten, als ob es sie vielleicht verkenne, wie man aus Mißverstand der gleichfolgenden Lehre zu glauben verleitet werden könnte.
\item Diejenigen, welche das Christenthum nicht kennen, und wäre es auch aus unverschuldeter Unwissenheit, können der Vortheile, die Jesus Christus dem menschlichen Geschlechte ausgewirkt hat, in ihrem ganzen Umfange nicht eher theilhaftig werden, als bis sie Christum, als den von Gott gesandten Erlöser, kennen gelernt haben. Diese Lehre, zu Folge welcher die Kenntniß der Person Jesu Christi zu einer unumgänglichen Bedingung der wahren Seligkeit gemacht wird, zeigt uns die Wichtigkeit dieser Kenntniß, und mithin auch die Größe der Gnade Gottes, vermöge der sie uns zu Theil geworden ist, und die Verbindlichkeit, sie dankbar zu benützen.
\item Daß uns endlich darüber, ob die Nichtchristen die ohne ihr Verschulden auf Erden nicht gefundene Gelegenheit, Jesum kennen zu lernen, nicht im andern Leben erhalten, keine ganz bestimmte Auskunft ertheilt sey, auch das hat seinen Nutzen. Wenn diese Frage, wie es sehr wahrscheinlich ist, eine bejahende Antwort erhalten müßte: so läßt sich leicht erachten, daß diese Antwort von tausend und tausend Menschen mißverstanden und mißbrauchet worden wäre. Sie würden nämlich aus ihr den Schluß ziehen, daß die Erkenntniß des Christenthums also kein~\RWSeitenw{200}\ so unschätzbar wichtiges Gut für Zeit und Ewigkeit sey; da es, versäumt in diesem Leben, noch in dem andern nachgeholt werden kann. Und wie falsch wäre dieß gleichwohl geschlossen! Wenn gleich derjenige, der das Christenthum ohne sein Verschulden auf Erden nicht kennen gelernt hat, dort noch Gelegenheit findet, zu dieser Kenntniß zu gelangen: so sind doch diejenigen immer unendlich besser daran, die es schon hier auf Erden kennen gelernt haben, die eben dadurch schon hier größere Fortschritte in der Vollkommenheit gemacht, und in jenem anderen Leben vielleicht die Lehrer der Uebrigen werden können. Auch ist kein Zweifel, daß jener große Eifer, mit dem katholische Christen durch alle Jahrhunderte hindurch für die Verbreitung des christlichen Glaubens gearbeitet haben (ein Eifer, von dem uns die Geschichte der Missionen so manches rührende Beispiel erzählt), niemals entstanden wäre, und so lange angehalten hätte. Und so bewährt sich denn auch hier, daß Gottes Offenbarung eben so weise verfahre in dem, was sie verschweigt, wie in demjenigen, was sie uns mittheilt.~\RWSeitenw{201}
\end{aufzb}
\end{aufza}

\clearpage%
\RWch{Drittes Hauptstück.\\
Christkatholische Moral.}
\RWpar{235}{Inhalt und Abtheilungen dieses Hauptstückes}
\begin{aufza}
\item Das katholische Christenthum nimmt alle diejenigen Pflichten, die schon die bloße Vernunft erkennt, gleichfalls in seinen Inhalt auf; nebst diesen aber schreibt es seinen Bekennern noch manche andere Verhaltungsregeln vor, deren sittliche Zuträglichkeit wir zwar schon durch die bloße Vernunft einsehen können und müssen, bevor wir sie als göttlich geoffenbart ansehen können, die aber gleichwohl nicht bestimmte Pflichten für uns wären, wenn sie die Kirche nicht dafür erklärte (1.~Hptthl. 4.~Hptst.). Da es die Kürze der Zeit auf keine Weise erlaubt, die einzelnen Pflichten, wie sie auch nur die bloße Vernunft in den verschiedenen Verhältnissen des Lebens anerkennt, zu entwickeln: so werden wir uns bloß mit der Angabe der allgemeinsten Lehren, und dann derjenigen, die dem katholischen Christenthume am Eigenthümlichsten sind, begnügen.
\item Hiebei befolgen wir die bereits aus der natürlichen Religion (1.~Hptthl. 2.~Hptst.) bekannte Abtheilung in die katholische \RWbet{Ethik} (oder Sittenlehre) und in die katholische \RWbet{Asketik} (oder Tugendmittellehre).
\end{aufza}

\RWabs{Erste Abtheilung}{Christkatholische Ethik}
\RWpar{236}{Die Lehre des Christenthums vom Daseyn eines Sittengesetzes überhaupt}
Das katholische Christenthum fängt seinen Unterricht in der Sittenlehre mit der Behauptung an, \RWbet{daß es ein Sollen}~\RWSeitenw{202}\ \RWbet{überhaupt gebe}; oder mit andern Worten, daß der bei allen, auch selbst den rohesten Menschen, sobald sie nur erst zu einigem Gebrauche ihrer Vernunft gelangt sind, erwachende Begriff von einem gewissen Sollen, von einer gewissen Verbindlichkeit oder Pflicht kein \RWbet{leerer}, \dh\  kein solcher Begriff sey, welchem kein wirklicher Gegenstand entspräche; sondern daß es in Wahrheit gewisse Handlungen, welche wir sollen, und andere, die wir nicht sollen, gebe. Die Gelehrten nennen den Inbegriff derjenigen Wahrheiten, welche ein Sollen aussagen, das \RWbet{Sittengesetz}; und somit kann man die katholische Lehre, die wir hier anführen, auch wohl die Lehre vom Daseyn eines Sittengesetzes nennen.

\RWpar{237}{Historischer Beweis dieser Lehre}
Daß die katholische Kirche wirklich so lehre, wie eben gesagt worden ist, beweisen alle Predigten, alle Gebetbücher, alle Lehrbücher der Moral, \usw\ Daß auch die Schrift so lehre, braucht nicht erst durch die Anführung einzelner Stellen derselben dargethan zu werden. Wenn aber Jemand ja daran zweifeln wollte, ob die heilige Schrift mit den Worten: Pflicht, Sollen, Verbindlichkeit, \udgl\  auch eben denselben Begriff verbinde: so könnten wir ihn \zB\  gleich durch folgende Stellen überführen. \RWbibel{Röm}{Röm.}{2}{12}\ schreibt der heil.\ Paulus: \erganf{Wer ohne das (mosaische) Gesetz zu kennen, gesündiget hat, wird seine Strafe auch nicht nach dem (mosaischen) Gesetze empfangen. Nur wer dasselbe kennt, und gleichwohl dawider gesündiget hat, wird nach demselben gerichtet werden; denn nicht Diejenigen, die das Gesetz bloß kennen, sind gerecht vor Gott, sondern nur Jene, die es auch befolgen. Denn wenn die Heiden, die das mosaische Gesetz nicht hatten, doch, durch das natürliche Gefühl (\RWgriech{f'usei}) getrieben, thaten, was das Gesetz befiehlt: so waren sie sich selbst ein Gesetz, und ihr Beispiel beweiset uns, daß die wichtigsten Pflichten, die das mosaische Gesetz enthält, in unsere Herzen geschrieben sind, und durch das Zeugniß des Gewissens selbst deutlich genug ausgesprochen werden, so zwar, daß unser eigenes Bewußtseyn den Handlungen, welche wir ausüben, bald seinen Beifall gibt, bald sie verdammt,} \usw~\RWSeitenw{203}

\RWpar{238}{Vernunftmäßigkeit, sittlicher und wirklicher Nutzen}
Daß diese Lehre auch von der bloßen Vernunft nothwendig angenommen werde, ist schon im ersten Haupttheile gezeigt worden. Ihr sittlicher Nutzen aber ist von der größten Wichtigkeit; denn wer das Daseyn eines Sittengesetzes läugnen oder auch nur bezweifeln würde, der würde eben darum auch alle Tugend bei sich selbst aufheben. Dennoch hat es in älterer sowohl als neuerer Zeit mehrere Weltweisen gegeben, welche das Daseyn eines Sittengesetzes, und bestimmte, nicht in der Willkür des Menschen, sondern in der Natur gegründete Pflichten bald mit ausdrücklichen Worten bestritten, bald zwar dem Worte nach zugestanden, aber der Sache nach doch verworfen. Zu diesen Letzteren gehören nämlich alle sogenannten \RWbet{Epikuräer} und \RWbet{Eudämonisten}. Da sie den Inhalt des Sittengesetzes in die Beförderung der eigenen Glückseligkeit setzten: so hoben sie eben darum allen Unterschied zwischen dem Sollen und dem Wünschen auf; denn wovon der Mensch eingesehen hat, daß es seine eigene Glückseligkeit befördert, das wünscht er auch schon und muß es wünschen. Eben deßhalb konnte bei dieser Lehre auch keine wahre Freiheit bestehen, das Sollen war also nicht nur mit dem Wünschen, sondern auch mit dem Müssen ein und derselbe Begriff; und die ganze Tugend verwandelte sich in eine bloße, dem Menschen nothwendig anklebende Selbstsucht, die sich nur dadurch von jener des Lasters unterschied, daß sie vielleicht glücklicher in der Wahl der Mittel, und somit klüger ist. Dieser verderblichen Lehre nun widersetzte sich das katholische Christenthum mit vielem Nachdrucke, wie dieß \zB\  noch erst in neuerer Zeit (bei Gelegenheit der Winherlischen Streitigkeiten) zu sehen war.

\RWpar{239}{Die Lehre des Christenthums von dem Gebiete des Sittengesetzes}
\begin{aufza}
\item Das Sittengesetz hat, nach der Darstellung des katholischen Christenthums, nicht nur ein Daseyn, sondern auch ein sehr ausgebreitetes, ja \RWbet{das umfassendste Gebiet}, das irgend ein Gesetz hat.~\RWSeitenw{204}
\item Es erstreckt sich nämlich, was das \RWbet{Subject} desselben anlangt, \RWbet{durchaus auf alle Wesen, welche Vernunft und eine gewisse Kraft, nach den Aussprüchen dieser Vernunft etwas zu thun, besitzen.}
\item Namentlich also zuerst schon auf \RWbet{Gott} selbst, als auf dasjenige Wesen, das die vollkommenste Vernunft mit der uneingeschränktesten Macht vereiniget; ferner auf \RWbet{alle Engel} und \RWbet{höhere Geister,} endlich auch auf \RWbet{alle Menschen ohne alle Berücksichtigung, weß Standes oder Ranges sie auch immer seyn mögen.}
\item Bei jedem dieser vernünftigen Wesen erstreckt sich ferner das Sittengesetz, was das \RWbet{Object} desselben anbelangt, \RWbet{genau so weit, als seine Einsichten und seine Macht sich erstrecken,} oder mit andern Worten: Jede Wirkung eines Wesens, die von der Wollkraft desselben unmittelbarer oder mittelbarer Weise abhängt, unterliegt dem Sittengesetze.
\item Was nun insonderheit den Menschen selbst anlangt, so bestimmt das katholische Christenthum, die unter dem Sittengesetze begriffenen Handlungen desselben noch etwas genauer, und zählet hieher
\begin{aufzb}
\item allerlei \RWbet{Veränderungen, welche wir an uns selbst hervorbringen} können, als
\begin{aufzc}
\item \RWbet{an unserem Geiste}, indem wir die Kräfte desselben entweder auszubilden, oder zu verwahrlosen vermögen; unseren Verstand mit nützlichen Kenntnissen entweder bereichern, oder die Entstehung mancher Einsichten auch wohl gar hintertreiben, und uns mit Wissen und Willen selbst täuschen können; indem wir in \RWbet{unserem Empfindungsvermögen} allerlei angenehme oder auch unangenehme Gefühle und Empfindungen hervorbringen, in \RWbet{unserem Begehrungsvermögen} verschiedene Wünsche und Neigungen, Affecte und Leidenschaften erzeugen, nähren, vergrößern, oder umgekehrt die schon entstandenen unterdrücken oder allmählich ausrotten können; \usw\
\item \RWbet{an unserem Leibe}; indem wir das Wachsthum, die Gesundheit und Vollkommenheit desselben befördern oder~\RWSeitenw{205}\ auch stören; unsere Lebensdauer verlängern oder auch abkürzen können, \usw\
\end{aufzc}
\item Allerlei \RWbet{Veränderungen, die wir in andern endlichen Wesen hervorbringen} können, nämlich in Menschen, in Thieren, in der leblosen Welt, \usw\
\end{aufzb}
\end{aufza}

\RWpar{240}{Historischer Beweis dieser Lehre}
Nur einige der hier angeführten Behauptungen bedürfen es, daß wir sie eigens nachweisen; denn von den übrigen kann Niemand zweifeln, daß sie Lehren der Kirche sind.
\begin{aufza}
\item[{\RWlat{ad} 3.}] Daß sich das Sittengesetz auch auf \RWbet{Gott} erstrecke, beweiset diejenige Lehre des katholischen Christenthums, die Gott Heiligkeit beilegt.
\item[{\RWlat{ad} 5. a, \RWgriech{a}.}] Daß auch das \RWbet{Glauben} oder \RWbet{Nichtglauben} unter gewissen Umständen dem Sittengesetze unterliege, beweist \zB\  die Stelle \RWbibel{Joh}{Joh.}{3}{19}: \erganf{Wer an ihn glaubt (an den Sohn Gottes), wird nicht gerichtet (verurtheilt) werden; wer aber nicht glaubt (nämlich da er doch alle Gelegenheit hätte, sich zu überzeugen), der ist schon gerichtet. Das aber ist die Ursache seines Gerichtes, weil, als das Licht in die Welt kam, die Menschen die Finsterniß mehr als das Licht liebten, weil ihre Werke böse waren.} (Woraus zu ersehen ist, daß nur eine freiwillige, aus Bosheit \udgl\  entsprungene Verkennung der Wahrheit für sündhaft erklärt werde.) \RWbibel{Joh}{Joh.}{9}{41}: \erganf{Wäret ihr Blinde, so hättet ihr keine Sünde. Nun aber sagt ihr selbst, daß ihr sehet, und eben darum bleibt eure Sünde.}
\end{aufza}

\RWpar{241}{Vernunftmäßigkeit}
Die Wahrheit aller dieser Lehren läßt sich auch durch die bloße Vernunft erweisen.
\begin{aufza}
\item Allerdings ist \RWbet{das Gebiet des Sittengesetzes sehr ausgebreitet,} indem es sich
\item \RWbet{auf jedes Wesen erstreckt, welches Vernunft und eine gewisse Kraft hat, nach den Aussprüchen}~\RWSeitenw{206}\ \RWbet{dieser Vernunft wirksam zu werden}. Wenn das Wesen Vernunft, \dh\  die Fähigkeit hat, gewisse allgemeine, oder, wie man sagt, apriorische Wahrheiten (\dh\ Wahrheiten, die aus bloßen Begriffen bestehen) zu erkennen: so muß es unter Andern gewiß auch die leicht zu erkennenden Wahrheiten des Sittengesetzes, \zB\  das oberste, daß man das allgemeine Wohl befördern solle, oder sonst einige aus diesem abgeleitete Gesetze, \zB\  daß man nicht lügen, rauben \udgl\  dürfe, mit mehr oder weniger Deutlichkeit anerkennen. Sollte es sich das Gegentheil vorstellen können; sollte es \zB\  glauben, daß man das allgemeine Wohl statt zu befördern, stören solle: so würden wir dasjenige, wodurch es geschieht, daß sich das Wesen dergleichen Dinge vorstellt, nicht mit dem Namen der Vernunft, sondern mit jenem der Verrücktheit belegen. -- Hat nun dieß Wesen ferner auch eine gewisse Kraft, nach den Aussprüchen seiner Vernunft thätig zu werden: so wird es die Aussprüche, daß man das allgemeine Wohl befördern, daß man nicht lügen, nicht stehlen \udgl\  solle, nicht bloß an Andere, sondern auch an sich selbst thun müssen, \dh\  es wird auch von sich selbst urtheilen, daß es unter dem Sittengesetze stehe, und darum steht es denn wirklich unter demselben.
\item Kein Zweifel also, daß sich das Sittengesetz auch \RWbet{auf Gott selbst} erstrecke. Wenn dieses gleichwohl von einigen Gelehrten bezweifelt und bestritten worden ist: so geschah es aus einem bloßen Mißverstande gewisser allerdings unschicklicher Redensarten, wie der: Gott unterstehe dem Sittengesetze, oder: er habe Pflichten oder Verbindlichkeiten, er sey genöthiget, dem Sittengesetze zu gehorchen, \udgl\  Durch solche Redensarten nämlich scheint man für's Erste vorauszusetzen, daß es noch etwas, das über Gott ist, gebe; woraus denn folgen würde, daß er nicht völlig unabhängig sey. Allein bei einer näheren Betrachtung verschwindet diese Schwierigkeit, indem diejenige Unabhängigkeit Gottes, die wir allein behaupten und beweisen können, nur in der Unabhängigkeit von jedem andern Wesen, nicht aber darin bestehet, daß er die apriorischen, \dh\ die reinen Begriffs-Wahrheiten nach seinem bloßen Belieben so oder anders einrichten könne; denn diese haben ja nicht nur selbst keine Existenz, sondern~\RWSeitenw{207}\ sind auch nicht einmal in existirenden Wesen (oder Substanzen) gegründet, sondern sie gründen sich entweder in andern dergleichen Wahrheiten (wenn sie Folgewahrheiten sind), oder sie haben gar keinen weiteren Grund (wenn es Grundwahrheiten sind). Ein Anderes ist es mit der Erkenntniß dieser Wahrheiten, die allerdings etwas Wirkliches, aber auch von Gott Abhängiges ist. Gott kann nach seinem Wohlgefallen machen, daß ein gewisses Wesen eine dergleichen Wahrheit erkenne, oder sie nicht erkenne; er könnte \zB\ machen, daß Jemand sich vorstellt, der Raum habe vier Dimensionen (in welchem Falle nämlich dieß Wesen verrückt seyn müßte); allein Gott kann nicht machen, daß es in der That wahr sey, daß der Raum vier Dimensionen habe. Da nun das Sittengesetz, wenigstens das oberste, eine reine Begriffswahrheit ist: so hängt es nicht von Gottes Belieben ab, daß es so laute, wie es lautet. Man kann also immerhin sagen, daß Gott das Sittengesetz mit Nothwendigkeit erkenne; wie auch, daß er mit Nothwendigkeit demselben gemäß handle; nur muß man von diesem Ausdrucke jeden Nebenbegriff einer Beschwerlichkeit, eines ihm unangenehmen Zwanges entfernen. Aber eben der Nebenbegriff von einem solchen Zwange ist der zweite Fehler, welchen die oben angeführten Redensarten haben. Bei dem Ausdrucke: Gott \RWbet{untersteht} oder \RWbet{unterliegt} dem Sittengesetze, und noch mehr bei dem Ausdrucke: Gott habe \RWbet{Pflichten} oder \RWbet{Verbindlichkeiten}, oder er sey \RWbet{genöthiget}, dem Sittengesetze zu gehorchen, denkt man an eine gewisse Beschwerlichkeit, die es Gott koste, das zu wollen, was dem Sittengesetze gemäß ist; und darum hat man sich an diese Ausdrücke gestoßen, und sie sind auch mit Recht als ganz unschicklich zu verwerfen.
\item Nicht minder gewiß ist es, daß sich das Sittengesetz bei jedem vernünftigen Wesen eigentlich nur auf dasjenige erstrecke, \RWbet{was durch den Willen und durch die nach Außen wirkende Kraft} desselben zu Stande gebracht werden kann, aber \RWbet{auf dieses auch ohne Ausnahme.} Wenn das oberste Sittengesetz in seiner einfachsten Gestalt (in der es ein echter Grundsatz ist und aus drei einfachen Begriffen besteht) ausgesprochen wird: so fordert es eigentlich nur das Wollen, nicht das Vollbringen: das Gewünschte~\RWSeitenw{208}\ (das allgemein, nicht bloß von dir allein Gewünschte) soll gewollt werden.
\begin{RWanm} 
Die Gelehrten drücken Nr.\,2 und 4 mit den Worten aus: das \RWbet{Subject} des Sittengesetzes sind alle vernünftigen Wesen, das \RWbet{Object} desselben sind alle Handlungen, die durch den Willen dieser Wesen zu Stande gebracht werden können. -- Der Unterschied, den sie hier zwischen den Worten \RWbet{Subject} und \RWbet{Object} machen, besteht bloß darin, daß sich Subject auf einen lebendigen, Object auf einen leblosen Gegenstand beziehet; oder besser: \RWlat{Subjectum alicujus propositionis id est, cui aliquid dicitur; objectum id, de quo dicitur.} -- Wirkungen, welche der Mensch ohne seinen Willen hervorbringt, \zB\  der Druck, den er durch die Schwere seines Körpers auf den Boden äußert, nennen die Gelehrten \RWlat{actiones hominis}; dagegen die Wirkungen, die er durch seinen Willen hervorbringt, \RWlat{actus humanos} (Handlungen).
\end{RWanm}
\item Daß der Mensch wirklich auf alle dort genannten Gegenstände durch seinen Willen einen gewissen Einfluß habe, ist vor dem Urtheile des gemeinen Menschenverstandes gar keinem Zweifel unterworfen. Sollten wir aber erklären, wie dieses eigentlich geschehe: so müßten wir etwa Folgendes sagen:
\begin{aufzb}
\item Auf eben die Art, wie der Mensch die Bemerkung macht, daß dieser oder jener äußere Gegenstand (\zB\  diese Blume) eine gewisse Wirkung (\zB\  die Empfindung eines Geruches) hervorbringe, und ihm deßhalb die Kraft, dergleichen Wirkungen hervorzubringen, beilegt, auf eben die Art macht er auch an sich selbst die Bemerkung, daß er durch seinen Willen im Stande sey, diese und jene Wirkungen (Veränderungen in der Sinnenwelt) unter bestimmten Umständen hervorzubringen, \zB\  seine Hand aufzuheben \udgl\  Treten nun Umstände ein, die jenen ähnlich sind, unter welchen er bereits mehrmal dieses oder jenes gethan: so wird er sich, nach dem bekannten Gesetze der Verknüpfung seiner Vorstellungen, an diese Handlungsweisen erinnern, und mit Wahrscheinlichkeit (nämlich nach dem Schlusse der Analogie) vermuthen, daß ihm die Hervorbringung einer dergleichen Veränderung in der Außenwelt wohl auch jetzt möglich seyn dürfte. Sogleich beurtheilt er nun Beides, sowohl die Sittlich\RWSeitenw{209}keit (\dh\  die Schicklichkeit dieser Handlung für die Beförderung des allgemeinen Wohles), als auch die Vortheile, die diese Handlung für ihn selbst haben würde. Trifft es sich nun, daß ihm die eine dieser Handlungen als sittlich gut (\dh\  als die zuträglichste für das gemeine Wohl), die andere als die zuträglichste für ihn selbst erscheint: so macht seine Vernunft, in Ansehung der ersteren Handlung, an ihn die Forderung: du sollst sie wollen; die letztere hingegen fängt sein Glückseligkeitstrieb an zu wünschen. Und nun ist es ihm absolut möglich, entweder die eine oder die andere zu wollen.
\item Entschließt er sich wirklich für die eine oder die andere, und fängt er an, sie zu wollen: so hat sein Wille auf eine nicht weiter zu erklärende Weise die Kraft, gewisse Veränderungen auch in der Maschine seines Körpers hervorzubringen, wodurch dann mittelbar gewisse Veränderungen in anderen Gegenständen, obgleich nicht immer genau dieselben, welche der Mensch beabsichtiget hatte, erfolgen. Vornehmlich äußert sich die Wirksamkeit unseres Willens:
\begin{aufzc}
\item in der Festhaltung, Belebung und weiteren Ausbildung gewisser, in unserem Bewußtseyn erwachter Vorstellungen, \dh\  wir haben die Kraft, wenn wir wollen, eine in uns so eben erwachte Vorstellung festzuhalten, sie zu beleben und auszubilden, oder auch umgekehrt, die Aufmerksamkeit unseres Geistes von ihr abzuziehen, sie wieder fahren zu lassen, \udgl\ 
\item Dann können wir bekanntlich auch allerlei Bewegungen mit den Gliedmaßen unseres Körpers hervorbringen;
\item hiedurch abermals gewisse Veränderungen in den leblosen sowohl als belebten Gegenständen, die uns umgeben, bewirken;
\item ja endlich auch Veränderungen, die auf uns selbst wieder zurückwirken, \zB\  ein Buch aufschlagen, darin lesen, \udgl\ 
\item Hiedurch geschieht es denn, daß wir insonderheit auch die Erkenntniß der Wahrheit bei uns beschleunigen oder verzögern, Begierden, Neigungen und Gewohn\RWSeitenw{210}heiten bewirken oder verhindern, und überhaupt alle diejenigen Wirkungen, die oben aufgezählt wurden, auf eine mehr oder weniger mittelbare Weise, hervorbringen können.
\end{aufzc}
\end{aufzb}
\end{aufza}

\RWpar{242}{Sittlicher Nutzen}
\begin{aufza}
\item Es dient offenbar sehr zur Vermehrung unserer Achtung gegen das Sittengesetz, wenn uns das Christenthum auf das weite und unermeßliche Gebiet desselben aufmerksam macht.
\item Könnten wir glauben, daß es gewisse vernünftige Wesenarten gebe, für welche das Sittengesetz keine Verbindlichkeit hat: so müßten wir eben darum glauben, daß sich entweder diese Wesen oder wir selbst irren, indem wir Dieses oder Jenes, \zB\  die Beförderung des allgemeinen Wohles, als eine Pflicht ansehen.
\begin{aufzb}
\item Würden wir glauben, daß sich der Irrthum nicht auf der unsrigen, sondern auf Seite derjenigen Wesen befinde, die diese Gesetze verwerfen, so würde uns dieses
\begin{aufzc}
\item zu gewissen Zeiten als ein wichtiger Einwurf gegen die Heiligkeit Gottes erscheinen, weil er dadurch, daß er vernünftige Wesen in einem so wichtigen Stücke im Irrthume läßt, es ihnen unmöglich macht, das allgemeine Wohl gehörig zu befördern.
\item Zu anderer Zeit wieder, wo wir uns etwa zur Uebertretung des Sittengesetzes mächtig versucht fühlen, würden wir jene Wesen, die, weil sie aus Irrthum dasselbe nicht anerkennen, auch nicht daran gebunden sind, beneiden, und würden versuchen, es ihnen gleich zu thun.
\end{aufzc}
\item Noch schlimmer wäre es, wenn wir annehmen würden, daß sich der Irrthum auf unserer Seite befinde, \dh\  daß das Gesetz von der Beförderung des allgemeinen Wohles, oder von der Wahrhaftigkeit, oder von der Dankbarkeit und andere dergleichen sittliche Gesetze nicht in der Wirklichkeit bestehen, sondern nur eine bloß uns Menschen eigentliche subjective Ansicht, aber objective Irrthümer sind. Dann würden wir nämlich gar keine Lust mehr fühlen, ja in der That nicht einmal Verbindlichkeit mehr haben, diesen Gesetzen zu gehorchen.~\RWSeitenw{211}
\item In jedem Falle endlich müßte uns doch der Gedanke, daß diese Gesetze nicht von allen vernünftigen Wesen gleichförmig anerkannt werden, ihre Gewißheit verdunkeln.
\end{aufzb}
\item Daß sich das Sittengesetz, namentlich \RWbet{auch auf Gott selbst} erstrecke, muß zum Verständnisse der Heiligkeit Gottes vorausgesetzt werden, und erst die Heiligkeit Gottes ist es bekanntlich, die uns den Glauben an ihn so wohlthätig macht. Zweckmäßig war es auch, zu erinnern, daß sich das Sittengesetz \RWbet{auf alle Menschen ohne Ausnahme} erstrecke; damit nicht Einige derselben, \zB\  Mächtige, wähnten, über dasselbe erhaben zu seyn. -- Daß auch noch eine zahllose Menge von \RWbet{Engeln und höheren Geistern} in Gottes Schöpfung lebe, die alle demselben Gesetze, wie wir, gehorchen, macht uns bereitwilliger, demselben zu gehorchen; muntert uns auf, wenn wir schon müde werden wollen; denn wir erfüllen ja eine Vorschrift immer um desto lieber, je größer die Anzahl derjenigen ist, die sich zugleich mit uns nach ihr richten.
\item Noch nothwendiger wird die Erinnerung, \RWbet{daß keine einzige mit Wissen und Willen von uns unternommene Handlung vom Sittengesetze ausgenommen sey;} denn es ist nur zu bekannt, wie gerne wir Menschen uns überreden wollen, daß diese und jene Handlung kein Gegenstand des Sittengesetzes sey. Insbesondere sey es höchst nothwendig, uns auf alle diejenigen Veränderungen aufmerksam zu machen, auf die wir einen nur entfernten oder mittelbaren Einfluß haben, \zB\  auf unsere Begriffe und Meinungen, auf unsere Wünsche und Triebe, auf unsere Lebensdauer, auf unsere Träume, \udgl\ ; denn diesen Einfluß pflegt man sehr oft zu übersehen, und zählt sich eben deßhalb von verschiedenen Pflichten, die man doch wirklich hat, los.
\end{aufza}

\RWpar{243}{Wirklicher Nutzen}
Der Nutzen, den das Christenthum durch diese Lehre der Wirklichkeit nach gestiftet hat, ist überaus groß. Wie häufig hat man \zB\  außerhalb des Christenthums die Wahrheit Nr.\,3.\ verkannt! Wie oft hat man nicht geglaubt, daß~\RWSeitenw{212}\ nur wir Menschen, und nicht einmal wir Alle verbunden wären, dem Sittengesetze zu gehorchen! Hat man nicht selbst in christlichen Ländern hie und da Stimmen vernommen, daß Fürsten nicht an die Gesetze der Moral gebunden seyen? -- Was wäre erst geschehen, wenn das Christenthum nicht das Gegentheil so ausdrücklich geprediget hätte!

\RWpar{244}{Die Lehre des katholischen Christenthums von dem Unterschiede zwischen Geboten und Räthen}
Nachdem das katholische Christenthum erklärt hat, daß sich das Sittengesetz oder der Begriff des Sollens auch auf uns Menschen erstrecke, verweiset es uns auf eine \RWbet{beträchtliche Anzahl von Regeln,} denen es nachrühmt, daß die Beobachtung derselben zur Führung des tugendhaften, Gott wirklich wohlgefälligen und uns endlich seligmachenden Wandels ersprießlich seyn dürfte. Inzwischen erinnert es selbst, daß die Art, wie diese Regeln von uns beobachtet werden sollen, nicht bei allen die nämliche sey, sondern es will, daß wir in dieser Hinsicht \RWbet{zwei Gattungen} derselben unterscheiden.
\begin{aufza}
\item Die eine Gattung von Regeln, welche uns unter dem Namen von \RWbet{Geboten} oder \RWbet{Gesetzen} vorgetragen werden, umfasset Handlungen, die wir als \RWbet{eigentliche Pflichten} in des Wortes strengster Bedeutung anzusehen haben, \dh\  Handlungen, durch deren Unterlassung wir uns straffällig machen würden.
\item Die andere Gattung von Regeln, die man uns unter dem Namen der bloßen \RWbet{Räthe} vorlegt, bezieht sich auf Handlungen, die nicht als eigentliche Pflichten, sondern als bloß \RWbet{verdienstlich} anzusehen sind, \dh\  auf Handlungen, durch deren Unterlassung wir uns noch keine Strafe zuziehen, sondern nur einer Belohnung berauben.
\end{aufza}

\RWpar{245}{Historischer Beweis dieser Lehre}
Einen gewissen Unterschied zwischen den Regeln, welche zur Führung  eines Gott wohlgefälligen und wahrhaft selig~\RWSeitenw{213}\ machenden Lebens ersprießlich wären, machte schon unser Herr selbst \RWbibel{Mt}{Matth.}{19}{16\,ff}, wo er einem Jünglinge, der ihm die Frage vorgelegt hatte, was er thun müsse, um das ewige Leben zu gewinnen, zuerst die Antwort gab: \erganf{Halte die \RWbet{Gebote}.} Als nun der Jüngling die weitere Frage erhob, welche Gebote hier gemeint seyen, nannte ihm Jesus jene bekannten: Du sollst nicht tödten, du sollst nicht ehebrechen, \udgl , du sollst deinen Nächsten lieben, wie dich selbst. Nachdem der Jüngling erwiederte, dieß Alles habe er von seiner Kindheit an befolgt, entgegnete ihm der Herr: \erganf{Wenn du vollkommen (\RWgriech{t'eleios}) werden willst: so verkaufe deine Güter, gib das gelösete Geld den Armen, und werde Einer von meinen Nachfolgern.} -- Diese letztere Weisung unterschied also der Heiland hier ausdrücklich von den Geboten. Es gab sonach Regeln in Jesu Sittenlehre, die nicht Gebote waren. Und wenn man, wie aus der gegenwärtigen Stelle hervorgeht, das ewige Leben gewinnen konnte, sobald man nur die sogenannten Gebote alle beobachtet hatte: so leuchtet ein, daß man durch die Nichtbefolgung jener andern Regeln zum Wenigsten nicht straffällig werde. Die katholische Kirche gab nun dergleichen Regeln den Namen der \RWbet{Räthe}, welche auch schon der heil.\ Paulus in diesem Sinne gebrauchte. Denn \RWbibel{1\,Kor}{1\,Kor.}{7}{6}\ heißt es: \erganf{Dieß sage ich euch als einen \RWbet{Rath} (\RWgriech{kat`a suggn'wmhn}), nicht aber als einen \RWbet{Befehl} (\RWgriech{o>u kat' >epitag'hn}).} Und \RWbibel[V.\,25]{1\,Kor}{}{7}{25}: \erganf{In Betreff der Jungfrauen habe ich keinen \RWbet{Befehl} des Herrn (\RWgriech{>epitag`hn Kur'iou}); einen \RWbet{Rath} aber (\RWgriech{gn'wmhn}) gebe ich euch.} Von diesen Räthen sagt Paulus nun mit ausdrücklichen Worten: daß man nicht sündige, wenn man sie nicht befolgt (\RWgriech[o>uq <'hmarths]{o>uq <'hmartes}). Das soll gewiß keinen andern Sinn haben, als daß man bei ihrer Nichtbefolgung keine Strafe zu befürchten habe. In des heil.\ Thomas \RWlat{Summa (prima secundae qu.\,108. art.\,4.)\RWlit{}{ThomasAquinas3} heißt es: Haec est differentia inter consilium et praeceptum, quod praeceptum importat necessitatem, consilium autem in optione ponitur ejus, cui datur.}
Unter jene Nothwendigkeit \RWlat{(necessitas)}, die ein Gebot erzeuge, kann offenbar keine eigentliche (physische) Nothwendigkeit, und unter der freien Wahl (\RWlat{optio}), die ein Rath übrig läßt, keine blinde Willkür verstanden~\RWSeitenw{214}\ werden; sondern der Sinn ist nur, daß man einem Gebote nothwendig folgen müsse, wenn man in keine Strafe verfallen will, von einem Rathe aber auch abgehen dürfe, ohne straffällig zu werden.

\RWpar{246}{Vernunftmäßigkeit}
Daß es zwischen den Regeln, welche zur Führung eines Gott wohlgefälligen und uns selig machenden Wandels mit Nutzen beachtet werden können, einen solchen Unterschied gebe, wie ihn die katholische Kirche zwischen den Geboten einerseits und den Räthen andererseits annimmt, läßt sich recht wohl begreifen. Da es, wie wir schon oben\editorischeanmerkung{RW IIIb 137--139.} gesehen, zwischen den sittlich guten Handlungen, die wir verrichten, einen Unterschied von der Beschaffenheit gibt, daß einige derselben bloß \RWbet{strenge Pflichtenerfüllungen}, andere dagegen \RWbet{verdienstlich} genannt werden dürfen: so muß es auch zwischen den Regeln, welche das sittlich gute Verhalten beschreiben, einen solchen Unterschied geben, daß einige sich nur auf Handlungen der ersten, andere auf Handlungen der zweiten Art beziehen.
\begin{aufza}
\item Regeln, deren Befolgung schlechterdings nothwendig ist, wofern das Wohl des Ganzen nicht sehr empfindlich verletzt werden soll, werden ohne Zweifel zu den \RWbet{Geboten} gehören.
\item Regeln, durch deren Uebertretung das Wohl des Ganzen nicht eben beeinträchtiget, sondern nur nicht so befördert wird, als es durch ihre Beobachtung befördert worden wäre, die überdieß eine Handlungsweise betreffen, welche von Seite des Handelnden so manche Opfer erfordert, und von denen es nicht gleich auf der Stelle einleuchtet, daß sie durch die für das Ganze entspringenden Vortheile überwogen werden, können ohne Zweifel nur als \RWbet{Räthe} aufgestellt werden.
\end{aufza}
\begin{RWanm} 
Man hat die katholische Kirche um dieser Lehre wegen sehr hart getadelt, und ihr den Grundsatz entgegengesetzt, daß der Mensch nie mehr thun könne, als er solle, weil er, zu Folge des obersten Sittengesetzes, das Wohl des Ganzen so sehr befördern müsse, als er nur immer vermag. Man hat behauptet, daß sie durch diese Lehre dem ausdrücklichen Befehle Jesu widerspreche: Wenn ihr Alles gethan, was euch geboten war, so sollet~\RWSeitenw{215}\ ihr sagen, wir sind unnütze Knechte, und haben nur das, was unsere Schuldigkeit war, geleistet. \RWbibel{Lk}{Luk.}{17}{10}\par
Hierauf ist zu erwiedern, daß die katholische Kirche durch ihre Unterscheidung zwischen Geboten und Räthen keineswegs läugne, wir sollen das Wohl des Ganzen immer so sehr befördern, als wir es nur vermögen; sondern sie lehret nur, daß nicht Alles, was wir sollen, von uns auch unter der Bedingung der Strafe gefordert werde, und meint, daß es nothwendig sey, uns auch auf jene guten Handlungen, die uns nicht unter der Bedingung einer Strafe vorgeschrieben werden können, durch Räthe aufmerksam zu machen, und zu ihrer Ausübung zu ermuntern. Mit dieser Lehre nun stehet der angezogene Befehl unseres Herrn schon darum nicht in dem geringsten Widerspruche, weil Jesus hier ausdrücklich nur von Menschen redet, die bloß gethan, was ihnen geboten war (\RWgriech{t`a diataqj'enta}). Aber auch, wenn er diese Einschränkung nicht beigefügt hätte, wäre es immer eine ganz richtige Verhaltungsregel, daß sich der Mensch seiner verdienstlichen Handlungen wegen nicht rühmen, sondern, nachdem er sie einmal gethan, lieber als seine bloße Schuldigkeit ansehen solle. 
\end{RWanm}

\RWpar{247}{Sittlicher Nutzen}
Wir haben schon oben gezeigt, daß die Unterscheidung zwischen zweierlei Arten von guten Handlungen, deren die einen uns als strenge Schuldigkeiten, die anderen als etwas bloß Verdienstliches erscheinen, sehr nützlich, ja sogar nothwendig sey. Soll aber dieser Unterschied von uns gehörig aufgefaßt und zweckmäßig angewendet werden: so ist es nöthig, daß uns die Sittenlehre gleich bei der Aufstellung jener einzelnen Regeln, nach welchen wir unser Verhalten einrichten sollen, erkläre, welche derselben als wahre Gebote, welche als bloße Räthe angesehen werden sollen.

\RWpar{248}{Wirklicher Nutzen}
Da die Unterscheidung zwischen bloß pflichtmäßigen und verdienstlichen Handlungen Jedem, der nicht leichtsinnig denkt, zu seiner Beruhigung überaus nothwendig ist: so läßt sich wohl erwarten, daß man in jeder besseren Religion auch zwi\RWSeitenw{216}schen den Vorschriften, welche zur Führung eines tugendhaften Lebens darin ertheilt werden, einen solchen Unterschied, wie ihn das katholische Christenthum zwischen Geboten und Räthen machte, wenigstens stillschweigend werde angenommen haben. Was aber der gesunde Menschenverstand stillschweigend anerkannte, das wurde zum größten Nachtheile für die gute Sache von mehreren Weltweisen bestritten und geläugnet. Nicht nur verschiedene heidnische Weltweise, sondern selbst einige christliche Philosophen von akatholischer Partei verwarfen aus dem schon oben angegebenen Grunde allen Unterschied zwischen Geboten und Räthen, woraus man denn sieht, daß die katholische Kirche durch ihre Aufstellung und Vertheidigung eines solchen Unterschiedes der Menschheit in der That einen nicht unwichtigen Dienst geleistet habe. Der Schade, den diese Unterscheidung hie und da etwa durch Mißverstand mag veranlasset haben, nämlich die Trägheit im Guten, zu der sie vielleicht einige Christen verleitet hat, ist gewiß nicht zu vergleichen mit ihren Vortheilen, mit jenem wohlthätigen Einflusse, den sie auf die Beruhigung gerade der edelsten Menschen gehabt, auf jene Freudigkeit, die sie in ihnen zu einer herrschenden Gemüthsstimmung erhoben, \usw\

\RWpar{249}{Die Lehre des katholischen Christenthums von den allgemeinsten Sittengesetzen}
So zahlreich und so verschiedenartig die Regeln sind, welche das Sittengesetz für das Benehmen verschiedener Wesen in den verschiedensten Verhältnissen ertheilt: so gibt es, wie das katholische Christenthum sagt, gleichwohl einige sehr einfache Sätze, die so beschaffen sind, daß sich, wenn auch vielleicht nicht alle, doch fast alle uns Menschen betreffenden sittlichen Vorschriften auf irgend eine Weise aus ihnen ableiten lassen. Die Kenntniß solcher Sätze (die wir mit der Benennung der \RWbet{allgemeinsten Sittengesetze} bezeichnen wollen) soll sehr vortheilhaft für uns seyn; das Christenthum macht uns daher auf die vorzüglichsten derselben aufmerksam, indem es namentlich folgende aufzählt:
\begin{aufza}
\item Handle immer so, wie es das allgemeine Beste, oder das Wohl des Ganzen erfordert.~\RWSeitenw{217}
\item Folge in Allem dem Willen Gottes.
\item Suche in Allem nur die Beförderung der Ehre Gottes.
\item Handle immer so, daß du Gott ähnlicher werdest.
\item Liebe Gott über Alles.
\item Ahme die Beispiele Jesu nach.
\item Liebe den Nächsten wie dich selbst.
\item Behandle Jeden so, wie du wünschtest, daß er auch dich behandle.
\end{aufza}
\begin{RWanm} Von einer schon minderen Allgemeinheit sind die Sprüche: Handle immer so, wie du einst wünschen wirst, gehandelt zu haben in deiner Todesstunde. -- Stelle dich nicht den Menschen dieser Welt gleich; \usw\ Diese wollen wir der Kürze wegen hier übergehen. 
\end{RWanm}

\RWpar{250}{Historischer Beweis dieser Lehre}
Der Begriff, den wir im vorigen Paragraphen unter der Benennung eines allgemeinen Sittengesetzes aufgestellt haben, ist in der katholischen Kirche gewiß nicht unbekannt, wenn er gleich nicht von allen katholischen Sittenlehrern mit eben demselben Worte bezeichnet, von vielen auch wohl ganz unbezeichnet gelassen worden ist.\par
Schon im Evangelio \RWbibel{Mt}{Matth.}{22}{40} hören wir Jesum von einem Paare von Geboten reden, in welchen alle übrigen enthalten wären: \erganf{In diesen zwei Geboten (der Liebe Gottes und des Nächsten) ist das ganze Gesetz und die Propheten (sind alle Gebote Gottes) enthalten.} Dasselbe thut der Apostel Paulus, wenn er (\RWbibel{Gal}{Gal.}{5}{14}) schreibt: \erganf{Das ganze Gesetz wird durch die Befolgung der einen Pflicht erfüllt: Liebe deinen Nächsten wie dich selbst.} Vgl.\ \RWbibel{Röm}{Röm.}{13}{8--10} Die Sittenlehrer der neueren Zeit sprechen in ihren Lehrbüchern häufig von einem gewissen \RWbet{obersten Sittengesetze,} welches der Eine so, der Andere anders erkläret, unter welchen sie aber Alle am Ende nichts Anderes verstehen, als einen Satz, aus dem sich die sämmtlichen Pflichten des Menschen ableiten lassen, gleichviel ob subjectiv oder objectiv, ob durch die Subsumtion eines bloß theoretischen Untersatzes, oder auch practischer Sätze. Sie verstehen also unter einem solchen obersten Sittengesetze gerade dasselbe, was ich so eben~\RWSeitenw{218}\ ein allgemeines Sittengesetz zu nennen vorgeschlagen habe. Daß aber in der katholischen Kirche insonderheit auch die so eben aufgezählten acht Sätze, als solche allgemeine Sittengesetze, oder als Sätze, aus welchen sich alle, oder doch fast alle sittlichen Vorschriften für uns Menschen ableiten ließen, aufgestellt worden seyen, kann man beinahe aus jedem Lehrbuche der katholischen Moral ersehen. Wir wollen zum Ueberflusse nur einiges Wenige über jeden beibringen.
\begin{aufza}
\item \RWbet{Handle immer so, wie es das allgemeine Beste} (oder das Wohl des Ganzen) \RWbet{erfordert}.\par
Diese Regel wird in den Lehrbüchern der katholischen Sittenlehre besonders in der Lehre von den Collisionsfällen, wo es sich um die Entscheidung der Frage handelt, welches von mehreren, einander widerstreitenden Gesetzen das wirklich verbindende sey, unzählige Male aufgestellt; ja sie wird überdieß in manchen Lehrbüchern der Moral sogar als das oberste Sittengesetz ausdrücklich angegeben, s.\ \zB\  des Herrn Raths und Professor Fritsch Einleitung in die christliche Sittenlehre. Bened.\ Stattler's \RWlat{Ethic. Christiana,}\RWlit{}{Stattler2} \uA\  Hindeutungen auf diese Regel kommen auch schon in den Büchern des alten Bundes vor, \zB\  \RWbibel{Est}{Esther}{16}{9}:\editorischeanmerkung{Bolzano zitiert die Stelle aus dem Buch Esther nach der Nummerierung der Vulgata. In der heute üblichen Zählung handelt es sich bei der Stelle vermutlich um Est 8,12h--i.} \erganf{Ihr müsset nicht wähnen, daß, wenn wir euch verschiedene Dinge befehlen, dieß aus Veränderlichkeit unseres Gemüthes geschehe; sondern wir schreiben vor, was wir nach dem Bedürfnisse der Zeit als förderlich \RWbet{für das Wohl des gemeinen Wesens} erkennen.} \RWbibel{2\,Makk}{2.\,Makkab.}{4}{5} heißt es von Onias: \erganf{Weil er in Allem den allgemeinen Nutzen suchte}, \usw\
\item \RWbet{Folge in Allem dem Willen Gottes.}\par
\begin{aufzb}
\item Die katholische Kirche verlangt, daß wir den Willen Gottes durchgängig als den letzten Zweck aller unserer Handlungen und unseres ganzen Strebens annehmen sollen. Dieß könnte nicht geschehen, wenn wir nicht glaubten, daß aus dem Satze: Folge dem Willen Gottes, alle unsere Pflichten und Obliegenheiten auf eine gewisse Art herleitbar seyen.
\item Wenn uns die heil.\ Schrift oder die Kirche irgend eine Pflicht empfiehlt: so geschieht es nicht sowohl dadurch, daß sie die Nothwendigkeit dieser Handlungsweise für die~\RWSeitenw{219}\ Beförderung des allgemeinen Wohles darstellt, als vielmehr meistens nur dadurch, daß sie erinnert, dieß sey der Wille Gottes. Vergl.\ auch \RWbibel{Joh}{Joh.}{4}{34}\ \RWbibel{Mt}{Matth.}{6}{10} \uma\ 
\end{aufzb}
\item \RWbet{Suche in Allem nur die Beförderung der Ehre Gottes.}\par
\begin{aufzb}
\item Das katholische Christenthum stellt bekanntlich die Ehre Gottes als den letzten Zweck der Weltschöpfung auf. Da es nun unsere Pflicht ist, den Willen Gottes, so viel es an uns liegt, auszuführen: so muß es auch unsere Pflicht seyn, die Ehre Gottes zu befördern.
\item Der heil.\ Paulus schreibt (\RWbibel{1\,Kor}{1\,Kor.}{10}{31}): \erganf{Ihr möget essen oder trinken, oder was ihr immer thuet: so thuet Alles zur Ehre Gottes.} Also müssen sich unter die Pflichten, welche die Ehre Gottes befördern, alle pflichtmäßigen menschlichen Handlungen bringen lassen. Vgl.\ \RWbibel{Joh}{Joh.}{17}{4} -- Bekannt ist übrigens auch das von dem Orden der Jesuiten überall Angebrachte: \RWlat{Omnia ad majorem Dei gloriam.}
\end{aufzb}
\item \RWbet{Handle immer so, daß du Gott ähnlicher werdest.}\par
Jesus sagt ausdrücklich (\Ahat{\RWbibel{Mt}{Matth.}{5}{48}}{5,44.}): \erganf{Werdet vollkommen (\RWgriech{t'eleioi}), wie euer Vater im Himmel vollkommen ist.} Und der heil.\ Paulus (\Ahat{\RWbibel{Eph}{Ephes.}{5}{1}}{5,11.}): \erganf{Werdet Nachahmer Gottes, wie wohlgerathene Kinder.} Vgl.\ auch \RWbibel{1\,Joh}{1\,Joh.}{3}{2} \ua
\item \RWbet{Liebe Gott über Alles.}\par
Auf die Anfrage eines Schriftgelehrten, welches das größte Gebot im Gesetze (in der ganzen göttlichen Offenbarung) wäre, erwiederte Jesus (\RWbibel{Mt}{Matth.}{22}{37}): \erganf{Liebe Gott, deinen Herrn, aus ganzer Seele, aus ganzem Herzen, und aus allen deinen Kräften (entlehnt aus \RWbibel{Dtn}{5\,Mos.}{6}{5}), das ist das erste und größte Gebot. Das andere aber ist jenem gleich: Liebe deinen Nächsten wie dich selbst. In diesen zwei Geboten ist das ganze Gesetz und die Propheten (alle Forderungen Gottes an die Menschen sind in diesen beiden Geboten) enthalten.} -- Vgl.\ \RWbibel{1\,Joh}{1\,Joh.}{4}{20}, wo diese Liebe erklärt wird.~\RWSeitenw{220}
\item \RWbet{Ahme dem Beispiele Jesu nach.}\par
Jesus selbst sagte: \erganf{Ich habe euch ein Beispiel gegeben, auf daß, wie ich gethan habe, ihr auch thun möget} (\RWbibel{Joh}{Joh.}{13}{15}). Und der heil.\ Petrus (\RWbibel{1\,Petr}{1\,Petr.}{2}{22}): \erganf{Christus litt für uns, und hinterließ uns hiedurch ein Beispiel, auf daß auch wir in seine Fußtapfen treten.} Gleich darauf wird wieder die Wahrhaftigkeit Jesu uns als ein Muster aufgestellt. -- Der heil.\ Paulus schreibt (\RWbibel{Eph}{Ephes.}{5}{1}): \erganf{Werdet Nachahmer Gottes, wie wohlgerathene Kinder, wandelt in der Liebe, so wie auch Christus uns geliebt hat}, \usw\ Ferner (\RWbibel{Phil}{Philipp.}{2}{5}): \erganf{Ein Jeglicher sey gesinnt, wie Jesus Christus es war.} Und (\Ahat{\RWbibel{1\,Kor}{1\,Kor.}{11}{1}}{10,32.}): \erganf{Werdet meine Nachahmer, wie ich es Christi bin.}
\item \RWbet{Liebe deinen Nächsten wie dich selbst.}\par
Hieher gehöret zum Theile die schon bei Nr.\,5. angeführte Stelle \RWbibel{Mt}{Matth.}{22}{37}\ Ingleichen \RWbibel{Röm}{Röm.}{13}{8--10}: \erganf{Wer seinen Nächsten liebt, der hat das ganze Gesetz erfüllet; denn die Gebote: Du sollst nicht ehebrechen, du sollst nicht tödten, du sollst nicht stehlen, du sollst kein falsches Zeugniß geben, du sollst nicht begehren, und jedes andere Gebot ist in der Vorschrift begriffen: Du sollst deinen Nächsten lieben wie dich selbst. Wer seinen Nächsten liebt, fügt ihm gewiß kein Leid zu. Die Liebe also ist die Erfüllung des Gesetzes.}
\item \RWbet{Behandle Jeden so, wie du wünschest, daß er auch dich behandle.}\par
Diese Regel stellte der Herr ausdrücklich auf (\RWbibel{Mt}{Matth.}{7}{12}): \erganf{Alles, was ihr wollet, das euch die Leute thun, das thuet ihr auch ihnen; denn also verlangt es das Gesetz und die Propheten.} -- Und schon \Ahat{\RWbibel{Tob}{Tob.}{4}{15}}{4,16.} (nach der Vulgata) heißt es: \erganf{Was du nicht willst, daß Andere dir thun, das thue auch du nicht den Andern.}
\end{aufza}

\RWpar{251}{Vernunftmäßigkeit}
Der Begriff eines \RWbet{allgemeinen Sittengesetzes}, wie wir ihn oben erklärten, enthält durchaus nichts Unmögliches. Es kann allerdings Sätze geben, aus denen sich alle~\RWSeitenw{221}\ oder doch fast alle sittlichen Vorschriften für den Menschen auf eine gewisse Art ableiten lassen. Das oberste Sittengesetz, dessen Daseyn wir im ersten Haupttheile dargethan haben, ist selbst ein solcher Satz, nämlich derjenige, aus dem sich die wesentlichen praktischen Wahrheiten, die es nur überhaupt gibt, objectiv, \dh\  so wie die Folgen aus ihrem Grunde, ableiten lassen. Da nun die Art, auf welche die einzelnen Pflichten aus jenem allgemeinen Satze ableitbar seyn sollen, nicht näher bestimmt wird: so kann es derselben auch mehrere geben. Wenn es sich um eine bloß subjective Herleitung, \dh\  um ein bloßes Erkennen oder Erinnern handelt: so gibt es der Wahrheiten mehrere, deren Betrachtung uns an alle oder doch fast alle unsere Pflichten erinnern, oder ihre Erkenntniß bei uns auf irgend eine andere Weise veranlassen kann. Namentlich alle diejenigen Wahrheiten, welche das Christenthum für solche Sätze ausgibt, sind wirklich von dieser Art.
\begin{aufza}
\item \RWbet{Handle immer so, wie es das allgemeine Beste oder das Wohl des Ganzen erfordert.}\par
Dieser Satz ist, wie wir wissen, das wahre oberste Sittengesetz, also auch ohne Zweifel tauglich zur Ableitung aller oder doch fast aller unserer Pflichten.
\item \RWbet{Folge in Allem dem Willen Gottes.}\par
\begin{aufzb}
\item Dieser Satz ist ohne Zweifel \RWbet{wahr}. Der Mensch ist in der That verpflichtet, dem Willen Gottes überall zu folgen, sobald man unter diesem Willen (wie hier von selbst sich verstehet) nicht Gottes eigentlichen Willen (nicht einen Actus seiner Wollkraft), sondern nur seine Gebote, \dh\  diejenigen Handlungsweisen versteht, deren Befolgung Gott für unsere Pflicht erklärt, welche er zu belohnen, und deren Uebertretung er zu bestrafen versprochen hat.
\item Aus diesem Satze lassen sich ferner auch \RWbet{alle unsere menschlichen Pflichten auf eine gewisse Art ableiten.} Von jenen Pflichten, die in der christlichen Offenbarung selbst (in der Bibel oder in der allgemeinen Lehre der Kirche) ausdrücklich aufgestellt werden (und dieses sind bei Weitem die meisten), versteht sich die Art ihrer Herleitung aus diesem Satze von selbst; denn Alles,~\RWSeitenw{222}\ was Gottes Offenbarung als eine Pflicht darstellt, erscheinet uns schon eben darum als ein Befehl, \dh\  Wille Gottes. So können wir \zB\  die Pflicht der Wahrhaftigkeit aus dem Satze: Folge dem Willen Gottes, herleiten, weil ja die heil.\ Schrift die Wahrhaftigkeit von uns ausdrücklich fordert. Aber auch selbst bei denjenigen Pflichten, welche die christliche Offenbarung nicht ausdrücklich als Gottes Gebote aufstellt, die wir jedoch durch unsere bloße Vernunft erkennen, läßt sich sehr leicht der Schluß anbringen: Alles, was gut und recht ist, was deine eigene Vernunft, nach fleißigem Nachdenken, dir als solches darstellt, das will auch Gott. Dieses und Jenes erkennst du aber für gut; also ist es auch Gottes Wille, daß du es thuest.
\end{aufzb}
\item \RWbet{Suche in Allem nur die Beförderung der Ehre Gottes.}\par
Auch dieses ist abermals
\begin{aufzb}
\item \RWbet{ein wahrer Satz}. Gottes Ehre befördern heißt nämlich thun, was die Anerkennung seiner erhabenen Vollkommenheiten, seiner Weisheit, Allmacht, Güte, insonderheit auch seiner Oberherrlichkeit über uns (\di\ des Rechtes, uns verbindliche Gesetze vorzuschreiben) bei uns und Andern befördert. Nun sind wir offenbar verpflichtet, diese Erkenntniß, so viel wir nur können, zu verbreiten; denn die Erkenntniß der göttlichen Vollkommenheiten, insonderheit seiner höchsten Oberherrlichkeit über uns Alle, ist gewiß überaus wohlthätig für alle Menschen.
\item Unter diesen Satz lassen sich ferner auch \RWbet{fast alle guten Handlungen auf eine gewisse Art beziehen}, in sofern nämlich, als jede gute Handlung zur mehreren Anerkennung der göttlichen Vollkommenheiten (zur Ehre Gottes), wenn nicht unmittelbar, wenigstens mittelbar, beiträgt:
\begin{aufzc}
\item zuvörderst dadurch, daß wir die Handlung (nach Nr.\,2.) auf Gottes Gebot unternehmen, und bei uns sowohl, als auch bei Andern, die unseren Beweggrund kennen, die Vorstellung von Gottes Oberherrlichkeit anschaulich machen; dann auch noch~\RWSeitenw{223}
\item dadurch, daß alles Gute, was wir Menschen zu Stande bringen, zugleich auch Gottes Werk ist, der uns erschaffen hat, und uns dazu die nöthigen Kräfte und Gelegenheiten verliehen hat, \usw\ Daher er denn auch gepriesen wird, so oft man uns preiset um unserer guten Werke wegen.
\end{aufzc}
\end{aufzb}
\item \RWbet{Handle immer so, daß du Gott ähnlicher werdest.}\par
Diese Vorschrift ist
\begin{aufzb}
\item \RWbet{richtig}. Zwar können Menschen und überhaupt alle endliche Wesen niemals zu einer völligen Gleichheit mit Gott gelangen; sollen und können ihm auch nicht in allen Stücken nachahmen; und sonach ist diese Vorschrift etwas unbestimmt. Allein aus der Erklärung, welche uns die katholischen Sittenlehrer geben, erhellet deutlich genug, daß sie die Aehnlichkeit mit Gott, nach der wir streben sollen, in lauter von uns nachahmbare und mit Recht nachzuahmende Vollkommenheiten setzen; vor Allem in die Nachahmung seiner Weisheit und seiner Heiligkeit mit ihren Unterarten, als der Gerechtigkeit, Güte, Langmuth, Milde, Barmherzigkeit, Wahrhaftigkeit, Treue, Parteilosigkeit, \usw\ Somit ist die Richtigkeit dieser Vorschrift außer Zweifel.
\item \RWbet{Sie umfaßt aber auch wirklich alle Pflichten}. Eine jede gute Handlung, worin sie immer bestehe, können wir aus dem Grunde der Nachahmung Gottes ableiten; denn selbst wenn die Handlung zu keiner der eben jetzt erwähnten Vollkommenheiten Gottes, die wir auch in uns ausbilden sollen, als eine Aeußerung derselben, oder als ein Mittel, das zu ihr führt, gehörte: so würde sie uns doch schon darum Gott ähnlicher machen, weil wir bei ihrer Verrichtung eben dasselbe wollen, was Gott will (Gott will alles Gute); weil wir auch ferner durch jede gute That mehr Fertigkeit in der Tugend erlangen, und also der unendlichen Heiligkeit Gottes immer näher treten.
\end{aufzb}
\item \RWbet{Liebe Gott über Alles.}\par
\begin{aufzb}
\item \RWbet{Richtigkeit dieser Vorschrift.}~\RWSeitenw{224}\par
Man sagt überhaupt, daß wir einen Gegenstand (auch einen leblosen) lieben, wenn wir an seiner Vorstellung ein Vergnügen finden können, ohne uns immer erst des Grundes, warum wir dieß Vergnügen finden, deutlich bewußt zu werden. Ist dieser Gegenstand ein lebendiges Wesen: so wird zu seiner Liebe eigentlich noch etwas mehr erfordert; und man sagt nur dann, daß wir ihn lieben, wenn es nicht bloß die Vorstellung desselben überhaupt, sondern insonderheit die Vorstellung von seinem Wohlbefinden ist, was uns Vergnügen macht, und zwar, ohne daß wir uns erst des Grundes dieses Vergnügens deutlich bewußt zu seyn brauchten. Es ist daher ein Mißbrauch des Wortes Liebe, und eine Entweihung desselben, wenn man von einem Wollüstlinge sagt, daß er den Gegenstand seiner Lüste -- liebe, da er doch um das Wohl desselben gar nicht bekümmert ist, sondern im Gegentheil es zerstört. Es ist zwar kein Bestandtheil des Begriffes, aber doch ein sicheres Merkmal der Liebe, daß wir an den geliebten Gegenstand oft denken, oft von ihm sprechen, sein Wohlseyn nach aller Möglichkeit zu befördern streben, und hierin einen Theil unserer eigenen Glückseligkeit suchen, daher auch Alles sehr gerne thun, wovon wir vermuthen, daß der geliebte Gegenstand es wünsche; es müßte denn seyn, daß wir deutlich einsehen, daß dieser Wunsch thöricht sey, und daß die Erfüllung desselben unseren Geliebten selbst unglücklich machen würde. Man erachtet bald, daß nicht alle diese so eben angegebenen Bestandtheile und Merkmale der Liebe zu endlichen Wesen bei einer vernünftigen Liebe zu Gott Statt finden können. Die Liebe zu Gott, die das katholische Christenthum von uns fordert, soll nach der eigenen Erklärung desselben sich vornehmlich äußern
\begin{aufzc}
\item in der genauesten Erfüllung aller Gebote Gottes, in einem öfteren Andenken an ihn, in öfterem Gespräche von ihm, \usw\
\item Dann aber auch mit jeder anderen vernünftigen Liebe zu einem lebendigen Gegenstande so viel gemein haben, als es uns möglich ist, in uns anzubilden.
\end{aufzc}
Die Forderung einer solchen Liebe enthält nun erstens nichts Unmögliches; denn es ist~\RWSeitenw{225}
\begin{aufzc}
\item Jedermann möglich, die Gebote Gottes alle genau zu erfüllen, öfters an ihn zu denken, von ihm zu sprechen, \usw\
\item Es ist auch ferner Jedem, vornehmlich aber jedem Gebildeten, möglich, es bei sich dahin zu bringen, daß der Gedanke an Gott ihm ein erfreulicher werde. Hiezu bedarf es nur, daß wir die großen Vollkommenheiten Gottes, besonders seine unendliche Weisheit und Güte, und die vermöge dieser uns selbst erwiesenen und noch in Zukunft zu erweisenden Wohlthaten öfters betrachten, nichts Böses thun, um sich nicht vor seiner Strafe fürchten zu müssen, sondern durch Gutesthun vielmehr in ihm unseren Vergelter und Belohner erwarten. Ungebildeten ist es überdieß eine sehr natürliche Vorstellung, daß Gott ein Vergnügen daran finde, wenn wir Menschen seine Gebote alle recht gewissenhaft befolgen, in seiner Verehrung, und in der Verbreitung seines Ruhmes recht eifrig sind, \usw\ Er kann es also bei sich dahin bringen, daß sich allmählich der herrschende Wunsch in seinem Herzen erzeuge, Gott durch die Erfüllung seiner Gebote Freude zu machen, \usw\
\end{aufzc}
Diese Aehnlichkeit seiner Gefühle gegen Gott mit den Gefühlen der Liebe wird der Gebildete sich freilich nicht geben können. Aber auch er wird es bei sich dahin bringen können, daß es ihm ein Vergnügen wird, den Willen Gottes zu thun. Dieß nämlich wird er bewirken können durch die Betrachtung der Heiligkeit und Vortrefflichkeit dieses Willens, durch den Gedanken, daß dieser Wille am Ende doch auf nichts Anderes, als auf die möglich größte Beglückung aller lebendigen Wesen abzwecke, und daß es so selige Folgen habe, den Willen Gottes zu vollziehen, \usw\par
Die Forderung einer solchen Liebe ist zweitens auch ganz der Vernunft gemäß.\par
Was erstlich den einen Bestandtheil dieser Forderung, nämlich die Erfüllung aller Gebote Gottes \usw\ anlangt: so ist die Vernunftmäßigkeit dieser Forderung bereits gezeigt. Daß aber auch der zweite Theil dieser Forderung vernunft\RWSeitenw{226}mäßig sey, erhellet aus den Vortheilen, die wir bei Betrachtung des sittlichen Nutzens dieser Lehre nachweisen wollen.
\item Diese Forderung kann auch als ein \RWbet{allgemeines Sittengesetz} angesehen werden; denn weil sich die Liebe Gottes, welche das Christenthum fordert, in der Erfüllung aller seiner Gebote äußern soll: so ist schon bei der Betrachtung des zweiten Gesetzes gezeigt worden, daß alle Pflichten des Menschen unter die Forderung, Gott zu lieben, bezogen werden können.
\end{aufzb}
\item \RWbet{Ahme dem Beispiele Jesu nach.}\par
\begin{aufzb}
\item Ein \RWbet{wahrer} Satz.\par
Die Nachahmung des Beispieles Jesu, welche das Christenthum von uns verlangt, soll keine Nachahmung einzelner Handlungen desselben, sondern nur eine Annahme seiner Grundsätze und Gesinnungen seyn, aus welchen jene Handlungen in seinen eigenthümlichen Verhältnissen entsprangen. Da nun die Grundsätze und die Gesinnungen Jesu, wie die katholische Kirche sie darstellt, (nämlich im Evangelio und in der Auslegung desselben) alle vollkommen richtig und sittlich gut waren: so ist die Richtigkeit dieser Vorschrift außer Zweifel.
\item Und \RWbet{umfaßt alle Pflichten.}\par
Da Jesus selbst nie eine einzige Sünde beging: so muß derjenige, der ihm ähnlich zu werden strebt, gleichfalls nach einer vollendeten Heiligkeit streben, also jede seiner Pflichten auf das Gewissenhafteste erfüllen, und so kann das Beispiel Jesu als ein subjectiver Anleitungsgrund zu jeder guten Handlung dienen; unzählige aber lassen sich aus der Betrachtung des Beispieles Jesu noch viel unmittelbarer ableiten, weil wir es nicht bloß durch diesen mittelbaren Schluß, sondern aus seiner Geschichte unmittelbar entnehmen, daß er diese Pflichten befolgt, und diese Tugenden an sich gehabt habe.
\end{aufzb}
\item \RWbet{Liebe deinen Nächsten, wie dich selbst.}\par
\begin{aufzb}
\item Eine \RWbet{richtige} Forderung.
Der Sinn dieser Forderung ist nach dem Evangelio und nach der Erklärung, welche das Christenthum von ihr gibt, daß wir~\RWSeitenw{227}
\begin{aufzc}
\item das Wohl des Nebenmenschen, wo es uns möglich ist, mit eben dem Fleiße befördern, wie unser eigenes, und
\item als Mittel hiezu eine möglichst starke Liebe zu unseren Nebenmenschen in unseren Herzen erwecken und unterhalten sollen.
\end{aufzc}
Der erste Theil dieser Forderung ergibt sich sehr leicht aus dem obersten Sittengesetze. Diesem gemäß muß man das Wohl des Ganzen befördern, folglich das Wohl eines jeden einzelnen Theiles mit einem Eifer, wie ihn die Wichtigkeit dieses Theiles verdient. Glieder von gleicher Empfänglichkeit für den Genuß des Glückes werden auch eine gleich große Sorgfalt verdienen. Wir werden, um dem Einen dieser Wesen einen Genuß zu verschaffen, nicht einen gleich großen Genuß Einem oder mehreren anderen entziehen dürfen. Da nun bei allen Menschen eine, der Regel nach, für den Genuß der Glückseligkeit gleiche Empfänglichkeit vorhanden ist: so müssen wir auch (soferne wir es vermögen) für Alle mit einem gleich großen Eifer sorgen; und also das eigene Wohl nicht mehr und nicht weniger als das Wohl eines Andern befördern.\par
Die Vernunftmäßigkeit des zweiten Theiles dieser Forderung betrachten wir bei dem sittlichen Nutzen dieses Gesetzes.
\item Diese Forderung \RWbet{umfaßt beinahe alle Pflichten.}\par
Nicht schlechterdings alle menschliche Pflichten, nicht die Obliegenheiten, die wir \zB\  gegen die thierische Welt haben, können aus diesem Satze (zum wenigsten nicht mit Vollständigkeit) entwickelt werden; aber doch alle Pflichten, die wir für die Beförderung des Wohles unserer Mitmenschen haben, lassen sich mit vieler Leichtigkeit aus ihm herleiten. Diese Pflichten sind aber bei Weitem die zahlreichsten und die wichtigsten, so daß wir auch diesen vortrefflichen Spruch mit dem Namen eines allgemeinen Sittengesetzes belegen, zumal da derjenige, der alle Pflichten, die er des Nächsten wegen hat, genau beobachtet, auch jene wenigen Pflichten, die sich aus diesem Grunde nicht ableiten lassen, schwerlich versäumen wird.
\end{aufzb}
\item \RWbet{Behandle Jeden so, wie du wünschtest, daß er auch dich behandle.}~\RWSeitenw{228}\par
Diese Vorschrift ist keineswegs so zu verstehen, als ob wir auf Andere immer so einwirken sollten, wie wir nach unserer Eigenthümlichkeit wünschten, daß sie auf uns einwirken möchten. Denn freilich kann es gar Manches geben, was wohl uns angenehm seyn würde, aber nicht Andern es ist; auch hegen wir vielleicht sehr unbescheidene Wünsche, und wollen, daß Andere Alles uns nur allein aufopfern sollen. Das sind nun weder die Andern uns, noch sind wir Andern es schuldig. Allein der Sinn jener Vorschrift ist nur, daß wir uns jedesmal in die Lage Anderer hineindenken sollen, um zu beurtheilen, was sie in dieser Lage billiger Weise von uns verlangen können; und daß wir, was ihnen angenehm und wahrhaft ersprießlich ist, mit eben dem Eifer vollziehen, mit dem wir wünschen und auch vernünftiger Weise verlangen dürften, daß auch von ihnen geschehe, was uns angenehm und wahrhaft ersprießlich ist. Hieraus ersieht man, daß diese Vorschrift eigentlich in der vorhin betrachteten: Liebe deinen Nächsten wie dich selbst, enthalten ist; daher denn ihre Vernunftmäßigkeit auch keines weiteren Beweises bedarf.
\end{aufza}

\RWpar{252}{Sittlicher Nutzen}
Allgemeine Sittengesetze, wie sie auch immer lauten, sind schon an sich von großem Nutzen. Wenn wir alle, oder doch fast alle unsere Pflichten in einen einzigen Satz, aus dem sie sich leicht wieder ableiten lassen, zusammenfassen können: so gibt dieß
\begin{aufzb}
\item der menschlichen Tugend ein schöneres und gefälligeres Ansehen. Die Uebereinstimmung, die unter so vielen, dem ersten Anscheine nach so ganz verschiedenartigen Pflichten und Obliegenheiten entdeckt wird, wenn man sie alle in einen einzigen sehr kurzen Satz zusammenfassen kann, ist unserem Streben nach Einheit überaus angenehm und für unseren Schönheitssinn ergötzend.
\item Ferner kommt uns die Tugend dann auch viel leichter vor, und wir entschließen uns eher zu ihrer Ausübung; denn weil sich der kurze Satz, der alle unsere Pflichten umfaßt, so leicht und schnell aussprechen läßt: so hoffen wir, auch die Erfüllung werde nichts so gar Schweres~\RWSeitenw{229}\ seyn. Im Gegentheil aber, wenn man uns unsere Pflichten so sehr zerstückelt, daß schon das bloße Behalten derselben im Gedächtnisse uns eine nicht unbedeutende Mühe verursacht: so werden wir muthlos bei dem Gedanken, um wie viel schwerer erst die Erfüllung durch die That seyn werde.
\item Doch der wichtigste Vortheil, den solche allgemeine Sittengesetze uns leisten, bestehet darin, daß sie die Anerkennung aller unserer einzelnen Pflichten und Obliegenheiten, oder doch wenigstens die Erinnerung an sie, zu der gehörigen Zeit gar sehr erleichtern und befördern können. Wer weiß es nicht, wie oft wir eine unserer Pflichten nur darum übertreten, weil wir sie nicht erkennen, oder uns wenigstens ihrer nicht zur gehörigen Zeit erinnern. Kennen wir einen Satz, aus dem sich alle oder doch fast alle unsere Pflichten auf eine leichte Art ableiten lassen; und haben wir uns nur diesen erst geläufig gemacht; so läßt sich hoffen, daß wir durch die Erinnerung an ihn in unzähligen Fällen erkennen, was unsere Pflicht hier sey. Dieser Vortheil wird in einem um desto höheren Grade Statt finden, wenn wir mehrere dergleichen Sittengesetze erhalten, deren ein jedes die menschlichen Pflichten und Obliegenheiten aus irgend einem eigenen Gesichtspuncte darstellt. Hiedurch geschieht nämlich, daß, wenn uns auch etwa der eine Satz dießmal nicht einfallen sollte, weil er in gar keiner Verbindung mit den in unserer Seele herrschenden Vorstellungen stehet, doch vielleicht irgend einer der übrigen unserm Bewußtseyn sich darstellt, und unsere Pflicht uns in's Gedächtniß ruft. -- Die allgemeinen Sittengesetze nun, die das katholische Christenthum eingeführt hat, sind unter allen, die man erdenken könnte, die vortrefflichsten, und es möchte wohl nicht leicht Jemand noch einen neuen Satz von großer Brauchbarkeit angeben können. Die Erfindung eines solchen verdiente in der That zu einer stehenden Preisaufgabe gemacht zu werden.
\end{aufzb}
\begin{aufza}
\item \RWbet{Handle immer so, wie es das allgemeine Beste oder das Wohl des Ganzen erfordert.}~\RWSeitenw{230}\par
\begin{aufzb}
\item Da dieser Satz das wahre oberste Sittengesetz ist, so dient er uns in unzähligen Fällen, um die verwickeltesten Collisionen schnell und richtig zu entscheiden.
\item Dann kann auch nichts der Tugend mehr zur Empfehlung gereichen, und nichts sie liebenswürdiger in unseren Augen machen, als die Bemerkung, daß der letzte Zweck all ihres Strebens nichts Anderes sey, als die Beförderung des allgemeinen Wohles; daß jede Pflicht und Obliegenheit, welche wir haben, am Ende nur in der Beförderung der eigenen oder der Glückseligkeit anderer Wesen gegründet sey.
\end{aufzb}
\item \RWbet{Folge in Allem dem Willen Gottes.}\par
Die Gewohnheit, alle unsere Pflichten uns als Gebote Gottes zu denken, ist von dem wohlthätigsten Einflusse auf unsere Tugend. Denn hiedurch erhalten wir
\begin{aufzb}
\item mehr Achtung und Ehrfurcht gegen alle diejenigen Pflichten, die wir nicht nur durch unsere eigene Vernunft erkennen, sondern auch in der göttlichen Offenbarung (etwa in der heiligen Schrift, oder in den Entscheidungen der Kirche, in den unter ihrer Auctorität eingeführten Lehrbüchern \udgl\ ) ausdrücklich aufgestellt finden. Sie werden uns jetzt um so sicherer, je weniger die göttliche Vernunft einem Irrthume ausgesetzt ist.
\item Auch selbst diejenigen Pflichten, die wir nicht ausdrücklich, und nicht unmittelbar in der göttlichen Offenbarung finden, sondern nur durch unsere eigene Vernunft herleiten, werden uns ehrwürdiger, wenn wir eben diese Vernunft als die Stimme Gottes in uns betrachten; wenn wir uns vorstellen, daß Gott es sey, der in uns spricht, durch das Gewissen, was wir, Geschöpfe seiner Hand, fliehen oder wählen müssen. (Gellert.)\RWlit{}{Gellert1}
\item Der Gedanke, daß diese oder jene Handlung der Wille Gottes sey, enthält auch den andern in sich, daß ihre Ausübung sicher belohnet werden, und
\item den noch ersteren, daß ihre Unterlassung eine ganz unausbleibliche und schwere Strafe finden werde.
\end{aufzb}
\item \RWbet{Suche in Allem nur die Beförderung der Ehre Gottes.}~\RWSeitenw{231}\par
Der eigenthümliche Gesichtspunct, unter dem dieser Satz unsere sämmtlichen Pflichten vereinigt, ist ganz geeignet, uns für sie einzunehmen, und einen heiligen Eifer für ihre Erfüllung in uns anzufachen. Oder wie mächtig erhebt nicht der Gedanke, daß wir durch jede unserer auch noch so gering scheinenden Handlungen, auch durch die Speise und den Trank, die wir zu uns nehmen, Gott zu verherrlichen im Stande sind! -- Welch einen Ernst vermag diese Vorstellung nicht über alle unsere Geschäfte auszubreiten! In schweren und großen Unternehmungen aber: welch ein Aufmunterungsgrund, der unsere schon müden und sinkenden Kräfte immer von Neuem wieder aufzurichten vermag!
\item \RWbet{Handle immer so, daß du Gott ähnlicher werdest.}\par
\begin{aufzb}
\item Indem uns das Christenthum auffordert, nach Aehnlichkeit mit dem Unendlichen zu streben, setzet es stillschweigend voraus, es sey uns nicht unmöglich, eine des Namens nicht ganz unwerthe Aehnlichkeit mit Gott zu erreichen. Dieser Gedanke nun hat etwas Erhebendes, etwas, das unserem Ehrgefühle schmeichelt, und uns eben deßhalb Lust macht, nach diesem erhabenen Ziele zu ringen. Beiläufig eben so, wie ein Anfänger in der Kunst des Zeichnens auf eine sehr angenehme Art überrascht wird, und eine gesteigerte Lust zu seiner Ausbildung erhält, wenn ihm nicht irgend eine unberühmte Arbeit eines Namenlosen, sondern die Zeichnung eines der berühmtesten Meister selbst vorgelegt wird: so werden auch wir nicht wenig überrascht, wenn uns Gott auffordert, daß wir ihn nachahmen sollen, und wir strengen uns nun um so eifriger an, etwas in seiner Art Vollkommenes zu leisten.
\item Aber so Großes wir auch zu Stande gebracht haben mögen, wenn wir es hinterher mit dem, was Gott thut, vergleichen: so werden wir immer einen unendlichen Abstand gewahren, und deßhalb zu keinem schädlichen Stolze, auch nicht zur Trägheit verleitet werden; sondern im Gegentheil in dieser Vorschrift des Christenthums die ernste Aufforderung finden, immer weiter zu streben.
\end{aufzb}
\item \RWbet{Liebe Gott über Alles.}~\RWSeitenw{232}\par
Wieder ein neuer Gesichtspunct, aus dem uns dieß Sittengesetz unsere sämmtlichen Verhältnisse zeigt. Wir sollen uns der Liebe zu Gott befleißen, und dann alle unsere Pflichten aus dieser Liebe zu Gott erfüllen. Das Bestreben, uns diese Liebe zu Gott zu verschaffen, sofern diese Liebe
\begin{aufzb}
\item in einem Wohlgefallen an dem Gedanken Gottes besteht, ist gewiß ein sehr ersprießliches Bestreben. Wenn wir es durch dieß Bestreben wirklich bei uns dahin bringen, daß der Gedanke an Gott uns ein erfreulicher Gedanke wird: so gewähret uns dieß mancherlei Vortheile:
\begin{aufzc}
\item Den Vortheil des Vergnügens, den uns das Wohlgefallen an dem Gedanken Gottes schon an sich selbst gewährt; ein Vergnügen, das uns um desto öfterer zu Theil wird, je öfter wir an Gott, an dieses wichtigste der Wesen, zu denken Veranlassung haben.
\item Eben um dieses Vergnügens willen, das der Gedanke an Gott uns gewährt, werden wir seiner recht oft gedenken, was, wie bekannt, eines der wirksamsten Beförderungsmittel der Tugend, eines der sichersten Abhaltungsmittel von allem Bösen ist.
\item Um dieses Wohlgefallens wegen, das wir an dem Gedanken Gottes finden, werden wir Gottes Gebote (Gebote eines von uns geliebten Wesens) auch um so leichter finden, \usw\
\end{aufzc}
\item Haben wir es in unserer Liebe zu Gott noch etwas weiter gebracht, \zB\  auch dahin, daß schon der bloße Gedanke: Dieß oder Jenes sey dem Willen Gottes gemäß, in uns die Begierde, es zu vollziehen, erzeugt (wie wir dieß etwa dadurch bewirken konnten, daß wir oft in Erwägung zogen, wie überaus gut und nur auf unser Glück abzielend alle göttlichen Gebote seyen); ja konnten wir uns sogar die Vorstellung beibringen, daß wir durch Gutes thun Gott selbst eine Art von Vergnügen machen, ihm wohlgefällig werden: so wird dieß noch überdieß folgende Vortheile haben:
\begin{aufzc}
\item Indem wir glauben, daß wir das Gute, das wir thun, nicht mehr bloß unsertwegen (aus bloßem Eigennutz, aus Furcht vor Strafe); sondern gewissermaßen um~\RWSeitenw{233}\ Gotteswillen thun, so hoffen wir auch um desto zuversichtlicher, von Gott Belohnung zu erhalten.
\item Eine solche Liebe zu Gott muß unser Herz nothwendig veredeln, und uns empfänglicher machen, auch für die Liebe gegen Menschen. Wer gegen Gott dankbar ist, wird es auch gegen Menschen seyn; wer Gott liebt, wird auch die Menschen lieben. (\RWbibel{1\,Joh}{1\,Joh.}{3}{17})
\end{aufzc}
\end{aufzb}
\item \RWbet{Ahme dem Beispiele Jesu nach.}\par
Dieses Sittengesetz hat erstlich zwei Vortheile, die ganz denjenigen ähnlich sind, die wir von dem Gesetze: Ahme Gott nach, so eben angezeigt haben; dann noch zwei eigene, nämlich:
\begin{aufzb}
\item Die Aufforderung, uns Jesum, den Vollkommensten aus allen Sterblichen, denjenigen, den Gott zur Würde seines eigenen Sohnes erhoben, vor dessen Namen sich alle Kniee beugen im Himmel, auf Erden und unter der Erde, als Muster der Nachahmung vorzusetzen, erhebt uns, und macht uns Lust, mit aller Anstrengung darnach zu ringen, daß wir ihm in der That einiger Maßen gleichen.
\item Da es uns aber nie einfallen kann, daß wir ihn schon erreicht hätten: so wird unser Eifer in der Vervollkommnung unserer selbst niemals erschlaffen können.
\item Die Aufforderung, dem Beispiele Jesu zu folgen, schließt auch die andere in sich, sein Beispiel oft zu betrachten. Thun wir dieß aber: so lernen wir hiedurch das wahre Wesen der menschlichen Vollkommenheit, und wie der Vollkommene sich in jeder Lage des Lebens benehme, weit besser kennen, als es auf irgend eine andere Weise, \zB\  durch abstracte Untersuchungen, oder durch die Betrachtung des Beispieles Anderer minder vollkommener Menschen, geschehen könnte.
\item Aus der Betrachtung des Beispieles Jesu erfahren wir nicht nur, wie sich der Vollkommene zu benehmen habe; sondern wir sehen auch, daß es den Menschen möglich sey, so zu verfahren.
\end{aufzb}
\item \RWbet{Liebe deinen Nächsten wie dich selbst.}\par
Zu Folge dieses Satzes soll der Ingebriff aller oder doch fast aller unserer Pflichten seyn, das Wohl des Näch\RWSeitenw{234}sten eben so, wie unser eigenes zu befördern; und als Mittel dazu sollen wir eine gewisse Liebe zu jedem Menschen, vornehmlich zu demjenigen, dem wir nahe genug stehen (zu unserem Nächsten), um ihm helfen zu können, in unseren Herzen auszubilden suchen.
\begin{aufzb}
\item Zu wissen, daß es der Inbegriff aller oder doch fast aller unserer Pflichten sey, das Wohl des Nächsten, wie das eigene, zu befördern, dient
\begin{aufzc}
\item der Tugend zur größten Empfehlung für uns, und muß uns alle ihre Vorschriften leicht und angenehm machen. Denn was kann mehr zur Empfehlung der Tugend gesagt werden, als daß alle ihre Vorschriften keinen andern Zweck haben, als das allgemeine Wohl, ja vornehmlich nur das Wohl des menschlichen Geschlechtes, zu befördern?
\item In jenen zweifelhaften Fällen, wo zwei oder mehrere Gebote in Collision gerathen, gewährt uns die Regel, daß wir nur das thun sollen, wodurch das Wohl unserer Mitmenschen am Meisten gewinnt, meistens die leichteste Entscheidung; eine noch leichtere, als der etwas allgemeinere Satz, daß wir das Wohl des Ganzen überhaupt (\dh\ des Inbegriffs aller geschaffenen Wesen) befördern sollen.
\item Ja dieser letztere Satz, wenn wir uns immer nur an ihn halten wollten, könnte uns selbst zu einer Art von Stolz und Eitelkeit, zur moralischen Schönrednerei verleiten; denn da es sich doch nur in den wenigsten Fällen ereignet, daß wir durch unsere Handlungen einen bemerkbaren Einfluß auf das Wohl aller lebendigen Wesen, ja auch nur aller Menschen, haben: so ist es eine unnütze Prahlerei, immer von seiner Rücksichtnahme auf das Wohl des Ganzen zu sprechen. Im Gegentheile aber kann man sich unmöglich bescheidener ausdrücken, als wenn man mit dem Christenthume spricht, daß man das Beste seines Nächsten (\dh\  desjenigen, auf den unsere Handlung so eben einen bemerkbaren Einfluß hat) berücksichtigen wolle.~\RWSeitenw{235}
\end{aufzc}
\item Nichts kann ersprießlicher seyn, als jene Liebe, die uns als Mittel zur Erfüllung unserer Pflichten gegen den Nächsten empfohlen wird.
\begin{aufzc}
\item Wenn wir nicht Liebe für einen Menschen empfinden, also bei seinem Unglücke uns nicht betrüben, und seines Glückes uns nicht freuen: so kennen wir keinen andern Antrieb, ihm nützlich zu werden, als den Beifall unseres Gewissens, und die allenfalls noch sehr zweideutigen und in manchen Fällen anwendbaren Triebfedern der Ehre und des eigenen Vortheils, der sich zuweilen mit dem Vortheile des Nächsten vereiniget. Dieß hat zur Folge, daß wir sehr vieles Gute, das wir die Gelegenheit hätten, dem Nächsten zu erweisen, gar nicht bemerken; denn nur Liebe macht scharfsichtig und übersieht keine Gelegenheit, wo dem Andern eine Freude gemacht, oder ein Leid erspart werden kann. In unzähligen andern Fällen, wo der Verstand es wohl deutlich genug erkennt, daß wir hier helfen könnten, werden wir es doch unterlassen, weil uns das Opfer zu groß seyn wird. Und selbst in den Fällen, wo wir uns zu einer Dienstleistung entschließen, werden wir es vielleicht auf so kalte und empfindungslose Art thun, daß unsere Gutthat dem Nächsten nicht halb so erfreulich wird, als wenn sie aus einem liebenden Herzen gekommen wäre.
\item In dem Gefühle der Mitfreude an dem Wohlseyn unseres Nächsten werden wir, vornehmlich dort, wo dieses Wohlseyn zum Theil unser eigenes Werk ist, eine der reinsten und erquickendsten Vergnügungen finden.
\item Und selbst die Wehmuth, die wir in andern Fällen über das Unglück des Nächsten empfinden, ist kein Schmerz solcher Art, daß ihm nicht eine gewisse Süßigkeit beigemischt wäre. Noch hat sich kein gefühlvoller Mensch im Ernste ein kaltes Herz gewünscht, nur um den Schmerz des Mitleids nicht empfinden zu müssen.
\end{aufzc}
\end{aufzb}
\item \RWbet{Behandle Jeden so, wie du wünschtest, daß er auch dich behandle}.~\RWSeitenw{236}\par
Diese Regel ist in unzähligen Fällen, mehr als eine jede andere, geeignet, uns das Widerrechtliche unseres Benehmens im höchsten Lichte zu zeigen, und uns zu vermögen, daß wir ein liebevolles Betragen annehmen. Zumal da das katholische Christenthum auch noch den Grundsatz aufstellt, daß mit eben dem Maße, mit dem wir Andern zumessen, auch uns selbst zugemessen werden solle; daß über den, der Unbarmherzigkeit geübt, ein unbarmherziges Gericht ergehen werde; daß uns vergeben werden soll, so wie wir selbst vergeben; \usw\
\end{aufza}

\RWpar{253}{Wirklicher Nutzen}
Da die so eben betrachteten Sätze alle eine so hohe Brauchbarkeit haben, und fast keinem erdenklichen Mißbrauche ausgesetzt sind: so ist kein Zweifel, daß sie durch einen Zeitraum von achtzehn Jahrhunderten des Guten ungemein viel gestiftet haben müssen. Zumal wenn wir bedenken, welche unvollkommene zum Theile selbst gefährliche Begriffe vor ihrer Einführung unter den Menschen herrschend gewesen, und durch sie verdrängt worden sind. --\par
Der Satz: \erganf{Liebe Gott über Alles}, findet sich zwar schon bei Moses (\RWbibel{Dtn}{5\,Mos.}{6}{5}); der Spruch: \erganf{Was du nicht willst, daß dir die Leute thun} \usw\ stehet im Buche Tobias (\Ahat{\RWbibel{Tob}{}{4}{15}}{4,16.}); aber Jesus erst war es, der diese Sätze heraushob, und sie bei seinen Anhängern in wirklichen Gebrauch setzte.
\begin{aufza}
\item Wir finden nicht, daß \RWbet{heidnische} Weltweise den letzten Zweck aller Pflichten jemals in die \RWbet{Glückseligkeit des Ganzen} gesetzt hätten; sondern diejenigen, die noch am Würdigsten von der Tugend dachten, setzten das wahre Wesen derselben in das \RWlat{honestum et utile}, Letzteres nur in Hinsicht auf den Handelnden verstanden. Inzwischen könnte man hier die Frage aufwerfen, warum das katholische Christenthum, obgleich es die Pflicht, das allgemeine Wohl zu befördern, lehrt, diesen Satz doch nicht ausdrücklich für das \RWbet{oberste Sittengesetz} erkläre, und somit nicht gestehe, daß es im Grunde gar keine andern Pflichten gebe, als nur die einzige, das allgemeine Wohl zu befördern? -- Hierauf ist~\RWSeitenw{237}\ zu erwiedern, daß der Begriff eines obersten Sittengesetzes in der oben (1.~Hauptthl.) aufgestellten rein wissenschaftlichen Bedeutung gar nicht zur Religion gehöre; daß ferner die ausdrückliche Erklärung, es gebe sonst keine andere Pflicht, nur die einzige, das allgemeine Wohl zu befördern, bei ungebildeten Leuten eher Schaden als Nutzen stiften könne. Denn würden Solche sich nicht von einer Menge sehr wichtiger Pflichten lossagen, sobald sie, aus Kurzsichtigkeit oder wohl gar aus Leidenschaft, nicht einsehen, wie durch Befolgung derselben das Wohl des Ganzen befördert werde? --
\item Wenn \RWbet{heidnische} Tugendfreunde sich zur Erfüllung ihrer Pflichten ermuntern wollten, so war es niemals der \RWbet{Wille Gottes}, sondern selbst bei den Besten aus ihnen nichts, als die innere Anständigkeit der Tugend, die sie in's Auge faßten. Eine Betrachtung, die bei Weitem nicht so wirksam seyn konnte, als die Betrachtung des göttlichen Willens, die gleich den Nebengedanken der Vergeltung herbeiführt, auch die Gefühle der Dankbarkeit und Liebe weckt, und als Beweggrund nützt. Jene Betrachtung dagegen, wie nahe verwandt war sie nicht mit Stolz und Hochmuth, und wie oft ging sie auch wirklich in diese Gefühle über.\par
Freilich kann man uns einwenden, daß auch der Grundsatz, dem Willen Gottes zu folgen, häufig gemißbraucht worden sey, und daß böse Menschen unter dem Vorwande, es sey der Wille Gottes, oft die schändlichsten Thaten verübten. Dasselbe gilt auch von dem Satze, der die Beförderung der Ehre Gottes verlangt. -- Aber so wahr das auch ist: so wird doch der Mißbrauch nicht so häufig Statt gefunden haben und des Bösen so viel erzeugt, daß es das Gute überwiegen sollte, was diese beiden Grundsätze bei unzähligen Christen veranlaßt haben; zumal da selbst die böse oder gemeinschädliche Handlung, die unter einem solchen Vorwande vollzogen ward, wegen der guten Meinung, welche der Handelnde bei ihrer Vollziehung hatte, Vieles von ihrer Schädlichkeit verlor. In Zukunft wird bei immer aufgeklärteren Begriffen darüber, was allein sich als Gottes Wille oder Gebot ansehen lasse, und was allein Gottes wahre Ehre befördere, dieser Mißbrauch ohnehin immer seltener werden.~\RWSeitenw{238}
\item Wo hatten die \RWbet{Heiden}, wo hat noch jetzt \RWbet{irgend ein Volk}, das nicht dem Christenthume zugethan ist, ein so vortreffliches Muster der Nachahmung, als wir an \RWbet{Jesu} haben? Und doch ist es wahr, daß alle Beispiele ungleich stärker als alle Belehrungen durch bloße Worte wirken. \RWlat{Verba movent, exempla trahunt!} \usw\
\end{aufza}

\RWpar{254}{Die Lehre des katholischen Christenthums von dem Vorhandenseyn natürlicher sowohl als geoffenbarter Pflichten}
Da es die Zeit schlechterdings nicht verstattet, eine auch noch so gedrängte Uebersicht von den sämmtlichen Pflichten zu geben, die das katholische Christenthum für die besonderen Verhältnisse des Lebens aufstellt: so hebe ich nur einige wenige Lehren, die mir vor andern merkwürdig dünken, heraus. Hieher gehört zuvörderst die katholische Lehre \RWbet{von dem Vorhandenseyn natürlicher sowohl als geoffenbarter Pflichten für uns Menschen.}
\begin{aufza}
\item Das katholische Christenthum erkläret ausdrücklich, daß wir alles dasjenige, was wir, auch ohne ein göttliches Zeugniß dafür auffinden zu können, \RWbet{durch unsere bloße sich selbst überlassene Vernunft als unsere Pflicht erkennen,} auch in der That dafür ansehen sollen. Es nennt dergleichen Pflichten bloß \RWbet{natürliche Pflichten,} und die Sätze, welche dergleichen Pflichten enthalten, besonders wenn sie nicht eine einzelne Handlung, sondern eine allgemeine Weise zu handeln bestimmen,\RWbet{ natürliche Gesetze.}
\item Nebst diesen natürlichen Pflichten, sagt das katholische Christenthum weiter, gibt es auch noch einige Pflichten für uns, die wir \RWbet{durch unsere bloße Vernunft nicht anerkennen} würden, wohl aber deßhalb annehmen sollen, weil sie das Zeugniß Gottes für sich aufweisen können. Es nennt diese Pflichten \RWbet{göttliche}, oder \RWbet{göttlich geoffenbarte.}
\end{aufza}

\RWpar{255}{Historischer Beweis dieser Lehre}
\begin{aufza}
\item Daß Alles, was uns die bloße sich selbst überlassene Vernunft auf dem gehörigen Wege des Nachdenkens als eine~\RWSeitenw{239}\ Pflicht darstellt, auch in der That Pflicht für uns sey, wurde in der katholischen Kirche von jeher geglaubet und gelehret. So schreibt auch der heil.\ Paulus (\RWbibel{Phil}{Philipp.}{4}{8}): \erganf{Im Uebrigen, liebe Brüder! was immer nur wahr, anständig und rechtschaffen ist, was rein, was liebenswürdig ist, was einen guten Ruf bringt, was immer Tugend heißt, was eines Lobes werth ist, dem strebet nach.} -- Und (\RWbibel{1\,Thess}{1\,Thessal.}{5}{15\,ff}): \erganf{Sucht Jedermann Gutes zu thun, nicht nur in eurer Gesellschaft, sondern auch gegen Andere. -- Prüfet Alles, und behaltet das Gute; von Allem, was böse heißt (oder was die Gestalt des Bösen hat) haltet euch entfernt.} -- Hieher gehört vielleicht auch \RWbibel{Röm}{Röm.}{14}{23}: \erganf{Alles, was immer nicht aus Ueberzeugung (\RWgriech{>ek p'istews}) hervorgeht, ist Sünde.}
\item Daß aber das katholische Christenthum glaube, zu diesen natürlichen Pflichten könnten durch Gottes Offenbarung auch noch gewisse neue hinzukommen, und daß es behaupte, es sey dieß wirklich geschehen, erhellet aus den später anzuführenden Vorschriften von dem Gebrauche der Heiligungsmittel; denn hier werden uns offenbar Pflichten aufgelegt, die wir durch unsere sich selbst überlassene Vernunft nie anerkannt haben würden.
\end{aufza}

\RWpar{256}{Vernunftmäßigkeit und sittlicher Nutzen}
Die Vernunftmäßigkeit und der sittliche Nutzen dieser, so wie der meisten rein praktischen Lehren fallen theils mit einander zusammen, theils kann man sie doch sehr bequem vereinigt abhandeln. Denn zur Vernunftmäßigkeit einer praktischen Lehre, \dh\  einer solchen, die irgend eine Handlungsweise uns als Pflicht aufstellt, oder doch wenigstens als verdienstlich empfiehlt, ist nöthig, daß wir von dieser Handlungsweise irgend einen Vortheil für das gemeine Beste, vornehmlich aber für die Handelnden selbst nachweisen. Gerade dieses aber entscheidet auch schon die sittliche Zuträglichkeit einer solchen Lehre.
\begin{aufza}
\item Was nun zuvörderst die Lehre von dem Vorhandenseyn der sogenannten natürlichen Pflicht anlangt: so ist es unser eigenes Bewußtseyn, welches uns diese Wahrheit be\RWSeitenw{240}stätigt. Daß aber das katholische Christenthum diese Lehre ausdrücklich aufstellt, ist ein sehr wichtiger Umstand; denn hiedurch nimmt es die sämmtlichen Lehren der natürlichen Moral in seinen Lehrbegriff auf; und wir werden uns als katholische Christen von keiner Pflicht, die uns die bloße Vernunft schon auflegt, aus dem Grunde lossagen dürfen, weil sie im Christenthume nicht ausdrücklich vorgetragen wird. Genug, daß jener Satz ausdrücklich vorkommt.
\item Die Vernunftmäßigkeit der Behauptung, daß uns durch eine göttliche Offenbarung auch neue, durch die bloße Vernunft gar nicht erkennbare Pflichten aufgelegt werden können, haben wir schon in dem ersten Haupttheile erwiesen. Der sittliche Nutzen aber, den uns diese Lehre gewähret, ist darum sehr beträchtlich, weil die katholische Kirche uns wirklich mehrere für die Beförderung unserer Tugend ungemein zuträgliche Vorschriften als solche uns von Gott selbst geoffenbarte Pflichten aufstellt.\par
\RWbet{Einwurf.} Könnten wir aber unter dem Vorwande, daß Dieß oder Jenes Gottes geoffenbarter Wille sey, einstens nicht auch zu Handlungen, die sittlich böse sind, verleitet werden?\par
\RWbet{Antwort.} Das haben wenigstens wir bei den Begriffen, die wir, in Uebereinstimmung mit dem gemeinen Menschenverstande, von den Kennzeichen einer wahren göttlichen Offenbarung angenommen haben, auf keinen Fall zu befürchten. Denn nach diesen Begriffen werden wir keinen Befehl, als von Gott selbst kund gemacht, annehmen, so lange wir nicht durch die bloße Vernunft eingesehen haben, daß die Befolgung desselben dem Sittengesetze nicht nur nicht widerspricht, sondern vielmehr irgend einen wirklichen Nutzen erzeugt.
\end{aufza}

\RWpar{257}{Die Lehre des katholischen Christenthumes vom Rechte}
\begin{aufza}
\item Wo immer Menschen in solchen Verhältnissen leben, daß sie der Eine in dem Zustande des Andern durch ihre freiwillige Thätigkeit eine vorherzusehende Veränderung bewirken können, da sagen uns die katholischen Sitten- und Rechtslehrer, gibt es für Jeden aus ihnen gewisse Handlungswei\RWSeitenw{241}sen, zu denen er ein \RWbet{Recht}, und wieder andere, zu denen er \RWbet{kein Recht} hat; die ersteren pflegt man auch \RWbet{rechtliche} oder \RWbet{rechtmäßige}, die letzteren \RWbet{widerrechtliche} oder \RWbet{rechtswidrige} zu nennen. In Hinsicht auf \RWbet{rechtliche Handlungen} liegt uns die Pflicht ob, denjenigen aus uns, der eine solche ausüben will, durch keinen Zwang zu hindern, selbst wenn wir einsehen, daß diese Handlung nicht eben die beste ist, die er verrichten könnte, und wenn wir auch im Besitze von Zwangsmitteln sind, durch deren Anwendung wir sein Vorhaben gar wohl vereiteln könnten. Nicht also ist es bei \RWbet{widerrechtlichen Handlungen}; in Betreff deren es uns, wenigstens unter gewissen Umständen und Bedingungen, erlaubt, zuweilen sogar eine Pflicht ist, ihre Vollziehung, wenn andere gelindere Mittel nicht helfen, durch Zwang zu verhindern.
\item \RWbet{Alle sittlich guten Handlungen,} wenigstens alle, die es nicht bloß subjectiv, sondern auch \RWbet{objectiv} sind, sollen wir, wie das Christenthum sagt, auch zu den rechtlichen zählen; und uns sonach wohl hüten, Andere in ihrer Ausübung zu stören, auch wenn unser eigener Vortheil dabei seine Erreichung nicht findet, und manche zu diesem Zwecke tauglichen Mittel uns zu Gebote stehen. Im Gegentheile ist es unsere Pflicht, dergleichen Handlungsweisen, so viel wir nur können, zu befördern. Allein auch \RWbet{böse Handlungen}, subjectiv böse sowohl als objectiv böse, \RWbet{können noch rechtlich seyn}, und in dieser Hinsicht verdienen, daß man sie dulde. Niemand möge sich also mit dem Gedanken, daß seine Handlungen rechtlich sind, daß er ein rechtlicher Mann genannt werden könne, beruhigen und begnügen.
\item Es gibt auch Handlungen, zu deren Vollziehung Andere uns zu \RWbet{zwingen}, wo nicht die Pflicht, doch das Recht haben; die eben deßhalb \RWbet{rechtlich erzwingbar} genannt werden. Von diesen Handlungen wird gelehret, daß wir, sobald derjenige, welchem das Recht, sie von uns zu erzwingen, zusteht, davon Gebrauch machen will, sofort die Pflicht haben, sie zu vollziehen; und daß es somit dann weder uns, noch Andern erlaubt ist, den versuchten Zwang durch einen Gegenzwang abzuwehren. Wohl gibt es aber auch erzwing\RWSeitenw{242}bare Handlungen, die für den Fall, wo derjenige, dem das Recht, sie zu erzwingen, zukommt, \RWbet{keinen Gebrauch davon macht}, auch keine Pflichten sind, ja vielleicht dann sogar pflichtwidrig wären. Man muß also \RWbet{rechtlich erzwingbare Handlungen} und \RWbet{rechtlich erzwingbare Pflichten} wohl unterscheiden. Die letzteren pflegt man auch \RWbet{Rechtspflichten} oder \RWbet{Zwangspflichten, juridische} oder \RWbet{vollkommene Pflichten} zu nennen; und im Gegensatze von ihnen nennt man diejenigen Pflichten, zu deren Erfüllung uns Niemand zu zwingen berechtiget ist, \RWbet{freie, bloß ethische} oder auch \RWbet{unvollkommene Pflichten.} Die katholischen Sittenlehrer erinnern uns nun, daß wir nicht glauben sollen, als ob die freie Pflicht, wenn sie gleich eine unvollkommene heißt, immer geringer seyn müßte, als eine Zwangspflicht, der man den Namen einer vollkommenen gegeben. Zuweilen kann uns die Uebertretung einer bloß freien Pflicht vor Gott wirklich strafwürdiger machen, als die Uebertretung einer Zwangspflicht.
\end{aufza}

\RWpar{258}{Vernunftmäßigkeit}
Ueber die wissenschaftliche Erklärung des Begriffes, den das Wort \RWbet{Recht} bezeichnen soll, sind unsere Rechtslehrer nicht einig; wir müssen es also \RWbet{nicht sowohl aus ihrer Erklärung,} als vielmehr nur \RWbet{aus dem Gebrauche}, welchen sie von diesem Worte machen, aus den Beschaffenheiten, die sie den rechtlichen Handlungen beilegen, den widerrechtlichen absprechen \udgl\ , zu entnehmen suchen, welchen Begriff sie eigentlich mit jenem Worte verbinden. Gesetzt nun, daß es uns durch dieses Mittel (bei dessen Anwendung wir begreiflicher Weise nur auf diejenigen Lehren hinsehen dürfen, in welchen alle katholischen Rechtslehrer einstimmig sind) nicht gelingen sollte, uns diesen Begriff zu einer vollkommenen Deutlichkeit zu erheben, \dh\  die Bestandtheile, aus welchen er zusammengesetzt ist, alle genau zu unterscheiden und anzugeben: so würden wir darum doch an der Vernunftmäßigkeit der oben angeführten Lehren unmöglich zweifeln können. Schon der Umstand, daß so viele Gelehrte (nämlich nicht nur die katholischen, sondern auch alle übrigen Rechtslehrer) in diesen~\RWSeitenw{243}\ Behauptungen übereinstimmen, und hiebei überdieß gestehen, daß ihnen diese Sätze als Wahrheiten erscheinen, die durch die bloße Vernunft erkannt werden können, muß jedem Bescheidenen eine hinreichende Bürgschaft für ihre Richtigkeit seyn; auch wenn er durch sein eigenes Nachdenken den Grund, auf welchem sie beruhen, nicht gleich zu entdecken vermöchte. Wenn ich inzwischen doch versuchen soll, eine Erklärung von dem Begriffe des Rechtes zu geben, welche mir als die wahrscheinlichste vorkommt: so glaube ich, daß wir uns unter der Rechtlichkeit einer Handlung (so oft wir das Wort in jener eigenthümlichen in der Rechtswissenschaft gebräuchlichen Bedeutung nehmen) eben nichts Anderes, als diejenige Beschaffenheit derselben denken, die wirklich alle Rechtslehrer als eine den rechtlichen Handlungen ausnahmslos zukommende Beschaffenheit ansehen, nämlich \RWbet{daß wir verpflichtet sind, die Ausübung einer dergleichen Handlung zu dulden,} \dh\  durch keine Anwendung eines Zwanges zu verhindern, und dieß zwar selbst in dem Falle, wenn wir die Mittel dazu in den Händen hätten. Diese Beschaffenheit wird, wie mir däucht, zu dem Begriffe der rechtlichen Handlungen nicht bloß hinzugedacht, sondern sie liegt schon in diesem Begriffe selbst, ja sie erschöpfet den ganzen Inhalt desselben; dergestalt, daß man die rechtlichen Handlungen am Besten als solche erklären würde, \RWbet{die wir zu dulden} (durch keine Anwendung eines Zwanges zu hindern) \RWbet{verpflichtet sind}, so gut wir es etwa auch vermöchten, sie zu hindern. Handlungen, denen diese Beschaffenheit fehlet, wären als widerrechtliche zu bezeichnen. Bei Annahme dieser Erklärung läßt sich die Richtigkeit der Lehren, die wir im vorigen Paragraph aufgestellt haben, auf das Deutlichste einsehen.
\begin{aufza}
\item Daß uns in Hinsicht auf rechtliche Handlungen erstlich die Pflicht obliegt, denjenigen aus uns, der eine solche ausüben will, \RWbet{durch keinen Zwang zu hindern}, selbst wenn wir einsehen, daß diese Handlung nicht eben die beste ist, die er ausüben könnte; und wenn wir auch im Besitze von Zwangsmitteln sind, durch deren Anwendung wir sein Vorhaben gar wohl vereiteln könnten: das ist nach jener Erkärung eine bloß analytische Wahrheit; denn solche Handlungen, die wir zu dulden verpflichtet sind, nennt man recht\RWSeitenw{244}lich. Ein Gleiches gilt von der Behauptung, daß es uns in Betreff \RWbet{widerrechtlicher Handlungen} öfters erlaubt, ja sogar Pflicht ist, ihre Vollziehung, wenn gelindere Mittel nichts helfen, durch Zwang zu verhindern; denn wenn dieß bei einer gewissen Weise zu handeln in keinem Falle erlaubt ist: so werden wir sie eben deßhalb nicht widerrechtlich nennen.
\item Daß aber \RWbet{alle sittlich guten Handlungen}, wenigstens alle, die es nicht bloß subjectiv, sondern auch \RWbet{objectiv} sind, den \RWbet{rechtlichen} beigezählt werden sollen: ist bei dem angenommenen Begriffe der Rechtlichkeit unwidersprechlich; denn solche Handlungen sollen wir nicht nur nicht hindern, sondern nach aller Möglichkeit befördern; sie müssen also allerdings rechtlich heißen. Bei dieser Erklärung begreift es sich ferner sehr wohl, wienach auch Handlungen, die \RWbet{sittlich böse} sind, \RWbet{rechtlich} genannt werden können. Sittlich böse kann eine Handlung genannt werden, sobald sie das Wohl des Ganzen störet, ja auch nur dasselbe nicht in dem Maße befördert, wie dieß durch eine andere Handlung, die unter den obwaltenden Umständen eben so möglich ist, geschehen würde; und doch begreift man leicht, daß der Schade, den eine solche Handlung stiftet, nicht immer groß genug sey, um jenen Schmerz, jene Erbitterung, jene Gefahr des Mißbrauchs, und alle die übrigen nachtheiligen Folgen, welche die Anwendung eines Zwanges nach sich zieht, vollkommen aufzuwiegen. In einem solchen Falle werden wir also den Menschen die Anwendung eines Zwanges verbieten, \dh\  die Handlung für rechtlich erklären müssen. So ist es \zB\  nicht sittlich gut, aber doch rechtlich gehandelt, wenn wir demjenigen, der uns um die Ertheilung eines Rathes ersucht, denselben abschlagen; denn wenn wir den Menschen die Erlaubniß einräumen wollten, diejenigen, die ihnen einen verlangten Rath abschlagen, zu seiner Ertheilung durch Zwang zu verhalten: so würden wir offenbar nur noch mehr Uebel anrichten, wie denn \zB\  in den meisten Fällen nicht einmal der nächste Zweck, den man bei einem solchen Zwange hätte, die Erlangung eines aufrichtigen, brauchbaren Rathes, erreicht werden würde. Sehr richtig also ist die Bemerkung der christlichen Sittenlehrer, daß sich Niemand bloß damit beruhi\RWSeitenw{245}gen und begnügen dürfe, daß seine Handlungen rechtlich (in der juridischen Bedeutung) sind.
\item Daß es auch \RWbet{rechtlich erzwingbare Handlungen} gebe, heißt bei dem angenommenen Begriffe wieder nichts Anderes, als daß es Handlungen von einer solchen Beschaffenheit gebe, daß derjenige, der uns durch Zwang dazu verhalten will, geduldet, \di\ durch keinen Gegenzwang daran verhindert werden müsse. Und dieser Fall wird begreiflicher Weise bei allen solchen Handlungen eintreten, bei denen der Zwang, durch den uns Andere zu ihrer Vollziehung zu verhalten suchen, ein geringeres Uebel ist, als die Erlaubniß jenes Gegenzwanges wäre, der zur Abwehrung des ersten von uns oder Andern angewandt werden müßte. So ist \zB\  die Bezahlung einer Schuld am festgesetzten Tage eine rechtlich erzwingbare Handlung, weil es ein größeres Uebel wäre, wenn man den Schuldnern oder andern Personen, welche es sehen, daß ein Gläubiger Zwang anwenden will, erlauben wollte, diesen Zwang durch einen Gegenzwang abzuwehren, als das Uebel, welches aus seinem eigenen Zwange hervorgeht. Sehr richtig wird aber bemerkt, daß solche erzwingbare Handlungen in dem besondern Falle, wo der, dem das Zwangsrecht zusteht, keinen Gebrauch davon macht, \RWbet{nicht immer Pflichten} wären, daß sie zuweilen sogar \RWbet{pflichtwidrig} seyn können. Man kann nämlich auch in Betreff der Handlungen, welche dem Wohle des Ganzen nicht immer zuträglich sind, doch gewissen Personen das Recht, sie zu erzwingen, zugestehen, weil die Erlaubniß, sie an diesem Zwange durch einen Gegenzwang zu hindern, noch weit nachtheiliger wäre. Wenn nun derjenige, dem dieses Recht zu zwingen eingeräumt ist, glücklicher Weise keinen Gebrauch davon macht, \dh\  nicht wirklich zwingt: so würde man durch die Vollziehung einer solchen Handlung das Wohl des Ganzen zuweilen eben nicht befördern, sondern nur stören; und sie sind also in einem solchen Falle nicht Pflicht, sondern vielmehr pflichtwidrig. Wenn ich \zB\  versprochen habe, Jemandem an einem bestimmten Tage eine Summe Geldes auszuzahlen, von der ich erst jetzt sehe, daß sie in seinen Händen sehr gefährlich werden könnte: so bin ich wohl verpflichtet, sie auszuzahlen,~\RWSeitenw{246}\ wenn er auf die Erfüllung meines Versprechens dringt, \dh\  diese Auszahlung ist eine rechtlich erzwingbare Handlung; macht er jedoch keinen Gebrauch von seinem Rechte, \dh\  fordert er diese Auszahlung nicht: so bin ich gewiß nicht verpflichtet, das Geld auszuzahlen, sondern ich bin im Gegentheile verpflichtet, es noch zurückzubehalten. Hiemit ist nun der \RWbet{Unterschied zwischen freien und erzwingbaren Pflichten} von selbst gerechtfertiget. Sehr richtig ist endlich noch die Erinnerung, daß \RWbet{freie Pflichten}, wenn sie gleich \RWbet{unvollkommene} heißen, \RWbet{nicht immer für geringer geachtet werden sollen}, als die erzwingbaren, ob man sie gleich vollkommene genannt hat. Die Größe einer Verpflichtung hängt nämlich nicht von denselben Umständen ab, von denen es abhängt, ob sie zu einer Zwangspflicht erhoben werden kann oder nicht. Die Größe einer Verpflichtung hängt von der Menge und Wichtigkeit der Nachtheile ab, die aus der Unterlassung der Handlung entspringen, von der mehr oder weniger deutlichen Einsicht, die man von diesen Nachtheilen hat, von den geringeren oder größeren Beschwerden, welche mit ihrer Vollziehung verbunden sind \usw\ Ob man dagegen eine gewisse Pflicht zu einer Zwangspflicht erheben könne oder nicht, hängt lediglich von dem Umstande ab, ob die Uebel des Zwanges geringer sind, als jene, die aus der Erlaubniß eines Gegenzwanges hervorgehen würden. Da kann es sich denn immerhin treffen, daß die Unterlassung einer gewissen Handlung einen viel größeren Schaden hat, als die einer andern, daß also jene eine viel größere Pflicht ist als diese; und daß die Menschen gleichwohl jene zu dulden, diese aber durch Zwang zu verhindern verpflichtet, oder doch berechtiget sind; weil sich bei jener, vielleicht ihrer Natur nach, sehr schwer ein Zwang anbringen läßt, oder des üblen Beispieles wegen nicht zu verstatten ist \udgl\ , während dieß bei der letzteren nicht der Fall ist. So ist es \zB\  gewiß eine viel größere Pflicht, demjenigen, den man durch Abforderung einer ihm vorgestreckten Geldsumme, deren man leicht entbehren kann, zur Verzweiflung bringen würde, die Summe noch länger zu überlassen, als eine kleine Summe, die man sich selbst von einem Reichen ausgeborgt hatte, zur bestimmten Zeit wieder zurückzubezahlen; und dennoch ist die erstere Pflicht~\RWSeitenw{247}\ nur eine freie, die letztere aber bekanntlich eine Zwangspflicht. \Usw\
\end{aufza}

\RWpar{259}{Sittlicher Nutzen}
\begin{aufza}
\item Offenbar ist es sehr nützlich, uns auf diejenige Gattung menschlicher Handlungen, die wir \RWbet{zu dulden verpflichtet} sind, oder die das Christenthum \RWbet{rechtliche} Handlungen nennt, eigends aufmerksam zu machen; denn dieses dient uns
\begin{aufzb}
\item theils dazu, daß wir wissen, wann für uns selbst der Fall vorhanden sey, daß wir die Handlung eines Andern dulden sollen, daß wir also nicht durch einen am unrechten Orte angebrachten Zwang uns versündigen, indem wir Andere sowohl als auch vielleicht uns selbst durch einen angewandten Zwang verletzen.
\item theils auch dazu, daß wir solche Handlungen, die widerrechtlich sind, um desto weniger selbst ausüben, je deutlicher wir einsehen, daß sogar Andere uns an ihrer Ausführung zu hindern verpflichtet oder doch berechtiget seyn würden.
\end{aufzb}
\item Machte uns aber das Christenthum mit dem Begriffe rechtlicher Handlungen bekannt: so war es auch höchst nöthig, uns auf den \RWbet{Unterschied, der zwischen bloß rechtlichen und zugleich sittlichen Handlungen obwaltet}, aufmerksam zu machen, weil wir sonst leicht verleitet werden könnten, es für genug zu halten, wenn wir nur rechtlich handeln.
\item Eben so nützlich ist es, zu wissen, daß es auch \RWbet{rechtlich erzwingbare Handlungen} gebe. Dieß nämlich dient, daß wir bei solchen Handlungen uns jederzeit für den Fall vorsehen, wenn etwa Jemand auftreten sollte, der sie von uns erzwingt. Daß aber
\begin{aufzb}
\item rechtlich erzwingbare Handlungen gibt, die gleichwohl \RWbet{keine Pflichten} sind, ist uns nöthig zu wissen, damit wir nicht irriger Weise glauben, daß wir auf jeden Fall zu ihrer Ausübung verbunden wären. Daß es aber
\item \RWbet{freie} (\dh\  nicht rechtlich erzwingbare) \RWbet{Pflichten darum nicht immer geringer sind, als rechtlich}~\RWSeitenw{248}\ \RWbet{erzwingbare oder Zwangspflichten}, ist nöthig zu erinnern, damit wir nicht glauben, es habe nur wenig auf sich, eine Pflicht der ersteren Art zu übertreten; ein Irrthum, zu dem uns ihr Name: Unvollkommene Pflichten, sehr leicht verleiten könnte.
\end{aufzb}
\end{aufza}

\RWpar{260}{Wirklicher Nutzen}
Zwar ist nicht zu läugnen, daß die so wichtige Lehre vom Rechte auch unter uns Christen noch nicht vollkommen entwickelt ist; daß auch noch unter uns einzelne Rechtslehrer Lehrsätze aufstellen, deren Falschheit zum Theile selbst durch den gesunden Menschenverstand eingesehen werden kann; daß endlich diese Irrthümer unserer Rechtslehrer auch einen sehr nachtheiligen Einfluß auf unsere bürgerlichen Verfassungen und selbst auf das Betragen einzelner Bürger haben: dennoch, verglichen mit andern nichtchristlichen Völkern, ist unter uns Christen bei Weitem das Meiste geschehen. -- Bei allen heidnischen Völkern sind die Begriffe des Rechtes viel unentwickelter, und die Verstöße gegen dasselbe viel häufiger. Auch ist der wohlthätige Einfluß, den die heil.\ Schrift des alten sowohl als neuen Bundes, dann die Entscheidungen der christlichen Lehrer (Bischöfe) in den Kirchenversammlungen, wie auch die ganze Einrichtung, welche die in der Kirche aufgekommene Verfassung (Hierarchie) auf die Entwickelung besserer Rechtsbegriffe hatte, nicht zu verkennen. Viel Mehreres aber, als noch bisher geschehen ist, läßt sich erst von der Zukunft erwarten! -- Daß dieses nicht schon früher geschehen ist, daß sich bis jetzt noch so mancher Irrthum unter den christlichen Lehrern der Rechtswissenschaft erhalten hat, darüber kann gewiß Niemand die göttliche Vorsehung anklagen, da es dem Allwissenden allein bekannt ist, wie vieles Licht die Menschheit in jedem ihrer Jahrhunderte zu ertragen vermag, wie viel zu wissen ihr frommt, und was für sie nachtheilig wäre.

\RWpar{261}{Die Lehre des katholischen Christenthums von der Obrigkeit}
Zu den wichtigsten Pflichten, welche aus dem Zusammenseyn mehrerer Menschen entspringen, gehören, vermöge der~\RWSeitenw{249}\ Lehre des katholischen Christenthums, die \RWbet{Pflichten der Unterthanen zu ihren Obrigkeiten, und dieser gegen jene.}
\begin{aufza}
\item Das katholische Christenthum behauptet zuvörderst, daß es \RWbet{rechtmäßige,} \dh\  solche \RWbet{Obrigkeiten,} die man in ihrer Macht nicht stören soll, gebe.
\item Es versichert uns ferner, daß es \RWbet{nie ohne die Leitung einer besondern göttlichen Vorsorge} geschehe, wenn gewisse Personen zu dem Besitze einer rechtmäßigen Obergewalt über Andere gelangen, und diese dagegen in das Verhältniß der Untergebenen treten.
\item Es lehret weiter, daß \RWbet{die Gebote solcher rechtmäßigen Obrigkeiten}, wenn sie anders nicht etwas ganz offenbar Böses betreffen, \RWbet{die Untergebenen zum Gehorsame verpflichten}, und dieß zwar \RWbet{nicht bloß wegen der Gefahr der Strafe,} sondern auch selbst in dem Falle, wo man gar keine Ursache hätte, eine Entdeckung oder Bestrafung seines Ungehorsams zu fürchten.
\item Es bemerkt überdieß, daß die katholischen Christen auch \RWbet{geistliche Obrigkeiten} unter sich anerkennen sollen, und den Verordnungen dieser gleichfalls Gehorsam schuldig sind.
\item Wie aber den Untergebenen die Pflicht des Gehorsams obliegt: so haben die Vorgesetzten die Pflicht, nur Gutes und wahrhaft Gemeinnütziges zu gebieten, und werden Gott für den Gebrauch ihrer Macht Rechenschaft ablegen müssen.
\end{aufza}
\begin{RWanm} 
Hieher gehört auch die Frage, ob ein Gesetz einer Obrigkeit bloß durch die allgemeine Nichtbeobachtung desselben aufgehoben werde? Da es mir aber däucht, daß die katholischen Schriftsteller diese Frage nicht alle auf einerlei Weise entscheiden: so will ich nur die Meinung, welcher die Meisten beipflichten, welche auch mir die richtige scheint, anführen. Weil ein Gesetz doch nichts Anderes ist, als der erklärte Wille einer Obrigkeit, daß eine gewisse Handlung geschehe oder unterbleibe: so kann man eigentlich nur dann behaupten, daß ein Gesetz aufgehoben sey, \RWbet{wenn jener Wille des Gesetzgebers aufgehört hat}, oder doch wenigstens aufgehört hat, \RWbet{für uns erkennbar zu}~\RWSeitenw{250}\ \RWbet{seyn}. Dazu ist nun freilich nicht immer eine ausdrückliche Widerrufung nöthig; sondern es kann auch eine stillschweigende genügen. Allein bloß aus dem Umstande, daß ein gewisses Gesetz allgemein übertreten wird, ja selbst noch aus dem hinzukommenden zweiten Umstande, daß diese allgemein gewordene Uebertretung von Seite der Obrigkeit nicht gestraft wird, kann man noch nicht sofort schließen, daß diese ihren Willen geändert habe, und das Gesetz als aufgehoben betrachtet wissen wolle. Es könnte auch seyn, daß die Obrigkeit von jener Uebertretung entweder nicht gehörig unterrichtet ist, oder daß ihr die Mittel zur Bestrafung fehlen, oder daß sie, eben wegen der allzugroßen Anzahl derer, die zu bestrafen seyn würden, Bedenken trägt, die Strafe zu vollziehen. Nur dann allein also, däucht mir, werden wir aus einer allgemein straflosen Uebertretung eines Gesetzes den Schluß ziehen dürfen, daß es die Obrigkeit als aufgehoben betrachtet sehen wolle: wenn keine der erwähnten Möglichkeiten Statt hat, wenn es im Gegentheile einleuchtend ist, daß die Obrigkeit die Uebertretung hätte bestrafen können und müssen, hätte sie noch den Willen, daß ihr Gesetz ferner beobachtet werde. \end{RWanm}

\RWpar{262}{Historischer Beweis dieser Lehre}
\begin{aufza}
\item Daß es überhaupt rechtmäßige Herren und Obrigkeiten gebe, wird in der heil.\ Schrift an unzähligen Orten vorausgesetzt, wenn die Versuche derer, die damit umgingen, eine bestehende Obrigkeit ihrer Macht zu berauben und zu stürzen, als die strafwürdigsten Verbrechen dargestellt werden. \ZB\  die Empörung Absolon's.
\item Daß aber jede Obrigkeit, wenigstens jede rechtmäßige, \RWbet{nur durch die Leitung einer besonderen Fürsorge Gottes zu ihrer Macht gelangt sey}, will der Apostel Paulus offenbar sagen, wenn er sich ausdrückt (\RWbibel{Röm}{Röm.}{13}{1}): \erganf{Unterwerfe sich Jeder der obrigkeitlichen Gewalt; denn es gibt keine Obrigkeit, die nicht von Gott eingesetzet wäre; sondern so viele es ihrer gibt, so sind sie alle von Gott verordnet.}
\item Die Pflicht, einer solchen rechtmäßigen Obrigkeit \RWbet{zu gehorchen}, lehrte uns der Heiland selbst durch Wort und That (\RWbibel{Mt}{Matth.}{22}{21}): \erganf{Gebet dem Kaiser, was des Kaisers ist}; und (\RWbibel{Mt}{Matth.}{17}{27}) als er die Tempelsteuer ent\RWSeitenw{251}richtete; ingleichen (\RWbibel{Mt}{Matth.}{26}{52}) bei seiner Gefangennehmung. Ein Gleiches lehrten und thaten auch die Apostel. So schreibt der heil.\ Petrus (\RWbibel{1\,Petr}{1\,Petr.}{2}{13}): \erganf{Unterwerft euch aus Gehorsam gegen den Herrn (Jesum Christum) jeder menschlichen Ordnung (\RWgriech{p'ash| anjrwp'inh| kt'isei}) es sey dem Könige, als der die höchste Gewalt hat, oder seinen Statthaltern, die zur Bestrafung der Bösen und zur Belohnung der Guten von ihm eingesetzt sind; denn also ist es der Wille Gottes.} Und der heil.\ Paulus (\RWbibel{Eph}{Ephes.}{6}{5}): \erganf{Ihr Knechte, gehorchet eurem Herrn, nicht als bloße Augendiener, die nur dem Herrn gefallen wollen, sondern als Knechte Christi, die Gottes Willen von Herzen erfüllen}, -- woraus erhellet, daß die Gebote der Obrigkeit \RWbet{auch im Gewissen} verbinden und auch selbst dort, wo keine Gefahr einer Entdeckung und Bestrafung drohet. Daß aber diese Pflicht die erwähnte Einschränkung habe, sagen die Apostel (\RWbibel{Apg}{Apostelg.}{4}{19}): \erganf{Ihr möget selbst urtheilen, ob es recht wäre vor Gott, wenn wir euch mehr, denn Gott, gehorchen wollten.}
\item Daß wir katholische Christen \RWbet{auch eine geistliche Obrigkeit anerkennen sollen,} und daß dieser das Recht zukomme, uns Gebote zu geben, die im Gewissen verbinden, wird tiefer unten bei dem Sacramente der Weihe gezeigt werden.
\item Daß alle Vorgesetzte von dem Gebrauche ihrer Macht Gott Rechenschaft ablegen müssen; beweiset \zB\  \RWbibel{Ps}{Psalm}{82}{}: \erganf{Jehova steht in der Versammlung der Fürsten (der Richter). Er hält Gericht über den Richter selbst: Wie lange noch richtet ihr unrecht und sehet auf die Person des Schuldigen? -- Schaffet Recht Wittwen und Waisen! sprecht den Bedrängten los, befreit die Unschuld, entreißt sie der Hand des Frevlers! -- Sie achten nicht darauf, werden nicht weiser, wandeln die Pfade der Finsterniß ungestört fort. Schon alle Grundfesten des Landes sind erschüttert. Wohl habe ich euch erklärt für meine Stellvertreter, wohl meine Söhne euch genannt: doch sollt ihr sterben, wie der Gemeinste; all ihr Tyrannen sollt vergehen! -- Ja, stehe auf, o Gott! und richte den Erdkreis; denn alle Völker der Erde sind doch nur dein Erbe!} -- So schreibt auch der heil.\ Paulus (\RWbibel{Kol}{Koloss.}{4}{1}):~\RWSeitenw{252}\ \erganf{Ihr Herren, betraget euch gegen eure Knechte, wie es gerecht und billig ist; bedenkend, daß auch ihr selbst einen Herrn im Himmel habt, (\RWbibel{Eph}{Ephes.}{6}{9}) bei dem kein Ansehen der Person gilt.}
\end{aufza}

\RWpar{263}{Vernunftmäßigkeit}
\begin{aufza}
\item Unter einer \RWbet{Obrigkeit} (in dieses Wortes weitester Bedeutung) verstehen wir \RWbet{eine jede} (es sey nun einzelne, oder aus der Verbindung mehrerer Menschen bestehende moralische) \RWbet{Person, die Anderen} (welche wir ihre Untergebenen nennen) \RWbet{die Verbindlichkeit zumuthet, etwas zu thun, bloß darum, weil sie es ihnen als ihren Willen} (als ihr Gebot oder Gesetz) \RWbet{erklärt hat}; zumal, wenn sie \RWbet{auch einige Macht hat, wenigstens eine und die andere dieser Handlungen, im Falle eines Widerstandes, von ihnen zu erzwingen.} Sind jene andern Menschen (die Untergebenen) \RWbet{wirklich verpflichtet}, jenen Geboten (wenn auch nicht allen, doch wenigstens einigen) Folge zu leisten: so heißt die Obrigkeit eine \RWbet{rechtmäßige}. Es gibt nun allerdings rechtmäßige Obrigkeiten; denn es gibt mancherlei Verhältnisse unter uns Menschen, um derentwillen es dem Wohle des Ganzen sehr zusagt, und also Pflicht ist, die Macht, die der eine Theil hat, ihm zu belassen, \zB\  wenn er sie größtentheils gut anwendet, oder wenn es uns nicht möglich ist, sie ihm zu nehmen, ohne ein ungleich größeres Uebel zu stiften, \udgl\ 
\item Da es, wie sich begreifen läßt, eine Sache von der größten Wichtigkeit für das gemeine Beste ist, von welcher Beschaffenheit die Personen sind, welche zu dem Besitze einer weit ausgebreiteten obrigkeitlichen Gewalt gelangen, besonders wenn sie zur Gattung der rechtmäßigen gehört: so ist es auch schon aus bloßen Gründen der Vernunft gewiß, daß die Besetzung solcher obrigkeitlichen Stellen ein Gegenstand \RWbet{einer besondern göttlichen Vorsehung} seyn müsse.
\item Was aber eine rechtmäßige Obrigkeit gebietet, das ist, der Regel nach, auch Pflicht. Betrifft es eine schon an sich selbst gemeinnützige Handlung: so ist kein Zweifel, daß ihre Ausübung durch den erklärten Willen der Obrigkeit zu~\RWSeitenw{253}\ einer um desto bestimmteren Pflicht werden müsse. Allein selbst wenn die Handlung, die eine Obrigkeit von uns verlangt, dem Wohle des Ganzen an sich nicht zuträglich ist, kann doch der Umstand, daß sie geboten worden ist, ihr eine gewisse Zuträglichkeit geben und machen, daß ihre Unterlassung nun ein noch größeres Uebel als ihre Vollziehung wäre.\par
Dergleichen Fälle sind:
\begin{aufzb}
\item Wenn zu befürchten ist, daß eine Obrigkeit, falls wir ihrem Befehle nicht gutwillig folgen, zur Anwendung gewisser Zwangsmittel, oder auf jeden Fall doch zu solchen Schritten sich entschließen würde, wodurch das Wohl des Ganzen viel mehr beeinträchtiget würde, als durch den Gehorsam selbst. So wäre \zB\  ohne Zweifel die Schuldigkeit eines Dieners, in einer Sache, die nicht von bedeutender Wichtigkeit ist, seinem Herrn zu gehorchen, wenn dieser die Handlung von ihm unter der Drohung, ihn zu erschießen, begehrte.
\item Wenn eine Obrigkeit sonst viele gute und dem Wohle des Ganzen zusagende Gebote zu geben pflegt, und es ist zu besorgen, daß unser Beispiel des Ungehorsames Andere verleiten werde, ihr auch in Fällen, wo Gehorsam besser ist, nicht zu gehorchen.
\item Wenn sich vorhersehen läßt, daß, wie wir der einen Obrigkeit den Gehorsam aufkündigen, gleich eine andere sich erheben werde, die ihre Macht noch ärger, als die andere mißbraucht. Wie \zB\  wenn durch den Umsturz einer zwar eben nicht fehlerfreien Verfassung leicht eine völlige Anarchie, \dh\  ein solcher Zustand eintreten könnte, wo gar keine Macht im Lande bestehet, die stark genug ist, die Angriffe, die sich der Einzelne auf den Einzelnen erlaubt, zu hindern. \Usw\
\end{aufzb}
Aus allem diesem ersieht man zur Genüge, daß das katholische Christenthum Recht hat, uns den Gehorsam gegen Obrigkeiten als eine unserer \RWbet{wichtigsten Pflichten} einzuschärfen. Daß aber diese Pflicht des Gehorsams zuweilen auch eine Ausnahme habe, und dieß zwar dann, wenn die befohlene Handlung offenbar böse ist, \dh\  wenn wir durch unseren Gehorsam des Uebels viel mehr anrichten würden,~\RWSeitenw{254}\ als durch das Aergerniß der Uebertretung, ist freilich nicht zu läugnen.
\item Ganz auf denselben Gründen, auf welchen die Pflicht des Gehorsames gegen Obrigkeiten überhaupt beruhet, beruhet auch die Pflicht, \RWbet{geistlichen Obrigkeiten} zu gehorchen.
\item Daß endlich jede Obrigkeit den Gebrauch ihrer Macht vor Gott zu verantworten habe, ergibt sich aus der Gerechtigkeit Gottes von selbst.
\end{aufza}

\RWpar{264}{Wirklicher Nutzen}
Da der sittliche Nutzen der eben angeführten Lehren schon aus Betrachtung ihrer Vernunftmäßigkeit einleuchtend genug wird: so wollen wir nur noch den \RWbet{wirklichen Nutzen}, den diese Lehren hervorgebracht haben müssen, mit wenigen Worten berühren. Wie viele und über alle Berechnung große Vortheile hat nicht der Glaube an die Pflicht des Gehorsames gegen weltliche sowohl als geistliche Gesetze und Anordnungen bewirket! Wohin wäre es gekommen, wenn diese Pflicht jemals von einem beträchtliche Theile der Christen verkannt worden wäre! Und gleichwohl, wie viele Versuchung, diese Pflicht, besonders in Betreff der weltlichen Gesetze, zu verkennen, gab nicht die Unvollkommenheit derselben durch alle christlichen Jahrhunderte hindurch! Wie oft traten auch wirklich einzelne Männer unter den Christen auf, die diese Pflicht läugneten, oder doch behaupteten, daß es nur in soferne Pflicht sey, die Gebote der weltlichen Obrigkeit zu beobachten, als mit der Uebertretung derselben die Gefahr einer Strafe, und somit eines eigenen Schadens verbunden sey, \dh\  daß die Gesetze der Obrigkeit \RWlat{leges mere poenales} wären. Doch solche verderbliche Irrthümer hat die katholische Kirche nie weit um sich greifen lassen.

\RWpar{265}{Die Lehre des katholischen Christenthums von den Versprechungen und Gelübden}
Unter Pflichten, die erst \RWbet{aus unseren eigenen Handlungen} entspringen, macht uns das katholische Christenthum~\RWSeitenw{255}\ besonders aufmerksam auf diejenigen, die aus \RWbet{Versprechungen} oder \RWbet{Verträgen} hervorgehen.
\begin{aufza}
\item Haben wir Jemand versprochen, etwas zu thun, entweder unbedingt oder bedingnißweise und vermittelst eines Vertrages, \dh\  nur unter der Bedingung, daß auch er selbst etwas thue: so sind wir im ersten Falle unbedingt, im zweiten unter der Bedingung, wenn auch der Andere das Seinige thut, verpflichtet, es zu erfüllen.
\item Doch hat diese Pflicht auch ihre \RWbet{Ausnahmen;} und zwar
\begin{aufzb}
\item wenn es uns \RWbet{unmöglich} wird, zu leisten, was wir versprochen hatten;
\item wenn uns derjenige, dem wir's versprochen hatten, von der Erfüllung selbst \RWbet{lossagt;}
\item wenn es etwas \RWbet{schlechterdings Böses} war, wovon auch derjenige, welchem wir das Versprechen gaben, gewußt, daß es unerlaubt sey.
\end{aufzb}
\item Auch \RWbet{Versprechungen, welche wir Gott thun, \dh\  Gelübde}, erzeugen die Pflicht der Erfüllung.
\item Die Erfüllung eines nur Gott gethanen Gelübdes \RWbet{hört auf}:
\begin{aufzb}
\item wenn sie uns \RWbet{unmöglich} ist;
\item wenn wir statt des Gelobten etwas Anderes thun, davon wir deutlich einsehen, daß es \RWbet{weit besser} ist.
\end{aufzb}
\item Doch sollen wir, wo möglich, in solchen Fällen \RWbet{nie bloß unserem eigenen Urtheile trauen}, sondern die Sache der Entscheidung eines vernünftigen Gewissensrathes anheimstellen.
\end{aufza}

\RWpar{266}{Historischer Beweis dieser Lehre}
\begin{aufza}
\item Daß man in der katholischen Kirche die Erfüllung gethaner \RWbet{Versprechungen}, und die Beobachtung eingegangener \RWbet{Verträge} für eine Pflicht erkläre, bedarf keines Beweises.
\item Gewiß ist aber auch, daß man \RWbet{von dieser Pflicht entbinde}, wenn einer der eben angegebenen Fälle sich ein\RWSeitenw{256}stellt. Der letzte Fall ist besonders aus dem einstimmigen Urtheile aller christlichen Rechtslehrer entschieden.
\item Daß aber auch \RWbet{Gelübde} verpflichten, beweisen  \RWbibel{Num}{4\,Mos.}{30}{3}: \erganf{Wenn Jemand vor Gott ein Gelübde thut, oder sich eidlich verpflichtet: so soll er sein Wort nicht wieder zurücknehmen, sondern, was er versprochen hat, soll er erfüllen.} So heißt es auch \Ahat{\RWbibel[{Psalm 50,14.}]{Ps}{}{50}{14}}{Psalm 49,14.}: \erganf{Opfere Gott Dankopfer und bezahle dem Höchsten die Gelübde.}
\item Daß aber auch Gelübde \RWbet{nicht verbinden}, wenn sie etwas \RWbet{Unmögliches} betreffen, oder wenn wir auch nur etwas \RWbet{entschieden Besseres} statt des Gelübdes thun wollen, wurde von den katholischen Sittenlehrern von jeher gelehrt; doch mit der Einschränkung,
\item daß man wo möglich sich nach Entscheidung seines Gewissensrathes richte.
\end{aufza}

\RWpar{267}{Vernunftmäßigkeit und sittlicher Nutzen}
\begin{aufza}
\item Sehr begreiflich ist, wie aus \RWbet{Versprechungen} und \RWbet{Verträgen} die Pflicht ihrer Erfüllung entspringe. Wäre es nämlich nicht Pflicht, dasjenige zu thun, wozu man sich in einem Versprechen oder Vertrage anheischig gemacht hat: so würde sich auch Niemand auf dergleichen Versprechungen oder Verträge verlassen können, was doch der menschlichen Gesellschaft überaus viele Vortheile gewährt.
\item Daraus ergibt sich aber auch, in welchen Fällen die Erfüllung eines solchen Versprechens oder Vertrages Pflicht sey. Wir sind nämlich zu einer versprochenen Handlung verpflichtet, nicht nur, wenn diese Handlung schon an sich selbst dem Wohle des Ganzen zuträglich ist (in welchem Falle wir eigentlich auch schon ohne Versprechen zu ihrer Ausübung verpflichtet wären); sondern auch wenn sie an sich nicht eben die allerzuträglichste ist, aber doch weniger schadet, als jene üblen Folgen, welche die Verletzung des gegebenen Wortes hätte. Zu den letzteren gehört der Schaden, den derjenige, welchem wir das Versprechen geleistet, dadurch erfährt, daß wir ihm unser Wort nicht halten; die Rache, die er vielleicht darüber nehmen wird; das Aergerniß, das unsere Wortbrüchig\RWSeitenw{257}keit andern Menschen gibt, \usw\ Hieraus erhellet, daß auch Handlungen, welche dem allgemeinen Wohle an sich nicht vortheilhaft sind, die folglich, abgesehen von dem Versprechen, unerlaubt wären, durch dasselbe zur Pflicht werden können. Habe ich \zB\  versprochen, Jemandem ein gewisses Geld zu geben: so werde ich es thun müssen, wenn mir auch hintenher ein viel gemeinnützigerer Gebrauch dieses Geldes einfallen sollte; es müßte denn seyn, daß die Nachtheile der Erfüllung meines Versprechens so beträchtlich wären, daß sie das Aergerniß, und alle die übrigen üblen Folgen einer Wortbrüchigkeit entscheidend überwiegen. Daß aber insbesondere
\begin{aufzb}
\item die auf einem Versprechen oder Vertrage beruhende Verbindlichkeit aufgehoben sey, wenn wir \RWbet{ganz außer Stand} sind, sie zu erfüllen, versteht sich von selbst.
\item Auch in dem Falle, wo diejenigen, denen wir unser Versprechen geleistet, und die durch die Nichterfüllung desselben allein verletzt werden könnten, uns davon \RWbet{lossagen}, ist kein vernünftiger Grund, der uns noch ferner binden könnte, vorhanden.
\item Endlich ist auch gewiß, daß eine Handlung, \RWbet{deren Unerlaubtheit derjenige, der sich dieselbe versprechen ließ, schon damals, als sie ihm versprochen ward, einsehen konnte}, durch kein auch noch so feierliches Versprechen (mithin auch durch keinen Vertrag) zur Pflicht werden könne. Der Grund ist, weil es dem allgemeinen Wohle nicht nachtheilig, sondern vielmehr zuträglich ist, wenn man dieser Art von Versprechungen keine verbindende Kraft zugestehet; denn so läßt sich erwarten, daß manche böse That, zu der sich die Menschen bereits verabredet hatten, durch das erwachte Gewissen des Einen oder des Andern noch unterbleiben werde, und die Verlegenheit, in welche der Böse geräth, wenn die versprochene Beihülfe zu einer bösen That ausbleibt, ist eine der gerechtesten Strafen desselben; und wenn er im Voraus weiß, daß er sich auf dergleichen Versprechungen nie verlassen könne: so fällt auch größtentheils die Versuchung weg, Andere zu solchen Versprechungen zu verleiten.~\RWSeitenw{258}
\end{aufzb}
\item Eben so leicht läßt sich auch die Behauptung rechtfertigen, daß selbst \RWbet{Gelübde} Pflichten erzeugen, wo nämlich die Handlung, die sie betreffen, \RWbet{dem Wohle des Ganzen schon an sich zuträglich und also sittlich gut ist}; denn eben deßhalb wäre ja die Vollziehung dieser Handlung schon ohne das Gelübde entweder eine Pflicht, oder doch wenigstens etwas Verdienstliches. Wenn nun Gott darum, weil wir die Handlung gelobten, unsere Verbindlichkeit dazu erhöht, wenn er sie \zB\  aus einer bloß verdienstlichen Handlung zu einer Schuldigkeit macht, durch deren Uebertretung wir uns Strafe zuziehen würden; oder wenn er in dem Falle, wo die Handlung ohnehin Pflicht war, die Strafe, die wir bei ihrer Unterlassung zu befürchten hatten, erhöhet: so wird hiedurch bewirkt, daß wir die Handlung nur um so gewisser vollziehen, und folglich wird das Wohl des Ganzen befördert.
\item Dagegen kann eine Handlung, von der wir entweder schon damals, als wir sie gelobten, oder auch nur hintenher einsehen, \RWbet{daß sie dem Wohle des Ganzen nicht ganz förderlich ist}, durch kein auch noch so feierliches Gelübde zur Pflicht werden. Hier fällt nämlich jener Grund, weßhalb Versprechungen, die wir den Menschen gethan haben, zuweilen auch dann noch verpflichten, wenn die Erfüllung keine an und für sich gemeinnützige That ist, hinweg. Dieser Grund war, weil Menschen sich widrigenfalls auf unsere Versprechungen nie verlassen könnten, und zu Schaden kämen, wenn sie auf sie gebaut. Dieses ist aber begreiflicher Weise niemals der Fall bei Gott. Das katholische Christenthum erlaubt daher wirklich die Auflösung solcher Gelübde.
\item Doch wird sehr weislich verlangt, daß wir uns das Urtheil, ob ein solcher Fall wirklich vorhanden sey, so fern es möglich ist, \RWbet{von unserem Seelenfreunde bestätigen} lassen. Dieses ist eine überaus zweckmäßige Verfügung, weil wir im widrigen Falle, wenn die Sache nur unserem eigenen Ermessen anheim gestellt bliebe, aus bloßer Trägheit manches Gute, das wir uns vorgenommen hatten, nicht vollenden würden, ohne etwas in Wahrheit Besseres statt dessen zu verrichten; weil es ferner auch eine sehr nöthige Beruhigung für unser Gewissen gewährt, wenn wir uns sagen können, daß wir in dieser Sache nicht unserem eigenen Ur\RWSeitenw{259}theil, sondern dem Rathe eines wohlunterrichteten Mannes gefolgt sind.
\end{aufza}
\begin{RWanm} Dieß gilt jedoch nur bei solchen Gelübden, \RWbet{bei welchen nicht zugleich auch den Menschen etwas versprochen wurde;} daher das Gelübde der Enthaltsamkeit, und andere dergleichen Gelübde, die man \zB\  dem geistlichen Stande abfordert, ihre verbindliche Kraft behalten, auch wenn man hintenher einsehen sollte, daß man besser gethan hätte, wenn man sie nicht abgelegt hätte. Hier wurde zugleich auch Menschen versprochen, worauf sie gerechnet haben, und dem gemäß sie ihre Anstalten getroffen. 
\end{RWanm}

\RWpar{268}{Die Lehre des katholischen Christenthums von der besonderen Art, wie wir Menschen das Sittengesetz erkennen, und von den Verbindlichkeiten, die wir in dieser Rücksicht haben}
\begin{aufza}
\item Das katholische Christenthum gibt zu, daß es uns Menschen \RWbet{nicht immer möglich} sey, mit völliger Richtigkeit und mit ungezweifelter Gewißheit zu erkennen, was in einem jeden vorhandenen Falle dem Sittengesetze an und für sich gemäß und also objectiv gut sey.
\item Eben deßhalb, sagt es, ist es Pflicht des Menschen, daß er \RWbet{wenigstens strebe}, zu der möglich größten Gewißheit hierüber zu gelangen. Er soll daher
\begin{aufzb}
\item \RWbet{seine sittliche Urtheilskraft überhaupt durch Uebung ausbilden;}
\item sich eine möglichst genaue \RWbet{Kenntniß aller natürlichen, göttlichen, und menschlichen Gesetze verschaffen};
\item \RWbet{auf seine jedesmaligen Verhältnisse genau aufmerksam seyn}; und
\item in jedem einzelnen Falle, bevor er sich zu einer bestimmten Handlung entschließt, so viel \RWbet{Bedachtsamkeit} und \RWbet{Ueberlegung} anwenden, als es theils an sich selbst möglich ist, theils die geringere oder größere Wichtigkeit des vorhandenen Falles verdient.
\end{aufzb}
\item Wenn er am Ende dieser Untersuchung sich zu demjenigen entschließt, \RWbet{was ihm mit größter Wahrschein}\RWSeitenw{260}\RWbet{lichkeit als dem Sittengesetze gemäß erschienen ist}: so ist nicht nur wahrscheinlich, sondern \RWbet{gewiß}, daß er gut und verdienstlich gehandelt hat, gesetzt auch, daß er sich durch einen unverschuldeten (\dh\  nicht durch einen früheren sittlichen Fehler herbeigeführten) Irrthum in dieser Untersuchung getäuscht haben sollte. Seine Handlung ist, wenn auch vielleicht nicht immer objectiv, doch \RWbet{jedesmal subjectiv gut}.
\item Ueber die Frage, ob derjenige, der aus mehreren Handlungen, welche ihm als dem Gesetze gemäß erscheinen, \RWbet{nicht gerade diejenige wählte, die ihm unter allen als die dem Sittengesetz gemäßeste erscheinet} (\RWlat{quod est perfectissimum, vel sententiam tutiorem}) -- \RWbet{eine strenge Pflicht verletze} und somit straffällig werde, herrscht keine allgemeine Uebereinstimmung unter den katholischen Sittenlehrern.
\item Wenn Jemand \RWbet{nicht gewiß} ist, ob eine Handlung ihm durch das Sittengesetz \RWbet{geboten}, aber doch \RWbet{sicher} ist, daß sie ihm \RWbet{nicht verboten} sey: so soll er sie lieber \RWbet{vollziehen}, besonders wenn es einen an sich sehr wichtigen Gegenstand, \zB\  sein Seelenheil, die Verwaltung eines Heiligungsmittels, \udgl\  betrifft.
\item Wenn er im Gegentheile \RWbet{nicht gewiß} ist, ob die Handlung durch das Sittengesetz \RWbet{verboten,} aber doch\RWbet{ sicher} ist, daß sie \RWbet{nicht geboten} sey: so soll er sie lieber \RWbet{unterlassen}.
\item Wenn ihm von einer Handlungsweise \RWbet{Beides zweifelhaft} ist, ob sie durch das Sittengesetz geboten oder wohl gar verboten sey: so besteht eben um dieses Zweifels willen \RWbet{gar keine Pflicht} für ihn, zu handeln; sondern er thut vielmehr besser, die zweifelhafte Handlung ganz zu unterlassen.
\item \RWbet{Wenn wir nicht gleich entscheiden können, welches unter zwei oder mehreren Gesetzen,} die sich auf einen gewissen Fall zu beziehen scheinen, \RWbet{das wirklich verbindende} sey: so werden uns folgende Regeln gegeben:
\begin{aufzb}
\item Dasjenige Gesetz, durch dessen Befolgung in dem vorhandenen Falle \RWbet{das Wohl des Ganzen mehr befördert} wird, ist das verbindliche.~\RWSeitenw{261}
\item Wenn wir von zwei Gesetzen, die so beschaffen sind, daß durch die Uebertretung des Einen das Wohl des Ganzen nur \RWbet{nicht befördert}, durch die Uebertretung des Andern aber sogar \RWbet{verletzet} wird, \RWbet{nur Eines befolgen können}: so ist das \RWbet{zweite} zu befolgen.
\item Wenn zwei Gesetze, deren das Eine \RWbet{Zweck,} das Andere \RWbet{Mittel} ist, mit einander im Widerspruche stehen: so ist bloß \RWbet{Ersteres} verbindlich.
\item Ein Gesetz, das etwas zu thun \RWbet{verbietet}, ist im Zusammenstoße mit einem andern, das etwas zu thun \RWbet{gebietet, meistens} das verbindende.
\end{aufzb}
\end{aufza}

\RWpar{269}{Vernunftmäßigkeit und sittlicher Nutzen}
\begin{aufza}
\item Es ist wohl nur ein bloßer Streit um Worte, wenn einige Gelehrte, \zB\  Kant, Mutschelle \uA\ die Möglichkeit eines irrenden Gewissens läugneten. Daß wir in jedem Falle völlig gewiß seyn können, wir handeln gut und löblich, wenn unser Bewußtseyn uns sagt, daß wir dasjenige thun, was unserer Einsicht nach dem Sittengesetze gemäß ist, ist freilich wahr, und wird vom Christenthume selbst (Nr.\,3.) eingestanden. Es wird nur behauptet, daß wir uns in der Beurtheilung, ob diese oder jene Handlung wirklich dem Sittengesetze gemäß sey, irren können. Und daß dieß wirklich nicht selten geschehe, beweiset uns unsere eigene Erfahrung, wenn wir eine gewisse Handlungsweise, die wir zu Einer Zeit als dem Sittengesetze gemäß ansahen, zu einer andern als nicht übereinstimmend mit demselben erkennen. So wahr es nun ist, daß wir uns in der Beurtheilung der objectiven Güte einer Handlung irren können; so wichtig ist es, dieß nie ganz außer Augen zu lassen. Denn könnte ein Mensch glauben, daß er in jedem Falle mit Richtigkeit und gleich auf der Stelle zu entscheiden vermöge, was hier geschehen oder nicht geschehen solle: so würde er
\begin{aufzb}
\item nie mit besonderem Fleiße untersuchen, nie Anderer Meinungen zu Rathe ziehen, und folglich in der That nur um so öfterer sehr unrichtig entscheiden.
\item Diejenigen, die ihm nicht beistimmen wollten, müßte er entweder für Thoren oder für Bösewichter halten, die~\RWSeitenw{262}\ der Erkenntniß der Wahrheit absichtlich widerstreben, oder sich doch so stellen, als ob sie nicht für Pflicht hielten, was sie doch in der That dafür erkennen.
\end{aufzb}
\item Die Richtigkeit der Vorschriften, die in dem zweiten Artikel aufgestellt werden, leuchtet von selbst ein; so wie auch der große Nutzen, den ihre Befolgung hervorbringt.
\item So nöthig es aber ist, uns auf die Möglichkeit eines Irrthums in unserem sittlichen Urtheile aufmerksam zu machen: so wahr und beruhigend ist die Versicherung, daß jeder Irrthum, den wir in Beurtheilung dessen, was dem Gesetze gemäß sey, in einem einzelnen Falle begehen, die Verdienstlichkeit unserer Handlung gar nicht vermindern, wenn anders wir diesen Irrthum nicht durch irgend eine andere sittlich böse That, \zB\  durch Unaufmerksamkeit auf den sittlichen Unterricht \udgl\ , veranlasset haben.
\item Die Frage, ob ein Mensch jedesmal, so oft er unter mehreren Handlungen, die ihm dem Sittengesetze gemäß zu seyn scheinen, nicht eben diejenige auswählt, die ihm als die gemäßeste erscheint, eine strenge Pflicht verletze und wirklich straffällig werde, läßt sich aus bloßen Vernunftgründen weder bejahend, noch verneinend entscheiden. So wahr es auch ist, daß derjenige, der auf die oben beschriebene Weise handelt, das Wohl des Ganzen weniger befördert, als wenn er nur immer die Handlung gewählt hätte, die unter allen ihm dem Gesetze die gemäßeste geschienen: so ist sein Fehler doch ungleich geringer, als der Fehler dessen, der eine dem Sittengesetze geradezu widersprechende Handlung vollzieht; und nur Gott allein mag wissen, ob es dem Wohle des Ganzen zuträglich seyn würde, ihn dafür positiv zu strafen. Wie groß würde dann die Menge der Bestraften!
\item Wenn wir das Wort \RWbet{Gewißheit} in seinem strengsten Sinne nehmen, in dem es einer jeden auch noch so hohen Wahrscheinlichkeit entgegengesetzt werden muß: so können wir freilich nie gewiß davon seyn, daß eine Handlung keinem Gesetze (weder einem göttlichen noch menschlichen) widerspreche; sondern wir könnten hievon jedesmal nur mit einem sehr hohen Grade von Wahrscheinlichkeit (ein Grad, der gemessen wird durch einen Bruch, der von der Einheit~\RWSeitenw{263}\ um einen unberechenbar kleinen Theil abweicht) versichert seyn; oder die sogenannte \RWbet{moralische Gewißheit} ist es, die das katholische Christenthum unter dem Worte Gewißheit allemal versteht. Dieses vorausgesetzt, müßten wir also die Regel, welche in Nr.\,5. aufgestellt wird, eigentlich so ausdrücken, wenn wir uns völlig streng ausdrücken wollten: Wenn Jemand ungewiß ist, ob eine bestimmte Handlung in irgend einem Gesetze geboten sey, aber doch sicher ist, daß sie nirgends verboten sey: so müßte er untersuchen, welche Folgen die Ausübung dieser Handlung, auch abgesehen davon, daß sie vielleicht durch ein Gesetz geboten ist, für das Ganze haben würde. Sind diese Folgen nicht nachtheilig: so wäre sie auszuüben, weil ihre Ausübung dann doch den Nutzen brächte, daß sie die Achtung des Gesetzes bei ihnen und bei Andern befördert. Hat sie dagegen auch einige nachtheilige Folgen: so wäre der Vortheil zu berechnen, den die durch ihre Ausübung an den Tag gelegte Achtung des Sittengesetzes hervorbringt, und der wahrscheinliche Schaden der Handlung abzuziehen. Wäre der Rest positiv: so wäre die Handlung auszuüben; im Gegentheile wäre sie zu unterlassen. -- Vergleicht man nun diese mit so viel Worten ausgedrückte Regel mit der kurzen Vorschrift, die das katholische Christenthum gibt: so sieht man, daß die letztere zwar keine völlige Genauigkeit, und eben deßhalb auch keine strenge Allgemeingültigkeit hat, aber doch so bestimmt und brauchbar für die meisten Fälle ist, als eine einfache und faßliche Regel über einen solchen Gegenstand nur immer seyn kann.
\item Dasselbe gilt von der Regel Nr.\,6.
\item Wenn wir zweifelhaft sind, ob eine Handlung geboten oder verboten sey, oder, allgemein zu reden: zweifelhaft sind, ob sie das Wohl des Ganzen mehr befördere oder störe (\dh\  wenn der Grad der Wahrscheinlichkeit für das Eine so groß als für das Andere ist): so sollen wir eben um dieses Zweifels willen die Handlung unterlassen; denn es ist jetzt kein Grund vorhanden, der uns zu ihrer Ausübung auffordern könnte; ja wir würden sogar leichtsinnig scheinen können, daß wir uns in die Gefahr der Uebertretung eines Gesetzes begeben.~\RWSeitenw{264}
\item Zwei oder mehrere einander widersprechende Vorschriften (\dh\  solche, die zwei oder mehrere nicht zugleich mögliche Handlungen fordern) können für Einen und denselben Fall nie wirklich verbindlich seyn; denn sagen, daß eine Vorschrift wirklich verbindlich sey für einen Fall, heißt eben so viel, als sagen, daß es in diesem Falle wirkliche Pflicht sey, so zu handeln. Behaupten also, daß für einen und eben denselben Fall mehrere einander widersprechende Vorschriften wirklich verbindlich wären, hieße behaupten, daß wir zugleich verpflichtet und auch nicht verpflichtet wären, etwas zu thun; welches ein offenbarer Widerspruch ist. Daß es gleichwohl oft scheint, zwei oder mehrere einander widersprechende Gesetze wären für einen und ebendenselben Fall verbindlich, rührt aus einem doppelten Grunde:
\begin{aufzb}
\item aus der \RWbet{Zweckwidrigkeit so mancher menschlicher Gebote;} indem von menschlichen Herren und Obrigkeiten zuweilen so zweckwidrige Befehle und Gebote aufgestellt werden, daß sie eben um dieser Zweckwidrigkeit willen keine wirklich verbindende Kraft haben. Wie wenn \zB\ ein Herr seinem Diener den Befehl gäbe, einen Andern zu ermorden, \udgl\ 
\item Aus einer \RWbet{Unvollständigkeit im Ausdrucke unserer meisten Gesetze}. Die wenigsten Gesetze gelten so völlig allgemein, als sie aus Liebe zur Kürze, oder auch deßhalb, weil wir uns die noch hinzuzudenkende nähere Beschränkung zu keinem deutlichen Bewußtseyn gebracht haben, ausgedrückt sind. Bei den meisten sollte, wenn sonst kein anderer, wenigstens der Zusatz beigefügt werden: Dieß soll geschehen, wofern diese Handlung in dem vorliegenden Falle das Wohl des Ganzen wirklich am Meisten befördert. Aber eben weil dieser Zusatz beinahe überall zu machen wäre: so können wir ihn füglich als einen, der sich von selbst versteht, weglassen; und so geschieht es nicht nur in der Moral, sondern auch in verschiedenen anderen Wissenschaften, daß wir eine Regel, die in vielen, ja fast allen Fällen gilt, als eine ganz allgemein geltende aussprechen, immer doch mit der noch stillschweigend hinzuzudenkenden Erinnerung, daß in gewissen sehr seltenen Fällen eine Art Ausnahme eintreten könne.~\RWSeitenw{265}\ So sagen wir \zB\  in der Naturgeschichte, der menschliche Körper habe zwei Hände, zwei Füße \udgl\ ; obwohl in einzelnen Fällen Ausnahmen Statt finden. Ganz eben so sagen wir auch in der Moral: Du sollst dein Leben, so lange du kannst, erhalten; obgleich dieß nur für viele, aber nicht für alle Fälle wahr ist, nämlich nicht für diejenigen, wo man durch die Erhaltung seines Lebens das Leben Anderer zerstören müßte, \udgl\  Durch diese zu allgemein lautenden Ausdrücke unserer Sittenregeln geschieht es denn, daß uns zuweilen zwei oder mehrere Gebote (menschliche, göttliche und natürliche) auf einen Fall anwendbar scheinen, ja ihren Worten nach auch wirklich anwendbar sind, welche doch in der That (ihrem Sinne nach) keineswegs gelten, oder zum wenigsten nicht alle verbindlich sind. -- Fordern nun diese Regeln die Eine dieses, die Andere ein anderes jenem zuwider laufendes Verhalten: so nennen wir dieses eine \RWbet{Collision der Pflichten}, obgleich es eigentlich nicht ein Widerstreit zwischen den Pflichten, sondern ein Widerstreit nur zwischen den Gesetzen, und zwar nur zwischen den unvollständigen Ausdrücken dieser Gesetze ist.
\end{aufzb}
Die Regeln nun, die das katholische Christenthum für solche Collisionen aufstellt, sind der Vernunft ganz gemäß.
\begin{aufzb}
\item Die erste stimmt mit dem obersten Sittengesetze so vollkommen überein, daß wir sie eigentlich als den unumstößlichsten Ausdruck desselben ansehen könnten.
\item Die zweite dürfte vielleicht nicht in völliger Allgemeinheit gelten, indem es zuweilen doch Fälle gibt, wo man dem Einzelnen, \zB\ sich selbst, einen Schmerz zufügen soll, wenn dadurch Andern auch nicht eben ein Schmerz erspart, aber doch ein Vergnügen von sehr großem Werthe verschafft wird. In den meisten Fällen aber bleibt es dabei, daß man nicht Einigen Schaden zufügen soll, um Andern zu nützen.
\item Die dritte Regel ist für sich selbst einleuchtend.
\item Verbietende Gesetze sind solche, die das Hervorbringen einer gewissen äußeren Veränderung untersagen; gebietende dagegen, welche sie vorschreiben. Jene haben gewöhnlich~\RWSeitenw{266}\ den Zweck, das Wohl des Ganzen nur nicht zu stören; durch die Befolgung der letzteren soll es auf eine positive Art befördert werden. Und somit ist diese Regel eine Folge der zweiten.
\end{aufzb}
\end{aufza}
\begin{RWanm} 
Gegen den wirklichen Nutzen, welchen die Lehre von der Möglichkeit eines irrenden Gewissens gestiftet hat, könnte man einwenden, daß Manche aus ihr die übereilte Folgerung gezogen: Kann ich nicht mit Gewißheit erkennen, was ich zu thun oder nicht zu thun habe: so kann ich eben darum auch nicht verbunden seyn, einem so unsichern Urtheile zu folgen. Doch die Erfahrung lehrt, daß dieser \RWbet{praktische Skepticismus} in der katholischen Kirche niemals geherrschet habe. Durch die Bestimmtheit nämlich, mit der diese Kirche die meisten unserer Pflichten entscheidet, und durch den hohen Ernst, mit dem sie jeder Verletzung derselben die Strafe Gottes androht, hat sie die Entstehung einer solchen Ausschweifung verhindert. Nirgends hat man in der That fleißiger über die Pflichten des Menschen nachgedacht, nirgends mehr wahre Gewissenhaftigkeit im wirklichen Leben bewiesen, als im Christenthume. Daß es auch Einige gab, welche es hierin übertrieben, die sogenannten Gewissensängstler oder Scrupulanten, ist freilich wahr; allein wer könnte glauben, daß dieser Schade den Nutzen, den das katholische Christenthum durch die Vertilgung des Leichtsinnes gestiftet hat, überwiege? -- Beträchtlicher dürfte das sittliche Verderben seyn, welches der \RWbet{Probabilismus} dadurch veranlaßte, daß mehrere katholische Sittenlehrer, namentlich aus der Gesellschaft Jesu, eine jede Meinung, welche nur irgend ein bewährter katholischer Schriftsteller aufgestellt hat, schon für wahrscheinlich genug erklärten, um sie im Leben befolgen zu dürfen. Doch so gefährlich auch diese Behauptungen gewesen: so haben sie des Bösen doch bei Weitem nicht so viel gestiftet, als uns die Feinde der Jesuiten aus bloßem Hasse gegen sie überreden wollen. 
\end{RWanm}

\RWpar{270}{Die Lehre des katholischen Christenthums von der verschiedenen Gut- und Bösartigkeit des Menschen}
\begin{aufza}
\item Eine Handlung ist nur dann \RWbet{sittlich gut} und in der That verdienstlich, wenn wir sie \RWbet{nicht nur in der Meinung, daß} sie dem Sittengesetze gemäß ist, sondern auch \RWbet{in der Absicht, weil} sie dieß ist, verrichten.~\RWSeitenw{267}
\item Und nur, wenn wir den \RWbet{herrschenden Willen} haben, \RWbet{immer so} sittlich gut zu handeln, verdienen wir den Namen eines \RWbet{Tugendhaften}. Die \RWbet{Tugend} also ist \RWbet{die herrschende Gesinnung, immer dem Sittengesetze gemäß zu handeln, und dieß zwar deßhalb, weil es so seyn soll.}
\item Es gibt nur \RWbet{Eine Tugend}, wiefern man unter derselben \RWbet{dieß herrschende Bestreben,} dem Sittengesetze immer gemäß zu handeln, versteht. Verstehen wir aber unter einer Tugend die \RWbet{Fertigkeit,} dem Sittengesetze \RWbet{in einer gewissen Art von Handlungen} gemäß zu leben: so gibt es der Tugenden allerdings \RWbet{mehrere}; und es ist möglich, es in einigen schon ziemlich weit gebracht zu haben, während man in gewissen anderen noch sehr zurück ist.
\item Wer \RWbet{nicht thut}, wovon er doch \RWbet{erkannte}, daß es das Sittengesetz als eine Pflicht verlangt, der \RWbet{sündiget}, und verdienet Strafe.
\item Es gibt aber gar mancherlei Arten und Grade der Sünden:
\begin{aufzb}
\item Wer seine Pflicht übertritt, \RWbet{ohne ein deutliches Bewußtseyn davon zu haben}, begeht eine bloße \RWbet{Gebrechlichkeitssünde}, von der bereits oben gelehrt worden ist, daß sie dem Menschen noch nicht den Namen eines Tugendhaften raubt.
\item Wer seine Pflicht übertritt \RWbet{mit deutlichem Bewußtseyn} (weil er den eigenen Vortheil mehr liebt), begeht eine \RWbet{vorsätzliche} oder \RWbet{Todsünde}.
\item \RWbet{Wer den herrschenden Willen hat, das Sittengesetz in gewissen Stücken zu übertreten}, der heißt ein \RWbet{Lasterhafter}; und dieser herrschende Wille selbst, oder was ihn erzeugt,  \RWbet{ein Laster}. Obgleich nun Niemand alle nur immer gedenkbaren Laster in sich vereinigen kann: so sagt das katholische Christenthum doch, daß Jeder, der nur \RWbet{Ein Laster} an sich hat, auch \RWbet{aller übrigen sich würde schuldig machen}, wenn er nur erst dazu von Außen versuchet werden sollte.
\item Wer an dem Bösen, \dh\  an dem, was dem Sittengesetze zuwider ist, ein \RWbet{eigenes Wohlgefallen} findet,~\RWSeitenw{268}\ schon \RWbet{darum, weil es böse ist}, und es \RWbet{aus diesem Grunde ausübt}, dem wirft das katholische Christenthum einen \RWbet{teuflischen Sinn} vor; und behauptet, daß wir durch fortgesetzten Wandel auf dem Pfade des Lasters allmählich auch selbst in diesen teuflischen Sinn verfallen können.
\item Das Christenthum spricht auch von \RWbet{Gewohnheitssünden}, \dh\  von Uebertretungen eines Gesetzes, die wir uns darum zu Schulden kommen lassen, \RWbet{weil eine lange Gewohnheit uns dazu verleitet}. Es empfiehlt uns nicht nur im Allgemeinen Behutsamkeit vor dieser Art von Sünden, sondern es nennt auch die einzelnen Arten von Sünden, die überaus leicht in Gewohnheit übergehen. Hieher gehören \zB\  die Sünde der Wollust, \uam
\item Das Christenthum warnet uns ferner vor \RWbet{Sünden, welche die Quelle von einer Menge anderer werden}, die es eben deßhalb \RWbet{Hauptsünden} nennt. Es gibt als solche an: den \RWbet{Stolz}, die \RWbet{Habsucht}, die \RWbet{Wollust}, die \RWbet{Unmäßigkeit im Genusse von Speise und Trank}, den \RWbet{Zorn}, die \RWbet{Trägheit}, den \RWbet{Neid}, \uam
\item Die christlichen Sittenlehrer stellen auch noch den Begriff von solchen Sünden auf, \RWbet{durch deren Begehung der Mensch sich den Weg zur Besserung gleichsam selbst abschneidet}. Sie nennen sie \RWbet{Sünden wider den heiligen Geist}; und rechnen dahin vornehmlich das \RWbet{Widerstreben gegen die bessere Einsicht}, das \RWbet{Sündigen auf die Barmherzigkeit Gottes}, die \RWbet{Verzweiflung an der Möglichkeit einer Besserung}, \usw\
\item Endlich finden wir auch in den Schriften der christlichen Sittenlehrer die Benennung \RWbet{himmelschreiende Sünden} zur Bezeichnung solcher Sünden, \RWbet{die eine ganz außerordentliche Größe und Abscheulichkeit haben}; wie die \RWbet{widernatürliche Wollust}, der \RWbet{vorsätzliche Todschlag}, die \RWbet{Unterdrückung der Armen, Wittwen und Waisen}, \udgl~\RWSeitenw{269}
\end{aufzb}
\item \RWbet{Keine Handlung, welche wir mit Bewußtseyn ausüben, ist durchaus gleichgültig}, \dh\  weder gut noch böse; sondern eine jede ist entweder gut oder böse, vermehrt entweder die Summe unserer Verdienste, oder vermindert sie, oder vermehrt wohl gar die Summe unserer Verschuldungen.
\end{aufza}

\RWpar{271}{Vernunftmäßigkeit und sittlicher Nutzen}
Ich übergehe den historischen Beweis auch dieser Lehren aus einem gleichen Grunde, wie schon einige Male; und will bloß ihre Vernunftmäßigkeit und ihren sittlichen Nutzen vereinigt betrachten.
\begin{aufza}
\item Daß eine Handlung, die bloß dem Sittengesetze gemäß, aber \RWbet{nicht wegen des Sittengesetzes} unternommen worden ist, keine Verdienstlichkeit habe, \dh\  keinen Grund enthalte, weßhalb die Glückseligkeit des Handelnden erhöht werden müßte, ist sehr leicht einzusehen. Weil der Handelnde zu einer solchen Handlung sich nicht aus Betrachtung ihrer Uebereinstimmung mit dem Sittengesetze entschloß: so folgt, daß ihn sein Glückseligkeitstrieb allein dazu vermocht habe; und somit wäre es sehr überflüssig, wenn man ihn noch durch eine angebotene Belohnung aufmuntern wollte. Dieß hätte im Gegentheile noch den Nachtheil, daß sich der Mensch gewöhnen würde, dem Sittengesetze nur dann zu folgen, wenn seine Forderungen mit den Wünschen seines Glückseligkeitstriebes übereinstimmen.
\item Es ist sehr nützlich, den \RWbet{Begriff der Tugend} gerade so zu bestimmen, wie es das Christenthum thut. Demjenigen, der nicht die herrschende Gesinnung hat, das Sittengesetz in allen Stücken zu befolgen, darf, so viele einzelne dem Gesetze gemäße Handlungen er auch sonst ausüben mag, doch kein belobender Name ertheilet werden, wie es geschähe, wenn man ihn tugendhaft nennen würde.
\item Es ist sehr nothwendig, zu erinnern, daß es in dieser Bedeutung des Wortes nur \RWbet{eine einzige Tugend,} (\di\ nur eine einzige tugendhafte Gesinnung) gebe; damit sich Niemand irre, wenn er in einer anderen Bedeutung des Wortes von mehreren Tugenden sprechen hört. Daß aber~\RWSeitenw{270}\ zwischen der bloßen\RWbet{ Gesinnung}, und zwischen der durch besondere Naturanlage erleichterten oder durch Uebung allmählich erworbenen \RWbet{Fertigkeit im Guten} ein wichtiger Unterschied obwalte, leuchtet von selbst ein. Der tugendhaften Fertigkeiten nun kann es begreiflicher Weise bei Einem und eben demselben Menschen gar manche geben, und während die Eine schon einen hohen Grad erreicht hat, kann eine andere noch sehr gering seyn. Es ist also auch nöthig, von \RWbet{mehreren Tugenden}, von tugendhaften Fertigkeiten nämlich, zu reden.
\item Daß Jeder Strafe verdient, der etwas unterläßt, wovon er doch \RWbet{erkennt}, daß es das Sittengesetz als eine Pflicht verlangt, unterliegt keinem Streite.
\item So ist auch der Unterschied, den das katholische Christenthum zwischen bloßen \RWbet{Gebrechlichkeits-} und \RWbet{vorsätzlichen Sünden} macht, schon oben gerechtfertiget worden.\par
\RWlat{ad}~c)~Die Erinnerung, daß ein Jeder, der nur Ein Laster wirklich an sich hat, auch aller übrigen sich würde schuldig machen, wenn er nur erst dazu von Außen versucht werden sollte: ist eben so wahr, als demüthigend für jeden Lasterhaften, und ganz geeignet, ihn zur Besinnung aufzuschrecken, wenn nur noch einiges Gefühl für Recht in seinem Busen schlummert.\par
\RWlat{ad}~d)~Die \RWbet{Möglichkeit eines teuflischen Sinnes} haben zwar Einige bestritten und gesagt, es sey nicht denkbar, daß ein Mensch, ja auch nur irgend ein Wesen, das Böse, als solches, verlangen könne. Ein bloßer Wortstreit! Es gibt gewiß Menschen, und, leider! nur zu viele, in deren Gemüthe sich, aus allerlei Gründen, \zB\  weil sie bisher schon sehr oft wider das Sittengesetz gehandelt, und also der Strafen schon sehr viele zu befürchten haben, \udgl\ , allmählich ein gewisser Abscheu gegen das Sittengesetz selbst entwickelt, \dh\  in deren Gemüthe sich die Vorstellung der Pflicht durch die bekannte Verknüpfung, in welche alle gleichzeitige Vorstellungen treten, mit allerlei unangenehmen und widrigen Vorstellungen und Gefühlen verbunden hat, so, daß sie alsbald einen gewissen Abscheu gegen eine Handlung verspüren, sobald sie~\RWSeitenw{271}\ nur hören, daß diese Handlung ihre Pflicht sey. Das ist es nun, was das katholische Christenthum ein \RWbet{Wohlgefallen am Bösen als solchem} (nämlich ein durch Ideenassociation erzeugtes Wohlgefallen) oder den \RWbet{teuflischen Sinn} nennt. -- Verdient es nun diese eben so böse als unglückliche Gemüthsstimmung nicht, daß man sie mit einem eigenen Namen bezeichne, und vor ihr warne? \par
\RWlat{ad}~e)~Die Unterscheidung der \RWbet{Gewohnheitssünden} ist besonders wichtig für die Asketik.\par
\RWlat{ad}~f)~Eben so dankenswerth ist es, daß uns das Christenthum auf die \RWbet{Verwandtschaften, die zwischen verschiedenen Lastern herrschen}, aufmerksam macht, und uns in dieser Hinsicht besonders vor denjenigen Lastern warnet, welche ein ganzes Heer anderer in ihrem Gefolge haben.\par
\RWlat{ad}~g)~Noch wichtiger ist die Warnung vor der ganz eigenen Art von Sünden, \RWbet{durch die wir uns den Weg zur Besserung gleichsam selbst abschneiden;} und wie richtig sind sie auch angegeben!\par
\RWlat{ad}~h)~Die Sünden, die das katholische Christenthum unter dem Namen der \RWbet{himmelschreienden} aufzählt, könnten nicht zweckmäßiger bezeichnet werden. Die durch die Ausübung solcher Sünden beleidigte Natur, das ist der Sinn dieses Namens, schreit hinauf zu Gott, daß er sie rächen möge! --
\item Daß eine jede Handlung, die wir mit deutlichem Bewußtseyn ausüben, entweder die Summe unserer Verdienste vermehre oder sie vermindere, oder wohl gar die Summe unserer Verschuldungen häufe, ergibt sich aus einem richtigen Begriffe von der Gerechtigkeit Gottes und mehreren schon vorgetragenen Lehren. Wie heilsam aber, daß uns dieß eigends zu Gemüthe geführt wird! Wie sehr wird durch diese Vorstellung nicht dem Leichtsinne gesteuert! und von welcher Wichtigkeit erscheint uns nicht von nun an jeder Augenblick, den wir mit einem deutlichen Bewußtseyn unser selbst verleben!~\RWSeitenw{272}
\end{aufza}

\RWabs{Zweite Abtheilung}{Die christkatholische Asketik}
\RWpar{272}{Die allgemeinsten Lehren des Katholicismus von dem Gebrauche der Tugendmittel}
\begin{aufza}
\item Das katholische Christenthum lehrt, daß es für uns Menschen, besonders in unserem gegenwärtigen durch den Sündenfall so vielfach verschlimmerten Zustande, zum Fortschreiten in unserer sittlichen Vollkommenheit bei Weitem noch nicht genug sey, wenn wir nur wissen, was wir zu thun haben, um vollkommen zu werden. Unsere Sinnlichkeit, unsere Leidenschaften und bösen Gewohnheiten, das böse Beispiel, das wir einander wechselseitig geben, tausenderlei äußere Versuchungen, welche durch zweckwidrige Einrichtungen in der Gesellschaft und andern Umständen, zuweilen wohl gar durch die Einwirkung gewisser übelgesinnter Wesen von höherer Art herbeigeführt werden, setzen uns einer beinahe unaufhörlichen Gefahr aus, dem Guten untreu zu werden, wenn wir es auch schon erkannt und selbst uns vorgenommen haben.
\item Indessen gibt es doch manche \RWbet{Mittel}, durch deren Anwendung wir diese Gefahr \RWbet{vermindern}, ja für einzelne Fälle wohl auch ganz \RWbet{aufheben} können. Der Gebrauch solcher Mittel, die das katholische Christenthum überhaupt \RWbet{Tugendmittel} nennet, ist uns nicht nur erlaubt, sondern verdienstlich, und in vielen Fällen sogar eine bestimmte Pflicht.
\item Dieses gilt insbesondere auch von dem Gebrauche der sogenannten \RWbet{Beweggründe zur Tugend}, \dh\  gewisser, von dem Begehrungsvermögen entlehnter Wünsche zu dem, was die Vernunft von uns fordert. Es ist nämlich nicht nur erlaubt, sondern verdienstlich, und in gewissen Fällen auch Pflicht, unser Begehrungsvermögen, so viel wir es können, so zu bearbeiten, daß seine Wünsche je mehr und mehr mit den Forderungen unserer Vernunft übereinstimmig werden.~\RWSeitenw{273}
\item Handlungen, zu deren Ausübung wir uns durch die Vermittlung eines solchen von unserem Begehrungsvermögen entlehnten Beweggrundes ermuntern, \RWbet{ermangeln keineswegs ihrer Verdienstlichkeit,} wenn anders der Wunsch unseres Begehrungsvermögens nur \RWbet{nicht der einzige Grund} war, der unseren Willen bestimmte.
\item So nützlich und nothwendig aber auch der Gebrauch gewisser Tugendmittel ist: so sind doch \RWbet{nicht alle von einem gleichen Werthe.} So ist \zB\  ein Tugendmittel um so vorzüglicher, \RWbet{je sicherer es den beabsichtigten Erfolg erreicht, je allgemeiner es in seiner Anwendbarkeit ist; je weniger es irgend einem Mißbrauche ausgesetzt ist; je weniger unangenehme Gefühle es bei übrigens gleicher Wirksamkeit verursacht; je leichter der Uebergang von ihm zu einem noch edlern ist}; \usw\
\item Unter den \RWbet{Antrieben zum Guten}, die vom Begehrungsvermögen entlehnt sind, gibt es mehrere, welche sogar \RWbet{gefährlich} werden können, und in Betreff derer daher eine größere Vorsicht nothwendig ist. Von dieser Art ist \zB\  der \RWbet{Ehrtrieb}, vollends der gröbere.
\item Der zweckmäßigste aus allen Antrieben zur Tugend, und wirklich derjenige, an den wir uns allmählich allein angewöhnen sollen, ist der vom \RWbet{Willen Gottes} entlehnte: Alles nur darum zu thun, weil Gott es will, und ewig lohnen wird.
\end{aufza}
\begin{RWanm} 
Den historischen Beweis dieser Lehre müssen wir abermals übergehen. 
\end{RWanm}

\RWpar{273}{Vernunftmäßigkeit und sittlicher Nutzen}
Die Vernunftmäßigkeit aller dieser Lehre ist aus Demjenigen, was wir bereits im 1.\,Haupttheil, 2.\,Hauptstück in der natürlichen Asketik gesagt, hinlänglich einzusehen, bis auf den letzten Punct, \RWbet{daß der vom Willen Gottes hergenommene Beweggrund der allerzuträglichste wäre.} Die Richtigkeit dieser Behauptung erhellet aus den in der natürlichen Asketik aufgestellten Regeln, wornach~\RWSeitenw{274}\ der verhältnißmäßige Werth der Tugendmittel beurtheilt werden kann. Dort hieß es nämlich
\begin{aufzb}
\item Ein Tugendmittel sey um so vorzüglicher, \RWbet{je sicherer es den beabsichtigten Zweck erreicht.} Nun kann man wohl behaupten, daß es für denjenigen Menschen, der an dem Daseyn und an der Gerechtigkeit Gottes nicht zweifelt, nicht möglich sey, Böses zu thun in einem solchen Augenblicke, da er daran denkt, daß alles Böse von Gott verboten sey, und bestraft werde, während daß alles Gute seine Belohnung findet. Denn im Bewußtseyn dieser Wahrheit kann sein Begehrungsvermögen, das immer nur dasjenige verlangt, wovon sich der Mensch vorstellt, daß es ihn glücklich machen werde, die böse Handlung gar nicht wünschen. Hieraus folgt aber freilich nicht, als ob der Mensch durch den Gebrauch dieser Vorstellung es bis dahin bringen könnte, daß er sich durchaus nie zum Bösen versucht fühlen sollte; denn es ist unmöglich, daß wir die Vorstellung vom Willen Gottes (wie überhaupt irgend eine bestimmte Vorstellung) uns immer gegenwärtig halten; unmöglich, zu verhindern, daß diese Vorstellung nicht doch zuweilen durch eine Menge anderer, die durch den Eindruck äußerer Gegenstände in uns angeregt werden, verdrängt werde. Aber so viel ist doch durch die Erfahrung selbst entschieden, daß wir der Vorstellung vom Willen Gottes durch Uebung eine so große Stärke, Geläufigkeit und Lebhaftigkeit ertheilen können, daß dieser Gedanke sich wenigstens bei jeder mit deutlichem Bewußtseyn angestellten Ueberlegung einfindet, und daß wir folglich, wenn wir beständig fortfahren, ihn fleißig zu benützen, es doch dahin bringen können, daß wir das Böse wenigstens nie mit einem deutlichen Bewußtseyn thun, also nie vorsätzlich sündigen.
\item Es hieß, ein Tugendmittel sey um so vorzüglicher, \RWbet{je allgemeiner es in seiner Anwendbarkeit ist}. Der von dem Willen Gottes entlehnte Beweggrund läßt sich auf jeden Fall, ohne Ausnahme, anwenden; denn es gibt auch nicht eine einzige gute Handlung, welche Gott unbelohnt ließe; nicht eine einzige, zu deren Ausübung uns also nicht der Gedanke an Gott ermuntern; und~\RWSeitenw{275}\ eben so auch nicht eine einzige böse Handlung, die Gott nicht strafte, von deren Begehung uns also dieser Gedanke nicht abhalten könnte.
\item Wir sagten ferner: ein Tugendmittel sey um so vorzüglicher, \RWbet{je weniger es dem Mißbrauche ausgesetzt ist.} -- Der von dem Willen Gottes entlehnte Beweggrund ist, bei einem Vernünftigen, eigentlich gar keiner Gefahr eines Mißbrauches ausgesetzt; denn ein Vernünftiger wird doch nie glauben, daß etwas Gottes Wille sey, wenn er nicht erst durch die Vernunft erkannt hat, daß es dem Wohle des Ganzen zuträglich sey.
\item Wir sagten auch noch, daß ein Tugendmittel um so vorzüglicher sey, \RWbet{je weniger unangenehme Gefühle es erzeugt.} Der von dem Willen Gottes entlehnte Beweggrund kann sich auf eine doppelte Art bei uns wirksam bezeugen, bald durch die Vorstellung der Belohnung, welche die Ausübung der guten Handlung haben wird, bald durch die Vorstellung der Strafe, die ihrer Unterlassung bevorsteht. Jene erzeugt angenehme, diese eine unangenehme Empfindung. Das katholische Christenthum stellt es uns frei, welcher von diesen beiden Vorstellungen wir uns bedienen wollen; oder vielmehr es räth uns, die erstere zu gebrauchen, so lange sich diese wirksam genug bezeugt; und erst, wenn diese zu schwach ist, unsere Zuflucht zur zweiten zu nehmen. Gebrauchen wir dieß Tugendmittel auf diese Art: so wirkt es größtentheils durch angenehme Gefühle, und verursacht uns nur, wenn es nothwendig ist, unangenehme Empfindungen.
\end{aufzb}

\RWpar{274}{Gedrängte Uebersicht der vornehmsten Tugendmittel von einer natürlichen Wirksamkeit, welche die christkatholische Religion ihren Gläubigen theils befiehlt, theils anräth, und zwar\newline 1.~das Mittel des Gebetes}
Keine Religion auf Erden hat ihren Bekennern mehrere und wirksamere Mittel zur Tugend empfohlen, ja auch in wirklichen Gebrauch bei ihnen zu setzen gewußt, als die katholische; so zwar, daß, wenn sie auch in ihren theoretischen~\RWSeitenw{276}\ Lehren (in ihrer Dogmatik) nicht den geringsten Vorzug vor einigen anderen Religionen hätte (und wir haben gezeigt, daß sie auch hier alle anderen hinter sich zurückläßt), ihr gleichwohl schon um der Vortrefflichkeit ihrer Asketik willen der Vorzug vor allen anderen Religionen eingeräumt werden müßte. Da es aber die Kürze der Zeit nicht erlaubt, von diesen Tugendmitteln mit der gehörigen Ausführlichkeit zu reden, und die Vernunftmäßigkeit, den möglichen und wirklichen Nutzen eines jeden abgesondert zu betrachten: so werden wir \RWbet{nur die wichtigsten} derselben in gedrängter Kürze berühren, und sie mit einigen Bemerkungen über ihre Vernunftmäßigkeit begleiten.\par
Zuvörderst ist zu bemerken, daß das katholische Christenthum eine doppelte Art von Tugendmitteln unterscheide:
\begin{aufzb}
\item \RWbet{natürliche}, worunter es solche versteht, denen kein anderer, als nur ein natürlicher Nutzen versprochen ist, und
\item \RWbet{übernatürliche}, worunter es solche versteht, deren gehörigen Gebrauch Gott nebst dem natürlichen Nutzen noch mit gewissen übernatürlichen Gnaden zu belohnen verheißen hat. Diese letzteren nennt man auch \RWbet{Heiligungsmittel} oder \RWbet{Sacramente.} Die Vernunftmäßigkeit dieses Unterschiedes ergibt sich aus der \RWparnr{219}.\ Nr.\,2. gemachten Bemerkung über den Sinn des Wortes übernatürlich von selbst.
\end{aufzb}
Unter den natürlichen Tugendmitteln, die das katholische Christenthum empfiehlt, nimmt einen der vorzüglichsten Plätze das \RWbet{Gebet} ein.\par
Es ist jedoch zu merken, daß die katholische Kirche das Beten keineswegs \RWbet{bloß als ein Tugendmittel} betrachtet; sondern auch abgesehen von jenem vortheilhaften Einflusse, den ein gehörig eingerichtetes Gebet auf die Beförderung der Tugend freilich in jedem Falle hat, kann es zuweilen noch manche andere Vortheile haben, und auch um dieser wegen verdienstlich, ja sogar Pflicht für uns seyn. Dergleichen Vortheile sind die Beruhigung, der Trost, die Freude, die das Gebet gewährt, \usw\ Daher kommt es denn, daß die Lehre vom Gebete nicht nur in der katholischen Asketik, sondern auch in der Pflichtenlehre oder Ethik abgehandelt wird.~\RWSeitenw{277}
\begin{aufza}
\item Die katholische Kirche versteht aber unter dem \RWbet{Gebete} jede \RWbet{Beschäftigung unseres Geistes mit Gott, durch welche Gefühle und Entschließungen, welche der Tugend und Glückseligkeit zuträglich sind, in uns angeregt werden.} Wenn wir also bloß nachdenken über Gott, ohne daß dieses Nachdenken in uns gewisse Gefühle und Entschließungen weckt; oder wenn unser Gedanke an Gott zwar von gewissen Gefühlen und Entschließungen begleitet ist, doch nicht von solchen, die sittlich gut sind, so beten wir nicht.
\item Das katholische Christenthum hat sich ein eigenes Verdienst um den Begriff des Gebetes schon darum beigelegt, daß es \RWbet{verschiedene Arten}, auf die wir beten können, deutlicher auseinander setzt, und sie unserer Aufmerksamkeit durch die \RWbet{Bezeichnung mit eigenen Namen} empfiehlt. Es unterscheidet nämlich
\begin{aufzb}
\item Das \RWbet{Gebet der Anbetung Gottes}, in welchem wir durch die Betrachtung der göttlichen Vollkommenheiten, besonders seiner Allmacht, Allwissenheit, höchsten Weisheit, Ewigkeit, und anderer, in unserem Gemüthe das Gefühl der Anbetung, \dh\  das Gefühl der höchsten uns möglichen Achtung und Ehrfurcht hervorzubringen suchen.
\item Das \RWbet{Gebet der Liebe Gottes}, in welchem wir durch die Betrachtung der göttlichen Vollkommenheiten, besonders seiner Güte und der unzähligen, uns selbst erwiesenen und noch in Zukunft zu erweisenden Wohlthaten, ein gewisses Gefühl der Liebe gegen ihn in uns zu erzeugen suchen.
\item Das \RWbet{Gebet des Dankes,} darin wir uns umständlich zu Gemüthe führen, wie viele und große Wohlthaten wir von Gott entweder überhaupt, oder erst eben jetzt erhalten haben, und welche Pflichten uns diese Wohlthaten auflegen, welchen Gebrauch er von uns gemacht wissen will, \usw\
\item Das \RWbet{Gebet der Hoffnung,} darin wir aus der Betrachtung der göttlichen Allmacht, Weisheit, Güte und Treue, ingleichen der vielen von ihm bisher schon empfangenen Gnaden als eben so vieler Unterpfänder dafür, daß~\RWSeitenw{278}\ er uns deren noch mehrere ertheilen wolle, die Hoffnung in uns zu beleben suchen, daß wir durch Gottes Beistand je länger je weiter fortschreiten werden in unserer sittlichen Vervollkommnung; daß es uns endlich gelingen werde, alle uns noch anklebenden bösen Gewohnheiten und Neigungen abzulegen, und so zuletzt der ewigen durch Christum uns erworbenen Seligkeit theilhaftig zu werden; ja, daß vielleicht auch viele andere durch uns für diese Seligkeit werden gewonnen werden.
\item Das \RWbet{Gebet der Selbsterforschung}, darin wir uns vor Gottes Angesichte, \dh\  im Andenken an jene Allwissenheit, der nichts verborgen ist, an jene Gerechtigkeit, welche nichts unvergolten läßt, erforschen, was sich noch Böses oder Mangelhaftes an uns befinde.
\item Das \RWbet{Gebet der Reue und des Vorsatzes}, in welchem wir die Betrachtung gewisser göttlicher Vollkommenheiten, besonders der göttlichen Gerechtigkeit, Güte, Langmuth \udgl\  dazu benützen, um unserer Reue, \dh\  unserem Schmerze über das viele Böse, das wir an uns finden, um unserem Vorsatze, es für die Zukunft zu verbessern, mehr Lebhaftigkeit, Stärke und Festigkeit zu ertheilen.
\item Das \RWbet{Gebet des Gelübdes}, darin wir einem guten Vorsatze,den wir gefaßt, dadurch mehr Festigkeit zu geben suchen, daß wir Gott selbst zu einem Zeugen desselben auffordern.
\item Das \RWbet{Bittgebet}, in dem wir von Gott verlangen, daß er einen gewissen (sittlich erlaubten) Wunsch, den wir hegen, erfüllen möge, und dieses um so zuversichtlicher hoffen, je lebhafter wir uns seine Allmacht, Weisheit und Güte vorstellen, und je fester wir uns vornehmen, und ihm auch angeloben, wenn unser Wunsch erfüllt werden sollte, einen recht guten Gebrauch von dieser göttlichen Wohlthat zu machen, gewisse verdienstliche Handlungen zu verrichten, \usw\
\end{aufzb}
Die Vernunftmäßigkeit und der sittliche Nutzen aller dieser Arten des Gebetes leuchtet von selbst ein; nur in Betreff des letzteren ist die Behauptung des Christenthums, daß solche~\RWSeitenw{279}\ Bittgebete Erhörung bei Gott finden, \dh\  daß sie den Willen Gottes bestimmen, uns, wenn nicht dieselbe Wohlthat, um die wir eben gebeten, doch eine andere, zuzuwenden, öfters bestritten worden. Allein ich glaube das Nöthige zur Rechtfertigung dieser Lehre schon oben beigebracht zu haben.
\item Um die Erweckung zweckmäßiger Gefühle beim Beten zu erleichtern, empfiehlt das katholische Christenthum eine \RWbet{angemessene Lage des Körpers} dabei zu beobachten, insonderheit \zB\  eine kniende Stellung, gefaltete Hände, gegen Himmel gerichtete Blicke, \usw\par
Sollte man glauben, daß sich erst neuerlich wieder Einer der angesehensten Lehrer des Christenthums, nämlich des protestantischen, \RWbet{Heinr.\ Gottl.\ Tzschirner}, in einem Buche, welches die Ueberschrift führet: \RWlat{De sacris ecclesiae nostrae publicis caute emendandis}\RWlit{}{Tzschirner1}, gegen die kniende Stellung beim Gebete erklären werde, als gegen eine nur aus dem sklavischen Sinne des Orientalen hervorgegangene Gewohnheit, die mit dem freien kindlichen Geiste durchaus im Widerspruche ist! -- Wie schlecht hätte also den Geist des Christenthums unser Herr Jesus gekannt, der (\RWbibel{Mt}{Matth.}{26}{39}) selbst kniend gebetet hat!
\item Als ein anderes Hülfsmittel des Gebetes, welches besonders anzuwenden wäre, wenn unser Herz eine gewisse Leere und Trockenheit empfindet, oder wenn wir laut und in Gegenwart Anderer beten sollen, und nicht geübt genug sind, um unsere Gedanken gehörig zu ordnen und verständlich auszudrücken, empfiehlt uns das katholische Christenthum den \RWbet{Gebrauch bereits vorhandener Gebetsformeln}, und wünscht, daß mehrere solcher Gebetsformeln für die verschiedensten Verhältnisse und Bedürfnisse des Lebens, und in den verschiedensten Sprachen, in eigenen Büchern \RWbet{(Gebetbüchern)} zu finden seyn möchten. Zu den Gebetsformeln, die in unserer Kirche wirklich sehr häufig angewandt werden, gehört das so vortreffliche Gebet, das wir \RWbet{von unserem Herrn Jesu selbst empfangen haben}, das jeder katholische Christ auswendig kann, der \RWbet{englische Gruß}, das \RWbet{apostolische Glaubensbekenntniß}, so viele ungemein schöne Gebete, die bei dem öffentlichen Gottesdienste, bei der Ausspendung~\RWSeitenw{280}\ der Sacramente, und bei verschiedenen anderen Gelegenheiten vorgeschrieben sind, und sich im Meßbuche, Rituale, im Breviere \uaO\ befinden.
\item Eine besondere Erwähnung verdient hier noch der in der katholischen Kirche ausgebreitete Gebrauch der \RWbet{Segnungen} und \RWbet{Einweihungen}. Es sind aber dergleichen Segnungen und Einweihungen nichts Anderes, als \RWbet{Gebete}, in denen wir Gott, als dem Geber aller guten Gaben, für den so eben erlangten Besitz eines gewissen Gutes danken, ihm den gehörigen Gebrauch desselben angeloben, und ihn bitten, er wolle gnädig verhüten, daß wir dasjenige, was uns zum Heile dienen soll, durch Mißbrauch nicht in unser Verderben verwandeln, er wolle vielmehr bewirken, daß es uns in der That gedeihlich und so gedeihlich werde, als es ein Gegenstand von dieser Art vermag. Obgleich wir nun solche Einsegnungen schon bei den Israeliten und vielen anderen Völkern finden: so hat man doch nirgends einen so vielfältigen und im Ganzen auch so zweckmäßigen Gebrauch davon gemacht, als unter uns; wobei ich jedoch gar nicht in Abrede stelle, daß hie und da auch manche abergläubige Vorstellung und mancher wirkliche Mißbrauch sich eingeschlichen habe. Tadel verdient es nur, wenn in der neuesten Zeit besonders diejenigen aus uns, welche auf Bildung Ansprüche machen, aus Furcht, für abergläubig gehalten zu werden, eine Sitte verlassen, die an sich selbst so vernünftig ist, und mit der Vorschrift des Apostels (\RWbibel{1\,Kor}{1\,Kor.}{10}{31}), daß wir Alles, was wir thun und genießen, zur Ehre Gottes thun und genießen sollen, so vollkommen übereinstimmt.
\item Weil es jedoch bekannt ist, daß oft die nützlichste Sache unterbleibt, wenn ihr nicht eine \RWbet{bestimmte Zeit} angewiesen wird: so ertheilet die katholische Kirche nicht nur im Allgemeinen den Rath, öfters, ja \RWbet{ohne Unterlaß} zu beten (\RWbibel{1\,Thess}{1\,Thess.}{5}{17}); sondern es werden uns auch \RWbet{bestimmte, überaus schickliche Zeiten zum Gebete} vorgeschlagen; wir werden namentlich ermahnt, den \RWbet{Morgen} und \RWbet{Abend} eines jeden Tages, die \RWbet{Mittagszeit}, besonders aber die \RWbet{Anfangszeit eines jeden wichtigeren Geschäftes} durch Gebet einzuweihen. Zu diesem~\RWSeitenw{281}\ Ende werden wir an bestimmten Zeiten des Tages selbst durch den \RWbet{Ruf einer Glocke} erinnert, einen kleinen Stillstand in unseren Geschäften zu machen, um unseren Geist vereinigt mit Tausenden, die es in eben diesem Augenblicke thun, zu unserem gemeinschaftlichen Schöpfer zu erheben.
\item Wahr ist es endlich, daß die Vorsteher der katholischen Kirche einige Andachten, \zB\  die sogenannten \RWbet{Rosenkränze}, gebilligt haben, welche durch eine zu oftmalige Wiederholung einer und eben derselben Gebetsformel leicht zu einem \RWbet{gedankenlosen Lippengebete} verleiten können. Allein es ist zu bemerken, daß solche Andachten nie allgemein vorgeschrieben wurden, für einzelne Menschen aber und in gewissen Verhältnissen unstreitig auch ihr Gutes haben.
\end{aufza}

\RWpar{275}{2.~Das Mittel der öffentlichen Gottesverehrung}
Das katholische Christenthum sagt, es sey nicht genug, daß wir ein Jeder Gott nur im Herzen verehren, sondern wir sollen dieß auch \RWbet{äußerlich} und \RWbet{in Gemeinschaft mit Andern} thun. Zu diesem Zwecke hat es einen eigenen \RWbet{öffentlichen Gottesdienst} eingerichtet, von welchem es freilich auch, wie vorhin vom Gebete, gilt, daß er, selbst abgesehen von jenem wohlthätigen Einflusse, den er als Tugendmittel hat, verdienstlich und pflichtgemäß sey, nämlich zur Erbauung für Andere, ingleichen zur eigenen Aufheiterung und Freude; keineswegs aber stellt die katholische Kirche sich vor, daß wir Gott diese Verehrung, obgleich sie dieselbe gleichfalls einen \RWbet{Dienst} nennet, \RWbet{um seiner selbst willen,} \dh\  zu seinem eigenen Vortheile zu leisten hätten. (\RWbibel{Ps}{Ps.}{50}{8--13}) Es ist ein Dienst, den wir Gott leisten \RWbet{um unsertwegen}, weil es zu unserer eigenen Tugend und Glückseligkeit nothwendig ist, so zu verfahren.
\begin{RWanm} 
Ueber den Zweck, den diese öffentliche Gottesverehrung oder \RWbet{Liturgie} haben soll, haben sich neuere Gelehrte sehr verschiedentlich erklärt. Einige gaben Belehrung, Andere Aufregung der Gefühle und Erwärmung für Religion an, \udgl\  Meines Erachtens liegt der vollständige Zweck dieser Anstalt in der \RWbet{Summe alles desjenigen Guten, was durch sie über}\RWSeitenw{282}\RWbet{haupt erreichbar ist.} Also \zB\  Versinnlichung der Größe und Majestät unseres Gottes, vor dem wir im öffentlichen Gottesdienste Alle, auch selbst die Vornehmsten, im Staube hingegossen sehen; Belebung aller religiösen Gefühle durch die Wahrnehmung, daß wir sie nicht allein, sondern daß so viele Andere sie mit uns gemeinschaftlich empfinden; sinnbildliche Darstellung der wesentlichen Gleichheit aller Menschen; brüderliche Annäherung derselben an einander, \usw\ 
\end{RWanm}
Zum Behufe dieser öffentlichen Gottesverehrung finden wir nun im katholischen Christenthume:
\begin{aufza}
\item \RWbet{Bestimmte Tage des Jahres}, namentlich den \RWbet{Sonntag}, und noch einige andere \RWbet{festliche Tage}, an welchen der Christ von der Arbeit ruhen, und die öffentliche Gottesverehrung als sein Hauptgeschäft ansehen soll; so jedoch, daß gewisse, diesem Hauptgeschäfte nicht widersprechende, ihn vielmehr noch unterstützende Zwecke, \zB\  der Erholung von Körper und Geist ermüdender Arbeit, der Geistesausbildung und des Vergnügens, nicht ausgeschlossen werden. Bei mehreren dieser Tage, besonders den Festtagen, wird der religiöse Gegenstand, dessen Gedächtniß an ihnen gefeiert werden soll, noch näher bezeichnet; und jeder Unparteiliche muß gestehen, daß diese Gegenstände überaus fruchtbar und glücklich gewählt sind. \ZB\  das Gedächtniß des traurigen Zustandes, in welchem wir Menschen uns befänden, wenn Christus nie erschienen wäre \RWbet{(Adventzeit)}, das Gedächtniß seiner Geburt, seiner Bekanntwerdung auch unter den Heiden \RWbet{(Erscheinung),} seiner Leiden und seines Todes \RWbet{(Fastenzeit),} seiner Auferstehung und Himmelfahrt, der Sendung des heil.\ Geistes, seiner jungfräulichen Mutter Maria, \usw\
\item \RWbet{Bestimmte Gebäude} in jeder Stadt, und durch das ganze Land hindurch in so beträchtlicher Anzahl, in denen die Gläubigen sich versammeln können, an Sonn- und Feiertagen auch kraft der bestimmten kirchlichen Verordnung versammeln \RWbet{sollen}, um dem öffentlichen Gottesdienste beizuwohnen. Was kann zweckmäßiger seyn, als daß der öffentliche Gottesdienst nur in bestimmten, zu diesem Geschäfte ausschließlich gewidmeten, Gebäuden vorgenommen werde? Wie vernünftig ist aber auch der Grundsatz der Katholiken, daß~\RWSeitenw{283}\ diese Gebäude, so viel es möglich ist, so eingerichtet werden sollen, daß sie schon durch die Größe und Erhabenheit ihres Baues, und durch die prächtige Ausschmückung, die sie erhalten, das Gemüth auch des sinnlichsten Menschen anziehen und erheben! Sollte nicht Alles, was kostbar und prächtig ist, nur in den Tempeln aufgestellt werden, damit es, statt die Eitelkeit Einzelner zu nähren, nur zur Verherrlichung Gottes diene?
\item \RWbet{Bestimmte Personen, welche den Gottesdienst in diesen Versammlungen leiten}, nämlich die Geistlichen. Man muß sehr verblendet durch Leidenschaft seyn, wenn man die Zweckmäßigkeit dieser Verfügung nicht einsehen und sich überreden kann, daß es in unsern gottesdienstlichen Versammlungen erbaulicher zugehen würde, wenn das Recht, hier zu sprechen, und das Ganze zu leiten, einem Jeden eingeräumt wäre. Man muß die Lehre der Kirche absichtlich mißdeuten, wenn man ihre Verfügung so auslegen kann, als ob, nach ihrer Ansicht, der Mensch der Geistlichen bedürfe, als gewisser, schlechterdings nothwendiger Vermittler zwischen ihm und Gott; als ob er in seinen Gebeten sich durchaus nicht unmittelbar an Gott selbst wenden dürfe, \usw\ Nur um der guten Ordnung wegen, nur um dem öffentlichen Gottesdienste mehr Vollkommenheit zu geben, läßt ihn die Kirche durch ihre Geistlichen, als Personen, die hierin eigens unterrichtet sind, besorgen.
\item So viele \RWbet{Gemälde, Bildsäulen und andere Verzierungen, welche an religiöse Gegenstände erinnern, und religiöse Gefühle zu erwecken vermögen,} \zB\  das Zeichen des Kreuzes; Bildnisse Jesu, \udgl\  Das Kreuzeszeichen hat erst in neuerer Zeit wieder verschiedene Feinde gefunden; so erkläret es \zB\  D. \RWbet{Kaiser} (in seiner biblischen Theologie oder Judäismus und Christianismus, 2.\,Thl., 1.\,Abschn., Erlangen 1814)\RWlit{}{Kaiser1a} für einen unästhetischen Gegenstand, der höchstens an Communiontagen und am Charfreitage in der christlichen Versammlung aufgestellt werden sollte. -- Mir däucht es vielmehr eine sehr löbliche Sitte der katholischen Kirche, daß sie dieß heilige Zeichen, das in der Form, in der wir es heut zu Tage gebrauchen,~\RWSeitenw{284}\ durchaus nichts Unästhetisches hat, nicht nur in Tempeln, sondern auch an öffentlichen Plätzen, auf Straßen \udgl\  aufpflanzt.
\item In unsern gottesdienstlichen Versammlungen selbst wird \RWbet{Alles vorgenommen, was sich hier schicklicher Weise vornehmen läßt.} Es wird
\begin{aufzb}
\item \RWbet{religiöser Unterricht ertheilet.} Es werden
\item \RWbet{gemeinschaftliche Gebete angestellt.} Es werden
\item \RWbet{Aufforderungen zur Wohlthätigkeit gemacht.} \Usw\
\end{aufzb}
\item Wenn es schon bei der häuslichen (oder vielmehr einsamen) Andacht zuweilen nothwendig ist, \RWbet{gewisse Gebetsformeln} zu haben: so ist dieß begreiflich um so nothwendiger bei einer öffentlichen. Daher finden sich denn auch in der katholischen Kirche mehrere zum Theil sehr treffliche Gebete, die zum Gebrauche in den gottesdienstlichen Versammlungen vorgeschrieben sind.
\item Wenn es schon bei demjenigen, der in der Einsamkeit betet, zur Förderung der eigenen Andacht dient, daß er bei diesem Geschäfte auch eine \RWbet{angemessene Haltung des Körpers} beobachte, so muß dieß um so nöthiger bei einer in Gesellschaft vorgenommenen Verehrung Gottes seyn. In der katholischen Kirche nun bestehen hierüber, besonders für den die ganze Handlung leitenden Geistlichen, sehr bestimmte, und gewiß größtentheils auch sehr zweckmäßige Anordnungen.
\item Auch der Umstand, daß ein Theil der gottesdienstlichen Gebete \RWbet{nicht in der Volkssprache}, sondern in einer \RWbet{gelehrten}, namentlich in der \RWbet{lateinischen Sprache} verrichtet wird, hat seinen wesentlichen Nutzen, wenn gleich die Kenntniß dieser Sprache unter uns Katholiken bei Weitem noch nicht so gemein ist, als sie es wohl seyn könnte, und nach dem Wunsche der Vorsteher unserer Kirche seyn sollte. Die Gebete, die in lateinischer Sprache gesprochen werden, gewinnen an Ehrwürdigkeit bei der gemeinen Menge, und werden vor Entweihung bewahret.\RWfootnote{%
Derjenige Theil unseres Gottesdienstes, der in der Landessprache gehalten werden kann und soll, beträgt zwar schon mehr, als der protestantische Gottesdienst im Ganzen; doch wäre zu wünschen, daß die Vorsteher der Kirche gestatteten, noch einen Theil der heil.\ Messe in die Volkssprache zweckmäßig zu übertragen, und solche Uebertragungen öffentlich einzuführen.}~\RWSeitenw{285}
\item Die katholische Kirche glaubt, daß sich die Wirksamkeit des \RWbet{Gesanges} und der \RWbet{Musik} zur Anregung sittlicher und religiöser Gefühle überaus wohl benützen lasse, und empfiehlt uns daher den Gebrauch derselben nicht nur bei unserer häuslichen Andacht und bei den verschiedenen Geschäften des Lebens, die eine solche Nebenbeschäftigung erlauben; sondern sie wendet dieses Erbauungsmittel auch zu der öffentlichen Verehrung Gottes in den Versammlungen an.
\item Zu der Zeit, wie Gott in der katholischen Kirche öffentlich verehrt wird, gehört auch die löbliche Sitte der Katholiken, sich bei Begegnungen, Zusammenkünften \udgl\  \RWbet{eines heiligen Grußes zu bedienen.} Es wäre sehr zu wünschen, daß diese Sitte nie abkommen möchte, oder, da einmal dasjenige, wessen die Menschen sich (es sey nun mit Recht oder Unrecht) zu schämen angefangen haben, nie ohne einige Veränderung wieder zu Ehren kommen mag: so wäre zu wünschen, daß man diesem Gruße durch irgend eine unbedeutende Abänderung im Ausdrucke das Gepräge der Neuheit gäbe, und die verdiente Achtung wieder verschaffte.

\item Eine sehr zweckmäßige Art des öffentlichen Gottesdienstes sind die \RWbet{öffentlichen Umgänge} oder \RWbet{Processionen}, die wir eben deßhalb auch schon bei dem israelitischen Volke, und bei so manchen Völkern alter und neuer Zeit, antreffen.
\item Auch das \RWbet{Wallfahrten}, wenn es \RWbet{gehörig eingerichtet} wird, ist zu den löblichen Arten der öffentlichen Gottesverehrung zu zählen, und verdiente für immer beibehalten zu werden. Denn Reisen, bald kürzere, bald wieder längere, sind doch beinahe für einen jeden Menschen ersprießlich, und können ihm bald zur Befestigung seiner Gesundheit, bald zur Erweiterung seiner Begriffe, bald zu noch anderen Zwecken dienen. Durch die Erhebung zu einer Wallfahrt, \dh\  durch die \RWbet{Verbindung mit einem religiösen Zwecke}, mit dem festen Vorsatze, jeden zu seiner Erbauung~\RWSeitenw{286}\ dienenden Anlaß auf dieser Reise bestens benützen zu wollen, werden sie nur um so ersprießlicher. Und werden solche Reisen in guter Gesellschaft und unter gehöriger Aufsicht unternommen: so fallen viele Gefahren, denen der Leib und die Seele auf Reisen ausgesetzt seyn können, bei solchen Wallfahrten hinweg.
\end{aufza}
\begin{RWanm} 
Merkwürdig ist es, daß die Protestanten, welche in früheren Zeiten den katholischen Gottesdienst nicht tief genug glaubten heruntersetzen zu können, jetzt viele seiner Vorzüge selbst eingestehen, und bei sich nachgeahmt wünschen. Am weitesten geht in dieser Rücksicht Hr.\ Kirchenrath G.~C.~\RWbet{Horst} in seiner Mysteriosophie (1817, Frankfurt am Mayn, 2 Theile)\RWlit{}{Horst1}, der \RWbet{eine Art von Messe}, ganz nach katholischer Weise, vorschlägt, den Actus der Elevation, das Schellen des Glöckleins, das Kreuzzeichen, das Aussetzen des Allerheiligsten, \usw\ eingeführt wissen will. 
\end{RWanm}

\RWpar{276}{3.~Uebungen in der Kunst der Selbstbeherrschung}
Eines der wichtigsten Tugendmittel ist ohne Zweifel die \RWbet{Uebung in der Kunst der Selbstbeherrschung}. Hierunter verstehen wir die Ueberwindung gewisser Begierden in einem solchen Falle, wo ihre Befriedigung an sich erlaubt wäre, nur in der Absicht, um sich auf diese Art etwas versagen zu lernen, damit man es auch dann vermöge, wenn es die Pflicht gebietet. Die Nützlichkeit dieses Verfahrens erkannten selbst die Juden und Heiden; doch wurde dieß Tugendmittel nirgends so zweckmäßig angewendet, als eben in der katholischen Kirche.
\begin{aufza}
\item Obgleich die Vorsteher der katholischen Kirche recht gut wußten, daß es der Begierden, an welchen wir die Kunst der Selbstbeherrschung zu üben Gelegenheit haben, gar mancherlei gebe; obgleich sie es auch nicht unterließen, uns auf verschiedene derselben in dieser Hinsicht noch eigens aufmerksam zu machen (namentlich auf die Begierde, etwas zu hören, zu sehen, zu sagen, \usw ): so erkannten sie doch, daß sich fast keine andere zum Gegenstande einer allgemein vorzuschreibenden Uebung eigne, als die -- \RWbet{Eßbegierde.} In der Beherrschung dieser sich zu üben, \RWbet{tritt die Gelegenheit für}~\RWSeitenw{287}\ \RWbet{Jeden täglich ein}, und zugleich wird durch diese Uebung \RWbet{eine beträchtliche Menge von Nahrungsstoffen erspart}, die wieder Anderen zum Genusse dienen können; \dh\ es ist dieß eine Uebung, die auch schon an sich selbst einen Vortheil für das Ganze gewähret. Damit aber diese so nützliche Uebung in der That vorgenommen werde, mußten die Vorsteher der Kirche sie ausdrücklich \RWbet{anbefehlen}; und um nicht etwa durch die Unbestimmtheit ihrer Verordnung die Gewissen Einiger zu beschweren, mußte das \RWbet{Alter}, innerhalb dessen solche Uebungen zu geschehen hätten, mußten die \RWbet{Tage}, an welchen sie angestellt werden sollen, genau bestimmt werden.
\item Allein noch wohlthätiger für jeden Einzelnen sowohl, als für die ganze Gesellschaft wurde die Fastenübung dadurch, daß man auf den Gedanken verfiel, nicht nur zu fordern, daß sich die Gläubigen zuweilen einen Abbruch in der Menge der Speisen, die sie genießen wollen, thun; sondern auch zu verlangen, daß sie sich zur Befriedigung ihrer Eßlust \RWbet{oft nur gewisser Speisen bedienen, gewisser anderer aber sich enthalten sollen}. Durch diese letztere Verfügung nämlich erhielt man Gelegenheit, noch ein Paar neue Vortheile zu erreichen:
\begin{aufzb}
\item zu verhindern, daß Niemand durch einen \RWbet{allzu ununterbrochenen} Genuß gewisser Nahrungsmittel seine Gesundheit verderbe, und in seinem eigenen Körper den Keim zu gewissen Versuchungen, \zB\  zur Trägheit, zur Wildheit, zur Wollust, \udgl\  erzeuge;
\item zu verhindern, daß von gewissen Nahrungsmitteln, die nur durch Aufopferung einer bedeutenden Menge anderer genießbaren Stoffe erzeugt werden können, \RWbet{nicht allzu viele verbraucht werden.}
\end{aufzb}
Freilich kann man nicht sagen, daß sich die Vorsteher der katholischen Kirche bisher bei der Entscheidung der Frage, \RWbet{welche Speisen sie zu den verbotenen zählen} oder nicht zählen sollen, dieser beiden Zwecke immer ganz deutlich bewußt gewesen wären; inzwischen ist es doch gewiß, daß ihre Entscheidungen diesem Zwecke zum größten Theile sehr angemessen waren.
So verboten sie den Genuß der \RWbet{Fleisch}\RWSeitenw{288}\RWbet{speisen}, und insbesondere des Fleisches solcher Thiere, \RWbet{die auf dem Lande leben,} deren Ernährung uns also insgemein um eine sehr große Menge genießbarer Stoffe ärmer macht;\RWfootnote{%
	Wie viele Pfunde vegetabilischen Nahrungsstoffes müssen wir aufopfern, um Ein Pfund Rindfleisch zu erhalten! -- Wenn die Erde allenthalben wird mehr bevölkert seyn: werden nicht die weltlichen Regierungen den Genuß der Fleischspeisen einschränken müssen, um eine immerwährende Hungersnoth zu verhindern?}
dagegen erlaubten sie den Genuß der Fische, der Wasserhühner \ua\,dgl. Wasserthiere, deren Ernährung nicht das Geringste kostet; \usw\
\item In unseren Staaten besteht seit 1809 folgende Fastenordnung: Erst bei dem Alter von 21 Jahren fängt die Verbindlichkeit zum \RWbet{Speiseabbruch} an, zu welchem man
\begin{aufzb}
\item in der vierzigtägigen Faste durch alle Wochentage;
\item in den Vigilien (\dh\  den Vorabenden) vor Pfingsten, Weihnachten, vor den Festtagen der heiligen Apostel Petrus und Paulus, der Himmelfahrt Mariens, und vor Aller Heiligen;
\item in den Quatemberzeiten (am Mittwoch, Freitag und Samstag); endlich
\item an allen Mittwochen und Freitagen des Advents verbunden ist. 
\end{aufzb}
Nebstdem ist noch der \RWbet{Genuß der Fleischspeisen} untersagt:
\begin{aufzb}
\item in der vierzigtägigen Faste: am Aschermittwoche, am grünen Donnerstage und an allen Samstagen derselben;
\item in den Vigilien vor Pfingsten und Weihnachten;
\item in den Quatemberzeiten am Samstage; und endlich
\item an allen Freitagen durch's ganze Jahr.\RWfootnote{%
	Es gibt also Fasttage von doppelter Art: solche, an denen \RWbet{bloß Speisenabbruch}, und solche, an denen \RWbet{zugleich Enthaltung von Fleischspeisen} geboten ist.}
\end{aufzb}
\end{aufza}
An den übrigen Fasttagen ist der Genuß des Fleisches nur unter der Bedingung erlaubt, daß man \RWbet{durch alle Sonn- und Feiertage der vierzigtägigen Faste in einer Kirche das Gebet des Herrn und den englischen Gruß fünfmal, und das apostolische}~\RWSeitenw{289}\ \RWbet{Glaubensbekenntniß bete, dann die drei göttlichen Tugenden und Reue und Leid erwecke.}\par
Wer müßte nicht gestehen, daß diese Forderungen sehr mäßig sind, und daß, wenn wir sie wirklich ganz in dem Geiste befolgten, in welchem die Kirche sie befolget sehen will, überaus viel Gutes aus ihnen hervorgehen würde? -- In welchem Geiste die Kirche ihr Fastengebot von uns beobachtet wissen wolle, zeigen die Worte in den Tagzeiten \RWlat{domin.\ I.\,Quadrag.\ ex Sermone S.~Leonis Papae: Ingressuri igitur, dilectissimi, dies mysticos, et purificandis animis atque corporibus sacratius institutos praeceptis Apostolicis obedire curemus; emundantes nos ab omni inquinamento carnis et spiritus, -- ut nemini dantes ullam offensionem, vituperationibus obloquentium non simus obnoxii. Digna enim ab infidelibus reprehensione carpemur, et nostro vitio linguae impiae in injuriam se religionis armabunt, si jejunantium mores a puritate perfectae continentiae discreparint. Non enim in sola abstinentia cibi stat nostri summa jejunii; aut fructuose corpori esca subtrahitur, nisi mens ab iniquitate revocetur?}\RWlit{}{LeoI1}

\RWpar{277}{4.~Uebungen der Wohlthätigkeit gegen Arme und Leidende}
\begin{aufza}
\item Es versteht sich von selbst, daß Handlungen der Wohlthätigkeit gegen Arme und Leidende schon an und für sich nützlich sind; sie sind es aber \RWbet{auch als Tugendmittel} in sofern, als sie das Herz erweichen, und für alles andere Gute empfänglicher machen. Die katholische Kirche hat uns denn diesen Gesichtspunct derselben besonders anempfohlen, damit, wenn wir dem Leidenden einen Dienst erweisen, wir glauben, daß wir selbst einen noch größeren Gewinn davon haben.
\item \RWbet{Spenden, im Gelde} bestehend, und überhaupt Abtretungen eines bald größeren, bald geringeren Theiles von seinem Eigenthume zu einem guten Zwecke sind zwar gar~\RWSeitenw{290}\ nicht die einzige, \RWbet{nicht einmal die verdienstlichste} Art des Wohlthuns; aber sie sind doch diejenigen, zu denen sich die Menschen am Liebsten herbeilassen; zu denen sie auch, als Mitglieder einer größeren Gesellschaft, mit gutem Erfolge am Häufigsten aufgefordert werden können. Auch die Vorsteher der katholischen Kirche haben es deßhalb an solchen Aufforderungen niemals ermangeln lassen. Sie thun es noch heute in jeder gottesdienstlichen Versammlung, bald stillschweigend durch die in jedem Gotteshause aufgepflanzten Opferaltäre, Almosenbüchsen \udgl\  bald ausdrücklich, wenn sie die anwesende Menge nach einer vorhergegangenen Ermahnung einen eigentlichen Opfersumgang anstellen lassen. In manchen Orten sind die Seelsorger beauftragt, sich zu gewissen Zeiten selbst in die Häuser ihrer Gläubigen zu begeben, um da Beiträge zur Unterstützung der Armen zu sammeln.
\item Wie reichlich die Spenden gewesen, welche durch solche Aufmunterungen von Seite der Vorsteher der katholischen Kirche zusammenkamen, das zeigte sich nicht nur gleich bei der ersten christlichen Gemeinde zu Jerusalem, von welcher der heil.\ Lukas bezeuget, ein Jeder habe sein Eigenthum als ein Gemeingut Aller betrachtet (\RWbibel{Apg}{Apostelg.}{4}{32}); sondern hievon sind ein bleibendes Denkmal auch die \RWbet{vielen frommen Anstalten}, welche nur diesen Gaben der Gläubigen ihre Entstehung und ihre Fortdauer in allen Ländern zu verdanken haben: so viele prächtige Tempelgebäude, so viele reiche Stiftungen zur Erhaltung der Seelsorger und der in ihren Kirchensprengeln befindlichen Armen, so viele Schulen, Armen- und Krankenhäuser, \usw\ \usw\
\end{aufza}

\RWpar{278}{5.~Verehrung und Nachahmung verklärter Tugendfreunde}
Daß gute Beispiele von der größten Wirksamkeit sind, wußte man wohl von jeher. Doch hatte es sich kein Volk der Erde einfallen lassen, von diesem Beförderungsmittel der Tugend einen so heilsamen Gebrauch zu machen, als es in der katholischen Kirche geschieht.
\begin{aufza}
\item Nach den Grundsätzen dieser soll es als ein \RWbet{Verlust für die Menschheit} betrachtet werden, wenn \RWbet{irgend}~\RWSeitenw{291}\ \RWbet{ein Beispiel der Tugend in der Verborgenheit bleibt.} Es werden daher alle diejenigen, die einen Tugendhaften kennen, aufgefordert, sein erbauliches Beispiel zu einer allgemeinen Kunde zu bringen.
\item Weil es jedoch gefährlich ist, den Tugendhaften, so lange er auf dieser Erde lebt, also noch den Versuchungen zur Eitelkeit, zum Stolze, zur Trägheit \usw\ ausgesetzt ist, allzu freigebig zu loben, und zu einem Muster der Nachahmung für Andere aufzustellen: so verlangt die katholische Kirche, daß dieses vornehmlich \RWbet{erst nach dem Tode} auch seiner nächsten Anverwandten geschehe.
\item Weil ferner Lebensbeschreibungen, die einen unbekannten Urheber haben, nur wenig Glauben finden, und eben deßhalb auch nur wenig Nutzen stiften können: so sind es die \RWbet{Vorsteher der Kirche} (die Bischöfe und insonderheit der Papst mit einem eigenen Collegio), die das Geschäft auf sich nehmen sollen, den sittlichen Werth desjenigen, dessen besonders tugendhaften Wandel man ihnen angerühmt hat, strenge zu untersuchen, und eine möglichst getreue und glaubwürdige Geschichte seines Lebens aus den verlässigsten Quellen zu schöpfen.
\item Bei dieser Untersuchung ist nicht zu fordern, daß der Mensch \RWbet{ganz vollkommen} gewesen sey; sondern man soll sich begnügen, wenn nur so viele und so erhabene Tugenden da sind, daß sich erwarten läßt, ihre Darstellung werde erbaulich einwirken.
\item Vornehmlich soll man bemüht seyn, \RWbet{Muster von jeder Art der Tugend} aufzustellen, auch solche Muster, wo möglich, aus jedem Geschlechte, aus jedem Lebensalter, aus jedem Stande und Gewerbe, bei jedem Volke und aus jedem Zeitalter hervorzuheben.
\item Aufzeichnen soll man auch jedes in dem Leben solcher Personen oder auch erst nach ihrem Tode sich etwa darbietende \RWbet{Ereigniß, das als ein auffallender Beweis von Gottes Wohlgefallen an ihrer Tugend,} oder als ein Zeichen, wodurch uns Gott irgend etwas Anderes will zu erkennen geben (Wunder), angesehen werden kann. Ereignisse von solcher Art der Vergessenheit überlie\RWSeitenw{292}fern, wenn man doch Mittel hätte, sie glaubwürdig aufzubewahren, heißt einen Raub an der Menschheit und an Gottes Ehre begehen. (\RWbibel{Tob}{Tob.}{12}{7})
\item Es ist gebräuchlich, den Personen, die man auf diese Art als Muster der Nachahmung und Erbauung für alle übrigen Gläubigen aufstellt, den Namen der \RWbet{Heiligen} oder der \RWbet{Seligen,} welchen sonst alle Christen (den letzteren wenigstens nach ihrem Tode) trugen, vorzugsweise zu geben; und eben darum wird auch das von dem päpstlichen Stuhle erlassene Urtheil, daß eine gewisse Person es werth sey, von der gesammten Christenheit als ein Gegenstand der Erbauung und Nachahmung betrachtet zu werden, eine \RWbet{Heilig-} oder \RWbet{Seligsprechung} derselben genannt. Nie hat man gelehrt, daß die Vorsteher der katholischen Kirche in diesem Geschäfte der Seligsprechungen, um so weniger in der Abfassung jener Lebensgeschichten und in der Erzählung der Wunder, die darein aufgenommen werden, \RWbet{unfehlbar} wären. Am Allerwenigsten darf man der katholischen Kirche die Meinung aufbürden, als ob diejenigen, welche von ihren Vorstehern selig gesprochen werden, alle und nur sie ausschließlich zur Zahl der seligen Himmelsbewohner gehörten, und wohl gar eben erst in dem Augenblicke, da ihre Seligsprechung auf Erden vorgehet, in den Himmel aufgenommen würden. Einer so thörichten und boshaften Mißdeutung wird vielmehr ausdrücklich widersprochen.
\item Obgleich es uns aber frei steht, in Betreff einzelner Heiligen, und der von ihnen erzählten Thaten und Wunder unsere eigene Meinung zu haben, wenn wir der herrschenden nicht glauben beitreten zu können: so sollen wir uns doch hüten, durch eine \RWbet{unvorsichtige Verbreitung unserer Zweifel Andere in ihrem frommen Glauben ohne Noth zu stören}. Ob aber dieß nöthig sey oder nicht, das sollen wir nicht ohne Erwägung der doppelten Wahrheit beurtheilen:
\begin{aufzb}
\item daß die Erbauung, die uns ein vorgehaltenes Muster der Tugend gewähret, verliere, wenn wir es für ein erdichtetes halten;
\item daß die Anrufungen, zu denen wir uns in der Voraussetzung entschließen, daß eine gewisse, von unseren kirch\RWSeitenw{293}lichen Vorstehern selig gesprochene Person sich in der That unter der Zahl der seligen Himmelsbewohner befinde, nicht nutzlos seyen, auch in dem Falle, daß wir uns in jener Voraussetzung irren.
\end{aufzb}
\item Um das Gedächtniß der Heiligen desto gewisser zu erhalten, und zwar nicht bloß in Büchern, und in den Köpfen der Gelehrten, sondern selbst bei der großen Menge hatten die Vorsteher der katholischen Kirche den überaus glücklichen Gedanken, für einen jeden Heiligen (oder doch wenigstens für die merkwürdigsten) \RWbet{einen eigenen Tag im Jahre auszuwählen, der seinem Andenken besonders geweiht seyn sollte}; eine Veranstaltung, die zugleich den andern Nutzen hatte, daß bald kein Tag im Jahre übrig blieb, an welchem die Gläubigen nicht durch das Beispiel irgend eines Heiligen zur Tugend aufgemuntert würden.
\item An solchen, dem Andenken eines Heiligen eigends gewidmeten Tagen findet sich nicht nur in jenen Lesestücken, die jeder katholische Geistliche täglich zu lesen verpflichtet ist (Brevier), eine kurze \RWbet{Lebensgeschichte} des Heiligen; sondern es kommt auch in dem öffentlichen Gottesdienste, in den \RWbet{Gebeten} der heiligen Messe, fast immer eine kurze Erwähnung desselben vor; eine Erwähnung, die meistens des Inhaltes ist, Gott wolle uns unter der Bedingung, daß wir die Tugenden dieses Heiligen nachahmen, durch seine Fürsprache gewisse Gnaden verleihen.
\item Um dem Gedächtnisse der Heiligen eine noch größere Feierlichkeit zu verschaffen, wurden selbst \RWbet{Tempel}, die man erbaute, und \RWbet{Altäre} (Opfertische), die man in ihnen errichtete, öfters dem Andenken berühmter Heiligen gewidmet.
\item Und um der Einbildungskraft, die diese Heiligen sich so gerne recht lebhaft vorstellen wollten, zu Hülfe zu kommen, werden auch \RWbet{Bildnisse} derselben (Gemälde, Zeichnungen, Statuen) verfertigt und verbreitet.
\item Konnte man etwa zu dem Besitze einiger (wahrer oder auch nur geglaubter) \RWbet{Ueberbleibsel} (Reliquien) von einer heiligen Person gelangen: so hielt man auch diese in Ehren; ja selbst Gemälde, Bücher, und andere dergleichen~\RWSeitenw{294}\ Gegenstände, die diese Ueberbleibsel auch nur berührt hatten, schätzte man nun um so höher.
\item Endlich wurde die ganz vortreffliche Gewohnheit eingeführt, daß wir ein Jeder gleich bei unserer Aufnahme in die Gemeinde der Christen (bei der Taufe), dann auch bei unserer Bestätigung im Glauben (bei der Firmung), auch wohl beim Eintritte in einen gewissen, besonders wichtigen Stand des Lebens, den \RWbet{Namen} eines oder einiger von diesen Heiligen zu dem unsrigen machen; wodurch wir uns denn verpflichten, die Tugenden dieses Verklärten zu einem ganz vorzüglichen Augenmerk unserer Nachahmung zu machen.
\end{aufza}
\begin{RWanm} 
Alle diese Lehren, Anstalten und Gebräuche haben eine so einleuchtende Vernunftmäßigkeit, und sittliche Güte, daß es für Denjenigen, der der Erkenntniß der Wahrheit nicht absichtlich widerstrebt, gewiß nicht nöthig ist, sie erst eigends zu rechtfertigen. Nur aus der \RWbet{falschen Auslegung}, welche man einigen von diesen Lehren gab, und aus den \RWbet{Mißbräuchen}, in die man hie und da verfiel, läßt sich begreifen, wie Lehren und Verfügungen, die so vortrefflich sind, auch ihre Gegner (namentlich im Protestantismus) gefunden haben können. Man stellte sich nämlich oft vor, daß sich die Vorsteher der katholischen Kirche in ihren Seligsprechungen für unfehlbar ausgeben wollen; man bildete sich ein, daß sie diejenigen Personen, die sie für heilig erklären, für durchaus vollkommen erklären; man fand Lebensbeschreibungen von diesen Heiligen, die voll Erdichtungen waren, den Charakter dieser Personen bis zum Unkenntlichen entstellten, die abgeschmacktesten Wundererzählungen enthielten, \usw ; man sah, daß viele Christen es in der Verehrung dieser Heiligen übertrieben, auf ihre Hülfe mehr als auf Gottes Hülfe bauten, wohl gar vergaßen, daß die Heiligen in ihrer Wirksamkeit nur von Gott abhängig wären; \usw\ \usw\ Statt gegen diese Mißbräuche und Mißdeutung zu eifern, hat man die Lehren und Gebräuche, die dazu Anlaß gaben, selbst angegriffen. Dieses verkehrte Benehmen ließe sich einiger Maßen entschuldigen, wenn man erweisen könnte, daß die besagten Mißbräuche allezeit und überwiegend eintreten müßten. Aber wer könnte dieß darthun? Im Gegentheile muß jeder Vernünftige einsehen, daß die befürchteten Mißbräuche und Mißdeutungen durch einen nur etwas sorgfältigeren Volksunterricht vermieden werden können; wie man~\RWSeitenw{295}\ denn auch schon gegenwärtig, wenigstens in unseren Ländern, kaum noch einige Spuren derselben bei den untersten Volksclassen antrifft. Daher kommt es auch, daß in neuerer Zeit mehrere Protestanten die Nützlichkeit der katholischen Heiligenverehrung eingesehen, und sich sehr günstig über sie ausgesprochen haben; \zB\  der schon vorhin erwähnte \RWbet{Horst, Marheineke,} \uA\ Nur wundern muß man sich über den Vorschlag, welchen der protestantische Gottesgelehrte \RWbet{Kaiser} offenbar nur aus dem Grunde erdachte, weil er die Heiligenverehrung nicht völlig so, wie sie unter uns Katholiken bestehet, aufnehmen wollte. Man solle nämlich der \RWbet{Unschuld}, der \RWbet{Mutterliebe}, der \RWbet{platonischen Liebe}, der \RWbet{Urania}, der \RWbet{ehelichen Treue}, und andern einzelnen Tugenden \RWbet{Tempel} errichten, um so den Eifer in der Verehrung und Ausübung dieser Tugenden zu erhöhen. Wie ungleich geeigneter für diesen Zweck ist doch das Mittel der katholischen Heiligenverehrung! Bei einem Heiligen kann wohl bald diese, bald jene einzelne Tugend vorherrschend seyn, aber es dürfen ihm doch auch nicht die übrigen fehlen; und so ist nicht zu fürchten, daß \RWbet{das Ideal der Tugend in der Einbildung zersplittert} werde, wie bei diesem Vorschlage Kaiser's. Ein Heiliger ist ferner keine bloße Idee, sondern hat wirklich gelebt; seine Verehrung ist also \RWbet{Nährung des Glaubens an die Möglichkeit, ja an die Wirklichkeit einer solchen Tugend} unter uns Menschen! wie viel ausgiebiger, als ein dem bloßen Ideale erbauter Tempel! -- 
\end{RWanm}

\RWpar{279}{6.~Benützung der schönen Künste zu sittlichen Zwecken}
\begin{aufza}
\item Auch die \RWbet{schönen Künste} sieht die katholische Kirche als ein überaus wirksames Mittel zur Beförderung sittlicher und religiöser Gefühle und Gesinnungen an; und erklärt, daß es nur dieser Zweck sey, den sich ein jeder echte Künstler bei allen seinen Hervorbringungen näherer oder entfernterer Weise vorsetzen müsse; ja sie versichert ihn, daß er im entgegengesetzten Falle, wenn er sich beikommen läßt, seine Talente im Widerspruche mit den Grundsätzen der Religion zu gebrauchen, nur der Vollkommenheit seiner Anstrengungen Abbruch thun werde; denn \RWbet{nichts sey vollkommen schön, was nicht auch sittlich gut ist.}~\RWSeitenw{296}
\item Die katholische Kirche behauptet, daß es in ihren religiösen Ideen, in den Erzählungen der Bibel, sowohl des alten als des neuen Bundes, in den Lebensgeschichten der Heiligen, in ihrer eigenen Geschichte, \RWbet{einen nie zu erschöpfenden Vorrath zu Darstellungen für eine jede Art der schönen Künste} gebe, und sie ermuntert Jeden, der dazu Kraft in sich fühlt, \RWbet{an der Verarbeitung dieses Stoffes Theil zu nehmen.} Wie viele der herrlichsten Kunstwerke, die wir auf dem Gebiete der Rede- und Dichtkunst, der Mahler-, Bildhauer- und Tonkunst aufzuweisen haben, verdanken ihr Daseyn nicht dieser Aufforderung?
\item Die katholische Kirche ermuntert ferner Alle, die nicht selbst Künstler sind, wenigstens \RWbet{den in ihnen liegenden Sinn für den Genuß schöner Kunstwerke auszubilden}, um sich derselben dann \RWbet{zur Belebung und Vervollkommnung ihrer sittlichen und religiösen Gefühle und Gesinnungen} bestmöglichst zu bedienen.
\item Die Vorsteher der katholischen Kirche haben von jeher auch Sorge dafür getragen, \RWbet{daß die gelungensten Darstellungen der Kunst, die sich zu sittlich religiösen Zwecken benützen lassen, dazu wirklich angewendet würden.} Sie haben \RWbet{die Schriften der besten Redner und Dichter,} nicht bloß der christlichen, sondern auch anderer, durch \RWbet{vielfältige Abschriften}, nach Erfindung der Buchdruckerkunst auch durch \RWbet{wohlfeile Auflagen} zu verbreiten gewußt; sie haben \RWbet{erbauliche Gemälde} und \RWbet{Bildsäulen}, so viele sie ihrer nur immer habhaft werden konnten, in Tempeln und an andern öffentlichen Orten \RWbet{zur allgemeinen Erbauung aufgestellt}; sie haben die rührendsten und erhebendsten \RWbet{Gesänge und Musikstücke} der Nachwelt aufbewahret, und wenn es thunlich war, selbst in den gottesdienstlichen Handlungen von ihnen Gebrauch machen lassen.
\item Was insbesondere \RWbet{Gemälde} und \RWbet{Bildsäulen} anlangt: so wird es den Gläubigen nicht nur erlaubt, sondern von ihnen sogar gefordert, daß sie diejenigen, die einen unserer Verehrung werthen Gegenstand, \zB\  unsern Herrn~\RWSeitenw{297}\ Jesum oder seine jungfräuliche Mutter Maria, oder sonst eine heilige Person, darstellen, \RWbet{mit einer angemessenen Ehrerbietung behandeln.} Wer könnte dieß tadeln? Es ist uns natürlich, daß wir die Ehrfurcht, die wir gegen eine gewisse Person empfinden, auch auf das Bild, welches sie vorstellen soll, übertragen; und wird dieß Letztere verunehrt: so ist es uns, als ob die vorgestellte Person selbst verunehret würde.
\end{aufza}
\begin{RWanm} 
Gesang und Musik sind ein so vortreffliches Mittel, zur Belebung der Andacht, besonders in einem öffentlichen Gottesdienste, daß man bei allen nur etwas gebildeten Völkern davon Gebrauch gemacht hat. Wie würdevoll aber, wie rührend und herzerhebend die in der katholischen Kirche gebräuchlichen Sangweisen sind, kann Jeder, der unsere gottesdienstlichen Versammlungen besucht, aus eigener Erfahrung kennen lernen. Sollte er die hie und da eingeführte Instrumentalmusik zu kunstreich und zu geräuschvoll finden: so wäre dieß für einen gegen die ausdrückliche Verordnung der Kirche eingeschlichenen Mißbrauch zu halten. Was die Verehrung der Bilder belangt: so ist nicht zu läugnen, daß sie bei Menschen, die auf einer noch \RWbet{sehr niedrigen Stufe der Bildung} stehen, leicht ausarten könne; und eben die Besorgniß einer solchen Ausartung war es, welche so manche Religionsstifter, \zB\  selbst Mosen, veranlasset hatte, die Aufstellung heiliger Bilder gänzlich zu untersagen. Auch in der christlichen Kirche trug man, besonders in Jahrhunderten, wo viele früher dem Bilderdienst ergebenen Heiden zum Christenthume übertraten, einiges Bedenken, Bilder in gottesdienstlichen Gebäuden aufzustellen und verehren zu lassen. Als aber die Vorsteher der christlichen Kirche erkannten, daß sich der Glaube an den alleinigen und wahren Gott schon befestiget habe, daß kein Rückfall in den Götzen- und Fetischdienst mehr zu besorgen stehe, nahmen sie (in dem bekannten Bilderstreite) den Gebrauch und eine angemessene Verehrung \RWbet{schicklicher} Bilder gegen die Gegner derselben (Ikonoklasten) selbst in Schutz; \RWbet{der zweite nicänische} (im Jahr 787) und auch noch der \RWbet{tridentinische Kirchenrath} erklärte, daß die Aufstellung heiliger Bilder im Tempel, und eine Verehrung derselben durch Entblößung des Hauptes, Beugung der Knie \udgl\  zulässig und zu loben sey; vorausgesetzt, daß man (wie dieses ohnehin kein nur etwas Unterrichteter thun wird) nicht von dem Bilde, sondern nur von demjeni\RWSeitenw{298}gen, welchen es vorstellt, und zuletzt nur von Gott, Hülfe erwarte. Wohl zu bemerken ist hier, daß man von dieser Lehre der Kirche noch eben nicht abgewichen seyn muß, wenn man in einem gewissen Sinne des Wortes an eigene \RWbet{Gnadenbilder} und \RWbet{Gnadenorte} glaubt. Es kann sich nämlich fügen, daß die vor einem gewissen Bilde verrichtete Andacht höher zu steigen pflegt, als vor anderen; es kann sich fügen, daß die im Anschauen dieses Bildes verrichteten Gebete mehrmals eine recht auffallende Erhörung finden: und es ist dann verzeihlich, wenn man ein solches Bild höher zu schätzen anfängt, als manches andere. Wenn man den Grund hievon ganz in \RWbet{natürlichen} Umständen sucht, \zB\  in der Beschaffenheit des Bildes selbst, das als ein vorzüglich gelungenes Werk der Kunst inniger rühren kann; oder in der Beschaffenheit des Ortes, an dem es aufgestellt ist, der durch Erinnerung an gewisse merkwürdige Begebenheiten, die sich hier einst zutrugen, oder durch andere Umstände, besonders geeignet ist, das Gemüth zu frommen Gefühlen zu stimmen, \udgl : so däucht mir, es liege in einem solchen Glauben an besondere Gnadenbilder und besondere Gnadenorten noch immer nichts, das durchaus Tadel verdiente. Allein weil solche Vorstellungen von ungebildeten Menschen sehr leicht gemißbraucht werden, weil diese die Wirksamkeit solcher Gnadenbilder und Orte leicht übertreiben, leicht ein zu großes Vertrauen auf die Erhörung einer daselbst zu verrichtenden Andacht setzen, sich dann mit allzu großer Begier nach dem Besuche eines solchen Ortes sehnen, \udgl : so lassen \RWbet{weise} Vorsteher der katholischen Kirche, wenn sie kein anderes Mittel, das Volk zu bescheiden wissen, das Bild selbst, welches der Gegenstand einer so ausschweifenden Verehrung zu werden drohet, lieber selbst entfernen. -- Jeder Unparteiliche wird übrigens eingestehen müssen, daß die Darstellungen selbst, die man in den von der Kirche \RWbet{allgemein} gebrauchten Bildern antrifft, durchaus nichts Unanständiges enthalten; obwohl nicht Alle dem Geschmacke Aller in einem gleichen Grade entsprechen und entsprechen können. 
\end{RWanm}

\RWpar{280}{7.~Zweckmäßige Bearbeitung der eigenen Neigungen}
\begin{aufza}
\item Eines der wichtigsten Mittel zur Tugend ist eine zweckmäßige Bearbeitung der eigenen Neigungen. -- Ich verstehe aber unter einer \RWbet{Neigung} diejenige \RWbet{Beschaffenheit unseres Begehrungsvermögens}, zu Folge der wir~\RWSeitenw{299}\ gewisse Handlungsweisen \RWbet{unwillkürlich wünschen}, so oft wir uns sie als möglich vorstellen, und zwar \RWbet{auch ohne uns erst der Gründe, warum wir sie wünschen, deutlich bewußt werden zu müssen}. Jeder Mensch hat dergleichen Neigungen; auch ohne daß er es darauf anlegt, entwickeln sich verschiedene Neigungen in seinem Herzen, und es ist gar nicht möglich, ganz ohne Neigungen zu bleiben. Begreiflicher Weise aber kommt auf die Neigungen eines Menschen Vieles an, und seine Tugend sowohl als auch seine Glückseligkeit hängt einem großen Theile nach von der Beschaffenheit seiner Neigungen ab. Hat er gute Neigungen, \dh\  Neigungen zu solchen Handlungsweisen, die entweder allgemein, oder doch in den meisten Fällen erlaubt und gut sind: so wird er das Gute um so sicherer thun, es wird ihm überdieß auch leicht und angenehm werden; also wird seine Tugend sowohl, als seine Glückseligkeit gewinnen. Im Gegentheile viel böse Neigungen, Neigungen zu solchen Handlungsweisen, die, wenn nicht allgemein, doch in den meisten Fällen unerlaubt sind, werden ihm die Erfüllung seiner Pflichten durch diesen Umstand gar sehr erschweren, und in vielen Fällen wird er sie übertreten.
\item Glücklicher Weise hat der vernünftige Mensch auf die Entstehung und Fortdauer, auf das Wachsthum oder die Unterdrückung und Ausrottung seiner Neigungen einen sehr großen Einfluß. Und eben darum macht es das Christenthum Jedem zur Pflicht, diesen Einfluß auf das Beste zu nützen, und also Neigungen zum Guten in sich hervorzubringen; die bösen Neigungen dagegen, welche der Zufall etwa erzeugt haben sollte, nach aller Möglichkeit bei sich zu schwächen und zu unterdrücken.
\item \RWbet{Um gute Neigungen in uns hervorzubringen}, empfiehlt uns aber das Christenthum den Gebrauch folgender Mittel:
\begin{aufzb}
\item Wir sollen \RWbet{die Pflichtmäßigkeit der Handlung, zu der wir uns Neigung beibringen wollen, oft in Erwägung ziehen}; ihre wohlthätigen Folgen für Andere sowohl, als für uns selbst, \zB\  die Freude, welche uns das Bewußtseyn der guten That gewähren~\RWSeitenw{300}\ wird, den Beifall der Vernünftigen, den wir so uns verschaffen, die Belohnungen, die selbst im andern Leben noch unser warten, \usw\ uns oft und lebhaft vergegenwärtigen.
\item Wir sollen \RWbet{die Vorstellung der Handlung, zu der wir Neigung in uns hervorbringen wollen, vermittelst der bekannten Gesetze der Ideenverknüpfung mit lauter solchen Vorstellungen zu verbinden trachten, die von gewissen angenehmen Empfindungen und Gefühlen bei uns begleitet zu seyn pflegen.} Hiezu dient unter Anderem
\begin{aufzc}
\item daß wir die Handlung oft in vergnügter Stunde verrichten, oder doch gleich darauf den Genuß irgend eines erlaubten Vergnügens folgen lassen. Hiezu dient ferner
\item der Umgang mit Menschen, die diese Neigung in einem höheren Grade besitzen, \usw\
\end{aufzc}
\end{aufzb}
\item Zur \RWbet{Unterdrückung unserer bösen Neigungen} dagegen empfiehlt uns das Christenthum folgende Mittel:
\begin{aufzb}
\item Wir sollen die \RWbet{Pflichtwidrigkeit und die verderblichen Folgen der Handlung recht oft und lebhaft betrachten}; und hiebei
\item vornehmlich \RWbet{diejenigen Gründe recht auseinander setzen, welche besonders auf uns den stärksten Eindruck zu machen fähig sind.}
\item Wir sollen \RWbet{die Schlußreihen, auf welchen diese Gründe beruhen, durch öftere Wiederholung uns recht geläufig machen}, und möglichst abkürzen.
\item Wir sollen \RWbet{mit dem Gegenstande unserer Neigung allerlei widerliche Vorstellungen in Verbindung setzen};
\item \RWbet{die wirkliche Ausübung der bösen That in uns nie ungestraft lassen;}
\item den \RWbet{Umgang mit Menschen, die dieser bösen Neigung gleich uns ergeben sind, meiden;}
\item die Siege, die wir uns zuweilen abgewinnen, \RWbet{uns selbst durch ein erlaubtes Vergnügen lohnen};~\RWSeitenw{301}
\item \RWbet{zur Zeit der Versuchung}, wenn uns die Vorstellung des Bösen ein Wohlgefallen daran entlocken will, \RWbet{Geberden und Mienen des Unwillens annehmen}, um ihn auf diese Art wirklich hervorzubringen;
\item \RWbet{Alles vermeiden, was uns an den Gegenstand unserer bösen Neigung erinnern könnte};\stepcounter{enumii}
\item uns deßhalb \RWbet{stets mit solchen Gegenständen beschäftigen, die unsere ganze Aufmerksamkeit für sich allein in Anspruch zu nehmen vermögen}, und besonders zur Zeit der Versuchung die bösen Gedanken dadurch mit aller Macht zu verdrängen suchen, daß wir die Aufmerksamkeit unseres Geistes auf etwas Anderes, das für uns Wichtigkeit hat, richten;
\item den ersten \RWbet{Anfängen zum Bösen}, der ersten mit Wohlgefallen verbundenen Vorstellung \RWbet{widerstehen}.
\item Uns mit \RWbet{bestimmten und recht in das Einzelne gehenden Vorsätzen} ausrüsten,
\item \RWbet{passende Denksprüche uns recht geläufig machen}; und 
\item an schicklichen Orten selbst auch \RWbet{zweckmäßige Erinnerungszeichen aufpflanzen.}
\item Ist die Befriedigung unserer Neigung sehr zur Gewohnheit geworden: so ist eine \RWbet{fast ununterbrochene Aufmerksamkeit} auf uns selbst nöthig, weil wir das Böse dann thun, öfters auch ohne uns dessen nur selbst deutlich bewußt zu seyn.
\item Nicht selten ist sogar die \RWbet{Hülfe Anderer} nöthig, die uns erinnern \usw\
\item \RWbet{Auch wenn es uns gelungen ist, das Böse Jahre lang} schon zu unterlassen, dürfen wir oft nicht glauben, schon Alles gewonnen zu haben; sondern \RWbet{noch immer müssen wir auf unserer Hut verbleiben}.
\item Wofern die Neigung, die wir bekämpfen, \RWbet{keine schon an sich selbst unerlaubte}, sondern nur eine durch ihre allzu oftmalige Ausübung schädliche Handlung betrifft: kann es oft nöthig, oft wenigstens dienlicher seyn, \RWbet{bei ihrer Entwöhnung nur stufenweise zu Werke zu gehen}, und also Anfangs nur kleine, in der Folge aber immer größer werdende Zeiträume fest\RWSeitenw{302}zusetzen, innerhalb deren wir uns die Ausübung dieser Handlung verbieten.
\item Um unseren guten \RWbet{Vorsätzen} mehr Kraft und Wirksamkeit zu geben, müssen wir sie \RWbet{mit dem Gedanken Gott verknüpfen} und zu Gebeten erheben.
\item Bei Ablegung böser Neigungen ist es, wo jene natürlichen Mittel, die das katholische Christenthum in seiner \RWbet{Buß- und Besserungsanstalt} vorschreibt, die hier auch noch mit gewissen übernatürlichen Segnungen verbunden sind, benützt werden sollen.
\end{aufzb}
\end{aufza}
\begin{RWanm} 
Auch zu den Tugendmitteln, welche in der katholischen Kirche nicht nur empfohlen, sondern selbst häufig angewendet wurden, dürfen wir die sogenannten \RWbet{drei evangelischen Räthe}, ingleichen die zu ihrer größeren Befolgung dienenden \RWbet{Klostergelübde} sammt einer Menge sich auf sie beziehenden Gebräuche und Einrichtungen zählen. Nur müssen wir, wenn wir den Zweck dieser Gegenstände nicht einseitig auffassen wollen, bemerken, daß die katholische Kirche sie uns nicht bloß als Tugendmittel, sondern in jeder Rücksicht empfehle, in der nur irgend etwas Gutes, es sey mittelbar oder unmittelbar, aus ihnen hervorgehen kann. Eine Handlungsweise, aus der in keinem Betrachte irgend etwas Gutes entspringt, hat die katholische Kirche nie empfohlen; in einer Entsagung eines Genusses, aus welchem durchaus kein Vortheil erwächst, hat sie nie etwas Verdienstliches oder Gott Wohlgefälliges gesucht, da ja auch sie mit dem Apostel Paulus erkannte, daß Alles, was sich auf Erden befindet, zu einem dankbaren Genusse für die Menschen von Gott erschaffen worden sey. Allein dieß hinderte sie nicht, gar Manches anzuempfehlen, was fast den Anschein einer zwecklosen Entsagung hat, und von den Gegnern unserer Religion auch wohl dafür erkläret worden ist. Von dieser Art sind besonders jene \RWbet{drei evangelischen Räthe}, in welchen sie die \RWbet{Ergreifung einer freiwilligen Armuth,} eine entweder lebenslängliche oder auch nur auf eine gewisse Zeit beschränkte \RWbet{Enthaltsamkeit von den Genüssen des Geschlechtstriebes}, endlich auch die \RWbet{Verläugnung des Eigenwillens} oder den Gehorsam, zwar eben \RWbet{nicht allgemein}, aber doch unter gewissen Umständen und für gewisse Personen als sehr verdienstliche Handlungen anpreiset. Und irret sie etwa in diesen Anpreisungen? Einmal ist doch gewiß, daß es bis auf den heutigen Tag, leider! nur zu viel~\RWSeitenw{303}\ Menschen gibt, die ihre nothwendigsten Lebensbedürfnisse nicht zu befriedigen vermögen, weil es ihnen an den hiezu erforderlichen irdischen Gütern gebricht. Nicht minder gewiß ist es ferner, daß diese Unglücklichen nicht darum Mangel leiden müssen, weil die Erde überhaupt die uns zum Leben nöthigen Güter nicht in hinreichender Menge hervorbringt, oder bei einer zweckmäßigen Bearbeitung nicht hervorbringen könnte, sondern vornehmlich nur darum, weil diese irdischen Güter \RWbet{ungleich vertheilt} sind, weil mancher Einzelne eine so große Menge derselben an sich zu bringen gewußt hat, als zur Befriedigung für die Bedürfnisse von Tausenden hinreichend wäre. Und unter solchen Verhältnissen sollte es nicht etwas Verdienstliches seyn, dem Streben nach Reichthum freiwillig zu entsagen? Es sollte nicht lobenswerth seyn, statt nur für sich selbst und für die Vermehrung seines Eigenthums zu sorgen, seine Kräfte vielmehr nur dem gemeinen Besten, nur der Befriedigung des Bedürfnisses Anderer zu widmen, und zur Belohnung für sich kaum so viel anzunehmen, als bei einer gleichen Vertheilung unter Alle auf jeden Einzelnen käme? Wie wohlthätig wirken Menschen, die so gesinnt sind, schon durch ihr bloßes Beispiel auf ihre Umgebung ein? Oder was kann beschämender seyn für jeden Selbstsüchtigen, als wenn er sehen muß, daß gerade Diejenigen, welche das Meiste verdienen, sich mit dem Wenigsten begnügen? -- Allein nicht minder als diese freiwillige Armuth verdienet es auch die \RWbet{Tugend der Enthaltsamkeit}, daß man die Menschen auf ihren Werth aufmerksam mache. Wie oft treten nicht selbst bei verehelichten Personen Umstände ein, die es bald räthlich, bald nothwendig machen, auf jeden Genuß einer sinnlichen Liebe für eine geraume Zeit zu verzichten? Und wie viel andere Menschen gibt es, welche sich durchaus nicht in solchen Verhältnissen befinden, daß es recht und vernünftig gethan wäre, sich zu verehelichen? Menschen \zB , bei welchen der Geschlechtstrieb keine besondere Lebhaftigkeit äußert, oder die mit gewissen körperlichen Gebrechen behaftet, oder nicht in der Lage sind, eine Familie gehörig ernähren und versorgen zu können, oder die einen allzugefährlichen Lebensberuf ergriffen haben, die einem baldigen Tode oder doch wenigstens einem sehr unruhevollen Leben und harten Verfolgungen entgegensehen müssen, thun gewiß sehr wohl, wenn sie unverehelichet bleiben. Allein es wird (\RWparnr{206}) das Opfer, das solche Menschen zu bringen haben, gar sehr erleichtert, wenn hohe Begriffe von der Verdienstlichkeit des jungfräulichen Lebens allgemein herrschen. -- Endlich ist auch kein~\RWSeitenw{304}\ Zweifel, daß überall, wo Menschen in naher Verbindung mit einander leben, und durch vereinigtes Wirken etwas zu Stande bringen sollen, nichts dem Gedeihen ihrer Unternehmung und ihrer eigenen Zufriedenheit mehr im Wege stehe, als jener \RWbet{Eigenwille}, der ohne Beachtung der Ansichten Anderer nur seinem eigenen Sinne stets folget. Wie sollte es also nicht eine Tugend seyn, wenn wir in solchen Verhältnissen den eigenen Willen gerne dem Willen der Uebrigen opfern, und demjenigen, der es durch seine Geistesüberlegenheit verdient, oder den die Wahl Mehrerer einmal dazu erkoren hat, daß er uns leite, freudig gehorchen in Allem, wovon wir nicht deutlich erkennen, daß der Gehorsam hier nachtheilig wäre? -- Muß man es aber aus den so eben angedeuteten Gründen gestehen, daß die Handlungsweisen, welche uns die katholische Kirche in ihren drei evangelischen Räthen empfiehlt, alle in Wahrheit lobenswerth sind: so kann man es auch nicht tadeln, wenn sie es zuläßt, ja sogar begünstigt, daß wir, zu desto sicherer Beobachtung eines solchen Verfahrens, oft durch ein eigenes Gelübde uns verbinden. Solche Gelübde sind es, welche man vornehmlich von denjenigen fordert, die sich in einen sogenannten \RWbet{geistlichen Orden} wollen aufnehmen lassen. Dergleichen Orden hat es in der katholischen Kirche gar mancherlei gegeben und es gibt deren noch jetzt eine nicht unbeträchtliche Anzahl, bestimmt bald für Personen des männlichen, bald auch des weiblichen Geschlechtes. Der Zweck, den man bei der Errichtung dieser Orden angab, war nie ein anderer, als den Personen, die eine solche Lebensweise nicht ohne göttlichen Beruf erwählen würden, Gelegenheit zu verschaffen, einen viel höheren Grad sittlicher Vollkommenheit zu erklimmen, als sie es außerdem vermocht haben würden; ingleichen sich durch die Ausführung gewisser in Gesellschaft leichter zu vollendender Arbeiten, \zB\  durch schriftstellerische Unternehmungen, durch öffentlichen Unterricht, durch Pflege der Kranken, durch Urbarmachung des Bodens \udgl\  irgend ein wichtiges Verdienst um die Menschheit beizulegen. Wirklich wird auch kein Unparteilicher, der die Geschichte der christlichen Kirche kennt, in Abrede stellen, daß durch die mancherlei Orden und Klöster, die in derselben allmählich aufgekommen sind, und zum Theile noch jetzt bestehen, ungemein viel Gutes gestiftet worden sey, und noch gestiftet werde. Ein Anderes aber ist es, ob diese Anstalten, so gut sie gemeint waren, nicht doch auch zu vielem Bösen eine Veranlassung gegeben, ja ob dieß Böse auch zuweilen nicht das Gute derselben sogar überwogen habe?~\RWSeitenw{305}\ Ein Anderes ist's, ob nicht die Regeln und Einrichtungen, die wir in diesen Orden antreffen, manche Gebrechen haben; und wohl so große, daß die Verbesserung, ja daß eine gänzliche Umschmelzung dieser Anstalten ein äußerst dringendes Bedürfniß, zumal für unsere Zeit, sey? Ueber dieß Alles kann Jeder urtheilen nach seiner eigenen Einsicht, ohne daß er darum aufhört, ein Verehrer der Kirche zu seyn; zumal, wenn er bedenkt, daß die Vorsteher der katholischen Kirche die Unvollkommenheiten dieser menschlichen Anstalten niemals verheimlichet, daß sie sich ihre Verbesserung immer sehr angelegen seyn ließen, und daß sie Manche derselben, die ihnen mehr Schaden als Nutzen zu stiften schienen, selbst aufgehoben haben; daß sich endlich auch kein einziger von den Vorstehern der Kirche gebilligter Orden dürfte aufweisen lassen, der in seinen Regeln und Einrichtungen irgend etwas der Sittlichkeit geradezu Widersprechendes enthalten hätte. Mögen inzwischen die jetzt bestehenden Orden auch noch so vollkommen seyn: so ist doch so viel gewiß, und durch das Benehmen der eifrigsten Verbreiter derselben, so wie durch die Gebräuche der Kirche selbst verbürget, man thue wohl daran, junge Personen zu warnen, daß sie in keinen solchen Orden eintreten mögen, ohne die sorgfältigste Prüfung, ob sie auch wirklich dazu berufen sind, vorausgeschickt zu haben. 
\end{RWanm}

\RWpar{281}{Die Lehre des Katholicismus von den sieben Heiligungsmitteln, und zwar zuerst von ihrem gültigen und pflichtmäßigen Gebrauche im Allgemeinen}
\begin{aufza}
\item Nebst den bisher beschriebenen Tugendmitteln, die eine bloß natürliche, \dh\  auch ohne Offenbarung einleuchtende Wirksamkeit haben, gibt es noch einige, die außer ihrer natürlichen Zuträglichkeit auch gewisse \RWbet{übernatürliche}, \dh\  nur aus der Offenbarung uns bekannte Wirkungen haben, und den besonderen Namen der \RWbet{Sacramente} oder der \RWbet{Heiligungsmittel} führen.
\item Die katholische Kirche zählt deren \RWbet{sieben}. --
\item Bei dem Gebrauche eines jeden sind wenigstens \RWbet{zwei} Personen, ein \RWbet{Ausspender} nämlich und ein \RWbet{Empfänger} nöthig.
\item Es unterscheidet aber die Kirche überhaupt eine \RWbet{doppelte Art des Gebrauches} bei jedem Heiligungsmittel, einen\RWbet{ bloß gültigen} nämlich und einen \RWbet{würdigen.}~\RWSeitenw{306}
\begin{aufzb}
\item \RWbet{Bloß gültig} ist der Gebrauch eines Heiligungsmittels, wenn er von solcher Art ist, daß in der That gewisse \RWbet{übernatürliche} (nur durch die Offenbarung bekannte) \RWbet{Wirkungen} eintreten, von welchen jedoch zu bemerken ist, daß sie auf Seite dessen, der hier bloß gültig, nicht aber auch würdig und wie er sollte, vorging, nicht wohlthätig, sondern \RWbet{nachtheilig} sind; \dh\  ein unwürdiger Ausspender oder Empfänger zieht sich nur schädliche Folgen, nur \RWbet{Strafen}, und dieß zwar übernatürliche, \dh\  solche zu, wie sie nur aus der Offenbarung selbst bekannt sind.
\item \RWbet{Würdig} dagegen heißt der Gebrauch eines Heiligungsmittels, wenn er \RWbet{vollkommen so beschaffen ist, wie er seyn soll.} Ein solcher bringt allemal nur überaus wohlthätige Wirkungen für den Empfänger sowohl als für den Ausspender hervor.
\end{aufzb}
\item Zur bloßen Gültigkeit eines Heiligungsmittels wird nach katholischer Lehre Folgendes erfordert:
\begin{aufzb}
\item Ein \RWbet{Ausspender}, welchem die Kirche die \RWbet{Fähigkeit}, ein Heiligungsmittel von dieser Art gültig auszuspenden, zuerkannt hat; wobei noch ferner erfordert wird, daß er diejenigen Handlungen, welche die Kirche bei diesem Heiligungsmittel \RWbet{wesentlich} nennt, in ernster Absicht verrichte.
\item Ein \RWbet{Empfänger}, welchem die Kirche die \RWbet{Fähigkeit} zuerkannt hat, ein Heiligungsmittel von dieser Art mit Gültigkeit zu empfangen; und von seiner Seite, falls er erwachsen ist, auch der \RWbet{geäußerte Wille,} daß man ihm dieses Heiligungsmittel ertheile.
\end{aufzb}
\begin{RWanm} 
Man hat es der katholischen Kirche verarget, daß sie zur bloßen Gültigkeit des Heiligungsmittels nicht noch viel Mehreres, nicht insbesondere auch eine andächtige Gesinnung von Seite des Ausspenders sowohl als des Empfängers fordere. Allein gerade darum, weil die Kirche lehrt, daß ein Heiligungsmittel, wenn es auch ohne Andacht gespendet oder empfangen wird, gültig sey, so bald nur gewisse Handlungen wirklich verrichtet worden sind: so weiß ein Jeder, der Ausspender sowohl als der Empfänger, daß er sich seinerseits auf das Schwerste versündige, und eine übernatürliche Strafe verwirke, wenn er nicht Alles das beobach\RWSeitenw{307}tet, was die Kirche zu einem nicht bloß gültigen, sondern selbst würdigen Gebrauche vorschreibet. 
\end{RWanm} 
\item Soll aber ein Heiligungsmittel nicht auf bloß gültige, sondern auch \RWbet{würdige Art} gespendet und empfangen werden, so wird noch ferner erfordert:
\begin{aufzb}
\item \RWbet{Von Seite des Ausspenders}, daß er
\begin{aufzc}
\item sich selbst \RWbet{im Stande der Gnade}, \dh\  in einer Gott wohlgefälligen Verfassung befinde, und also wenigstens von jeder schweren Sünde frei sey; daß er
\item auch \RWbet{den Empfänger des Empfanges würdig halte}, oder ihm doch nicht ohne Verletzung irgend einer höheren, \zB\  natürlichen, Pflicht den verlangten Empfang verweigern könne; daß er
\item endlich auch alle ihm zu Gebote stehenden Mittel zu seiner sowohl, als des Empfängers und aller Anwesenden \RWbet{möglich größten Erbauung} anwende. 
\end{aufzc}
\item \RWbet{Von Seite des Empfängers}, daß er sich
\begin{aufzc}
\item zu dem Empfange dieses Heiligungsmittels \RWbet{so gut, als es in seiner Lage möglich ist, bereite}; allenfalls also auch durch den Gebrauch gewisser anderer Heiligungsmittel, wenn er derselben schon empfänglich ist; daß er
\item auch den \RWbet{Ausspender nicht für unwürdig halte}, oder durch eine höhere Pflicht gedrungen sey, ihn zur Leistung dieser geistlichen Hülfe aufzufordern, \zB\  weil eben kein Anderer da ist \udgl ; daß er endlich
\item zur Zeit des Empfanges selbst Alles anwende, was zu seiner eigenen sowohl, als auch der Anwesenden \RWbet{Erbauung} dienen kann.
\end{aufzc} 
\end{aufzb}
\begin{RWanm} 
Zu Folge dieser Lehre kann jeder, der bei dem Gebrauche eines Heiligungsmittels, es sey als Ausspender oder Empfänger, beschäftiget ist, aller hier zu gewinnenden Segnungen für sich selbst theilhaftig werden, wenn der Andere nur jene äußere Handlungen, welche die Kirche bei diesem Sacramente als wesentlich erkläret hat, gleichviel mit welcher Gesinnung und Andacht, verrichtet; vorausgesetzt, daß nur er für sich selbst Alles thue, was oben angegeben wurde. Die katholische Theologie drückt dieß durch die Redensart aus: \RWlat{Sacramenta conferunt gratiam}~\RWSeitenw{308}\ \RWlat{\RWbet{ex opere operato, non ex opere operantis} (Concil.\ Trident.\ sess.\,7.\ c.\,8.)}\RWlit{}{Tridentinum1} -- eine Aeußerung, welche die Protestanten so auslegen, als ob die Gnade, welche uns durch das Sacrament zu Theil wird, von der Gesinnung Beider, des Ausspenders sowohl als des Empfängers, ganz unabhängig wäre, so daß sie hierin eine stillschweigende Erlaubniß finden wollen, die Sacramente ohne alle Andacht sowohl ausspenden als auch empfangen zu dürfen. So ist es aber gar nicht; sondern die Kirche lehrt, daß nur derjenige auch ohne eigene Andacht gewisse Gnaden durch ein Sacrament empfange, der außer Stand ist, Gefühle der Andacht zu erwecken, \zB\  ein Kind, wenn es getauft wird, oder ein Sterbender, dem man die letzte Oelung reichet. Wer aber andächtig seyn könnte und es nicht ist, von dem behauptet die Kirche, daß er nicht nur keine Gnade erlange, sondern im Gegentheile sich schwer versündige und einen Gottesraub begehe, dieß zwar nicht nur bei dem Empfange, sondern selbst bei der bloßen Ausspendung des Sacramentes. Wenn sie aber sagt, daß die den Sacramenten eigenthümliche Gnade \RWbet{nicht von dem Handelnden, sondern von der verrichteten Handlung an sich} abhänge: so will sie hiedurch nur zweierlei andeuten:
\begin{aufzb}
\item daß weder der Ausspender noch der Empfänger sich einbilden solle, die Segnungen, die sie erlangen, durch ihre eigene Andacht verdient zu haben, da sie vielmehr nur ein unverdientes Geschenk sind, das Gottes unendliche Huld mit der Verrichtung jener heil.\ Handlung verknüpft hat;
\item daß ferner weder der Ausspender noch der Empfänger, wenn er das Seinige thut, einen wesentlichen Verlust an jener übernatürlichen Gnade des Sacramentes bloß dadurch erleide, daß nicht auch der Andere das Seinige thue. Das ist nun wohl sehr billig; und im entgegengesetzten Falle, wenn die einem Sacramente verheißenen Segnungen an die Bedingung gebunden wären, daß auch der Andere Alles das Seinige thue, so könnten wir nie völlig sicher seyn, daß wir dieser Segnungen theilhaftig werden. 
\end{aufzb}
\end{RWanm}
\end{aufza}

\RWpar{282}{Die Lehre des Katholicismus von der Aufnahmsfeier, oder der heiligen Taufe}
\begin{aufza}
\item Wer immer nach reiflicher Prüfung des katholischen Christenthums die Wahrheit desselben erkannt hat, an den~\RWSeitenw{309}\ macht eben dieses Christenthum die Aufforderung, \RWbet{daß er die ihm gewordene Ueberzeugung so öffentlich als möglich kund thue,} und zugleich feierlich gelobe, daß er sein Leben künftig ganz nach den Vorschriften desselben einrichten wolle.
\item Auf dieses Gelübde soll er, wenn sich nach Gründen der Wahrscheinlichkeit vermuthen läßt, \RWbet{daß es ihm Ernst damit sey}, und daß er der Kirche nicht zur Schande leben werde, wenn ferner auch eigene \RWbet{Zeugen} (Taufpathen) da sind, welche für seinen bereits seit längerer Zeit geführten rechtschaffenen Wandel sprechen, von einem Vorsteher der Gemeine unter die Zahl der Christen so feierlich als möglich aufgenommen werden. Das Zeichen dieser Aufnahme soll eine \RWbet{Reinigung des Leibes}, durch ein Bad, oder durch ein diese Reinigung nur andeutendes Begießen oder Besprengen mit natürlichem Wasser seyn; wobei der Bischof, oder wer sonst diese Handlung verrichtet, den eigentlichen Zweck derselben durch die bestimmten Worte ausdrücken soll: \anf{Hiemit taufe (\dh\  widme) ich dich der Verehrung Gottes des Vaters, des Sohnes und des heiligen Geistes.}
\item Nebst den natürlichen Vortheilen der Erbauung, welche aus der gehörigen Verrichtung dieser Handlungen nicht nur der Täufling selbst, sondern auch alle anwesenden Personen erfahren, gewinnt der Erstere auch noch folgende \RWbet{übernatürliche Gnaden}:
\begin{aufzb}
\item Die \RWbet{Nachlassung aller Sünden}, welche er etwa noch vor der Taufe begangen, wenn anders er sie gehörig bereuet und den festen Vorsatz der Besserung hat.
\item Die Nachlassung auch selbst der \RWbet{Erbsünde}; daher er denn, der vorhin ein Gegenstand des göttlichen Mißfallens war, sich jetzt betrachten darf als einen \RWbet{Gegenstand des göttlichen Wohlgefallens}, als Gottes \RWbet{Ebenbild}, und als erhoben zur Kindschaft Gottes und Brüderschaft Jesu Christi.
\item Eine besondere \RWbet{Stärkung des heil.\ Geistes}, durch die es ihm möglich gemacht werden wird, sein Leben in allen Stücken nach den jetzt anerkannten Pflichten des Christenthums einzurichten.~\RWSeitenw{310}
\item Da er jetzt in die Gemeinschaft der Christen eintritt: so gewinnt er auch \RWbet{alle Rechte und Vortheile, welche nur eben erst aus dieser Gemeinschaft entspringen.} Hieher gehört das Recht zur Theilnahme an allen übrigen Heiligungsmitteln, die Antwartschaft auf den Himmel oder auf jene ewige Seligkeit, die nur erst Christus seinen Anhängern im anderen Leben ausgewirkt hat. Ihm kommen von nun an auch alle Fürbitten, die von den Gläubigen für ihre Brüder tagtäglich angestellt werden, zu Statten, \usw\
\end{aufzb}
\item Wenn es besonderer Umstände wegen (\zB\  wegen einer Krankheit) nicht füglich angeht, daß die Taufe öffentlich, oder vom Bischofe selbst verrichtet werde: so mag sie von seinem Stellvertreter, dem Seelsorger des Ortes oder in seiner Ermanglung von einem seiner Gehülfen, im Nothfalle wohl auch von einem andern Mitgliede der christlichen Gemeine vollzogen werden. Selbst Taufen der Abtrünnigen, wenn sie doch noch Anhänger Christi sind und die bei der Taufe oben (Nr.\,2.) als wesentlich angegebenen Handlungen beobachten, sind gültig.
\item Es ist nicht nur erlaubt, sondern selbst löblich ist es, Kinder in einem Alter, da sie die Lehren des Christenthumes noch gar nicht zu verstehen vermögen, \zB\  gleich in den ersten Stunden ihres Lebens, zu taufen, wofern sie nur christlichen, oder doch solchen Eltern und Vormündern zugehören, die ihre Einwilligung hiezu ertheilen, und wenn Personen da sind, die diese Kinder einst auch im Christenthume zu unterrichten versprechen. Solche Personen (Taufpathen) sollen, wofern es möglich ist, zu jeder Taufe beigezogen werden.
\item Doch sollen Menschen, welche auf diese Art als Kinder schon getauft wurden, in der Folge der Jahre, wenn sie das Christenthum kennen gelernt, die feierliche Versicherung ihrer Ueberzeugung von der Göttlichkeit dieser Religion, und das Gelübde, nach den Grundsätzen derselben zu leben, öffentlich nachtragen, was in dem Sacramente der Firmung geschieht.
\item Ist Gelegenheit da, die Taufe feierlich zu verrichten: so werden heut zu Tage von den Vorstehern der katholischen~\RWSeitenw{311}\ Kirche in unsern Ländern folgende Gebräuche vorgeschrieben. Eine solche Taufe wird
\begin{aufzb}
\item an einem geheiligten Orte in der Versammlung der Gläubigen,
\item in Gegenwart mehrerer Zeugen, besonders der Pathen ertheilet.
\item Das Wasser, dessen man sich bedient, ist dazu eigends geweihet, \dh\  durch gewisse eigene Gebete und Feierlichkeiten dazu bestimmt.
\item Der Taufende gebietet dem Teufel, unter dessen Macht der zu Taufende bisher gleichsam gestanden war, ihn zu verlassen, bezeichnet ihn mit dem Zeichen des heiligen Kreuzes, legt ihm die Hände auf, \udgl\ 
\item Er legt dann Salz auf seine Zunge, um anzudeuten, daß Christen (nach \RWbibel{Mt}{Matth.}{5}{13}) das Salz der Erde seyn sollen.
\item Er fordert ihn endlich auf, daß er selbst laut und öffentlich dem Teufel abschwöre; auch
\item das Bekenntniß seines Glaubens ablege, und
\item tauft ihn erst, nachdem er ausdrücklich erklärt hat, daß dieses sein ernster, wohl überlegter Wille sey.
\item Nun salbt er ihn noch als einen Kämpfer für das Reich der Himmeln mit geheiligtem Oele;\stepcounter{enumii}
\item legt ihm ein weißes Gewand als Sinnbild der Unschuld an; und
\item reicht ihm endlich eine brennende Kerze, um anzudeuten, daß er nun seyn müsse (\RWbibel{Mt}{Matth.}{5}{14}) ein Kind des Lichtes, ein Freund der Wahrheit, und ein Beförderer heilsamer Entschlüsse auf Erden.
\end{aufzb}
\item Wer immer mit Beobachtung dessen, was allein wesentlich bei dieser Handlung ist, und wenn er erwachsen war, mit seiner Einwilligung, und also gültig getauft worden ist, dessen Seele ward, bildlicher Weise zu reden, ein \RWbet{Merkmal eingeprägt,} das nie vertilgt werden kann, er ist und bleibt nun für immer ein Mitglied der christlichen Kirche, und eben deßhalb kann es nie nöthig werden, daß man die Taufe wiederhole, was daher auch ausdrücklich untersagt wird.~\RWSeitenw{312}
\item Auch selbst derjenige endlich, der nicht Gelegenheit besitzt, die heilige Taufe zu empfangen, erhält \RWbet{schon durch den bloßen feurigen Wunsch}, sie zu empfangen, \dh\ durch die sogenannte \RWbet{Begier-} oder \RWbet{Bluttaufe}, alle wesentlichen Vortheile, die eine wirkliche Taufe gewähret.
\end{aufza}

\RWpar{283}{Historischer Beweis dieser Lehre}
Nur über einige dieser Lehren soll zum Beweise ihrer historischen Richtigkeit Einiges beigebracht werden.
\begin{aufza}
\item Die Pflicht des öffentlichen Bekenntnisses, daß man sich von der Wahrheit des Christenthums aus Gründen überzeugt habe, legt uns schon Jesus in den Worten auf (\RWbibel{Mt}{Matth.}{10}{32}): \erganf{Wer mich bekennen wird vor den Menschen, den werde auch ich vor meinem himmlischen Vater bekennen. Wer mich verläugnen wird vor den Menschen, den werde auch ich vor meinem himmlischen Vater verläugnen.}
\item Die bei der Taufe gewöhnliche Formel ist aus den Worten Jesu (\Ahat{\RWbibel{Mt}{Matth.}{28}{19}}{28,29.}) entlehnt: \erganf{Geht zu allen Völkern, lehrt und taufet sie im Namen des Vaters, des Sohnes und des heiligen Geistes.} Das völlige Untertauchen unter das Wasser scheint man schon in dem Zeitalter der Apostel nicht für wesentlich gehalten, sondern geglaubt zu haben, daß auch ein bloßes Begießen oder Besprengen genüge. In den ersten Jahrhunderten, da häufig noch Erwachsene zur Taufe kamen, wurde die Taufe nur an den zwei Hauptfesten des Jahres, Ostern und Pfingsten, vom Bischofe selbst verrichtet.
\item Ueber die Wirkungen der Taufe.
\begin{aufzb}
\item \RWbibel{Apg}{Apostelg.}{2}{38}: \erganf{Lasset euch taufen zur \RWbet{Vergebung eurer Sünden}.}
\item Daß insbesondere auch die \RWbet{Erbsünde} nachgelassen werde, erhellet aus dem Gegensatze, welchen die heilige Schrift zwischen den Heiden (den Kindern des Zorns) und den durch das Bad der Wiedergeburt erneuerten Kindern Gottes macht, \usw\
\end{aufzb}
\item Als sich das Christenthum immer mehr ausbreitete, wurde die heil.\ Handlung der Taufe nicht nur vom Bischof,~\RWSeitenw{313}\ sondern auch von \RWbet{Priestern, Diakonen} (von einem solchen auch schon \RWbibel{Apg}{Apostelg.}{8}{12}) im Nothfalle selbst von \RWbet{Laien} ertheilt. Gegen die Mitte des dritten Jahrhunderts wurde in dem Streite zwischen dem \RWbet{Papste Stephan} und dem karthaginiensischen Bischofe \RWbet{Cyprian} (dem Heiligen) gegen den Letzteren entschieden, daß auch die \RWbet{Taufe der Ketzer} gültig sey, wenn nur die wesentlichen Handlungen dabei beobachtet worden wären.
\item Die Sitte, \RWbet{Kinder zu taufen,} scheint schon aus dem apostolischen Zeitalter herzurühren; denn \RWbibel{Apg}{Apostelg.}{10}{44\,ff}\ wird uns erzählt, daß der Apostel Petrus den Hauptmann Cornelius sammt allen seinen Angehörigen (worunter sich wahrscheinlich auch Kinder befunden haben werden) getaufet habe. Der gelehrte \RWbet{Origenes} hielt die Kindertaufe für eine apostolische Tradition. Der heil.\ \RWbet{Cyprian} redet von Kindern, denen das heil.\ Abendmahl gereicht worden, welches voraussetzt, daß sie auch schon getauft waren.
\item \RWbet{Begierd-} und \RWbet{Bluttaufe}. -- Der Hauptmann Cornelius empfing die Gaben des heil.\ Geistes ( \RWbibel{Apg}{Apostelg.}{10}{44\,ff}) noch vor der Taufe mit Wasser. Der heil.\ \RWbet{Ambrosius} schreibt vom Kaiser \RWbet{Valentinian}, der noch als Katechumenus, also ohne Taufe, verstarb: \RWlat{Hunc sua pietas abluit et voluntas. -- Martyres suo abluuntur sanguine.} -- Dasselbe behauptet der heil.\ \RWbet{Augustin}, und bezeugt, daß in der römischen, mailändischen, afrikanischen, gallikanischen, spanischen Kirche für jene Katechumenen, die ohne Taufe gestorben, nicht anders als für Christen gebetet worden sey, \usw\
\end{aufza}

\RWpar{284}{Vernunftmäßigkeit und sittlicher Nutzen}
\begin{aufza}
\item Dadurch, daß Jeder, der sich von der Göttlichkeit des katholischen Christenthums aus Gründen überzeugt hat, diese Ueberzeugung und seinen Vorsatz, von nun an nach den Vorschriften dieser Religion zu leben, \RWbet{so öffentlich als möglich kund thut}, gewinnt er selbst sowohl, als Andere. Er selbst, indem die Verpflichtung, nach den Grundsätzen dieser Religion sein Leben einzurichten, um desto stärker wird, je öffentlicher er es bekannt macht, daß er sich von ihrer~\RWSeitenw{314}\ Göttlichkeit überzeugt habe; denn je mehrere Menschen dieß wissen, um desto mehrere würde er im entgegengesetzten Falle durch seinen lasterhaften Wandel ärgern. Auch Andere gewinnen, weil jede Nachricht von einem Menschen, der sich von der Vortrefflichkeit des Christenthums aus Gründen überzeugt hat, Alle jene, die dieser Religion bereits zugethan sind, in ihrem Glauben bestärken, die Uebrigen aber zu einer um so genaueren Prüfung derselben aufmuntern muß.
\item Sehr zweckmäßg ist es, den Vorstehern der Kirche aufzutragen, daß sie nicht Jeden, der sich zur Aufnahme in die katholische Kirche meldet, ohne die sorgfältigste \RWbet{Prüfung seiner Gesinnungen und seines Lebenswandels} aufnehmen sollen; denn durch die Aufnahme unwürdiger Glieder wird nur die Ehre des Christenthums beeinträchtiget, und sie selbst werden nicht besser, sondern nur schlechter und strafwürdiger. (\RWbibel{Mt}{Matth.}{23}{15}) -- Die sinnbildliche Handlung, welche bei dieser Aufnahme vorgeschrieben ist, ist eben so leicht zu verstehen, als passend. Eine Reinigung des Leibes durch Waschen ist nämlich das schicklichste Sinnbild jener geistigen Reinigung, die durch die Taufe in unserer Seele vorgehen soll und bei einem würdigen Empfange derselben auch wirklich vorgeht. Um aber das Bad der Taufe von jedem gewöhnlichen zu unterscheiden, mußten noch einige Worte hinzugefügt werden. Dieß hätten zwar schon die Worte: Ich taufe dich hiemit, geleistet; allein da es auch bei den Juden und Heiden Taufen gab: so war es nöthig, die Taufe der Christen durch einen noch näheren Zusatz zu unterscheiden, \zB : Ich taufe dich hiemit zum Christen. Besser ist aber ein Zusatz, welcher schon durch sich selbst an gewisse wichtige Lehren des Christenthums erinnert. Wie zweckmäßig also die Worte: Ich taufe dich im Namen des Vaters, des Sohnes und des heil.\ Geistes; da eben die Lehre von Gottes dreieiniger Natur dem Christenthume nicht nur charakteristisch eigen ist, sondern auch die Grundlage der meisten andern Lehren ausmacht, und an die wichtigsten Pflichten erinnert. Da aber die Ehre des Christenthums, wie schon gesagt wurde, fordert, daß sich nicht jeder Mensch, dem es nur einfällt, für einen Christen ausgeben dürfe, ohne von den Vorstehern~\RWSeitenw{315}\ der Kirche dazu ermächtiget zu werden; da also eigentlich nur diese es sind, die ihn in's Christenthum aufnehmen können: so sollte auch dieser Umstand in der Taufe sinnbildlich dargestellt seyn, und daher tauft der Bischof als oberster Vorsteher einer Gemeinde, oder in seiner Verhinderung ein anderes Mitglied, welches dann gleichsam einen Abgeordneten des Ersteren vorstellt.
\item Da jene segensreichen Wirkungen nicht dem Bade, als solchem, sondern dem göttlichen Rathschlusse zugeschrieben werden, der es beschlossen hat, demjenigen, der die Taufe würdig empfängt, diese Gnaden mitzutheilen: so liegt nichts Widersprechendes in diesen Verheißungen; denn wie wir eben gesehen, so sind die Handlungen, die hier von Seite des Täuflings sowohl als auch des Taufenden gefordert werden, durchgängig gute und gemeinnützige Handlungen; sie können also vernünftiger Weise der Gegenstand einer Belohnung von Seite Gottes werden.
\begin{aufzb}
\item Dem Getauften wird die Nachlassung aller derjenigen Sünden zu Theil, die er schon vor der Taufe begangen. Wer den Vorsatz faßt, ein neues und besseres Leben, ganz nach den Grundsätzen des Christenthums, zu führen, den soll man billig seiner früheren Fehler wegen nicht mehr beunruhigen, in der Hoffnung, daß er dergleichen künftig nicht mehr begehen wolle.
\item Wie zweckmäßig ferner, ihm die erfreuliche Vorstellung zu erlauben, daß sich nun durchaus nichts Gott Mißfälliges an ihm befinde, auch Adams Sünde nicht mehr; daß er zur Kindschaft Gottes, zur Brüderschaft Christi erhoben sey; \usw\ Wie stärket ihn dieß, wie gibt es ihm Muth und Lust, den neuen Lebenspfad, den er nun anfangen soll, in aller Unschuld zu wandeln!
\item Um glauben zu können, daß Gott demjenigen, der die heil.\ Taufe auf die gehörige Weise empfangen hat, alle diejenigen Gnaden und Stärkungen mittheilt, welche zur Führung eines recht christlichen Wandels nothwendig sind, brauchen wir eben nicht vorauszusetzen, daß ihm alle diese göttlichen Wohlthaten durch eine unmittelbare Einwirkung Gottes auf seinen Geist zukommen; sondern Gott kann~\RWSeitenw{316}\ sich auch unzähliger natürlicher Mittel und Kräfte bedienen, um in dem Getauften gute Gedanken und Entschließungen zu wecken, Versuchungen zum Bösen abzuwenden, \usw\ Nur dürfen wir auch nicht wähnen, daß solche mittelbare Wirkungen Gottes immer von minderer Wichtigkeit und weniger dankenswerth seyen als unmittelbare.
\end{aufzb}
\item Da unser Herr die Taufe allen seinen Bekennern vorgeschrieben: so mußten die Vorsteher der Kirche billig die Erlaubniß geben, daß man im Nothfalle sie auch mit Weglassung aller derjenigen Gebräuche, die Jesus nicht selbst ausdrücklich vorgeschrieben hatte, ertheile. Und da er über die Person des Taufenden gleichfalls nichts Näheres festgesetzt hatte: so war es abermals billig, im Falle der Noth die Taufe Jedermann zu gestatten, der nur selbst ein Mitglied der Kirche ist. Wäre die übernatürliche Wirkung der Taufe an die Bedingung der Rechtschaffenheit des Spenders gebunden: wann könnte man gewiß seyn, die Segnungen derselben wirklich empfangen zu haben? -- Eben so zweckmäßig war es, selbst die von Abtrünnigen ertheilte Taufe für gültig zu erklären, wenn anders nur der von Christo vorgeschriebene Gebrauch dabei beobachtet worden; denn durch diese Erklärung wird nichts als Gutes erreicht. Ist nämlich die von Abtrünnigen ertheilte Taufe gültig: so müssen sie alle als Christen angesehen werden; und so haben dann rechtgläubige Christen die fortwährende Verbindlichkeit auf sich, für sie als Brüder zu beten und darauf hinzuwirken, daß sie den Irrthum, der sie von uns noch trennt, erkennen lernen. Sind diese Abtrünnigen als Christen anzusehen: so ist es das Christenthum, dem alles Gute, das sie an sich haben, oder das durch sie geschieht, zur Ehre gereicht. Was sie dagegen Tadelnswürdiges thun oder haben, kann nicht auf Rechnung des Christenthums kommen, sondern es wird nur ihrer abweichenden Meinung in sofern wenigstens zur Last gelegt werden können, als sich nachweisen läßt, daß diese Meinung etwas der Tugend Nachtheiliges habe.
\item Sehr zweckmäßig ist es, unter den oben angegebenen Bedingungen einen Menschen auch schon als Kind zu taufen; denn kann er auch jetzt aus einer solchen Handlung nicht~\RWSeitenw{317}\ den natürlichen Nutzen der Erbauung schöpfen: so kann sie doch verschiedene andere natürliche Vortheile für ihn sowohl, als für Andere haben.
\begin{aufzb}
\item Eltern, Vormünder, Pathen und überhaupt alle diejenigen Christen, die auf die Erziehung des getauften Kindes jetzt oder in der Folge einen Einfluß gewinnen, erhalten nun eine verstärkte Verbindlichkeit, das Kind zu einem guten, dem Christenthume nicht zur Schande gereichenden Menschen zu erziehen.
\item Gehört das Kind schon zur Zahl der Christen: so wird es auch aller der Fürbitten theilhaftig, welche die Christen in ihren gottesdienstlichen Versammlungen, oder auch häuslichen Andachten, anstellen.
\item Wenn der Getaufte in der Folge erfährt, daß er dem Christenthume, in das man ihn in seiner Kindheit aufgenommen, so viel zu verdanken habe, die bessere und sorgfältigere Erziehung, die für ihn angestellten Fürbitten, \usw : so muß ihn dieß mit Dankgefühl gegen diese Religion erfüllen, und ihn um desto gewisser bestimmen, nach ihren Vorschriften sein Leben einzurichten.
\item Alle erwachsenen Personen, welche bei dieser Handlung der heil.\ Taufe auch nur als Zuschauer zugegen sind, werden erbaut \usw\
\end{aufzb}
Eben um dieser natürlichen Vortheile willen ist es nun ganz vernünftig, zu glauben, daß Gott jene übernatürlichen Segnungen, die er mit dieser Handlung sonst zu verknüpfen verheißen hat, auch hier eintreten lasse, \dh\  daß er das Kind, welches die fromme Sorgfalt der Christen getauft hat, auf solchen Wegen leite, wodurch es schnellere Fortschritte in seiner Tugend und Glückseligkeit macht, als es im Gegentheile geschehen wäre. -- Sehr begreiflich aber, daß eine Taufe, die wider den Willen des zu Taufenden, oder wider den Willen seiner Eltern vorgenommen wird, bald durchaus ungültig, bald doch sehr unerlaubt sey; denn eine solche Handlung hat natürlich schädliche Folgen; sie muß Denjenigen, an dem man sie mit Gewalt vollziehen will, oder (falls er ein Kind ist) die Eltern und Vormünder desselben äußerst erbittern; sie setzet das Christenthum der Gefahr aus, ein Mit\RWSeitenw{318}glied mehr zu erhalten, das (weil es nicht christlich erzogen worden ist) dem Christenthume nur zur Schande leben wird, \usw\
\item Die Behauptung, daß die Taufe der Seele ein unvertilgbares Merkmal eindrücke, hat nur den Sinn, daß ein Getaufter von nun an für immer ganz anders zu betrachten sey, als ein noch Ungetaufter. Und dieses ist sehr vernünftig. Strenge genommen hat ja eine jede Handlung des Menschen, die er entweder selbst oder die Andern an ihm vollziehen, Folgen, die sich in das Unendliche erstrecken, so, daß er von der Zeit an, da diese Handlung vollzogen wurde, für immer als ein Anderer betrachtet werden kann, als er geworden wäre, wenn diese Handlung nicht Statt gefunden hätte. Aber es wäre nicht gut, diesen Gedanken bei einer jeden Handlung zu fassen. -- Nur bei Verrichtungen, die von besonderer Wichtigkeit sind, die ferner gewisse sehr wohlthätigen Folgen haben, die endlich dem Menschen eigene Pflichten und Verbindlichkeiten auflegen, wird es gerathen seyn, dieß zu Gemüthe zu fassen. Nirgends gerathener also, als bei der Handlung der heil.\ Taufe. Aus diesem Gesichtspuncte nämlich erscheint die Taufe dem Menschen in ihrer ganzen Wichtigkeit; indem er erwägt, daß sie Folgen habe, die sich bis in die Ewigkeit erstrecken. So freut er sich nun des Glückes, ein Getaufter zu seyn, um desto inniger; so fühlt er auch um so stärker die Verbindlichkeit, sein ganzes Leben bei den in ihr gemachten Gelübden zu verbleiben; so haben auch alle seine Mitchristen die Pflicht, so lange er lebt, sich für ihn eifriger zu verwenden, wenn er etwa auch vom Glauben abgefallen wäre, doch noch als Christen zu betrachten, und also an seiner Zurückbringung zu arbeiten; \usw\ Hieraus ergibt sich nun auch, daß die Taufe nicht mehr wiederholt werden dürfe; denn
\begin{aufzb}
\item es tritt der Zustand, für den sie ganz passend eingerichtet ist, nur ein einziges Mal im Leben ein, nämlich beim ersten Uebertritt zum Christenthume. Hat Jemand nach seiner Taufe gesündiget, oder ist er ganz abgefallen vom christlichen Glauben: so mögen ihm wohl bei seiner Rückkehr andere Heiligungsmittel nothwendig seyn, aber nur nicht die Taufe, in welcher Mehreres vorkommt, was sich~\RWSeitenw{319}\ für einen solchen Fall nicht schickt; \zB\  die Nachlassung der Erbsünde, \udgl\ 
\item Würde die Taufe öfter wiederholt: so würde sie eben hiedurch manchen der oben angeführten Vortheile verlieren; sie würde \zB\  aufhören, ein sicheres Kennzeichen eines Christen zu seyn.\par
Die übrigen Puncte bedürfen keiner Rechtfertigung.
\end{aufzb}
\end{aufza}

\RWpar[Die Lehre des Katholicismus von der Bestätigungsfeier oder der heil.\ Firmung]{285}{Die Lehre des Katholicismus von der Bestätigungsfeier oder der heil.\ Firmung.\\ Historischer Beweis und Vernunftmäßigkeit.}
\begin{aufza}
\item Wer durch die Taufe als Mitglied der christlichen Gemeine aufgenommen ist, soll, wenn er die erstere in seiner Kindheit, oder doch auf keine ganz feierliche Art und nicht vom Bischofe selbst empfangen hat, in einem Alter, da er die Lehren des Katholicismus bereits vollkommen inne hat, seine Ueberzeugung von ihrer Wahrheit und sein Gelübde, nach ihnen zu leben, vor der Gemeine, der er als Mitglied zugehört, und vor dem Bischofe derselben kund thun.\par
\RWbet{Historischer Beweis.}\par
Freilich wird heut zu Tage bei der Ausspendung dieses Heiligungsmittels vom Firmlinge selbst gewöhnlich nichts gesprochen; doch ist auch schon sein stummer Hintritt vor den Bischof, um sich firmen zu lassen, ein stillschweigendes Bekenntniß.\par
\RWbet{Vernunftmäßigkeit.}\par
Diese öffentliche Kundmachung dient, wie bei der Taufe, zur mehreren Befestigung des Vorsatzes selbst bei dem Firmling, und zur Erbauung für seine Mitchristen, ingleichen zur Erregung der Aufmerksamkeit auf unsere Religion auch selbst bei Nichtchristen. Im Gegentheile, wenn diese Einrichtung nicht getroffen wäre, könnte in Ländern, wo beinahe alle Einwohner als Kinder getauft worden sind, der für die Ehre unserer Religion äußerst nachtheilige Zweifel entstehen, ob wohl auch Alle, die hier Christen heißen, weil sie als Kinder getauft worden sind, als Jünglinge und Männer  Ueberzeugung gefunden haben? -- Durch die Aufforderung zu dem~\RWSeitenw{320}\ so eben erwähnten Schritte fordert die Kirche Alle, die schon als Kinder getauft worden sind, auf, Eines von Beiden zu thun, entweder öffentlich auszutreten, oder öffentlich zu bekennen, daß sie als Jünglinge und Männer die Göttlichkeit der Anstalt, in die man sie als Kinder aufgenommen hat, erkennen, und nun aus eigener Wahl in ihr verbleiben wollen. Doch war es zweckmäßig, dieses Bekenntniß nicht zur bestimmten Pflicht und Schuldigkeit zu erheben, um das Gewissen Einzelner nicht zu beschweren, indem die äußere Gelegenheit hiezu oft fehlen kann.
\item Diese Erklärung soll, wenn sich, zu Folge des Urtheils eigener Personen, der \RWbet{Firmpathen}, vermuthen läßt, daß sie ernstlich gemeint sey, von Seite des Bischofs und der Gemeine mit Freuden aufgenommen und damit erwiedert werden, daß man für den zu Firmenden bei Gott gemeinschaftlich fürbitte, auf daß ihn der heil.\ Geist mit seinen Gaben und Gnaden ausrüsten und zur Erfüllung seines löblichen Vorsatzes stärken wolle. Zum Zeichen, daß diese Stärkung auch wirklich erfolgen werde, soll der Bischof den zu Firmenden salben, wie Kämpfer\RWfootnote{%
	Eigentlich ist diese, und eine ähnliche bei der Taufe übliche, Salbung des Hauptes mit Chrysam das Zeichen einer gewissen höheren Würde.}
gesalbt werden, mit heiligem Oele. Es ist gebräuchlich, diese Salbung nur an der Stirne und in Gestalt eines Kreuzes zu verrichten, und den Zweck derselben durch die bestimmten Worte anzudeuten: Ich bezeichne dich mit dem Zeichen des Kreuzes, und bestärke dich durch die Salbe des Heils im Namen des Vaters, des Sohnes und des heil.\ Geistes. -- Diese sinnbildliche Handlung wird nach Umständen und nach der größeren oder geringeren Bildung des Zeitalters noch mit gewissen anderen schicklichen Ceremonien begleitet.
\item Nebst dem natürlichen Vortheile der Erbauung wird der Gefirmte nun auch den übernatürlichen Vortheil gewinnen, daß der Geist Gottes seine sittlichen Kräfte erhöhen, ihn zur Verrichtung alles Guten stärken, und ihm die Gnade geben wird, dem Christenthume, trotz aller inneren und äußeren Versuchungen zum Abfalle, treu zu bleiben.~\RWSeitenw{321}\par
\RWbet{Historischer Beweis.}\par
\RWbibel{Apg}{Apostelg.}{8}{14}\ heißt es, als die Apostel vernommen, daß auch zu Samaria auf die Predigt des Diakons Philippus Mehrere zum Christenthume übergetreten wären, hätten sie Petrum und Paulum hingeschickt, die über sie beteten, auf daß sie den heil.\ Geist empfangen möchten. Denn noch war derselbe über Keinen herabgekommen, sondern sie waren bloß getauft auf den Namen Jesu. Da legten sie ihnen die Hände auf, und sie empfingen den heil.\ Geist. (So auch \RWbibel{Apg}{Apostelg.}{19}{6})\par
\RWbet{Vernunftmäßigkeit.}\par
Warum dürfte Gott nicht dem Einen mehrere, dem Andern weniger Kräfte und Gnaden verleihen? Warum nicht Jenen, welche so eifrig darum bitten, mehr schenken, als Andern, die es aus Trägheit unterlassen? --
\item Dieß Heiligungsmittel soll nur ein Bischof, oder höchstens, wenn er im Nothfalle dazu von seinem Bischofe eigens aufgefordert worden wäre, ein Priester ausspenden können.\par
\RWbet{Vernunftmäßigkeit.}\par
Dieses der größeren Feierlichkeit wegen. Auch ist es gewiß sehr zweckmäßig, daß jeder Gläubige wenigstens einmal in seinem Leben dem Bischofe vorgestellt werde, und gleichsam in seine Hände gelobe, daß er die Vorschriften des Christenthums befolgen wolle.
\item Dieses und alle die folgenden Heiligungsmittel kann nur der schon Getaufte gültig empfangen; und um es auch würdig zu thun, muß er sich erst in eine Gott wohlgefällige Verfassung versetzen; und also \zB , sofern es nöthig ist, erst noch Gebrauch von der christlichen Besserungsanstalt machen.\par
\RWbet{Vernunftmäßigkeit.}\par
Daß nur Getaufte dieses und alle die folgenden Heiligungsmittel gültig empfangen können, hat seinen Grund darin, daß alle Heiligungsmittel doch nur für Christen eingesetzt sind, oder nur von demjenigen, der an das Christenthum glaubt, vernünftiger Weise gebraucht werden können; die Taufe aber~\RWSeitenw{322}\ dasjenige derselben ist, durch welches wir erst zu Christen aufgenommen werden. Billig muß also die Taufe vorausgeschickt werden. Da wir ferner ein jedes Heiligungsmittel nur dann erst würdig empfangen, wenn wir zu unserer sittlichen Vervollkommnung, die der Zweck aller ist, von unserer Seite so Vieles beitragen, als wir nur immer vermögen: so folgt, daß wir zur Taufe uns nur durch die selbst außer dem Christenthume bekannten Tugendmittel, zu den übrigen Heiligungsmitteln aber nur durch den Gebrauch der christlichen Besserungsanstalt vorbereiten können und sollen.
\item Auch dieses Heiligungsmittel drückt der Seele ein unauslöschliches Merkmal ein, und soll um deßwillen nur einmal empfangen werden. --\par
\RWbet{Vernunftmäßigkeit.}\par
Die Gründe sind hier beinahe die nämlichen, wie bei der Taufe; denn auch der Fall, für den die Bestätigungsfeier bestimmt ist, tritt nur einmal im Leben ein; und die Wichtigkeit, die Erfreulichkeit und die Verbindlichkeit dieser heil.\ Handlung wird durch die Vorstellung, daß ihre Folgen nie wieder vertilgt werden können, gehoben.
\end{aufza}

\RWpar{286}{Die Lehre des Katholicismus vom heiligen Abendmahle}
\begin{aufza}
\item Die katholische Kirche verlangt, wie schon erwähnt, von ihren Bekennern, daß sie, wenn es die äußeren Umstände gestatten, öftere \RWbet{Zusammenkünfte} halten, um Gott auch \RWbet{öffentlich und gemeinschaftlich zu verehren.} Diese Gottesverehrungen soll der \RWbet{Geistliche}, wenn Einer da ist, leiten. Und ist ein Bischof oder auch nur ein Priester zugegen: so soll unter Anderm auch ein \RWbet{gemeinschaftliches Mahl} gefeiert werden, das eine Nachahmung sey von jenem Abendmahle, mit welchem Jesus Christus von seinen Jüngern schied. Der Bischof oder der Priester soll die Person des Herrn, die übrigen Gläubigen die der Jünger vorstellen. Der Erstere soll in Nachahmung dessen, was wir im Evangelio von dem Herrn lesen, Brod und Wein nehmen, und nachdem er darüber die Worte des Herrn: Das ist mein Leib, das~\RWSeitenw{323}\ ist mein Blut, gesprochen, davon genießen, und auch den Anwesenden zum Genusse darbieten.
\item Wie jene Worte über das Brod und den Wein nur ausgesprochen sind: so hat schon \RWbet{das Wesen des Brodes und des Weines aufgehört}, da zu seyn, und unter den bloßen Gestalten, welche hier noch verweilen, ist \RWbet{Jesus Christus nach seiner göttlichen sowohl als menschlichen Natur wahrhaft und wesentlich zugegen}; nicht etwa so, daß er den Raum dieser Gestalten ausfüllte, und in dem einen Theile derselben mit diesem, in einem andern mit einem anderen seiner Theile zugegen wäre, und, wenn man auf diese Gestalten verändernd einwirkt, selbst auch Veränderungen erlitte; sondern \RWbet{in jedem Theile dieser Gestalten ist Christus ganz gegenwärtig}, wird durch die Veränderung, die sie erfahren, nicht mitverändert; obgleich seine Gegenwart in denselben aufhört, sobald sie selbst aufhören, Gestalten von Brod und Wein (genußbare Nahrungsstoffe) zu seyn.
\item Es ist uns nicht nur erlaubt, sondern sogar unsere Pflicht, \RWbet{den unter diesen Gestalten sich uns darbietenden Gottmenschen anzubeten}; diese Gestalten sonach mit der größten uns nur immer möglichen Ehrerbietigkeit zu behandeln, bei ihrem Anblicke niederzuknieen, \usw\ Insonderheit sollen wir uns in der Gegenwart jener gesegneten Gestalten einer so feurigen Andacht befleißen, als wir es für unsere Pflicht erachten würden, wenn unser Herr in sichtbarer menschlichen Gestalt vor uns erschiene.
\item Und wenn wir dieß thun, und nachdem wir uns erst durch alle uns zu Gebote stehenden Mittel gehörig vorbereitet haben, im demüthigen Vertrauen auf die eigene Aufforderung Jesu es wagen, von jenen gesegneten Gestalten auch etwas zu \RWbet{genießen}: so wird unsere Seele durch Jesum Christum auf eine eben so wohlthätige Art \RWbet{genährt und gestärket}, wie Brod und Wein die stärkendsten Nahrungsmittel des Leibes zu seyn pflegen, und eben so innig, wie solche Nahrungsmittel sich mit dem Leibe vereinigen, \RWbet{wird sich auch Jesus Christus mit unserer Seele vereinen.}
\item Selbst wenn wir uns, im Gefühle unserer Unwürdigkeit, oder aus Besorgniß, daß uns die öftere Wiederholung~\RWSeitenw{324}\ gleichgültig machen dürfte, dem wirklichen Genusse des heil.\ Abendmahles entziehen, aber dabei doch mit aller uns möglichen Andacht zugegen sind, und um die Mittheilung einiger Segnungen bitten: so bitten wir nicht unerhört.
\item Wer dagegen dieser heil.\ Handlung ohne Andacht beiwohnt, macht sich nur strafwürdig; und wer es wagt, den Leib des Herrn zu empfangen, ohne sich erst in eine Gott gefällige Gemüthsverfassung versetzt zu haben, isset und trinket sein eigenes Gericht.
\item Um der gesegneten Wirkungen des heil.\ Abendmahles theilhaftig zu werden, genüget es völlig, dasselbe auch nur in \RWbet{Einer der beiden Gestalten} des Brodes oder des Weines genossen zu haben.
\item Die schon im alten Bunde und selbst bei heidnischen Völkern bestandene Gewohnheit, Gott unter Anderm durch \RWbet{Opfer} zu verehren, soll auch noch unter uns fortdauern; doch so, daß wir nicht ferner glauben, Gott wohlgefällig zu werden, wenn wir eines seiner Geschöpfe des Lebens berauben, oder irgend einen nutzbaren Gegenstand zweckloser Weise zerstören; sondern wir sollen wissen, daß Gott kein anderes Opfer verlange, als die Verzichtleistung auf jeden eigenen Vortheil, und selbst auf das Leben, wenn das gemeine Beste, wenn die Gesetze der Tugend es erheischen. Wir sollen eben deßhalb wissen, \RWbet{daß unter allen Opfern, die Gott je dargebracht wurden, keines ihm wohlgefälliger gewesen sey und bleibe, als das Opfer, das Jesus Christus durch seinen freiwilligen Versöhnungstod am Kreuze dargebracht hat}. Wir sollen ferner glauben, daß Jesus Christus, so oft wir dieß heilige Mahl auf Erden feiern, im Himmel Kenntniß davon erlange, daß es ihm angenehm sey, daß er beim Vater fürbitte für uns, ja daß er, bildlicher Weise zu reden, den Vater neuerdings an seine Verdienste um uns und seinen Kreuzestod erinnere, um ihn für die Erhörung unserer Bitte geneigter zu machen. Wir sollen eben deßhalb die Gelegenheit dieses heil.\ Mahles benützen, um nicht nur ein Jeder für uns, sondern auch noch für Andere, für unsere Verwandte und Freunde, für Lebende und Verstorbene mit dem gewissesten Erfolge einer Erhörung zu beten.~\RWSeitenw{325}
\item Nebst dieser Lehre gibt es mehrere von den Vorstehern der katholischen Kirche herrührende Verfügungen in Betreff des heil.\ Abendmahles, deren wichtigste wir in Kürze andeuten wollen.
\begin{aufzb}
\item Es ist verordnet, daß man das heil.\ Abendmahl nicht anders als \RWbet{in der Versammlung mehrerer Gläubigen} und (außer in gewissen Fällen der Nothwendigkeit) an einem \RWbet{eigens dazu geweiheten Orte} feiere.
\item Auch hier nur soll es empfangen werden, es sey denn, daß Jemand durch Krankheit oder eine andere gleichwichtige Ursache gehindert wäre, in der Versammlung zu erscheinen.
\item Von dem Brode, dessen man sich hiebei bedienen will, verlangt die lateinische Kirche, daß es, dem immerwährenden Gebrauche bei ihr gemäß, ein \RWbet{ungesäuertes} sey. Der Wein soll ungekünstelt (nicht gekocht \udgl ) rein und genießbar seyn.
\item Von den gesegneten Gestalten soll, wenn auch nicht eben von allen Anwesenden, so doch von Einigen, auf jeden Fall aber vom Priester selbst, etwas \RWbet{genossen} werden.
\item Personen, welche durch ihre Lasterhaftigkeit oder durch ihre hartnäckige Vertheidigung falscher Grundsätze ein \RWbet{öffentliches Aergerniß} geben, sollen zu dem Genusse des heil.\ Abendmahles, auch wenn sie sich dabei einstellen wollten, \RWbet{nicht zugelassen werden}.
\item Das heil.\ Abendmahl soll \RWbet{nüchtern} empfangen werden, es sey denn von jenen, die es in einer schweren Krankheit als heil.\ Wegzehrung zu empfangen wünschen.
\item Jeder Christ soll das heil.\ Abendmahl \RWbet{wenigstens einmal im Jahre} zur österlichen Zeit empfangen; der Feier desselben in den gottesdienstlichen Versammlungen aber soll er \RWbet{an jedem Sonn- und Feiertage mit Andacht beiwohnen}.
\item Außer dem die heil.\ Handlung verrichtenden Priester soll man den übrigen Anwesenden, zumal soferne sie nicht zum geistlichen Stande gehören, das heil.\ Abendmahl \RWbet{nur unter der Gestalt des Brodes} darreichen.
\item Die Gebete und Ceremonien, die bei der Feier des heil.\ Mahles vom Priester verrichtet werden, soll nicht ein Je\RWSeitenw{326}der nach seinem Belieben so oder anders einrichten dürfen, sondern den Vorstehern nur soll das Recht zustehen, zweckmäßige Abänderungen an dem, was bisher üblich ist, zu treffen.
\end{aufzb}
\end{aufza}

\RWpar{287}{Historischer Beweis dieser Lehre}
Zu~1.--6. Bei \RWbibel{Mt}{Matth.}{26}{26\,ff}\ \RWbibel{Mk}{Mark.}{14}{22\,ff}\ \Ahat{\RWbibel{Lk}{Luk.}{22}{19\,ff}}{29,19\,ff.}\ und \RWbibel{1\,Kor}{1\,Kor.}{11}{23\,ff}\ wird die Geschichte der Einsetzung des heil.\ Abendmahles erzählt. Die Worte, deren sich Jesus hier (nach diesem vierfachen Berichte) bedient, wären doch in der That sehr unbequem gewesen, wenn er nichts Mehreres gewollt hätte, als daß wir uns unter jenen Gestalten des Brodes und Weines seinen Leib und sein Blut nur eben so \RWbet{bildlich vorstellen} sollten, als wir von einem Gemälde, welches ihn vorstellt, bildlicher Weise sagen, er sey hier gegenwärtig. Bei \RWbibel{Joh}{Joh.}{6}{55}\ sagt Jesus ausdrücklich: \erganf{\RWbet{Mein Fleisch ist wahrhaft eine Speise, mein Blut ist wahrhaft ein Trank}}, und läßt es zu, daß mehrere seiner bisherigen Schüler um dieser Einen ihnen so hart, \dh\  so unverständlich dünkenden Lehre willen von ihm abtrünnig werden. Er hat es gewiß also vorhergesehen, daß man auch in der Zukunft seine Worte nicht in einer bloß bildlichen, sondern viel eigentlicheren Bedeutung auslegen werde; und hätte er nicht gewollt, daß sie so ausgelegt werden: wie leicht hätte er dieß durch eine beigefügte Erklärung nicht verhindern können! Aber, selbst wenn wir annehmen könnten, daß Jesus eine so buchstäbliche Auslegung seiner Worte gar nicht vorhergesehen, auch nicht gewollt habe: auch hieraus würde noch nichts Nachtheiliges gegen diese Lehre folgen. Denn da, wie sich bald zeigen wird, der Glaube an eine nicht bloß figürliche, sondern wirkliche Gegenwart Jesu im heil.\ Abendmahle für unsere Tugend und Glückseligkeit überaus zuträglich ist: so müßten wir diese Lehre noch immer als eine derjenigen ansehen, die der Geist Gottes unter uns nur darum aufkommen ließ, damit wir an sie glauben. Der Umstand, daß unser Herr gerade Ausdrücke von einer solchen Art gebrauchte, die eine sehr natürliche Veranlassung zu jener Auslegung gaben; oder daß wenigstens die vier Evangelisten~\RWSeitenw{327}\ ihm gerade solche Ausdrücke über diesen Gegenstand in den Mund legten, dieser Umstand, sage ich, müßte uns als ein eigenes Zeichen erscheinen, durch welches uns Gott an den Tag legt, daß er diese Lehre von uns geglaubt wissen wolle. In der That finden wir auch den Glauben an eine nicht bloß figürliche, sondern wirkliche Gegenwart Christi im heil.\ Abendmahle schon bei den heil.\ Aposteln. Der heil.\ Paulus schreibt (\RWbibel{1\,Kor}{1\,Kor.}{10}{16}): \erganf{Der Kelch des Segens, den wir segnen, ist er nicht eine Gemeinschaft des Blutes Christi? Das Brod, das wir brechen, ist es nicht eine Gemeinschaft des Leibes Christi?} -- Und (\RWbibel{1\,Kor}{}{11}{27}): \erganf{Wer immer unwürdig ißt von diesem Brode, oder unwürdig trinkt aus diesem Kelche, der macht sich schuldig des Leibes und des Blutes des Herrn. Der Mensch erforsche sich also selbst, und dann erst esse er von diesem Brode und trinke aus diesem Kelche. Denn wer unwürdig ißt oder trinkt, der isset und trinket sich selbst das Gericht, weil er den Leib des Herrn nicht unterscheidet. Und eben um dieses Vergehens willen sind bereits Mehrere von euch erkrankt, und Einige sogar gestorben!} \usw\ Wer könnte sich solcher Ausdrücke bedienen, wenn er das Brod und den Wein im heil.\ Abendmahle für nichts Weiteres als für bloße Sinnbilder Jesu ansieht? -- \par
Auch in den Schriften der christlichen Schriftsteller aus den ersten Jahrhunderten lassen sich Stellen in Menge beibringen, die ihren Glauben an eine wirkliche Gegenwart Christi im heil.\ Abendmahle beweisen. Wir wollen nur einige Wenige kennen lernen:
\begin{aufzb}
\item Der heil.\ \RWbet{Ignatius} lebte in einer Gegend, von der man vorzugsweise vor Andern annehmen darf, daß sich das Christenthum hier in seiner Urform erhalten habe, die eben deßhalb auch in besonderem Ansehen stand \RWlat{(Iren.\ adv.\ haeres.\ l.\,3.\ c.\,3.\ v.\,2.)}\RWlit{}{Irenaeus1} in jenem Kleinasien nämlich, in welchem nebst Paulus auch Johannes so viele Jahre gelebt. Dieser heil.\ Ignatius nun nennet, in seinem Briefe an die Epheser (\RWlat{c.\,20.}), das heil.\ Abendmahl ein Arzneimittel zur Unsterblichkeit und ein Gegengift gegen den Tod, um in Christo ewiglich zu leben. Gegen Doketen oder Leute, welche die wahre Menschheit~\RWSeitenw{328}\ Christi in Zweifel zogen, sagt er \RWlat{(epist.\ ad Smyrn.\ c.\,7.)}\RWlit{}{Ignatius2}: \erganf{Sie enthalten sich des Abendmahles (der Eucharistia), weil sie nicht bekennen, dasselbe sey das Fleisch unseres Heilandes Jesu Christi. Menschen, die diesem Geschenke Gottes so widersprechen können, gehen durch ihren eigenen Widerspruch zu Grunde.}
\item \RWbet{Justin der Philosoph und Martyrer} schreibt in seiner ersten Apologie (an den Kaiser Antonius Pius): \erganf{Wir genießen dieß nicht als ein gemeines Brod oder als einen gemeinen Trank; sondern wie der zu unserem Heile Mensch gewordene Christus Fleisch und Blut hatte, so sind wir belehrt, daß die durch seine Worte geheiligte Speise, wodurch unser Fleisch und Blut genährt wird, das \RWbet{Fleisch und Blut des Mensch gewordenen Christus sey.}}
\item \RWbet{Irenäus} schreibt \RWlat{(adv.\ haeres.\ l.\,4.\ c.\,18.)}\RWlit{}{Irenaeus1}: \erganf{So wie das irdische Brod, wenn die Anrufung Gottes hinzugekommen ist, kein gemeines Brod mehr, sondern das Abendmahl (Eucharistia) ist, welches aus zwei Theilen besteht, einem irdischen und einem \RWbet{himmlischen}: so sind auch unsere Körper, welche das Abendmahl genießen, nicht mehr zerstörbar, sondern haben die Hoffnung der Auferstehung.}
\item \RWbet{Origenes} (in seiner 13.~Homil. über \RWbibel[2\,Mos.]{Ex}{}{}{}) unterweiset diejenigen, die zum heil.\ Abendmahle hinzutreten wollen: \erganf{Ihr, die ihr den heil.\ Geheimnissen beiwohnen dürfet, wisset wohl, wie ihr den \RWbet{Leib des Herrn,} indem ihr ihn (auf eure Hand) empfanget, mit aller Vorsicht und Ehrerbietigkeit zu behandeln habet, damit von der geheiligten Speise nichts auf die Erde falle.}
\item Aehnlicher Weise schreibt der heil.\ \RWbet{Cyrillus}: \erganf{Wenn du hinzutrittst zu dem Tische des Herrn, so lege die linke Hand gleich einem Throne unter die rechte, \RWbet{welche den König empfangen soll}; gebückt, und wie es der Verehrung und Anbetung zukommt, erscheine.}
\item Der heil.\ \RWbet{Ambrosius} beschäftigt sich schon mit der Widerlegung verschiedener Einwürfe gegen diese Lehre: \erganf{Du sprichst allezeit: Ich sehe doch etwas Anderes, wie~\RWSeitenw{329}\ kann man mir sagen, daß ich den \RWbet{Leib des Herrn} empfange? -- Auch dieß muß ich dir noch begreiflich machen. Das Wort Jesu Christi, welches aus Nichts machen konnte, was noch nicht war (die Welt), sollte es nicht auch dasjenige, was schon vorhanden ist, in etwas Anderes \RWbet{umwandeln} können?} \usw\ Ueberhaupt kann es nur dieser allgemein herrschende Glaube der ersten Christen, daß sie im heil.\ Abendmahle den wahren Leib Jesu empfangen, gewesen seyn, der jene Lästerung der Heiden, daß die Christen in ihren gottesdienstlichen Versammlungen \RWbet{Menschenfleisch} äßen (thyestische Gastmahle), veranlasset hatte, eine Lästerung, auf die schon\RWbet{ Plinius} in seinem 110ten Briefe\RWlit[: \eanf{\RWlat{ad capiendum cibum promiscuum tamen et innoxium}}, dt.\ Übers.: \eanf{um Speise zu sich zu nehmen, und zwar ganz gewöhnliche und unschädliche}]{ep.\,CX}{Plinius1a} rechtfertigend hinzudeuten scheint, wenn er sagt: \RWlat{\Ahat{cibum}{libum} sumunt promiscuum tamen et innocuum}.
\end{aufzb}
In späterer Zeit ward dann die Frage: wie jenes Brod und jener Wein in den Leib und das Blut Christi übergehen, dadurch beantwortet, daß man sagte: Die \RWbet{Substanzen} oder das \RWbet{Wesen des Brodes und des Weines} würden entfernt oder vernichtet, und die \RWbet{Substanzen des Leibes und Blutes Jesu Christi} träten an ihre Stelle. Dieß konnte man nun füglich die \RWbet{Transsubstantiation} nennen. Und der Kirchenrath von Trient erklärte: \RWlat{(sess.\,13.\ c.\,8.\ can.\,1.) \anf{Si quis negaverit, in sanctissimo Eucharistiae sacramento \RWbet{contineri vere, realiter et substantialiter corpus et sanguinem, una cum anima et divinitate Domini nostri Jesu Christi,} ac proinde totum Christum; sed dixerit, tantummodo esse in eo, ut in signo vel figura aut virtute, anathema sit.}} -- Und \RWlat{(Can.\,2.) \anf{Si quis dixerit, in sacrosanctae Eucharistiae sacramento remanere \RWbet{substantiam panis et vini una cum corpore et sanguine Domini nostri Jesu Christi}, negaveritque mirabilem illam conversionem totius substantiae vini in sanguinem, remanentibus duntaxat speciebus panis et vini, quam quidem conversionem catholica ecclesia aptissime \RWbet{transsubstantiationem} appellat, anathema sit.}}\RWlit[, dt.\ Übers.: (1651) \eanf{Wer leugnet, dass im Sakrament der heiligsten Eucharistie wahrhaft, wirklich und substanzhaft der Leib und das Blut zusammen mit der Seele und Gottheit unseres Herrn Jesus Christus und daher der ganze Christus enthalten ist, vielmehr sagt, er sei lediglich wie in einem Zeichen bzw. Abbild oder der Wirkkraft nach in ihm: der sei mit dem Anathema belegt.} (1652) \eanf{Wer sagt, im hochheiligen Sakrament der Eucharistie verbliebe zusammen mit dem Leib und Blut unseres Herrn Jesus Christus die Substanz des Brotes und des Weines, und jene wunderbare und einzigartige Verwandlung der ganzen Substanz des Brotes in den Leib und der ganzen Substanz des Weines in das Blut, wobei lediglich die Gestalten von Brot und Wein bleiben, leugnet -- und zwar nennt die katholische Kirche diese Wandlung sehr treffend Wesensverwandlung --: der sei mit dem Anathema belegt.}]{Nr. 1651--1652}{DH}~\RWSeitenw{330}\par
Aus den schon vorhin angeführten Zeugnissen erhellet auch, daß man das unter den Gestalten des Brodes und Weines Verborgene \RWbet{schon in den ältesten Zeiten verehrt und angebetet}, und also keineswegs (wie die Evangelischen lehren) geglaubt habe, daß Jesus nur im Augenblicke des Genusses gegenwärtig sey.\par
Zu 7.\ und 9. Schon in den ältesten Zeiten wurde das heil.\ Abendmahl zuweilen nur unter \RWbet{Einer Gestalt} vertheilt, \zB\  Kindern oder Kranken. Im \RWbet{dreizehnten Jahrhunderte} wurde die Mittheilung des gesegneten Kelches wegen verschiedener Gefahren und Mißbräuche den Laien gänzlich untersagt. Inzwischen erhielten nach der Hand doch einige Kirchen, \zB\  die böhmische, die Erlaubniß der Kommunion unter beiden Gestalten vom Basler sowohl als von dem Tridentiner Kirchenrathe. Durch die Bemühungen der Jesuiten aber wurden die Böhmen allmählich dahin gebracht, daß sie von diesem Gebrauche freiwillig abstanden.\par
8.~Zur Rechtfertigung ihres Begriffes von einem \RWbet{Opfermahle} beruft sich die katholische Kirche auf die Stellen \RWbibel{Mal}{Malach.}{1}{11}\  \RWbibel{Ps}{Psalm}{110}{4}\ \RWbibel{Hebr}{Hebr.}{13}{10--15}\ \uam

\RWpar{288}{Vernunftmäßigkeit und sittlicher Nutzen}
1.~Die Forderung der katholischen Kirche, daß wir das Andenken Jesu in unseren gottesdienstlichen Versammlungen durch eine Art von Nachahmung seines letzten Abendmahles feiern sollen, wird Jeder zweckmäßig finden; da es bekannt ist, daß uns die letzten Reden und Thaten einer Person, die sich um uns verdient gemacht hat, immer die wichtigsten und folglich die geeignetsten sind, ihr Andenken unter uns zu erneuern.\par
2.--6.\ \textbf{A.}~\RWbet{Vernunftmäßigkeit dieser Artikel}.\par
\begin{aufzb}
\item Es ist nichts Unmögliches, daß nach den Segensworten des Priesters die sogenannte Substanz oder das Wesen des Brodes und Weines auf uns zu wirken aufhören. Wenn wir uns erinnern, daß die katholische Kirche die Ausdrücke, deren sie sich zur Darstellung ihrer Lehren be\RWSeitenw{331}dienet, nicht etwa in dem Sinne, welchen nur irgend ein Gelehrter mit ihnen verknüpfet, sondern in dem Sinne nimmt, welchen der \RWbet{allgemeine Sprachgebrauch} festgesetzt hat; daß es ihr ferner bei diesen Ausdrücken immer nur darum zu thun sey, daß sie uns nur in keinem derjenigen Stücke, wo irren schädlich ist, zu einer unrichtigen Vorstellung verleiten: so wird uns einleuchten, zur Wahrheit dieses Artikels sey keineswegs nothwendig, daß die erwähnten Substanzen im eigentlichsten Sinne des Wortes vernichtet würden; nicht einmal daß sie auf unsichtbare Weise entfernt, oder daß ihr Einwirken auf unsere Sinne durch eine eigene Gegenkraft gehemmt und aufgehoben werde, sondern es sey schon genug, wenn nur die Art dieses Wirkens durch den Hinzutritt gewisser anderer Umstände und Kräfte ganz wesentlich, \dh\  gerade darin geändert wird, worein wir das Wesen des Brodes und Weines sonst insgemein setzen. Wir können schon dann sagen, nicht mehr das Wesen des Brodes oder des Weines sey da, wenn nur die Wirkungen, welche die unter diesen Gestalten vorhandenen Stoffe auf uns ausüben, durch den Hinzutritt gewisser anderer Umstände jetzt ganz anders und unendlich wohlthätiger geworden sind, als es die Wirkungen eines gemeinen Brodes oder Weines werden können. Das ist nun hier der Fall; weil das gesegnete Brod und der gesegnete Wein das Mittel ist, dessen sich Jesus Christus bedient, um die wohlthätigsten Einwirkungen auf unsere Seele hervorzubringen.
\item Es ist auch zweitens nichts Unmögliches, daß von jener Einsegnung an Christus als Gott und Mensch unter den Gestalten des Brodes und des Weines \RWbet{gegenwärtig} sey. Denn wie auch immer die schulgerechte Erklärung des Begriffs der Gegenwart lauten möge: so ist doch so viel gewiß, daß wir uns unter der Gegenwart Christi im Altarsacramente, so fern der Glaube an sie von einiger Wichtigkeit für uns ist, nichts Anderes vorstellen, als eine gewisse überaus wohlthätige \RWbet{Wirksamkeit} Jesu. Ob diese Wirksamkeit eine ganz unvermittelte sey, oder ob Jesus Christus sich hiebei einiger Mittelkräfte von die\RWSeitenw{332}ser oder jener Art bediene, das muß uns, wenn wir vernünftig denken, gleichgültig seyn; und eben darum gehört die Entscheidung dieser Frage einer bloß müßigen Neugier gar nicht zur Religion und in das Gebiet der kirchlichen Unfehlbarkeit. Denn sicher kann uns doch nur daran liegen, ob, wenn wir das heil.\ Abendmahl mit der gebührenden Andacht empfangen, Christus in Wahrheit auf uns wohlthätig einwirke; ob dieses aber durch eine ganz unvermittelte Wirksamkeit, oder auf irgend eine andere Art geschehe, das brauchen wir nicht zu wissen. Daß aber der Gottmensch durch die gesegneten Gestalten des Brodes und des Weines, wenn nicht durchgängig unmittelbar, doch wenigstens mittelbarer Weise überaus wohlthätig auf uns einzuwirken im Stande seyn müsse, wer wollte dieses in Abrede stellen? Wirkt er doch
\begin{aufzc}
\item einmal schon dadurch überaus wohlthätig auf uns, daß er uns die Erlaubniß gegeben, unter diesen Gestalten uns ihn selbst vorzustellen; denn wenn wir nur recht lebhaft an Jesum Christum denken; wenn wir uns sagen: \anf{Er, der Erlöser der Welt, ist hier zugegen; sein heiliger Leib, derselbe, der für dich an's Kreuz geschlagen wurde, das Blut, das auch für dich vergossen worden, ist unter diesen Gestalten zugegen!} -- : so wirket dieser Gedanke gewiß sehr wohlthätig auf uns.
\item Wer aber an die endlose Fortdauer unserer Seele, an einen Zustand höherer Vollkommenheit im andern Leben, und an die innige Verbindung glaubt, in welcher Jesus Christus schon auf der Erde mit Gott gestanden, und jetzt noch fortwährend stehet, der wird auch gerne glauben, daß unser Herr in seinem gegenwärtig \Ahat{verklärten}{erklärten} Zustande unendlich viele und verschiedene Mittel besitze, um einem Jeden, der sein heil.\ Abendmahl auf die gehörige Weise feiert, alle die Segnungen zukommen zu lassen, welche uns die katholische Lehre von seiner Gegenwart in demselben verheißet.
\end{aufzc}
\item Aber eben, weil der würdigen Feier des heil.\ Abendmahles so große Segnungen versprochen sind, und weil wir den Herrn der Herrlichkeit selbst hier in uns aufnehmen sollen: so ist nichts begreiflicher, als daß sich~\RWSeitenw{333}\ derjenige, der unwürdig Antheil nimmt an diesem Mahle, gröblichst versündige, und darum auch der Strafgerechtigkeit Gottes anheim fallen müsse.
\end{aufzb}
\begin{aufza}
\item \RWbet{Einwurf.} Wie kann die Gestalt, eine Adhärenz, noch übrig bleiben, wenn die Ursache dieser Gestalt, die Substanz des Brodes und Weines, entfernt ist? -- \par
\RWbet{Antwort.} Nach dem Gesagten haben wir nicht nöthig, anzunehmen, daß eben dasjenige, was die Substanz der Sache, nach der Bedeutung der Schule, ausmacht, verschwinde oder wohl gar vernichtet werde. Aber auch wenn das wäre, so müßte daraus doch nichts Unmögliches folgen; sondern wir brauchten nur anzunehmen, daß Jesus Christus oder Gott selbst jene Erscheinung bewirke, welche vorhin durch die Substanzen des Brodes und Weines bewirkt wurden. Bekanntlich können auch durch ungleiche Ursachen gleiche Wirkungen hervorgebracht werden.
\item \RWbet{Einwurf.} Wie kann die bloße Gestalt des Brodes doch noch Geruch und Schwere haben, den Körper nähren? \usw\par
\RWbet{Antwort.} Alle diese Erscheinungen gehören zu jener Gestalt, von welcher die Kirche sagt, daß sie nicht abgeändert werde. Unter Gestalt nämlich verstehen wir hier alle sinnlich wahrnehmbaren Beschaffenheiten, gleichviel, ob sie durch das Auge, oder durch was für einen andern Sinn sie aufgefaßt werden mögen.
\item \RWbet{Einwurf.} Wie kann derselbe Jesus mit Fleisch und Blut an tausend Orten zugleich zugegen seyn?\par
\RWbet{Antwort.} Hier ist die zweite göttliche Person, sodann die Seele und der Leib Jesu zu unterscheiden. Die zweite göttliche Person ist, als Gott selbst, allgegenwärtig; doch hat, wie wir schon wissen, diese Gegenwart Gottes an allen Orten ihre verschiedenen Grade, und in dem allerheiligsten Altarsacramente ist Gott, und insbesondere die zweite göttliche Person, in einem ganz vorzüglichen Grade zugegen. Die Seele Jesu kann als ein Geist gleichfalls an mehreren Orten zugleich gegenwärtig seyn, selbst in dem Sinne, wenn unter der Gegenwart eines Geistes an einem Orte eine gewisse~\RWSeitenw{334}\ ganz unvermittelte Wirksamkeit an diesem Orte verstanden werden soll. Was aber den Leib Jesu Christi anlangt: so kann dieser, als ein Körper, \dh\  ein Gegenstand, der einen Raum ausfüllt, freilich nur immer an einem einzigen Orte so gegenwärtig seyn, daß er denselben ausfüllt. Allein von einer solchen Gegenwart redet die Kirche auch nicht; es wird vielmehr ausdrücklich angemerkt, daß die \RWbet{sacramentalische} Gegenwart Jesu Christi im heil.\ Abendmahle zwar eine wirkliche (reale), aber nicht eine körperliche und den Raum ausfüllende Gegenwart sey.
\item \RWbet{Einwurf.} Ist es nicht unanständig, daß Gott im heil.\ Abendmahle (falls er darin wirklich zugegen ist) von Maden und Würmern benaget werden, in Faulniß übergehen könne? \usw\par
\RWbet{Antwort.} Nicht Gott, sondern nur jene Gestalten werden benagt, \udgl\  Nehmen wir aber nichts sie Verunehrendes und zwar freiwillig vor: so kann es auch weder uns selbst zu einem Anstoße gereichen, noch uns bei Gott verantwortlich machen, wenn wider unsern Willen und bei der größten Sorgfalt etwas dergleichen vorfällt. Auch lehrt ja die Kirche, Christus sey nur so lange unter diesen Gestalten gegenwärtig, als sie Gestalten von Brod und Wein sind. Geht \zB\  der Wein in Essig \udgl\  über, so hört auch Christus auf, unter ihm gegenwärtig zu seyn; und eben deßhalb kann uns, was mit diesem Essig vorgeht, nicht mehr ärgerlich werden.
\item \RWbet{Einwurf.} Ist es nicht unanständig, daß, nach katholischer Lehre, einem Menschen die Macht gegeben seyn soll, durch das Aussprechen gewisser Worte, gleich einem Zauberspruch, Jesum zu zwingen, daß er das vorliegende Brod in seinen Leib verwandle, \usw\ ?\par
\RWbet{Antwort.} Christus wird nicht gezwungen, sondern es ist sein eigener Wille, daß er in diesen Gestalten erscheint. Er thut es nämlich, weil er's verheißen, oder (was eben so viel ist) den allgemeinen Glauben, daß er es verheißen habe, in der katholischen Kirche hat aufkommen lassen. Damit aber Alles, was nur im Geringsten unanständig seyn könnte, bei dieser heil.\ Handlung vermieden würde, weiset die Kirche den~\RWSeitenw{335}\ Priester an, Christum bei dieser Gelegenheit immer von Neuem in den demüthigsten Ausdrücken zu bitten, daß er uns die versprochene Gnade seiner Gegenwart auch dießmal schenken wolle. Sehr weise lehrt sie aber, daß Christus erscheine, auch wenn der Priester die ihm zur Pflicht gemachte Bitte nicht mit gehöriger Andacht verrichtet; denn wäre diese Bedingung hiezu erforderlich: so könnten die Gläubigen nie recht gewiß seyn, daß sie des Glückes, Jesum in ihrer Mitte zu haben, genießen.
\end{aufza}

\vabst \textbf{B.}~\RWbet{Sittlicher Nutzen dieser Artikel}.\par
Der Glaube an die wirkliche Gegenwart Christi im heil.\ Abendmahle ist das ergiebigste Beförderungsmittel der Andacht.
\begin{aufzb}
\item Einmal schon darum, weil er den stärksten Verpflichtungsgrund dazu enthält. Wenn wir glaubten, daß Christus im heil.\ Abendmahle nicht in der That und unabhängig von unserer Vorstellung, sondern (wie die Reformirten lehren) nur bildlich und nur in so ferne zugegen sey, als wir ihn uns unter diesen Gestalten vorstellen wollen: so würden wir, eben weil es von unserer bloßen Willkür abhängt, ob er hier gegenwärtig sey oder nicht, keine so starke Verbindlichkeit zur möglichsten Andacht fühlen. Glauben wir aber, daß er, sobald der Priester nur jene Segensworte ausgesprochen hat, zugegen sey, auch wenn wir nicht an ihn denken, und daß diese Gegenwart in einer eigentlichen und, wenn wir es anders wollen, für uns selbst überaus wohlthätigen Wirksamkeit bestehe: so sehen wir deutlich ein, daß wir uns auf das Gröblichste versündigen würden, wenn wir in seiner Gegenwart zerstreut und unandächtig wären.
\item Bei allem Nachdenken, besonders über abstracte Gegenstände, also bei allem Gebete, kommt es uns sehr zu Statten, wenn irgend ein passender sinnlicher Gegenstand da ist, welchen wir mit den Blicken unserer Augen ohngefähr eben so festhalten können, wie wir jetzt mit den Augen unseres Geistes den Gegenstand unseres Nachdenkens festhalten wollen. Niemand schaut hin und her, wenn er ernst nachdenken will. Zu diesem Gegenstande können uns freilich gar mancherlei Dinge dienen, auch~\RWSeitenw{336}\ solche, die mit der Sache, worüber wir nachdenken wollen, in keiner Beziehung stehen. So heftet \zB\  Mancher den Blick beim Nachdenken zur Erde. Viel besser aber ist es, wenn der Gegenstand, auf dem unsere Augen ruhen, mit der Betrachtung, die wir anstellen sollen, in einer gewissen Beziehung steht. So wird uns also \zB , wenn wir die Absicht haben, unser Gemüth zu Jesu zu erheben, ein Kreuzeszeichen, ein Bildniß, das unsern Herrn auf eine nicht ganz unwürdige Weise darstellt, sehr gute Dienste leisten. Bildnisse aber, und alle Gegenstände, die aus sehr mannigfaltigen Theilen zusammengesetzt sind, können uns auch leicht auf fremdartige Gedanken ableiten, \zB\  dadurch, daß wir in eine jetzt sehr unzeitige Betrachtung über den Grad der Vollendung, welche die künstliche Darstellung habe, und über das, was ihr noch mangelt, verfallen. Nichts von dem Allen steht zu befürchten, wenn wir auf die sehr einfachen Gestalten des Brodes und Weines unsere Blicke zu heften angewiesen werden, und dieß zwar mit dem Bedeuten, daß, so wie Brod und Wein die stärkendsten Nahrungsmittel für unsern Leib sind, und sich bei ihrem Genusse auf's Genaueste vereinigen, so sey auch Jesus die stärkendste Nahrung für unsere Seele, und sey bereit, sich auf das Innigste mit ihr zu vereinigen, wenn wir nur selbst dazu thun.
\item Wenn der Gegenstand, an den wir denken sollen, Wirklichkeit hat, so ist es uns, wenn unsere Vorstellung lebhafter werden soll, ein Bedürfniß, ihn wenigstens in unserer Einbildung zu vergegenwärtigen. Bei diesem Geschäfte stört uns aber gewöhnlich nichts mehr, als das immer wiederkehrende Bewußtseyn, daß jene Vergegenwärtigung bloße Einbildung, und daß unser Gegenstand doch in der Wirklichkeit nicht da sey. Im Gegentheile also kann uns das Andenken an einen Gegenstand, also auch die Beschäftigung des Geistes mit Jesu, nichts so sehr erleichtern, als seine wirkliche Gegenwart. Es fragt sich nur, ob er uns etwa in seiner ehemaligen ganz menschlichen oder in was für einer Gestalt er sonst erscheinen müßte, damit uns sein Erscheinen am Wohlthätigsten wäre. Das Erstere würde zwar Anfangs den~\RWSeitenw{337}\ stärksten Eindruck auf uns machen; in der Folge aber, je öfter wir Jesum gesehen haben würden, um desto gleichgültiger würde selbst dieser Anblick uns lassen. Wenn uns dagegen Jesus, verborgen unter irgend einem natürlichen Gegenstande, erscheint, so zwar, daß bei seinem Erscheinen eigentlich gar keine sichtbare Veränderung eintritt: so wird, je weniger die Sinne angesprochen werden, um desto mehr unser Geist beschäftiget werden. Unser lebhaftes Andenken an Jesum wird jetzt auch ein bloß aus Pflichtgefühl, durch freie Anstrengung des Geistes hervorgebrachtes, also verdienstliches Denken an ihn seyn, während es in dem ersten Falle ein Werk des bloßen äußern Eindruckes auf unsere Sinne wäre, und also gar keine Uebung für unsern Geist darböte. Gehen wir nun die verschiedenen natürlichen Gegenstände durch, die sich zu diesem Zwecke in Vorschlag bringen ließen: so werden wir uns gewiß vergeblich bemühen, etwas ausfindig zu machen, was in allem Betrachte schicklicher wäre, als jene Gestalten des Brodes und Weines, die als die stärksten und erquickendsten Nahrungsmittel des Leibes, die Stärkung, die unserer Seele im heil.\ Abendmahle zu Theil werden soll, wie auch die innige Vereinigung, die Jesus Christus hier mit uns eingehen will, so anschaulich machen. -- Sollen wir aber unter diesen Gestalten des Brodes und Weines Christum als wirklich gegenwärtig verehren dürfen, sollen wir sagen können: Hier ist er --: so müssen nur die\RWbet{ Gestalten} von Brod und Wein, nicht mehr die \RWbet{Stoffe} selbst verbleiben. Was hier auf unsere Sinne einwirkt, was wir hier sehen, \usw , das müssen wir nicht als Wirkungen jener natürlichen Stoffe, sondern als mittelbare oder unmittelbare Einwirkungen Jesu betrachten dürfen.
\end{aufzb}
\begin{RWanm} 
Hieraus ersieht man zugleich, daß weder die Lehre der \RWbet{Reformirten} von einer \RWbet{bloß bildlichen} Gegenwart Jesu, noch jene der \RWbet{Evangelischen} von einer zwar wirklichen, aber \RWbet{auf den Augenblick des Genusses beschränkten} Gegenwart Jesu im heil.\ Abendmahle so vernünftig und sittlich zuträglich sey, wie die katholische. Denn sagen, daß wir im heil.\ Abendmahle nichts Mehreres als eine bloß bildliche Gegenwart~\RWSeitenw{338}\ Jesu annehmen dürfen, heißt sagen, daß wir uns von dem Genusse dieses Mahles durchaus keinen andern und größeren Nutzen versprechen dürfen, als denjenigen, den wir von irgend einem lebhaften Andenken an Jesum haben. Behaupten, daß Jesus zwar wirklich gegenwärtig sey, aber bloß im Genusse, heißt einen der wichtigsten natürlichen Vortheile, den uns der Glaube an die wirkliche Gegenwart Jesu in diesem Mahle gewähren könnte, aufheben. Dieser Vortheil ist, daß wir, wenn Jesus auch schon vor dem Genusse in Wahrheit gegenwärtig ist, ihn nicht nur anbeten dürfen, sondern auch sollen; und daß somit unsere Andacht eine viel längere Dauer erhält und einen weit höheren Grad der Innigkeit ersteigen kann. Wenn uns dagegen gesagt wird, daß jene Gegenwart Jesu erst im Genusse, \dh\  erst in dem Augenblicke eintritt, da alle Sichtbarkeit dieser Gestalten aufhört: so fällt dieser wichtige Vortheil nicht nur hinweg; sondern durch diese Bestimmung wird es uns \Ahat{eigends}{nirgends} zu einer Pflicht gemacht, uns wohl in Acht zu nehmen, daß wir beim Anblicke jener Gestalten nicht etwas von der Andacht, die nur für unsern Herrn Jesum selbst gebührt, in uns aufkommen lassen. Kann nun in einer wahren göttlichen Offenbarung nur eine einzige Bestimmung vorkommen, welche statt unsere Erbauung zu fördern, ihr Abbruch thut? -- 
\end{RWanm}
\begin{aufza}\setcounter{enumi}{6}
\item Es ist ein merkwürdiger Beweis, wie aufgeklärt die Vorsteher der katholischen Kirche dachten, als sie die Meinung, daß man das heil.\ Abendmahl \RWbet{unter beiden Gestalten} genießen müsse, um seiner Segnungen theilhaftig zu werden, als einen \RWbet{Aberglauben} verwarfen. -- Den Buchstaben der Schrift hatten sie allerdings wider sich; denn Jesus sprach: \erganf{Trinket Alle aus diesem Kelche} (\RWbibel{Mt}{Matth.}{26}{27}\ vergl.\ \Ahat{\RWbibel{Mk}{Mark.}{14}{23}}{17,23.}\ und \RWbibel{Joh}{Joh.}{6}{53}). Allein weil es in der Folge sich zeigte, daß der Genuß des Kelches in einer sehr großen gemischten Versammlung manchen Uebelstand verursache, und daß im Gegentheile durch eine kluge Verfügung, die diesen Genuß nur dem einzigen Stande der Priester einräumt, das so nothwendige Ansehen dieses Standes noch mehr gehoben werden konnte (daß also nichts als Vortheile aus einer solchen Anordnung hervorgehen würden): so trugen die Vorsteher der Kirche kein Bedenken, hier von dem Buchstaben der Vorschrift Jesu abzuweichen. Und sie blieben bei ihrer Verfügung desto standhafter, je mehr sich aus den Klagen~\RWSeitenw{339}\ Vieler verrieth, daß sie der thörichten Meinung wären, als ob nur derjenige Christum ganz empfinge, der jene \RWbet{beiden} Gestalten empfängt. Dieses erklärten sie für eine thörichte Meinung, weil es doch in der That thöricht und ungeziemend ist, Gott zuzumuthen, daß er den Menschen eine verheißene Gnade entziehen, ja auch nur schmälern werde, wenn sie an jenen Handlungen, welche er ihnen als Bedingung vorgeschrieben hat, einen oder den andern an sich ganz \RWbet{gleichgültigen} Umstand \RWbet{aus guter Absicht} ändern, oder die Handlung durch eine gewisse Veränderung nur noch erbaulicher machen.
\item Daß man das heil.\ Abendmahl auch als ein \RWbet{Opfermahl} in dem oben angegebenen Sinne betrachten dürfe, wird kein Vernünftiger bestreiten.
\item Was in dem 9ten Absatze vorkommt, sind keine Glaubenslehren, sondern bloße Vorschriften der katholischen Kirchenvorsteher, die in einigen Stücken immerhin unvollkommen seyn können. In der That sind aber diese Vorschriften von einer solchen Beschaffenheit, daß wir wohl Ursache haben, mit ihnen zufrieden zu seyn, bis etwa auf die unter i) erwähnten Gebete und Ceremonien, in Betreff deren denn doch zu wünschen wäre, daß Einiges bald besser eingerichtet würde, so gut und \Ahat{unübertrefflich}{übertrefflich} auch vieles Andere ist.
\end{aufza}

\RWpar{289}{Die Lehre des Katholicismus von der Buß- und Besserungsanstalt}
\begin{aufza}
\item Um in seiner Vollkommenheit desto gewisser und leichter fortzuschreiten, soll jeder katholische Christ, weß Standes er auch sey, zu diesem wichtigen Geschäfte sich einen \RWbet{freundschaftlichen Gehülfen} wählen; und dieß zwar aus der Classe der von ihrem Bischofe eigends dazu \RWbet{geprüften und bevollmächtigten Priester} denjenigen, der unter Allen, welche ihm zu Gebote stehen, durch seine Weisheit und Rechtschaffenheit ihm als der \RWbet{tauglichste} erscheint.
\item Jedem katholischen Christen wird es zur Pflicht gemacht, daß er von Zeit zu Zeit eine \RWbet{geheime Prüfung} mit sich selbst vornehme, \dh\  daß er im Andenken an Gott,~\RWSeitenw{340}\ den Allwissenden, und nach vorausgeschicktem Gebete um seine Erleuchtung, die Frage untersuche, welche Fehltritte er seither begangen habe, und was ihm zu seiner sittlichen Vollkommenheit noch mangle? Alle an sich bemerkten Fehltritte, die er mit deutlichem Bewußtseyn beging, soll er \RWbet{dem gewählten Seelenfreunde mit der Gemüthsstimmung bekennen, als ob er sich vor Gott selbst anklagte;} eine möglichst getreue Schilderung der Mängel seines sittlichen Charakters beifügen, und gerne und aufrichtig Alles beantworten, worauf zur richtigen Beurtheilung seines Seelenzustandes von Seite seines Seelenfreundes gefragt wird. Wohl aber hüte er sich hiebei, daß er \RWbet{nicht, ohne Noth, die Ehre Anderer verletze}, oder auch etwas sage, was seinem \RWbet{Seelenfreunde selbst zu einem Aergerniß gereichen könnte}. Zu loben ist es, wenn man auch bloße Schwachheitsfehler und sogenannte geringe Sünden angibt, doch ist es keine Schuldigkeit dieses zu thun. Ein solches Bekenntniß seiner Fehler (eine \RWbet{sacramentalische Beichte} genannt) soll jeder katholische Christ je eher, je lieber ablegen, sobald er so unglücklich war, sich schwer versündiget zu haben, und vollends wenn er Gefahr läuft, durch Aufschub in noch mehrere Sünden zu verfallen. Aber auch wenn unser Gewissen uns keine schweren Sünden vorrückt, sollen wir doch \RWbet{wenigstens einmal des Jahres}, um die sogenannte österliche Zeit, mit unserem Seelenfreunde uns über unsern sittlichen Zustand berathen.
\item Ueber die Fehler, die wir begangen haben, und über die Unvollkommenheiten, die wir an uns entdecken, sollen wir einen \RWbet{möglichst lebhaften Schmerz} und Unwillen in unserm Herzen zu erwecken trachten; und dieß bei einer jeden unserer Selbstprüfung, vornehmlich aber bei jener, auf die wir das Sündenbekenntniß von unserem Seelenzustande abzulegen gedenken. Die Gründe, aus welchen es uns unserer Sünden gereuet, mögen immerhin zum Theile von ihrer eigenen Häßlichkeit, von den verderblichen Folgen und Strafen derselben, hergenommen seyn, und also in sofern auf einer Art von Selbstliebe beruhen. Nur wäre diese Reue eine noch \RWbet{unvollkommene}, bei der wir nicht stehen bleiben sollen, sondern wir müssen uns bemühen, auch eine \RWbet{vollkommene} Reue in uns zu erwecken, \dh\  eine Reue, die auf der Be\RWSeitenw{341}trachtung beruht, daß wir Unrecht gethan, und gegen Gott und Jesum Christum uns versündiget haben.
\item Der Zweck dieser Reue ist unserem \RWbet{Vorsatze}, nicht mehr zu sündigen, eine um so größere Stärke und Festigkeit zu ertheilen. Die Bildung dieses Vorsatzes ist etwas, das uns ganz vornehmlich angelegen seyn muß. Wir sollen nicht bloß mit einem solchen uns begnügen, der \RWbet{im Allgemeinen} festsetzt, künftig nichts Böses zu thun; sondern wir sollen, so viel es nur immer möglich ist, \RWbet{im Einzelnen} bestimmen, wie wir bei diesen und jenen Gelegenheiten in Zukunft vorgehen wollen. Und diese Vorsätze der Besserung sollen wir auch unserem Seelenfreunde in der Gemüthsstimmung, als ob wir sie Gott selbst gelobten, mittheilen.
\item Wir sollen endlich darüber nachdenken, ob und auf welche Weise sich der durch unsere Thorheiten \RWbet{angerichtete Schaden zum Theile wenigstens wieder gut machen ließe}, durch welche Mittel wir ferner unsere bisherigen bösen Gewohnheiten und Neigungen am Sichersten ablegen, uns vor dem Rückfalle in unsere Fehler oder vor anderen neuen Verirrungen bewahren, und unsere sittliche Vollkommenheit erhöhen könnten. Alle Verhaltungsregeln, die unser eigenes Gewissen uns bei diesem Nachdenken uns als unsere Pflicht darstellt, sind es auch wirklich; und wir werden recht thun, was wir in dieser Hinsicht uns vornehmen, auch unserem Seelenfreunde wissen zu lassen; besonders aber in Betreff solcher Puncte, welche uns selbst noch zweifelhaft sind, um seine Entscheidung zu bitten.
\item Dem Seelenfreunde liegt es zuvörderst als strengste Verbindlichkeit ob, nichts von dem Allen, was er aus unserer Beichte erfuhr, \RWbet{selbst wenn es sehr ehrenvoll für uns wäre, anderen Personen mitzutheilen}, ja auch nur \RWbet{irgend einen Gebrauch} davon in seinen Reden und Handlungen zu machen, dadurch man auf den Inhalt unserer Beichte zurückschließen könnte. Er darf sogar \RWbet{nicht einmal mit uns selbst außerhalb der Beichte} von dem, was wir ihm in derselben anvertraut hatten, zu sprechen anfangen; es sey denn, daß wir ihm die Erlaubniß dazu ertheilet hätten. Er soll sich ferner bemühen, uns bei dem schweren~\RWSeitenw{342}\ Geschäfte der Besserung in allen denjenigen Stücken behülflich zu werden, in welchen wir es bedürfen und es ihm möglich ist, uns zur Hülfe zu kommen. Er soll uns also
\begin{aufzb}
\item auf unsere Fehler aufmerksam machen und sie in ihrer Häßlichkeit, in ihren verderblichen Folgen, in ihrer ganzen Strafwürdigkeit darstellen, wenn es sich zeigt, daß wir sie, etwa durch Eigenliebe und durch was immer für andere Umstände gehindert, nicht so, wie wir sollten, erkennen. Und wenn er
\item wahrnimmt, daß wir nicht die gehörige Reue über unsere Fehler empfinden: so soll er durch alle ihm zu Gebote stehenden Mittel dahin wirken, daß diese Reue erwache. So wie im Gegentheile, wenn es sich finden sollte, daß unser Schmerz zu unmäßig sey, daß er \zB\  unsere Gesundheit zerstöre, oder uns an der Vollziehung selbst jenes Guten, wozu noch Gelegenheit da ist, verhindere: so soll er denselben in seine gehörigen Schranken zurückzuführen suchen.
\item Ganz vornehmlich müsse er aber darauf sehen, daß die guten Vorsätze, die wir für die Zukunft fassen, möglichst bestimmt, lebhaft und ausdauernd werden. Um uns an ihre Erfüllung, die unser eigener Vortheil erheischt, noch fester zu binden, verlange er, wenn wir es nicht von selbst thun, daß wir die wichtigsten derselben ihm namentlich angeloben.
\item Er rathe uns in allen denjenigen Stücken, in welchen wir uns selbst nicht zu rathen wissen; und scheue sich nicht, uns auch solche Wahrheiten, die uns recht unangenehm sind, zu sagen, sobald es zu unserem Heile nothwendig ist.
\item Unser Seelenfreund hat nicht nur das Recht, sondern sogar die Pflicht, uns die Verrichtung einiger schon an sich guter Handlungen, die gleichwohl noch keine bestimmte Schuldigkeit für uns wären, zur Pflicht aufzulegen, und von uns zu verlangen, daß wir sie ohne allen Aufschub, und in dem Geiste der Buße, \dh\  mit der frommen Meinung, verrichten, daß Gott, der Barmherzige, sie als eine Art von \RWbet{Genugthuung} für den Schaden, den wir durch unsere Sünden angerichtet haben, annehmen wolle.~\RWSeitenw{343}\ Zu diesen guten Werken, die man \RWbet{Genugthuungen} oder \RWbet{Bußübungen} nennt, müsse der Seelenfreund nur solche Handlungen wählen, die unserem Seelenzustande und unseren Verhältnissen genau anpassen, deren Verrichtung uns ferner gewiß nicht unmöglich seyn wird, die endlich auch so bestimmt sind, daß wir uns hintenher nicht leicht durch den Gewissenszweifel, ob wir auch unserer Schuldigkeit entsprochen, beunruhiget fühlen können.
\end{aufzb}
\item Wenn wir auf diese Weise alle natürlichen und uns zu Gebote stehenden Mittel zu unserer Besserung und zur Wiedergutmachung des durch die Sünde angerichteten Schadens theils schon wirklich angewendet haben, theils noch in Zukunft anzuwenden gedenken: so dürfen wir auch die frohe Hoffnung fassen, \RWbet{daß uns um der Verdienste Jesu Christi willen die Schuld unserer begangenen Sünden vor Gottes Richterstuhle von nun an dergestalt nachgelassen werde,} daß uns höchstens nur gewisse zeitliche Strafen derselben zu ertragen bleiben; wir dürfen uns vorstellen, daß wir jetzt gleichsam neue Menschen geworden und unsere verlorene Unschuld wieder erlanget haben; wir dürfen hoffen, daß uns Gott jetzt wieder eine höhere und übernatürliche Unterstützung und Hülfe werde angedeihen lassen, durch die es uns möglich seyn wird, die neu erlangte Unschuld für immer zu bewahren; wir dürfen hoffen, daß wir die Größe und Dauer selbst der bloß zeitlichen Strafen, die uns bevorstehen, durch einen stets regen Eifer im Guten um ein Beträchtliches werden vermindern können; daß endlich selbst jenes Unheil, welches wir durch die Sünde angerichtet haben, durch Gottes unendliche Weisheit und Macht auch manches Gute veranlassen werde.
\item Allein das Urtheil, ob wir die hier erwähnten Bedingungen alle erfüllt haben, sollen wir nicht selbst über uns aussprechen wollen, sondern \RWbet{dieß müssen wir der Einsicht und Beurtheilung unseres Seelenfreundes überlassen}. Er ist es, der uns die eben beschriebenen erfreulichen Zusicherungen (den sogenannten \RWbet{Sündenerlaß} oder die \RWbet{Absolution}) in der Person eines Richters an Gottes Statt entweder zu ertheilen oder vorzuenthalten hat,~\RWSeitenw{344}\ je nachdem er nach Gründen der Wahrscheinlichkeit vermuthen kann, daß wir für den Zweck unserer Besserung wirklich Alles gethan haben, was in unseren Kräften stand oder nicht. Er wird es zu verantworten haben, wenn er aus überwindlichem Irrthume auf Erden löste, was nicht im Himmel gelöset ward, oder umgekehrt.
\item Wer endlich außer Stande ist, die eine oder die andere der hier gegebenen Vorschriften zu erfüllen, oder sie nur durch Verletzung höherer Pflichten erfüllen könnte, der hat den übernatürlichen Erfolg der Nachlassung seiner Sünden auch dann noch zu erwarten, wenn er nur die erwähnte vollkommene Reue empfindet, und übrigens alles dasjenige thut, was er in seinen Umständen noch vermag.
\end{aufza}

\RWpar{290}{Historischer Beweis dieser Lehre}
Aus den Worten Jesu \RWbibel{Joh}{Joh.}{20}{22}: \erganf{Wem ihr die Sünden nachlasset, dem sind sie nachgelassen, wem ihr sie vorenthaltet, dem sind sie vorenthalten}, schloß die katholische Kirche, daß die Apostel, und, aus demselben Grunde, wie sie, auch ihre Nachfolger, die Bischöfe und die Priester, eine gewisse Macht erhalten hätten, Sünden entweder nachzulassen, oder auch vorzuenthalten. Um nun von dieser Macht einen vernünftigen Gebrauch zu machen, müssen sie, schloß man weiter, Kenntniß von den Gesinnungen und dem inneren Seelenzustande der ihnen anvertrauten Gläubigen erhalten; und folglich muß diesen auch die Verbindlichkeit obliegen, ihre Gesinnungen und ihren Seelenzustand den Priestern anzuvertrauen. Wer möchte diese Auslegung der Worte Jesu und diese Folgerung aus ihnen nicht sehr natürlich finden? -- Inzwischen dürfte sich doch nicht darthun lassen, daß schon in dem frühesten, namentlich in dem apostolischen Zeitalter, eine bestimmte Verordnung über die Beichte (eine Verpflichtung zu einer eigentlichen und sogenannten \RWbet{Ohrenbeichte}, bei der man alle, auch die heimlich begangenen Sünden bekennen muß) bestanden habe. Eine solche Verordnung nämlich hätte bei der damals noch so geringen Anzahl von Priestern nicht einmal befolgt werden können. Genug, daß heut zu Tage, ja daß seit Jahrhunderten der allgemeine Glaube~\RWSeitenw{345}\ in unserer Kirche herscht, daß die Verpflichtung zur Beichte eine nicht bloß auf menschlicher, sondern auf göttlicher (nämlich auf einer durch Christum gegebenen) Anordnung beruhe. Ob ein gewisses Gebot als ein bloß menschliches oder echt göttliches anzusehen sey, das hängt nicht davon ab, durch welche Vermittlung es entstanden, sondern bloß davon, ob es die Kennzeichen einer göttlichen Offenbarung habe. Uebrigens finden wir schon im dritten Jahrhunderte in den Schriften der Kirchenväter Stellen, die für die Nothwendigkeit einer Ohrenbeichte sprechen.\par
In des \RWbet{Hermas} \RWlat{Pastor (simil.\ 7.\ Mand.\ 4. c.\,3.)}\RWlit{}{Hermas1} kommt die erste sichere Stelle über die öffentliche Beichte vor; bei \RWbet{Tertullian} aber \RWlat{(de Poenit.\ c.\,9.)}\RWlit{c.\,9 (PL 1, 1355)}{Tertullian5} ist der Gedanke zu lesen: \RWlat{In quantum non peperceris tibi, in tantum tibi, crede, Deus parcet,} \dh\  wenn du deine falsche Schamhaftigkeit dich nicht abhalten lässest, die Sünde zu bekennen, so wird sie Gott dir vergeben. \RWbet{Origenes} bemerkt, daß man die Sünden nicht immer der ganzen Gemeinde, sondern nur \RWbet{Einem Seelenarzte} bekennen müsse. Dieser mag urtheilen, \erganf{\RWlat{an languor talis sit, qui in conventu totius ecclesiae exponi debeat et curari. Multa hac deliberatione et satis perito medici illius consilio procurandum est}}. Und \RWlat{Homil.\ 3.\ in Levit.}\RWlit{}{Origenes4} schreibt er: \erganf{Wir müssen hier Alles, was wir begangen haben, bekannt machen. Haben wir etwas heimlich verbrochen, haben wir uns in Reden oder auch nur in Gedanken verfehlt: Alles müssen wir offenbaren; denn wenn wir uns selbst anklagen, vereiteln wir einst die Anklage Satans.}\par
Der heil.\ \RWbet{Cyprian} schreibt \RWlat{(de lapsis)}\RWlit[: \eanf{\RWlat{Confiteantur singuli \auslass\ delictum suum, dum adhuc qui deliquit in saeculo est, dum admitti confessio ejus potest, dum satisfactio et remissio facta per sacerdotes apud Dominum grata est}}]{c.\,29 (PL 4, 503)}{Cyprianus3}: \erganf{Beichte doch Jeder seine Verbrechen, so lange er lebt, so lange sein Bekenntniß noch kann angenommen werden, so lange die vom Priester ertheilte Lossprechung bei Gott noch angenommen wird,} \usw\par
Die übrigen Artikel dieser Lehre müssen wir der Kürze wegen übergehen.

\RWpar{291}{Vernunftmäßigkeit und sittlicher Nutzen}
\begin{aufza}
\item Die Forderung, daß wir uns zu dem Geschäfte unserer sittlichen Ausbildung einen \RWbet{freundschaftlichen Ge}\RWSeitenw{346}\RWbet{hülfen} wählen sollen, ist von der größten sittlichen Wohlthätigkeit für uns.
\begin{aufzb}
\item Hiedurch gewinnt erstlich schon das Geschäft der sittlichen Ausbildung in unseren eigenen Augen an Wichtigkeit; denn daß wir angehalten werden, uns einen Gehülfen dazu zu erwählen, geschieht ja doch offenbar, weil es ein überaus wichtiges Geschäft ist, von dessen glücklichem oder unglücklichem Fortgange unser Loos durch eine ganze kommende Ewigkeit abhängt.
\item Ueber unzählige Fehler, welche wir an uns selbst entweder gar nicht oder doch nicht in ihrer ganzen Häßlichkeit erblicken würden, kann uns der Seelenfreund die Augen öffnen.
\item Von manchen Sünden wird uns die Scham, die wir empfinden würden, wenn wir uns ihrer anklagen müßten,
\item von manchen andern ein noch viel edlerer Beweggrund, die Furcht, unsern Seelenfreund dadurch in neue Betrübniß zu stürzen, abhalten.
\item In Fällen, wo wir uns nicht selbst zu rathen vermögen, wo wir zu nachgiebig oder zu strenge gegen uns wären, wird unser Seelenfreund entscheiden.
\item Je höher etwa der Rang, welchen wir in der bürgerlichen Gesellschaft begleiten, und je größer der Ruf unserer Gelehrsamkeit und unserer Einsichten ist, um desto erbaulicher auch selbst für Andere wird das Beispiel seyn, welches wir ihnen geben, wenn wir uns dieser vortrefflichen Anstalt nicht stolz entziehen.
\item Und können wir wohl, wenn auch der Mann, zu dem wir in Ermangelung jedes Andern unsere Zuflucht zu nehmen genöthiget sind, noch so geistesarm wäre, jemals im Voraus schon gewiß seyn, daß uns nicht irgend ein Wort von seinen Lippen rühren, einen heilsamen Gedanken in uns veranlassen, kurz, daß die Handlung, die uns die Kirche hier vorschreibt, gar keinen sittlichen Nutzen für uns selbst haben werde?
\end{aufzb}
Hieraus ergibt sich aber, daß wir gewiß auf keine Weise uns einst vor Gott verantworten und die Vergebung unserer Sünden von ihm erwarten könnten, wenn wir die~\RWSeitenw{347}\ Vorschrift von der Zuziehung eines Gehülfen zu dem Geschäfte unserer sittlichen Besserung nicht befolgten, selbst in dem Falle, daß sie eine bloß menschliche Anordnung wäre; denn schon nach dem Grundsatze der bloßen natürlichen Religion ist es ja entschieden, daß wir uns keine Hoffnung auf Vergebung machen können, wenn wir nicht Alles erfüllen, was unsere Besserung und unser künftiges Fortschreiten in der Tugend befördern kann. Um wie viel stärker muß diese Verbindlichkeit nicht durch den Umstand werden, daß wir hier keine bloß menschliche, sondern eine im echten Sinne des Wortes göttliche (nämlich durch Gott geoffenbarte) Anordnung vor uns haben!\par
Sehr zweckmäßig ist hiebei auch verordnet, daß wir uns diesen Seelenfreund aus dem Stande der Priester, und zwar aus derjenigen Classe derselben auswählen sollen, welche ihr Bischof eigens geprüft und zu diesem Geschäfte tauglich befunden hat; wodurch uns im Uebrigen gar nicht verwehret wird, nebst diesem noch andere Menschen, wenn wir es nöthig finden, bei unseren sittlichen Angelegenheiten zu Rathe zu ziehen. Im Allgemeinen läßt sich doch immer erwarten, daß unter der hier erwähnten Classe von Menschen, welche in allen Wahrheiten der Religion und Sittenlehre sorgfältiger als Andere, und in dem Geschäfte der Leitung menschlicher Seelen ganz eigens unterrichtet wurden, die tauglichsten Führer für uns anzutreffen seyn werden.
\item Richtige Erkenntniß unserer Fehler ist die erste Bedingung zu ihrer Besserung, öftere \RWbet{Selbstprüfungen} aber sind das ausgiebigste Mittel, um zu dieser Selbstkenntniß zu gelangen. Wie zweckmäßig ferner, daß wir dieser Selbstprüfung erst ein eigenes Gebet um die Erleuchtung des göttlichen Geistes vorausschicken, und dann auch während derselben stetes Andenken an Gott, den Heiligen und Allwissenden, unterhalten sollen! Wie nothwendig ist dieß, um uns bei diesem Geschäfte vor jeder gefährlichen Selbsttäuschung zu bewahren, und nicht über Vieles, was Tadel verdient, leichtsinnig wegeilen zu lassen! \usw\ Daß wir die Fehler, die wir an uns gewahren, besonders alle diejenigen Sünden, die wir mit deutlichem Bewußtseyn begingen, dem gewählten Seelenfreunde, wie vor Gott selbst, bekennen sollen, dient uns~\RWSeitenw{348}
\begin{aufzb}
\item zur mehreren Selbstbeschämung. Ungleich lebhafter fühlen wir das Häßliche und das Entehrende der Sünde, wenn wir sie einem Andern eingestehen sollen. Um diese Beschämung nicht wieder fühlen zu müssen, oder vielleicht auch, um unserem Freunde die Kränkung zu ersparen, die unser Rückfall ihm verursachen würde, nehmen wir uns künftig viel sorgfältiger in Acht vor allem Bösen.
\item Durch die Erzählung unserer Fehler setzen wir unsern Seelenfreund in den Stand, ein richtiges Urtheil über unsern sittlichen Zustand zu fällen. Er kann uns nun über so manchen Fehler, den wir nur halb oder gar nicht erkennen, die Augen öffnen, kann uns die Schädlichkeit und Verderblichkeit gewisser Sünden in ein helleres Licht setzen, kann uns manchen Rath, wie wir die Sünde in Zukunft vermeiden können, geben, \usw\
\end{aufzb}
Weislich entschied jedoch die Kirche, daß wir die sogenannten läßlichen Sünden nicht nothwendig anzeigen müßten, weil es in manchen Fällen sehr schwer, wohl gar unmöglich wäre, sie richtig und vollständig anzugeben, und unser Gewissen also durch eine solche Vorschrift nur beschweret würde.
\item Das schmerzliche Gefühl der \RWbet{Reue} hat man zwar häufig für eine unnütze Selbstquälung erklärt, und behauptet, der Vorsatz, nicht mehr zu sündigen, genüge. Allein die Erweckung der Reue hat eben nur den Zweck, zu bewirken, daß unser Vorsatz, nicht ferner zu sündigen, mehr Stärke und Festigkeit erhalte; denn je lebhafter wir eine That bereuen, um desto sorgfältiger pflegen wir uns vor ihrer Wiederholung zu hüten. Wer glauben würde, daß eine Reue, die keine Besserung zur Folge hat, das bloße Weinen und Seufzen an und für sich verdienstlich sey, der würde sich freilich sehr irren; aber diesen Irrthum hat die katholische Kirche nie gelehrt, sondern ausdrücklich verworfen. Wie vernunftmäßig ferner auch das sey, was die katholische Kirche von den Beweggründen zu dieser Reue sagt, erhellet aus den Begriffen, die wir im 1sten Haupttheile von den Beweggründen zur Tugend aufstellten, von selbst.
\item Sehr gut ist es gewiß, daß die katholische Kirche die Fassung des \RWbet{Vorsatzes} von der Erweckung der Reue~\RWSeitenw{349}\ noch unterscheidet; denn so kann das Vorurtheil, als ob die Reue auch ohne Vorsatz und somit ohne Besserung von einem Werthe sey, um so weniger Platz greifen. Eben so gut ist es auch, daß sie verlangt, wir sollen diesen Vorsatz vor unserem Seelenfreunde, gleich als vor Gott, aussprechen; denn dadurch gewinnt er an Feierlichkeit, Verbindlichkeit und Stärke.
\item Nichts ist nothwendiger, als daß man uns anhalte, auf eine \RWbet{Wiedergutmachung des durch die Sünde angerichteten Schadens} zu denken, bei Zeiten in Ueberlegung zu ziehen, auf welche Weise wir uns vor den Gefahren eines Rückfalles bewahren wollen, \usw\
\item Die \RWbet{Pflicht der Verschwiegenheit} muß ganz den Umfang erhalten, den der katholische Lehrbegriff ihr anweiset, wenn die heil.\ Handlung der Beichte nicht als Mittel, um die Geheimnisse der Menschen auszuspähen, gemißbraucht, oder auf jeden Fall allmählich verhaßt werden soll. Dieß Letztere würde nämlich schon dann erfolgen, wenn es dem Seelenfreunde erlaubt wäre, den, der ihm sein Vertrauen einmal geschenkt, in der Folge von freien Stücken wieder an das, was er ihm einst gestanden, zu erinnern; geschähe es auch in der besten Absicht. Die Zweckmäßigkeit der übrigen in diesem Abschnitte ertheilten Vorschriften leuchtet von selbst ein. Nur über den letzten Punct einige Worte. Wenn wir demjenigen, der etwas Böses gethan, gleich von dem Augenblicke, da er den Vorsatz gefaßt, es künftig nicht wieder zu thun, die Erlaubniß geben wollten, sich über sein vorhergegangenes Betragen zufrieden zu stellen: so würden wir nur das sittliche Gefühl der Menschen abstumpfen; denn dann hätte ja Jener, der nie etwas Böses gethan, nichts vor demjenigen, der sich der gröbsten Vergehungen schuldig gemacht hat, voraus; es wäre denn, daß es dem Letztern vielleicht nur etwas schwerer fällt, sich von dem Bösen, dessen verführerische Reize er kennen gelernt hat, zu enthalten. Wenn wir dagegen verlangen, daß er ein und das andere Gute, das an sich selbst nicht seine Schuldigkeit gewesen wäre, mit der Gesinnung verrichte, daß es nun seine Schuldigkeit geworden sey, weil er doch eine Art von Ersatz für den gestifteten Schaden zu leisten verbunden sey, und wenn wir~\RWSeitenw{350}\ ihm erst, nachdem er dieß Alles vollendet hat, gestatten, daß er sich über die Vergangenheit einiger Maßen beruhige: so wird nicht nur das sittliche Gefühl bewahret, sondern es kommt auch des Guten in der Welt offenbar mehr zu Stande. Wenn aber einige Gegner des Katholicismus sich an den Namen: \RWbet{Genugthuung} stoßen, welchen wir diesen guten Werken zuweilen wohl ertheilen: so vergessen sie, daß wir ausdrücklich lehren, daß diese guten Werke nicht durch ihren eigenen Werth, welchen wir allerdings als sehr gering betrachten, sondern \RWbet{nur durch den Werth der unendlichen Verdienste Jesu Christi} und durch die Barmherzigkeit Gottes eine genugthuende Kraft erhalten.
\item Die Möglichkeit einer Vergebung der Sünden durch die Verdienste Jesu Christi wurde bereits gezeigt. Die Bedingungen aber, unter welchen sie uns die katholische Kirche hier verspricht, können nicht vorsichtiger ausgewählt seyn. Sie fordert, daß wir zum Zwecke unserer Besserung erst Alles das leisten, was wir durch unsere eigenen Kräfte vermögen; und worin dieß Alles bestehe, setzet die Kirche so vortrefflich auseinander, wie es kein Sittenlehrer je früher gethan hatte.
\item Daß wir aber \RWbet{das Urtheil, ob wir auch Alles gethan haben}, nicht eigenmächtig über uns aussprechen, sondern \RWbet{es unserem Seelenfreunde anheim stellen sollen}, ist dem Gesetze der Billigkeit vollkommen angemessen, daß Niemand Richter in seiner eigenen Sache seyn soll. Hier würde uns nämlich gar oft Trägheit und Eigenliebe überreden, daß wir Alles gethan, was nur gefordert werden könne, während wir doch viel Mehreres thun könnten und sollten. Zuweilen würden wir auch zu viel von uns verlangen. In jedem Falle aber muß uns der Umstand, daß wir in einer so wichtigen und so schwer zu beurtheilenden Angelegenheit nicht unserm eigenen Bedünken allein überlassen sind, sondern daß wir die Meinung noch eines Andern, und eines wohl Unterrichteten einholen müssen, und daß auch dieser mit uns gleichförmig denkt, und uns versichert, wir hätten das Unsrige jetzt gethan, zur Beruhigung gereichen. Wir müssen entweder sehr leichtsinnig und um unser Seelenheil unbesorgt seyn, oder ein eben so thörichtes als gefährliches~\RWSeitenw{351}\ Vertrauen in die Unfehlbarkeit unserer eigenen Ansichten setzen, wenn wir gar kein Bedürfniß einer solchen Beruhigung fühlen.\par
\RWbet{1.~Einwurf.} Die Kirche selbst lehrt, daß eine Lossprechung aus des Priesters Mund, wenn wir nicht unser Möglichstes gethan haben, vor Gott nicht gelte; daß aber im Gegentheile, wenn wir dieß gethan, die Nachlassung unserer Sünden auch ohne die priesterliche Lossprechung erfolge. Hieraus erhellet denn deutlich, daß diese ganze Handlung der Lossprechung unnütz und überflüssig sey!\par
\RWbet{Antwort.} Nicht doch! Immer muß es uns sehr tröstend und beruhigend seyn, über die wichtige und so schwer zu entscheidende Frage, ob wir auch unser Möglichstes gethan, das Urtheil noch eines Andern zu hören.\par
\RWbet{2.~Einwurf.} So ist doch wenigstens die Behauptung unrichtig, daß der Priester bei dem Geschäfte der Lossprechung als ein Richter verfahre; sondern er macht hier nur einen bloßen Boten Gottes, der den beunruhigten Sünder an die allgemeine Wahrheit erinnert, daß Gott dem Menschen verzeihe, wenn er erst Alles, was er zu seiner Besserung und zur Wiedergutmachung des angerichteten Schadens zu thun im Stande war, wirklich geleistet hat.\par
\RWbet{Antwort.} Ein Seelenfreund, der sein Amt auf gehörige Weise verwaltet, thut ungleich mehr, als man hier sagt. Er hilft dem Sünder zur Erkenntniß seiner Fehler; vermehrt und veredelt seine Reue; stärkt seine Vorsätze; leitet das ganze Geschäft seiner Besserung; entscheidet in \Ahat{jenen}{jeden} zweifelhaften Fällen, was hier zu thun oder nicht zu thun sey; setzet die guten Werke, welche der reuige Sünder als eine Art von Ersatz für das verübte Böse verrichten soll, fest; \usw\ Wenn anders der Büßende nicht deutlich erkennt, daß die Entscheidungen und Forderungen seines Freundes unrichtig und unbillig sind: so ist er verpflichtet, denselben nachzukommen; und nur erst, wenn er dieß thut, erhält er von Gott die gewünschte Nachlassung seiner Sünden, die ihm der Priester verkündigt. Dieser handelt hier also wirklich nicht als ein bloßer Bote, sondern er handelt auch als ein Richter, nicht zwar nach eigenen, wohl aber nach den Gesetzen Gottes.
\item Die Vernunftmäßigkeit und der sittliche Nutzen dieses Artikels leuchtet von selbst ein.~\RWSeitenw{352}
\end{aufza}
\begin{RWanm} 
Hier mag auch etwas von dem in der katholischen Kirche bestehenden Gebrauche der \RWbet{Ablässe} beigebracht werden. Als in den ersten Jahrhunderten der christlichen Zeitrechnung die höchsten weltlichen Obrigkeiten noch Heiden waren, wurde von Seite des Staates das sittliche Betragen, wie aller übrigen Unterthanen, so auch insonderheit der Christen, einer nur sehr geringen Aufmerksamkeit gewürdiget. Die gröbsten Verbrechen konnte man öffentlich begehen, ohne zu besorgen, daß man von seiner weltlichen Obrigkeit darüber werde zur Rechenschaft gezogen und bestrafet werden. Wie nun ein solcher Mangel an Aufsicht und Bestrafung die Menschen jedesmal verschlimmert: so nahm das Sittenverderbniß auch damals je mehr und mehr überhand, und wirkte begreiflicher Weise auch auf die Christen sehr nachtheilig ein. Um nun dem Uebel wenigstens unter den Christen einiger Maßen zu steuern, bedienten sich ihre geistlichen Vorsteher eines Rechtes, das ihnen einerseits durch das hohe Ansehen, dessen sie sich bei ihren Gläubigen erfreueten, eingeräumt, andererseits aber durch die Gleichgültigkeit der weltlichen Regierung nicht streitig gemacht wurde. Sie schloßen jeden Christen, der sich auf eine gröbere Weise vergangen, und hiedurch ein öffentliches Aergerniß gegeben hatte, nicht nur aus ihrer Gemeinde feierlich aus; sondern sie nahmen ihn auch, wenn er in der Folge seine Verirrung bereuet, und Besserung angelobt hatte, nicht eher wieder auf, als bis er den Ernst seiner Vorsätze durch einen mehrjährigen, durchaus erbaulichen Lebenswandel und durch verschiedene, oft sehr beschwerliche Bußübungen bewiesen. Durch öftere Wiederholung derselben Forderung in ähnlichen Fällen bildeten sich bestimmte Regeln (Bußgesetze, \RWlat{canones poenitentiales}), die festsetzten, was für eine Art von Bußbübung bei jeder Art des Vergehens verlangt werden sollte. War aber für Jene, die sich auf eine öffentliche Weise vergangen hatten, eine so strenge Buße verordnet: so fiel es Jedem von selbst auf, daß es für Andere, die zwar nicht öffentlich, aber doch eben so schwer oder wohl gar noch schwerer gesündiget hatten, geziemend wäre, sich einer ähnlichen, obgleich nicht öffentlichen Bußübung zu unterziehen, bevor sie in ihrem eigenen Gewissen sich beruhigen, und sich der Theilnahme an dem Mahle des Herrn würdig erachten konnten. Und so wurden denn allmählich auch für geheime Vergehun\RWSeitenw{353}gen, Bußregeln vorgeschlagen, und in Ausübung gebracht. Allein nach wenigen Jahrhunderten zeigte es sich, leider! daß fast alle diese Regeln, die für den öffentlichen sowohl als auch die für den geheimen Gebrauch, für das schon zu sehr über Hand genommene Verderben zu strenge abgefaßt seyen; und es stand zu besorgen, daß Viele, die von dem Pfade der Tugend abgeirrt waren, nur darum nicht wieder zurückkehren würden, weil sie die harten Bußübungen scheuten, welche man ihnen dann zumuthen würde. Sehr richtig erkannten die Vorsteher der katholischen Kirche, daß sie unter solchen Umständen von der alten Strenge zwar allerdings ablassen müßten; aber doch nicht, ohne statt dessen, wovon sie abließen, irgend etwas Anderes, wenn auch bei Weitem Leichteres, zu fordern. Sie verordneten also, daß man zu einem immerwährenden Gedächtnisse in der katholischen Kirche erzähle, wie viele, langwierige und beschwerliche Bußübungen unter uns Christen einst im Gebrauche gewesen, und wohl von Tausenden schon mit der größten Bereitwilligkeit geleistet worden seyen; daß man jedoch hinzufüge, nur wegen des traurigen Verderbens der Zeit habe es nöthig geschienen, jene Strenge zu mildern und künftig nur diese und jene äußerst leicht zu erfüllenden Forderungen an diejenigen zu machen, die nach den alten Satzungen so große Opfer zu bringen gehabt hätten. Die Handlungen, die man jetzt forderte, bestanden, oder sollten doch wenigstens der Regel nach nur in gewissen Gebeten, im Fasten, im Almosengeben und in allerlei andern, durchgängig guten und gemeinnützigen Werken bestehen; und wurden, weil sie uns Erlaß von anderen, ungleich beschwerlicheren Leistungen gewährten, \RWbet{Ablässe} oder \RWbet{Ablaßübungen} genannt. Hiernächst erräth man von selbst, was die Benennung: ein \RWbet{vollkommener} oder ein \RWbet{theilweiser} Ablaß, ein Ablaß von so und so viel Jahren, Monaten oder Tagen bedeutet haben mochte. Nach dem verschiedenen Werthe der guten Handlungen nämlich, zu deren Verrichtung man bei der Verkündigung eines Ablasses einlud, erklärte man bald, daß durch die Ausübung derselben die ganze, bald daß nur ein gewisser Theil der Kirchenbuße erlassen seyn sollte, \usw\ Doch eine sehr wesentliche Veränderung mit diesen Ablässen ergab sich, als in der neueren Zeit (ohngefähr im 14ten Jahrhunderte) das Ansehen und die Macht der Kirchenvorsteher so bedeutend herabgesetzt wurde, und als die Gesinnungen und das Verhalten der weltlichen Obrigkeiten gegen die Geistlichkeit ohngefähr so geartet wurde, wie wir es heut zu Tage noch finden. In unsern Tagen~\RWSeitenw{354}\ würden gewiß die wenigsten weltlichen Obrigkeiten, wenn sie auch noch so eifrige Verehrer des Christenthums sind, den geistlichen Vorstehern die Macht einräumen wollen, über diejenigen ihrer Unterthanen, die irgend etwas verbrochen, Strafen von der Art zu verhängen, wie sie vor Zeiten gebräuchlich waren, späterhin aber durch jene Ablässe, ihrer ursprünglichen Bestimmung nach, wieder erlassen werden sollten. Hieraus ergibt sich nun so fort, \RWbet{daß Ablässe ganz nach dem ursprünglichen Sinne in unseren jetzigen Tagen nicht mehr verkündiget werden können;} denn, wo keine Macht zu strafen bestehet, da ist auch keine Macht, etwas von dieser Strafe zu erlassen. Aber noch gar nicht folgt, daß man die Ablässe auch nicht in etwas veränderter Gestalt hätte beibehalten können. So übereilt schloßen weder die zu Trident versammelten Väter, noch irgend ein katholischer Theolog, der über den Ablaß auf eine Weise geschrieben, welche den Beifall Anderer gefunden; obgleich der überaus ärgerliche Mißbrauch, der mit dem Ablaßwesen seither getrieben worden war, eine nicht geringe Versuchung enthielt, die ganze Sache zu verwerfen. Allein so wenig Gutes die bisher verkündigten Ablässe auch gestiftet haben mochten: doch übersahen die Theologen der katholischen Kirche nicht, daß jene Handlungen, zu deren Verrichtung man die Gläubigen durch die Ausschreibung eines Ablasses aufforderte, an sich erbaulich und gemeinnützig waren, oder daß man doch mindestens in Zukunft lauter solche Handlungen dazu erwählen könnte, deren Gemeinnützigkeit außer Streit ist. Sie irrten sich ferner auch, wie der Erfolg dargethan hat, gar nicht, wenn sie den Vorstehern der Kirche versprachen, daß keine weltliche Obrigkeit, wenn sie auch noch so eifersüchtig auf ihre Rechte wäre, gegen die Ausschreibung eines Ablasses, darin man nur die Verrichtung entschieden guter und gemeinnütziger Thaten verlangt, und diese keineswegs als eine bestimmte Schuldigkeit auflegt, sondern bloß als verdienstlich \RWbet{empfiehlt}, etwas einwenden werde; besonders wenn man sie erst um ihre Einwilligung anspricht. Die Theologen erinnerten ferner, daß man die Ablässe nicht mehr als Handlungen darstellen könne, durch die gewisse kirchliche Strafen erlassen werden sollen; aber das könne man, sagten sie, in aller Wahrheit erklären, daß es entehrend für uns wäre, wenn wir die Aufforderung unserer geistlichen Vorsteher zu diesen Ablaßhandlungen nur darum nicht mehr beachten wollten, weil sie nicht mehr die Macht haben, jene kirchlichen Strafen über uns zu verhängen. Das könne man versichern, daß~\RWSeitenw{355}\ die Vollziehung der von unseren Vorstehern uns eigens anempfohlenen Werke bei übrigens gleichen Umständen erbaulicher und verdienstlicher sey, als die Verfolgung gewisser willkürlicher, von uns selbst angenommener Zwecke. Das könne man lehren, daß wir um solcher Handlungen willen, wenn wir sie mit gebessertem Herzen und auf gehörige Weise verrichten, von Gott die Nachlassung so mancher von jenen zeitlichen Strafen, welche bekanntlich auch dann noch stehen bleiben, wenn uns die ewige verziehen worden ist, erwirken können. Und um uns dieser erfreulichen Hoffnung hingeben zu können, ohne doch auf den Werth dieser an sich oft so geringfügigen Werke uns zu Viel einzubilden, setzten sie weise bei, es wäre nicht sowohl der innere Werth der Werke an sich, auf welchem diese Wirksamkeit derselben beruhe, sondern es seyen vielmehr \RWbet{nur die Verdienste Jesu Christi, ingleichen die Verdienste, die sich so viele Tausend bei Weitem bessere Menschen, als wir, durch ihre Tugend erworben haben}, um derentwillen Gott auch das Wenige, welches wir thun, so gnädig ansehe. Dieser letzte gewiß sehr schöne Gedanke ist es, welcher der Lehre der Theologie von dem sogenannten \RWbet{Schatze der Gnade}, die Christus und seine Heiligen uns erworben haben, zu Grunde lieget. Auch die Behauptung endlich, \RWbet{daß Ablässe selbst für Verstorbene gewonnen werden können}, sobald wir sie so verstehen, wie sie die besten katholischen Theologen ausgelegt haben, hat nichts Anstößiges. Wenn es überhaupt vernünftig ist, zu glauben, daß Gott auch Fürbitten, die wir für das Wohl Anderer zu seinem Throne senden, erhöre; und daß ihm Gebete, die wir mit Verrichtung guter Werke verknüpfen, besonders angenehm sind; daß endlich auch Verstorbene gar oft noch einer Hülfe bedürftig seyn können: warum sollte es ungereimt seyn, wenn wir die guten Werke, zu deren Uebung uns die Vorsteher der Kirche bei der Ausschreibung eines Ablasses auffordern, in der frommen Meinung verrichten, damit Gott irgend einer derjenigen Personen, die diesen Schauplatz der Erde früher, als wir, verlassen mußten, Barmherzigkeit erweise? -- Hier verdient endlich auch noch die Weisheit angerühmt zu werden, mit welcher die Vorsteher unserer Kirche \RWbet{besondere Zeitpuncte} benützten, um den schon erschlafften Bußeifer unter uns Christen von Neuem anzuregen. Nicht nur die Tage einer allgemein drohenden Gefahr und Noth, die selbst bei heidnischen Völkern oft zu dem Zwecke der sittlichen Besserung benützt wurden, sondern auch jeder größere Abschnitt~\RWSeitenw{356}\ in dem Verlaufe der Zeit, jeder Abschluß eines neuen Jahrhundertes, ja auch noch kürzere Zeiträume wurden und werden noch heut zu Tage in der katholischen Kirche mit vielem Glücke angewandt, um durch die feierliche Ausschreibung eines Ablasses, verbunden mit passenden Predigten und Andachtsübungen von der verschiedensten Art, den Geist der Buße zu wecken. 
\end{RWanm}

\RWpar{292}{Die Lehre des Katholicismus von der Einweihung in den heil.~Stand der Ehe}
\begin{aufza}
\item Unter den Christen soll es nur lauter \RWbet{einfache} und \RWbet{lebenslängliche Eheverbindungen} geben, \dh\  Verbindungen, welche nur zwischen je zwei und zwei Personen verschiedenen Geschlechtes geschlossen, und, waren sie gültig geschlossen, nur durch den Tod des einen oder des andern Theils aufgelöst werden sollen. In diesen Verbindungen, und nur in ihnen allein, ist es erlaubt, den Geschlechtstrieb zu befriedigen, und zwar auch hier nur auf eine Art, welche den Zweck der Zeugung nicht vereitelt, und auf die Gesundheit der beiden Gatten nachtheilig einwirkt.
\item Der guten Ordnung wegen soll auch in dieser Gesellschaft Ein Theil, und zwar der \RWbet{Mann,} \RWbet{das Oberhaupt} seyn. Diesem soll in streitigen Fällen das Recht der Entscheidung zustehen; dagegen soll ihm auch die Pflicht obliegen, dieses Rechtes der Obergewalt sich nicht willkürlich, sondern nur zum gemeinsamen Besten zu bedienen, und seine Gattin nicht als seine Dienerin, sondern als seine Freundin und treueste Lebensgefährtin zu achten und zu behandeln, sie wie sich selbst, und wie Christus seine Gemeinde geliebt hat, zu lieben.
\item Nicht nur der Staat, sondern \RWbet{auch die Kirche hat das Recht, gewisse Bedingnisse für die Abschließungen der Ehen festzusetzen}, die man nicht übertreten kann, ohne sich zu versündigen; ingleichen andere, bei deren Nichtbefolgung die Ehe nicht einmal gültig wäre \RWlat{(Impedimenta dirimentia).}
\item Katholische Christen, welche gesonnen sind, sich in den Stand der Ehe zu begeben, sollen, nachdem sie zuvor einen~\RWSeitenw{357}\ für ihr Verhältniß und für ihre Bedürfnisse geeigneten geistlichen Unterricht empfangen haben, und auch nach diesem noch bei ihrer Gesinnung verbleiben, \RWbet{ihr Vorhaben der Gemeinde verkündigen lassen,} damit Jeder, der etwas dagegen einzuwenden hätte, Gelegenheit erhielte, es noch bei Zeiten zur Sprache zu bringen.
\item Ist diese für die Verkündigung gesetzlich bestimmte Zeit vorüber, ohne daß eine gültige Einwendung vorgebracht wurde: so stehet es den Brautleuten zu, ihr Vorhaben auszuführen; doch soll es \RWbet{im Angesichte ihres geistlichen Vorstehers,} des Bischofs oder des Pfarrers, vor der ganzen Gemeinde seyn, wo sie einander feierlich geloben, die Pflichten zu erfüllen, die sie als Gatten übernehmen.
\item Der Bischof oder der Pfarrer, der ihnen dieß wechselseitige Gelübde abgenommen hat, erkläre den Bund ihrer Ehe nun geschlossen, und fordere alle Anwesenden auf, in Vereinigung mit ihm Gott den Allmächtigen um Segen für dieses neue Ehepaar zu bitten.
\item Und Gott wird dieß Gebet erhören, und das neue Paar auf eine in Wahrheit übernatürliche Weise segnen und stärken, auf daß sie fähig werden, die neuen, mitunter so schweren Pflichten ihres ehrwürdigen Standes vollkommen zu erfüllen.
\end{aufza}

\RWpar{293}{Historischer Beweis dieser Lehre}
\begin{aufza}
\item Daß der Vertrag der Ehe ein \RWbet{lebenslänglicher} seyn müsse, folgert die katholische Kirche aus den Worten Jesu (\RWbibel{Mt}{Matth.}{19}{4}): \erganf{Habt ihr nicht gelesen, daß der, so den Menschen erschaffen, sie als Mann und Weib geschaffen hat? und daß es heiße: der Mensch wird Vater und Mutter verlassen, um sich mit seinem Weibe zu verbinden? \RWbet{Was nun Gott selbst verbunden hat, das soll der Mensch nicht scheiden}. Moses zwar hat euch um eurer Hartherzigkeit wegen die Ehescheidung erlaubt; allein so war es nicht im Anfange gewesen; und ich erkläre (setze für meine Anhänger fest): Wer immer sich scheidet von seinem Weibe, es sey denn um eines Ehebruchs willen, und eine~\RWSeitenw{358}\ Andere heirathet, wird selbst ein Ehebrecher; und wer die Verstoßene heirathet, wird gleichfalls selbst ein Ehebrecher.} -- Eben dieß folgerte man auch aus den Worten Pauli (\RWbibel{1\,Kor}{1\,Kor.}{7}{39}): \erganf{Das Weib ist durch das Gesetz gebunden, so lange der Mann lebt.}
\item Daß in dieser Gesellschaft \RWbet{der Mann das Oberhaupt} sey, \usw\ folgerte man aus der Stelle \RWbibel{Kol}{Kol.}{3}{18}: \erganf{Ihr Weiber! seyd euren Männern unterthan, wie dem Herrn. Ihr Männer! liebt eure Weiber und kränket sie nicht.} Und  \Ahat{\RWbibel{Eph}{Ephes.}{5}{23}}{5,21.}: \erganf{Der Mann ist das Haupt}; und \RWbibel[V.\,25.]{Eph}{}{5}{25}: \erganf{Ihr Männer! liebet eure Weiber, wie Christus seine Kirche geliebt, der sich für sie dahin gegeben hat}; -- und \RWbibel[V.\,33.]{Eph}{}{5}{33}: \erganf{Liebet eure Weiber, wie euch selbst}; \usw\
\item In den ersten drei christlichen Jahrhunderten, in denen sich die heidnische Obrigkeit um die Aufrechthaltung der guten Sitten sehr wenig kümmerte, mehr aber noch vom 5ten bis 11ten Jahrhunderte, wo die christlichen Kaiser der Geistlichkeit so große Vorrechte einräumten, erließen die Vorsteher der katholischen Kirche gar mancherlei Verordnungen, betreffend den Ehestand und die Bedingungen eines gültigen, oder nicht gültigen, erlaubten oder nicht erlaubten Vertrages bei demselben. Obgleich nun in unsern Tagen dergleichen Verordnungen nicht ohne die Einwilligung der weltlichen Obrigkeit erlassen werden dürfen: so behaupten die katholischen Theologen doch immer noch, daß der Kirche ein Recht zu solchen Verordnungen an und für sich genommen zustehe; und daß sie somit durchaus nicht ihre Macht überschritten, als sie in früherer Zeit dergleichen Verordnungen wirklich erließen. Dieß Letztere wurde ganz förmlich von dem Tridentinischen Kirchenrathe (\RWlat{sess.\,24.\ cap.\,4.\ 12.})\RWlit{Nr. 1804, 1812}{DH} entschieden.
\item[4.--6.]\stepcounter{enumi}\stepcounter{enumi}\stepcounter{enumi} Aehnliche Oeffentlichkeiten sind in der katholischen Kirche bei der Knüpfung eines Ehebandes von jeher beobachtet worden. Das neue Ehepaar mußte vor der Gemeinde erscheinen, wurde vom Priester eingesegnet \udgl\ 
\item Daß auch die Ehe gewisse \RWbet{übernatürliche Segnungen} habe, und somit ein \RWbet{Sacrament} oder \RWbet{Heiligungsmittel} sey, bewies man ehedem aus den Worten des Apostels (\RWbibel{Eph}{Ephes.}{5}{32}), der von der Ehe (nach der~\RWSeitenw{359}\ Vulgata) sagt: \RWlat{Sacramentum hoc magnum est, dico autem in Christo et in ecclesia} (\RWgriech{t`o must'hrion to~uto m'ega >est'in, >eg`w d`e l'egw e>is Qrist'on, ka`i e>is t`hn >ekklhs'ian}). Neuere Theologen gestehen, daß hier das Wort \RWlat{sacramentum} in einer andern Bedeutung vorkomme; allein es ist genug, daß man von jeher geglaubt, daß Gott die eheliche Verbindung auf eine übernatürliche Weise segne. So schrieb schon Tertullian \RWlat{(ad uxorem II.)}:\RWlit[: \eanf{\RWlat{Unde sufficiamus ad enarrandum felicitatem ejus matrimonii, quod Ecclesia conciliat, et confirmat oblatio, et obsignat benedictio, angeli renuntiant, Pater rato habet?}}]{l.\,2 (PL 1, 1415--1416)}{Tertullian4a} \erganf{Wie sollte ich wohl im Stande seyn, die Seligkeit jener ehelichen Verbindung zu schildern, welche die Kirche knüpft, welche das heil.\ Meßopfer bestätigt, welche der Segen des Priesters versiegelt, welche die Engel im Himmel mit Frohlocken verkündigen, und Er, der Allvater, genehmiget!}
\end{aufza}

\RWpar{294}{Vernunftmäßigkeit und sittlicher Nutzen}
\begin{aufza}
\item Es ist höchst merkwürdig, daß, so zufällig und an keine Regel gebunden es auch zu seyn scheint, ob die in einer Familie erzeugten Kinder vom männlichen oder vom weiblichen Geschlechte seyen; ingleichen, ob einmal bei diesem, einmal bei jenem Geschlechte eine größere Sterblichkeit einreiße, dennoch die Anzahl der Erwachsenen (im Alter der Mannbarkeit befindlichen) Personen bei beiden Geschlechtern beinahe zu allen Zeiten ein und dasselbe Verhältniß (24 zu 25) beobachte. Schon diese Erfahrung allein entscheidet, \RWbet{daß die Natur nur einfache Ehen verlange}, und daß die Polygamie sowohl als auch die Polyandrie mit dem Vortheile des Ganzen im Widerspruche stehe. Daß aber die Ehe zu einer \RWbet{nur durch den Tod auflöslichen Verbindung} erhoben werde, gewährt folgende sittliche Vortheile:
\begin{aufzb}
\item Durch diese Verfügung erhalten beide Theile den stärksten Beweggrund, einen so wichtigen Schritt nicht unüberlegt, und nicht ohne die genaue Prüfung zu thun, ob auch der andere Theil es werth sey, eine so unauflösliche Verbindung mit ihm zu knüpfen.
\item Eben deßhalb muß sich auch jeder Theil durch die Wahl des Andern doppelt geehrt finden, und kann nun um so eher glauben, daß es nicht bloß die schnell vorübergehenden Reize des Körpers sind, die dieser an ihm schätze;~\RWSeitenw{360}\ sondern daß er an ihm auch wohl noch gewisse wichtigere und bleibendere Vorzüge des Geistes und Herzens bemerkt haben müsse.
\item Der Umstand, daß alle Hoffnung einer Auflösung dieser Gesellschaft, eher als durch den Tod, beiderseits abgeschnitten ist, macht, daß sich beide Theile viel sorgfältiger in Acht nehmen, die Liebe zum andern sowohl als auch die Gegenliebe desselben nicht zu verlieren, weil sie wohl einsehen müssen, daß sie hiedurch nur sich selbst unglücklich machen würden. So unterdrücken sie also gleich in ihrer ersten Entstehung jede Neigung zu einer andern Person, \usw\
\item Offenbar wird durch diese Unauflöslichkeit der Ehe auch für die Erziehung der Kinder und für die wechselseitige Hülfeleistung am Besten gesorgt.
\end{aufzb}
\begin{RWanm} 
Aus diesen Gründen ist zur Genüge erwiesen, daß eine jede Ehe \RWbet{auf Lebenslang geschlossen} werden müsse; wie denn dieß auch schon die Natur einer wahren Liebe fordert, die jedesmal ewig zu dauern hoffet. Daß aber die einmal gültig geschlossene auch hintenher \RWbet{nie wieder aufgelöst werden} solle, folgt hieraus freilich noch nicht. In der katholischen Kirche wird diese gänzliche Unauflösbarkeit einer jeden gültig geschlossenen Ehe zwar von den meisten, aber doch nicht von allen Theologen behauptet. Man könnte also noch darüber streiten, ob sie auch als ein eigentlicher Glaubensartikel oder als eine bloße Disciplinarvorschrift anzusehen sey. 
\end{RWanm} 
\item Zur Erhaltung der guten Ordnung, und zur Verhinderung unzähliger Zwistigkeiten ist es gewiß nöthig, daß auch in der ehelichen Gesellschaft, bestehet sie gleich nur aus zwei Personen, Ein Theil als \RWbet{Oberhaupt} erkläret werde. Und daß das Christenthum hiezu den \RWbet{Mann} ernennt, dürfte wohl sehr in der Natur gegründet seyn, weil in der Regel doch wirklich der Mann Beides, der stärkere sowohl, als auch der weisere Theil zu seyn pflegt.
\item Die Vorsteher der Kirche haben (wie jeder Mensch) das Recht und die Verbindlichkeit, sich ihres Ansehens bei der Gemeinde, und der Macht, die ihnen dasselbe gewähret, zur Ausführung alles desjenigen Guten zu bedienen, das sie zu~\RWSeitenw{361}\ Stande bringen können. Wenn nun die weltlichen Obrigkeiten entweder zu träge waren, um über gewisse wichtige Gegenstände selbst schickliche Anordnungen zu entwerfen; oder wenn sie die Macht hiezu aus was immer für einem Grunde den Vorstehern der Kirche freiwillig überließen, \zB\  weil sie vermeinten, dieses gezieme sich so, oder die Geistlichen verstünden es besser als sie; \usw : so thaten diese doch gewiß recht daran, wenn sie die ihnen zu Gebote stehende Gelegenheit benützten, und Verordnungen, die ihnen heilsam dünkten, erließen. Hieraus ergibt sich nun schon von selbst, daß die Vorsteher der katholischen Kirche ein Recht gehabt hatten, auch über die Ehe und über die Bedingungen, welche zur Gültigkeit oder Erlaubtheit ihres Vertrages gehören, allerlei Anordnungen zu machen.
\begin{RWanm} 
Ein Anderes ist die Frage, ob diese Anordnungen auch an sich selbst immer ganz zweckmäßig gewesen seyen. Was namentlich jene Verordnung anbelangt, durch welche eine Menge theils natürlicher, theils geistiger \RWbet{Verwandtschaftsgrade}, innerhalb deren die Ehe verboten seyn sollte, eingeführt wurden: so liegt zwar diesen auch eine sehr richtige Ansicht zu Grunde; denn es ist, überhaupt zu reden, zuträglich, daß Ehebündnisse nur unter Personen, die sich in keinem Betrachte verwandt sind, geknüpfet werden; aber es scheint doch, daß man in diesem Verbote einst etwas zu weit gegangen sey, und sie als ein Mittel gemißbraucht habe, um sich durch die in vorkommenden Fällen ertheilten \RWbet{Dispensen} zu bereichern. 
\end{RWanm}
\item[4.--6.]\stepcounter{enumi}\stepcounter{enumi}\stepcounter{enumi} Der heilsame Zweck dieser Anordnungen leuchtet von selbst ein. Die Gelegenheit des hier zu ertheilenden Unterrichtes kann und soll jeder gewissenhafte Seelsorger benützen, um in den Begriffen der Brautleute Alles dasjenige nachzuholen, was in dem Jugendunterrichte vielleicht versäumt worden ist, und was ihnen in jenem Alter füglich noch gar nicht beigebracht werden konnte; er kann und soll sie vor jeder Uebereilung bei ihrem Vorhaben warnen, und wenn vorherzusehen ist, daß ihre Verbindung unglücklich seyn werde, etwa, weil sie einander nicht wahrhaft lieben, aus bloßer Furcht oder aus Eigennutz ihr Jawort geben wollen \udgl : so kann und soll er versuchen, sie noch bei Zeiten zu einem vernünftigen Entschlusse zu bestimmen. Aehnliche Vortheile~\RWSeitenw{362}\ haben auch die Verkündigungen. Schon durch den bloßen Aufschub, den sie verursachen, gewähren sie dem neuen Brautpaare Zeit, sich, wenn es nöthig seyn sollte, noch eines Besseren zu besinnen. Viel wichtiger ist aber noch, daß eine solche Kundmachung Jedem, dessen Rechte vielleicht durch die beabsichtigte Verbindung verletzt würden, oder der sonst etwas dagegen vorzubringen hat, nicht nur eine Gelegenheit gebe, sondern ihn selbst auffordere, sich am gehörigen Orte zu melden. Der Umstand, daß die beiden Eheleute das wechselseitige Versprechen der Liebe und Treue öffentlich ablegen müssen, ertheilt demselben ungleich mehr Feierlichkeit und bindende Kraft. Die ganze Gemeinde kann nun als Zeuge wider sie aufstehen. Und wenn sie hierauf nicht nur den Priester, sondern auch die versammelte Gemeinde für sich beten sehen: von welcher Wichtigkeit muß ihnen der Schritt, den sie jetzt eben thun, erscheinen! -- Wie muß auch diese Theilnahme Aller sie rühren und sie inniger verbinden mit dem Ganzen!
\item Wenn man den großen Einfluß betrachtet, welchen der Ehestand nicht nur auf die Tugend und Glückseligkeit des verehelichten Paares selbst, sondern auch auf ihre Kinder und auf so viele andere Personen hat; wenn man erwägt, wie vieles Unheil aus unglücklichen Ehen, aus verwahrloster Kindererziehung \usw\ entspringt: so muß man erkennen, daß es der Güte und Heiligkeit Gottes gewiß entsprechend sey, die Personen, welche in diesen Stand treten, durch Mittheilung \RWbet{solcher Gnaden} zu stärken, welche den Namen der \RWbet{übernatürlichen} in der schon oft erklärten Bedeutung verdienen.
\end{aufza}

\RWpar{295}{Die Lehre des Katholicismus von der Kirche und von den Vorstehern derselben oder dem geistlichen Stande}
\begin{aufza}
\item Der katholische Lehrbegriff macht es einem Jeden, der sich von seiner Wahrheit überzeugt hat, zur Pflicht, sich an die Religionsgesellschaft anzuschließen, die der Herr Jesus gestiftet hat, der er auch fortwährend als ihr oberster Leiter und Gesetzgeber vorsteht; daher sie wahrscheinlich auch den~\RWSeitenw{363}\ Namen der \RWbet{Kirche}, \dh\  einer vom Herrn gestifteten Religionsgesellschaft (\RWgriech{<h kuriak`h >ekklhs'ia}) annahm.
\item Es besteht aber der \RWbet{Zweck dieser Gesellschaft}, um ihn zuvörderst nur \RWbet{im Allgemeinen} zu bestimmen, \RWbet{in der Erreichung alles desjenigen Guten in Zeit und Ewigkeit, was sich durch eine solche Summe von Kräften hervorbringen läßt, als da zusammenkommen}, wenn nicht nur Allen, die von der Wahrheit des katholischen Lehrbegriffes überzeugt worden sind, zur Pflicht gemacht wird, sich zu vereinigen, sondern wenn überdieß auch noch so viele andere Menschen, als es nur zuträglich ist, in den Verein mit aufgenommen werden.
\item Insonderheit soll dieser Verein für die Erreichung \RWbet{unserer Glückseligkeit} nicht bloß \RWbet{in diesem,} sondern, und zwar ganz vornehmlich, \RWbet{in dem zukünftigen Leben} sorgen; wie er denn eben deßhalb auch noch dort fortgesetzt werden, ja dort noch ungleich wirksamer sich bezeugen soll.
\item Für diese Erde sind es vornehmlich folgende Zwecke, deren möglichster Verwirklichung die Mitglieder dieses Vereins nachstreben sollen:
\begin{aufzb}
\item Durch eine \RWbet{wechselseitige Mittheilung ihrer Gedanken} über dasjenige, worüber sie noch \RWbet{verschiedener Meinung} sind, sollen sie trachten, je mehr und mehr \RWbet{einstimmig} mit einander zu werden, und in dieser Einstimmigkeit eben das sicherste Kennzeichen der Wahrheit finden;
\item sie sollen ferner bestrebt seyn, ihre besseren Religionsbegriffe auch \RWbet{unter andere Menschen immer weiter auszubreiten}, \zB\  dadurch, daß sie zu diesem Geschäfte einige Prediger (Missionäre) aussenden, \udgl ;
\item sie sollen einander wechselseitig ermuntern und anhalten, \RWbet{ein der erkannten Wahrheit gemäßes Leben} zu führen; sollen also \zB\  diejenigen, welche durch ihren Wandel der Lehre widersprechen, dafür zurechtweisen, und wenn es sich thun läßt, selbst bestrafen, \usw\;
\item sie sollen durch einen zweckmäßig eingerichteten \RWbet{öffentlichen Gottesdienst} für ihre eigene sowohl als für die Erbauung Anderer sorgen;~\RWSeitenw{364}
\item sie sollen einander auch \RWbet{selbst in allerlei leiblichen Nöthen} nach Kräften unterstützen, \usw ;
\end{aufzb}
\item Glieder der Kirche sind aber zuerst und vornehmlich alle diejenigen Menschen, \RWbet{die von der Wahrheit des katholischen Lehrbegriffes überzeugt, und wirklich gesonnen sind, zu den Zwecken, die der Herr Jesus nach Angabe eben dieses Lehrbegriffes vorhat, Alles, was ihre Kräfte vermögen, beizutragen}. Glieder sind noch ferner alle diejenigen, die \RWbet{wenigstens etwas für diese Zwecke zu thun bereitwillig sind.} Wir dürfen endlich auch alle Jene der Kirche beizählen, \RWbet{durch deren Beizählung zu ihr ein überwiegender Vortheil erreicht wird}. So insonderheit Kinder, wenn ihre Eltern selbst Glieder der Kirche sind. Wir dürfen im Gegentheile alle diejenigen von der Kirche ausschließen, deren Ausschließung einen entscheidenden Vortheil gewähret. So ist es namentlich
\begin{aufzb}
\item mit denjenigen, die einer irrigen religiösen Meinung nicht nur für ihre eigene Person hartnäckig zugethan sind, sondern den Irrthum auch unter Andern auszubreiten suchen (Ketzern); ingleichen
\item mit Allen, die durch ihr lasterhaftes Leben den übrigen Gliedern nur zum Aergernisse und zur Schande gereichen.
\end{aufzb}
\item Es umfasset also die Kirche nicht nur \RWbet{jetzt lebende}, sondern auch \RWbet{vorlängst schon verstorbene} Menschen. Der Theil derselben, der aus noch lebenden Menschen bestehet, heißt die \RWbet{sichtbare} oder auf Erden befindliche oder noch \RWbet{streitende}; der Theil dagegen, der die bereits Verstorbenen umfaßt, die \RWbet{unsichtbare} Kirche. Diese zerfällt noch weiter in die \RWbet{siegreiche,} darin sich Jene befinden, die bereits im Genusse der ewigen Seligkeit sind, um in die noch \RWbet{leidende} Kirche, welche diejenigen begreift, die erst noch durch Leiden geläutert werden müssen.
\item Der sichtbaren Kirche können wir bald nur innerlich \dh\  nur der Gesinnung nach; bald nur äußerlich, \dh\  nur durch die Annahme gewisser äußerer Zeichen, die dafür gelten, daß man ein Mitglied der Kirche sey; bald auch auf beiderlei Art angehören.~\RWSeitenw{365}
\item Wer die \RWbet{echt gläubige Gesinnung} hat, \dh\  bereit ist, Alles zu glauben, was Gott geoffenbaret hat, oder was sich ihm überhaupt nur als Wahrheit darstellt; wer ferner auch bestrebt ist, nach der erkannten Wahrheit sein Leben einzurichten; der gehört der Kirche bloß hiedurch schon \RWbet{innerlich} an; gesetzt auch, daß ihn irgend ein ungerechter Spruch irdischer Richter der äußeren Zeichen beraubt, oder daß er dergleichen Zeichen der Aufnahme noch nie empfangen, oder die Lehren der katholischen Religion nicht einmal noch kennen gelernt hätte.
\item Wer immer auf eine gültige Weise \RWbet{getaufet}, und nicht wieder hintenher aus der Gemeinschaft ausgeschlossen wurde, der gehört der Kirche wenigstens \RWbet{äußerlich} an, gesetzt auch, daß er in seinem Innern bisher noch nicht gesonnen wäre, die Zwecke Jesu zu befördern, wie etwa ein Kind, dem man die Taufe ertheilt hat; oder daß er sogar bestrebt wäre, diesem Zwecke heimlich entgegenzuwirken, wie etwa der Kaiser Julian, bevor er noch öffentlich vom Christenthume abfiel.
\item Auch schon \RWbet{vor Christo gab es eine Art Kirche auf Erden}, in sofern wenigstens, als wir uns vorstellen können, daß es vom Anbeginn des menschlichen Geschlechtes eine gewisse Gesellschaft von Menschen gegeben habe, die den alleinigen und wahren Gott erkannte, und auf den verheißenen Erlöser hoffte. Auch diese Gesellschaft war von dem Sohne Gottes gestiftet, und arbeitete dem Zwecke Jesu vor. Zu dieser Kirche gehörten, wenigstens äußerlich, alle diejenigen, die, wenn sie vom Volke Israel waren, sich an das mosaische Gesetz hielten, ingleichen diejenigen, die aus dem Heidenthume kommend, die bessern Religionsbegriffe der Juden annahmen, und sich als Verehrer Jehova's betrugen. Zum Unterschiede von dieser vorchristlichen Kirche kann man die von Christo selbst gestiftete die \RWbet{christliche} nennen.
\item Zwar gibt es noch einige andere Religionsgesellschaften, die sich auch rühmen, von Christo gestiftet zu seyn, und sich daher gleichfalls christliche Kirchen nennen; allein sie haben den Fehler, daß sie \RWbet{den Zweck Jesu nicht vollständig} befördern; dieß thut bloß diejenige, die diese Zwecke~\RWSeitenw{366}\ so auffaßt, wie sie in dem katholischen Lehrbegriffe ausgelegt werden. Nur diese ist also die \RWbet{einzig wahre Kirche.}
\item Wie Christus ihr Stifter, so sind die heil.\ Apostel ihre ersten Verbreiter gewesen; und sie kann sich rühmen, \RWbet{daß sie noch gegenwärtig in dem Geiste dieser Apostel fortwirke}, viel besser, als einige andere Gemeinden, die sich wohl auch rühmen oder einst gerühmt haben, von den Aposteln gegründet zu seyn. Sie darf sich deßhalb zu ehrendem Andenken an die Männer, denen sie ihre Verbreitung verdankt, und deren Geist bisher noch immer fortlebt, eine \RWbet{apostolische} Gesellschaft nennen.
\item Sie hat sich ferner auch die \RWbet{katholische}, \dh\  die allgemeine Religionsgesellschaft genannt, nicht sowohl wegen der großen Ausbreitung, die sie bisher schon \RWbet{hat}; sondern vielmehr wegen derjenigen, die sie zu haben \RWbet{verdient}, auch einst noch zu erlangen hoffet. Denn alle übrigen Religionsgesellschaften sollten eigentlich aufgelöst werden, und ihre Glieder sich in diese aufnehmen lassen; und über kurz oder lang wird dieß auch wirklich geschehen. Von diesem Beinamen: katholisch, der eigentlich nur ihrer Gesellschaft gebühret, nahmen die Glieder selbst den Namen: \RWbet{Katholiken} an.
\item Diese katholische Kirche verdienet auch den Lobspruch der \RWbet{einigen}, weil sie, im Gegensatz mit den andern christlichen Kirchen, die Einigkeit, \dh\  die Uebereinstimmung in den Gesinnungen, in den Gebräuchen und in den Bestrebungen ihrer Glieder in einem solchen Maße befördert, als es nur möglich ist, ohne der Eigenthümlichkeit eines Jeden Abbruch zu thun.
\item Nicht minder verdient sie es, eine \RWbet{heilige} Kirche zu heißen, nicht zwar als ob ihre Glieder selbst alle heilig wären, wohl aber, weil sie Anweisungen, Ermunterungsgründe und Hülfsmittel zur Heiligkeit anbeut, wie keine andere Religionsgesellschaft in einer gleichen Vortrefflichkeit aufweisen kann.
\item Sie kann sich auch den Beinamen einer \RWbet{unfehlbaren} beilegen, wenn er nicht so verstanden wird, als ob~\RWSeitenw{367}\ die Verrichtungen aller ihrer einzelnen Glieder, ja auch nur die ihrer Vorsteher, unfehlbar gut wären; sondern nur so, daß diese Gesellschaft \RWbet{in ihrem religiösen Lehrbegriffe}, \dh\  in jenen Lehren, die von allen, oder doch fast allen ihren Gliedern, für welche sie eine religiöse Wichtigkeit haben, einstimmig vorgetragen werden, nie fehle und nie fehlen könne.
\item Sie kann endlich auch in aller Wahrheit behaupten, daß man ihr angehören, \RWbet{wenigstens innerlich angehören müsse}, wenn man der ewigen Seligkeit theilhaftig werden wolle, daß \RWbet{außer ihr kein Heil} sey.
\item In dieser sichtbaren Kirche soll es fortwährend \RWbet{einen eigenen Stand} geben, dem es als seine lebenslängliche Beschäftigung obliege, sich mit den Wahrheiten der Religion zuvörderst selbst auf das Vollkommenste vertraut zu machen, dann aber auch andere Mitglieder der Gesellschaft darin zu unterrichten, ja, wenn es möglich ist, diese Wahrheiten auch unter der gesammten übrigen Menschheit allmählich auszubreiten.
\item Dieser Stand, den wir gewöhnlich den \RWbet{geistlichen} nennen, soll von den übrigen Ständen, die man mit dem gemeinsamen Namen der \RWbet{Laien} (\RWgriech{la`os}, \RWlat{populus}) umfaßt, sehr scharf gesondert werden. Den Geringsten aus diesem Stande soll in gewisser Hinsicht ein Vorrang selbst vor dem Vornehmsten der Laien eingeräumt werden.
\item Nebst dem Geschäfte des Unterrichtes soll diesem Stande auch die Leitung des Gottesdienstes und die \RWbet{Ausspendung der Heiligungsmittel} (etwa mit Ausnahme jenes der Taufe im Nothfalle und etwa des Heiligungsmittels der Ehe) ausschließlich anvertraut seyn. Und eben dieß Letztere, diese Fähigkeit der Ausspendung so erhabener Heiligungsmittel, besonders des heil.\ Abendmahles, sollen die Mitglieder dieses Standes als den wahren Grund ihres Vorranges vor allen übrigen betrachten.
\item Das Ansehen, das die bisher genannten Rechte und Obliegenheiten diesem Stande geben, und die Macht, die ihm die weltlichen Obrigkeiten freiwillig einräumen werden, soll er benützen, um \RWbet{allerlei heilsame Anordnungen}~\RWSeitenw{368} zu ertheilen über jeden Gegenstand, worüber zu verfügen die weltliche Obrigkeit ihm gestattet.
\item Die Mitglieder dieses Standes sollen der ganzen übrigen Christenheit zu einem \RWbet{Muster der Nachahmung} und zur Ermunterung in ihrem Streben nach sittlicher Vollkommenheit dienen; sie sollen sich also auf einer Stufe der Vollkommenheit befinden, welche entschieden höher, als die der Uebrigen ist; sie sind auch mehr noch, als jeder andere Christ, verpflichtet, sich nach dem Beispiele Jesu zu bilden.
\item Bevor man noch Jemanden in diesen Stand aufnimmt, soll man sich erst so gewiß, als es uns Menschen nur möglich ist, überzeuget haben, \RWbet{daß er schon gegenwärtig auf einer Stufe der Vollkommenheit stehe, zu der nur Wenige sich erhoben haben}.
\item Wenn es die weltliche Obrigkeit erlaubt, soll die \RWbet{Gemeine das Recht der Wahl} ausüben. Immer nur öffentlich und nach vorhergegangener Aufforderung an jedes einzelne Mitglied soll die Aufnahme in diesen Stand geschehen, beim Gottesdienste, unter Gebet und mit Beobachtung gewisser sinnbildlicher Gebräuche, die auf die wichtigsten Verrichtungen und Pflichten des Aufzunehmenden hindeuten.
\item Gott selbst wird diese \RWbet{Handlung der Weihe mit übernatürlichen Gnaden} segnen, und Jedem, der sie empfängt, nicht nur die Fähigkeit zur Ausspendung jener Heiligungsmittel verleihen, die man hier seiner Verwaltung anvertraut; sondern auch, wenn er dieß Heiligungsmittel auf eine würdige Art empfängt, die Kraft ihm schenken, alle die schweren und wichtigen Pflichten seines Standes vollkommen zu erfüllen.
\item Es kann und soll aber in diesem Stande Beides, \RWbet{verschiedene Abstufungen} in der einem jeden zustehenden \RWbet{Fähigkeit zur Ausspendung bestimmter Heiligungsmittel} sowohl als auch in der \RWbet{äußeren ihm wirklich eingeräumten Amtsgewalt} geben.
\item In Hinsicht des Ersteren, oder in Hinsicht auf die Fähigkeit zur Ausspendung gewisser Heiligungsmittel soll es folgende Abstufungen (\RWlat{ordines}) geben:~\RWSeitenw{369}
\begin{aufzb}
\item \RWbet{Bischöfe} (\RWgriech{>ep'iskopoi}, Aufseher), \dh\  Personen, denen die Fähigkeit der Ausspendung aller Heiligungsmittel ertheilt ist.
\item \RWbet{Priester} (\RWgriech{presb'uteroi}, Aelteste), die nur fünf Heiligungsmittel (nämlich mit Ausnahme jener der Firmung und der Weihe) zu spenden fähig sind.
\item \RWbet{Diakonen} (\RWgriech{di'akonoi}, Diener, Gehülfen), die nur die Macht haben, das Heiligungsmittel der \RWbet{Taufe} zu spenden, und bei Verwaltung der übrigen Heiligungsmittel dem Bischofe oder dem Priester behülflich zu seyn, \zB\  das heil.\ Abendmahl den Gläubigen darzureichen \udgl\ 
\item Endlich auch \RWbet{Subdiakonen} und andere von noch minderer Weihe (\RWlat{ordines minores}), \zB\  Akolythen, Lektoren \udgl , deren Weihe zu gewissen untergeordneten Verrichtungen beim öffentlichen Gottesdienste und bei der Ausspendung der Heiligungsmittel befähiget.
\end{aufzb}
\item Niemand soll zu einer höheren Stufe der Weihe zugelassen werden, ohne erst alle untergeordneten empfangen, und sich durch eine löbliche Verwaltung gewisser für diese passenden Aemter der Erhaltung einer höheren Weihe würdig bewiesen zu haben.
\item Jede Weihe, die einmal gültig ertheilt worden ist, soll nicht mehr wiederholt werden, indem auch sie der Seele ein Merkmal, das ewig unauslöschbar ist, eindrückt.
\item In Betreff der Aemter soll es in der katholischen Kirche:
\begin{aufzb}
\item \RWbet{Einen Primas}, \di\ Vorsteher der ganzen Christenheit geben, der in der religiösen Gesellschaft, die Jesus Christus auf Erden zu stiften erschien, und die er als ihr unsichtbares Oberhaupt noch fortwährend leitet, das \RWbet{sichtbare Oberhaupt}, der Mittelpunct der Vereinigung und somit gleichsam sein Stellvertreter seyn soll. Als diesen Primas erkennt die Kirche bis jetzt den \RWbet{Bischof von Rom.}
\item Jedem der \RWbet{übrigen Bischöfe} soll das Aufseher- oder Seelsorgeramt über einen bestimmten nicht allzu ausgedehnten Theil der katholischen Christenheit angewiesen seyn.~\RWSeitenw{370}
\item Kann dieser Bischof nicht alle geistlichen Bedürfnisse seiner Gemeine allein befriedigen: so soll er untergeordnete Seelsorger oder \RWbet{Pfarrer} (\RWgriech[paroiko`us]{paro'ikous}) zur Aufsicht über einzelne kleinere Theile seiner Gemeine aufstellen, denen er auch noch Gesellschafter (Kapläne), jene sowohl als diese aus dem Stande der Priester, beifügen kann.
\item Eben so mag er auch \RWbet{Diakonen} oder andere geistliche Personen zu Aemtern anstellen, zu denen sie sich kraft ihrer Weihe schicken.
\end{aufzb}
\item \RWbet{Der Primas der Kirche hat als solcher die Pflicht}:
\begin{aufzb}
\item Verordnungen zu ertheilen, die er für die gesammte Christenheit heilsam erachtet, sofern es die weltlichen Mächte gestatten. Er hat insonderheit die Pflicht:
\item Einzelne Länder und Städte mit tauglichen Bischöfen zu besetzen, oder, falls sich die weltliche Obrigkeit das Recht der Erwählung selbst vorbehalten hätte, die Gewählten wenigstens zu bestätigen. Ihm liegt es vornehmlich ob,
\item für die Verbreitung des Christenthums auch unter andern Völkern durch Aussendung schicklicher Missionäre zu sorgen;
\item Mißbräuche, die hie und da eingerissen sind, bei Zeiten abzustellen;
\item entstandene Streitigkeiten zu schlichten;
\item wenn es zu diesem oder zu irgend einigen anderen gemeinnützigen Zwecken nothwendig ist, und wenn die weltlichen Mächte dazu ihre Einwilligung geben, eine allgemeine Versammlung aller Bischöfe an einem schicklichen Orte in Vorschlag zu bringen und wirklich auszuschreiben; \usw\
\end{aufzb}
\item \RWbet{Die Bischöfe als die obersten Vorsteher in ihren einzelnen, ihnen vom Primas anvertrauten Gemeinen sollen}
\begin{aufzb}
\item durch ihren ausgezeichneten frommen Lebenswandel der ganzen Gemeine zu einem Vorbilde dienen,
\item die religiösen Bedürfnisse derselben theils in eigener Person, theils durch die Anstellung Anderer (nämlich der~\RWSeitenw{371}\ Seelsorger mit ihren Kaplänen \usw ) zu befriedigen suchen; und hiebei nicht vergessen, daß jene Geistlichen, welche sie angestellt haben, diese Aemter nur an ihrer Statt (\RWlat{jure delegato}) verwalten; daher denn alle Fehler, die diese begehen, ihnen zur schärfsten Verantwortung gereichen, wenn sie bei ihrer Auswahl und Anstellung nicht mit der möglichsten Sorgfalt verfuhren.
\item Sie sollen, wenn sie die nöthige Geschicklichkeit dazu haben, auch selbst durch Schriften erbauen, \zB\  durch Hirtenbriefe, Sendschreiben \udgl\  an ihre eigene, oder auch an benachbarte Gemeinen.
\item Sie sollen in allen denjenigen Stücken, darin es die weltliche Macht gestattet, zweckmäßige Anordnungen in ihrer eigenen Gemeine treffen.
\item Wenn einst der Primas der Kirche seiner Schuldigkeit offenbar nicht entspräche: dann sollen die Bischöfe in allgemeinen Versammlungen an seiner Statt handeln, ihn nöthigenfalls absetzen, einen Andern erwählen, \usw\ \usw\
\end{aufzb}
\item \RWbet{Aehnliche Pflichten haben nun auch die Seelsorger}, mit jenen Einschränkungen, die sich aus der Natur ihres Amtes ergeben.
\item Wer irgend \RWbet{ein geistliches Amt aus bloßem Eigennutze} sucht, oder aus Eigennutz verleihet, begehet eines der schwersten Verbrechen, das man vom Simon dem Magier (\RWbibel{Apg}{Apostelg.}{8}{18--24}) die \RWbet{Simonie} nennt; wo es erwiesen ist, soll es durch Ausschließung aus der Gemeine (Excommunication) bestrafet und jede Anstellung von dieser Art soll an sich ungültig seyn.
\item In der lateinischen Kirche besteht für alle Geistlichen vom Subdiakon anzufangen, die Pflicht, ein \RWbet{ehelos Leben} zu führen.
\item Für eben diese Geistlichen, wie auch für alle jene, die eine geistliche Pfründe genießen, ob sie gleich selbst keine heilige Weihe empfangen haben, besteht die Pflicht, täglich gewisse, von der Kirche eigens bestimmte Lesungen und Ge\RWSeitenw{372}bete zu verrichten (\RWlat{horas canonicas recitandi}, oder das sogenannte \RWbet{Brevier zu beten}). Jene Lesungen aber enthalten allerlei Auszüge aus der heil.\ Schrift, dazu gehörige Auslegungen der Kirchenväter, Lebensbeschreibungen der Heiligen \usw\
\item Kein Geistlicher darf die \RWbet{Einkünfte}, die ihm sein geistliches Amt oder auch nur eine einzelne geistliche Verrichtung einbringt, als sein durch diese Dienste erworbenes Eigenthum betrachten; er hat sie vielmehr \RWbet{als ein der Kirche gehöriges Gut} (\RWlat{patrimonium Christi}), \dh\  als ein Gut anzusehen, das bloß zu wohlthätigen Zwecken (\RWlat{ad pios usus}), \zB\  namentlich zur Unterstützung der Armen, oder zum Besten der Religion verwendet werden darf. Für sich darf er nur dann, wenn er sich seinen Lebensunterhalt auf keine andere Weise verschaffen kann, so viel nehmen, als er zu diesem Zwecke nothwendig braucht. Wer mehr genommen, der ist zur Rückstellung verpflichtet.
\end{aufza}

\RWpar{296}{Historischer Beweis dieser Lehre}
\begin{aufza}
\item Daß Jesus in der That gewollt, daß seine Anhänger einen gewissen religiösen Verein mit einander bilden, dem er als Oberhaupt vorzustehen versprochen, erweisen mehrere Stellen der heil.\ Schrift; \zB\  \RWbibel{Mt}{Matth.}{16}{18}: \erganf{Du bist Petrus, und auf diesem Felsen will ich meine Gemeine (\RWgriech{>ekklhs'ian}) gründen} \usw\ \RWbibel{Mt}{Matth.}{18}{17}: \erganf{Achtet er auch nicht den Ausspruch der Gemeine (\RWgriech{>ekklhs'ia}) so betrachte ihn gleich einem Heiden und Zöllner.} \RWbibel{Lk}{Luk.}{12}{32}: \erganf{Sey ohne Furcht, du kleine Herde! (\RWgriech{po'imnion}); denn eueres Vaters Rathschluß ist es, gerade euch die Herrschaft (über den Erdkreis) zu geben.} \RWbibel{Joh}{Joh.}{15}{5}: \erganf{Ich bin der Weinstock, ihr seyd die Schossen \usw\ Weil ihr es nicht mit der Welt haltet, und ich euch von der Welt ausgesondert habe: so hasset euch die Welt.} \RWbibel{Joh}{}{17}{20}: \erganf{Doch nicht für sie allein (für die Apostel) bitte ich, sondern auch für die, welche durch ihre Lehre an mich glauben werden, damit alle einig seyen, wie du, o Vater! mit mir, und ich mit dir einig bin.} \RWbibel{Mt}{Matth.}{28}{20}: \erganf{Ich bleibe bei euch durch alle Tage bis an der Zeiten Ende.} \uma~\RWSeitenw{373}\par
Die Artikel 2--24 übergehe ich, weil sie theils schon in den Vorhergehenden erwiesen, theils für sich selbst bekannt sind, oder doch kaum bestritten werden dürften.\par
\item[25.] Daß die heil.\ Handlung der Weihe eine gewisse \RWbet{übernatürliche Stärkung} verleihe, schließen die Katholiken unter Anderem auch aus den Worten Pauli (\RWbibel{1\,Tim}{1\,Tim.}{4}{14}): \erganf{Vernachlässige nicht die Gnade (oder Gabe, \RWgriech{q'arisma}), die dir durch die Auflegung der Hände des Priesterthums mitgetheilt worden ist.} Deßgleichen \Ahat{\RWbibel{2\,Tim}{2\,Tim.}{1}{6}}{2,16}
\item[26.] Daß ein \RWbet{Unterschied zwischen Bischöfen und Priestern} selbst schon in dem apostolischen Zeitalter gemacht worden sey, obgleich man den Namen da noch zuweilen verwechselte, beweiset \zB\  gleich die Stelle \RWbibel{Apg}{Apostelg.}{15}{22}: \erganf{Hier fanden die Apostel, die Priester, und die ganze Gemeine für gut}, \usw\ Man unterschied also die Apostel, die sich bekanntlich als Bischöfe ansahen, von den Priestern, und beide noch von der übrigen Gemeine. So schreibt auch Paulus an Titus, den er als Bischof zu Kreta angestellt hatte (\RWbibel{Tit}{}{1}{5}): \erganf{Ich ließ dich deßwegen in Kreta zurück, damit du das Fehlende in Ordnung brächtest, und in jeder Stadt Priester anstelltest, wie ich dir aufgetragen habe}; und \RWbibel{1\,Tim}{1\,Tim.}{5}{19}\ erinnert eben dieser Apostel den Timotheus, Bischof von Ephesus, er möge gegen einen Priester keine Klage annehmen, außer vor zwei oder drei Zeugen. Diese Stellen beweisen deutlich, daß es schon zu den Zeiten der Apostel und durch ihre eigene Anordnung geistliche Vorsteher in der Kirche gegeben habe, denen eine gewisse Amtsgewalt über andere geistliche Vorsteher, die Priester genannt wurden, anvertraut worden sey. Es muß uns erlaubt seyn, die ersteren \RWbet{Bischöfe} zu nennen. \RWbibel{Apg}{Apostelg.}{6}{1\,ff}\ wird die Einsetzung der \RWbet{Diakonen} erzählt, und aus \RWbibel{Apg}{}{8}{14\,ff} erhellet, daß nicht ein Diakon, wohl auch kein bloßer Priester, sondern nur ein Apostel (in der Folge also ein Bischof) ermächtiget gewesen, durch Auflegung der Hände und Gebete den einmal schon Getauften die sogenannten Gaben des heil.\ Geistes mitzutheilen, \dh\  sie zu firmen. 
\item[30.] Daß schon \RWbet{Jesus Christus das Primat eingesetzt} habe, erhellet aus mehreren Stellen der heil.\ Schrift~\RWSeitenw{374}\ sehr deutlich. Zuvörderst aus den schon oft angeführten Worten des Herrn zu Petrus, \RWbibel{Mt}{Matth.}{16}{18}; indem es vergeblich ist, unter dem \RWbet{Felsen}, auf den hier Jesus seine Kirche zu gründen verspricht, Jemanden Andern, als Simon, den Felsenmann (denn so ohngefähr ließe sich das hebräische Kephas, oder das griechische Petrus übersetzen) -- verstehen zu wollen. War aber hier Petrus gemeint: so ist auch offenbar, daß ihm ein Vorzug vor allen übrigen Aposteln eingeräumt worden sey. Einen solchen Vorzug ertheilte diesem Apostel auch der Auftrag Jesu \RWbibel{Joh}{Joh.}{21}{15\,ff}\ sich als den Hirten der Herde zu beweisen. Denn ohne uns hier in eine gekünstelte Unterscheidung zwischen den Schafen (\RWgriech{pr'obata}) und Lämmern (\RWgriech{>arn'ia}) einzulassen, ist so viel offenbar, daß der an Petrus ergangene Auftrag ein ganz vorzüglicher, nicht alle übrigen Apostel gleicher Weise betreffender Auftrag seyn konnte, weil es sonst ungereimt gewesen wäre, als Bedingung dazu eine Liebe von Petrus zu fordern, die stärker als jene der übrigen sey. Zum Hirten also auch über diese Anderen, zu einem Aufseher über die ganze Gemeine wurde hier Petrus erhoben (vgl.\ auch \RWbibel{Lk}{Luk.}{22}{32}). Uebrigens ist es auch unverkennbar, daß Petrus schon bei den Lebzeiten Jesu, um so mehr aber nach seinem Tode sich als der Vornehmste unter den Brüdern betragen habe. Er ist es, der bald nach der Himmelfahrt Jesu den Vorschlag zur Erwählung eines neuen Apostels an die Stelle des unglücklichen Judas Ischkarioth thut (\RWbibel{Apg}{Apostelg.}{1}{15}); der am ersten Pfingstsonntage die Predigt des Evangeliums eröffnet (\RWbibel{Apg}{}{2}{14\,ff}); der die Heuchelei des Ananias und der Saphira, welche die Apostel des Herrn zu hintergehen vermeinten, bestrafet (\RWbibel{Apg}{}{5}{3\,ff}); der vor dem hohen Rathe erklärt, daß die Apostel dem Verbote desselben, betreffend die Predigt des Evangeliums, nicht gehorchen könnten (\RWbibel{Apg}{}{5}{29\,ff}) \usw\ \usw\ Zwar hat man von Seite der Protestanten eingewendet, daß jener Vorrang, den Jesus dem Petro einräumte, und den dieser auch behauptete, ein bloßer Vorrang der Ehre (\RWlat{praerogativa honoris}) gewesen sey. Es scheinet aber, daß man bei dieser Behauptung entweder selbst nicht eingesehen, oder erwartet habe, es werde Andern nicht einfallen, daß ein Vorrang dieser Art eine sehr~\RWSeitenw{375}\ zwecklose Verfügung, eine bloße Nahrung der Eitelkeit, und somit der Person unseres Herrn durchaus unwürdig gewesen wäre. Hat Jesus Christus gewollt, daß Petrus einen gewissen Vorrang vor den Uebrigen habe: so hat er gewollt, daß Petrus durch diesen Vorrang in den Stand gesetzt werde, verschiedenes Gute zu bewirken, wozu er außerdem nicht ermächtigt gewesen wäre; und somit mußte es ein Vorrang seyn, der gewisse, wenn auch nicht mit Genauigkeit bestimmte, doch eigenthümliche Rechte und Obliegenheiten erzeugte. Andere sagen, daß der Vorrang Petri vor den übrigen Aposteln allerdings in gewissen, ihm eigenen Rechten und Verbindlichkeiten bestanden habe; aber dieser Vorrang sey ihm nur seiner persönlichen Eigenschaften wegen eingeräumt worden, und habe deßhalb nicht auf seine Nachfolger, und um so weniger auf alle römischen Bischöfe übertragen werden sollen, da es noch zweifelhaft sey, ob Petrus gerade zu Rom seinen Sitz aufgeschlagen habe. Hierauf erwiedere ich, wenn der Herr Jesus schon zu den Zeiten der Apostel, da die Anzahl der Christen \Ahat{noch}{nicht} so gering war, und da diejenigen, die an verschiedenen Orten lebten, auch bei dem besten Willen so wenig Verkehr mit einander zu unterhalten vermochten, gewollt hat, daß man den Petrus als das (mit einer noch so unbestimmten Gewalt ausgerüstete) Oberhaupt Aller, oder (wenn man den Ausdruck lieber will) als den Mittelpunct der Vereinigung Aller betrachte: so ist die Nothwendigkeit eines solchen Primas für alle folgenden Zeiten um desto größer, und das Gute, welches durch ihn gestiftet werden kann, um desto mannigfaltiger. Wie könnten wir also glauben, Jesus habe gewollt, daß der von ihm gestiftete Primat mit Petrus untergehe? Nur dieses aber, daß ein Primat, und zwar ein fortdauerndes, von Christo eingesetzt sey, ist eine Glaubenslehre; daß der Primas eben der Bischof von Rom seyn müsse, betrachten die Katholiken keineswegs als ein Dogma. Auch geben sie zu, daß es die persönlichen Eigenschaften Petri waren, welche den Herrn bestimmten, gerade ihm das Primat anzuvertrauen, sie wissen auch recht wohl, daß nicht ein Jeder, der dieses wichtige Amt nach ihm verwaltete, in einem gleichen Grade, wie Petrus, desselben würdig gewesen; aber nur däuchte es ihnen bis auf den heutigen Tag zuträglich,~\RWSeitenw{376}\ diese Würde dem Bischofe von Rom zu überlassen, und dieß zwar nicht eben, weil Rom der Sitz Petri gewesen, sondern weil mehrere Umstände sich für diesen Ort vereiniget hatten, und noch jetzt vereinigen. Rom war die Hauptstadt der Welt, ihr Bischof der mächtigste schon in den frühesten Zeiten, um so mehr jetzt, da er ein eigener Landesherr ist, \usw\ Daß übrigens der Bischof von Rom wirklich schon seit den ersten Jahrhunderten als Primas der Kirche angesehen worden sey, beweiset die Kirchengeschichte. Schon \RWbet{Clemens, der erste Nachfolger Petri}, wurde von den Korinthern angesprochen, um die bei ihnen entstandenen Streitigkeiten zu schlichten. Und \RWbet{Irenäus} im zweiten Jahrhunderte schreibt: \erganf{Es ist nothwendig, daß alle Gläubigen sich zu dieser (der römischen) Kirche wegen ihrer Vorzüge (\RWlat{propter potiorem principalitatem}) halten; denn hier ist die apostolische Tradition aufbewahrt worden.} \UmA\par
Fragen wir aber, welche Rechte und Obliegenheiten es wären, welche der Bischof von Rom kraft eines obersten Vorsteheramtes in der katholischen Kirche habe: so wird uns in verschiedenen Jahrhunderten und von verschiedenen Personen so Verschiedenes erwiedert, daß eben aus dem Mangel an Uebereinstimmung folgt, daß keine dieser näheren Bestimmungen, etwa mit Ausnahme derer, die wir Nr.\,31. angegeben haben, als eine eigentliche Glaubenslehre angesehen werden könne.
\end{aufza}

\RWpar{297}{Vernunftmäßigkeit und sittlicher Nutzen}
\begin{aufza}
\item Dadurch, daß alle diejenigen Menschen, deren religiöse Begriffe bis auf einen gewissen Grad einstimmig sind, in eine nähere Verbindung miteinander treten, kann, vornehmlich wenn diese Begriffe selbst vernünftig sind, gar manches Gute zu Stande gebracht werden. Und so wird es denn eigentlich jede bessere Religion ihren Bekennern zur Pflicht machen, in eine gewisse Verbindung zu treten. Da nun eine Gesellschaft, zu welcher beizutreten es unsere Pflicht schon bloß aus dem Grunde wäre, weil wir uns von der Wahrheit eines gewissen religiösen Lehrbegriffes überzeugt haben, eine \RWbet{Religionsgesellschaft} (oder im weiteren Sinne~\RWSeitenw{377}\ auch eine \RWbet{Kirche}) genannt wird: so muß es wohl jeder Stifter einer neuen Religion für dienlich erachten, auch eine eigene Religionsgesellschaft oder Kirche zu stiften. Dieß durfte um so weniger unser Herr Jesus unterlassen; denn hätte er seine Schüler nicht eigens dazu angehalten, daß sie in eine nähere Verbindung miteinander treten, in einer solchen auch für alle kommenden Zeiten verbleiben, und sich fortwährend nur als Glieder eines einzigen Ganzen betrachten: so wäre sehr zu besorgen gewesen, daß jene besseren Begriffe, die er uns mitgetheilt hat, bald wieder verloren gehen würden; oder es hätte wenigstens einer viel größeren Mitwirkung von Seite Gottes bedurft, um dieß zu verhindern, und einer noch größeren, wenn seine Anhänger, wie es sein Wille war, in der Erkenntniß heilsamer Wahrheiten sogar mit der Zeit fortschreiten sollten.
\item Wie hier der Zweck der Kirche bestimmt worden ist: so wäre eigentlich der Zweck einer jeden anderen Vereinigung mehrerer Kräfte, ja auch der Zweck jeder einzelnen Kraft zu bestimmen. Immer muß von dem Zwecke, der die Entstehung einer solchen Verbindung mehrerer, oder auch einer solchen einzelnen Kraft veranlaßt hat (von dem \RWbet{Entstehungszwecke}), der Zweck, den man sich bei dem \RWbet{Gebrauche} derselben vorsetzen soll, unterschieden, und dieser letztere immer dahin bestimmt werden, daß er sich nicht bloß auf Dasjenige beschränke, was man im Sinne gehabt, als man die Kräfte hergab; sondern daß er vielmehr den Inbegriff alles desjenigen Guten umfasse, was sich durch Kräfte dieser Art nur immer hervorbringen läßt; gleichviel, ob man im Anfange daran gedacht oder nicht, und selbst in dem Falle, wenn es sich zeigt, daß der Zweck, der die Veranlassung war, aus der man die Kräfte hergab, als ein unwürdiger, verlassen zu werden verdiene.
\item Der dritte Punct bedarf keiner weiteren Rechtfertigung.
\item Auch die Vernunftmäßigkeit des vierten Punctes ist außer Zweifel; denn daß die hier genannten Zwecke durch einen religiösen Verein mehr oder weniger befördert werden können, leuchtet von selbst ein.~\RWSeitenw{378}
\item Wenn es richtig ist, daß wir unter einer \RWbet{Gesellschaft} im eigentlichen Sinne nie etwas Anderes verstehen, als einen Inbegriff vernünftiger Wesen, deren jedes bereit ist, wenigstens etwas zu thun, weil es erkannte, daß es der Wille der Uebrigen, oder, was eben so viel ist, des Oberhauptes sey: so sind freilich Menschen, die nicht gesonnen sind, die Zwecke Jesu zu befördern, auch keine Glieder der Kirche im eigentlichen Sinne des Wortes; und im Gegentheil Alle, an denen sich dieses Merkmal findet, sind es. Dieß kann uns aber nicht hindern, der Ersteren Einige gleichwohl der Kirche beizuzählen, der Letzteren Einige aus ihr auszuschließen, sobald durch diese Beizählung oder Ausschließung irgend ein wesentlicher Vortheil erreicht werden kann. Nehmen wir doch selbst bei Gesellschaften, deren Zwecke von einer ungleich geringeren Wichtigkeit sind, nebst den wirklichen Gliedern derselben auch manche an, die es dem bloßen Namen nach sind, \RWbet{Titularmitglieder}. Daß es nun insbesondere vortheilhaft sey, Kinder, deren Eltern selbst Glieder der Kirche sind, der Kirche beizuzählen, und deßhalb taufen zu lassen, wurde schon gezeigt. Nicht minder einleuchtend ist der Nutzen, der durch Ausschließung der hier angegebenen zwei Arten von Menschen entsteht.
\item[6.--9.]\setcounter{enumi}{9} Diese Artikel bedürfen abermals keiner Rechtfertigung.
\item Eine gewiß sehr schöne Lehre des Katholicismus ist es, daß es eine Kirche auch schon vor Christo gegeben. Wer könnte es läugnen, daß die frommen Männer des alten Bundes den Zwecken Christi vielfältig vorgearbeitet haben? Gesetzt nun auch, daß sie Manches davon nicht eben mit dem deutlichsten Bewußtseyn von den wohlthätigen Folgen, die es durch die Benützung Jesu erhalten würde, gethan: so wissen wir doch, daß Jedem, der etwas Gutes in guter Absicht verrichtet, auch sebst diejenigen ersprießlichen Folgen desselben, die er nicht vorhergesehen hatte, zum Verdienst angerechnet werden. Was aber den Umstand belangt, ob das Volk Israel, und schon lange vorher die sogenannten Erzväter bis zu dem ersten Stammpaare hin, alle auf den verheißenen Erlöser gehofft, und welche Begriffe sie von ihm gehabt haben mögen:~\RWSeitenw{379}\ so leuchtet ein, daß diese Frage eigentlich keine zur Religion gehörige Sache betreffe; -- daß aber die Vorstellung, welche die große Menge der Katholiken annimmt, die erbaulichste sey, und also wenigstens als bildliche Vorstellung unstreitig beibehalten zu werden verdiene.
\item Was die katholische Kirche hier von sich sagt, muß jede andere Religionsgesellschaft, will sie sich selbst nicht widersprechen, gleicher Weise behaupten; nur darf freilich keine verlangen, daß man ohne Beweis ihr glaube. Das thut auch die katholische Kirche nicht; sondern die Richtigkeit dieser Behauptung erweiset sich aus der Vergleichung der Lehre des Katholicismus mit jener des Protestantismus und der übrigen \RWbet{christlich} sich nennenden Religionsgesellschaften. Es zeigt sich, daß keine der Tugend und Glückseligkeit der Menschen so zuträglich sey, wie die erste; und da es im Voraus gewiß ist, daß die Beförderung der Tugend und Glückseligkeit der Zweck Jesu Christi gewesen: so ist schon entschieden, daß keine Religionsgesellschaft den Absichten Jesu so vollständig, als die der Katholiken entspreche.
\item Hiemit ist auch schon die Benennung: \RWbet{apostolisch} gerechtfertiget.
\item Und wenn es wahr ist, daß die Religionsgesellschaft der Katholiken die Tugend und Glückseligkeit der Menschen am Meisten befördert: so verdient sie auch einst \RWbet{allgemein herrschend} zu werden.
\item Daß die katholische Kirche die \RWbet{Einigkeit} durch Uebereinstimung in den Gesinnungen, Gebräuchen und Bestrebungen ihrer Glieder mehr als eine jede andere Religionsgesellschaft befördere, ist gleichfalls außer Zweifel. Die Verschiedenheit, ja man kann sagen, die Verwirrung der Begriffe, die sich in unseren Tagen, namentlich des protestantischen \RWbet{Deutschlands} bemächtiget hat, eine Folge des einem Jeden eingeräumten Rechtes, sich seine religiösen Ansichten selbst zu bilden, -- ist sichtbar höher gestiegen, als es je zuträglich seyn kann. Selbst protestantische Gelehrte haben dieß vielfältig eingestanden.
\item Der Beiname: \RWbet{heilig}, in der erklärten Bedeutung, ist durch das Gesagte bereits gerechtfertiget.~\RWSeitenw{380}
\item Die Vernunftmäßigkeit und der sittliche Nutzen der Lehre von der Unfehlbarkeit der Kirche, ingleichen,
\item daß außer ihr kein Heil sey, wurde schon besprochen.
\item Daß man in der katholischen Kirche fortwährend einen eigenen Stand unterhalte, dem die Auffindung und Verbreitung religiöser Wahrheiten zu seiner lebenslänglichen Beschäftigung obliegt, ist gewiß eine sehr heilsame Verfügung; denn nun läßt sich erwarten, daß es der Menschen, welche in religiösen Wahrheiten unwissend sind, unter uns Katholiken vergleichungsweise doch immer weniger geben werde, als in irgend einer anderen Religionsgesellschaft, die keinen eigenen Stand der Religionslehrer hat.
\item Die Forderung, daß alle katholischen Christen, so groß auch übrigens ihr bürgerlicher Rang seyn möchte, jedem, auch dem Geringsten, der \RWbet{Geistlichen} einen gewissen Vorzug vor sich einräumen sollen, ist die beste Vorbereitung, die Menschen allmählich dahin zu bringen, daß sie die Nichtigkeit aller \RWbet{Rangunterschiede} erkennen.
\item \begin{aufzb} \item Nichts könnte zweckmäßiger seyn, als die Leitung des \RWbet{öffentlichen Gottesdienstes} eben denjenigen Personen anzuvertrauen, die schon durch ihren Stand verpflichtet sind, sich die gründlichsten Kenntnisse von der Religion zu verschaffen. Man höre, wie Cicero (\RWlat{de divin.\ l.\,2.})\RWlit[: \eanf{\RWlat{de officio num quis haruspicem consulit quem ad modum sit cum parentibus, cum fratribus, cum amicis vivendum, quem ad modum utendum pecunia, quem ad modum honore, quem ad modum imperio? Ad sapientes haec, non ad divinos referri solent.}}]{}{Cicero3b} über diejenigen Personen sich äußert, welchen zu seiner Zeit (nämlich im Heidenthume) die Leitung des öffentlichen Gottesdienstes anvertraut war: \erganf{Was für eine Beziehung ist zwischen dem Dienste der Götter und unserer Pflichten? Hat man je einen Haruspex über das Betragen, welches der Mensch gegen Eltern, Brüder und Freunde zu beobachten hat, über den Gebrauch, den man von seinen Gütern, seinen Ehrenstellen, seinem Ansehen machen soll, zu Rathe gezogen? Diese Sorge gehet die \RWbet{Weltweisen} an, nicht die Besorger des Gottesdienstes.} So kann man von den Besorgern des \RWbet{christlichen} Gottesdienstes nicht sprechen.
\item Sehr weise ist es auch, den wahren Grund des Vorzuges, welchen die Geistlichen vor den Laien haben sollen, in ihre \RWbet{Fähigkeit} zur Ausspendung der Heiligungsmittel zu setzen; denn so wird~\RWSeitenw{381}
\begin{aufzc}
\item dem Stolze vorgebeugt, der sehr leicht entstehen könnte, wenn sie den Grund dieses Vorzuges in ihren \RWbet{Einsichten,} in ihrer vorzugsweisen Tugend, oder in anderen, von ihrer freien Thätigkeit abhängigen Vorzügen, und nicht in einer so ganz ohne ihr Verdienst ihnen verliehenen \RWbet{Gnade} des Himmels suchen und finden dürften. Auch wird auf diese Art
\item \RWbet{der Werth und die Vortrefflichkeit dieser Heiligungsmittel} in ein noch helleres Licht gesetzt; und dabei steht gleichwohl bei Menschen, welche so unterrichtet sind, wie die \RWbet{Geistlichen} fast durchgängig, kaum zu befürchten, daß sie, mit dem bloßen Ausspenden der Sacramente sich zufrieden stellend, der \RWbet{Wichtigkeit ihrer übrigen Pflichten} und der \RWbet{Nothwendigkeit} einer mehr als gewöhnlichen Vollkommenheit vergessen werden; zumal da ihnen unaufhörlich eingeprägt wird, daß sie sich eben nur durch die Erfüllung dieser Pflichten der Ausspendung jener Heiligungsmittel erst \RWbet{würdig} machen können, und im widrigen Falle das Verbrechen eines Gottesraubes begehen.
\end{aufzc}
\end{aufzb}
\item Sehr vernünftig ist es, die \RWbet{Grenzen der geistlichen Macht} auf keine andere Art zu bestimmen, als durch die Grenzen der \RWbet{Möglichkeit} selbst; denn eine jede andere Bestimmung würde fehlerhaft seyn oder höchstens auf eine kurze Zeit nur taugen. So ist es \zB\  ganz falsch, was viele neuere Gelehrte gethan haben, die Grenzen der geistlichen Macht durch den \RWbet{Gegenstand} zu bestimmen, und \zB\  zu sagen, nur was die Tugend und das Heil der Seele, oder, nur was die \RWbet{Religion} betrifft, darüber hätte die \RWbet{geistliche Obrigkeit} zu entscheiden. Schon \RWbet{Mendelssohn} in seinem Jerusalem\RWlit{}{Mendelssohn1} zeigte, wie ungereimt dieses sey, weil ja die Tugend und das Heil der Seele von unserer \RWbet{irdischen} Glückseligkeit, von unseren äußeren Verhältnissen größtentheils abhängen; so, daß Demjenigen, dem man das Recht einräumt, alle für die Beförderung unserer Tugend und unseres Seelenheiles nöthigen Anstalten zu treffen, eben deßhalb auch das Recht eingeräumt werde, über alle Dinge, ohne Ausnahme, zu verfügen. Eben so fehlerhaft ist es, zu sagen: \RWbet{Was die}~\RWSeitenw{382}\ \RWbet{Religion betrifft,} man mag dieses Wort in seiner \RWbet{engeren} oder \RWbet{weiteren} Bedeutung nehmen. In beiden Bedeutungen versteht man unter Religion nur den Inbegriff gewisser \RWbet{Meinungen;} die geistliche Obrigkeit würde also nur zu bestimmen haben, was die Christen zu \RWbet{glauben}, nie aber, was sie zu \RWbet{thun} hätten; sie würde eigentlich gar keine Macht, Gesetze zu geben, besitzen; sie würde also \zB\ nicht einmal die Art und Weise der Gottesverehrung bestimmen dürfen; denn dieß sind Handlungen. Die gesunde Vernunft sagt dagegen, daß jeder Mensch des Guten so viel thun solle, als er nur immer, unter Anderem auch dadurch vermag, daß er es Anderen, welche ihm zu gehorchen bereit sind, befehle. Also versteht es sich von selbst, daß auch die geistliche Obrigkeit die Pflicht habe, über Alles Befehle zu ertheilen, worüber die Gläubigen ihr zu gehorchen bereitwillig sind, und woran die weltliche Obrigkeit sie nicht hindert. Da aber die Gesinnungen der weltlichen Obrigkeiten hier in verschiedenen Ländern und zu verschiedenen Zeiten gar sehr verschieden sind; da sie der geistlichen Macht bald einen größeren, bald einen geringeren Spielraum gestatten: so ließe sich eben deßhalb keine bestimmtere Regel, als diese allein ertheilen.
\item Daß jeder Geistliche verpflichtet sey, Andern zum Muster zu dienen, und also sich zu einer höheren Stufe sittlicher Vollkommenheit, als die Uebrigen, zu erheben; daß er auch stärker, als diese, verpflichtet sey, sich nach dem Beispiele Jesu zu bilden, das Alles würde, auch wenn es der katholische Lehrbegriff nicht ausdrücklich sagte, bloß aus dem Umstande folgen, weil jeder Geistliche
\begin{aufzb}
\item viel mehr Gelegenheit hat, an seiner sittlichen Vervollkommnung mit Glück zu arbeiten und besonders sich nach dem Beispiele Jesu zu bilden; denn ihm liegt die Beschäftigung mit religiösen den Menschen bessernden Wahrheiten als seine Lebensaufgabe ob; er kann und soll sich der wirksamsten Heiligungsmittel viel öfter, als die meisten übrigen Christen, bedienen; er hat beim Unterrichte Anderer und bei Ausspendung der Heiligungsmittel an sie eine beständige Aufforderung, sein Herz von jeder, auch der geringsten Schuld rein zu bewahren; er kann die~\RWSeitenw{383}\ Lebensgeschichte unseres Herrn Jesu mit aller Muße studiren, und sich die vollständigste Kenntniß von jener eigenthümlichen Art, wie Jesus dachte, empfand und handelte, verschaffen.
\item Sittliche Fehler und Schwachheiten, die sich an einem Geistlichen befinden, stiften insgemein ein viel größeres Aergerniß, als wenn sie an einem Andern sich befänden. Schon der Vorrang, den der katholische Lehrbegriff dem Geistlichen eingeräumt wissen will, macht, daß auf ihn die Augen Aller sich richten. Auch ist es ferner sehr natürlich, daß uns bei dem Manne, der uns so viel von unseren Pflichten vorspricht, die Frage einfällt, ob denn auch er die seinigen erfülle; und so geschieht es, daß wir seinen Lebenswandel schärfer in's Auge fassen, nicht selten sogar mit dem geheimen Wunsche, Fehler an ihm zu entdecken. Glaubt Jemand erst, Entdeckungen von dieser Art gemacht zu haben: wie beeilt er sich nicht gewöhnlich, sie Anderen mitzutheilen, und meistens noch mit Vergrößerungen, mit Darstellung des bloß Vermutheten als etwas, das völlig gewiß ist, \usw\ Und was für Folgerungen nun, die wenigstens die gemeine Menschenmenge aus solchen, gleichviel ob wahren oder nur angedichteten, Fehlern und Schwächen ihrer Geistlichkeit ziehet! Ein Theil derselben erlaubt sich sofort den Schluß, ein Geistlicher, der so lebt, könne unmöglich selbst glauben, was er Andern vorprediget, und also müßten wohl alle oder doch die meisten Lehren der Religion nichts weiter, als Priestertrug seyn! Andere bezweifeln nicht die Aufrichtigkeit des Priesters bei seinem Unterrichte; aber sie schließen aus seiner Nichtbefolgung dessen, was er doch selbst für eine Pflicht erklärt, daß es nicht einmal ihm, um wie viel weniger ihnen, möglich sey, Alles zu erfüllen, was die Religion von uns verlangt. Noch Andere endlich lassen auch dieses dahingestellt; halten sich aber an den Gedanken, daß wenn so Viele, mitunter selbst Geistliche, Gottes Gebote nicht halten, die Strafe der Uebertreter doch auf keinen Fall so hart seyn könne, weil ja sonst Niemand selig werden könnte und der Himmel leer bleiben würde. --~\RWSeitenw{384}
\end{aufzb}
\item Ist aber das Aergerniß, welches durch unsittliche Geistliche angerichtet wird, so groß: so ist es Pflicht für Alle, die bei der Aufnahme der Personen in diesen Stand etwas zu sagen haben, Niemand zuzulassen, von dem sie sich nicht überzeugten, daß er schon gegenwärtig auf einer Stufe sittlicher Vollkommenheit stehe, zu der nur Wenige sich erheben; denn wer sich nicht schon vor seiner Aufnahme in diesen Stand durch seine Sittlichkeit auszeichnet, der gibt auch keine hinreichend sichere Hoffnung, daß er es nachher thun werde. Ja, da die Pflichten des geistlichen Standes so schwer sind: so läßt sich nur von einem Manne, der eine mehr als gemeine sittliche Kraft an den Tag gelegt hat, erwarten, daß er sie alle standhaft erfüllen werde; von einem Schwächern dagegen steht zu befürchten, daß er, unfähig, Alles zu thun, was ihm obläge, durch das Bewußtseyn verletzter Pflichten in seinem sittlichen Gefühle immer mehr abgestumpft werden, und so nur immer tiefer herabsinken werde. Begeht man den Fehler einer zu großen Nachgiebigkeit bei der Aufnahme der Personen in diesen Stand öfters; glaubt man \zB\  hiezu bemüßiget zu seyn, weil es der Geistlichen noch nicht so viele gibt, als wohl zu wünschen wäre: so tritt, besonders wenn das Zeitalter überhaupt ein verdorbenes ist, die äußerst traurige Erfahrung ein, daß nun gerade der geistliche Stand, der Stand, der noch der unbescholtenste seyn sollte, die größte Entartung zur Schau trägt! --
\item Ließe sich auch an der Art, wie man die heilige Handlung der \RWbet{Weihe} in unseren Tagen verrichtet, Manches vervollkommnen: so ist doch gewiß, daß sie Erbauung bezwecke, und bei Gemüthern, die für sie empfänglich sind (und solche sollen ja wohl die jungen Geistlichen haben), auch zu bewirken vermöge.
\item Die übernatürliche Wirksamkeit des Sacramentes der Weihe wird Niemand anstößig finden, der die Erklärung dieser Wirksamkeit bei den schon früher abgehandelten Heiligungsmitteln begriffen hat.
\item Die Unterscheidung zwischen der \RWbet{inneren Fähigkeit} zur Ausspendung bestimmter Heiligungsmittel und zwischen der \RWbet{äußeren Amtsgewalt} ist eine sehr zweckmäßige~\RWSeitenw{385}\ Unterscheidung; denn erstlich erfordert es schon die gute Ordnung, daß einem Geistlichen in gewissen Fällen zwar wohl die innere Fähigkeit zur Ausspendung eines Heiligungsmittels, aber doch nicht die rechtmäßige Amtsgewalt dazu eingeräumt werde. So ist es namentlich, wenn sich der Geistliche in ein fremdes, der Obsorge eines Anderen zugetheiltes Gebiet begibt. Hier muß ihm die Ausspendung gewisser Heiligungsmittel untersagt seyn, ob man ihm gleich die Fähigkeit, sie gültig auszuspenden, nicht abläugnen kann, ja auch wohl Fälle eintreten können (Fälle der Nothwendigkeit), wo er dieß rechtmäßig thut. Auch dient es, dem Stolze Einzelner auf eine heilsame Art zu steuern, und zu gleicher Zeit das Ansehen der Heiligungsmittel noch immer höher zu heben, wenn der katholische Lehrbegriff verlangt, daß jeder Geistliche, der sich in Betreff seiner Fähigkeit zur Ausspendung gewisser Heiligungsmittel auf einer nur untergeordneten Stufe befindet, eben um deßwillen, so ausgebreitet auch seine übrige Amtsgewalt seyn möchte, dem willig nachstehe, der eine höhere Weihe erhalten.
\item[27.--29.]\setcounter{enumi}{29} Diese Artikel bedürfen keiner besonderen Rechtfertigung.
\item Die Nützlichkeit, ja, man kann sagen, die Nothwendigkeit eines eigenen \RWbet{Primas in der katholischen Kirche} ist nicht schwer darzuthun. Wenn auch Jesus versprochen, er selbst werde das unsichtbare Oberhaupt der Kirche durch alle Zeit verbleiben: so versteht sich doch, daß er uns durch dieses Versprechen nicht der Pflicht entheben wollte, alle diejenigen Einrichtungen zu treffen, durch welche, nach menschlicher Einsicht, am ehesten Böses verhütet und Gutes herbeigeführt werden könnte. Und unter diese gehört der Primat. In einer jeden Gesellschaft, wenn sie der Kräfte mehrere hat, und eine längere Zeit hindurch fortbestehen soll, wird es entschiedene Vortheile haben, wenn nicht bloß Alle im Allgemeinen ersucht werden, zu überlegen, was für verschiedenes Gute die Gesellschaft durch die ihr angehörigen Kräfte hervorbringen könnte, und dieses den Uebrigen mitzutheilen; sondern wenn noch überdieß irgend ein Einzelner da ist, dem man es dadurch, daß man ihn für den \RWbet{Vorsteher}~\RWSeitenw{386}\ \RWbet{der ganzen Gesellschaft} erklärte, zu einer eigenthümlichen Obliegenheit machte, über das, was die Gesellschaft thun soll, fortwährend nachzudenken, und, was ihm das Beste scheint, nicht nur bekannt zu machen, sondern durch Bitten, durch Aufforderungen, ja, wenn es nöthig ist, selbst durch Befehle dahin zu wirken, daß es zur Ausführung komme; wobei man denn auch zu gleicher Zeit den Uebrigen die Verbindlichkeit auflegt, auf diese Bitten, Aufforderungen oder Befehle ihres Vorstehers zu achten und sie so oft zu vollziehen, als nicht ganz offenbar ist, daß ihre Vollziehung überwiegend schädlich wäre. Die Nützlichkeit einer solchen Einrichtung ist dem gemeinen Menschenverstande von jeher so einleuchtend gewesen, daß es wohl keinen einzigen auf längere Dauer geschlossenen Verein unter den Menschen\RWfootnote{%
	Merkwürdig ist, daß selbst \RWbet{Thiere}, wenn sie einen gemeinsamen Zweck ausführen wollen, einer Art von Anführer sich unterwerfen.}
gibt, wo man nicht Einem der Glieder eine Art Vorsteheramt übertragen hätte. Der Unterschied besteht nur in der bald größeren bald geringeren Ausdehnung, die man den Rechten sowohl als auch den Obliegenheiten eines solchen Vorstehers gab, wie man ihm denn hiernächst bald diesen, bald jenen Namen ertheilte. Ein König und ein Präsident sind beide Vorsteher in einer bürgerlichen Gesellschaft; der Unterschied ist nur, daß jenem gewöhnlich viel ausgedehntere Rechte als diesem eingeräumt sind. Daß nun die Rechte und Pflichten, welche dem obersten Vorsteher der katholischen Kirche zustehen und obliegen sollen, nicht von der göttlichen Offenbarung selbst festgesetzt sind, hat seinen guten Grund. Wären sie dieses: so hätte nicht füglich etwas an ihnen geändert werden können, was doch zu verschiedenen Zeiten und unter verschiedenen Umständen äußerst nothwendig war. Gewiß gab es Zeiten, wo es sehr gut war, wenn der Primas der Kirche nur bittend vortrug, was er von Anderen ausgeführt zu sehen wünschte. Gewiß gab es andere Zeiten, wo er allmählich einen befehlenden Ton (\RWlat{rogantes mandamus}) annehmen durfte und sollte. Auch gab es Zeiten, wo es nicht nur für die gesammte Christenheit, sondern auch noch für andere Menschen recht heilsam war, daß sich der Papst herausnahm, in Dingen zu entscheiden, von denen in unseren Tagen gesagt wird, daß sie ganz außerhalb des Bereiches~\RWSeitenw{387}\ seiner Macht gelegen. Und wieder gibt es jetzt eine Zeit, wo es nothwendig ist, daß er von solchen Forderungen abstehe, Anderes nur erbitte, \usw\
\item[31.--34.]\setcounter{enumi}{34} An diesen Artikeln wird Niemand Anstoß nehmen.
\item Was hier den Geistlichen in Betreff ihrer Einkünfte zur Pflicht gemacht wird, ist eigentlich eine auch allen übrigen Menschen, nur nicht in gleichem Grade obliegende Verbindlichkeit. Der Unterschied zwischen demjenigen, was wir unser \RWbet{Eigenthum} nennen, und was \RWbet{Gemeingut} ist, bestehet nicht in dem Zwecke, zu dem das Eine oder das Andere verwendet werden soll; nicht ist es uns erlaubt, das Eigenthum bloß für uns selbst zu verbrauchen, unangesehen, was wir damit für Andere ausrichten könnten; sondern wir sollen es, wie das Gemeingut, immer nur so verwenden, daß das gemeine Beste dadurch am Meisten gewinnt; also für uns, wenn wir es nöthiger als Andere, für Andere aber, wenn diese es nöthiger haben, als wir selbst. Der einzige Unterschied, der zwischen Eigenthum und Gemeingut obwaltet, ist lediglich der, daß wir von unserer Verwendung bei dem ersteren nur unserem Gewissen, von unserer Verwendung des Gemeingutes aber auch Anderen (etwa dem Staate) Rechenschaft zu legen haben. In ihrem Gewissen also sind eigentlich alle Menschen, auch die Nichtchristen, und um so mehr die nichtgeistlichen Christen, verbunden, ihr Eigenthum wie ein Gemeingut zu betrachten, und zur Befriedigung ihrer eigenen Bedürfnisse davon nur so viel zu verbrauchen, als durchaus nothwendig ist, das Uebrige aber an Arme oder doch zu Zwecken, die gemeinnützig sind, zu verwenden. Für den geistlichen Stand konnte man dieses Verfahren zu einer um desto strengeren Obliegenheit erheben; erstlich schon darum, weil er den Uebrigen mit seinem Beispiele vorleuchten soll; dann aber auch, weil die mit geistlichen Aemtern verbundenen Einkünfte insgemein nur aus den frommen Spenden der Gläubigen entstanden, die diese nur in der bestimmten Absicht gaben, damit sie für Arme oder zu anderen wohlthätigen Zwecken verwendet würden, so daß die Geistlichen nur in soferne auch selbst einen theilweisen Anspruch darauf~\RWSeitenw{388}\ hätten, als sie den Armen beizuzählen wären, \dh\  als sie außerdem gar nicht zu leben vermöchten.
\item Ueber die Zweckmäßigkeit des \RWbet{Cölibatgebotes} ist viel gestritten worden; und es gibt der Gründe für und wider dasselbe so viele und so hochwichtige, daß ich mich, wenigstens bis jetzt, außer Stand fühle, eine bestimmte Meinung zu fassen. Doch gesetzt auch, daß es bei der Ausnahmslosigkeit dieses Gebotes, bei dem Umstande, daß man Diejenigen, die sich zum geistlichen Stande melden, in einem größtentheils noch sehr jugendlichen Alter, und ohne ihren Beruf erst mit gehöriger Sorgfalt geprüft zu haben, aufnimmt, die einmal Aufgenommenen aber schon für ihr ganzes Leben unwiederruflich bindet, bei dem so ausgebreiteten Verderben der Sitten in unserer Zeit, und bei der so großen Anzahl der Mitglieder, die wir aber dem geistlichen Stande aus sehr gutem Grunde wünschen, -- gesetzt auch, sage ich, daß unter diesen Verhältnissen die Nachtheile des Cölibatgebotes überwiegend wären: so würde es doch für uns, die wir die Macht, es aufzuheben, nicht haben, so lange, als es noch in rechtlicher Kraft besteht, ersprießlicher seyn, unser Augenmerk auf die Vortheile, die es gewähret, zu richten. Ich glaube aber, daß man besonders folgende ohne Widerspruch zugeben müsse:
\begin{aufzb}
\item Der Vorsatz einer lebenslänglichen Enthaltsamkeit, den jeder Geistliche gleich beim Eintritt in diesen Stand (bei dem Empfange des Subdiakonats) vermöge des bestehenden Cölibatgebotes fassen muß, setzt, wenn er mit aller Ueberlegung gefaßt wird, eine so hohe Stärke der Seele, eine so edle Geringschätzung sinnlicher Freuden voraus; und die getreue Erfüllung dieses Vorsatzes erzeugt eine so große Fertigkeit in der wichtigen Tugend der Selbstbeherrschung, daß man wohl hoffen sollte, wer so Großes vermag, würde viel Anderes, das ungleich leichter ist, um so gewisser leisten. Wer sich versagen gelernt hat, was die Natur doch so mächtig fordert: sollte der noch der Völlerei, dem Trunke, der Spielsucht sich ergeben können?
\item Der Cölibat, wenn er (vom geistlichen Stande doch größtentheils) gehörig beobachtet wird, trägt ungemein~\RWSeitenw{389}\ viel zur Vermehrung jener Hochachtung bei, die diesem Stande so sehr zu wünschen ist. Zu wünschen ist es doch gewiß, daß der Stand der Geistlichen allgemein hochgeschätzt werde; aus ihrem Munde kommt ja der Unterricht in den erhabensten Religionswahrheiten, aus ihren Händen soll die Schaar der übrigen Christen die heil.\ Sacramente empfangen, den Vorschriften, die dieser Stand uns aufstellt, sollen wir zwangslos gehorchen, \usw\ Ist es nun wohl ein Zweifel, ob diese Personen in unseren Augen verlieren oder gewinnen werden, wenn wir uns vorstellen dürfen, daß sie keinen derjenigen Genüsse kennen, deren wir Menschen uns, es geschehe aus welchem Grunde es immer will, schämen? --
\item In jeder Gemeine gibt es und muß es der Menschen mehrere geben, die sich durch mancherlei Umstände genöthiget sehen, ein eheloses Leben zu führen, auch wenn der Geschlechtstrieb in ihnen schon lange erwacht ist; und wieder Andere gibt es, die selbst im Ehestande auf die sinnlichen Genüsse desselben durch eine geraume Zeit Verzicht leisten müssen. Für alle Diese dienet das Beispiel ihres Geistlichen, wenn er so lebt, wie es das Cölibatgebot vorschreibt, zu einer Art des Trostes und der Ermunterung.
\item Der ehelose Stand bewahret den Geistlichen vor einer Menge von Sorgen für seine Angehörigen, die ihn von der Erfüllung seiner Standespflichten abziehen oder sie ihm wenigstens sehr erschweren würden (1 Kor. 7, 32.). Um wie viel leichter kann er, wenn er allein da stehet, jede Gefahr einer Ansteckung bei dem Besuche der Kranken verachten; um wie viel muthiger die Wahrheit predigen auch dort, wo sich vorhersehen läßt, daß sie ihm Feinde und Verfolger zuziehen werde!
\item Lebt der Geistliche ehelos: so kostet sein Unterhalt der Gemeine um desto weniger (1 Kor. 9, 5 ff.), ein Umstand, der besonders darum sehr zu berücksichtigen ist, weil, wie schon angemerkt wurde, die Einkünfte, welche ein geistliches Amt trägt, ihrer ursprünglichen Bestimmung nach meistens ein Armengut sind.~\RWSeitenw{390}
\end{aufzb}
\item Daß man es nicht ganz dem Belieben eines jeden einzelnen Mitgliedes des geistlichen Standes anheimstellt, ob er und was für fromme Lesungen und Gebete er tagtäglich vornehmen wolle, dürfte in mancherlei Hinsicht sehr gut seyn. Wie viele Geistliche würden, wenn keine bestimmte Vorschrift hierüber bestände, ganze Wochen und Monate vorbeigehen lassen, ohne irgend etwas für den Zweck ihrer Erbauung zu thun. Und wenn sie auch etwas thäten: würden es Andere wissen? Und bietet mir nicht selbst der Umstand, daß eben das Lesestück, welches ich jetzo lese, von allen anderen Genossen meines Standes an eben dem Tage gleichfalls gelesen wird, eine eigene Ermunterung dar, es mit der größt möglichen Erbauung zu lesen? -- Nur über die Auswahl dieser Lesestücke also, über die allzuhäufige Wiederholung einiger Gebete, Psalmen \udgl\  ließe sich rechten. Aber vergesse man nicht, daß sich kein Buch dieser Art so abfassen ließe, daß nicht immer noch etwas mit einem mehr oder weniger scheinbaren Rechte daran zu tadeln wäre.
\end{aufza}
\begin{RWanm} 
In dem Falle einer Krankheit oder auch dann, wenn man eine dringende Liebespflicht gegen seinen Nächsten darüber versäumen müßte, ist die Verbindlichkeit des Brevierbetens nach der eigenen Entscheidung der Kirche als behoben anzusehen. Allein der bloße Umstand, daß sich Jemand für seine eigene Person auf eine zweckmäßigere Weise glaubt erbauen zu können, als es durch Lesung des Breviers geschieht, ist noch kein gültiger Grund, sich von jener Verbindlichkeit, loszuzählen; und zwar schon darum nicht, weil das \RWbet{Aergerniß,} das man durch Uebetretung eines so ausdrücklichen und so oft wiederholten Gebotes der Kirche geben würde, die \RWbet{Verstellung,} die man beobachten müßte, um jenes Aergerniß einiger Maßen zu vermeiden, offenbar größere Uebel wären, als aller hier zu besorgende Zeitverlust seyn kann. 
\end{RWanm}

\RWpar{298}{Die Lehre des Katholicismus von der den Kranken und Sterbenden zu leistenden Hülfe oder der letzten Oelung}
\begin{aufza}
\item Wenn ein Christ auf längere Zeit durch Krankheitsumstände, ja auch wohl andere Ursachen verhindert ist, die gottesdienstlichen Versammlungen zu besuchen: so soll er~\RWSeitenw{391}\ berechtiget seyn, die Geistlichen der Gemeine zu sich bitten zu zu lassen, damit sie ihm die entbehrte Gabe des Unterrichtes reichen, ihn seinen Bedürfnissen nach belehren, berathen, trösten, mit ihm gemeinschaftlich beten, ihn mit den heil.\ Sacramenten der Buße und des Altars versehen, \usw\
\item Die Priester sollen sich weder durch die Beschwerlichkeiten eines Weges, noch durch die Gefahren der Ansteckung, noch durch andere Rücksichten auf bloß leibliche Güter abhalten lassen, demjenigen, der sie um einen solchen der Seele zu leistenden Dienst ersuchet, wer er auch sey, zu willfahren.
\item Wird die Gefahr des Todes bei einem Kranken größer: so soll der Geistliche (Bischof oder Priester) abermals herbeigerufen werden, und nachdem er manch herzliches Gebet um die Vergebung seiner (des Kranken) Sünden, um die Linderung seiner Leiden, ja, wenn es dienlich ist zu seinem Seelenheile, auch um die gänzliche Genesung zum Throne Gottes gesandt, den Kranken an verschiedenen Theilen seines Leibes, wo es der Wohlstand erlaubt, besonders an den fünf Sinneswerkzeugen, mit heiligem Oele salben.
\item Auch diese Gebete und Salbungen wird Gott, wenn es dem Seelenheile des Kranken dienlich ist, seine Genesung befördern; und wenn dieß nicht ist, ihm jene übernatürlichen Gnaden verleihen, die er, um eines seligen Todes zu sterben, nöthig hat.
\end{aufza}

\RWpar{299}{Historischer Beweis dieser Lehre}
Krankenbesuche wurden in der katholischen Kirche von jeher als eine der heiligsten Pflichten des geistlichen Standes erachtet; und schon bei \RWbibel{Jak}{Jak.}{5}{14}\ heißt es: \erganf{Ist Jemand unter euch krank: so rufe er die Priester der Kirche zu sich, daß sie über ihn beten und ihn mit Oel salben im Namen des Herrn. Und das Gebet des Glaubens wird dem Kranken helfen, und der Herr wird ihn aufrichten, und wenn er in Sünden ist: so werden sie ihm vergeben werden.} Hieraus aber ergibt sich das Wesentliche der obigen Lehren von selbst.~\RWSeitenw{392}

\RWpar{300}{Vernunftmäßigkeit und sittlicher Nutzen}
\begin{aufza}
\item[1.\ und 2.]\stepcounter{enumi}\stepcounter{enumi} Eine gewiß sehr heilsame Anordnung ist es, daß Personen, welche auf eine längere Zeit verhindert sind, an den gottesdienstlichen Versammlungen Antheil zu nehmen, die Geistlichen des Ortes zu einem ihre Erbauung bezweckenden Besuche einzuladen berechtiget, die Letzteren aber verpflichtet seyn sollen, auf diese Vorladung zu erscheinen, so viele Unbequemlickeiten oder Gefahren für sie selbst damit verbunden seyn möchten.
\begin{aufzb}
\item Wer durch eine längere Zeit gehindert ist, die gottesdienstlichen Versammlungen zu besuchen, der entbehret des darin ertheilten Unterrichtes; und es ist billig, daß ihm dieses Versäumniß durch einen eigenen Besuch des Geistlichen zum Theile wenigstens eingebracht werde.
\item Er kann das heil.\ Abendmahl nicht am Altare genießen; es ist also nöthig, daß es in seiner Wohnung ihm dargereicht werde. Aber was könnte schicklicher seyn, als die Verfügung, daß es ihm nur aus der Hand eines Geistlichen, eines Diakons wenigstens, dargereicht werde? --
\item Ein Kranker, ein Gefangener bedürfen in ihrer schweren und ungewöhnlichen Lage gerade am Meisten der weisen Belehrung, des sanften Trostes eines Geistlichen; zumal wenn ihr durch körperliche Leiden geschwächter Geist jetzt vielleicht weniger, als sonst, im Stande ist, die Quelle des Trostes in sich selbst zu finden. Zu wie viel sittlichen Zwecken also wird ein vernünftiger Priester einen solchen Zeitpunct nicht zu benützen wissen! Wie manche heilsame Entschließung wird er in dem Gemüthe auch selbst Desjenigen, der sich zu anderer Zeit nicht eben sehr lenksam bewies, hervorzubringen vermögen!
\item Und je beschwerlicher und gefahrvoller ein Krankenbesuch bisweilen werden mag, um desto unzweideutiger beweiset sich der Eifer des Seelsorgers in seiner Pflichterfüllung und die nachahmungswürdige Liebe desselben zu sei\RWSeitenw{393}nen Mitbrüdern; um desto offenkundiger wird es, wie viel höher die Seele zu schätzen sey, als der Leib!\RWfootnote{%
	Doch wäre zu wünschen, daß man weise und christlich genug dächte, um in Fällen, wo die Gefahr der Ansteckung sehr groß ist, sich der Barmherzigkeit Gottes vertrauensvoll zu überlassen, ohne das Leben des Seelsorgers zu gefährden.}
\end{aufzb}
\item Da die Besinnungskraft eines Kranken, besonders bei schon herannahendem Tode meistens sehr schwach zu seyn pflegt: so bedarf es gewisser \RWbet{gröberer Eindrücke} auf seine Sinne, wenn man sich ihm auch jetzt noch verständlich mittheilen will. Der \RWbet{Sinn des Gefühles} dauert gewöhnlich am Längsten; und so ist es denn wohl am Schicklichsten, durch \RWbet{diesen} Sinn auf das Bewußtseyn des Kranken, wo möglich, noch eine wohlthätige Einwirkung zu versuchen. Dieß ist der \RWbet{Zweck der Salbung mit Oel}.\RWfootnote{%
	Daß nicht bloße Berührungen des Kranken, sondern eine Salbung mit Oel vorgeschrieben wurde, hat auch den Nutzen, daß die Gefahr der Ansteckung für den Seelsorger vermindert wird.}
Fühlt es der Kranke noch, daß diese Handlung jetzt an ihm vorgenommen werde, und ward er in seinen gesunden Tagen von der Beschaffenheit und von den Wirkungen dieses Heiligungsmittels gehörig unterrichtet: so wird sich auch jetzt noch ein Strahl des Trostes und der Beruhigung in seine Seele senken. Er erfährt so, daß er nicht einsam und verlassen auf seinem Lager liege, daß gute, mitleidige Menschen voll der Bekümmerniß um ihn herumstehen, daß sie für seine Rettung, wenn sie nichts Anderes vermögen, wenigstens beten. Dieses Gebet dient übrigens, wie man von selbst einsieht, nicht nur zur Erbauung des Kranken, wenn er es etwa, da es laut gesprochen wird, vernehmen sollte; sondern eben so auch zur Erbauung aller Anwesenden.
\item Die Stunde des Todes ist und bleibt doch für uns Alle mehr oder weniger furchtbar. Der große Schritt aus der Zeit in die Ewigkeit; die bange Rückerinnerung an alle die tausendfältigen kleineren und größeren Vergehungen, deren wir uns in diesem Leben schuldig gemacht; der schreckliche Zweifel, der sich jetzt häufig einfindet, ob wir uns auch in unserem Glauben an eine Fortdauer und an alle Hoffnungen~\RWSeitenw{394}\ der Religion nicht etwa doch getäuscht; der heftige Schmerz, der unsere Glieder entweder schon durchtobet, oder den wir in Kurzem noch zu erfahren fürchten: dieß Alles wirkt auf unseren geschwächten Geist mit einer solchen Heftigkeit ein, daß es in Wahrheit viel erfordert, mit Ruhe sterben zu können, und nicht zu zittern und zu zagen. Wie wohlthätig also erweiset sich uns Gott im katholischen Christenthume, daß er uns nicht verläßt auch selbst in der bangen Todesstunde, daß er uns Hülfe sendet von Menschen und Hülfe von Oben! -- Nur eben durch diese Hülfe empfinden wir Muth und Kraft, aus diesem Leben zu treten, ganz mit jener zuversichtlichen Ergebung in Gottes Rathschluß, die uns erst völlig würdig machet des Eingangs in die Wohnungen der ewigen Freude. 
\end{aufza}

\begin{center}*\ *\ *\end{center}

\RWch{Beschluß dieser Religionswissenschaft.}
\RWpar{301}{Wahrheit und Göttlichkeit des katholischen Christenthums}
\begin{aufza}
\item Wir haben nun, obgleich nicht alle, doch die meisten, und besonders alle diejenigen Lehren des katholischen Christenthums kennen gelernt, deren Vernunftmäßigkeit oder sittlicher Nutzen noch am ehesten bezweifelt werden könnte. Ich glaube deßhalb erwarten zu dürfen, daß Niemand, der alle hier abgehandelten Lehren mit der Vernunft übereinstimmend und sittlich zuträglich befunden hat, an irgend einer der übrigen Lehren, die ich nur wegen der Kürze der Zeit oder aus menschlicher Vergessenheit übergangen habe, einen Anstoß nehmen werde.
\item Und so kann ich es denn als erwiesen ansehen, daß die katholische Religion ein Inbegriff von durchaus solchen Wahrheiten sey, die nicht nur mit der Vernunft in keinem Widerspruche stehen, sondern die vielmehr alle einen sehr wohlthätigen Einfluß auf unsere Tugend und Glückseligkeit zu äußern versprechen, sobald wir sie erst mit voller Zuversicht annehmen werden. Dieß können und sollen wir aber~\RWSeitenw{395}\ wirklich; denn dieser Lehrbegriff hat ja nebst dem so eben nachgewiesenen \RWbet{inneren} Merkmale auch noch das zweite \RWbet{äußere} Kennzeichen einer wahren göttlichen Offenbarung, die Beglaubigung nämlich durch Wunder.
\item Einmal ist schon diese so hohe Vortrefflichkeit des katholischen Lehrbegriffes, diese durchgängige Vernunftmäßigkeit desselben und diese Uebereinstimmung mit unsern sittlichen Bedürfnissen eine Art von Wunder, oder ein Zeichen zu nennen, dadurch uns Gott deutlich genug zu erkennen gegeben hat, daß er die Wahrheit dieses Lehrbegriffes bezeuge, daß er ihn von uns geglaubt wissen wolle als seine Offenbarung. Denn wie viel Ungewöhnliches, ja wie Unzähliges, das auch ganz anders hätte ausfallen können nach dem gewöhnlichen Laufe der Dinge, hat sich seit einem Zeitraume von achtzehn Jahrhunderten zugetragen, um diesem Lehrbegriffe gerade diejenige Art von Ausbildung zu ertheilen, welche er wirklich erhielt! Wie kam es doch nur, daß Menschen auf alle diese wohlthätigen Lehren verfielen? Verfielen und daran festhielten in Zeiten, wo ihre sittliche Verdorbenheit sie das gerade Gegentheil von dem, was sie hier lehrten, thun hieß? Wie kam es, daß die Vorsteher unserer Kirche in so vielen Streitigkeiten immer für diejenige Meinung sich entschieden, die einzig die richtige war, obgleich sie den innern Grund oft gar nicht einsahen? Wie ist es zu erklären, daß andere Religionsgesellschaften, die durch den Abfall von der katholischen Kirche entstanden, nicht eben so richtig wie diese lehrten, obwohl sie häufig gelehrtere Männer in ihrer Mitte hatten? \usw\
\item Doch in dem zweiten Haupttheile wurde gezeigt, daß der katholische Lehrbegriff auch noch viel größere Wunder zu seiner Bestätigung aufzuweisen habe. Dort nämlich wurde erwiesen, daß sich gleich bei der Entstehung des Christenthums die unverkennbarsten Wunder in Menge zugetragen haben; Ereignisse, die sich bei einem richtigen Begriffe von dem, was ein Wunder sey, als solche darstellen, wie auch der eigentliche Hergang derselben beschaffen seyn möchte. Da nun in eben diesen, meistens von Jesu und den Aposteln verrichteten Wunderwerken der vornehmste Grund darin liegt,~\RWSeitenw{396}\ daß die katholische Religion bis auf den heutigen Tag sich erhalten und so weit ausgebreitet hat; da man den Lehrern derselben nur in sofern gläubig beipflichtete, als man sich vorstellte, daß sie von Jesu und den Aposteln selbst seyen vorgetragen worden, oder sich wenigstens auf das von ihnen Vorgetragene stützen: so habe ich schon dort gezeigt, daß in diesen Wundern eine Bestätigung des katholischen Lehrbegriffes als einer göttlichen Offenbarung liege, wenn anders aus einer näheren Prüfung hervorgeht, daß dieser Lehrbegriff der Tugend und Glückseligkeit der Menschen zuträglicher sey, als jeder der übrigen christlichen Lehrbegriffe. Diese Prüfung ist nun angestellt worden, und durchgängig nur zum Vortheile des Katholicismus ausgefallen. Und somit glaube ich den Beweis für die Wahrheit und Göttlichkeit dieser Religion jetzt als beendigt ansehen zu können.
\item Dieser Beweis ist, wie ich mir überdieß schmeichle, von einer solchen Beschaffenheit, daß er in dem Gemüthe eines Jeden, der ihn gehörig auffaßt, einen Grad der Zuversicht erzeuget, der zur vollkommensten Beruhigung hinreicht. Denn wer sich die Mühe nicht verdrießen lassen will, ihn nochmals ganz zu überblicken, die Sätze, aus welchen er bestehet und die man nothwendig zugeben muß, um zu dem Schlußsatze desselben zu gelangen, herauszuheben und neben einander zu stellen; der wird gewahr werden, daß die Anzahl dieser Sätze eben nicht allzu groß ist, und daß sie alle theils von der Art sind, daß ihre Wahrheit schon durch den bloßen gesunden Menschenverstand unwidersprechlich einleuchtet, theils auf Erfahrungen beruhen, die Jeder täglich zu wiederholen vermag. Auch nicht ein einziges Mal war es zur Führung unseres Beweises nöthig, uns in das unsichere Gebiet der Metaphysik zu versteigen, und erst von dorther die Gründe zu holen, auf welchen das Gebäude unserer religiösen Ueberzeugungen aufgeführt werden sollte; auch nicht ein einziges Mal bedurften wir es, erst mit Bestimmtheit auszumitteln, wie ein gewisses Ereigniß an sich beschaffen gewesen, und durch was für verschiedener Kräfte Zusammenwirken es hervorgebracht worden sey, welches der wahre, von dem Verfasser selbst beabsichtigte Sinn dieser oder jener Schriftstelle sey, über deren Auslegung die Gelehrten uneinig~\RWSeitenw{397}\ sind; und was es sonst noch für unsichere, oder doch schwer einzusehende Voraussetzungen gibt, auf die man die Wahrheit und Göttlichkeit unserer heil.\ Religion sonst nur zu oft gegründet. Indem ich dieß sage, habe ich keineswegs vergessen, daß auch in diesem Lehrbuche gar manche schwierige Untersuchungen vorgekommen sind, und daß ich es mehr als einmal gewagt habe, Meinungen auszusprechen, in denen mir schwerlich ein Jeder beipflichten wird, von denen einigen ich vielleicht selbst mit der Zeit wieder abgehen werde. Allein nur dieses wird man mir, wenn man gerecht seyn will, zugestehen müssen, daß solche strittige und unsichere Meinungen hier nur gelegenheitlich, zur Uebung im Denken oder zu sonst einem andern Zwecke vorgebracht wurden, ohne daß der Beweis, der für die Wahrheit und Göttlichkeit des katholischen Christenthums geführt ward, ihrer nur im Geringsten bedurft hätte. Man kann diese Meinungen annehmen oder verwerfen, und alle wesentlichen Gründe, auf welchen jener Beweis beruhet, bleiben noch immer unverrückt stehen. Ich erlaube mir denn zu sagen, hier sey die Wahrheit und Göttlichkeit des katholischen Christenthums auf eine Art erwiesen, bei welcher auch der bescheidenste Denker nicht zu befürchten braucht, daß seine Schlüsse durch was immer für eine einst noch zu machende Entdeckung auf dem Gebiete der Weltweisheit, oder auf jenem der Geschichte und Auslegungskunst, oder in sonst einer andern Wissenschaft, je könnten umgestoßen werden.
\end{aufza}

\RWpar{302}{Beurtheilung anderer Religionen}
Zwar habe ich es bei der bisherigen Untersuchung des katholischen Lehrbegriffes nicht ganz unterlassen, auch auf dasjenige zu achten, was die übrigen Religionen auf Erden, welche sich für geoffenbart ausgeben, über dieselben Gegenstände lehren, und so oft diese Lehren etwas besonders Merkwürdiges hatten, habe ich nicht ermangelt, ihrer ausdrücklich zu erwähnen, dergestalt, daß auch aus diesen bloß gelegenheitlichen Erwähnungen schon hervorgehen dürfte, daß keine jener Religionen es verdiene, dem katholischen Christenthume von uns vorgezogen zu werden. Dennoch wird es nicht überflüßig seyn, bevor wir schließen, noch eine jede dieser~\RWSeitenw{398}\ Religionen einzeln in's Auge zu fassen, um auf das Vollkommenste überzeugt zu werden, wie wenig es der Wille Gottes seyn könne, daß wir mit Hintansetzung der katholischen Religion zu irgend einer von ihnen übertreten. Zu diesem Zwecke ist aber durchaus nicht nothwendig zu untersuchen, ob die Entstehung, Erhaltung oder Ausbreitung dieser Religionen nicht auf gewisse außerordentliche Begebenheiten sich gründe, sondern wir brauchen bloß auf ihre \RWbet{Lehren} einige Blicke zu werfen; und schon aus der Beschaffenheit dieser wird sich ergeben, daß keine einzige jener Religionen als die vollkommenste für uns angesehen werden könne, gesetzt auch, daß sie Wunder in Menge aufzuweisen hätten.
\begin{aufza}
\item Was zuerst die sehr beträchtliche Anzahl \RWbet{christlicher} Lehrbegriffe anlangt, welche es noch nebst dem katholischen gibt, sowohl diejenigen, zu denen sich ganze Völkerschaften bekennen, als auch diejenigen, die nur von einzelnen Gelehrten ausgedacht worden sind; so können sie keiner sich rühmen, so viele der Tugend und Glückseligkeit der Menschen zuträgliche Lehren zu haben, als die katholische Kirche aufstellt, auch wenn wir uns nur an dasjenige halten, was sie mit Allgemeinheit lehret. In allen diesen christlichen Lehrbegriffen fehlt es an jener herrlichen Lehre von der Unfehlbarkeit des allgemeinen Glaubens, deren sittlichen Nutzen wir oben betrachtet haben. Die \RWbet{griechische} (nicht unirte) Kirche, welche in ihren Lehren sich der katholischen noch am Meisten nähert, gestehet zwar neben der Bibel auch eine mündliche Ueberlieferung als Erkenntnißquelle der göttlichen Offenbarung zu, aber sie untersagt es doch ihren eigenen Patriarchen und Synoden, Entscheidungen in Sachen des Glaubens zu treffen. Sie will ferner auch den Primat des Papstes nicht anerkennen, und läugnet das Vorhandenseyn eines Reinigungszustandes nach dem Tode. Gewiß sehr wichtige Irrthümer! Gesetzt also auch, daß diese Kirche in ihren nicht zum Glauben gehörigen Disciplinarvorschriften und Gebräuchen, in ihrem Gottesdienste \usw\ der katholischen nicht im Geringsten nachstünde, ja sie wohl gar überträfe (was doch kein Unparteilicher behaupten wird); so könnten wir dennoch auf keine Weise zweifeln, daß der~\RWSeitenw{399}\ katholische Lehrbegriff den Vorzug vor jenem der griechischen Kirche verdiene. Daß diese letztere noch nebstdem das Ausgehen des heiligen Geistes vom Sohne verwirft, ist eine Sache, die zwar von keiner so großen sittlichen Wichtigkeit ist, aber doch offenbar irrig. -- Noch viel abweichender, und eben deßhalb auch mit noch mehreren Irrthümern untermengt ist der Lehrbegriff der \RWbet{evangelischen} Kirche. Diese erklärte die Bibel bekanntlich für die alleinige Erkenntnißquelle der ganzen göttlichen Offenbarung, beging aber gleich bei ihrer Entstehung den Fehler, daß sie gewisse \RWbet{symbolische Bücher} entwarf, bei deren Lehrsätzen sie immer stehen zu bleiben sich selbst verbindlich machte. Sie lehrte ferner: die Bibel sey nicht nur die einzige Erkenntnißquelle, sondern sie sey auch für jeden Menschen verständlich, die Kirche bedürfe gar keines sichtbaren Oberhauptes zur Erhaltung der Einheit, sondern es sey genug, daß Christus ihr unsichtbares Oberhaupt zu seyn versprochen habe; durch den Sündenfall sey unser freier Wille ganz aufgehoben worden; der Glaube allein mache selig, die guten Werke aber hätten gar keine Verdienstlichkeit; Christus sey auch als Mensch allwissend, allgegenwärtig, \usw ; es gebe nur zwei Sacramente, nämlich die Taufe und das heil.\ Abendmahl, in welchem letztern Christus nur im Augenblicke des Genusses seine segensreiche Gegenwart äußere; die Beichte sey zur Vergebung der Sünden nicht nothwendig, auch keine Bußwerke und Genugthuungen wären erforderlich; die Heiligen dürfe man nicht verehren, \umA\  Die \RWbet{reformirte} Kirche wollte noch überdieß die wirkliche Gegenwart Christi im heil.\ Abendmahle nicht einmal im Augenblicke des Genusses selbst zugeben, \dh\  sie wollte, daß wir von dem Genusse dieses Mahles durchaus keine anderen Segnungen erwarten, als solche, die auch ein jedes andere lebhafte Andenken an Christum gewähret. Auch lehrte Calvin, daß Gott gewisse Menschen zu einer ewigen Verdammniß vorherbestimmt habe nach einem unbedingten, \dh\  nicht aus Vorhersehung ihrer Lasterhaftigkeit hervorgehenden Rathschlusse. Wahr ist es, daß man in neuerer Zeit in beiden Kirchen versucht habe, einige dieser anstößigen Lehren zu mildern; doch ein noch größeres Uebel, ein alle positiven Lehren des Christenthums aufhebender \RWbet{Ra}\RWSeitenw{400}\RWbet{tionalismus} erhob sich, und gewinnt mit jedem Tage mehr und mehr Ausbreitung, zum Wenigsten in Deutschland. Wer kann nun, wenn er von dem katholischen Christenthume gut unterrichtet ist, und also dasjenige, was bloße Volksmeinung ist, oder zu den abänderlichen Gebräuchen und Disciplinarvorschriften gehört, nicht mit den Glaubenslehren verwechselt, den großen Vorzug des Katholicismus verkennen?
\item Die Religion der \RWbet{Juden}, wenn sie so aufgefaßt wird, wie wir sie in den Büchern des a.\,B.\ antreffen, wird auch von uns Katholiken als eine wahre göttliche Offenbarung betrachtet. Da aber diese Offenbarung nur wenig mehr als bloße natürliche Religion enthält, und die Erscheinung eines vollkommneren Lehrers verheißt, den sie mit Zügen schildert, welche an Jesu in Wahrheit anzutreffen sind: so kann es uns wohl unmöglich beikommen, das Christenthum zu verlassen, um zur mosaischen Religion zurückzukehren. Jedoch die meisten der jetzt lebenden Juden sind noch ungleich schlimmer daran, da sie nebst der Bibel auch den \RWbet{Talmud} (eine angebliche Auslegung und Ergänzung des mosaischen Gesetzes) annehmen, ein Werk, das ihnen nicht nur eine unerträgliche Last von Satzungen aufbürdet, sondern auch voll von den abgeschmacktesten Thorheiten ist. Unter den wenigen Juden, die auf Gelehrsamkeit Ansprüche machen können, gibt es noch eine eigene Partei, die \RWbet{Kabbalisten,} die sich des Besitzes einer viel höheren geheimen Weisheit rühmen, durch welche sie auch selbst dasjenige, was in der Bibel vorkommt, nicht nach dem Buchstaben, sondern auf eine ganz andere geistige Weise zu deuten wissen. Zu dem Besitze dieser geheimen Weisheit wollen sie auf dem Wege einer Ueberlieferung (Kabbala) gelangt seyn, welche nach Einigen sogar von Adam selbst ausgegangen seyn soll. Was nun von Rabbi \RWbet{Akibha} und einigen Andern in Schriften (vornehmlich in den zwei Büchern \RWbet{Hezirah} und \RWbet{Sohar}) mitgetheilt worden ist, bestehet seinem wichtigsten Inhalte nach darin, daß aus dem Einen göttlichen Urwesen oder Urlichte (\RWbet{Ensoph}) alle übrigen Dinge ausgeflossen wären, und zwar mittelst der sogenannten \RWbet{Schehina} oder dem Ebenbilde Gottes; daß es solcher Ausflüsse oder Entwicklungsstufen~\RWSeitenw{401}\ (\RWbet{Sephiroth}) eigentlich zehen, und vier Welten gebe, deren Eine immer vollkommener als die andere sey. Der praktische Theil dieser Kabbala ertheilt, nicht eine Anleitung zur Tugend, sondern der Kunst, Wunder zu wirken, durch eine eigenthümliche Anwendung göttlicher Namen und Sprüche. Alle diese Lehren sind so beschaffen, daß durch eine gläubige Annahme derselben der wahren Tugend und Glückseligkeit der Menschen nicht der geringste Vorschub geschähe.
\item Bei den Persern finden wir gewisse heilige Bücher (\RWbet{Zend-Avesta}), die angeblich von \RWbet{Zoroaster}, einem Magier aus Medien, ungefähr 600 Jahre vor Christo geschrieben worden seyn sollen. Das Wichtigste, was wir aus diesen Büchern, wenn wir sie gläubig annehmen könnten, erlernen würden, wäre: \RWbet{Zeruane-akherene} sey der letzte Urgrund aller Dinge, aus welchem selbst \RWbet{Ormuzd} (das gute) und \RWbet{Arihmann} (das böse Grundwesen) selbst erst hervorgegangen wären, zwei Wesen, die mit einander in stetem Kampfe sind, und dieser Kampf auch durch die Geister, die ihre Thronen umgeben, fortsetzen lassen. Wir würden ferner wissen, daß die Stammeltern unsers Geschlechtes durch Arihmann's Verführung Sünder geworden seyen, daß aber einst Alle, auch Arihmann, selbst noch gebessert werden sollen. Fragen wir aber nun nach den Mitteln, wodurch wir diese Besserung zu Stande bringen könnten: so fänden wir in dem ganzen Werke nichts Anderes, als allerlei geheimnißvolle Gebetsformeln und nutzlose Ceremonien verzeichnet. Könnten wir also wohl im Ernste glauben, Gott wolle, daß wir die weisen Vorschriften des Christenthums verlassen, und diesen Parsesdienst annehmen?
\item Viel unvollständiger sind die Nachrichten, die wir von den religiösen Begriffen der \RWbet{Indier} haben. Ihre Heiligen in der sanskritten Sprache geschriebenen Bücher (Vedam) sind überaus dunkel und selbst ihre Echtheit ist noch nicht genug erwiesen. Nach der Behauptung Einiger soll \RWbet{Parabrahma} das höchste Grundwesen seyn, das eben deßhalb auch \RWbet{Param Atma}, \di\ der erste Geist, und \RWbet{Antrojanis}, \di\ das einzige Wesen genannt wird. Aber wie sonderbar, daß diesem höchsten Wesen gleichwohl eine~\RWSeitenw{402}\ Gattin, \RWbet{Paraschakli}, beigesellt wird? Eben so sonderbar ist die indische Dreieinigkeit (\RWbet{Trimurti}), da ihre dritte Person (\Ahat{\RWbet{Shiva}}{\RWbet{Schuven}}) nur damit beschäftiget ist, zu zerstören, was die beiden andern (\RWbet{Brahma} und \RWbet{Wischnu}) Gutes veranstaltet haben. Auch gibt es eine Menge \RWbet{Menschwerdungen} des Brahma, die aber weit mehr Aehnlichkeit mit den Metamorphosen der griechischen Mythologie haben, als daß sie mit der christlichen Lehre von einer Menschwerdung des Sohnes Gottes verglichen werden könnten. Unläugbar ist es endlich, daß diese Bücher nirgends auf Besserung des Herzens, sondern auf ein bloßes Beobachten nutzloser Ceremonien als Mittel zur Entsündigung dringen. Wer weiß es endlich nicht, was für abscheuliche Gebräuche in dieser indischen Religion theils zugelassen, theils sogar vorgeschrieben werden? Wer hat nicht von der grausamen Sitte gehört, vermöge deren die Weiber bei dem Tode ihrer Gatten sich verbrennen lassen? Wo hat das Christenthum je etwas so Thörichtes gefordert, oder nur geduldet?
\item Die heiligen Bücher der \RWbet{Chinesen} geben zwar einen ganz leidentlichen Begriff von der Tugend, wenn sie dieselbe in die Mäßigung unserer Begierden setzen; aber die Mittel, welche sie uns zu diesem Zwecke vorschlagen, sind äußerst unvollkommen. Astrologie und Zauberei werden als Hauptbeschäftigungen des Weisen dargestellt; Gestirne und Flüsse und alle Elemente werden von höheren Geistern bewohnt, die unsere Opfer verlangen; und von dem einigen und höchsten Grundwesen wird fast nichts gesprochen.
\item Welches die religiösen Begriffe der \RWbet{alten Aegypter}, nicht sowohl jene des Volkes, als vielmehr die ihrer Priester gewesen seyn mochten, ist eine schwer zu beantwortende Frage; besonders da es bisher noch nicht gelungen ist, die \RWbet{Hieroglyphen}, in welchen diese einen Theil ihrer Geheimnisse niedergelegt haben dürften, vollständig zu entziffern. Den Begriff eines höchsten Urwesens glauben Einige in der ägyptischen \RWbet{Athor} zu finden, obgleich die ägyptischen Priester selbst diese Athor nur mit Venus Urania vergleichen. Sie soll ein Ei geboren haben, aus welchem \RWbet{Pthah} (der Urvater) und \RWbet{Neitha} (die Urmutter) hervorgegangen wären.~\RWSeitenw{403}\ Die Seele des Menschen soll nach dem Tode 3000 Jahre lang in allerlei thierischen Körpern umherirren, bis sie dann wieder in ihren menschlichen Leib, wenn er durch Einbalsamirung bis dahin gehörig aufbewahrt blieb, zurückkehrt, \usw\ Kann man dergleichen Fabeln dem christlichen Lehrbegriffe wohl an die Seite stellen?
\item Die wichtigsten Lehren der \RWbet{muhamedanischen} Religion sind nach dem Koran folgende: Es gibt nur einen einzigen Gott, und Muhamed ist sein Prophet, nicht zwar der einzige; denn allerdings waren auch Moses und Christus Propheten; aber Muhamed ist doch der letzte und vornehmste unter ihnen. Das Schicksal hat einem Jeden aus uns den Tag, an dem wir sterben müssen, dergestalt festgesetzt, daß wir seine Erscheinung weder durch Unvorsichtigkeit beschleunigen, noch durch die größte Sorgfalt verzögern können. Die Freuden des Himmels bestehen in sinnlichen Lüsten. Durch Tugend, durch Beten, Fasten kann dieser Himmel verdient werden. Doch ist auch jeder Mensch in seinem Leben wenigstens einmal zu einer Wallfahrt nach Mekka verbunden; den wahren Glauben soll man durch Feuer und Schwert verbreiten; \usw\ Wie verblendet müßte man seyn, wenn man die sittliche Verderblichkeit dieser Lehren nicht einsehen wollte! -- Und wie ärgerlich ist nicht der eigene Lebenswandel dieses angeblichen Propheten, dem Gott für seine eigene Person jegliche Ausschweifung erlaubet!
\item Von den religiösen Begriffen noch einiger anderer Völker, \zB\  der \RWbet{Griechen} oder der \RWbet{Römer} zu sprechen, verlohnt sich gar nicht der Mühe; da Niemand so unverschämt war, sich anzustellen, als ob er in demjenigen, was diese Völker glaubten, die wahre göttliche Offenbarung vermuthe. Ueberhaupt ist zu bemerken, daß wohl kein einziger von den Gelehrten neuerer Zeit, die uns in irgend einer der bisher angeführten Religionen die Ueberbleibseln einer das Christenthum weit übertreffenden Weisheit bewundern lassen wollen, ganz redlich zu Werke gegangen sey. Sie scheinen dieß Alle nur in der Absicht gethan zu haben, um den Werth des Christenthums in unsern Augen möglichst herabzusetzen. Nur deßhalb, scheint es, wählten sie insgemein zum Gegenstande ihrer so übertriebenen Lobpreisungen eine derjenigen~\RWSeitenw{404}\ Religionen, die sich bei alten und schon durch ihr Alterthum uns ehrwürdigen Völkern befinden. Sie gestehen auch klüglich schon im Voraus ein, daß keine dieser Religionen so, wie wir sie gegenwärtig finden, ganz vollkommen sey; aber die fruchtbarsten Wahrheiten und die erhabensten Geheimnisse, sagen sie, wären bereits verloren gegangen; und manche uns thöricht dünkende Gebräuche und Lehren, die wir hier antreffen, wären ihrem Ursprunge nach nichts Anderes als sinnbildliche Darstellungen, die eine spätere Zeit nicht mehr verstanden und eben deßhalb je länger je mehr nur verunstaltet hätte. Darum behaupten sie endlich auch von allen denjenigen Gebräuchen und Lehren des Christenthums, die eine auch noch so entfernte Aehnlichkeit mit gewissen, in jenen alten Religionen vorkommenden Gebräuchen und Lehrsätzen haben, die erstern wären den letztern nachgebildet. Mir däucht es nun, um die gute Sache des Christenthums zu retten, sey es gar nicht nöthig, uns in eine immer nur mühsame Widerlegung des historischen Theils aller dieser Behauptungen einzulassen. Möchte es wahr seyn, daß uns in jenen alten Religionen gewisse erhabene Lehren der Weisheit untergegangen wären; eben weil sie uns \RWbet{untergegangen} sind, diese Lehren, ist es nicht Gottes Wille, daß wir sie wissen sollen, sie müssen nicht wahrhaft nützlich für uns seyn, und Thorheit wäre es, ihren Untergang sehr zu bedauern. Möchte es eben so wahr seyn, daß sich gewisse Gebräuche und Lehren des Christenthums aus diesen alten Religionen herschreiben: verdienen sie darum nicht ferner beibehalten zu werden? könnten sie deßhalb schon keinen Anspruch auf den Namen göttlicher Offenbarungen haben? Nach den Begriffen, die oben aufgestellt wurden, ist dieses gar nicht der Fall. Die Gründe, auf welchen der oben gelieferte Beweis für die Wahrheit und Göttlichkeit des katholischen Christenthums beruhet, werden durch keine historische Untersuchung, zu welchen Ergebnissen sie auch immer führen mag, umgestoßen.
\end{aufza}

\RWpar{303}{Beschluß}
\begin{aufza}
\item Wenn wir von irgend einer uns vorhin unbekannten, oder doch nicht von uns angenommenen Wahrheit Ueber\RWSeitenw{405}zeugung erlangen; so dürfen wir nie versäumen, uns zu befragen, wozu diese neu gewonnene Ueberzeugung von uns benützt werden könne und solle? Ob sie insonderheit uns nicht gewisse Pflichten auflege? \usw\
\item Auch diejenigen also, die sich durch den hier vorgetragenen Beweis, oder auf sonst eine andere Art von der Wahrheit und Göttlichkeit des katholischen Christenthums überzeugt fühlen, mögen es nicht unterlassen, die beschriebene Frage an sich zu stellen; und sie werden gewiß bald inne werden, daß ihnen hier eine Ueberzeugung zu Theil geworden sey, welche gehörig angewandt den wohlthätigsten Einfluß auf ihr ganzes künftiges Glück in Zeit und Ewigkeit haben könne; so daß sie
\begin{aufzb}
\item erstlich schon alle Ursache haben, Gott, als durch den uns alle heilsame Erkenntniß wird, mit dem gerührtesten Herzen dafür zu danken; dann aber müssen sie es
\item als ihre nächste Pflicht betrachten, jede der einzelnen Lehren, die der katholische Religionsbegriff enthält, auf das Genaueste kennen zu lernen und sich geläufig zu machen, weil nur so möglich ist, nach diesen Lehren zu leben, und aus ihnen allen den Trost, den zu gewähren sie im Stande sind, zu schöpfen. Sie müssen eben deßhalb
\item bei jeder Gelegenheit, wo eine Anwendung von diesen Lehren Statt findet, sich ihrer erinnern, und sie in der That befolgen. Sie dürfen
\item nicht eher mit sich zufrieden werden, als bis ihr eigenes Bewußtseyn ihnen das Zeugniß gibt, daß sie durch ihren auf so feste und deutlich erkannte Gründe sich stützenden Glauben an das katholische Christenthum in Wahrheit besser und glückseliger geworden sind, als sie es ehedem waren, oder als Andere es sind, die dieses Glückes entbehren. Sie müssen sich
\item endlich auch bestreben, die Ueberzeugung, die ihnen selbst geworden ist, so viel es ihre Verhältnisse gestatten, auch Andern beizubringen; und das um so eifriger thun, wenn sie finden, daß es auch unter uns Katholiken noch~\RWSeitenw{406}\ Viele gibt, denen es an einer gehörigen Einsicht in die Gründe unserer Religion, oft selbst an einem richtigen Verständnisse des Sinnes ihrer Lehren mangelt.
\end{aufzb}
\item Wer nichts von dem Allen zu thun bereit wäre, dem würde die Ueberzeugung von der Wahrheit und Göttlichkeit des katholischen Christenthums nicht nur keinen Nutzen gewähren, sondern sie würde ihn sogar verantwortlich machen. \erganf{Denn \RWbet{der Knecht, der den Willen seines Herrn kennen gelernt hat, und ihn nicht thut, wird eine doppelte Strafe empfangen.}} (\RWbibel{Lk}{Luk.}{12}{47})
\item Wenn wir dagegen thun, was so eben verlangt worden ist; so werden wir mit jedem Tage vollkommener an uns selbst inne werden, daß der katholische Glaube eine bewunderungswürdige Kraft habe, besser und glückseliger zu machen. Auch noch an uns wird sich der Ausspruch Jesu bewähren: \erganf{\RWbet{Wer meine Lehre befolgt, wird an sich selbst inne, ob sie von Gott sey, oder nicht.}} (\RWbibel{Joh}{Joh.}{7}{17})\par
Denn allerdings gilt auch von dem katholischen Christenthume, was der Apostel schon vom Evangelio gerühmt hat, daß es eine \erganf{\RWbet{Kraft Gottes sey, selig zu machen Jeden, der es mit gläubigem Sinne annimmt}}. (\RWbibel{Röm}{Röm.}{1}{16}) Nur diese Seligkeit, eine auf Tugend gegründete wahrhafte Seligkeit, welche in diesem Leben nur anfangen, in dem andern erst unaufhörlich fortschreiten soll, unter uns Menschen je mehr und mehr zu befördern, das ist der Zweck, zu dem der Sohn Gottes erschienen, und eine Kirche gegründet, welche die Pforten der Hölle nicht überwältigen werden. Möge denn dieser heilsame Zweck an einem Jeden aus uns in Erfüllung gehen!
\end{aufza}

\begin{center}[Ende.]\end{center}
\endinput