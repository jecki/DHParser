\clearpage\linenumbers%
\RWch[Einleitung.]{Einleitung.\RWSeitenwohne{1}}
\RWpar{1}{Inhalt und Zweck dieser Einleitung}
Bei jedem Unterrichte, besonders wenn er in \RWbet{wissenschaftlicher} Form ertheilet wird, ist es gewöhnlich, mit einer \RWbet{Einleitung} in denselben anzufangen. Es finden sich nämlich fast immer mehre Wahrheiten vor, deren Kenntniß dem Empfänger des neuen Unterrichtes gleich Anfangs nothwendig ist, und die man gleichwohl keineswegs bei ihm voraussetzen darf. Diese Wahrheiten sind es denn, die man ihm in der Einleitung zuvörderst beizubringen trachtet. Hieraus ergibt sich sogleich die nähere Beschaffenheit der Untersuchungen, die in einer Einleitung von Rechtswegen vorgenommen werden. Es müssen dieß nämlich
\begin{aufza}[a)]
\item Wahrheiten seyn, von denen wir nicht füglich annehmen können, daß sie dem Anfänger schon von anderer Seite her bekannt sind. Es müssen ferner
\item Wahrheiten seyn, die mit dem Unterrichte, den wir nun zu ertheilen haben, in einer gewissen \RWbet{Verbindung} stehen, die eben den Grund enthält, weßhalb wir sie vielmehr bei diesem, als bei irgend einem andern Unterrichte beibringen. Es müssen also Bemerkungen seyn, welche entweder zum gehörigen Verständnisse des zu ertheilenden Unterrichtes, oder doch dazu nothwendig sind, um den Anfänger geneigt zu machen, Aufmerksamkeit und Fleiß auf diesen Unterricht zu verwenden. So pflegt man \zB\ in der Einleitung zur Raumwissenschaft oder Geometrie einige arithmetische Sätze,~\RWSeitenw{2}\ die zum Verständnisse gewisser geometrischer nothwendig sind, oder etwas über den Nutzen der Geometrie \udgl\ vorauszuschicken.
\item Es müssen endlich Wahrheiten seyn, die, wenn sie einerseits mit dem abzuhandelnden Gegenstande in \RWbet{Verbindung} stehen, andererseits doch auch keine \RWbet{Belehrungen} über \RWbet{ihn selbst} enthalten. Denn wäre dieses der Fall, so würde ihr Vortrag, als ein Bestandtheil des schon \RWbet{angefangenen} Unterrichtes, nicht aber erst als eine vorläufige \RWbet{Einleitung} in denselben betrachtet werden müssen. So gehört \zB\ die Erklärung des Begriffes vom Raume, und noch offenbarer die Erklärung der Begriffe: Linie, Fläche, Körper \usw\ nicht mehr in die Einleitung zur Geometrie, sondern schon in den Vortrag selbst, weil man schon Unterricht über den Raum ertheilt, wenn man den Begriff vom Raume, und noch mehr, wenn man die Begriffe von Linie, Fläche, Körper \usw\ erklärt.
\end{aufza}\par

Die \RWbet{erste} aus diesen drei Bedingungen enthält den Grund, warum von solchen Wahrheiten \RWbet{überhaupt einmal}; die \RWbet{zweite}, warum von ihnen gerade \RWbet{bei diesem} Unterrichte; die \RWbet{dritte}, warum von ihnen eben in der \RWbet{Einleitung} gesprochen wird.\par
Diese allgemeinen Bemerkungen zeigen, daß in der Einleitung zu einer Wissenschaft vornehmlich folgende Stücke mit allem Fug und Recht abgehandelt werden:
\begin{aufza}[a)]
\item Die Erklärung des \RWbet{Begriffes} der Wissenschaft;
\item die Darstellung des \RWbet{Nutzens};
\item die Anzeige ihrer \RWbet{Hülfswissenschaften};
\item die Anzeige der wichtigsten \RWbet{Bücher}, die über sie geschrieben worden sind (\RWbet{Literatur});
\item der \RWbet{Plan} und die \RWbet{Eintheilung}, die man bei ihrem Vortrage beobachten will;
\item die Regeln, die beim Vortrage dieser Wissenschaft wegen der eigenthümlichen Natur ihres Gegenstandes noch nebst den gewöhnlichen Regeln, welche man über den Vortrag einer jeden Wissenschaft aufstellt, zu beobachten sind; \uam~\RWSeitenw{3}
\end{aufza}\par
Da ich nun gegenwärtig auch einen Unterricht, nämlich in der Religionswissenschaft zu ertheilen gedenke, so wird es dienlich seyn, gleichfalls erst eine kurze \RWbet{Einleitung} vorauszuschicken, in der ich
\begin{aufza}[a)]
\item den \RWbet{eigentlichen Begriff} dieser Wissenschaft erklären;
\item den \RWbet{Nutzen ihres Studiums} bestimmen;
\item ihrer \RWbet{Hülfswissenschaften} erwähnen;
\item die \RWbet{wichtigsten Werke}, die über sie bereits geschrieben sind, anzeigen;
\item endlich auch den ohngefähren \RWbet{Plan} meines Vortrages, und eine \RWbet{Uebersicht seiner Haupttheile} vorlegen will.
\end{aufza}
\begin{RWanm}
Viele pflegen auch \RWbet{die Geschichte einer Wissenschaft} in ihre Einleitung aufzunehmen. Dieses däucht mir aber größtentheils zweckwidrig, weil der Anfänger, so lange er noch keine deutliche Kenntniß von dem Inhalte einer Wissenschaft (von den ihr eigenthümlichen Lehrsätzen und Beweisen) hat, die Geschichte der Veränderungen ihres Vortrages (und dieses heißt doch die \RWbet{Geschichte} der Wissenschaft) theils gar nicht zu begreifen vermag, theils doch ohne gehörigen Nutzen vernimmt, indem er noch nicht beurtheilen kann, auf welcher Seite etwa bei jeder der ihm erzählten Streitigkeiten die Wahrheit liegen möge. Mir däucht es daher zweckmäßiger, am Schlusse des Vortrages einer Wissenschaft, oder noch besser, am Schlusse des Vortrages ihrer einzelnen Abschnitte jederzeit die diesen Theil betreffenden historischen Nachrichten in Kürze mitzutheilen. Doch werde ich hier selbst dieses nur selten thun dürfen, um nicht zu weitläufig zu werden.
\end{RWanm}

\RWpar{2}{Begriff der Religionswissenschaft}
Unter dem Namen der \RWbet{Religionswissenschaft}, die man, obwohl schon minder schicklich, auch \RWbet{Religionsphilosophie} oder \RWbet{philosophische Religionslehre} nennt, verstehe ich \RWbet{die Wissenschaft von der vollkommensten Religion}. Damit man diese Erklärung um so richtiger auffassen könne, muß ich erst die Bedeutung der einzelnen in ihr vorkommenden Worte einiger Maßen erläutern.
\begin{aufza}
\item Ich fange von dem bekanntesten, nämlich dem Worte \RWbet{Religion}, an, in Betreff dessen ich hier nur zu bemerken brauche, daß ich unter Religion nicht, wie es häufig geschieht, eine bloße~\RWSeitenw{4} \RWbet{Lehre von Gott}, sondern \RWbet{den Inbegriff aller derjenigen Lehren und Meinungen eines Menschen verstehe, die einen Einfluß auf seine Tugend und Glückseligkeit haben}. Genauer werde ich diesen Begriff im Vortrage der Religionswissenschaft selbst, wohin er eigentlich gehört, bestimmen. S.~\RWparnr{20}. 
\item Bekanntlich gibt es aber sehr viele und verschiedene Religionen; und nicht alle haben einerlei Einfluß auf die Tugend und Glückseligkeit der Menschen. Diejenige aus ihnen also, die unter allen den wohlthätigsten äußert, nenne ich die \RWbet{vollkommenste}.
\item Bei dem Worte \RWbet{Wissenschaft} müssen wir drei verschiedene Bedeutungen unterscheiden:
\begin{aufzb}
\item \RWbet{erstlich die subjective}; in der es eben so viel als das Wort \RWbet{Kenntniß} bedeutet. In diesem Sinne nimmt man das Wort, wenn man \zB\ sagt: ich habe Wissenschaft davon; oder: ich habe keine Wissenschaft davon; oder: dieser Mensch besitzt viele Wissenschaften; oder: die Wissenschaften bilden den Geist des Menschen; \usw\
\item \RWbet{die objective, aber weitere Bedeutung}, wo ich darunter einen Inbegriff aller über einen und denselben Gegenstand bekannten und merkwürdigen Behauptungen verstehe, wenn diese so geordnet sind, daß sie in Jedem, der sie in dieser Anordnung durchdenkt, die Ueberzeugung von ihrer Wahrheit bewirken, gleichviel, ob er auch immer den eigentlichen \RWbet{Grund dieser Wahrheit} erfahre oder nicht; --
\item \RWbet{endlich die objective engere Bedeutung}, wo ich darunter nur einen Inbegriff aller über einen und eben denselben Gegenstand bekannten und merkwürdigen Behauptungen verstehe, wenn diese so geordnet sind, daß sie bei Jedem, der sie in dieser Anordnung durchdenkt, nicht nur die Ueberzeugung von ihrer Wahrheit bewirken, sondern ihn auch den \RWbet{Grund} dieser Wahrheit, so oft es möglich ist, einsehen lassen. -- 
\end{aufzb}
\begin{RWanm}
Man nennt die Bedeutung, in der das Wort \RWbet{Wissenschaft} in b) und c) genommen wird, eine \RWbet{objective}, weil hier unter Wissen\RWSeitenw{5}schaft ein gewisser \RWbet{Inbegriff von Wahrheiten} verstanden wird, ohne vorauszusetzen, ob diese Wahrheiten von Jemand, \di\ von einem \RWbet{Subjecte} wirklich erkannt werden. Hieraus ist zugleich zu entnehmen, warum die \RWbet{erst} angeführte Bedeutung eine \RWbet{subjective} heißt. Eine \RWbet{Kenntniß} nämlich kann nur gedacht werden als vorhanden in einem \RWbet{Subjecte}.
\end{RWanm}
\item Nur in der dritten \RWbet{engern} Bedeutung nehme ich das Wort \RWbet{Wissenschaft} in meiner obigen Erklärung. Einen Inbegriff von Behauptungen also, die so geordnet sind, daß sie zwar wohl \RWbet{Ueberzeugung} bewirken, aber doch nirgends den eigentlichen \RWbet{Grund}, auf dem ihre Wahrheit beruht, zu erkennen geben, ob er sich gleich hie und da nachweisen ließe, nenne ich noch keine \RWbet{Wissenschaft} im strengsten Sinne des Wortes.
\item Hieraus ergibt sich nun deutlich, was ich mir unter der \RWbet{Religionswissenschaft} denke. Sie ist mir ein Unterricht in der vollkommensten Religion, \dh\ in denjenigen Lehren, welche die Tugend und Glückseligkeit des Menschen am allermeisten befördern, und zwar ein solcher Unterricht, dabei man den eigentlichen Grund der vorgetragenen Wahrheiten, wenn auch nicht immer, doch so oft es möglich ist, angibt.
\end{aufza}

\RWpar{3}{Rechtfertigung dieses Begriffes}
Es ist nöthig, der jetzt gegebenen Erklärung der Religionswissenschaft noch einige Bemerkungen beizufügen, welche zur \RWbet{Rechtfertigung} derselben dienen werden.
\begin{aufza}
\item Es könnte nämlich bezweifelt werden, ob der im vorigen §, unter 3, b) und c) angenommene Unterschied zwischen einer Wissenschaft im weitern und engern Sinne auch in der Wirklichkeit bestehe. Auf den ersten Blick könnte man vielmehr glauben, daß es, um \RWbet{Ueberzeugung} von einer Wahrheit zu bewirken, nothwendig sey, auch ihren \RWbet{eigentlichen Grund} anzugeben; und wenn dieß wäre, dann würde freilich kein Unterschied zwischen der Wissenschaft im weitern und engern Sinne bestehen, indem auch jene, weil sie doch gleichfalls Ueberzeugung hervorbringen soll, den eigentlichen Grund einer jeden Wahrheit nachweisen müßte. Allein so ist es nicht; denn eine nähere~\RWSeitenw{6}\ Betrachtung zeigt, es sey eben nicht nöthig, daß man, um Ueberzeugung von einer Wahrheit zu bewirken, immer den eigentlichen Grund, auf welchem sie beruhet, aufdecke. So kann man \zB\ von der Wahrheit, daß es im Winter kälter sey als im Sommer, eine sehr sichere Ueberzeugung schon durch Berufung auf das bloße Gefühl, noch mehr durch Hinweisung auf den Thermometerstand bewirken. Aber berührt man wohl da den eigentlichen Grund, warum es im Winter kälter ist, als im Sommer? -- Eben so kann man Jeden, auch selbst den blödesten Menschen, von der Wahrheit, daß die gerade Linie die kürzeste zwischen zwei Puncten sey, sehr sicher überzeugen, wenn man ihn auffordert, einen Bindfaden zwischen zwei Puncten auszuspannen, und ihn bemerken läßt, wie dieser Faden, je straffer er ihn anzieht, \dh\ je kürzer er ihn macht, um desto vollkommener die Lage der geraden Linie zwischen den beiden Puncten annimmt. Aber wird ihm auf diese Art wohl der eigentliche Grund, warum die gerade Linie die kürzeste sey, zum Bewußtseyn gebracht? -- Daß Lügen Unrecht sey, wird Jeder einleuchtend finden, sobald wir ihm nur ein einzelnes Beispiel von einer Lüge erzählen, oder ihn an sein eigenes Urtheil in Fällen, wo er belogen ward, erinnern; gleichwohl erfährt er auf diese Weise noch gar nicht, warum Lügen unerlaubt sey. -- Daß ein Mann, der uns eine sittlich zuträgliche Lehre im Namen Gottes vorträgt, und zur Bestätigung seiner göttlichen Sendung die außerordentlichsten Thaten verrichtet, \zB\ Todte erweckt \udgl , allerdings Glauben verdiene, sieht wohl ein Jeder auch ohne alle Beweise ein; aber den eigentlichen Grund, warum wir dieses thun sollen, zu entwickeln, dürfte nicht völlig so leicht seyn. -- Aus diesen Beispielen erhellet zur Genüge, daß Ueberzeugung von einer Wahrheit bewirket werden könne, ohne den eigentlichen Grund derselben anzugeben, folglich bestehet auch der oben aufgestellte Unterschied zwischen der Wissenschaft in weiterer und in engerer oder strengerer Bedeutung dieses Wortes.
\item Gelegenheitlich mag hier noch angemerkt werden, daß wir die eine oder die mehren Wahrheiten, durch deren Betrachtung die bloße \RWbet{Erkenntniß} einer bestimmten anderen Wahrheit bewirkt wird, ihren \RWbet{Erkenntnißgrund} oder auch wohl den \RWbet{sub}\RWSeitenw{7}\RWbet{jectiven Grund} derselben nennen. Zum Unterschiede von diesem nenne ich die eine oder die mehren Wahrheiten, welche das Warum einer bestimmten anderen enthalten, den \RWbet{eigentlichen} oder den \RWbet{objectiven} Grund derselben. Eine Reihe von Sätzen, durch welche ein bloßer Erkenntnißgrund einer Wahrheit angegeben, und also bewirket wird, daß derjenige, der diese Reihe von Sätzen durchdenkt, die Wahrheit anerkenne, nenne ich einen \RWbet{Beweis}, insonderheit einen bloß \RWbet{subjectiven Beweis} oder auch eine bloße \RWbet{Gewißmachung}. Eine Reihe von Sätzen dagegen, die uns den objectiven Grund einer Wahrheit angibt, nenne ich eine \RWbet{Begründung}, oder einen objectiven oder streng \RWbet{wissenschaftlichen Beweis} derselben.
\item Wenn das Wort Wissenschaft in einer von den zwei \RWbet{objectiven} Bedeutungen genommen werden soll, so kann man die \RWbet{Wissenschaft an sich} von einer \RWbet{Darstellung} derselben unterscheiden. Die \RWbet{Darstellung einer Wissenschaft}, die auch ein \RWbet{Unterricht} in ihr, ein \RWbet{Vortrag} oder \RWbet{Lehrbegriff} derselben heißt, ist eine (schriftlich entworfene oder bloß mündlich vorgetragene, oder auch nur gedachte) Reihe von Sätzen, die in der Absicht gewählt und angeordnet wurden, um einem Jeden, der sie in dieser Anordnung durchdenkt, die Ueberzeugung von ihrer Wahrheit beizubringen; und (wenn es die Darstellung einer Wissenschaft im engern Sinne seyn soll) ihn auch zugleich den \RWbet{Grund} ihrer Wahrheit einsehen zu lassen. Ob aber, und in welchem Grade diese Absicht wirklich erreicht worden sey, bleibt dahingestellt; auch wenn sie mehr oder weniger verfehlt worden wäre, würden wir den Inbegriff jener Sätze doch einen \RWbet{Lehrbegriff}, nämlich einen mehr oder weniger \RWbet{fehlerhaften} Lehrbegriff nennen. \RWbet{Von der Wissenschaft an sich} mag es in Betreff eines jeden Gegenstandes nur eine \RWbet{einzige} geben, indem es wohl nur eine \RWbet{einzige} Auswahl und Anordnung von Sätzen gibt, bei welcher der Zweck der \RWbet{Ueberzeugung}, und vollends jener der \RWbet{Erkenntniß ihres Grundes} am Besten erreicht werden kann. Der \RWbet{Darstellungen} aber gibt es begreiflicher Weise sehr viele, und eine ist mehr oder minder vollkommen als die andere. Auch in dem Unterrichte, der hier ertheilt werden soll, wird eigentlich nicht die \RWbet{Religionswissenschaft an sich}, son\RWSeitenw{8}dern nur ein \RWbet{bestimmter Lehrbegriff} derselben vorgetragen; nämlich derjenige, der dem Lehrer der beste scheint, der aber gleichwohl noch seine Unvollkommenheiten und Mängel haben wird.
\item Wie in der Folge erwiesen werden soll, ist die vollkommenste aus allen nicht nur vorhandenen, sondern auch nur gedenkbaren Religionen \RWbet{die katholisch-christliche}. Wir könnten also, da die Religionswissenschaft ein Unterricht in der vollkommensten Religion seyn soll, auch sagen, daß sie die \RWbet{Wissenschaft von der katholisch-christlichen Religion} sey. Allein es wäre nicht zweckmäßig, von diesem Satze als \RWbet{Erklärung} auszugehen, indem wir auf diese Art den \RWbet{Nutzen}, den das Studium der Religionswissenschaft gewährt, nicht so leicht zeigen könnten, als wir es jetzt vermögen. Denn wenn wir unter der Religionswissenschaft einen Unterricht in der \RWbet{katholischen} Religion verstünden; so würden wir erst dann darthun können, daß das Studium der Religionswissenschaft von Nutzen sey, wenn wir erwiesen hätten, daß die katholische Religion einen Werth habe. Verstehen wir aber unter der Religionswissenschaft einen wissenschaftlichen Unterricht in derjenigen Religion, welche die \RWbet{vollkommenste} ist: so wird uns Jeder ohne Schwierigkeit zugestehen, daß dieses Studium seinen Nutzen haben werde.
\end{aufza}

\RWpar{4}{Nutzen der Religionswissenschaft}
Die Vortheile, welche das Studium der Religionswissenschaft solchen, die dazu \RWbet{Fähigkeit} haben, und es auf die \RWbet{gehörige Weise} betreiben, gewährt, sind von so großer Wichtigkeit, daß es zu seiner Empfehlung wahrlich keiner Uebertreibung bedarf. Ich will sie daher mit einer solchen Mäßigung beschreiben, daß Jeder fühlen mag, wie ich hier eher zu wenig, als zu viel sage.\par
Ich sage aber nicht, daß dieses Studium \RWbet{Allen ohne Ausnahme} zuträglich, geschweige denn \RWbet{nothwendig} wäre; sondern ich sage bloß, daß es \RWbet{denjenigen Menschen ersprießlich sey, die dazu Fähigkeit haben, und es auf die gehörige Weise betreiben}. Denn daß es Leuten, die entweder~\RWSeitenw{9}\ gar keine \RWbet{Fähigkeit} zu einem wissenschaftlichen Unterrichte, nämlich kein wissenschaftliches Talent, besitzen; oder bei denen dieses Talent noch gar nicht entwickelt worden ist; oder die sich nicht mit dem gehörigen \RWbet{Fleiße} und mit der nöthigen \RWbet{Beharrlichkeit} auf dieses Studium verlegen, statt heilsam nur \RWbet{schädlich und verderblich} werden könne: will ich so wenig in Abrede stellen, daß ich es vielmehr selbst behaupte, und davor warne. Es ist nämlich nicht zu verwundern, wenn Leute solcher Art Vieles nur \RWbet{halb auffassen}, Vieles ganz \RWbet{mißverstehen}, die einzelnen aus dem Zusammenhange gerissenen Sätze, die sie hie und da auffangen, nicht \RWbet{richtig anzuwenden} wissen, und so zuletzt nur in \RWbet{Unruhe und Verwirrung} gerathen, oder veranlaßt werden, Verschiedenes nur zur Beschönigung ihrer verderblichen Leidenschaften zu mißbrauchen.\par
Von einem \RWbet{gehörig betriebenen Studium der Religionswissenschaft} aber hat der Talentvolle keine dergleichen Nachtheile zu befürchten; es wird ihm vielmehr \RWbet{bestimmte Vortheile einer doppelten Art} gewähren: einige, die es mit einem jeden  wissenschaftlichen Studium \RWbet{gemein} hat; andere, die demselben \RWbet{eigenthümlich} sind.
\begin{aufza}
\item Mit jeder Wissenschaft hat es die Religionswissenschaft gemein, daß sie demjenigen, der sie mit Fleiß studirt, \RWbet{eine höchst schätzbare Uebung im richtigen Denken, und ein sehr edles Seelenvergnügen gewähret}.
\begin{aufzb}
\item Die \RWbet{Kraft zu Denken} ist gewiß das schätzbarste Kleinod, welches der Schöpfer unserm Geschlechte verliehen. Daher ist auch der wichtigste und \RWbet{schätzbarste Vorzug}, den ein Mensch vor Andern besitzen kann, mit Ausnahme der Tugend, der Besitz einer \RWbet{geübteren Denkkraft, als sie Andere haben}; wie denn dieser Vorzug zugleich auch das wirksamste Beförderungsmittel der Tugend selbst, und also nicht selten die vermittelnde Ursache zur Erreichung des höchsten Vorzuges zu seyn pflegt, den Menschen überhaupt besitzen können. -- Das Mittel nun, sich diesen Vorzug einer geübteren Denkkraft eigen zu machen, das sicherste und ausgiebigste~\RWSeitenw{10}\ ist die \RWbet{Erlernung der Wissenschaften}, besonders solcher, in denen viele Begriffe entwickelt, lange Reihen von Schlüssen gebildet, täuschende Trugschlüsse aufgedeckt werden, \usw\ Solche Wissenschaften sind die mathematischen, die \RWbet{Philosophie} und die \RWbet{Religionswissenschaft}. Zwar möchte man glauben, \RWbet{Erlernung der Logik}, \dh\ der Lehre von den Regeln des \RWbet{richtigen Denkens} wäre ein schon hinreichendes Mittel, zum richtigen Denken zu gelangen; allein es verhält sich mit dem \RWbet{richtigen Denken} wie mit dem richtigen \RWbet{Sprechen} und mit mehren andern Dingen, die einer Fertigkeit bedürfen. Wie \RWbet{Logik} die Regeln des richtigen Denkens aufstellt, so die \RWbet{Sprachlehre} die Regeln des richtigen Sprechens. Wie aber Niemand aus der Sprachlehre allein eine Sprache erlernt, sondern in dieser Sprache erst sehr Vieles \RWbet{lesen} oder hören und endlich auch \RWbet{selbst zu sprechen} versuchen muß: so lernt auch Niemand aus der Logik allein richtig und geläufig denken, wenn er nicht nebenbey noch eine große Menge richtiger Gedanken \RWbet{Anderen} nachgedacht, und dann auch selbst versucht hat, über Verschiedenes zu denken. Jenes geschieht nun zwar bei einem \RWbet{jeden Unterrichte}, der nicht unrichtig und falsch ist, vornehmlich aber geschieht es bei einem \RWbet{wissenschaftlichen}, indem sich ein solcher überall das \RWbet{Warum} anzugeben bemühet, und eben deßhalb sich in die Zergliederung einer Menge von Begriffen einlassen, lange Reihen von Schlüssen anstellen, und eine Menge von Trugschlüssen aufdecken muß. Je mehr dieß nun in einer Wissenschaft der Fall ist, desto größer der Vortheil, welchen sie unserer Denkkraft gewährt. In der Religionswissenschaft gibt es eine bedeutende Anzahl von Begriffen, die entwickelt, Schlußreihen von sehr beträchtlicher Länge, die durchgeführt, und sehr viele, überaus täuschende Trugschlüsse, die in ihrer Nichtigkeit aufgedeckt werden müssen.
\begin{RWanm} 
Die \RWbet{Fertigkeit im richtigen Denken}, welche die \RWbet{Religionswissenschaft} gewährt, dürfte in gewisser Rücksicht noch einen Vorzug vor derjenigen haben, die man aus \RWbet{andern} Wissenschaften schöpft. Denn weil die meisten Gegenstände, die man in der Religionswissenschaft betrachtet, auch im gemeinen Leben alltäglich wiederkehren, und von der~\RWSeitenw{11}\ höchsten \RWbet{Wichtigkeit} sind: so wird die \RWbet{Fertigkeit im Denken}, die man aus ihrer Betrachtung sich beigelegt hat, auf das wirkliche Leben \RWbet{viel leichter und sicherer angewendet} werden, als es der Fall ist, wenn man seine Denkkraft an der Betrachtung bloß solcher Gegenstände geübt hat, die im geselligen Leben nur selten vorkommen, oder von keiner so großen Wichtigkeit sind. Man hat dann wohl eine Fertigkeit, über \RWbet{Dinge gewisser Art} richtig zu urtheilen, Trugschlüsse \RWbet{einer gewissen Art} schnell zu bemerken \usw ; aber auf Gegenstände einer andern Art, gerade auf solche, die uns am Häufigsten vorkommen, erstreckt sich diese Fertigkeit vielleicht sehr wenig.
\end{RWanm}
\item Nächst dem Vergnügen, welches uns das Bewußtseyn der \RWbet{Tugend} gewährt, ist dasjenige, so \RWbet{aus dem Gewahrwerden unserer Kräfte, besonders geistiger}, entspringt, das höchste und edelste. Oder was kann dem Menschen wichtiger und erfreulicher seyn, was ihm mehr Hoffnung auf eine glückliche Zukunft gewähren, als wenn er sich der in ihm lebenden Kräfte, besonders der \RWbet{geistigen}, bewußt wird? -- Dieses Vergnügen aber kann uns nicht anders zu Theil werden, als wenn wir \RWbet{Wirkungen} von unsern Kräften sehen, wenn wir durch ihre Anwendung etwas zu Stande bringen. Nur durch den Anblick dessen, was wir durch unsere Kräfte bewirkt haben, erhalten wir eine anschauliche Vorstellung von ihrer Größe. Daher denn die Freude, die schon das kleine Kind darüber empfindet, wenn es ihm eben zum ersten Male gelingt, einige Schritte zu thun, ohne gestrauchelt zu haben; daher das Frohlocken des Knaben, wenn er zum ersten Male den Hügel in seiner Nachbarschaft erklommen; daher aber auch das Hochgefühl eines Correggio, wenn er bei der Vergleichung eines Gemäldes von eigener Hand mit einer der Arbeiten Raphaels erkennet, -- auch er sey ein Mahler! -- Wie jedoch unter allen Kräften des Menschen die Kraft des Denkens die wichtigste ist; so ist es natürlich, daß auch das Bewußtseyn keiner andern Kraft uns angenehmer seyn könne, als das \RWbet{Bewußtseyn einer mehr als gemeinen Kraft zu denken}. Dieses Bewußtseyn nun ist es, das wir uns durch die Erlernung der Wissenschaften erwerben. Bei jeder \RWbet{neuen} Wahrheit, die wir da kennen lernen, bei jeder \RWbet{deutlichen Aufklärung} eines bisher nur dunkel~\RWSeitenw{12} von uns gedachten Begriffes, bei jeder glücklichen Entdeckung der \RWbet{verborgenen Gründe} einer an sich uns schon \RWbet{bekannten} Wahrheit, bei jeder Enthüllung eines täuschenden \RWbet{Trugschlusses}, bei jedem \RWbet{neuen Fortschritte}, ja auch bei jeder \RWbet{Erinnerung} an bereits \RWbet{gemachte} Fortschritte, an die schon überstiegenen Schwierigkeiten \usw\ empfinden wir ein Vergnügen, weil uns da anschaulich wird, wie unsere Denkkraft Manches doch schon geleistet und wie viel mehr Fertigkeit im Denken wir jetzt besitzen, als wir vor Jahren besaßen, oder als viele Andere haben. Dieses Vergnügen ist um so schätzbarer, weil es als ein \RWbet{rein geistiges} den Körper nicht so heftig angreift und erschöpft, als es bei den Vergnügungen der Sinne der Fall zu seyn pflegt; weil es ferner durch \RWbet{Wiederholung} nicht zum Ekel und Ueberdruß wird, wie dieß bei sinnlichen Vergnügungen geschieht; weil es endlich auch zu seinem Genusse gar keines \RWbet{Aufwandes}, keiner besonderen \RWbet{Umstände} und \RWbet{Gelegenheiten} bedarf, wie dieses alles bei den Sinnesvergnügungen der Fall ist.
\begin{RWanm} 
Die \RWbet{Religionswissenschaft} dürfte sich auch in dieser Hinsicht vor mancher \RWbet{andern} Wissenschaft eines nicht unbedeutenden Vorzuges rühmen, indem die Rückerinnerung an unsere Kenntnisse in ihr durch die Geschäfte des Lebens so oft gleichsam von selbst herbeigeführt wird. Denn Begriffe und Wahrheiten, welche in das Gebiet der Religion gehören, kommen ja täglich und stündlich vor, während Begriffe aus andern Wissenschaften, \zB\ der Mathematik, gewiß viel seltener angeregt werden. So oft \zB\ von einer \RWbet{Pflicht} die Rede ist, so wird derjenige, der aus dem Unterrichte in der Religionswissenschaft eine deutlichere Einsicht in die Natur der menschlichen Pflichten, als hundert andere Menschen besitzt, eine Gelegenheit haben, sich dieses Vorzuges mit Freuden zu erinnern.
\end{RWanm}
\end{aufzb}
\item Nebst diesen Vortheilen, welche das Studium unserer Wissenschaft mit jeder andern mehr oder weniger gemein hat, gewährt es auch solche, die demselben \RWbet{eigenthümlich} sind. Durch eine gründliche Erlernung der Religionswissenschaft gewinnt nämlich überaus viel unser \RWbet{Vertrauen} sowohl, als auch die \RWbet{Achtung}, die wir der vollkommensten Religion zu allen Zeiten des Lebens schenken sollen.
\begin{aufzb}
\item Nothwendig ist es zwar keineswegs, daß wir, um Zutrauen zur vollkommensten Religion zu fas\RWSeitenw{13}sen, erst einen \RWbet{wissenschaftlichen} Unterricht in ihr empfangen haben müßten; sondern auch ein \RWbet{gemeiner} Unterricht kann hiezu genügen. Denn es ist allgemein, um Vertrauen zu einer Lehre zu fassen, genug zu sehen, \RWbet{daß} es so sey, wie diese Lehre aussagt; und man braucht eben nicht zu wissen, \RWbet{warum} es so sey. Aber gewiß ist doch, daß unser Vertrauen \RWbet{wächst}, wenn auch noch dieses Letztere hinzukommt, \dh\ wenn wir auch deutlich einsehen lernen, \RWbet{warum} etwas so, und nicht anders sey. Dieses geschieht nun in Betreff der meisten und wichtigsten religiösen Wahrheiten, sobald wir einen \RWbet{wissenschaftlichen} Unterricht in der vollkommensten Religion erhalten; und somit gewinnt durch einen solchen Unterricht unser Vertrauen zur Religion. 
\item  Wir Menschen sind nämlich gewohnt, \RWbet{den Werth eines Gutes} insgemein um desto höher zu schätzen, \RWbet{je größer die Mühe} war, welche uns sein Erwerb gekostet. Wir pflegen ferner das, was uns anschaulich macht, daß wir gegenwärtig auf einer höhern Stufe der Vollkommenheit stehen, als es ehedem war, oder als es bei Andern noch jetzt ist, höher zu schätzen, als was uns zu dieser Bemerkung keine Gelegenheit gibt. In diesen zwei Wahrheiten liegt nun ein doppelter Grund, weßhalb derjenige, der sich eine \RWbet{wissenschaftliche Kenntniß} der Religion erworben hat, mehr \RWbet{Achtung} für sie im Herzen fühlen wird, als bei übrigens gleichen Umständen ein Anderer. Er nämlich besitzt an seiner Religion eine \RWbet{Kenntniß}, deren Erlangung ihm nicht wenig Mühe und Anstrengung gekostet hat. Um desto \RWbet{wichtiger} erscheint sie ihm, um desto williger räumt er ihr den gehörigen Einfluß auf seine Handlungsweisen ein. Durch diese Kenntniß wird ihm so oft anschaulich, daß er jetzt eine bedeutend höhere Denkfertigkeit besitze, als er in früheren Jahren gehabt, oder als viele seiner Genossen noch jetzt haben; die Freude, die ihm dieß verursacht, gehet auf die Vorstellung von der Religion selbst über, und macht ihm auch diese werther.~\RWSeitenw{14}
\end{aufzb}
\item Ich könnte noch weiter gehen, und nebst der \RWbet{Nützlichkeit} eines gründlichen Studiums der Religionswissenschaft, die ich bisher behauptete, sogar eine gewisse \RWbet{Nothwendigkeit} desselben \RWbet{für solche Menschen} behaupten, \RWbet{die eine wissenschaftliche Bildung entweder bereits erhalten haben, oder erst noch erhalten sollen}. Ich nenne aber eine \RWbet{wissenschaftliche} oder \RWbet{gelehrte Bildung} oder \RWbet{Erziehung} eine jede solche, bei der unter Anderm auch manche \RWbet{wissenschaftliche} Kenntnisse beigebracht wurden. Es läßt sich darthun, daß bei Personen, die eine solche Bildung erhielten, die Religion weder das \RWbet{nöthige Zutrauen}, noch die \RWbet{nöthige Achtung} erhalten könne, wenn sich zu ihren übrigen gelehrten Kenntnissen nicht frühzeitig auch eine gelehrte Kenntniß der Religion gesellet.
\begin{aufzb}
\item Personen, die eine gelehrte Bildung genossen haben, die eben deßhalb eine größere Fertigkeit im Denken besitzen, und darum in diesem Geschäfte auch ein eigenes Vergnügen finden, lassen den Vorrath ihrer Begriffe nie ganz unbearbeitet liegen; es drängt sie vielmehr, darüber nachzudenken, das Eine mit dem Andern zu vergleichen, und auf die weiteren Folgen zu sehen, die sich aus dieser und jener Behauptung ergeben. Dieß thun sie denn auch, und zwar ganz vornehmlich mit dem Vorrathe ihrer religiösen Begriffe. Eine fast unvermeidliche Folge hievon ist aber, daß sie auf eine Menge \RWbet{scheinbarer Widersprüche und Zweifel} gerathen, welche dem Ungebildeten nie in den Sinn zu kommen pflegen. Diese Zweifel werden dann noch durch Umgang mit andern Religionsverwandten, mit Menschen von dem entschiedensten Unglauben, durch das Lesen religionswidriger Bücher und durch andere dergleichen Umstände vermehret. Offenbar kann nur ein \RWbet{wissenschaftlicher} Unterricht in der Religion, bei dem man bemühet war, von einer jeden aufgestellten Behauptung nicht nur zu zeigen, \RWbet{daß}, sondern auch \RWbet{warum} sie richtig sey, der Entstehung solcher Zweifel theils vorbeugen, theils doch die besten Mittel zu ihrer Auflösung an die Hand geben. -- Personen von einer gelehrten Bildung verlangen um so mehr nach diesem~\RWSeitenw{15} \RWbet{Warum}, weil sie bereits gewohnt sind, dieß Warum auch in andern Fächern des menschlichen Wissens zu erfahren. Verbirgt es sich ihnen nur bei den Lehren der Religion: so fühlen sie sich versucht, zu argwöhnen, daß diese Lehren vielleicht gar keinen Grund haben; ihr \RWbet{Zutrauen} zu denselben sinkt dann von Tage zu Tage. Gleichwohl ist es gewiß, daß die Religion ihren wohlthätigen Einfluß auf unser Herz nur dann erst äußere, nur dann erst einen Bestimmungsgrund bei unsern Handlungen und eine Richtschnur bei unsern Hoffnungen und Besorgnissen abgeben könne, wenn wir ein unbeschränktes Vertrauen in ihre Lehren setzen. Denn wenn wir zweifeln, und nach Art der Zweifler uns bald auf diese, bald jene Seite neigen: so ist unser Glaube meistens zu schwach, um uns zur Folgsamkeit zu bestimmen, und doch nicht schwach genug, um unsere Unfolgsamkeit ganz zu rechtfertigen; und so sind wir schlimmer daran, als die, welche gar nicht glauben. So nothwendig also ist es, daß jeder Gebildete einen wissenschaftlichen Unterricht in der Religion erhalte, weil diese widrigenfalls nicht das ihr nöthige Zutrauen behaupten kann. 
\item Wer in allen Fächern des menschlichen Wissens gelehrte Kenntnisse mit vieler Mühe gesammelt, und nur dem Studium der Religion nie eine \RWbet{eigene Zeit und Anstrengung} gewidmet hat, dem muß es nothwendig scheinen, als ob die Religion viel minder wichtig wäre, als alles Andere. Wer über tausend Dinge mit einer gewissen \RWbet{gelehrten Einsicht} zu sprechen vermag, nur über \RWbet{Puncte der Religion} noch an den unvollständigen Begriffen seiner Kindheit hängt, nur hier eben nichts Mehres zu sagen weiß, als der gemeinste Landmann, der wird allmählich anfangen, sich seiner religiösen Begriffe zu schämen. Soll dieß verhütet werden, so muß er, wie über andere Gegenstände des menschlichen Wissens, so auch über die Gegenstände der Religion einen gelehrten Unterricht erhalten.
\end{aufzb}
\begin{RWanm}
Was ich in diesem Absatze sagte, hat die \RWbet{Erfahrung selbst} nur allzusehr bestätiget. Auffallend war es, wie mit der gelehrten Bil\RWSeitenw{16}dung, die sich seit einigen Jahrhunderten in den verschiedenen Staaten Europa's allmählich ausgebreitet hatte, auch ein so allgemeiner Unglaube um sich griff, daß beinahe schon Niemand mehr, als nur der ungebildete gemeine Mann noch mit voller Herzlichkeit an dem ererbten Glauben seiner Väter hing, während die Uebrigen, die sich \RWbet{Gebildete} nannten, theils Zweifler, theils entschiedene Ungläubige waren. Kaum läßt sich diese traurige Erscheinung aus einem andern Grunde erklären, als aus der Vernachlässigung eines den Fortschritten der Zeit angemessenen gelehrten Vortrages der Religion. Denn wirklich, seitdem man diesen Fehler in einigen Staaten zu verbessern angefangen, dürfte sich nachweisen lassen, daß wieder mehr Glaube unter dem gebildeten Theile der Nation herrsche, und daß man allmählich aufhört, sich des Bekenntnisses, daß man ein Christ sey, zu schämen.
\end{RWanm}
\item Nebst dem bisher entwickelten Nutzen, welcher dem Studium der Religionswissenschaft \RWbet{wesentlich} eigen ist, gewähret dasselbe noch häufig gewisse \RWbet{zufällige Vortheile}, die auf der \RWbet{Unvollkommenheit desjenigen Religionsunterrichtes} beruhen, der \RWbet{nicht auf wissenschaftliche Art} ertheilt wird, den ich mithin den \RWbet{gemeinen} nennen möchte. Diese Vortheile, wenn sie gleich zufällig sind, weil sie bei einer \RWbet{verbesserten Beschaffenheit des gemeinen Unterrichtes} wegfallen müßten, sind doch für Zeiten und Orte, wo diese Verbesserungen noch nicht getroffen worden, nicht minder wichtig und beherzigungswerth. In der bisherigen Betrachtung nämlich dachte ich mir Menschen, welche noch vor Empfang des \RWbet{wissenschaftlichen} einen \RWbet{gemeinen} Unterricht in der Religion erhalten hatten, der, obgleich nicht den Grund der abgehandelten Wahrheiten angebend, doch \RWbet{vollständig} und \RWbet{überzeugend} war, und ich zeigte die Vortheile an, die ihnen daraus erwachsen, wenn zu diesem \RWbet{gemeinen} Unterrichte nun noch ein \RWbet{wissenschaftlicher} hinzukommt. Begreiflich müssen die Vortheile zahlreicher seyn, wenn man nicht einmal voraussetzen darf, daß der \RWbet{gemeine} Unterricht in der Religion die gehörige Vollständigkeit und Ueberzeugungskraft hatte. Dieses ist aber wirklich sehr oft der Fall. Denn der \RWbet{gemeine} Unterricht ist
\begin{aufzb}
\item \RWbet{selten vollständig genug}. Man trägt nicht \RWbet{alle} Lehren der vollkommensten Religion vor, oder man stellt sie doch nicht in das gehörige Licht; sondern begnügt sich mit Begriffen, die äußerst mangelhaft, ja selbst unwürdig sind. Man unterläßt es fast durchgängig, die Fruchtbarkeit und~\RWSeitenw{17} den wohlthätigen Einfluß und Gebrauch einer jeden Lehre zu zeigen.
\item Eben so selten pflegt der gemeine Unterricht \RWbet{überzeugend genug} zu seyn. Man verabsäumt es entweder ganz, die \RWbet{Wahrheit} der religiösen Lehrsätze, welche man aufgestellt hat, durch eigene Beweise darzuthun, oder man bringt Beweise vor, die aber nicht zulänglich sind, und deren Schwäche, wenn auch nicht auf der Stelle, doch in spätern Jahren bemerkt wird, und dann zu Zweifeln und zum Unglauben verleitet.
\end{aufzb}
In Fällen dieser Art ist dann der wissenschaftliche Religionsunterricht ein um so dringenderes Bedürfniß; er leistet dem, der ihn empfängt, nebst den schon angegebenen Vortheilen auch noch den wichtigen Dienst, daß er ihn mit den Lehren der vollkommensten Religion erst vollständig bekannt macht und von der Wahrheit derselben beruhigend überzeuget.
\end{aufza}

\RWpar{5}{Ueber die schicklichste Zeit des Studiums der Religionswissenschaft}
Nachdem wir gesehen, daß das Studium der Religionswissenschaft für einen Jeden, der dazu Fähigkeit hat, nützlich, unter besonderen Umständen aber, nämlich für alle Jene, die eine \RWbet{gelehrte Bildung} erhalten, sogar \RWbet{nothwendig} sey: so fragt sich zunächst, \RWbet{welches die schicklichste Zeit für dieses Studium seyn möge}? Und hierauf antworte ich, diese schicklichste Zeit seyen \RWbet{die philosophischen Studienjahre}. Wie nämlich jeder Unterricht, der auf eine wissenschaftliche Art ertheilt werden soll, schon ein gewisses reiferes Alter voraussetzt, wo man zur Aufmerksamkeit bereits gewöhnt worden ist, und eine längere Reihe von Schlüssen bis an ihr Ende zu begleiten vermag, so ist es auch mit der Religionswissenschaft der Fall. Auch hier ist es nöthig, daß man gewisse Vorkenntnisse aus der Weltweisheit und Geschichte zu diesem Unterrichte, wo nicht schon mitbringe, doch wenigstens zu gleicher Zeit mit ihm erhalte. Hieraus ergibt sich denn, daß dieser Unterricht auf keine Weise noch vor Eintritt~\RWSeitenw{18}\ der akademischen Studienjahre (also nicht etwa schon in den Gymnasialjahren) ertheilt werden dürfe. Da dieser Unterricht ferner für einen jeden Menschen, der eine höhere Bildung empfängt, Bedürfniß wird, er mag sich übrigens dem oder jenem Lebensstande widmen: so wird es wohl am Zweckmäßigsten seyn, den \RWbet{wissenschaftlichen Religionsunterricht} in jene Zeit zu verlegen, wo man die künftigen Mitglieder der höhern Lebensstände noch \RWbet{unabgesondert} bei einander hat, um sie mit dem bekannt zu machen, was ihnen \RWbet{gemeinschaftlich} nothwendig ist; mit andern Worten, daß man den Unterricht in der Religionswissenschaft in die \RWbet{philosophischen Studienjahre} verlege.

\RWpar{6}{Hülfswissenschaften der Religionswissenschaft}
So eingeschränkt auch das Wissen des Menschen in vielerlei Hinsichten ist, so ist doch die Menge der Wahrheiten, welche der menschliche Fleiß bereits gesammelt hat, so groß, daß man sie zum Behufe des Gedächtnisses sowohl, als auch zu manchen andern Zwecken \RWbet{in gewisse Fächer abtheilen mußte}. Dieß die Entstehung der \RWbet{verschiedenen Lehrbegriffe}, die man auch \RWbet{Wissenschaften} nennt. So hat man \zB\ der \RWbet{Geometrie} bloß diejenigen Wahrheiten, welche den \RWbet{Raum} betreffen, der \RWbet{Logik} bloß jene, welche die \RWbet{Kunst zu denken} angehen, \usw\ zugetheilt. Obgleich man nun diese \RWbet{Abtheilungen} größtentheils so gemacht hat, wie die Natur der Wahrheiten selbst sie erfordert, \dh\ so, daß man solche Wahrheiten, die mit einander zusammenhängen, einander erläutern, auch in Ein Fach gestellt hat: so gibt es gleichwohl \RWbet{mehre Wissenschaften}, die zum Beweise ihrer Sätze einzelne Wahrheiten erst aus einer andern erborgen müssen. Eine Wissenschaft, die aus keiner andern borgt, heißt eine \RWbet{selbstständige}. Eine solche scheint \zB\ die \RWbet{Metaphysik} zu seyn. Eine Wissenschaft dagegen, die gewisse Wahrheiten aus andern borgen muß, \dh\ gewisse Lehren hat, deren Gründe zugleich in eine andere Wissenschaft gehören, und~\RWSeitenw{19}\ daselbst vorgetragen werden, heißt eine \RWbet{abhängige} Wissenschaft. Die Wissenschaften aber, \RWbet{von denen} sie abhängig ist, heißen ihre \RWbet{Hülfswissenschaften}.\par
Die Religionswissenschaft rühmt sich nun keineswegs, eine der \RWbet{selbstständigen} zu seyn; vielmehr gibt es zwei Wissenschaften, aus denen sie beinahe Alles, was sie lehrt, erborget. Diese zwei Hülfswissenschaften derselben sind: die \RWbet{Weltweisheit} (Philosophie, theoretische sowohl als praktische) und die \RWbet{Geschichte}, namentlich die \RWbet{christliche Kirchengeschichte}.\par
Daß und warum die Religionswissenschaft der jetzt genannten Hülfswissenschaften bedürfe, wird sich erst in der Folge zeigen. Indessen ist es mit der Andeutung dieser Hülfswissenschaften hier gar nicht so gemeint, als ob derjenige, der die Religionswissenschaft studieren will, erst jene beiden sich bekannt machen müßte; man wird vielmehr dasjenige, was hier aus ihnen erborgt werden muß, in so weit, als es dem Zuhörer noch nicht bekannt seyn kann, mit hinlänglicher Umständlichkeit im Vortrage der Religionswissenschaft selbst aus einander setzen.


\RWpar{7}{Lehrbücher der Religionswissenschaft}
\begin{aufza}
\item Die Pflicht eines akademischen Lehrers erheischt, seine Zuhörer auch mit der \RWbet{Literatur} seines Gegenstandes, \dh\ mit den vorzüglichsten Schriften, die über seine Wissenschaft geschrieben worden sind, einigermaßen bekannt zu machen.
\begin{aufzb}
\item Dieses ist \RWbet{einmal} schon darum nothwendig, damit der Anfänger sich nicht genöthiget sehe, bei dem, was ihm sein Lehrer beibringt, für immer stehen zu bleiben; vielmehr Gelegenheit erhalte, sich, wenn er Zeit und Kräfte hat, aus andern Schriften noch weiter auszubilden.
\item Besonders in solchen Fällen, wo der Lehrer seine eigenen, von Andern abweichenden Meinungen vorträgt, ist es doppelte Pflicht, die Meinungen Anderer nicht nur historisch zu~\RWSeitenw{20} erwähnen, sondern auch die Schriften anzuzeigen, in welchen man sie noch weiter ausgeführt findet. Denn so gewiß der Lehrer auch seiner Sache zu seyn glaube, so ist er doch nicht unfehlbar; und es geziemt sich also, daß er den Anfänger in den Stand setze, seine Begriffe mit jenen anderer Männer zu vergleichen, um dann aus eigener Einsicht, wer hier Recht haben mag, zu entscheiden.
\item Daß diese Verbindlichkeit beim Vortrage der \RWbet{Religionswissenschaft} um destso stärker sey, folgt aus der \RWbet{Wichtigkeit} des Gegenstandes, den diese Wissenschaft behandelt.
\end{aufzb}
\item Aus diesem Grunde will denn auch ich gleich jetzt einige der \RWbet{vorzüglichsten Schriften} anführen, welche das Ganze der Religionswissenschaft umfassen; Schriften dagegen, welche nur einzelne Theile derselben betreffen, gedenke ich erst in der Folge gehörigen Orts zu erwähnen, wenn es der Raum verstattet:
\begin{aufzb}
\item \RWlat{\RWbet{Marsilius Ficinus}} scheint einer der Ersten gewesen zu seyn, der \RWlat{de veritate religionis christianae} (beiläufig\editorischeanmerkung{Bolzano verwendet \erganf{beiläufig} im Sinne von \erganf{ungefähr}.} 1520)\RWlit{}{Ficinus1} in systematischer Anordnung schrieb. Ein sehr kleines und unbedeutendes Büchlein, noch nach den Grundsätzen der platonischen Philosophie.
\item Etwas ausführlicher war schon des \RWlat{\RWbet{Ludovicus Vives}} Buch \RWlat{de veritate fidei christianae}, welches zum ersten Mal in Basel 1544 in 12.\editorischeanmerkung{Gemeint ist das Buchformat \erganf{Duodez}.}\ erschien.\RWlit{}{Vives1}
\item Auch \RWbet{Erasmus von Rotterdam}, und \RWbet{Peter Ramus} schrieben über denselben Gegenstand minder bedeutende Werke.
\item Wichtiger aber war das Werk des \RWbet{Philipp Mornäus}, welches er zuerst in französischer Sprache, 1607 aber in Leiden lateinisch herausgab.
\item Doch \RWbet{Hugo Grotius} übertraf alle seine Vorgänger durch sein jetzt noch sehr lesenswerthes, zwar kleines, aber eben so gründlich als faßlich geschriebenes Buch: \RWlat{de veritate religionis christianae}\RWlit{}{Grotius1}, das er im Kerker beiläufig 1620, zunächst zum Behufe für die holländischen Schiffsleute, damit sie das Christenthum auf ihren Reisen zu vertheidigen~\RWSeitenw{21}\ wüßten, in holländischen Versen verfaßte, dann in's Latein übertrug, worauf es auch noch in das Französische, Deutsche, Englische, Schwedische, Dänische, Griechische, Arabische, Persische, Malay'sche und Sinesische übersetzt worden ist, und in der lateinischen Uebersetzung allein mehr als zwanzig Auflagen erlebt hat. Obgleich der Verfasser kein Katholik war, also auch nicht nach katholischen Grundsätzen vorging, so wurde sein Buch doch selbst von Katholiken, \zB\ von dem berühmten Bischofe Bossuet, vom Cardinal Barberin \uA\ sehr hoch geschätzt und empfohlen.
\item Der Bischof \RWbet{Peter Daniel Huetius} schrieb ein weitläufiges und gelehrtes Werk zum Gebrauch für den Dauphin: \RWlat{demonstratio evangelica}\RWlit{}{Huet1} (Paris 1680), das dann noch öfter aufgelegt wurde. Es ist in mathematischer Methode abgefaßt. Die Lehre von den messianischen Weissagungen, wie auch die Untersuchung über die Aechtheit und Glaubwürdigkeit der Bücher des alten und neuen Bundes wird sehr umständlich und mit vieler Gelehrsamkeit abgehandelt. Die Lehren des Christenthums selbst aber werden nur äußerst kurz berührt.
\item Einer der gelehrtesten unter den protestantischen Theologen, \RWbet{Georgius Calixtus}, den seine Zeitgenossen des \RWbet{Kryptokatholicismus} beschuldigten, schrieb gleichfalls ein sehr schönes Werk: \RWlat{de veritate religionis christianae}\RWlit{}{Calixtus1}, worin vornehmlich die Widerlegung der Mohamedaner und Juden sehr lesenswerth ist.
\item Da man im 17\hoch{ten} Jahrhunderte besonders in \RWbet{England} anfing, das Christenthum und alle positive Religionen sehr heftig zu bestreiten (Hobbes, Cherbury, Shaftesbury, Bolingbroke, Tindal, Morgan \uA ): so traten auch gründliche Vertheidiger der guten Sache der göttlichen Offenbarung zuerst ganz vornehmlich in England auf. Zu diesen gehört besonders: \RWbet{Foster}'s Nützlichkeit, Wahrheit und Vortrefflichkeit des Christenthums\RWlit{}{Foster1}; \RWbet{Stackhousen}'s Vertheidigung des Christenthums\RWlit{}{Stackhouse1}; \RWbet{Champelle}'s, \RWbet{Irving}'s und~\RWSeitenw{22}\ \RWbet{Paley}'s Werke, welche fast alle auch in das Deutsche übersetzt worden sind.
\item\stepcounter{enumii} Unter den \RWbet{Franzosen}, die sich gleichfalls durch eigene, in ihrer Mitte aufgetretene Freigeister (worunter besonders \RWbet{Voltaire}) aufgefordert fühlten, die Wahrheit und Göttlichkeit des Christenthumes zu vertheidigen, zeichneten sich vornehmlich aus: \RWbet{Houteville} (\RWlat{La Religion chrétienne prouvée par les faits.\RWlit{}{Houteville1} Paris 1740. 3~Fol.} Deutsch. Frankfurt 1745\RWlit{}{Houteville1a}), \RWbet{Vernet} (\RWlat{Traité de la vérité de la rel. chrétienne.\RWlit{}{Vernet1} Genève} 1748--88. 10 Vol.), \RWbet{Karl Bonnet} (\RWlat{Recherches philosophiques sur les preuves du Christianisme. Genève} 1771\editorischeanmerkung{%
	\lit{Bonnet2}. Bolzano hatte in seiner Bibliothek die deutsche Übersetzung von 1773 (\lit{Bonnet2a}), erwähnt aber unten die Ausgabe von 1769 (\lit{Bonnet1a}).}); und \RWbet{Bergier} (\RWlat{Traité historique et dogmatique sur la vraie Religion}\RWlit{}{Bergier1}, und \RWlat{Preuves du Christianisme}\RWlit{}{Bergier2}, das erste Werk erschien auch deutsch zu Bamberg und Würzburg, wie auch im Nachdrucke zu Budweis 1788 in 12~B.\RWlit{}{Bergier1a}); das Bonnet'sche Werk übersetzte \RWbet{Lavater} in's Deutsche mit Anmerkungen (Zürich 1769)\RWlit{}{Bonnet1a}, und hielt es für einen so unwiderleglichen Beweis der Wahrheit des Christenthumes, daß er den berühmten israelitischen Weltweisen \RWbet{Mendelssohn} aufforderte, Eines von Beidem zu thun, entweder dieß Buch zu widerlegen, oder ein Christ zu werden.
\item In \RWbet{Deutschland}, zuvörderst unter den \RWbet{Protestanten}, sind folgende eine ganz vorzügliche Erwähnung verdienende Werke erschienen: \RWbet{Lilienthal}'s gute Sache der göttlichen Offenbarung (Königsberg 1750. 16~Bde.)\RWlit{}{Lilienthal1}; \RWbet{Basedow}'s Versuch über die Wahrheit des Christenthumes als der besten Religion\RWlit{}{Basedow1} 
%???
%Rosenmüller, Prüfung der vornehmsten Gründe für und wider die Religion. Erlangen. Sehr compendiarisch. 
%???
(Berlin 1766), ein Buch, bei welchem nur zu bedauern ist, daß der Verfasser den Nutzen der sogenannten Geheimnißlehren nicht einsah, weßhalb er von der ganzen Christuslehre wenig mehr beibehielt, als die bloßen Wahrheiten der natürlichen Religion; \RWbet{Jerusalem}'s Betrachtungen über die vornehmsten Wahrheiten der Religion\RWlit{}{Jerusalem1} (3~Thle., Braunschweig 1776); das Werk ist durch den Tod des Verfassers unbeendigt geblieben. \RWbet{Nösselt}'s Vertheidigung der Wahrheit und Göttlichkeit der christlichen Religion (2\hoch{te} Ausg. 1783. In der 5\hoch{ten} Ausgabe 1784 ist nur die erste Hälfte erschienen\editorischeanmerkung{%
	In Bolzanos Bibliothek befand sich die 3.\,Aufl.\ von 1769 (\lit{Noesselt1}) und die 5.\,Aufl.\ von 1783 (\lit{Noesselt1a}).});~\RWSeitenw{23}\ \RWbet{Leß}'s Wahrheit der christlichen Religion (5\hoch{te} Ausg.\ 1785)\RWlit{}{Less1}. Von diesem Werke sagte \RWbet{Lessing}, daß er es nie zur Hand genommen, ohne sich daraus zu belehren und zu erbauen. Im Jahre 1786 fing der Verfasser die Bearbeitung desselben nach einem erweiterten Plane, unter dem Titel: \RWbet{Ueber die Religion, ihre Geschichte, Wahl und Bestätigung in 3 Theilen}\RWlit{}{Less2}, an, kam aber nur mit den beiden ersten Theilen zu Stande. Besonders ausführlich sind die historischen Partien, von der Aechtheit, Unverfälschtheit und Glaubwürdigkeit der heil.\ Schrift, \usw\ abgehandelt; daher auch spätere Schriftsteller Vieles von ihm entlehnten, so wie er selbst in dieser Rücksicht, obwohl nicht ohne eigenes Urtheil, des Engländers \RWbet{Lardner}'s Schriften über die Glaubwürdigkeit der biblischen Geschichte benützte. Auf die kritische Philosophie nahm \RWbet{Leß} noch keine Rücksicht; \RWbet{D.~Balth.~Münter} Bekehrungsgeschichte des Grafen Struensee (2\hoch{te} Aufl., Leipzig 1773)\RWlit{}{Muenter1}; ingleichen: \RWbet{Unterhaltungen eines nachdenkenden Christen mit sich selbst} (2 Bände.\RWlit{}{Muenter2} In diesem Werke wird der sittliche Nutzen gewisser Lehren des Christenthumes sehr gut aus einander gesetzt); \RWbet{Kleuker}'s neue Prüfung und Erklärung der vorzüglichsten Beweise für die Wahrheit und Göttlichkeit des Christenthumes\RWlit{}{Kleuker1} (Riga 1787--94. 4.~Thle.); \RWbet{Storr}'s \RWlat{doctrinae christianae pars theoretica e sacris literis repetita} (\RWlat{Tub.} 1793.\editorischeanmerkung{%
	Bolzano hat in seiner Bibliothek die 2. Aufl., Stuttgart: Mezler 1807, die von ihm erwähnte erste Auflage ist ebenfalls in Stuttgart erschienen (\lit{Storr1}).}) zeichnet sich durch besondere Gründlichkeit aus; \RWbet{Georg Friedrich Seiler}'s Schriften (kurze Apologie des Christenthumes\RWlit{}{Seiler1}; der vernünftige Glaube an die Wahrheit des Christenthumes\RWlit{}{Seiler2} \uam\ ) suchen vornehmlich die Vernunftmäßigkeit der christlichen Geheimißlehren zu zeigen; \RWbet{Köppen}'s, die Bibel, ein Werk der göttlichen Weisheit (3.\,Aufl. 1797. 8.).\editorischeanmerkung{%
	1797 erschien die zweite Auflage dieses Werks (\lit{Koeppen1}), während die dritte (\lit{Koeppen1a}) erst 1837 erschien. Mit der Ziffer \anf{8} dürfte Bolzano das Oktav-Format angegeben haben.} 
Der Verfasser bemüht sich, den göttlichen Geist, der in der Bibel herrscht, nachzuweisen.
\item Die zuletzt angeführten Schriften, und schon einige frühere, welche nicht von katholischen Verfassern herrühren, sind bei aller ihrer Vortrefflichkeit höchstens nur dazu brauchbar, uns von der Wahrheit derjenigen Lehren des katholischen Religionsbegriffes zu überzeugen, die dieser mit den übrigen~\RWSeitenw{24}\ christlichen Lehrbegriffen gemein hat. Zu einem Beweise der Wahrheit und Göttlichkeit des Katholicismus in seinem ganzen Umfange können wir sie nicht brauchen. Zu diesem Zwecke sind besonders folgende Werke zu empfehlen: Die in lateinischer Sprache geschriebenen Werke von \RWbet{Gazzaniga} \RWlat{Theologia polemica}\RWlit{}{Gazzaniga1} und \RWlat{praelectiones theologicae}\RWlit{}{Gazzaniga2}, sind zwar eigentlich für Theologen bestimmt, und dürften einem Laien etwas zu weitläufig erscheinen; indessen kann sie ein solcher gleichfalls zum Nachschlagen benützen, und wird sich daraus leicht überzeugen können, wie schwach die Gründe sind, die von den Gegnern des Katholicismus wider denselben vorgebracht worden sind. \RWbet{Benedict Stattler}'s \RWlat{Demonstratio evangelica (Aug.~Vind.} 1771\RWlit{}{Stattler1}) ist für Nicht-Theologen bestimmt, und mit vieler Gründlichkeit geschrieben. Ein gleiches Lob verdient auch \RWbet{Simon Jordan}'s Werk: \RWlat{De Religione contra libertinos libri tres in usum auditorum (Pragae 1773. 4.\,tom.\RWlit{}{Simon1})}, nach welchem Buche der Verfasser zu seiner Zeit an dieser Universität Vorlesungen hielt. Eins der besten Werke ist aber unstreitig des Benediktiner \RWbet{Beda Mayer}'s Vertheidigung der natürlichen, christlichen und katholischen Religion nach dem Bedürfnisse unserer Zeit (Augsb.~1787. 4 Bände\RWlit{}{Mayr1}). Zu bedauern ist nur, daß der sehr gründliche Verfasser auf die Einwürfe der kritischen Philosophie noch keine Rücksicht genommen. Viel Beifall fand auch \RWbet{Storchenau}'s Philosophie der Religion (7 Bde.\ Augsb.~1780\RWlit{}{Storchenau1}) und \RWbet{Schwarzhueber}'s praktisches Religionsbuch für nachdenkende Christen (3 Bände, Salzb. 1795.\RWlit{}{Schwarzhueber1}); aber des trefflichen \RWbet{Ildephons Schwarz} Handbuch der christlichen Religion (Bamberg und Würzburg, 4.\,Aufl., 3 Bde. 1803\editorischeanmerkung{%
	In Bolzanos Bibliothek befand sich die erste Auflage von 1793 (\lit{Schwarz1}), vermutlich ein Ersatzexemplar, das ihm \priho\  geschickt hatte, da Bolzano die hier genannte Ausgabe (\lit{Schwarz1a}) verlorengegangen war; vgl.\ \lit{BergMorscher2002b}, 349.})
dürfte unter allen bisher erschienenen Werken, die zur Empfehlung der christlichen Religion geschrieben sind, das geeignetste seyn, um einem Verächter derselben bessere Gesinnungen gegen sie einzuflößen; und diesen Zweck wird es, wenn schon durch nichts Anderes, durch die vielseitige Belesenheit, den feinen Geschmack und sanften, liebenswürdigen Charakter, den der Verfasser überall verräth, erreichen. Viele gute Gedanken enthalten auch \RWbet{Peutinger}'s~\RWSeitenw{25}\ Religion, Offenbarung und  Kirche im Lichte der reinen Vernunft aufgefaßt\RWlit{}{Peutinger1} (Salzb.\ 1795); ingleichen \RWbet{Joh.~Mich.~Sailer}'s Grundlehren der Religion. Ein Leitfaden zu den Religionsvorlesungen an die akademischen Jünglinge aus allen Fakultäten\RWlit{}{Sailer1a} (Sulzbach 1832). Doch zu dem Zwecke, um als ein Leitfaden für die in allen k.~k. österreichischen Erbstaaten vorgeschriebenen philosophischen Religionsvorlesungen zu dienen, war keines der bisher vorhandenen Werke geeignet. Daß nun dasjenige, welches Hr. \RWbet{Jakob Frint}, ehemaliger Professor der philosophischen Religionslehre zu Wien, gegenwärtiger k.~k. Burgpfarrer,\RWfootnote{Nachmals Bischof zu St.~Pölten, gestorben den [11.\,Oktober 1834].}
herausgab (Lehrbuch der Religionswissenschaft.\RWlit{}{Frint1a} Wien 1813--1820, 3\hoch{te} Auflage, 6 Bände), den sämmtlichen Forderungen, die man an einen solchen Leitfaden zu machen berechtiget ist, in einem sehr hohen Grade entspreche, daß der Verfasser eine vertraute Bekanntschaft mit den besten Schriften, die für und wider die Religion erschienen sind, verrathe, die erstern auf eine sehr verständige Weise benütze, die Einwürfe aber, welche die letztern enthalten, gründlich zu widerlegen bemüht sey, daß in seinem Buche eine sehr lichtvolle Anordnung, und ein überaus klarer und deutlicher Vortrag herrsche, daß der sittliche Nutzen der einzelnen Lehren des Christenthumes noch nirgends so umständlich aus einander gesetzt worden sey: das und noch mehres Andere wird diesem Werke auch derjenige nachrühmen müssen, der nicht allen Ansichten des gelehrten Verfassers beipflichten kann, und der Meinung ist, daß sich die gute Sache der katholischen Religion mit noch siegreicheren Waffen vertheidigen lasse.
\end{aufzb}
\end{aufza}

\RWpar{8}{Plan und Eintheilung der Religionswissenschaft}
\begin{aufza}
\item Bevor die wissenschaftliche Religionslehre die einzelnen Lehren der vollkommensten Religion selbst vortragen, und ihre Wahrheit darthun kann, muß sie nach der Natur der Sache erst einige \RWbet{Vorbereitungen} treffen, die gleichwohl nicht mehr zur Einleitung, sondern schon in den eigentlichen Unterricht gehö\RWSeitenw{26}ren, weil sie die vollkommenste Religion schon selbst betreffen. Um nämlich beweisen zu können, daß diese oder jene Religion (\zB\ die katholische) wirklich die vollkommenste sey, muß man erst deutlich erklären, was man unter der \RWbet{vollkommensten Religion} verstehe. Um dieses gehörig zu thun, muß man erst den Begriff der \RWbet{Religion überhaupt} bestimmen. Da ferner nur daraus allein sicher erkannt werden kann, daß die katholische Religion die vollkommenste sey, weil sie von Gott geoffenbaret worden ist, so muß erwiesen werden, daß sie dieß wirklich sey. Um nun dieß zu vermögen, wird man erst wieder den Begriff einer Offenbarung erklären, und \RWbet{die sichern Kennzeichen einer wahren göttlichen Offenbarung festsetzen} müssen. Bevor man aber Jemand zumuthen kann, daß er nun ausgehe und untersuche, ob irgend eine der auf Erden vorhandenen Religionen die Kennzeichen einer göttlichen Offenbarung an sich habe, wird man ihm erst die Pflicht, nach der Erkenntniß dieser vollkommensten Religion zu streben, nach ihrer Wichtigkeit darstellen, und auch die einzelnen Verbindlichkeiten, die diese Pflicht enthält, auseinandersetzen, und ihm beweisen müssen, daß, falls eine Offenbarung für ihn vorhanden wäre, sie ohne Zweifel auch zum Inhalte der für ihn vollkommensten Religion gehören würde. Da aber Gott nichts Ueberflüßiges thut: so ist einleuchtend, daß es auch keine wahre Offenbarung für den Menschen geben könnte, wenn eine bloß natürliche Religion schon völlig hinreichend für ihn wäre. Und da dieß Letztere häufig behauptet worden ist, so muß die \RWbet{Unzulänglichkeit der natürlichen Religion}, etwa in ihrer vollkommensten Gestalt, vorausgeschickt werden. -- Ist dieses Alles geschehen, und hat sich am Ende gezeigt, daß die natürliche Religion auch selbst in ihrer vollkommensten Gestalt eine gewisse Unzulänglichkeit habe: so wird die Hoffnung erwachen, daß Gott uns vielleicht wirklich eine Offenbarung geschenkt habe. Allein von Einigen ist selbst ihre \RWbet{Möglichkeit} bestritten worden, mithin wird auch diese erst noch vertheidiget werden müssen. Bei der Untersuchung über ihre Kennzeichen findet es sich, daß eine wahre Offenbarung nur an dem vereinigten Daseyn \RWbet{zweier Kennzeichen} erkennbar sey, nämlich an einer gewis\RWSeitenw{27}sen \RWbet{inneren Vortrefflichkeit} oder sittlichen Zuträglichkeit \RWbet{ihrer Lehren} und an gewissen \RWbet{außerordentlichen Ereignissen}, die keinen sichtbaren Nutzen und Zweck ihres Daseyns hätten, wenn es nicht der seyn sollte, daß sie uns zur Bestätigung jener Lehren dienen. Nun werde ich umständlich zeigen, daß diese beiden Kennzeichen dem \RWbet{katholisch-christlichen Religionsbegriffe} im vollesten Maße zukommen; woraus dann folgen wird, daß er eine wahre göttliche Offenbarung, mithin auch die vollkommenste Religion für uns sey. Da nun der Zuhörer auch schon die \RWbet{Lehren} dieser Religion, nämlich bei Prüfung ihrer innern Vortrefflichkeit, wird kennen gelernt haben, so wird der Vortrag jetzt die beiden Zwecke eines jeden Unterrichtes, die Bekanntmachung mit den beizubringenden Lehren, und die Ueberzeugung von ihrer Wahrheit erreicht haben, und hiemit geschlossen werden können.
\item Wenn ich mich überdieß allenthalben, wo es nur thunlich seyn wird, bemühen werde, \RWbet{den eigentlichen Grund}, worauf die Wahrheit jeder aufgestellten Behauptung beruhet, anzugeben, so wird man diesem Unterrichte auch nicht den Namen eines \RWbet{wissenschaftlichen} absprechen können.
\item So wird sich denn der Vortrag der ganzen Religionswissenschaft in \RWbet{drei}, einander am Umfange nicht allzu ungleiche \RWbet{Haupttheile} zerlegen lassen. Der \RWbet{erste} nämlich wird \RWbet{die nöthigen Vorbereitungen zur Aufsuchung der vollkommensten Religion} enthalten, und abermals in vier Unterabtheilungen oder \RWbet{Hauptstücke} zerfallen, deren das \RWbet{erste von dem Begriffe der Religion, von den verschiedenen Arten derselben, und von dem pflichtmäßigen Verhalten gegen sie} handeln; das \RWbet{zweite} einen kurzen \RWbet{Abriß der natürlichen Religion}; das \RWbet{dritte} eine \RWbet{Würdigung dieser natürlichen Religion, und einen Beweis für die Nothwendigkeit einer Offenbarung}; das \RWbet{vierte} endlich die Lehren von der \RWbet{Möglichkeit und den Kennzeichen einer Offenbarung} enthalten wird. Der \RWbet{zweite} und \RWbet{dritte Haupttheil} wird dann erweisen, daß die \RWbet{katholische Religion} die beiden Kennzeichen einer göttlichen Offen\RWSeitenw{28}barung wirklich an sich habe; der \RWbet{eine}, daß ihre Lehre die \RWbet{höchste sittliche Zuträglichkeit} habe, der \RWbet{andere}, daß ihr Daseyn auch mit solchen \RWbet{außerordentlichen Begebenheiten} verbunden sey, die keinen sichtbaren Nutzen und Zweck hätten, wenn es nicht der seyn sollte, daß sie uns zur Bestätigung dieser Lehre dienen. Aus Gründen, die sich hier nicht wohl anführen lassen, werde ich die zuletzt genannten zwei Untersuchungen in der umgekehrten Folge vornehmen; und somit wird der \RWbet{zweite Haupttheil} der Religionswissenschaft die \RWbet{Wunder, die zur Bestätigung des Katholicismus} dienen; der \RWbet{dritte} aber die Lehren desselben betrachten. Die \RWbet{Unterabtheilungen}, in welche auch noch diese zwei letztern Haupttheile zerfallen, hier schon anzeigen zu wollen, wäre von gar keinem Nutzen.
\end{aufza}

\cleardoublepage\RWteil{I}{Erster Haupttheil.}{Nöthige Vorbereitungen zur Aufsuchung der vollkommensten Religion.}{\RWSeitenwohne{29}}

\cleardoublepage\RWch[Erstes Hauptstück.\\
		Von dem Begriffe der Religion, den verschiedenen Arten derselben und dem pflichtmäßigen Verhalten gegen sie.]{Erstes Hauptstück.\RWSeitenwohne{31} \\
		Von dem Begriffe der Religion, den verschiedenen Arten derselben und dem pflichtmäßigen Verhalten gegen sie.}
\RWpar{9}{Inhalt und Zweck dieses Hauptstückes}
Zu Folge des eben beschriebenen Planes muß ich erstlich den \RWbet{Begriff} bestimmen, der mit dem Worte \RWbet{Religion} entweder \RWbet{wirklich} verbunden wird, oder doch verbunden werden \RWbet{sollte}. Hiebei verstehet sich von selbst, daß ich erweisen müsse, daß diesem Begriffe auch ein \RWbet{wirklicher Gegenstand} entspreche, ja daß es nicht Eine, sondern mehrerlei Religionen gebe. Es muß insonderheit der \RWbet{Unterschied zwischen natürlicher und geoffenbarter Religion} in sein gehöriges Licht gesetzt, und endlich der \RWbet{Begriff der allervollkommensten Religion} festgestellt werden. Hierauf werde ich die \RWbet{Pflichten, welche der Mensch in Betreff seiner eigenen religiösen Ueberzeugungen hat}, umständlich auseinander setzen müssen. Daß ich auch von den \RWbet{Pflichten} rede, die Jeder in Ansehung der \RWbet{religiösen Ueberzeugungen seiner Mitmenschen} hat, ist zwar nicht eben \RWbet{nothwendig}; da aber diese Pflichten mit jenen so nahe zusammenhängen, und da man so oft sich gegen sie verstößt: so werde ich sie in Kürze mitnehmen. Aus einem ähnlichen Grunde werde ich auch \RWbet{der gewöhnlichsten Abweichungen von diesem pflichtmäßigen Betragen gegen die Religion}~\RWSeitenw{32}\ erwähnen, und ihre \RWbet{Quellen}, ihre \RWbet{Schädlichkeit}, ihre \RWbet{Strafwürdigkeit} mit wenigen Worten besprechen. Aber zuerst noch

\RWpar{10}{Eine Untersuchung, die allen übrigen vorangehen muß}
Wenn schon ein jeder Unterricht von den Behauptungen, welche er aufstellt, \RWbet{Ueberzeugung} zu bewirken bemüht seyn muß; so ist einleuchtend, daß mir beim Unterrichte in der vollkommensten Religion, also beim Vortrage der Religionswissenschaft diese Verbindlichkeit in einem ganz vorzüglichen Grade obliegt. Hier muß ich bestrebt seyn, eine Ueberzeugung hervorzubringen, die nicht nur in der Gegenwart durch keine Zweifel beunruhiget wird, sondern von der sich auch erwarten läßt, daß sie in dem ganzen künftigen Leben meiner Zuhörer durch nichts werde umgestoßen, ja nur erschüttert werden. Dieß würde ich jedoch keineswegs leisten, wenn \RWbet{die Voraussetzungen}, auf die ich meine Beweise stützte, \dh\ die Sätze, die ich in meinen Schlüssen als \RWbet{Prämissen} oder \RWbet{Vordersätze} aufnähme, nicht alle \RWbet{gewiß} und \RWbet{unläugbar} wären. Denn der Grad der Wahrscheinlichkeit, welchen die \RWbet{Conclusion} eines Schlusses hat, ist niemals größer, sondern im Gegentheile meistens noch etwas kleiner, als der Grad der Wahrscheinlichkeit, den seine schwächste Prämisse hat.\par
Nun gibt es aber fast keine \RWbet{einzige Wahrheit}, die nicht von \RWbet{einzelnen Gelehrten} einmal in Zweifel gezogen, ja wohl geradezu \RWbet{bestritten} worden wäre. Schon unter den \RWbet{Griechen} gab es eine sehr zahlreiche Secte von Philosophen, die \RWbet{Skeptiker} (\dh\ \RWbet{Zweifler}) genannt, die sich ein eigenes Geschäft daraus machten, jede auch noch so einleuchtende Wahrheit durch scheinbare Einwürfe, und durch die Nachweisung angeblicher Widersprüche derselben mit andern schwankend und zweifelhaft zu machen. So läugneten sie \zB\ die Möglichkeit aller Bewegung, das Daseyn einer Körperwelt, das Daseyn anderer denkender Wesen außerhalb ihrer, ja Einige sogar ihr \RWbet{eigenes} Daseyn.~\RWSeitenw{33}\par
Allein das Schlimmste ist, daß die Gefahr, in eine Art von \RWbet{Skepticismus} zu verfallen, wirklich für \RWbet{jeden Menschen}, wenigstens für jeden Gebildeten, vorhanden ist, wenn nicht eigends vorgebeugt wird; ich will sagen, daß es \RWbet{fast keine einzige Wahrheit} gibt, die zu bezweifeln sich ein \RWbet{Gebildeter} nicht irgend einmal in seinem Leben \RWbet{versucht} fühlen könnte. Diese Versuchung kann nämlich eintreten:
\begin{aufza}[a)]
\item \RWbet{wenn diese Wahrheit mit einer unserer Leidenschaften in Widerspruch geräth}. Denn nur zu oft geschieht es, daß Menschen an einer Wahrheit zu zweifeln anfangen, die sie noch nie bezweifelt hatten, bloß weil sich jetzt irgend ein leidenschaftlicher Wunsch in ihnen erhebt, dem diese Wahrheit hinderlich ist. Und je geübter sie dann im Denken sind, um so leichter bringen sie es bei sich selbst dahin, sich durch hervorgesuchte scheinbare Einwürfe zu überreden, daß das nicht wahr sey, was sie bisher immer für wahr gehalten hatten.
\item \RWbet{wenn wir erfahren, daß diese Meinung von vielen andern Menschen bezweifelt und bestritten werde}.
\item wenn wir durch Nachdenken und Vergleichen zu entdecken glauben, daß der vorliegende Satz \RWbet{mit gewissen andern, von uns für wahr gehaltenen Sätzen in einem Widerspruch stehe}.
\item wenn wir uns vorgenommen hatten, \RWbet{die Gründe des Satzes aufzusuchen}, und bei diesem Geschäfte so unglücklich waren, \RWbet{sie nicht zu finden}. Denn dann entstehet in uns der Verdacht, daß der Satz vielleicht gar keine Gründe \RWbet{habe}; und wir sind um so geneigter, dieß anzunehmen, je bekannter es uns aus vielen Beispielen ist, daß so manche Meinung, welcher die große Menge der Menschen mit vieler Zuversicht anhängt, bei einer genaueren Untersuchung gleichwohl als grundlos und falsch befunden werde.
\end{aufza}\par
Um diesem Skepticismus, so gut es von meiner Seite geschehen kann, vorzubeugen, werde ich in meinem folgenden Vor\RWSeitenw{34}trage durchaus nichts als gewiß und ausgemacht annehmen, was sich auf irgend eine Art zweifelhaft machen läßt, wenn ich es nicht zuvor durch einen eigenen Beweis dargethan habe. Ich gehe deßhalb so weit, mir vorzustellen, daß ich selbst Zuhörer von einer Art vor mir hätte, die sich zuweilen wenigstens in \RWbet{einer solchen Stimmung} befinden, bei der ihnen durchaus nichts als gewiß erscheinet. Und ich frage mich, \RWbet{was wohl die erste Wahrheit wäre}, von der ich diese in einem solchen Zustande zu überzeugen vermöchte.\par
Offenbar dürfte das keine derjenigen Wahrheiten seyn, die zu ihrem Beweise schon der \RWbet{Voraussetzung} von \RWbet{einer} oder \RWbet{etlichen andern} bedürfen. Es muß vielmehr eine Wahrheit seyn, \RWbet{die für sich selbst gewiß ist}, indem ihr \RWbet{Gegentheil sich widerspricht}. Von dieser Art ist nun wohl nur die Wahrheit, \RWbet{daß es Wahrheiten überhaupt gebe}, wie dieß der folgende § mit Mehrem zeigen wird.

\RWpar{11}{Es gibt Wahrheiten}
\begin{aufza}
\item Daß es überhaupt \RWbet{einige}, wenigstens \RWbet{Eine} Wahrheit gebe, wird Jedermann einleuchtend finden, sobald ich nur erst den Sinn, den ich mit dieser Behauptung verbinde, gehörig festgesetzt haben werde. Ich muß nämlich erinnern, daß ich das Wort: \RWbet{Wahrheit} hier in seiner \RWbet{objectiven} Bedeutung nehme, in der ein jeder Satz, der etwas so aussagt, wie es ist, eine Wahrheit genannt wird, gleichviel, ob dieser Satz von irgend Jemand wirklich erkannt und ausgesprochen werde oder nicht. Wir unterscheiden also \RWbet{Wahrheiten an sich}, von \RWbet{Wahrheiten}, die von irgend Jemand \RWbet{gedacht} und \RWbet{anerkannt} werden. -- Daß es auch Wahrheiten letzterer Art, \dh\ \RWbet{erkannte} Wahrheiten, oder (was eben so viel ist) \RWbet{Erkenntnisse} gebe, wollen wir jetzt noch nicht behaupten; denn diese Behauptung setzt wirklich schon Mehres voraus,
\begin{aufzb}
\item daß es Wahrheiten an sich gebe, und
\item daß es ein Wesen gebe, das urtheilt.~\RWSeitenw{35}
\end{aufzb}
Erst wenn dieses Beide vorausgesetzt wird, läßt sich die Frage untersuchen, ob die Urtheile dieses, alle oder doch einige, wahr sind?
\item Daß es aber \RWbet{einige}, oder wenigstens doch \RWbet{Eine} objective Wahrheit (oder Wahrheit an sich) gebe, das läßt sich ohne Voraussetzung von irgend etwas Anderem darthun. Denn der entgegengesetzte Satz, \dh\ die Behauptung, daß gar nichts wahr sey, hebt ja sich selbst auf. Wenn nämlich nichts wahr wäre, so wäre auch dieß falsch, daß nichts wahr sey. Durch diesen leichten Schluß kann sich denn also jeder auch noch so arge Zweifler für sich selbst überzeugen, daß es einige oder doch wenigstens Eine objective Wahrheit gebe.
\end{aufza}

\RWpar{12}{Es gibt auch mehre und zwar unendlich viele Wahrheiten}
\begin{aufza}
\item Daß es mindestens \RWbet{Eine} Wahrheit gebe, ist aus dem eben Gesagten gewiß. Gesetzt nun, daß Jemand den Zweifel hegte, ob es auch \RWbet{mehre} Wahrheiten gebe; so würde ich auf folgende Art versuchen, ihn auch hievon zu überzeugen.
\item Ich kann die Eine Wahrheit, deren Vorhandenseyn mir der Zweifler zugibt, wie sie auch immer laute, durch die Formel: \RWbet{$A$ ist $B$}, bezeichnen, und behaupte nun, daß es nebst dieser zum wenigsten noch \RWbet{Eine} zweite gebe. Denn wer das Gegentheil annehmen wollte, der müßte ja die Behauptung, \RWbet{daß es außer der Wahrheit: $A$ ist $B$, sonst keine andere gibt}, selbst als wahr aufstellen. Diese Behauptung ist aber offenbar von der Behauptung: \RWbet{$A$ ist $B$}, ganz verschieden, und wäre also, sofern sie wahr wäre, gleich eine \RWbet{zweite} Wahrheit. Es ist daher nicht wahr, daß es nur \RWbet{Eine} Wahrheit gebe, sondern es gibt derselben wenigstens zwei.
\item Auf eine ähnliche Weise läßt sich aber darthun, daß es auch mehr als zwei Wahrheiten geben müsse. Denn die zwei Wahrheiten, deren alleiniges Vorhandenseyn Jemand zugeben wollte, mögen wie immer lauten, so lassen sie sich durch~\RWSeitenw{36}\ die zwey Formeln: \RWbet{$A$ ist $B$}, und: \RWbet{$C$ ist $D$}, bezeichnen, und nun läßt sich wie vorhin zeigen, \RWbet{daß die zwei Wahrheiten: $A$ ist $B$, und: $C$ ist $D$, noch nicht die einzigen seyn können}, indem das Gegentheil durch die Behauptung, \RWbet{daß nichts wahr sey, als nur, daß $A$, $B$ und $C$, $D$ ist}, wofern sie wahr wäre, gleich eine \RWbet{neue} und \RWbet{dritte} Wahrheit ausmachen würde.
\item  Man sieht von selbst, daß diese Schlußart sich ohne Ende fortsetzen lasse; und es muß also der Wahrheiten \RWbet{unendlich viele} geben.
\end{aufza}

\RWpar{13}{Wir Menschen sind im Stande, Wahrheiten zu erkennen, und erkennen auch wirklich einige}
\begin{aufza}
\item Waren die vorigen Betrachtungen vermögend, den Zweifler erst dahin zu bringen, daß er sich eingestehe, es müsse Wahrheiten, und zwar unendlich viele Wahrheiten geben: so versuche ich ferner, ihn auch dahin zu bringen, daß er sich selbst die Fähigkeit der \RWbet{Erkenntniß} wenigstens einiger dieser Wahrheiten beilege.
\item Dieß kann ich freilich nicht anders, als dadurch, daß ich ihn vermöge, irgend ein Urtheil zu fällen; denn würde er in der That auch nicht ein einziges Urtheil fällen: so würde er eben deßhalb auch keine einzige Wahrheit erkennen, denn jedes Erkenntniß setzt doch ein Urtheil voraus.
\item So wie ich aber bewirke, daß er auch nur ein einziges Urtheil, es laute wie es wolle, fällt; so legt er sich eben hierdurch schon die Fähigkeit bei, Wahrheiten zu erkennen; denn dieses Urtheil selbst hält er ja doch für Wahrheit, wenigstens zu der Zeit, da er es eben ausspricht.
\item Doch der Zweifler gesteht, daß er dem Drange, Urtheile zu fällen, allerdings nicht widerstehen könne: aber er zweifelt nur, ob diesen Urtheilen auch Wahrheit beiwohne. Verstände er dieß bloß so, daß wohl nicht \RWbet{alle} seine Urtheile wahr sind, so wäre hiegegen, wie wir bald sehen werden, nicht das Geringste zu sagen. Allein der Zweifler meint, daß vielleicht auch \RWbet{nicht ein einziges} seiner~\RWSeitenw{37}\ Urtheile wahr sey. Dieß nun ist ungereimt; denn die Voraussetzung, daß \RWbet{alle} unsere Urtheile falsch sind, hebt sich ja selbst auf, weil auch sie diesen Urtheilen beigezählt werden müßte. Wer dieses erwägt, muß, wie ich glaube, sich selbst gestehen, daß doch nicht alle seine Urtheile falsch sind, daß doch in einigen, in einem einzigen wenigstens, Wahrheit liege. 
\item Auf eine ähnliche Weise, wie \RWparnr{12}, können wir ihm dann zeigen, daß er der Wahrheiten in der That \RWbet{mehre}, ja unbestimmt \RWbet{viele} (nämlich nur nach und nach) erkenne.
\end{aufza}

\RWpar{14}{Wir irren uns zwar zuweilen in unseren Urtheilen; aber es gibt doch auch gewisse Umstände, bei deren Vorhandenseyn wir uns mehr oder weniger versichern können, daß wir nicht irren}
So deutlich es uns durch das Gesagte geworden seyn mag, daß nicht alle unsere Urtheile falsch sind; so ist es doch von der andern Seite gewiß, daß wir nur gar zu oft irren; denn wie oft behaupten wir nicht das Gegentheil von dem, was wir zu irgend einer früheren Zeit behauptet, und halten für wahr, was wir früher für falsch; oder für falsch, was wir früher für wahr gehalten hatten. In solchen Fällen müssen wir uns entweder früher geirrt haben, oder wir irren uns eben jetzt. Sollen wir aber wohl deßhalb, weil wir zuweilen uns irren, gar keinem unserer Urtheile trauen? Da würden wir abermals in die sich selbst widersprechende Behauptung, daß alle unsere Urtheile falsch sind, verfallen. Wir müssen also eine Art von Auswahl treffen, und werden dann bei allem Bewußtseyn unserer Fehlbarkeit in gewissen Stücken, doch einigen unserer Urtheile mit voller Zuversicht vertrauen können. Es ist von höchster Wichtigkeit, daß wir die Regeln, nach welchen diese Auswahl vernünftiger Weise zu geschehen habe, kennen lernen. Ich kann hier aber nur einige derselben anführen, und sage dann, daß wir nach Beschaffenheit folgender Umstände \RWbet{bald mehr bald weniger versichert seyn} können, daß wir in einem gewissen Urtheile nicht irren:~\RWSeitenw{38}
\begin{aufza}
\item \RWbet{Wenn sich uns ein Urtheil unwiderstehlich aufdringt}, und wir die Sache nicht einmal anders zu denken vermögen, so dürfen wir einem solchen Urtheile vollkommen trauen; \zB\ dem Urtheile, daß zwei gerade Linien keinen Raum einschließen können. Denn wenn wir nicht einmal \RWbet{solchen} Urtheilen trauen wollten: so könnten wir offenbar um desto weniger \RWbet{andern}, die sich nicht so unwiderstehlich aufdringen, wir dürften also \RWbet{gar keinem unserer Urtheile trauen}, \dh\ wir müßten glauben, daß wir gar keine Wahrheit erkennen, welches der obige Widerspruch wäre.
\item \RWbet{Urtheile, die nichts Anderes, als eine bloße, so eben gegenwärtige Wahrnehmung (\dh\ Empfindung oder Vorstellung) in uns aussagen}, ohne die Ursache derselben bestimmen zu wollen, \dh\ \RWbet{reine Wahmehmungsurtheile} haben vollkommene Gewißheit; \zB\ das Urtheil: Ich habe die Vorstellung von einer \RWbet{rothen Farbe}, -- wo ich nicht damit sagen will, daß irgend ein rother Gegenstand wirklich vor mir stehe, \udgl\ Denn auch solche Urtheile dringen sich uns unwiderstehlich auf.
\item Urtheile, in denen wir uns erlauben, \RWbet{über die Ursache einer gehabten Wahrnehmung zu entscheiden}, in denen wir \zB\ aussagen, daß derselbe Gegenstand, der die Empfindung $A$ in uns hervorbrachte, auch Ursache der Empfindung $B$ sey, kurz \RWbet{Erfahrungsurtheile} im engern Sinne des Wortes haben als solche immer nur \RWbet{Wahrscheinlichkeit}; allein wenn die beiden Empfindungen oder Vorstellungen $A$ und $B$ schon überaus oft von uns \RWbet{zugleich} empfunden worden sind; dagegen niemals die eine ohne die andere: so wird die Wahrscheinlichkeit des Urtheiles, daß der Gegenstand, der die Wahrnehmung $A$ hervorbringt, auch Ursache der Wahrnehmung $B$ sey, so groß, daß wir sie einer vollen \RWbet{Gewißheit} gleich achten können. Ein Beispiel von einem solchen Erfahrungsurtheile wäre das Urtheil: der Ofen wärmt mich. Das, was ich am Ofen \RWbet{sehe}, ist $=A$; die \RWbet{Wärme}, die ich empfinde ist $=B$; der Ofen selbst ist jener Gegenstand, den ich als Ursache der Wahrnehmung $A$ bezeichne, und, indem ich nun sage: \RWbet{der Ofen}~\RWSeitenw{39}\ \RWbet{wärmet mich}, sage ich aus, daß eben derselbe Gegenstand (Ofen), der Ursache von der Empfindung der Farbe, Gestalt \udgl\ ($A$) ist, auch Ursache von der Empfindung der \RWbet{Wärme} ($B$) sey.
\item Urtheile, deren gesammte Bestandtheile bloße \RWbet{Begriffe} sind, die man eben deßhalb sehr wohl \RWbet{Begriffsurtheile} oder \RWbet{Urtheile aus Begriffen} nennen könnte, gewöhnlich aber als \RWbet{Urtheile \RWlat{a priori}} bezeichnet, erhalten auch ohne alle anderen Gründe einen sehr hohen Grad von Wahrscheinlichkeit, wenn sie durch viele \RWbet{Erfahrungen bestätiget werden}, \dh\ wenn wir viele Erfahrungen machen, die gerade so ausfallen, wie sie ausfallen \RWbet{müßten}, wenn jenes Begriffsurtheil \RWbet{wahr} wäre, und ganz anders ausfallen könnten, wenn es \RWbet{nicht} wahr wäre. Beispiele geben die Gesetze vom Stoße elastischer Körper; von der Reflexion der Lichtstrahlen, \udgl\
\item \RWbet{Ein Urtheil, welches durch einen sehr leichten und kurzen Schluß aus andern sehr sichern Urtheilen abgeleitet wird, ist auch selbst sehr gewiß}; \zB\ das Urtheil, daß dieser Stein, wenn ich ihn nicht halte, herabfallen werde.
\item Auch wenn ein Urtheil \RWbet{durch eine lange Reihe von Schlüssen abgeleitet werden} muß, kann es doch viele Gewißheit erhalten, wenn sich nur \RWbet{Einer} oder \RWbet{mehre von folgenden Umständen} beigesellen: 
\begin{inparaenum}[a)] 
\item wenn wir die Schlußreihe schon \RWbet{mehrmals} durchgegangen sind; 
\item besonders \RWbet{zu verschiedenen Zeiten}; 
\item wenn wir dasselbe Urtheil auf \RWbet{mehr als eine Art abgeleitet} haben; 
\item wenn wir unsere Schlußkette auch \RWbet{durch mehre andere Menschen} haben prüfen lassen. So wird das Urtheil, daß eine Addition aus sehr vielen Summanden richtig sey, einen hohen Grad von Wahrscheinlichkeit erhalten, wenn wir sie etliche Mal schon durchgegangen sind; dieß auf verschiedene Art; \usw 
\end{inparaenum}
\item Urtheile, in welchen fast alle Menschen auf Erden mit einander einstimmen, die ich eben deßhalb \RWbet{Urtheile des (all-)~gemeinen Menschenverstandes} nenne, sind unter folgenden drei Bedingungen beinahe für \RWbet{unfehlbar} zu~\RWSeitenw{40}\ erachten: \begin{inparaenum} \item wenn sie einen Gegenstand betreffen, zu dessen Beurtheilung nichts als \RWbet{Vernunft}, oder doch nur \RWbet{solche Erfahrungen} nothwendig sind, welche ein jeder Mensch anstellen kann, und anstellt; nicht aber, wenn Erfahrungen oder Versuche nothwendig sind, die nur sehr wenige Menschen angestellt haben und anstellen können; \item wenn es \RWbet{nicht ganz gleichgültig} für den Menschen ist, ob er sein Urtheil über diesen Gegenstand so oder anders fälle; \item wenn das Urtheil so, wie es lautet, gar nicht der \RWbet{Sinnlichkeit des Menschen} schmeichelt, sondern ihr vielmehr \RWbet{Abbruch} thut.\end{inparaenum}\ Beispiele von Urtheilen, denen diese Eigenschaften zukommen, wären die Urtheile, daß \RWbet{Lügen Sünde}, \RWbet{Dankbarkeit eine Pflicht sey}, und hundert andere dergleichen \RWbet{Pflichtsurtheile}. -- Ein Beispiel dagegen von einem Urtheile, das, auch wenn es ganz allgemein verbreitet wäre, doch keine Verlässigkeit hätte, wäre das Urtheil, daß Auf- und Untergang der Sonne durch eine Bewegung derselben von Osten nach Westen entstehe; denn diesem Urtheile fehlt jede der drei hier angegebenen Bedingungen. Daß nun ein Urtheil des gemeinen \RWbet{Menschenverstandes} unter den angegebenen Bedingungen die größte Verlässigkeit habe, erweise ich so: Läßt sich der Gegenstand des Urtheiles durch bloße \RWbet{Vernunft} (ohne besondere Erfahrungen oder Versuche \udgl ) entscheiden; so hat ein jeder Mensch die \RWbet{Fähigkeit}, ihn zu beurtheilen. Ist dieser Gegenstand auch nicht \RWbet{gleichgültig} für uns; so findet es jeder der \RWbet{Mühe werth}, über ihn nachzudenken. Wenn endlich Alle in einem Urtheile einstimmen, das ihrer Sinnlichkeit \RWbet{nicht schmeichelt}, sondern ihr nur Abbruch thut; so gibt es wohl kaum einen anderen Grund, aus dem sich diese Uebereinstimmung erklären ließe, als den, daß die \RWbet{Kraft der Wahrheit selbst} Allen dieß gleichlautende Geständniß abgedrungen 
habe.
\end{aufza}
An die hier beigebrachten Regeln, ersuche ich jedes der Urtheile, das ich in diesen Vorträgen aufstellen werde, zu halten; um seinen jedesmaligen Grad der Gewißheit zu schätzen. Man wird, wie ich hoffe, finden, daß derjenige Theil meiner Behauptungen, die man nothwendig zugeben muß, wenn man den hier geführten Beweis für die Wahrheit und~\RWSeitenw{41}\ Göttlichkeit des katholischen Christenthumes überzeugend finden will, volle Verlässigkeit habe, und dieß zwar meistens, weil es Behauptungen sind, die zu den Aussprüchen des gemeinen Menschenverstandes von der so eben beschriebenen Art gehören. Bei einigen werde ich dieß auch eigends nachweisen.

\RWpar{15}{Der Mensch ist ein der Tugend und Glückseligkeit fähiges Wesen}
Um zu dem Begriffe der Religion auf einem Wege zu gelangen, dabei sich von selbst findet, daß dieser Begriff auch \RWbet{real} sey, \dh\ daß er auch einen wirklichen Gegenstand habe, oder daß es ein Etwas von der Art, wie wir die Religion erklären, in der That gebe, müssen wir von einer kurzen Betrachtung unserer menschlichen Natur selbst ausgehen.
\begin{aufza}
\item Wir Menschen haben die Fähigkeit, unsern eigenen \RWbet{Zustand} entweder \RWbet{angenehm} oder \RWbet{unangenehm} zu empfinden, oder mit andern Worten, wir sind für Lust und Schmerz \RWbet{empfänglich}. Drücken wir diese Behauptung, was hier ohne Beeinträchtigung ihrer Folgen geschehen kann, statt in der \RWbet{vielfachen} Zahl, in der \RWbet{einfachen} aus: \RWbet{Ich} (urtheilendes Wesen) \RWbet{habe die Fähigkeit, angenehm oder unangenehm zu empfinden}; so ist sie eine sehr unmittelbare Folgerung aus dem bloßen Wahrnehmungsurtheile: \RWbet{Ich empfinde Lust} -- \RWbet{Schmerz}. Also nach \RWparnr{14} Nr.\,1 und 5. gewiß. Diese Fähigkeit, seinen eigenen \RWbet{Zustand} entweder \RWbet{angenehm} oder \RWbet{unangenehm} zu \RWbet{empfinden}, nennen wir das \RWbet{Empfindungsvermögen}, (minder richtig auch) \RWbet{Gefühlsvermögen}. Wir Menschen haben also ein eigenes Empfindungsvermögen.
\item Wir finden Wesen um uns, denen wir, wegen ihrer bald \RWbet{gänzlichen}, bald doch sehr \RWbet{vielfältigen} Aehnlichkeit mit uns, ein dem unsrigen \RWbet{gleiches} oder doch \RWbet{ähnliches} Empfindungsvermögen (Empfänglichkeit für Lust und Schmerz) beilegen müssen; namentlich Menschen und Thiere. Dieser Satz ist ein \RWbet{Erfahrungsurtheil}; hat also seiner Natur nach zwar nur \RWbet{Wahrscheinlichkeit}, hier aber eine so hohe, daß kein Vernünftiger im Ernste zweifeln wird,~\RWSeitenw{42}\ ob die ihn umgebenden Menschen und Thiere auch wirklich lebendige und der Empfindung fähige Wesen wären. Ob sich Des Cartes, wenn er die Thiere gewisser vorgefaßter Meinungen wegen für leblose Automate erklärte,\RWlit{}{Descartes1} in der That überredet hatte, daß sie nichts Mehres wären, stehet noch dahin.
\item Wir haben die Fähigkeit, \RWbet{verschiedene Veränderungen in uns und außerhalb unser} hervorzubringen, \zB\ unsere Gliedmaßen auf verschiedene Weise zu bewegen, und dadurch wieder allerlei äußere Gegenstände zu verändern. Zu diesem \RWbet{Erfahrungsurtheile} gelangen wir ganz auf dieselbe Weise, durch welche wir sonst erfahren, daß dieser oder jener äußere Gegenstand gewisse Kräfte (\dh\ Fähigkeiten zu wirken) habe. Wie wir \zB\ erfahren, daß der \RWbet{Magnet} das Eisen an sich ziehe, so erfahren wir auch, daß \RWbet{unsere eigene Hand} den Magnet halte, daß wir diese Hand so oder anders bewegen können \usw\ Hier nämlich sind wir uns \RWbet{selbst} das Object der Beobachtung. Wirkungen, welche wir \RWbet{selbst} hervorbringen, und zwar mit dem Bewußtseyn, \RWbet{daß} wir sie hervorbringen, nennen wir unsere \RWbet{Handlungen}. Wir haben also ein Vermögen zu \RWbet{handeln}.
\item Viele unserer Handlungen wirken auf unsern \RWbet{eigenen} Zustand wieder zurück, und bringen bald angenehme bald unangenehme Empfindungen in uns hervor.
\item Und nicht nur auf \RWbet{uns}, sondern auch auf \RWbet{andere} empfindende Wesen, die uns umgeben, Menschen und Thiere, können wir durch unsere Handlungen einwirken, und ihren Zustand oft sehr verändern, in ihnen bald angenehme, bald unangenehme Empfindungen erzeugen.
\item In vielen Fällen können wir auch mit mehr oder weniger Wahrscheinlichkeit vorherwissen, was für einen Einfluß eine gewisse Handlung theils auf uns selbst, theils auch auf Andere haben werde.
\item Wenn wir eine gewisse Handlung schon mehrmals \RWbet{verrichtet}, oder wenn wir dieselbe von andern uns ähnlichen Wesen, \zB\ Menschen, verrichten \RWbet{gesehen}~\RWSeitenw{43}\ haben; so schließen wir hieraus bald mehr bald weniger sicher, daß wir auch \RWbet{jetzt} im Stande seyn werden, sie zu verrichten. So glaube ich \zB , daß ich im Stande wäre, die Gellert'sche Fabel vom Rhinozeros\RWlit{}{Gellert1} herzusagen, oder diesen Schuldbrief zu zerreißen, \udgl
\item Wenn wir uns eine gewisse Veränderung vorstellen, von der wir angenehme Folgen für uns erwarten: so bewirkt die Vorstellung dieser Veränderung ein eigenes Etwas in uns, welches wir einen \RWbet{Wunsch}, eine \RWbet{Begierde}, oder auch ein \RWbet{Verlangen} nennen; wir wünschen, begehren oder verlangen diese Veränderung. Die Fähigkeit, etwas zu wünschen, zu begehren oder zu verlangen, nennen wir das \RWbet{Begehrungsvermögen}, auch in bestimmten Beziehungen einen \RWbet{Trieb}, eine \RWbet{Neigung} \udgl\ Weil nun der Inbegriff alles Angenehmen auch \RWbet{Seligkeit} oder \RWbet{Glückseligkeit} heißt; so nennen wir die Fähigkeit \RWbet{zu wünschen überhaupt} auch den \RWbet{Glückseligkeitstrieb}. Wir Menschen haben also einen Glückseligkeitstrieb, und alle \RWbet{anderen} Triebe sind nur besondere Aeußerungen von diesem.
\item Wenn wir uns vorstellen, daß es uns möglich seyn dürfte, eine gewisse Handlung zu verrichten, durch welche eine Veränderung, die wir \RWbet{wünschen}, herbeigeführt würde: so wünschen wir auch die Verrichtung \RWbet{dieser} Handlung, wir haben ein Verlangen, so zu handeln.
\item Von gewissen Handlungen, welche wir uns als \RWbet{möglich} vorstellen, sprechen wir nach einem \RWbet{Obersatze}, dessen wir uns oft selbst nicht deutlich bewußt sind, das Urtheil aus, daß wir sie ausüben \RWbet{sollen}. Dieser Begriff des \RWbet{Sollens} ist einer der merkwürdigsten, die es im ganzen Umfange des menschlichen Wissens gibt. Es scheint ein völlig \RWbet{einfacher Begriff} zu seyn, der nicht weiter erklärt oder zerlegt werden kann. Mit andern Worten nennen wir das, was Jemand \RWbet{soll}, auch seine \RWbet{Pflicht}, seine \RWbet{Obliegenheit}, \RWbet{Verbindlichkeit} \usf\ Freilich hat jedes dieser Worte auch seine eigenen Nebenbegriffe, wir können hier aber von diesen absehen, und nur auf dasjenige merken, was jene Worte alle gemeinschaftlich bezeichnen. Etwas \RWbet{sollen}, dazu \RWbet{verpflichtet}, dazu \RWbet{verbunden} seyn, \usw~\RWSeitenw{44}\ gilt uns hier also für dasselbe. Die Fähigkeit unsers Erkenntnißvermögens, nach der es ein solches \RWbet{Sollen} oder dergleichen \RWbet{Pflichten} erkennet, hat man die \RWbet{praktische Vernunft} genannt, weil man die Urtheile selbst, in welchen wir ein Sollen oder eine Pflicht aussprechen, \RWbet{praktische Urtheile}, alle andern dagegen bloß \RWbet{theoretische} nennt. Die praktischen Urtheile, \dh\ diejenigen, die irgend ein \RWbet{Sollen} aussagen, werden zuweilen auch \RWbet{sittliche Gesetze}, \RWbet{Sittengesetze}, \RWbet{Forderungen}, \RWbet{Gesetze}, \RWbet{Gebote}, \RWbet{Imperative}, \RWbet{kategorische Imperative} der \RWbet{praktischen Vernunft} genannt. Wenn wir genauer reden, so unterscheiden wir zwischen Urtheilen, die sich auf einen \RWbet{einzelnen bestimmten Fall} beziehen, und solchen, die \RWbet{allgemein} sind. Nur diesen letztern gibt man die Namen: \RWbet{Gesetze}, \RWbet{sittliche Gesetze}, \RWbet{Sittengesetze}. Der Inbegriff aller sittlichen Gesetze heißt auch zuweilen das \RWbet{Sittengesetz} in collectiver Bedeutung. In einem andern Sinne aber verstehen wir unter dem \RWbet{Sittengesetze}, nämlich unter dem \RWbet{obersten Sittengesetze}, nur einen einzigen Satz, aus dem sich alle übrigen praktischen Wahrheiten objectiv herleiten lassen, \dh\ so wie die Folgen aus ihrem Grunde fließen.
\item Wenn uns in einem gegebenen Falle \RWbet{zwei oder mehre Handlungen zugleich als möglich vorschweben}, \dh\ wenn es uns scheint, daß wir im Stande wären, sowohl auf diese, als auch auf jene Art zu handeln: so überlegen wir, was für verschiedene \RWbet{Folgen} aus einer jeden dieser Handlungen hervorgehen würden, und vergleichen sie auch mit den sittlichen Gesetzen. Scheint uns nun Eine dieser Handlungen $A$ besonders \RWbet{angenehme Folgen für uns selbst} zu haben; eine \RWbet{andere} $B$ dagegen dem \RWbet{Sittengesetze gemäß} zu seyn: so fühlen wir von diesem Augenblicke an mit der größten Deutlichkeit, \RWbet{daß es uns möglich sey, die Eine oder die andere aus diesen beiden Handlungen zu wollen}. Der Begriff dieses \RWbet{Wollens} darf nicht mit jenem des \RWbet{Wünschens} (\no\,8) verwechselt werden. Denn daraus, daß wir etwas \RWbet{wünschen}, folgt noch nicht, daß wir es schon \RWbet{wollen}. Die Fähigkeit \RWbet{wollen zu können}, heißt \RWbet{Wille}, \RWbet{Willensvermögen},~\RWSeitenw{45}\ \RWbet{Wollkraft}; und den besonderen Umstand, daß es uns möglich ist, von zwei Handlungen eben sowohl die Eine, als die andere zu wollen, nennt man die \RWbet{Freiheit}\var{\ unsers Willens. Wir Menschen besitzen also \RWbet{Freiheit}}{{\ unsers Willens. Wir Menschen besitzen also \RWbet{Freiheit}}{}{[vacat]}}. Statt zu sagen, wir können von zwei Handlungen die eine oder die andere wollen, pflegt man auch kurz zu sagen: \RWbet{wir können wählen}. Die Freiheit kann also auch als eine \RWbet{Fähigkeit zu wählen} erkläret werden. Noch ist zu erinnern, daß man das \RWbet{Wollen} einer Handlung ja nicht mit dem \RWbet{Hervorbringen} derselben für einerlei halte. Dieß Letztere ist nur eine sehr gewöhnliche, aber zuweilen doch auch ausbleibende \RWbet{Wirkung} des Erstern. Wenn ich \zB\ dieses Papier verbrennen \RWbet{will}, so wird es mir wohl vermuthlich gelingen, es wirklich zu verbrennen; aber es kann sich doch auch fügen, daß ich meine Hand plötzlich gelähmt fühle, oder daß das Papier aus \RWbet{Asbest} bereitet, und also unverbrennlich ist \udgl\ Daher können wir auch niemals mit völliger Gewißheit sagen, daß Eine von jenen beiden Handlungen uns möglich seyn werde; wohl aber, daß es uns möglich sey, \RWbet{Eine von beiden, welche es immer ist, zu wollen}. Aus diesem Grunde wird auch das \RWbet{Sollen} in einem praktischen Satze, wenn man ihn recht genau ausdrücken will, nie auf die \RWbet{Handlung selbst}, auf die Hervorbringung des äußeren Erfolges, sondern nur auf das \RWbet{Wollen} dieses Erfolges gerichtet, \dh\ wir sagen nicht: du sollst Dieses oder Jenes \RWbet{thun}, sondern wir sagen nur: du sollst es \RWbet{wollen}.
\item Wenn wir die Handlung wollen, welche wir wollen \RWbet{\var{sollen}{{sollen}{}{sollten}}}, \dh\ welche zu wollen wir als Pflicht erkennen; so heißt dieser bestimmte Willensentschluß oder Wille \RWbet{ein guter, sittlich guter, moralischer Wille}. Ein Wesen, das den \RWbet{herrschenden} Willen hat, dasjenige zu thun, was es soll, heißt man \RWbet{ein gutes, sittlich gutes, moralisches}, auch \RWbet{tugendhaftes}, ja wohl \RWbet{heiliges} Wesen. Und die Eigenschaft, vermöge deren es das ist, heißt man \RWbet{die Güte, die sittliche Güte, die Moralität, die Tugend}, oder auch \RWbet{die Heiligkeit} desselben. Wenn wir im Gegentheile diejenige Handlung wollen, welche der Glückseligkeitstrieb im Widerspruche mit der~\RWSeitenw{46}\ Forderung unserer Vernunft \RWbet{wünscht}; so heißt dieser bestimmte Willensentschluß \RWbet{ein böser, sittlich böser, unmoralischer, sündhafter Wille}. Ein Wesen, das einen herrschenden Willen hat, Handlungen, welche die Vernunft verbietet, auszuüben, heißt ein \RWbet{böses}, \RWbet{sittlich böses}, \RWbet{unmoralisches}, auch \RWbet{sündhaftes, lasterhaftes} Wesen; und diese Eigenschaft an ihm \RWbet{Bösartigkeit, sittliche Bösartigkeit, Immoralität, Sündhaftigkeit, Lasterhaftigkeit}. -- Wir Menschen sind also eben dadurch, daß wir ein Sittengesetz erkennen, und einen Glückseligkeitstrieb haben, Beides, \RWbet{der Tugend sowohl als auch des Lasters fähig}. Nicht selten versteht man unter dem Namen eines \RWbet{sittlichen} oder \RWbet{moralischen} Wesens auch nur ein Wesen, welches der \RWbet{Tugend} oder des \RWbet{Lasters} fähig ist, und also \RWbet{sittlich} oder auch \RWbet{unsittlich}, \RWbet{moralisch} oder auch \RWbet{unmoralisch} handeln kann. Hier also nimmt man den Ausdruck eines \RWbet{sittlichen} oder \RWbet{moralischen} Wesens in einer \RWbet{weitern} Bedeutung, und versteht darunter ein Wesen, das, wenn es auch nicht in der That \RWbet{sittlich} oder \RWbet{moralisch} ist, es doch seyn könnte. Dann setzt man den \RWbet{sittlichen} oder \RWbet{moralischen} Wesen die \RWbet{nicht freien}, \RWbet{physischen} oder \RWbet{bloßen Naturwesen}, \zB\ die Thiere, entgegen. Es ist auch wohl zu merken, daß jeder Willensentschluß eines vernünftigen Wesens, der nur dem Sittengesetze \RWbet{gemäß} ist, \RWbet{sittlich gut} heiße, auch wenn kein widersprechender Wunsch des Glückseligkeitstriebes, und also keine \RWbet{Freiheit} in der so eben erklärten Bedeutung Statt gefunden. So heißen \zB\ alle Handlungen Gottes \RWbet{heilig}, obgleich kein Glückseligkeitstrieb und keine \RWbet{solche} Freiheit, welche in einer Möglichkeit, das Gegentheil zu thun, bestände, bei Gott anzutreffen ist.
\item Wenn wir mit \RWbet{Freiheit} handeln, \dh\ wenn wir uns entschließen, die \RWbet{eine} oder die andere von zwei Handlungen zu wollen, deren die \RWbet{eine} unsere Vernunft \RWbet{fordert}, die \RWbet{andere} unser Glückseligkeitstrieb \RWbet{wünscht}: so geschieht das immer ohne bestimmenden \RWbet{Grund}. Denn wer behaupten wollte, daß ein \RWbet{bestimmender Grund} vorhanden sey, der uns das eine Mal zur \RWbet{Erfüllung unserer Pflicht}, das andere Mal zur Uebertretung derselben, nämlich zur~\RWSeitenw{47}\ \RWbet{Ausübung der Handlung, die unser Glückseligkeitstrieb wünscht}, bestimme: der könnte diesen Grund höchstens nur darin suchen, daß in dem Einen Falle die Forderung der Vernunft, in dem andern der \RWbet{Wunsch des Glückseligkeitstriebes} stärker sey. Allein eine genauere Betrachtung zeigt, daß diese zwei Dinge nicht gleichartig sind, \dh\ nicht mit einerlei Maß \RWbet{gemessen} werden können; daher man durchaus nicht sagen kann, das Eine sey stärker oder schwächer, als das Andere. Ein \RWbet{Wunsch} ist kein Theil eines \RWbet{Urtheiles}, ein Urtheil kein Theil eines Wunsches. Der so berühmte \RWbet{Satz vom Grunde}, (\RWlat{principium causalitatis}) \RWbet{daß Alles, was geschieht, seinen Grund haben müsse}, ist also nicht allgemein wahr; und sollte eigentlich nur so ausgedrückt werden: \RWbet{Forsche bei Allem nach einem Grunde}; untersuche, \RWbet{ob} einer vorhanden sey, und \RWbet{worin} er bestehe.
\item Inzwischen besteht doch einmal die \RWbet{Redensart}, daß man, wenn wir dasjenige wählen, was unsere \RWbet{Pflicht} war, sagt, \RWbet{es sey die Pflicht} -- -- und im entgegengesetzten Falle, wenn wir dasjenige wählen, was wir \RWbet{gewünscht hatten}, es sey der \RWbet{Glückseligkeitstrieb} -- der \RWbet{Beweggrund unsers Willensentschlusses} gewesen. Hier muß man also unter dem Worte \RWbet{Beweggrund} nicht einen \RWbet{eigentlichen}, \dh\ bestimmenden, sondern nur einen \RWbet{uneigentlich} sogenannten \RWbet{Grund}, eine \RWbet{Bedingung} oder einen \RWbet{Theilgrund} (eine \RWlat{conditio sine qua non}) verstehen. Die Handlungen freier Wesen erfolgen zwar \RWbet{ohne Grund}; aber sie setzen doch dreierlei als \RWbet{Theilgrund} oder \RWbet{Bedingung} (\RWlat{conditio sine qua non}) voraus: 
\begin{inparaenum}[a)] 
\item \RWbet{Das Daseyn eines handelnden Wesens} (wo es kein handelndes Wesen gibt, kann es auch keine freie Handlung geben); 
\item in diesem Wesen die \RWbet{Eigenschaft der Freiheit}, welche wieder 
\begin{inparaenum}[g)] \item eine \RWbet{Kenntniß des Sittengesetzes} (oder eine praktische Vernunft), und 
\item einen \RWbet{Glückseligkeitstrieb}
\end{inparaenum} 
voraussetzt; endlich \item einen wenigstens uneigentlich sogenannten \RWbet{Beweggrund}, \dh\ die Vorstellung zweier Handlungen, welche dem Wesen beide als \RWbet{möglich} erscheinen, und deren eine es durch seinen %
\var{Glückseligkeitstrieb \RWbet{wünscht}, während es von der andern durch~\RWSeitenw{48}\ seine praktische Vernunft erkennet, 
 daß es sie wollen \RWbet{soll}.
\end{inparaenum} 
-- Nur \RWbet{Sittlichkeit} (in der Bedeutung Nr.\,12), nicht aber \RWbet{Freiheit} (in der Bedeutung, wie sie der \RWbet{Mensch} hat) ist eine \RWbet{wahre Vollkommenheit} eines Wesens. Diese Freiheit  nämlich entspringt aus dem Vorhandenseyn eines \RWbet{Glückseligkeitstriebes}}{{Glückseligkeitstrieb \RWbet{wünscht}, während es von der andern durch~\RWSeitenw{48}\ seine praktische Vernunft erkennet, daß es sie wollen \RWbet{soll}. -- Nur \RWbet{Sittlichkeit} (in der Bedeutung Nr.\,12), nicht aber \RWbet{Freiheit} (in der Bedeutung, wie sie der \RWbet{Mensch} hat) ist eine \RWbet{wahre Vollkommenheit} eines Wesens. Diese Freiheit  nämlich entspringt aus dem Vorhandenseyn eines \RWbet{Glückseligkeitstriebes}}{}{[vacat]}},
einer \RWbet{Vernunft} und aus dem Umstande, daß diese beiden zuweilen in einen so genannten Widerspruch treten, \dh\ widerstreitende Handlungsweisen verlangen. Das Daseyn eines \RWbet{Glückseligkeitstriebes} (gewisser Bedürfnisse) und noch mehr jener Streit zwischen den Wünschen des Glückseligkeitstriebes und den Forderungen der Vernunft sind an sich Unvollkommenheiten. Nur \RWbet{relativ}, \dh\ nur im Vergleiche mit noch unvollkommneren Wesen, wie solche, denen alle Empfindung fehlt, oder die keine Vernunft besitzen, ist die moralische \RWbet{Freiheit} ein \RWbet{Vorzug}; verglichen mit \RWbet{Gott}, und an sich selbst betrachtet, ist sie eine Unvollkommenheit; und je weiter der Mensch in der Vollkommenheit fortrückt, desto seltener entsteht ein Streit zwischen den Wünschen seines Glückseligkeitstriebes und den Forderungen seiner Vernunft, desto öfter handelt er \RWbet{tugendhaft}, ohne eigentlich frei gehandelt zu haben. --
\end{aufza}
\begin{RWanm}
Am Schlusse dieses § darf ich nicht unterlassen, zu erinnern, daß man dasjenige, was hier über die \RWbet{Freiheit} des menschlichen Willens gesagt ist, bloß als \RWbet{meine} Ansicht von dieser Sache zu betrachten habe, die irrig seyn kann, ohne daß darum das Uebrige wegfällt. Da bei dieser Ansicht vorausgesetzt wird, daß ein menschlicher Willensentschluß (ein freier wenigstens) gar keinen \RWbet{bestimmenden} Grund habe, so hat man ihr den Namen des \RWbet{Indeterminismus} gegeben. Ihr steht entgegen der \RWbet{Determinismus}, der einen bestimmenden Grund bei jedem Willensentschlusse annimmt. Welche Meinung auch immer die richtige seyn mag: daß unser menschliche Wille eine gewisse Beschaffenheit habe, welche wir unter dem Worte \RWbet{Freiheit} verstehen, daß wir nach dem verschiedenen Gebrauche, den wir von dieser Freiheit machen, bald sittlich gut, bald sittlich böse handeln, daß die sittlich guten Handlungen eine Belohnung, die sittlich bösen eine Bestrafung verdienen, das Alles sind Wahrheiten, die ewig stehen bleiben müssen, weil schon das Urtheil des bloßen gemeinen Menschenverstandes auf eine unfehlbare Art für sie entscheidet.~\RWSeitenw{49}
\end{RWanm}

\RWpar{16}{Viele Begriffe und Meinungen des Menschen haben auf seine Tugend sowohl, als auch auf seine Glückseligkeit Einfluß}
\begin{aufza}
\item Bei jeder Ansicht, welche man von der Freiheit des Menschen annehmen mag, muß man es doch als etwas, das durch die Erfahrung selbst unwidersprechlich dargethan wird, betrachten, daß wir Menschen den Forderungen unserer Vernunft um so öfter und sicherer folgen, also um so tugendhafter werden,
\begin{aufzb}
\item je \RWbet{geläufiger}, und mit je mehr \RWbet{Richtigkeit} und \RWbet{Lebhaftigkeit} wir unsere Pflichten erkennen. Denn wenn uns oft nicht einmal einfällt, was unsere Pflicht sey: so ist es auch nicht möglich, daß wir dieselbe erfüllen;
\item je \RWbet{seltener} die Wünsche unseres Glückseligkeitstriebes mit den Forderungen unserer Vernunft in einen \RWbet{Widerspruch} treten. Denn, nur wenn ein solcher Widerspruch eintritt, ist es uns möglich, unserer Pflicht untreu zu werden;
\item je \RWbet{schwächer} endlich und je \RWbet{gemäßigter} selbst in dem Falle eines solchen Widerspruches die Wünsche unsers Glückseligkeitstriebes sind. Denn um so eher läßt sich dann hoffen, daß diese Wünsche die Vorstellung der Pflicht aus unserem Bewußtseyn nicht verdrängen werden; und so lange nur dieß nicht geschehen ist, ist es uns möglich, ihr zu gehorchen. --
\end{aufzb}
\item Nach dieser Vorausschickung ist es leicht zu zeigen, daß wir sehr viele Begriffe und Meinungen\footnote{%
	Ein für allemal werde erinnert, daß das Wort: \RWbet{Meinung} hier und in allen ähnlichen Fällen in seiner weitesten Bedeutung genommen werde, dergestalt, daß es jeden Satz, den Jemand als wahr annimmt, jedes Urtheil also, welches er fällt, bezeichnet, gleichviel, ob dieses Urtheil an sich selbst wahr oder nicht wahr sey, und gleichviele auch, mit welchem Grade der Zuversicht man dasselbe fälle. Nicht also bloß diejenigen unserer Urtheile, die ungewiß oder wohl gar unrichtig sind, sondern auch diejenigen, die wir mit voller Zuversicht annehmen, und die der Wahrheit ganz gemäß sind, werden seine \RWbet{Meinungen} genannt.}
haben, die~\RWSeitenw{50}\ einen entscheidenden Einfluß bald auf unsere Tugend, bald auf unsere Glückseligkeit, bald auch auf beide äußern. Es ist aber wohl zu bemerken, daß ich hier eigentlich nicht von dem Einflusse rede, den solche Meinungen \RWbet{zufälliger} Weise, sondern bloß von demjenigen, den sie wegen gewisser, in der Natur des Menschen liegender Gründe, und daher \RWbet{allgemein} haben. So kann jede rein mathematische oder physikalische Wahrheit zufällig einen Schaden erzeugen, wenn sie uns \zB\ zu einem Experimente veranlaßt, dabei wir verunglücken; aber dieß ist keine Wirkung, die solche Wahrheiten vermöge eines in der Natur des Menschen liegenden, allgemein geltenden Grundes hervorbringen.
\item Einen solchen allgemein Statt findenden Einfluß äußern zuvörderst auf unsere \RWbet{Tugend}:
\begin{aufzb}
\item die Begriffe, die wir uns von \RWbet{unseren eigenen Pflichten und Obliegenheiten} bilden. Es kann nicht anders als höchst vortheilhaft für unsere Tugend seyn, wenn wir alle unsere Pflichten genau, lebhaft und mit Geläufigkeit erkennen; es muß im Gegentheile einen sehr nachtheiligen Einfluß auf dieselbe haben, wenn wir von mehren unserer Pflichten entweder gar keine, oder doch nur eine dunkle, unsichere und sehr ungeläufige Kenntniß besitzen; denn wie wäre es uns in diesem letztern Falle nur möglich, jeder unserer Pflichten gehörig nachzukommen? --
\item Einen großen Einfluß auf unsere Tugend haben ferner \RWbet{unsere Begriffe von den Folgen, welche aus der Erfüllung oder Nichterfüllung unserer Pflichten für uns selbst hervorgehen werden}. Es kann nicht anders als vortheilhaft für unsere Tugend seyn, wenn wir überzeugt sind, daß fast eine jede gute Handlung schon durch den natürlichen Lauf der Dinge früher oder später sich lohne, daß jedes Laster dagegen seine Strafe bei sich führe; daß überdieß ein gerechter Gott lebe, der dafür sorgt, daß nicht die geringste gute oder böse Handlung je unvergolten bleibe. Es muß dagegen einen sehr schädlichen Einfluß auf unsere Tugend haben, wenn wir vermeinen, daß jede Pflicht\RWSeitenw{51}erfüllung etwas Beschwerliches habe; daß die Tugend den Menschen wenigstens hienieden immer nur unglücklich mache; daß sich der Lasterhafte mindestens hier auf Erden in einem beneidenswerthen Glückszustande befinde; daß die Vergeltung, die erst im andern Leben eintreten soll, zweifelhaft sey, \usw\ Im ersten Falle wird sich viel seltener ein Streit zwischen den Forderungen unserer Vernunft und den Wünschen unsers Glückseligkeitstriebes erheben; und selbst wenn er sich erhebt, werden die Wünsche nicht so heftig und unbändig seyn, als in dem letztern Falle.
\end{aufzb}
\item Es gibt auch Begriffe und Meinungen, die \RWbet{auf unsere Glückseligkeit} einen allgemein geltenden Einfuß äußern:
\begin{aufzb}
\item Die Begriffe, die wir uns \RWbet{von dem wahren Wesen der uns erreichbaren Glückseligkeit und von den Mitteln und Bedingungen derselben bilden}, haben begreiflich den größten Einfluß auf unsere Glückseligkeit. Es kann nicht anders als überaus vortheilhaft seyn, wenn diese Begriffe \RWbet{richtig} und \RWbet{vollständig} sind; es muß uns dagegen die größten Nachtheile zuziehen, wenn wir das wahre Wesen der Glückseligkeit verkennen, wenn wir von manchen Mitteln und Wegen, die zur Glückseligkeit führen, nichts wissen, und dafür Manches für ein Mittel unserer Beglückung ansehen, was es in Wirklichkeit nicht ist. In diesem Falle nämlich steht zu besorgen, daß wir so Manches unterlassen werden, was unsere Glückseligkeit vermehrt haben würde, und Manches thun, wodurch wir uns am Ende nur unglücklich machen.
\item Einen großen Einfluß auf unsere Glückseligkeit haben auch gewisse \RWbet{erfreuliche oder unerfreuliche Begriffe, die wir uns von solchen Dingen bilden, deren Aenderung gar nicht in unserer Macht steht}, \zB\ vom Tode und seiner Furchtbarkeit, vom andern Leben \usw\ Wer sich von solchen Dingen \RWbet{erfreuliche Vorstellungen} macht, gewinnt an Heiterkeit, dagegen derjenige, der von eben diesen Din\RWSeitenw{52}gen allerlei Schlimmes besorgt, das er doch nicht abwehren kann, durch seine furchtbaren Vorstellungen ohne Noth gequält wird.
\item Endlich haben \RWbet{alle solche Meinungen, die einen Einfluß auf unsere Tugend haben}, schon eben darum auch einen \RWbet{mittelbaren Einfluß} auf unsere Glückseligkeit. Denn weil es, wie in der Folge noch eigends dargethan werden soll, einen Gott gibt, der jede gute That belohnt und jede böse bestrafet; so wird ein jedes vernünftige und der Tugend fähige Wesen während seines ganzen Daseyns nur in dem Maße glücklich, in welchem es tugendhaft lebt, und um so unglücklicher, je lasterhafter es wird. Was also immer unsere Tugend befördert, befördert mittelbar auch unsere Glückseligkeit; und was unsere Tugend beeinträchtiget, macht auch unserem Glücke Abbruch.
\end{aufzb}
\item  Daß es übrigens auch Begriffe und Meinungen gebe, denen kein solcher allgemein geltender Einfluß auf unsere Tugend und Glückseligkeit beigelegt werden kann, bedarf nicht besonders erwiesen zu werden. Wer wird \zB\ läugnen, daß es für die Zwecke der Tugend und Glückseligkeit des Menschen, um allgemein zu reden, gleichgültig sey, ob er sich vorstelle, daß die dunklen Flecken in der Mondscheibe Thäler oder Meere sind? \udgl
\item Sollte es nöthig seyn, den hier bemerkbar gemachten Unterschied zwischen den menschlichen Meinungen durch ein Paar eigene Benennungen zu bezeichnen; so könnte man die Begriffe und Meinungen der letztern Art wohl am füglichsten -- \RWbet{gleichgültige}; jene der ersten aber \RWbet{wichtige}, \RWbet{allgemein wichtige} nennen.
\end{aufza}

\RWpar{17}{Wir wünschen es auch zuweilen, gewisse Meinungen zu haben}
Aus demjenigen, was ich so eben über die Art und Weise gesagt, wie viele unserer Meinungen einen bald größern bald geringern Einfluß auf unsere Tugend und Glück\RWSeitenw{53}seligkeit haben, gehet hervor, daß wir gar oft im Stande seyn müssen, diesen Einfluß selbst zu bemerken, ja auch bei Meinungen, die wir nicht wirklich haben, zu beurtheilen, \dh\ uns mit mehr oder weniger Richtigkeit vorzustellen, von welchem Einflusse sie, wenn wir sie hätten, seyn würden. Hieraus ergibt sich nun weiter, es könne oft auch der Wunsch, daß wir doch einer gewissen Meinung zugethan wären, in uns zum Vorschein kommen. Wenn wir nämlich vorhersehen, daß ein gewisser Glaube, \zB\ der an die Unsterblichkeit unserer Seele, unserer Tugend sehr zuträglich seyn würde: so werden wir, wenn wir das Gute aufrichtig wollen, alsbald den Wunsch verspüren, daß wir doch in der That diesen Glauben besäßen. In diesem Falle also ist es unsere sittliche Gesinnung, von welcher der Wunsch, daß eine gewisse Meinung die unsrige wäre, hervorgehet. Noch öfter tritt der Fall ein, daß wir vorhersehen, oder uns wenigstens einbilden vorherzusehen, daß eine gewisse Meinung, wenn wir ihr zugethan wären, uns von verschiedenen uns lästigen Verbindlichkeiten befreien, oder gewisse andere wahre oder nur scheinbare Vortheile bringen würde. Wer nicht sehr tugendhaft ist, erwehrt sich in solchen Fällen schwerlich des Wunsches, daß er doch wirklich so glaubte!

\RWpar{18}{Wir haben auf die Entstehung unserer Meinungen einen beträchtlichen Einfluß durch unsern Willen}
\begin{aufza}
\item Wahr ist es allerdings, daß das Geschäft des \RWbet{Urtheilens}, also auch all unser Meinen und Führwahrhalten unserm Willen nicht \RWbet{unmittelbar} unterstehe. Wir können nicht unmittelbar, bloß weil wir es wollen, das Urtheil, daß etwas so oder anders sey, in uns hervorbringen, sondern es hängt im Gegentheil von der Beschaffenheit der in unserem Gemüthe so eben vorhandenen \RWbet{Vorstellungen} ab, daß unsere Urtheilskraft gerade dieses und nicht ein anderes Urtheil über die vorgestellten Gegenstände fället. Sie thut dieß mit einer Art von Nothwendigkeit. Nichts desto weniger bleibt es gewiß, daß unser Wille~\RWSeitenw{54}\ einen bedeutenden \RWbet{mittelbaren} Einfluß auf die Beschaffenheit unserer Urtheile, und somit auch auf die Entstehung unserer Meinungen habe. Unser Wille hat nämlich einen unläugbaren Einfluß:
\begin{aufzb}
\item auf unsere \RWbet{Aufmerksamkeit}. Wir können, wenn wir wollen, (ob dieß schon unmittelbar, oder noch abermals durch eine Vermittlung erfolge, will ich hier nicht untersuchen) die Aufmerksamkeit unseres Geistes auf gewisse Vorstellungen richten, bei ihnen verweilen, sie durch dieses Verweilen lebhafter, stärker machen \usw ; oder wir können im Gegentheile unsere Aufmerksamkeit von diesen Vorstellungen abziehen, und dadurch machen, daß sie allmählich aus unserem Bewußtseyn sich verlieren.
\item Unser Wille hat ferner einen unläugbaren Einfluß auf unsern \RWbet{Körper}, und dadurch abermals auf eine Menge anderer \RWbet{äußerer Gegenstände}. Wir können, wenn wir wollen, verschiedene Bewegungen in den Gliedmaßen unsers Körpers hervorbringen, und hierdurch auf die uns umgebenden Körper und Gegenstände verändernd einwirken, sie bald uns nähern, bald entfernen \usw\ Durch alles Dieses aber wird sehr begreiflich, wie unser Wille einen beträchtlichen Einfluß auf unsere Meinungen ausüben könne. Je nachdem wir nämlich die Aufmerksamkeit unseres Geistes auf diese oder jene sinnlichen Gegenstände richten, entstehen bald diese, bald jene sinnlichen Vorstellungen, Erfahrungen und Erkenntnisse in uns; je nachdem wir mit unserem Nachdenken bei diesen oder bei jenen Vorstellungen länger verweilen, sie mit einander vergleichen \usw : kommen bald diese, bald jene Ueberzeugungen bei uns zu Stande; je nachdem wir uns an diesem oder an jenem Ort viel aufhalten, mit diesen oder mit jenen Personen umgehen, diese oder jene Bücher lesen \udgl , werden bald diese, bald jene Begriffe und Meinungen in uns entwickelt.
\end{aufzb}
\item Doch ich behaupte noch mehr: Wir können den Einfluß, den unser freiwilliges Betragen auf die Ausbildung unserer Begriffe und Meinungen haben wird, oft auch \RWbet{vorhersehen}. Oder können wir nicht vorherwissen, daß wir zu dem~\RWSeitenw{55}\ Besitze gewisser Kenntnisse und Einsichten gewiß nicht gelangen werden, wenn wir die Mittel, die zu denselben führen, die nöthige Aufmerksamkeit, das erforderliche Nachdenken, die Bücher \udgl\ nicht anwenden? So können auch viele Menschen von sich vorauswissen, daß sie die Meinungen derer, mit denen sie viel umgehen, annehmen werden, wenn sie bereits aus ihrer frühern Erfahrung wissen, daß sie gewöhnlich den Meinungen derer, mit denen sie umgingen, beitraten.
\item Ich behaupte endlich: Der Einfluß, welchen der Mensch durch seinen Willen auf die Entstehung seiner Meinungen hat, reicht so weit, daß wir uns, wenn wir wollen, oft \RWbet{sogar selbst täuschen}, oder (wie man auch sagt) \RWbet{überreden} können, \dh\ daß wir uns absichtlich so betragen können, daß eine gewisse Meinung, die wir anfänglich selbst noch für unrichtig oder doch unerwiesen hielten, zuletzt uns eigen werde. Der Mensch kann nämlich:
\begin{aufzb}
\item (wie wir im vorigen § schon sahen) den Wunsch fassen, daß er einer gewissen Meinung selbst wirklich zugethan wäre; und weil er einen Einfluß auf seine Meinungen hat, den er vorherzusehen vermag: so kann er es
\item \RWbet{versuchen}, diese Meinung wirklich in sich hervorzubringen; kann seine Aufmerksamkeit absichtlich auf alle wahren oder nur scheinbaren Gründe derselben richten; von Allem, was ihr entgegensteht, sein Augenmerk abziehen; Umgang mit Menschen pflegen, die dieser Meinung zugethan sind; Bücher, in denen sie vertheidiget wird, lesen \usw\ Durch alle diese Mittel kann es endlich
\item in der That geschehen, daß er dieser Meinung mit einem bald größern, bald geringern Grade der Zuversicht anhängt.
\end{aufzb}
\item Gegen die hier behauptete Möglichkeit einer absichtlichen Selbsttäuschung lassen sich jedoch einige \RWbet{Einwürfe} machen, die ich in Kürze anführen und beantworten muß. Man kann nämlich
\begin{aufzb}
\item sagen, daß es schon dem natürlichen \RWbet{Triebe nach Wahrheit} zuwider laufe, daß Jemand den Vorsatz~\RWSeitenw{56}\ fassen sollte, sich selbst zu täuschen. Wohl geschehe es nur zu oft, daß sich der Mensch, ohne es selbst zu wissen und zu wollen, täuscht; aber nie könne er getäuscht zu werden \RWbet{wünschen}. Allein gesetzt, daß er es wirklich \RWbet{wünschte}: so wäre es doch
\item nicht möglich, es auszuführen. Denn sich \RWbet{mit Wissen täuschen}, ist ja ein Widerspruch. Wenn Jemand es weiß, daß eine Meinung falsch ist, so ist es eben darum nicht \RWbet{seine} Meinung; er kann höchstens \RWbet{vorgeben}, daß er so glaube, aber nicht wirklich so glauben.
\item Endlich erfolgt ja alles Urtheilen oder Führwahrhalten nur nach gewissen nothwendigen Gesetzen. Sehen wir die Gründe für eine Wahrheit ein, so ist es uns unmöglich, sie zu läugnen; und im Gegentheile, so lange wir noch \RWbet{keine} Gründe haben, ist es uns unmöglich, sie zu glauben.
\end{aufzb}
\item Hierauf erwidere ich nun:
\begin{aufzb}
\item Der natürliche Trieb nach Wahrheit ist bei uns Menschen nicht so groß, daß wir in einzelnen Fällen nicht wünschen könnten, etwas auch lieber \RWbet{nicht zu wissen}, oder uns wohl gar das Gegentheil von dem, was wirklich ist, einzubilden. So ist es \zB\ eine nur zu bekannte Sache, daß sich derjenige, der etwas Böses gethan hat, gerne überreden möchte, daß er es nicht gethan habe, oder daß es nichts so Böses gewesen \usw\
\item Es ist freilich nicht möglich, daß man in eben dem Augenblicke, da man den Vorsatz der Selbsttäuschung faßt, die Täuschung auch schon ausgeführt habe; aber es kann doch bald nachher geschehen. So lange wir uns der Absicht, uns selbst zu täuschen, noch deutlich bewußt sind, so lange haben wir uns freilich noch nicht getäuscht; aber auf diese Absicht vergessen wir in der Folge, und dann erst tritt die eigentliche Täuschung ein.
\item Das Urtheilen geschieht freilich nach nothwendigen Gesetzen; aber die Richtung der Aufmerksamkeit unseres Geistes ist frei, und je nachdem wir diese bald da bald~\RWSeitenw{57}\ dorthin wenden, können wir bald Gründe \RWbet{für} eine Meinung, bald Gründe \RWbet{wider} sie zu sehen glauben.
\end{aufzb}
\end{aufza}

\RWpar{19}{Begriff eines sittlichen oder moralischen Satzes}
Es gibt (wie schon im vorigen § nachgewiesen wurde) Meinungen, die wir uns wünschen, oder (was eben so viel heißt) für deren Annahme irgend ein Wunsch, (man sagt zuweilen auch ein \RWbet{Interesse}) in unserem Herzen sich reget. Da aber zu jeder Meinung andere angeblich sind, welche ihr widersprechen: so ist leicht einzusehen, daß es auch Meinungen gebe, die wir uns nicht wünschen, die uns vielmehr zuwider sind, oder gegen deren Annahme sich ein gewisser Wunsch unsers Herzens sträubet. Weil es \zB\ angenehm ist, sich zu denken, daß man Verdienste um Andere habe: so ist es im Gegentheil unangenehm, überführt zu werden, daß man vielmehr geschadet als genützet habe; \udgl\ Wir können also im Allgemeinen sagen, es gebe Meinungen, die unserem Begehrungsvermögen \RWbet{nicht gleichgültig} sind, sondern für oder wider deren Annahme dasselbe stimmet. Da nun (wie in dem letzten § gezeigt ward) unser Wille einen beträchtlichen Einfluß auf die Beschaffenheit unserer Meinungen hat; so läßt sich wohl erachten, daß bei den Meinungen der eben beschriebenen Art, die unserm Begehrungsvermögen nicht gleichgültig sind, zwar eben nicht immer, doch in sehr vielen Fällen, eine eigene Versuchung eintrete, sie, ohne hiezu gehörig berechtigt zu seyn, entweder anzunehmen oder auch zu verwerfen. So fühlen wir uns, \zB\ wenn wir etwas Böses gethan, nicht nur versucht, uns zu überreden, daß es nichts so gar Strafwürdiges sey, sondern wir überreden uns wirklich nur allzuoft von dieser Meinung. Bei einer nähern Betrachtung aber zeigt sich, es könne beinahe kein Satz so gleichgültig seyn, daß nicht durch zufällige Verhältnisse wenigstens auf einige Zeit ein Wunsch für~\RWSeitenw{58}\ oder wider seine Annahme, und mithin auch ein Verlangen, ihn ohne hinreichende Gründe entweder zu behaupten, oder im Gegentheil zu verwerfen, in uns hervorgebracht werden könnte. Was mag \zB\ für unser Begehrungsvermögen an sich gleichgültiger seyn, als ob irgend eine rein mathematische Wahrheit so oder anders laute? Dennoch, wenn wir es selbst gewesen sind, die diesen Satz gefunden, und wenn wir befürchten müssen, falls er unrichtig wäre, darüber verspottet zu werden: so wünschen wir alsbald, er möchte wahr seyn, und gerathen in Versuchung, wo möglich, uns und Andere von seiner Richtigkeit zu überreden. Wohl gibt es aber auch Sätze, die nicht vermöge eines bloß zufälligen Verhältnisses, sondern durch einen in unserer Natur liegenden und mithin allgemein geltenden Grund unser Begehrungsvermögen für oder wider sich einnehmen, und dadurch eine Versuchung erzeugen, sie ohne hinlänglichen Grund entweder anzunehmen oder zu verwerfen. Von dieser Art sind \zB\ die wichtigen Sätze, daß keine Lüge erlaubt sey; daß wir der Seele nach unsterblich sind; daß eine einzige mit deutlichem Bewußtseyn verübte böse That uns ein ewig dauerndes Unglück als Strafe zuziehen könne; \udgl. Mir däucht nun diese Art von Sätzen so merkwürdig, daß ich ein eigenes Wort zu ihrer Bezeichnung wünschte; und da ich sonst keinen schicklicheren Ausdruck kenne: so erlaube ich mir, sie \RWbet{sittliche} oder \RWbet{moralische Sätze} zu nennen. Ein sittlicher oder moralischer Satz heißt mir also ein Satz, in Betreff dessen es einen in der Natur des Menschen liegenden Grund zu der Versuchung gibt, ihn, ohne hiezu hinlänglich berechtiget zu seyn, entweder als wahr anzunehmen, oder als falsch zu verwerfen. Ich habe die Benennung: \RWbet{sittlich} oder \RWbet{moralisch} gewählt, weil das Verhalten, welches wir gegen solche Sätze beobachten (unser Fürwahrhalten oder auch unser Verwerfen derselben) insgemein etwas \RWbet{Sittliches} oder \RWbet{Moralisches} in dieser Worte weiterer Bedeutung, \dh\ eine Handlung ist, die entweder sittlich-gut oder sittlich-böse genannt werden darf. Denn widerstehen wir der Versuchung, einen solchen~\RWSeitenw{59}\ Satz ohne hinreichenden Grund entweder anzunehmen oder zu verwerfen, prüfen wir ihn mit aller nur möglichen Strenge, obgleich wir den Wunsch, ihn wahr zu finden, haben: so ist unser Betragen gewiß sehr lobenswerth. Eben so tadelnswerth ist es in dem entgegengesetzten Falle, wenn wir, nachgebend der Versuchung, uns von der Wahrheit des Satzes überreden, weil er uns angenehm ist, oder seiner Erkenntniß widerstreben, weil er uns unangenehm ist.

\RWpar{20}{Begriff des Wortes Religion}
Erst nach allen diesen Vorausschickungen glaube ich den Begriff erklären zu können, der meiner Meinung nach mit dem Worte \RWbet{Religion} verbunden werden sollte.
\begin{aufza}
\item Wir haben nämlich (\RWparnr{16}) gesehen, daß es sehr viele Begriffe und Meinungen gibt, die für die Zwecke der Tugend und Glückseligkeit des Menschen nicht gleichgültig sind, die vielmehr einen allgemein geltenden entweder wohlthätigen oder verderblichen Einfluß bald auf die Tugend, bald auf die Glückseligkeit der Menschen äußern.
\item Es läßt sich leicht erachten, daß ein großer Theil dieser Meinungen zugleich auch zu der Art derer gehöre, die ich (im vorigen §) \RWbet{sittliche} oder \RWbet{moralische Sätze} nannte; \dh\ daß in denselben ein eigener aus unserer menschlichen Natur hervorgehender Grund zu der Versuchung liege, sie, ohne hiezu berechtiget zu seyn, als wahr anzunehmen, oder als falsch zu verwerfen.
\item Gleichwohl ist dieses nicht immer der Fall, und es gibt Sätze, die einen beträchtlichen Einfluß auf unsere Tugend und Glückseligkeit haben, ohne daß sie doch eine Versuchung enthalten, sich entweder für oder wider sie zu erklären. So ist dieß namentlich dort, wo durch die neue Kenntniß nicht sowohl neue beschwerliche Pflichten für uns entstehen, sondern nur neue willkommene Mittel zur Befriedigung unserer Bedürfnisse uns dargeboten werden, wie \zB\ bei medicini\RWSeitenw{60}schen Wahrheiten, ingleichen da, wo wir die neuen Pflichten, die uns die Annahme eines Satzes auflegen wird, oder das Unangenehme, welches für unsere sinnliche Natur aus ihm hervorgehen mag, noch gar nicht ahnen; oder auch dort, wo es uns alsbald als etwas völlig Unmögliches einleuchtet, uns der Anerkennung des Satzes erwehren zu wollen, weil seine Wahrheit zu offenbar in's Auge strahlet; \udgl\
\item Wie es aber Sätze gibt, die einen beträchtlichen Einfluß auf unsere Tugend und Glückseligkeit haben, und doch nicht sittlicher Art sind; so kann es auch umgekehrt zuweilen Sätze geben, die zu den sittlichen gerechnet werden dürfen, und dennoch von keinem beträchtlichen Einflusse auf Tugend und Glückseligkeit sind. So ist es namentlich, wenn es vermöge einer uns Menschen natürlichen Täuschung den Anschein gewinnt, daß irgend ein Satz, wofern er wahr wäre, von lästigen Pflichten uns befreien würde, während es doch bei einer nähern Betrachtung sich zeigt, daß er zu solchen Folgerungen gar nicht berechtige.
\item Es kann also jede der zwei Beschaffenheiten: die \RWbet{Wichtigkeit} eines Satzes sowohl, als auch die \RWbet{Sittlichkeit} desselben ohne die andere seyn; wenn aber beide vereinigt sind, \dh\ wenn der Satz nicht nur von einer solchen Art ist, daß seine Betrachtung in unserem Herzen eine eigene Versuchung weckt, sich entweder für oder wider ihn zu bestimmen, sondern wenn durch seine Annahme oder Verwerfung auch der Grad unserer Tugend und Glückseligkeit eine Veränderung erfährt: so ist derselbe gewiß einer ganz eigenen Aufmerksamkeit werth. Ich erlaube mir also einen solchen Satz einen \RWbet{religiösen} zu nennen und verstehe sonach unter \RWbet{Religion}, wenn ich dieß Wort in seinem \RWbet{subjectiven} Sinne nehmen soll, einen Inbegriff aller derjenigen Meinungen eines Menschen, die religiös sind, oder mit anderen Worten, die \RWbet{Religion} eines Menschen, heißt mir der Inbegriff aller derjenigen Meinungen dieses Menschen, die einen entweder wohlthätigen oder~\RWSeitenw{61} nachtheiligen Einfluß auf seine Tugend oder auf seine Glückseligkeit äußern, und zugleich so beschaffen sind, daß eine eigene Versuchung da war, sich ohne gehörigen Grund entweder für oder wider sie zu bestimmen.
\item Aus dieser subjectiven Bedeutung des Wortes ergibt sich leicht seine \RWbet{objective}. Wenn wir uns nämlich einen Inbegriff religiöser Sätze mit der Bestimmung denken, daß diese Sätze die Religion eines Menschen wohl ausmachen \RWbet{könnten}, ohne jedoch vorauszusetzen, daß sie \RWbet{wirklich} von irgend Jemand geglaubt und angenommen werden: so denken wir uns den Begriff der Religion in \RWbet{objectiver} Bedeutung.
\item In beiden Fällen nehmen wir aber dieses Wort in einer Bedeutung, die \RWbet{weiter} als die gewöhnliche ist. Denn in dieser wird, wenn ich nicht irre, unter Religion nichts Anderes verstanden, als der Glaube an Gott und der Inbegriff aller derjenigen sittlichen Meinungen eines Menschen, die seine Verhältnisse und Pflichten gegen Gott betreffen.
\end{aufza}

\RWpar{21}{Rechtfertigung dieser Erklärungen}
\begin{aufza}
\item Es wird zweckmäßig seyn, vor allem Andern zu zeigen, daß die Erklärung, die ich im vorigen § von dem Begriffe der Religion in der \RWbet{gewöhnlichen} Bedeutung (Nr.\,9.) gab, wirklich das ausdrücke, was man mit diesem Worte bezeichnet. Zu diesem Zwecke muß ich verlangen, daß man sich die verschiedenen Fälle, in welchen das Wort \RWbet{Religion} und einige davon abgeleitete Worte gebraucht werden, vergegenwärtige, und untersuche, ob nicht in allen diesen Fällen die hier gegebene Erklärung angewendet werden könne. Wir sagen \zB\ von Jemand, er habe \RWbet{keine Religion}, wenn wir so viel sagen wollen, als er glaube an keinen Gott, oder er glaube doch wenigstens nicht, daß er gewisse Pflichten gegen Gott habe. Wir nennen \RWbet{religiöse} Gefühle, Gefühle solcher Art, die durch den Glauben an Gott und durch die Vorstellung, welche Jemand von Gott besitzt, gebildet oder doch~\RWSeitenw{62}\ davon abgeleitet werden. Wir erzählen Jedem, dem wir die Religion eines Volkes, \zB\ der Griechen, darstellen wollen, welche Begriffe sie von Gott gehabt, wie sie geglaubt, diesen Gott verehren zu müssen, \usw\ Alles genau, wie unsere Erklärung verlangt.
\item In der That sind auch die meisten Erklärungen, die man von dem Begriffe der Religion bisher gegeben hat, mit jener obigen fast ganz übereinstimmend. Nur den Beisatz, daß die Lehren der Religion \RWbet{sittlicher} Art seyn müßten, \dh\ daß man nur solche Meinungen über die Verhältnisse und Pflichten des Menschen gegen Gott zur Religion zu zählen habe, die einer eigenen Versuchung ausgesetzt sind, ohne hinreichenden Grund entweder geglaubt oder bestritten zu werden, hat man, so viel ich wüßte, bisher noch nie gemacht. Allein mir däucht, daß man sich diese Bedingung immer \RWbet{stillschweigend} hinzu gedacht habe. Denn
\begin{aufzb}
\item erstlich ist doch außer Zweifel, daß es gar manche Lehre und Behauptung gebe, die von Gott handelt, und die man gleichwohl nie zur Religion gezählt hat, bloß weil sie gleichgültig ist, oder weil sich im menschlichen Herzen kein Grund zu der Versuchung vorfindet, sich mit Leidenschaft entweder für oder wider sie zu erklären. So hat man die Frage, ob diese oder jene von den Gelehrten versuchte \RWbet{Erklärung des Begriffes von Gott}, ingleichen der oder jener versuchte Beweis für das Daseyn Gottes, den Regeln der Logik auf das Vollkommenste entspreche, niemals zur Religion gezählt.
\item Und hat man nicht oft erinnert, die Lehren der Religion (der wahren) wären von einer solchen Beschaffenheit, daß zu ihrer Anerkennung ein sittlich-guter Wille erforderlich sey; ein sittlich-böser Mensch könne es, wenn er wolle, immer dahin bringen, daß sie ihm zweifelhaft erscheinen? Hieraus aber folgt schon von selbst, daß alle Lehren der Religion das seyen, was ich (\RWparnr{19}) \RWbet{sittliche} Lehren nannte.
\end{aufzb}
\item Vorausgesetzt nun, daß die im vorigen § (\no\,7) versuchte Erklärung den Begriff der Religion in der gewöhnlichen Bedeutung richtig darstelle: so muß ich zwar gestehen, daß die Erklärung der \no\,5 u.\ 6 einen~\RWSeitenw{63}\ Begriff, der etwas \RWbet{weiter} ist, enthalte; allein, ich glaube behaupten zu dürfen, daß der Unterschied lange so groß nicht sey, als man sich ihn beim ersten Anblicke vorstellt; und daß es seine entschiedenen Vortheile hätte, wenn man, ohne die engere, bisher gewöhnliche Bedeutung des Wortes zu verdrängen, auch jene weitere einführte und zuweilen gebrauchte. 
\item \RWbet{Weiter} ist der Begriff, den ich in \no\,5 u.\ 6 aufgestellt habe; denn er umfasset alles Dasjenige, was man bisher zur Religion gezählt hat, und noch Einiges, was man nicht immer dazu zählet.
\begin{aufzb}
\item Er umfasset Alles, was man bisher zur Religion gezählt hat; denn wirklich ist doch Alles, was nicht etwa nur von einem einzelnen Gelehrten, sondern von ganzen Gesellschaften, \zB\ von allen Christen, zur Religion gezählt worden ist, von einem nicht unbemerkbaren Einflusse auf die Tugend und Glückseligkeit der Menschen, und zugleich von der Art, daß eine eigene Versuchung da ist, zu Einem von Beidem, entweder sich davon, daß es wahr, oder sich davon, daß es falsch sey, zu überreden. Also verhält es sich schon mit der Behauptung, daß ein Gott sey; also verhält es sich auch mit den verschiedenen Begriffen, die man uns von unseren Verhältnissen und Pflichten gegen Gott in den verschiedenen Religionen beibringen will.
\item Jener Begriff umfasset aber auch noch ein und das Andere, was man bisher nicht allgemein zur Religion gezählt hat. So hat man \zB\ nicht immer daran gedacht, auch die Begriffe, welche ein Mensch über das wahre Wesen der ihm auf Erden möglichen Glückseligkeit, und über die Mittel und Bedingnisse zu ihrer Erreichung heget, zu seiner Religion zu zählen. Nach der Erklärung aber, die oben aufgestellt wurde, gehören diese Begriffe allerdings auch dazu, weil sie von großer Wichtigkeit sind, und eine sittliche Beschaffenheit haben. Nach der hier vorgeschlagenen neuen Bedeutung des Wortes dürfte man (was das Befremdendste ist) sagen, daß selbst der Atheist (\dh\ der Gottesläugner) Religion noch habe; denn auch er hat gewisse Meinungen, die einen Einfluß, einen verderblichen nämlich, auf seine Tugend und Glückseligkeit äußern, und~\RWSeitenw{64}\ sittlicher Art sind. Nach dem gewöhnlichen Sprachgebrauche dagegen sagt man von solchen Menschen, sie hätten gar keine Religion, sie wären \RWbet{irreligiöser Gesinnung}.
\end{aufzb}
\item Indessen ist doch der Unterschied zwischen dem einen und dem andern Begriffe lange so groß nicht, als es beim ersten Anblicke scheint. Denn auf den ersten Blick möchte man glauben, daß zur Religion in dem gewöhnlichen Sinne des Wortes nicht einmal die Lehre von den sämmtlichen \RWbet{Pflichten} des Menschen, von seinen Pflichten gegen sich selbst, gegen andere Menschen \usw\ gezählt werden könne; daß eben so wenig die Lehre von der Unsterblichkeit unserer Seele, von den Belohnungen und Strafen des andern Lebens, und viele andere dergleichen Wahrheiten von größter Wichtigkeit hiehier gehörten. Allein so ist es nicht; sondern über alle so eben genannte, und über noch gar manche andere Gegenstände von ähnlicher Art, verbreitet sich ja selbst der gemeinste Religionsunterricht, wenigstens wie er in unsern Tagen ertheilt wird. Man findet Mittel hiezu, indem man (was sehr zu loben ist) die sämmtlichen Pflichten des Menschen als göttliche Gebote, und somit als Pflichten, die er auch gegen Gott selbst hat, darstellt. Andere Lehren dagegen, \zB\ die von der Unsterblichkeit der menschlichen Seele, von ihren künftigen Schicksalen \usw\ bezieht man in das Gebiet der Religion gewöhnlich durch den Umstand, daß man sie als \RWbet{von Gott geoffenbaret} ansieht.
\item Die bisher so gewöhnliche Bedeutung ganz verdrängen zu wollen, wäre ein zweckwidriges Beginnen.
\begin{aufzb}
\item Für's Erste schon, weil es ein vergebliches wäre; denn da das Wort Religion eines derjenigen ist, die sich in Aller Munde befinden, und da wir ganze auf jener Bedeutung desselben gegründete Redensarten haben, wie: Ein Mann ohne Religion; Ein irreligiöses Betragen \udgl : so würde es irgend ein Einzelner, der uns den bisherigen Gebrauch des Wortes verbieten wollte, gewiß nicht dahin bringen, daß ihm alle Uebrigen gehorchten.~\RWSeitenw{65}
\item Um so weniger, da diese alte Bedeutung des Wortes, wenn sie nicht ihre eigenen Vortheile hat, wenigstens unschädlich ist. Sollte es nämlich nicht seine eigenen Vortheile haben, wenn wir unsere sämmtlichen Pflichten, auch jene, welche uns gegen uns selbst und unsere Nebenmenschen obliegen, aus einem Gesichtspuncte auffassen lernen, aus welchem sie uns als Pflichten gegen Gott selbst erscheinen? Sollte es nicht gut seyn, um unsern Abscheu vor dem Systeme des Atheismus desto sichtbarer zu machen, ihm den Namen einer Religion gar nicht zugestehen zu wollen?
\end{aufzb}
\item Allein ohne von der gewöhnlichen Bedeutung ganz abzugehen, kann man zuweilen wenigstens mit Nutzen auch jene weitere gebrauchen.
\begin{aufzb}
\item Erstlich ist es doch gewiß gut, daß man die Menschen auch auf denjenigen Theil ihrer Meinungen, die einen beträchtlichen Einfluß auf ihre Tugend und Glückseligkeit haben, und zugleich so beschaffen sind, daß eine eigene Versuchung zur Parteilichkeit bei ihnen obwaltet, besonders aufmerksam mache. Denn gerade in Betreff dieser Meinungen liegen ihnen eigene Pflichten ob, die sie unmöglich erfüllen können, wenn sie nicht immer aufmerksam auf jene Meinungen sind. Um diese Aufmerksamkeit zu bewirken, ist nun das Erste und Nöthigste, jene Meinungen insgesammt unter einen \RWbet{eigenen Begriff} zusammen zu fassen und diesem Begriffe ein eigenes \RWbet{Zeichen} in der Sprache zu geben. In unserer \RWbet{deutschen} Sprache aber besteht kein anderes Wort, welches man schicklicher zu diesem Zeichen wählen könnte, als das Wort -- \RWbet{Religion}. Zwar ist noch \RWbet{Ein} Wort vorhanden, das sich gewissermaßen auch hieher schicken würde, nämlich das Wort: \RWbet{Weisheit}, oder das noch bestimmtere: \RWbet{Lebensweisheit}. So nämlich könnte man allenfalls den Inbegriff aller Meinungen nennen, die auf die Tugend und Glückseligkeit des Menschen einen \RWbet{wohlthätigen} Einfluß haben. Das \RWbet{Gegentheil} der Weisheit, oder Meinungen, die für die Tugend und Glückseligkeit \RWbet{nachtheilig} sind, könnte man \RWbet{Thorheiten} nennen. Allein nun würde es noch an einem~\RWSeitenw{66}\ Worte fehlen, welches die \RWbet{Gattung} bezeichnete, von welcher Weisheit und Thorheit die beiden \RWbet{Arten} sind, \dh\ das den Inbegriff aller für die Tugend und Glückseligkeit \RWbet{nicht gleichgültiger} Meinungen, \RWbet{noch unentschieden, ob sie vortheilhaft oder nachtheilig sind}, umfaßte. Und dazu scheint nur das Wort \RWbet{Religion} zu taugen; besonders, wenn noch der beschränkende Umstand hinzugefügt wird, daß diese Meinungen zugleich sittlicher Art seyn sollen.
\item Will man die Wichtigkeit und die Nothwendigkeit eines eigenen Unterrichtes in der Religion für einen jeden Menschen recht einleuchtend darstellen; so ist kein Begriff dazu tauglicher, als unser obige. Denn nun liegt es schon in dem Begriffe derselben, daß sie nur lauter Lehren, die für uns wichtig sind, umfasse.
\item Wenn eigene \RWbet{Lehrer} mit der Verbindlichkeit angestellt werden, um in der \RWbet{Religion} Unterricht zu ertheilen, (wie dieses der Fall bei allen Bischöfen, Seelsorgern \usw\ ist); so wird es abermals sehr gut seyn, mit dem Worte Religion den von mir vorgeschlagenen weiteren Begriff zu verknüpfen. Denn dann werden solche Personen nicht mehr, wie es häufig geschah, vorgeben können, ihre Pflicht erfüllt zu haben, wenn sie nur jene wenigen Wahrheiten, die sich unmittelbar auf \RWbet{Gott} beziehen, vortragen, während das Volk noch einer Menge von Irrthümern anhängt, die den verderblichsten Einfluß auf seine Tugend und Glückseligkeit äußern.
\item Von jeher hat man die Lehre von Gottes \RWbet{Offenbarung}, (nämlich von ihrer Möglichkeit, Nützlichkeit, und von ihren Kennzeichen) zur \RWbet{Religionslehre} gezählt, und nirgend anders als in \RWbet{dieser} abgehandelt. Durch dieß Verfahren hat man stillschweigend vorausgesetzt, daß jede Offenbarung \RWbet{Religion} seyn müsse; oder mit anderen Worten, daß Alles, was Gott einem Menschen durch \RWbet{Offenbarung} bekannt macht, zu seiner \RWbet{Religion} gehöre. Nun ist es aber einleuchtend, daß Gott sich in einer Offenbarung noch über weit \RWbet{Mehres}, als was \RWbet{nach dem gewöhnlichen Sprachgebrauche}~\RWSeitenw{67}\ zur Religion gezählt wird, auslassen könne. Oder wie Manches wird nicht selbst in unsern h.~Schriften als eine gewissen einzelnen Menschen gewordene göttliche Offenbarung erzählt, was jenem Sprachgebrauche nach gar nicht zur Religion gerechnet werden dürfte? \zB\ die dem Abraham geschehene Verheißung; die Offenbarungen, welche der Nährvater unsers Herrn Jesu Christi im Traume erhielt; \udgl\
\end{aufzb}
In der Folge werden wir sehen, daß Alles, was uns Gott offenbaret, einen wohlthätigen Einfluß auf unsere Tugend oder Glückseligkeit haben müsse; woraus denn erhellet, daß nach dem hier vorgeschlagenen Begriffe allerdings jede Offenbarung zur \RWbet{Religion} gezählt werden könne. Und so wird also jene \RWbet{Inconsequenz} behoben.
\item Was mich zur Aufstellung dieses weitern Begriffes besonders ermuntert, ist die Bemerkung, daß sich die Lehre des \RWbet{Christenthums}, die doch gewiß nur \RWbet{Religionslehre} seyn will, nicht etwa erst in unseren Tagen, sondern gleich Anfangs über Alles ausbreitete, was zur Beförderung der Tugend und Glückseligkeit unter den Menschen dienen kann, und zugleich sittlicher Art ist. Er selbst, der göttliche Stifter des Christenthums, predigte, wie man hinlänglich weiß, beinahe überall nur \RWbet{Moral}. Ein Gleiches thaten auch seine ersten Schüler, die heiligen Apostel, wie wir in ihren noch übrigen Briefen und in der Apostelgeschichte an mehren Orten (\zB\ \RWbibel{Apg}{Apostelg.}{24}{25}) sehen. \RWbibel{Apg}{Apostelg.}{20}{20}\ lesen wir vollends Pauli ausdrückliche Erklärung: \erganf{\RWbet{Keine ersprießliche Lehre habe ich euch vorenthalten!}}; und \RWbibel{Apg}{Apostelg.}{13}{14}\ sprechen die Vorsteher der Synagoge von Antiochien den Apostel und seinen Begleiter selbst darum an, wenn er \RWbet{irgend eine Lehre des Trostes für sie wüßte}, sie ihnen mitzutheilen. Seinen Brief an Titus beginnt derselbe mit den Worten: \erganf{\RWbet{Paulus, der Diener Gottes und der Gesandte Jesu Christi, der beauftragt ist, den Glauben (die Religion) der von Gott Auserwählten, und die Erkenntniß aller Wahrheiten, welche zur Frömmigkeit führen, und die Hoffnung des ewigen Lebens gewähren, zu}~\RWSeitenw{68}\ \RWbet{fördern}}, \usw\ Gewiß hielt Paulus sich selbst nur für einen Lehrer der Religion; dennoch fühlte er sich, wie diese Stellen beweisen, berufen, jede ersprießliche Lehre, jede Wahrheit, die zur Frömmigkeit führet, und die Hoffnung des ewigen Lebens gewähret, \dh\ mit andern Worten, die unsere Tugend und Glückseligkeit befördern kann, seinen Schülern vorzutragen. Mußte er also wohl nicht glauben, daß alle solche Lehren in das Gebiet der Religion gehören?
\item Endlich kann ich in diesem Versuche, den Begriff der Religion zu erweitern, auch schon auf manchen sehr achtungswürdigen \RWbet{Vorgänger} hinweisen. Die Erklärung, welche Hr.~\RWbet{Jakob Frint} in seinem Lehrbuche\RWlit{}{Frint1} aufstellt, läuft mit der hier gegebenen ganz auf dasselbe hinaus. Dieser Gelehrte erkläret nämlich die Religion für eine Anleitung des Menschen zur Erreichung seiner Bestimmung; die Bestimmung des Menschen aber setzt er in Tugend und Glückseligkeit, und so ist also die Religion nach ihm eine Anleitung zur Tugend und Glückseligkeit; mithin gehören alle Lehren, die einen Einfluß auf Tugend oder Glückseligkeit haben, nach ihm zur Religion. Dieser Begriff ist, wie man sieht, im Grunde nur noch etwas weiter, als der oben vorgeschlagene; in sofern nämlich, als hier der Beysatz fehlt, daß diese Lehren \RWbet{sittlicher Art} seyn müssen, \dh\ daß eine eigene Versuchung obwalten müsse, sich ohne hinlängliche Berechtigung entweder für oder wider sie zu erklären. Einen solchen Beisatz habe ich für nöthig erachtet, weil es sonst eine Menge von Wahrheiten aus dem Gebiete der Mathematik, Physik, Oekonomie, Heilkunde und anderer Wissenschaften gäbe, die alle zur Religion gezählt werden müßten; indem sie alle, wenn auch nicht eben zur Beförderung unserer Tugend, doch zur Beförderung unserer Glückseligkeit beitragen, und zur Erreichung unserer Bestimmung nothwendig sind. Wahrheiten dieser Art wird gleichwohl Niemand der Religion beigezählt wissen wollen; es scheint also nöthig, daß zu dieser Erklärung noch ein beschränkender Beisatz hinzugefügt werde. Ob aber der, den ich hier angegeben habe, der richtige sey, das ist freilich noch ungewiß.~\RWSeitenw{69}
\end{aufza}
\begin{RWanm}
Zuweilen ergibt sich aus der bloßen Betrachtung des \RWbet{Ursprunges} und der \RWbet{Bestandtheile} eines Wortes (aus seiner \RWbet{Etymologie}) mancher nicht zu verachtende Aufschluß über die wahre Bedeutung desselben. Bei dem Worte \RWbet{Religion} gibt jedoch diese Ableitung keine besondere Belehrung, sondern sie selbst ist noch im Streite.\editorischeanmerkung{Hier war in einem von Bolzanos Manuskripten ursprünglich ein Hinweis auf die Ableitungen des Religionsbegriffs bei Cicero und Laktanz, der aber nicht mehr rekonstruiert werden kann. Vgl.\ GA 1,6/1, S.\,104, Anm.\,j.}
\end{RWanm}

\RWpar{22}{Verschiedene Arten der Religion}
Durch die so eben versuchte Rechtfertigung des aufgestellten Begriffes der Religion wurde zu gleicher Zeit auch seine \RWbet{Gegenständlichkeit} (Realität) erwiesen, \dh\ erwiesen, daß es dergleichen Gegenstände, wie wir sie uns in dem Begriffe der Religion vorstellen, in der That gebe. Wir müssen uns nun noch mit einigen der merkwürdigsten Arten derselben bekannter machen.\par
\begin{aufza}
\item Eine Religion oder ein Inbegriff sittlicher, auf Tugend und Glückseligkeit Bezug habender Meinungen, der in der That bei \RWbet{Einem} oder \RWbet{einigen lebenden} Menschen vereinigt anzutreffen ist, heißt eine \RWbet{lebende Religion}; auf eben die Art, wie man auch Sprachen, die wirklich von Menschen gesprochen werden, lebende Sprachen nennt.
\item Eine Religion oder ein Inbegriff sittlicher, auf Tugend und Glückseligkeit Bezug habender Meinungen, der \RWbet{ehemals} wohl bei Menschen angetroffen wurde, nun aber \RWbet{nicht mehr} herrschet, heißt eine \RWbet{todte} oder \RWbet{ausgestorbene Religion}; eben so, wie im ähnlichen Falle auch eine Sprache todt oder ausgestorben heißt.
\item Eine Religion oder ein Inbegriff sittlicher, auf Tugend und Glückseligkeit Bezug habender Meinungen, der \RWbet{niemals} wirklich bei irgend einem Menschen vereinigt anzutreffen war, sondern nur \RWbet{in der Einbildung} vorhanden ist, heißt eine \RWbet{bloß gedachte}, \RWbet{imaginirte} oder \RWbet{bloß mögliche Religion}.
\item Ein Inbegriff aller auf den Zweck der Tugend und Glückseligkeit sich beziehender Meinungen sittlicher Art, zu welchen sich \RWbet{alle} oder doch \RWbet{fast alle} Mitglieder einer Gesellschaft, für welche diese Meinungen verständlich und wichtig sind, auf eine gleichlautende Weise bekennen, heißt mir \RWbet{die Reli}\RWSeitenw{70}\RWbet{gion dieser Gesellschaft}. So heißt \zB\ der Inbegriff aller derjenigen religiösen Meinungen, die man bei \RWbet{allen} oder doch \RWbet{fast allen} Katholiken, für welche diese Meinungen verständlich und wichtig sind, antrifft, \RWbet{die Religion der Katholiken}, oder der \RWbet{Katholicismus}. So gehört \zB\ der Satz, \RWbet{daß Christus zwei Naturen in einer einzigen Person vereinige}, zu dieser Religion, weil alle gelehrten Katholiken, für die allein jener Satz verständlich und wichtig ist, hierüber einstimmig sind. Die hier gegebene Erklärung ist, wie mir däucht, dem herrschenden Sprachgebrauche vollkommen angemessen. Denn so oft wir den Namen einer ganzen \RWbet{Gesellschaft}, \RWbet{eines Volkes}, \RWbet{Geschlechtes}, \udgl\ irgend einem \RWbet{Gattungsbegriffe} als Beiwort beigesellen, um den Begriff einer besondern \RWbet{Art} daraus zu bilden; so bezeichnet der neue Begriff immer nur dasjenige, was allen oder doch beinahe allen Individuen jener Gesellschaft, jenes Volkes \usw\ gemeinschaftlich zukömmt. So nennen wir \zB\ \RWbet{französischen} Charakter einen Charakter, den entweder alle, oder doch fast alle \RWbet{Franzosen} haben; \RWbet{griechische} Gesichtsbildung eine Gesichtsbildung, die allen oder beinahe allen \RWbet{Griechen} eigen gewesen; \RWbet{deutsche} Sprache eine Sprache, die von allen oder fast allen \RWbet{Deutschen} gesprochen wird; \usw\
\item Besonders wichtig ist endlich die Eintheilung der Religionen in \RWbet{geoffenbarte}, die man auch \RWbet{positive} -- und \RWbet{nicht geoffenbarte}, die man auch \RWbet{natürliche} nennt. Um aber diese Eintheilung gehörig zu verstehen, müssen wir erst den \RWbet{Begriff des Offenbarens} selbst erklären.
\end{aufza}

\RWpar{23}{Erklärung des Begriffes Offenbaren in der weitesten Bedeutung}
Das Wort \RWbet{Offenbaren}, dessen Erklärung ich jetzt geben soll, wird in verschiedener Bedeutung genommen. Wir können, däucht mir, in Allem \RWbet{vier} Bedeutungen desselben unterscheiden. Drei \RWbet{weitere}, in denen es gleichgeltend ist mit den Worten \RWbet{Bekanntmachen}, \RWbet{Unterrichten}, \RWbet{Lehren} \udgl ; und eine \RWbet{engere}, in der es gleichgeltend mit~\RWSeitenw{71}\ den Worten \RWbet{Bezeugen}, oder \RWbet{Zeugenschaft ablegen}, gebraucht wird. Unter den \RWbet{drei weiteren} Bedeutungen des Wortes \RWbet{Offenbaren}, welche dasselbe mit den Worten \RWbet{Bekanntmachen}, \RWbet{Unterrichten} \udgl\ gemein hat, ist die \RWbet{erste} und \RWbet{weiteste} Bedeutung diejenige, in der es nichts Anderes anzeigt, als \RWbet{Ursache seyn von der Entstehung einer Meinung}; oder in welcher man sagt, \RWbet{daß ein Wesen $A$ einem andern $B$ eine gewisse Meinung $M$ geoffenbaret oder bekannt gemacht habe}, wenn $A$ die \RWbet{Ursache} (wenigstens eine \RWbet{Theilursache}) von der Entstehung der Meinung $M$ in $B$ war. -- Die Meinung $M$ selbst nennt man die \RWbet{geoffenbarte oder bekannt gemachte Meinung}. Die \RWbet{Thätigkeit}, durch welche das Wesen $A$ die Meinung $M$ in $B$ hervorbringt, nennet man die \RWbet{Offenbarung} oder \RWbet{Bekanntmachung} in der activen Bedeutung des Wortes. Oft aber pflegt man die geoffenbarte Meinung selbst eine \RWbet{Offenbarung} zu nennen; wo denn dieß Wort in einer passiven Bedeutung gebraucht wird.\par
Es versteht sich von selbst, daß das Wesen $B$, dem eine Meinung geoffenbaret werden soll, ein \RWbet{denkendes Wesen} seyn müsse, weil es sonst keiner Annahme von \RWbet{Meinungen} fähig wäre. Das Wesen $A$ aber, oder das \RWbet{Offenbarende} muß bei dieser Bedeutung des Wortes \RWbet{Offenbaren} nicht eben immer ein \RWbet{denkendes} Wesen, sondern es kann selbst ein \RWbet{lebloser} Gegenstand seyn. Die Meinung $M$, welche das Wesen $A$ in $B$ durch seine Wirksamkeit hervorbringt, kann \RWbet{wahr} oder \RWbet{falsch} seyn; die \RWbet{Art und Weise} endlich, wodurch es sie hervorbringt, kann seyn, welche sie immer will. So ist es \zB\ nicht einmal nöthig, daß das Wesen, dem etwas geoffenbaret wird, wisse, \RWbet{von wem} es ihm geoffenbaret werde; und eben so wenig braucht das Offenbarende zu wissen, \RWbet{daß} es, und \RWbet{was} es offenbaret, da es, wie gesagt, sogar ein \RWbet{lebloses} Wesen seyn könnte. In dieser \RWbet{weitesten} Bedeutung werden die Worte \RWbet{Offenbaren} oder \RWbet{Bekanntmachen} wirklich sehr oft genommen, wenn wir \zB\ sagen, ein Brief habe uns den~\RWSeitenw{72}\ Tod unsers Freundes \RWbet{geoffenbaret} oder \RWbet{bekannt gemacht}; oder ein \RWbet{Ungenannter} habe uns durch einen Brief Dieses oder Jenes \RWbet{geoffenbaret} oder \RWbet{bekannt gemacht}.

\RWpar{24}{Erklärung des Begriffes Offenbaren in der zweiten Bedeutung}
In einer \RWbet{zweiten} schon etwas \RWbet{engeren} Bedeutung werden die Worte \RWbet{Offenbaren}, \RWbet{Bekanntmachen}, \RWbet{Unterrichten} \usw\ genommen, wenn man darunter versteht, mit \RWbet{Wissen und Willen} (\dh\ mit \RWbet{Absicht}) \RWbet{Ursache seyn von der Entstehung einer Meinung}, oder \RWbet{wenn man sagt, ein Wesen $A$ habe einem andern $B$ eine gewisse Meinung $M$ geoffenbaret oder bekannt gemacht, sofern $A$ mit Wissen und Willen die Ursache} (wenigstens eine \RWbet{Theilursache}) ist \RWbet{von der Entstehung der Meinung $M$} in $B$. Auch jetzt wieder heißt die \RWbet{Wirksamkeit}, durch welche das Wesen $A$ diese Meinung in $B$ mit Wissen und Willen hervorbringt, eine \RWbet{Offenbarung} in der \RWbet{activen} Bedeutung; die Meinung selbst, die es geoffenbaret hat, eine \RWbet{Offenbarung} in der \RWbet{passiven} Bedeutung.\par
Hier muß nicht bloß das Wesen, dem etwas geoffenbaret wird, sondern auch das \RWbet{Offenbarende} selbst ein \RWbet{denkendes} Wesen seyn, weil es nach der Erklärung mit \RWbet{Wissen und Willen} (oder mit \RWbet{Absicht}) wirken soll. -- Die übrigen Stücke aber bleiben wie vorhin; daher die dortigen Erläuterungen auch hier wieder gelten. In dieser Bedeutung nimmt der Sprachgebrauch die Worte \RWbet{Offenbaren} und \RWbet{Bekanntmachen} in Sätzen, wie folgender: Mein Freund hat mir sein Innerstes \RWbet{geoffenbart}.

\RWpar{25}{In keiner dieser beiden Bedeutungen kann man die Religionen auf Erden in geoffenbarte und nicht geoffenbarte eintheilen}
Wenn man die Religionen auf Erden in \RWbet{geoffenbarte} und \RWbet{nicht geoffenbarte} eintheilen will; so meint~\RWSeitenw{73}\ man dieß vornehmlich nur in \RWbet{Beziehung auf Gott}, \dh\ man nennt diejenigen Religionen \RWbet{geoffenbaret}, oder zu mehrer Deutlichkeit auch \RWbet{göttlich geoffenbaret}, von denen man sich vorstellt, daß sie von \RWbet{Gott} geoffenbaret wären; \RWbet{nicht geoffenbaret} dagegen nennt man diejenigen, von denen man sagen will, daß sie von Gott \RWbet{nicht} geoffenbaret wären. Wollte man aber das Wort \RWbet{Offenbarung} in einer von den zwei jetzt erklärten Bedeutungen nehmen; so würde sich zeigen, daß eine \RWbet{jede Religion} den Namen einer \RWbet{göttlich geoffenbarten} verdiene; woraus denn erhellt, daß diese zwei Bedeutungen \RWbet{zu weit} sind, um eine \RWbet{Eintheilung} der Religionen auf Erden in \RWbet{geoffenbarte} und \RWbet{nicht geoffenbarte} zu begründen.\par
Ich werde dieses nur von der \RWbet{zweiten} Bedeutung zeigen; denn wenn selbst \RWbet{diese} zu weit ist, so versteht es sich von der ersteren, die noch weiter ist, von selbst. Ich behaupte also, wenn unter einer \RWbet{Offenbarung} nichts Anderes verstanden werden soll, als eine mit Wissen und Willen begleitete Thätigkeit eines Wesens, wodurch es Ursache von der Entstehung einer Meinung in einem andern Wesen wird: so könne man eine \RWbet{jede Religion} auf Erden, ja überhaupt eine \RWbet{jede Meinung}, auf welche irgend ein \RWbet{endliches} Wesen auf was immer für einem Wege geräth, \RWbet{göttlich geoffenbaret} nennen. Alle Meinungen nämlich, auf die ein geschaffenes Wesen auf was immer für einem Wege geräth, entstehen in demselben durch \RWbet{Gottes Wirksamkeit}, und zwar mit \RWbet{Wissen} und \RWbet{Willen} Gottes. Denn
\begin{aufza}[a)]
\item Alles, was immer geschieht, also auch \RWbet{jede Meinung}, auf die ein geschaffenes Wesen auf was immer für einem Wege kommt, hat ihren letzten Grund in Gott; indem, wenn Gott nicht diese Welt und dieses denkende Wesen in ihr geschaffen, nicht diese und jene besonderen Umstände herbeigeführt oder zugelassen hätte, auch jene Meinung in dem Wesen nicht hätte entstehen können.~\RWSeitenw{74}
\item Aus Gottes \RWbet{Allwissenheit} folgt aber auch, daß er von Allem, was geschieht, \RWbet{weiß}; also geschieht es nicht nur durch seine \RWbet{Wirksamkeit}, sondern auch mit seinem \RWbet{Wissen}, daß jenes Wesen die erwähnte Meinung annimmt.
\item Aus Gottes \RWbet{Allmacht} folgt endlich, daß er auch \RWbet{Alles}, was er \RWbet{will, vermöge}, und mithin auch zur \RWbet{Wirklichkeit} bringe. Daher kann nichts von Allem, was immer geschieht, dem Willen Gottes \RWbet{zuwider} geschehen. Alles geschieht also seinem Willen \RWbet{gemäß}, und folglich entsteht auch jene Meinung nicht nur durch seine \RWbet{Wirksamkeit}, und mit seinem Wissen, sondern auch mit seinem \RWbet{Willen}.
\end{aufza}

\RWpar{26}{Auflösung einiger Einwürfe}
In dem so eben geführten Beweise kommen zwei Behauptungen vor, die einigen Anstoß verursachen könnten, nämlich, daß Alles, was immer geschieht, 1.~durch \RWbet{Gottes Wirksamkeit}, und 2.~sogar \RWbet{nach seinem Willen} erfolge.
\begin{aufza}
\item In Rücksicht der \RWbet{ersten} Behauptung könnte man einwerfen, daß es
\begin{aufzb}
\item der \RWbet{Heiligkeit Gottes} zu widersprechen scheine, wenn auch die \RWbet{Uebel in der Welt} als eine Wirkung seiner Thätigkeit dargestellt werden; und daß es
\item mit der \RWbet{Freiheit der geschaffenen Wesen streite}, Gott als die Ursache auch jener \RWbet{guten oder bösen Handlungen}, die diese verrichten, anzusehen. -- \par
Hierauf erwidere ich aber:
\begin{aufzc}
\item Die Behauptung, daß auch die \RWbet{Uebel} dieser Welt durch Gottes Wirksamkeit erfolgen, würde nur dann der \RWbet{Heiligkeit Gottes} widersprechen, wenn sich erweisen ließe, daß irgend eines dieser Uebel ohne einen noch größern Nachtheil für das Ganze hätte wegbleiben können. Dieß läßt sich aber schlechterdings~\RWSeitenw{75}\ nicht erweisen. Sind nun die Uebel in der Welt nothwendig zu ihrer größeren Vollkommenheit; so ist es auch gar nicht gegen die Heiligkeit Gottes, zu denken, daß er sie durch seine Einwirkung herbeigeführt habe.
\item Die Behauptung, daß Gott die Ursache auch jener \RWbet{guten oder bösen Handlungen} sey, welche die freien Naturwesen verrichten, würde der \RWbet{Freiheit} der letztern höchstens dann widersprechen, wenn man sie so verstände, daß Gott der \RWbet{vollständige} und \RWbet{nöthigende} Grund zu diesen Handlungen sey. Aber man meint nur, daß Gott eine \RWbet{Theilursache} von ihnen, eine \RWbet{Bedingung} (\RWlat{conditio sine qua non}) zu denselben sey; in sofern nämlich, als diese Handlungen nicht hätten erfolgen können, wenn er nicht jene Wesen geschaffen, ihnen nicht jene Kräfte gegeben hätte, \usw\
\end{aufzc}
\end{aufzb}
\item In Rücksicht der \RWbet{zweiten} Behauptung könnte man einwerfen,
\begin{aufzb} 
\item sie \RWbet{widerspreche} für's \RWbet{Erste} einer Wahrheit, die uns nicht nur der gemeine Menschenverstand lehrt, sondern die auch die \RWbet{heilige Schrift} und die \RWbet{katholische Kirche} ausdrücklich aufstellen, der Wahrheit nämlich, \RWbet{daß alles Böse in der Welt, besonders das sittlich Böse dem Willen Gottes entgegen sey}.
\item Was kann auch in der That \RWbet{anstößiger} seyn, als zu sagen, \RWbet{daß das sittlich Böse in der Welt} (die Sünde) \RWbet{nach Gottes Willen erfolge}? -- \par
Hierauf erwidere ich:
\begin{aufzc} 
\item Die Behauptung, daß auch das Böse in der Welt nach Gottes Willen geschehe, würde der Wahrheit, daß alles Böse dem Willen Gottes \RWbet{entgegen} sey, nur dann widersprechen, wenn man den Ausdruck: \RWbet{Wille Gottes} -- in beiden Behauptungen in derselben Bedeutung nähme. Aber so ist es nicht; denn dieser Ausdruck ist in zwei verschiedenen Bedeutungen gebräuchlich; 
a) in einer \RWbet{eigentlichen}, in der man~\RWSeitenw{76}\ unter dem \RWbet{Willen Gottes} einen eigentlichen \RWbet{Actus der göttlichen Wollkraft}, einen göttlichen \RWbet{Rathschluß} versteht; in welchem Sinne des Wortes Alles, was immer Gott \RWbet{will}, wirklich zu Stande kommt; dann aber auch b) in einer \RWbet{uneigentlichen}, in der man unter dem \RWbet{Willen Gottes} jeden Erfolg verstehet, den Gott so sehr, als es von seiner Seite nur immer geschehen kann, befördert. -- In dieser uneigentlichen Bedeutung nimmt man den Ausdruck: \RWbet{Wille Gottes}, wenn man \zB\ sagt: \RWbet{Gott will, daß alle Menschen tugendhaft und glückselig werden.} Denn hiedurch will man nichts Anderes sagen, als daß Gott Alles thue, was sich nur immer von seiner Seite thun läßt, um alle Menschen zur Tugend und Glückseligkeit zu bringen. In dieser Bedeutung kann man nicht sagen, daß \RWbet{jeder} Wille Gottes zur Wirklichkeit gelange; indem nicht Alles, was Gott so sehr befördert, als es nur immer von seiner Seite geschehen kann, wirklich erfolgt, weil es zum Theil auch von der Freiheit der geschaffenen Wesen abhängt. Insonderheit sind alle \RWbet{bösen} Handlungen dem Willen Gottes in dieser zweiten Bedeutung des Wortes zuwider. Und so erhellet, wie jene beiden Behauptungen recht wohl neben einander bestehen können, indem man den Ausdruck: \RWbet{Wille Gottes} -- in der \RWbet{ersten} in seiner \RWbet{eigentlichen}, in der \RWbet{zweiten} aber in seiner \RWbet{uneigentlichen} Bedeutung nimmt.
\begin{RWanm}
Fragt man, woher der Gebrauch des Ausdruckes: \RWbet{Wille Gottes} -- in der \RWbet{uneigentlichen} Bedeutung rühre; so glaube ich erwidern zu dürfen, er habe seinen Ursprung einer Art von \RWbet{Anthropomorphismus} (\dh\ einer nicht ganz richtigen Uebertragung \RWbet{menschlicher} Eigenschaften auf das Wesen \RWbet{Gottes}) zu danken. Wenn nämlich wir \RWbet{Menschen} etwas so sehr befördern, als es von unserer Seite nur immer geschehen kann, so ist es meistens der Fall, daß wir auch \RWbet{wünschen}, es möchte erfolgen. So wollte man denn auch etwas Aehnliches von Gott annehmen. Weil man jedoch fühlte, es wäre gar zu \RWbet{menschlich} und wirklich \RWbet{unschicklich} von Gott~\RWSeitenw{77}\ gesprochen, wenn man ihm \RWbet{Wünsche} beilegte: so wählte man lieber die Redensart, daß er das \RWbet{wolle}, was er so sehr, als es von seiner Seite nur immer geschehen kann, befördert. Dieß \RWbet{Wollen} findet aber nur bei einem Theile desjenigen, was er auf eine solche Art befördert, Statt; nämlich nur bei demjenigen, was auch wirklich zu Stande kommt. Denn Gott will jeden Erfolg, auch selbst den besten, \RWbet{nur in so weit}, als er ihn auch zu Stande zu bringen \RWbet{vermag}, und wirklich \RWbet{bringt}. Was nicht zu Stande kommt, (weil es entweder an sich selbst unmöglich ist, oder nur durch Verhinderung eines noch größeren Gutes bewirkt werden könnte), das \RWbet{will} Gott auch weder, noch \RWbet{wünscht} er, es bewirken zu \RWbet{können.}
\end{RWanm}
\item Die Behauptung, daß auch das \RWbet{Böse in der Welt nach Gottes Willen} erfolge, wäre nur dann anstößig, wenn man den Ausdruck: \RWbet{Wille Gottes} -- in seiner \RWbet{uneigentlichen} Bedeutung nähme, und also behauptete, daß das Böse von Gott \RWbet{so sehr befördert werde}, als es nur immer von seiner Seite geschehen kann. Allein das sagt man nicht; sondern man redet hier von Gottes \RWbet{eigentlichem} Willen; und damit es um so weniger Jemand anstößig finde, daß auch das Böse in der Welt dem \RWbet{eigentlichen} Willen Gottes gemäß sey, so muß man erwägen, daß dieser Wille abermals doppelt, ein \RWbet{unbedingter} (oder absoluter) nämlich, und ein \RWbet{bedingter} (oder hypothetischer) sey. Der \RWbet{unbedingte} ist derjenige, vermöge dessen Gott eine Sache um \RWbet{ihrer selbst} wegen will; der \RWbet{bedingte} derjenige, vermöge dessen er die Sache nur als \RWbet{Bedingung} (weil sie ein Mittel zu etwas Gutem ist) will. Was sollte es nun Anstößiges haben zu sagen, daß Gott das Böse in der Welt \RWbet{bedingnißweise} wolle, nämlich in wiefern es als Bedingung zu einem noch größern Guten nothwendig ist, oder nicht ohne ein noch größeres Übel wegbleiben könnte?
\begin{RWanm}
Obgleich sich aber die beiden Redensarten, \RWbet{daß auch das Böse in der Welt durch Gottes Wirksamkeit erfolge, und dem Willen Gottes gemäß sey}, nach den gegebenen Erklärungen vollkommen rechtfertigen lassen: so haben sie doch ihr Unbequemes, und die katholische Kirche thut~\RWSeitenw{78}\ wohl, wenn sie statt dessen sagt, das Böse in der Welt erfolge nur durch Gottes \RWbet{Zulassung}, und sey seinem \RWbet{Willen}, nämlich dem \RWbet{uneigentlichen} oder seinen \RWbet{Geboten} oder \RWbet{Absichten} \RWbet{zuwider}.
\end{RWanm}
\end{aufzc}
\end{aufzb}
\end{aufza}

\RWpar{27}{Erklärung des Begriffes einer göttlichen Offenbarung in der dritten Bedeutung}
Da wir so eben zweierlei Arten des \RWbet{göttlichen Willens} kennen gelernt haben, einen \RWbet{eigentlichen} nämlich, (der wieder ein \RWbet{unbedingter} oder \RWbet{bedingter} seyn kann) und einen \RWbet{uneigentlichen}; so gibt dieß Gelegenheit zur Bildung einer \RWbet{dritten}, noch etwas engeren Bedeutung, in welcher der Ausdruck: \RWbet{göttliche Offenbarung}, genommen werden kann und wird. Man versteht nämlich unter \RWbet{einer göttlichen Offenbarung} in der activen Bedeutung \RWbet{eine solche Wirksamkeit Gottes, durch die er Meinungen in uns erzeugt, welche sein unbedingter Wille billiget}, \dh\ Meinungen, die \RWbet{an sich selbst gut und ersprießlich} sind. Sie, diese Meinungen selbst nennt man dann \RWbet{göttliche Offenbarungen} in der \RWbet{passiven} Bedeutung. In diesem Sinne können nur \RWbet{alle solche Meinungen, die an sich gut und zuträglich sind, göttlich geoffenbaret}, oder \RWbet{göttliche Offenbarungen} heißen; diese aber auch ohne Unterschied, wir mögen übrigens zu ihrer Annahme auf was immer für eine Weise gelangt seyn; und nur \RWbet{verderbliche Irrthümer} werden in diesem Sinne des Wortes den \RWbet{göttlichen Offenbarungen} entgegengesetzt werden dürfen.\par
Hieraus ist zu ersehen, daß man die mancherlei Religionen auf Erden in dieser Bedeutung allerdings schon in \RWbet{geoffenbarte} und \RWbet{nicht geoffenbarte} eintheilen könnte. \RWbet{Geoffenbarte} nämlich wären nun solche, die vernünftig und zuträglich sind, \zB\ die \RWbet{christliche; nicht geoffenbaret} dagegen jene, die unvernünftig und nachtheilig sind, \zB\ die \RWbet{heidnischen}.\par
Gleichwohl ist es gewiß, daß man nicht \RWbet{diese}, sondern noch irgend eine \RWbet{engere} Bedeutung des Wortes~\RWSeitenw{79}\ \RWbet{Offenbaren} im Sinne haben müsse, wenn man eine \RWbet{geoffenbarte} Religion der \RWbet{natürlichen} entgegensetzt; denn dasjenige, was man nach allgemeiner Uebereinstimmung \RWbet{natürliche Religion} nennt, ist doch wohl eine \RWbet{vernünftige} und den Menschen \RWbet{zuträgliche} Religion, die folglich, wenn man das Wort \RWbet{Offenbaren} in der jetzt eben erklärten Bedeutung nehmen würde, gleichfalls \RWbet{geoffenbart} heißen müßte. Dennoch hat man in unserer neuesten Zeit häufig behauptet, daß es nur diese Bedeutung allein, sonst aber keine noch engere gebe, in der das Christenthum, oder sonst eine andere Religion, auf den Namen einer göttlichen Offenbarung Anspruch zu machen hätte, weil jeder andere noch engere Begriff dieses Wortes, namentlich jeder solche, dabei die geoffenbarte Religion der natürlichen entgegengesetzt werden könnte, bei einer nähern Prüfung als unhaltbar verschwinde. Um desto nothwendiger ist es, daß wir jetzt diese engere Bedeutung recht deutlich darstellen.

\RWpar{28}{Erklärung des Begriffes Offenbaren in seiner vierten und engsten Bedeutung}
\begin{aufza} 
\item In seiner \RWbet{engsten} Bedeutung wird das Wort Offenbaren, wie ich bereits erwähnte, gleichbedeutend mit dem Worte \RWbet{bezeugen} genommen. Jemand \RWbet{etwas offenbaren} heißt nämlich, wie ich behaupte, oft auch so viel, als, ihm \RWbet{etwas bezeugen}, oder \RWbet{ihn zur Annahme einer gewissen Meinung durch die Dazwischenkunft eines Zeugnisses bestimmen}; wobei zu bemerken ist, daß man in Hinsicht auf \RWbet{Menschen} gewöhnlich das Wort \RWbet{Zeugniß}, in Hinsicht auf \RWbet{Gott} aber wenigstens heut zu Tage fast insgemein nur das Wort \RWbet{Offenbarung} gebraucht. Daß nun dieses Wort \RWbet{wirklich} und zwar gerade \RWbet{dann} in der Bedeutung eines \RWbet{Zeugnisses} gebraucht werde, wenn man die Religionen auf Erden in \RWbet{natürliche} und \RWbet{geoffenbarte} eintheilen will, und daß es nach dem herrschenden Sprachgebrauche auch schon keine \RWbet{engere} Bedeutung für dieses Wort gebe; dieß Alles glaube ich so zu erweisen:~\RWSeitenw{80}
\begin{aufzb} 
\item Bei dieser Bedeutung des Wortes \RWbet{Offenbarung} läßt sich die Eintheilung in \RWbet{natürliche} und \RWbet{geoffenbarte} Religionen sehr füglich anbringen; indem ja doch gewiß ist, daß die \RWbet{natürliche Religion} nicht eine solche sey, deren Wahrheiten wir auf Gottes Zeugniß hin annehmen.
\item Wenn wir die \RWbet{christliche} oder \RWbet{israelitische} Religion göttliche Offenbarungen nennen, so wollen wir im Grunde nichts Anderes anzeigen, als dieses wären Religionen, deren Wahrheiten \RWbet{Gott selbst bezeuget hat.}
\item In der \RWbet{heiligen Schrift} selbst wird die göttliche Offenbarung an unzähligen Stellen ausdrücklich nur das \RWbet{Zeugniß Gottes} (\RWhebr{`edAh}, \RWgriech{mart'urion}, \RWlat{testamentum}) genannt.
\end{aufzb}
\item Vorausgesetzt also, es habe seine Richtigkeit, daß das Wort \RWbet{Offenbarung}, wenn es in seiner engern Bedeutung genommen werden soll, nichts Anderes als ein göttliches Zeugniß bedeute: so wird es nöthig seyn, den Begriff eines \RWbet{Zeugnisses} in seine einzelnen Bestandtheile aufzulösen. Die genaue Erklärung dieses Begriffes ist nun nach meiner Meinung diese: Ein \RWbet{Zeugniß}, (in der \RWbet{activen} Bedeutung) heißt \RWbet{jede Handlung oder Thätigkeit, zu der sich Jemand in der bestimmten Absicht entschließt, damit ein Anderer, wenn er nach seiner besten Einsicht vorgeht, aus der Bemerkung jener Thätigkeit schließe, es sey der Wille des Ersteren, daß er eine gewisse Meinung annehme, weil Jener selbst sie für wahr hält.} Die Meinung, um die es sich hier handelt, heißt die \RWbet{bezeugte Meinung}, oder das \RWbet{Zeugniß} in der \RWbet{passiven} Bedeutung. -- Ich sage also, daß $A$ dem $B$ eine gewisse Meinung $M$ bezeuge, wenn $A$ irgend eine Handlung in der bestimmten Absicht vornimmt, damit $B$, wenn er nach seiner besten Einsicht vorgeht, aus der Bemerkung derselben schließe, es sey der Wille des $A$, daß $B$ die Meinung $M$ annehme, weil $A$ selbst sie für wahr hält. Daß diese Erklärung mit dem bisherigen Sprachgebrauche des Wortes \RWbet{Zeugniß} ganz übereinstimme, glaube ich aus folgender~\RWSeitenw{81}\ genaueren \RWbet{Betrachtung ihrer Bestandtheile}, die zugleich eine \RWbet{Erläuterung} derselben seyn wird, erweisen zu können.
\begin{aufzb} 
\item Daß \RWbet{erstlich} keiner der angegebenen Bestandtheile \RWbet{wegbleiben} dürfe, wo man dem Sprachgebrauche nach behaupten soll, daß ein Zeugniß Statt gefunden habe; erweise ich so:
\begin{aufzc}
\item Soll man mit Recht behaupten, daß $A$ dem $B$ etwas bezeugt habe, so wird \RWbet{erstlich} erfordert, daß $A$ \RWbet{eine gewisse Handlung verrichtet}, irgend eine Wirksamkeit geäußert habe. Denn wenn $A$ \RWbet{gar nichts gethan}, gar keine \RWbet{Veränderung in der Sinnenwelt hervorgebracht}, auch keine, die von selbst erfolgt wäre, \RWbet{verhindert hat}, also \zB\ weder irgend etwas \RWbet{gesprochen}, noch \RWbet{geschrieben}, noch irgend ein sonstiges \RWbet{Zeichen} von sich gegeben, nicht einmal \RWbet{geschwiegen} hat, wo man sonst zu reden pflegt, und so durch dieses Stillschweigen selbst etwas zu erkennen gegeben hat: so kann man gewiß nicht sagen, $A$ habe etwas \RWbet{bezeugt}.
\item Eben so nothwendig aber gehört zum Daseyn eines Zeugnisses auch \RWbet{zweitens}, daß $A$ bei der Handlung, die er verrichtet, die \RWbet{bestimmte Absicht habe} zu bewirken, daß $B$, wenn er nach seiner besten Einsicht verfährt, aus der Wahrnehmung dieser Handlung schließe, es sey der Wille des $A$, daß $B$ die Meinung $M$ annehme, weil $A$ selbst sie für wahr hält. Denn wer gewisse \RWbet{Veränderungen} entweder ganz \RWbet{unwissentlich}, oder doch nicht in \RWbet{dieser}, sondern in einer ganz \RWbet{andern} Absicht hervorgebracht hätte, von dem kann man nicht sagen, daß er ein \RWbet{Zeugniß} abgelegt habe. Z.\,B.\ wenn Jemand in einer mit Convulsionen begleiteten Ohnmacht, wo er nichts von sich selbst weiß, eine \RWbet{Zuckung} macht, die ein Anderer für die bejahende Antwort einer so eben an ihn gestellten Frage ansieht, \zB\ ob er im Schooße der Kirche zu sterben gedenke; -- eben so, wenn Jemand etwas in einer dem Andern unbekannten~\RWSeitenw{82}\ Sprache redet, was dieser, weil es zufällig auch in seiner Sprache eine, obwohl ganz andere Bedeutung hat, falsch auslegt \udgl : -- so kann man in keinem dieser Fälle sagen, daß der Erste das wirklich bezeugt habe, was nun der Andere glaubt.
\item Der Zeuge $A$ muß ferner die bestimmte Absicht haben, daß $B$ aus der Betrachtung seiner Handlung den Schluß ableite, es wäre sein Wille, daß $B$ die Meinung $M$ annehme, \RWbet{weil $A$ selbst sie für wahr hält}. Denn wenn uns Jemand nur zu erkennen gibt, \RWbet{er wolle, daß wir eine gewisse Meinung annehmen}, aber nicht darum, weil er selbst sie für wahr hält, sondern nur darum, weil ihre Annahme vielleicht unserer Tugend und Glückseligkeit zuträglich wäre, oder aus sonst einem anderen Grunde: so kann man wohl sagen, er habe uns die Annahme der Meinung $M$ \RWbet{gerathen}, oder \RWbet{befohlen}, oder uns darum \RWbet{gebeten}; keineswegs aber, er habe sie uns \RWbet{bezeugt}.
\item Der Zeuge muß endlich noch \RWbet{beabsichtigen}, daß die Annahme der Meinung $M$ bei $B$ \RWbet{bloß in dem Falle erfolge}, wenn dieser bei Beurtheilung des Zweckes der von $A$ geäußerten Thätigkeit \RWbet{nach seiner besten Einsicht vorgeht}, \dh\ jeden vermeidlichen Irrthum bei dieser Auslegung vermeidet. Denn wenn die Worte oder Handlungen des $A$ von einer solchen Beschaffenheit sind, daß $B$ nur durch eine willkührliche Selbsttäuschung aus ihnen den Schluß ableiten kann, es sey der Wille des $A$, daß er die Meinung $M$ annehme, weil $A$ selbst sie für wahr hält: so kann man nicht sagen, daß $A$ ihm diese Meinung \RWbet{bezeuget} habe, gesetzt auch, daß er diese Selbsttäuschung des $B$ \RWbet{erwartet} und wohl gar \RWbet{gewünscht} hätte. Wenn uns \zB\ ein vornehmer Mann die Frage vorlegt, ob wir ein gewisses, von ihm verfertigtes Gedicht nicht schön finden, und wir erwidern ein kaltes: \RWbet{O ja!}, so darf er sich nicht rühmen, daß wir die Schönheit seines Gedichtes ihm \RWbet{bezeuget} hätten;~\RWSeitenw{83}\ gesetzt auch, daß wir vorausgesehen, er werde sich unsere Antwort auf das Günstigste auslegen. Und selbst, wenn wir dieß wünschen, verdienen wir wohl vielleicht den Vorwurf einer Zweideutigkeit, nicht aber den einer Lüge, wenn anders der vornehme Mann so viel Vernunft hat, daß er, wofern er sich nicht hätte täuschen wollen, recht wohl begriffen haben würde, daß unsere Antwort aus bloßer Höflichkeit so gelautet habe.
\end{aufzc}
\item Alle in der gegebenen Erklärung vorkommenden Bestandtheile sind also zum Daseyn eines Zeugnisses nöthig; daß aber auch \RWbet{nur sie}, und sonst \RWbet{kein anderer} mehr nothwendig sey, erhellet daraus, weil auf \RWbet{alles Uebrige, was hier noch unbestimmt gelassen ist}, nichts ankommt, so daß, wo nur diese vier Stücke vorhanden sind, ein Zeugniß Statt findet, die übrigen Umstände mögen beschaffen seyn, wie sie wollen. So kommt es
\begin{aufzc}
\item nicht darauf an, ob die Meinung $M$, deren Entstehung der Zeuge veranlaßte, \RWbet{wahr} oder \RWbet{falsch} sey, oder auch nur von ihm selbst für \RWbet{wahr} oder \RWbet{falsch} gehalten werden. Von diesem Umstande hängt es wohl ab, ob das Zeugniß ein \RWbet{richtiges} oder \RWbet{unrichtiges}, ein \RWbet{wahrhaftes} oder \RWbet{lügenhaftes} sey; ein Zeugniß aber bleibt es in jedem Falle, wenn nur die übrigen Umstände da sind.
\item Eben so wenig kommt \RWbet{zweitens} darauf an, \RWbet{worin eigentlich die Handlung des Zeugen bestanden} habe; ob er \zB\ \RWbet{gesprochen}, oder \RWbet{geschrieben}, oder gewisse andere \RWbet{Zeichen} von sich gegeben habe. Von diesen Umständen hängt ab, ob man das Zeugniß ein \RWbet{mündliches} oder \RWbet{schriftliches} \udgl\ nenne: aber ein Zeugniß bleibt es in jedem Falle; selbst dann, wenn der Zeuge eigentlich gar \RWbet{keine sichtbare Veränderung hervorbrachte}, sondern vielmehr nur das \RWbet{Entstehen einer}, die erfolgen \RWbet{sollte}, verhinderte; wenn er \zB\ durch ein bloßes \RWbet{Stillschweigen} seinen Willen, daß wir bei einer gewissen Meinung bleiben sollen, zu erkennen gab. Diesen besondern Fall nennt man ein \RWbet{stillschweigendes Zeugniß}.~\RWSeitenw{84}\
\item Auch darauf kommt es nicht an, ob der Zeuge die Absicht, die er bei seiner Handlung hatte, wirklich erreichte oder nicht, \dh\ ob der Andere diese Handlung bemerkte oder nicht bemerkte, ob er den Zweck derselben errieth oder nicht errieth, ob er endlich die Meinung $M$ in der That annahm oder nicht annahm. Von diesen Umständen hängt ab, ob man das Zeugniß ein \RWbet{bemerktes} oder \RWbet{unbemerktes, verstandenes} oder \RWbet{mißverstandenes, geglaubtes} oder \RWbet{nicht geglaubtes} nenne; \RWbet{gegeben} wurde es aber in jedem dieser Fälle.
\item Auch darauf kommt es endlich nicht an, ob die \RWbet{Schlüsse}, durch welche der Andere aus der von dem Zeugen verrichteten Handlung auf dessen Willen schließt, daß er die Meinung $M$ annehmen solle, an sich selbst \RWbet{richtige} oder \RWbet{unrichtige Schlüsse} sind, \RWbet{wenn sie nur nach seiner besten Einsicht geschehen}. Denn wenn auch $B$ \RWbet{unrichtig} schließt, $A$ aber hat es vorausgesehen, daß $B$ unrichtig \RWbet{schließen werde}, und nach \RWbet{seiner} besten Einsicht nicht anders \RWbet{könne}, und eben, um ihn zu diesen Fehlschlüssen zu verleiten, nahm er jene Handlungen vor: so kann man mit allem Rechte sagen, er habe sein \RWbet{Zeugniß} für die Meinung $M$ bei $B$ abgelegt. Z.\,B.\ Wenn Jemand eine an ihn gestellte Frage \RWbet{zweideutig} beantwortet, in der Erwartung, daß der unverständige Frager sich nur an die eine Auslegung derselben halten werde und \RWbet{könne}.
\end{aufzc}
\end{aufzb}
\end{aufza}

\RWpar{29}{Erklärung der Begriffe eines bloß vermeintlichen und eines bloß vorgeblichen oder angenommenen Zeugnisses}
\begin{aufza}
\item Wenn der besondere Fall eintritt, daß Jemand eine Meinung $M$ annimmt, weil er aus einer gewissen Erscheinung nach seiner besten Einsicht schließt, daß ein Zweiter die Meinung $M$ von ihm geglaubt wissen wolle, weil er sie selbst für wahr hält, \RWbet{ohne daß dieser Andere das wirklich will}: so nennt man dieses ein \RWbet{vermeintliches}~\RWSeitenw{85}\ \RWbet{Zeugniß}. Ein solcher Fall tritt \zB\ ein, wenn jener Andere die Veränderung in der Sinnenwelt gar nicht hervorgebracht, oder doch nicht zu diesem Zwecke hervorgebracht hat; \zB\ wenn dem $A$ ein Brief an $B$ unterschoben wird, $B$ diesen Brief für ächt hält, und daher nun eine gewisse Nachricht in der Meinung, daß sie von $A$ ihm mitgetheilt sey, zu glauben anfängt.
\item Wenn der besondere Fall eintritt, daß Jemand eine Meinung $M$ annimmt, weil er mit \RWbet{absichtlicher Selbsttäuschung sich überredet}, es folge aus einer gewissen Erscheinung, daß ein Zweiter die Meinung $M$ von ihm geglaubt wissen wolle, weil er sie selbst für wahr hält, so nennt man dieß ein \RWbet{angenommenes} oder \RWbet{vorgebliches} Zeugniß. Z.\,B.\ Wenn $B$ aus bloßer Eitelkeit sich überredet, daß die zweideutige Antwort, welche ihm $A$ auf die Frage, wie seine Gedichte ihm gefallen, ertheilt hatte, beifällig auszulegen sey.
\item Im Gegensatze von solchen bloß \RWbet{vermeintlichen} oder bloß \RWbet{angenommenen} Zeugnissen pflegt man das Zeugniß des \RWparnr{28}\ der mehren Deutlichkeit wegen auch ein \RWbet{wirkliches Zeugniß} zu nennen.
\item Man muß die jetzt erklärten, bloß \RWbet{vermeintlichen} oder \RWbet{angenommenen} Zeugnisse nicht mit den \RWbet{falschen} verwechseln. Ein \RWbet{falsches} Zeugniß ist ein \RWbet{wirkliches} Zeugniß; das bloß \RWbet{vermeintliche} aber, wie auch das \RWbet{angenommene} sind beide gar \RWbet{kein eigentliches Zeugniß}.
\end{aufza}

\RWpar{30}{Erklärung der Begriffe eines menschlichen, höheren und göttlichen Zeugnisses, oder einer göttlichen Offenbarung}
\begin{aufza}
\item Je nachdem Derjenige, auf dessen Zeugniß Jemand eine Meinung annimmt, bald nur ein bloßer \RWbet{Mensch}, bald irgend ein \RWbet{höheres Wesen}, \zB\ ein Engel, bald auch \RWbet{Gott selbst} ist, nennet man das Zeugniß bald ein bloß \RWbet{menschliches}, bald ein \RWbet{höheres}, bald auch ein \RWbet{göttliches}.~\RWSeitenw{86}\ Nur bei den \RWbet{höheren} Zeugnissen, \zB\ bei den \RWbet{göttlichen}, ist es gewöhnlich, das Wort \RWbet{Offenbarung} zu brauchen. Und so ist also eine \RWbet{göttliche Offenbarung} in dieses Wortes \RWbet{engster und activer} Bedeutung, jede Veränderung in der Sinnenwelt, welche Gott in der Absicht hervorgebracht hat, damit ein geschaffenes Wesen, wenn es nach seiner besten Einsicht vorgeht, daraus entnehmen möge, es sey der Wille Gottes, daß es eine gewisse Meinung annehme, weil sie Gott selbst für wahr erkennet. Die so bezeugte Meinung heißt eine \RWbet{göttliche Offenbarung} in der \RWbet{passiven} Bedeutung des Wortes.
\item Da schon erinnert wurde, daß man den Ausdruck: \RWbet{Wille Gottes}, bald in eigentlicher, bald in uneigentlicher Bedeutung nehme; so wird zur vollständigen Bestimmung des Begriffes einer göttlichen Offenbarung gehören, daß wir auch noch erörtern, in welcher Bedeutung dieser doppelsinnige Ausdruck in der so eben gegebenen Erklärung vorkomme. Derselbe erscheint hier aber zweymal. Zuerst gleich, wenn wir sagen, daß eine göttliche Offenbarung eine Veränderung sey, die Gott in der bestimmten \RWbet{Absicht} oder (was eben so viel ist) mit dem bestimmten \RWbet{Willen} hervorbringt, daß \usw. Das zweite Mal, wenn wir dann weiter sagen, daß wir aus der Bemerkung dieser Veränderung den Schluß ziehen sollen, es sey der \RWbet{Wille Gottes}, daß \usw. Das erste Mal ist die Rede von einem Willen Gottes, der außerhalb unserer Vorstellung \RWbet{vorhanden seyn} muß, wenn eine wahre göttliche Offenbarung vorhanden seyn soll; das zweite Mal dagegen ist die Rede von einem Willen Gottes, den wir bloß \RWbet{voraussetzen}, den wir uns als vorhanden \RWbet{vorstellen} müssen, wenn eine wirkliche Offenbarung für uns vorhanden seyn soll. Das \RWbet{erste} Mal nehmen wir, meine ich, den Willen Gottes in seiner \RWbet{uneigentlichen}, das \RWbet{zweite} Mal in seiner \RWbet{eigentlichen} Bedeutung. Derjenige Wille Gottes, dessen \RWbet{Vorhandenseyn} nothwendig ist, wo eine wirkliche göttliche Offenbarung vorhanden seyn soll, ist, däucht mir, ein bloß \RWbet{uneigentlicher}; denn es genügt zu dem Vorhandenseyn~\RWSeitenw{87}\ einer göttlichen Offenbarung, daß von Seite Gottes nur alles dasjenige geschehen sey, was von seiner Seite geschehen konnte, um ihre Anerkennung bei uns zu bewirken. Derjenige Wille Gottes dagegen, den wir uns als vorhanden \RWbet{vorstellen}, auf dessen Daseyn wir (nach unserer besten Einsicht verfahrend) \RWbet{schließen} müssen, wenn eine Offenbarung vorhanden seyn soll, ist Gottes \RWbet{eigentlicher} Wille, und zwar ein \RWbet{unbedingter}. Denn wir müssen glauben, daß Gott gewisse Erscheinungen wirklich in der Absicht hervorgebracht habe, damit wir eine bestimmte Lehre künftig als wahr annehmen; und wir müssen glauben, daß er dieß wolle, nicht bloß \RWbet{zulassungsweise}, sondern \RWbet{unbedingt} deßhalb, weil es etwas an sich selbst Gutes ist. So, meine ich, verhält sich die Sache auch bei jedem Zeugnisse, welches von einem endlichen Wesen, \zB\ von einem Menschen, gegeben wird. Der Wille, der von Seite des Zeugen wirklich vorhanden seyn muß, damit er ein Zeugniß gegeben habe, braucht eben kein anderer zu seyn, als nur ein solcher, dabei er alles dasjenige thut, (und zwar mit Wissen und Willen) was sich von seiner Seite dabei thun läßt. Dieß kann Jemand, ohne daß er darum in Wahrheit zu wollen braucht, daß sein Zeugniß angenommen werde. Der Wille dagegen, den wir, die wir das Zeugniß annehmen sollen, uns bei dem Zeugen als vorhanden \RWbet{vorstellen} müssen, ist ein eigentlicher Wille, wir müssen uns denken, er habe nicht etwa bloß gewünscht, sondern in der That \RWbet{gewollt}, daß wir die von ihm bezeugte Meinung annehmen.
\end{aufza}

\RWpar{31}{Es gibt keine bloß vermeintliche göttliche Offenbarung}
Ob die so eben aufgestellte Erklärung des Begriffes einer göttlichen Offenbarung einen \RWbet{realen Begriff} enthalte; mit andern Worten, ob irgend ein Gegenstand angeblich sey, der diesem Begriffe entspricht, oder ob sich vielmehr ein \RWbet{innerer Widerspruch} in der Verbindung der Merkmale, die hier vorkommen, befinde: das müssen wir jetzt noch dahin gestellt seyn lassen, weil es erst im vierten Hauptstücke (wo von der Möglichkeit und den Kennzeichen göttlicher Offen\RWSeitenw{88}barungen die Rede seyn wird) näher untersucht werden kann. Folgender Lehrsatz dagegen läßt sich schon jetzt einsehen, und mag seiner Merkwürdigkeit wegen gleich hier mitgenommen werden. Es ist der Satz, \RWbet{daß es keine bloß vermeintliche göttliche Offenbarungen gebe.}\par
Eine bloß \RWbet{vermeintliche göttliche} Offenbarung wäre nach der Erklärung des \RWparnr{29}\ vorhanden, wenn Jemand eine Meinung annähme, weil er aus irgend einer Erscheinung nach seiner besten Einsicht geschlossen, daß Gott von ihm diese Meinung darum geglaubt wissen wolle, weil er sie selbst als wahr erkennt, \RWbet{während doch Gott dieß in der That nicht wollte}. Allein vermöge dessen, was ich schon \RWparnr{24}\ erinnert und \RWparnr{25}\ noch mehr gerechtfertiget habe, geschieht von Allem, was immer geschieht, nichts wider Gottes Willen, sondern vielmehr Alles gemäß seinem Willen. -- Somit kann sich der Fall, daß Gott nicht will, daß Jemand etwas glaube, und daß er es gleichwohl glaube, niemals ereignen; mithin kann es auch keine bloß vermeintliche Offenbarung geben.
\begin{RWanm}
Woher es komme, daß es \RWbet{vermeintliche Zeugnisse} wohl in Hinsicht auf \RWbet{Menschen}, nicht aber in Hinsicht auf \RWbet{Gott} geben könne, begreift sich leicht. Es kommt dieß nämlich nur daher, weil der Mensch eingeschränkt, Gott aber uneingeschränkt ist; weil Vieles wider den Willen und ohne das Wissen \RWbet{des Menschen}, nichts aber wider \RWbet{Gottes} Wissen und Willen erfolgt. Bei einem \RWbet{Menschen} also kann es sich eben um seiner Eingeschränktheit wegen ereignen, daß Andere eine Meinung annehmen, weil sie aus gewissen Erscheinungen nach ihrer besten Einsicht schließen, daß er diese Meinungen von ihnen geglaubt wissen wolle, ohne daß jener nur etwas davon weiß; oder, falls er auch davon weiß, ohne daß er im Stande ist, es zu verhindern. Dieses kann aber begreiflicher Weise nicht so bei \RWbet{Gott} geschehen.
\end{RWanm}

\RWpar{32}{Erklärung des Begriffes einer vorgeblichen oder bloß angenommenen göttlichen Offenbarung}
Wohl aber kann es \RWbet{vorgebliche} oder bloß \RWbet{angenommene} göttliche Offenbarungen geben.~\RWSeitenw{89}\ 
Hierunter verstehen wir nämlich vermöge der \RWparnr{29}\ gegebenen Erklärung \RWbet{eine Meinung, die Jemand darum annimmt, weil er mit absichtlicher Selbsttäuschung sich überredet, aus einer gewissen Erscheinung folge, es sey der Wille Gottes, daß er diese Meinung annehme, weil sie Gott selbst für wahr erkennt.} Wenn sich \zB\ \RWbet{Mahomed} aus Stolz, Eroberungssucht oder anderen Leidenschaften zu überreden suchte, daß er ein göttlicher Gesandter sey, daß diese und jene Einfälle, die seine Leidenschaft ihm eingab, Befehle Gottes seyen, \udgl : so waren dieß bloß \RWbet{vorgebliche}, bloß \RWbet{angenommene Offenbarungen} Gottes. Im Gegensatze von einer solchen bloß \RWbet{angenommenen} oder \RWbet{vorgeblichen Offenbarung} pflegt man die \RWparnr{30}\ erklärte der größern Deutlichkeit wegen auch eine \RWbet{wirkliche} göttliche Offenbarung zu nennen. Da Gott vermöge seiner höchsten \RWbet{Wahrhaftigkeit} nicht zulassen kann, daß eine Meinung, die wir auf sein wirkliches Zeugniß annehmen, falsch sey: so nennt man alle wirklichen Offenbarungen Gottes eben darum auch \RWbet{wahre}. Nur die vorgeblichen also kann man, je nachdem die Meinungen, die ihren Inhalt ausmachen, wahr oder falsch sind, in wahre und falsche eintheilen.

\RWpar{33}{Erklärung der Begriffe einer geoffenbarten und natürlichen Religion}
Jetzt erst bin ich im Stande, die \RWparnr{22}\ versprochene Erklärung des Begriffes einer \RWbet{geoffenbarten} und \RWbet{natürlichen} Religion zu geben.
\begin{aufza}
\item Eine Religion, deren einzelne Lehren auf irgend ein \RWbet{höheres Zeugniß}, es sey nun der \RWbet{Gottheit selbst}, oder doch irgend eines \RWbet{höhern Wesens}, \zB\ eines Engels, angenommen werden, wird eine \RWbet{positive} oder~\RWSeitenw{90}\ \RWbet{geoffenbarte}, nach Umständen auch \RWbet{göttlich geoffenbarte} Religion genannt. Eine Religion, deren einzelne Lehren auf eine bloß \RWbet{angenommene} oder \RWbet{vorgebliche} Offenbarung Gottes gegründet sind, heißt eine \RWbet{angeblich geoffenbarte Religion}. Und im Gegensatze mit dieser kann man die erstere der größern Deutlichkeit wegen auch eine \RWbet{wirkliche, göttlich geoffenbarte Religion} nennen. Gewöhnlich pflegt man auch eine bloß angeblich geoffenbarte Religion schon eine \RWbet{positive} zu nennen.
\item Jede Religion dagegen, deren Sätze auf kein, weder wirkliches, noch bloß angenommenes höheres Zeugniß gegründet werden, heißt eine \RWbet{natürliche} oder \RWbet{Vernunft-Religion}. Den Namen \RWbet{Vernunftreligion} hat sie erhalten, weil man von ihren Lehren zu sagen pflegt, daß man sie durch die \RWbet{bloße Vernunft} erkenne, nämlich durch die Vernunft \RWbet{allein}, ohne dabey der Hülfe eines \RWbet{höheren Zeugnisses} zu bedürfen. Warum sie den Namen \RWbet{natürliche Religion} erhalten, ist nicht so sicher zu bestimmen. Vielleicht weil ihre meisten Lehren aus der Betrachtung der \RWbet{Natur} hergeleitet werden; vielleicht auch im Gegensatze mit der \RWbet{geoffenbarten Religion}, von der man sich häufig vorgestellt hatte, daß sie gewisser \RWbet{übernatürlicher} Mittel zu ihrer Entstehung oder Ausbreitung bedürfe.
\end{aufza}

\RWpar{34}{Andere Erklärungen der Begriffe einer natürlichen und geoffenbarten Religion sammt ihrer Beurtheilung}
Obgleich die eben vorgetragenen Erklärungen der natürlichen sowohl, als der geoffenbarten Religion, wenn ich mich anders nicht irre, nur die Begriffe darstellen, welche der allgemein herrschende Sprachgebrauch mit diesen Worten verbindet; so muß ich doch gestehen, sie bisher noch nirgends genau so angetroffen zu haben. Es wird sich also geziemen,~\RWSeitenw{91}\ daß ich die wichtigsten Erklärungen, die man sonst gegeben hat, anführe, und die Gründe anzeige, warum ich geglaubt, sie verlassen zu müssen.
\begin{aufza} 
\item Ich kann hier füglich von der Erklärung anfangen, die man von der \RWbet{natürlichen Religion} gegeben, weil man geglaubt hat, daß diese unabhängig von dem Begriffe der Offenbarung erklärt werden könne, und weil man diese Erklärung fast immer auf einerlei Art abgefaßt hat. Fast überall nämlich liest man, daß die natürliche Religion diejenige sey, von deren Wahrheit man durch die Vernunft allein überzeugt werden könne. Gegen diese Erklärung erinnere ich aber:
\begin{aufzb}
\item Nimmt man das Wort \RWbet{Vernunft} in seiner engeren Bedeutung, in der es die Fähigkeit \RWbet{reiner Begriffs-Erkenntnisse} bezeichnet: so ist es \RWbet{falsch}, daß alle Wahrheiten der natürlichen Religion durch die Vernunft allein erkannt werden können. Denn mehre Sätze von größter Wichtigkeit, die man insgemein in das Gebiet der natürlichen Religion bezieht, sind aus der \RWbet{Erfahrung} entlehnt, oder bedürfen doch der Erfahrung zu ihrer völligen Bestätigung und Begründung, \zB\ die Sittenregeln der Mäßigkeit, der Keuschheit, \usw , die wichtige Lehre von der Unsterblichkeit der Seele \usw. In dieser Bedeutung des Wortes \RWbet{Vernunft} ist also die Erklärung der natürlichen Religion \RWbet{zu enge}.
\item Nimmt man dagegen das Wort \RWbet{Vernunft} in jener \RWbet{weiteren} Bedeutung, in der es eben so viel als \RWbet{Urtheilskraft} überhaupt, \dh\ das Vermögen zu urtheilen, bezeichnet: so ist es wieder nicht wahr, daß jede religiöse Wahrheit, welche der Mensch durch die Vernunft erkennt, in das Gebiet der natürlichen Religion gehöre. Denn in dieser weiteren Bedeutung erkennt der Mensch Alles, was er erkennet, durch die Vernunft. Auch die Wahrheiten der geoffenbarten Religion also würden in sofern, als sich der Mensch durch Gottes Zeugniß von denselben zu versichern, sie sonach zu \RWbet{erkennen} vermag, in das Gebiet der natürlichen Religion gehören. Folglich ist die Erklärung jetzt viel zu \RWbet{weit}.~\RWSeitenw{92}
\item Auch dadurch würde man sie nicht retten, daß man dem Worte \RWbet{Erkennen} eine besondere Bedeutung unterlegte; daß man erklärte, \RWbet{eine Wahrheit erkennen}, heiße so viel als \RWbet{den inneren Grund derselben einsehen}. Denn nun wäre die Erklärung der natürlichen Religion abermals zu \RWbet{enge}; weil es doch eine Menge wichtiger Lehren in ihr gibt, deren objectiven Grund wir nicht einsehen, ja gar nicht einzusehen vermögen.
\item Der einzige Fall, in welchem sich diese Erklärung rechtfertigen ließe, wäre also der, wenn man die Redensart: \RWbet{Etwas durch die Vernunft allein erkennen}, in einer solchen Bedeutung nähme, daß sie das Gegentheil von der Erkenntniß durch ein \RWbet{Zeugniß}, die man auch \RWbet{Glauben} nennt, bezeichnete; dann aber wäre jene Erklärung einerlei mit unserer oben gegebenen, und hätte billig der Deutlichkeit wegen auch so ausgedrückt werden sollen.
\end{aufzb}
\item Nicht so einstimmig, wie in der Erklärung des Begriffes der natürlichen Religion, war man bisher in der Erklärung dessen, was man eine \RWbet{Offenbarung} nenne. Daß der Begriff einer \RWbet{göttlichen Offenbarung} im Grunde nur der eines \RWbet{Zeugnisses} sey, welches Gott für die Wahrheit einer gewissen Lehre ablegt, konnte um so weniger ganz übersehen werden, da, wie ich oben schon sagte, in der \RWbet{heiligen Schrift} selbst die göttliche Offenbarung das \RWbet{Zeugniß Gottes} genannt wird. Daher erklärten denn schon mehre Gelehrte, \zB\ \RWbet{Kleuker, Reinhardt} \uA\ die göttliche Offenbarung als eine \RWbet{göttliche Zeugenschaft}; nur unterließen es Alle, den Begriff einer Zeugenschaft selbst in seine näheren Bestandtheile zu zerlegen. Von den Meisten wurden jedoch ganz andere Erklärungen gegeben. So trifft man
\begin{aufzb} 
\item namentlich bei den älteren Gelehrten häufig bloß die Erklärung an, eine göttliche Offenbarung wäre \RWbet{ein die Religion betreffender Unterricht Gottes an die Menschen.} (\RWlat{Revelatio est divina hominum ad veram et moralem religionem institutio}.)~\RWSeitenw{93}
\item Einige setzten hier noch die Worte bei: \RWbet{über Dinge, die der Mensch durch seine bloße Vernunft nicht zu erkennen vermag.} (So unter Andern \RWbet{Nösselt}.)
\item Andere sagten, \RWbet{die göttliche Offenbarung sey eine unmittelbare Belehrung Gottes}. (So \RWbet{Frint}, \RWbet{Tittmann}, \uA )
\item Statt des Wortes \anf{\RWbet{unmittelbar}} hatten Andere sonst das Wort \anf{\RWbet{übernatürlich}} gebraucht. (\RWlat{Revelatio est institutio supernaturalis}.) (So \RWbet{Kant} \uA )
\item Noch Andere setzten das Wort \anf{\RWbet{außerordentlich}}. (So \RWbet{Bonnet}, \RWbet{C.~Ludw.~Nitzsch} \uA )
\end{aufzb}
\RWbet{Ueber diese Erklärungen denke ich nun so}:
\begin{aufzb} 
\item Die Erklärung a) däucht mir \RWbet{zu weit}; denn ein \RWbet{Unterricht Gottes} kann ja gemäß dem \RWbet{Sprachgebrauche} des Wortes \RWbet{Unterricht} jede Erkenntniß, die wir auf irgend eine Weise uns erworben haben, heißen; weil eine jede uns von \RWbet{Gott} kommt, da er immer mit \RWbet{Wissen} und \RWbet{Willen} Ursache davon ist, daß wir zu ihr gelangen. (\RWparnr{27})
\item In Rücksicht des Beisatzes: \RWbet{über Dinge} \usw\ gelten dieselben Erinnerungen, welche ich über die Erklärung von der \RWbet{natürlichen} Religion so eben vorgebracht habe.
\item Bei einer Offenbarung kommt es dem Sprachgebrauche zu Folge gar nicht darauf an, ob Gott die \RWbet{unmittelbare} oder die \RWbet{mittelbare} Ursache jener Erkenntnisse ist, die uns durch sie zu Theil werden sollen (\RWparnr{28}, b, $\beta$.). So nennen wir \zB\ die christliche Religion noch heut zu Tage eine \RWbet{von Gott uns mitgetheilte Offenbarung}, ob sie gleich nichts weniger als unmittelbar von Gott zu uns gekommen ist. Von der anderen Seite könnte es sogar \RWbet{unmittelbare Belehrungen Gottes} geben, die gleichwohl gar keine Offenbarungen wären. Wenn Gott \zB\ durch eine unmittelbare Einwirkung auf unsern Verstand die Erkenntniß irgend einer Wahrheit zugleich mit der Einsicht ihres Grundes in uns hervorbrächte: so wäre dieß keine Offenbarung zu nennen, weil wir~\RWSeitenw{94}\ diese Kenntniß nicht auf \RWbet{Gottes Zeugniß} annehmen würden.
\item Auch die Erklärung d) däucht mir in jeder Bedeutung, die man dem Worte \RWbet{übernatürlich} gegeben hat, oder noch geben könnte, verwerflich.
\begin{aufzc}
\item Erklärt man das Wort \RWbet{Natur}, (wie es häufig geschieht) \RWbet{als den Inbegriff aller geschaffenen Wesen und Kräfte}: so würde eine \RWbet{übernatürliche Belehrung} Gottes so viel heißen, als eine durch kein geschaffenes Wesen und keine geschaffene Kraft \RWbet{vermittelte}, also ganz \RWbet{unmittelbare} Belehrung; und somit würde diese Erklärung ganz so, wie die nächst vorhergehende, zu beurtheilen seyn.
\item Einige haben daher unter \RWbet{natürlichen Wesen und Kräften} nur solche verstehen wollen, die Gott gleich im Anfange der Schöpfung erschaffen hatte; Wesen und Kräfte dagegen, die er erst \RWbet{in der Zeit} hervorruft, nennen sie \RWbet{übernatürliche}. Nach ihrer Erklärung also wäre die Offenbarung ein Unterricht, welchen uns Gott durch Wesen und Kräfte ertheilt, die er \RWbet{erst in der Zeit}, etwa zu eben der Zeit, da dieser Unterricht ertheilt wird, erschaffen hat. Ich lasse es nun dahin gestellt, ob der Gedanke, daß Gott \RWbet{erst in der Zeit} Wesen und Kräfte erschaffe, seine vollkommene Richtigkeit habe; und bemerke nur, daß es dem Sprachgebrauche zu Folge gar nicht darauf ankommt, ob jene Wesen und Kräfte, durch die uns Gott eine Offenbarung ertheilt, in dieser oder in jener Zeit geschaffen werden.
\item Wollte man etwa unter übernatürlichen Wirkungen Gottes solche verstehen, die er \RWbet{durch eigends für diese Wirkungen allein geschaffene Kräfte hervorbringt}: so wäre die Offenbarung ein solcher Unterricht Gottes an die Menschen, zu dessen Hervorbringung sich Gott gewisser Kräfte bedient hat, die er nur ausschließlich für diesen Zweck allein geschaffen. -- Aber wer siehet nicht, daß auch diese Erklärung dem Sprachgebrauche gar nicht gemäß wäre?~\RWSeitenw{95}\ Denn warum sollten wir einen gewissen Unterricht Gottes nur deßwegen keine Offenbarung nennen, weil die Kräfte, durch die ihn Gott hervorzubringen wußte, nach seiner unendlichen Weisheit auch noch zu manchen andern, etwa uns unbekannten Zwecken benützet worden sind? --
\item Verstehet man endlich (wie auch \RWbet{dieß} Einige gethan) unter \RWbet{Natur} den Inbegriff aller auf Erden \RWbet{gewöhnlich} wirkenden Kräfte: dann heißt eine \RWbet{übernatürliche Wirkung} eine solche, welche durch Kräfte \RWbet{höherer} Wesen, die nur selten auf Erden einwirken, erfolgte. Es ist nun wahr, daß eine jede Offenbarung von einigen \RWbet{seltenen Erscheinungen} begleitet seyn müsse; daß aber die Wesen, die diese seltenen Erscheinungen erzeugen, eben Wesen von \RWbet{höherer} Art, solche, die sich nur selten auf Erden aufhalten, seyn müßten, ist keineswegs nöthig. Auch gilt es nicht umgekehrt, daß eine jede Erkenntniß, die uns durch eine seltene Erscheinung, und durch Vermittlung von Wesen, die sich nur selten auf Erden zeigen, zu Theil wird, eine göttliche Offenbarung seyn müsse. Durch jede \RWbet{ungewöhnliche Naturerscheinung}, \zB\ durch einen Steinregen, bringt uns Gott ebenfalls mancherlei neue Kenntnisse bei. Nennen wir aber dergleichen Kenntnisse wohl göttliche \RWbet{Offenbarungen} in jener engeren Bedeutung, in der wir die Offenbarung der natürlichen Religion entgegen setzen? Diese Erklärung wäre dann also abermals zu weit.
\end{aufzc}
\item Den zuletzt gerügten Fehler hat auch die Erklärung, daß eine Offenbarung nichts Anderes als eine \RWbet{außerordentliche} Belehrung Gottes wäre. Es ist sehr wahr, daß Gott bei einer jeden Offenbarung auf eine außerordentliche Weise einwirken müsse; aber es läßt sich nicht umgekehrt behaupten, daß eine jede Erkenntniß, die uns auf außerordentliche Art von Gott zu Theil wird, eine göttliche Offenbarung zu nennen sey.
\end{aufzb}
\end{aufza}

\RWpar{35}{Formelle und materielle Offenbarung}
Richten wir unser Augenmerk auf die \RWbet{Beschaffenheit} der Lehren, welche uns Gott in seinen Offenbarungen mittheilt;~\RWSeitenw{96}\ so kann es einen gar nicht unwichtigen Unterschied unter denselben geben, welcher die Eintheilung in \RWbet{formelle} und \RWbet{materielle} Offenbarungen begründet.\par
Der Inhalt einer göttlichen Offenbarung kann nämlich eine Lehre von solcher Art seyn, deren Wahrheiten wir einsehen könnten, auch ohne des göttlichen Zeugnisses gerade zu bedürfen; oder es kann im Gegentheil eine Lehre seyn, die wir, wenn Gott uns nicht ihre Wahrheit bezeugt hätte, durch unsere sich selbst überlassene Vernunft keineswegs zu erkennen vermögten. Im ersten Falle nennt man die Offenbarung eine bloß \RWbet{formelle}, weil es nicht ihren Lehren an sich (nicht der \RWbet{Materie}), sondern nur jener Art (der \RWbet{Form}), wie diese Lehren uns mitgetheilt wurden, anzusehen ist, ob sie geoffenbaret seyen oder nicht. Im zweiten Falle nennt man sie eine \RWbet{materielle} Offenbarung; weil es nicht bloß aus der Form, sondern auch aus der Materie der Lehren erhellet, daß sie uns nur durch eine Offenbarung bekannt geworden seyn können.

\RWpar{36}{Begriff der vollkommensten Religion}
\begin{aufza}
\item Da wir nur solche Lehren zu dem Inhalte einer Religion zählen, die für die Tugend und Glückseligkeit nicht gleichgültig sind; so ist leicht zu erachten, daß verschiedene Religionen auch einen sehr verschiedenen Einfluß auf die Tugend und Glückseligkeit der Menschen haben werden. Wohl muß es aber unter allen gedenkbaren Religionen Eine, oder es kann vielleicht auch etliche geben, welche der Tugend und Glückseligkeit eines bestimmten Menschen, wenn nicht für immer, wenigstens für einen gewissen Zeitraum seines Lebens so zuträglich sind, daß keine andere noch zuträglicher wäre. Es sey mir erlaubt, eine solche Religion die \RWbet{vollkommenste} für diesen Menschen und für diese Zeit seines Lebens zu nennen.
\item Ein Ähnliches wird sich von einem ganzen Inbegriffe von Menschen, von einem ganzen Volke, oder von sonst einer andern Gesellschaft mehrer Menschen behaupten lassen. Auch für diese muß jedesmal ein religiöser Lehrbegriff~\RWSeitenw{97}\ angeblich seyn, der, wenn auch vielleicht nicht eben auf jeden Einzelnen, doch auf die Gesammtheit so wohlthätig einwirkt, daß keine andere Religion auf diese Menschen im Ganzen wohlthätiger einwirken würde. Diese Religion will ich denn die \RWbet{vollkommenste für diese Gesellschaft} nennen.
\item Eine Religion endlich, welche alle diejenigen Lehren und Ansichten in sich schließt, die für den Menschen im \RWbet{Allgemeinen}, \dh\ wie wir ihn uns mit Weglassung aller bloß bei dem Einzelnen Statt findenden fehlerhaften Eigenheiten denken, in dem Maße zuträglich sind, daß keine anderen noch zuträglicher wären, nenne ich \RWbet{die vollkommenste Religion an sich}, oder auch schlechtweg nur die \RWbet{vollkommenste}.
\item In diesen drei Erklärungen habe ich mit Absicht nicht gesagt, daß eine Religion, welche den Namen der vollkommensten, es sey nun für einen einzelnen Menschen oder für Mehre, oder auch für die Menschheit im Allgemeinen, verdienen soll, die \RWbet{allerzuträglichste} seyn, \dh\ daß jede andere \RWbet{minder} zuträglich seyn müsse; sondern nur, daß keine andere da seyn dürfe, die noch \RWbet{mehr} Zuträglichkeit hat, als sie. So nämlich wird es mir Niemand abläugnen können, daß jeder dieser Begriffe ein \RWbet{gegenständlicher} (realer) sey, \dh\ daß es ein Ding der Art, wie er es vorstellet, gebe. Denn daß es unter allen gedenkbaren Religionen jederzeit wenigstens eine geben müsse, die so zuträglich ist, daß keine andere noch mehr Zuträglichkeit hat, ist außer allem Zweifel; weil ja das Maß der menschlichen Fassungskraft nur ein endliches, und mithin auch die Menge der religiösen Meinungen, für die er einige Empfänglichkeit hat, nur eine endlich ist. Würde ich aber zu der vollkommensten Religion verlangen, daß jede andere \RWbet{minder zuträglich} seyn müsse; so könnte Jemand bezweifeln, ob es auch jedesmal eine solche gebe. Denn man könnte sagen, es sey der Fall gedenkbar, daß zwei verschiedene Religionen einen beiderseits gleichen und so hohen Grad der Zuträglichkeit haben, daß keine dritte noch zuträglicher ist; und in diesem Falle würde ja keine den Namen der vollkommensten verdienen.~\RWSeitenw{98}
\item Mit Absicht habe ich es ferner in diesen drei Erklärungen unentschieden gelassen, ob alle einzelnen Lehren, aus welchen diese Religionen zusammengesetzt sind, durchaus nur \RWbet{Wahrheiten} seyn müßten oder nicht. Hierüber ließe sich nämlich, besonders wenn eine Religion nur für diesen oder jenen einzelnen Menschen, oder für eine gewisse Gesellschaft von Menschen die vollkommenste seyn soll, noch streiten. Eine Erklärung aber soll nur so abgefaßt werden, daß ein Gegner so wenig als möglich Veranlassung finde, schon den hier aufgestellten Begriff selbst in Anspruch zu nehmen.
\item Daß sich übrigens diese drei Begriffe nicht nur in ihren in der Erklärung angegebenen Bestandtheilen, sondern auch ihrem \RWbet{Umfange} nach unterscheiden, \dh\ daß sie nicht einerlei Gegenstände bezeichnen, wird man bald einsehen. Eine Religion, welche die vollkommenste an sich ist, muß darum nicht auch die vollkommenste für jeden einzelnen Menschen seyn; denn weil ein jeder Mensch auch seine eigenthümlichen Verhältnisse und Bedürfnisse hat, so läßt sich gar wohl die Möglichkeit denken, daß es noch nebst den Lehren, die jene vollkommenste Religion enthält, gar manche nur ihm allein nöthige Aufschlüsse und Belehrungen gebe; und eben so ist es auch möglich, daß gewisse Wahrheiten, welche zur vollkommensten Religion an sich gehören, ihm noch nicht einmal zuträglich sind, daß er \zB\ wenn er sich in einem noch allzu kindlichen Alter befindet, noch keine Empfänglichkeit für sie hat, \udgl\ Und wie einzelne Menschen, so kann es auch ganze Gesellschaften und Völker geben, die noch nicht vorbereitet genug sind, um alle Lehren der Religion, welche die vollkommenste an sich ist, mit Nutzen aufzufassen.
\item Der letzte dieser Begriffe, nämlich der einer \RWbet{vollkommensten Religion an sich}, ist eben derjenige, der in dem oben aufgestellten Begriffe der \RWbet{Religionswissenschaft} gemeint war; denn jene vollkommenste Religion, mit welcher diese uns bekannt machen soll, ist keine andere, als die vollkommenste Religion \RWbet{an sich}. Nicht in der Religion, die nur für diesen oder jenen einzelnen Menschen, sondern in derjenigen, die für die \RWbet{Menschheit überhaupt} die vollkommenste ist, soll die Religionswissenschaft unterrichten.~\RWSeitenw{99}
\end{aufza}

\RWpar{37}{Hauptpflicht des Menschen in Hinsicht auf seine Religion.}
Schon \RWparnr{18}\ wurde gezeigt, daß wir Menschen auf die Entstehung unserer Meinungen und Ueberzeugung einen beträchtlichen Einfluß durch unsern Willen haben; und aus der Art, wie der Begriff der Religion in dieses Wortes weiterer Bedeutung festgesetzt ward, erhellet, daß es insonderheit unsere \RWbet{religiösen} Meinungen sind, bei welchen dieser Einfuß von einer besondern Stärke und Wichtigkeit ist; weil dieses Meinungen sind, die für den Zweck unserer Tugend und Glückseligkeit nicht gleichgültig sind, und eine eigene Versuchung darbieten, uns ohne gehörigen Grund entweder für oder wider sie zu bestimmen. Dieß leitet uns denn auf die Frage, \RWbet{welchen Gebrauch wir von diesem Einfluße unsers Willens auf unsere Religion zu machen schuldig seyen?} oder mit andern Worten, was es für Pflichten gebe, die wir in Ansehung unserer eigenen religiösen Meinungen und Ueberzeugungen haben? -- Wenn ich die \RWbet{sämmtlichen Pflichten}, welche uns Menschen in dieser Hinsicht obliegen, \RWbet{zuerst} in eine \RWbet{einzige Regel} zusammenfassen soll; so werde ich sie ohngefähr so ausdrücken müssen: \RWbet{Ein Jeder hat nach der Erkenntniß der für ihn vollkommensten Religion zu streben}, \dh\ ein Jeder hat, so viel er vermag, immer nur jene Begriffe und Meinungen zu ergreifen, die seiner Tugend und Glückseligkeit entweder die zuträglichsten, oder doch so zuträglich sind, daß keine anderen noch zuträglicher wären. Wir sollen nämlich ein Jeder alles dasjenige thun, was zur Beförderung unserer Tugend und Glückseligkeit nur immer beitragen kann, und uns nicht unmöglich ist. Nun ist es für den Zweck unserer Tugend und Glückseligkeit nichts weniger als gleichgültig, ob wir diesen oder jenen religiösen Meinungen und Begriffen anhängen; vielmehr gibt es Begriffe und Meinungen, die einen überaus wohlthätigen Einfluß auf unsere Tugend oder Glückseligkeit haben; es stehet auch häufig in unserer Macht, ob wir uns diese Begriffe und Meinungen aneignen oder nicht: also ist's außer Zwei\RWSeitenw{100}fel, wir sollen darnach streben, daß wir uns alle jene Begriffe und Ansichten, die unserer Tugend oder Glückseligkeit so zuträglich sind, daß keine anderen sie noch übertreffen, bekannt und zu eigen machen.

\RWpar{38}{Umständlichere Auseinandersetzung der einzelnen in dieser Hauptpflicht enthaltenen Verbindlichkeiten.}
Die eben erwiesene Pflicht, die wir mit einem Worte auch die \RWbet{Glaubenspflicht} des Menschen nennen könnten, enthält so viele einzelne Verbindlichkeiten von größter Wichtigkeit in sich, daß es nothwendig ist, sie etwas umständlicher auseinanderzusetzen.
\begin{aufza} 
\item Von der Zeit an, da wir zum vollen Gebrauche unserer Vernunft gelangen, sind wir verpflichtet, auf alle jene Meinungen, die einen Einfluß auf unsere Tugend oder Glückseligkeit vermuthen lassen, vollends, wenn sie zugleich von einer solchen Art sind, daß eine eigene Versuchung, sich entweder für oder wider sie zu erklären, in unsern Herzen sich reget, besonders aufmerksam zu seyn; die Beschaffenheit ihres Einflußes von Zeit zu Zeit zu prüfen, und diejenigen Meinungen, die unserer Tugend und Glückseligkeit nachtheilig sind, nach aller Möglichkeit zu entfernen. Daß diese Pflicht erst eintrete, wenn wir \RWbet{den vollen Gebrauch unserer Vernunft} erhalten, leuchtet von selbst ein; denn früher konnten wir ja noch gar nicht beurtheilen, welche Meinungen einen Einfluß auf unsere Tugend oder Glückseligkeit haben, und ob dieser Einfluß wohlthätig oder nachtheilig sey. Seitdem wir dieß aber vermögen, sind wir verpflichtet, auf solche Meinungen besonders \RWbet{aufmerksam} zu seyn, weil sie die \RWbet{wichtigsten}, ja gewissermaßen die einzig wichtigen sind. -- Wir müssen den Einfluß derselben \RWbet{von} \RWbet{Zeit zu Zeit} untersuchen, weil er sich
\begin{aufzb} 
\item mit der Zeit \RWbet{ändern} kann, und eine Meinung, die für eine gewisse Zeit wohlthätig für uns war, in der Folge schädlich werden kann; weil unsere Einsichten~\RWSeitenw{101}
\item auch \RWbet{zunehmen}, und wir bei einer späteren Prüfung vielleicht entdecken können, was unserer Aufmerksamkeit bei einer früheren entgangen war; weil wir endlich auch
\item allmählich verschiedene neue Begriffe und Meinungen annehmen, deren Einfuß wir vorhin, da wir sie noch nicht \RWbet{hatten}, auch nicht zu \RWbet{prüfen} vermochten.
\end{aufzb}
\item Wenn es sich fügen sollte, daß wir für die Wahrheit einer gegebenen Ansicht zwar keine völlig entscheidende Gründe fänden; aber doch sicher beurtheilen könnten, daß ihre Annahme, selbst auf den Fall ihrer Unrichtigkeit, nicht anders als wohlthätig auf unsere Tugend und Glückseligkeit einwirken werde: so wird es nicht nur erlaubt, sondern sogar etwas Verdienstliches seyn, wenn wir alle uns zu Gebote stehende Mittel benützen, uns von dieser Ansicht wirklich zu überreden. Wir werden also \zB\ wohl thun, wenn wir alle wahren oder nur scheinbaren Gründe, die diese Meinung für sich hat, öfters in's Auge fassen; von den Gründen dagegen, die ihr entgegenstehen, unsere Aufmerksamkeit, so viel als möglich, abziehen; Umgang mit Menschen pflegen, die dieser Meinung zugethan sind; Schriften, in denen sie vertheidigt wird, lesen, \usw\ Durch ein solches Betragen werden wir nämlich, gesetzt auch, es komme allmählich wirklich dahin, daß wir die Meinung als Wahrheit annehmen, ohngeachtet sie doch nur ein Irrthum ist, weder uns noch Andern schaden; sondern im Gegentheil, weil unsere Tugend und Glückseligkeit vollkommener werden wird, wird auch das Ganze gewinnen.
\end{aufza}\par
\RWbet{Einwurf.} \RWbet{Irrthum} kann nie wohlthätig seyn; nur \RWbet{Wahrheit} ist es. Der Irrthum aber ist das gerade Gegentheil der Wahrheit, wie kann er also dieselben Wirkungen mit ihr hervorbringen? Können wohl \RWbet{Licht} und \RWbet{Finsterniß} einerlei Wirkungen äußern?\par
\RWbet{Antwort.}
\begin{aufza} \item Allerdings können auch entgegengesetzte Dinge unter verschiedenen, ja zuweilen selbst unter gleichen Umständen gleiche Wirkungen äußern. So führt man in diesem Einwurfe \RWbet{Licht} und \RWbet{Finsterniß} als zwei einander entgegengesetzte Dinge an, und doch können beide die gleiche~\RWSeitenw{102}\ Wirkung äußern, daß sie das deutliche Sehen verhindern. Zu starkes Licht nämlich kann dieses gerade so, wie zu starke Finsterniß thun. So folgt also auch bloß daraus, weil Wahrheit und Irrthum einander entgegengesetzt sind, noch gar nicht, daß sie unter verschiedenen, ja vielleicht gar unter denselben Umständen, nicht beide nützlich werden könnten.
\item Ferner ist auch gewiß, daß \RWbet{nicht alle} Wahrheit wohlthätig für den Menschen wirke. Denn wenn auch bei einem Wesen, dessen \RWbet{Verstand} ohne alle Irrthümer, und dessen \RWbet{Empfindungs-} und \RWbet{Begehrungsvermögen} ohne alle Schwächen wäre, jede neue Wahrheit nur lauter wohlthätige Folgen hervorbringen müßte: so ist dieses doch nicht der Fall, in welchem wir \RWbet{Menschen} uns befinden. Auch dem \RWbet{Weisesten} klebt noch so mancher Irrthum, und auch dem \RWbet{Besten} und \RWbet{Vollkommensten} so manche Schwäche des Empfindungs- und Begehrungsvermögens an. Wo aber \RWbet{Irrthum} und \RWbet{Schwäche}\ vorhanden ist, kann eine Wahrheit, die hinzukommt, unläugbar so verderbliche Folgen erzeugen, daß der Zustand der Unwissenheit viel vortheilhafter gewesen wäre. Kommt nämlich \RWbet{Wahrheit} zu vorhandenen \RWbet{Irrthümern}, so kann sie leicht \RWbet{neue} und sehr verderbliche Irrthümer erzeugen. Wenn \zB\ Jemand den Irrthum hegt, daß die heil.\ Schrift nicht Gottes Werk seyn könnte, falls irgend eine der in ihr vorkommenden Jahreszahlen oder eine andere historische Angabe unrichtig wäre: wird dieser nicht, wenn man ihm beibringt, daß wirklich einige dieser Angaben nicht ganz verlässig sind, in den neuen sehr schädlichen Irrthum verfallen, daß die heil.\ Schrift nicht Gottes Werk sey? -- Eben so kann dort, wo gewisse Schwachheiten des \RWbet{Empfindungs-} und \RWbet{Begehrungsvermögens} obwalten, die Erkenntniß einer neuen Wahrheit statt gehofften Nutzens nur Schaden anrichten, weil sie uns eine Betrübniß verursacht, die wir ersparen konnten, oder Begierden anfacht, die nicht befriedigt werden dürfen, oder uns Mittel und Wege zur Ausführung eines Verbrechens an die Hand gibt, \udgl\ So kann es \zB\ sehr schädlich werden, wenn man einem Kranken den so eben erfolgten Tod seines Freundes erzählt; oder einem Blindgebornen, der nicht~\RWSeitenw{103}\ zu heilen ist, erklärt, wie viele Vortheile ihm der Sinn des Gesichtes verschaffen könnte; oder dem Bösewichte, der mit Mordgedanken umgeht, die geheime Bereitungsart des \RWlat{aqua toffana} lehrt.
\item Ist nun \RWbet{Unwissenheit} in vielen Fällen nützlich, warum sollte es nicht auch der \RWbet{Irrthum} seyn können? Warum sollte nicht auch ein \RWbet{Irrthum} zuweilen vortrefflich dienen können, um die schlimmen Wirkungen eines andern schwer zu vermeidenden Irrthumes, oder die üblen Folgen eines bestimmten Fehlers in unserem Empfindungs- und Begehrungsvermögen aufzuheben? Wenn Jemand den Ekel, den er vor einem gewissen Arzneimittel empfindet, so oft er an die Bestandtheile desselben denkt, nicht überwinden kann: werden die schädlichen Folgen, die dieser Fehler seines Empfindungsvermögens hat, nicht durch den Irrthum, daß die Arznei aus etwas Anderem bereitet sey, am Leichtesten aufgehoben? Wenn Jemand irgend eine verbotene Begierde nicht unterdrücken kann, so lange er das Mittel zu ihrer Befriedigung vor Augen hat: werden nicht die Gefahren, die dieser Fehler seines Begehrungsvermögens hat, am Sichersten beseitiget, wenn er sich vorstellt, daß keine Möglichkeit sie zu befriedigen da sey? Und eben so, wenn ein sehr ungebildeter Mensch von dem Irrthume, daß nur sinnliche Schmerzen die größten wären, schwer loszureißen ist: wird ihm da nicht die Vorstellung, daß die Strafen des andern Lebens in sinnlichen Schmerzen bestehen, auch falls sie unrichtig ist, vortreffliche Dienste leisten, indem sie die schlimmen Wirkungen, die jener erste Irrthum hervorbringen könnte, verhindert? Wird er sich nicht jetzt um so sorgfältiger vor jeder Sünde hüten?
\end{aufza}\par
\RWbet{Einwurf.} Wenn man auch zugeben wollte, daß ein gewisser Irrthum für einen \RWbet{einzelnen} Fall wohlthätig sey, so wird er für hundert andere Fälle doch um so nachtheiliger wirken.\par
\RWbet{Antwort.} Aber vielleicht lassen sich Irrthümer denken, aus denen gar keine oder wenigs[tens] keine überwiegend nachtheilige Folgen hervorgehen. Von dieser Art dürfte das zuletzt angeführte Beispiel von den sinnlichen Strafen im andern Leben, \uma\ seyn.~\RWSeitenw{104}


\RWpar{39}{Fortsetzung. Besondere Pflichten in Hinsicht auf eine Offenbarung}
\begin{aufza}\setcounter{enumi}{2}
\item Insonderheit bestehet für Jeden, der den Begriff einer Offenbarung kennen gelernt hat, und einigermaßen begreift, wie nützlich ihm eine solche werden könnte, die Pflicht zu untersuchen, ob es nicht in der That eine göttliche Offenbarung gebe, und falls er eine findet, sie gläubig anzunehmen. Eine Offenbarung ist nämlich eine Lehre, die Gott von uns geglaubt wissen \RWbet{will mit seinem unbedingten Willen, und dieß zwar darum, weil er sie selbst für wahr erkennet}. Der letztere Beisatz heißt eben so viel, als \RWbet{weil Gott die Annahme dieser Lehre für uns am zuträglichsten findet.} Da sich nun Gott unmöglich irren kann, so muß eine solche Lehre auch \RWbet{wirklich} den höchsten Grad der Zuträglichkeit für uns besitzen, und somit ist es Pflicht, sie aufzusuchen, und wenn wir sie gefunden, sie gläubig anzunehmen; denn dadurch werden wir gewiß unsere Tugend sowohl als unsere Glückseligkeit in einem hohen Grade befördern.
\item Es läßt sich aber das Daseyn einer wahren göttlichen Offenbarung für uns -- wenn wir einige ganz außerordentliche Fälle jetzt nicht beachten wollen -- in der Regel nur unter den mancherlei auf Erden vorhandenen \RWbet{Religionen} vermuthen. Es liegt uns also die Pflicht ob, diese, so viel wir es vermögen, der Reihe nach kennen zu lernen, und zu untersuchen, ob irgend eine derselben die Kennzeichen einer wahren göttlichen Offenbarung habe. Da diese Untersuchung nicht nur von äußerster Wichtigkeit, sondern auch ihrer Natur nach eine sehr viel umfassende ist: so sollen wir sie erst dann, bis unsere Vernunft zu einem gewissen Grade der Ausbildung gediehen ist, (etwa in den Jahren der angehenden Mannbarkeit) mit allem Fleiße vornehmen.
\item Begreiflicher Weise werden wir bei dieser Untersuchung auch nicht zu schnell vorgehen dürfen; weil wir uns sonst leicht übereilen könnten. Wir müssen ihr also einen, der Wichtigkeit und der Schwierigkeit ihres Gegenstandes angemessenen Zeitraum, \zB\ von etlichen Jahren, widmen.
\item Je nachdem uns Gott mehr oder weniger Verstandeskräfte und äußere günstige Gelegenheiten gegeben, werden wir bei unserer Untersuchung bald mehr, bald weniger tief in das Innere dringen, bald mehr bald weniger auf die Aussagen Anderer uns verlassen dürfen, \usw\ Je weiter wir~\RWSeitenw{105}\ nämlich in unsern Nachforschungen fortgehen, vorausgesetzt, daß wir nichts wagen, was unsere Kräfte übersteigt, um desto gesicherter sind wir vor einem Irrthum, um desto fester wird unsere Ueberzeugung, um desto mehr gewinnt auch die zuletzt gefundene Wahrheit an Wichtigkeit in unsern Augen, und wird uns um so werther.
\item Am schwierigsten ist die Frage, wie wir uns mittlerweile, bevor unsere Untersuchung über die wahre göttliche Offenbarung noch ganz geschlossen ist, zu benehmen haben, wenn wir in Verhältnisse kommen, in welchen verschiedene Religionen auch ein verschiedenes Betragen vorschreiben? -- Die \RWbet{allgemeine} Antwort, die ich auf diese Frage glaube ertheilen zu können, ist diese: \RWbet{Wir sollen in jedem einzelnen Falle nur derjenigen Regel folgen, deren Befolgung für diesen Fall -- auch abgesehen von der Religion, aus welcher dieselbe entlehnt ist -- unserer Tugend und Glückseligkeit am Zuträglichsten ist}; wovon nur dann eine Ausnahme zu machen wäre, wenn die Befolgung jener Regel nicht ohne ein Aufsehen erregendes Bekenntniß zu einem Glauben, von dessen Wahrheit wir noch keine zureichende Ueberzeugung haben, und also nicht ohne eine Art von \RWbet{Lüge} möglich wäre. Einer Regel, die unserer Tugend und Glückseligkeit zuträglich ist, sind wir verbunden zu folgen, wenn es auch gar keine \RWbet{göttliche Offenbarung} wäre, die uns dieselbe vorschreibt; um wie vielmehr, wenn sie in einer Religion aufgestellt wird, von der es uns bisher mehr oder weniger wahrscheinlich ist, daß sie eine göttliche Offenbarung sey. -- Daß wir jedoch nicht auch in dem Falle verbunden seyn können, die Regel zu befolgen, wenn sie von uns eine Handlung fordert, durch die wir feierlich erklären würden, daß wir die Religion, aus der sie entlehnt ist, für eine wahre göttliche Offenbarung erkennen, folgt aus der \RWbet{Pflicht der Wahrhaftigkeit}. Wir würden da nämlich eine \RWbet{Lüge}, und zwar in einem höchst wichtigen Gegenstande begehen, indem wir feierlich erklären würden, daß wir eine gewisse Religion für eine wahre göttliche Offenbarung halten, während wir doch hievon noch nicht hinreichend überzeugt sind. -- So kann und soll also \zB\ ein geborener~\RWSeitenw{106}\ Katholik, auch so lange er sich von der Wahrheit der katholischen Religion noch nicht befriedigend überzeugt hat, zwar wohl die heil.\ Sakramente der Buße und des Altars empfangen; jenes der heil.\ Firmung aber sollte er füglich nicht eher verlangen, als bis seine Zweifel hinlänglich gelöst sind.
\item Waren wir endlich so glücklich, eine Religion, welche die Kennzeichen einer wahren göttlichen Offenbarung an sich trägt, zu entdecken: so wird es unsere Pflicht, wenn es nicht etwa schon früher geschehen, uns von allen einzelnen Lehren derselben \RWbet{eine möglich genaue und vollständige Kenntniß zu verschaffen}, und diesen Ansichten durch Uebung alle diejenige Geläufigkeit, Lebhaftigkeit und Festigkeit zu ertheilen, welche sie zur Erfüllung ihrer Bestimmung nur immer nöthig haben. Wir sollen daher jene Lehren sammt ihren Gründen vielfältig wiederholen, sie aus verschiedenen Gesichtspuncten betrachten, sie in geläufige Verbindung mit unsern übrigen Begriffen bringen, sie auf die eigenthümlichen Verhältnisse unsers Lebens anwenden, \usw\ Daß wir uns von den Lehren einer Religion, welche Gott selbst von uns geglaubt wissen will, zuvörderst eine möglichst genaue und vollständige Kenntniß verschaffen müssen; leuchtet von selbst ein. Oder was könnte es uns nützen, zu wissen, hier habe sich Gott geoffenbart, wenn wir nicht frügen, \RWbet{was} er geoffenbaret habe? Wir müssen ferner trachten, diese Begriffe uns recht geläufig zu machen, und sie zu einem recht hohen Grade der Lebhaftigkeit und Festigkeit zu erheben; weil wir im widrigen Falle die Vorschriften unsers Glaubens oft aus Vergessenheit übertreten, das unserer Sinnlichkeit Lästige schwerlich beobachten, die Beweggründe, die unser Glaube uns zur Erfüllung unserer Pflichten darbeut, nur wenig gelten lassen, die heiteren Aussichten endlich, welche er uns in die Zukunft eröffnet, wie nicht verlässig, so auch nicht wirksam genug finden würden.
\item Wenn die göttliche Offenbarung, die wir gefunden haben, eine \RWbet{Gesellschaftsreligion} ist; so können wir höchstens erwarten, daß sie die sämmtlichen Lehren enthalte, die zu der \RWbet{vollkommensten Religion an sich} (\RWparnr{36}\ \no\,3) gehören; daß sie uns aber auch alle diejenigen Belehrungen ertheile,~\RWSeitenw{107}\ die uns nach unserer ganz eigenthümlichen Lage (Individualität) nothwendig sind, und somit die vollkommenste Religion auch \RWbet{für uns selbst} sey, können wir nicht verlangen. Nicht dürfen wir also vermeinen, unsere Glaubenspflicht schon ganz erfüllt zu haben, wenn wir uns nur mit den sämmtlichen Lehren der wahren göttlichen Offenbarung bekannt gemacht haben; sondern wir müssen auch jetzt noch eifrig trachten, jede in unseren Verhältnissen uns nur ersprießliche Wahrheit, auf welche Art wir es immer vermögen, kennen zu lernen.
\item Wenn uns, nachdem wir eine Ueberzeugung angenommen haben, hinterher wieder neue sie betreffende Zweifel und Einwürfe aufstoßen: \RWbet{so dürfen wir uns diese nicht sofort nur aus dem Sinne schlagen}; sondern wir müssen sie vielmehr sobald als möglich \RWbet{in eine nähere Erwägung ziehen}, und sie zu lösen und zu beantworten suchen. Durch eine solche Prüfung kann nämlich
\begin{aufzb}
\item unsere bisherige Ansicht vielleicht noch \RWbet{berichtiget} werden, indem wir einen früher gemachten Fehlschluß entdecken; oder wir können
\item so glücklich seyn, den Zweifel zu lösen, wodurch dann unsere \RWbet{Ueberzeugung} erhöht wird; in jedem Falle endlich
\item erhalten wir so doch eine \RWbet{Gelegenheit, uns in vernünftigem Denken zu üben}.
\end{aufzb}
\item Finden wir aber, nachdem wir uns eine hinlängliche Mühe gegeben, keine beruhigende Auflösung unseres Zweifels: so müssen wir unterscheiden, ob es ein Zweifel ist, \RWbet{der unseren bisherigen Beweis für die Wahrheit einer religiösen Meinung ganz umstößt}, oder ob dieser fortbestehet. Im \RWbet{ersten} Falle, wo der Beweisgrund, den wir für unsere bisherige Ansicht gehabt, aufhört überzeugend zu seyn, müssen wir uns nach einem neuen umsehen, und wenn wir keinen finden, unsere bisherige Meinung zwar nicht als einen \RWbet{Irrthum} verwerfen, aber doch nicht mehr für entschieden ansehen. -- Im \RWbet{zweiten} Falle dagegen, wo der Grund, den wir für unsere bisherige Meinung gehabt, noch immer seine Kraft behält, müssen wir dieser auch immer noch zugethan bleiben; weil~\RWSeitenw{108}\ aber die stete Gegenwart des Zweifels in unserem Bewußtseyn gleichwohl die Lebhaftigkeit unserer Ueberzeugung schwächen würde: so wird es zweckmäßig seyn, ihn nicht ununterbrochen im Sinne zu behalten.
\end{aufza}

\RWpar{40}{Verschiedene fehlerhafte Verhaltungsarten der Menschen gegen ihre Religion}
Von dem bisher (\RWparnr{37--39}) beschriebenen \RWbet{pflichtmäßigen Verhalten gegen die Religion}, weichet dasjenige, welches die Menschen \RWbet{gewöhnlich} beobachten, nur allzu häufig ab.
\begin{aufza}
\item Bei Weitem die \RWbet{meisten} Menschen stellen ihr ganzes Leben hindurch gar keine eigentliche Prüfung der mancherlei Religionen, die es auf Erden gibt, an; sondern sie bleiben, ohne erst einen Beweis für ihre Wahrheit zu verlangen, nur bei derjenigen Religion stehen, in der sie, wie man sagt, \RWbet{geboren}, \dh\ von Kindheit an unterrichtet worden sind. Ich nenne dieses Verhalten die \RWbet{fehlerhafte Anhänglichkeit an die Jugendreligion}.
\item Ein sehr beträchtlicher Theil dieser Menschen nimmt nicht nur an, was man in seinem Vaterlande \RWbet{allgemein} glaubt; sondern ergreift auch noch eine Menge von \RWbet{Zusätzen} auf, die ihm von noch so unglaubwürdigem Munde vorgetragen werden. Ich nenne diesen Fehler die \RWbet{religiöse Leichtgläubigkeit oder den Aberglauben}.
\item Nicht selten findet man sogar, daß sich gewisse Menschen für Bekenner einer Religion ausgeben, ohne sich um ihre Lehren nur je bekümmert zu haben. Ich nenne dieß \RWbet{den blinden Glauben}.
\item Unter den \RWbet{Aufgeklärteren} gibt es, besonders in unseren Tagen, viele, welche der Meinung sind, es wäre gleichgültig, ob ein Mensch diesem oder jenem von den verschiedenen religiösen Lehrbegriffen, die es auf Erden gibt, anhange. Man nennt diese Meinung den \RWbet{Indifferentismus}.
\item Endlich gibt es auch ungemein viele Menschen, welche die mancherlei Religionen auf Erden gar nicht mit aufrichtigem Her\RWSeitenw{109}zen prüfen, sondern im Voraus schon den Entschluß gefaßt haben, bald diese, bald jene, bald wohl auch alle diese Religionen überhaupt zu bezweifeln und falsch zu finden. Mann nennt dieß \RWbet{Zweifelsucht} oder \RWbet{Ungläubigkeit}.
\end{aufza}\par
Jeden dieser Fehler werde ich nun im Einzelnen mit Wenigem betrachten, und seine Entstehung, Schädlichkeit und Strafwürdigkeit zeigen.

\RWpar{41}{Fehlerhafte Anhänglichkeit an die Jugendreligion}
\indent\textbf{A.} Die \RWbet{Quellen} dieses Fehlers sind:
\begin{aufza} 
\item Ein großer Theil der Menschen in allen Ländern lebt noch in einer solchen \RWbet{Unwissenheit}, daß er nicht einmal \RWbet{weiß}, es gebe noch \RWbet{andere} Vorstellungen von Gott, als es diejenigen sind, die er in seinem Lande antrifft; er ist auch viel zu \RWbet{ungeübt im Denken} und zu geistesschwach, als daß er im Stande wäre, sich eine neue, noch nie gehörte Vorstellung von Gott und andern religiösen Gegenständen selbst auszudenken. So bleibt er also beinahe nothgedrungen bei den Vorstellungen, die man ihm in der Jugend beigebracht hatte, stehen.
\item Mehrere haben zwar wohl gehört, daß es noch andere Religionen gebe, aber man hat ihnen die \RWbet{allerunvortheilhaftesten Vorstellungen} von denselben beigebracht, so, daß sie niemals nur auf den Gedanken, sie mit der ihrigen zu vertauschen, kommen konnten; besonders da man ihnen
\item jeden auch noch so geringen Zweifel an ihrer angeborenen Religion als eine Sünde darstellte; ohne zu unterscheiden, ob es die Sinnlichkeit oder die Liebe zur Wahrheit und Tugend sey, die diesen Zweifel eingibt.
\item Endlich ist nicht zu läugnen, daß bei vielen Tausenden auch eine gewisse \RWbet{Trägheit des Geistes} und der Mangel an einem hinlänglichen \RWbet{Eifer für ihre sittliche Vervollkommnung} die wahre Ursache ist, daß sie die Frage, ob die Religion, in der man sie von Kindheit an unterrichtet hat, wohl auch die wahre sey, nie bei sich aufkommen lassen.~\RWSeitenw{110}
\end{aufza}\par
\vabst\textbf{B.} Die \RWbet{Schädlichkeit} dieses Fehlers wird von selbst einleuchten, sobald wir unten bei Widerlegung des Indifferentismus dargethan haben werden, daß nicht alle Religionen die Tugend und Glückseligkeit der Menschen in einem gleichen Grade befördern.\par
\vabst\textbf{C.} Seine \RWbet{Strafwürdigkeit} ist nach Umständen sehr verschieden. Denn wo die Anhänglichkeit an die Jugendreligion aus einem der drei \RWbet{erst} angeführten Gründe herrührt, kann sie allerdings zu keinem Verbrechen angerechnet werden; wohl aber dort, wo sie aus \RWbet{Trägheit oder Gleichgültigkeit für die Tugend entspringt}. Je nachdem nun der Antheil, den diese beiden letzteren Gründe haben, bald größer, bald geringer ist, wird auch die Strafwürdigkeit bald größer, bald geringer seyn.

\RWpar{42}{Religiöse Leichtgläubigkeit oder Aberglaube}
Unter \RWbet{religiöser Leichtgläubigkeit} verstehe ich die Gewohnheit eines Menschen, alles dasjenige, was ihm in Betreff religiöser Gegenstände gesagt wird, es komme aus einem auch noch so unglaubwürdigen Munde, ungeprüft anzunehmen. Unter dem \RWbet{religiösen Aberglauben} aber verstehe ich die einzelnen irrigen, und als solche Irrthümer leicht zu erkennenden Meinungen, die man aus jener Leichtgläubigkeit annimmt.\par

\vabst\textbf{A.} \RWbet{Quellen dieses Fehlers.}
\begin{aufza}
\item Die nächste Quelle der \RWbet{religiösen Leichtgläubigkeit} und somit auch des \RWbet{Aberglaubens}, ist und bleibt immer ein \RWbet{allzuniedriger Grad der Verstandesbildung}. Denn wenn der Verstand eines Menschen nicht gehörig entwickelt ist, so sieht er das Ungereimte gewisser Vorstellungsarten nicht ein; wenn er nicht einige Uebung im Denken hat, so vermag er nicht, sich andere Vorstellungen von den Dingen, als jene, die man ihm eben beigebracht hat, zu bilden; wenn er ganz unbekannt ist mit den verschiedenen Arten, wie Irrthum und Betrug eintreten, und mit den mancherlei Triebfedern, die zu den Letztern verleiten: was ist begreif\RWSeitenw{111}licher, als daß er Alles, was man ihm nur erzählt, aufs Wort glaubt?
\item Hiezu gesellt sich dann noch die \RWbet{Liebe zum Wunderbaren}, die, sofern sie in nichts Anderem, als in der Neigung, wunderbare Ereignisse kennen zu lernen, bestehet, bei allen Menschen zu finden ist, und noch keinen Tadel verdient; denn sie gründet sich auf den Umstand, daß wunderbare Ereignisse unserer \RWbet{Einbildungskraft} sowohl, als auch unserem \RWbet{vernünftigen Nachdenken} viel Stoff gewähren, und uns überdieß manche Erweiterung unserer Einsichten versprechen. Allein bei ungebildeten und eben deßhalb leichtgläubigen Menschen artet diese Liebe zum Wunderbaren gewöhnlich dahin aus, daß sie das Wunderbare nicht nur gern \RWbet{hören}, sondern auch alsbald \RWbet{glauben}; ja daß sie unter mehren Darstellungen, welche man ihnen von dem Hergange eines und desselben Ereignisses gibt, meistens diejenige ergreifen, welche die wunderbarste ist. Daher glauben sie denn an Ahnungen, an Geistererscheinungen, an die geheimen Kräfte gewisser Sprüche und Zeichen, an die Möglichkeit, den Zufall zu berechnen, in die Zukunft zu sehen, \udgl\
\item Nicht selten mischen sich auch \RWbet{allerlei eigennützige Neigungen} in das Spiel. So ist es namentlich bei allem demjenigen Aberglauben, der unsere Sinnlichkeit, unsere Trägheit, unsere Ausschweifungen begünstiget, \udgl\
\item Endlich gibt es noch \RWbet{sehr viele äußere Veranlassungen und Nahrungsmittel des Aberglaubens}; dahin besonders schlechte Erziehungsanstalten, Erzählungen, Bücher, Bilder und ähnliche Mittel, durch welche der Aberglaube der Väter auf ihre Enkel fortgepflanzt wird, gerechnet werden müssen.
\end{aufza}\par
\vabst \textbf{B.} \RWbet{Schädlichkeit.}\par
Ob und in welchem Grade jede einzelne abergläubige Meinung schädlich sey, hängt von ihrer besonderen Beschaffenheit und von den übrigen Begriffen des Menschen ab. Es kann einzelne abergläubige Meinungen geben, die gleichgültig sind; diese werden denn unserer Erklärung nach gar~\RWSeitenw{112}\ nicht zur \RWbet{Religion} gehören. Es gibt selbst einige, die unter besonderen Umständen sogar ihr Gutes haben. So kann \zB\ die Meinung, daß derjenige, der einen falschen Eidschwur begeht, innerhalb \RWbet{Jahr} und \RWbet{Tag} sterben müsse, bei manchen Menschen dienen, sie um so eher von dem Begehen eines Meineides abzuhalten. Einen ähnlichen Nutzen kann auch die Meinung, daß jeder Selbstmörder ewig verdammt sey, bei solchen Menschen haben, welche zu wenig Kenntnisse von der Natur des Menschen besitzen, um zu begreifen, daß mancher Selbstmord auch aus einer ganz unverschuldeten Geisteszerrüttung hervorgehen könne; \udgl\ Inzwischen, da die Erfindung und Ausbreitung des Aberglaubens größtentheils das Werk menschlicher Leidenschaften ist: so pflegt er auch größtentheils aus \RWbet{solchen Meinungen} zu bestehen, welche der Tugend und somit auch der wahren Glückseligkeit des menschlichen Geschlechtes nur allzunachtheilig sind. Im Allgemeinen also kann man immer behaupten, daß der Aberglaube den Menschen schädlich sey, und um so mehr, da er überdieß auch die \RWbet{Ausbildung ihres Verstandes verhindert}. Denn wie er einerseits eine \RWbet{Wirkung} des Unverstandes ist, so ist er andererseits auch wieder \RWbet{Ursache} davon, daß die begonnene Verstandesbildung nicht weiter fortschreiten kann. Je größer nämlich die Menge der Albernheiten ist, die sich ein Mensch einmal in den Kopf gesetzt hat, und je mehre Genossen er in seinem Glauben findet: um desto hartnäckiger widersetzt er sich jeder bessern Einsicht, die man ihm beibringen will.\par
\vabst \textbf{C.} Die \RWbet{Strafwürdigkeit} dieses Fehlers hängt abermals von dem Grade der Freiheit, und von der größern oder geringern Bösartigkeit der Beweggründe ohngefähr eben so ab, wie bei der Anhänglichkeit an die Jugendreligion gesagt ward.

\RWpar{43}{Blinder Glaube}
Unter dem \RWbet{blinden Glauben} verstehe ich jene sehr sonderbare Anhänglichkeit an den \RWbet{bloßen Namen} einer Religion, dabei man es für genug hält, auf alle Lehren~\RWSeitenw{113}\ derselben im Voraus geschworen zu haben, ohne sich um die wirkliche Kenntniß derselben zu bekümmern.\par
\vabst \textbf{A.} \RWbet{Quellen}.\par
Der blinde Glaube hat seinen Grund entweder in einer ungemeinen \RWbet{Stumpfheit des Verstandes}, nach der man im Ernste glaubt, daß es bei der Anhänglichkeit an eine Religion auf nichts, als auf das \RWbet{bloße Bekenntniß zu ihrem Namen ankomme}, oder wohl gar in einer versteckten \RWbet{Gleichgültigkeit gegen die religiösen Wahrheiten selbst}.\par
\vabst \textbf{B.} \RWbet{Schädlichkeit}.\par
Offenbar ist es nicht das \RWbet{bloße Glauben, oder Behaupten, daß diese oder jene Religion die wahre sey}, was zu beseligen vermag; sondern das \RWbet{Glauben} und \RWbet{Befolgen} ihrer Lehrsätze allein kann für die Tugend und Glückseligkeit des Menschen zuträglich seyn.\par
\vabst \textbf{C.} \RWbet{Strafwürdigkeit}.\par
Wenn es auch möglich seyn sollte, sich von der Wahrheit einer Religion vernünftiger Weise zu überzeugen, ohne erst ihre einzelnen Lehren kennen gelernt zu haben: so wird es \RWbet{doch von dem Augenblicke an}, da man sich überzeugt hat, die heiligste Pflicht, nach einer möglichst vollständigen Kenntniß der Lehren dieser Religion zu streben. Wenn anders also der Blindgläubige nicht etwa einen so hohen Grad von Blödigkeit hat, daß er diese Pflicht nicht einsieht: so ist er gewiß äußerst strafwürdig zu nennen.

\RWpar{44}{Religiöser Indifferentismus}
Unter dem \RWbet{religiösen Indifferentismus} versteht man die Meinung, es wäre ganz gleichgültig, welcher von den verschiedenen auf Erden herrschenden Religionen Jemand zugethan sey.\par
\vabst \textbf{A.} \RWbet{Entstehungsart} dieser Meinung.
\begin{aufza} 
\item Als \RWbet{erste} und vorzüglichste Veranlassung zu dieser Meinung ist die theils wirkliche, theils auch nur scheinbare~\RWSeitenw{114}\ Erfahrung zu betrachten, daß die Bekenner der mancherlei auf Erden herrschenden Religionen im Durchschnitte beinahe insgesammt \RWbet{auf einer und eben derselben Stufe der Tugend und Glückseligkeit} stehen. Es gibt, meint man, eine verhältnißmäßig fast gleiche Anzahl tugendhafter und durch Tugend auch glücklicher Menschen unter den Heiden, den Türken, den Juden, wie unter uns Christen; und die katholischen Christen scheinen eben nicht tugendhafter und durch die Tugend glücklicher als die protestantischen Christen. Hieraus nun glaubt man gleichsam wie aus der \RWbet{Erfahrung selbst} beweisen zu können, daß alle Religionen, die es auf Erden gibt, für die Zwecke der Tugend und Glückseligkeit gleichgültig wären.
\begin{RWanm} 
Aber das ist nicht richtig geschlossen; denn wäre es auch, daß sich keine Religion auf Erden fände, deren Bekenner sich auf eine ganz entschiedene Weise durch Tugend und Glückseligkeit vor den Bekennern aller übrigen auszeichneten: doch würde immer nicht folgen, daß unter diesen Religionen nicht Eine sey, welche die übrigen an sittlicher Zuträglichkeit übertrifft. Religion nämlich ist nicht das Einzige, was auf die Tugend und Glückseligkeit der Menschen Einfluß hat; auch die Verfassung eines Staates, die Beschaffenheit seiner Erziehungsanstalten, das Klima des Landes, die Schicksale, welche ein Volk erlebt hat, vor allem Andern aber eine Menge abergläubiger Meinungen, die wir dem Inhalte der Religion eines Volkes nicht beizählen dürfen, wenn sie, obgleich von Vielen, doch nicht von Allen angenommen werden, diese und andere dergleichen Umstände können einen sehr schädlichen Einfluß auf die Tugend und Glückseligkeit eines Menschen haben, und daran Ursache seyn, daß er auch bei dem Besitze einer Religion, die an sich selbst betrachtet der Tugend und Glückseligkeit der Menschen viel zuträglicher, als manche andere wäre, dennoch vor Andern sich nur wenig auszeichnet. Ueberdieß ist es nicht einmal wahr, daß die Bekenner aller Religionen auf Erden im Durchschnitte auf einerlei Stufe der Tugend und Glückseligkeit stehen. Es gibt zwar unter allen Religionsverwandten einzelne \RWbet{tugendhafte} und wohl auch \RWbet{glückliche} Menschen; nur ist die \RWbet{Anzahl} dieser Tugendhaften und der \RWbet{Grad} ihrer Vollkommenheit nicht gleich. Ich werde später zeigen, daß die Bekenner der christlichen, und~\RWSeitenw{115}\ namentlich der katholischen Religion in dieser Hinsicht wirklich vor allen Andern sich auszeichnen.
\end{RWanm}
\item Eine \RWbet{zweite} Veranlassung zur Entstehung des Indifferentismus haben gewisse an sich sehr \RWbet{richtige Bemerkungen} gegeben, mit welchen dieser Irrthum eine entfernte Aehnlichkeit hat, die man für völlige Gleichheit gehalten. Diese Bemerkungen sind:
\begin{aufzb}
\item Daß ein bloßer \RWbet{Formularglaube} (\dh\ eine bloße Anhänglichkeit an Formeln, die man nicht einmal versteht) keinen Nutzen habe, und daß es mithin gleichgültig sey, ob man an diesen ober jenen Formeln hänge. Da man nun fand, daß bei den meisten Menschen bloßer Formularglaube herrsche; so schloß man übereilt, es wäre gleichgültig, welcher Religion Jemand zugethan sey.
\item Man bemerkte ferner, daß sich die Menschen oft über gewisse Meinungen, die uns nach unseren jetzigen Begriffen sehr \RWbet{gleichgültig} seyn können, und also nicht mehr zur Religion gehören, mit vieler Heftigkeit stritten, weil sie sich vorstellten, daß es wichtige Glaubensartikel wären. So wurde \zB\ vom 15ten bis zu 17ten Jahrhunderte hin die Behauptung, daß die Erde still stehe, als ein zur Religion gehöriger Glaubensartikel angesehen, weil man vermeinte, daß mit Verwerfung dieses Satzes zugleich das Ansehen der heil.\ Schrift und die Göttlichkeit der christlichen Religion, also eine Menge höchst wichtiger Lehren und Meinungen umgestoßen würden. Aus solchen Beispielen nun folgerte man, daß alle religiösen Meinungen von der Art wären, und schloß dies um so lieber, weil man durch diese Behauptung allen Religionsverfolgungen mit einem Male ein Ende zu machen hoffte.
\item Man bemerkte richtig, daß bloßes \RWbet{Wissen ohne Thun} nichts nütze, und daß man auch bei den besten Kenntnissen ein schlechter Mensch bleiben könne; ingleichen, daß die meisten Menschen es nur beim Wissen der religiösen Wahrheiten bewenden lassen, und nicht zur~\RWSeitenw{116}\ wirklichen Ausübung übergehen. Hieraus nun folgerte man, daß alle religiösen Kenntnisse im Grunde überflüßig wären.
\item Man erkannte, daß die Anhänglichkeit an einen falschen Glauben und die Verwerfung des wahren, wenn sie aus unvermeidlichem Irrthume herrühren, \RWbet{Niemand zur Sünde angerechnet werden dürfen}; und dieses verwechselte man mit der Behauptung, daß es ganz gleichgültig sey, ob Jemand zu diesem oder jenem Glauben sich wende.
\end{aufzb}
\begin{RWanm}
Bei einigem Nachdenken leuchtet das Irrige all dieser Schlüsse ein. Denn
\begin{aufzb} 
\item \RWbet{Formularglaube} ist keine eigentliche \RWbet{Religionskenntniß}; daraus, daß Formularglaube gleichgültig ist, folgt also gar nicht, daß auch der Glaube an verschiedene Religionen gleichgültig sey.
\item Meinungen, die heut zu Tage schon keinen Einfluß mehr auf Tugend und Glückseligkeit haben, konnten ihn doch in früheren Zeiten haben; daß man sie aber mit Hitze verfocht, und jene, die sie nicht anerkennen wollten, verfolgte, war freilich gefehlt: doch diesem Uebel der \RWbet{Intoleranz} kann gesteuert werden, ohne dem Indifferentismus das Wort zu reden. Sehr übereilt aber ist es, daraus, weil \RWbet{einige} Meinungen, über die man sich einst mit vieler Heftigkeit stritt, jetzt nicht mehr wichtig sind, zu schließen, daß dieß von \RWbet{allen} Lehren gelte, die man zur Religion beziehet.
\item Richtige \RWbet{Kenntniß} hat rechtschaffenes \RWbet{Thun} freilich nicht eben zu einer ihrer \RWbet{nothwendigen} Folgen; aber die erstere ist doch die unentbehrlichste \RWbet{Bedingung} zu dem letzteren.
\item Wenn gleich die Verwerfung der wahren Religion, wo sie aus unvermeidlichem Irrthume herrührt, nicht \RWbet{an sich selbst} zur \RWbet{Schuld} gerechnet werden kann: so hat sie doch für unsere Tugend sowohl als auch für unsere Glückseligkeit sehr \RWbet{nachtheilige Folgen}; ent\RWSeitenw{117}stehet sie aus Gleichgültigkeit, so ist sie auch schon an sich betrachtet \RWbet{sträflich}.
\end{aufzb}
\end{RWanm}
\item Die Vertheidiger des katholischen und so mancher anderer Religionsbegriffe bemühten sich häufig nur, die \RWbet{Wahrheit} ihrer Lehrsätze zu zeigen, den \RWbet{Nutzen} aber, den ihre gläubige Annahme hat, übergingen sie meistens mit Stillschweigen. Das mußte natürlich Einige auf den Gedanken leiten, daß diese Lehrsätze wohl gar \RWbet{keinen} Nutzen hätten; hierauf verfielen sie um so eher, da es zuweilen wirklich einiges Nachdenken kostet, diesen Nutzen zu einem ganz \RWbet{deutlichen Bewußtseyn} zu erheben, obwohl er sich unserm \RWbet{Gefühle} fast immer gleich auf der Stelle kund gibt.
\item Und so wenig die Vertheidiger der Religion bemühet waren, den natürlichen Nutzen ihrer Lehrsätze zu zeigen, so laut behaupteten sie doch ihre \RWbet{unumgängliche Nothwendigkeit} für Jeden, der nicht ewig verdammt werden wolle. Diese so übertriebene Behauptung reizte den \RWbet{Geist des Widerspruches}, sich die entgegengesetzte Behauptung zu erlauben, daß jene Lehrsätze, weit entfernt, nothwendig zu seyn, nicht einmal den geringsten Nutzen hätten.
\item Endlich ist nicht zu vergessen, daß auch bei diesem Irrthume, wie bei allen sittlichen, das \RWbet{Herz} einen sehr großen Antheil habe. Man hatte sich nicht aus vorhergegangener Ueberzeugung, sondern aus Menschenfurcht, Bequemlichkeit oder anderen sinnlichen Beweggründen entschlossen, sich zu einer gewissen Religion zu bekennen, oder sie zu verwerfen. Man wünschte nun, sich über dieses Verfahren bei sich selbst rechtfertigen zu können, und darum überredete man sich von der Behauptung des Indifferentismus. Denn wenn es gleichgültig wäre, welcher Religion man zugethan ist, dann könnte es freilich kein so großes Vergehen seyn, aus bloßer Bequemlichkeit, Menschenfurcht \udgl\ sich bald zu dieser, bald jener Religion zu bekennen.
\end{aufza}\par
\vabst \textbf{B.} \RWbet{Irrigkeit und Schädlichkeit des Indifferentismus.}~\RWSeitenw{118}\par
Wir müssen erst das \RWbet{Irrige} des Indifferentismus erkennen, dann wird uns bald auch seine \RWbet{Schädlichkeit} einleuchtend werden.
\begin{aufza}\par
\item \RWbet{Irrigkeit.}\\
Es ist nicht gleichgültig, weder für unsere \RWbet{Tugend} noch für unsere \RWbet{Glückseligkeit}, ob wir dieser oder jener auf Erden herrschenden Religion zugethan leben.
\begin{aufzb}
\item \RWbet{Nicht gleichgültig für unsere Tugend}; denn verschiedene dieser Religionen enthalten eine bald mehr bald minder \RWbet{richtige}, bald mehr bald minder \RWbet{vollständige Sittenlehre}, bieten uns bald mehr bald weniger \RWbet{ausgiebige Mittel} und \RWbet{Beweggründe} zur \RWbet{wirklichen Erfüllung} unserer Pflichten an. Sie müssen also auch unser Wachsthum in der Tugend auf eine ungleiche Weise befördern, einige uns sogar hinderlich seyn.
\item \RWbet{Nicht gleichgültig für unsere Glückseligkeit}. \RWbet{Einmal} schon wegen des ungleichen Einflusses auf unsere Tugend, indem mit dieser auch unsere Glückseligkeit steigt oder fällt; dann aber auch \RWbet{unmittelbar}, denn verschiedene Religionen geben uns bald mehr bald weniger richtige Ansichten von dem \RWbet{wahren Wesen der irdischen Glückseligkeit} und von den Mitteln und Bedingnissen derselben, bald mehr bald minder beruhigende Aufschlüsse über so manche \RWbet{Räthsel in der Welteinrichtung}, bald mehr bald minder erfreuliche \RWbet{Aussichten in die Zukunft}, einige täuschen mit leeren Hoffnungen, andere beängstigen mit eben so grundlosen Besorgnissen \usw 
\end{aufzb}
\item \RWbet{Schädlichkeit}.\\
Wer dem Irrthume des Indifferentismus huldigt, der wird sich keine Mühe geben, aus den verschiedenen Religionen auf Erden diejenige, die für seine Tugend und Glückseligkeit am allerzuträglichsten wäre, herauszufinden; sondern bei jener bleiben, zu der ihn andere Rücksichten, etwa Bequemlichkeit, Menschenfurcht \udgl\ bestimmen. Das kann nun zufälliger Weise eine für ihn sehr nachtheilige Religion seyn.~\RWSeitenw{119}
\end{aufza}


\RWpar{45}{Zweifelsucht und Unglaube}
\RWbet{Zweifel} und \RWbet{Zweifelsucht}, und eben so \RWbet{Nichtglaube} und \RWbet{Ungläubigkeit} sind wohl zu unterscheiden.
\begin{aufza}
\item \RWbet{Zweifel} und \RWbet{Nichtglaube} bezeichnen ein Verhältniß unseres \RWbet{Verstandes} (oder Erkenntnißvermögens) zu einer Religion; \RWbet{wir sind im Zustande des Zweifels gegen sie}, wenn wir an ihrer Wahrheit \RWbet{zweifeln}, und in \RWbet{dem Zustande des völligen Nichtglaubens} gegen sie, wenn wir ihre Wahrheit mit Entschiedenheit \RWbet{verneinen}. Wer nun fast alle auf Erden herrschende Religionen, und selbst die Lehre von Gottes Daseyn bezweifelt, der heißt uns vorzugsweise ein \RWbet{Zweifler}; und wer diese Religionen, insonderheit auch den Glauben an Gott geradezu verwirft, heißt ein \RWbet{Nichtgläubiger}. Das Eine sowohl als das Andere könnte man vielleicht ganz unschuldiger Weise werden; dann nämlich, wenn man das Seinige gethan hätte, sich von der Wahrheit zu überzeugen; aus einem unüberwindlichen Irrthum aber nicht überzeugt worden wäre.
\item \RWbet{Zweifelsucht} und \RWbet{Ungläubigkeit} dagegen bezeichnen ein Verhältniß unseres \RWbet{Willens} gegen die Religion. \RWbet{Zweifelsucht} nämlich ist ein nicht aus Liebe zur Tugend entsprungenes \RWbet{Bestreben}, Zweifel und Einwürfe gegen die Wahrheit einer bestimmten, oder auch aller auf Erden herrschenden Religionen bei sich hervorzubringen und zu unterhalten. Und eben so ist \RWbet{Ungläubigkeit} ein nicht aus Liebe zur Tugend herrührendes \RWbet{Bestreben}, entweder nur eine \RWbet{gewisse} oder auch \RWbet{alle} Religionen auf Erden geradezu \RWbet{falsch zu finden}. Es gibt also eine \RWbet{besondere Zweifelsucht und Ungläubigkeit}, welche nur gegen eine \RWbet{bestimmte} Religion, und eine \RWbet{allgemeine}, die gegen alle auf Erden herrschende Religionen (namentlich solche, welche den Glauben an einen Gott lehren) gerichtet ist.
\end{aufza}
\begin{RWanm}
Offenbar ist der Unterschied zwischen \RWbet{Zweifelsucht} und \RWbet{Ungläubigkeit} nur gering; denn es ist eben kein großer Unterschied, ob man bestrebt ist, eine gewisse Behauptung nur zu~\RWSeitenw{120}\ \RWbet{bezweifeln}, oder geradezu zu \RWbet{läugnen}; von dem Einen geht man bald zu dem Anderen über. Daher können wir von diesen beiden Fehlern hier in Vereinigung reden.
\end{RWanm}

\RWpar{46}{A.~Quellen der Zweifelsucht und Ungläubigkeit.}
\begin{aufza} 
\item Die \RWbet{ausgebreitetste} Quelle der Zweifelsucht und Ungläubigkeit ist ein \RWbet{lasterhafter Wille}. Das Bewußtseyn nämlich, gewisse Handlungen begangen zu haben, die in einer bestimmten, oder auch überhaupt in allen auf Erden herrschenden Religionen für unerlaubt und strafwürdig erklärt werden, besonders aber der \RWbet{Wille}, solche Handlungen auch noch in Zukunft auszuüben, erzeugt unmittelbar den Wunsch, die Religionen, welche ein solches Benehmen verbieten, für falsch halten zu können, um auf diese Art
\begin{aufzb}
\item \RWbet{minder strafwürdig zu seyn}. Denn wer jene Religionen, oder wenigstens die Vorschriften, in denen sie dergleichen Handlungsweisen verbieten, für \RWbet{falsch} halten würde, der würde eben deßhalb durch ihre Uebertretung nicht sehr verantwortlich werden.
\item \RWbet{um ruhiger leben zu können}. Denn wer an die Strafen, mit denen die meisten Religionen die Uebertreter ihrer Vorschriften theils schon in diesem, theils erst im künftigen Leben bedrohen, nicht glaubt, der entgeht zwar darum noch nicht diesen Strafen, aber er lebt doch so lange, als sie nicht wirklich über ihn einbrechen, ruhiger fort; während ein Anderer schon durch die bloße \RWbet{Vorstellung} und Erwartung dieser Strafen unaufhörlich gepeiniget wird. Aus diesem doppelten Grunde wünscht sich jeder Lasterhafte, wo möglich, zu überreden, daß alle Religionen, die seine Lebensweise verdammen, falsch wären. Er gibt sich also auch jede erdenkliche Mühe, diese Ueberredung bei sich hervorzubringen, und es gelingt ihm dieß meistens nur allzubald; indem die wenigsten religiösen Wahrheiten von einer solchen Art sind, daß sie sich unserm Gefühle unwiderstehlich aufdringen würden.
\end{aufzb}
\begin{RWanm}
Auf diese Weise kann jede \RWbet{einzelne lasterhafte Neigung} oder \RWbet{Gewohnheit} zur Zweifelsucht und Ungläubig\RWSeitenw{121}keit verleiten, besonders wenn sie bereits so tief gewurzelt ist, daß man ihrer fast nicht wieder los werden kann, \zB\ Wollust, Herrschsucht, Geiz.
\end{RWanm}
\item Eine \RWbet{zweite}, gleichfalls sehr ausgebreitete Quelle der Zweifelsucht und Ungläubigkeit ist die \RWbet{Eitelkeit}, obwohl nicht darum, weil sie an und für sich betrachtet schon zu den lasterhaften Neigungen gehört; denn den Fehler der Eitelkeit pflegen sich die Menschen nicht leicht so hoch anzurechnen, daß sie um seinetwegen die Strafen der Hölle befürchten, und somit ihren Glauben lieber ganz abschwören sollten; sondern die Eitelkeit pflegt Zweifelsucht und Ungläubigkeit auf ihre \RWbet{eigene Weise} zu erzeugen; wobei sie übrigens bald als \RWbet{Unterscheidungssucht}, bald als \RWbet{Nachahmungssucht}, bald endlich als \RWbet{Rechthaberei} sich äußert.
\begin{aufzb} 
\item Als \RWbet{Unterscheidungssucht}.\\
Aus Eitelkeit möchte man gern etwas Vorzüglicheres als andere Menschen scheinen. Dazu ist vor Allem nothwendig, daß man sich von Andern \RWbet{unterscheide}; denn nur derjenige, der sich von Andern unterscheidet, zieht die Aufmerksamkeit der Menschen auf sich, und erregt die Vermuthung, daß er vielleicht etwas Vorzüglicheres als alle Uebrigen sey. So geht aus Eitelkeit Unterscheidungssucht hervor. Nun gibt es zwar mancherlei Dinge, durch die sich der Unterscheidungssüchtige von Andern zu unterscheiden vermögte; aber nicht alle stehen so leicht zu Gebote, oder sind so wirksam, als die \RWbet{Unterscheidung in der Religion}. Das beste Mittel, sich nicht nur zu unterscheiden, sondern auch ehrwürdig zu machen, wäre wohl freilich \RWbet{ein hoher Grad von Sittlichkeit und Tugend}. Aber dieß Mittel fällt den Menschen insgemein viel zu beschwerlich. Ein anderes gleichfalls sehr ausgiebiges Mittel wäre die \RWbet{Ausführung großer gemeinnütziger Thaten}; aber dazu gehört mehr \RWbet{Kraft und Ausdauer}, als die meisten Menschen haben, oder doch anwenden wollen. Ein sehr beliebtes Mittel, sich vor Andern auszuzeichnen, ist die \RWbet{Verschwendung}; aber nur Wenige können Gebrauch von diesem Mittel machen. \RWbet{Abweichungen in Kleidung, Sprache}, \udgl\ 
%%!!! folgt ein größerer Zusatz in A, der in dem GA-Digitalisat fehlt !!!%%
erregen wohl Aufsehen; werden aber, statt Bewunderung zu finden,~\RWSeitenw{122} gewöhnlich nur verlacht. Abweichende Meinungen in \RWbet{Kunst} und \RWbet{Wissenschaft} erregen allerdings viele Aufmerksamkeit doch nur bei derjenigen Classe von Menschen, die sich mit eben dieser Kunst oder Wissenschaft befassen; die Uebrigen kümmern sich wenig darum. Ausgiebiger also als vieles Andere sind \RWbet{abweichende Meinungen im Fache der Religion, Verwerfung dieser oder jener, ja wohl gar aller auf Erden herrschenden Religionen}. Aufsehen wird auf diese Art um so gewisser erregt, je wichtiger es für \RWbet{alle Menschen} ist, zu hören, daß Jemand im Fache der Religion gewisse, von den gewöhnlichen abweichende Meinungen vortrage. Die große Menschenmenge ermangelt dann insgemein nicht, den \RWbet{Scharfsinn} des Mannes, der dort Irrthümer gesehen haben will, wo bisher Niemand einige fand, die \RWbet{Stärke des Geistes}, mit der er sich von allen Jugendvorurtheilen so gänzlich loszureißen vermochte, endlich den \RWbet{Muth} zu bewundern, mit dem er die ihm gewordene Ueberzeugung so öffentlich auszusprechen waget. Also aus bloßer Sucht, sich zu unterscheiden, entschließen sich Manche, Zweifler und Nichtgläubige zu scheinen; aber sie könnten es nicht einmal recht \RWbet{scheinen}, wenn sie es nicht auch im Herzen \RWbet{wären}, so trachten sie dann sich selbst zu überreden, daß die Religionen, gegen welche sie sich erklären, keine Wahrheit hätten.
%%!!! Ende des größeren Zusatzes aus A!!!%%
\item Als \RWbet{(eitle) Nachahmungssucht}.\\
Eben dieselbe Eitelkeit, welche so viele Menschen zur Unterscheidungssucht verleitet, verführt Andere wieder zur \RWbet{Nachahmungssucht}. Wenn nämlich irgend ein Mann sich erst viel \RWbet{Ruhm} und \RWbet{Ansehen} in der menschlichen Gesellschaft erworben hat, so gibt es gleich eine Menge eitler Thoren, welche durch Nachahmung dessen, was sich am Leichtesten an ihm nachahmen läßt, \zB\ durch Nachahmung seiner Kleidung, Sprache, geäußerter Meinungen \udgl , eine gewisse Aehnlichkeit mit ihm zu erreichen trachten, um so an ihrem eigenen Ansehen zu gewinnen. Sie hoffen nämlich, daß man nach dem bekannten Gesetze der Ideenverknüpfung die Vorstellungen von Größe, Wichtigkeit, Macht \usw\ von jenem Manne auch auf sie übertragen werde. Geschieht es also, daß~\RWSeitenw{123}\ sich ein angesehener Mann einer, oder vielleicht auch allen auf Erden herrschenden Religionen abhold bezeugt: dann geben sich gleich eine Menge seiner Nachahmer die Miene, als ob auch sie diese Religionen für unrichtig hielten; sie suchen Einwürfe auf, und überreden sich dergleichen wirklich gefunden zu haben.
\begin{RWanm}
Von dieser aus Eitelkeit entsprungenen Nachahmungssucht, welche ich eben deßhalb die \RWbet{eitle} genannt, ist die \RWbet{schwachsinnige} zu unterscheiden. Diese findet Statt, wo Menschen aus bloßer Schwachheit des Geistes Alles, was irgend ein großer gelehrter Mann gesagt hat, gleich für unfehlbar halten. Auch sie kann \RWbet{Zweifel} und \RWbet{Nichtglauben} -- aber wenigstens nicht unmittelbar Zweifelsucht und Ungläubigkeit bewirken.
\end{RWanm}
\item Als \RWbet{Rechthaberei}.\\
Gar viele Menschen gibt es, die, wenn sie geirrt haben, und den begangenen Irrthum schon ahnen, aus bloßer Eitelkeit sich schämen, ihn offen einzugestehen, und um dieß nicht zu müssen, um ihre Behauptung noch ferner beibehalten zu können, mit aller Anstrengung sich von ihrer Richtigkeit zu überreden suchen. Aus diesem Fehler der Rechthaberei ist besonders bei \RWbet{Gelehrten} schon öfters Zweifelsucht und Ungläubigkeit hervorgegangen. Wenn nämlich ein solcher dem Fehler der Rechthaberei ergebene Gelehrte irgendwo zufällig sich einen Irrthum beikommen ließ, und wenn nun Andere, die seinen Irrthum bemerkten, ihn aus gewissen Lehrsätzen der Religion zu widerlegen versuchen: so wird er, um nur nicht nachgeben zu müssen, auch diese religiösen Lehrsätze, so viel nur möglich ist, bestreiten; und um dieß zu vermögen, sich selbst von ihrer Falschheit zu überreden suchen.
\end{aufzb}
\item Noch eine eigene Quelle der Zweifelsucht und Ungläubigkeit, die sich besonders bei \RWbet{Schriftstellern} zuweilen äußert, ist die \RWbet{Gewinnsucht}. In unsern Tagen gibt es, leider! eine beträchtliche Anzahl von Schriftstellern, die mit der Schriftstellerei eine Art von Gewerbe treiben. Sie schreiben, um \RWbet{bezahlt} zu werden. Da aber Bücher, welche gewisse von der gewöhnlichen abweichende religiöse Meinungen vortragen, Bücher, welche \zB\ die herrschende Religion~\RWSeitenw{124}\ bestreiten, oder die Tugend verspotten, \udgl\ gerade am Theuersten bezahlt zu werden pflegen: so entschließen sich Manche aus bloßer Gewinnsucht, Prediger des Unglaubens zu werden, und, wie natürlich, trachten sie dann sich selbst zuerst zu überreden, daß jene Grundsätze, die sie bestreiten, ein bloßes Vorurtheil wären.
\item Endlich kann Alles, was immer \RWbet{Zweifel} erregt, auch eine Veranlassung zu \RWbet{Zweifelsucht} und \RWbet{Ungläubigkeit} werden. Wenn nämlich erst durch was immer für Umstände Zweifel gegen gewisse Lehren der Religion in unserm Gemüthe erregt worden sind; so kann das peinliche Gefühl, in welches der Zustand des Zweifels allemal versetzt, den Entschluß veranlassen, dasjenige, was uns schon einmal als zweifelhaft erscheint, lieber \RWbet{ganz} zu verwerfen, um so der Ungewißheit los zu werden; besonders, wenn wir bemerken, wie die Ablegung der Religion auch unsere sinnlichen Neigungen begünstigen werde. Auf diese Art können \RWbet{irreligiöse Bücher, Umgang mit ungläubigen Menschen}, ja auch selbst \RWbet{seichte Vertheidigungsgründe und Beweise der Religion}, erst zu \RWbet{Zweifeln} und zum \RWbet{Nichtglauben}, zuletzt zur Zweifelsucht und Ungläubigkeit verleiten.
\end{aufza}

\RWpar{47}{B.~Schädlichkeit der Zweifelsucht und Ungläubigkeit}
Das Bestreben des Zweifelsüchtigen oder Ungläubigen, Zweifel und Einwürfe gegen die Wahrheit einer bestimmten, oder auch aller auf Erden herrschenden Religionen in sich hervorzubringen, pflegt nicht ohne Wirkung zu bleiben; sondern indem er die Aufmerksamkeit seines Geistes absichtlich von allen Gründen, die \RWbet{für} diese Religionen sprechen, abzieht, sich dagegen mit lauter scheinbaren Einwürfen \RWbet{wider} sie beschäftiget, gelingt es ihm meistens nur allzubald, daß ihm als zweifelhaft oder auch durchaus falsch erscheint, was er so finden will. Ist nun die Religion, von der er auf eine solche Weise abfällt, von der Art, daß seine Tugend und Glückseligkeit durch sie befördert worden wäre: so liegt der Schade, den er sich selbst zuziehet, am Tage. Da aber fast bei jeder Zweifelsucht und Ungläubigkeit die vornehmste Triebfeder \RWbet{Leidenschaft} ist;~\RWSeitenw{125}\ so sind die religiösen Lehren, welche der Zweifelsüchtige verwirft, insgemein solche, die seiner Sinnlichkeit einen wohlthätigen Zaum angelegt hätten, und also recht heilsam für ihn gewesen wären.

\RWpar{48}{C.~Sträflichkeit der Zweifelsucht und Ungläubigkeit}
\RWbet{Zweifel} und \RWbet{Nichtglaube} können zuweilen unverschuldet und unsträflich seyn; \RWbet{Zweifelsucht} und \RWbet{Ungläubigkeit} dagegen sind immer strafwürdig.
\begin{aufza}
\item Wer sich vornimmt, eine gewisse \RWbet{Religion}, unangesehen, ob sie wahr oder falsch, zuträglich oder nicht zuträglich für ihn sey, zu bezweifeln und falsch zu finden, der setzt sich eben hiedurch in die Gefahr, eine Religion zu verwerfen, die vielleicht in der That überaus zuträglich für ihn gewesen wäre. Das ist nun offenbar sehr böse gehandelt, und strafwürdig in einem hohen Grade.
\item Wer eine Religion vollends aus der bestimmten Absicht bezweifelt und verwirft, weil sie seinen lasterhaften Lebenswandel verdammt, der handelt um so strafbarer; denn er beraubt sich ja selbst des Mittels zu seiner Besserung. (Das Christenthum sieht dieß als eine der Sünden gegen den heil.\ Geist an, von denen es sagt, sie würden selten oder nie vergeben.)
\item Nicht minder strafwürdig ist die Zweifelsucht, wenn sie aus \RWbet{niedriger Gewinnsucht} hervorgeht.
\item Weniger schändlich, aber doch immer sehr ahndungswerth ist es, wenn Jemand aus bloßer \RWbet{Eitelkeit} (es sey schon aus Unterscheidungs- oder Nachahmungssucht oder Rechthaberei) eine Religion bezweifelt und bestreitet. Wie gering ist doch nicht der Vortheil, den er sich verschafft, verglichen mit der Zerstörung, die er hiedurch in seinem sittlichen Charakter sowohl als auch in seiner innern, von diesem abhängigen Glückseligkeit anrichtet! Oder kann der je ruhig werden, dem sein Gewissen vorwirft, daß er die Wahrheit nicht mit lauterem Sinne gesucht, sondern sich vorgenommen habe, was er wahr oder falsch finden wolle? --
\end{aufza}

\RWpar{49}{Hauptpflicht des Menschen in Hinsicht auf die religiösen Ueberzeugungen seiner Nebenmenschen}
Wir haben ein Jeder nicht nur auf unseren eigenen Glauben, sondern auch auf die religiösen Ueberzeugungen~\RWSeitenw{126}\ \RWbet{anderer Menschen} einen mehr oder weniger bedeutenden Einfluß. Wie wir uns nun in Hinsicht auf unsere eigenen Ueberzeugungen zu verhalten oder nicht zu verhalten haben, das ist bisher besprochen worden; es erübriget noch, daß ich mit Wenigem auch \RWbet{das pflichtmäßige Betragen} bestimme, das \RWbet{wir in Hinsicht auf die religiösen Ueberzeugungen unserer Mitmenschen} zu beobachten haben.

Wie wir in Hinsicht auf unsere eigene Ueberzeugung die Pflicht haben, nach der Erkenntniß der für uns vollkommensten Religion zu streben: so haben wir in Hinsicht auf den Glauben unserer Mitmenschen die Pflicht, auch bei ihnen \RWbet{die Anerkennung der für sie vollkommensten Religion, so viel es in unsern Kräften stehet, zu befördern}. Denn wir sind nicht nur alles dasjenige zu thun verbunden, was unsere eigene Tugend und Glückseligkeit erhöhet, sondern so viel wir können, auch Alles, was die Tugend und Glückseligkeit unserer Mitmenschen befördert. Dazu gehört nun, daß wir sie, wie möglich, einen Jeden zur Erkenntniß derjenigen Religion zu bringen suchen, welche für sie die \RWbet{vollkommenste} ist.

\RWpar{50}{Umständlichere Auseinandersetzung der einzelnen in dieser Hauptpflicht enthaltenen Verbindlichkeiten}
\begin{aufza} 
\item Es ist uns nicht nur erlaubt, sondern selbst unsere Pflicht, religiöse Ansichten, die wir für zuträglich halten, \RWbet{Kindern} auch selbst in einem solchen Alter, da sie noch keine Fähigkeit haben, sie ihren Gründen nach zu prüfen, beizubringen.
\begin{aufzb}
\item Es gibt religiöse Ansichten, welche dem Kinde auch schon in einem Alter, da es sie noch nicht zu prüfen im Stande ist, ersprießlich werden können, wohl gar nothwendig sind. Es ist
\item auch \RWbet{möglich}, dergleichen Ansichten dem Kinde beizubringen, ohne daß es ihre Wahrheit noch zu prüfen vermag. Denn werden ihm diese Begriffe nur öfter \RWbet{vorgetragen}, und hört es nur nichts denselben Widersprechendes;~\RWSeitenw{127}\ so nimmt es sie an, obwohl es die eigentlichen Beweisgründe derselben noch nie gehört hat, auch nicht im Stande gewesen wäre, sie zu fassen.
\item Wenn wir dagegen mit der Mittheilung dieser Begriffe so lange warten wollten, bis der heranwachsende Mensch sie zugleich mit ihren Beweisen zu fassen im Stande ist; so würden wir ihn der Gefahr aussetzen, sie niemals anzunehmen. Denn wie leicht kann es geschehen, daß er in späteren Jahren keine Gelegenheit mehr findet, einen religiösen Unterricht zu empfangen, oder daß er aus Trägheit ihn nicht anhört, oder in Verbindungen geräth, in denen man ihm ganz andere verderbliche Begriffe beibringt, die er um so begieriger ergreift, je weniger bessere Grundsätze er bisher erhalten hatte, die er denselben entgegensetzen könnte?
\item Endlich ist auch gewiß, daß eine jede Ueberzeugung um desto fester haftet, je älter sie bereits geworden ist. Begriffe und Grundsätze also, die man uns schon in frühester Kindheit beigebracht hat, lassen sich nicht so leicht durch andere verdrängen. Nun ist es aber nöthig, daß die religiösen Ueberzeugungen des Menschen den möglich höchsten Grad der Lebhaftigkeit und Festigkeit erhalten. Also müssen wir auch schon aus diesem Grunde den Religionsunterricht so frühzeitig als möglich anfangen.
\end{aufzb}
\end{aufza}

\RWpar{51}{Auflösung einiger Einwürfe}
\RWbet{1.~Einwurf.} Auf diese Art gewöhnt man das Kind zu einem \RWbet{grundlosen Glauben}; und hat man es daran gewöhnt, so wird es in Zukunft, wie unseren \RWbet{richtigen}, aber doch \RWbet{unerwiesenen} Behauptungen, so auch den ungereimtesten Meinungen, die es nur irgendwo hört, seinen Beifall schenken.\par
\RWbet{Antwort.} Wenn der Erzieher dem Kinde Beweise von seiner \RWbet{Einsicht} und \RWbet{Redlichkeit} gegeben hat: so ist der Glaube, womit dasselbe Alles, was es aus seinem Munde hört, auch ohne fernere Beweise annimmt, kein \RWbet{grundloser}~\RWSeitenw{128}\ \RWbet{Glaube}. Der Lehrer kann für seine Versicherungen Glauben fordern und finden, und darum doch das Kind gewöhnen, nicht den Aussprüchen eines \RWbet{Jeden} zu trauen.\par
\RWbet{2.~Einwurf.} Kinder vermögen die wenigsten Lehrsätze noch zu \RWbet{fassen}; will man sie gleichwohl darin unterrichten, so ist es ein bloßer \RWbet{Formularglaube}, den man auf diese Art erzeuget; ein \RWbet{Formularglaube}, der nicht nur keinen Nutzen gewährt, sondern in so weit noch \RWbet{schädlich} ist, als er den Verstand \RWbet{abstumpft}, und die Begierde nach einer \RWbet{gründlicheren Kenntniß} erstickt. Ein Mensch, der die Formeln der religiösen Wahrheiten von seiner Kindheit an auswendig weiß, glaubt schon Alles zu wissen, und kümmert sich daher um keinen weiteren Unterricht.\par
\RWbet{Antwort.} Man muß den \RWbet{Lehrsatz} und seinen \RWbet{Beweis} unterscheiden. So lange das Kind noch nicht im Stande ist, den Lehrsatz selbst zu fassen, muß man es freilich mit der Erlernung der Formel, in welcher dieser Lehrsatz ausgesprochen wird, verschonen. Kann es aber den Lehrsatz fassen, obgleich es noch nicht seinen Beweis verstehen könnte, dann bringe man ihm immerhin den ersteren bei, und es wird doch kein bloßer \RWbet{Formularglaube} entstehen. Dem Umstande aber, daß sich ein Kind, wenn es von den religiösen Lehren doch etwas kennen gelernt hat, einbilden sollte, schon Alles zu wissen, läßt sich leicht vorbeugen; wie denn dieser Grund, wenn er etwas bewiese, zu viel bewiese, weil aus ihm folgen würde, man thue am besten, die Menschen in jedem Fache ganz unwissend zu lassen.\par
\RWbet{3.~Einwurf.} Wenn man die Mittheilung gewisser religiöser Wahrheiten, \zB\ vom Daseyn Gottes, erst für das reifende Jünglingsalter aufsparen wollte; so würden sie dann durch Neuheit überraschen, und eben darum, weil sie jetzt \RWbet{ganz} begriffen werden könnten, auch einen viel tiefern Eindruck hervorbringen, als wenn der Mensch schon durch die Gewohnheit dagegen abgestumpft ist.\par
\RWbet{Antwort.} 
\begin{aufzb} 
\item Es würde uns \RWbet{einen sehr großen Zwang auflegen}, wenn wir die meisten religiösen Wahrheiten vor unserer Jugend, vom Kinde an, bis in~\RWSeitenw{129}\ das reifere Jünglingsalter hin geheim halten sollten. Wir dürften da nie von Gott und göttlichen Dingen vor ihnen reden, müßten alle unsere Andachtsübungen vor ihnen geheim halten, \usw\ Welche Beschwerlichkeit wäre das nicht!
\item Und eben weil dieser Zwang äußerst beschwerlich und daher auch ganz \RWbet{unnatürlich} wäre, würde der Jungling beim ersten Anfange des Unterrichtes, den wir ihm über Gott ertheilen wollten, \RWbet{zweifeln, ob wir auch Wahrheit reden}. Wenn dieses wahr ist, würde er sagen, warum habt ihr es mir so lange vorenthalten? Dieser Zweifel würde den tiefen Eindruck, den wir erwarten, gar sehr vermindern.
\item Des mannigfaltigen Nutzens, den die Bekanntschaft mit diesen Wahrheiten dem Kinde schon frühzeitig geleistet haben würde, hätten wir es beraubt.
\item Ueberzeugungen, die erst so spät gegründet wurden, könnten wohl nimmermehr zu jenem Grade der Festigkeit gedeihen, wie Grundsätze, welche das Kind, wenn man so sagen darf, schon mit der Muttermilch einsog.
\item Endlich ist auch nicht nothwendig, daß jene Wahrheiten der längern Bekanntschaft wegen dem Menschen \RWbet{gleichgültig} werden müßten. Man sehe nur darauf, daß man dem Kinde, wo möglich, gleich das \RWbet{erste Mal}, da man es mit einem erhabenen Gegenstande, \zB\ mit Gott, bekannt macht, einen recht würdigen Begriff beibringe, und wähle hiezu einen besonders \RWbet{günstigen} Augenblick aus; man sorge ferner dafür, daß der ertheilte Begriff in der Folge immer vervollkommnet werde: so wird man den Eintritt einer gewissen Gleichgültigkeit nie zu befürchten haben.
\end{aufzb}


\RWpar{52}{Fortsetzung von \RWparnr{50}}
\begin{aufza}\setcounter{enumi}{1}
\item Von jener Zeit an, da der Mensch zum Gebrauche seiner Vernunft kommt, müssen wir ihn besonders auf jenen Theil seiner Meinungen aufmerksam machen, die einen Ein\RWSeitenw{130}fluß auf seine Tugend oder Glückseligkeit haben, und ihm die \RWbet{Pflichten} beibringen, die er in Ansehung derselben hat, diejenigen nämlich, die oben (\RWparnr{37}\ bis \RWparnr{39}) beschrieben worden sind. Denn nur durch die Bekanntmachung mit diesen Pflichten setzen wir ihn in den Stand, sie zu beobachten, und tragen eben dadurch am meisten bei, daß er zur Anerkennung der für ihn vollkommensten Religion gelange.
\item Man schaffe ihm nun auch Gelegenheit, jene religiösen Ansichten, die für ihn die zuträglichsten sind, vollständig kennen zu lernen, und sich von ihrer Wahrheit zu überzeugen. Zu diesem Zwecke sind anzuwenden
\begin{aufzb}
\item \RWbet{gelegenheitliche Belehrungen}, welche bei jedem schicklichen Anlasse, dergleichen das gesellige Leben alltäglich darbietet, ertheilt werden sollen. So haben Mütter und Väter täglich Gelegenheit, ihren Kindern, ältere Geschwister ihren jüngeren, Erzieher ihren Zöglingen, alle Menschen endlich denjenigen, mit denen sie umgehen, religiöse Begriffe und Grundsätze beizubringen.
\item \RWbet{förmlicher Unterricht}, \dh\ solche Belehrungen, die man nicht bloß aus gelegenheitlicher Veranlassung, sondern zu \RWbet{festgesetzten Zeiten und Stunden} ertheilt. Von Seite des \RWbet{Staates} muß dafür Sorge getragen werden, daß dieser Unterricht, wie fern er durch öffentlich angestellte Personen theils in den Schulen, theils von den Kanzeln herab zu ertheilen ist, zweckmäßig eingerichtet werde.
\item \RWbet{religiöse Schriften}, \dh\ Schriften, in denen die Lehren der vollkommensten Religion auf eine recht faßliche und überzeugende Weise vorgetragen werden. Nach dem verschiedenen Grade der Fassungskraft und Bildung der Menschen, nach ihren verschiedenen Bedürfnissen sollte es von diesen Schriften gar mancherlei Arten geben.
\end{aufzb}
\item Es ist erlaubt, die Menschen selbst durch \RWbet{manche sinnliche Triebfedern}, nicht zwar zur \RWbet{Annahme} bestimmter religiöser Meinungen, wohl aber zur \RWbet{Aufmerksamkeit} auf sie und auf ihre Beweisgründe anzureizen. Der\RWSeitenw{131}gleichen Triebfedern wären ein reizendes Gewand, das man dem Vortrage der religiösen Lehren ertheilt, gewisse Belohnungen oder ehrenvolle Auszeichnungen, die man denjenigen angedeihen ließe, welche dem religiösen Unterrichte mit Fleiße beigewohnt, \udgl\ Durch ein solches Verfahren würde einerseits \RWbet{nichts Böses} -- nämlich noch keine Häuchelei, wohl aber andererseits das Gute bewirkt, daß die Menschen, indem sie den religiösen Lehren und ihren Beweisgründen mehr Aufmerksamkeit schenken, auch häufiger von ihrer Wahrheit überzeugt werden.
\end{aufza}\par
\RWbet{Einwurf.} Wer nur aus Sinnlichkeit aufmerkt, \zB\ weil ihm der Vortrag gefällt, hat kein Verdienst von dieser Aufmerksamkeit.\par
\RWbet{Antwort.} Wahr; aber wenn dieses Mittel nicht gebraucht worden, und er nicht aufmerksam gewesen wäre, so hätte er auch kein Verdienst gehabt. Nun da er aufmerksam ist, hat er zwar kein Verdienst von dieser Aufmerksamkeit, aber sie ist ihm doch nützlich.
\begin{aufza}\setcounter{enumi}{4}
\item Man sehe endlich darauf, daß die religiösen Wahrheiten, die man den Menschen beibringt, kein \RWbet{todtes Wissen} bei ihnen verbleiben, sondern den \RWbet{eigentlichen Zweck}, den alle religiösen Kenntnisse haben, wirklich \RWbet{erreichen}, \dh\ daß sie darnach empfinden und handeln lernen. Zu diesem Ende bemühe man sich,
\begin{aufzb}
\item die Ueberzeugung von diesen Wahrheiten zu einem möglichst hohen Grade der Lebhaftigkeit zu erheben. Dazu ist wieder nöthig, daß man den Unterricht nicht in bloß abgezogenen Begriffen ertheile, sondern die allgemeine Wahrheit durch einzelne Beispiele anschaulich mache. Man trachte ferner
\item jede einzelne religiöse Wahrheit mit vielen solchen Vorstellungen, welche den Menschen recht geläufig sind, in Verbindung zu bringen, damit sie, so oft sie der letzteren gedenken, nach dem Gesetze der Ideenverknüpfung, sich auch der ersteren erinnern. Auch dazu dient ein Unterricht durch concrete Beispiele viel besser, als ein abstracter. Man muß ihnen endlich~\RWSeitenw{132}
\item Gelegenheit geben, nach den erlangten Einsichten auch zu handeln. Hat man \zB\ einem Kinde die Pflicht der Wohlthätigkeit erklärt; so verschaffe man ihm bald eine bequeme Gelegenheit, diese Pflicht auszuüben, \udgl\,m.
\end{aufzb}
\item Aus allem diesem ist zu ersehen, daß sich der Unterricht in der vollkommensten Religion, besonders für Kinder und in öffentlichen Schulen, schwerlich auf eine zweckmäßigere Weise ertheilen ließe, als durch ein \RWbet{Buch}, in welchem die Lehren und Vorschriften dieser Religion \RWbet{an dem Leitfaden einer theils wirklichen, theils erdichteten Geschichte} so angereihet wären, daß ihre \RWbet{Vernunftmäßigkeit} und ihr \RWbet{Nutzen} aus jener Anwendung, in der sie hier vorkommen, schon von selbst anschaulich würde.
\begin{RWanm}
Schriften, in denen die Wahrheiten der natürlichen Religion auf eine solche Weise behandelt werden, sind bereits mehre vorhanden. \RWbet{Campe, Salzmann} \uA\ haben vortreffliche Arbeiten dieser Art geliefert. Hieher gehört vorzüglich der erste Band des \RWbet{moralischen Elementarbuches} von \RWbet{Salzmann}\RWlit{}{Salzmann1}; dessen \RWbet{Erster Unterricht in der Sittenlehre}\RWlit{}{Salzmann2}; dessen \RWbet{Heinrich Gottschalk}, oder \RWbet{erster Unterricht in der natürlichen Religion}\RWlit{}{Salzmann3}; \uam\  An einem Buche aber, welches das Ganze der vollkommensten Religion, also alle Lehren des Christenthums, und dieß zwar des katholischen, auf eine gleiche Art darstellte, fehlt es noch ganz. Lossius' \RWbet{Gumal und Lina\RWlit{}{Lossius1}}, oder der Karoline Pichler \RWbet{Agathokles}\RWlit{}{Pichler1}, oder Sailer's und Pa\v{r}ic\v{z}ek's \RWbet{Religion der Unmündigen}\RWlit{}{Pavrizek1}, oder Mutschelle's \RWbet{Unterredungen eines Vaters mit seinen Kindern}\RWlit{}{Mutschelle1}, oder die \RWbet{Religion in Briefen} \ua\,dgl.~Schriften können noch gar nicht als eine Auflösung der Aufgabe, von der ich hier spreche, angesehen werden. Wer also das hiezu nöthige Talent besäße, würde sich wahrlich ein bleibendes Verdienst um die Menschheit erwerben, wenn er ein solches Werk verfaßte!
\end{RWanm}
\item Erst nachdem man das Kind durch die Vermittlung eines solchen Buches, oder auf irgend eine andere Art mit allen einzelnen Wahrheiten der vollkommensten Religion bekannt gemacht, und ihm den Nutzen derselben durch die erwähnte Anwendung auf das wirkliche Leben recht anschaulich~\RWSeitenw{133}\ dargestellt hätte, wäre es zweckmäßig, ihm auch ein Buch in die Hände zu geben, in welchem die sämmtlichen Lehren und Vorschriften dieser Religion \RWbet{systematisch} verzeichnet sind, um ihn so eine Uebersicht des Ganzen zu verschaffen.
\end{aufza}

\RWpar{53}{Einige der wichtigsten Fehler, die man sich gegen die jetzt beschriebenen Pflichten zu Schulden kommen läßt, und zwar 1.~der Fehler des systematischen Unterrichtes}
Wie das Betragen, welches die Menschen gegen ihre \RWbet{eigenen} religiösen Ueberzeugungen beobachten, sehr von demjenigen abweicht, das die Vernunft für pflichtmäßig erklärt: so und noch stärker weicht das Betragen, welches sie gegen die religiösen Ueberzeugungen ihrer \RWbet{Nebenmenschen} befolgen, von dem pflichtmäßigen ab. Es würde mich viel zu weit führen, wollte ich alle Fehler, die man in dieser Hinsicht sich zu Schulden kommen läßt, umständlich anführen; nur einige der wichtigsten und gewöhnlichsten werde ich kurz berühren.\par
Ein Fehler, den man in unserer Zeit \RWbet{beinahe allgemein} begeht, und überdieß noch zu einem \RWbet{Verdienste} sich anrechnet, ist der, daß man den \RWbet{ersten Unterricht in der Religion} auf eine \RWbet{abstracte} und fast \RWbet{systematische} Weise ertheilt. Alle Lehrbücher, nach denen man, selbst in den kleinsten Schulen, den Religionsunterricht ertheilet, sind in einem abstracten Vortrage, und nach systematischer Form geschrieben. Dieses hat nun meines Erachtens folgende Nachtheile:
\begin{aufza}
\item Ein solcher Unterricht wird von den Wenigsten gefaßt, während ein durch wahre oder erdichtete Erzählungen anschaulich gemachter Unterricht selbst von dem schwächsten Verstande wäre begriffen worden.
\item Wer ihn aber auch faßt, bei dem bleibt er doch nur ein \RWbet{todtes Wissen}, weil die auf diese Art bewirkte Ueberzeugung keine \RWbet{Lebhaftigkeit} hat, und weil so abstracte Wahrheiten aus Mangel an einer hinlänglichen Menge von~\RWSeitenw{134}\ Berührungspuncten mit andern Vorstellungen, welche der Seele des Zöglings geläufiger sind, nicht oft genug in sein Bewußtseyn zurückkehren können.
\item Ein solcher Unterricht hat endlich auch \RWbet{zu wenig Reiz} für das flüchtige Kindesalter, als daß man sich die nöthige Aufmerksamkeit für ihn versprechen könnte. Sucht man nun diese thörichter Weise vielleicht durch \RWbet{Strafmittel} zu erzwingen; so macht man das Uebel noch ärger, indem man auf diese Art einen nicht ungerechten Unwillen gegen den Unterricht erregt, der alsbald auch auf die Religion selbst übergehet.
\end{aufza}

\RWpar{54}{2.~Noch einige andere Fehler, wodurch die religiösen Wahrheiten verhaßt gemacht werden}
Nebst dem nur eben erwähnten abstracten Unterrichte, und der bei ihm so häufig gebrauchten \RWbet{Strafmittel} gibt es noch manche andere Fehler, durch welche die religiösen Wahrheiten \RWbet{verhaßt} gemacht werden, als:
\begin{aufza}
\item Wenn man von ihnen \RWbet{verhaßte Anwendungen} macht; \zB\ wenn man am Ende jedes religiösen Unterrichtes dem Zöglinge zeigt, wie so ganz fehlerhaft sein bisheriges Betragen gewesen; wenn man aus jeder Lehre, die man ihm vorträgt, beschwerliche Pflichten als Folgerungen, die sich aus ihr ergeben, ableitet; wenn man den Zögling immer nur dann, oder doch dann zuerst auf die Stimme seines Gewissens aufmerksam macht, wenn er so eben unrecht gehandelt, so daß er diese Stimme zuerst als eine strafende kennen lernet.
\item Wenn man \RWbet{zu lange und zu viel vorprediget}, und vielleicht überdieß noch in einem widerlichen Tone.
\item Wenn man die Schüler \RWbet{zu viel auswendig lernen} läßt.
\item Wenn man mit seinen Belehrungen \RWbet{zudringlich} ist; denn jede Zudringlichkeit ist lästig, und erregt überdieß noch den Verdacht einer eigennützigen Absicht.~\RWSeitenw{135}
\end{aufza}

\RWpar{55}{3.~Unhaltbare Beweisgründe}
\begin{aufza}
\item Von jeher haben sich Lehrer der Religion des Fehlers schuldig gemacht, daß sie die wichtigsten religiösen Wahrheiten auf Beweisgründe stützten, die bei genauerer Prüfung als \RWbet{unhaltbar} befunden wurden.
\item Diese Erscheinung ist auch sehr begreiflich. Es kann dieß
\begin{aufzb}
\item zum Theile schon daher kommen, weil es dem menschlichen Verstande überhaupt leichter ist, die Wahrheit, als ihren eigentlichen Beweisgrund einzusehen. So trug es sich also sehr oft zu, daß man wohl von der Wahrheit eines Satzes vollkommen überzeugt war, aber noch immer nicht wußte, welches sein eigentlicher Beweisgrund wäre. Indem man sich nun bemühte, auch diesen zu finden, gerieth man auf manche Scheingründe, die man für wirkliche ansah. Dieses begegnete den Lehrern jeder Wissenschaft. Wie viele unhaltbare Beweise hat man \zB\ nicht schon in der Mathematik für die Theorie der Parallelen zum Vorschein gebracht!
\item Bei religiösen Wahrheiten geschah dieses um so leichter, weil man aus der guten Absicht, diesen Wahrheiten ein desto allgemeineres und festeres Zutrauen zu verschaffen, wünschte, recht vielerlei Beweise für sie zu finden.
\item Endlich erlaubte man sich auch öfter aus frommem oder auch nicht frommem Betruge manchen Scheinbeweis für einen wirklichen auszugeben, obbleich man seine Unhaltbarkeit selbst einsah. Man erlaubte sich dieß, entweder weil man diese Art von Täuschung ihres guten Zweckes willen für etwas Unsträfliches, wohl gar Verdienstliches hielt; oder weil man in ihr ein bequemes Mittel zur Erreichung gewisser eigennütziger Absichten fand.
\end{aufzb}
\item Daß aber durch Aufstellung solcher unhaltbarer Beweise in dem Gebiete der Religion ungleich mehr Schaden als Nutzen gestiftet werde, erhellet aus folgenden Gründen:
\begin{aufzb}
\item Früher oder später wird die Unhaltbarkeit eines solchen Beweises entdeckt, und wer sie gemacht, ermangelt~\RWSeitenw{136}\ nicht, seine Entdeckung allgemein zu verbreiten, wäre es auch nur, um seinen Scharfsinn, oder seine Gelehrsamkeit zu zeigen. Wem nun kein anderer, als dieser unhaltbare, nunmehr entkräftete Beweis gegeben wurde, für den ist die Wahrheit jetzt so gut, als unerwiesen, er kann ihr nicht ferner sein Zutrauen mehr schenken.
\item Haben wir aber erst die Entdeckung gemacht, daß mehre der uns gegebenen Beweise unhaltbar sind: so argwöhnen wir, daß dieses auch bei den übrigen der Fall seyn dürfte; und so verlieren wir das Zutrauen auch selbst zu solchen Wahrheiten, deren Beweise wir bisher noch nicht entkräften konnten.
\item Viele schließen sogar, weil die \RWbet{Beweise}, welche man ihnen für gewisse Lehren gegeben, falsch und unhaltbar sind: so müßten wohl auch die \RWbet{Lehren} selbst falsch und unhaltbar seyn.
\end{aufzb}
\item Da unhaltbare Beweise, von welcher Art sie auch seyn mögen, immer die eben erwähnten Nachtheile mehr oder weniger befürchten lassen: so ergibt sich, daß es nie, oder nur in den seltensten Fällen erlaubt seyn könne, sich ihrer wissentlich zu bedienen; daß dieses wenigstens dort nie geschehen dürfe, wo man bei ihrem Gebrauche zugleich eine \RWbet{Lüge} begehen, und \zB\ ein Factum, das man für unerwiesen hält, doch für erwiesen, einen Schluß, den man für unrichtig hält, doch für richtig ausgeben müßte. \RWbet{Täuschungen} nämlich, die nicht zugleich auch Lügen sind, können wohl \RWbet{manchmal, nie} aber können Lügen erlaubt seyn.
\end{aufza}

\RWpar{56}{4.~Irreligiöse Schriften}
\begin{aufza}
\item Irreligiöse Schriften, \dh\ Schriften, in denen Behauptungen, \RWbet{die der vollkommensten Religion mehr oder weniger widersprechen}, oft mit den scheinbarsten Gründen vertheidiget werden, gibt es in einer, leider! nur zu großen Menge.
\item Die Entstehung solcher Schriften ist aus demjenigen, was ich (\RWparnr{46}.~2.~b.~c.) gesagt, nicht eben unbegreiflich,~\RWSeitenw{137}\ und wird aus Manchem, was erst in Zukunft gelegenheitlich, \zB\ im 1.~Hauptstück des 2.~Haupttheiles vorkommen soll, noch begreiflicher werden.
\item Ihre Schädlichkeit ist theils nach ihrer eigenen, theils nach der Beschaffenheit derer, die sie zu lesen wagen, verschieden. Ein und dasselbe Buch dieser Art kann für gewisse Leser gefährlich, Andern nicht nur unschädlich, sondern, wiewohl auf eine Weise, die der Verfasser schwerlich beabsichtigt haben möchte, selbst nützlich seyn. Die Wahrheit folgender Bemerkungen wird, wie ich hoffe, jeder Unparteiliche zugeben:
\begin{aufzb} 
\item \RWbet{Bücher, die irrige Vernunftschlüsse vortragen, sind für alle diejenigen gefährlich, welche nicht Scharfsinn genug besitzen, um das Unrichtige in diesen Schlüssen gewahren zu können. Wer aber dieß vermag, kann solche Bücher ohne Schaden, zuweilen wohl gar mit einem Nutzen lesen, in sofern wenigstens als er durch Widerlegung ihrer Fehlschlüsse im Denken geübt, und in seiner bisherigen bessern Ueberzeugung noch mehr befestiget wird.} -- Nur möge sich Niemand ohne hinlängliche Gründe zutrauen, daß er zu dieser letztern Classe von Lesern gehöre; denn in der That besitzen die wenigsten Menschen Scharfsinn und Uebung im Denken genug, um jeden verdeckten Fehl- und Trugschluß gewahr zu werden, und das, was eigentlich daran unrichtig ist, zu erkennen. Hiezu kommt noch, daß die Verfasser solcher Schriften sich oft sehr \RWbet{feiner Kunstgriffe} bedienen, um ihre Trugschlüsse desto scheinbarer zu machen, und die Aufmerksamkeit der Leser von dem, was darin irrig ist, desto gewisser abzulenken. Ich will hier einige der gewöhnlichsten in Kürze andeuten:
\begin{aufzc}
\item Sie tragen die falschesten Behauptungen mit dem Tone der \RWbet{festesten Zuversicht}, und als ob es die ausgemachtesten Wahrheiten wären, vor.
Dieß blendet den Leser, er meint, die Ursache, daß er die Wahrheit der hier ausgesprochenen Behauptung nicht sofort einsehe, müsse wohl nur in seiner eigenen Blödigkeit liegen, glaubt dem Verfasser aufs Wort, und ist betrogen.~\RWSeitenw{138}
\item Sie führen \RWbet{berühmte Namen} als Gewährsmänner an, und suchen dagegen unser Vertrauen auf die Einsicht oder Redlichkeit aller derjenigen, welche einst die entgegengesetzte Wahrheit behauptet hatten, herabzustimmen. Die \RWbet{Feinde des Christenthums}, ein Manes, Celsus, Porphyrius \uA\ sind ihnen lauter \RWbet{Weise}; die Schriftsteller dagegen, die für die Sache der Religion geschrieben, können sie kaum tief genug herabwürdigen.
\item Sie geben sich den \RWbet{Anschein der größten Unparteilichkeit}, als ob sie die ruhigsten, die uneingenommensten Forscher der Wahrheit wären, und je mehr Unbefangenheit sie häucheln, um desto parteilicher ist ihre Darstellung der Sache.
\item Sie schieben das Falsche, das sie uns beizubringen wünschen, unvermerkt unter viel Wahres ein, und der durch so viele unläugbare Sätze sicher gemachte Leser nimmt auch das Falsche an, ohne es einer genauen Prüfung erst unterzogen zu haben; oder sie mengen
\item die Wahrheit, die sie verdächtig machen wollen, unter einen Haufen der offenbarsten Ungereimtheiten. Z.\,B.\ den katholischen Lehrsatz von einem Reinigungszustande nach dem Tode tragen sie vor vermengt mit allen den albernen Vorstellungen, die nur der gemeine Mann von einem Fegefeuer sich bildet.
\item Sie \RWbet{häufen nur lauter Einwürfe}, und verschweigen dasjenige, was sich zu ihrer Widerlegung sagen ließe, oder führen auch wohl Widerlegungen an, aber gerade nur solche, die seicht und unbefriedigend sind.
\item Wenn sie dem Leser eine Meinung beibringen wollen, von der sie besorgen, sie dürfte im Anfange seinen Abscheu erregen, oder ihn wenigstens mißtrauisch gegen sie machen: so verschweigen sie lieber dieselbe und \RWbet{tragen bloß die Vordersätze, aus denen sie sich von selbst als Schlußsatz ergibt, behutsam und an zerstreuten Orten vor}. Der Leser, der nun nichts Arges ahnet, nimmt diese Vordersätze ohne viele Prüfung an, und leitet am Ende selbst die Folgerung aus ihnen ab, die sie ihm eigentlich beibringen wollten, und hängt derselben nur um so fester an, da er sie für seine eigene Erfindung ansieht.
\item Dasjenige, was sie im Anfange nur bedingungsweise und unter bestimmten Voraussetzungen für einen ein\RWSeitenw{139}zelnen Fall \udgl\ erwiesen, gebrauchen sie in der Folge, als wäre es allgemein dargethan worden.
\item Vorne versprechen sie, etwas erst später zu beweisen; später berufen sie sich auf den schon vorhin gelieferten Beweis.
\item Statt alle Gründe ihrer Behauptung auf einmal vorzutragen, und so die Uebersicht derselben zu erleichtern, zerstückeln sie Alles, sprechen an zehnerlei Orten von einem und demselben Gegenstande, bringen die nämlichen Gründe unter verschiedener Einkleidung zu wiederholten Malen vor; und der Leser weiß nun am Ende nicht mehr, wie viele und welche Gründe gebraucht worden sind; weil er sich aber erinnert, daß die Sache so oft besprochen worden ist: so bekommt er ein gewisses Zutrauen, und sieht die öftere Wiederholung einer Behauptung für einen Beweis derselben an.
\end{aufzc}
\item \RWbet{Bücher, die unrichtige historische Behauptungen vorbringen, sind allen denen gefährlich, welche den wahren Hergang der Sache nicht anderswoher richtiger kennen gelernt haben, und der hier vorkommenden Darstellung trauen. Nur wer so viel historische Kenntnisse hat, daß er jede in der Erzählung begangene Untreue sogleich gewahr wird, oder wer wenigstens der hier gegebenen Darstellung auf keinen Fall vertraut, kann solche Bücher ohne Nachtheil, aber er wird sie doch kaum mit einem Nutzen lesen.} -- Gegen historischen Betrug kann selbst die größte Gewandtheit in der Aufdeckung versteckter Trugschlüsse nicht schützen. Auch läßt sich kaum erwarten, daß nicht selbst derjenige, der mit dem größten Mißtrauen liest, doch ein und das Andere annehmen werde. Denn auch wer ausgebreitete historische Kenntnisse hat, wird nicht im Stande seyn, jede Unredlichkeit, die man sich in der Darstellung gewisser, anfangs nur unbedeutend scheinender Nebenumstände eines Ereignisses erlaubet, wahrzunehmen; und was wir auch nicht in dem ersten Augenblicke, als wir es lesen, glauben, bleibt doch in unserm Gedächtnisse zurück, vermischt sich allmählich mit unsrer bisherigen Vorstellung von der Sache, und trübt so die richtige Ansicht. -- Wie erst, wenn sich dergleichen Schriftsteller, um das Vertrauen der Leser desto gewisser zu gewinnen, nebst mehren (bei~a.) schon erwähnten Kunstgriffen, auch noch folgender bedienen:~\RWSeitenw{140}
\begin{aufzc}
\item Sie stellen sich an, als ob sie aus Furcht nicht Alles sagen dürften, was sie doch wissen, und lassen den unerfahrenen Leser auf diese Art ungleich ärgere Dinge vermuthen, als sie je sagen könnten.
\item Um jene Wunder, die zur Bestätigung der göttlichen Offenbarung gewirkt worden sind, zweifelhaft zu machen, erzählen sie uns die unglaubwürdigsten Wundergeschichten anderer Religionen, \zB\ der heidnischen, und stellen sich an, als meinten sie, daß diese letzteren eben so starke Beweise der Wahrheit für sich hätten, als jene ersteren.
\item Erzählungen, die mit den Lehren der göttlichen Offenbarung in einem wirklichen oder nur scheinbaren Widerspruche stehen, oder die wenigstens ein nachtheiliges Licht auf sie werfen, geben sich allenthalben für völlig sichere Nachrichten aus, so unglaubwürdig auch ihre Quellen seyn mögen; Nachrichten aber, welche der wahren Religion günstig sind, finden sie niemals strenge genug erwiesen. Daß \zB\ Kaiser Julian von einem \RWbet{Christen} getödtet worden, daß auf dem römischen Stuhle einst eine Frauensperson, \RWbet{Johanna}, gesessen, das Alles sind ihnen entschiedene Facta; ob aber \RWbet{Petrus} wirklich zu Rom gewesen, ob \RWbet{Jesus} in der That zu Bethlehem geboren worden, \udgl\ sind ihnen noch sehr zweifelhafte Dinge.
\end{aufzc}
\item \RWbet{Bücher endlich, welche die Leidenschaften des Lesers rege machen, und seinen Verstand durch seine Sinnlichkeit zu bestechen suchen, sind mehr oder weniger für einen Jeden gefährlich, am gefährlichsten aber für junge reizbare Gemüther}. Aller Scharfsinn, und alle Kenntnisse, die Jemand haben mag, nützen ihm wenig, wenn seine Leidenschaften aufgereizt, und der sündhafte Wunsch in seinem Herzen erzeugt worden ist, die Fesseln der Religion zu zerbrechen. Nun schenkt er Allem, was irreligiöse Schriftsteller behaupten, seinen Beifall; nun sind die abgeschmacktesten Trugschlüsse, wenn sie der Religion widersprechen, die schlagendsten Beweise; und die unverschäm\RWSeitenw{141}testen Lügen Enthüllung einer lange verborgenen Wahrheit. Die \RWbet{Kunstgriffe} aber, die solche Schriftsteller anwenden, um die Gemüther ihrer Leser zum Nachtheile der Wahrheit einzunehmen, und sie erst lasterhaft zu machen, damit sie ungläubig werden, sind in der That empörend:
\begin{aufzc}
\item Noch eine der unschuldigsten Leidenschaften, die sie in's Spiel zu ziehen trachten, ist die \RWbet{Eitelkeit}. Sie reden von der wahren göttlichen Offenbarung in den verächtlichsten Ausdrücken, spotten der Frömmigkeit und Tugend, schildern denjenigen, der noch an Himmel oder Hölle glaubt, als einen abergläubigen, blöden und schwachsinnigen Menschen, der keine Kraft besäße, sich von den Vorurtheilen seiner Jugend loszureißen \usw\ Der Leser schämt sich, mit solchen Menschen in einerlei Classe zu stehen; er fürchtet, daß auch er selbst ein Gegenstand des Spottes werden würde, wenn er noch ferner gläubig bliebe; und aus falscher Schamhaftigkeit, aus bloßer Eitelkeit bekennt er sich zum Unglauben, und thut dieß um so eher, da er auf diese Art nicht nur dem Spotte entgeht, sondern auch die freigebigsten Lobpreisungen erfährt. Denn wer so gefällig ist, jenen Schriftstellern beizupflichten, beweist ja Stärke des Geistes, ist ja ein aufgeklärter Denker, ein vorurtheilfreier Mann, \usw\
\item Noch ungleich wirksamer, aber auch ungleich verderblicher noch ist die Leidenschaft der \RWbet{Wollust}, welche nur zu viele Schriftsteller in ihren Lesern anzufachen suchen, um ihre Herzen von aller Religion am sichersten abwendig zu machen. Die üppigsten Schilderungen sinnlicher Lüste sind es, die man in ihren Schriften, oft selbst in solchen, wo man nicht einmal einen Anlaß dazu erwarten möchte, antrifft, und von den sträflichsten Verirrungen des Geschlechtstriebes wird als von Dingen gesprochen, die sehr verzeihlich, ja am Ende wohl gar etwas Löbliches wären. Kaum ist es möglich, Bücher von dieser Art zu lesen, ohne einigen Schaden zu nehmen; denn hat sich irgend ein~\RWSeitenw{142}\ unreines Bild unserer Einbildungskraft einmal bemächtiget; so ist es äußerst schwer, es wieder zu verbannen.
\end{aufzc}
\item Gewöhnlich werden die bisher aufgezählten Mittel und Kunstgriffe zur Verbreitung irreligiöser Gesinnungen \RWbet{vereinigt angewendet}, und hieraus mag ein Jeder von selbst erachten, wie äußerst verderblich und gefährlich das Lesen irreligiöser Schriften fast allgemein, besonders aber in einem Alter seyn müsse, wo die Begriffe noch schwanken, wo keine Fertigkeit in der Entwirrung verwickelter Trugschlüsse, und keine ausgebreiteten Kenntnisse in irgend einem Fache verlangt werden können, wo endlich das Gemüth die größte Empfänglichkeit hat, sich von den Leidenschaften der Eitelkeit und Wollust einnehmen zu lassen.
\end{aufzb}
\item Ein Schriftsteller, der \RWbet{nicht aus Leidenschaft an der Wahrheit gewisser religiöser Lehren zweifelt}, mag wohl zuweilen berechtiget seyn, seine Bedenklichkeiten auch selbst in öffentlichen Schriften vorzutragen; nur muß er es auf eine Weise thun, daß er hiebei
\begin{aufzb}
\item \RWbet{Niemand ärgere}, \dh\ Niemand eine Stütze der Tugend und einen Trost im Leiden raubt, ohne ihm etwas Besseres dafür zu geben. Auch darf er
\item \RWbet{Niemand durch Scheingründe zu hintergehen suchen}, also sich keiner künstlich versteckten Trugschlüsse bedienen, und keine historischen Unwahrheiten absichtlich vorbringen; am allerwenigsten aber darf er es sich beikommen lassen,
\item die \RWbet{Leidenschaften} seiner Leser rege zu machen. Wie sträflich nun derjenige sey, der sich das Gegentheil von diesem Allen erlaubt, erhellet aus dem Schaden, den er hiedurch theils wirklich anrichtet, theils doch anrichten könnte. Der eine sowohl als der andere ist unberechbar groß, wie Jeder zugeben wird, der in Erwägung ziehen will, in wie viel Hände oft solche Schriften gerathen; wie viele verderbliche Wirkungen auch nur ein einziges zweideutiges Wort zuweilen nach sich zieht; wie schwer dasjenige, was einmal geschrieben, und durch den~\RWSeitenw{143}\ Druck verbreitet worden ist, wieder vertilgt werden kann; wie eine Schrift noch hinterbleibt, um fortwährend Böses zu stiften, wenn ihr Verfasser schon längst mit Tode abgegangen ist; ja vielleicht nach Jahrhunderten erst nur noch mehr Schaden und Unheil anrichtet, als gleich bei ihrer Erscheinung; wie selbst ein nachheriger Widerruf dessen, was man in früherer Zeit gelehrt, selten im Stande ist, die schädliche Wirkung desselben aufzuheben, indem es häufig geschieht, daß das Publicum wohl die Schrift, aber nicht ihren Widerruf liest, oder dem Eindrucke, den die erstere hervorbringt, mehr als den Warnungen des letzteren folgt.
\end{aufzb}
\end{aufza}


\RWpar{57}{5.~Aergerliches Beispiel}
\begin{aufza}
\item Auch eine, leider! nur zu gewöhnliche Art, wie Menschen sich an den religiösen Ueberzeugungen ihrer Mitmenschen versündigen, ist das \RWbet{ärgerliche Beispiel}, das sie durch ihren eigenen \RWbet{Lebenswandel} geben. Gewöhnlich pflegt man den Vorwurf eines ärgerlichen Beispieles nur solchen Personen zu machen, die ein sehr \RWbet{lasterhaftes} Leben führen; allein, wenn wir die Sache genauer betrachten, so müssen wir sagen, \RWbet{religiöses Aergerniß durch seinen Lebenswandel} gebe ein Jeder, der nur in irgend einer Rücksicht \RWbet{durch seine Lebensweise} verursacht, \RWbet{daß seine Mitmenschen in der Erkenntniß der vollkommensten Religion mehr oder weniger beirret werden}; und dieß thut Jeder, der zu den Lehren der vollkommensten Religion ganz oder theilweise sich bekennt, und doch durch sein Leben beweist, daß er nicht auf derjenigen Stufe der Tugend und Glückseligkeit stehe, zu der er durch Benützung dieser Lehren in seinen Verhältnissen füglich hätte gelangen können; also ein Jeder, der, wenn  auch eben nicht \RWbet{lasterhaft}, doch nicht so \RWbet{tugendhaft} ist, als er es nach den Grundsätzen seines Glaubens billig seyn könnte und sollte; oder der zwar wohl tugendhaft, aber nicht \RWbet{ruhig} und \RWbet{zufrieden}, nicht froh und glücklich, sondern \zB\ sehr trübsinnig ist, immer das Schlimmste erwartet, sich vor dem Tode~\RWSeitenw{144}\ und der Zukunft fürchtet, über ein Jedes auch noch so kleine Versehen, das er aus menschlicher Schwachheit sich beikommen läßt, sich ohne Maß beängstiget, \usw\
\item Ein solches Betragen hat nämlich folgende Nachtheile:
\begin{aufzb} 
\item Menschen, welche die innere Beschaffenheit der Religion, welcher derselbe zugethan ist, nicht anders woher genauer kennen, ziehen aus seinem Verhalten den Schluß, daß seine Religion die Fehler, die sie an ihm bemerken, entweder selbst erzeuge, oder doch wenigstens keine kräftige Gegenmittel wider sie habe. So erhalten sie also eine ganz falsche Vorstellung von dem Werthe und der Brauchbarkeit dieser Religion für die Beförderung der Tugend und Glückseligkeit der Menschen, und geben sich eben deßhalb auch keine Mühe, sie näher kennen zu lernen.
\item Andere Menschen, denen die Lehren und Vorschriften dieser Religion genau genug bekannt sind, um einzusehen, daß der Wandel jenes Mannes mit ihnen nicht übereinstimmt, gerathen auf die Vermuthung, er müsse im Herzen wohl gar nicht an die Wahrheit dieser Vorschriften glauben, weil er sich sonst in seinem Lebenswandel mehr nach ihnen richten würde; und sein vermeintlicher Unglaube macht sie in ihrem eigenen Glauben wankend. Dieß ist besonders der Fall, wenn es ein sehr gelehrter Mann ist, von dem man allgemein vermuthet, daß er die Gründe für oder wider die Wahrheit des Glaubens, zu dem er sich äußerlich bekennt, geprüft haben werde, \zB\ ein Gottesgelehrter, ein Geistlicher.
\item Ist solch ein Mann, dessen Leben nicht mit den Grundsätzen seiner Religion übereinstimmet, \RWbet{Erzieher} oder \RWbet{Vater}, oder lebt er nur sonst in einer genauen Verbindung mit andern, besonders jungen Leuten: so steht sehr zu befürchten, daß diese, gesetzt  er trüge ihnen auch die richtigsten Begriffe vor, doch seinen \RWbet{Werken} mehr als seinen \RWbet{Worten} folgen, \dh\ daß sie die Fehler, die er in seiner eigenen Lebensweise begehet, aus Nachahmung annehmen werden\Hstreicht{(\RWlat{verba movent, exempla trahunt})}.~\RWSeitenw{145}\ Dann wird denn auch ihr Wandel nicht mit den Grundsätzen ihres Glaubens übereinstimmen, und auch sie werden religiöses Aergerniß geben.
\end{aufzb}
\end{aufza}

\RWpar{58}{6.~Verläugnung des Glaubens}
\begin{aufza} 
\item Noch eine sehr wichtige Versündigung an den religiösen Ueberzeugungen Anderer ist die \RWbet{Verläugnung} oder \RWbet{Abläugnung} des Glaubens, \dh\ eine durch ausdrückliche Worte, oder durch was immer für eine absichtliche Handlung von sich gegebene Erklärung, daß man eine gewisse religiöse Lehre nicht für wahr halte, die man doch in der That für wahr hält.
\item Eine solche Verläugnung oder Abläugnung des Glaubens ist
\begin{aufzb}
\item erstlich schon eine \RWbet{Lüge}, und bringt sonach alle die Nachtheile hervor, welche bei einer jeden Lüge, von welcher Art sie auch immer seyn mag, Statt finden. Durch eine jede Lüge wird nämlich das wechselseitige Vertrauen der Menschen unter einander geschwächt, und folglich alles das Gute verhindert, welches ein solches Vertrauen hervorbringen könnte.
\item Eine Lüge aber, welche die \RWbet{Religion} betrifft, und durch die man einen dem Menschen zuträglichen Glauben wider seine eigene Ueberzeugung für falsch erkläret, hat noch den eigenthümlichen Nachtheil, daß sie die Ausbreitung dieses Glaubens hindert, und ist aus diesem Grunde um so viel strafwürdiger.
\end{aufzb}
\end{aufza}\par
\RWbet{Einwurf.} Es kann gleichwohl Fälle geben, wo der Schaden, den wir durch die Verläugnung unsers Glaubens anrichten, weit geringer ist, als derjenige, den wir uns durch das Bekenntniß desselben zuziehen würden. Z.\,B.\ wenn wir darüber das Leben verlieren sollten. Ja zuweilen kann eine augenblickliche Verläugnung der Religion das einzige Mittel werden, ein Leben zu erhalten, das wir in Zukunft eben nur einer um desto ungehinderteren Verbreitung derselben widmen wollen.~\RWSeitenw{146}\par
\RWbet{Antwort.} Das Leben eines Menschen ist ein geringes Gut, verglichen mit den in das Unendliche sich erstreckenden Vortheilen, welche die Ausbreitung nur einer einzigen wichtigen Wahrheit hervorbringt; derjenige aber, der seinen Glauben auf einige Zeit nur darum verläugnen wollte, um ihn dann ungehinderter verbreiten zu können, würde seinen eigenen Zweck vereiteln. Denn eben durch sein gegenwärtiges Verläugnen würde er seiner zukünftigen Anpreisung dieser Religion alle Glaubwürdigkeit benehmen; weil wer auch nur einmal gelogen hat, in Zukunft nie wieder vollen Glauben findet. Dagegen ein für die Wahrheit der Religion bestandener Martyrertod macht auch die rohesten und gedankenlosesten Menschen aufmerksam auf dieselbe; Jeder fühlt sich gedrungen, den zu bewundern, der die Kraft hat, sein Leben für seinen Glauben zu geben. Jeder vermuthet im Voraus, daß eine Lehre, die den Menschen solche Kraft verleihet, kein leerer Wahn seyn könne; man prüft sie, und wird gläubig. Die Richtigkeit dieser Bemerkungen wird durch die bekannte Geschichte der schnellen Ausbreitung des Christenthums zur Zeit der blutigsten Verfolgungen desselben auf eine recht augenfällige Weise bestätiget.
\begin{RWanm}
Daraus, daß eine eigentliche Abläugnung seines Glaubens in keinem Falle erlaubt ist, folgt aber nicht, daß man zu einem offenen \RWbet{Bekenntnisse} desselben überall verpflichtet wäre. Zwischen der Abläugnung und dem Bekenntnisse einer Wahrheit gibt es noch einen \RWbet{Mittelweg}, welcher im \RWbet{Schweigen} besteht; denn wer von einer Wahrheit schweigt, der läugnet sie darum noch nicht. Zu einem offenen Bekenntnisse seines Glaubens ist man nur dann verpflichtet, wenn sich vermuthen läßt, daß es zur weiteren Verbreitung der Wahrheit etwas beitragen, und daß der Nutzen, der so hervorgeht, den Schaden überwiegen werde. Wann solch ein Fall vorhanden sey, muß aus Betrachtung aller Umstände entschieden werden. Ich erinnere nur:
\begin{aufzb}
\item daß dieser Fall nicht allemal dann schon vorhanden sey, wenn wir nur über unsere religiösen Meinungen \RWbet{befragt} werden. Oft thut man dieß \zB\ in der sichtbaren Absicht, um über unsere Erklärung zu spotten; da sind wir keineswegs verpflichtet, uns zu erklären, sondern wir sollen vielmehr die Frage auf eine kluge Art von uns ablehnen.
\item Wenn man uns dagegen aus einer \RWbet{aufrichtigen Lernbegierde}, oder von Seite einer \RWbet{rechtmäßigen Obrigkeit},~\RWSeitenw{147}\ also in ernster Absicht befragt; ingleichen, so oft wir vorhersehen können, daß unsere Antwort Mehren zu Ohren kommen werde, \udgl ; dann sind wir allerdings schuldig, zur Rede zu stehen.
\end{aufzb}
\end{RWanm}

\RWpar{59}{7.~Proselytenmacherei und Intoleranz}
\begin{aufza} 
\item Man hat es nicht selten versucht, durch die eröffnete Aussicht auf irdische Vortheile, und noch weit öfter durch Androhung, ja auch wirkliche Zufügung verschiedener Uebel, \zB\ durch Beschimpfungen, Verlust des Amtes, körperliche Züchtigungen \udgl\ zu bewirken, daß Andersdenkende sich zu dem Glauben, den man für den allein wahren hielt, bekehren. Das Eine oder die Bemühung, seinem Glauben Anhänger zu gewinnen, indem man sie durch angebotene Belohnungen verlockt, kann man das \RWbet{Proselytenmachen} nennen; das Andere oder die Androhung, ja auch selbst wirkliche Zufügung verschiedener Uebel zu dem gleichen Zwecke erhält den Namen der \RWbet{Intoleranz} oder \RWbet{Unduldsamkeit}. Wir wollen von beiden Fehlern hier in Vereinigung sprechen; und zuerst
\item untersuchen, woher es \RWbet{gekommen}, daß man sich dieser fehlerhaften Handlungsweisen so oft schuldig gemacht habe.
\begin{aufzb}
\item Man hat geglaubt, daß man durch angetragene Belohnungen oder durch angedrohte Strafen die dem Menschen so natürliche Trägheit zum Denken überwinden müsse. \RWlat{Vexatio dat intellectum}, sagten Juristen und Politiker. Und selbst der heil.\ Augustinus \RWlat{(adversus Donatistas)\RWlit{}{Augustinus3} meinte: Qui nescio, qua vi consuetudinis nullo modo mutari in melius cogitarent, nisi hoc terrore percussi sollicitam mentem ad considerationem veritatis intenderent}.
\begin{RWanm}
Es ist allerdings wahr, daß die Menschen träge zum Nachdenken sind, und selbst zum Nachdenken über ihr eigenes Heil; es ist auch ferner wahr, daß man bemühet seyn müsse,~\RWSeitenw{148}\ etwas zu thun, um diese ihre Trägheit zu überwinden; allein zu diesem Zwecke ist es nicht nöthig, eine Belohnung auf die \RWbet{Annahme}, sondern es genügt, sie auf das bloße \RWbet{Anhören} der religiösen Lehrsätze und ihrer Gründe zu setzen. Thut man ein Mehres, so treten Nachtheile ein, die ich bald angeben werde.
\end{RWanm}
\item Man glaubte ferner, wenn man auch Anfangs nur eine erhäuchelte Anhänglichkeit bewirke, so würde aus dieser doch mit der Zeit, besonders bei den Nachkommen eine recht \RWbet{herzliche} hervorgehen.
\begin{RWanm}
Auch dieses ist wahr; wie die Geschichte so mancher Völkerbekehrungen beweiset. Nur meine ich, daß der Zweck, um den es sich bei solchen Bekehrungen eigentlich handelt, die wahre Besserung und Beglückung der Bekehrten, durch Mittel von anderer Art, sanfte Belehrungen \udgl\ noch schneller und vollkommener hätte erreicht werden können.
\end{RWanm}
\item Zuweilen machte man sich auch eine etwas überspannte Vorstellung von jenen \RWbet{übernatürlichen Wirkungen}, welche selbst eine bloß äußere, eine aus Furcht nur erzwungene Annahme des wahren Glaubens hervorbringen müsse.
\begin{RWanm}
Durch das katholische Christenthum wird diese Vorstellung durchaus nicht unterstützt; denn es erkläret ausdrücklich, daß eine wider Willen vollzogene Taufe gar keine Gültigkeit habe.
\end{RWanm}
\item Es ist uns allerdings an sich selbst \RWbet{unangenehm}, zu erfahren, daß sich ein Anderer von unsern Meinungen nicht überzeugen läßt; denn dieß beleidiget
\begin{aufzc}
\item unseren Stolz; wir sehen, daß er nicht genug Zutrauen zu unsern Einsichten habe; es erregt
\item die Besorgniß, daß wir am Ende uns wohl gar selbst irren; es erschwert
\item endlich jeden Falls unseren Umgang mit ihm. Diese Unannehmlichkeiten lassen wir nun die Menschen selbst entgelten, und so verfolgen wir sie, wenn sie nicht gutwillig sich überzeugen lassen.
\end{aufzc}
\item Hiezu kommt noch, daß wir oft (richtiger oder unrichtiger Weise) glauben, es sey nur \RWbet{ihre eigene Schuld},~\RWSeitenw{149}\ daß sie nicht überzeugt werden, es sey nur Unaufmerksamkeit auf unsern Vortrag, oder ein sträflicher Eigensinn, der von seinen Meinungen durchaus nicht abgehen will, \udgl\ Bei dieser Vorstellung glauben wir denn eine Art von Recht zu haben, solche Personen für ihren sittlichen Fehler zu strafen.
\item Endlich ist nicht zu läugnen, daß man bei der Verfolgung der Andersdenkenden nur allzu oft auch eine erwünschte Gelegenheit fand, gewisse \RWbet{eigennützige Vortheile} zu erreichen; \zB\ Rache zu nehmen für eine persönliche Beleidigung, den Mitwerber um ein einträgliches Amt zu entfernen, sich durch das Vermögen des Verketzerten selbst zu bereichern \usw\
\end{aufzb}
\item Es ist ein, überhaupt zu sagen, \RWbet{schädliches} Beginnen, sey es durch Anerbietung irdischer \RWbet{Vortheile}, sey es durch Androhung verschiedener \RWbet{Uebel}, bewirken zu wollen, daß Jemand den Glauben annehme, der uns der wahre dünkt.
\begin{aufzb} 
\item Erstlich ist die Anwendung dieses Mittels allemal mit der Gefahr verbunden, der Andersdenkende dürfte sich aus \RWbet{bloßem Eigennutz} oder aus \RWbet{bloßer Furcht} vor zeitlichen Nachtheilen von unserem Glauben zu überreden suchen; oder wohl gar die Annahme desselben nur mit dem Munde \RWbet{häucheln}, während er in der That etwas ganz Anderes glaubt. Beides aber wäre ein Verbrechen, zu dessen Begehung wir unserem Nächsten keine Versuchung geben sollen.
\item Durch dieses Mittel wird auch \RWbet{die gute Sache selbst nur verdächtig gemacht}. Man denkt, eine Religion, der wir auf solche Art Anhänger werben wollen, müsse keine Beweisgründe für ihre Wahrheit haben, welche durch ihre eigene Kraft zu überzeugen vermögen; oder walte vermuthlich irgend ein eigennütziger Zweck ob, der uns zur Ergreifung so leidenschaftlicher Maßregeln verleitet.
\item Was insbesondere den Gebrauch der \RWbet{Zwangsmittel} oder die \RWbet{Intoleranz} betrifft; so liegt es in der Natur der Sache, daß ein solches Verfahren in dem Gemüthe dessen, der so behandelt wird, den lebhaftesten~\RWSeitenw{150}\ Unwillen gegen seine Unterdrücker erregen, und daß dieser sodann selbst auf ihre Religion übergehen, \dh\ daß auch diese ihm \RWbet{verhaßt} werden müsse, weil er sie als die veranlassende Ursache seiner Leiden ansieht.
\item Dieß Letztere muß besonders dann geschehen, wenn der Mißhandelte glaubt, daß seine Verfolger sich ganz \RWbet{nach den Grundsätzen ihrer Religion richten}, indem sie ihn verfolgen. Dann erlaubt er sich nämlich den Schluß, daß eine Religion, welche die Menschen so grausam macht, die eine solche Verfolgung der Andersdenkenden gestattet oder wohl gar befiehlt, unmöglich die wahre und Gott gefällige seyn könne.
\end{aufzb}
\item Bei der Beurtheilung der \RWbet{Strafwürdigkeit} dieses Verfahrens kommt es abermals darauf an, aus welcher Quelle es entsprungen. Wenn lediglich die bei 2.\,a, b und c genannten Irrthümer zu Grunde lagen, und wenn man nicht Schuld daran war, daß man das Irrige dieser Vorstellungen nicht einsah: so war das Böse, das man in dieser Meinung that, wohl freilich kein Verbrechen. Schon nicht mehr unschuldig aber kann eine Verfolgungssucht heißen, an welcher die bei 2.\,d erwähnten Beweggründe einigen Antheil haben. Abscheulich vollends ist ein Verfahren zu nennen, das, wie in den Fällen 2.\,f, unter dem Deckmantel der Religion die schändlichsten eigennützigsten Absichten birgt.
\end{aufza}

\RWch{Einschaltung.\\ Etwas über die kritische und einige neuere Philosophieen in Deutschland.}
\RWpar{60}{Inhalt und Zweck dieser Einschaltung}
Unter den mancherlei philosophischen Systemen, die in der neueren Zeit zum Vorschein gekommen sind, ist jenes der \RWbet{kritischen Philosophie}, deren Urheber \RWbet{Imm. Kant}~\RWSeitenw{151}\ war, eines von denen, die unsere Aufmerksamkeit am Meisten verdienen; und dieß zwar
\begin{aufzb}
\item schon wegen der \RWbet{beträchtlichen Anzahl der Gelehrten}, die diesem Systeme nicht nur vor mehreren Jahrzehenden, sondern selbst gegenwärtig, wenn auch nicht ganz, doch theilweise zugethan sind;
\item wegen des \RWbet{Scharfsinnes}, den der Erfinder unläugbar an den Tag gelegt hat;
\item wegen der \RWbet{scheinbaren Gründlichkeit und Vollendung}, welche er seinem Systeme, unter Anderem auch dadurch zu geben wußte, daß er die sämmtlichen Begriffe und Lehren desselben in einen sehr innigen Zusammenhang, und in eine durch ihre Symmetrie ergötzende Ordnung setzte;
\item wegen des \RWbet{Einflusses}, den dieses System auf alle übrigen philosophischen Systeme, die seitdem in Deutschland erschienen sind, gehabt hat; denn alle sind durch jenes gleichsam veranlaßt, und daraus hervorgegangen.
\end{aufzb}\par
Da nun dieses kritische System den Ansichten, die ich in diesem Unterrichte der Religionswissenschaft vortrage, \RWbet{gar häufig widerspricht}; so wird es nicht genug seyn, daß ich der einzelnen abweichenden Lehren desselben bloß \RWbet{gelegenheitlich} erwähne, und sie zu widerlegen bemühet bin; sondern es dürfte sich der Mühe verlohnen, an einem eigenen Orte auch die \RWbet{Hauptsätze}, aus welchen jene einzelnen Lehren entspringen, im Zusammenhange darzustellen; aber auch in gedrängter Kürze die \RWbet{Gründe}, aus welchen ich diesen Sätzen \RWbet{nicht beipflichten kann}, hinzuzufügen. Dieser kurzen Prüfung der vornehmsten Grundsätze der kritischen Philosophie werde ich dann noch \RWbet{einige allgemeine Bemerkungen über den Geist, der in den meisten und beliebtesten Philosophieen}, die seit der \RWbet{Kant'schen} vornehmlich nur in Deutschland erschienen sind, nachfolgen lassen.

\RWpar{61}{Kurze Uebersicht des Kant'schen Systemes}
Aus \RWbet{Kant's} Prolegomenen zu einer künftig aufzustellenden Metaphysik\RWlit{}{Kant1}, aus seiner Logik\RWlit{}{Kant3}, Kritik der reinen Ver\RWSeitenw{152}nunft\RWlit{}{Kant2} \uma\  Schriften dürfte sich darthun lassen, daß Folgendes eine nicht ungetreue Darstellung seiner Behauptungen sey.
\begin{aufza}
\item Alle unsere \RWbet{Vorstellungen} sind theils \RWbet{Anschauungen}, theils \RWbet{Begriffe}. Anschauungen werden als Vorstellungen erklärt, welche sich \RWbet{unmittelbar} auf einen Gegenstand beziehen. Begriffe dagegen als Vorstellungen, die sich nur \RWbet{mittelbar}, nämlich nur \RWbet{mittelst der Anschauung} auf einen Gegenstand beziehen.
\item Alle \RWbet{Urtheile} sind theils \RWbet{Erfahrungsurtheile}, theils \RWbet{Urtheile} \RWlat{a priori}. Letztere werden als solche erklärt, die \RWbet{nicht aus der Erfahrung entlehnt} sind, ob wir gleich zum \RWbet{Bewußtseyn} derselben nur durch Erfahrung gelangen. Als untrügliche Merkmale derselben wird \RWbet{Allgemeinheit} und \RWbet{Nothwendigkeit} angegeben.
\item Für eben so wichtig und vor ihm noch unbemerkt erklärte \RWbet{Kant} die Eintheilung der Urtheile in \RWbet{analytische} und \RWbet{synthetische}. Die ersteren sind, wie er sich ausdrückte, solche, bei welchen \RWbet{das Prädicat in dem Subjecte schon enthalten} ist, während in den synthetischen dem Subjecte ein Prädicat beigelegt wird, das man noch \RWbet{nicht in ihm gedacht} hatte; daher er sie auch \RWbet{Erweiterungsurtheile} nannte. Das Urtheil: Alle Dreiecke sind Figuren, soll analytisch; die Urtheile: alle Körper sind schwer, alle Winkel in einem Dreiecke betragen zwei rechte \udgl , sollen synthetisch seyn.
\item \RWbet{Kant} legte sich selbst die Frage vor, was uns zur Bildung dieser beiden Arten von Urtheilen berechtige? -- Was nun zuvörderst die analytischen betrifft; so glaubte er, die Bildung dieser unterliege gar keiner Schwierigkeit, indem sie alle auf einem und eben demselben Grundsatze, nämlich auf dem bekannten \RWbet{Satze des Widerspruches} oder der Identität ($A$ ist $A$; $x$ ist entweder $A$ oder Nicht-$A$) beruhen.
\item Schwieriger erschien ihm dagegen die Erklärung der Möglichkeit eines synthetischen Urtheiles. Da man hier dem Subjecte ein Prädicat beilegt, welches in dem Begriffe desselben doch ganz und gar nicht liegt: so soll sich die Frage,~\RWSeitenw{153}\ was uns zu dieser Verbindung gleichwohl veranlassen und berechtigen könne? nicht anders beantworten lassen, als durch die Voraussetzung, daß wir das Prädicat in einer \RWbet{Anschauung}, die wir mit dem Begriffe des Subjectes verbinden, antreffen. Und also soll sich die Möglichkeit aller synthetischen Urtheile nur auf Anschauungen gründen, die wir mit dem Begriffe des Subjectes verbinden. Wir legen \zB\ dem Begriffe \RWbet{Körper}, das in ihm selbst gar nicht enthaltene Prädicat \RWbet{schwer} bei, weil wir in jeder concreten Anschauung, die wir mit dem Begriffe Körper verbinden, das Merkmal der Schwere antreffen.
\item Aus dieser Erklärung der Sache ergab sich, daß wir nur über \RWbet{solche} Gegenstände synthetisch urtheilen können, von welchen wir \RWbet{Anschauungen} haben.
\item Zur Bildung eines \RWbet{empirisch-synthetischen} Urtheiles ist keine andere, als eine bloß \RWbet{empirische Anschauung} nöthig; soll aber das Urtheil \RWlat{\RWbet{a priori}} seyn: so muß auch jene Anschauung, welche dasselbe vermittelt, eine \RWbet{apriorische} seyn. Und folglich muß es, wenn synthetische Urtheile \RWlat{a priori} möglich seyn sollen, auch apriorische Anschauungen geben.
\item Die meisten Lehrsätze der reinen Mathematik (Arithmetik und Geometrie), ingleichen der reinen Naturwissenschaft (Mechanik) sind offenbar synthetisch. An ihrer Wahrheit aber kann kein Vernünftiger zweifeln; und darum ist entschieden, daß es für diese zwei Wissenschaften synthetische Urtheile \RWlat{a priori} gebe. Die \RWbet{apriorischen Anschauungen} aber, durch welche diese Urtheile vermittelt werden, sind keine anderen, als die \RWbet{Zeit} und der \RWbet{Raum}. Zeit und Raum sind nämlich keineswegs, wie manche ältere Philosophen geglaubt, \RWbet{Dinge an sich}; auch nicht, wofür sie \RWbet{Leibnitz} ausgab, bloße \RWbet{Verhältnisse}, (also Begriffe); sondern \RWbet{Anschauungen}; doch nicht empirische, sondern \RWbet{rein apriorische}, wie dieß aus ihrer Allgemeinheit und Nothwendigkeit erhellen soll; denn es ist schlechterdings unmöglich, sich eine Erscheinung in unserm Gemüthe (oder Innern) zu denken, die nicht in Zeit, und eine Erscheinung außerhalb Unser, die nicht in Zeit und Raum wäre. Die~\RWSeitenw{154}\ Zeit also ist die Bedingung oder Form aller innern; der Raum aller äußern Erscheinungen.
\item Hierdurch nun glaubte \RWbet{Kant} die Möglichkeit synthetischer und dabei doch apriorischer Urtheile über solche Gegenstände, welche in Zeit und Raum erscheinen, \dh\ welche von \RWbet{sinnlicher} Art, oder Gegenstände einer möglichen Erfahrung sind, hinlänglich aufgeklärt zu haben.
\item Zum Ueberflusse fügte er dieser Erklärung noch eine (wie er sich rühmte) ganz \RWbet{vollständige Aufzählung} aller rein apriorischen und zugleich einfachen Verstandesbegriffe (\RWbet{Kategorien}) bei, aus welchen alle übrigen apriorischen Begriffe zusammengesetzt seyn sollten. Wenn man aus einem Urtheile, sagte er, Subject und Prädicat, \dh\ die Materie desselben, welche in jedem eine andere ist, hinwegdenkt: so bleibt nur noch die Form desselben übrig; \zB\ alle $A$ sind $B$. Auch diese Form ist nicht bei allen Urtheilen dieselbe. Jeder eigenen Form des Urtheilens aber liegt irgend ein anderer, rein apriorischer und einfacher Verstandesbegriff zu Grunde. In dem erwähnten Urtheile \zB\ der Begriff der \RWbet{Allheit}. Wenn man daher alle möglichen Formen der Urtheile aufstellt: so wird man aus ihnen alle einfachen Verstandesbegriffe oder Kategorien erkennen. Nun suchte \RWbet{Kant} vermittelst einer sehr künstlichen Classification zu zeigen, daß es in Allem gerade zwölf Formen der Urtheile gebe.
\begin{longtable}{ll}
Der \RWbet{Quantität} nach drei: & allgemeine, \\
& besondere, und \\
& einzelne. \\
Der \RWbet{Qualität} nach drei: & bejahende,\\
& verneinende, und \\
& limitirende.\\
Der \RWbet{Relation} nach drei: & kategorische,\\
& hypothetische, und \\
& disjunctive.\\
Der \RWbet{Modalität} nach drei: & problematische,\\
& assertorische, und \\
& apodictische.~\RWSeitenw{155}
\end{longtable}
Aus diesen Formen der Urtheile nun leitete er folgendes Verzeichniß aller ursprünglich reinen Verstandesbegriffe oder Kategorien ab:
\begin{longtable}{lll}
%\begin{aufzb}\item
a) &  drei Kategorien der \RWbet{Quantität}: & Allheit, \\
& & Vielheit, \\
& & Einheit. \\
b) &  drei Kategorien der \RWbet{Qualität}: & Realität, \\
& & Negation, \\
& & Limitation. \\
c) & drei Kategorien der \RWbet{Relation}: & Substanz und Adhärenz, \\
& & Ursache und Wirkung, \\
& & Wechselwirkung. \\
d) & drei Kategorien der \RWbet{Modalität}: & Möglichkeit u. Unmöglichkeit, \\
& & Daseyn und Nichtseyn, \\
& & Nothwendigkeit u. Zufälligkeit. 
\end{longtable}
%\end{aufzb}
\item Ueber Gegenstände, die weder im Raume noch in der Zeit erscheinen, sondern von \RWbet{übersinnlicher} Art sind, \zB\ über \RWbet{Gott}, über unsere \RWbet{Seele} \udgl\ sollen eben darum, weil es uns in Betreff ihrer gänzlich an Anschauungen mangelt, auch \RWbet{gar keine synthetischen Urtheile möglich} seyn.
\item Gleichwohl besitzen wir Begriffe, oder, genauer zu reden, \RWbet{Ideen} von einigen dergleichen übersinnlichen Gegenständen, als die Idee von Gott, von einem \RWbet{Weltganzen}, von einer \RWbet{Seele}, von ihrer \RWbet{Freiheit} und \RWbet{Unsterblichkeit} \udgl\ Wie aber die Gegenstände, die diesen Ideen entsprechen, \RWbet{an sich} beschaffen seyen, ja ob es überhaupt nur dergleichen Gegenstände gebe, darüber sollen wir mittelst der \RWbet{theoretischen} Vernunft gar nichts entscheiden können; sondern bloß analytische Sätze, die unsere Erkenntniß nicht im Geringsten erweitern, \zB\ Gott ist Gott, sind wir über dergleichen Gegenstände zu fällen berechtiget.
\item Die Unmöglichkeit, über diese Gegenstände synthetisch zu urtheilen, soll auch der \RWbet{ärmliche Zustand} bestätigen, in welchem sich die ganze \RWbet{Metaphysik}, \dh\ die Lehre von den übersinnlichen Gegenständen, nämlich die Ontologie, die natürliche Theologie, die Kosmologie und Psychologie bis~\RWSeitenw{156}\ auf den heutigen Tag befindet. Denn während man in den zwei Wissenschaften, die einen sinnlichen Gegenstand haben, (der Mathematik und reinen Naturwissenschaft) schon seit Jahrtausenden in dem Besitze unbestrittener Wahrheiten ist, konnte man in der Metaphysik noch keinen festen Schritt fassen, noch \RWbet{keine einzige Wahrheit} aufstellen, die nicht von Andern mit eben so starken Gründen bestritten und widerlegt worden wäre. Ja \RWbet{Kant} behauptete im Ernst, daß unsere Vernunft, sobald sie es wagen will, die Begriffe und Grundsätze des Verstandes (die Kategorien) auch auf übersinnliche Gegenstände anzuwenden, \RWbet{viermal zwei einander widersprechende Sätze} (Antinomien) mit gleich wichtigen Gründen beweisen könne, nämlich:
\begin{aufzb}[1.]
\item Die Welt hat einen Anfang in der Zeit, und auch bestimmte Grenzen im Raume; und sie hat keinen Anfang in der Zeit, und keine Grenzen im Raume.
\item Jede zusammengesetzte Substanz in der Welt besteht aus einfachen Theilen; und es gibt nichts Einfaches in der Welt.
\item Nebst dem Gesetze der Causalität gibt es noch ein Gesetz der Freiheit in der Welt; und es gibt keine Freiheit, sondern Alles erfolgt nach den Gesetzen der Nothwendigkeit.
\item Es gibt ein schlechthin nothwendiges Wesen; und es gibt überall kein schlechthin nothwendiges Wesen.
\end{aufzb}
\item Nichts desto weniger soll es, wenn auch nicht durch theoretische Vernunft, oder auf dem Wege der Erkenntniß, dennoch auf einem anderen Wege, durch \RWbet{praktische Vernunft} möglich seyn, über alle diese Gegenstände zur völligen Gewißheit zu gelangen. Es ist nämlich ein unläugbares Factum in unserem Selbstbewußtseyn, daß es ein \RWbet{Sollen} (eine Pflicht, ein Sittengesetz) für uns gebe. Alles dasjenige nun, was zum Behufe dieses Sollens schlechterdings nothwendig ist, das müssen wir praktisch für wahr halten oder \RWbet{glauben}. Und dieser praktische Glaube ist nicht etwa schwankend und ungewiß, sondern er gewährt die völligste (obgleich nicht logische, doch \RWbet{moralische}) \RWbet{Gewißheit} und Ueberzeugung; und steht in dieser Rücksicht dem festesten theo\RWSeitenw{157}retischen Wissen nicht nach, sondern er unterscheidet sich von diesem nur in dem Grunde, auf welchem er beruht. Das Wissen nämlich beruht auf Anschauung, das Glauben aber auf praktischer Vernunft oder auf dem Factum in unserem Selbstbewußtseyn, daß es ein Sollen gibt. Der Inhalt des Sollens, oder die Forderung der praktischen Vernunft ist nun die \RWbet{Realisirung des höchsten Gutes} (die Zustandebringung des möglich höchsten Grades der Tugend, und einer ihr entsprechenden Glückseligkeit). Alles also, was immer nothwendig vorhanden seyn muß, damit das höchste Gut zu Stande kommen könne, davon müssen wir glauben, daß es vorhanden sey. Die Realisirung des höchsten Gutes aber ist durchaus unmöglich ohne Freiheit, Gott und Unsterblichkeit; daher können und \RWbet{müssen} wir mit aller Gewißheit glauben, daß wir frei sind, daß ein Gott sey, und daß unsere Seele unsterblich sey. Diese drei Wahrheiten nannte nun \RWbet{Kant} die \RWbet{Postulate der praktischen Vernunft}, und meinte, daß uns von übersinnlichen Dingen kein Mehres zu wissen nothwendig sey.
\end{aufza}

\RWpar{62}{Beurtheilung dieser Lehren}
\begin{aufza}
\item Daß alle Vorstellungen entweder \RWbet{Anschauungen} oder \RWbet{Begriffe} sind, gebe auch ich zu; nur dürfte ich unter Anschauung nicht ganz dasselbe verstehen, was \RWbet{Kant}, indem ich bei einer jeden Vorstellung zur \RWbet{Anschauung} nur das allein, was in ihr \RWbet{individuell} ist; Alles dagegen, was sie \RWbet{mit andern gemein haben} kann, schon zum \RWbet{Begriffe} rechne. So zähle ich in der Vorstellung von \RWbet{dieser Farbe} oder von \RWbet{diesem Schmerz}, die Vorstellungen \RWbet{Farbe} und \RWbet{Schmerz} schon zum \RWbet{Begriffe}. Alles, wofür die Sprache ein Wort besitzt, gehört eben deßhalb schon zu den Begriffen; denn Worte erfindet man nicht für \RWbet{eine einzige} Vorstellung, sondern für eine Vorstellung, die öfters wiederkehrt, die sich auf Vieles anwenden läßt, kurz für Begriffe.
\item Die Eintheilung der Urtheile in \RWbet{Erfahrungsurtheile} und \RWbet{apriorische} habe ich gleichfalls schon oben angenommen, und nur auf eine etwas andere Weise erklärt.~\RWSeitenw{158}
\item Dafür, daß uns \RWbet{Kant} auf den Unterschied zwischen \RWbet{analytischen} und \RWbet{synthetischen} Urtheilen (von dem man zwar schon bei \RWbet{Aristoteles} einige Spuren findet) aufmerksamer gemacht hat, sollte man ihm, wie ich glaube, viel Dank wissen; nur hat er (meinem Dafürhalten nach) diesen Unterschied noch immer nicht deutlich genug aufgefaßt, und nicht gehörig angewendet. Ich denke mir ihn ohngefähr so. Alle Begriffe sind entweder \RWbet{einfach}, oder aus mehreren einfachen \RWbet{zusammengesetzt}. Ist nun das \RWbet{Prädicat} eines Urtheiles Einer von den \RWbet{Bestandtheilen}, aus welchen der Begriff des \RWbet{Subjects} zusammengesetzt ist, so heißt das Urtheil \RWbet{analytisch}, und wenn es \RWbet{ganz dasselbe} mit dem Subjecte ist, \RWbet{identisch}. So ist \zB\ das Urtheil: Ein rechtwinkeliges Dreieck ist ein Dreieck, analytisch; das Urtheil: Ein Dreieck ist eine dreiseitige Figur, identisch. Jedes andere Urtheil dagegen, \dh\ jedes Urtheil, in welchem der Prädicatbegriff keiner von den Bestandtheilen des Subjectbegriffes ist, nennt man \RWbet{synthetisch}. Nach dieser Erklärung läßt sich sehr leicht beweisen, daß es synthetische Urtheile gebe; weil ja auch Urtheile gebildet werden können, deren Subjectvorstellung eine \RWbet{einfache} Vorstellung ist. Der Behauptung, daß es sich in der Wissenschaft nicht sowohl um analytische, als vielmehr um synthetische Urtheile handle, pflichte auch ich bei; ja ich glaube, daß bloß analytische Sätze in einer Wissenschaft gar nicht aufgestellt zu werden verdienen.
\item  Daher möchte ich denn auch die Sätze der \RWbet{Identität} oder des \RWbet{Widerspruches}, und mehre andere dergleichen Sätze, mit denen man in den speculativen Wissenschaften so viel Aufhebens gemacht hat, lieber gänzlich beseitiget sehen. Statt zu sagen, daß sich die analytischen Sätze auf den Satz des Widerspruches \RWbet{gründen}, möchte ich vielmehr sagen, daß sie nichts Anderes als \RWbet{specielle Beispiele} von diesem Satze sind.
\item Doch dieses Alles wäre von geringer Wichtigkeit; allein der \RWbet{folgenreichste Irrthum} in dem Systeme der kritischen Philosophie ist meiner Meinung nach die Art, wie \RWbet{Kant} die Möglichkeit \RWbet{synthetischer Urtheile} erklärte.
\begin{aufzb}
\item Zwar ist wohl nicht zu läugnen, daß viele Urtheile, namentlich alle sogenannten \RWbet{Erfahrungsurtheile} durch~\RWSeitenw{159}\ \RWbet{Anschauungen} vermittelt werden; und ich habe die Art, wie dieses meiner Meinung nach geschieht, schon oben (\RWparnr{14}) an einem Beispiele erläutert. Sind mehrmals Anschauungen von einer solchen Art, daß wir sie sämmtlich unter den Begriff $A$ (\zB\ hellroth) subsumiren konnten, in uns entstanden; und hatten wir zu gleicher Zeit mit ihnen jedesmal auch gewisse andere Anschauungen, welche wir unter den Begriff $B$ (\zB\ Schmerz) subsumiren konnten: so berechtigt uns dieses zu dem Urtheile: Derselbe Gegenstand $x$, welcher die Ursache der unter dem Begriffe $A$ stehenden Anschauungen ist, ist \RWbet{wahrscheinlicher Weise} auch die Ursache der unter dem Begriffe $B$ stehenden Anschauungen. Der Gegenstand, der die Ursache von der Empfindung des Hellrothen war, oder das Hellrothe ist wahrscheinlich auch die Ursache von der Empfindung des Schmerzes; oder das Hellrothe verursachet Schmerz. Von dieser Form sind alle Erfahrungsurtheile; nimmermehr aber kann auf diese Weise ein Urtheil \RWlat{a priori}, ein reiner Begriffssatz, entstehen.
\item Wie unstatthaft die \RWbet{Kant'sche} Erklärungsart der Entstehung unserer \RWbet{reinen Begriffsurtheile} sey, zeigt sich auf folgende Art. Um das synthetisch-apriorische Urtheil: \RWbet{Alle $S$ sind $P$}, zu bilden, verbinde ich (heißt es nach Kant) mit dem Begriffe des Subjectes $S$ irgend eine Anschauung, in der ich das Prädicat $P$, und zwar, wenn meine Anschauung eine rein apriorische ist, mit dem Gefühle entdecke, daß dieß $P$ \RWbet{allen} unter $S$ stehenden Anschauungen zukomme; und daher lege ich es dem ganzen $S$ bei, und sage: Alle $S$ sind $P$. -- Hiegegen erinnere ich nun, \RWbet{in einer Anschauung könne nie ein Begriff} ($P$) enthalten seyn, sondern höchstens könne geschehen, daß sich zu gleicher Zeit mit jener Anschauung, welche sich unter den Begriff $S$ subsumiren läßt, in unserem Gemüthe noch eine \RWbet{andere Anschauung} einstellt, welche sich unter den Begriff $P$ subsumiren läßt, und die somit den Begriff $P$ in uns anregt. Allein durch diesen Umstand, durch diese gleichzeitige Erscheinung zweier Anschauungen in meinem Bewußtseyn, deren die~\RWSeitenw{160}\ Eine unter den Begriff $S$, die andere unter den Begriff $P$ gehört, werde ich noch gar nicht berechtiget, den Begriff $P$ den Gegenständen, welche sich unter den Begriff $S$ subsumiren lassen, \RWbet{allgemein} beizulegen; sondern erst, wenn die erwähnten Anschauungen \RWbet{sehr oft} in unser Bewußtseyn gleichzeitig einkehren, schließen wir bloß mit \RWbet{Wahrscheinlichkeit}, und in einem Urtheile, das ein \RWbet{empirisches} ist, auf den Zusammenhang zwischen $S$ und $P$. Dieses gesteht auch \RWbet{Kant} selbst ein; doch nur bei \RWbet{empirischen} Anschauungen; ein Anderes aber soll es bei den \RWbet{apriorischen} Anschauungen seyn. Allein ich sehe den Unterschied durchaus nicht ein, der zwischen apriorischen und empirischen Anschauungen Statt finden soll, um bei den ersteren jenen Schluß zu rechtfertigen, der bei den letztern nicht angeht. Freilich beruft sich \RWbet{Kant} auf ein gewisses Bewußtseyn oder Gefühl der \RWbet{Nothwendigkeit}, welches die apriorische Anschauung begleite; aber ist solch ein Bewußtseyn oder Gefühl wohl etwas Anderes, als selbst schon ein (dunkel gedachtes) Urtheil, und dieß zwar ein \RWbet{synthetisches}? Und so dreht sich denn (wie ich glaube) diese Erklärung der Entstehungsart der synthetischen Urtheile in einem Zirkel.
\item Da ich nun diese \RWbet{Kant'sche} Erklärung verwerfe, so muß ich wohl \RWbet{eine andere} geben. Wie ich mir die Entstehung der synthetischen \RWbet{Erfahrungsurtheile} denke, habe ich bereits gesagt. In Rücksicht der reinen \RWbet{Begriffsurtheile} aber unterscheide ich die \RWbet{Wahrheit an sich selbst}, und unsere \RWbet{Erkenntniß} von derselben.
\begin{aufzc}
\item Was die \RWbet{Wahrheit an sich} belangt; so sind die meisten reinen Begriffswahrheiten, namentlich alle, deren Subject oder Prädicat ein \RWbet{zusammengesetzter} Begriff ist, \RWbet{Folgewahrheiten}, \dh\ sie haben ihren Grund in gewissen anderen. So hat \zB\ die Wahrheit, daß im gleichschenkeligen Dreiecke die Winkel an der Grundlinie gleich sind, ihren Grund in den zwei andern Wahrheiten, daß zwei Seiten und der eingeschlossene Winkel ein Dreieck bestimmen, und daß es nur Einen Winkel im Dreiecke gibt, der einer gege\RWSeitenw{161}benen Seite desselben entgegensteht. Nun gibt es aber auch \RWbet{Grundwahrheiten}, \dh\ Wahrheiten, die keinen weitern Grund ihrer Wahrheit haben. Bei diesen ist es denn auch ungereimt, nach einem Grunde zu fragen. Ist das Urtheil: Alle $S$ sind $P$, eine solche Grundwahrheit; so kann auf die Frage, warum dem Subjecte $S$ das Prädicat $P$ zukomme? nichts Anderes erwidert werden, als: Weil es das Subject $S$ und kein anderes ist.
\item Betreffend die \RWbet{Möglichkeit unserer Erkenntniß apriorischer Wahrheiten}; so werden wir
\begin{aufzb}[a.]
\item auf viele derselben schon durch die \RWbet{Erfahrung} (auf die bereits beschriebene Weise) geleitet; erkennen sie aber dann auch nur mit Wahrscheinlichkeit.
\item Diejenigen aber, die \RWbet{Folgewahrheiten} sind, lernen wir häufig durch die \RWbet{Einsicht in ihre Gründe} kennen. Wenn wir \zB\ die beiden Wahrheiten erkennen, aus welchen eine dritte folgt; so leuchtet uns nicht eben jedesmal, aber doch oft auch diese dritte ein.
\item \RWbet{Grundwahrheiten} endlich erkennen wir zu Folge der Vorstellung, die wir uns von den Begriffen derselben machen. Ist der Satz: Alle $S$ sind $P$, eine Grundwahrheit, so erkennen wir, daß alle $S$ die Eigenschaft $P$ haben, kraft dessen, daß wir die Begriffe $S$ und $P$ besitzen. -- Dieß aber noch weiter zu erklären, ist eine Unmöglichkeit, indem die Sache selbst eine ganz \RWbet{einfache Verrichtung} unserer Urtheilskraft ist.
\end{aufzb}
\end{aufzc}
\end{aufzb}
\item Falsch ist daher (meiner Ueberzeugung nach) die absprechende Behauptung \RWbet{Kant's}, daß wir nur über solche Gegenstände synthetisch urtheilen können, von welchen wir \RWbet{Anschauungen} haben. -- Ich behaupte vielmehr, daß wir über alle Gegenstände, von welchen wir nur \RWbet{Begriffe} haben, eben kraft dieser Begriffe auch gewisse Urtheile und zwar mitunter auch synthetische zu fällen im Stande sind. Denn einen gewissen Begriff $S$ haben, heißt doch nichts Anderes, als ihn von anderen $T$, $U$, $V$, \textsymmdots\ unterscheiden; und dieß wieder heißt nichts Anderes, als ihm oder vielmehr den unter~\RWSeitenw{162}\ ihm enthaltenen Gegenständen gewisse Beschaffenheiten (Prädicate) \RWbet{beilegen}, und andere \RWbet{absprechen}, also \RWbet{Urtheile fällen}, in welchen er die Subjectvorstellung bildet.
\item Wurde bereits besprochen in 5.
\item[8.~und 9.]\setcounter{enumi}{9} Die von \RWbet{Kant} aufgestellte, und seit dem fast allgemein angenommene Behauptung, \RWbet{daß Zeit und Raum Anschauungen wären}, und daß die mathematischen Wahrheiten nicht aus Begriffen, sondern aus diesen Anschauungen erwiesen würden, däucht mir ganz falsch zu seyn. Ich halte Zeit und Raum mit \RWbet{Leibnitz} für bloße \RWbet{Verhältnisse}, und zwar ist mir die Zeit dasjenige Verhältniß, das als \RWbet{Bedingung zur Veränderung} nothwendig ist, \dh\ das die Bedingung enthält, unter welcher einem und eben demselben wirklichen Gegenstande mehre einander \RWbet{widersprechende} Beschaffenheiten zukommen können; der Raum, oder vielmehr die \RWbet{Orte} der Dinge aber sind mir diejenigen Verhältnisse unter denselben, welche den Grund enthalten, daß sie bei diesen und jenen Kräften, in dieser und jener Zeit, gerade so, und nicht anders, auf einander einwirken. Aus diesen Begriffen von Zeit und Raum lassen sich alle Eigenschaften derselben, \dh\ die ganze Chrono- und Geometrie ableiten, ohne daß man je nöthig hätte, sich auf eine sogenannte Anschauung zu berufen; obgleich ich gestehe, daß dieses viel bequemer sey, als das Beweisen aus bloßen Begriffen. Daß aber die \RWbet{Arithmetik}, wie \RWbet{Kant} behauptete, auf der Vorstellung der Zeit beruhe, scheint mir ein offenbarer Irrthum. Die \RWbet{Zeitlehre} (Chronometrie) ist es, die mit der Zeit sich beschäftiget, mit den Lehren der Arithmetik dagegen hat der Begriff der Zeit gar nichts zu schaffen.
\item Ich glaube keineswegs, daß die \RWbet{Kant'sche Tafel der Kategorien alle einfachen Begriffe} umfasse; auch däucht mir, daß mehre Begriffe, die sie enthält, schon \RWbet{nicht mehr einfach}, sondern \RWbet{zusammengesetzt} sind. Mir däucht auch die logische Classification der Urtheile, aus der \RWbet{Kant} seine Kategorien ableitete, sehr fehlerhaft zu seyn. So sind mir \zB\ die \RWbet{besonderen} Urtheile nur anders ausgedrückte \RWbet{Möglichkeitsurtheile}, manche~\RWSeitenw{163}\ \RWbet{hypothetische} sind gleichfalls \RWbet{nur im Ausdrucke} von \RWbet{kategorischen} verschieden, \usw\
\item Schon darum, weil wir im Besitze der Begriffe \RWbet{Gott, Seele} \usw\ sind, können wir auch gewisse (nicht bloß analytische) Urtheile über die durch diese Begriffe bezeichneten Gegenstände fällen.
\item Die \RWbet{Kant'sche} Behauptung, daß wir über die \RWbet{übersinnlichen} Gegenstände nichts urtheilen können, widerspricht nicht nur dem gemeinen Menschenverstande, sondern \RWbet{Kant} selbst hat sich bei dieser Gelegenheit in einen Widerspruch verwickelt. Er zählte nämlich unter diese übersinnlichen Gegenstände auch die Dinge, welche die \RWbet{Ursachen} der \RWbet{Erscheinungen} sind, die wir haben, die sogenannten \RWbet{Noumena}, und nahm das Daseyn solcher Dinge überall an, welches er doch, zu Folge des eben Gesagten, nicht hätte thun sollen, weil schon die Annahme des Daseyns eines Dinges ein Urtheil über dasselbe ist; und noch offenbarer ist die Behauptung, daß wir über diese Dinge an sich nicht zu urtheilen vermögen, schon \RWbet{selbst ein Urtheil} über diese Dinge.
\item Die Fortschritte, welche die Mathematik und die reine Naturwissenschaft gemacht, während die Metaphysik bis auf den heutigen Tag in einem so ärmlichen Zustande verblieben ist, lassen sich wohl auch auf eine andere Weise erklären; vornehmlich daraus, weil die Richtigkeit der meisten Lehrsätze der Mathematik und reinen Naturwissenschaft \RWbet{durch die Erfahrung erprobt} werden kann. Das Blendwerk jener vier \RWbet{Antinomien} aber verschwindet bei einer genauen logischen Beleuchtung der Beweise; weil sich dann jedesmal zeigt, daß der Beweis des Einen von den zwei widersprechenden Sätzen, oder wohl gar, daß die Beweise beider fehlerhaft sind.
\item Die von \RWbet{Kant} vorgeschlagene \RWbet{Aushülfe durch die praktische Vernunft} taugt meiner Einsicht nach zu nichts, als daß sie uns zeigt, wie unwiderstehlich doch dem gesunden Menschenverstande die Wahrheit von Gottes Daseyn, und einige andere Wahrheiten sich aufdringen müssen; indem selbst derjenige Philosoph, der aus einem Irrthume seiner Philoso\RWSeitenw{164}phie keine befriedigenden Beweise für diese Wahrheiten zu finden wußte, dennoch bloß darum, weil er ein rechtschaffener Mann war, der seinem innersten Gefühle nicht widersprechen wollte, eingestand, daß er an einen Gott, an Freiheit und Unsterblichkeit glaube. Aus welchen Gründen ich aber diese Methode des \RWbet{Postulirens} für unzulänglich erachte, wird tiefer unten (\RWparnr{170}) gezeigt werden.
\begin{RWanm}
Das ganze \RWbet{Kant'sche} System, besonders aber die inconsequente Behauptung desselben \RWbet{von der Subjectivität aller unserer Urtheile} (Nr.\,12.) hätte (meinem Dafürhalten nach) gar nicht zum Vorschein kommen können, wenn \RWbet{Kant} den Anfang des Philosophirens dort angenommen hätte, wo er (nach meiner Meinung) einzig genommen werden sollte. Der Philosoph muß nämlich schlechterdings nichts als schon bekannt und ausgemacht voraussetzen, nicht einmal, wie \RWbet{Kant}, sein eigenes Daseyn, noch weniger das Daseyn gewisser anderer Gegenstände, welche Vorstellungen in ihm hervorbringen. Er kann also, wie ich glaube, nirgends, als von der Behauptung anfangen, \RWbet{daß es Wahrheiten überhaupt} gebe. Dieß muß man ihm nothwendig zugestehen, weil die entgegengesetzte Behauptung sich selbst widerspricht. Er untersucht nun die Natur dieser Wahrheiten, findet, daß sie in einem gewissen \RWbet{objectiven Zusammenhange} mit einander stehen, \dh\ sich zu einander wie \RWbet{Gründe} und \RWbet{Folgen} verhalten, \usw , dieß Alles, ohne noch sich selbst als denkendes Wesen dabei vorauszusetzen, oder mit einzumengen. Hierauf versucht er dann das System aller dieser Wahrheiten nach ihrem objectiven Zusammenhange darzustellen; und erst, wenn er zur Darstellung der sogenannten empirischen Wahrheiten kommt, erfährt er, daß es ein Ich und Dinge außerhalb dieses Ichs gebe \usw\ Die Frage \RWbet{Kant's} (4), was dieses Ich zur Bildung seiner Urtheile berechtige? wenn sie nicht etwa von irgend einem einzelnen, sondern von den gesammten Urtheilen desselben verstanden werden soll, ist, meiner Meinung nach, ungereimt, und soll gar nicht aufgeworfen werden; denn wer so frägt, der zweifelt, ob irgend Eines seiner Urtheile~\RWSeitenw{165}\ objectiv wahr sey, will dieß eben erst untersuchen; und doch, wie kann er es, wenn er nicht während der Untersuchung selbst sich die Fähigkeit, objectiv wahre Urtheile zu bilden, zutraut?
\end{RWanm}
\end{aufza}

\RWpar{63}{Einige Bemerkungen über die neueste Art des Philosophirens in Deutschland}
\begin{aufza} 
\item So ausgebreitet der Beifall war, den sich die kritische Philosophie, vornehmlich durch die Bemühungen eines K.~L.~\RWbet{Reinhold}, Joh.~\RWbet{Schulz}, C.~Erh.~\RWbet{Schmid}, \RWbet{Mellin}, \RWbet{Bendavid}, \RWbet{Heidenreich}, \RWbet{Snell}, \RWbet{Kiesewetter}, \RWbet{Buhle}, \RWbet{Tieftrunk}, \umA\  nach und nach in ganz Deutschland erwarb, so kurz war die Dauer desselben. Sehr frühzeitig wurde bemerkt, daß \RWbet{Kant}, da er von dem \RWbet{Factum des Bewußtseyns} ausging, zwei Stücke, ein Gemüth, in welchem sich dieß Bewußtseyn vorfindet, und Dinge außerhalb desselben, die das Bewußtseyn anregen, vorausgesetzt habe. Auch fühlte man die Unvollkommenheit der Art, wie \RWbet{Kant} die drei wichtigen Wahrheiten des Daseyns Gottes, der Freiheit und Unsterblichkeit vermittelst der praktischen Vernunft postulire, ingleichen das Gekünstelte seiner Deduction der Kategorien. Besonders aber war die enge Grenze, die \RWbet{Kant} dem theoretischen Vernunftgebrauche anwies (das bloße Feld einer entweder wirklichen oder doch möglichen Erfahrung) viel zu beschränkend für das freie Emporstreben des menschlichen Geistes, als daß man nicht auf alle Weise gesucht haben sollte, sich dieser Fesseln wieder zu entledigen. Und so erschienen denn von Zeit zu Zeit neue Versuche von philosophischen Systemen, von denen jedoch keines so glücklich war, einen sehr ausgebreiteten Anhang zu finden.
\item Da es nicht möglich ist, diese verschiedenen Systeme, deren Anzahl sich mit jeder Messe vermehret, im Einzelnen näher zu beschreiben: so werde ich mir bloß einige \RWbet{allgemeine Bemerkungen über den Geist} erlauben, der in den \RWbet{meisten} derselben und vornehmlich in denjenigen wehet, die am \RWbet{Beliebtesten} sind.
\begin{aufzb} 
\item Je mehre einander wechselseitig umstoßende Systeme man in so kurzer Zeit entstehen und wieder ver\RWSeitenw{166}schwinden sah, um desto mehr, hätte man glauben sollen, werde ein Jeder, der im Begriffe stand, ein neues zu Tage zu fördern, besorgen, daß auch das seinige ein gleiches Schicksal mit den bisherigen erfahre; um desto \RWbet{bedächtiger} werde er bei der Bearbeitung desselben zu Werke gehen; um desto \RWbet{bescheidener} endlich dasselbe ankündigen. Von allem Diesen geschah in unserer neuesten Zeit häufig das Gegentheil. Beinahe Jeder trat im Tone der völligsten Zuversicht auf, daß er und \RWbet{er allein} das richtige System gefunden; und was das Lächerlichste war, diese Sprache der Zuversicht hörte man selbst Philosophen führen, welche ihr eigenes System schon mehrmals umgeschmolzen hatten. Die bescheidenen Redensarten: \RWbet{mich däucht}, es ist mir \RWbet{gegenwärtig} Dieß und Jenes \RWbet{wahrscheinlich} \usw\, sind diesen Gelehrten ganz fremd; ja sie behaupten sogar im Ernste, für einen Weltweisen gezieme es sich nicht, etwas zu meinen, sondern er müsse alles, was er sagt, mit apodiktischer Gewißheit wissen! -- Statt daß man, um sich nicht zu übereilen, seine vermeintlichen Entdeckungen in einer Wissenschaft, wo es so leicht ist, zu irren, theilweise, etwa in einzelnen Abhandlungen, vorgetragen hätte, um erst das Urtheil des Publicums über dieß Wenige zu hören und zu benützen, trat vielmehr Jeder auf mit einem schon vollendeten Systeme, oder gab wenigstens sich das Ansehen, als wäre er schon im Besitze eines solchen.
\item Eine zweyte Eigenheit der Weltweisen unserer Zeit ist es, daß sie sich an die unerläßlichsten \RWbet{Regeln der Logik}, namentlich an die Pflicht, immer bestimmt und deutlich zu erklären, \RWbet{wovon} man eigentlich spreche, in welcher \RWbet{Bedeutung} man dieß oder jenes Wort nehme; dann deutlich anzugeben, aus welchen \RWbet{Gründen} man etwas behaupte \usw , gar nicht gebunden halten, oder sie wenigstens überaus schlecht befolgen. In ihren Schriften bleibt der Leser gewöhnlich im Zweifel, in welchem Sinne das, was er so eben gelesen hat, genommen werden solle; muß dieß erst mühsam aus dem Zusammenhange des Ganzen zu errathen suchen; faßt eben deßhalb keine bestimmten, sondern bald da bald dort\RWSeitenw{167}hin schwankende Begriffe auf; weiß in den wenigsten Fällen, ob, was auf eine gewisse Behauptung unmittelbar folgt, schon der Beweis derselben, oder nur eine Erläuterung des Gesagten, oder wohl gar schon eine Folgerung daraus, ob es der einzige Beweis, den der Verfasser kennt, seyn solle, oder ob er noch andere im Vorrath habe, \usw\ Es ist betrübend, bemerken zu müssen, daß es \RWbet{Kant} selbst gewesen, der die erste Veranlassung zu diesem Unfuge gegeben durch die unglückliche Behauptung, daß eine streng logische Methode nur auf die Mathematik und reine Naturwissenschaft, nicht aber auf Philosophie anwendbar wäre.
\item Eine fernere Eigenheit der jetzt in Deutschland herrschend gewordenen Art des Philosophirens ist auch die \RWbet{Liebe zur Bildersprache}. Sonst glaubte man, der Philosoph müsse, so viel es thunlich ist, sich immer nur in den \RWbet{allereigentlichsten Ausdrücken} erklären, und alle Bilder als Zeichen, die ihren Gegenstand nur \RWbet{unbestimmt} bezeichnen, möglichst vermeiden: nicht also thun es unsere neuesten Weltweisen, sie häufen Bild auf Bild, und, was das Schlimmste ist, gebrauchen ihre Bilder, ohne den eigentlichen Sinn derselben (dasjenige, worin die Aehnlichkeit bestehen soll) je mit Bestimmtheit anzugeben.
\item Aus dieser Liebe zu bildlichen Ausdrücken entspringt eine weitere Unart der deutschen Weltweisen unserer Zeit, daß sie \RWbet{mit Aehnlichkeiten spielen}, und überall, wo sie nur eine entfernte Aehnlichkeit zwischen zwei Dingen erhascht zu haben glauben, diese wie eine \RWbet{völlige Gleichheit} derselben behandeln.
\item Auf eine fast kindliche Weise gefällt man sich in Sätzen, die etwas \RWbet{Paradoxes} haben, ja ihrem buchstäblichen Sinne nach einen baaren Widerspruch enthalten.
\item Ueberhaupt achtet man wenig darauf, ob man~\RWSeitenw{168}\ auch etwas \RWbet{Brauchbares} sage; sondern ist schon zufrieden, wenn man nur eine Menge von \RWbet{neuen Behauptungen} vorgebracht hat, die sich in einem gewissen Sinne vielleicht wohl rechtfertigen lassen, in diesem Sinne aber etwas schon längst Bekanntes, oder auch eine bloße Tautologie enthalten.
\end{aufzb}
\item Das beste Mittel, all diesen Uebeln zu steuern, däucht mir ein \RWbet{gründliches Studium der Logik}, von welcher ich glaube, daß man sie keineswegs, wie \RWbet{Kant} behauptete, als eine schon \RWbet{seit Aristoteles' Zeiten vollendete Wissenschaft} ansehen sollte; sondern daß allerdings noch manche Verbesserungen von Wichtigkeit sich in ihr anbringen ließen.
\item Uebrigens läugne ich nicht, daß in den Schriften der neuesten Weltweisen, denen ich die so eben aufgezählten Vorwürfe mache, gar \RWbet{mancher brauchbare Gedanke} vorkomme. Junge Leute jedoch muß man, so scheint es wenigstens mir, vor solchen Büchern schon aus dem Grunde warnen, weil sie diese spielende Art des Philosophirens, die freilich bei Weitem leichter als gründliches Denken ist, nur zu gern annehmen, und sich so zur Gewohnheit machen können, daß sie ihr ganzes künftiges Leben hindurch für jedes regelmäßige Denken verdorben sind.~\RWSeitenw{169}
\end{aufza}

\RWch{Zweites Hauptstück.\\ Kurzer Abriß der natürlichen Religion.}
\RWpar{64}{Inhalt und Zweck dieses Hauptstückes}
\begin{aufza} 
\item Ich nenne (nach \RWparnr{33}) den Inbegriff aller religiösen Lehren, deren Wahrheit ein Mensch \RWbet{ohne Dazwischenkunft eines göttlichen Zeugnisses}, oder (wie man dieß auszudrücken pflegt) \RWbet{durch seine eigene Vernunft} einzusehen glaubt, \RWbet{seine natürliche} oder \RWbet{Vernunft-Religion}. Da aber die Urtheilskraft der Menschen, ihre Erfahrungen \usw\ \RWbet{verschieden} sind; so kann dem Einen etwas gewiß dünken, das einem Andern nur zweifelhaft, einem Dritten wohl ganz irrig vorkommt. Daher wird der Inhalt der natürlichen Religion bei verschiedenen Menschen auch verschieden seyn. In dieser Bedeutung des Wortes also gibt es
\begin{aufzb}
\item der natürlichen Religionen \RWbet{mehre}, beinahe so viele, als es Menschen gibt.
\item \Hhat{Manche}{Mehrere} dieser Religionen können auch \RWbet{falsche} Sätze enthalten und einander widersprechen.
\end{aufzb}
\item Nicht eben so ist es mit derjenigen Religion, welche ich die natürliche Religion \RWbet{des ganzen Menschengeschlechtes} nenne. Unter dieser verstehe ich nämlich (zu Folge der \RWparnr{20}\ gegebenen Erklärung von dem Begriffe der Religion einer ganzen Gesellschaft) den Inbegriff bloß aller derjenigen religiösen Lehren, deren Wahrheit alle, oder doch fast alle Menschen, ohne ein göttliches Zeugniß, einzusehen glauben. Von dieser Religion, die ich zuweilen auch schlechtweg die \RWbet{natürliche} nenne, glaube ich behaupten zu dürfen,~\RWSeitenw{170}
\begin{aufzb}
\item \RWbet{daß sie nur eine einzige sey}, weil es nur einen einzigen Inbegriff von Lehren gibt, von denen gesagt werden kann, daß alle oder fast alle Menschen ihre Wahrheit ohne Dazwischenkunft eines göttlichen Zeugnisses einzusehen glauben.
\begin{RWanm}
Gegen diese Behauptung ließe sich höchstens einwenden, daß es dem Sprachgebrauche nach wohl schon erlaubt wäre, eine Lehre zur natürlichen Religion des Menschengeschlechtes zu zählen, wenn ihre Wahrheit nur von allen, oder doch fast allen Menschen \RWbet{eines gewissen Zeitalters} eingesehen wird. Und in diesem Falle ließe sich immerhin gedenken, daß auch die natürliche Religion der Menschheit zu verschiedenen Zeiten eine verschiedene gewesen sey, ja bei dem steten Fortschreiten, das ich bei unserem Geschlechte annehme, ließe sich hoffen, daß diese Religion an Vollständigkeit ihres Inhaltes von einem Jahrhunderte zum andern zunehmen werde.
\end{RWanm}
\item \RWbet{daß diese Religion nur lauter Wahrheiten enthalte}. Denn nach demjenigen, was \RWparnr{13}\ von der hohen Verlässigkeit der \RWbet{Urtheile des gemeinen Menschenverstandes} gesagt wurde, läßt sich kaum denken, daß eine Behauptung, deren Wahrheit alle, oder doch fast alle Menschen durch ihre bloße Vernunft einzusehen glauben, dennoch ein Irrthum seyn sollte, am Wenigsten, wenn sie von einer solchen Beschaffenheit ist, daß sie der Sinnlichkeit nicht schmeichelt.
\end{aufzb}
\item In dem gegenwärtigen Hauptstücke ist es nun meine Absicht, eine gedrängte Uebersicht der Lehren der natürlichen Religion aus dem Grunde aufzustellen, um in der Folge
\begin{aufzb}
\item die Unzulänglichkeit dieser Religion, und dadurch die Nothwendigkeit einer göttlichen Offenbarung desto verlässiger beurtheilen zu können; ingleichen
\item die hier entwickelten Grundsätze später bei Prüfung der Lehren der geoffenbarten Religion zu benützen.
\end{aufzb}
Aus diesem doppelten Zwecke begreift man, daß es uns hier nicht sowohl um die natürliche Religion dieses oder jenes einzelnen Menschen, als vielmehr um die natürliche Religion des menschlichen Geschlechtes selbst zu thun seyn könne. Nur diese also ist es, die ich hier darzustellen gedenke.~\RWSeitenw{171} Ich werde aber bei Weitem nicht alle Lehren derselben abhandeln, sondern nur solche ausheben, von denen es sich nicht schon von selbst verstehet, daß sie zum Inhalte dieser Religion gezählt werden müssen. Uebrigens werde ich bei dieser Gelegenheit auch einige der Lehren vortragen, die wirklich noch nicht so allgemein anerkannt sind, daß sie den Lehrsätzen der natürlichen Religion der Menschheit beigezählt werden dürften, die mir aber doch sehr einleuchtend scheinen, und in der Folge angewandt werden sollen, um die Vernunftmäßigkeit der Lehren des Katholicismus in ein desto helleres Licht zu stellen. Ich werde es endlich, damit meine Darstellung auch dem Zwecke der \RWbet{Wissenschaft} entspreche, bei mehren dieser Lehren versuchen, auch jenen eigentlichen Grund, auf dem ihre Wahrheit nach meiner Vorstellung beruhet, anzudeuten.
\end{aufza}

\RWpar{65}{Abheilungen, welche bei einem jeden systematischen Vortrage einer Religion gewöhnlich sind}
\begin{aufza} 
\item Jede Religion muß sowohl \RWbet{praktische} als \RWbet{theoretische} Sätze enthalten.
\begin{aufzb}
\item \RWbet{Praktische}; weil alle praktischen Meinungen, die Jemand annimmt, \dh\ alle Begriffe, die er sich von seinen Pflichten macht, recht eigentlich zu seiner Religion gehören. Denn es ist offenbar, daß es keine für die Tugend und Glückseligkeit des Menschen wichtigere Begriffe geben kann, als eben diese; und es ist sehr begreiflich, daß alle praktischen Sätze eine dem menschlichen Herzen natürliche Versuchung darbieten, sich entweder für oder wider sie zu bestimmen. (\RWparnr{20})
\item \RWbet{Theoretische.}
\begin{aufzc}
\item Einmal schon \RWbet{wegen der praktischen}; denn die meisten praktischen Sätze werden und können erst unter Voraussetzung gewisser rein theoretischer Sätze erkannt werden. Um seine Pflichten zu erkennen, muß der Mensch erst \RWbet{seine Verhältnisse zu Gott, zu}~\RWSeitenw{172}\ \RWbet{sich selbst, zu seinen Nebenmenschen und zu vielen andern Dingen} kennen gelernt haben, weil nach Verschiedenheit derselben auch seine Pflichten verschieden sind. Wer unter andern Verhältnissen lebt, der hat auch andere Pflichten. Nun sind aber Sätze, die unsere Verhältnisse ausdrücken, in sofern bloß theoretisch. Also wird jeder Religionsunterricht gewisse rein theoretische Sätze enthalten müssen; und dieß zwar nicht bloß als \RWbet{Hülfslehren}, die zum Beweise der religiösen Lehren erforderlich sind, sondern die meisten dieser theoretischen Sätze werden den \RWbet{eigentlichen Religionssätzen} beigezählt werden müssen, weil ihre Annahme oder Verwerfung nicht nur von großer Wichtigkeit ist, wegen der Folgen, die sich aus ihnen für unsere Pflichten ergeben, sondern weil diese Sätze eben um ihrer Folgen wegen meistens auch sittlicher Art sind.
\item Dann gibt es aber auch theoretische Sätze, aus denen zwar nicht neue Pflichten, doch manche \RWbet{Beweggründe} zur Erfüllung unserer Pflichten, \RWbet{oder Gründe des Trostes im Unglücke}, oder \RWbet{erfreuliche Hoffnungen} folgen. Auch alle solche Sätze haben einen wichtigen Einfluß auf unsere Tugend und Glückseligkeit, gehören also zur Religion, sofern sie überdieß sittlicher Art sind.
\end{aufzc}
\end{aufzb}
\item Bei einem systematischen Vortrage pflegt man die theoretischen Sätze von den praktischen zu sondern, und ihnen vorauszuschicken. Den Inbegriff jener nennt man \RWbet{Dogmatik} oder auch die \RWbet{theoretische Religionslehre}, den Inbegriff dieser \RWbet{Moral} oder die \RWbet{praktische Religionslehre}. Man schickt die Dogmatik der Moral voraus, weil man erst seine Verhältnisse kennen muß, bevor man aus ihnen seine Pflichten herleiten kann.
\item Die Lehren der Dogmatik pflegt man abermals und zwar auf mancherlei Art in gewisse Unterabtheilungen zu bringen, und mit eigenen Namen zu bezeichnen. So nennt man \zB\ die Lehren von Gott \RWbet{Theologie}, die von der Welt \RWbet{Kosmologie}, jene vom Menschen \RWbet{Anthropologie}, \udgl~\RWSeitenw{173}
\item Die Lehren der \RWbet{Moral} bringt man gewöhnlich unter \RWbet{zwei} Abtheilungen:
\begin{aufzb}
\item Die \RWbet{Ethik} oder \RWbet{Pflichtenlehre}, welche die Pflichten, die dem Menschen in Hinsicht auf verschiedene Gegenstände \RWbet{unmittelbar} obliegen, abhandelt; und
\item die \RWbet{Asketik} oder \RWbet{Tugendmittellehre}, die von den \RWbet{Mitteln} handelt, durch deren Gebrauch es der Mensch dahin bringen kann, seine Pflichten am Sichersten und Vollkommensten zu erfüllen. Da es nun eine Pflicht ist, Gebrauch von diesen Mitteln zu machen; so handelt auch die Asketik von gewissen Pflichten des Menschen, die sich jedoch von jenen in der Ethik behandelten dadurch unterscheiden, daß sie dem Menschen nur \RWbet{mittelbarer Weise}, nämlich nur wegen seiner übrigen Pflichten obliegen.
\end{aufzb}
\end{aufza}
\begin{RWanm}
Zur \RWbet{Asketik} gehört unter Anderm auch die \RWbet{Lehre von den Beweggründen zum Guten}, deren der Mensch sich bedienen oder auch nicht bedienen soll; eine Lehre, die Einige unter dem Namen der \RWbet{Motivenlehre} als einen eigenen \RWbet{Haupttheil} der Religionslehre ansehen.
\end{RWanm}

\RWabs{Erster Abschnitt}{Natürliche Dogmatik}
\RWpar{66}{Begriff Gottes}
\begin{aufza} 
\item Daß die \RWbet{Lehre von Gott} zur Religion, \dh\ zum Inbegriffe derjenigen Lehren gehöre, die einen Einfluß auf unsere Tugend und Glückseligkeit haben, und zugleich eine eigene Versuchung, sich entweder für oder wider sie zu bestimmen, darbieten, wird Niemand in Abrede stellen; Jeder wird vielmehr zugestehen, daß diese Lehre unter den theoretischen eine der \RWbet{wichtigsten} sey, mit deren Vortrage billig der Anfang gemacht wird.~\RWSeitenw{174}
\item Zuvörderst handelt es sich hier um die Festsetzung des \RWbet{Begriffes} von Gott; denn ohne diesen gehörig festgestellt zu haben, würde Alles, was ich in der Folge über sein Daseyn und seine Eigenschaften zu sagen habe, schwankend und undeutlich bleiben.
\item Es muß aber ein Begriff angegeben werden, der folgende zwei Beschaffenheiten hat:
\begin{aufzb}
\item Daß er, wenn auch nicht alle Eigenschaften, die etwa einem geübten Theologen gleich bei der Aussprache des Namens: Gott, beifallen, doch alle wesentlichen Merkmale Gottes schon als Bestandtheile enthalte, und somit ausschließlich \RWbet{nur dieses} Wesen allein zu seinem Gegenstande habe.
\item Daß er so einfach als möglich sey, und daß sich Beides, sowohl das Daseyn, als auch die Eigenschaften Gottes aus ihm am Leichtesten ableiten lassen.
\end{aufzb}
\begin{RWanm}
Das Erstere fordert die Ehrfurcht, die wir dem Worte schuldig sind; denn Worte, die einmal gewählt worden sind, um einen so ehrwürdigen Gegenstand zu bezeichnen, als es derjenige ist, zu dessen Bezeichnung in deutscher Sprache das Wort: \RWbet{Gott}, gewählt ist, müssen nie zur Bezeichnung eines minder ehrwürdigen Gegenstandes angewandt und dadurch entweihet werden. Darum muß denn auch ich in der hier aufzustellenden Erklärung einen Begriff mit dem Worte Gott verbinden, der wenn auch nicht eben in seinen \RWbet{Bestandtheilen}, doch nach \RWbet{den Eigenschaften}, die aus ihm gefolgert werden können, ganz mit demjenigen übereinstimmt, was man nach einem in unsern Tagen allgemein herrschenden Sprachgebrauche unter Gott verstehet.
\end{RWanm}
\item Dieser Forderung nun wird, wie ich glaube, genug gethan, wenn ich erkläre, daß ich mir unter Gott das \RWbet{unbedingt Wirkliche} denke.
\item Der Begriff des \RWbet{Wirklichen} braucht keine weitere Erläuterung; wohl aber jener des Unbedingten. \RWbet{Unbedingt} heißt mir dasjenige, was keine Bedingung hat. Der Begriff der \RWbet{Bedingung} aber und sein Correlat, der Begriff des \RWbet{Bedingten} sind zwei Begriffe, die eine große~\RWSeitenw{175}\ Aehnlichkeit mit den zwei Begriffen des \RWbet{Grundes} und der \RWbet{Folge} haben, ohne doch \RWbet{Einerlei} mit ihnen zu seyn. Jene sind nämlich \RWbet{weiter} als diese. Ein jeder Grund ist wohl auch eine Bedingung; aber nicht umgekehrt ist jede Bedingung auch schon ein Grund, und ein vollständiger Grund. Eben so ist eine jede Folge wohl etwas Bedingtes; aber nicht umgekehrt ist alles Bedingte auch eine Folge. So haben \zB\ die freien Handlungen wohl eine Bedingung (nämlich ein frei handelndes Wesen), aber keinen Grund; sie sind etwas Bedingtes, aber noch keine Folge. Eben so ist das Daseyn einer Feder wohl eine Bedingung zum Schreiben, aber noch nicht der vollständige Grund dazu, sondern nur ein Theil desselben. -- Allgemein besteht die \RWbet{Aehnlichkeit} zwischen Bedingung und Grund, Bedingtem und Folge darin, daß wie die Folge nicht ohne Grund, so das Bedingte nicht ohne Bedingung Statt finden kann; der \RWbet{Unterschied} aber darin, daß die Folge nothwendig ist, sobald der Grund gesetzt wird, das Bedingte dagegen, auch wenn die Bedingung gesetzt wird, noch immer wegbleiben kann.
\begin{RWanm}
Grund und Folge, wenn sie etwas Wirkliches (Existirendes) sind, pflegt man auch \RWbet{Ursache} und \RWbet{Wirkung} zu nennen. So heißen \zB\ die beiden Prämissen wohl der Grund, aber nicht die Ursache der Conclusion, wenn alle drei Wahrheiten nicht als existirende Gedanken, sondern als Wahrheiten an sich betrachtet werden. Dagegen die Erkenntniß der Prämissen heißt die Ursache von der Erkenntniß der Conclusion.
\end{RWanm}
\item Nach dieser Erklärung verstehe ich also unter Gott dasjenige Wirkliche, das nicht nur keinen Grund, sondern nicht einmal eine Bedingung seiner Wirklichkeit hat.
\item Und dieser Begriff, glaube ich, paßt auf kein anderes Wesen, als nur auf dasjenige, was wir uns allgemein unter dem Worte Gott denken. Allgemein nämlich verstehen wir unter Gott ein Wesen, das \RWbet{keine weitere Bedingung seines Daseyns hat}, das überdieß auch \RWbet{alle Vollkommenheiten}, namentlich Allwissenheit, Allmacht, Heiligkeit \usw\ in sich vereiniget. Dieß Letztere fordern wir zu dem Begriffe Gottes so strenge, daß wir denjenigen,~\RWSeitenw{176}\ der zwar zulassen wollte, daß es ein Wesen gebe, das keine weitere Bedingung seines Daseyns hat, der aber diesem Wesen Verstand und Willen abspräche, für einen \RWbet{Gottesläugner} erklären würden. Gleichwohl ist es, wie mir däucht, keineswegs nöthig, den Begriff der Allvollkommenheit, ja auch nur den eines Wesens, um wie viel weniger die Begriffe aller jener einzelnen Vollkommenheiten, der Allmacht, Allwissenheit \usw\ in die Erklärung des Begriffes von Gott schon als Bestandtheile aufzunehmen; denn Alles dieses ergibt sich aus dem bloßen Begriffe einer \RWbet{unbedingten Wirklichkeit} als Folge. Ein Wirkliches, das keine weitere Bedingung seiner Wirklichkeit hat, muß eben deßhalb ein Wesen, und zwar ein solches seyn, das alle Vollkommenheiten, Allmacht, Allwissenheit, \usw\ in sich vereiniget. Hätte ich also in meine Erklärung von Gott noch jene mehren Merkmale aufnehmen wollen, daß dieses Wirkliche ein \RWbet{Wesen} sey, und zwar ein solches, das alle möglichen Vollkommenheiten vereiniget \usw : so hätte ich besorgt, den Fehler der \RWbet{Ueberfüllung} zu begehen. \RWbet{Ueberfüllt} nämlich nenne ich einen Begriff, der aus Bestandtheilen zusammengesetzt ist, deren der Eine Beschaffenheiten aussagt, die schon eine Folge der übrigen sind. Die Aufstellung solcher überfüllter Begriffe betrachtet man aber insgemein als einen Fehler in einem Unterrichte, der auf strenge Wissenschaftlichkeit Ansprüche macht. So sieht man es durchgängig für einen Fehler an, wenn Jemand das \RWbet{Parallelogramm} als ein Viereck erklärt, dessen je zwei gegenüberstehende Seiten einander \RWbet{gleichlaufend} und \RWbet{gleich lang} sind; denn das Eine ist schon eine Folge von dem Anderen. Eben so fehlerhaft wäre es, Gott als das Wesen zu erklären, das eine unbedingte Wirklichkeit hat, und alle Vollkommenheiten in sich vereinigt.

\begin{RWanm}
Die hier versuchte Erklärung ist schon von mehren Weltweisen angenommen worden, nur daß sie statt des Ausdruckes, daß Gott keinen weitern \RWbet{Grund} seines Daseyns habe, meistens die Redensart gebrauchten, \RWbet{daß er den Grund seines Daseyns in sich selbst habe}; daher sie ihn auch wohl das \RWbet{von sich selbst} bestehende Wesen (\RWlat{ens a se}) nannten. Daß nun diese Redensart dort, wo es sich um bloße Gemeinverständlichkeit (Popularität) handelt, vorzuziehen sey, will ich gar nicht~\RWSeitenw{177}\ in Abrede stellen; wie ich denn dort nicht einmal mißbilligen würde, wenn man noch beisetzte, daß dieses Wesen der Inbegriff aller Vollkommenheit sey. Wenn es sich aber, wie es hier meine Absicht ist, um eine möglichst deutliche Angabe der \RWbet{Bestandtheile} eines Begriffes handelt; so dürfte der obige Ausdruck zweckmäßiger seyn. Zwar sind wir es Alle und sehr mit Recht gewohnt, bei einem jeden Gegenstande nach dem Grunde seines Daseyns zu fragen; um also einen solchen auch bei Gott anzugeben, hat man gesagt, er habe den Grund seines Daseyns \RWbet{in sich}. Dieses ist aber in der That unrichtig gesprochen; denn es ist überhaupt ungereimt, daß etwas Grund von sich selbst seyn könne, sondern der Grund und seine Folge, und eben so die Bedingung und das Bedingte sind immer zwei von einander verschiedene Gegenstände. Man sollte also, statt zu sagen, daß der Grund von Gottes Daseyn in ihm selbst liege, eigentlich sagen, daß es gar keinen Grund, ja nicht einmal eine Bedingung seines Daseyns gebe. Durch jenen unrichtigen Ausdruck gibt man dem Gottesläugner nur Anlaß, zu behaupten, daß der Gedanke Gottes schon \RWbet{in sich selbst} einen Widerspruch enthalte. Woher es aber gekommen, daß man in diesen Fehler verfiel, ist wohl nicht schwer zu errathen. \RWbet{Der Satz vom Grunde}, den man aus Mißverstand häufig zu weit ausdrückte, nämlich, \RWbet{daß nichts ohne Grund sey}, gab die Veranlassung, daß man auch von dem Daseyn Gottes einen Grund nachweisen wollte; und weil man denn einsah, daß dieser Grund unmöglich in etwas außerhalb Gottes gesetzt werden könne: so glaubte man am Besten zu thun, wenn man erkläre, er liege in Gott selbst.
\end{RWanm}
\end{aufza}

\RWpar{67}{Daseyn Gottes}
\begin{aufza} 
\item Die Behauptung, daß ein Gott sey, \dh\ daß etwas Wirkliches sey, das keine weitere Bedingung seiner Wirklichkeit hat, kann als ein völlig ausgemachter Lehrsatz der natürlichen Religion des Menschengeschlechtes angesehen werden.
\item Diese Wahrheit wird ja von Allen ohne Ausnahme zugegeben. Wohl haben mehre Weltweise noch gezweifelt, ob das unbedingte Wirkliche auch ein mit Verstand und Willen begabtes Wesen sey; wohl haben Einige diese Welt selbst für das Unbedingte gehalten: aber fast Niemand hat noch~\RWSeitenw{178}\ bezweifelt, daß es ein Unbedingtes überhaupt gebe.
\item Also nicht, um zu einer größern Gewißheit zu gelangen bei einer Wahrheit, die durch den Ausspruch des bloßen gesunden \Ahat{Menschenverstandes}{Mehrverstandes} entschieden genug ist, sondern nur um die Gründe, aus welchen die Vernunft diese Wahrheit erkennt, zu einem, wo möglich, deutlicheren Bewußtseyn zu erheben, versuche ich nachstehenden Beweis für diese Wahrheit.
\begin{aufzb}
\item \RWbet{Es gibt doch überhaupt einiges Wirkliche}. Dieses mein eigenes Urtheil, als eine \RWbet{Erkenntniß}, oder als ein \RWbet{Gedanke} betrachtet, ist schon selbst etwas \RWbet{Wirkliches}.
\item Ich hebe nun irgend Eines von diesen Wirklichen, \zB\ das Wirkliche $A$, heraus, und frage denjenigen, der an dem Daseyn eines unbedingten Wirklichen noch zweifelt, wofür er das Wirkliche $A$ erklären wolle; ob für unbedingt, oder nicht? Thut er das Erstere; so gestehet er selbst, daß es ein unbedingt Wirkliches, \dh\ einen Gott gebe.
\item Will er das Wirkliche $A$ nicht für unbedingt erklären: so verlange ich, daß er gleich alle die wirklichen Dinge $A$, $B$, $C$, \textsymmdots , welche nicht unbedingt sind, in einen Inbegriff zusammenfasse. Dieses muß wenigstens im Gedanken möglich seyn, gesetzt auch, daß ihre Menge unendlich wäre.
\item Ich frage nun ferner, wofür er diesen Inbegriff \RWbet{aller} bedingten Wirklichkeit, der gewiß selbst auch etwas Wirkliches ist, erkläre, ob für ein unbedingtes oder bedingtes Wirkliches? Thut er das Erstere, so gesteht er uns abermals ein unbedingtes Wirkliches zu.
\item Im widrigen Falle aber erinnere ich, daß jedes Wirkliche, welches nicht unbedingt ist, das Daseyn eines andern Wirklichen voraussetzt, durch welches es bedingt ist. Auch der Inbegriff aller bedingten Wirklichkeit also setzt noch~\RWSeitenw{179}\ ein anderes Wirkliches, durch welches er eben bedingt ist, voraus.
\item Dieß andere Wirkliche nun muß etwas Unbedingtes seyn; denn wäre es bedingt, so würde es eben darum mit zu dem Inbegriff aller bedingten Wirklichkeit gehören.
\item Sonach gibt es in einem jeden Falle ein Wirkliches, welches unbedingt ist, \dh\ einen Gott.
\end{aufzb}
\begin{RWanm}
Die Annahme d, daß der Inbegriff aller bedingten Wirklichkeit selbst etwas Unbedingtes wäre, gestatte ich dem Gegner \RWbet{nur vor der Hand}, weil ich es zum Beweise des gegenwärtigen Satzes nicht nöthig habe, sie zu widerlegen. In der Folge werde ich gleichwohl die Unrichtigkeit dieser Behauptung, wenn unter diesen Dingen \RWbet{Substanzen} verstanden werden sollen, zeigen.
\end{RWanm}
\end{aufza}

\RWpar{68}{Fehlerhaftigkeit eines sehr gewöhnlichen Schlusses für das Daseyn Gottes}
\begin{aufza}
\item Obwohl hier nicht der Ort ist, die verschiedenen Beweise zu prüfen, die man bisher für das Daseyn Gottes vorgebracht hat; so lohnt es sich doch der Mühe, vor Einem \RWbet{Fehlschlusse zu warnen}, den man nicht nur in den Schriften der Gelehrten antrifft, sondern auch im \RWbet{gemeinen} Leben täglich zu begehen pflegt, wenn man das Daseyn Gottes darzuthun sucht. Man muß Gebildete vor diesem Fehlschlusse warnen, damit sie den Gegnern der Religion keine Gelegenheit geben, sie zu beschuldigen, daß sich ihr Glaube an Gott auf einen Irrthum gründe.
\item Wir erfahren es, sagt man, an so viel tausend und tausend Dingen um uns her, daß sie \RWbet{veränderlich} sind, daß sie nicht von jeher da waren, sondern erst in der Zeit \RWbet{entstanden} sind, und immer eine \RWbet{Ursache} ihrer Entstehung hatten. Wir können also schließen, daß dieses auch von den übrigen Dingen, die wir nicht wirklich entstehen sahen, gelte; daß folglich jedes aus ihnen eine gewisse Ursache seiner Entstehung, entweder alle dieselbe, oder jedes eine eigene, habe. Diese Ursache mag allenfalls wieder entstanden seyn, und eine andere Ursache voraussetzen: so kann~\RWSeitenw{180}\ dieß gleichwohl nicht in's Unendliche so fortgehen; denn keine Reihe von Gründen kann in's Unendliche gehen. Es muß also irgend eine \RWbet{erste Ursache} geben, die keine weitere Bedingung ihres Daseyns hat, die folglich \RWbet{Gott ist}.
\item In dieser Kette von Schlüssen ist nur der einzige Satz, \RWbet{daß eine Reihe von Gründen} (oder auch von Bedingungen) \RWbet{nie in's Unendliche fortgehen könne}, wie ich mir vorstelle, irrig; obgleich es, von \RWbet{Aristoteles} an, sehr viele Weltweise gegeben, die ihn für eine ausgemachte Wahrheit ansahen. Ich glaube nämlich, es gebe allerdings \RWbet{auch unendliche Reihen von Bedingungen oder Gründen}, oder mit andern Worten: es gebe Reihen, bei denen man, anzufangen von einem Gliede $A$, nach seiner Bedingung, und nach der \RWbet{Bedingung} dieser \RWbet{Bedingung} und so fort ohne Ende fortfragen kann, ohne je auf ein Glied zu kommen, das nur \RWbet{Bedingung} und nicht mehr ein \RWbet{Bedingtes}, oder, wie man zu sagen pflegt, ein \RWbet{erstes Glied}, oder der \RWbet{Anfang} der Reihe wäre. Solche unendliche Reihen von Bedingungen finden sich, wie ich mir vorstelle, allenthalben; \zB\ schon bei einer jeden noch so geringen \RWbet{Bewegung} eines Körpers. Man denke sich einen materiellen Körper, oder noch besser, nur einen einzigen materiellen Atom $A$, der sich Anfangs in einem gewissen Orte $\alpha$ befindet, dann aber (etwa durch einen Stoß von Außen) in Bewegung gesetzt, nach Verlauf einiger Zeit im Orte $\beta$ anlangt. Um von $\alpha$ nach $\beta$ zu kommen, hat dieser Atom eine gewisse \RWbet{Linie} beschreiben müssen. Lasset uns annehmen, daß es eine gerade gewesen. Bekanntlich gibt es nun in einer jeden Linie unendlich viele Puncte, und in allen diesen muß sich der Atom befunden haben, bevor er aus $\alpha$ nach $\beta$ eintreffen konnte. Es sey nun $\mu$ einer von diesen Puncten, und $\lambda$ einer, der \RWbet{vor} ihm, \dh\ näher an $\alpha$ zu liegt: so müßte der Atom auch in $\lambda$ gewesen seyn, bevor er in $\mu$ noch anlangen konnte, und erst durch $\lambda$ hindurch gegangen seyn. Wir können also sagen, daß des Atoms Durchgang durch $\lambda$ eine \RWbet{Bedingung} seines Eintretens in den Ort $\mu$ gewesen. Nun gibt es aber auch zwischen $\lambda$ und $\alpha$ noch einen Punct $\kappa$, und zwischen $\kappa$ und $\alpha$ noch~\RWSeitenw{181}\ einen Punct $\iota$, \usw\ Ja es gibt überhaupt der Puncte, deren der Eine \RWbet{vor} dem Andern liegt, obgleich sie alle noch \RWbet{hinter} $\alpha$ stehen, unendlich viele; und der Durchgang durch jeden vorhergehenden ist als eine Bedingung des Durchganges durch den folgenden anzusehen. Sonach ist das Anlangen unsers Atoms in $\beta$ als ein Ereigniß zu betrachten, das der Bedingungen unendlich viele vor sich hat. Es gibt also \RWbet{Reihen von Bedingungen, die in's Unendliche fortgehen}.
\item Hieraus folgt aber nicht das Geringste, das den im vorigen § gegebenen Beweis für das Daseyn Gottes umstoßen könnte. Denn in diesem haben wir uns, wie man sieht, auf diese Voraussetzung gar nicht gestützt. Auch darf man nicht glauben, daß eine Reihe von Bedingungen, die in's \RWbet{Unendliche} fortgeht, niemals ein \RWbet{erstes Glied} haben \RWbet{könne}. Denn daraus allein, daß man, wenn man, von einem Gliede derselben anzufangen, nach seiner Bedingung, und nach der Bedingung dieser Bedingung, und so beständig fortfragt, nie auf ein Glied kommt, das keine weitere Bedingung hat, folgt noch nicht, daß dieses gar nicht vorhanden seyn müsse. Das obige Beispiel zeigt uns das Gegentheil. Die Reihe von Bedingungen, welche die Ankunft des Atoms $A$ in $\beta$ hat, ist unendlich, und hat doch gleichwohl ein erstes Glied; dieses ist nämlich das Daseyn des Atoms im Orte $\alpha$. Denn wenn die Bewegung, die wir annahmen, vom Orte $\alpha$ anfing; so setzt das Daseyn des Atoms in $\alpha$ keinen Durchgang desselben durch einen andern Ort voraus; sein Daseyn in $\alpha$ ist also nur Bedingung von seinem Durchgange durch alle übrigen, selbst aber nichts Bedingtes, \dh\ \RWbet{es ist das erste Glied in der unendlichen Reihe von Bedingungen}, auf denen die Beschreibung der Linie $\alpha\beta$ beruhet.
\end{aufza}

\RWpar{69}{Unbedingte Nothwendigkeit Gottes}
\begin{aufza} 
\item Ich nenne \RWbet{nothwendig} alles dasjenige, dessen Nichtseyn unmöglich ist, \dh\ mit irgend einer reinen Be\RWSeitenw{182}griffswahrheit im Widerspruche stehet. Hat die Unmöglichkeit eines gewissen Gegenstandes einen Grund (oder doch eine Bedingung), der selbst nicht nothwendig ist (dessen Nichtseyn also keiner Begriffswahrheit widerspricht): so nenne ich diesen Gegenstand bloß \RWbet{bedingt nothwendig}. So ist \zB\ die Wirkung einer Ursache nur bedingt nothwendig; indem ihr Nichtseyn nur dann unmöglich ist, wenn ihre Ursache gesetzt wird. -- \RWbet{Unbedingt nothwendig} dagegen heißt mir ein Gegenstand, dessen Nichtseyn ohne Bedingung unmöglich ist.
\begin{RWanm}
Wenn ich den Begriff der \RWbet{Nothwendigkeit} hier recht erklärt habe: so läßt er sich eigentlich nur auf \RWbet{Wirklichkeiten}, \dh\ auf Gegenstände, die \RWbet{existiren}, anwenden; nur, was ein \RWbet{Daseyn} hat, was ist, kann \RWbet{Nothwendigkeit} haben. Dem widerspricht aber, daß man zuweilen auch Gegenständen, die gar kein Daseyn haben, nämlich den \RWbet{reinen Begriffswahrheiten} eine Art von Nothwendigkeit beilegt. So sagt man \zB\ die Wahrheit, daß die Summe aller Winkel in einem Dreiecke zwei rechte beträgt, sey eine \RWbet{nothwendige Wahrheit}. Ich meine, daß man den Ausdruck nothwendig hier in einer uneigentlichen Bedeutung nehme.
\end{RWanm}
\item Auch die Wahrheit, \RWbet{daß Gott nothwendig, und zwar unbedingt nothwendig sey}, gehört zu den entschiedenen Lehrsätzen der natürlichen Religion des menschlichen Geschlechtes. Wir finden nicht, daß irgend ein Weltweiser, der an das Daseyn Gottes, \dh\ an das Daseyn eines unbedingt Wirklichen, geglaubt, behauptet hätte, daß dieses Wirkliche nicht mit \RWbet{Nothwendigkeit}, und zwar mit \RWbet{unbedingter} Nothwendigkeit bestehe.
\item Der Schluß, durch den die Vernunft diese Wahrheit erkennt, dürfte folgender seyn. Alles Wirkliche, das nicht \RWbet{nothwendig} ist, \dh\ dessen Nichtseyn möglich ist, muß durch \RWbet{Freiheit} wirklich seyn, und setzt also das Daseyn eines frei handelnden Wesens als Bedingung voraus; es ist daher ein \RWbet{bedingtes Wirkliches}. Das unbedingt Wirkliche also, oder Gott, ist mit \RWbet{Nothwendigkeit} wirklich. Diese Nothwendigkeit aber muß eine \RWbet{unbedingte} seyn, sonst hätte Gott abermals eine bedingte Wirklichkeit.~\RWSeitenw{183}
\end{aufza}

\RWpar{70}{Substanzialität Gottes}
\begin{aufza} 
\item Obgleich die Gelehrten noch sehr darüber im Streite sind, wie der Begriff einer \RWbet{Substanz} zu erklären sey: so ist es doch leicht, Jeden, der die Bedeutung des Wortes noch nicht versteht, damit bekannt zu machen, wenn dazu nicht verlangt wird, daß er sich diese Vorstellung zu einem ganz \RWbet{deutlichen Bewußtseyn} bringe. Zu diesem Zwecke brauche ich nämlich nur zu erinnern, daß alles \RWbet{Wirkliche} zu einer von folgenden zwei Arten gehöre, daß es entweder \RWbet{Substanz} oder \RWbet{Adhärenz} sey, daß \RWbet{Adhärenz} bloß ein solches Wirkliche heiße, das sich \RWbet{an} einem Andern als eine Beschaffenheit desselben befindet. Was sich nun \RWbet{nicht an} einem Andern befindet, sondern, wie man dieß auszudrücken pflegt, \RWbet{für sich bestehet}, das heißt \RWbet{Substanz} oder \RWbet{Wesen}. Noch deutlicher wird die Bedeutung dieser Worte, wenn wir den Lehrsatz anführen, daß Alles, was anfängt oder aufhört, nur eine Adhärenz sey, während jede Substanz etwas Beständiges ist, das weder anfangen, noch vergehen kann, sondern, so ferne es einmal ist, zu aller Zeit seyn muß.
\item Daß nun Gott \RWbet{eine Substanz sey}, ist abermals eine Behauptung, die man den ausgemachtesten Lehrsätzen der natürlichen Religion des Menschengeschlechtes beizählen darf. Denn sicher hat noch Niemand, der das Daseyn eines unbedingt Wirklichen zugab, daran gezweifelt, daß dieses Wirkliche eine Substanz seyn müsse.
\item Auch der \RWbet{Grund}, auf dem diese Wahrheit beruht, ist wohl leicht einzusehen. Denn jede \RWbet{Adhärenz} setzet das Daseyn einer Substanz, an der sie sich befindet, als eine Bedingung zu ihrem eigenen Daseyn voraus. Das unbedingt Wirkliche kann also keine Adhärenz seyn, und folglich bleibt nichts übrig, als daß es eine Substanz sey.
\end{aufza}

\RWpar{71}{Unabhängigkeit Gottes}
\begin{aufza} 
\item Jede Substanz, ja jeder Gegenstand überhaupt hat gewisse Beschaffenheiten, die man auch \RWbet{Adhärenzen} nennet.~\RWSeitenw{184}\ Ueber den Begriff, den dieses Wort bezeichnet, habe ich mich bereits im vorigen § verständigt. Beschaffenheiten, die einem aus mehren Theilen $A$, $B$, $C$, $D$, zusammengesetzten \RWbet{Ganzen} $M$ zukommen, heißen in Beziehung auf die einzelnen Theile $A$, $B$, $C$, $D$, \RWbet{Verhältnisse derselben} unter einander. So ist \zB\ die Gleichheit zweier Seiten in einem Dreiecke in Hinsicht auf das Dreieck eine \RWbet{Beschaffenheit} desselben, in Hinsicht auf die einzelnen Seiten aber nennt man sie ein \RWbet{Verhältniß} derselben unter einander. Nicht selten aber pflegt man auch dasjenige, was eigentlich ein bloßes Verhältniß eines Gegenstandes $A$ zu einem andern $B$ ist, eine Beschaffenheit des $A$ zu nennen.\par
So sagt man \zB , es wäre eine Beschaffenheit der Seite $ca$, daß sie der Seite $ab$ \RWbet{gleich} sey; oder es wäre eine Beschaffenheit der Linie $ca$, daß sie die Länge eines Zolles hat, obwohl dieß eigentlich bloße \RWbet{Verhältnisse} der Linie $ca$ zur Linie $cb$ oder zu einem Zolle sind. Um solche Beschaffenheiten eines Gegenstandes, die es nur \RWbet{uneigentlicher} Weise heißen, und die im Grunde bloße \RWbet{Verhältnisse} desselben zu einem andern sind, von den eigentlichen zu unterscheiden, nennt man die ersteren auch wohl \RWbet{äußere}, die letzteren dagegen \RWbet{innere Beschaffenheiten} oder \RWbet{Eigenschaften}. Zu diesen innern Beschaffenheiten oder zu den Eigenschaften eines Wesens gehören auch seine \RWbet{Kräfte}. Diese sind nämlich solche Beschaffenheiten desselben, vermöge deren es irgend etwas Wirkliches hervorbringt, oder hervorbringen kann. So heißt man \zB\ die Beschaffenheit eines Wesens, vermöge deren es Vorstellungen in sich hervorbringen kann, seine \RWbet{Vorstellungskraft}; und die Beschaffenheit, vermöge deren es gewisse Veränderungen in andern Wesen hervorbringen kann, seine \RWbet{Veränderungskraft}; \usw\ Wie man aber das Wort \RWbet{Beschaffenheit} zuweilen uneigentlich gebraucht, so auch das Wort \RWbet{Kraft}. Wenn nämlich ein Wesen $A$ nur in Vereinigung mit einem andern $B$ im Stande ist, eine gewisse Wirkung $x$ zu erzeugen: so sollte man die Möglichkeit dieser Wirkung eigentlich eine \RWbet{Kraft der Wesen $A$ und $B$ in Vereinigung} nennen; man schreibt sie aber sehr oft dem Einen der beiden Wesen, demjenigen, von dem man sich vorstellt,~\RWSeitenw{185}\ daß es den größten Antheil bei dieser Wirkung habe, allein zu. So sagt man, wir haben die Kraft zu \RWbet{schreiben}, was wir doch eigentlich nicht allein, sondern nur mittelst einer Feder und anderer Schreibmaterialien vermögen. Zum deutlicheren Unterschiede kann man daher die Kräfte eines Wesens, die es in eigentlicher Bedeutung sind, vermöge deren es also gewisse Wirkungen für sich allein hervorbringt, die \RWbet{eigentlichen} oder die \RWbet{wahren Kräfte} desselben nennen. Wenn wir nun sagen, ein Wesen $A$ sey von einem andern $B$ \RWbet{unabhängig}: so wollen wir hiedurch nichts Anderes zu verstehen geben, als daß weder das \RWbet{Daseyn}, noch irgend eine der \RWbet{eigentlichen} oder \RWbet{wahren Kräfte} des Wesens $A$ von dem Wesen $B$ \RWbet{abhängig, \dh\ bedingt} durch dasselbe sey. Und wenn es ein Wesen gibt, das durchaus von gar keinem andern abhängig ist, \dh\ dessen Daseyn sowohl, als alle eigentlichen oder wahren Kräfte desselben durch kein anderes Wesen bedingt sind: so werden wir es schlechtweg ein \RWbet{unabhängiges} nennen.
\item Es ist nun abermals ein Lehrsatz der natürlichen Religion des Menschengeschlechtes, daß Gott selbst ein solches ganz unabhängiges Wesen sey. Denn noch Niemand hat sich beikommen lassen, das Gegentheil zu behaupten, und zu sagen, daß entweder das Daseyn, oder nur irgend eine von Gottes eigentlichen Kräften durch ein anderes Wesen bedingt sey.
\item Lasset uns nun noch den Beweis dieser Wahrheit versuchen! Daß Gottes \RWbet{Daseyn unbedingt sey}, liegt schon in seinem Begriffe; daß aber auch keine von seinen eigentlichen Kräften durch ein Wesen außer ihm bedingt sey, erhellet, däucht mir, aus Nachstehendem. Weil zu den eigentlichen Kräften eines Wesens nur solche gezählt werden sollen, durch die es im Stande ist, etwas Wirkliches für sich allein hervorzubringen: so ist ein jedes Wesen im Grunde nichts Anderes, als dasjenige Etwas, welches den Grund zur Möglichkeit aller dieser Kräfte enthält. Behaupten also, daß irgend eine der eigentlichen Kräfte eines Wesens durch ein anderes Wesen bedingt sey, heißt eben so viel, als sagen, daß dieses Wesen \RWbet{selbst} durch jenes andere bedingt sey. Bei~\RWSeitenw{186}\ Gott, der durch kein anderes Wesen bedingt ist, kann also auch keine seiner eigentlichen Kräfte durch ein anderes Wesen bedingt seyn; er ist daher von jedem unabhängig.
\end{aufza}

\RWpar{72}{Unveränderlichkeit Gottes}
\begin{aufza} 
\item Man sagt von einem Gegenstande, daß er sich ändere, wenn ihm gewisse \RWbet{einander widersprechende innere Beschaffenheiten zukommen}. Dieses ist nach der Erklärung des \RWparnr{62}\ \no\,8 immer nur unter der Bedingung einer verschiedenen \RWbet{Zeit} möglich; daher man auch sagt, daß jede Veränderung nur in der Zeit geschehe. Wenn es dagegen ein Wesen gibt, dem unter keiner Bedingung (nämlich auch nicht unter der einer verschiedenen Zeit) gewisse einander widersprechende innere Beschaffenheiten beigelegt werden können: so sagt man, dasselbe sey \RWbet{unveränderlich}.
\begin{RWanm}
Nur auf die \RWbet{inneren Beschaffenheiten} erstreckt sich der Begriff der \RWbet{Unveränderlichkeit} eines Wesens; denn Niemand sagt, daß sich ein Wesen verändert habe, wenn sich bloß eine seiner \RWbet{äußern} Beschaffenheiten oder Verhältnisse geändert hat, \dh\ wenn es bloß einige einander widersprechende \RWbet{Verhältnisse} zu andern Dingen gibt, die man demselben (unter der Bedingung einer verschiedenen Zeit) beilegen kann. So sagen wir nicht, daß sich ein Mensch geändert habe, wenn er einen seiner Verwandten durch den Tod verliert; denn der Besitz dieses Verwandten ist keine innere Beschaffenheit des Menschen, sondern ein bloßes Verhältniß desselben zu andern Wesen.
\end{RWanm}
\item In dieser Bedeutung muß man den Begriff der Unveränderlichkeit auch nehmen, wenn man ihn Gott beilegen will. Denn in den äußern Beschaffenheiten Gottes, \dh\ in seinen Verhältnissen zu uns und andern Wesen, ändert sich Manches, weil ein Verhältniß sich ändert, sobald nur Ein Glied desselben sich ändert. So stehet Gott \zB\ heute in dem Verhältnisse eines \RWbet{Gesetzgebers}, morgen in dem eines \RWbet{Richters} zu uns; heute straft er, morgen lohnt er; vor Jahrhunderten gab er uns jene, jetzt wieder diese Gebote; Alles, nicht weil er sich geändert hat, sondern weil wir Menschen uns ändern.~\RWSeitenw{187}
\item Auch die Behauptung der Unveränderlichkeit Gottes kann man den sichern Lehrsätzen der natürlichen Religion des menschlichen Geschlechtes beizählen; denn alle Weltweisen, wenigstens neuerer Zeit, haben diese Wahrheit erkannt, wenn sie gleich in der Art, sie zu beweisen, von einander abweichen.
\item Meinem Dafürhalten nach wäre diese Beschaffenheit Gottes etwa so darzuthun. Auf eben die Art, wie ich im vorigen § erwies, daß keine von Gottes \RWbet{eigentlichen Kräften} durch ein Wesen außerhalb seiner bedingt sey, läßt sich auch darthun, daß \RWbet{keine der innern Beschaffenheiten Gottes} sich ändere. Die sämmtlichen inneren Beschaffenheiten Gottes müssen nämlich aus seinem bloßen \RWbet{Begriffe} bestimmbar seyn. Denn im entgegengesetzten Falle, wenn sich nicht aus dem Begriffe Gottes allein entscheiden ließe, ob er eine gewisse Beschaffenheit besitze oder nicht; müßte, weil doch irgend ein Entscheidungsgrund für diese Frage vorhanden seyn muß, derselbe in einem \RWbet{andern} Wesen liegen; und so müßten einige von Gottes inneren Beschaffenheiten in einem \RWbet{andern} Wesen gegründet seyn. Wenn aber alle innern Beschaffenheiten Gottes aus seinem bloßen Begriffe bestimmbar sind: so ist es nicht möglich, ihm unter irgend einer Bedingung (selbst unter Voraussetzung einer verschiedenen Zeit) einander widersprechende Beschaffenheiten beizulegen; und also ist er schlechterdings unveränderlich.
\begin{RWanm}
Nicht also ist es der Fall bei einem \RWbet{bedingten} Wesen; bei einem solchen kann man nicht alle inneren Beschaffenheiten desselben aus dem Begriffe, \RWbet{daß es bedingt sey}, ableiten; sondern mehre derselben lassen sich erst aus den Beschaffenheiten des Einen oder der mehren anderen Wesen, durch die es bedingt ist, beurtheilen.
\end{RWanm}
\end{aufza}

\RWpar{73}{Einheit Gottes}
\begin{aufza}
\item Unter der \RWbet{Einheit} Gottes verstehet man, daß es nur eine einzige Substanz gebe, die \RWbet{Gott} ist, \dh\ die keine Bedingung ihrer Wirklichkeit hat.
\item Auch diese Wahrheit glaube ich den Lehrsätzen der natürlichen Religion des menschlichen Geschlechtes beizählen~\RWSeitenw{188}\ zu dürfen. Zwar scheint es, als widerspräche ihr der fast bei allen Völkern der Erde vorfindliche \RWbet{Polytheismus}, ein Glaube an mehrere sogenannte Gottheiten, deren jeder man doch gewiß auch eine eigene Substanz beigelegt haben wird. Allein hier müssen wir uns nicht durch den \RWbet{Namen} irre führen lassen. Was die heidnischen Völker \RWbet{Gottheiten} nannten, darunter verstanden sie nie oder selten \RWbet{Wesen, die keinen weitern Grund ihres Daseyns hätten}; denn sie erzählen uns ja vom Ursprung dieser Götter, von ihrer Abstammung \usw\ Am Ende nehmen sie aber fast alle \RWbet{irgend ein einziges letztes Grundwesen} an, von welchem diese Götter und Alles, was ist, seinen Urprung habe. Nur dieses Wesen ist es, was wir in unserer Bedeutung des Wortes \RWbet{Gott} nennen. Und so kann man sagen, daß wohl auch alle polytheistischen Völker im Grunde nur einen \RWbet{einzigen} Gott geglaubt, daß aber ihr Fehler gewesen, über der Verehrung der vielen \RWbet{untergeordneten Götter} (bloßer Geschöpfe) des wahren Gottes beinahe vergessen zu haben.
\item So einleuchtend es aber auch für jeden Nachdenkenden ist, daß nur \RWbet{Ein} wahrer Gott seyn könne; so schwer ist es, sich des Grundes, aus dem die Vernunft dieses Urtheil fällt, deutlich bewußt zu werden. Im vorigen § wurde bereits gezeigt, daß alle inneren Beschaffenheiten Gottes aus seinem bloßen \RWbet{Begriffe} herleitbar seyn müssen. Daraus ergibt sich nun, daß, wenn es mehre Götter gäbe, alle von durchaus gleichen innern Beschaffenheiten (allmächtig, höchst weise, heilig \usw ) gedacht werden müßten. Ließe sich also \RWbet{Leibnitzen's} sogenannter Grundsatz \RWbet{von der Einerleiheit des Nichtzuunterscheidenden}, \dh\ der Satz erweisen, \RWbet{daß es nicht mehre einander durchaus gleiche Gegenstände geben könne}; so würde schon daraus die \RWbet{Einheit} Gottes folgen. Doch warum sollte nicht auch folgender bereits oft angeführte Beweis für die Einheit Gottes genügen? Es gibt nur einen Gott, weil die Voraussetzung, daß es der Götter mehre gäbe, auf einen Widerspruch führet. Sollte es nämlich mehre Götter geben; so müßte, da sie einander alle durchgängig gleich seyn müssen, jeder von ihnen dasselbe vermögen, und auch dasselbe wirken.~\RWSeitenw{189}\ Dieselbe Welt also, die als die Wirkung des Einen zu denken ist, müßte auch als die Wirkung des Andern gedacht werden, was doch ungereimt ist.
\end{aufza}

\RWpar{74}{Allvollkommenheit Gottes}
\begin{aufza}
\item Ich nenne ein Wesen \RWbet{allvollkommen}, wenn es alle Kräfte, die neben einander möglich sind, und jede in jenem höchsten Grade, in welchem sie neben den übrigen und bei vorausgesetzter Unabhängigkeit dieses Wesens möglich ist, vereinigt. Andere haben ein solches Wesen auch das \RWbet{allervollkommenste} oder \RWbet{allerrealeste} genannt.
\item Obgleich der Begriff der Allvollkommenheit ein etwas \RWbet{schwer} zu fassender Begriff ist, und diese Eigenschaft Gottes von ungebildeten Leuten nicht aufgefaßt wurde: so haben doch \RWbet{Gelehrte} jederzeit erkannt, daß Gott, das unbedingte Wesen, auch allvollkommen seyn müsse. Ja diese Beschaffenheit Gottes leuchtete ihnen so deutlich ein, und stellte sich ihnen bei jedem Gedanken an Gott so lebhaft vor Augen, daß Viele den Begriff der \RWbet{Allvollkommenheit} für durchaus einerlei mit dem \RWbet{Begriffe von Gott} selbst hielten, wie sie denn eben deßhalb die \RWbet{Erklärung} aufstellten, daß unter Gott das Wesen zu verstehen sey, das alle Vollkommenheiten vereinigt. Hieraus erhellet, daß wir auch diese Beschaffenheit Gottes den gewissen Lehrsätzen der natürlichen Religion des Menschengeschlechtes beizählen dürfen.
\item Weltweise, die von der so eben erwähnten Erklärung Gottes, daß er das allervollkommenste Wesen sey, ausgingen, hatten freilich nicht nöthig zu beweisen, daß Gott diese Eigenschaft habe; sondern sie mußten bloß darthun, daß es ein solches Wesen, wie sie Gott nennen, gebe. Mir dagegen, der ich in meiner Erklärung bloß sagte, daß ich mir unter Gott dasjenige Wirkliche denke, das keine weitere Bedingung seiner Wirklichkeit hat, liegt es jetzt ob zu beweisen, daß dieses Wirkliche auch Allvollkommenheit habe. Es scheint dieß auf folgende Weise möglich. Ich suche darzuthun, daß ein jedes Wesen, welches nicht allvollkommen ist, ein abhängiges Wesen seyn müsse; woraus sich denn von~\RWSeitenw{190}\ selbst ergeben wird, daß Gott, weil er nicht abhängig ist, allvollkommen seyn müsse. Wenn irgend ein Wesen nicht allvollkommen ist, \dh\ wenn es entweder nicht alle Kräfte hat, die neben einander möglich sind, oder zwar alle, aber doch nicht in jenem höchsten Grade, in dem sie neben einander möglich sind: so muß sich ein \RWbet{Grund} hievon angeben lassen. Dieser kann nicht in dem Wesen selbst liegen; denn wenn er in ihm liegen sollte, so müßte er in seinen Kräften liegen, \dh\ die Kräfte, welche das Wesen besitzt, müßten den Grund enthalten, weßhalb es einige andere entweder gar nicht, oder doch nicht in dem zur Allvollkommenheit verlangten Grade hat. Dieß könnte nur dann geschehen, wenn die ersteren, \dh\ die Kräfte, welche das Wesen hat, mit jenen andern oder mit dem zur Allvollkommenheit verlangten Grade derselben in einem Widerspruche ständen; dann aber dürften wir sie (meiner Erklärung zu Folge) gar nicht zur Allvollkommenheit verlangen, indem wir zu dieser nur lauter solche Kräfte verlangen dürfen, die neben einander möglich sind, und alle nur in dem Grade, in dem sie neben einander möglich sind. Also kann der Grund, weßhalb ein Wesen nicht allvollkommen ist, nie in ihm selbst liegen. Er muß sich daher in einem andern Wesen befinden. Wenn aber der Grund, weßhalb ein Wesen gewisse Kräfte hat, oder nicht hat, in einem andern Wesen liegt: so ist es von diesem \RWbet{abhängig}. Also ist jedes Wesen, welches nicht allvollkommen ist, ein abhängiges Wesen. Gott also, der nicht abhängig ist, muß Allvollkommenheit haben.
\end{aufza}

\RWpar{75}{Einzelne Kräfte Gottes}
\begin{aufza}
\item Wenn wir, gemäß der eben erwiesenen Allvollkommenheit Gottes, bestimmen wollen, \RWbet{welche einzelne Kräfte} in diesem Wesen vorhanden sind; so müssen wir untersuchen, wie vielerlei Arten von Kräften es überhaupt gebe.
\item Da eine jede Kraft in einer \RWbet{Möglichkeit, etwas zu wirken}, besteht; diese Wirkung aber entweder in dem Wesen selbst, darin die Kraft sich befindet, oder außerhalb des Wesens zum Vorschein kommen kann: so unterscheide ich~\RWSeitenw{191}\ zuerst \RWbet{zwei Gattungen von Kräften}, deren eine ich \RWbet{inwohnende}, die andere \RWbet{außer sich wirkende} nenne. Die inwohnenden oder immanenten Kräfte sind solche, deren Wirkungen in dem Wesen selbst, das diese Kräfte hat, Statt finden; die \RWbet{außer sich wirkenden} dagegen solche, deren Wirkungen in einem andern Wesen vor sich gehen.
\item Was nun zuvörderst die \RWbet{immanenten} Kräfte anlangt; so finden wir an uns selbst folgende vier vereinigt:
\begin{aufzb}
\item \RWbet{Die Kraft zu denken} (zu welcher die Vermögen des \RWbet{Vorstellens}, \RWbet{Erkennens}, \RWbet{Schließens} \udgl\ gehören).
\item Die Kraft \RWbet{zu empfinden} (\dh\ unseres Zustandes entweder angenehm oder unangenehm inne zu werden.)
\item Die Kraft \RWbet{zu wünschen}, und
\item die Kraft \RWbet{zu wollen}. Mit einem gemeinschaftlichen Namen pflegen wir alle diese vier Kräfte wohl auch \RWbet{Vorstellungskräfte}, und Wesen, die eine oder die andere derselben haben, \RWbet{geistige}, die übrigen aber (falls es dergleichen gibt) \RWbet{materielle} Wesen zu nennen. Hiebei ist noch zu bemerken, daß wir uns außer diesen vier an uns selbst befindlichen immanenten Kräften keine Vorstellung von einer fünften zu machen vermögen, die nicht entweder unter einer von diesen enthalten, oder aus etlichen von ihnen zusammengesetzt wäre.
\end{aufzb}
\item \RWbet{Alle außer sich wirkenden Kräfte} können nur Eins von Beidem bewirken: entweder einer Substanz \RWbet{das Daseyn selbst} ertheilen, oder an einer bereits vorhandenen Substanz gewisse \RWbet{Veränderungen} erzeugen. Eine Kraft, die einer Substanz das Daseyn selbst zu geben vermag, wird eine \RWbet{Schöpferkraft} genannt. Eine Kraft dagegen, die nur an einer schon vorhandenen Substanz gewisse \RWbet{Veränderungen} hervorbringt, mag man \RWbet{Veränderungskraft} nennen. Diese letztere läßt sich nun wieder von sehr verschiedener Art denken, je nachdem das Wesen, an dem die Veränderung vorgehen soll, ein geistiges oder nichtgeistiges ist. Ist es ein \RWbet{geistiges} Wesen; so kann die Einwirkung auf dasselbe entweder eine Veränderung in seinen \RWbet{Gedanken},~\RWSeitenw{192}\ oder in seinen \RWbet{Empfindungen}, oder in seinen \RWbet{Wünschen}, oder in seinem \RWbet{Willen} zur Folge haben. Bei \RWbet{materiellen} Wesen endlich lassen sich wenigstens dreierlei Arten von Veränderungen: \RWbet{mechanische, chemische} und \RWbet{organische} unterscheiden, und daher ihnen gemäß auch dreierlei Arten von Kräften, die solche Veränderungen hervorbringen können, gedenken.
\item Alle so eben aufgezählten Kräfte, mit Ausschluß der einzigen Kraft zu schaffen, sind selbst bei uns Menschen in einem gewissen Grade vorhanden. Wir \RWbet{denken, empfinden, wünschen, wollen}, und \RWbet{bringen auch Veränderungen} in geistigen sowohl als materiellen Substanzen (theils mittel-, theils unmittelbar) hervor.
\item Daraus ergibt sich, daß diese Kräfte auch in Gott anzutreffen seyn müssen, falls nicht etwa Eine aus ihnen, wenn sie in einem höhern Grade vorhanden ist, das Daseyn einer anderen aufhebt. Dieses ist nur bei der Kraft zu \RWbet{wünschen} der Fall, welche dann wegfallen muß, wenn sich ein Wesen durch den Besitz seiner übrigen Kräfte in dem Genusse der höchsten Seligkeit befindet. Nebst diesen auch bei uns anzutreffenden Kräften aber muß Gott, das allvollkommene Wesen, auch noch die \RWbet{Schöpferkraft} haben, weil es sonst keine Substanzen außerhalb seiner gäbe. Daß aber durch Setzung der Schöpferkraft keine der vorigen Kräfte wieder aufgehoben werde, leuchtet von selbst ein. Mit der Kraft zu denken, so wie mit der zu \RWbet{empfinden}, oder mit der zu \RWbet{wollen} stehet doch die Kraft zu schaffen gewiß in keinem Widerspruche; zum Daseyn einer \RWbet{Veränderungskraft} aber wird sie sogar erfordert; denn wenn Gott keine Schöpferkraft hätte, so würde es auch keine Substanzen, die er verändern könnte, geben.
\item So hätten wir also in Gott überhaupt folgende fünf Kräfte anzunehmen: eine Kraft zu \RWbet{denken}, eine Kraft zu \RWbet{empfinden}, eine Kraft zu \RWbet{wollen}, eine Kraft zu \RWbet{schaffen}, und eine Kraft zu \RWbet{verändern}. Es ist nun nöthig, daß wir den Grad, in welchem diese Kräfte bei Gott vorhanden sind, und einige
der merkwürdigsten Beschaffen\RWSeitenw{193}heiten, die ihm in dieser Hinsicht zukommen, in etwas nährere Erwägung ziehen.
\end{aufza}

\RWpar{76}{Eigenschaften der göttlichen Erkenntnißkraft}
\begin{aufza}
\item Gottes Erkenntnißkraft ist:
\begin{aufzb}
\item \RWbet{allwissend}, \dh\ sie umfaßt alle Wahrheiten; oder wie man durch Induction dieß zu beschreiben pflegt, Gott weiß das Vergangene, das Gegenwärtige, und das Zukünftige; das Wirkliche und das bloß Mögliche, so nie zur Wirklichkeit gelangt, \usw\ Gottes Erkenntnißkraft ist ferner
\item auch \RWbet{höchst weise}, \dh\ er kennt zu allen Zwecken auch die tauglichsten Mittel.
\end{aufzb}
\item Daß Gott eine Erkenntißkraft habe, und daß sie die eben genannten beiden Eigenschaften besitze, kann als ein zuverlässiger Lehrsatz in der natürlichen Religion der Menschheit angesehen werden; weil beydes fast von allen nachdenkenden Menschen von jeher anerkannt worden ist.
\item Ich würde diese Wahrheiten auf folgende Art herleiten: Nach seiner Allvollkommenheit muß Gott, wie wir so eben gesehen, auch eine \RWbet{Denkkraft} haben, und diese Denkkraft muß so groß angenommen werden, als sie es neben den übrigen Kräften nur immer seyn kann. Nun wird die Denkkraft durch keine der übrigen Kräfte beschränkt, sondern sie selbst ist es vielmehr, die einige andere, \zB\ die Wollkraft und die Kraft, nach Außen zu wirken, bestimmt. Also muß diese Kraft in Gott ganz so groß angenommen werden, als sie nur \RWbet{an sich selbst} seyn kann. Allein der eigentliche Zweck alles Denkens ist nur \RWbet{das Erkennen der Wahrheit}, und es ist jede Wahrheit an und für sich erkennbar. Also muß Gottes Erkenntnißkraft alle Wahrheiten ohne Ausnahme umfassen, \dh\ \RWbet{allwissend} seyn. Und folglich auch \RWbet{höchst weise}; denn wer alle Wahrheiten erkennt, muß auch erkennen, welche Mittel zu jedem Zwecke die tauglichsten sind; weil dieß gleichfalls Wahrheiten sind.
\begin{RWanm}[Anm.~1.]
Man könnte fragen, ob es nicht eine zu enge Erklärung der Allwissenheit sey, daß sie bloß alle \RWbet{Wahrheiten}~\RWSeitenw{194}\ umfasse, da es ja doch gewiß ist, daß Gott auch unsere \RWbet{Irrthümer} kennt. Hierauf ist aber zu erwidern, daß die Erkenntniß, daß sich dieses oder jenes Wesen in diesem oder jenem Irrthum befinde, auch eine \RWbet{Wahrheit} sey, also auch nach dieser Erklärung zur göttlichen Allwissenheit gehöre.
\end{RWanm}
\begin{RWanm}[Anm.~2.]
Es gibt noch verschiedene mehr oder weniger schwierige Fragen bei der Erkenntnißkraft Gottes, die ich nur kurz berühre.
\begin{aufza}
\item Erkennt Gott jede Wahrheit \RWbet{unmittelbar}, oder die \RWbet{Folgewahrheiten} nur mittelbar, nämlich durch die Erkenntniß ihrer Gründe?\par
\RWbet{Antwort.} Ich glaube das Letztere. --
\item Erkennt Gott die Ereignisse der Welt darum, \RWbet{weil sie sich zutragen}; oder liegt vielmehr umgekehrt in seiner Vorstellung von ihnen eben der Grund, \RWbet{warum} sie erfolgen? und wäre im ersteren Falle seine Erkenntniß nicht von der Welt \RWbet{abhängig}?\par
\RWbet{Antwort.} Man muß die \RWbet{nothwendigen} Ereignisse in der Welt von den \RWbet{zufälligen}, die durch die Freiheit der geschaffenen Wesen erfolgen, unterscheiden. Die ersteren erkennt Gott nicht daraus, weil sie erfolgen, sondern aus ihren nothwendigen Gründen; die letzteren aber, weil sie sich zutragen. Dieß dürfte aber seiner Unabhängigkeit nicht widersprechen, weil die Erkenntniß solcher Ereignisse, wie es scheint, gar nicht den \RWbet{innern Beschaffenheiten} Gottes, sondern bloß seinen \RWbet{Verhältnissen zur Welt} beigezählt werden muß.
\item Erkennt Gott solche Ereignisse erst dann, wenn sie sich zutragen, oder weiß er sie schon von Ewigkeit vorher?\par
\RWbet{Antwort.} Allerdings muß das Letztere der Fall seyn; indem das Erstere einen Wechsel von Vorstellungen bei Gott voraussetzen würde, der seiner Unveränderlichkeit widerspräche. -- Auch gibt es factische Beweise dafür, daß Gott die freien Handlungen vorhersieht, weil er sie ja oft selbst \RWbet{voraussagt}, oder verschiedene Anstalten, die für sie passen, oft schon Jahrhunderte voraus trifft.
\item Wenn sich die göttliche Erkenntniß solcher Ereignisse darauf, daß sie sich zutragen, gründet, wie kann sie Gott früher, als sie geschehen sind, kennen? Müßte die Wirkung da nicht früher als ihre Ursache seyn?~\RWSeitenw{195}\par
\RWbet{Antwort.} Diese Schwierigkeit sucht man gewöhnlich durch die Bemerkung zu heben, daß die Dinge nicht \RWbet{an sich selbst} in der Zeit wären, sondern nur \RWbet{uns} in einer Zeitfolge erschienen. Wenn wir also finden, daß eine schon vor Jahrhunderten von uns bemerkte Verfügung Gottes ganz für gewisse freie Handlungen passe, die uns erst jetzt erscheinen, \zB\ eine Vorhersagung derselben: so liegt zwar freilich ein Theilgrund jener göttlichen Verfügung in diesen freien Handlungen; aber wir dürfen gleichwohl nicht sagen, daß der Grund später als seine Wirkung \RWbet{vorhanden sey}, sondern nur, daß er von uns später als seine Wirkung \RWbet{angeschaut} werde.
\end{aufza}
\end{RWanm}
\end{aufza}

\RWpar{77}{Eigenschaften der göttlichen Empfindungskraft}
\begin{aufza}
\item Vermöge seiner Empfindungskraft ist Gott in dem Besitze \RWbet{einer ganz reinen, ununterbrochenen, und ihrem Grade nach unendlich hohen Seligkeit}, die eben deßhalb keine \RWbet{Wünsche} zuläßt.
\item So haben sich den Zustand Gottes alle christlichen sowohl als heidnischen Weltweisen gedacht.
\item Und in der That, die Kraft zu empfinden, \dh\ sich seines Zustandes angenehm oder unangenehm bewußt zu werden, ist doch gewiß nicht eine Unvollkommenheit, ein Mangel an Kräften; sondern höchst wahrscheinlicher Weise ist jede \RWbet{angenehme} Empfindung nichts als die Wirkung, die aus dem Innewerden einer uns beiwohnenden Kraft oder Vollkommenheit; jede \RWbet{unangenehme} Empfindung dagegen nichts als die Wirkung, die aus dem Innewerden einer gewissen Beschränkung unserer Kräfte, oder eines Mangels hervorgehet. Wenn nun dieß richtig wäre, so müßte freilich Gott, als das allvollkommene Wesen, das alle Kräfte vereinigt, welche nur neben einander möglich sind, und alle in dem höchsten Grade, in welchem sie neben einander möglich sind, die höchste angenehme Empfindung von seinem Zustande haben. \RWbet{Wünsche} dagegen setzen unwidersprechlich die Vorstellung von einem Gute, das man nicht hat, voraus; sie können also bei Gott nicht Statt finden.~\RWSeitenw{196}
\end{aufza}

\RWpar{78}{Eigenschaften der göttlichen Wollkraft}
\begin{aufza}
\item Die Wollkraft Gottes ist
\begin{aufzb} 
\item \RWbet{frei} zu nennen; nicht zwar in dem Sinne, in dem ich oben nach dem Systeme des Indeterminismus die Wollkraft des Menschen frei genannt habe, \dh\ nicht so, als ob sich Gott zu etwas Anderem, als zu demjenigen, wozu er sich wirklich bestimmt, bestimmen könnte; wohl aber in einer derjenigen Bedeutungen, in welchen man auch nach dem Systeme des Determinismus von einer Freiheit redet. Der göttliche Wille ist ferner
\item \RWbet{heilig}, \dh\ in der allervollkommensten Uebereinstimmung mit dem, was gut ist, oder, wie man zu sagen pflegt, was das Sittengesetz verlangt; er will nichts Anderes, als was das allgemeine Wohl möglichst befördert.
\item \RWbet{höchst wirksam}, \dh\ Alles, was Gott \RWbet{will}, geschieht.
\end{aufzb}
\item Diese drei Eigenschaften des göttlichen Willens haben die Weltweisen zu allen Zeiten angenommen.
\item Auch dürfte es nicht schwer seyn, einen Beweis für sie zu finden.
\begin{aufzb}
\item Eine \RWbet{Freiheit} in dem Sinne, wie sie nach meiner Ansicht bei Menschen anzutreffen ist, eine Möglichkeit nämlich, unter denselben Umständen auch das Gegentheil von dem zu beschließen, was er wirklich beschließt, kann bei Gott freilich nicht angetroffen werden. Diese Freiheit findet ja bei uns Menschen lediglich darum Statt, weil oft zwei Handlungen da sind, deren die Eine von unserer Vernunft gefordert, während die andere von unserm Glückseligkeitstriebe gewünscht wird. Bei Gott gibt es aber keine Wünsche des Glückseligkeitstriebes, also auch keine Freiheit von dieser Art. Allein schon die große Ausbreitung, die das System des Determinismus hat (es zählt der Anhänger bei Weitem mehrere als der Indeterminismus), beweiset, daß man das Wort \RWbet{Freiheit} auch noch in einer andern Bedeutung nehme; wie dann, wenn im gemeinen Leben davon gesprochen wird, ob Jemand etwas~\RWSeitenw{197}\ aus \RWbet{freiem Willen} gethan habe oder nicht, schwerlich daran gedacht wird, ob er auch unter denselben Umständen, und bei derselben Beschaffenheit seines Gemüthes etwas Anderes hätte beschließen können. Und so wird es wohl eine Bedeutung des Wortes Freiheit geben, in der wir sie unbedenklich auch Gott beilegen dürfen und müssen. Namentlich scheint es, daß wir uns unter der \RWbet{Freiheit} eines Wesens gar oft nichts Anderes vorstellen, als eine Art von \RWbet{Unabhängigkeit} seiner Willensentschlüsse von \RWbet{äußeren Gegenständen}, wobei wir zulassen, daß es wohl irgend einen, doch nur in dem Wesen selbst (in seinem Innern) liegenden Grund zu diesen Entschließungen gebe. In dieser Bedeutung nun ist der Wille Gottes ohne Zweifel frei, und im höchsten Grade frei zu nennen; denn nur ein in ihm selbst befindliches Etwas, nämlich nur seine eigene Vernunft ist es, die seinen Willen bestimmt.
\item Damit ist aber zugleich auch die \RWbet{Heiligkeit} des göttlichen Willens bewiesen. Es lassen sich nämlich überhaupt nur zwei Kräfte denken, welche die Wollkraft bestimmen, die \RWbet{Vernunft} und der \RWbet{Glückseligkeitstrieb}. Da dieser letztere bei Gott nicht anzutreffen ist, so kann sein Wille nur durch die Forderungen seiner Vernunft bestimmt werden; und folglich will Gott immer nur dasjenige, was dem Wohle des Ganzen am Meisten zusagt; sein Wille ist also höchst heilig.
\item Daß Gottes Wille endlich auch die vollkommenste \RWbet{Wirksamkeit} habe, dürfte schon aus der \RWbet{Allwissenheit} folgen. Denn es gilt allgemein, daß derjenige, der etwas will, sich wenigstens vorstellen müsse, daß er es \RWbet{könne}. Wer aber \RWbet{allwissend} ist, kann sich in seiner Vorstellung von dem, was er vermag, nicht irren; und so muß denn Gott Alles, was er will, auch \RWbet{vermögen}, und in Wirklichkeit setzen. Es versteht sich übrigens von selbst, daß wir den Willen Gottes hier in seiner \RWbet{eigentlichen} Bedeutung nehmen; denn in der uneigentlichen Bedeutung, in der er oft (\RWparnr{26}\ \no\,2.) genommen wird, könnten wir nicht sagen, daß Alles, was Gott will, geschehe.~\RWSeitenw{198}
\end{aufzb}
\end{aufza}

\RWpar{79}{Eigenschaften der nach Außen wirkenden Kräfte Gottes}
\begin{aufza}
\item Wir bestimmen die Beschaffenheit der \RWbet{nach Außen wirkenden Kräfte Gottes} am besten, wenn wir erinnern, daß Gott \RWbet{allmächtig} sey, \dh\ daß er die Kraft habe, Allem das Daseyn zu geben, was 
\begin{inparaenum}[a)] 
\item \RWbet{an sich selbst möglich} ist, (\dh\ keiner reinen Begriffswahrheit widerspricht); auch 
\item mit dem Zwecke der möglich größten Summe der Glückseligkeit übereinstimmt; und endlich 
\item eines bestimmenden Grundes zu seinem Daseyn bedarf.
\end{inparaenum}
\begin{RWanm}
Dieser letzte Beisatz ist nöthig, um von der Classe der Dinge, die Gott hervorbringen kann, zweierlei auszuschließen: erstlich \RWbet{ihn selbst}, der an sich möglich ist, aber schon seinem Begriffe nach ein \RWbet{unbedingtes} Daseyn hat; dann aber auch die freien Willensentschließungen der geschaffenen Wesen, die zwar wohl einer Bedingung, aber keines bestimmenden Grundes zu ihrem Daseyn bedürfen.
\end{RWanm}
\item Zu dieser Allmacht nun gehört, daß Gott
\begin{aufzb}
\item \RWbet{Schöpferkraft}, \dh\ die Kraft, Substanzen das Daseyn zu geben; und
\item \RWbet{Veränderungskraft}, \dh\ die Kraft habe, auf die geschaffenen Substanzen mittelbar sowohl als auch unmittelbar einzuwirken.
\end{aufzb}
\item Daß Gott allmächtig sey, haben fast alle Weltweise von jeher zugegeben, und es kann also zu den ausgemachten Lehrsätzen der natürlichen Religion des Menschengeschlechtes gezählt werden. Daß aber diese Allmacht auch eine \RWbet{schöpferische} Macht sey, \dh\ daß Gott auch Substanzen das Daseyn geben könne, haben die heidnischen Weltweisen fast insgemein verkannt. Unter den neueren Gelehrten hat es dagegen wieder Einige gegeben, welche Gott zwar die Macht zu schaffen beilegten, aber bezweifelten, ob er die Macht habe, auf die geschaffenen Substanzen noch fortwährend, und nicht bloß mittelbar, sondern selbst unmittelbar einzuwirken. 
\item Meiner Meinung nach läßt sich das Eine sowohl als das Andere auf folgende Art beweisen. Eine gewisse~\RWSeitenw{199}\ \RWbet{Kraft nach Außen zu wirken}, ist mit den \RWbet{immanenten} Kräften, die wir in Gott bereits angenommen haben, nicht nur verträglich, sondern wird bei denselben sogar vorausgesetzt. Namentlich könnte es gar keine Wollkraft in Gott geben, wenn er nicht eine Kraft nach Außen zu wirken, hätte; denn Gottes \RWbet{Wollkraft} kann, da sie von der Vernunft allein bestimmt wird, auf nichts Anderes, als auf die Hervorbringung der möglich größten Summe von Glückseligkeit gerichtet seyn. Hätte aber Gott keine Kraft nach Außen zu wirken, so gäbe es auch keine Glückseligkeit, welche durch ihn erst hervorgebracht werden sollte. Es fragt sich also nur, wie groß wir die Kraft, nach Außen zu wirken, bei Gott annehmen müssen? Bei uns Menschen wird die Kraft, nach Außen zu wirken, auf mancherlei Art beschränkt:
\begin{aufzb}
\item durch die Beschränkungen, die unserer \RWbet{Denkkraft} gesetzt sind; indem wir nicht wollen, und also auch nicht versuchen und hervorbringen können, wovon wir aus Unwissenheit glauben, daß es uns unmöglich sey, oder daran wir aus Vergessenheit nicht denken.
\item durch unsere Kraft zu \RWbet{wollen}, indem wir oft Manches, was wir hervorbringen könnten, wovon wir auch wissen, daß wir's hervorbringen können, doch nicht hervorbringen wollen, und eben deßhalb auch nicht in der That hervorbringen. Diese Kraft zu wollen wird bei uns wieder bedingt
\begin{inparaenum}[(g)]
\item durch unsere praktische Vernunft, und
\item durch den Glückseligkeitstrieb.
\end{inparaenum}
\end{aufzb}
Bei Gott fallen alle diese Beschränkungen weg, bis auf die einzige durch seine praktische Vernunft, welche bei ihm eine \RWbet{bestimmende} Wirksamkeit äußert. Die Kraft nach Außen zu wirken, welche wir Gott beilegen, muß sonach Alles hervorbringen können, was eine Kraft dieser Art an sich hervorbringen kann, mit Ausnahme dessen, was entweder dem Sittengesetze zuwider wäre, oder schon seiner Natur nach keinen bestimmenden Grund seines Daseyns zuläßt. Nun ist wohl offenbar, daß es für Alles, was \RWbet{an sich möglich ist}, (\dh\ was keiner reinen Begriffswahrheit widerspricht) und doch kein unbedingtes Daseyn hat, (\dh\ nicht Gott selbst~\RWSeitenw{200}\ ist) auch irgend eine Kraft, die ihm das Daseyn gibt, möglich seyn müsse. Also wird sich die göttliche Kraft, nach Außen zu wirken, auf Alles erstrecken, was
\begin{inparaenum}[a)]
\item an sich selbst möglich ist,
\item dem Sittengesetze nicht widerspricht, und
\item irgend eines bestimmenden Grundes zu seinem Daseyn bedarf.
\end{inparaenum}
\item Daß diese göttliche Allmacht \RWbet{die Kraft zu schaffen} und die \RWbet{Kraft zu verändern} in sich schließe, ist schon gezeigt worden. Hier will ich also nur noch die \RWbet{Schwierigkeiten} zu heben versuchen, welche gewisse Weltweise bei der Annahme dieser Kräfte fanden.
\begin{aufzb} 
\item Eine Kraft zu \RWbet{schaffen} sprachen die heidnischen Weltweisen Gott aus dem Grunde ab, weil sie das Schaffen für etwas an sich selbst Unmögliches hielten. Dazu veranlaßte sie aber die irrige Vorstellung, \RWbet{daß jede Ursache der Zeit nach früher seyn müsse als ihre Wirkung}. Wenn nämlich dieses wäre; so müßte das schaffende Wesen der Zeit nach früher vorhanden gewesen seyn, als das geschaffene, weil jenes die Ursache vom Daseyn des Letztern seyn soll. Es müßte also eine Zeit gegeben haben, in der das geschaffene Wesen noch nicht vorhanden war, also einen Augenblick, in dem es \RWbet{anfing} zu seyn. Dieß aber widerspricht dem Lehrsatze, daß Substanzen weder entstehen noch vergehen, sondern, wofern sie sind, zu aller Zeit vorhanden seyn müssen. Diese Schwierigkeit nun ist gehoben, sobald man die Vorstellung, daß eine jede Ursache der Zeit nach früher seyn müsse als ihre Wirkung, widerlegt hat. Ich erinnere also, daß es für's Erste Ursachen und Wirkungen gebe, die beide gar nicht in einer Zeit sind, in Betreff deren es also eine offenbare Ungereimtheit wäre, zu sagen, die Ursache müsse früher als ihre Wirkung seyn. Ein Beispiel hievon gibt uns die \RWbet{Allvollkommenheit} und \RWbet{höchste Seligkeit} Gottes. Die letztere ist in der erstern gegründet, Gott fühlt sich höchst selig, weil er sich seiner Allvollkommenheit bewußt ist; gleichwohl ist weder diese noch jene Eigen\RWSeitenw{201}schaft Gottes in einer Zeit vorhanden. Oefters ist nur die Ursache allein in der Zeit, nicht aber ihre Wirkung; oder die Wirkung allein in der Zeit, nicht aber ihre Ursache; und da ist es gleichfalls sehr ungereimt, behaupten zu wollen, daß die eine früher sey als die andere. Wenn endlich Ursache und Wirkung \RWbet{beide} in der Zeit sind; so sind sie beide allemal \RWbet{gleichzeitig}, sie fangen zu gleicher Zeit an, und hören zu gleicher Zeit auf. Denn sagen, daß die Ursache jetzt eben \RWbet{angefangen} oder \RWbet{aufgehört} habe zu seyn, heißt doch gewiß nichts Anderes als sagen, daß sie jetzt eben \RWbet{zu wirken} angefangen oder aufgehört habe, und dieses heißt wieder nichts Anderes, als daß die \RWbet{Wirkung} ihr Daseyn angefangen oder beendigt habe. -- Aber wie kommt es, wird man mir einwenden, daß der gemeine Menschenverstand diese Wahrheit verkannt hat, und durchgängig spricht, die Ursache müsse früher als ihre Wirkung seyn, und die letztere dauere oft lange noch fort, wenn schon die erstere aufgehört hat? Dieß kommt nur daher, sage ich, weil im gemeinen Leben die Worte \RWbet{Ursache} und \RWbet{Wirkung} selten in ihrer strengsten Bedeutung, sondern in einem Sinne genommen werden, in welchem es allerdings ganz wahr ist, daß die Ursache früher als ihre Wirkung vorhanden sey, und daß diese oft noch lange fortdauere, wenn jene schon aufgehört hat. Denn wir nennen im gewöhnlichen Leben eine Sache $A$ schon \RWbet{Ursache} von einer andern $B$, und diese \RWbet{Wirkung} von jener, wenn $A$ eigentlich nur ein in gewisser Hinsicht besonders merkwürdiger \RWbet{Theil} der ganzen Ursache von $B$ ist, während das Daseyn der übrigen Stücke, die zur Entstehung von $B$ mitgewirkt haben, sich von selbst verstehet; oder auch wohl schon dann, wenn $A$ nur eine (in der so eben erklärten Bedeutung genommene) Ursache einer \RWbet{Veränderung} an einem Dinge war, das hiedurch fähig wurde, die Wirkung $B$ zu erzeugen. So sagen wir, die Ursache des Gefühls von Wärme, das wir jetzt eben haben, sey das im Ofen brennende Feuer, da gleichwohl zur Entstehung dieses Gefühls noch viele andere Dinge gehören, \zB\ die Luft zwischen uns und dem Ofen, die~\RWSeitenw{202}\ Wände des Zimmers, die diese Luft einsperren, die Empfänglichkeit unsers Körpers für solche Einwirkungen der Wärme, \usw\ Eben so sagen wir, der Baumeister sey die Ursache von dem Daseyn des Hauses, während doch der Baumeister und seine Bauleute höchstens Ursache von jenen Ortsveränderungen sind, die mit den Steinen und übrigen Materialien, aus denen das Haus zusammengesetzt wird, vorgingen. Daß nun ein oder der andere einzelne \RWbet{Theil} einer Ursache viel früher vorhanden sey, als die \RWbet{ganze} Ursache, und mithin auch als die Wirkung; ingleichen, daß die Wirkung fortdauere, wenn die Ursache von der \RWbet{Veränderung} eines Dinges, durch welche dasselbe erst in den Stand gesetzt wurde, diese Wirkung hervorzubringen, längst aufgehört hat: das Alles sind sehr begreifliche Dinge. Hieraus erkläret sich zugleich, wie man von einer und eben derselben Erscheinung in verschiedenen Rücksichten \RWbet{verschiedene} Dinge als Ursachen angeben könne; was dann auch die verschiedenen \RWbet{Eintheilungen} der Ursachen in \RWbet{moralische, psychologische, physische} \udgl\ veranlasset hat. So sagt man \zB\ von einem Selbstmorde, die \RWbet{psychologische} Ursache desselben (\dh\ derjenige Theil seiner ganzen Ursache, der in der Seele lag) sey eine Geisteszerrüttung, das \RWbet{physische} Ausführungsmittel eine Pistole gewesen \udgl\
\item Daß einige Gelehrte neuerer Zeit daran gezweifelt, ob Gott auf die geschaffenen Substanzen auch \RWbet{unmittelbar} einwirken könne, hat seinen Grund darin, weil solche Einwirkungen in der \RWbet{Zeit} vorgehen müssen. Wenn aber Gott in der \RWbet{Zeit} wirken sollte: so, meinten sie, müßte er selbst in der Zeit seyn. Dieses folgt aber gar nicht; denn nur wenn sich ein Wesen \RWbet{verändert}, ist man gezwungen zu sagen, daß es sich in einer Zeit befinde. Wenn aber Gott auf die Substanzen der Welt verschieden einwirkt (theils mittel-, theils unmittelbar): so setzt dieß keine Veränderung in seinen innern Beschaffenheiten, also auch kein Vorhandenseyn desselben in einer Zeit voraus; denn die \RWbet{Einwirkungen} Gottes auf die Substanzen der Welt gehören ja zu seinen bloßen Verhältnissen (\RWparnr{72}).~\RWSeitenw{203}
\end{aufzb}
\end{aufza}

\RWpar{80}{Ewigkeit und Allgegenwart Gottes}
\begin{aufza}
\item Noch ein Paar Eigenschaften Gottes, die hier erwähnt werden müssen, sind seine Ewigkeit und Allgegenwart. Unter der \RWbet{Ewigkeit} Gottes verstehen wir, daß er zu allen Zeiten \RWbet{sey}, und also auch \RWbet{wirke}. Unter der \RWbet{Allgegenwart}, daß er in allen Theilen des Raumes \RWbet{gegenwärtig} sey, \dh\ eine \RWbet{gewisse Art von Wirksamkeit} äußere.
\begin{RWanm}
Wie sehr sich auch noch darüber streiten ließe, was unter dem Worte \RWbet{Gegenwart} verstanden werde, wenn es von irgend einem endlichen, zumal von einem geistigen Wesen gebraucht wird: so darf man mir doch so viel zugeben, daß diejenige Gegenwart, die wir \RWbet{Gott} beilegen, wiefern sie zur \RWbet{Religion} gehört, \dh\ von einiger Wichtigkeit für uns ist, in einer gewissen \RWbet{Wirksamkeit} Gottes bestehe. Wenn wir behaupten, daß Gott \RWbet{allgegenwärtig} sey; so wollen wir damit nichts Anderes, wenigstens nichts, was uns sonst wichtig seyn könnte, behaupten, als daß Gott überall \RWbet{wirke}.  Ob diese Wirksamkeit unmittel- oder mittelbar erfolge, ist ein Umstand, der uns gewiß ganz gleichgültig seyn kann.
\end{RWanm}
\item Gottes \RWbet{Ewigkeit} ist noch von Niemand, der an das Daseyn eines Wesens von unbedingter Wirklichkeit glaubte, in Zweifel gezogen worden. Gottes \RWbet{Allgegenwart} aber wurde vor Einführung des Christenthums nur von Weltweisen eingesehen; und auch selbst diese hatten sich nicht immer einen ganz richtigen Begriff von ihr gebildet; wir werden sie also nicht zu den ausgemachten Lehrsätzen der natürlichen Religion des menschlichen Geschlechtes zählen dürfen.
\item Meiner Ansicht nach folgt die Ewigkeit Gottes schon aus seiner \RWbet{Substanzialität}. Wer aber auch diese Folge nicht zugeben wollte, müßte die Ewigkeit Gottes doch aus seiner \RWbet{Nothwendigkeit} schließen. -- Gottes \RWbet{Allgegenwart} aber bezieht man insgemein nicht auf diejenigen Theile des Raumes (wenn es ja solche gibt), die durchaus \RWbet{leer}, sondern nur auf diejenigen, die mit Substanzen erfüllt sind. Daß nun Gott allerdings in einem jeden Theile des Raumes, in welchem sich eine \RWbet{Substanz} befindet, wirke, und zwar \RWbet{unmittelbar} wirke, und also gewiß auch \RWbet{gegenwärtig} sey; folgt schon daraus, weil die fortwährende Erhaltung dieser Sub\RWSeitenw{204}stanzen eine fortwährende \RWbet{Wirkung} (und dieß zwar eine \RWbet{unmittelbare}) von Gottes Willen ist.
\end{aufza}

\RWpar{81}{Folgerungen aus diesen Eigenschaften Gottes}
Aus den bisher erwiesenen göttlichen Eigenschaften ergeben sich mehre wichtige Folgerungen, die das \RWbet{Verhalten Gottes} zur Welt betreffen. Einige derselben, auf die ich mich künftig zu berufen gedenke, will ich hier in gedrängter Kürze zusammenstellen.
\begin{aufza}
\item \RWbet{Gott mußte Geschöpfe, die der Glückseligkeit empfänglich sind, erschaffen}. Denn daß es ihm \RWbet{möglich} gewesen, solche zu schaffen, ist wenigstens jetzt, da wir sie \RWbet{wirklich} sehen, außer allem Zweifel. Ist es ihm aber möglich; so mußte er es um seiner Heiligkeit wegen auch wirklich \RWbet{thun}, weil er sonst keine \RWbet{Glückseligkeit} hätte befördern können.
\item \RWbet{Und zwar eine unendliche Menge derselben}. Denn daß eine \RWbet{unendliche Menge} von Geschöpfen allerdings \RWbet{möglich} sey, wird einem Jeden, der daran zweifeln wollte, aus der Unendlichkeit des Raumes einleuchtend. So gut es \zB\ in dem Räume, den diese Erde einnimmt, eine gewisse (allenfalls endliche) Menge lebendiger Geschöpfe geben kann und wirklich gibt, so gut kann es in jedem andern Raume außerhalb der Erde, und folglich, weil dieser Räume unendlich viele sind, eine unendliche Menge lebendiger Geschöpfe geben. Sind aber unendlich viele lebendige Geschöpfe \RWbet{möglich}; so wäre Gott nicht heilig, wenn er sie nicht wirklich erschaffen hätte, weil er keineswegs die größte Summe von Glückseligkeit, die an sich möglich ist, zur Wirklichkeit brächte.
\begin{RWanm}
Die Welt ist also dem \RWbet{Raume} nach \RWbet{unendlich}.
\end{RWanm}
\item \RWbet{Es muß auch zu aller Zeit lebendige Geschöpfe gegeben haben, und auch in Zukunft geben}. Eine Wahrheit, die auf dieselbe Art, wie die nächstvorhergehende erwiesen werden kann.~\RWSeitenw{205}
\begin{RWanm} 
Die Welt ist also der Zeit nach \RWbet{ewig}, und dennoch \RWbet{abhängig von Gott}. So haben auch \RWbet{Wolff, Daries} (\RWlat{Dissert.\ de creatione ab aeterno possibili})\RWlit{}{Darjes1} \uA\ gelehrt. Irrig war nur die Meinung der älteren Weltweisen, eines \RWbet{Okellus} aus \RWbet{Lukanien, Aristoteles, Zeno}, daß die Welt in ihrem ewigen Daseyn auch von Gott \RWbet{unabhängig} sey.
\end{RWanm}
\item \RWbet{Wofern es leblose Dinge in der Welt gibt, so können diese nur der lebendigen Geschöpfe wegen vorhanden und für sie eingerichtet seyn}. Weil nämlich Gott nichts zu wirken vermag, was nicht das oberste Sittengesetz verlangt; so muß Alles, was er hervorbringt, zur Beförderung der Glückseligkeit dienen. Also auch leblose Geschöpfe, wofern er einige geschaffen hat, sind nur zur mehren Beförderung der Glückseligkeit geschaffen. Diese findet aber nur in lebendigen Geschöpfen Statt, also sind jene nur der lebendigen Geschöpfe wegen vorhanden und für sie eingerichtet.
\item \RWbet{Dagegen kein lebendiges Geschöpf wird als ein bloßes Mittel angesehen, sondern bei der Behandlung eines jeden nimmt Gott auch Rücksicht auf das Gefühl der Lust oder des Schmerzes, welches durch diese Behandlung in ihm selbst entsteht}. Jedes lebendige Geschöpf ist nämlich eines gewissen Grades von Lust oder Schmerz empfänglich. Es sey nun dieser Grad noch so gering, so darf ihn doch Gott, wenn er die Summe der Glückseligkeit in der Welt (falls wir so sagen dürfen) berechnen will, \dh\ wenn er der Welt diejenige Einrichtung geben will, bey welcher die Summe der Glückseligkeit ein Größtes wird, nicht übersehen. Also ist sicher bei jeder Verfügung, die Gott mit einem solchen Wesen trifft, unter Anderem auch auf den Eindruck, den sie auf dieses Wesen selbst hervorbringen wird, Rücksicht genommen.
\item \RWbet{\Hhat{Es stehet in keinem erweislichen Widerspruche mit der Heiligkeit Gottes}{Die Vernunft hat keinen hinlänglichen Grund, Gott darüber zu tadeln}, daß er auch freie Wesen geschaffen}, (selbst in der oben angenommenen indeterministischen Bedeutung dieses Wortes). Beinahe könnte man zwar glauben, daß die Erschaffung freier Wesen in jener obigen Bedeutung etwas Zweckwidriges sey; indem Gott, wenn er einigen seiner Geschöpfe die Freiheit schenkt, eben hiedurch einen Theil der allgemeinen Glückseligkeit in \RWbet{ihre}~\RWSeitenw{206}\ Gewalt gibt, und es von ihnen nun abhängt, ob sie das Gute, das möglich ist, auch wirklich machen werden. Einige werden dieß sicher nicht thun, und so wird denn die Welt, so scheint es wenigstens, nur unvollkommener werden, als sie es wäre, wenn Gott Alles entweder selbst, oder durch Geschöpfe, die keine dergleichen Freiheit haben, bewirkte. Allein wenn anders die von mir (\RWparnr{15}) gegebene Erklärung der Freiheit ihre Richtigkeit hat; so ist das Daseyn derselben eine unvermeidliche Folge von dem Vorhandenseyn eines gewissen Streites zwischen den Forderungen der Vernunft und den Wünschen des Glückseligkeitstriebes. Sollte es also keine freien Wesen geben, so hätte Gott nur Eins von Beiden thun müssen: entweder er hätte nur lauter solche Wesen erschaffen müssen, die keine Vernunft haben, oder nur lauter solche, die neben der Vernunft keinen mit ihr jemals in Streit gerathenden Trieb der Glückseligkeit haben. Das Erstere wird kaum Jemand im Ernste verlangen, da es einleuchtend ist, daß die vernunftlosen Wesen auf einer viel niedrigeren Stufe der Ausbildung, und eben darum auch der Glückseligkeit stehen, als die vernünftigen. Denn je niedriger das Bewußtseyn (oder Gefühl) eines Wesens, um desto niedriger auch der Grad der Lust, dessen dasselbe empfänglich ist. So scheint das Gefühl eines Polypen dunkler, als jenes des \RWbet{Elephanten}, und dieses viel dunkler als das des \RWbet{Menschen}, darum ist aber auch der \RWbet{Elephant} eines viel höhern Grades der \RWbet{Lust} empfänglich, als der \RWbet{Polype}, und der \RWbet{Mensch} eines viel höhern als der \RWbet{Elephant}. Eben so wenig ist aber auch das Zweite zu begehren, daß nämlich alle Wesen auf einer so hohen Stufe der Ausbildung stehen sollen, als dazu nöthig ist, damit die Wünsche ihres Glückseligkeitstriebes niemals in einen Widerspruch mit den Forderungen ihrer Vernunft gerathen. Ein Zustand dieser Art kann nur dann eintreten, wenn sich das Wesen die Ueberzeugung, daß eine jede Abweichung von seiner Pflicht früher oder später bestraft werde, nicht nur erworben hat, sondern sie auch so fest zu halten vermag, daß sie für keinen Augenblick aus seinem Bewußtseyn wieder verschwindet. Da aber Einsichten und Ueberzeugungen nicht angeboren seyn können, so kann ein geschaffenes Wesen nur~\RWSeitenw{207}\ erst allmählich und mit der Zeit zu dieser Stufe der Vollkommenheit gelangen, und es muß also früher eine Zeit für dasselbe geben, wo es noch ohngefähr so, wie wir Menschen Versuchungen zum Bösen ausgesetzt ist, und somit Freiheit in der hier angenommenen Bedeutung hat.
\item \RWbet{Es läßt sich nicht verlangen, weder}
\begin{inparaenum}[a)] \item \RWbet{daß Gott alle lebendigen Geschöpfe mit gleichen Kräften ursprünglich ausgerüstet habe, noch}
\item \RWbet{daß er die gleichgeschaffenen auf eine gleiche Art behandle}.
\end{inparaenum}
\begin{aufzb}
\item Was das \RWbet{Erstere} betrifft, so könnte es ja wohl seyn, daß jedes lebendige Geschöpf eines gewissen Stoffes zur thätigen Bearbeitung desselben und zum Genusse seiner eigenen Glückseligkeit bedarf. So bedürfen \zB\ wir Menschen einer Erde, auf der wir wohnen, gewisser Nahrungsmittel, die wir genießen, \usw\ Nun läßt sich leicht erachten, daß dieser Stoff nicht bis zur Würde eben desselben Geschöpfes, das ihn zur Uebung seiner Kräfte und zum Genusse seiner Glückseligkeit gebrauchen soll, erhoben werden könne. Statt also diesen Stoff \RWbet{ganz leblos} zu lassen, wird es ohne Zweifel besser gethan seyn, ihn zu \RWbet{beleben}, irgend ein Geschöpf daraus zu bilden, das zwar einen niedrigeren Rang, als das \RWbet{erstere} hat, aber doch immer eines \RWbet{Genusses der Glückseligkeit} empfänglich ist. So ist \zB\ ein Theil der Erdmasse zu Thieren, und weil diese abermals einer Nahrung bedürfen, ein anderer zu kleineren Thieren gebildet worden, \usw\
\item Vorausgesetzt, daß es auch \RWbet{freie Wesen} gebe; so liegt schon in dem Vorhandenseyn dieser ein leicht begreiflicher Grund, warum Gott auch \RWbet{gleichgeschaffene} Wesen nicht immer auf eine gleiche Art behandle, und behandeln lasse; denn freie Geschöpfe werden, selbst wenn ihre Anlagen und alle Verhältnisse, in welche sie Gott gesetzt hat, vollkommen gleich seyn sollten, vermöge ihrer Freiheit doch ungleich handeln. Das Eine derselben wird seine Freiheit gut, das andere übel gebrauchen. Gott also, der diesen ungleichen Gebrauch vorhersah, mußte~\RWSeitenw{208}\ eben darum sie in verschiedene Verhältnisse, jedes in solche setzen, die sich zu seinen freien Handlungen am Besten schicken. Weil endlich diese freien Wesen auch einander, und die Geschöpfe, welche sich innerhalb ihres Wirkungskreises befinden, ungleich behandeln werden: so mußte Gott auch diesen Geschöpfen eine ungleiche Empfänglichkeit und ungleiche Kräfte geben, so wie es sich für die Behandlung, welche sie erfahren werden, am Besten schickt. Wenn Gott \zB\ vorhersieht, daß dieser oder jener Mensch eine sehr harte Behandlung von seinen Mitmenschen erfahren werde: so konnte es gut seyn, diesem einen geringern Grad von \RWbet{Empfänglichkeit} zu ertheilen; denn hiedurch wurden ihm viele Leiden erspart \usw\
\end{aufzb}
\item In Rücksicht der Behandlung, die Gott den freien Wesen angedeihen lassen muß, erkennt die Vernunft mit größter Deutlichkeit die Regel, \RWbet{daß Gott auf eine jede sittlich gute Handlung eine Erhöhung, auf jede sittlich böse eine Verminderung der Glückseligkeit des handelnden Wesens müsse erfolgen lassen}. Daß die \Ahat{Beobachtung}{Beobachtungen} einer solchen Regel nichts an sich selbst Unmögliches enthalte, wird Niemand in Abrede stellen. Wenn aber Gott diese Regel festsetzt, und wenn ein jedes freie Wesen erwartet, daß Gott nach ihr vorgehen werde: so erhalten wir alle den größten Aufmunterungsgrund zu allem Guten, und den stärksten Abhaltungsgrund von allem Bösen; wir werden uns also der Tugend viel mehr, als es sonst geschähe, befleißigen, und das Laster viel eifriger fliehen; und eben hiedurch wird das Wohl des Ganzen viel mehr befördert werden.
\begin{RWanm}
Die Erhöhung der Glückseligkeit, die einem freien Wesen um seiner guten Handlungen willen zu Theil wird, heißt \RWbet{Belohnung}; die Unglückseligkeit aber, die ihm um seiner bösen Handlungen willen zu Theil wird, \RWbet{Strafe}. Der Tugendhafte \RWbet{verdient Belohnung}, heißt, es ist ein Grund in ihm vorhanden, weßhalb ein anderes Wesen, welches die Macht dazu hat, ihn glücklich machen soll. Der Lasterhafte \RWbet{verdient (verschuldet) Strafe}, heißt, es ist ein Grund in ihm vorhanden, weßhalb ein anderes Wesen, welches die Macht dazu hat, ihm Strafe zufügen soll.~\RWSeitenw{209}
\end{RWanm}
\item Was die \RWbet{Beschaffenheit} und den \RWbet{Grad} dieser Belohnungen sowohl als Strafen anlangt: so kann die Vernunft hierüber nur so viel zu behaupten wagen:
\begin{aufzb}
\item Belohnungen müssen immer so groß seyn, als es nur möglich ist; denn aus der Erhöhung derselben entspringt kein Nachtheil, wohl aber ein doppelter Vortheil: das Gute geschieht um so gewisser, und die Glückseligkeit des Handelnden gewinnt um so mehr.
\item Strafen dagegen müssen immer nur so groß angesetzt werden, bis die Summe des Bösen, welches durch ihre Abschreckung verhindert worden ist, die Summe der Leiden, die sie in den Bestraften (die sich nicht abschrecken ließen) hervorbringen, am Meisten übertrifft.
\item Die Strafe, die Gott auf ein Verbrechen ankündigt, muß daher jedesmal wenigstens größer seyn, als der sichtbare Vortheil des Verbrechens im Augenblicke der Versuchung: denn sonst würde sich der Sünder durch die Androhung einer solchen Strafe nicht abhalten lassen.
\item Je größer der Nutzen einer guten Handlung, um desto größer muß, bei übrigens gleichen Umständen, ihre Belohnung seyn; und umgekehrt bei jeder bösen Handlung.
\item Je größer das Opfer einer guten Handlung ist, um desto größer muß bei übrigens gleichen Umständen ihre Belohnung seyn; und umgekehrt, je größer die Lust und die Versuchung zu einer bösen Handlung, um desto größer die auf sie gesetzte Strafe.
\item Je deutlicher die Einsicht in das Gesetz war, \dh\ je bestimmter das freie Wesen gewußt hat, daß es durch seine Handlung das allgemeine Wohl störe, um desto größer muß, bei übrigens gleichen Umständen, die Strafe ausfallen.
\item Je weiter ein Tugendhafter in seiner sittlichen Vollkommenheit bereits fortgerückt ist, um desto mehr Glückseligkeit muß er sich als Belohnung versprechen dürfen; und umgekeht mit dem Lasterhaften.
\end{aufzb}
\item Mit keinem Grunde dagegen läßt sich behaupten,
\begin{aufzb}
\item daß die Belohnung oder Strafe \RWbet{immer gleich auf der Stelle eintreten} müsse. Denn für ein vernünf\RWSeitenw{210}tiges Wesen ist schon ein hinreichender Aufmunterungsgrund zum Guten und Abhaltungsgrund vom Bösen vorhanden, wenn es nur weiß, es werde in Zukunft einmal für alles Gute belohnt, für alles Böse bestraft werden. Daß aber die Belohnung oder Strafe gleich auf der Stelle eintrete, ist vielleicht wegen des Zusammenhanges des Ganzen nicht möglich; Gott müßte den natürlichen Lauf der Dinge zu oft gewaltsam unterbrechen.
\item daß der Grad der Belohnung bei mehren Wesen, die einen \RWbet{gleichen Grad von Tugend} haben, ebenfalls \RWbet{gleich} seyn müsse, und so auch bei der Bestrafung. Denn jeder Tugendhafte erhält schon Aufmunterungsgrund genug, wenn er nur weiß, daß er für seine Person sich nicht anders glücklicher machen könne, als wenn er tugendhafter wird, und umgekehrt. Zu wissen, ob ein Anderer, der einen gleichen Grad von Tugend erreicht, auch einen gleichen Grad der Belohnung erhalte oder nicht, kann ihm kein neuer Beweggrund zur Tugend werden.
\begin{RWanm}
Hier ist nur erwiesen, daß eine gleiche Behandlung gleicher Tugendgrade nicht eben \RWbet{nothwendig} sey, um den Geschöpfen einen hinlänglichen Aufmunterungsgrund zu geben. Wenn aber Gott das Gegentheil thun, und Geschöpfe von einem gleichen Grade des Verdienstes in der That ungleich behandeln soll: so bedarf es hiezu erst noch eines \RWbet{besondern Grundes}. Daß aber auch ein solcher möglich sey, läßt sich aus Nr.\,6 erachten. So könnten \zB\ einige Tugendhafte \RWbet{nur darum reichlicher} als Andere belohnt werden, weil es bei dem Zusammenhange der Dinge in der Welt nicht möglich ist, alle \RWbet{gleichreichlich} zu belohnen.
\end{RWanm}
\item daß jedes unfreie Geschöpf während des ganzen Zeitraumes seines Daseyns mehr Lust als Schmerz genieße. Denn wenn das Daseyn eines solchen Geschöpfes eine nothwendige Bedingung zum Daseyn oder zur Glückseligkeit anderer Geschöpfe wäre; und wenn die Glückseligkeit, welche die letzteren durch die Leiden des ersteren gewinnen, wichtiger wäre, und sich auf keine andere Weise erreichen ließe: so würde die Summe der Glückseligkeit des Ganzen durch die Hervorbringung dieses Geschöpfes immer erhöhet; also kein Grund vorhanden, die Erschaffung desselben zu verbieten.~\RWSeitenw{211}
\begin{RWanm}
Bei \RWbet{freien} Geschöpfen gilt dieses freilich nicht, wegen der Nr.\,8 aufgestellten Regel.
\end{RWanm}
\begin{RWanm}[\RWbet{Schlußanmerkung.}] 
Diese zehn Sätze wage ich nicht \RWbet{alle} für gewisse Lehrsätze der natürlichen Religion des menschlichen Geschlechtes auszugeben. Die Sätze 1, 4, 8, 9 sind wohl mit ziemlicher Allgemeinheit erkannt worden; besonders 8. Dagegen die Sätze 2, 3, 6, und 10,\,b und c sind vielleicht öfter bestritten als vertheidigt worden. Ich stelle sie also hier nur als \RWbet{meine Meinungen} hin.
\end{RWanm}
\end{aufzb}
\end{aufza}

\RWpar{82}{Daseyn der Welt}
\begin{aufza} 
\item Unter dem Worte \RWbet{Welt} in seiner weitesten Bedeutung verstehen wir den Inbegriff aller Wesen, welche kein unbedingtes Daseyn haben.
\item Daß es nun solche Wesen gebe, und daß insonderheit wir selbst und alle jene Gegenstände, deren Daseyn wir durch bloße Wahrnehmung allein erfahren, namentlich alle die mancherlei theils leblosen theils belebten Gegenstände, die wir auf Erden antreffen, alle Himmelskörper, \usw\ zu den Dingen gehören, welche kein unbedingtes Daseyn haben, sondern durch Gott bedingt sind: ist eine Wahrheit, die so allgemein anerkannt wurde, daß wir sie abermals den Lehrsätzen der natürlichen Religion des menschlichen Geschlechtes beizählen dürfen.
\item Die Reihe der Schlüsse, durch welche sie \RWbet{meiner} Meinung nach herzuleiten wäre, ist etwa folgende:
\begin{aufzb}
\item Jeder Gegenstand, der seine eigene, von andern abgesonderte Empfindung hat, ist eine eigene, von andern abgesonderte Substanz. Auch jeder Gegenstand, der einen eigenen Theil des Raumes, und wäre es auch nur den eines Punctes, ausfüllt, ist eine eigene Substanz. Die Wahrheit dieser zwei Sätze wird Jeder, der nur die Bedeutung des Wortes Substanz versteht, zugeben, auch ohne erst einen Beweis für sie zu verlangen.
\item Also ist jeder einzelne Mensch, jedes einzelne Thier, und alle jene Dinge, deren Daseyn wir durch bloße Wahrnehmung erfahren, sind Eines von Beydem, entweder einzelne Substanzen, oder ein Inbegriff mehrer Substanzen. Denn sie besitzen alle~\RWSeitenw{212}\ entweder Empfindung, oder sie füllen doch irgend einen Raum aus.
\item Diese Substanzen aber haben kein unbedingtes Daseyn, sonst müßten sie die Eigenschaften einer Substanz von unbedingtem Daseyn, also Allwissenheit, Allmacht \usw\ besitzen. Ihr Daseyn ist sonach bedingt, und es gibt folglich eine \RWbet{Welt}.
\item Da endlich jeder bedingte Gegenstand irgend einen andern als die Bedingung seines Daseyns voraussetzt, nebst den bedingten Wesen aber Niemand als Gott vorhanden ist: so folgt, daß alle bedingten Wesen, \dh\ die ganze Welt durch \RWbet{Gott} bedingt sey.
\end{aufzb}
\end{aufza}

\RWpar{83}{Auch die Beschaffenheit der Welt bestätiget das Daseyn Gottes und seine Eigenschaften}
Die vorhin aufgestellten Eigenschaften Gottes sind von einer solchen Wichtigkeit, daß wir uns eine möglichst sichere Ueberzeugung von ihnen zu verschaffen, und daher wünschen müssen, sie nicht nur durch eine Reihe apriorischer Schlüsse erwiesen zu sehen, sondern auch durch die \RWbet{Erfahrung selbst} bestätiget zu finden. Dieser Wunsch ist um so gerechter, je mehr sich erwarten läßt, daß, wenn es in der That einen solchen Gott gibt, wie wir ihn oben beschrieben, auch mehre für uns bemerkbare Spuren seiner Eigenschaften in der Welt als seinem Werke anzutreffen seyn werden. So ist es auch wirklich; und die Beobachtung dieser Spuren ist es viel mehr als die Bekanntschaft mit jenen apriorischen Schlüssen, der wir die allgemeine Verbreitung sowohl als auch die Zuverlässigkeit unsers Glaubens an Gott verdanken. Eben deßhalb will ich hier in Kürze zeigen, wie dieser Schluß aus der Betrachtung der Welt auf das Daseyn Gottes eigentlich geschehe.
\begin{aufza}
\item Allgemein sind wir berechtigt anzunehmen, \RWbet{daß ein gewisser Gegenstand das Werk eines verständigen Wesens und von demselben zu einem bestimmten Zwecke hervorgebracht sey}, wenn wir~\RWSeitenw{213}
\begin{aufzb}
\item keinen Grund finden, das Daseyn und die Einrichtungen dieses Gegenstandes für an sich nothwendig zu halten; auch
\item keine Unmöglichkeit darin erkennen, daß ein Wesen, wie jenes, dem wir sein Daseyn zuschreiben wollen, vorhanden sey, den Gegenstand hervorgebracht und ihm diese Beschaffenheiten ertheilt habe; wenn wir vielmehr
\item wahrnehmen, daß der Gegenstand durch seine Einrichtungen zur Hervorbringung jener Wirkung, die wir als seinen Zweck ansehen wollen, in der That tauge; daß ferner
\item diese Wirkung ein Erfolg von solcher Art ist, daß ihn ein Wesen, wie das unsrige, sehr wohl beabsichtigen konnte; und wenn er endlich
\item bei einer andern Einrichtung nicht mehr im Stande wäre, diesen Erfolg zu erzeugen.
\end{aufzb}
Je größer sodann die Anzahl dieser nicht an sich nothwendigen Beschaffenheiten des Gegenstandes ist, und je weniger er, wenn auch nur einige derselben anders wären, zur Hervorbringung jener Wirkung, die wir als seinen Zweck angeben wollen, noch ferner tauglich seyn würde: um desto zuversichtlicher können wir sagen, er sey von diesem verständigen Wesen, und nur zu diesem Zwecke hervorgebracht.
\item Wenden wir dieses auf die Welt an; so haben wir
\begin{aufzb}
\item nicht den geringsten Grund, zu behaupten, daß alle jene Dinge, deren Daseyn wir aus bloßer Wahrnehmung kennen, schon \RWbet{an sich selbst}, \dh\ auch wenn wir nicht annehmen, daß sie Gott eben um seiner Vollkommenheit willen erschaffen mußte, \RWbet{nothwendig} wären. Wer könnte \zB\ beweisen, daß es aus einem andern Grunde, als weil es eben so am Zuträglichsten für das Wohl der Lebendigen, und also der Vollkommenheit Gottes am Angemessensten ist, nothwendig sey, daß die Erde gerade diese und jene Gattungen von Pflanzen und Thieren hervorbringe, \usw ?
\item Eben so wenig kann Jemand die Unmöglichkeit eines Wesens von der Art, wie wir uns Gott denken, oder die Unmöglichkeit, daß dieses Wesen allen jenen Dingen das Daseyn gegeben habe, beweisen.~\RWSeitenw{214}
\item Dagegen finden wir an allen diesen Dingen eine zahllose Menge von Einrichtungen, durch welche sie fähig werden, das Wohl der Lebendigen zu befördern.
\item Diese Beförderung des Wohles der Lebendigen ist aber eine Wirkung, welche ein Wesen, wie wir uns Gott denken, sehr wohl beabsichtigen kann, ja sogar muß.
\item Zahllos ist endlich die Menge der Einrichtungen, welche wir uns nur etwas verändert vorstellen dürfen, um zu erkennen, daß sie sodann dem Wohle der Lebendigen entweder gar nicht, oder doch nicht mehr so gut wie jetzt entsprechen würden.
\end{aufzb}
Es ist sonach kein Zweifel, daß ein Gott da sey, und daß er alle diese Dinge wirklich nur zu dem Zwecke des Wohles der Lebendigen hervorgebracht habe.
\begin{RWanm}[Anm.\,1.] 
Diese Reihe von Schlüssen pflegt man den \RWbet{physicotheologischen Beweis für Gottes Daseyn} zu nennen; während derjenige, der \RWparnr{67}\ vorkam, etwa ein \RWbet{ontologischer} genannt werden müßte.
\end{RWanm}
\begin{RWanm}[Anm.\,~2.] 
Bei dieser Gelegenheit mag noch erinnert werden, daß die in der Welt bemerkbare Tauglichkeit so vieler Einrichtungen für das Wohl der Lebendigen selbst für den \RWbet{Atheisten} sehr trostreich und erbaulich seyn müsse. Denn das Vorhandenseyn so vieler, dem Wohle der Lebendigen zusagender Einrichtungen, mag auch die \RWbet{Ursache}, der sie ihr Daseyn zu verdanken haben, diese oder jene seyn, berechtiget doch jederzeit zu dem Schlusse, daß auch diejenigen Theile der Welt, die wir nicht kennen, oder deren Zweckmäßigkeit uns nicht einleuchtet, nicht minder zweckmäßig für das Wohl der Lebendigen eingerichtet seyn mögen.
\end{RWanm}
\begin{RWanm}[Anm.\,~3.] 
Von einigen Einwürfen, die man gegen die bemerkte Zweckmäßigkeit der Welt, besonders aus dem Vorhandenseyn so vieler Uebel in ihr, herzuleiten pflegt, wird in der Folge gesprochen.
\end{RWanm}
\end{aufza}

\RWpar{84}{Unsterblichkeit der menschlichen Seele}
\begin{aufza} 
\item Eine der wichtigsten Wahrheiten für den vernünftigen Menschen, die aber in der natürlichen Religion des menschlichen Geschlechtes nicht mit vollkommener Gewißheit, sondern (so wie ich wenigstens glaube) nur mit sehr hoher Wahrscheinlichkeit aufgestellt werden kann, ist die von der \RWbet{Unsterblichkeit der Seele}.~\RWSeitenw{215}
\item Der Glaube an die Fortdauer unsers Bewußtseyns nach dem Tode ist zwar bei allen nicht völlig rohen Völkerschaften zu finden, und hieraus möchte man vielleicht den übereilten Schluß ziehen wollen, daß die Unsterblichkeit der Seele zu den ganz \RWbet{sichern Lehrsätzen} der natürlichen Religion gehöre. Aber es ist nicht zu vergessen, daß jener Glaube allenthalben nicht als ein Lehrsatz, dessen Wahrheit man durch die bloße sich selbst überlassene Vernunft einsehen könne, sondern als eine \RWbet{Offenbarung} angenommen werde. Auf dem ganzen Erdenrunde glauben die Menschen nur darum an eine Unsterblichkeit der Seele, weil sie hierüber eine Versicherung Gottes (oder der Götter) selbst zu haben glauben. Wie sie nun immer zu diesem Glauben gekommen seyn mögen; sey es, daß Gott einigen einzelnen Völkern wirklich gewisse Offenbarungen ertheilte; sey es, daß eine den \RWbet{ersten} Menschen geschehene Offenbarung durch mündliche Ueberlieferungen sich bis zu ihnen fortgepflanzt hat; sey es, daß auch vielleicht der bloße \RWbet{Wunsch}, Gott \RWbet{möchte} sich hierüber offenbaren, am Ende den Glauben, es sey wirklich geschehen, hervorgebracht habe: in jedem Falle beweiset der Umstand, daß sich die Menschen in Ansehung dieser Wahrheit nur \RWbet{auf eine Offenbarung beriefen}, deutlich genug, daß die Vernunft sich selbst zu schwach gefühlt habe, hierüber mit Gewißheit zu entscheiden. So finden wir denn auch wirklich, daß alle diejenigen Menschen, welche die Wahrheit ihrer Volksreligion in Zweifel zogen, und nun keinen andern Führer, als die \RWbet{Vernunft}, hatten, die Fortdauer ihres Geistes zwar \RWbet{hofften}, aber nicht mit völliger \RWbet{Gewißheit} annahmen. Nur einige Wenige aus diesen Weltweisen rühmten sich, entscheidende Beweise für die Unsterblichkeit der Seele gefunden zu haben, an welchen aber die Andern alsbald wichtige Mängel entdeckt haben wollten.
\end{aufza}

\RWpar{85}{Aus welchen Gründen die Unsterblichkeit der Seele gleichwohl auch ohne Offenbarung schon erwiesen werden könne}
Meiner Meinung nach fehlt es den Gründen, welche für die Unsterblichkeit der Seele vorgebracht werden können,~\RWSeitenw{216}\ nicht sowohl an der benöthigten Strenge und Verlässigkeit, als vielmehr nur daran, daß sie nicht leicht genug einzusehen sind; wozu noch kommt, daß die Menschen in einer Sache von solcher Wichtigkeit auch gegen die sichersten Schlüsse mißtrauisch werden. Diejenigen dieser Gründe, die ich meinerseits für die befriedigendsten halte, will ich hier mit wenigen Worten andeuten.
\begin{aufza} 
\item Schon \RWparnr{82}\ stellte ich es als eine Wahrheit, die Niemand bezweifeln könne, dar, daß jeder Gegenstand, der ein \RWbet{eigenes}, von andern \RWbet{abgesondertes Bewußtseyn} hat, \zB\ jeder Mensch, auch eine \RWbet{eigene}, von andern abgesonderte \RWbet{Substanz} seyn müsse. Hier ist noch nicht gesagt, daß dieser Gegenstand nur eine \RWbet{einzige} Substanz sey, wie denn auch wirklich der Mensch nicht eine einzelne Substanz, sondern nach seinem \RWbet{Körper} wenigstens eine Zusammensetzung \RWbet{unzählig vieler Substanzen} ist. Aber nicht alle diese Substanzen, die wir zu unserem Körper und sonach zum \RWbet{Menschen} selbst zählen, gehören zu demjenigen Subject in uns, dem unser Bewußtseyn zukommt, oder das unser \RWbet{eigentliches Ich} ausmacht; sondern es läßt sich im Gegentheil darthun, daß dieses eigentliche Ich und überhaupt ein jedes einzelne Subject, welches \RWbet{Bewußtseyn} hat, und sich daher nur als ein \RWbet{einziges Ich} denkt, an und für sich genommen, auch nur \RWbet{eine einzige Substanz sey}, die eben deßhalb den Namen einer \RWbet{geistigen} verdient, während wir solche Substanzen, die gar kein Bewußtseyn, und nicht einmal die geringste Vorstellungskraft besitzen (falls es dergleichen gibt), \RWbet{materielle} Substanzen zu nennen pflegen.
\item Bloß aus dem Bewußtseyn der \RWbet{Identität}, das wir ein Jeder von unserem Ich haben, \dh\ aus dem Bewußtseyn, daß eben dasselbe Ich, das diese Eine Vorstellung oder Empfindung gehabt hat, auch jene andere habe, folgt schon, daß unzählig viele Theile, die sich an uns (\dh\ an unserem Körper) befinden, nicht unser eigentliches Ich ausmachen. Denn weil es ein und dasselbe Ich ist, das so verschiedenartige Vorstellungen und Empfindungen in sich faßt, welches \zB\ jetzt etwas sieht, jetzt etwas hört, \usw , so muß dieses Ich ein Ding von solcher Art seyn, welches bei~\RWSeitenw{217}\ allen diesen Vorstellungen eine Veränderung erleidet. Dieses gilt aber von keinem der größeren sichtbaren Theile unsers Körpers, die nur bei Vorstellungen gewisser Art, nicht aber bei allen Vorstellungen, die wir haben, eine Veränderung erleiden. So wird \zB\ unser Auge wohl bei Vorstellungen, die wir von Farben, allein nicht bei denjenigen, die wir von Tönen haben; unser Ohr dagegen wohl bei Vorstellungen, die wir von Tönen, nicht aber bei denjenigen, die wir von Farben haben, gerührt; \usw\ Also kann weder Auge, noch Ohr einen Bestandtheil unsers eigentlichen Ichs ausmachen. Weil wir uns ferner, so weit unsere Rückerinnerung in die verflossenen Jahre unsers Lebens reicht, bewußt sind, immer \RWbet{dieselben} zu seyn: so muß dasjenige, was unser eigentliches Ich ausmacht, fortdauernd auch dieselbe \RWbet{Eine}, oder falls es aus \RWbet{mehren} besteht, derselbe Inbegriff mehrer Substanzen seyn. Allein bekanntlich können wir beträchtliche Theile unsers Körpers, \zB\ ganze Gliedmaßen, Sinneswerkzeuge, selbst Theile des Gehirnes durch einen Zufall verlieren, ohne daß jenes Bewußtseyn der Einerleiheit unsers jetzigen Ichs mit dem vorigen aufhört. Die Physiologie versichert sogar, daß bloß durch die regelmäßigen, theils sichtbaren, theils unsichtbaren Aussonderungen, die bei dem menschlichen Körper unaufhörlich Statt finden, in einem Zeitraume von etwa zehn Jahren fast alle die einzelnen Theilchen, aus denen er Anfangs bestand, entfernt, und dafür andere an ihre Stelle gesetzt würden. Unmöglich kann also irgend einer von diesen \RWbet{wandelbaren} Theilen zu unserm eigentlichen Ich gehören.
\item Damit ist aber freilich noch nicht erwiesen, daß es nicht vielleicht doch \RWbet{einige} Theile unsers Körpers (materielle Substanzen) gebe, die uns unwandelbar von unserer Kindheit an beiwohnen, und zu unserem Ich gehören. Auch diese Vermuthung wird wegfallen, wenn ich nun darthue, daß \RWbet{Empfindungen, Gedanken, Wünsche} und \RWbet{Willensentschließungen} überhaupt nie in \RWbet{materiellen Substanzen} vorgehen können. Um dieses einzusehen, müssen wir erst den Unterschied fassen, der zwischen dem \RWbet{Hervorbringen} einer Wirkung \RWbet{durch} eine gewisse Substanz, und zwischen dem \RWbet{Vorgehen} derselben in einer Substanz obwaltet. Eine Wir\RWSeitenw{218}kung kann nämlich durch die vereinigte Thätigkeit mehrer Substanzen $A$, $B$, $C$, \ldots\ \RWbet{hervorgebracht} werden, ohne daß sie in allen \RWbet{vorgehen} müßte. So ist \zB\ das Hören eines Schalles, wenn die Uhr schlägt, eine Wirkung, die durch die vereinigte Thätigkeit sehr vieler Substanzen hervorgebracht wird; denn zur Erzeugung derselben müssen so viele Substanzen, aus denen die Uhr, ferner die zwischen der Uhr und meinem Körper befindliche Luftschichte, mein eigenes Ohr, \usw\ zusammengesetzt ist, das Ihrige beitragen. Sicherlich sind es nicht alle diese Substanzen, \RWbet{in denen} das Hören dieses Schalles \RWbet{vorgeht}; denn Niemand wird zweifeln, daß dieses Hören nicht in der Uhr, nicht in der Luft, sondern nur in demjenigen Dinge, das ich mein eigentliches \RWbet{Ich} nenne, vorhanden sey. Nach dieser Vorerinnerung behaupte ich nun, daß \RWbet{Empfindungen, Gedanken, Wünsche} und \RWbet{Willensentschließungen}, wenn sie auch Wirkungen sind, zu deren \RWbet{Hervorbringung} bei uns materielle Substanzen mitwirken, doch in solchen Substanzen nie \RWbet{vorgehen} können. Dieses scheint nämlich schon daraus zu folgen, weil alle Wirkungen oder Veränderungen, die in \RWbet{materiellen} Substanzen vorgehen, Veränderungen in einem \RWbet{Raume}, \dh\ \RWbet{Bewegungen} sind.\RWfootnote{%
	Nicht immer Bewegungen \RWbet{allein}, aber doch \RWbet{auch} Bewegungen. So sind \zB\ chemische und organische Erscheinungen solche Wirkungen, die in materiellen Substanzen vorgehen, und \RWbet{mehr} als bloße Bewegungen in denselben, aber doch \RWbet{auch} Bewegungen sind.}
Nun ist es aber gewiß, daß Empfindungen, Gedanken, Wünsche und Willensentschließungen, wenn sie gleich durch Bewegungen materieller Substanzen hervorgebracht werden, auch dergleichen wieder veranlassen können, doch an sich selbst nichts weniger als Bewegungen sind. Daher kann auch der Gegenstand, in welchem diese Wirkungen vorgehen, keine materielle Substanzen enthalten.
\item Ja er darf überhaupt nicht aus \RWbet{mehren} Substanzen zusammengesetzt, sondern er muß nur eine \RWbet{einzige} Substanz seyn. Dieses erhellt meines Erachtens daraus, weil alle Wirkungen, welche in einer Zusammensetzung mehrer Substanzen vorgehen, auch selbst zusammengesetzt und also zerlegbar seyn müssen. Wohl läßt es sich denken, daß eine einzige Substanz der Wirkungen \RWbet{mehre} habe;~\RWSeitenw{219}\ daß aber eine Wirkung, die doch in \RWbet{mehren} Substanzen vorgeht, nur einfach, und nicht in mehre zerlegbar sey, ist undenkbar. Denn sagen, daß eine gewisse Wirkung $M$ in den Substanzen $A$, $B$, $C$, \ldots\ \RWbet{vorgehe}, heißt doch wohl sagen, daß in jeder aus ihnen eine \RWbet{gewisse} Wirkung, ein \RWbet{Theil} von $M$, vorgehe; und also muß $M$ wenigstens in so viele Theile, als es Substanzen $A$, $B$, $C$, \ldots\ gibt, zerlegbar seyn. So ist \zB\ der Tanz eine Wirkung, welche in einer Verbindung mehrer Personen vorgeht, und ist auch eben deßhalb eine zusammengesetzte Wirkung, die wir in die Bewegungen jedes einzelnen Mittänzers zerlegen können. Nicht also ist es mit den Empfindungen, Gedanken, Wünschen und Willensentschließungen, welche wir haben. Unter diesen nämlich gibt es gewiß einige, die durchaus einfach sind, und nicht weiter zerlegt werden können. So ist ein einzelner Willensentschluß eine ganz \RWbet{einfache} Wirkung; so kann eine einzelne Vorstellung von einem \RWbet{einfachen} Begriffe gewiß nicht in mehrere Vorstellungen zerlegt werden; \usw\ Nothwendig also muß auch dasjenige, worin diese einfachen Empfindungen, Gedanken, \usw\ vor sich gehen, eine ganz \RWbet{einfache} Substanz seyn.
\item Substanzen, einfache Substanzen, können nun, wie mir däucht, weder entstehen noch vergehen; sondern, wofern sie einmal vorhanden sind, sind sie immer vorhanden; entweder ohne einer Bedingung ihres Daseyns zu bedürfen, wie Gott, oder -- wie alle übrigen Substanzen -- durch Gottes \RWbet{Wirksamkeit}. Unsere \RWbet{Seele} muß also durch Gottes Wirksamkeit immer bestehen.
\item Aber freilich nicht immer in einerlei \RWbet{Zustande}; denn der Zustand, in dem sich ein geistiges Wesen wie unsere Seele befindet, hängt der Erfahrung zu Folge gar sehr ab von der Beschaffenheit desjenigen Leibes, mit dem es verbunden ist, und von den übrigen Dingen, welche durch ihn auf dasselbe einwirken. Es fragt sich also, welche Veränderung die menschliche Seele besonders dann zu erwarten habe, wenn ihr jetziger Leib durch den Tod zerstört wird? In Hinsicht auf diese Frage nun bemerke ich zuerst, es sey nicht wahrscheinlich, daß irgend eine Substanz, die einmal geistige~\RWSeitenw{220}\ Kräfte hat, diese je wieder verliere; sondern, wofern es ja Sustanzen gibt, die gar keine geistigen Kräfte haben, so sind sie den geistigen so ganz entgegengesetzt, daß kein Uebergang aus der einen in die andere Classe Statt findet. Es scheint mir nämlich, daß alle \RWbet{Veränderungen}, die eine Substanz erleiden kann, ohne vernichtet zu werden, nur darin bestehen, daß die in ihr bereits vorhandenen Kräfte entweder vermehrt oder vermindert, nicht aber daß sie durchaus vernichtet und mit ganz andern ausgetauscht würden. Eine Substanz also, die einmal geistige Kräfte besitzt, muß, wie ich glaube, \RWbet{immer} dergleichen besitzen und besessen haben. Wohl können diese Kräfte zu einer gewissen Zeit sehr unentwickelt gewesen, zu einer andern schon mehr ausgebildet seyn; wohl kann sich die Substanz einmal in solchen Umständen, die den Gebrauch ihrer Kräfte beschränken, ein andermal wieder in solchen, die ihn erweitern, befinden; nie aber können ihr diese Kräfte selbst gänzlich entzogen werden.
\item Wenn aber dieses ist, dann können wir wohl schon aus der \RWbet{Zweckmäßigkeit}, die wir in allen Einrichtungen des Weltalls antreffen, mit vieler Wahrscheinlichkeit schließen, daß keine geistige Substanz (es müßte denn etwa nur durchaus ihr eigenes Verschulden seyn) jemals in einen solchen Zustand gerathe, in dem es ihr für alle künftige Zeit unmöglich würde, die in ihr schlummernden Kräfte und Fähigkeiten zu entwickeln, und durch ihren Gebrauch sich selbst und Andere zu beglücken. Wir dürfen dieß schließen, weil es höchst unwahrscheinlich ist, daß irgend eine Substanz in der Welt eine Kraft oder Fähigkeit erhalten habe, die sie entweder \RWbet{schlechterdings niemals}, oder doch anzufangen von einem gewissen Zeitpuncte ihres Daseyns nie wieder gebrauchen kann. Und so dürfen wir auch in Rücksicht auf \RWbet{uns selbst} erwarten, daß unsere Seele, wenn dieser Leib zerfällt, über kurz oder lang, hier auf Erden oder irgendwo anders, in solchen Verhältnissen wieder aufleben werde, wo es ihr möglich werden soll, die in ihr schlummernden Kräfte noch vollkommener, als es in diesem Leben geschah, zu entwickeln, und sich auch einen Wirkungskreis, der diesen Kräften angemessen ist, zu verschaffen. Ja strenge genommen scheint es schon aus der bloßen \RWbet{Natur einer Vorstel}\RWSeitenw{221}\RWbet{lungskraft} zu folgen, daß Wesen, welche mit dieser Kraft versehen sind, immer Gelegenheit zu ihrer Ausbildung, und dadurch auch Gelegenheit, noch immer vollkommener zu werden, finden. Vorstellungen nämlich muß eine Substanz, welche mit Vorstellungskraft versehen ist, überall sich zu verschaffen vermögen, ja zum Theil wider Willen erhalten. Da aber jede Vorstellung eine gewisse Spur und Nachwirkung von sich zurückläßt; so muß durch diese beständige Uebung der Vorstellungskraft das Vermögen der Vorstellungen selbst immer größer werden. Wächst nun dieß Eine Vermögen, welches die Grundlage zu allen übrigen ist, beinahe wider Willen in uns unaufhörlich: wie sollten wir nicht erwarten, daß sich uns auch Gelegenheit zur Entwicklung aller unserer übrigen Kräfte darbieten werde?
\item Diese Erwartung muß um so zuversichtlicher werden, wenn wir die vielen \RWbet{Einrichtungen} in der Natur betrachten, die mit dem \RWbet{Wiederaufleben in einem vollkommeneren Zustande} die größte Aehnlichkeit haben. Alle lebendigen Wesen auf Erden entstehen aus organischen Keimen, die früher da waren, aber erst zur Entwicklung gelangten, als irgend ein äußerer Reiz, \zB\ die Wärme der Sonnenstrahlen \udgl\ dazu kam. Nimmt man nun an, daß die Seelen dieser lebendigen Wesen (denn \RWbet{Seelen} müssen wir aus dem \no\,4 angegebenen Grunde allerdings \RWbet{allen} empfindenden Wesen, auch den \RWbet{Thieren} beilegen) mit den organischen Keimen, aus denen sie sich entwickeln, von jeher verbunden gewesen: so war der Zustand, in dem sich diese Seelen früher befanden, gewiß viel unvollkommener, als derjenige, in den sie durch die Entwicklung des Keimes übergehen; und diese Erscheinung gibt uns also ein Beispiel, wie eine geistige Substanz aus einem unvollkommeneren Zustande in einen vollkommeneren übergehe. Nimmt man dagegen an, daß sich die geistige Substanz mit dem organischen Keime erst um die Zeit seiner Entwicklung verbinde: so gibt diese Erscheinung einen Beweis, daß sich geistige Substanzen im Verlaufe der Zeit auch wohl mit neuen organischen Leibern vereinigen können; und wir dürfen dann hoffen, daß unsere Seele, wenn sie den gegenwärtigen Leib im Tode verläßt, in einer andern Welt einen neuen auffinde, in welchem sie ein vollkommeneres Leben, als das ver\RWSeitenw{222}flossene war, anfängt. Allein noch mehr, auf dem ganzen Erdenrunde lehrt die Beobachtung uns ein stetes Fortschreiten von dem unvollkommeneren Zustande zum vollkommeneren. Aus unorganischen Massen entstehen (durch Gährung, Nahrung und andere Mittel) Gebilde, die auf der niedrigsten Stufe der Organisation stehen, aus diesen allmählich immer vollkommenere Wesen, Pflanzen, Thiere, \usw\ Kein Zweifel also, daß unsere Seele auch bei dem Tode unsers Leibes nur wieder in einen andern vollkommeneren Zustand übergehe.
\item Aber wird sie in ihrem künftigen Leben sich auch noch des Lebens auf Erden, hier geübter Thaten und erlebter Schicksale erinnern? Wenn es wahr seyn sollte, daß die Seele lange vor unserer Geburt bereits vorhanden gewesen; so erzeugt der Umstand, daß wir uns gleichwohl dieses frühern Daseyns gar nicht erinnerlich sind, die Besorgniß, daß es im künftigen Leben derselbe Fall seyn werde. Allein diese Besorgniß verschwindet wieder, wenn wir, wie billig, annehmen, daß unsere früheren Zustände alle noch unvollkommener waren als dieser gegenwärtige, und daß dieser der \RWbet{erste} sey, in dem wir ein \RWbet{deutliches Bewußtseyn unserer selbst}, Vernunft und Freiheit erlangten. Hatten wir früher noch keine Vernunft, so konnten wir auch noch keine eigentlichen Erfahrungen machen, nichts lernen, nichts thun, dessen uns deutlich zu erinnern jetzt einen Nutzen für uns hätte; und darum brauchte uns Gott die Rückerinnerung an jene früheren Zustände auch nicht mitzugeben, oder vielmehr, es war noch gar nicht möglich, uns diese mitzugeben, weil wir uns jener Zustände, selbst als sie gegenwärtig waren, nicht deutlich bewußt gewesen.
\item 	Aber von jetzt an, seit dem wir \RWbet{Vernunft} und \RWbet{Freiheit} besitzen, wäre es zweckwidrig, wenn Gott bei unserem Austritte aus dieser Welt eine solche Veränderung mit uns vorgehen ließe, bei der wir alle Rückerinnerung auf unser Erdenleben verlören; denn
\begin{aufzb}
\item wenn wir anders das Alter, in dem die Vernunft sich entwickelt, auf Erden zurückgelegt haben; so haben wir auch ein Jeder so manche Erfahrungen gemacht, so man\RWSeitenw{223}che Ueberzeugungen gewonnen, die uns in jenem künftigen Leben, wie es auch immer beschaffen seyn mag, zu Statten kommen müssen; \zB\ die Erfahrung, daß sich das Gute fast immer belohne, das Böse dagegen bestrafe; die Ueberzeugung, daß ein Gott sey, der mit unendlicher Weisheit Alles erschaffen hat, Alles erhält und regieret, \usw\ Dieß Alles sind Kenntnisse, die sicher in jeder Welt brauchbar seyn müssen, die aber verloren gehen und wieder von Neuem erlernt werden müßten, wenn wir gar keine Erinnerung an dieses Leben in das andere mitnehmen würden.
\item So manches Gute, das wir auf Erden zu Stande gebracht, könnte uns wohl noch in jenem andern Leben freuen, wenn wir die Rückerinnerung daran behielten.
\item Hat uns die göttliche Vorsehung auf Erden in verschiedene besonders günstige Verhältnisse gesetzt, durch die es uns möglich ward, unsere Kräfte zu einem seltenen Grade der Vollkommenheit zu entwickeln; so wird die Rückerinnerung hieran im andern Leben uns ein verstärkter Antrieb, diese Kräfte auf das Gewissenhafteste für die Beförderung des allgemeinen Wohles zu benützen.
\item Ein Aehnliches muß erfolgen, wenn wir uns erinnern, daß wir hier manches Böse gethan, Manchen beschädigt \udgl\
\item Sollte es vollends der Fall seyn, daß wir im andern Leben mit eben denselben Menschen, mit denen wir schon hier auf Erden gelebt, die hier unsere Wohlthäter oder geliebten Freunde waren, oder von uns beleidigt wurden \udgl , wieder zusammenkommen: so wäre der Vortheil, den die Rückerinnerung an jene früheren Verhältnisse, und das \RWbet{Erkennen} dieser Menschen hätte, besonders einleuchtend.
\item Durch diese Einrichtung würde ferner der Fürsehung Gottes auch eine vollkommenere Weltregierung möglich. Denn wenn wir im andern Leben keine Rückerinnerung hätten, so würde dort jede Belohnung oder Strafe für etwas, das wir in diesem Leben gethan, zweckwidrig seyn, und folglich müßte sich Gott die Regel machen, schon hier auf~\RWSeitenw{224}\ Erden alles genügend zu vergelten. Um dieses zu erreichen, müßte er weit öfter Eingriffe in den natürlichen Lauf der Dinge machen, und würde sich überhaupt viel enger beschränkt sehen in seiner Weltregierung als im entgegengesetzten Falle, wo er die Vergeltung bald in dem gegenwärtigen Leben erfolgen lassen, bald in das künftige verschieben darf; und so nach Umständen bald das Eine bald wieder das Andere thun kann, wie sich bald dieses bald jenes besser schickt.
\item Ja genau betrachtet, ist es nicht einmal möglich, daß eine ganz allgemeine und vollständige Vergeltung alles Guten und Bösen Statt finde, wenn es kein anderes Leben und keine Rückerinnerung in demselben gibt. Denn es gibt gute Handlungen, die ihrer Natur nach auf Erden nur Leiden, wohl gar den Tod herbeiziehen, \zB\ wenn sich Jemand zur Rettung Anderer opfert. Solche gute Handlungen könnte Gott unmöglich nach ihrem Werthe belohnen, wenn es kein anderes Leben und keine Rückerinnerung gäbe. Der Lasterhafte dagegen, wenn er des Bösen erst recht viel verübt hätte, könnte sich durch einen schnellen Selbstmord allen einbrechenden Folgen seiner bösen Thaten entziehen, und es so gleichsam Gott selbst unmöglich machen, ihn zu bestrafen.
\end{aufzb}
\item Betrachten wir endlich, \RWbet{wie der Lauf des Schicksals in der Welt wirklich beschaffen sey}, so wird unser Glaube an Unsterblichkeit durch ihn noch mehr bestätigt. Denn wirklich herrscht doch ein großes und, wie es scheint, weit größeres \RWbet{Mißverhältniß zwischen der Tugend und Glückseligkeit} auf Erden, als es seyn müßte. Die besten, die tugendhaftesten Menschen sind oft die unglücklichsten! Wir sehen, daß gerade sie meistens die wenigsten Freuden des Lebens genießen, bei den gemeinnützigsten Unternehmungen den stärksten Widerstand erfahren, und eines frühzeitigen Todes sterben. Böse und lasterhafte Menschen dagegen sehen wir oft recht begünstigt vom Glücke. Alles, was ihnen nur wünschenswerth dünkt, die verschiedensten Arten der sinnlichen Lüste, Reichthum und Ehre und hohes Lebens\RWSeitenw{225}alter wird ihnen zu Theil; und kaum bemerken wir, daß sie auch nur durch einige Gewissensangst in ihrem Lebensgenusse gestört, und beim herannahendem Tode beunruhiget würden. Dergleichen Ereignisse scheint Gott eigends zuzulassen, um uns hiedurch zu verstehen zu geben, daß es ein Land der Vergeltung jenseits der Gräber gebe.
\end{aufza}
\begin{RWanm}
Es ist noch nöthig, die wichtigsten Gründe, die man dem Glauben an Unsterblichkeit entgegengestellt hat, kennen zu lernen. Gewisse Verletzungen an unserem Körper, sagt man, rauben der Seele das Vermögen des Denkens, und versetzen sie in eine Art bewußtlosen Zustandes, welchen man \RWbet{Ohnmacht} nennt, wenn er vorübergehend ist, der aber vom wirklichen \RWbet{Tode} durch nichts, als durch die Fortdauer des letzteren unterschieden zu seyn scheint. Wie also die Seele im Zustande einer Ohnmacht einige Minuten oder Stunden hindurch bewußtlos ist: so verliert sie beim Tode ihr Bewußtseyn für immer. Dieses ist auch um so begreiflicher, da, wie die Beobachtungen der Psychologen lehren, die Seele zu jeder Vorstellung eines eigenen Organs und einer gewissen Bewegung desselben bedarf, und da die Rückerinnerung an schon gehabte Vorstellungen dem Mechanismus jener Organe folgt. Menschen, die einen Theil ihres Gehirnes durch Zufall einbüßten, verloren mit diesem auch einen Theil ihrer Begriffe und Erinnerungen. Beim Tode also, wo unser ganzer Körper zerstört wird, muß unsere Seele nothwendig entweder ganz aufhören zu denken und zu seyn, oder sie wird, falls sie mit einem andern Körper verbunden werden sollte, doch keine Rückerinnerung an dieses Erdenleben behalten. -- \par
Auf diese Einwürfe antworte ich:\par
\begin{aufza}
\item Das Denken hört in einer Ohnmacht eben so wenig als im Schlafe ganz auf, sondern wir können uns nur desjenigen, was wir in einem solchen Zustande gedacht, bei unserem Erwachen nicht mehr erinnern. Denn da die Rückerinnerung an einmal gehabte Vorstellungen von sehr zufälligen Umständen abhängt, \zB\ von ihrer größeren oder geringeren Lebhaftigkeit, von ihrem Zusammenhange und ihrer Aehnlichkeit mit unseren übrigen Vorstellungen, \usw ; da wir uns auch so mancher anderer Vorstellungen, die wir doch unläugbar gehabt, nicht wieder erinnern; auf einige nur nach einer langen Zeit, bei der zufälligen Entstehung einer ähnlichen, erinnert werden: so schließen wir~\RWSeitenw{226}\ wohl mit Recht, daß unsere Seele auch in dem festesten Schlafe und in der tiefsten Ohnmacht denke, ob wir gleich diese Gedanken beim Erwachen nicht wissen.
\item Doch wie dieß auch sey; so sind die Erscheinungen des Schlafes und der Ohnmacht für unsern Glauben an Unsterblichkeit eher begünstigend, als daß sie ihn umstoßen sollten. Denn wenn die Aehnlichkeit zwischen Tod und Schlaf oder Ohnmacht wirklich so groß ist: so dürfen wir aus dem Tode wohl eben so, wie aus dem Schlafe oder der Ohnmacht ein Erwachen erwarten, welches dann eintreten wird, wenn ein hinlänglicher Reiz von Außen auf unsere Seele eingewirkt hat. Und wie bei dem Schlafenden das Tageslicht, bei dem Ohnmächtigen irgend ein starker Geruch \udgl\ dieses Reizmittel abgibt: so muß es wohl auch für den Verstorbenen hinlänglich starke Reizmittel geben, die ihn zu einem neuen Leben, nicht zwar in dieser, aber in einer andern Welt erwecken.
\item Daß die Seele gewisser Organe bedürfe, um mit der Welt in Verbindung zu stehen, mag seine Richtigkeit haben; daß sie aber ohne Organe und ohne eine bestimmte Bewegung derselben gar keine Vorstellungen haben könne, und daß ihre Organe keineswegs gewechselt werden dürften, wenn sie die Rückerinnerung an einmal gehabte Vorstellungen und das Bewußtseyn ihrer Identität nicht verlieren soll: das alles hat noch Niemand erwiesen. Es könnte ja seyn. daß die Organe des Leibes der Seele nur nöthig wären, um ihr einen gewissen Vorrath von Vorstellungen zuzuführen, den sie nun unabhängig von ihnen weiter verarbeiten kann. Das Factum aber, daß sie sich dieser Organe gleichwohl bei all ihrem gegenwärtigen Denken noch fortbedient, ließe sich daraus erklären, weil sie es einmal gewohnt ist, und weil es ihr leichter fällt; etwa eben so, wie das Kind die Finger, an denen es zählen gelernt hat, beim Zählen noch immer in eine gewisse Bewegung setzt, obgleich es derselben nicht nöthig hätte. Können wir aber auch ohne Organe denken; so ist kein Zweifel, daß wir in Zukunft, wenn wir mit einem ganz neuen Leibe vereinigt werden, die Rückerinnerung an unsere wichtigsten Begriffe aus früherer Zeit noch immer behalten, und wenn dieß ja nöthig seyn sollte, sie zum Behufe für unser Gedächtniß auch an gewisse Organe desselben anknüpfen werden. Es wird uns ergehen, wie einem Menschen, der sein Notatenbuch verliert. Er hat darum nicht Alles, was~\RWSeitenw{227}\ in demselben stand, vergessen; macht sich ein neues, und trägt die wichtigsten Notaten aus dem alten in das neue über.
\item Gesetzt aber, daß die Seele gewisser Organe zum Denken schlechterdings nöthig hätte, und daß die Materie derselben durchaus nicht dürfte geändert werden, soll die Seele das Bewußtseyn ihrer Identität behalten: auch so noch wird aus den Erscheinungen, die sich beim Tode äußern, nichts gegen die Unsterblichkeit der Seele und unsere Rückerinnerung folgen. Denn wissen wir wohl, wie groß oder klein an Umfang und Gewicht jene \RWbet{nothwendigen Organe} sind? Und können wir behaupten, daß bei der Verwesung im Tode durchaus \RWbet{alle} Organe des Leibes aufgelöst werden? Können, ja müssen nicht einige feinere Theile in ihrer vorigen Verbindung bleiben, und sich, von unsern Sinnen unbemerkt, dieser Erde entziehen, in jene Gegenden hinüber, in denen wir erneuert aufleben sollen? -- Keine Zerstörung in der Natur ist eine Auflösung in durchaus einfache Theile; also auch die Zerstörung, welche der Tod bei uns verursacht, trennt unsere Seele nicht von allen Theilen des Körpers, sondern die feineren Theile desselben, diejenigen, die uns zur Rückerinnerung an dieses gegenwärtige Leben und zu einer ununterbrochenen Fortsetzung unserer Thätigkeit nothwendig sind, wird uns kein Tod entreißen.
\end{aufza}
\end{RWanm}
\clearpage

\RWabs{Zweiter Abschnitt}{Natürliche Moral}

\RWpar{86}{Inhalt dieser Abtheilung}
Die Zeit erlaubt nicht, einen auch noch so kurzen Umriß der \RWbet{einzelnen Pflichten} des Menschen, wie sie durch bloße Vernunft erkennbar sind, zu entwerfen. Nur Zweierlei will ich also in dieser Abtheilung leisten: \RWbet{erstlich} die Wahrheit aufstellen, die, meiner Meinung nach, der oberste Grundsatz der ganzen Sittenlehre oder das \RWbet{oberste Sittengesetz} ist; dann einige besonders wichtige Folgerungen, die sich aus dieser Wahrheit in der Lehre von den \RWbet{Tugendmitteln} ergeben, ableiten.~\RWSeitenw{228}

\RWpar{87}{Begriff und Daseyn eines obersten Sittengesetzes}
\begin{aufza}
\item Unter dem \RWbet{obersten Sittengesetze} verstehe ich eine praktische Wahrheit, aus der sich jede andere praktische Wahrheit (also auch jede einzelne Pflicht, die den Menschen betrifft) \RWbet{objectiv}, \dh\ so wie die Folge aus ihrem \RWbet{Grunde} ableiten läßt.
\item Daß es ein solches oberstes Sittengesetz gebe, erweise ich so:
\begin{aufzb}
\item Es gibt doch überhaupt einige praktische Wahrheiten. Wer dieses läugnen wollte, widerspräche dem Urtheile des gesundenen Menschenverstandes, der auf dem ganzen Erdenrunde mit der größten Zuversicht annimmt, daß es gewisse Handlungen, die man \RWbet{ausüben soll}, und wieder andere, die man \RWbet{nicht ausüben soll}, gebe. Diese Meinung kann auch um so weniger ein leeres Vorurtheil seyn; da ein geringes Nachdenken zeigt, daß der Begriff des \RWbet{Sollens} nicht ein zusammengesetzter, sondern ein durchaus einfacher Begriff sey. Nur Begriffe, die zusammengesetzt sind, können zuweilen einen innern Widerspruch enthalten, um deßwillen es unmöglich ist, daß man sie irgend einem Gegenstande als Prädicat beilege, obgleich dieß zuweilen erst durch ein langes Nachdenken einleuchtet. Ein Beispiel von einem solchen Begriffe wäre der eines Dreieckes mit drei rechten Winkeln, oder der Begriff eines Gottes, der als ein unbedingtes Wesen doch \RWbet{Leidenschaften} hätte \udgl\  Einfache Begriffe aber können eben darum, weil sie nur einfach sind, nie einen innern Widerspruch enthalten; sie sind daher alle \RWbet{reell}, \dh\ können als Prädicat auf gewisse Gegenstände angewendet werden. Auch der Begriff des \RWbet{Sollens} muß sich also auf gewisse Gegenstände anwenden lassen, \dh\ es gibt Handlungen, von denen mit Wahrheit gesagt werden kann, daß man sie ausüben \RWbet{soll}, und eben so andere, die man \RWbet{nicht soll}.
\item Alle praktischen Wahrheiten sind in der Form: \RWbet{$A$ soll gewollt werden}, enthalten. Denn nicht das wirk\RWSeitenw{229}liche \RWbet{Hervorbringen}, sondern das bloße
\RWbet{Wollen} ist der unmittelbare Gegenstand, auf den sich ein jedes praktische Urtheil bezieht. (\RWparnr{15}\ \no\,11.)
\item Wenn es zwei Wahrheiten gibt, eine praktische von der Form: \RWbet{$A$ soll gewollt werden}, und eine theoretische von der Form: \RWbet{Wer $A$ will, muß auch $B$ wollen} (nämlich weil $B$ ein Mittel zur Hervorbringung von $A$ ist, oder doch wenigstens dafür gehalten wird); so folgt aus diesen beiden Wahrheiten die neue praktische: \RWbet{$B$ soll gewollt werden}. Und diese neue Wahrheit ist in den beiden vorigen auf eine \RWbet{objective} Art gegründet; sie selbst ist mithin eine bloße \RWbet{Folgewahrheit}.
\item Wenn sich die Wahrheit: \RWbet{$S$ soll gewollt werden}, aus der Wahrheit: \RWbet{$R$ soll gewollt werden}, diese aus der Wahrheit: \RWbet{$P$ soll gewollt werden} \usw\ objectiv herleiten läßt; so muß es in dieser Reihe von Wahrheiten, deren die Eine durch die Andere bedingt ist, immer Eine Wahrheit: \RWbet{$A$ soll gewollt werden}, geben, die keine weitere Bedingung hat. Denn sicher kann es im Reiche der Wahrheiten keine Reihe von Folgen ohne ersten Grund geben. Die bedingte Wahrheit besteht nur, weil es unbedingte Wahrheiten gibt. Eine solche Wahrheit nun, die sich wie jene: \RWbet{$A$ soll gewollt werden}, aus keiner andern auf die beschriebene Art ableiten läßt, will ich eben deßhalb eine \RWbet{ursprüngliche} oder \RWbet{unbedingte praktische} Wahrheit nennen.
\item Es gibt also wenigstens Eine ursprüngliche praktische Wahrheit; und wenn es \RWbet{nur} eine einzige gibt: so ist diese selbst schon das \RWbet{oberste Sittengesetz}; denn alle übrigen praktischen Wahrheiten lassen sich aus ihr objectiv herleiten.
\item Gesetzt aber, es gebe mehre solche ursprüngliche praktische Wahrheiten, \zB\ \RWbet{$A$ soll gewollt werden, $A'$ soll gewollt werden}, \usw : so könnte man alle diese in eine einzige zusammenfassen, nämlich: \RWbet{$A$ und $A'$ \usw\ soll gewollt werden}. Aus dieser Wahrheit ließen sich dann alle übrigen praktischen Wahrheiten~\RWSeitenw{230}\ objectiv herleiten. So wäre denn sie das oberste Sittengesetz. Ein oberstes Sittengesetz gibt es daher in einem jeden Falle.
\end{aufzb}
\end{aufza}

\RWpar{88}{Herleitung dieses obersten Sittengesetzes}
\begin{aufza}
\item Was Jemand \RWbet{wollen} soll, muß ihm auch \RWbet{möglich} seyn, oder er muß es wenigstens für möglich \RWbet{halten}. Wenn wir daher alle, nicht nur dem Menschen, sondern auch selbst dem vollkommensten Wesen \RWbet{mögliche} Wirkungsarten durchgehen: so kann es nicht fehlen, daß sich darunter auch die \RWbet{Eine} oder die \RWbet{mehren}, welche die Vernunft \RWbet{unbedingt} gebietet, befinden. Haben wir diese entdeckt, so ist durch sie zugleich das oberste Sittengesetz gegeben. Lasset uns daher alle diese Wirkungsarten der Reihe nach betrachten, und den gesunden Menschenverstand befragen, welche derselben er unbedingt gebiete.
\item Aus den Erörterungen, die schon \RWparnr{75}\ angestellt wurden, läßt sich entnehmen, daß \RWbet{alle Wirkungen}, welche nur irgend ein vernünftiges Wesen hervorzubringen vermag, unter folgende Classen gebracht werden können:
\begin{aufzb}
\item Das \RWbet{Schaffen}, durch welches einer Substanz das Daseyn selbst gegeben wird;
\item \RWbet{Einwirkungen auf den Zustand lebloser Wesen};
\item \RWbet{Einwirkungen auf den Zustand lebendiger Wesen}, und zwar entweder
\begin{aufzc}
\item auf ihr \RWbet{Empfindungsvermögen}, indem man angenehme oder unangenehme Empfindungen in denselben erzeugt; oder
\item auf ihr \RWbet{Erkenntnißvermögen}, indem man bald diese, bald jene Vorstellungen, Begriffe und Erkenntnisse; oder
\item auf ihr \RWbet{Willensvermögen}, indem man bald diese bald jene Willensentschließungen; oder
\item auf ihr \RWbet{Begehrungsvermögen}, indem man bald diese bald jene Begierden in ihnen befördert, oder verhindert.~\RWSeitenw{231}
\end{aufzc}
Jede hier nicht ausdrücklich genannte Wirksamkeit läßt sich auf Eine oder etliche der hier genannten zurückführen. So läßt sich \zB\ der Einfluß, den wir auf die nach Außen wirkenden Kräfte eines Andern haben, auf eine der vorhin genannten Wirkungen zurückführen, je nachdem wir durch ihn eine Veränderung entweder bloß in leblosen Substanzen, oder in lebendingen, in ihrem Empfindungs-, Erkenntniß-, Begehrungs- oder Willensvermögen hervorbringen; \usw\
\end{aufzb}
\item Um nun aus diesen verschiedenen Arten des Wirkens die \RWbet{eine} oder die \RWbet{etlichen} herauszufinden, die von der Vernunft unbedingt geboten werden, in deren Vereinigung also das oberste Sittengesetz besteht: fange ich mit Untersuchung derjenigen an, von denen es am Klärsten einleuchtet, daß sie in diese Classe \RWbet{nicht} gehören.
\begin{aufzb}
\item \RWbet{Das Schaffen wird durch die Vernunft nicht unbedingt gefordert}. Das Schaffen \RWbet{lebloser} Dinge kann einleuchtender Weise nur um \RWbet{lebendiger} willen von der Vernunft gefordert werden; und das Erschaffen \RWbet{lebendiger} Wesen, die aber nichts als \RWbet{Schmerzen} empfinden, und gleichwohl Niemanden \RWbet{nützen}, verbietet die Vernunft gewiß. Hieraus erhellet deutlich, daß das Gebot \RWbet{zu schaffen} nur ein bedingtes sey, daß es von einem \RWbet{andern} abhänge, und eben deßhalb auch in denjenigen Fällen, in denen es die Vernunft (von Gott nämlich) fordert, nur wegen irgend eines \RWbet{anderen} Grundes gefordert werde. Das Gebot des Schaffens macht also weder das ganze oberste Sittengesetz, noch einen Bestandtheil desselben aus.
\item \RWbet{Auch Veränderungen in den leblosen Theilen der Welt werden durch kein ursprüngliches Gesetz geboten.} Alle Gestalten und Umbildungen, die man der leblosen Materie geben mag, sind an und für sich genommen, \dh\ abgesehen von jenem Einflusse, den sie vielleicht auf \RWbet{lebende} Geschöpfe haben, der praktischen Vernunft ganz gleichgültig. Wenn sie von uns verlangt, daß wir bald diese bald jene Veränderung mit den leblosen Theilen der Welt vornehmen sollen; so verlangt sie dieß immer nur, weil etwas Anderes, eine~\RWSeitenw{232}\ gewisse Veränderung im Zustande der \RWbet{lebendigen} Wesen hiedurch erreicht werden soll; und so sind alle diese Forderungen immer nur \RWbet{abgeleitet}. Wenn es ja scheinen sollte, daß einige Gebote dieser Art unbedingt sind, so wären es etwa nur folgende: \RWbet{schöne Gestaltungen häßlichen vorzuziehen}, und, \RWbet{keinen organischen Körper ohne Noth zu zerstören}. Aber auch diese Gebote haben einen Grund. Wir sollen schöne Gestaltungen häßlichen vorziehen, weil der Anblick des Schönen uns selbst und unseren Mitmenschen ein Vergnügen gewährt, auch durch die Regelmäßigkeit, an die er uns gewöhnt, einen wohlthätigen Einfluß auf unsere Tugend hat. Wir sollen organische Körper nicht ohne Noth zerstören, weil sie entweder lebendig und also der Glückseligkeit empfänglich sind, oder doch von dem lebendigen Theile der Welt besser als unorganische Körper benützt werden können; zum Theile auch darum, weil sie insgemein einen schönern Anblick als unorganische gewähren. Wer aber auch nicht zugeben wollte, daß die hier angegebenen Gründe beider Gebote die wahren sind, der müßte doch bloß aus dem Umstande, daß diese Gebote nicht \RWbet{ausnahmslos} sind, daß sie oft anderen weichen müssen, erkennen, daß sie nicht unbedingt sind. Denn Ausnahmen kann nur eine Regel haben, welche aus einer höheren, und zwar nicht mit völliger Strenge folgt.
\item \RWbet{Auch Einwirkungen auf das Erkenntnißvermögen werden nie unbedingt geboten}. Wenn es von irgend einer Einwirkung auf das Erkenntnißvermögen (das eigene oder das anderer Wesen) scheinen könnte, sie werde unbedingt geboten; so wäre es diese, \RWbet{Erkenntniß der Wahrheit bei sich und Andern zu befördern}. Daß aber auch diese Vorschrift nicht eine unbedingte sey, erhellet schon daraus, weil die Vernunft verlangt, einen Unterschied zwischen den Wahrheiten zu machen, und die nützlicheren den minder nützlichen vorzuziehen; weil es sogar Wahrheiten gibt, deren Erforschung und Verbreitung sie uns verbietet, \zB\ gewisse Heimlichkeiten eines Andern, deren Entdeckung und Verbreitung seiner Ehre nachtheilig, oder uns und Andern nur~\RWSeitenw{233}\ zum Aergerniß gereichen könnte \usw\ Hieraus ist hinlänglich zu ersehen, daß das Gebot, die Erkenntniß der Wahrheit bei sich und Andern zu befördern, unter irgend einem höhern Gebote stehe, aus dem es abgeleitet ist.
\item \RWbet{Auch Einwirkungen auf das Begehrungsvermögen werden nie unbedingt geboten}. Der gesunde Menschenverstand erkennt auf das Deutlichste, daß, wenn wir zuweilen verpflichtet sind, gewisse Begierden und Wünsche in uns oder Andern anzuregen, wir hiezu immer durch irgend einen höheren Grund verpflichtet werden, \zB\ um das Vergnügen zu erhöhen, das die Erlangung eines Gutes gewähren wird, wenn man so eben eine gewisse Begierde darnach empfunden hatte; oder um durch die Einwirkung auf das Begehrungsvermögen mittelbar auch auf den Willen und auf die Handlungen eines Menschen einzuwirken.
\item \RWbet{Auch Einwirkungen auf das Willensvermögen gehören nicht in den Inhalt des obersten Sittengesetzes}. Jede Einwirkung auf das Willensvermögen eines Wesens kann nur in Einem von Beidem bestehen: man sucht das Wesen entweder dahin zu bringen, daß es das wolle, was gewollt werden soll; oder man sucht es umgekehrt davon abzuhalten. Das Letztere verbietet die Vernunft ohne Zweifel, das Erste gebietet sie aber; sie will, \RWbet{daß wir ein jedes Wesen, so viel es uns möglich ist, dahin bringen sollen, daß es das wolle, was gewollt werden soll}, oder, was eben so viel heißt, daß es tugendhaft werde. Allein nur muß man nicht glauben, daß diese Forderung als ein \RWbet{wesentlicher Bestandtheil} zum Inhalte des obersten Sittengesetzes gehöre. Denn der Satz: \RWbet{Man soll sich bemühen,  Andere dahin zu bringen, daß sie das wollen, was gewollt werden soll}, ist ein \RWbet{identischer Satz}, \dh\ ein Satz, dessen Prädicat mit dem Subjecte einerlei ist. Die Redensart: \RWbet{Man soll sich bemühen}, heißt doch wohl eben so viel, als: \RWbet{man soll wollen}, oder: \RWbet{es soll gewollt werden}. Und also sagt der Satz: \RWbet{Man soll sich bemühen,}~\RWSeitenw{234}\ \RWbet{Andere dahin zu bringen, daß sie das wollen, was gewollt werden soll}, im Grunde nichts Anderes aus, als der: \RWbet{Es soll gewollt werden, daß das gewollt werde, was gewollt werden soll}. Wer sieht nicht, daß dieß \RWbet{identisch} sey? Aus einem \RWbet{identischen} Satze aber läßt sich nie eine Wahrheit objectiv herleiten, und folglich kann auch das oberste Sittengesetz kein identischer Satz seyn, und keinen solchen Satz als Bestandtheil enthalten.
\item \RWbet{Beförderung der Glückseligkeit ist ein ursprüngliches Vernunftgebot}. Wenn keine der bisher betrachteten Wirkungsarten von der Vernunft unbedingt befohlen wird, so bleibt nichts übrig, als daß es unter den Einwirkungen auf das Empfindungsvermögen der Wesen eine unbedingt gebotene gebe; denn wenn auch hier keine solche vorhanden wäre, so würde es überhaupt gar keine unbedingte, mithin auch keine bedingte Pflicht geben. Nun gibt es aber offenbar nur zwei Zustände, in die das Empfindungsvermögen eines Wesens versetzt werden kann, einen \RWbet{angenehmen} und einen \RWbet{unangenehmen}. Einen von beiden zu befördern, muß also ursprüngliche Pflicht seyn. Niemand wird sagen wollen, daß die Vernunft Hervorbringung unangenehmer Gefühle, oder Beförderung des Schmerzes, sondern Jeder ist gewiß, daß sie Hervorbringung angenehmer Empfindungen oder Glückseligkeit verlange. Freilich erhebt sich hier gleich die Bedenklichkeit, daß es doch auch Fälle gebe, in denen die Regel, angenehme Empfindungen zu befördern, eine wenigstens scheinbare Ausnahme erleidet. So ist es \zB\ kein Zweifel, daß wir einem freien Wesen, das \RWbet{Böses} gethan hat, nicht lauter angenehme, sondern zuweilen auch einige unangenehme Empfindungen als Strafe zufügen dürfen. Allein schon daraus, daß es sonst keine andere ursprüngliche Pflicht gibt, folgt, daß diese Ausnahme nur scheinbar seyn müsse; und wir können also voraussetzen, daß es sich bei einem längeren Nachdenken überall zeigen werde, wie man nur eben, um die Regel zu befolgen, gerade so vorgehen müsse. So ist es bei dem angeführten Beispiele offenbar, daß man das Böse strafe, nur~\RWSeitenw{235}\ um durch Verminderung desselben die Glückseligkeit des Ganzen zu vermehren.
\end{aufzb}
\item \RWbet{Beförderung der Glückseligkeit} ist also das einzige wahrhaft ursprüngliche Vernunftgebot, folglich der einzige Inhalt des obersten Sittengesetzes. Frägt man nun, \RWbet{welcher Wesen Glückseligkeit es sey, welche befördert werden soll:} so ist dem gesunden Menschenverstande vollkommen einleuchtend, daß wir die Glückseligkeit eines \RWbet{jeden} der Empfindung fähigen Wesens \RWbet{wenigstens dann gewiß} zu befördern verpflichtet sind, wenn dieses ohne den mindesten Eintrag der Glückseligkeit \RWbet{anderer} Wesen (auch selbst des \RWbet{Handelnden}) geschehen kann. Wird aber gefragt, \RWbet{in welchem Grade wir die Glückseligkeit eines solchen Wesens befördern sollen}; so ist die Antwort einleuchtend: \RWbet{in dem höchsten Grade, der uns nur immer möglich ist}. Wird ferner gefragt, was in dem Falle zu geschehen habe, wenn die Wahl zwischen zwei oder mehren Handlungen ist, durch deren jede man einigen Wesen Glückseligkeit verschafft, aber in ungleichem Grade: so wird Jeder (wenigstens dann, wenn es bloß thierische Wesen sind, bei denen kein verschiedener Tugendgrad Statt findet) die Antwort geben: man müsse diejenige Handlung wählen, durch welche die Summe der hervorgebrachten Glückseligkeit, gleichviel in welchen Individuen, ein Größtes wird. Wird endlich gefragt, ob man wohl seine \RWbet{eigene} Glückseligkeit der Beförderung der Glückseligkeit \RWbet{Anderer} vorziehen dürfe, wenn gleichwohl der Gewinn für Andere wichtiger ist, als für uns: so wird die \RWbet{Eigenliebe} vielleicht diese Frage zu \RWbet{bejahen} wünschen; die \RWbet{Vernunft} aber wird sie verneinen. Denn wollten wir uns die Erlaubniß ertheilen, den eigenen Vortheil dem größeren Vortheile Anderer vorzuziehen: so würde die neue Frage entstehen, ob wir in allen Fällen berechtiget wären, den Vortheil Anderer dem eigenen aufzuopfern, jener mag diesen auch noch so sehr übertreffen, oder (wenn dieß nicht überall geschehen soll) bei welchem Grade des Uebermaßes dieses zuerst geschehen dürfe? Daß wir den eigenen Vortheil dem Vortheile Anderer jedesmal vorziehen dürfen, wäre eine Entscheidung, vor welcher jeder bessere Mensch zurückbebt. Für den zweiten Fall~\RWSeitenw{236}\ aber ist kein Maßstab bekannt, nach dem wir den Grad jenes Uebermaßes bestimmen könnten. Wir können also nicht umhin, auch von uns selbst zu verlangen, daß wir den eigenen Vortheil dem Vortheile Anderer überall aufopfern, wo er nur immer der kleinere ist. Und so wird sich denn das oberste Sittengesetz etwa auf folgende Art ausdrücken lassen: \RWbet{Wähle von allen dir möglichen Handlungen immer diejenige, die, alle Folgen erwogen, das Wohl des Ganzen, gleichviel in welchen Theilen, am meisten befördert}.
\item Ohne den Inhalt dieses Satzes wesentlich zu vermehren, können wir noch den Begriff der \RWbet{Tugend} einschalten, und sagen: \RWbet{Wähle unter allen dir möglichen Handlungen immer diejenige, die, alle Folgen erwogen, die Tugend und Glückseligkeit des Ganzen am meisten befördert}. Daß durch diesen Zusatz nichts Wesentliches geändert werde, erhellet aus dem, was wir bereits \no\,3, e gesehen, daß nämlich der Satz: \RWbet{Befördere Tugend}, für sich ein identischer Satz sey. Gewinnt aber unser Princip durch diese Behauptung auch nichts in Hinsicht auf seine wissenschaftliche Vollkommenheit, so gewinnt es doch um so mehr in Hinsicht auf seine Brauchbarkeit für das gesellige Leben. Denn durch die ausdrückliche Erinnerung, man müsse stets so handeln, daß nicht nur die \RWbet{Glückseligkeit}, sondern auch die \RWbet{Tugend} befördert werde, wird dem gefährlichen Mißverstande vorgebeugt, als könne irgend eine Handlung, welche der Tugend der Menschen Abbruch thut, die wahre Glückseligkeit befördern, und somit als erlaubt angesehen werden. Daher werde auch ich das oberste Sittengesetz in der Folge jederzeit mit diesem Beisatze gebrauchen.
\end{aufza}
\begin{RWanm}
So einleuchtend mir die Wahrheit des hier aufgestellten obersten Sittengesetzes däucht, so muß ich doch gestehen, daß es bisher nur von einer sehr kleinen Anzahl von Gelehrten ausdrücklich angenommen, von Einigen sogar bestritten worden sey. Sonach darf ich die Wahrheit, daß der letzte Grund aller Pflichten und Obliegenheiten nicht nur des Menschen, sondern auch aller übrigen Wesen in nichts Anderem liege, als in Beförderung der Tugend und Glückseligkeit des Ganzen, keineswegs für einen Lehr\RWSeitenw{237}satz der natürlichen Religion des menschlichen Geschlechtes ausgeben. Nicht ohne gehässige Nebenabsicht hat man die Vertheidiger dieses Princips mit dem Namen der \RWbet{Kosmopoliten} oder \RWbet{Weltbürger} bezeichnet, während es schicklicher gewesen wäre, dieses Princip \RWbet{den Satz vom allgemeinen Wohle} zu nennen. Einige seiner vornehmsten Anhänger waren: \RWbet{Joh.~Jak.~Rousseau}, der es jedoch eigentlich so ausdrückte: \RWbet{Befördere dein Wohl mit so wenigem Schaden deines Nebenmenschen, als es nur möglich ist.} (\RWlat{Sur l'origine et les fondements de l'inégalité parmi les hommes})\RWlit{}{Rousseau3}. \RWbet{Struve}, \RWbet{Bernh.~Basedow} (der es in seiner \RWbet{praktischen Philosophie für alle Stände}\RWlit{}{Basedow2} am Consequentesten durchgeführt hat). \RWbet{Leß, Meiners, Trapp, Herder, Wieland, Okel, Garven, E.~Plattner} (der es im 2ten Buche seiner \RWbet{philosophischen Aphorismen}\RWlit{}{Platner1} mit dem meisten Aufwande von Scharfsinn deducirte), \umA
\end{RWanm}

\RWpar{89}{Einwürfe gegen dieß oberste Sittengesetz}
Ich halte es für meine Pflicht, die wichtigsten \RWbet{Einwürfe}, die man gegen dieß oberste Sittengesetz vorgebracht hat, aufrichtig anzuzeigen, aber auch diese Anzeige mit kurzen Gegenbemerkungen zu begleiten.\par
\RWbet{1.~Einwurf.} Es gibt mehre Pflichten, deren Richtigkeit uns der \RWbet{gesunde Menschenverstand} und unser \RWbet{innerstes Gefühl} (das \RWbet{Gewissen}) verbürgt, die gleichwohl aus dem Satze: \RWbet{Befördere das Wohl des Ganzen}, sich nicht ableiten lassen. Z.\,B.\ die Pflicht, wenn ich nur Einem von zwei Menschen, die einen für beide gleich wichtigen Dienst von mir verlangen, willfahren kann, demjenigen den Vorzug einzuräumen, von dem ich selbst einmal eine Wohlthat empfangen hatte.\par
\RWbet{Antwort.} Alles, was der gesunde Menschenverstand, was unser innerstes Gefühl (das Gewissen) für Pflicht erklärt, halte auch ich für Pflicht (\RWparnr{14}\ \no\,7.), glaube aber auch, daß es sich durch ein genaueres Nachdenken aus unserem Grundsatze ableiten lasse. Nur muß man nicht bloß auf die nächsten, sondern auch auf die weitern Folgen einer jeden Handlung sehen. So läßt sich \zB\ die im Einwurfe angegebene Pflicht auch aus unserem Grundsatze sehr wohl ableiten; denn nur~\RWSeitenw{238}\ die nächsten Folgen der beiden Handlungen, zwischen denen hier die Wahl ist, sind einander gleich, und würden es also nach unserem Grundsatze unentschieden lassen, welche von beiden gewählt werden soll. Sehen wir aber auf die entfernteren Folgen; so erkennen wir bald, daß das Wohl des Ganzen viel mehr gewinne, wenn wir demjenigen dienen, der uns früher selbst gedient hat. Durch ein solches Betragen befördern wir nämlich die Tugend der Wohlthätigkeit, indem wir zeigen, daß, wer Andern wohlgethan hat, bei Gelegenheit wieder eine Entgeltung dafür finde. \Usw\par
\RWbet{2.~Einwurf.} Nach diesem Grundsatze könnte es keine \RWbet{allgemeine Sittenregeln} geben, keine Pflichten, die ohne Ausnahme verbinden; oder wie wollte man von irgend einer Verhaltungsweise darthun, daß sie dem Wohle des Ganzen in \RWbet{jedem} Falle ohne alle Ausnahme förderlich sey? Gleichwohl erkennt die Vernunft verschiedene Sittengesetze, die ohne Ausnahme gelten, \zB\ die Pflicht der Wahrhaftigkeit bei einem Eidschwure; die Regeln der Keuschheit; \udgl\par
\RWbet{Antwort.} Ich entgegne, daß wohl manche Regel, welche die Sittenlehrer als eine \RWbet{allgemein geltende} aufstellten, in gewissen Fällen eine billige Ausnahme erleide. Der gemeine Menschenverstand pflichtet mir bei, und trägt keinen Anstand, in Fällen, wo es das Wohl des Ganzen offenbar fordert, eine solche Ausnahme für erlaubt zu erklären. Und eben daher kommt wohl das Sprichtwort, \RWbet{daß keine Regel ohne Ausnahme sey}. Doch gibt es, wie ich glaube, auch \RWbet{einige ganz ausnahmslose Sittenregeln}, und die im Einwurfe erwähnten sind ohne Zweifel von dieser Art. Aber diese \RWbet{Ausnahmslosigkeit} derselben läßt sich auch aus unserem Grundsatze darthun. Wenn nämlich eine Regel von solcher Beschaffenheit ist, daß ihre Befolgung in den meisten Fällen überaus nothwendig, die Erlaubniß einer Abweichung in einigen einzelnen Fällen aber solcher Abweichungen alsbald \RWbet{zu viele} nach sich ziehen, und hiedurch fast den ganzen Nutzen der Regel wieder aufheben würde: so muß man sie eben deßhalb als eine \RWbet{ausnahmslose} Regel betrachten und befolgen. Von dieser Art ist \zB\ die Regel der Wahrhaftigkeit, welche so überaus noth\RWSeitenw{239}wendig ist, wenn sich die Menschen auf die Aussagen wenigstens eines Jeden, den sie als einen \RWbet{sittlich guten} Menschen erkannt haben, getrost verlassen sollen. Dieser Nutzen der Regel würde aber fast ganz verschwinden, sobald man Ausnahmen von ihr in gewissen Fällen erlauben wollte; denn nun würde Niemand wissen, ob sich der Andere, der ihm etwas bezeugt, nicht eben jetzt in einem solchen Falle, wo Lüge erlaubt seyn soll, befinde, oder doch zu befinden glaube. Auch würden sich Viele überreden, daß sie in einem solchen Falle sind, ohne es wirklich zu seyn. Es ist daher nothwendig für die Menschheit, ein Mittel zu besitzen, bei dessen Anwendung sie sich in äußerst wichtigen Fällen mit der größten Sicherheit überzeugen kann, daß eine gewisse Aussage die lautere Wahrheit ist. Dieß Mittel ist der \RWbet{Eidschwur}, doch nur so lange, als man die Pflicht der Wahrhaftigkeit bei ihm als eine völlig ausnahmlose Regel ansieht. Kein Zweifel also, daß man sie wirklich dafür ansehen müsse! -- Eben so offenbar ist es, daß die \RWbet{Regel der Keuschheit} ohne Ausnahme verbindet; denn jener kleinliche Vortheil, den eine Uebertretung dieser Regel in einzelnen Fällen etwa hervorbringen kann, ein flüchtiger Sinnengenuß, kann er je aufwiegen, ja auch nur in Vergleich gestellt werden mit den unsäglich großen und vielfachen Nachtheilen, welche die Erlaubniß einer Abweichung in einzelnen Fällen durch eine Ausdehnung auf viel mehre nach sich ziehen würde? \usw\par
\RWbet{3.~Einwurf.} Der gesunde Menschenverstand fragt, wenn er untersuchen will, ob etwas \RWbet{Pflicht} sey, nicht, ob es \RWbet{nütze}? sondern er hält diese beiden Fragen vielmehr für durchaus verschieden. Schon die alten \RWbet{Stoiker} machten einen Gegensatz zwischen dem \RWlat{\RWbet{honesto}} und dem \RWlat{\RWbet{utili}}. Und die \RWbet{Juristen} sagen: \RWlat{Fiat justitia, et pereat mundus!}\par
\RWbet{Antwort.} Es bestehet allerdings ein Unterschied zwischen dem \RWbet{Nützlichen} und dem \RWbet{sittlich Guten}, wenn man unter dem Ersteren dasjenige versteht, was nur dem \RWbet{Handelnden} allein, oder wohl auch der ganzen Menschheit, aber nur in den \RWbet{nächsten}, nicht aber auch in seinen \RWbet{entferntesten Folgen} nützt. Nur so versteht der gemeine Menschenverstand das \RWbet{Nützliche}, wenn er es dem \RWbet{sittlich}~\RWSeitenw{240}\ \RWbet{Guten} entgegensetzt. Nur so verstanden es auch die Stoiker, wenn sie sich anders selbst verstanden. Der Satz der Juristen aber: \RWlat{Fiat justitia, et pereat mundus!} ist eine kleine Uebertreibung, wenn man unter dem \RWlat{\RWbet{mundus}} das ganze Weltall versteht. Die \RWbet{wahre} Gerechtigkeit verlangt nie, und \RWbet{kann} nie verlangen, was die ganze Welt zu Grunde richten würde.\par
\RWbet{4.~Einwurf.} Nach diesem Grundsatze würde der sittliche Werth unserer Handlungen vom \RWbet{bloßen Zufalle} abhangen. Wenn Jemand in der Absicht, seinen Nächsten zu tödten, den Dolch gegen ihn zückte, zufälliger Weise aber nur ein Geschwür öffnete, von welchem dieser jetzt geneset; so hätte er ein gutes Werk verrichtet.\par
\RWbet{Antwort.} Keineswegs; denn die \RWbet{sittliche Güte} einer Handlung (\dh\ der Anspruch, den sie auf Belohnung hat) richtet sich immer nur darnach, ob die Handlung in der \RWbet{Meinung}, daß sie mit dem Gesetze übereinstimmt, und eben nur \RWbet{aus diesem Grunde} unternommen worden sey; ingleichen nach den bald größern bald geringern Vortheilen, welche der Handelnde seiner Pflicht wegen aufopferte. Dieses erkennet jeder gesunde Menschenverstand, und aus unserem Grundsatze läßt es sich sehr natürlich erklären. Denn eben um das allgemeine Wohl zu befördern, muß man die \RWbet{Tugend}, \dh\ die Gesinnung, alles dasjenige zu thun, was das allgemeine Wohl verlangt, so viel es nur möglich ist, befördern; muß also jede Handlung, die in dieser Absicht unternommen wird, weil es das allgemeine Wohl fordert, zu belohnen, und um so mehr zu belohnen versprechen, je mehr Ueberwindung sie fordert, \usw\par
\RWbet{5.~Einwurf.} Wir Menschen können sehr wenig oder gar nicht beurtheilen, was jedesmal dem Wohle des Ganzen zuträglich sey oder nicht. Wir übersehen ja nur den geringsten Theil der Welt, kennen nur die nächsten, nicht aber die entfernteren Folgen von irgend einer Handlung. Mithin ist der Grundsatz von der Beförderung des allgemeinen Wohles für uns Menschen so gut als unbrauchbar.\par
\RWbet{Antwort.} Es ist freilich wahr, daß wir von keiner einzigen Handlung, die wir verrichten, alle auch die enfern\RWSeitenw{241}testen Folgen derselben vorherzusehen vermögen; allein dieß ist auch nicht nothwendig, um nach dem Grundsatze von der Beförderung des allgemeinen Wohles vorgehen zu können. Dieser verlangt, daß wir uns nur jederzeit zu der Handlung entschließen, die nach denjenigen Folgen derselben, \RWbet{die wir von ihr vorhersehen können}, dem Wohle des Ganzen am zuträglichsten scheinet. Uebrigens haben wir auch alle Ursache zu glauben, daß diejenigen Folgen unserer Handlungen, die wir mit mehr oder weniger Wahrscheinlichkeit vorhersehen, meistens die \RWbet{wichtigsten} sind, und daß diejenigen, die wir auf keine Art bemerken können, meistentheils auch nur unwichtig sind. Wenn dieß ist, so können wir immerhin mit einer hinlänglichen Sicherheit behaupten, daß gewisse Handlungsweisen dem Wohle des Ganzen zuträglich, andere demselben nachtheilig sind. So wird \zB\ kein Vernünftiger in Abrede stellen, daß Lügen, Stehlen, Mord \udgl\  Handlungsweisen sind, welche das Wohl des Ganzen stören; daß mithin die ihnen entgegengesetzten Handlungsweisen dem gemeinen Besten beförderlich sind; \usw\par
\RWbet{6.~Einwurf.} Wenn dieser Grundsatz auch wahr ist, so ist er wenigstens \RWbet{gefährlich}; denn man kann ihn zur Rechtfertigung der bösesten Handlungen, sind sie nur von der Art, daß ihre schlimmen Folgen nicht auf den ersten Blick einleuchtend sind, mißbrauchen.\par
\RWbet{Antwort.} Ich verneine es nicht, daß eine unbehutsame Verbreitung dieses Grundsatzes gefährlich werden könne. Insonderheit muß man Jeden, den man mit diesem Grundsatze bekannt macht, warnen, daß er nie eine Pflicht, die ihm ein inneres \RWbet{Gefühl} oder (wie man sagt) sein \RWbet{Gewissen} ankündiget, bloß darum verwerfe, weil er nicht deutlich einsieht, auf welchem Grunde sie beruhe, \dh\ auf welche Art ihre Befolgung das Wohl des Ganzen befördern möge. Er denke nur länger nach, so wird er dann meistens diese Gründe finden. Aber er finde, oder finde sie nicht; so glaube er nie, daß jenes innere Gefühl sich irre. Denn es ist eine allgemein bekannte Sache, daß wir gar viele Wahrheiten mit voller Bestimmtheit erkennen (fühlen), ohne im Stande zu seyn, uns auch die Gründe, auf denen sie beruhen, zu einem deutlichen Bewußtseyn zu bringen.~\RWSeitenw{242}

\RWpar{90}{Kurze Beurtheilung der gewöhnlichsten andern Ansichten über das oberste Sittengesetz}
Um desto völliger zu überzeugen, daß der hier aufgestellte Grundsatz der wahre sey, will ich noch die gewöhnlichsten \RWbet{anderen} Ansichten über das oberste Sittengesetz anführen, und einer jeden nur ein Paar Worte zur Beurtheilung an die Seite stellen.
\begin{aufza}
\item Zuerst muß ich erinnern, daß es \RWbet{Weltweise} gegeben, die unsere sämmtlichen Begriffe über \RWbet{Gut} und \RWbet{Böse} für ein bloßes Vorurtheil erklärten. Hieher gehört \zB\ \RWbet{Michel Montaigne}, wenn er behauptete, die \RWbet{Erziehung} wäre der einzige Grund unserer Begriffe über Gut und Böse.
\end{aufza}\par
Hätte er nur gesagt, daß die Erziehung einen beträchtlichen Einfluß auf die bestimmte Art und Weise habe, wie die Begriffe der Menschen über Gut und Böse sich ausbilden und gestalten: so könnten wir dieß allerdings zugeben, obgleich erinnert werden müßte, daß dieser Einfluß der Erziehung bei keiner Gattung unserer Begriffe weniger Vorurtheile erzeugt hat, als bei den \RWbet{sittlichen}, wie dieses schon aus der \RWbet{Gleichförmigkeit} derselben auf dem ganzen Erdenrunde, trotz allen Abweichungen in der Erziehungsart, erhellet. Will man aber behaupten, daß nicht nur \RWbet{manche} unserer Meinungen über das, was in gewissen Fällen gut oder böse sey, sondern auch der \RWbet{Begriff} von Gut und Böse \RWbet{selbst}, \dh\ der Glaube, daß es überhaupt ein gewisses \RWbet{Sollen} gebe, ein bloßes uns durch Erziehung eingepflanztes \RWbet{Vorurtheil} sey: so wird durch diese Behauptung die ganze Sittenlehre mit einem Mahle umgestürzt. Allein das Irrige derselben wurde bereits \RWparnr{87}\ \no\,2,\,a gezeigt.
\begin{aufza}\setcounter{enumi}{1}
\item Nach \RWbet{Mandeville's} Meinung soll alle Tugend nichts als eine \RWbet{Wirkung des Ehrtriebes} seyn. Aus Liebe zur öffentlichen Achtung streben wir, das allgemeine Wohl zu befördern.
\end{aufza}\par
Wäre es auch wahr, daß die Menschen meistentheils, ja daß sie immer aus bloßem Ehrtriebe thun, was gut ist: so würde daraus doch nicht im Geringsten folgen, daß der Begriff des Guten mit jenem des \RWbet{Ehrenvollen} einerlei sey;~\RWSeitenw{243}\ folglich wäre auch noch gar nicht bewiesen, daß das oberste Sittengesetz nicht etwa so lauten könne, wie ich es aufgestellt habe.
\begin{aufza}\setcounter{enumi}{2}
\item \RWbet{Thomas von Aquino} glaubte den Inbegriff all unserer Pflichten in die Formel: \RWbet{Thue, was gut ist}, einschließen zu können. Dieselbe Formel stellte auch erst vor einigen Jahren noch ein Gelehrter auf.
\end{aufza}\par
Der Satz ist sehr wahr, aber bloß identisch: denn unter dem Guten (dem sittlich \RWbet{Guten}, und nur dieß allein kann hier gemeint seyn) versteht man ja eben nichts Anderes, als was gewollt werden soll. \RWbet{Thue, was gut ist}, heißt also: \RWbet{Es soll gewollt werden, was gewollt werden soll.}
\begin{aufza}\setcounter{enumi}{3}
\item \RWbet{Folge der Vernunft}, sagte \RWbet{Richard Price}, handle so, wie die Vernunft es unmittelbar für wahr und Recht erkennt.
\end{aufza}\par
Gleichfalls identisch. Unter der \RWbet{Vernunft} versteht man hier offenbar nichts Anderes, als die Fähigkeit, Wahrheiten zu erkennen, und unter derjenigen Vernunft, der man immer \RWbet{folgen} soll, die Fähigkeit, \RWbet{praktische} Wahrheiten zu erkennen. \RWbet{Folge der Vernunft} heißt also eben so viel, als: \RWbet{Thue, was du sollst}. Uebrigens hat der Satz noch das Gefährliche, daß er aus Mißverstand leicht so ausgelegt werden kann, als ob man behauptete, daß nur solche Pflichten, die (wie man sagt) durch die \RWbet{bloße Vernunft}, \dh\ ohne Dazwischenkunft eines fremden Zeugnisses \zB\ der Gottheit selbst, erkennbar sind, verbindlich für den Menschen wären.
\begin{aufza}\setcounter{enumi}{4}
\item Mehre \RWbet{Engländische Weltweisen}, \zB\ \RWbet{David Hume, Jakob Oswald, Heinrich Home, Hutcheson} \uA\ behaupteten, daß sich die Pflichten des Menschen nicht aus \RWbet{Begriffen} herleiten ließen, sondern daß es einen eigenen \RWbet{Sinn} für sie, sonst auch das \RWbet{sittliche Gefühl} genannt, gebe. Sie stellten daher als obersten Grundsatz in der Sittenlehre den Satz auf: \RWbet{Folge dem sittlichen Gefühle in Dir}. Unter uns Deutschen war es besonders \RWbet{Fr.~H.~Jakobi}, der alle unsere theoretischen sowohl als praktischen Erkenntnisse übersinnlicher Dinge auf das \RWbet{Gefühl} gegründet wissen wollte.
\end{aufza}\par
Auch dieser Satz ist, so wie die vorigen, \RWbet{identisch}. Denn auch hier versteht man unter dem \RWbet{sittlichen Gefühle}, ganz wie vorhin unter der \RWbet{Vernunft}, eine Fähigkeit, prak\RWSeitenw{244}tische Wahrheiten zu erkennen, nur mit dem einzigen hier gleichgültigen Unterschiede, daß man sich vorstellt, diese Wahrheiten würden eine jede \RWbet{unmittelbar} erkannt, nicht aber einige erst durch Schlüsse aus andern abgeleitet. Diese letztere Meinung ist nun, meinem Dafürhalten nach, unrichtig; da ich vielmehr glaube, daß nur das einzige oberste Sittengesetz, als ein Grundsatz unmittelbar erkannt werden müsse; von allen übrigen praktischen Wahrheiten aber mir vorstelle, daß es durch längeres Nachdenken nicht unmöglich sey, sie aus dem obersten Sittengesetze vermittelst Zuziehung eines bloß theoretischen Untersatzes objectiv abzuleiten. Gleichwohl verkenne ich nicht das Gute, das dieses Princip hat, uns durch die Vorstellung, daß wir eine jede unserer Pflichten unmittelbar erkennen, vor jener gefährlichen \RWbet{moralischen Zweifelsucht} zu bewahren, die jede Pflicht, deren Grund sie sich nicht deutlich auseinander zu setzen weiß, gleich als ein Vorurtheil verwerfen will.
\begin{aufza}\setcounter{enumi}{5}
\item \RWbet{Gut ist}, sagen Einige, \RWbet{was die Uebereinstimmung aller Menschen dafür erklärt}.
\end{aufza}\par
Wahr ist es allerdings, daß eine Handlungsweise, die durch das übereinstimmende Urtheil aller Menschen für gut oder böse erklärt wird, auch in der That gut oder böse seyn müsse. Wahr ist es auch, daß die Bemerkung einer so allgemeinen Uebereinstimmung in einem Urtheile unsere Zuversicht zu seiner Richtigkeit ungemein verstärken müsse, und daß es eben deßhalb sehr anzurathen sey, an diese Uebereinstimmung zu denken, so oft uns die Leideschaft versucht, an einer allgemein erkannten Pflicht zu zweifeln; nur dürfte sich dieß Mittel nicht überall anwenden lassen, weil es doch auch Fälle gibt, worüber die Urtheile der Menschen nicht so einstimmig sind. Allein wenn dieß auch nicht wäre, so könnte man den Satz: \RWbet{Gut ist, was die Uebereinstimmung aller Menschen dafür erklärt}, doch niemals für das oberste Sittengesetz ausgeben, weil sich aus ihm keine einzige Pflicht \RWbet{objectiv}, \dh\ so wie die Folge aus ihrem Grunde herleiten läßt; denn sicher ist doch etwas nicht \RWbet{darum} gut, weil alle Menschen es für gut \RWbet{erkennen}, sondern umgekehrt, weil es \RWbet{gut ist, erkennen} es alle Menschen für gut.~\RWSeitenw{245}
\begin{aufza}\setcounter{enumi}{6}
\item \RWbet{Gut ist}, sagen Andere, \RWbet{was die Gesetze des Landes dafür erklären.}
\end{aufza}\par
Auch gegen diesen Satz läßt sich dieselbe Einwendung machen, die ich so eben bei dem vorigen machte, nur daß es überdieß nicht einmal so sicher als vorhin geschlossen ist, daß Alles dasjenige, was die Gesetze eines Landes für gut oder böse erklären, es auch wirklich sey. Wollte man aber vollends den Ausdruck: \RWbet{was die Gesetze des Landes für gut erklären}, so verstehen: \RWbet{was sie nicht zu bestrafen drohen}, \dh\ für \RWbet{recht} (in der juridischen Bedeutung dieses Wortes) erklären: so wäre es ein ganz falscher Satz, weil es unzählig viele \RWbet{sittlich böse} Handlungen gibt, die gleichwohl selbst in den besten Staaten \RWbet{geduldet}, \dh\ für recht erklärt werden und erklärt werden müssen. 
\begin{aufza}\setcounter{enumi}{7}
\item \RWbet{Handle deinen sämmtlichen Verhältnissen gemäß}, lehren \RWbet{Frint} \uA
\end{aufza}\par
Es kommt darauf an, was man unter einer Handlung, welche den sämmtlichen Verhältnissen des Handelnden \RWbet{gemäß} ist, verstehen will. Verstehet man hierunter nichts Anderes als eine Handlung, die unter den obwaltenden Verhältnissen gut ist, oder was eben so viel heißt, von der Vernunft gebilligt und für Pflicht erklärt wird; so ist der Satz: Handle deinen sämmtlichen Verhältnissen gemäß, sehr richtig, aber offenbar \RWbet{identisch}, und kann also nicht zum obersten Sittengesetze in der von mir angenommenen Bedeutung taugen. Wohl hat er übrigens den Nutzen, uns an die Wahrheit zu erinnern, daß wir nicht im Stande sind zu beurtheilen, ob Jemand gut oder böse handelt, wenn wir nicht alle \RWbet{Verhältnisse}, in denen er sich befindet, betrachten. Doch eben so wahr ist auch, daß man diesen Satz leicht mißverstehen und schändlich mißbrauchen könne, wenn man ihn so deutet, daß Alles erlaubt wäre, was die Verhältnisse, wie man sagt, \RWbet{gebieterisch fordern}, \dh\ was zu unterlassen uns in diesen Umständen schwer fällt. Welche Verbrechen ließen sich da nicht entschuldigen! und wie oft hört man nicht wirklich böse Menschen eine solche Entschuldigung brauchen!
\begin{aufza}\setcounter{enumi}{8}
\item \RWbet{Folge der Natur}, war das Princip der \RWbet{Stoiker}.~\RWSeitenw{246}
\end{aufza}\par
Dieses Princip scheint im Wesentlichen nicht sehr von dem so eben betrachteten unterschieden; indem es wohl keinen andern Sinn haben dürfte, als den: Man soll dasjenige thun, was nach Betrachtung aller Einrichtungen der Natur als das Beste erscheint. Uebrigens hat dieser Satz doch das Verdienst, uns an die wichtige Wahrheit zu erinnern, daß man die Einrichtungen, welche die Natur getroffen hat, fleißig beobachten müsse; eine Erinnerung, die um so nöthiger ist, je öfter es, leider! geschieht, daß wir auf die Natur und ihre unabänderlichen Gesetze vergessen.
\begin{aufza}\setcounter{enumi}{9}
\item \RWbet{Aristoteles} stellte als oberstes Sittengesetz den Satz auf: \RWbet{Beobachte überall das Mittelmaß}.
\end{aufza}\par
Es frägt sich, wie man dieß \RWbet{Mittelmaß} verstehe? Versteht man darunter das \RWbet{rechte} Maß, \dh\ dasjenige, was die Vernunft in einem jeden Falle billiget: so ist der Satz identisch. Versteht man aber das \RWbet{mathematische Mittel}, \dh\ eine Thätigkeit, die von zwei gegebenen Aeußersten gleichweit entfernt bleibt: so frägt es sich wieder, welches diese zwei \RWbet{Aeußersten} sind? Sagt man, um doch etwas Bestimmtes anzugeben, daß man die \RWbet{kleinste} und die \RWbet{größte} in jeder Art mögliche Thätigkeit so nenne: so ist es gewiß sehr falsch, daß der Mensch überall nur das Mittelmaß halten solle. Oder soll man wohl auch bei einer Gelegenheit, wo man Jemand eine \RWbet{Wohlthat} erweisen kann, das Mittel zwischen der größten und kleinsten \RWbet{Wohlthat}, die man ihm zu erweisen im Stande wäre, wählen? -- Uebrigens ist dieser Satz recht brauchbar, um uns zu erinnern, daß es fast in aller Art von Dingen nicht bloß auf die Beschaffenheit, den Zweck \usw , sondern auch auf das Maß unsers Bestrebens ankomme, daß wir bald zu wenig, bald wieder zu viel thun können. Wie viel aber eben das \RWbet{rechte Maß} ausmache, das kann nur aus Betrachtung des Verhältnisses, in welchem unsere Thätigkeit zum allgemeinen Wohle steht, beurtheilet werden.
\begin{aufza}\setcounter{enumi}{10}
\item \RWbet{Gut ist}, lehrten verschiedene \RWbet{Theologen, was Gott will}, und das oberste Sittengesetz lautet sonach: \RWbet{Folge dem Willen Gottes}. So erklärten sich \RWbet{Melanchthon}, \RWbet{Crusius}, \uA ~\RWSeitenw{247}
\end{aufza}\par
Versteht man dieß so, als wäre der Wille Gottes der \RWbet{letzte Grund} davon, warum etwas gut oder böse ist; so däucht mir dieß irrig. Denn obgleich die \RWbet{einzelnen} Pflichten des Menschen, \zB\ die Pflicht, daß ich jetzt diesem Armen dieß Almosen geben soll, allerdings auf Verhältnissen beruhen, die nur der Wille Gottes in die Welt eingeführt hat: so gibt es doch wenigstens Eine praktische Wahrheit, nämlich das oberste Sittengesetz, die nicht durch den Willen Gottes bedingt, sondern, wie alle reine \RWbet{Begriffswahrheiten}, von diesem Willen ganz unabhängig da stehet. Wie könnte man auch sonst sagen, daß Gott selbst heilig sey, \dh\ daß auch sein Wille immer mit dem übereinstimme, was er als gut erkennt, wenn es von seinem \RWbet{Willen allein} abhinge, was gut oder nicht gut seyn soll? Versteht man aber unter dem Satze: \RWbet{Gut ist, was Gott will}, nur so viel, \RWbet{daß Alles, was Gott will, \dh\ gebietet} (also in einer uneigentlichen Bedeutung des Wortes will), \RWbet{sittlich gut} sey; so ist der Satz allerdings wahr. Er ist im Grunde auch \RWbet{praktisch}; denn er sagt eben so viel, als: \RWbet{Thue dasjenige, was du als Gottes Willen, \dh\ als Gottes Gebot, erkennest}; aber er ist gleichwohl nicht das oberste Sittengesetz, weil sich nicht \RWbet{alle} praktischen Wahrheiten, nämlich nicht diejenigen, die das Verhalten Gottes selbst bestimmen, aus ihm herleiten lassen, und weil auch diejenigen, die sich aus ihm herleiten lassen, aus ihm nicht objectiv, nicht wie die Folge aus ihrem Grunde, fließen. Denn nicht darum, weil Gott dieß oder jenes will, \dh\ gebietet, soll es von uns gewollt werden; sondern umgekehrt, weil es von uns gewollt werden \RWbet{soll}, will oder gebietet es uns Gott. Inzwischen ist dieser Satz doch immer von einer großen Brauchbarkeit für uns, wie dieß an seinem Orte (in der katholischen Moral) gezeigt werden soll.
\begin{aufza}\setcounter{enumi}{11}
\item Wie einige Theologen den \RWbet{Willen}, so haben Andere die \RWbet{Ehre} Gottes zum obersten Grundsatze der ganzen Sittenlehre erhoben.
\end{aufza}\par
Daß wir nun Gottes Ehre zu befördern verpflichtet sind, unterliegt keinem Zweifel. Ich gebe auch zu, daß viele, ja wenn man will, \RWbet{alle} menschlichen Pflichten auf die Pflicht~\RWSeitenw{248}\ der Beförderung der Ehre Gottes theilweise gegründet, oder doch aus ihr hergeleitet werden können, indem wir durch jede sittlich gute Handlung auch Gottes Ehre befördern. Gleichwohl kann man diese Pflicht nicht als das oberste Sittengesetz betrachten, weil sich die meisten Pflichten des Menschen nur \RWbet{theilweise} und nicht \RWbet{vollständig} auf sie gründen; weil ferner diese Pflicht selbst noch einen weitern \RWbet{Grund} hat, nämlich den Nutzen, den die Beförderung der Ehre Gottes für die Geschöpfe hat; weil überdieß die meisten unserer Handlungen nicht darum Pflichten sind, weil sie die Ehre Gottes befördern, sondern vielmehr die Ehre Gottes befördern, weil sie Pflichten sind; weil sich endlich auch nicht alle praktischen Wahrheiten, nämlich nicht diejenigen, die das Verhalten Gottes selbst bestimmen, aus dieser Regel herleiten lassen.
\begin{aufza}\setcounter{enumi}{12}
\item Noch andere Theologen stellten \RWbet{die Nachahmung Gottes}, oder den Satz: \RWbet{Strebe nach Aehnlichkeit mit Gott}, als oberstes Sittenprincip auf.
\end{aufza}\par
Wenn man diejenige Aehnlichkeit, deren Erreichung der Mensch sich vorsetzen soll, in die Heiligkeit, und in gewisse dem Menschen nachahmbare und von ihm nachzuahmende Wirkungsarten Gottes setzet: so ist dieser Satz allerdings wahr, aber zugleich auch \RWbet{identisch}; denn er sagt dann nichts Anderes aus, als: \RWbet{Du sollst Alles thun, wovon du aus Betrachtung der göttlichen Vollkommenheiten erkennst, daß du es sollst}.
\begin{aufza}\setcounter{enumi}{13}
\item \RWbet{Plato, Wolf, Baumgarten, Daries, Eberhard, F.~V.~Reinhard} u.~viele A.~stellten das \RWbet{Princip der Vollkommenheit}, \dh\ den Satz auf: \RWbet{Strebe nach Vollkommenheit}. Die Meisten aus ihnen verstanden darunter nur die Vollkommenheit des handelnden Subjectes selbst, daher sie das Princip auch wohl so ausdrückten: \RWbet{Vervollkommne dich selbst} (\RWlat{perfice te ipsum}, Wolf). Einige aber, \zB\ \RWbet{Daries}, verstanden die \RWbet{allgemeine Vollkommenheit}, und verlangten also, daß man die \RWbet{eigene} sowohl, als \RWbet{Anderer} Vollkommenheit befördern solle.\end{aufza}\par
Wir müssen noch fragen, was man hier unter der \RWbet{Vollkommenheit} verstanden habe. \RWbet{Plato} erklärte sich hierüber~\RWSeitenw{249}\ nicht sehr deutlich, indem er sich bloß mit einem Gleichnisse behalf, und die Vollkommenheit, nach welcher der Mensch streben soll, durch diejenige, die sich in einem wohleingerichteten Staate befindet, erläuterte. \RWbet{Wolf} gab die genauere Erklärung, Vollkommenheit sey die Uebereinstimmung des Mannigfaltigen an einem Dinge zu einem Zwecke. Hiernächst erhielt die Formel: Strebe nach Vollkommenheit, den Sinn: \RWbet{Alle deine Handlungen seyen auf einen und eben denselben letzten Zweck gerichtet}. Das ist nun allerdings wahr; denn weil die Größe jener Wirkung, deren Hervorbringung das oberste Sittengesetz verlangt (nämlich die Glückseligkeit des Ganzen) keine andere Grenze als die der Möglichkeit kennt, \dh\ weil wir die Glückseligkeit des Ganzen so sehr befördern sollen, als es nur immer möglich ist: so muß sich, um diesem Gesetze Genüge zu thun, der Gebrauch unserer sämmtlichen Kräfte, und alle unsere Handlungen müssen sich (unmittel- oder mittelbar) auf diesen Einen Zweck beziehen; und diese wichtige Wahrheit ist es, an die uns das Princip der Vollkommenheit, so ausgelegt, erinnert. Aber so wahr dieses Princip bei einer solchen Auslegung ist; so kann man es doch nicht als den obersten Grundsatz der ganzen Sittenlehre ansehen, weil man, um irgend eine Pflicht aus diesem Satze herleiten zu können, erst \RWbet{jenen letzten Zweck}, auf den all unser Streben gerichtet seyn soll, bestimmen muß. Allein der Satz, der diesen Zweck bestimmt, wird selbst ein praktischer seyn, von der Form: \RWbet{Der letzte Zweck aller deiner Bestrebungen soll dieß und dieß seyn}. Hat man ihn einmal gefunden (und ich glaube, wirklich gezeigt zu haben, daß dieser letzte Zweck aller unserer Bestrebungen in der Beförderung des allgemeinen Wohles zu bestehen habe): so ist offenbar, daß sich die sämmtlichen Pflichten des Menschen schon aus ihm allein herleiten lassen, ohne daß es des Satzes: Deine sämmtlichen Handlungen sollen auf Einen Zweck gerichtet seyn, bedürfte. -- Erklärt man die \RWbet{Vollkommenheit} noch bestimmter dahin, daß sie die Uebereinstimmung aller Beschaffenheiten eines Gegenstandes mit demjenigen Zwecke sey, den er erreichen \RWbet{soll}; so wäre der Satz: strebe nach Vollkommenheit, sogar \RWbet{identisch.} Einige Gelehrte neuerer Zeit scheinen den Be\RWSeitenw{250}griff der Vollkommenheit ohngefähr so zu verstehen, wie ich die \RWbet{Allvollkommenheit} oben (\RWparnr{74}) in Beziehung auf \RWbet{Gott} erklärte, \dh\ sie scheinen sich unter der Vollkommenheit eines Wesens die möglich größte Entwicklung und Wirksamkeit seiner sämmtlichen Kräfte zu denken. Dann würde das Princip: Strebe nach Vollkommenheit, eigentlich den Sinn haben: \RWbet{Alles, was möglich ist} (jedes mögliche Wesen und jeder mögliche \RWbet{Zustand} desselben) \RWbet{soll wirklich gemacht}, oder noch besser, \RWbet{soll gewollt werden.} -- Hiegegen würde nun Mancher schon die Bedenklichkeit erheben, daß auch das sittlich \RWbet{Böse} möglich ist. Aber auf diesen Einwurf ließe sich vielleicht erwidern, daß alles sittlich Böse (\zB\ ein jeder \RWbet{Schmerz}, den wir Jemanden ohne Noth verursachen) zwar etwas \RWbet{Wirkliches}, aber nur etwas Solches sey, das die Entstehung viel mehrerer \RWbet{anderer} Wirklichkeiten verhindert. Diese Antwort hat viele Wahrscheinlichkeit; und wenn wir annehmen (was ich schon \RWparnr{77}\ bemerkte), daß das Bewußtseyn jeder Kraft eine angenehme Empfindung gewähre, und daß sonst nichts Anderes angenehm sey als das Bewußtseyn des Wirkens: so würde das Princip: \RWbet{Strebe darnach, daß Alles, was möglich ist, zur Wirklichkeit gelange}, im Wesentlichen ganz übereinstimmen mit dem hier aufgestellten: \RWbet{Strebe, die möglich größte Summe der Glückseligkeit zu bewirken}. Es ließe sich dann nur noch darüber streiten, welche von beiden Wahrheiten den Grund von der andern enthalte, und somit ursprünglich sey. -- Einige Gelehrte, \zB\ \RWbet{Eberhard} erklärten die \RWbet{Vollkommenheit} als die Tauglichkeit eines Wesens zum Genusse der Glückseligkeit. Verlangten sie nun, daß man nicht bloß die \RWbet{eigene}, sondern auch \RWbet{Anderer} Glückseligkeit befördern soll; so war ihr Princip im Grunde nicht von dem meinen verschieden. Glaubten sie aber, daß Jeder nur für seine eigene Vervollkommnung zu sorgen habe; so war ihr Princip einerlei mit dem gleich folgenden.
\begin{aufza}\setcounter{enumi}{14}
\item \RWbet{Prodikus, Aristippus, Demokritus, Epikurus} und seine Anhänger, \RWbet{Lukretius, Gassendi, Buddeus, Helvetius}, der Verfasser des \RWbet{Systems der Natur}, \RWbet{d'Alembert}, und noch viele Andere behaupteten, daß es gar keine andere Pflicht~\RWSeitenw{251}\ für den Menschen gebe als die der \RWbet{Selbstbeglückung}. \RWbet{Mache dich selbst glücklich}, war ihrer Meinung nach der oberste Grundsatz der ganzen Sittenlehre. Einige dieser Gelehrten, die man mit einem verächtlichen Nebenbegriffe \RWbet{Epikuräer} nennt (obgleich es wahrscheinlich ist, daß Epikur selbst nicht diese Meinung gehegt), setzten die Glückseligkeit, nach der man streben soll, in bloße \RWbet{Sinnenlust}. Andere dagegen, die man eben deßhalb mit dem glimpflicheren Namen der \RWbet{Eudämonisten} bezeichnet, verstanden unter der Glückseligkeit nicht bloße Sinnenlust, sondern auch geistige Vergnügungen, kurz Alles, was immer den Menschen glücklich macht. Diesem Eudämonismus waren vor Erscheinung der kritischen Philosophie die meisten Weltweisen Deutschlands, besonders unter den Protestanten, ergeben.
\end{aufza}\par
Ich glaube aber aus folgenden Gründen erweisen zu können, daß dieses Princip in jeder Bedeutung des Wortes Glückseligkeit falsch und verwerflich sey.
\begin{aufzb}
\item Wenn nur \RWbet{eigene Glückseligkeit} der letzte Zweck all unsers Strebens seyn dürfte; so wäre derjenige, der Millionen Menschen das Leben retten könnte, ohne daß es ihm auch nur das geringste Opfer kostet, keineswegs zu dieser Rettung verpflichtet, sobald er nicht irgend einen \RWbet{eigenen} Vortheil dabei fände. Wessen Gefühl empört sich nicht gegen diese Behauptung? So empörend sie ist, so falsch muß auch der Grundsatz seyn, aus dem sie folgen würde.
\item Dieses Princip verstattet keine Anwendung auf Gott; denn Gott kann seine eigene Glückseligkeit durch keine seiner Handlungen erhöhen. Aus welchem Grunde also hätte er die Welt geschaffen, wenn kein vernünftiges Wesen je etwas wollen \RWbet{soll}, ja auch nur wollen \RWbet{kann}, was nicht zu seiner eigenen Beglückung dient? In welchem Sinne könnte man ihn denn \RWbet{heilig} nennen?
\item Wenn der \RWbet{Tugendhafte} zu Folge des Eudämonismus nichts Anderes als seine eigene Glückseligkeit sucht; so ist zwischen ihm und dem \RWbet{Lasterhaften} kein Unterschied im Zwecke, sondern nur in den Mitteln, durch die sie ihren Zweck zu erreichen glauben, also in ihrem Verstande.~\RWSeitenw{252}\ Tugend ist also nichts Anderes als Klugheit, Laster nichts Anderes als Thorheit. Der gesunde Menschenverstand findet aber diesen Unterschied ganz anders.
\item Nach dem Systeme des Eudämonismus haben, wie man sieht, die Begriffe des \RWbet{Wünschens}, des \RWbet{Sollens} und des \RWbet{Wollens}, drei Begriffe, die der gesunde Menschenverstand doch so genau unterscheidet, Einen und eben denselben Umfang; denn alles dasjenige, was die Vernunft uns als zuträglich für unsere Glückseligkeit darstellt, und also zur \RWbet{Pflicht} uns auflegen würde, müssen wir auch vermöge unsers Glückseligkeitstriebes \RWbet{wünschen}, und somit würden und müßten wir es ohne Zweifel auch \RWbet{wollen}.
\item Eben deßhalb könnte es auch nach dem Systeme des Eudämonismus gar keine \RWbet{Freiheit}, wenigstens nicht in der indeterministischen Bedeutung, geben; denn eine solche Freiheit findet nur Statt, wenn wir uns zwei Handlungen als möglich vorstellen, deren die eine wir \RWbet{sollen}, und die andere \RWbet{wünschen}; also wenn ein Widerspruch zwischen dem \RWbet{Wünschen} und dem \RWbet{Wollen} da ist.
\item Wäre aber keine Freiheit; so würde Alles, was \RWbet{wirklich} ist, zugleich auch \RWbet{nothwendig}, und Alles, was \RWbet{nicht wirklich} ist, zugleich auch \RWbet{unmöglich} seyn. Das \RWbet{Mögliche}, das \RWbet{Wirkliche} und das \RWbet{Nothwendige} wären sonach abermals drei Begriffe von einerlei Umfang, was sehr befremdend klingt.
\item Einige Eudämonisten gaben als \RWbet{vornehmsten}, ja wohl als \RWbet{einzigen} Bestandtheil jener Glückseligkeit, nach welcher der Mensch streben soll, den Beifall \RWbet{seines Gewissens}, die \RWbet{innere Freudigkeit} an, die er nach Ausübung einer guten That empfinden würde. Allein um diese Freudigkeit zu empfinden, müssen wir erst eingesehen haben, daß eine gewisse That \RWbet{gut}, \dh\ dem obersten Sittengesetze gemäß ist. Dann also wäre die Formel abermals ein identischer Satz. Uebrigens würde sie an die Wahrheit erinnern, daß jenes frohe Bewußtseyn, das uns die Ausübung guter Thaten gewähret, wirklich einer der wichtigsten Bestandtheile unserer Glückseligkeit sey.~\RWSeitenw{253}
\end{aufzb}
\begin{aufza}\setcounter{enumi}{15}
\item \RWbet{Immanuel Kant} drückte das oberste Sittengesetz so aus: \RWbet{Handle nach derjenigen Maxime deines Willens, von der du wollen kannst, daß sie Gesetz einer allgemeinen Gesetzgebung} (auch wohl Naturgesetz) \RWbet{würde}, oder auch so: Thue jederzeit das, wovon du vernünftiger Weise wollen kannst, daß jeder Andere in deiner Lage eben so verführe, ja allenfalls selbst durch ein Naturgesetz so zu verfahren gezwungen wäre. Allein auch noch bei dem zweiten Ausdrucke liegt eine Zweideutigkeit in den Worten: \RWbet{etwas vernünftiger Weise wollen}. Dieses kann nämlich bedeuten a)~etwas so wollen, wie die Vernunft es billiget, oder b)~nur etwas wollen, das möglich ist, das sich nicht selbst widerspricht. Nimmt man die \RWbet{erste} Bedeutung, so erhält das Kantische Princip folgenden Sinn: \RWbet{Thue in jeder Lage das, wovon du gestehen mußt, daß es auch jeder Andere in dieser Lage thun soll}. Diese Vorschrift ist allerdings wahr, aber sie macht nicht das oberste Sittengesetz aus, sondern ist eine bloße Folgerung aus ihm, und zwar eine solche, die sich aus ihm ergibt, wie man auch immer sich vorstellen mag, daß es laute; sobald man nur voraussetzt, daß es ein Urtheil \RWlat{a priori}, eine reine \RWbet{Begriffswahrheit} sey. Denn eine solche muß, weil sie aus bloßen \RWbet{Begriffen} zusammengesetzt ist, \RWbet{Allgemeingültigkeit} haben, \dh\ von allen Individuen, die unter dieselben Begriffe subsummirt werden können, also von allen handelnden Wesen, die sich in gleichen Verhältnissen befinden, Gleiches fordern. Wie wenig aber dieser Satz die Dienste eines obersten Sittengesetzes vertreten könne, erhellet daraus, weil sich aus ihm allein noch gar nicht beurtheilen läßt, wie Jemand handeln soll, wenn nicht noch eine \RWbet{andere} praktische Wahrheit der Form: \RWbet{In dieser und jener Lage soll dieses und jenes geschehen}, zu Hülfe gezogen wird. -- Nimmt man dagegen die \RWbet{zweite} Bedeutung der Redensart: etwas vernünftiger Weise wollen, an; so erhält das Kantische Princip den Sinn: \RWbet{Du sollst in jeder Lage das thun, was ohne einen Widerspruch zu begehen, von allen Wesen, die in eben diese Lage gerathen, gefordert werden könnte}. Und diese Bedeutung scheint Kant selbst angenommen zu haben, wie man aus einigen seiner Beispiele sieht.~\RWSeitenw{254}\ So beweiset er die Pflicht, ein anvertrautes Gut nicht vorzuenthalten, durch die Bemerkung, daß die entgegengesetzte Maxime, wenn sie als allgemeines Gesetz aufgestellt werden sollte, sich selbst vernichten müßte; indem es dann gar keine anvertrauten Güter mehr geben würde, weil Niemand einem Andern etwas anvertrauen würde, wenn dieser nicht verpflichtet wäre, es wieder zurückzustellen. Meiner Meinung nach ist zwar kein Zweifel, daß ein Satz, der sich selbst widerspricht, falsch, und also sein contradictorisches Gegentheil wahr seyn müsse; aber ich glaube nicht, daß irgend eine der praktischen Wahrheiten, irgend eine der menschlichen Pflichten, auf diese Art erweislich sey. Denn keine dieser Wahrheiten ist, däucht mir, von einer solchen Beschaffenheit, daß ihr contradictorisches Gegentheil ein sich selbst widersprechender Satz wäre. Der Widerspruch, den Kant in dem angeführten und einigen ähnlichen Beispielen zu bemerken glaubte, ist nur scheinbar. Um dieß zuerst an dem von ihm selbst angeführten Beispiele zu zeigen, so ist es ja nicht völlig wahr, was hier gesagt wird, daß es gar keine anvertrauten Güter mehr geben würde, wenn es nicht Pflicht wäre, sie wieder zurückzustellen. Denn wenn es nur nicht Pflicht wäre, sie vorzuenthalten; so könnte ja Mancher noch hoffen, das Gut, das er einem Andern anvertraut, wieder zurück zu erhalten. Aber gesetzt auch, es würde dann keine anvertrauten Güter mehr geben: doch würde die Regel: ein anvertrautes Gut darfst du vorenthalten, noch nicht in einen Widerspruch mit sich selbst gerathen. Denn sie sagt ja nur, daß wir ein Gut, \RWbet{wenn es uns anvertraut worden ist}, nicht aber, \RWbet{wenn es uns nicht anvertraut worden ist}, vorenthalten dürfen. Daher kommt es denn auch, daß sich nach jenem Kantischen Princip viele ganz falsche Regeln vertheidigen ließen, \zB\ die Maxime: \RWbet{Mache, so viel du nur immer vermagst, Andere um dich her unglücklich}. Auch diese Regel nämlich kann ich, ohne einen Widerspruch zu begehen, allen Wesen, die sich in meiner Lage befinden, zur Befolgung vorschreiben. Denn wenn sich auch alle Menschen alle erdenkliche Mühe gäben, Alles um sich her unglücklich zu machen: so würde die Welt doch nicht gleich ausgestorben seyn, sondern es würde noch lange Geschöpfe geben, die sie zu quälen und zu martern im Stande wären. Und gesetzt auch, es käme~\RWSeitenw{255}\ endlich dahin, daß gar keine Geschöpfe, auf die sich dieses Gesetz anwenden ließe, vorhanden wären: so würde es sich darum noch immer \RWbet{nicht widersprechen}, denn es sagt ja nur, daß man, so lange es \RWbet{möglich} ist, quäle. -- So unbrauchbar aber dieß Kantische Princip, meiner Ansicht nach, zu einem obersten Sittengesetze ist, so ist doch nicht zu läugnen, daß es an eine sehr wichtige Wahrheit erinnert, nämlich an die, daß man das \RWbet{Schädliche} und eben deßhalb auch \RWbet{Unerlaubte} einer Handlungsweise öfters erst dann recht anschaulich finde, wenn man sich vorstellt, daß sie allgemein ausgeübt würde.
\item Doch auch noch auf folgende Art glaubte Kant das oberste Sittengesetz ausdrücken zu können: \RWbet{Behandle kein vernünftiges Wesen als bloßes Mittel, sondern betrachte ein jedes auch als Zweck an sich}.
\end{aufza}\par
Diese Formel scheint folgenden Sinn zu haben: Bei jeder Handlung, durch welche du eine gewisse Veränderung in dem Zustande eines vernünftigen Wesens hervorbringen würdest, sollst du auf die Beschaffenheit dieser Veränderung merken, und sie soll einen von den Bestimmungsgründen, aus denen du dich entweder zur Ausübung oder zur Unterlassung der Handlung entschließest, ausmachen. Das wäre nun allerdings sehr wahr, allein es ist gewiß nicht das oberste Sittengesetz; schon darum nicht, weil sich aus dieser Regel nicht \RWbet{alle} Pflichten des Menschen herleiten lassen. Denn es gibt \zB\ gewiß auch Handlungen, die keinen uns bemerkbaren Einfluß auf den Zustand \RWbet{vernünftiger}, aber wohl \RWbet{thierischer} Wesen haben; und der gesunde Menschenverstand gebietet, daß wir auch auf diesen Einfluß merken, und \zB\ unter zwei Handlungen, die einen gleich vortheilhaften Einfluß auf vernünftige Wesen haben, deren die Eine aber einem Thiere Schmerz macht, während dieß bei der andern nicht der Fall ist, die letztere vorziehen sollen. Es ist also gewiß unrichtig zu behaupten, daß man lebendige Wesen, wenn sie vernunftlos sind, als bloße Mittel gebrauchen dürfe, und nicht verpflichtet sey, auch in ihnen einen Zweck an sich anzuerkennen.
\begin{aufza}\setcounter{enumi}{17}
\item \RWbet{Fichte} lehrte: \anf{\RWbet{Das Princip der Sittlichkeit ist der nothwendige Gedanke der Intelligenz, daß sie ihre Freiheit nach dem Begriffe der Selbstständigkeit schlechthin ohne Ausnahme bestimmen}~\RWSeitenw{256}\ \RWbet{soll},} \dh\ ein jedes freie Wesen soll seine Freiheit unter ein \RWbet{Gesetz} bringen, welches kein anderes seyn soll, als der Begriff der \RWbet{absoluten Selbstständigkeit}, \dh\ der absoluten Unbestimmbarkeit durch irgend etwas außer ihm. Oder (wie es an einer andern Stelle heißt): der Endzweck aller Handlungen des sittlich guten Menschen ist, \RWbet{daß die Vernunft und nur sie in der Sinnenwelt herrsche}. So erwies er \zB\ die Pflicht, unser und Anderer Leben zu erhalten, daraus, weil wir ja Werkzeuge zur Realisirung des Sittengesetzes in der Sinnenwelt sind, \udgl
\end{aufza}\par 
Verstehe ich anders die Ausdrücke recht, so ist dieß Princip ein bloß identischer Satz, was auch aus der besondern Art, wie Fichte sein Princip deducirte, schon zu erwarten war. Die Vernunft und nur sie soll in der Sinnenwelt herrschen, heißt doch nichts Anderes als: Alles, was von der Freiheit des Menschen abhängt, soll nach den Vorschriften, welche die Vernunft darüber aufstellt, eingerichtet werden, oder es soll eingerichtet werden, wie es eingerichtet werden soll!
\begin{RWanm} 
Wer die hier angeführten, und noch so viele andere Formeln, durch die man das oberste Sittengesetz geglaubt hat ausdrücken zu können, der Reihe nach aufzählen hört, wird vielleicht kaum begreifen, wie man auf so verschiedenartige Abwege kommen, und das so nahe liegende, und so natürlich sich darbietende \RWbet{wahre} Gesetz habe verfehlen können. Die Ursache dieser Erscheinung liegt, wie ich glaube,
\begin{aufzb}
\item vornehmlich in dem Umstande, daß sich die wenigsten Gelehrten deutlich bewußt wurden, \RWbet{was} sie eigentlich unter dem obersten Sittengesetze verstehen sollten. Die Meisten scheinen es nämlich nie recht bedacht zu haben, daß sie den letzten \RWbet{objectiven Grund} aller Verbindlichkeiten und Pflichten, oder überhaupt aller praktischen Wahrheiten angeben sollen. Sie scheinen das oberste Sittengesetz mit einem Satze, der nur irgend einen \RWbet{subjectiven Erkenntnißgrund} unserer Pflichten, ein Mittel zur Erinnerung an sie, abgeben kann, verwechselt zu haben. Nur daraus wird begreiflich, wie man Formeln der Art, wie: Thue, was gut ist! Folge der Vernunft! Handle deinen sämmtlichen Verhältnissen gemäß! Ahme Gott nach! \udgl\  für das oberste Sittengesetz habe ausgeben können.~\RWSeitenw{257}
\item Daß man das \RWbet{wahre} (nämlich von mir für wahr gehaltene) Princip nicht häufiger annahm, kam \RWgriech{a}) aus der Vorstellung von der Gefährlichkeit desselben; \RWgriech{b}) aus der Meinung, daß sich nicht alle Pflichten aus demselben herleiten ließen; \RWgriech{g}) aus Abscheu gegen den Eudämonismus, mit dem es häufig verwechselt worden ist.
\item An den Verirrungen, in welche die Gelehrten der neuesten Zeit verfielen, hat nebst der Sucht, sich durch die Aufstellung eines bisher noch unerhörten Principes auszuzeichnen, den größten Antheil das logische Vorurtheil, daß eine jede recht unwidersprechliche Herleitung einer Wahrheit von dem Satze der Identität $A = A$ ausgehen müsse.
\item Wie das Princip des \RWbet{Eudämonismus}, das so verderblich ist, gleichwohl so viele Anhänger gefunden habe, läßt sich aus folgenden Gründen erklären: \RWgriech{a}) Aus Sinnlichkeit ist es dem Menschen sehr natürlich zu wünschen, daß er keine andere Verbindlichkeit als die der Selbstbeglückung hätte; und dieser Wunsch erzeugte die Ueberredung, daß es auch wirklich so sey. Zumal da man \RWgriech{b}) die Beschwerlichkeiten, welche es kostet, dem Triebe nach Glückseligkeit zu widersprechen, öfters für eine völlige Unmöglichkeit ansah; und freilich, wofern es unmöglich wäre, je etwas Anderes zu beschließen, als was der Glückseligkeitstrieb verlangt: so würde es eigentlich gar keine Pflichten geben; oder, wenn man ihr Daseyn doch annehmen wollte: so könnte ihr oberster Grundsatz nicht anders lauten, als: Thue, was du wünschest, oder, was dich selbst glücklich macht. Denn was die Vernunft uns als eine Pflicht auflegen soll, muß möglich seyn; nach jener Voraussetzung aber wäre uns nichts Anderes möglich, als unseren Wünschen zu folgen. \RWgriech{g}) Andere sahen die hohe Lebhaftigkeit, die der Glückseligkeitstrieb hat, wenn gleich für keinen Beweis der Unmöglichkeit, ihm zu widerstehen, doch wenigstens für einen Wink Gottes selbst an, daß wir ihm folgen sollen. \RWgriech{d}) Man glaubte ferner bemerkt zu haben, und es ist wohl auch wahr, daß die meisten Handlungen der Menschen aus einer bald deutlichen, bald minder deutlichen Rücksicht auf eigenen Vortheil entspringen; und daraus schloß man, daß dieses bei allen der Fall sey, daß es auch gar nicht anders seyn könne, und daß die Glückseligkeit der einzig mögliche Beweggrund aller Handlungen wäre; da sie doch in der That oft nur mitwirkender Grund ist, zuweilen auch gar keinen Antheil hat. \RWgriech{e}) Was aber den Eudämonismus mehr als alles Uebrige begünstigte, ist der Umstand,~\RWSeitenw{258}\ daß sich die meisten, oder fast alle Pflichten des Menschen auch aus dem Grunde der Selbstbeglückung wenigstens scheinbarer Weise herleiten lassen. 
\end{aufzb}
\end{RWanm}

\RWpar{91}{Unsicherheit aller menschlichen Tugend}
Gemäß dem Versprechen \RWparnr{86}\ sollen jetzt noch einige der wichtigsten Sätze aus der \RWbet{natürlichen Tugendmittellehre} folgen.
\begin{aufza}
\item Vor Allem muß ich jedoch erst die Wahrheit, \RWbet{daß alle menschliche Tugend unsicher sey}, erweisen, \dh\ ich muß erklären, daß und warum Niemand von seiner eigenen Beharrlichkeit in der Tugend im Voraus vollkommen sicher seyn könne, so ernst und lebhaft auch immer sein gegenwärtiger Vorsatz, allen Vorschriften der Tugend getreu zu bleiben, seyn mag. Es versteht sich aber von selbst, daß ich hier
\begin{aufzb}
\item nicht von \RWbet{Vergehungen einer bestimmten Art}, sondern von Uebertretungen des Sittengesetzes \RWbet{überhaupt} rede; denn von gewissen Fehlern, \zB\ solchen, zu denen wir gar keine Anlage, keine Gelegenheit, nicht einmal die nöthigen Kräfte haben, können wir wohl mit ziemlicher Gewißheit sagen, daß wir sie nie begehen werden. Auch nehme ich
\item den Fall aus, wo es Jemand durch eine göttliche Offenbarung selbst bekannt geworden wäre, daß er bis an das Ende seines Lebens in der Tugend beharren werde.
\end{aufzb}
\item Nachdem ich auf diese Art den \RWbet{Sinn} meiner Behauptung bestimmt, schreite ich zu ihrem \RWbet{Beweise}. Unmittelbar aus der \RWbet{Erfahrung}, aus einer Erfahrung, die wir ein Jeder theils \RWbet{an uns selbst} (wenn wir auf die verflossenen Jahre unsers Lebens zurücksehen), theils auch an \RWbet{Andern} machen können, ergibt sich, daß es nichts Ungewöhnliches beim Menschen sey, die ernstesten und lebhaftesten Vorsätze zu fassen, und doch in kurzer Zeit darauf ihnen ganz entgegen zu handeln. Nicht bloß in Fehler, die uns schon zur unglücklichen Gewohnheit und zweiten Natur geworden sind, pflegen wir, leider! auch nach den festesten Vorsätzen wieder zurückzufallen;~\RWSeitenw{259}\ sondern selbst neue Fehler, Fehler, zu deren Begehung wir uns früher noch nie versucht gefühlt hatten, lassen wir uns oft zu Schulden kommen, wenn eine unerwartete Versuchung dazu eintritt. Menschen, die bis in ihr hohes Alter den Pfad der Tugend gewandelt, verirren sich zuweilen noch in ihren letzten Lebenstagen auf die Bahn des Lasters. Diese nur allzutraurigen Erfahrungen müssen jeden vernünftigen Menschen aufmerksam machen, und was so vielen Andern begegnete, muß er auch für sich selbst, und was vielleicht ihm selbst schon so oft ehedem geschah, muß er auch für die Zukunft besorgen. So fest also auch seine Vorsätze seyn mögen; so darf er sich doch niemals für ganz versichert halten, daß er den Vorschriften der Tugend in keiner Rücksicht je wieder werde ungetreu werden.
\end{aufza}

\RWpar{92}{Was uns zur Untreue an unsern tugendhaften Vorsätzen verleite?}
\begin{aufza}
\item Nach einer schon mehrmals vorgetragenen Bemerkung finden wir Menschen uns nur damals frei, oder nur damals ist es uns möglich, das Sittengesetz zu übertreten, wenn die Vernunft und der Glückseligkeitstrieb bei uns in Streit gerathen. Man kann also sagen, daß es nur die Wünsche unsers Glückseligkeitstriebes sind, die uns zur Untreue an unsern guten Vorsätzen und zu Abweichungen von der Tugend verleiten. Je öfter nun dieser Widerspruch zwischen den Wünschen unsers Glückseligkeitstriebes und den Forderungen unserer Vernunft eintritt, und je lebhafter unsere Wünsche dann sind, um desto größer ist auch die Gefahr, daß wir ihnen folgen, obgleich wir dazu niemals genöthiget sind.
\item Vermöge des Glückseligkeitstriebes aber kann der Mensch immer nur das wünschen, was sich ihm als ein taugliches Mittel zur Beförderung seiner wahren Glückseligkeit wenigstens in diesem Augenblicke darstellt. Wenn nun ein Mensch an Gottes Daseyn glaubt, oder auch sonst auf irgend eine Art von der wichtigen Wahrheit überzeugt ist, daß Tugend allein der sicherste Weg zur wahren Glückseligkeit sey, und daß eine jede böse That früher oder später unausbleiblich unglücklich mache: so~\RWSeitenw{260}\ entsteht die Frage, wie sich ein Solcher noch zu irgend einer Abweichung vom Sittengesetze versucht fühlen könne? -- Ich antworte hierauf: Nur dadurch, daß er sich dieser wichtigen Wahrheit in einzelnen Augenblicken entweder gar nicht erinnert, oder wohl an sie denkt, aber sie nicht verlässig genug findet. Dieses nun ist wieder nur durch die besondere Einrichtung unseres Gemüthes möglich, zu Folge der es immer nur Eine oder einige mit einander verwandte Vorstellungen gibt, die wir zu einerlei Zeit mit einem lebhaften Bewußtseyn zu umfassen vermögen. Stellt sich der Mensch \zB\ die Lust der Sünde lebhafter vor; so verschwinden in eben dem Augenblicke die Vorstellungen von Gott, von den Strafen des andern Lebens \udgl ; und nur so kann es geschehen, daß wir die Ausübung der bösen That jetzt für uns vortheilhaft und somit \RWbet{erwünschlich} finden. Nach dieser Erklärung ist also die Freiheit des Menschen auf der Ideenassociation und auf der Eingeschränktheit seines Bewußtseyns gegründet. Von jeder Sünde kann man, nach dieser Darstellung, behaupten, daß sie aus einer Art von \RWbet{Gottesvergessenheit}, wenn nicht gar Läugnung oder Bezweiflung des Daseyns Gottes entspringe.
\end{aufza}

\RWpar{93}{Es gibt gewisse Tugendmittel}
Was immer auf irgend eine Weise zur Sicherung, Erleichterung oder Vervollkommnung unserer Tugend beitragen kann, will ich in dieser Rücksicht ein \RWbet{Tugendmittel} nennen.\par
Daß nun dieser Begriff nicht ohne Gegenstand sey, \dh\ daß es gewisse Mittel zur Beförderung oder Sicherung unserer Tugend in Wirklichkeit gebe: ist leicht zu beweisen. Alles nämlich, was dazu beiträgt, uns eine vollständigere und geläufigere Kenntniß von unseren Pflichten zu verschaffen, ingleichen Alles, was dazu beiträgt, daß die Wünsche unsers Glückseligkeitstriebes mit den Forderungen unserer Vernunft seltener in Widerspruch gerathen, und selbst in solchen Fällen schwächer und gemäßigter werden, das Alles trägt auch zur Sicherung, Erleichterung und~\RWSeitenw{261}\ Vervollkommnung unserer Tugend bei. Nun gibt es aber ohne Zweifel gar manche Mittel, die das so eben Erwähnte zu leisten vermögen. So muß \zB\ durch öfteres Nachdenken über unsere Pflichten unsere Erkenntniß derselben immer vollständiger und geläufiger werden; durch öfteres Nachdenken über die Wahrheit, daß nur die Tugend allein wirklich beglücke, durch die Verbindung verschiedener angenehmer Vorstellungen mit dem, was unsere Schuldigkeit ist \udgl , muß sich unsere Abneigung vor der Erfüllung derselben vermindern. Es gibt also ohne Zweifel gewisse Tugendmittel. Unter Anderen ist, wie leicht einzusehen, auch die \RWbet{Vorstellung gewisser Vortheile}, die aus der Erfüllung unserer Pflichten \RWbet{für uns selbst} hervorgehen werden, ein Mittel, uns zu ihrer Beobachtung sicherer zu bestimmen, und ihre Vollziehung uns leichter zu machen, also ein \RWbet{Tugendmittel}. Diese besondere Gattung von Tugendmitteln will ich \RWbet{Beweggründe} zur Tugend, \RWbet{und zwar vom Glückseligkeitstriebe entlehnte Beweggründe}, oder auch kürzer \RWbet{sinnliche Beweggründe} nennen.

\RWpar{94}{Es ist nicht nur erlaubt, sondern auch Pflicht, sich aller Tugendmittel, zu denen man nur Gelegenheit hat, zu bedienen}
\begin{aufza}
\item Daß wir Alles thun sollen, was zur \RWbet{Vervollkommnung} unserer Tugend nur immer beitragen kann, wird kein Vernünftiger bezweifeln, da diese Behauptung im Grunde ein bloß identischer Satz ist. (\RWparnr{88}\ \no\,5.)
\item Aber auch Alles, was zur mehren \RWbet{Sicherung} unserer Tugend beitragen kann, sollen wir benützen. Denn weil wir \RWbet{nie völlig} sicher seyn können, daß wir uns keine Abweichung von unseren tugendhaften Vorsätzen werden zu Schulden kommen lassen (\RWparnr{92}); so müssen wir diese Sicherheit, so sehr wir können, \RWbet{vermehren}.
\item Daß wir endlich auch Alles, was zur \RWbet{Erleichterung} unserer Pflichten beitragen kann, benützen dürfen und sollen, erhellet~\RWSeitenw{262}
\begin{aufzb}
\item daraus, weil durch Erleichterung unserer Pflichten unsere \RWbet{eigene} Glückseligkeit gewinnt, die eines \RWbet{Andern} aber nicht beeinträchtiget, und also die Summe des Wohlseyns im Ganzen gewiß vermehrt wird.
\item Was uns \RWbet{leichter fällt}, thun wir auch \RWbet{sicherer}; und so sind wir also auch, um unsere Tugend zu sichern, verpflichtet, sie uns so viel als möglich zu erleichtern.
\end{aufzb}
\item So offenbar dieß Alles auch für den gesunden Menschenverstand ist; so haben doch \RWbet{Gelehrte} dagegen viele sehr scheinbare Einwürfe vorgebracht, deren wichtigste wir genauer kennen lernen müssen.
\end{aufza}\par
\RWbet{1.~Einwurf.} Durch den Gebrauch der Tugendmittel wird unsere Tugend eigentlich nur darum gesichert, weil sie erleichtert wird. Allein je leichter uns die Ausübung der Tugend fällt, um desto geringer wird auch ihr Werth. Denn es ist doch ein allgemein anerkannter Satz: \RWbet{je schwerer der Kampf, um desto verdienstvoller der Sieg.} Durch den Gebrauch der Tugendmittel also schmälern wir nur das Verdienst unserer Tugend.\par
\RWbet{Antwort.} Der Satz: je schwerer der Kampf, um so verdienstvoller der Sieg, hat nur dann seine Richtigkeit, wenn der Kampf von unserer Seite nicht zu vermeiden war. Wenn aber Jemand durch die Vernachlässigung solcher Tugendmittel, die ihm doch zu Gebote standen, selbst daran Ursache ist, daß er nun einen schweren Kampf zu kämpfen hat: so wird ihm Gott seine Belohnung um dieses Grundes willen gewiß nicht vermehren, sondern vielmehr ihn dafür strafen, daß er sich einer so großen Gefahr ohne Noth ausgesetzt hat. Umgekehrt also denjenigen, der es durch fleißigen Gebrauch der Tugendmittel dahin bringt, daß ihm die Erfüllung aller seiner Pflichten je länger je leichter wird, und daß er eben deßhalb nun seltener sündiget, den wird Gott darum nicht weniger belohnen, sondern vielmehr noch eigens \RWbet{dafür} belohnen, daß er die Ausübung der Tugend bei sich so sicher gestellt hat, als es ihm möglich war. Und wie sehr widerspricht man sich in diesem Einwurfe auch selbst! Man fürchtet, daß man das Verdienst der Tugend schmälere, wenn man sich ihre Ausübung durch Tugendmittel erleichtert; und eben, indem~\RWSeitenw{263}\ man aus diesem Grunde (\dh\ um mehr Belohnung zu erlangen) sich die Tugend erschweren will, bestimmt man sich ja nach einem eigennützigen Beweggrunde.\par
\RWbet{2.~Einwurf.} Aber indem wir uns öfteren Versuchungen zum Bösen aussetzen, wird unsere sittliche Kraft geübt und gestärkt, und wir sind dann im Stande, auch solche Versuchungen, die unvermeidlich sind, desto sicherer zu besiegen.\par
\RWbet{Antwort.} Wir fern es ein Mittel zur Uebung der sittlichen Kräfte wäre, sich gewissen Versuchungen (Reizen) zum Bösen auszusetzen, in sofern dürfte dieß selbst als ein \RWbet{Tugendmittel} betrachtet werden. In der That aber wird dieß überaus selten der Fall seyn, und wir thun meistens besser, jede Versuchung zum Bösen, die an sich vermeidlich ist, auch zu vermeiden. -- Uebrigens ist die Vermeidung der versuchenden Gelegenheit gar nicht das einzige Tugendmittel, wie man in diesem Einwurfe annimmt, sondern es gibt noch viele andere Mittel, die unsere Kraft zur Tugend stärken können; \zB\ die Vermehrung der Ueberzeugung, daß Tugend allein beglücke \udgl\par 
\RWbet{3.~Einwurf.} Aber so wird sich der Tugendhafte doch nie der Kraft, die seine Tugend besitzt, deutlich bewußt werden können!\par
\RWbet{Antwort.} Er wird sich immer bewußt werden können, daß er die Tugend um ihrer selbst willen liebe; daß er aber den bestimmten \RWbet{Grad} seiner Tugendkraft erfahre, ist theils etwas Unmögliches, theils doch unnöthig und selbst schädlich; denn es könnte ihn nur zu Stolz und Uebermuth verleiten.\par
\RWbet{4.~Einwurf.} So ist denn wenigstens der Gebrauch jener \RWbet{vom Glückseligkeitstriebe entlehnten Beweggründe zur Tugend} unerlaubt; denn durch diese wird unsere Tugend selbst vernichtet, und ein schnöder Eigennutz tritt an ihre Stelle; oder wenigstens wird doch die Reinheit ihrer Triebfeder, welche bloß die Achtung gegen das Gesetz seyn soll, durch die Beimischung sinnlicher Vortheile besteckt.~\RWSeitenw{264}\par
\RWbet{Antwort.} Ich setze voraus, daß derjenige, der sich solcher Beweggründe bedient, den Vorsatz der Tugend nicht bloß um jener Vortheile \RWbet{willen} fasse, sondern vielmehr sich zur Befolgung des Sittengesetzes schon vorher aus der Erkenntniß, daß es so \RWbet{recht} sey, entschlossen habe, und daß er dieser Vorstellungen sich jetzt bloß als eines Mittels bediene, um den gefaßten Vorsatz desto \RWbet{gewisser} auszuführen. Wenn dieses der Fall ist; so kann man nicht sagen, daß seine Tugend durch den Gebrauch solcher Beweggründe \RWbet{in schnöden Eigennutz} verwandelt werde, nicht einmal daß die \RWbet{Reinheit ihrer Triebfeder} dabei verliere. Denn auch bey dem Gebrauche aller nur möglichen Tugendmittel wird es der Mensch doch nie dahin bringen, daß ihm die Ausübung der Tugend gar keine Mühe mehr koste, daß sein Glückseligkeitstrieb mit seiner Vernunft in gar keinen Widerspruch mehr gerathe. Immer also wird es noch Fälle geben, wo er mit Freiheit handeln, und sich zur Ausübung des Guten nur darum, weil es gut ist, wird bestimmen können. Allein selbst in denjenigen Fällen, wo sein Glückseligkeitstrieb mit den Forderungen der Vernunft zusammentrifft, wo er also eigentlich nicht mehr mit Freiheit handelt, wird seine Handlung noch \RWbet{verdienstlich} seyn. Denn sie ist eine dem Wohle des Ganzen noch immer zuträgliche Handlung, und ihre Ausübung ist, wenn nicht unmittel-, doch mittelbarer Weise ein Werk seiner Freiheit. Damals nämlich, als er durch seine freie Thätigkeit seinem Glückseligkeitstriebe jene besondere Richtung ertheilte, durch welche derselbe jetzt mit der Vernunft übereinstimmt, damals schon legte er den Grund zu seiner jetzigen Handlung. Und die Belohnung, die er für jene Bearbeitung seines Glückseligkeitstriebes verdient, muß größer angesetzt werden, als die Summe aller der Belohnungen, die ihm für seine jetzigen Handlungen, wenn er sie noch im Kampfe mit der Sinnlichkeit vollzöge, zugedacht werden müßten. Denn eben dadurch, daß er sich ganz von diesem Kampfe befreit hat, erfolgt ja die Ausübung des Guten nur um so sicherer, das Wohl des Ganzen gewinnt nur um so mehr; die Belohnung also, die ihm Gott zuweiset, muß um so größer seyn.\par
\RWbet{5.~Einwurf.} Aber durch den Gebrauch der vom Glückseligkeitstriebe entlehnten Beweggründe zur Tugend bringt~\RWSeitenw{265}\ man zwar einige dem Gesetze gemäße Handlungen hervor, vergrößert aber die Herrschaft des Glückseligkeitstriebes über seinen Willen, und es erfolgen von nun an um so mehre dem Gesetze zuwiderlaufende Handlungen.\par
\RWbet{Antwort.} Ich läugne nicht, daß es Beweggründe zur Tugend geben kann, von denen dieser Vorwurf gilt; Beweggründe von dieser Art darf man eben deßhalb nicht zu den \RWbet{Tugendmitteln} zählen, und als solche anwenden. Aber sicher gibt es auch \RWbet{andere} vom Glückseligkeitstriebe entlehnte Beweggründe zur Tugend, die dieser Vorwurf nicht trifft, von denen man keineswegs sagen kann, daß sie uns in der Folge zu desto mehren dem Sittengesetze zuwiderlaufenden Handlungen verleiten. Hieher gehört \zB\ die Vorstellung, daß Tugend allein wahrhaft beglücke; \uam

   
\RWpar{95}{Einige Regeln, nach denen der vergleichungsweise Werth verschiedener Tugendmittel geschätzt werden kann}
Schon aus dem Bisherigen ist zu ersehen, daß der Werth verschiedener Tugendmittel verschieden ist. Ich sage nun, ein Tugendmittel sey, wenn die übrigen Umstände gleich sind, um so vorzüglicher:
\begin{aufza}
\item \RWbet{je sicherer es den beabsichtigten Zweck erreichet}. -- Je sicherer die Wirksamkeit des Mittels ist, um desto mehr wird unsere Tugend durch den Gebrauch desselben befördert, weil die Abweichugnen um desto seltener erfolgen. So ist \zB\ die gänzliche Vermeidung einer versuchenden Gelegenheit (so oft sie möglich ist) einem auch noch so festen und bestimmten Vorsatze, wie man sich in der Versuchung benehmen wolle, vorzuziehen; denn solche Vorsätze pflegen ihres Erfolges doch häufig zu verfehlen.
\item \RWbet{je allgemeiner dasselbe in seiner Anwendbarkeit ist}. In je mehren Fällen ein Tugendmittel anwendbar ist, um so mehr Gutes wird es bewirken, wenn wir es uns geläufig machen. So ist \zB\ der Gebrauch gewisser \RWbet{Denksprüche} ein besseres Tugendmittel, als der Gebrauch \RWbet{gewisser Erinnerungszeichen}, die nur an \RWbet{einzelnen}~\RWSeitenw{266}\ \RWbet{Orten} aufgestellt werden, also auch nur dort auf uns wirken können.
\item \RWbet{je weniger es dem Mißbrauche ausgesetzt ist}. Bei gewissen Tugendmitteln, besonders bei den \RWbet{Beweggründen} trifft es sich nämlich, daß sie zuweilen auch in Mißbrauch übergehen, und statt uns zum Guten anzutreiben, zu einer bösen That verleiten können. Wofern nun die Gefahr des Schadens größer als die Hoffnung des Nutzens ist; so ist ein solches Mittel ganz zu verwerfen. Im Vergleiche des einen mit dem andern aber ist immer jenes Tugendmittel das vorzüglichere, das der Gefahr des Mißbrauches weniger ausgesetzt ist. So ist \zB\ der Beweggrund des Beifalls bei Vernünftigen ein besseres Tugendmittel, als der Beweggrund des Beifalls bei der großen Menge, \udgl\ 
\item \RWbet{je mehr es bei seinem Gebrauche nicht sowohl Schmerz, als Lust hervorbringt}; denn um so mehr trägt es zur Beförderung unserer Glückseligkeit bei. So ist \zB\ die Hoffnung ein schätzbareres Tugendmittel als die Furcht, wenn beide von gleicher Wirksamkeit sind.
\item \RWbet{je verwandter ein Tugendmittel mit einem andern edlerer Art ist}, \dh\ je leichter der Uebergang von jenem zu diesem Statt findet. So ist \zB\ die Furcht vor einem noch weit entfernten, vielleicht erst im Alter bevorstehenden Schaden ein edleres Tugendmittel, als die Furcht vor einer Strafe, die gleich auf der Stelle eintritt; denn die erstere macht uns empfänglicher für das sehr wirksame Tugendmittel, das in der Furcht vor den Bestrafungen des andern Lebens bestehet.
\item \RWbet{je mehr es die Bildung des Geistes befördert}. Aus dem vorhergehenden Grunde; denn um so geschickter macht es uns zur Benützung der \RWbet{vorzüglichsten Tugendmittel}, welche gerade viel Bildung des Geistes bedürfen.
\begin{RWanm}
Aus allem Diesen ergibt sich, daß der vorzüglichste aller Beweggründe zur Tugend der Beweggrund sey, den der Gedanke an Gottes Willen mit sich führt.~\RWSeitenw{267}
\end{RWanm}
\end{aufza}

\RWch{Drittes Hauptstück.\\ Würdigung der natürlichen Religion\\ und\\ Nothwendigkeit einer Offenbarung.}
\RWpar{96}{Inhalt und Zweck dieses Hauptstückes}
Nachdem wir die wichtigsten Wahrheiten der bloßen Vernunftreligion in einem kurzen Umrisse kennen gelernt, sind wir im Stande, auch ihren Werth zu würdigen, und zu untersuchen, ob diese Religion für alle Bedürfnisse der Menschen so völlig hinreichend sey, daß kein Vernünftiger Ursache hat, das Daseyn einer göttlichen \RWbet{Offenbarung} weder für sich noch für Andere zu wünschen. Diese Untersuchung ist es, mit der wir uns in dem gegenwärtigen Hauptstücke zu beschäftigen haben. Warum dieß geschehen müsse, ist in der Einleitung bereits gesagt. Wer nämlich glauben könnte, daß die so eben aufgestellte Frage \RWbet{bejahend} zu beantworten sey, \dh\ daß schon die bloße Vernunftreligion für alle Bedürfnisse des Menschen vollkommen zureiche, der könnte es eben deßhalb nicht der Mühe werth finden, eine göttliche Offenbarung zu suchen, ja er würde sich sogar berechtigt glauben, im Voraus zu behaupten, daß sie, als etwas Ueberflüssiges, nicht vorhanden sey.\par
Dieß Letztere hat man nun wirklich in unseren Tagen behauptet, hat jede Offenbarung, die uns von Gott zu Theil werden sollte, für überflüssig, wohl gar für schädlich erklärt, und eben daraus den Schluß gezogen, daß es zu Folge der höchsten Weisheit und Heiligkeit Gottes, nach der er nichts Ueberflüssiges, um so weniger etwas Schädliches~\RWSeitenw{268}\ thun kann, im Voraus gewiß sey, daß er sich nirgends geoffenbaret habe. Um desto nöthiger ist es, daß wir mit ausführlichen Gründen und mit Berücksichtigung der dagegen erhobenen Einwürfe darthun, eine göttliche Offenbarung sey -- wie für das menschliche Geschlecht im Ganzen, so auch für jeden Einzelnen -- so viel wir die Folgen derselben nur immer beurtheilen können -- nicht bloß sehr heilsam und ersprießlich, sondern sogar ein dringendes Bedürfniß.

\RWpar{97}{Was unter der hier zu beweisenden Nothwendigkeit einer Offenbarung verstanden werde?}
Aus dem so eben Gesagten kann man schon einiger Maßen entnehmen, was ich eigentlich unter derjenigen \RWbet{Nothwendigkeit einer Offenbarung}, zu deren Beweise ich mich hier anheischig mache, verstehe. Ich will nämlich
\begin{aufza}
\item für's Erste keineswegs sagen, daß eine Offenbarung dem Menschen im strengsten Sinne des Wortes \RWbet{unentbehrlich} wäre, etwa bloß um sein Daseyn auf Erden fortzusetzen, oder sich nur zu irgend einem Grade der Tugend und Glückseligkeit zu erheben. Denn sowohl in dem Einen als in dem andern Falle würde mich die Erfahrung selbst widerlegen, die zeigt, daß Menschen auch ohne den Besitz einer göttlichen Offenbarung bei dem bloßen Lichte der natürlichen Religion gelebt, und sich zu einem gewissen Grade der Tugend und Glückseligkeit emporgearbeitet haben.
\item Meine Meinung gehet nur dahin,
\begin{aufzb}
\item daß jene natürliche Religion, mit der sich der aufgeklärteste Theil des menschlichen Geschlechtes begnügen müßte, wenn es keine göttliche Offenbarung gäbe, nicht so vollkommen sey, daß sich nicht eine geoffenbarte Religion zur Beförderung seiner Tugend und Glückseligkeit ungleich wirksamer bezeigen könnte; ja daß die erstere gewisse Mängel habe, welche auch in den besten und weisesten Menschen den bestimmten Wunsch nach einer vollkommeneren Belehrung durch Gott erwecken.~\RWSeitenw{269}
\item daß noch weit mehr die übrige Menge der Menschen, das menschliche Geschlecht im Ganzen, göttlicher Offenbarungen bedürfe, um in der Tugend und Glückseligkeit nicht immer tiefer zu sinken.
\item daß aber dieses Alles nur mit dem Beisatze: \RWbet{so viel wir Menschen es zu beurtheilen vermögen}, gelte; wodurch erklärt werden soll, daß Gott, der Allwissende, die Sache doch vielleicht noch anders finden, daß er gewisse uns Menschen nicht erkennbare Nachtheile bei einer Offenbarung vorhersehen, und durch diese bestimmt werden könnte, sie entweder dem ganzen menschlichen Geschlechte, oder doch einzelnen Theilen desselben für immer oder auf eine gewisse Zeit vorzuenthalten, daher wir denn nicht berechtiget wären, über ihn zu klagen, wenn wir oft einzelne Menschen, oft ganze Völker antreffen, welche der Wohlthat einer Offenbarung entbehren.
\begin{RWanm} 
Man könnte fragen, ob es nicht rathsamer wäre, eine so zu verstehende Nothwendigkeit einer Offenbarung lieber nur eine \RWbet{Nützlichkeit} derselben zu nennen, und eben hiedurch allen Verdacht, als ob man die Sache übertreiben wolle, zu entfernen? Hierauf erwidere ich nun, daß der Sprachgebrauch eine Sache bloß \RWbet{nützlich} nennt, wenn ihr Besitz uns zwar gewisse Vortheile gewährt, ihr Mangel aber gleichwohl nicht schmerzliche Empfindungen verursacht; daß er dagegen sie \RWbet{nothwendig} nenne, wenn auch dieß Letztere der Fall ist. Es scheint mir daher zu wenig gesagt, wenn man die Offenbarung bloß nützlich und nicht auch nothwendig nennen wollte; denn es ist doch, so viel wir nur absehen können, gewiß, daß unser Geschlecht, hätte sich Gott ihm nicht vielfältig zu erkennen gegeben, und es nicht an so manche vergessene Wahrheit zu rechter Zeit erinnert, in noch viel schlimmere Thorheiten und Laster, und eben darum in noch viel größere Uebel und Leiden verfallen wäre, als es wirklich geschehen ist. Der Umstand aber, daß dieses Alles doch nur mit dem Beisatze: \RWbet{so viel wir absehen}, gelte, kann keine Irrung veranlassen, wenn man so vorsichtig ist, sich einmal ausdrücklich darüber zu erklären; am Allerwenigsten könnte mich dieser Umstand bestimmen, das Wort Nothwendigkeit mit dem Worte Nützlichkeit zu vertauschen, weil auch bei diesem derselbe Beisatz stillschweigend hinzugedacht werden müßte.~\RWSeitenw{270}
\end{RWanm} 
\end{aufzb}
\end{aufza}

\RWpar{98}{Wie diese Nothwendigkeit bewiesen werden könne?}
Die zweckmäßigste Art nun, wie die so eben erklärte Nothwendigkeit einer Offenbarung erwiesen werden könne, scheint mir folgende zu seyn:
\begin{aufza}
\item Wir müssen für's Erste zeigen, \RWbet{wie unzulänglich die bloße Vemunftreligion in ihrer höchsten bisher bekannten Vollständigkeit auch für Gebildete und Aufgeklärte sey}, und wie viel Ursache daher selbst diese haben, eine göttliche Offenbarung zur Auflösung verschiedener Zweifel zu wünschen. Nun gibt es zwar, wie ich schon \RWparnr{64}\ erinnert, der natürlichen Religionen mehre; allein diejenige, die wir bei dieser Untersuchung zu Grunde legen müssen, wird die natürliche Religion des \RWbet{Menschengeschlechtes} selbst seyn. Denn einerseits kann die Lehren, die sie enthält, Niemand bezweifeln, ohne höchst ungläubig zu seyn; andererseits kann aber auch kein Bescheidener eine Lehre, die sie aus ihrem Inhalte ausschließt, die also noch viel Widerspruch erfährt, für völlig gewiß und ausgemacht ansehen, so entscheidend ihm auch die Gründe, die er für sie zu haben glaubt, bedünken mögten. Wenn es also wahr ist, daß die weisesten Menschen auch die bescheidensten sind; so ist diejenige Religion, an die sich der aufgeklärteste Theil der Menschheit halten müßte, wenn keine Offenbarung da wäre, keine andere, als die natürliche Religion der Menschheit selbst. Wollen wir also den Grad der Nothwendigkeit, den eine göttliche Offenbarung für diese Classe der Menschen hat, gehörig beurtheilen; so brauchen wir nur die Zweifel und Lücken aufzuzählen, die sich in der natürlichen Religion des menschlichen Geschlechtes finden.
\item Dann müssen wir aber für's Zweite zeigen, \RWbet{wie noch weit größer das Bedürfniß einer Offenbarung für das Menschengeschlecht im Ganzen sey}. Dieß läßt sich meines Erachtens auf eine doppelte Art darstellen:
\begin{aufzb}
\item \RWbet{einmal aus der Geschichte der religiösen Irrthümer}, in welche nicht nur die \RWbet{große Menge} der Menschen, sondern auch selbst die \RWbet{Gelehrten} so oft~\RWSeitenw{271}\ verfielen, als sie des Lichtes der Offenbarung, es sey nun mit oder ohne Schuld, entbehrten.
\item \RWbet{Sodann auch aus der Natur der Sache selbst}, indem man die Ursachen angibt, warum es kaum möglich sey, die Wahrheiten der natürlichen Religion in der gehörigen Reinheit und Vollständigkeit zu einer allgemeinen Anerkennung bei allen Menschen zu bringen, wenn keine Offenbarung dazu behülflich ist.\par
Nach diesem Plane werde ich nun in diesem Hauptstücke vorgehen.
\end{aufzb}
\end{aufza}
   
\RWabs{Erste Abtheilung}{Nothwendigkeit einer göttlichen Offenbarung selbst für die gebildetsten Menschen}
\RWpar{99}{Dunkelheiten der natürlichen Religion in ihrer Dogmatik, und zwar 1.~in der Lehre von Gottes Eigenschaften}
\begin{aufza}
\item  Auch der gebildetste Mensch besitzt, wie wir so eben sahen, keine anderen Wahrheiten, an die er sich mit voller Zuversicht halten könnte, als jene, welche die natürliche Religion des menschlichen Geschlechtes aufstellt. Diese gewährt ihm nun wohl eine hinlängliche Sicherheit darüber, daß es einen Gott, \dh\ ein Wesen von unbedingter Wirklichkeit gebe; allenfalls auch, daß dieses Wesen als der Inbegriff aller Vollkommenheiten, und sonach begabt mit einer unendlichen Macht, Weisheit und Heiligkeit gedacht werden müsse: allein der \RWbet{erste} Mangel in dieser Religion verräth sich, wenn wir uns die so eben erwähnten Eigenschaften Gottes noch etwas deutlicher vorstellen wollen. Weil nämlich alle göttlichen Kräfte und Beschaffenheiten eine gewisse \RWbet{Unendlichkeit} haben, das Unendliche aber von unserem endlichen Verstande schwer aufgefaßt werden kann; so wird der bescheidene Mensch um desto schüchterner in seinen Urtheilen über Gott, je reiflicher er erwägt, daß es das \RWbet{unendliche Wesen} sey, das~\RWSeitenw{272}\ er hier zu beurtheilen waget. Er fühlt, daß einem jeden Begriffe, den er von Gott sich macht, so vieles Bildliche anklebt, so Vieles, was wohl auf Menschen und andere endliche Wesen, aber nicht auf Gott angewendet werden kann. Bemüht er sich aber, dieß Alles wegzudenken; so verschwindet ihm darüber der Begriff selbst. -- Um nur ein einziges, aber recht merkwürdiges Zeugniß hierüber anzuführen, sey es dasjenige, welches uns Cicero \RWlat{(de natura Deorum l.\,3.\ c.\,21.)}\RWlit{}{Cicero1} erzählet: \RWlat{De Simonide quum quaesivisset tyrannus Hiero, quid aut quale sit Deus? deliberandi sibi \RWbet{unum diem} postulavit; quum idem ex eo postridie quaereret, \RWbet{biduum} petivit. Quum saepius duplicaret numerum dierum, admiransque Hiero requireret, cur ita faceret? Quia, quanto, inquit, diutius considero, tanto mihi videtur res obscurior.}
\item Daß dieser Mangel der natürlichen Religion allerdings nachtheilig sey, wird kaum Jemand in Abrede stellen; denn je bestimmter, sicherer und lebhafter unsere Begriffe von Gott sind, um desto wirksamer kann sich auch der Gedanke an Gott bei uns bezeigen.
\item Durch eine Offenbarung könnte, so viel wir einsehen, dem Uebel beträchtlich abgeholfen werden. Denn wer so sicher als Gott selbst kann uns belehren, wer er eigentlich sey, wie und unter welchen Bildern wir uns ihn vorstellen sollen, wenn diese Vorstellungen der Wahrheit am Getreuesten, und für uns selbst am Ersprießlichsten seyn sollen? Nur durch den Gebrauch solcher Bilder aber kann der Gedanke an Gott die nöthige Lebhaftigkeit, und eben darum auch die gehörige Wirksamkeit erhalten.
\begin{RWanm}
Daß man auch über den wissenschaftlichen \RWbet{Beweis} für das Daseyn Gottes bis jetzt gestritten habe, das ist nicht als ein Mangel der natürlichen Religion zu erachten. Denn weil man sich von der Wahrheit, daß ein Gott sey, hinlänglich überzeugen kann, auch ohne den streng wissenschaftlichen Beweis zu erfahren; so gehört der letztere nicht zu den Lehren, die einen Einfluß auf unsere Tugend und Glückseligkeit haben, also nicht zur Religion. Die Ungewißheit, in der sich die Weltweisen in dieser Rücksicht befinden, ist somit als ein bloßer Mangel der Philosophie, nicht aber als ein Mangel der Religion zu betrachten.~\RWSeitenw{273} 
\end{RWanm}
\end{aufza}
   
\RWpar{100}{2.~Ungewißheit der natürlichen Religion in der Lehre von der Unsterblichkeit der Seele}
\begin{aufza}
\item Ich habe schon \RWparnr{84}\ gezeigt, daß und warum der Lehrsatz von der Unsterblichkeit unserer Seele nicht als ein ganz ausgemachter Lehrsatz der natürlichen Religion des Menschengeschlechtes angesehen werden könne. Nun will ich diese Behauptung noch durch das Geständniß einiger einzelnen Weltweisen erhärten.
\begin{aufzb}
\item \RWbet{Sokrates}, dieser Ehrwürdigste der Weisen Griechenlands, der so berühmt geworden ist durch seine Lehre von der Unsterblichkeit, hatte doch selbst keine völlige Gewißheit von dieser Wahrheit; denn noch in seinen letzten Lebenstagen sprach er von diesem Gegenstande, als von einer Sache, die nur wahrscheinlich ist, und beschloß seine Rede mit den Worten: \anf{Entweder gehe ich nun in eine gänzliche Fühllosigkeit über, oder ich komme in die Gesellschaft der Götter.} In Xenophon's Apologie spricht er: \anf{Schon ist sie da, die Stunde des Scheidens; denn ich muß sterben, ihr aber werdet fortleben. Wer von uns das bessere Loos gefunden, ist Jedem unbekannt, nur Gott nicht.} Fast eben dieselben Worte führt uns auch Cicero \RWlat{Quaest.\ Tuscul.\ I.\ 14}\RWlit{}{Cicero8} an; und von sich selbst macht
\item eben dieser \RWbet{Cicero} das traurige Geständniß, er wäre so lange nur von der Unsterblichkeit der Seele überzeugt, so lange er das Buch des Plato (den Phädon) lese; er fange aber alsbald zu zweifeln an, wenn er das Buch aus den Händen lege, und über die Sache selbst zu denken anfange. \RWlat{Feci me hercule, et quidem saepius, sed nescio quomodo, dum lego, assentior, cum posui librum, et mecum ipse de animarum immortalitate ceopi cogitare, assensio illa omnis dilabitur}. Im Buche \RWlat{de senectute c.\,23.}\RWlit{}{Cicero7} äußert er folgende sehr nachahmungswerthe Gesinnung: \RWlat{Quodsi in hoc erro, quod animas hominum immortales credam, lubenter erro, nec mihi hunc errorem, quo delector, dum vivo, extorqueri volo; si mortuus -- ut quidam minuti philosophi censent -- nihil sentiam; non~\RWSeitenw{274}\ vereor, ne hunc errorem meum mortui philosophi irrideant}.
\item L.~A.~\RWbet{Seneca}, der schon am Ende der heidnischen Zeiten, im ersten christlichen Jahrhunderte, lebte, und also die Versuche aller heidnischen Weltweisen vor sich hatte, schreibt \RWlat{epist.\,102}\RWlit{450}{Seneca4a}: \erganf{\RWlat{Quomodo molestus est jucundum somnium videnti, qui excitat; aufert enim voluptatem etiamsi falsam, effectum tamen verae habentem: sic epistola tua mihi fecit injuriam. Revocavit enim me cogitationi aptae traditum, et iturum, si licuisset, ulterius. Juvabat de aeternitate animarum quaerere, imo me hercule credere: credebam enim facile opinionibus magnorum virorum rem gratissimam \RWbet{promittentium} magis, quam probantium.}}\RWuebers{%
\anf{Wie mich jemand stört, der mich aus einem angenehmen Traum aufweckt -- er nimmt mir ja ein Vergnügen, auch wenn es nicht echt ist, aber doch die Wirkung eines echten Vergnügens hat --, so hat mich auch Dein Brief geärgert; denn er hat mich in die Wirklichkeit zurückgerufen, als ich einem zu dem Traum passenden Gedanken nachgegangen war und diesen hatte fortsetzen wollen, falls es möglich gewesen wäre. Es machte mir Freude, mich mit der Frage der Unsterblichkeit der Seele zu befassen, nein, vielmehr, beim Herkules, an diese zu glauben; ich schloss mich nämlich ohne weiteres den Auffassungen großer Männer an, die eine sehr angenehme Angelegenheit eher versprachen als bewiesen} (\lit{Seneca4b}, 389).}
\end{aufzb}
\item Wer nun die großen Vortheile kennt, die ein fester Glaube an Unsterblichkeit für unsere Tugend sowohl als auch für unsere Glückseligkeit gewährt (sie werden im dritten Haupttheile \RWparnr{170}\ umständlicher auseinandergesetzt), der wird begreifen, daß sich wohl in der ganzen natürlichen Religion kein wichtigerer Mangel findet als der, daß sie uns von der endlosen Fortdauer unsers Geistes nicht völlig gewiß machen kann.
\item Daß eine Offenbarung dieses vermöchte, bedarf keines Beweises.
\end{aufza}

\RWpar{101}{3.~Ungewißheit der natürlichen Religion über die Frage von der Vergebung der Sünden}
Daß jede Sünde, \dh\ jede freiwillige Uebertretung des Sittengesetzes, eine gewisse \RWbet{Bestrafung} verdiene, und unter Voraussetzung eines moralischen Weltregierers auch wirklich finde, ist keinem Zweifel unterworfen. (\RWparnr{81}) Es frägt sich aber, ob es so gar keine Mittel gebe, durch welche der Sünder die einmal verdiente Strafe wieder von sich abwenden, und die Heiligkeit Gottes dahin bestimmen könne, ihn nicht mehr zu strafen, oder, (wie man zu sagen pflegt) ihm seine Sünden zu \RWbet{vergeben}? Daß dieses unmöglich sey, so lange der Sünder sich nicht \RWbet{bessert}, ist abermals außer Zweifel; die Frage ist~\RWSeitenw{275}\ aber, ob diese Besserung allein schon hinreiche, ihn von der Strafe, die er für seine vorhergehenden Sünden verdient hätte, zu befreien; oder ob sonst noch etwas Anderes geschehen, noch irgend eine besondere \RWbet{Genugthuung} geleistet werden müsse; oder ob endlich schlechterdings nichts vorhanden sey, was die einmal verdiente Strafe wieder aufheben könnte? Diese Frage nun ist es, die man mit einem Worte \RWbet{die Frage von der Sündenvergebung}, oder mit einem minder zweckmäßigen Ausdrucke \RWbet{die Frage von der Genugthuung} nennt.\par
Die bloße, sich selbst überlassene Vernunft vermag über diese Frage nicht nur mit keiner Gewißheit, sondern nicht einmal mit \RWbet{Wahrscheinlichkeit} zu entscheiden; oder genauer zu reden: es kommt dem Menschen so vor, als dürfte die Besserung allein nicht hinreichend seyn zur Vergebung; -- aber indem er sich nach einem bestimmten Genugthuungsmittel umsieht, entdeckt er bald seine gänzliche Unvermögenheit, ein taugliches anzugeben, und fühlt sich daher genöthigt, es bei der Besserung allein bewenden zu lassen.

\RWpar{102}{Beweis dieser Ungewißheit aus den Gebräuchen, die sich bei allen Völkern finden}
Bei allen Völkern der Erde finden wir den Gebrauch gewisser \RWbet{Genugthuungs-} oder \RWbet{Versöhnungsmittel}, durch welche man die Schuld vergangener Verbrechen von sich abzuwälzen suchte. Aus der \RWbet{Allgemeinheit} dieses Gebrauches möchte man schließen, es sey das Urtheil des gemeinen Menschenverstandes, daß Besserung allein zur Tilgung der Schuld vergangener Verbrechen nicht hinreiche. Aus der \RWbet{Verschiedenheit} dieser Versöhnungsmittel aber, und aus der einleuchtenden \RWbet{Untauglichkeit} der meisten können wir lernen, daß es der bloßen Vernunft nicht möglich sey, ein taugliches auszudenken. Denn bei dem Einen Volke sah man Verrichtung gewisser Gebete nach vorgeschriebenen Formeln, bei einem andern das Fasten, bei einem dritten das Baden in diesem oder jenem Flusse, bei den meisten gewisse Opfer, Schlacht\RWSeitenw{276}opfer, die nur unschuldigen Thieren das Leben kosteten, nicht selten sogar Menschenopfer, als ein wirksames Mittel an, die Schuld begangener Sünden zu tilgen. Daß nun solche Genugthuungsmittel Verwerfung verdienen, erkannte selbst der leichtsinnige Ovid:\RWlit{}{Ovidius2} \par
\RWlat{Ah nimium faciles, qui tristia crimina caedis}\par
\RWlat{Fluminea tolli posse putetis aqua!}\par
\mbox{}\hfill \RWlat{Fastor.\ l.\,2\ \Ahat{v.\,45--46}{v.\,48}}.

\RWpar{103}{Fernerer Beweis dieser Ungewißheit aus dem Streite der Weltweisen hierüber}
\begin{aufza}
\item Die meisten \RWbet{heidnischen Weltweisen}, und unter den \RWbet{christlichen} jene, die der Offenbarung im Herzen abgeneigt waren, behaupteten, \RWbet{daß Besserung allein schon hinreichend wäre, um die Vergebung der Sünden zu gewinnen}. Die Gründe, die sie für ihre Behauptung anführten, waren ohngefähr folgende:
\begin{aufzb}
\item Der Zweck aller Strafen, sagten sie, sey ja kein anderer als Besserung; ist also diese erfolgt, so müssen auch die Strafen wegbleiben.
\item Die Menschen haben, wenn sie gesündigt, kein anderes Mittel in den Händen, als das der \RWbet{Besserung;} denn Fasten, Bäder \usw\ können doch offenbar unsere Strafwürdigkeit nicht mindern. Sollte also Besserung nicht hinreichen; so müßten, weil doch alle Menschen mehr oder weniger gesündigt haben, auch Alle Strafe erfahren, also auch Alle unglücklich werden, welches der Güte Gottes widerspricht.
\item Wer sich gebessert hat, ist nunmehr \RWbet{sittlich gut}, der sittlich Gute darf aber keine Strafen erfahren; also kann die Strafe, wenn sie nicht früher an ihm vollzogen worden ist, nun nicht mehr nachgetragen werden.
\item Endlich wenn auch der Gebesserte noch Strafe für seine vorigen Sünden zu befürchten hätte; so würde die Lust und der Muth zur Besserung verschwinden, und Verzweiflung eintreten.
\end{aufzb}
\item Andere, namentlich die \RWbet{meisten christlichen Weltweisen} vertheidigten dagegen die Behauptung, daß Besserung allein~\RWSeitenw{277}\ noch nicht hinreichend sey, sondern daß es eines eigenen Genugthuungsmittels bedürfe. Die Gründe der Ersteren entkräfteten sie durch folgende Gegenbemerkungen:
\begin{aufzb}
\item Es ist ein Irrthum, daß der Zweck aller Strafe die bloße Besserung, und zwar nur Besserung desjenigen, an dem sie vollzogen wird, sey. Man kann auch strafen, um Andere, die sich zu einem ähnlichen Verbrechen versucht fühlen, durch das Beispiel des Bestraften abzuschrecken, \udgl\,m. Der vollständige Zweck der Strafe besteht überhaupt in der Entfernung alles desjenigen Bösen, und in der Herbeiführung alles desjenigen Guten, das sich durch eine schickliche Einrichtung, die man der Strafe gibt, entfernen und herbeiführen läßt. Es ist auch also falsch, daß die Strafe jedesmal aufgehoben werden müsse, sobald die Besserung eintritt; denn wenn nun auch jener Eine Zweck wegfällt, so können noch andere bleiben.
\item Daß die sich selbst überlassene Vernunft kein taugliches Genugthuungsmittel anzugeben wisse, ist allerdings wahr; daraus folgt aber nur, daß Menschen, die keine Offenbarung haben, auch nicht zur Leistung einer Genugthuung verbunden seyn können; keineswegs aber folgt, weder
\begin{aufzc}
\item daß ihnen deßhalb die Strafe, die sie für ihre früheren Sünden verdient, erlassen werden müsse; noch
\item daß auch \RWbet{Gott selbst} kein solches Genugthuungsmittel in einer Offenbarung bekannt machen könne; noch endlich
\item daß alle diejenigen Menschen, denen dieß Genugthuungsmittel ohne Verschulden unbekannt bleibt, \RWbet{völlig} und \RWbet{ewig} unglücklich werden müßten; denn wäre es nicht möglich, daß ihre Strafe nur eine endliche Dauer hat, oder sie nur eines Theils ihrer Glückseligkeit beraubt?
\end{aufzc}
\item Wer \RWbet{sittlich gut} ist, und es auch \RWbet{vorher immer war}, verdient freilich keine Strafe. Allein daß auch derjenige, der \RWbet{einmal böse war}, und erst seit Kurzem sittlich-gut geworden ist, nun keine Strafe mehr verdiene, ist eben zu beweisen.
\item Es ist nicht nothwendig, daß sich gerade Verzweiflung einstelle, weil ja der Sünder durch Besserung wenigstens~\RWSeitenw{278}\ eine Verminderung seiner Strafe erwarten darf; auch könnte der Gebesserte hoffen, daß ihm Gott ein Mittel zur Genugthuung vielleicht noch bekannt machen werde.
\end{aufzb}
Für ihre eigene Meinung führten diese Weltweisen noch folgende Gründe an:
\begin{aufzb}
\item Einen, der auch mir richtig dünkt. Es würde das Ansehen des Sittengesetzes leiden, wenn jede Uebertretung desselben durch eine nachherige Besserung gleich wieder straflos gemacht würde. Der Leichtsinn der Menschen würde sich überreden, daß es sonach um die Verletzung der Tugend keine so große Sache sey. Bessere man sich nur am Schlusse des Lebens, so könne man allen Strafen entgehen, und werde demjenigen, der immer tugendhaft gelebt hat, gleich geachtet. Dieses erkannte schon Cicero, wie wir aus folgender uns von Lactantius (\RWlat{de vero cultu c.\,24}.)\RWlit{}{Lactantius1} aufbewahrten Stelle aus dem verloren gegangenen 3.~Buche \RWlat{Academ.} ersehen: \RWlat{Quodsi liceret, ut iis, qui in itinere deerravissent, sic vitam deviam secutos corrigere errorem poenitendo, felicior esset emendatio temeritatis.}
\item Andere stellten die Sache lieber so vor. Sie nahmen es erstlich als einen Grundsatz, den man ohne Beweis zugeben müsse, an, daß jede Verletzung des Sittengesetzes Eines von Beidem, entweder Strafe oder Genugthuung fordere. Unter Genugthuung verstanden sie eine Handlung, durch welche mehr geleistet wird, als man zu leisten schuldig gewesen wäre, wenn man nie gesündiget hätte; und sie behaupteten nun, daß diese Genugthuung nie in der bloßen Besserung selbst bestehen könne. Denn wenn wir uns bessern, und dann auch noch so tugendhaft leben; so thun wir (sagten sie) doch immer nicht mehr, als wir zu thun verpflichtet gewesen wären, auch wenn wir uns vorhin nie versündiget hätten. So fordert denn also die Vernunft ein Genugthuungsmittel, und weiß doch selbst keines anzugeben. Es ist uns folglich eine Offenbarung nöthig, damit sie uns mit einem solchen bekannt mache.
\begin{RWanm}
Denjenigen, welche die Sache so darstellten, machte man den Einwurf, daß doch in Rücksicht auf menschliche Strafen~\RWSeitenw{279}\ Genugthuung oft wirklich Statt finde, und geleistet werde; und fragte, warum sonach nur in Beziehung auf Gott keine möglich seyn sollte? -- Hierauf erwiderten Jene, Menschen wären nicht berechtiget, unter Bedingung der Strafe von uns zu fordern, daß wir zu der Beförderung ihres Wohles allemal so viel, als wir nur können, beitragen; sondern es gebe nur bestimmte Leistungen, welche sie unter Strafe von uns fordern könnten. Haben wir also eine Leistung dieser Art einmal unterlassen, und uns hiedurch vor Menschen straffällig gemacht; so können wir dadurch, daß wir ein anderes Mal mehr thun, als sie von uns zu fordern berechtiget sind, die verdiente Strafe wieder von uns abwenden, \dh\ Genugthuung leisten. So etwas fände aber bei Gott nicht Statt. Gott nämlich muß, vermöge seiner Heiligkeit, unter Bedingung der Strafe fordern, daß wir des Guten bei jeder Gelegenheit so viel leisten, als wir nur immer vermögen. Haben wir also irgendwo weniger geleistet; so sind wir vor ihm schon straffällig geworden, und wir können jetzt nicht, wie bei Menschen, dadurch Genugthuung leisten, daß wir ein andermal mehr thun, als er zu fordern berechtiget ist. -- Meinem Dafürhalten nach ist aber diese Darstellung der Sache nicht eben die richtigste. Die Vernunft fordert, wenn wir genau reden wollen, kein Genugthuungsmittel, sondern sie fürchtet nur mit mehr oder weniger Wahrscheinlichkeit, daß der gebesserte Sünder auch nach der Besserung noch Strafe verdienen dürfe; und nur, um von dieser Furcht befreit zu werden, ist eine Offenbarung über diesen Gegenstand erwünschlich.
\end{RWanm}
\end{aufzb}
\item Auch \RWbet{Kant} gab zu, daß Besserung, an und für sich betrachtet, noch nicht hinreichend sey, um die Schuld vorhergegangener Gesetzesübertretungen zu tilgen. Allein weil dieser Weltweise gleichwohl die Nothwendigkeit einer Offenbarung, die hieraus folgen würde, nicht eingestehen wollte: so behauptete er, die menschliche Vernunft wäre für sich selbst im Stande, uns ein Genugthuungsmittel zu lehren. Dieses bestehe nämlich in jenen Beschwerlichkeiten, welche der Sünder beim Anfange seiner Besserung zu überwinden hat. Jede Sünde, sagte er, macht, daß uns der sittliche Lebenswandel in der Folge beschwerlicher fällt, als er es gewesen wäre, wenn wir nie gesündiget hätten. Diese Beschwerlichkeiten sind also Strafen der Sünden; allein sie treffen den Menschen zu einer Zeit, da er bereits gute Gesinnungen hegt, und also~\RWSeitenw{280}\ keine Strafe mehr verdient; daher wird die geduldige Ertragung derselben ein überschüßiges Verdienst, eine Genugthuung für die vorigen Sünden. -- Nach \RWbet{Kant's} Meinung sind die christlichen Lehren vom Sündenfalle und von der Erlösung durch die Leiden des Sohnes Gottes in der That nichts Anderes, als symbolische Darstellungen dieser so eben vorgetragenen abstracten Wahrheiten.
\end{aufza}\par
Hiegegen bemerke ich aber:
\begin{aufzb}
\item Dem bloßen gemeinen Menschenverstande wird diese Entscheidung der Sache gewiß \RWbet{keine Befriedigung} gewähren. Er wird sie anhören, vielleicht den Scharfsinn, der darin liegt, einiger Maßen bewundern; aber noch immer die Frage aufwerfen, ob dieß auch so gewiß sey, ob sich Gott selbst erklärt habe, daß er jene geduldige Ertragung der Beschwerlichkeiten, die mit der Besserung verbunden sind, für eine vollständige Genugthuung ansehen wolle?
\item Und wirklich, wenn wir die Sache näher betrachten, so zeigt sich, daß \RWbet{Kant} den rechten Gesichtspunct bei dieser Frage verfehlt habe. Die Begriffe von \RWbet{Genugthuung}, von \RWbet{überschüßigem Verdienste} \usw\ sind von bloß \RWbet{menschlichen Gerichten} entlehnt, und können in dem hier angegebenen Sinne auf Gott nicht angewendet werden. Von Gott wissen wir nur, daß er stets so handeln müsse, wie es das Wohl des Ganzen am Meisten befördert, und daraus folgern wir, daß er auf jede gute Handlung eine \RWbet{Belohnung}, auf jede böse eine \RWbet{Strafe} setzen müsse; indem wir deutlich einsehen, daß dieses nothwendig sey, um die Geschöpfe zur Tugend anzutreiben und vom Laster abzuhalten, und eben hiedurch das allgemeine Wohl zu befördern. Wenn also die Frage aufgeworfen wird, ob die Beschwerlichkeiten, die mit der Besserung verbunden sind, die Nachlassung aller Strafen bewirken können oder nicht: so muß man dieß nicht durch Untersuchungen von einer solchen Art, wie sie hier \RWbet{Kant} angestellt hatte, entscheiden wollen; sondern es hängt dieses lediglich von der andern Frage ab, ob diese Einrichtung dem Wohle des~\RWSeitenw{281}\ Ganzen zuträglich wäre oder nicht? -- Nun ist es aber offenbar, daß die in Rede stehende Einrichtung eine wohlthätige sowohl als auch eine nachtheilige Folge hätte. Die wohlthätige wäre, daß alle Menschen, die sich gebessert haben, nun keine weitere Strafen mehr zu befürchten hätten; die nachtheilige dagegen, daß die Hoffnung der Straflosigkeit, die mit der Besserung eintritt, Viele zu desto mehren Sünden verleiten würde. Wer mag nun sagen, welche von diesen beiden Folgen die andere überwiegen würde? Wer sieht nicht ein, daß sich kein menschlicher, ja überhaupt kein endlicher Verstand eine bestimmte Beantwortung dieser Frage zutrauen könne?
\item Endlich ist noch zu bemerken, daß sich in dieser \RWbet{Kant'schen} Theorie ein innerer Widerspruch befinde. Denn die geduldige Ertragung jener Beschwerlichkeiten, die mit der Besserung verknüpft sind, kann man nicht als ein \RWbet{überschüßiges} Verdienst betrachten, wenn anders, -- wie es \RWbet{Kant} selbst behauptet -- bloß pflichtmäßige Handlungen nie ein überschüßiges Verdienst gewähren. Geduldige Ertragung aller Leiden ist doch für jeden, auch für denjenigen Menschen, der vorher nie gesündiget hatte, Pflicht, um wie viel mehr für den, der sich diese Leiden durch seine Sünden selbst zugezogen hat.
\end{aufzb}
\begin{aufza}\setcounter{enumi}{3}
\item Aus allem diesen ersieht man, wie wenig die sich selbst überlassene Vernunft bis auf den heutigen Tag im Stande sey, die Frage von der Vergebung der Sünden auf eine sichere Art zu entscheiden.
\end{aufza}

\RWpar{104}{Wichtigkeit dieses Zweifels}
Die Unentschiedenheit der Frage von der Vergebung der Sünden hat einen nachtheiligen Einfluß auf unsere Tugend sowohl, als auch auf unsere Glückseligkeit.
\begin{aufza}
\item Wenn es dem Menschen an \RWbet{hinlänglichen} Gründen zur Entscheidung über irgend einen Gegenstand fehlt; so pflegt er sich häufig aus bloß \RWbet{subjectiven} Gründen, aus der so eben herrschenden \RWbet{Gemüthsstimmung} bei ihm, \udgl\  bald auf die Eine, bald auf die andere Seite hinzu\RWSeitenw{282}neigen. Ist sein Gemüth so eben in einer heitern Stimmung, so beurtheilt er Alles auf eine solche Art, wie es der Sinnlichkeit am Meisten zusagt; es dünkt ihm nun wahrscheinlicher, daß bloße Besserung hinreichend sey. Wird er wohl gar zu einer Sünde versucht, so ergreift er diese Meinung um so begieriger; und fügt noch den weitern Trugschluß hinzu, daß er, wenn durch Besserung die Strafe getilgt wird, von seiner Sünde gar keinen Schaden habe, wofern er sich nur in Zukunft bessert; und daß er dieß vermögen werde, darüber läßt er sich jetzt keinen Zweifel beikommen und entschließt sich so zur Sünde. -- Im Gegentheile aber wenn Unglücksfälle, Krankheiten oder der heranrückende Tod die Seele in eine traurige Stimmung versetzen: dann kehrt auch jene Frage wieder, und mit je weniger Gründen er sie zu anderer Zeit bejahend entschieden hatte, um desto fürchterlich glaubwürdiger kommt ihm nun ihre verneinende Entscheidung vor. Die Besserung allein, spricht er nun zu sich selbst, ist doch nicht hinreichend; denn sie macht ja das Böse, das ich verübte, nicht ungeschehen. Ich kann in Zukunft immer nicht mehr thun; bei meinen geschwächten Kräften vielleicht nicht einmal so viel, als ich schuldig gewesen wäre, auch wenn ich ehedem nicht gesündiget hätte! Durch meine Sünden habe ich die Majestät Gottes, des Unendlichen, beleidigt; wie dieses ein Verbrechen von unendlicher Größe ist, so dürfte es wohl auch eine Strafe von unendlicher Größe fordern; und wie der Schaden, den ich durch meine Sünden angerichtet habe, in alle Ewigkeit fortwährt, so dürfte auch die Strafe, die meiner wartet, in alle Ewigkeit dauern! Diese und andere dergleichen schreckliche Gedanken können den Sünder selbst in Verzweiflung stürzen. Und ob ich auch eben nicht behaupten mag, daß diese nachtheiligen Folgen bei Jedermann in einem gleichen Grade zu befürchten wären; so dürfte doch Niemand ganz sicher seyn, daß er nicht einige derselben an sich erfahren werde. Zur Sicherung seiner eigenen Tugend muß also Jeder eine bestimmte Erklärung über jene Fragen durch Gottes Offenbarung wünschen.
\item Da aber alle Menschen wenigstens aus ihren frühern Jahren sich gewisser, bald mehr bald minder wichtiger Vergehun\RWSeitenw{283}gen bewußt sind: so kann die Frage, wie es Gott einst mit diesen Vergehungen halten werde, für keines Einzigen Ruhe und Glückseligkeit gleichgültig seyn. Nur ein sehr leichtsinniger Mensch könnte sich gar nicht um diese Frage bekümmern. Je ernsthafter dagegen Jemand ist, je sorgsamer bedacht für sein nicht irdisches bloß, sondern auch jenseitiges und ewig währendes Heil, je wichtiger vielleicht auch die Vergehungen sind, die er in frühern Jahren sich hatte zur Schuld kommen lassen, oder je zarter nur sein eigenes Gewissen ist, je näher endlich und immer näher er jenen Augenblick heranrücken sieht, der uns die Pforten der Ewigkeit aufschließt: um desto sehnlicher wird er eine erfreuliche Beantwortung dieser Frage durch Gottes Offenbarung wünschen.
\end{aufza}
   
\RWpar{105}{4.~Ungewißheit der natürlichen Religion in der Frage vom Ursprunge und Zwecke des Uebels}
Bei so vielem Guten in dieser Welt, durch dessen Betrachtung das Daseyn eines höchst weisen und höchst gütigen Urhebers derselben die völligste Bestätigung (\RWparnr{83}) erhält, gibt es auch \RWbet{manche Einrichtungen und Ereignisse in ihr, welche mit Gottes Vollkommenheit in einem scheinbaren Widerspruche stehen.}\par
Hieher gehören
\begin{aufzb}
\item so viele \RWbet{Schmerzen} und \RWbet{Leiden}, denen die ganze lebendige Schöpfung, besonder aber
\item \RWbet{wir Menschen} selbst ausgesetzt sind;
\item der \RWbet{starke Reiz zum Bösen}, den wir in unsern Trieben und Neigungen finden, aus dem so viele Versuchungen zum Bösen, so viele Unruhe, und
\item so viele wirkliche \RWbet{Verbrechen} und \RWbet{Laster} entspringen; in Betreff deren Gott nicht nur den innern Entschluß, sondern auch dessen wirkliche Ausführung zuläßt.
\item Das arge \RWbet{Mißverhältniß}, das zwischen Tugend und Glückseligkeit obwaltet. Der Tugendhafte ist so oft unglücklich auf Erden, indem das Schicksal sowohl als böse Menschen ihn oft in Vereinigung verfolgen, während das Laster in ungestörtem Glücke fortlebt.~\RWSeitenw{284}
\item Die große \RWbet{Ungleichheit} in der Vertheilung der \RWbet{Gelegenheiten} zu unserer \RWbet{sittlichen} Ausbildung sowohl, als auch zum \RWbet{Genusse der Glücksgüter} dieser Welt. -- 
\end{aufzb}
Unwiderstehlich dringt sich bei der Betrachtung all dieser Einrichtungen und Ereignisse die Frage auf, woher dieß Alles komme, und wie es sich mit Gottes Vollkommenheiten vereinigen lasse? Dieß nennen wir denn \RWbet{die Frage von dem Ursprunge und Zwecke des Uebels}, von der ich behaupte, daß sie durch bloße Vernunft nicht bis zur völligen Befriedigung beantwortet werden könne.


\RWpar{106}{Beweis dieser Ungewißheit aus den verschiedenen Verirrungen, auf welche die menschliche Vernunft bei dieser Untersuchung gerathen ist}
Daß die sich selbst überlassene Vernunft die Frage vom Ursprunge und Zwecke des Uebels nicht befriedigend zu beantworten vermöge, beweisen für's Erste schon die mannigfaltigen \RWbet{Verirrungen}, auf welche die Weltweisen eben durch das Bestreben, eine genügende Antwort derselben zu finden, gerathen sind.
\begin{aufza}
\item Einige Weltweise, weil sie glaubten, das Daseyn so vieler Uebel lasse sich mit einer unendlichen Macht, Weisheit und Güte gar nicht vereinigen, verirrten sich so weit, daß sie aus diesem Grunde lieber das Daseyn Gottes selbst in Zweifel zogen; oder zwar einen Gott annahmen, aber doch die Entstehung des Weltalls nicht von ihm selbst herleiteten, sondern dasselbe entweder für das Werk einer blinden Naturnothwendigkeit (Fatum), oder (was eigentlich dasselbe mit andern Worten war) für das Werk eines bloßen Zufalls, eines ohngefähren Zusammenstoßes gewisser von Ewigkeit her in Bewegung befindlicher Atome \udgl\  erklärten.
\item Andere wieder glaubten, die Schwierigkeit nicht anders lösen zu können, als daß sie zwei letzte Grundwesen aller Dinge (zwei Gottheiten), ein gutes und ein böses, annahmen, und wie das Gute in der Welt von jenem, so rühre dagegen das Böse in ihr von diesem her. Solcher Meinung waren \zB\ die Perser, die Manichäer \uA ~\RWSeitenw{285}
\item Noch Andere nahmen zwar nur einen einzigen, höchst weisen und heiligen Gott an, räumten ihm auch einen Einfluß auf diese Welt ein, jedoch nur so, daß er dieselbe aus einem von Ewigkeit her vorhandenen, von ihm selbst unabhängigen Stoffe (Materie) \RWbet{gebildet habe}. Aus der natürlichen Unvollkommenheit dieser Materie, aus ihrer Trägheit, \udgl\  erklärten sie nun die Unvollkommenheiten in dieser Welt. In diesem Irrthume waren selbst die besten heidnischen Weltweisen, \zB\ die Stoiker, befangen.
\item \RWbet{Plato} stellte zur Erklärung der vielen Leiden, welche der Mensch von seiner frühesten Kindheit an auf Erden auszustehen hat, und zur Erklärung jenes so starken Hanges zum Bösen in uns die äußerst sonderbare Behauptung auf, wir Alle hätten schon vor diesem Leben irgendwo gelebt, hätten gesündiget, und wären zur Strafe dafür in diesen gegenwärtigen Leib, gleichsam in einen Kerker eingesperrt worden.
\item \RWbet{Cicero} sagt, dieß wäre die Meinung mehrer alten Weisen gewesen, die auch für ihn etwas Wahres zu haben scheine. \RWlat{Ex quibus humanae vitae erroribus et aerumnis fit, ut interdum veteres illi, sive Vates, sive in sacris initiisque tradendis divinae mentis Interpretes, qui nos ob aliqua scelera suscepta in vita superiori, poenarum luendarum causa esse natos dixerunt, aliquid vidisse videantur.} (\RWlat{Aug. contra Julianum}\RWlit{}{Augustinus6} l.\,4.\ c.\,15., wobei er Cicero citiert.) Derselbe klagte an einer andern Stelle, welche uns gleichfalls Augustinus aufbewahrt hat, die Natur selbst sehr hart an über die wahrhaft stiefmütterliche Behandlung des Menschen, und weiß sie darüber nicht zu rechtfertigen. \RWlat{Homo non ut a matre, sed ut a noverca natura editus in vitam, corpore nudo et fragili et infirmo, animo autem anxio ad molestias, humili ad timores, molli ad labores, prono ad libidines, in quo tamen inesset tamquam obrutus quidam divinus ignis ingenii et mentis.}\RWlit{}{Cicero6}
\item  Noch Andere sahen sich zum Wenigsten genöthigt, die Vorsehung Gottes zu läugnen; und gaben sie auch eine \RWbet{allgemeine} zu, so läugneten sie doch die \RWbet{besondere}. So~\RWSeitenw{286}\ heißt es \zB\ selbst bei \RWbet{Cicero} in dem Buche \RWlat{de natura deorum:\RWlit{}{Cicero1} Magna dii curant, parva negligunt.}
\end{aufza}

\RWpar{107}{Beweis dieser Ungewißheit aus jenen Unvollkommenheiten, welche selbst die gelungensten Theodiceen haben}
Nicht alle Weltweisen sind bei Erörterung der Frage vom Ursprunge und Zwecke des Uebels in so grobe Irrthümer verfallen, wie wir jetzt eben angeführt haben. Einige, besonders \RWbet{neuere}, denen die Offenbarung selbst hiebei zu einer Leuchte diente, waren glücklicher in ihren \RWbet{Theodiceen} oder Versuchen, Gott wegen des vielen Uebels in der Welt zu rechtfertigen, \zB\ \RWbet{Leibnitz, Bilfinger, Clarke, Villaume, Teller, Werdermann, Wagner} \uA\ Daß aber auch die gelungensten dieser Versuche noch keine völlige Befriedigung gewähren, wird sich am Besten zeigen, wenn ich das Gründlichste, was meiner Meinung nach vorgebracht worden ist, in einer kurzen Uebersicht mittheile; zugleich aber auch in beigefügten Bemerkungen nachweise, was diese Rechtfertigungen noch immer zu wünschen übrig lassen.
\begin{aufza}
\item Für's Erste hat man erinnert, daß wir bei keiner einzigen Einrichtung oder Begebenheit in der Welt ihren Zusammenhang mit dem Ganzen vollständig übersehen, und alle ihre Folgen kennen. Daraus ergibt sich denn, daß wenn irgend eine Einrichtung oder Begebenheit uns auch noch so \RWbet{nachtheilig} für das Ganze, dabei auch noch so \RWbet{entbehrlich} scheinen sollte, wir gleichwohl nie berechtiget sind, sie wirklich für das, was sie scheint, zu erklären.\par
\end{aufza}
Diese Erinnerung hat ihre völlige Richtigkeit; indessen bleibt es doch immer wahr, daß solche Einrichtungen und Ereignisse in der Welt, die nach demjenigen Theile ihrer Folgen, die unser Auge zu überschauen vermag, verderblich sind, uns eben deßhalb beunruhigen, und daß die Kraft des Beweises, den wir aus der bemerkten Zweckmäßigkeit anderer Welteinrichtungen für das Daseyn einer göttlichen Fürsehung herleiten, durch sie geschwächt werde.
\begin{aufza}\setcounter{enumi}{1}
\item Doch in Betreff der meisten Einrichtungen und Ereignisse, worüber die Menschen sich beklagen, reicht ja (so sagt man weiter) selbst unser beschränkter Verstand hin, gewisse Vortheile derselben einzusehen, oder doch sonst etwas zu entdecken, was uns zufrieden mit ihnen machen kann.
\begin{aufzb}
\item  \RWbet{In Rücksicht der vielen Schmerzen und Leiden, denen die ganze lebendige Schöpfung ausgesetzt ist}, läßt sich erinnern,
\begin{aufzc}
\item  daß wir die \RWbet{Empfindlichkeit} dieser Schmerzen in unserer Einbildung leicht \RWbet{übertreiben} dürften. Ge\RWSeitenw{287}wisse Zuckungen, Convulsionen \udgl\  haben zwar ein sehr fürchterliches Ansehen, aber der Schmerz, den das Geschöpf dabei empfindet, dürfte nur sehr gering seyn.
\item  Durchaus \RWbet{allen Schmerz} zu vermeiden, wäre \RWbet{nicht einmal möglich}, wenn zugleich nicht auch alles \RWbet{Lustgefühl} aus der Welt verbannt werden sollte. Denn jedes endliche Wesen, wenn es Empfindung hat, muß nebst der Lust zugleich auch Schmerz empfinden, weil Schmerz nichts Anderes ist, als das Gefühl begrenzter Lust, und jede Lust eines endlichen Wesens eben darum, weil es nur endlich ist, begrenzt seyn muß. Empfindende Wesen aber mußte Gott schaffen, weil sonst gar keine Glückseligkeit in der Welt Statt finden könnte.
\item  Endlich scheint es, daß alles Schmerzgefühl meistens den Nutzen hat, das Empfindungsvermögen selbst zu \RWbet{erhöhen}, und so das leidende Wesen in der Folge auch eines höhern Gefühles der Lust empfänglich zu machen. Aus eigener Erfahrung wissen wir nämlich, daß Schmerzen insgemein die Lebhaftigkeit unsers Bewußtseyns erhöhen, während \RWbet{angenehme Gefühle} etwas \RWbet{Einschläferndes} haben. Diesen Zweck scheinen insonderheit jene Qualen und Martern zu haben, die gewissen Thieren noch in dem Augenblicke ihres Sterbens zugefügt werden, und dieß zwar öfters durch andere Thiere, denen ein eigener Instinct dazu gegeben ist, das Thier, das sie zu ihrer Nahrung verzehren wollen, erst so zu quälen. Es scheint, das Gefühlsvermögen des sterbenden Thieres selbst werde gerade durch diesen letzten Todeskampf gesteigert zu jenem höhern Grade der Vollkommenheit, dessen dasselbe auf seiner künftigen Stufe des Daseyns bedarf.
\end{aufzc}
Ich erinnere nur, daß diese Betrachtungen uns zwar wohl darüber beruhigen können, daß Schmerzen \RWbet{überhaupt} vorhanden sind; aber sie lassen noch immer den Zweifel übrig, ob auch so große und so viele nöthig wären? --
\item  \RWbet{In Rücksicht der Leiden, die der Mensch selbst erfährt}, finden nebst den \RWlat{sub a} angeführten Beruhigungsgründen noch folgende neue Statt:~\RWSeitenw{288}
\begin{aufzc}
\item  Gewisse Schmerzen und Leiden sind zur Entwicklung der Geisteskräfte für jeden Menschen nöthig; denn nicht die angenehmen, sondern die schmerzlichen Gefühle sind es, die schon das \RWbet{Kind} veranlassen, sich nach der Ursache derselben umzusehen; und die Erfahrung lehrt uns, daß Menschen, die in ihren früheren Lebensjahren manche empfindliche Krankheiten bestanden, geistvoller sind als andere.
\item  Die Leiden und Unbequemlichkeiten, denen das menschliche Geschlecht ausgesetzt ist, haben die meiste Veranlassung zu jenen herrlichen Erfindungen gegeben, durch welche unser irdische Zustand so sehr vervollkommnet wird.
\item  Leiden sind auch Veranlassung zu vielen Tugenden. So sind \zB\ die Tugenden der Geduld, der Standhaftigkeit, des Muthes, des Mitleids, der Wohlthätigkeit, der großmüthigen Vergebung erlittener Unbilden, ohne Leiden gar nicht gedenkbar.
\item  Die vielen Leiden dieser Erde ziehen unsern Sinn vom Irdischen ab, und flößen uns die Hoffnung und den gemäßigten Wunsch nach einem andern Leben ein, was dann zum Wenigsten den Nutzen hat, daß wir uns bei der Herannahung unseres eigenen, oder des Todes unserer Anverwandten leichter beruhigen können.
\item  Endlich ist auch ein großer Theil unserer Leiden das Werk unserer eigenen Thorheiten und Laster, und Gott erscheint dabei nicht als \RWbet{bestimmender Grund}.
\end{aufzc}
Diese Betrachtungen erklären, warum die Leiden der Menschen zahlreicher sind, als jene der Thiere; allein, daß es gerade so \RWbet{viele und so große seyn müssen}, wird immer nicht bewiesen.
\item \RWbet{In Hinsicht des starken Reizes zum Bösen} läßt sich Nachstehendes erinnern:
\begin{aufzc}
\item  Daß ein \RWbet{gewisser Streit} zwischen \RWbet{Vernunft} und \RWbet{Sinnlichkeit}, oder zwischen Pflicht und Wunsch vorhanden seyn müsse, ist eine nothwendige Folge unserer Beschränktheit.~\RWSeitenw{289}
\item  Auf der Beschwerlichkeit dieses Streites beruht auch das \RWbet{Verdienstliche unserer Tugend}; je größer der Streit, desto verdienstlicher der Sieg.
\item  Endlich scheint es auch, daß der Mensch selbst zum Theile Schuld daran ist, daß manche seiner sinnlichen Begierden eine so hohe Lebhaftigkeit erstiegen haben.
\end{aufzc}
Auch diese Gründe haben den nämlichen Mangel wie die vorhergehenden.
\item \RWbet{In Rücksicht der vielen Verbrechen, welche Gott zuläßt}, kann man erinnern:
\begin{aufzc}
\item Was erstlich den \RWbet{innern Entschluß} betrifft, so kann Gott diesen nicht verhindern, es wäre denn, daß er die Freiheit selbst aufhübe, und mit ihr zugleich die Vernunft, mithin auch alle jene höhere Glückseligkeit, deren wir nur als vernünftige Wesen fähig und würdig werden. Da läßt sich denn wohl begreifen, daß der Nachtheil größer als der Vortheil wäre.
\item  Was aber die \RWbet{Ausführung} des bösen Entschlusses betrifft, so kann Gott diese oft um so eher zulassen, weil sie doch eigentlich nicht mehr das \RWbet{wahre Böse} ist, sondern im Gegentheile noch manches Gute veranlassen kann, als: 
\begin{aufzb}[a.]
\item Gerade dadurch, daß dem Lasterhaften zuweilen die Ausführung einer bösen That gelingt, zieht er sich eine \RWbet{Bestrafung} zu, welche um so geeigneter ist, entweder ihn selbst, wenn es noch möglich ist, oder doch Andere durch Abschreckung zu bessern, je deutlicher es Jedem einleuchtet, daß diese Strafe eine ganz natürliche Folge seines Verbrechens ist; \zB\ wenn der Unmäßige sich eben durch seine Unmäßigkeit eine Krankheit zuzieht, oder der Lügner eben um seiner Lügen willen keinen Glauben mehr findet, \udgl 
\item Die bösen Handlungen, welche der Eine verübt, geben manchen Andern Gelegenheit zur Uebung \RWbet{guter Thaten}, \zB\ zur Geduld, zur Vergebung, zur Entgeltung des Bösen mit Gutem \usw
\item Endlich entspringen wohl auch manche \RWbet{zufällige Vortheile} aus bösen Handlungen, indem Gott das\RWSeitenw{290}jenige, was böse beabsichtiget war, ganz gegen die Absicht des Unternehmers zu einem guten Zwecke leitet.\end{aufzb}\par
Dieß sehen wir freilich in \RWbet{manchen}, aber nicht in \RWbet{allen}, nicht einmal in den mehresten Fällen; und so fühlen wir uns immer noch versucht zu klagen, warum des Bösen so gar viel geschehe?
\end{aufzc}
\item \RWbet{In Rücksicht des Mißverhältnisses zwischen Tugend und Glückseligkeit} ist zu bemerken, daß es
\begin{aufzc}
\item noch nicht entschieden sey, ob der Zusammenhang des Ganzen auch erlaube, daß schon hier auf Erden eine hinreichende Belohnung alles Guten, und eine hinreichende Bestrafung alles Bösen eintrete. Vielmehr scheint es gewisse \RWbet{Fälle} zu geben, wo eine \RWbet{hinlängliche Entgeltung} in diesem Leben schlechterdings unmöglich ist (\RWparnr{85}\ \no\,10, 9).
\item Wenn eine solche Welteinrichtung, bei der die Entgeltung einer Handlung noch in diesem Leben eintritt, auch nicht an sich unmöglich wäre; so dürfte vielleicht doch eine andere, bei der es Gott freigestellt ist, ob er in diesem oder jenem Leben entgelte, wichtige Vorzüge vor der ersteren haben. Denn wenn sich Gott die Regel vorschreibt, Jedem schon gleich auf Erden zu entgelten; so wird er eben hiedurch in seinen Einrichtungen und in der Regierung unsers Schicksals beschränkt; darf dann so manche Anstalten, die in anderer Rücksicht vielleicht sehr große Vortheile hätten, nicht treffen, weil sie mit dieser Regel im Widerspruche ständen.
\item Der Nutzen, den ein genaues Verhältniß zwischen der Tugend und Glückseligkeit schon hier auf Erden dadurch gewähren könnte, daß es uns von der Gerechtigkeit Gottes recht augenscheinlich überzeugen würde, kann wohl auch erreicht werden, wenn Gott nur einige Male recht sichtbar lohnt oder straft, in einer Offenbarung aber erkläret, daß erst das andere Leben das eigentliche Land der Vergeltung \Hhat{sey}{seyn werde}.
\item Würden wir finden, daß schon hier immer eine hinlängliche Vergeltung eintrete; so würde Einer der~\RWSeitenw{291}\ stärksten Gründe für den Glauben an Unsterblichkeit wegfallen (\RWparnr{85}\ \no\,11.).
\item Wir würden uns endlich erlauben, den sittlichen Werth eines Jeden bloß nach dem Glücke, dessen er genießt, zu messen. Wenn wir dagegen uns jetzt bewogen fühlen, den sittlichen Charakter eines Menschen auszuforschen: so sind wir genöthigt, Betrachtungen und Schlüsse ganz anderer Art anzustellen; Untersuchungen, die uns zwar ungleich mehr Mühe verursachen, aber auch viel lohnender sind, weil wir uns mit der Natur unseres eigenen Herzens, mit dem inneren Wesen der Tugend sowohl als des Lasters, mit dem Zusammenhange, der zwischen diesen und jenen einzelnen Tugenden und Lastern obwaltet, bekannt machen müssen.
\end{aufzc}\par
   So wahr dieß Alles ist, so wenig reichet es doch zu unserer völligen Beruhigung hin, weil es nicht zeigt, daß und warum jenes Mißverhältniß zwischen der Tugend und Glückseligkeit gerade so groß, als wir es antreffen, seyn müsse.
\item Was endlich die Vernunft zur Rechtfertigung Gottes wegen der \RWbet{ungleichen Vertheilung der Gelegenheiten zu unserer sittlichen Ausbildung sowohl als zum Genusse der Glückseligkeit} vorbringen kann, wurde schon \RWparnr{77}\ \no\,7.\ angeführt.\par
Auch gegen diese Gründe gilt die schon mehrmals gemachte Bemerkung; und so ergibt sich denn aus Allem, daß die sich selbst überlassene Vernunft die Frage vom Ursprunge und Zwecke des vielen Uebels auf Erden nicht so befriedigend beantworten könne, daß es nicht sehr erwünscht wäre, eine Offenbarung möchte uns hierüber ein Mehres sagen.
\end{aufzb}
\end{aufza}
   
\RWpar{108}{Noch einige Zweifel in der natürlichen Dogmatik}
Nebst den bisher erwähnten Zweifeln in der Dogmatik der natürlichen Religion gibt es noch einige andere, für den gebildeten Menschen eben nicht unwichtige Fragen, über welche~\RWSeitenw{292}\ eine nähere Erklärung Gottes er äußerst erwünschlich finden müßte. Ich will sie nur kurz berühren.
\begin{aufza}
\item Ob unsere \RWbet{Bittgebete} bei Gott Erhörung finden; ob insbesondere auch
\item \RWbet{Fürbitten}, die wir für das Wohl Anderer zu seinem Throne emporsenden, von ihm berücksichtiget werden können? -- Wie mächtig dränget uns nicht das Herz bei gewissen Anlässen, in eigenen Leibes- und Seelennöthen sowohl als bei dem Unglücke unserer Freunde zu Gott um Hülfe zu rufen! Und dennoch können wir dieß vernünftiger Weise nicht thun, wenn wir nicht die Erhörbarkeit solcher Gebete voraussetzen. Aber ohne daß Gott sich hierüber selbst erklärt hat, muß uns diese Voraussetzung nicht nur sehr ungewiß, sondern sie kann uns beinahe kühn erscheinen. Oder wer sind wir, zu glauben, daß Gott um unserer Bitten willen Veränderungen im Weltlaufe vornehmen werde? --
\item Ob es noch \RWbet{außer uns Menschen} gewisse \RWbet{vernünftige} Wesen in Gottes Schöpfung gebe, ob und in welcher Verbindung wir etwa mit diesen Wesen stehen? -- Daß es vernünftige Wesen auch außer uns Menschen in Gottes Schöpfung gebe, ist zwar auch ohne alle Offenbarung sehr wahrscheinlich; aber um desto unbeantwortlicher, in welcher Verbindung sie mit uns stehen? Und dennoch wie wichtig für jeden edleren Menschen, den es unendlich freuen und erheben würde, wenn er es wüßte, daß ihn auch höhere Geister einer Aufmerksamkeit würdigen, der sich dem wohlthätigen Einflusse derselben so gerne hingeben wollte, wenn er erst wüßte, daß ein solcher Statt findet!
\item In welchem Zustande unsere Seele in jenem andern Leben fortdauern werde? Von welcher Art die Belohnungen sowohl als auch die Strafen der andern Welt seyn werden? Ob wir gleich in dem Augenblicke, da wir aus diesem Leben austreten, ein anderes anfangen, oder ob es einen gewissen vielleicht Jahrhunderte dauernden Seelenschlaf gebe? Ob wir mit einem neuen Leibe bekleidet werden? Ob wir auch dort noch in Verbindung mit andern Wesen stehen, und thätig seyn werden? Ob wir stets fortschreiten werden in unserer Vollkommenheit? Ob wir die Unsrigen dort wieder~\RWSeitenw{293}\ antreffen werden? Ob Jene, die uns bereits vorangegangen sind, fortwährend mit uns verbunden sind? Kunde von unserm Befinden erhalten? auf uns wohl gar noch einzuwirken vermögen? \usw\ Dieß Alles sind Fragen, bei deren versuchter Beantwortung wir, selbst wenn ein fortgesetztes Nachdenken uns der Gründe mehre kennen lehrt, als wir uns Anfangs nur als möglich vorgestellt hatten, immer doch schmerzlich fühlen, hier mangle uns die sonst überall so willkommene Bestätigung unserer Schlüsse durch die Erfahrung. Und gleichwohl, wie viel würde es zu unserer Beruhigung, zur Minderung unserer Furcht vor dem Tode, zu unserem Troste beim Absterben geliebter Anverwandten, zur Erhöhung des Eifers, mit dem wir an unserer sittlichen Vervollkommnung auf Erden arbeiten, beitragen können, wenn wir bestimmtere Antwort auf alle diese Fragen aus Gottes Munde erhielten!
\item Wird dieses Menschengeschlecht, zu dem wir als Glieder gehören, von Zeit zu Zeit in der Vollkommenheit fortschreiten, oder wird es im Gegentheil stets auf derselben Stufe stehen bleiben, oder wohl gar je länger je mehr sich verschlimmern? -- Keinem, der um das Wohl seiner Brüder besorgt ist, \dh\ Keinem, der wahren \RWbet{Adel der Seele} besitzt, kann diese Frage gleichgültig seyn. Ohne Offenbarung aber, bloß aus demjenigen, was die Geschichte uns hierüber sagt, oder aus bloßen \RWbet{Vernunftgründen} können wir nie mit völliger Gewißheit über sie entscheiden.
\end{aufza}

\RWpar{109}{In ihrer Ethik ist die natürliche Religion vergleichungsweise noch am Vollkommensten}
Weit vollkommener als in ihrer Dogmatik ist die natürliche Religion in ihrem praktischen Theile, besonders in ihrer \RWbet{Pflichtenlehre} oder \RWbet{Ethik}; hier gibt es der Zweifel nur wenige, und eine viel größere Vollständigkeit. Denn 
\begin{aufza}
\item so widersprechend auch oft die Meinungen der Menschen über bloß theoretische Fragen sind, so übereinstimmend sind sie fast durchgängig, wenn es sich um die Beantwortung~\RWSeitenw{294}\ einer praktischen Frage, ob diese oder jene Handlungsweise recht, billig und gut sey, um die Entscheidung eines sittlichen Gesetzes handelt. Bei allen Völkern und zu allen Zeiten hat man nicht völlig, aber doch beinahe dieselben unveränderlichen Begriffe vom Inhalte des Sittengesetzes an den Tag gelegt. Allenthalben erklärt man die Unzucht, den Diebstahl, die Lüge, den Neid, die Rachsucht für unrecht; allenthalben dagegen preist man die Mäßigkeit, die Keuschheit, die Gerechtigkeit, die Wohlthätigkeit, die Nächstenliebe, die Wahrhaftigkeit, die Treue \usw\ Aus dieser großen Uebereinstimmung läßt sich der Schluß ziehen, daß die menschliche Vernunft mit einem sehr hohen Grade der \RWbet{Gewißheit} erkenne, welche Handlungsweisen dem Wohle des Ganzen im Allgemeinen zuträglich oder nachtheilig, und also pflichtmäßig oder unerlaubt sind.
\item Daß wir sie aber auch beinahe \RWbet{vollständig} kennen, \dh\ daß es kaum viele Handlungsweisen von einer wirklichen Gemeinnützigkeit gibt, die von uns übersehen werden; schließe ich so. Fände das Gegentheil Statt; gäbe es viele Handlungsweisen von einer allgemeinen und von uns gleichwohl nicht erkannten Nützlichkeit: so müßte es doch gewiß das erste und wichtigste Geschäft einer Offenbarung seyn, uns mit diesen bekannt zu machen, und sie uns als Gebote aufzustellen. Allein wir finden in allen Religionen auf Erden, die hie und da für geoffenbaret gelten, beinahe nur eben dieselben Gesetze, die schon die bloße Vernunft erkennt. Selbst auf den Fall also, daß Eine oder mehre dieser Religionen wirkliche göttliche Offenbarungen sind, bleibt es doch wahr, daß die natürliche Religion in ihrer Pflichtenlehre \RWbet{fast} eben die Vollständigkeit habe, wie eine Offenbarung.
\end{aufza}
\begin{RWanm}
Können wir wohl umhin, bei dieser Gelegenheit die Weisheit Gottes zu bewundern, die sich in dieser Einrichtung unsers Verstandes und in dieser Leitung unserer Begriffe kund gibt? In allem dem, was unser gesetzmäßiges Verhalten, was unser Thun oder Lassen anlangt, \dh\ in dem, was uns zu wissen am \RWbet{Allernöthigsten} ist, erkennen wir die Wahrheit mit vieler Uebereinstimmung und Vollständigkeit, und nur in Dingen, die uns viel weniger zu wissen nöthig sind, weichen unsere Urtheile häufig theils von einander, theils von der Wahrheit selbst ab.~\RWSeitenw{295} 
\end{RWanm}

\RWpar{110}{Doch könnte uns eine Offenbarung auch in der Pflichtenlehre wichtige Dienste leisten}
Die eben gerühmte Vollkommenheit der natürlichen Ethik dürfen wir gleichwohl nicht übertreiben. Es gibt doch immer
\begin{aufza}
\item \RWbet{einige Pflichten}, über welche die sich selbst überlassene Vernunft wirklich mit keiner völligen Gewißheit zu entscheiden vermag, weil ihre Gründe etwas verborgener liegen. Dergleichen sind \zB\
\begin{aufzb}
\item die Art und Weise, wie Gott am zweckmäßigsten von uns verehrt werden solle? Da die Vernunft so wenig vermag, sich einen deutlichen Begriff von Gottes Eigenschaften zu bilden; so ist es ihr auch schwer zu bestimmen, welche Verehrungsweise Gottes in aller Rücksicht die zweckmäßigste wäre.
\item Ob und in welchen Fällen Jemand verbunden seyn könne, für das Wohl Anderer sogar sein Leben hinzugeben?
\item Ob und in welchen Fällen es ein Recht gebe, Andere des Lebens zu berauben? -- Wie sehr haben nicht die größten Weltweisen über diese beiden Fragen gestritten! --
\end{aufzb}\par
Gleichwohl sind Zweifel über unsere Pflichten immer sehr gefährlich; denn wenn es, leider! eine tägliche Erfahrung lehrt, daß wir uns Uebertretungen unserer Pflichten zur Schuld kommen lassen selbst dort, wo das Gewissen mit aller Bestimmtheit spricht: um wie viel weniger werden wir uns entschließen, eine Verbindlichkeit zu erfüllen, welche uns zweifelhaft ist!
\item Ferner ist es bekannt, daß wir in Stunden der Versuchung zuweilen selbst an solchen Verbindlichkeiten zu zweifeln anfangen, die wir bei ruhiger Ueberlegung deutlich genug erkennen, und die auf sehr sichern Gründen beruhen. Welch ein Vortheil also für unsere Tugend, wenn eine Offenbarung dergleichen Zweifeln der Leidenschaft dadurch ein Ende machte, daß sie mit lauter Stimme uns zuriefe, es sey die göttliche Entscheidung selbst, die dieses fordert!~\RWSeitenw{296}
\item Dadurch daß eine Offenbarung uns unsere sämmtlichen Pflichten als \RWbet{Gebote Gottes} aufstellt, könnte sie nebst dem so eben erwähnten Vortheile der größern Gewißheit auch noch den zweiten höchst wichtigen Vortheil erreichen, daß wir die stärksten Beweggründe zu ihrer Erfüllung erhielten, wenn diese Erfüllung als eine Handlung, die uns das Wohlgefallen, jede Uebertretung dagegen als eine That, die uns das Mißfallen und die Strafe Gottes zuziehen muß, dargestellt würde.
\item Endlich dürfte eine Offenbarung selbst einige \RWbet{neue} der bloßen Vernunft gar nicht erkennbare Pflichten und Sittenvorschriften aufstellen können. Denn da wir Menschen nicht alle unsere Verhältnisse kennen, auch bei Weitem nicht alle Folgen unserer Handlungen zu überschauen vermögen: so kann es denn doch seyn, daß gewisse Handlungsweisen das allgemeine Beste befördern würden, ohne daß wir es eben so deutlich einsehen. Wenn nun dieß wirklich der Fall seyn sollte; so würden wir offenbar weit glücklicher seyn in dem Besitze einer Offenbarung, als wir es ohne sie sind, weil diese mehren Pflichten, die sie uns auflegt, so beschränkend sie auch unserer Sinnlichkeit erscheinen mögten, das Wohl der menschlichen Gesellschaft dennoch erhöhen würden, wodurch am Ende auch unser eigenes gewönne.
\end{aufza}
   
\RWpar{111}{Unzulänglichkeit der natürlichen Asketik oder Tugendmittellehre}
\begin{aufza}
\item Bedeutend mangelhafter als in der Ethik ist die natürliche Religion in ihrer \RWbet{Asketik} oder \RWbet{Tugendmittellehre}. Die Asketik begreift nämlich denjenigen Theil unserer Pflichten, die uns nur mittelbarer Weise, nur um uns die Erfüllung unserer übrigen Pflichten zu sichern und zu erleichtern, obliegen. (\RWparnr{65}\ \no\,4,\,b.) Daß nun die Ethik der natürlichen Religion eine so große Vollständigkeit besitzt, rührt eigentlich nur daher, weil ihre Sätze größtentheils auf Verhältnissen beruhen, welche leicht in die Sinne fallen. Denn ob diese oder jene Weise zu handeln, wenn sie erst allgemein angenommen würde, das Wohl des Ganzen stören oder befördern müßte,~\RWSeitenw{297}\ fällt meistens deutlich genug in die Augen. Die Pflichten dagegen, welche die Tugendmittellehre aufstellt, beruhen auf Wahrheiten, die ihrer Natur nach schon nicht so leicht in die Augen fallen, die auf so manchen zum Theile sehr feinen psychologischen Bemerkungen beruhen, \udgl\ Nothwendig muß also die natürliche Asketik viel unsicherer und unvollständiger seyn als die natürliche Ethik.
\begin{RWanm}
Vorhin (\RWparnr{109}\ \no\,2) bewies ich die Vollständigkeit der natürlichen Ethik daraus, weil auch alle positiven Religionen, die es auf Erden gibt, beinahe nicht \RWbet{mehre} Pflichten als jene aufstellen. Sollte es einem Verehrer der natürlichen Religion einfallen, auf eben die Art beweisen zu wollen, daß auch die natürliche Asketik vollständig sey; so würde er sehen, daß dieser Beweis sich hier nicht führen lasse. Denn in der That finden wir bei den verschiedenen positiven Religionen, \zB\ im Christenthume, eine beträchtliche Anzahl von Tugendmitteln aufgestellt, von denen die natürliche Religion nichts weiß. 
\end{RWanm}
\item Um nun auch einige \RWbet{einzelne Tugendmittel} anzugeben, in denen die natürliche Religion mangelhaft ist, will ich nur folgende erwähnen:
\begin{aufzb}
\item Eine recht zweckmäßig eingerichtete Gottesverehrung könnte ohne Zweifel auch ein sehr wichtiges Tugendmittel werden. Aber wir haben schon angemerkt, daß die bloße sich selbst überlassene Vernunft sehr wenig zu bestimmen vermöge, wie eine solche beschaffen seyn müßte.
\item Ob tugendhafte Menschen, die ihren Lebenswandel bereits vollendet haben, oder andere höhere und vollkommenere Geister durch unsere Anrufung auf eine wirksame Art aufgefordert werden können, uns bei unseren sittlichen Angelegenheiten beizustehen, darüber kann die sich selbst überlassene Vernunft abermals nicht entscheiden.
\item Eines der wichtigsten Beförderungsmittel der Tugend ist wohl der \RWbet{frohe Muth}, mit dem wir das Geschäft der Besserung anfangen sollen. Der natürlichen Asketik aber mangelt es sehr an einem Mittel, uns diesen frohen Muth, diese Hoffnung eines gesegneten Erfolges besonders dann zu geben, wenn es sich um die Ablegung einer verjährten bösen Gewohnheit handelt. Denn wer den Vorsatz der Besserung schon unzählige Male gefaßt, ihn aber eben~\RWSeitenw{298}\ so oft wieder gebrochen hat: mit welchem Zutrauen zu sich selbst kann er hoffen, daß jener neue Vorsatz, den er jetzt faßt, der eben nicht stärker, eben nicht lebhafter scheint, als alle vorhergehenden waren, dem aber eine noch immer älter gewordene, noch immer hartnäckigere Gewohnheit entgegensteht, von einem glücklicheren Erfolge begleitet seyn werde? Hofft er dieß aber nicht, so kann er auch mit keinem frohen Muthe an seine Besserung gehen; und sie muß eben deßhalb schon mißlingen. Da nun die Anzahl der Menschen, die irgend eine üble Gewohnheit an sich haben, leider! nur allzugroß ist; und da Niemand wissen kann, ob er nicht einst sich selbst noch unter dieser Classe von Menschen befinden werde: so erhellet hieraus die Wichtigkeit dieses Mangels. Die göttliche Offenbarung hilft diesem Uebel durch ihre Lehre von Gottes Gnade ab, in welcher es heißt, daß sich der Allmächtige in der Ausspendung seiner Gnaden nicht eben nach unserer bisherigen Benützung der schon empfangenen Wohlthaten richte; daß er daher auch selbst demjenigen, der seine bisherigen Gnaden schlecht benützt hatte, öfters doch große neue Gnaden schenke; daß eben jenes besonders lebhafte Gefühl unserer eigenen Ohnmacht, welches in uns entsteht, schon eine Wirkung der göttlichen Gnade sey, die uns noch weiterer Mittheilungen empfänglich macht, und sie erwarten läßt, weil es ja heißt: \RWbet{daß Gott stark sey in den Schwachen}, \dh\ in denjenigen, die ihre Ohnmacht erkennen, und alles Gute, das in ihnen zu Stande kommen soll, nicht sich, sondern nur Gott zuzuschreiben bereit sind; \usw\
\item Sehr ohnmächtig zeigt sich auch die natürliche Asketik, wenn es sich handelt um Mittel, wie wir die wichtigen Pflichten der \RWbet{Feindesliebe}, der \RWbet{großmüthigen Vergebung erlittener Unbilden}, und mehre dergleichen schwere Pflichten uns erleichtern sollen. U.\,s.\,w.
\end{aufzb}
\item Endlich, wenn es auch keine \RWbet{bemerkbare} und von uns \RWbet{selbst empfundene} Mängel in der natürlichen Asketik gäbe: so wäre es doch immer möglich, daß eine Offenbarung uns \RWbet{neue}, der Vernunft nicht erreichbare,~\RWSeitenw{299}\ von ihr \RWbet{nicht einmal geahnete Mittel} zur Tugend kennen lehre; und so bestände dann schon deßhalb allein die Pflicht, uns umzusehen, ob eine solche göttliche Offenbarung für uns nicht etwa wirklich da sey?
\end{aufza}
   
\RWabs{Zweite Abtheilung}{Nothwendigkeit einer Offenbarung in Rücksicht auf das menschliche Geschlecht im Ganzen}
\RWabs{I}{Aus der Geschichte der Irrthümer, in welche nicht nur die große Menge der Menschen, sondern auch die Gelehrten zu allen Zeiten verfallen sind}

\RWpar{112}{Uebersicht dieses Beweises}
So viele Mängel die natürliche Religion selbst in der \RWbet{vollkommensten} Gestalt hat, in der wir sie bisher betrachteten; so ist doch die Ansicht, die sich die \RWbet{einzelnen} Glieder des menschlichen Geschlechtes von ihr gebildet hatten, fast immer noch viel \RWbet{unvollkommener} gewesen. Nehmen wir die Bekenner des Juden- und Christenthums aus, so haben die meisten übrigen Menschen bald diesen, bald jenen der in dem zweiten Hauptstücke aufgestellten wichtigen Lehrsätze der natürlichen Religion verkannt, und noch viel häufiger ihren Glauben daran bald durch diese, bald jene irrigen Zusätze verunreiniget und unwirksam gemacht. Die Geschichte beweiset dieß auf das Deutlichste, indem sie zeigt, daß es kein Volk auf Erden gegeben, das nicht gewissen sehr groben die Religion betreffenden Irrthümern gehuldiget hätte, daß ferner selbst die größten Gelehrten älterer sowohl als neuerer Zeit nie völlig frei von solchen Irrthümern geblieben sind. Ich will dieses Alles in den gleich folgenden §§ mehr durch einzelne Beispiele als durch eine vollständige Aufzählung beweisen. Auch so noch wird sich zeigen, daß das Bedürfniß einer Offenbarung, das ich so eben~\RWSeitenw{300}\ selbst für den gebildetsten Theil der Menschheit dargethan habe, in einem noch ungleich höheren Grade für unser Geschlecht, wiefern wir es als ein Ganzes betrachten wollen, vorhanden sey, indem dasselbe, bloß seinen eigenen Einsichten überlassen, so wenig im Stande ist, auch nur die wichtigsten Wahrheiten der natürlichen Religion in einer gewissen Vollständigkeit und Reinheit aufzufassen.
\begin{RWanm}
Es könnte vielleicht Jemandem scheinen, als stände meine gegenwärtige Behauptung, daß fast kein einziger Mensch die Lehren der natürlichen Religion vollständig aufgefaßt habe, im Widerspruche mit der Erklärung des \RWparnr{64}\ \no\,3, vermöge deren ich nur solche Meinungen als sichere Lehrsätze der natürlichen Religion der Menschheit angesehen wissen wollte, die fast von Allen anerkannt werden. Um durch diesen scheinbaren Widerspruch nicht beirret zu werden, braucht man bloß zu bemerken, daß ich zur Aufstellung eines sichern Lehrsatzes der natürlichen Religion des menschlichen Geschlechtes, nicht eben eine \RWbet{vollkommene}, sondern nur eine \RWbet{beinahe allgemeine} Uebereinstimmung aller Menschen verlange. Da kann es also bei einer jeden Lehre noch einzelne Menschen geben, die ihr nicht zugethan sind; und eben so kann jeder einzelne Mensch noch einige besondere Irrthümer hegen, die aber eben deßhalb, weil sie bei einem Jeden von einem andern Inhalte sind, zu keinen Glaubensmeinungen der natürlichen Religion des ganzen menschlichen Geschlechtes erhoben werden können. 
\end{RWanm}

\RWpar{113}{1.~Irrthümer einiger Weltweisen, wodurch die ganze natürliche Religion auf einmal umgestoßen oder doch schwankend gemacht wird}
Es wurde schon (\RWparnr{10}\ \no\,3.) erinnert, daß mehre nicht nur alte, sondern auch neuere Weltweisen es sich zu einem eigenen Geschäft gemacht hatten, die Gewißheit \RWbet{aller} menschlichen Erkenntnisse zu zerstören. Eben darum wollten sie auch keine der wichtigen Wahrheiten, welche die natürliche Religion enthält, für sicher gelten lassen. Von dieser Secte waren unter den Alten \RWbet{Arkesilaus, Karneades, Klitomachus, Agrippa}; besonders berühmt aber machte sich \RWbet{Pyrrhon},~\RWSeitenw{301}\ dessen Lehren \RWbet{Sextus Empirikus} in einem sehr ausführlichen Werke zusammenstellte. Unter den neuern Skeptikern war Einer der berühmtesten \RWbet{Peter Bayle} (Professor zu Rotterdam, der \RWbet{1706}, die Feder in der Hand, starb). In seinem \RWlat{Dictionnaire critique}\RWlit{}{Bayle1} trug er hin und wieder den äußersten Skepticismus vor, und behauptete \zB , man könne das Daseyn Gottes nicht mit Gewißheit darthun, sondern man müsse es bloß auf den Ausspruch der Bibel glauben. Auch der engländische Bischof \RWbet{Peter Brown} gerieth durch den Satz, daß alle unsere Erkenntniß von Gott und übersinnlichen Dingen nur analogisch wäre, auf die Meinung, daß diese Kenntnisse keine Gewißheit hätten. \RWbet{Arthur Ashley Sykes} trug unter dem Namen \RWbet{Eugenius Philalethes} gleichfalls verschiedene skeptische Meinungen vor, und stellte unter Anderm die sehr gefährliche Behauptung auf, daß man für keinen Irrthum verantwortlich seyn könne. Einer der scharfsinnigsten war \RWbet{David Hume}, der sich \RWbet{in seinen Gesprächen über die natürliche Religion}\RWlit{}{Hume1a}, in seinem \RWbet{Versuche über die Wunder} und in anderen Schriften bemühte, allen Glauben an Gott, Unsterblichkeit und Offenbarung zu vernichten.\par
Welch einen verderblichen Einfluß dieses System des Skepticismus, wenn es ja den Namen eines Systems verdient, auf die Tugend und Glückseligkeit seiner Anhänger äußern müsse, läßt sich von selbst erachten.
   
\RWpar{114}{2.~Irrthümer in Hinsicht auf Gott}
\begin{aufza}
\item Mag auch nach einer Meinung, welche selbst die Geschichte bestätiget, der erste Glaube der Menschen \RWbet{Monotheismus} gewesen seyn; so hat sich doch \RWbet{Polytheismus} oder Vielgötterei sehr zeitig eingefunden, und war bis auf Christi Zeiten bei allen Völkern der Erde, mit Ausnahme des einzigen gar nicht zahlreichen Volkes der Israeliten, herrschender Glaube gewesen. Ja selbst noch heut zu Tage herrscht, wenn wir die Christen, Juden und Mahomedaner ausnehmen, bei allen übrigen Völkern der Erde Polytheismus. Und nicht~\RWSeitenw{302}\ genug, daß man der Gottheiten mehre annahm; wie äußerst unwürdig waren auch die Begriffe, die man sich von ihnen bildete, oder noch heut zu Tage bildet! Die \RWbet{Perser} glaubten, wie schon (\RWparnr{106}) erwähnt wurde, daß es nebst einem \RWbet{guten} Gotte, auch einen \RWbet{bösen} gebe, der sein einziges Wohlgefallen nur an der Ausübung des Bösen finde. Die weisen Aegyptier konnten selbst Hunden und Katzen, Krokodilen, und andern dergleichen Thieren göttliche Ehre erweisen. Die \RWbet{Tyrier} hatten noch zu Alexanders des Großen Zeiten eine so kindische Vorstellung von ihren Göttern, daß sie die Statue des Apollo in Ketten legten, damit er nicht etwa zu Alexander übergehe. (\RWlat{Curtius Rufus l.\RWlit{}{Curtius1}\,4.\ c.\,3.}) Bei eben dieser Gelegenheit erneuerten sie auch die lange schon unterlassene Gewohnheit der \RWbet{Menschenopfer}. Dergleichen Menschenopfer zur Ehre der Götter waren auch bei den Griechen, Römern, Galliern und Germanen nicht ungewöhnlich. Noch gewöhnlicher aber war es, die Götter durch allerlei Ausschweifungen zu verehren. Wie verführerisch zur Wollust war nicht die ganze \RWbet{Götterlehre} der Griechen und Römer! Ihren verderblichen Einfluß auf die Sitten erkannte selbst \RWbet{Terentius}, wenn er (\RWlat{Eunuch.\ Act.~3.\ Sc.~5}.)\RWlit{}{Terentius1} einen Jüngling schildert, der sich durch Jupiters Vorbild zu einer Schandthat ermuntert. Und \RWbet{Seneca} schreibt (\RWlat{de brevitate vitae cap.\,6.})\RWlit{}{Seneca2}: \RWlat{Quid aliud est, vitia nostra incendere, quam auctores illis inscribere deos, et dare morbo exemplo divinitatis excusatam licentiam?} Nur aus diesem Grunde wollte der weise \RWbet{Plato} den Homer, Hesiod und andere Dichter aus seiner idealischen Republik gänzlich verbannt wissen.
\item Aber auch selbst die \RWbet{weisesten Männer} erhoben sich zu keinem ganz richtigen Begriffe von Gott. \RWbet{Parmenides}, der Stifter der eleatischen Schule, sein Schüler \RWbet{Zeno} und alle Weltweisen aus dieser Schule lehrten den \RWbet{Pantheismus}, \dh\ sie behaupteten, daß es nur eine einzige Substanz, nämlich die Gottheit, gebe, alles Uebrige sey nur Schein. Auch \RWbet{Plato} drückte sich (\RWlat{de leg.}) auf eine so sonderbare Weise über das höchste Wesen aus, daß man am Ende fast meinen muß, sein höchster Gott sey die \RWbet{Welt}~\RWSeitenw{303}\ \RWbet{selbst} gewesen. Merkwürdig ist auch der Eingang, den \RWbet{Cicero} zu seinem bekannten Werke \RWlat{de natura deorum}\RWlit{}{Cicero1} macht: Die meisten Philosophen behaupten (schreibt er), es gebe Götter, Protagoras zweifelt daran. Diagoras von Miletus und Andere behaupten, es gebe keine. Diejenigen ferner, welche sie annehmen, sind über ihre Gestalt, ihren Aufenthaltsort, und, was das Wichtigste ist, ob sie auch wirksam oder unwirksam sind, verschiedener Meinung. Unter so vielen einander widersprechenden Meinungen ist es gar wohl möglich, daß keine die wahre sey. Doch wer auch glauben sollte, etwas Gewisses über diesen Gegenstand zu haben, den wird die große Verschiedenheit der gelehrtesten Männer alsbald zu Zweifeln bringen! --
\item Daß selbst die besten heidnischen Philosophen Gott nicht als Schöpfer, sondern nur als Weltbildner annahmen, wurde schon (\RWparnr{106}) erinnert, wie ich denn auch schon einige andere hieher gehörige Irrthümer gelegenheitlich berührte. Unter den \RWbet{neuern Weltweisen} machte besonders \RWbet{Benedikt Spinoza}, der ein geborner Jude, von seinen eigenen Glaubensgenossen ausgeschlossen, zu Amsterdam als Privatmann lebte, und \RWbet{1677} starb, sich durch eine scheinbar noch viel consequentere Darstellung des pantheistischen Systems, als alle früheren waren, berühmt; und der französische \RWbet{Graf von Boulainvilliers} suchte dieses System unter dem Scheine, als ob er es widerlegte, nur weiter auszubreiten. Auch der Irländer \RWbet{Johann Toland} huldigte demselben. Ein Gleiches gilt von dem deutschen Gelehrten \RWbet{Joseph Edelmann}, und was die neuesten Weltweisen in Deutschland, \RWbet{Schelling}, \RWbet{Hegel} und ihre Anhänger lehren, ist ebenfalls nur eine Art \RWbet{Pantheismus}. Selbst Männer, die als Lehrer des Christenthums auftraten, wie der berühmte Reformator \RWbet{Calvin}, und in unseren Tagen Prof.~\RWbet{Schleiermacher} (in s.~Reden über die Religion) huldigten dem Pantheismus. Noch weiter ging \RWbet{Lamettrie}, ein französischer Arzt, der später Mitglied der Berliner Akademie ward, indem er in mehren seiner Schriften (\RWlat{l'homme une machine} \usw )\RWlit{}{LaMettrie1} den gröbsten Materialismus vortrug; ingleichen der Verfasser des \RWlat{Système de la nature}\RWlit{}{Dholbach1}, welcher das Daseyn Gottes geradezu läugnete. Der französische Gelehrte \RWbet{Robinet} dehnte die Lehre von der \RWbet{Unbegreiflichkeit} Gottes so weit aus, daß er dem menschlichen Verstande alles Recht absprach, irgend etwas Positives über Gott auszusagen, weil zwischen den Attributen Gottes und den Kräften und Fähigkeiten der Geschöpfe gar~\RWSeitenw{304}\ keine Analogie Statt finde; man könne daher \zB\ nicht einmal sagen, daß Gott ein \RWbet{vernünftiges, heiliges, gütiges} Wesen sey, und eben so wenig auch das Gegentheil.
\end{aufza}
   
\RWpar{115}{3.~In Ansehung des Menschen}
\begin{aufza}
\item Eine gewisse Fortdauer nach dem Tode glaubten zwar fast alle Völker, allein die Begriffe, welche sie sich von unserem \RWbet{Zustande} in jenem andern Leben und von den \RWbet{Freuden} des Tugendhaften daselbst gebildet hatten, waren sehr falsch und zweckwidrig. So träumten sie sich in dem anderen Leben meistens eine Seligkeit, die in ununterbrochenem Genusse der ausschweifendsten sinnlichen Lüste bestehen würde, eine Vorstellung, welche den ohnehin zu großen Werth, den der Mensch sinnlichen Freuden beilegt, noch mehr erhöhen und nach dem Genusse derselben um so begieriger machen mußte. Diese verderbliche Meinung nahm dann auch \RWbet{Mahomed} in sein Glaubenssystem auf. Wie thöricht war ferner nicht die Fabel vom \RWbet{Lethe}, der den Verstorbenen die Rückerinnerung an dieses Leben raubt, \umA\ 
\item Die \RWbet{Weltweisen} verfehlten es häufig darin, daß sie die Fortdauer der menschlichen Seele nicht nur als etwas durch bloße Vernunft nicht streng Erweisliches ausgaben, sondern geradezu verwarfen. Dieß thaten \zB\ in älterer Zeit die \RWbet{Epikuräer}, in neuerer die \RWbet{Materialisten}, \uA\  Selbst \RWbet{Aristoteles} gab nur eine Fortdauer der menschlichen Seele ohne Bewußtseyn und Vorstellungen, oder wenigstens ohne Rückerinnerung zu. Und um nur Eines Beispieles der Verirrung auch noch aus unserer Zeit zu erwähnen, das letzte Werk, das \RWbet{Wieland} schrieb, hatte den Zweck, den Glauben zu verbreiten, daß wir nach dem Tode zwar fortdauern in einem bewußten und empfindenden Zustande, doch ohne Rückerinnerung an dieses Erdenleben; welches am Ende eben so viel ist, als ob man die Unsterblichkeit selbst läugnete. Denn wenn ich fortdauere, ohne zu wissen, daß ich derselbe bin: ist es nicht eben so viel, als ob ich aufgehört hätte, zu seyn, und ein Anderer wäre an meine Stelle getreten? Und diesen Glauben, was das Merkwürdigste ist, empfahl uns Wieland durch den Titel seines Buches (\RWbet{Euthanasia}\RWlit{}{Wieland1}) als ein Mittel, dem Tode seine Furchtbarkeit zu benehmen. Wir wollen es glauben, für den, der hier Böses gethan hat.
\item Auch der Begriff, den die heidnischen Weisen von unserem \RWbet{gegenwärtigen} Zustande gaben, war nicht sehr~\RWSeitenw{305}\ tröstlich und zweckmäßig. \RWbet{Theognis}, Verfasser der Gnomen, sagt (v.\,425--428): \erganf{Das Beste für den Bewohner der Erde ist wohl, niemals geboren zu seyn, die Strahlen der eilenden Sonne niemals gesehen zu haben; ist man aber geboren, auf's Baldigste zu gehen durch Pluto's Pforten, und bedeckt zu liegen mit vieler Erde!} -- \RWbet{Plato} schilderte, wie ich schon \RWparnr{106} bemerkte, unsern gegenwärtigen \RWbet{Zustand} auf Erden als einen \RWbet{Zustand der Strafe}, den Leib als einen Kerker, in welchem die Seele zur Abbüßung gewisser Sünden, die sie in einem andern Leben begangen, verbannt worden wäre.
\end{aufza}

\RWpar{116}{4.~In Hinsicht auf die Moral}
\begin{aufza}
\item \RWbet{Beim Volke.}\par
Es gibt Verbrechen, die nicht nur von einzelnen Menschen, sondern von ganzen Völkerschaften mit einer so großen Allgemeinheit begangen worden sind, daß man vermuthen muß, ihr sittliches Gefühl sey in Beziehung auf diese Fehler so abgestumpft gewesen, daß sie das Sündhafte derselben gar nicht erkannten, weil sonst nicht zu begreifen wäre, wie man sich diese Vergehungen so allgemein hätte zur Schuld kommen lassen. Das ist vornehmlich der Fall, wenn es Verbrechen sind, die an sich eben von keinem besondern Reize sind, Verbrechen, vor deren Begehung sich vielmehr die menschliche Natur in ihrem gewöhnlichen Zustande entsetzt, oder wenn wir bemerken, daß man die Ausübung solcher Handlungen sogar für etwas Verdienstliches, für eine eigene Verehrungsweise der Götter angesehen habe. Hieher gehören so manche \RWbet{schändliche Ausschweifungen}, die bei vielen Völkern als ein eigentlicher Gottesdienst vollzogen wurden. Hieher auch jene gräßlichen \RWbet{Menschenopfer}, zu denen wir fast ein jedes Volk zuweilen, besonders wenn es von irgend einem ungewöhnlichen Mißgeschick heimgesucht wurde, seine Zuflucht nehmen sehen, wie zu dem ausgiebigsten Mittel, welches den Menschen zur Besänftigung der erzürnten Gottheit nur zu Gebote stehe. Eltern sogar haben ihre Kinder zum Opfer dargebracht! Setzt dieß nicht, nebst unrichtigen Begriffen von Gott, auch sehr unrichtige Begriffe von den menschlichen Pflichten voraus? Hieher gehört ferner die überaus grausame Behandlung der Feinde und Kriegsgefangenen, \zB\ in Amerika, wo die Gefangenen langsam zu Tode gemartert, mit Messern zerschnitten werden, \usw; die schreck\RWSeitenw{306}lichen Mißhandlungen der Sclaven, die man bei vielen Völkern, \zB\ gleich bei den Spartanern, den Römern \uA , als eine Waare, und nicht als Menschen ansah. \RWlat{Res, non personae}, hießen sie im römischen Rechte ausdrücklich. Hieher gehöret ferner das Tödten der altgewordenen Eltern in einem großen Theile von Südamerika, auch, wie Eusebius erzählt, bei den \RWbet{Massageten}; der Genuß des Menschenfleisches bei den Karaiben, die ihre eigenen Kinder verzehren, und bei so vielen anderen Völkern, \zB\ den Brasilianern, und selbst bei den gutmüthigen Taheiten; das Verbrennen der Weiber beim Ableben ihrer Ehegatten, bei den Indiern in Guinea und anderwärts; das Abschlachten ganzer Herden unschuldiger Sclaven beim Tode ihres Herrn, das nicht nur in Guinea, sondern auch bei den gesitteten Römern gebräuchlich war; die Vielweiberei, und die so \RWbet{harte Behandlung, welche das weibliche Geschlecht} im ganzen Morgenlande erfährt; die grausamen Schauspiele der Römer, die Fechterspiele, die Kämpfe der Menschen mit wilden Thieren (selbst der menschenfreundliche Titus gab solche Schauspiele, und erst Kaiser Honorius schaffte sie gänzlich ab); die übertriebene Verehrung und Gewalt, die man den Kaisern einräumte (die Römer apotheosirten ihre Imperatoren bei ihren Lebzeiten; Domitian wagte, seine Edicte mit den Worten anzufangen: \RWbet{Unser Herr und Gott gebietet}, \usw; Caligula ließ sich selbst einen Tempel erbauen); \usw
\item \RWbet{Bei den Gelehrten}.\par
Selbst die Weltweisen ließen sich wichtige Irrthümer in der Moral zur Schuld kommen.
\begin{aufzb}
\item Kaum gelang es den weisesten Männern des Alterthums, das \RWbet{wahre innere Wesen der Tugend} aufzufassen. Die so berühmte Tugend der \RWbet{Stoiker}, gewiß die vorzüglichste, welche das Alterthum kannte, war doch noch weit entfernt von wahrer echter Tugend; sie forderte eine zweckwidrige Verläugnung der unschuldigsten Gefühle und war, genau betrachtet, nur ein versteckter Hochmuth, der sich der menschlichen Natur schämte. -- Noch weit unwürdiger des Namens einer Tugend war jene der \RWbet{Epikuräer}, die eigentlich nichts Anderes als kluge Sinn\RWSeitenw{307}lichkeit war. -- Die Tugend, die der vergötterte \RWbet{Plato}, der weise \RWbet{Aristoteles}, der schon am Ende der heidnischen Zeiten lebende \RWbet{Cicero} lehrten, war größtentheils nur eine \RWbet{bürgerliche Tugend}, \dh\ sie erstreckte sich fast nur auf solche Pflichten, welche der Mensch als Bürger hat, nicht auf dasjenige, was er den Menschen, die außerhalb seines Vaterlandes leben, und endlich selbst allen übrigen Geschöpfen schuldig ist. Eine bedeutende Anzahl von Gelehrten älterer sowohl als neuerer Zeit hoben durch Läugnung der Freiheit zugleich auch alle Sittlichkeit und Zurechnung auf. So der Arzt la Mettrie in den Schriften: \RWlat{l'homme une machine,\RWlit{}{LaMettrie1} une plante\RWlit{}{LaMettrie2}}, \umA
\item Um nun auch Beispiele von \RWbet{einzelnen Pflichten} zu geben, deren Daseyn die Philosophen verkannten, führe ich folgende an:
\begin{aufzc}
\item \RWbet{Sokrates, Plato, Seneca} \uma\ heidnische Weltweisen, obgleich sie das Irrige der Vielgötterei einsahen, erklärten es doch für erlaubt, die Götter nach der herrschenden Landessitte, die manchmal eine sehr unsittliche war, zu verehren. Seneca sagt: \erganf{Dergleichen Dinge wird der Weise beobachten, bloß weil sie durch die Gesetze vorgeschrieben sind, nicht aber darum, als wenn die Beobachtung derselben den Göttern gefällig wäre. Jene ganze Schaar der Götter, die ein verjährter Aberglaube seit uralten Zeiten zusammengebracht hat, werden wir so anbeten, daß wir dabei bedenken, ihre Verehrung sey mehr ein Gegenstand der hergebrachten Sitte als der Pflicht.} -- Cicero schreibt (\RWlat{de divinat.\ 2.\,33.})\RWlit{}{Cicero3}: \erganf{Das Alterthum irrte in vielen Dingen, doch wird die hergebrachte Sitte, Religion, Disciplin, Gesetzgebung, und das schon in uralten Zeiten gestiftete Collegium der Wahrsager, theils um der Meinung des Volkes willen, theils wegen der großen Vortheile, welche die Republik daraus zieht, beibehalten.}
\item Die Würde des weiblichen Geschlechtes, die Gleichheit seiner Ansprüche auf den Genuß aller irdischen Güter, haben die meisten heidnischen Weltweisen und Gesetz\RWSeitenw{308}geber verkannt; daher daß sie die Ehescheidungen so sehr begünstigen, daß Plato in seiner idealischen Republik eine Gemeinschaft der Weiber eingeführt wissen wollte, daß sie bestimmte Personen des weiblichen Geschlechtes der öffentlichen Unzucht widmeten; \usw
\item Den Werth der Ehre, des Ruhmes und Nachruhmes haben die meisten heidnischen Weltweisen viel zu hoch angesetzt, und die Begierde darnach als die einzige Triebfeder zur Tugend angesehen, ja mit der Tugend selbst verwechselt.
\item Ein gewisses \RWbet{Uebermaß} im Genusse von Speise und Trank erklärten Mehre für erlaubt. Sagt Horaz doch selbst von dem strengen Cato: \RWlat{virtus Catonis incaluit mero}.
\item Den \RWbet{Selbstmord }nahmen sie häufig in Schutz. Seneca vertheidigt und erhebt ihn an mehren Orten seiner Schriften, \zB\ \RWlat{epist.\,70}\RWlit[ -- die Worte in den runden Klammern sind von Bolzano so eingefügt, die eckigen Klammern zeigen Auslassungen Bolzanos an.]{215--216}{Seneca4a}: \erganf{\RWlat{Non vivere bonum est, sed bene vivere. Itaque sapiens vivit, quantum debet, non quantum potest. Videbit, ubi victurus sit, cum quibus, quomodo, quid acturus: cogitat semper, qualis vita, non quanta sit. Si multa occurrunt molesta, et tranquillitatem turbantia, emittit se; nec hoc tantum in necessitate ultima facit, sed cum primum illi coeperit suspecta esse fortuna, diligenter circumspicit, numquid illo die desinendum sit. Nihil existimat sua referre, faciat finem, an accipiat, tardius fiat an citius. -- -- Itaque effeminatissimam vocem illius Rhodii (Telesphori) existimo, qui, cum in caveam conjectus esset a tyranno (Lysimacho), et tanquam ferum aliquod animal aleretur, suadenti cuidam, ut abstineret cibo: omnia, inquit, homini, dum vivit, speranda sunt.}}\RWuebers{%
\anf{Nicht zu leben ist gut, sondern gut zu leben. Daher lebt der Weise, solange er muss, nicht solange er kann.} [Quelle?]} \Usw\ Etwas gemäßigter drückt er sich \RWlat{epist.\,58.}\RWlit{174}{Seneca4a} aus: \erganf{\RWlat{Morbum morte non fugiam, duntaxat sanabilem, nec officientem animo: non afferam mihi manus propter dolorem; sic mori, vinci est. Hunc tamen si sciero, perpetuo mihi esse patiendum,~\RWSeitenw{309}\ exibo non propter ipsum, sed quia impedimento mihi futurus est ad omne, propter quod vivitur.}}
\item Wie unvollkommen ihre Begriffe über die \RWbet{Pflicht der Feindesliebe} gewesen, mögen folgende Stellen beweisen. In den \RWbet{Gnomen}, die dem Theognis zugeschrieben werden, wird (v.\,363) der Rath ertheilt: \erganf{Schmeichle dem Feinde; wenn er aber in deiner Gewalt ist, dann räche dich ohne Nachsicht.} \RWbet{Cicero} \RWlat{de officiis l.\,3.\ c.\,19}\RWlit{}{Cicero3} sagt: \erganf{Der ist ein guter Mann, der Jedem nützt, dem er nützen kann, und Niemandem schadet, so lange er nicht durch Beleidigungen gereizt worden ist.} Und (\RWlat{l.\,1.\ c.\,42.}): \erganf{Gegen Menschen von fremden Nationen fordert die Gerechtigkeit, daß man ihnen gebe, was uns nichts kostet, Feuer, Licht, \usw }
\item Auch ihre Ansichten über den \RWbet{Werth der irdischen Güter} waren sehr schwankend. Denn so sehr es die \RWbet{Stoiker} in der Verachtung aller irdischen Güter übertrieben, so sehr übertrieben es die \RWbet{Epikuräer} wieder in ihrer Hochschätzung, da sie die Wollust, und zwar die körperliche, zum höchsten Gute auf Erden erhoben.
\end{aufzc}
\end{aufzb}
\end{aufza}
   
\RWpar{117}{5.~In Hinsicht auf die Asketik}
\begin{aufza}
\item \RWbet{Beim Volke.}\par
Nicht nur, daß man die große Volksmenge beinahe nirgends belehrte, wie eigentlich und durch welche Mittel der Mensch sein Herz bessern, die bösen Leidenschaften leichter im Zaume halten könnte, \udgl: so waren im Gegentheile Gebräuche eingeführt, die recht verderblich einwirken mußten, \zB\ die Verehrung so vieler schändlicher Gottheiten, die albernen Genugthuungsmittel, die man sie in dem Gebrauche eines Bades, in der Verrichtung eines Menschenopfers \udgl\  finden lehrte.
\item \RWbet{Bei den Weisen.}\par
Die \RWbet{Stoiker} glaubten, daß man, um tugendhaft zu werden, eine gänzliche \RWbet{Fühllosigkeit} (Apathie) gegen alle Freuden und Leiden in sich hervorbringen müsse; -- die \RWbet{Neuplatoniker} (im zweiten Jahrhunderte, von \RWbet{Plotinus} in Alexandrien gestiftet) fanden das vornehmste Mittel zur Tugend darin, daß man den Körper durch Fasten,~\RWSeitenw{310}\ Kasteiungen \udgl\ möglichst zu schwächen und zu entkräften suche. \Usw\par Wenn also weder die große Menge noch die Gelehrten den Inhalt der natürlichen Religion jemals in der gehörigen Reinheit und Vollständigkeit erkannten, sondern so vielen und so groben Irrthümern anhingen: so erhellet, wie viele Ursache unser Geschlecht auch schon dann haben würde, das Geschenk einer göttlichen Offenbarung dankbar anzunehmen, wenn diese nichts Anderes als die bloßen Wahrheiten der natürlichen Religion enthielte.
\end{aufza}
   

\RWabs{II}{Aus der Natur der Sache}
\RWpar{118}{Uebersicht dieses Beweises}
Aus dem Bisherigen (\RWparnr{113--117}) war zu ersehen, daß die sich selbst überlassene Vernunft der Menschen den Inhalt der natürlichen Religion zu keiner Zeit weder \RWbet{vollständig} noch \RWbet{rein} und \RWbet{unverfälscht} erkannt habe. Es muß doch ein Grund seyn, der dieß verhindert hat. Wir werden ihn entdecken, wenn wir nur etwas tiefer in die Natur der Sache selbst eindringen. Dann nämlich wird sich zeigen, \RWbet{daß eine allgemeine Verbreitung der natürlichen Religion in gehöriger Reinheit und Vollständigkeit gewissen Schwierigkeiten unterliege, denen nicht anders, als durch eine göttliche Offenbarung selbst abgeholfen werden kann}. Indem wir dieß darthun, werden wir einen \RWbet{neuen Beweis} für die Nützlichkeit der letzteren erhalten. Es gibt aber überhaupt zwei Wege, auf denen man sich die Ausbreitung der natürlichen Religion vorstellen kann. Der Eine, den ich den \RWbet{Weg der Ueberzeugung} nennen will, wird eingeschlagen, wenn man die Menschen durch vernünftigen Unterricht dahin bringt, daß sie die Lehren der natürlichen Religion aus eigener Einsicht, \dh\ nicht bloß darum annehmen, weil sie ein \RWbet{Anderer} von ihrer Wahrheit versichert, sondern weil sie die Wahrheit derselben durch eigenes Nachdenken erkannten. Der andere Weg, den ich den \RWbet{Weg des fremden Zeugnisses} nenne, wird betreten, wenn man die Menschen dahin bringt, daß sie die Wahrheiten der natürlichen Religion um irgend eines Zeugnisses willen (\zB\ eines Menschen, oder der Gottheit) annehmen. Ich werde nun zeigen, daß und warum es seine Schwierigkeiten habe, die Wahrheiten der natürlichen Religion allgemein auszubreiten, man mag diese Ausbreitung auf dem Einen oder dem andern dieser beiden Wege versuchen.~\RWSeitenw{311}

\RWpar{119}{1.~Schwierigkeiten, die der Verbreitung der natürlichen Religion auf dem Wege der Ueberzeugung entgegenstehen}
Ich sage keineswegs, daß es schon darum schwer oder wohl gar unmöglich wäre, die Wahrheiten der natürlichen Religion auf dem Wege der Ueberzeugung auszubreiten, \RWbet{weil diese Wahrheiten etwa auf Schlüssen beruhen, die nur von wenigen besonders scharfsinnigen Menschen begriffen werden könnten}. Nein! die Wahrheiten der Vernunftreligion und überhaupt alle Wahrheiten solcher Art, die für den Einen wie für den andern Menschen, für Ungelehrte wie für Gelehrte von \RWbet{gleicher Wichtigkeit} sind, beruhen, insoweit sie dem Menschen überhaupt erkennbar sind, meistens auf sehr einfachen und leicht faßlichen Gründen, so daß der größte Gelehrte in Ansehung ihrer nichts oder wenig vor dem Ungelehrten voraus hat. Aber obgleich so ziemlich alle Menschen die Fähigkeit haben, die Wahrheiten der natürlichen Religion zu fassen: so findet sich doch
\begin{aufza}
\item die erste Schwierigkeit, welche der Ausbreitung derselben auf dem Wege der Ueberzeugung entgegensteht, darin, daß der gewöhnliche Mensch das ernste Nachdenken scheuet, das zur Auffindung dieser Wahrheiten erfordert wird, daß er nicht einmal einem Unterrichte, der ihm nichts Anderes als dergleichen trockene Belehrungen verspricht, ein aufmerksames Ohr leiht. Denn in der That braucht man sich eben nicht viel in der Welt umzusehen, um sich zu überzeugen, daß es nur eine geringe Anzahl von Menschen gibt, die ein ernstes Nachdenken über sich selbst und ihre Pflichten, über Gott,  Unsterblichkeit \ua\,dgl.\ Gegenstände lieben, oder die nur geneigt wären, sich bei dem Unterrichte eines Andern einzufinden, der ihnen eben nichts Unterhaltenderes, als Untersuchungen von diesem Inhalte verspricht. Schon darum also, weil es uns kaum gelingen wird, die Menschen zur Aufmerksamkeit, und zu einem ernsten Nachdenken über religiöse Wahrheiten zu bringen, wird es auch schwerlich gelingen, die natürliche Religion in einiger Reinheit~\RWSeitenw{312}\ und Vollständigkeit unter den Menschen auszubreiten. -- Durch eine Offenbarung könnte dieser Schwierigkeit dadurch sehr abgeholfen werden, weil eine solche auch ihre trockensten Lehren durch die Verbindung mit gewissen Erzählungen, welche die Aufmerksamkeit selbst des rohesten Menschen auf sich ziehen; vornehmlich aber durch die Erzählung von jenen Wundern, die sich zu ihrer Bestätigung zugetragen haben, reizend und wichtig machen könnte.
\item[\RWbet{Einwurf.}] Dieses Mittels könnte man sich auch ohne Offenbarung bedienen. Man könnte ja die Wahrheiten der natürlichen Religion am Leitfaden einer Geschichte vortragen; man könnte die merkwürdigsten Ereignisse, welche die ganze Geschichte der Menschheit aufzuweisen hat, benützen, um jene Wahrheiten an ihnen zu versinnlichen, oder wenn etwa wirkliche Begebenheiten nicht anziehend genug seyn sollten, so könnte man selbst zu Erdichtungen seine Zuflucht nehmen.
\item[\RWbet{Antwort.}] Erdichtete Erzählungen haben bei Weitem nicht das Interesse, das eine wirkliche Geschichte hat. Und selbst die merkwürdigsten Ereignisse, die uns die menschliche Geschichte aufweisen kann, sind nicht so anziehend, wie jene Wunderbegebenheiten, die zur Bestätigung einer Offenbarung dienen.
\item Obgleich es wahr ist, daß die nothwendigsten Lehren der natürlichen Religion auf so gemeinfaßlichen Gründen beruhen, daß selbst der Ungelehrteste sie zu begreifen vermag: so ist es doch eben so wahr, daß es eine Menge ihnen entgegenstehender \RWbet{Irrthümer} gibt, die ihre Scheingründe gleichfalls für sich haben, und zwar Scheingründe solcher Art, deren gänzliche Nichtigkeit der minder geübte Verstand nicht so leicht einzusehen vermag, besonders dann, wenn zu den Zweifeln des Verstandes sich auch noch ein geheimes Interesse des Herzens hinzugesellt. Dieß Letztere ist nun sehr häufig der Fall, indem die meisten Wahrheiten der natürlichen Religion, besonders aber die praktischen, die Sinnlichkeit des Menschen einschränken. Je sinnlicher also der Mensch ist, um desto mehr setzt seine Leidenschaft sich ihrer Anerkennung entgegen, um desto eher gelingt es seinem Bestreben, sich selbst zu überreden, daß es nicht wahr, wenigstens noch nicht gewiß und~\RWSeitenw{313}\ ausgemacht sey, was ihm die bessere Vernunft und das Gewissen mit leiser Stimme zuruft. Auf diese Art geschieht es, daß auch sehr einleuchtende Wahrheiten der natürlichen Religion doch nicht zu einer allgemeinen Anerkennung bei allen Menschen gelangen. -- Eine Offenbarung kann dieser Schwierigkeit steuern, indem sie alle bestrittenen oder in Zweifel gezogenen Lehren der bloßen Vernunftreligion mit einer göttlichen Auctorität entscheidet.
\item[\RWbet{Einwurf.}] Um an eine Offenbarung zu glauben, muß man erst an das Daseyn Gottes, an seine Wahrhaftigkeit \umA\  glauben. Alle diese Wahrheiten, vollends aber die Kennzeichen, aus denen entschieden werden muß, ob eine gewisse Religion eine göttliche Offenbarung sey oder nicht, sind nicht so einleuchtend, daß es nicht Jedem, der sie verkennen \RWbet{will}, ein Leichtes wäre, sie zu verkennen. Jener Schwierigkeit wird also durch eine Offenbarung nicht im Geringsten abgeholfen; denn sinnliche Menschen werden die Ueberzeugung von ihrem Daseyn nicht bei sich aufkommen lassen.
\item[\RWbet{Antwort.}] Es ist -- wie in der Folge gezeigt werden soll -- nicht einmal wahr, daß der Glaube an eine Offenbarung den Glauben an Gott, an seine Wahrhaftigkeit \usw\ in der Art schon voraussetzt, daß er durch jene nicht erst noch befestigt werden könnte. Die \RWbet{Wunder} aber, in denen das vornehmste Kennzeichen einer Offenbarung bestehet, ziehen die Aufmerksamkeit auch des sinnlichsten Menschen an sich, erschüttern ihn, und flößen ihm eine gewisse Furcht ein vor dem Verbrechen, einen Glauben, der solche Wunder für sich hat, ohne gehörige Prüfung zu verwerfen. Endlich ist es, selbst der Erfahrung zu Folge, viel leichter, sich von der Einen Wahrheit: \RWbet{Dieses oder jenes ist eine göttliche Offenbarung}, zu überzeugen, als von so vielen Wahrheiten der natürlichen Religion, deren jede auf einem \RWbet{eigenen Grunde} beruhet.
\item  Ohne Offenbarung bleibt die Vernunft, wie wir schon oben (\RWparnr{99--111}) gesehen, über so manche wichtige Frage in Zweifel. Weil aber der Zustand des Zweifels dem Menschen lästig ist, und insbesondere seine \RWbet{Eitelkeit} kränkt, so ergreifen die Meisten lieber irgend eine nur halb erwiesene Behauptung, als daß sie im Zustande der Unentschieden\RWSeitenw{314}heit verblieben. Und so kommen denn dort, wo keine Offenbarung herrscht, bald eine Menge der verschiedensten oft sehr gewagten, oft auch sehr ungereimten Vermuthungen zum Vorschein, vermittelst deren die Einbildungskraft die Lücken der gründlicheren Erkenntniß auszufüllen bestrebt ist, und die der bestochene Verstand im Kurzen gleich ausgemachten Wahrheiten annimmt. Aus diesem Grunde vermag sich die bloße Vernunft-Religion bei keinem Volke \RWbet{in der gehörigen Reinheit} zu erhalten, sondern wird immer sehr bald durch allerlei Zusätze entstellt, welche gewöhnlich um so verderblicher für die Sittlichkeit ausfallen, je größer der Antheil ist, den menschliche Leidenschaften an ihrer Entstehung nehmen. -- Durch eine Offenbarung kann dieses Uebel zwar nicht ganz verhindert werden, indem auch sie noch manche Fragen der menschlichen Neugierde unbeantwortet läßt und lassen muß; da sie uns aber doch über sehr Vieles und gerade über das Wichtigste gehörig befriediget: so bleibt den Dichtungen unserer Einbildungskraft kein so weiter und gefährlicher Spielraum mehr übrig.
\end{aufza}\par
   
\RWpar{120}{2.~Schwierigkeiten, die der Ausbreitung der natürlichen Religion auf dem Wege des Zeugnisses entgegenstehen}
Da es aus obigen Gründen so viele Schwierigkeiten hat, die natürliche Religion in der gehörigen Reinheit und Vollständigkeit allgemein auszubreiten, wenn man den Weg der Ueberzeugung eines Jeden einschlagen will: so bleibt nichts Anderes übrig, als daß man das Mittel des \RWbet{fremden Zeugnisses} benütze, und den Wahrheiten der natürlichen Religion dort, wo die eigene Einsicht nicht auslangen will, durch das Ansehen eines Zeugen aufzuhelfen suche. Soll dieses ohne die Dazwischenkunft einer göttlichen Offenbarung geschehen; so ist einleuchtend, das Zeugniß, dessen man sich bedient, darf nur ein menschliches seyn. Denn beriefe man sich auf das Zeugniß Gottes; so würde man eben hiedurch eine, wenn gleich von Seite des Urhebers nur vorgebliche göttliche Offenbarung einzuführen versuchen. Ich behaupte aber, daß ein bloß menschliches Ansehen zu diesem großen Zwecke nicht zureichend sey, und dieß aus folgenden Gründen:~\RWSeitenw{315}
\begin{aufza}
\item Es ist zwar nicht zu läugnen, daß auch das bloße menschliche Zeugniß unter gewissen Umständen, besonders wenn es nur über bloße sinnliche Wahrnehmungen (Facta) abgelegt werden soll, einen hinlänglichen Grad der Gewißheit erreichen könne. Beruhet doch auch die göttliche Offenbarung selbst auf solchen Zeugnissen der Menschen, in sofern wenigstens, als wir die Wunder, die sich zu ihrer Bestätigung zugetragen haben, nicht alle selbst wahrgenommen, sondern auf fremdes Zeugniß glauben. Viel unbefriedigender aber ist menschliches Zeugniß dort, wo es über allgemeine Vernunftwahrheiten, vornehmlich eines religiösen Inhaltes, abgelegt werden soll. Denn über solche Wahrheiten will die Vernunft eines jeden Menschen ein gleiches Recht zur Entscheidung haben. \eanf{Kann ich dieß nicht einsehen}, heißt es, \eanf{wie willst du es eingesehen haben?} Wollte man aber, um den, der also fragt, zu bescheiden, erwidern, daß es sich hier um eine Wahrheit handle, die nur durch tiefes Nachdenken vermittelst eines mehr als gewöhnlichen Grades von Scharfsinn erkannt werden könne: so wäre schon dieser Umstand ein hinreichender Grund, um demjenigen, der diese Wahrheit entdeckt zu haben vorgibt, nicht mit so ganzer Zuversicht zu trauen, wie dem, der irgend eine sinnliche Erscheinung so oder anders wahrgenommen zu haben betheuert. Der Erstere konnte sich nämlich viel leichter irren als der Letzte. Daher sehen wir denn, daß nicht bloß religiöse, sondern auch andere Lehren, selbst mathematische, auf das Wort der Gelehrten nicht unbedingt von der übrigen Menge der Menschen angenommen werden; sondern nur solche Behauptungen der Gelehrten nimmt die große Menge mit voller Zuversicht an, von deren Wahrheit sie sich durch sinnliche Beispiele wenigstens einiger Maßen selbst überführen kann; andere Lehren, \zB\ was unsere Astronomen über die Größe und Entfernung der Himmelskörper, \udgl\  mit größter Uebereinstimmung vortragen, erlaubt sich der gemeine Mann immer in Zweifel zu ziehen. Und dieß sind doch Wahrheiten, vor deren Annahme sich unsere sinnliche Natur nicht sträubt. Um wie viel weniger läßt sich erwarten, daß religiöse Wahrheiten auf das bloße Wort der Gelehrten jemals mit aller hier nöthigen Zuversicht geglaubt werden würden! -- Bei einer Offenbarung fällt diese Schwierigkeit weg. Hier nämlich legen die Menschen zwar auch ein Zeugniß ab, allein~\RWSeitenw{316}\ nicht über Vernunftwahrheiten, sondern über bloße \RWbet{sinnliche Wahrnehmungen}.
\item Nebst dieser eben erwähnten Schwierigkeit, welche in der Natur der Sache selbst liegt, und daher niemals ganz zu vermeiden wäre, besteht bis auf den heutigen Tag noch eine andere, nämlich das sehr geringe Ansehen, dessen sich unsere Weltweisen bei der übrigen Menge des Volkes zu erfreuen haben. Denn wenn die natürliche Religion auf menschliches Zeugniß verbreitet werden sollte: so ist begreiflich, daß die geschicktesten Zeugen hiezu die Weltweisen, \dh\ diejenigen Personen wären, die schon von Natur mit mehr als gewöhnlichen Gaben des Geistes ausgerüstet, ihr ganzes Leben nur dem edlen Zwecke der Aufsuchung wichtiger Wahrheiten widmen. Allein gerade diese Classe der Menschen hat sich bis jetzt noch in kein besonderes Ansehen bei der großen Volksmenge zu setzen gewußt. Daran mag Ursache seyn:
\begin{aufzb}
\item daß selten Personen vornehmer Stände unter dieser Classe erscheinen, indem diese selten den dornigten Pfad der Weltweisheit betreten;
\item daß unsere größten Weisen den äußeren Prunk und Zierath gewöhnlich verachten, während die große Menschenmenge doch immer nur diejenigen zu schätzen weiß, welche viel äußern Glanz um sich her verbreiten;
\item daß sie ferner sich auch keine Reichthümer sammeln; noch
\item hohe Aemter und Würden durch Schleichwege, Bestechungen und Schmeicheleien zu erlangen suchen, \udgl\ 
\item daß endlich mehre aus ihnen sich nebst der eigentlichen (praktischen) Weltweisheit auch noch mit vielen sehr unfruchtbaren, und von dem gemeinen Manne sogar für ungereimt gehaltenen rein wissenschaftlichen Untersuchungen (mit sogenannter speculativer Philosophie) befassen, oder daß wenigstens Weltweise und speculative Philosophen noch häufig einerlei Namen führen, und mit einander verwechselt werden. Die Letzteren aber können aus sehr begreiflichem Grunde bei der großen Menschenmenge niemals in hoher Achtung stehen; denn es sind größtentheils Menschen, die für die Welt entweder wirklich, oder doch~\RWSeitenw{317}\ scheinbar ganz unbrauchbar sind, Menschen, die bei den gemeinsten Vorfällen des Lebens die auffallendste Unbehülflichkeit verrathen, unter einander in ewigem Streite liegen, und häufig Dinge behaupten, die dem gemeinen Manne die aufgelegtesten Ungereimtheiten scheinen, ja es zuweilen auch wirklich sind. --
\end{aufzb}
\end{aufza}\par
Bei einer Offenbarung würde auch diese Schwierigkeit wegfallen; denn eine solche wird durch Personen gepredigt, die sich für göttliche Gesandte, oder doch wenigstens für Diener Gottes erklären. Obgleich nun nicht zu läugnen ist, daß der Stand der Diener Gottes, der Priester, schon sehr oft gemißbraucht wurde, ja obgleich, wie \RWbet{Haller} sagt, \RWbet{kein Böses ist geschehen, so nicht ein Priester that}: so stehen doch diese Art Menschen noch immer in einem viel höheren Ansehen, als jene Weltweisen; indem es zu offen vorliegt, daß alles Böse, das sich die Mitglieder dieses Standes etwa erlaubt haben mögen, nur ihrer Persönlichkeit, nie aber ihrem Stande selbst zur Last gelegt werden dürfe.

\RWpar{121}{Diese Schwierigkeiten können und dürfen durch keinen frommen Betrug gehoben werden}
Da nun, wie ich so eben gezeigt, die Wahrheiten der natürlichen Religion auf dem Wege der Ueberzeugung nicht ohne die größte Schwierigkeit allgemein ausgebreitet werden können; ihrer Verbreitung auf dem Wege des Zeugnisses aber der Umstand entgegenstehet, daß menschliches Ansehen in diesem Stücke zu schwach ist, um vollen Glauben zu fordern und zu finden: so könnte man etwa auf den Gedanken verfallen, dem, was das menschliche Nachdenken gefunden hat, durch vorgespiegelte Eingebungen und Wunder das Ansehen einer göttlichen Offenbarung zu ertheilen. Dieß wäre zwar ein Betrug; aber um seiner guten Absicht willen dürfte er sich, meint man vielleicht, entschuldigen lassen, und würde den Namen eines \RWbet{frommen} mit vollem Rechte verdienen.\par
Hierauf entgegne ich:
\begin{aufza}
\item Wer die Nothwendigkeit eines Betruges von dieser Art behauptet, der gibt schon eben hiedurch die Nützlichkeit einer Offenbarung -- obwohl einer bloß vorgegebenen -- zu.~\RWSeitenw{318}\ Wenn aber diese schon den Menschen nützlich, ja sogar nothwendig seyn sollte: um wie viel nützlicher müßte nicht eine wahre seyn, welche gewiß noch manche andere Vortheile zugleich gewähren könnte, und deren Wunder von einer solchen Art seyn würden, daß sie doch keinem Verdachte einer Täuschung ausgesetzt wären.
\item Bei einer näheren Betrachtung zeigt sich jedoch, daß dieses Mittel auch durchaus unerlaubt wäre, und daß sich tugendhafte Personen desselben niemals bedienen dürften; denn eine richtig urtheilende Vernunft verbietet jeden Betrug, um wie viel mehr einen Betrug in so wichtigen Dingen, der auch so unsicher wäre, und wenn er entlarvt würde, die Menschen mißtrauisch machen müßte selbst gegen dasjenige, was völlig wahr und richtig ist, durch welchen endlich auch, wenn man ihn zulassen wollte, den schändlichsten Mißbräuchen Thor und Riegel geöffnet würden!
\end{aufza}
   
\RWpar{122}{Also ist uns eine formelle sowohl, als materielle Offenbarung nöthig} 
Aus dem Bisherigen ersieht man nun zur Genüge, daß unserem Geschlechte Beides, eine \RWbet{formelle} sowohl, als eine \RWbet{materielle Offenbarung} erwünschlich sey.
\begin{aufzb}
\item Eine \RWbet{formelle}, weil es verschiedene Wahrheiten gibt, welche, ob sie zwar durch die bloße, sich selbst überlassene Vernunft erkannt werden könnten, doch nicht nur von der großen Menge der Menschen fast niemals, sondern selbst von den Gelehrten nur selten erkannt worden sind. (\RWparnr{113--117})
\item Aber auch eine \RWbet{materielle}, weil es auch mehre Fragen von großer Wichtigkeit gibt, deren Beantwortung die Kräfte der menschlichen Vernunft schlechterdings überschreitet. Z.\,B.\ die Fragen von der Vergebung der Sünden, vom Ursprunge und Zwecke des Uebels \uma\  (\RWparnr{99--111})
\end{aufzb}

\RWpar{123}{Ob das Bedürfniß einer Offenbarung für den Gebildeten oder Ungebildeten größer sey, und von wem es lebhafter empfunden werde?}
Aus den eben erwähnten Mängeln der natürlichen Religion ergibt sich die Nützlichkeit einer solchen Offenbarung, in~\RWSeitenw{319}\ der diese Mängel gehoben werden, auf eine einleuchtende Weise. Da ferner einige dieser Mängel wirklich von der Art sind, daß ihr Vorhandenseyn den sittlich denkenden Menschen nicht wenig beunruhiget: so können wir wohl ohne Uebertreibung sagen, daß es ein eigenes \RWbet{Bedürfniß} einer Offenbarung gebe. In dieser Hinsicht ließe sich aber die Frage aufwerfen, ob dieß Bedürfniß für den Gebildeten oder den Ungebildeten ein größeres sey? und von wem es lebhafter empfunden werden möge? -- Um diese Frage gehörig beantworten zu können, erinnere ich zuerst, daß zwischen den beiden Redensarten: ein \RWbet{Bedürfniß haben}, und \RWbet{es empfinden}, ein wichtiger Unterschied obwalte; daß überdieß auch das Wort \RWbet{Bedürfniß} selbst noch eine Zweideutigkeit habe, indem man sich seiner bedient, um die Nothwendigkeit einer Sache sowohl zur Erreichung eines gewissen Vortheiles, als auch zur Abwehrung eines bloßen Schadens zu bezeichnen. Ich sage nun:
\begin{aufza}
\item \RWbet{Ungebildete haben einen größern Schaden davon, wenn ihnen keine Offenbarung zu Theil wird; der Nutzen aber, den der Gebildete aus dem Besitze einer Offenbarung zu ziehen vermag, ist der größere.} Der Ungebildete ist eben, weil seine Geisteskräfte wenig entwickelt worden sind, nur wenig aufgelegt, den abstracten Wahrheiten der natürlichen Religion eine gehörige Aufmerksamkeit zu widmen; er ist noch weniger fähig, den Täuschungen seiner Einbildungskraft und den in seinem Zeitalter und Lande herrschenden Vorurtheilen und Irrthümern zu widerstehen. Ueber Kurz oder Lang wird er sonach ein Raub des Aberglaubens, und hat somit einen sehr großen Schaden davon, daß ihm keine göttliche Offenbarung zu Hülfe gekommen ist. -- Einen so großen Schaden hat der Gebildete nicht zu befürchten, da sein höherer Grad der Aufklärung ihn das Irrige des Volksaberglaubens bald einsehen läßt, weßhalb nicht zu besorgen ist, daß er sich von den Irrthümern seiner Zeit so ganz werde hinreißen lassen. Hat er ein feines sittliches Gefühl, so wird schon dieses ihn lehren, selbst bei den zweifelhaftesten Sätzen der natürlichen Religion sich immer auf jene Seite zu neigen, wo mehr Gewinn für seine Tugend und Glückseligkeit zu hoffen ist. Aber so wahr dieses ist, so~\RWSeitenw{320}\ gewiß ist es auch von der andern Seite, daß, wenn einem solchen Manne das Geschenk einer göttlichen Offenbarung zu Theil würde, er einen noch ungleich größern Nutzen aus ihr zu schöpfen wissen würde als jeder Ungebildete. Der Letztere nämlich kann eben, weil seine Kräfte und Fähigkeiten nicht gehörig entwickelt sind, die höheren Lehren der Offenbarung nicht alle ganz so benützen, wie sie benützt werden sollten, und von Gebildeten wirklich benützt werden. Er faßt nur das Gröbere auf, fühlt nur die unmittelbarsten, die einleuchtendsten Folgen, die sich aus ihren bildlichen Lehren ergeben; doch alles Höhere entgeht ihm. Nur der Gebildete verstehet es ganz, was Gott zu ihm gesprochen, und weiß es anzuwenden.
\item \RWbet{Obgleich aber der Schaden, welchen der Ungebildete durch die Entbehrung einer Offenbarung leidet, der größere ist: so ist doch seine Empfindung dieses Schadens, und sein Wunsch nach einer Offenbarung schwächer, als bei dem Gebildeten, der dieß Bedürfniß stärker zu fühlen vermag}. Ungebildete Menschen denken insgemein nicht viel nach über ihren Zustand, und sie vermögen nicht leicht, sich eine deutliche Vorstellung von einem andern bessern Zustande zu machen. Dieß hat zur Folge, daß sie zufrieden sind mit ihrer Lage, auch wenn sie eben nicht die beste seyn sollte. Der Gebildete dagegen denkt nach, sieht, daß es besser seyn könnte, und sehnet sich nach diesem Besseren. Je höher der Grad seiner Bildung, je edler sein Herz ist, um desto stärker ist sein Wunsch, von gewissen übersinnlichen Dingen ein Mehres und ein Sichreres zu wissen. Je mehr er \zB\ vermag, sich von den sinnlichen Gegenständen, die ihn umgeben, loszureißen, um desto Mehres wünscht er von Gott und der Geisterwelt zu wissen; je größere Opfer er der Tugend zu bringen gedenkt, um desto wichtiger ist es für ihn, von der Unsterblichkeit seiner Seele auf das Gewisseste überzeugt zu werden; je mehr sich seine Wirksamkeit und sein edles Bestreben auf das Ganze der Menschheit erstreckt, um desto dringender wird für ihn die Frage, ob es auch möglich sey, das menschliche Geschlecht in seiner Vollkommenheit weiter zu bringen; \usw ~\RWSeitenw{321}
\begin{RWanm}
Auch die Erfahrung dürfte mit dieser Entscheidung der Frage sehr übereinstimmen. Die \RWbet{große}, \dh\ die ungebildete Menge der Menschen hatte doch sicher den größten Schaden davon, wenn ihr zu irgend einer Zeit in irgend einem Lande keine göttliche Offenbarung zu Theil geworden war; sie war es, die sodann in die verderblichsten Irrthümer und in die schändlichsten Laster und Ausschweifungen verfiel: gleichwohl bemerken wir nicht, daß sie sich jemals nach einer Offenbarung besonders gesehnet hätte. Nur die Gebildetsten und die Edelsten unseres Geschlechtes, obgleich sie auch ohne Offenbarung vor jenen Thorheiten sich zu bewahren gewußt, waren doch gerade diejenigen, die sich am Innigsten nach einer Offenbarung sehnten, und die gefundene mit der dankbarsten Freude begrüßten. 
\end{RWanm}
\end{aufza}

\RWpar{124}{Ob dieß Bedürfniß einer Offenbarung jemals aufhören werde?}
\begin{aufza}
\item Wenn sich das menschliche Geschlecht von Jahrhundert zu Jahrhundert sowohl in wissenschaftlicher als sittlicher Hinsicht immer vervollkommnet: so könnte man vermuthen, es werde vielleicht einmal dahin kommen, daß unser Verstand, so unfähig er auch noch jetzt dazu seyn mag, alle die oben erwähnten Zweifel der natürlichen Religion aus sich selbst zu entscheiden im Stande seyn werde. Da nun in einem solchen Falle, so scheint es wenigstens, keine Offenbarung mehr für uns nöthig seyn würde: so erhebt sich die Frage, ob wohl mit Recht behauptet werden könne, daß das Bedürfniß einer Offenbarung jemals aufhören werde?
\item Gegen diese Frage erinnere ich zuvörderst, daß sie mir als eine Frage der bloßen Neugier erscheine. Es ist uns keineswegs nöthig, daß wir das wissen, und mit völliger Bestimmtheit wissen, was hier gefragt wird. Gesetzt, es wäre einem künftigen erst nach Jahrtausenden auf dieser Erde zu erscheinendem Geschlechte nicht nothwendig, das durch ein göttliches Zeugniß bestätigt zu erhalten, was uns die Offenbarung jetzt lehret; darum verdiente sie von uns jetzt Lebenden nicht weniger dankbar benützet zu werden, denn jetzt noch ist sie uns Bedürfniß und unschätzbare Wohlthat.
\item Allein es scheint nicht, daß unser menschliche Verstand bei allem Fortschreiten in der Vollkommenheit, welches~\RWSeitenw{322}\ ich ihm sehr gerne einräume, jemals zu einer ganz sichern Entscheidung aller der oben angeführten Zweifel in der natürlichen Religion gelangen werde. Denn es ist wohl zu bemerken, daß einige dieser Zweifel zu ihrer völligen Entscheidung einer Uebersicht des ganzen Weltgebäudes und aller Einrichtungen in demselben, oder doch wenigstens einer Uebersicht des ganzen menschlichen Geschlechtes und eines Vorhersehens aller einzelnen freien Handlungen bedürfen. So wäre \zB , um über die Frage von der Vergebung der Sünden zu entscheiden, nach Ausweis des \RWparnr{103} nöthig, daß man vorhersehen könnte, wie viele Menschen diese Vergebung zu ihrem Vortheile benützen, wie viele Andere sie zu ihrem eigenen Verderben mißbrauchen werden.
\item In jedem Falle bleibt wenigstens so viel gewiß, wenn auch der \RWbet{Schade}, der unserem Geschlechte aus der Entbehrung einer Offenbarung zuwächst, immer geringer werden sollte: so wird doch der \RWbet{Nutzen}, welchen dasselbe aus dem \RWbet{Besitze} einer solchen schöpfen kann, nicht nur nie ganz verschwinden, sondern im Gegentheile je später, je größer werden, weil die Empfänglichkeit für eine höhere Belehrung um desto größer werden wird, je mehr wir an Bildung überhaupt gewinnen. So viel wir auch noch in Zukunft lernen mögen, so wird doch Gott immer noch mehr als wir wissen, und immer noch wird es gar manches Nützliche, was er uns beibringen kann, geben.
\end{aufza}
   
\RWpar{125}{Geständnisse einzelner Weltweisen}
Daß wir in den vielen Schriften von einem so mannigfaltigen Inhalte, als uns die heidnischen Weltweisen hinterlassen haben, kaum Eine erweislich echte Stelle finden, in der sie ihr Verlangen nach einer höheren göttlichen Offenbarung ausgesprochen hätten, gereichet ihnen zu keiner besonderen Ehre, und zeigt, daß sie an tausend andere Dinge fleißiger, als an dasjenige gedacht, was ihren sittlichen Bedürfnissen hätte abhelfen können.
\begin{aufza}
\item Inzwischen derjenige aus ihnen, der uns als der Bescheidenste und Beste bekannt ist, der weise \RWbet{Sokrates}, scheint dieses Bedürfniß gleichwohl erkannt, und davon mehrmals zu seinen Schülern gesprochen zu haben, wie aus den Schriften \RWbet{Plato}'s hervorgeht. In dem Gespräche \RWbet{Timäus}~\RWSeitenw{323}\ legt dieser \RWbet{Plato} dem \RWbet{Timäus} folgende Worte in den Mund: \erganf{Eine sehr schwere Sache ist es, den Urheber und Vater des Weltalls ausfindig zu machen; wenn man ihn aber auch selbst gefunden hat, \RWbet{ihn allen Uebrigen bekannt zu machen}, halte ich für \RWbet{unmöglich}.} -- In den Gesprächen \RWbet{Aeschines} des Sokrates (deren Echtheit übrigens sehr zu bezweifeln) läßt dieser seinen Lehrer behaupten, daß \RWbet{ohne eine höhere Leitung} Sittlichkeit unter den Menschen unmöglich befördert werden könne. In dem Platonischen Gespräche \RWbet{Epinomis} sagt Sokrates, \RWbet{daß nur Gott allein der wahre Lehrer der Tugend sey.} -- In \RWbet{Alcibiades II.}, dessen Aechtheit man jedoch abermals sehr bezweifelt, kommt folgende Stelle vor:
\end{aufza}\par
\erganf{\RWbet{Sokrates} (der vom Gebete gesprochen, und dem Alcibiades gezeigt hatte, daß der Mensch im Grunde nicht einmal wisse, um was er die Götter bitten solle). Wir müssen also erwarten, bis Jemand kommt, der uns belehre, wie wir uns gegen die Götter und Menschen verhalten sollen.\par
\RWbet{Alcibiades.} Wann wird er kommen, o \RWbet{Sokrates}! dieser Lehrer, und wer wird er seyn? Ich wünschte sehr, ihn zu sehen.\par
\RWbet{Sokrates.}  Es ist der nämliche, der bereits Sorge für dich trägt. (Einige glauben, daß \RWbet{Sokrates} hier sich selbst gemeinet habe.) Aber wie dort \RWbet{Homer} von \RWbet{Athene} erzählt, daß sie den Nebel erst von \RWbet{Diomedis} Auge abnehmen mußte, sollte er die Göttliche vom Menschen unterscheiden: so muß der Nebel auch von deiner Seele erst hinweggenommen werden, bevor du das Gute vom Bösen zu unterscheiden vermagst.\par
\RWbet{Alcibiades.} O so nehme er den Nebel hinweg von mir! Ich bin bereit, Alles zu thun, was er mich heißen wird, wenn ich nur besser werde} \usw \par
In dem Gespräche \RWbet{Phädrus} heißt es: \erganf{Ich stimme dir vollkommen bei, o \RWbet{Sokrates}!, und ich glaube, eine vollständige Kenntniß dieser Dinge könne uns hienieden nicht gewährt werden. Aber nur schwache Seelen lassen nach, bevor sie nach Möglichkeit untersucht haben. Wir müssen entweder selbst nachdenken, um zur Beruhigung zu gelangen, oder wenn wir die Unmöglichkeit der Gewißheit einsehen, uns mit dem Wahrscheinlichsten begnügen, und unser Leben nach diesem einrichten, \RWbet{wenn anders uns nicht irgend ein Ausspruch der Götter sicherer leitet.}}~\RWSeitenw{324}
\begin{aufza}\setcounter{enumi}{1}
\item \RWbet{Jamblichus} -- ein Eklektiker des 3ten Jahrhundertes -- sagt (\RWlat{de vita Pythagorae cap.\,28.})\RWlit{}{Jamblichus1} ausdrücklich, es sey nicht so leicht zu wissen, was Gott gefällig sey, wenn man nicht etwa von Gott selbst, oder von einer Person, die mit Gott in einem nahen Umgange steht, belehrt worden ist.
\item Auch der berüchtigte \RWbet{Voltaire} legt einmal folgendes Geständniß über die Nothwendigkeit einer höheren Offenbarung ab: \erganf{\RWlat{Nous savons \RWbet{avec toute la terre}, qu'il y a du mal sur la terre, ainsi que du bien, savons, qu' aucun \RWbet{philosophe} n'a pas jamais expliqué l'origine du mal moral et du mal physique; disons, que la \RWbet{révélation seule} peut dénouer ce grand noeud, que tous les philosophes ont embrouillé. C'est \RWbet{le seul asyle}, auquel l'homme puisse recourir dans les ténèbres de sa raison et dans les calamités de sa nature faible et mortelle.}}
\item Auch \RWbet{Rousseau} schildert in seinem Emil\RWlit{}{Rousseau1} (l.\,3.\ c.\,23.) die Ungewißheit, in der sich die sich selbst überlassene Vernunft befindet, sehr stark, wenn er sich ausdrückt: \erganf{Wir Sterbliche schwimmen hier auf dem Meere menschlicher Meinungen ohne Steuer, ohne Magnetnadel, nur unseren stürmischen Leidenschaften überlassen, ohne einen andern Führer als einen unerfahrenen Steuermann zu haben, der seinen Weg nicht kennt, und weder weiß, woher er gekommen, noch wohin er gehet.}
\end{aufza}
   
\RWabs{III}{Prüfung der vornehmsten Einwürfe gegen die Nothwendigkeit einer höhern Offenbarung}
\RWpar{126}{1.~Einwurf. Aus der Vollkommenheit der natürlichen Religion}
Die bisher behauptete Nothwendigkeit einer göttlichen Offenbarung ist von verschiedenen Gelehrten neuerer Zeit nicht nur nicht zugestanden, sondern durch Gründe angegriffen worden, welche fast durchaus so abgeschmackt sind, daß es nur~\RWSeitenw{325}\ der Umstand, weil es sich hier um die Bestreitung einer der sinnlichen Natur des Menschen widerstreitenden Lehre handelt, begreiflich macht, wie Männer, die doch für Weltweise angesehen werden wollten, so alberne Einwürfe vorbringen konnten; und wie das, was sie vorbrachten, andererseits nicht sogleich mit der verdienten allgemeinen Verachtung zurückgewiesen wurde. Ich will diejenigen Einwürfe, die noch den meisten Anschein von Gründlichkeit haben, hier anführen, und mehr zu einer Uebung im Nachdenken, als weil ich besorge, daß sie irgend Jemand irre leiten könnten, der nicht selbst irre geleitet werden will, sie kürzlich widerlegen.\par
\RWbet{Einwurf.} Die natürliche Religion ist Gottes Werk; Gottes Werk muß vollkommen seyn; was vollkommen ist, muß seinem Zwecke entsprechen. Der Zweck jeder Religion ist Anleitung zur Tugend und Glückseligkeit. Folglich muß die natürliche Religion zu diesem Zwecke hinreichen, und also ist jede Offenbarung entbehrlich; und wer die Nothwendigkeit einer Offenbarung behauptet, läugnet die Vollkommenheit der Werke Gottes.\par
\RWbet{Antwort.} In diesem Einwurfe ist Alles richtig behauptet, bis zu dem Satze, \RWbet{daß der Zweck jeder Religion Anleitung zur Tugend und Glückseligkeit sey}. Dieses ist nämlich unbestimmt ausgedrückt, und kann daher wahr und falsch seyn, je nachdem man es auslegt. Wahr ist es, daß jede Religion den Zweck, zur Tugend und Glückseligkeit zu leiten habe, wenn man es so auslegt, daß eine jede zur Hervorbringung dieser Wirkung so Vieles beitragen soll, als sie ihrer Natur nach vermag. Falsch aber ist es, wenn man sich vorstellt, Gott habe gewollt, daß jede Religion, und also auch die natürliche, \RWbet{vollkommen hinreichend} zu dieser Wirkung sey. \RWbet{Es ist an sich unmöglich}, daß die bloße Vernunftreligion für die Beförderung der Tugend und Glückseligkeit Alles das leiste, was eine Offenbarung zu leisten vermag. Folglich hat Gott dieß auch nicht gewollt, und mithin ist es nicht die Bestimmung der natürlichen Religion; also ist sie \RWbet{auch nicht unvollkommen zu heißen}, wenn sie dieß nicht leistet. Immerhin also kann man behaupten, daß eine Offenbarung dem menschlichen Ge\RWSeitenw{326}schlechte nützlich oder gar nothwendig sey, ohne die Weisheit Gottes und sein Werk, die natürliche Religion, herabzusetzen oder zu tadeln.

\RWpar{127}{2.~Einwurf. Die Nothwendigkeit einer Offenbarung streitet mit Gottes Allmacht und Weisheit}
Wenn eine höhere Offenbarung dem menschlichen Geschlechte nothwendig seyn sollte; so müßte Gott nicht \RWbet{allmächtig}, oder \RWbet{nicht höchst weise} seyn, daß er den menschlichen Geist nicht vollkommener erschaffen. Denn sicher würde es mehr Macht und Weisheit verrathen, wenn Gott das menschliche Erkenntnißvermögen so eingerichtet hätte, daß wir die Wahrheiten, die uns zu wissen nothwendig sind, alle selbst und ohne mühsame Dazwischenkunft einer göttlichen Offenbarung erkennen.\par
\RWbet{Antwort.} 1.~Es gibt für's Erste einige Wahrheiten, welche kein endlicher Verstand, so vollkommen er auch immer seyn möchte, von selbst erkennen kann, und deren Mittheilung gleichwohl sehr wünschenswerth und sogar nothwendig für uns ist. Von dieser Art ist \zB\ die Beantwortung der Frage von einem Genugthuungsmittel. (\RWparnr{103})
\begin{aufza}\setcounter{enumi}{1}
\item Ueberdieß könnten es auch wohl gewisse andere Umstände verhindert haben, uns ein \RWbet{noch vollkommeneres Erkenntnißvermögen}, als unser gegenwärtiges, zu geben, oder es könnte seyn, daß wir durch diesen vollkommeneren Verstand nicht wirklich vollkommener, \dh\ nicht wirklich besser und glücklicher geworden wären. Für unsere Verhältnisse auf dieser Erdenwelt wäre uns vielleicht ein jedes höhere Maß der Verstandeskräfte eher nachtheilig als ersprießlich; denn zur \RWbet{Vollkommenheit} eines Wesens gehört, daß alle Kräfte desselben in einem gewissen Verhältnisse unter einander und mit den Dingen außer ihm stehen.
\end{aufza}

\RWpar{128}{3.~Einwurf. Die Offenbarung ist nicht allgemein nothwendig, weil sie nicht allgemein verbreitet ist}
Wenn eine Offenbarung allgemein nothwendig wäre, so müßte sie auch allgemein verbreitet seyn. Nun ist aber keine~\RWSeitenw{327}\ einzige jener Religionen, welche sich für geoffenbaret ausgeben, allgemein verbreitet, keine derselben zählt \ergaenzt{auch} nur die Hälfte des menschlichen Geschlechtes unter ihre Anhänger. Also müßte Gott ungerecht und parteilich seyn, wenn eine Offenbarung wirklich nothwendig für die Menschen wäre. \RWbet{Ungerecht}, weil er so vielen Millionen Menschen eine Religion entzieht, die ihnen doch so nothwendig seyn soll. \RWbet{Parteilich}, weil er sie Einigen gibt, ohne daß ihn ein vernünftiger Grund, irgend ein vorhergehendes Verdienst, das sich gerade nur diese Menschen erwarben, bei jener Austheilung bestimmte.\par
\RWbet{Antwort.} 1.~Gesetzt, wir wüßten auf diesen Einwurf nichts zu erwidern: doch würde er unsern oben gegebenen Beweis für die Nothwendigkeit einer göttlichen Offenbarung nicht umstoßen; weil er nicht \ergaenzt{einen} einzigen der von uns aufgestellten Gründe angreift. Eine Bemerkung, die wir auch gegen die beiden früheren Einwürfe hätten vorbringen können.
\begin{aufza}\setcounter{enumi}{1}
\item Allein, wenn dieser Einwurf etwas bewiese; so würde er \RWbet{zu viel beweisen}, nämlich, daß auch die natürliche Religion den Menschen überflüssig sey; was die Gegner doch selbst nicht behaupten. Auch der natürlichen Religion mangelt es nämlich an jener allgemeinen Verbreitung, die man in diesem Einwurfe von einer jeden den Menschen nöthigen oder doch nützlichen Offenbarung fordert.
\item Übrigens habe ich oben keine so \RWbet{unumgängliche Nothwendigkeit} der Offenbarung behauptet, daß ohne sie Niemand zu einem gewissen Grade der Tugend und einer davon abhängigen inneren Glückseligkeit gelangen könnte; ich habe nicht einmal behauptet, daß eine und eben dieselbe göttliche Offenbarung für alle Menschen ganz ohne Ausnahme nothwendig sey; ich lasse zu, es könne Völker geben, die wegen des allzu niedrigen Grades der Bildung, auf dem sie bis jetzt stehen, einer höhern Offenbarung gar nicht empfänglich sind, und statt des gehofften Nutzens nur Schaden von ihr hätten. Setzen wir dieses voraus, so streitet es weder mit Gottes Gerechtigkeit noch Unparteilichkeit, wenn er bisher Einen und eben denselben geoffenbarten Glauben noch nicht an alle Menschen hat gelangen lassen.
\item Endlich wenn selbst hie und da einzelne Menschen oder auch ganze Völker bereits Empfänglichkeit für die geoffenbarte Re\RWSeitenw{328}ligion haben; doch könnte Gott, ohne parteilich oder ungerecht zu handeln, ihnen dieß herrliche Geschenk versagen, wenn ihn nur irgend ein hinreichend wichtiger äußerer Grund, \dh\ ein Grund, der nicht in diesen Menschen, sondern in dem Zusammenhange des Ganzen liegt, dazu bestimmte.
\end{aufza}

\RWpar{129}{4.~Einwurf. Eine geoffenbarte Religion kann niemals allgemein verbreitet werden}
Wenn eine Offenbarung dem menschlichen Geschlechte allgemein nothwendig wäre; so müßte sie, wenn auch jetzt noch nicht allgemein verbreitet seyn, wenigstens einmal zu einer allgemeinen Anerkennung gelangen. Aber kein geoffenbarter Glaube kann jemals allgemeiner Glaube werden; wie dieses aus folgenden Gründen erhellet:
\begin{aufzb}
\item weil jene übervernünftigen Wahrheiten, die eine Offenbarung aufstellen soll, niemals von allen Menschen auf eine gleiche Art verstanden werden können. Wie schwer wird es den Menschen nicht schon, sich über sinnliche Gegenstände zu verständigen, um wie viel schwerer erst über dergleichen übersinnliche Dinge! Selbst wenn man es einmal dahin gebracht hätte, daß auf dem ganzen Erdenrunde einerlei Redensarten über diese Gegenstände herrschten: so würde sich doch Jeder bei diesen Redensarten etwas Anderes denken. (\RWbet{Bahrdt}.)
\item Die Beweise, worauf sich eine Offenbarung stützt, sind \RWbet{Wundererzählungen}; solche Erzählungen aber verlieren in eben dem Maße an Glaubwürdigkeit, in welchem sie sich nach Raum und Zeit verbreiten. (\RWbet{Kant}.) Im Jahre \RWbet{3150} nach Christi Geburt -- hat \RWbet{Johann Craig} berechnet\RWfootnote{\RWlat{Joh.~Craigii principia mathematica Theologiae christianae. Lond. 1699. 1.\,edit.}}\RWlit{}{Craig2} -- wird die evangelische Geschichte alle Glaubwürdigkeit verloren haben.
\item Die \RWbet{vielen Sprachen}, die es auf diesem Erdenrunde gibt, und deren Anzahl sich auf \RWbet{500} erstreckt, machen es gleichfalls unmöglich, daß eine Religion zu allen Völkern vordringe: zumal da die meisten dieser Sprachen so äu\RWSeitenw{329}ßerst unvollkommen sind, daß man die abstracten Lehren der Offenbarung, \zB\ der christlichen, in ihnen gar nicht ausdrücken kann. (\RWbet{Bahrdt.})
\item Hiezu kommt noch die \RWbet{physische Lage} mehrer Länder, die für uns Europäer beinahe unzugänglich sind, wie auch verschiedene politische Verhältnisse, zu Folge deren gewisse Völker keinem Ausländer einen Aufenthalt unter sich gestatten. (\RWbet{Fragmentist}.)
\item Endlich ist die Behauptung, daß eine Offenbarung jemals ganz allgemein werde, \RWbet{in sich selbst widersprechend}. Um allgemein zu werden, müßte sie den Charakter der Nothwendigkeit annehmen; sie beruht aber auf \RWbet{Erfahrungen}; Erfahrungen aber geben immer nur das Bewußtseyn, daß ein Gegenstand \RWbet{sey}, nie daß er so seyn \RWbet{müsse}, \dh\ sie haben immer nur den Charakter der Zufälligkeit. Also muß auch jede Offenbarung als ein Erfahrungsgegenstand den Charakter der Zufälligkeit haben, sie kann demnach nicht nothwendig seyn. (\RWbet{Kant.})
\end{aufzb}\par
\RWbet{Antwort.} Es ist an und für sich betrachtet nicht einmal wahr, daß eine jede Offenbarung darum, weil sie dem menschlichen Geschlechte nützlich ist, einst völlig allgemein verbreitet werden müßte. Indessen verspricht dieses die christliche Offenbarung wirklich, und darum will ich die Gründe näher erwägen, aus denen man dieß für etwas Unmögliches erklärt hat.
\begin{aufzb}
\item Ich gebe es zu, \RWbet{daß völlig gleiche Ansichten} über religiöse Gegenstände bei allen Menschen zu bewirken, etwas Unmögliches sey; ja meiner Meinung nach wäre dieß auch etwas \RWbet{Ueberflüssiges} oder gar \RWbet{Schädliches}. Aber die Offenbarung hat dieß auch nicht zur Absicht; sondern sie will nur \RWbet{gleichförmigere} Gesinnungen, als etwa jetzt schon vorhanden sind, unter den Menschen zu Stande bringen, sie verspricht nur eine Gleichförmigkeit, die gewiß genug wäre, damit sich alle Menschen dereinst als Mitglieder einer und eben derselben religiösen Gesellschaft (Kirche) betrachten und betragen können. Das aber wird doch Niemand für eine Unmöglichkeit erklären; das kann und muß auch seinen sehr großen Nutzen haben, wenn anders jene gemeinschaftlich angenommenen Ansichten selbst wohlthätig sind, und wenn diejenigen Begriffe, welche die Menschen gegenwärtig haben, wie Jeder zugeben wird, großentheils noch sehr unvollkommen und nachtheilig sind.~\RWSeitenw{330}
\item Daß alle Erzählungen in eben dem Maße an Glaubwürdigkeit verlieren, in welchem sie sich nach Zeit und Raum verbreiten, ist keineswegs erweislich. Aber sollten sie auch etwas verlieren; so folgt doch gar nicht, daß die Erzählung einer bestimmten Begebenheit nicht über das ganze Erdenrund und bis an das Ende des menschlichen Geschlechtes mit hinlänglicher Glaubwürdigkeit fortgepflanzt werden könnte. Es gibt ja doch wirklich Erzählungen, die auf dem ganzen Erdenrunde Glauben finden, und schon Jahrhunderte alt sind. Die Berechnung \RWbet{Craig}'s beruht auf den willkürlichsten Voraussetzungen, die keiner ernstlichen Widerlegung werth sind. Gesetzt aber, es sollte wirklich für ein gewisses Zeitalter oder in einem gewissen Lande einmal der Fall eintreten, daß man von den zu Jesu Zeiten geschehenen Wundern nicht mehr mit hinlänglicher Sicherheit sich überzeugen könnte: so dürfte ja Gottes Vorsehung, ohne die christliche Offenbarung darum zu Grunde gehen zu lassen, nur einige neue Wunder zu ihrer Bestätigung wirken.
\item Möchte es auch noch mehr als 500 Sprachen geben: aus dieser Menge wird doch gegen die Möglichkeit einer allgemeinen Ausbreitung der christlichen Offenbarung nicht das Geringste folgen. Oder ist es denn nöthig, daß alle diese Sprachen von einem Einzigen erlernet werden? Wenn aber mehre von diesen Sprachen gegenwärtig noch so unvollkommen sind, daß man die höheren Lehren der Offenbarung in ihnen nicht einmal ausdrücken kann: so sind auch eben darum die Völker, die diese Sprachen reden, jener erhabenen Lehren noch nicht bedürftig. Man wird sie erst bilden, wird ihre Begriffe erst entwickeln müssen, dann wird sich auch ihre Sprache von selbst vervollkommnen, und man wird ihnen jene Lehren der Offenbarung recht bequem beibringen können.
\item Keine Klippen und Berge sind unübersteiglich, kein Klima schlechterdings unerträglich, wenn auch für Europäer, doch nicht für nähere Nachbaren; alle politischen Verhältnisse sind der Veränderung unterworfen; \usw\
\item Wenn der \RWlat{sub e} angeführte Grund, der so gelehrt aussieht, etwas bewiese: so ließe auf eben die Art sich auch die Nothwendigkeit von Speise und Trank~\RWSeitenw{331}\ für die Erhaltung des Lebens wegdemonstriren. Denn auch Speise und Trank müssen in der Erfahrung gegeben werden, \usw\ Dem großen Weltweisen ist hier nämlich etwas sehr Menschliches begegnet. Zwei verschiedene Begriffe hat er um ihrer gleichen Bezeichnung wegen verwechselt: die etwas uneigentlich so genannte Nothwendigkeit einer Offenbarung, welche eigentlich nur eine \RWbet{Nothwendigkeit} zu einem gewissen \RWbet{Zwecke} ist, hat er für eine absolute, und aus \RWbet{Begriffen} erweisliche Nothwendigkeit gehalten. Von dieser letzteren gilt, was er behauptet, nicht von der ersteren.
\end{aufzb}

\RWpar{130}{5.~Einwurf. Jede fernere Belehrung soll überflüssig seyn}
\begin{aufza}
\item Die einzigen drei Gegenstände, worüber uns eine Offenbarung in \RWbet{theoretischer} Hinsicht Belehrung mittheilen könnte, müßten \RWbet{Gott, Freiheit} und \RWbet{Unsterblichkeit} seyn. Aber was könnte sie uns von diesen drei Gegenständen noch über dasjenige, was wir schon durch die bloße Vernunft wissen, lehren, das uns nicht überflüssig oder gar schädlich wäre?
\begin{aufzb}
\item \RWbet{In Rücksicht auf Gott?} Wollte die Offenbarung uns etwa die Gottheit, wie sie an sich ist, darstellen? Aber dann würde die Vorstellung von Gottes unendlicher Majestät uns zur Befolgung des Sittengesetzes mit Gewalt treiben, und unsere Freiheit aufheben.
\item \RWbet{In Rücksicht auf die Freiheit?} Wollte sie uns etwa die Verbindung der Freiheit in uns mit der Natur-Nothwendigkeit außer uns erklären? Aber was würde uns dieß nützen?
\item \RWbet{In Rücksicht auf Unsterblichkeit?} Wollte sie uns etwa die Belohnungen und Strafen des künftigen Lebens wie gegenwärtig vormahlen? Aber dieß würde ja unsere Tugend in bloßen Eigennutz verwandeln.
\end{aufzb}
\item Eben so wenig kann eine Offenbarung \RWbet{praktischer} Wahrheiten von einem Nutzen für uns seyn. Denn wie die Vernunft in uns spricht, so spricht sie auch in Gott; dasselbe oberste Sittengesetz, welches wir Menschen anerkennen, muß auch Gott anerkennen; jede Pflicht also, die uns Gott auf\RWSeitenw{332}stellen will, muß sich auch schon aus unserem obersten Sittengesetze herleiten lassen. Folglich gäbe es hier nur zwei Fälle: entweder Gott würde uns nur die Pflicht -- die Regel -- aufstellen, ohne uns ihren Ableitungsgrund zu zeigen; oder er würde uns auch mit ihrem Ableitungsgrunde bekannt machen. Das Erstere würde keine wahre Moralität, sondern nur Legalität bewirken; im letzteren Falle aber würde uns Gott einen Dienst zu leisten versuchen, den jeder Weisere seinen unweiseren Mitbürgern gleichfalls zu leisten vermag. (\RWbet{Fichte} d.~Aelt.)
\end{aufza}\par
\RWbet{Antwort.} 1.~Es ist
\begin{aufzb}
\item falsch, daß jede Offenbarung theoretischer Wahrheiten gerade nur die drei Gegenstände: Gott, Freiheit und Unsterblichkeit betreffen müßte. Es gibt, wie wir oben gesehen, noch manche andere Gegenstände, worüber einen näheren Aufschluß zu erhalten sehr wünschenswerth, um nicht zu sagen, nothwendig für uns wäre. Es ist
\item falsch, daß jede Offenbarung, die diese drei Gegenstände behandelt, unsere Einsichten über sie gerade \RWbet{erweitern} müßte; auch eine bloß formelle Offenbarung über diese Objecte kann uns von Nutzen seyn, weil selbst Gelehrte oft über die Eigenschaften Gottes, über die Freiheit und Unsterblichkeit des Menschen, und über andere dergleichen Wahrheiten der bloßen Vernunftreligion geirrt. Es ist
\item falsch, daß jede Erweiterung unserer Einsichten über diese drei Gegenstände, jede materielle Offenbarung jene im Einwurfe angegebenen Folgen hervorbringen müßte. Zwar daß eine weitere Belehrung über die Freiheit, darin uns die Verbindungsart derselben mit der Naturnothwendigkeit erklärt würde, unnütz wäre, gebe ich selbst zu. Auch das mag wahr seyn, daß eine erschöpfende Kenntniß von Gott, und eine so lebhafte Darstellung der künftigen Belohnungen und Strafen, wie wenn sie gegenwärtig wären, der Tugend nachtheilig seyn müßte; aber dergleichen Belehrungen verspricht ja die Offenbarung nirgends, und könnte sie auch auf keine Weise leisten.
\end{aufzb}
\begin{aufza}\setcounter{enumi}{1}
\item Es ist freilich wahr, daß die Vernunft in Gott eben das spreche, was sie in uns spricht, wenn sie nicht unrichtig spricht; aber daraus folgt nur, daß Alles, was unsere Ver\RWSeitenw{333}nunft -- wenn sie nicht irrt -- für wahr erkennt, auch von Gott dafür anerkannt werde; nicht aber umgekehrt, daß Alles, was Gott erkennt, auch von uns eingesehen werden müßte. Also ist es zwar richtig, daß jede Pflicht, die uns Gott aufstellt, auch aus dem obersten Sittengesetze, das eine richtig urtheilende Menschenvernunft erkennt, herleitbar seyn müsse; aber es ist nicht nothwendig, daß auch wir Menschen diese Herleitungsart der Pflicht einsehen müssen, sondern es kann auch Pflichten (gemeinnützige Handlungsweisen) geben, die aus Verhältnissen folgen, welche wir nicht kennen, und es ist falsch, daß die Befolgung solcher Pflichten bloße Legalität und keine Moralität bewirken würde. Wenn wir dergleichen Gesetze darum befolgen, weil sie Gott aufgestellt hat, und weil wir aus dieser Aufstellung Gottes schließen, daß sie richtige Herleitungen aus dem obersten Sittengesetze sind: so hat unser Gehorsam gewiß einen sittlichen Werth. Auf diese Art könnte uns also auch eine \RWbet{materielle} Offenbarung praktischer Wahrheiten von Nutzen seyn. Und eben so auch eine \RWbet{formelle}. Denn auch, wenn eine gewisse Pflicht sich aus denjenigen Verhältnissen, die wir selbst kennen, herleiten läßt, kann es noch immer von Nutzen seyn, wenn uns die Offenbarung mit ihr bekannt macht. Jener weisere Mann, der uns Unweisere darüber belehren könnte, steht uns vielleicht nicht immer und überall zu Gebote. Uns die Art der Herleitung zu zeigen, ist freilich nicht die Sache einer Offenbarung als solcher; denn diese läßt sich in keine wissenschaftlichen Untersuchungen ein; aber veranlassen kann sie gleichwohl auch diese Herleitung dadurch, daß sie den Lehrsatz ausspricht, zu dem wir dann den Beweis leichter auffinden können. Wie viele zum Theil ganz handgreifliche Irrthümer und Uebertreibungen also sind nicht in diesem einzigen Einwurfe zusammengehäuft!
\end{aufza}

\RWpar{131}{6.~Einwurf. Positive Religionen schaden der Aufklärung und der Sittlichkeit der Menschen}
\begin{aufza}
\item Jede positive Religion, \dh\ jede, die zu den Lehren der Vernunft noch einige Zusätze macht, die sie nicht aus Gründen, sondern auf Auctorität will angenommen wissen,~\RWSeitenw{334}\ schadet der \RWbet{Aufklärung} der Menschen und verbreitet Aberglauben. Sie zeigt, daß man etwas auch \RWbet{ohne}, ja auch sogar \RWbet{gegen} die Gründe der gesunden Vernunft glauben und annehmen könne. Dann meint der Mensch, er dürfe dieß überall thun, und wird auf diese Art eine Beute des Aberglaubens. Insonderheit ist der schädliche Glaube an Geistererscheinungen nur den vorgeblichen Offenbarungen zuzuschreiben.
\item Alle positiven Religionen schaden ferner auch der \RWbet{Sittlichkeit}. Sie veranlassen einmal schon Secten und Spaltungen unter den Menschen, Streitigkeiten, aus denen dann Haß und Verfolgungen entspringen; sie verlöschen das Bild der wesentlichen Gleichheit aller Menschen, \usw\ Ueberdieß theilen sie unsere Aufmerksamkeit, und lenken sie von dem Einen, was Noth thut, von der Tugend ab, auf allerlei sehr unwichtige Dinge; sie schwächen endlich auch die Beweggründe zur Tugend, indem sie dem Menschen Mittel und Wege zeigen, sich die ewige Seligkeit auch ohne Tugend zu erwerben; \udgl\  (\RWbet{Bahrdt}.)
\end{aufza}\par
\RWbet{Antwort.} 1.~Es ist unrichtig, daß jede positive Religion dem Menschen ein Beispiel gebe, wie er auch ohne und sogar wider die Gründe der gesunden Vernunft etwas annehmen könne. Die Lehren, die eine wahre göttliche Offenbarung ertheilt, werden nicht \RWbet{wider} die gesunde Vernunft seyn; auch werden die Menschen solche nicht ohne Gründe anzunehmen brauchen, sondern sie wird ihre Wahrheit durch Wunder hinlänglich darthun. Wollte man aber sagen, daß eine Offenbarung der Aufklärung der Menschen schon darum schade, weil sie den Menschen an einen Auctoritätsglauben gewöhne: so dient zur Antwort, daß Auctoritätsglaube -- besonders derjenige, der historische Gegenstände betrifft -- dem menschlichen Geschlechte höchst nothwendig sey, und wenn auch keine Offenbarung da ist, so oft in Anspruch genommen werde, daß durch die wenigen Fälle, in denen auch die Offenbarung Glauben auf menschliche Aussagen fordert, wahrlich nichts verdorben werden kann. Der Glaube an Geistererscheinungen, sollte er immerhin sein Daseyn nur den wahren oder vorgeblichen Offenbarungen zu verdanken haben, ist nicht so schädlich, als man es darstellt; wenigstens ist sein Schade in keinen Ver\RWSeitenw{335}gleich zu bringen mit jenem Nutzen, den der Glaube an höhere Wesen und ihren wohlthätigen Einfluß auf uns, so wie überhaupt an alle übrigen Lehren einer wahren göttlichen Offenbarung hervorbringen kann.
\begin{aufza}\setcounter{enumi}{1}
\item Es ist unrichtig, daß Offenbarungen Secten und Spaltungen unter den Menschen befördern, vielmehr vermindern sie die Zahl derselben. Denn wenn es auch keine wahre göttliche Offenbarung gäbe, so würden die Menschen nichts desto weniger über alle diejenigen Fragen, welche nur eine Offenbarung zu beantworten vermag, etwas entscheiden wollen; und weil sie dann keine hinreichenden Gründe dazu hätten: so würde jeder anders entscheiden, und hieraus würden erst zahllose Secten und Streitigkeiten entstehen. Dieß bestätiget auch wirklich die Geschichte. Höchstens könnte man sagen, daß solche Streitigkeiten dort, wo Jeder nicht seine eigene fehlbare Menschenmeinung, sondern den Ausspruch der Gottheit zu vertheidigen glaubt, mit größerer Hitze geführt werden dürften; aber auch dieser Schade wiegt den Nutzen noch nicht auf, den eine wahre göttliche Offenbarung leistet. Je inniger und fester wir von ihren Lehren überzeugt sind, um desto wohlthätiger wirkt sie auf unser Herz, und selbst die Hitze, mit der wir Andere von unserer Meinung zu überzeugen bestrebt sind, entspringt aus keinem so sträflichen Grunde. Eben so falsch ist es, daß eine Offenbarung unsere Aufmerksamkeit nothwendig theilen müßte. Sie könnte im Gegentheile sehr Vieles beitragen, um unsere Aufmerksamkeit auf das, was einzig Noth thut, ununterbrochen zu richten, indem sie mit jedem Gegenstande auf Erden gewisse Nebenbegriffe verbände, die uns an Gott, an unsere Pflichten \usw\ erinnern. -- Daß man auch ohne Tugend selig werden könne, wird keine wahre Offenbarung lehren. -- Endlich begeht man in diesem Einwurfe noch einen andern Fehler. Da man behauptet, daß alle positiven Religionen dem menschlichen Geschlechte aus den hier angegebenen Gründen nachtheilig wären; so muß man eben darum auch behaupten, daß sich Gott niemals wirklich geoffenbaret habe. Allein wenn alle vorhin gerügten Vorurtheile und Mißbräuche, \zB\ der Glaube an Geistererscheinungen, die heftigen Streitigkeiten \usw , auch ohne daß je eine göttliche Offenbarung vorhanden gewesen,~\RWSeitenw{336}\ zum Vorschein gekommen sind: so folgt ja eben hieraus, daß sie durch eine göttliche Offenbarung nicht erst erzeugt, nicht einmal vermehrt, sondern im Gegentheil nur vermindert würden; indem es doch Niemand läugnen wird, daß dort, wo zwischen einer wahren und einer bloß vorgeblichen göttlichen Offenbarung die Wahl ist, die erstere sicher die minder schädliche, und also wenigstens vergleichungsweise nützlich seyn werde.
\end{aufza}

\RWpar{132}{7.~Einwurf. Aus der Unbegreiflichkeit einer Offenbarung}
Was könnte uns eine Offenbarung auch wirklich nützen, da ihre Belehrungen alle ganz über unsere Vernunft erhaben seyn müßten, und folglich von uns gar nicht begriffen, sondern nur dunkel geahnet würden? Wie könnten uns solche dunkle und schwankende Erkenntnisse zu einer Richtschnur in unserem Lebenswandel dienen?\par
\RWbet{Antwort.} Freilich können die Wahrheiten einer Offenbarung nicht ganz \RWbet{begriffen}, \dh\ nicht nach ihren innern Gründen eingesehen werden, aber darum kann ihre Erkenntniß doch sicher und lebhaft genug seyn. \RWbet{Sicher}, weil sie auf Gottes Zeugniß beruht; \RWbet{lebhaft}, weil mehre ihrer Lehren in Bildern enthalten seyn können. Auch kann das Schwankende, das solche Bilder haben, durch negative Bestimmungen hinlänglich eingeschränkt werden. Und wenn dieß Alles geschieht, dann werden uns solche Wahrheiten gar wohl zum Troste, zur Richtschnur in unserem Lebenswandel \usw\ dienen können, ob wir gleich ihre inneren Gründe nicht einsehen; wie dieß ein gleicher Fall mit so vielen mathematischen, physikalischen, medizinischen und anderen Wahrheiten ist, von denen auch derjenige Gebrauch macht, der ihre inneren Gründe nicht kennt.

\RWpar{133}{8.~Einwurf. Aus der Unveränderlichkeit der Natur}
Die Vertheidiger der Offenbarung geben selbst zu, daß in dem ersten (originellen) Zustande des menschlichen Geschlechtes die natürliche Religion vollkommen hinreichend gewesen sey. War sie es aber damals, so muß sie es auch noch jetzt seyn;~\RWSeitenw{337}\ denn die Natur bleibt unveränderlich immer dieselbe; wir haben jetzt noch immer dieselben Verhältnisse, wie wir sie ehemals gehabt; mithin auch jetzt noch die nämlichen Pflichten und Beweggründe zur Tugend, folglich die nämliche Religion.\par
\RWbet{Antwort.} 
\begin{aufza}
\item Die Vertheidiger der Offenbarung behaupten keineswegs, daß nicht auch den ersten Menschen schon eine Offenbarung recht nützlich gewesen wäre; höchstens haben Einige aus ihnen behauptet, daß in dem gegenwärtigen verdorbenen Zustande des menschlichen Geschlechtes die Nothwendigkeit einer höhern Belehrung noch größer sey, als sie es Anfangs war.
\item Uebrigens ist es falsch, daß die Natur so ganz unabänderlich ist. Wenn auch in dem leblosen und belebten thierischen Theile der Schöpfung ein unveränderlicher Kreislauf herrschen sollte; bei dem Geschlechte der Menschen ist dieß offenbar nicht der Fall. Bei uns Menschen ändern sich die Bedürfnisse, die Fähigkeiten, die Verhältnisse \usw\ gar sehr. Immerhin könnte also heute eine Offenbarung Bedürfniß für uns seyn, wenn sie es auch nicht schon vor Jahrtausenden gewesen.
\end{aufza}

\RWpar{134}{Anhang. Prüfende Uebersicht der sonst gewöhnlichen Arten, die Nothwendigkeit einer Offenbarung zu beweisen}
Ich habe gleich \RWparnr{97}\ erklärt, daß ich nicht eine \RWbet{entschiedene}, eine auch von \RWbet{Gott selbst} anzuerkennende \RWbet{Nothwendigkeit} einer göttlichen Offenbarung beweisen, sondern nur darthun wolle, \RWbet{daß eine Offenbarung nützlich und nothwendig scheine, so weit wir Menschen es absehen können;} ein Zusatz, durch welchen angezeigt wurde, daß ich es immer noch dahin gestellt seyn lasse, ob nicht Gott selbst vielleicht die Sache anders finden, und gewisse uns nicht bemerkbare Nachtheile von einer Offenbarung vorhersehen könnte, die ihn bestimmen, sie entweder der ganzen Menschheit oder doch einem gewissen Theile derselben entweder für immer oder doch für eine gewisse Zeit vorzuenthalten. Nicht also sind die bisherigen Bearbeiter dieses Gegenstandes, so viel ich wenigstens wüßte, verfahren; sondern fast Alle haben~\RWSeitenw{338}\ sich hiebei so unbestimmter und allgemein lautender Ausdrücke bedient, als ob es eine \RWbet{objectiv gültige} und somit eine auch von Gott selbst anzuerkennende \RWbet{Nothwendigkeit} oder \RWbet{Nützlichkeit} einer Offenbarung wäre, die sie erweisen wollten; sie haben überdieß diese Nothwendigkeit auf alle \RWbet{Zeitalter}, und auf alle einzelne \RWbet{Glieder} der Menschheit ausgedehnt; es sey denn, daß Einige höchstens bei dem ersten Menschenpaare im Stande der Unschuld im Paradiese eine Ausnahme von dieser Nothwendigkeit machten.\par
Zu diesem Zwecke unterschieden sie häufig zweierlei Arten von Gründen für ihre Behauptung: solche, welche die \RWbet{Nothwendigkeit}, und \RWbet{andere}, die eine bloße \RWbet{Nützlichkeit} einer göttlichen Offenbarung beweisen sollten. Als Grund für die \RWbet{Nothwendigkeit} einer höhern Offenbarung führten die Meisten an, \RWbet{daß die sich selbst überlassene Vernunft kein Genugthuungsmittel für unsere Sünden kenne}, eine Behauptung, die sie durch ähnliche Betrachtungen, wie die des \RWparnr{103}, zu beweisen suchten. Hiebei behaupteten sie, um ihre Sache desto einleuchtender zu machen, daß jene Schuld, die sich der Mensch durch die Uebertretung eines göttlichen Gebotes zuzieht, von einer unendlichen Größe sey. -- Einige gründeten die Nothwendigkeit einer höhern Offenbarung auch noch auf Gottes \RWbet{Oberherrlichkeit über uns}. Weil Gott, sagten sie, unser oberster Herr ist, so ist es nothwendig, ihm zu dienen mit Allem, was sich an uns befindet, mit unserem Verstande, mit unserem Willen, mit unserem Empfindungsvermögen \usw\ Um ihm aber zu dienen mit unserem Verstande, muß er in einer Offenbarung verschiedene, uns durchaus unbegreifliche Lehren eröffnen, durch deren gläubige Annahme wir unsern Verstand dem seinigen unterwerfen.\par
Zum Beweise der \RWbet{Nützlichkeit} einer Offenbarung wurden fast alle die übrigen Gründe, die auch ich oben zum Beweise der Mangelhaftigkeit der natürlichen Religion angeführt habe, gebraucht. Einige setzten noch bei, daß die natürliche Religion nicht im Stande sey, die Einheit Gottes, oder auch seine bis auf das kleinste Geschöpf sich erstreckende Fürsorge mit völliger Sicherheit zu erkennen. Man sehe \zB\ \RWbet{Leß, Beda Mayr}, \uA ~\RWSeitenw{339}
   
\begin{center}\RWbet{Beurtheilung dieses Verfahrens.}\end{center}

\begin{aufza}
\item Meiner Ansicht nach ist die Behauptung, daß eine göttliche Offenbarung für unser ganzes Geschlecht eine \RWbet{entschiedene Nothwendigkeit}, ja auch nur \RWbet{Nützlichkeit} habe,
\begin{aufzb}
\item für's Erste \RWbet{unerweislich}, und deßhalb \RWbet{anmaßend}. Denn da wir von keiner einzigen Einrichtung oder Begebenheit in der Welt \RWbet{alle} Folgen zu übersehen vermögen; so können wir auch nicht alle Folgen berechnen, welche die Mittheilung einer göttlichen Offenbarung hätte; und eben so wenig mit Bestimmtheit angeben, welche Veränderungen im Laufe der Welt zu ihrer Einführung getroffen werden müßten. Wie sollten wir also ganz zuversichtlich behaupten können, daß die vortheilhaften Folgen derselben alle nachtheiligen überwiegen werden? Könnten wir dieß, so wären wir ja eben deßhalb auch berechtiget, von Gott geradehin \RWbet{zu fordern}, daß er \RWbet{sich offenbare}, weil er nach seiner Heiligkeit doch gewiß Alles, was seinen Geschöpfen \RWbet{entschieden} vortheilhaft ist, zu Stande bringen muß. Nun fühlt aber Jeder, daß es \RWbet{zu viel gewagt}, ja wirkliche \RWbet{Anmaßung}\RWfootnote{%
Auch schon einige Andere, namentlich \RWbet{Döderlein} (in seinem Religionsunterrichte.\ Nürnberg 1785.\ Th.\,I.) haben auf das Anmaßende dieser Behauptung aufmerksam gemacht.}
wäre, eine solche Forderung an Gott zu stellen. Also ist es auch gefehlt, eine objectiv geltende Nothwendigkeit einer göttlichen Offenbarung behaupten zu wollen.
\item Durch diese Behauptung verwickelt man sich in allerlei \RWbet{Schwierigkeiten.} Eine solche ist \zB\ die Beantwortung der Frage, warum, wenn eine Offenbarung für Alle entschieden nothwendig ist, Gott sie nicht Allen mitgetheilt habe?
\item Um eine so überspannte Behauptung auf eine wenigstens scheinbare Art zu beweisen, mußte man seine Zuflucht zu \RWbet{Uebertreibungen} nehmen, welche der guten Sache gar nicht zum Vortheil gereichen, weil sie nicht unbemerkt bleiben, wohl aber den Geist des Widerspruches reizen, und Veranlassung geben, daß man am Ende nicht einmal das, was doch recht einleuchtend bewiesen werden konnte,~\RWSeitenw{340}\ zugestehen will, \RWbet{daß nämlich -- wenigstens so weit wir Menschen es absehen können -- eine Offenbarung uns nützlich, ja selbst ein Bedürfniß für uns sey}.
\item Und dieses Alles kann um so füglicher erspart werden, da es zu jenem Zwecke, zu welchem wir die Nothwendigkeit einer Offenbarung behaupten, vollkommen hinreicht, wenn diese Nothwendigkeit nur so verstanden wird, wie ich es oben gethan; denn wir behaupten die Nothwendigkeit einer Offenbarung,
\begin{aufzc}
\item um diejenigen zu widerlegen, die aus der \RWbet{Ueberflüssigkeit} derselben im Voraus darthun wollen, daß keine da sey; dann aber auch
\item um einen jeden Menschen desto geneigter zu ihrer Aufsuchung und Annahme zu machen.
\end{aufzc}
Das Eine so wie das Andere wird erreicht, wenn man nur so viel beweiset, als ich mir oben aufgab; denn wenn eine Offenbarung, auch nur \RWbet{so viel wir absehen können}, nützlich und nothwendig für uns ist; so dürfen wir schon nicht sagen, sie wäre überflüßig, und so muß Jeder die Verbindlichkeit zu ihrer Aufsuchung und Annahme eingestehen.
\item Es scheint aber, daß man jene Behauptung zum Theile auch noch zur mehren Ehre der göttlichen Offenbarung aufgestellt habe. Man hat sich nämlich vorgestellt, daß Gottes Offenbarung etwas an ihrem Werthe verliere, wenn nicht behauptet würde, daß sie für \RWbet{alle} Menschen \RWbet{schlechterdings} nothwendig sey. Hiegegen erinnere ich jedoch, daß wir dem Ansehen der göttlichen Offenbarung gar keinen Abbruch thun, daß sie vielmehr \RWbet{gewinne}, wenn wir uns nicht anmaßen, aus unserer eigenen Vernunft darüber entscheiden zu wollen, ob sie uns nur nützlich und nothwendig \RWbet{scheine}, oder es in der That \RWbet{sey}, da wir aus ihrer eigenen Erklärung am Ende erfahren, daß das Letztere der Fall sey.
\end{aufzb}
\item Die Unterscheidung der Gründe, die eine bloße \RWbet{Nützlichkeit}, und anderer, die eine eigentliche \RWbet{Nothwendigkeit} der göttlichen Offenbarung beweisen, ist meiner Mei\RWSeitenw{341}nung nach sehr schwankend; denn was der Eine bloß sehr nützlich nennt, das mag ein Anderer nothwendig finden.
\item Meine Gedanken darüber, daß die sich selbst überlassene Vernunft kein Genugthuungsmittel für unsere Sünden kenne, habe ich oben schon auseinandergesetzt. Daß aber die Strafe, die wir durch Uebertretung eines göttlichen Gebotes verschulden, \RWbet{unendlich} groß seyn müsse, wenn man das so versteht, daß diese Strafe diejenige, die wir durch Uebertretung eines bloß menschlichen Gebotes verschulden, unendliche Male übertreffen müsse, dürfte schwer darzuthun seyn. Noch weniger dürfte der aus dem Begriffe der göttlichen Oberherrlichkeit über uns geführte Beweis der Nothwendigkeit einer Offenbarung Jedem befriedigend seyn. Denn wie folgt wohl aus Gottes Oberherrlichkeit, daß wir ihm unter Andern auch mit unserm Verstande dienen müssen, und wie, daß wir dieß eben nicht anders als durch die gläubige Annahme unbegreiflicher Offenbarungslehren vermöchten?
\item So schwer es auch seyn mag, den wissenschaftlichen Beweis für die \RWbet{Einheit Gottes} zu finden, so sehe ich doch nicht, daß irgend einer von jenen Weltweisen, welche das Daseyn Gottes überhaupt zugaben, an der Einheit dieses Wesens gezweifelt hätte; und darum dächte ich, daß man die Einheit Gottes immerhin den sichern Lehrsätzen der natürlichen Religion des menschlichen Geschlechtes beizählen könne (\RWparnr{73}). Daß sich aber die \RWbet{Vorsehung Gottes} auf jedes einzelne Geschöpf erstrecke, ist allerdings mehren heidnischen Weltweisen, bevor sie das Christenthum darüber aufgeklärt hatte, \RWbet{zweifelhaft} vorgekommen; wer also diese Wahrheit den unsichern Lehrsätzen der natürlichen Religion, und folglich denjenigen Mängeln derselben, aus denen sich das Bedürfniß einer Offenbarung erweisen läßt, beizählen will, mit dem werde ich eben nicht rechten. Ich habe es bloß darum unterlassen, weil es mir schien, daß wenigstens auf der Stufe der Vollkommenheit, auf der sich die natürliche Religion in unseren Tagen befindet, (einer Stufe, die wir zum Theile der durch das Christenthum selbst herbeigeführten Aufklärung unsers Verstandes zu verdanken haben) an der Allgemeinheit der göttlichen Vorsehung nicht mehr gezweifelt werde. Indessen habe ich doch nicht ermangelt, der Lehre von der Einheit sowohl als von der Fürsehung Gottes dort zu erwähnen, wo ich die Nothwendigkeit einer Offenbarung aus den Verirrungen des menschlichen Verstandes darzuthun suchte. (\RWparnr{114})
\end{aufza}
   
\RWch[Viertes Hauptstück.\\ Von der Möglichkeit und den Kennzeichen einer Offenbarung.]{Viertes Hauptstück.\RWSeitenwohne{342}\\ Von der Möglichkeit und den Kennzeichen einer Offenbarung.}
\RWpar{135}{Inhalt und Zweck dieses Hauptstückes}
Nachdem die Nützlichkeit und das Bedürfniß einer Offenbarung erwiesen ist, dringt sich die Frage auf, \RWbet{ob eine vorhanden und welche es sey}? Um diese wichtige Frage beantworten zu können, muß erst die allgemeinere entschieden seyn: \RWbet{welches sind überhaupt die sichern Kennzeichen einer Offenbarung}? Da es aber ungereimt wäre, eine Offenbarung zu suchen, wenn man schon in Voraus von ihrer Unmöglichkeit überzeugt werden könnte; und da man wirklich behauptet hat, daß eine solche Unmöglichkeit aus Gründen \RWlat{a priori} erweislich wäre: so ist es nothwendig, daß wir die Unrichtigkeit dieser Behauptung eigens darthun. Es muß denn also noch von der \RWbet{Möglichkeit} sowohl, als von den \RWbet{Kennzeichen} einer Offenbarung gehandelt werden. Beide Untersuchungen hängen jedoch so innig zusammen, daß sie nicht wohl in zwei verschiedene Hauptstücke getrennt werden können, sondern viel schicklicher verbunden werden; denn um den Beweis der Möglichkeit einer Offenbarung vollständig zu führen, muß man beweisen, daß es sichere Kennzeichen für ihr Vorhandenseyn gebe; dieß kann man aber nicht anders, als wenn man angibt, welche? Und daß die Kennzeichen, welche man angibt, vollkommene Sicherheit gewähren, folgt zum Theil wieder daraus, weil keine Unmöglichkeit einer Offenbarung erweislich ist. So greifen also die beiden Untersuchungen über die Möglichkeit und die Kennzeichen einer Offenbarung die Eine in die andere, und können nur in Verbindung angestellt werden.~\RWSeitenw{343}

\RWpar{136}{Ueber den Begriff und die verschiedenen Arten der Möglichkeit}
Da ich jetzt von der \RWbet{Möglichkeit} einer Offenbarung zu sprechen habe, so muß ich mich erst darüber erklären, welchen \RWbet{Begriff} ich mit dem Worte: \RWbet{Möglichkeit}, verbinde, zumal da man dieß Wort in sehr verschiedenen Bedeutungen zu nehmen pflegt, die leicht verwechselt werden können, und dann nur Mißverstand veranlassen müssen.
\begin{aufza}
\item Ohne mich aber in eine vollständige Zerlegung dieses Begriffes in seine einfachen Theile einlassen zu wollen, werde ich mich schon hinlänglich über ihn verständigen, wenn ich bloß sage, daß ich etwas in sofern \RWbet{möglich} nenne, wiefern es Daseyn haben \RWbet{kann}, und deutlicher, wiefern dessen Nichtseyn aus keiner reinen Begriffswahrheit fließet. Das Gegentheil des Möglichen, oder dasjenige, dessen Nichtseyn aus einer reinen Begriffswahrheit folgt, nenne ich in sofern \RWbet{unmöglich.}
\item Ich unterscheide aber mehre Arten des Möglichen und des Unmöglichen, und zwar zuvörderst das \RWbet{innerlich oder an sich Mögliche} und das \RWbet{äußerlich} oder \RWbet{beziehungsweise Mögliche}.
\begin{aufzb}
\item Unter dem \RWbet{innerlich oder an sich Möglichen}, welches man auch (minder schicklich) das \RWbet{logisch Mögliche}, manchmal wohl gar das \RWbet{Denkbare} genannt hat, verstehe ich dasjenige, dessen Nichtseyn aus keiner reinen Begriffswahrheit fließt, wenn man bloß solche reine Begriffswahrheiten allein, und keine andere Wahrheiten, nämlich nicht solche, die auch eine Anschauung enthalten, betrachtet. Das Gegentheil oder dasjenige, dessen Nichtseyn aus einer bloßen reinen Begriffswahrheit folgt, auch wenn man keine andere Wahrheit dazu nimmt, nenne ich \RWbet{innerlich, an sich selbst unmöglich.} So ist \zB\ ein gleichschenkliches Dreieck mit einem rechten Winkel innerlich möglich; ein gleichseitiges Dreieck mit einem rechten Winkel dagegen innerlich unmöglich. Denn daß es ein Dreieck der erstern Art nicht gebe, folgt aus keiner reinen Begriffswahrheit; daß es aber kein Dreieck der letztern Art gebe, folgt allerdings aus einer reinen Begriffswahrheit; es widerspricht nämlich dem Satze, daß~\RWSeitenw{344}\ jeder Winkel eines gleichseitigen Dreieckes zwei Drittel eines rechten sey.
\item \RWbet{Aeußerlich} oder \RWbet{beziehungsweise möglich} nenne ich dasjenige, dessen Daseyn nicht nur keiner reinen Begriffswahrheit, sondern auch noch gewissen andern Sätzen, die auf Anschauungen beruhen, nicht widerspricht. Das Gegentheil nenne ich \RWbet{äußerlich unmöglich}.
\end{aufzb}
\item Eine besondere Art des innerlich Unmöglichen ist dasjenige, wobei der Widerspruch schon im Begriffe liegt, oder doch gleich auf der Stelle (schon aus den bloßen Worten) bemerkt werden kann. Man pflegt es das \RWbet{Ungereimte}, auch eine \RWlat{Contradictio in adjecto, in ipsis terminis}, ein \RWgriech{xulos'idhron} (hölzernes Schüreisen) zu nennen.
\item Das \RWbet{äußerlich Mögliche oder Unmögliche} umfasset noch mehre merkwürdige Arten. Ich zähle hieher
\begin{aufzb}
\item das \RWbet{bedingt} (oder hypothetisch) \RWbet{Mögliche}, und das \RWbet{bedingt} (oder hypothetisch) \RWbet{Unmögliche}. Bedingt möglich nenne ich dasjenige, das in Beziehung auf eine gewisse Voraussetzung oder Bedingung, die man so eben macht, möglich ist; \dh\ dessen Nichtseyn sich aus keiner reinen Begriffswahrheit ableiten läßt, auch wenn man diese (empirische) Voraussetzung dazu nimmt. Das Gegentheil nenne ich bedingt unmöglich. Es ist leicht einzusehen, daß ein und derselbe Gegenstand bald bedingt möglich, bald bedingt unmöglich seyn könne, je nachdem man bald diese, bald jene Bedingung oder Voraussetzung macht. So ist \zB\ eine mondhelle Nacht bedingt möglich zu nennen, wenn vorausgesetzt wird, daß eben Vollmond sey; bedingt unmöglich aber, wenn vorausgesetzt wird, daß eben Neumond sey.
\item \RWbet{Das physisch Mögliche und physisch Unmögliche}. Ich nenne physisch möglich, was mit keinem sogenannten Gesetze der Natur in einem Widerspruche stehet, \dh\ dessen Nichtseyn auch nach Voraussetzung aller sogenannten Naturgesetze aus keiner Begriffswahrheit fließet. Das Gegentheil nenne ich physisch unmöglich. So nenne ich es \zB\ physisch möglich, daß es in einem Tage schneie und regne; physisch unmöglich aber, daß ein schwerer~\RWSeitenw{345}\ Körper, wenn er nicht gehalten wird, nicht falle; denn das Erstere widerspricht keinem der sogenannten Naturgesetze, wohl aber das Letztere.
\item \RWbet{Das psychologisch Mögliche und psychologisch Unmögliche}. Psychologisch möglich nenne ich, was mit keinem der sogenannten Gesetze der Psychologie, \dh\ mit keinem derjenigen Gesetze, nach denen die Seele sich in ihren Vorstellungen, Empfindungen, Wünschen, Willensentschließungen und Handlungen richten muß, im Widerspruche steht. Das Gegentheil heißt mir psychologisch unmöglich. So nenne ich es \zB\ psychologisch möglich, daß Jemand seine Muttersprache vergesse; psychologisch unmöglich aber, daß er etwas thue, das er für Unrecht hält, und wovon er auch nicht den geringsten Vortheil für sich selbst absieht.
\item \RWbet{Das problematisch Mögliche}. So nenne ich alles dasjenige, was \RWbet{mit nichts uns Bekanntem} in einem von uns \RWbet{erkannten} Widerspruche steht, oder, was eben so viel heißt, von dem wir nicht \RWbet{wissen, daß} es unmöglich sey. So nenne ich \zB\ das Luftsegeln problematisch möglich, weil wir nicht einsehen, daß es unmöglich sey.
\item \RWbet{Das vollkommen}, oder \RWbet{schlechthin}, oder \RWbet{in allem Anbetrachte}, oder \RWbet{absolut Mögliche}. So nenne ich nur dasjenige, was in Beziehung auf alle Gegenstände, die es nur immer gibt, sie mögen uns bekannt oder unbekannt seyn, Möglichkeit hat, \dh\ was mit keinem derselben in einem Widerspruche stehet, oder dessen Nichtseyn aus keiner Wahrheit (sie sey nun eine reine Begriffswahrheit, oder auch eine andere) fließet.
\end{aufzb}
\item Diese zwei letzten Arten der Möglichkeit, die \RWbet{problematische} nämlich und die \RWbet{vollkommene}, sind die zwei wichtigsten; doch sind auch die übrigen vier nicht ohne allen Nutzen. Im gemeinen Leben wird das Wort: Möglich, fast nie anders als in der Bedeutung des problematisch Möglichen genommen. Wenn wir \zB\ auf die Frage, ob es heute regnen werde, erwidern: Es ist möglich! so wollen wir hiedurch nichts Anderes sagen, als uns wäre nichts bekannt, woraus die Unmöglichkeit, daß es heute regnen werde, folge. Und~\RWSeitenw{346}\ eben so, wenn Jemand erzählt, er habe gefunden, daß \RWbet{113} eine Primzahl sey, und wir erwidern darauf: \RWbet{Es ist möglich}; so wollen wir hiedurch nichts Anderes sagen, als seine Behauptung stehe mit keiner uns bekannten Wahrheit in einem uns bekannten Widerspruche; wir wüßten nicht, daß dieses nicht seyn könne.
\begin{RWanm}
Nicht zufrieden mit diesen sechs Arten der Möglichkeit hat man noch die des \RWbet{moralisch} oder \RWbet{sittlich Möglichen} erdacht. So nämlich wollte man dasjenige genannt wissen, was zu Folge des Sittengesetzes geschehen soll oder darf, \dh\ was sittlich gut ist; das dagegen, was nicht geschehen soll, oder was sittlich böse ist, nannte man etwas \RWbet{moralisch oder sittlich Unmögliches}. Das ist nun meines Erachtens eine nur Mißverstand erregende Benennung; denn um das sittlich Böse als eine Art des Unmöglichen betrachten zu können, müßte man das letztere Wort in einer ganz andern Bedeutung nehmen, als es diejenige ist, die ich in Uebereinstimmung mit dem allgemeinen Sprachgebrauche festgesetzt habe. Offenbar ist die \RWbet{sittlich böse Handlung} nicht eine \RWbet{unmögliche}, sondern eine und zwar \RWbet{vollkommen mögliche Handlung}; denn könnte sie nicht vollzogen werden, so würde es ungereimt seyn, sie zu verbieten. Ja, leider! \RWbet{kann} sie nicht nur, sondern sie \RWbet{wird} auch häufig vollzogen, und dann ist sie nicht bloß \RWbet{möglich}, sondern auch \RWbet{wirklich}. -- Was diese unrichtige Benennung veranlaßt hat, ist die Redensart, zu Folge deren man sich erlaubt, zu sagen, die Forderungen der Vernunft und die Wünsche des Glückseligkeitstriebes ständen oft in einem \RWbet{Widerspruche} mit einander, und die böse Handlung \RWbet{widerspreche} dem Sittengesetze. Nun ist das, was einer Wahrheit widerspricht, allerdings unmöglich zu nennen; allein man vergaß, wie es scheint, daß man das Wort \RWbet{Widerspruch} hier nicht in seiner eigentlichen Bedeutung, wie es in der Erklärung des Begriffes der Möglichkeit genommen werden muß, sondern in einer uneigentlichen nehme, in der es nichts Anderes anzeigen soll, als daß die Forderung der Vernunft und der Wunsch des Glückseligkeitstriebes jeder ein eigenes \RWbet{Object} haben, oder daß die sittlich böse Handlung eine andere sey als jene, welche das Sittengesetz verlangt. Darum ist aber noch kein eigentlicher Widerspruch zwischen den beiden Sätzen: \RWbet{Ich soll $A$ thun}, und: \RWbet{Ich wünsche $B$ zu thun}; ingleichen auch nicht zwischen den beiden Sätzen: \RWbet{Cajus hat die Handlung $B$ verrichten}~\RWSeitenw{347}\ \RWbet{sollen}, und \RWbet{Cajus hat die Handlung $A$ verrichtet}. -- Uebrigens gab man dem \RWbet{sittlich Bösen} den Namen des \RWbet{sittlich Unmöglichen}, vielleicht auch deßhalb um so lieber, weil man erwartete, daß die Menschen das Laster gewisser meiden würden, wenn man es ihnen als etwas \RWbet{Unmögliches} schildert. In wiefern diese Erwartung nicht ganz ungegründet seyn dürfte, mag man es für den Gebrauch des geselligen Lebens wohl zugeben, daß man das \RWbet{sittlich Böse} zuweilen auch \RWbet{sittlich unmöglich} nenne; in der Wissenschaft aber wird diese Benennung schwerlich am rechten Orte seyn. 
\end{RWanm}
\end{aufza}
   
\RWpar{137}{Einige Lehrsätze über die Möglichkeit}
Bevor ich weiter gehen, und bestimmen kann, welche Art von Möglichkeit einer göttlichen Offenbarung ich in diesem Hauptstücke zu beweisen mich wolle anheischig machen, muß ich noch einige Lehrsätze über den Begriff der Möglichkeit vorausschicken.
\begin{aufza}
\item \RWbet{Alles, was wirklich ist, ist auch vollkommen möglich}. Die Wahrheit dieses Satzes braucht nicht erst dargethan zu werden.
\item \RWbet{Alles Mögliche, das nicht zugleich unbedingt nothwendig -- also nicht Gott -- ist, bedarf zu seiner vollkommenen Möglichkeit auch noch der Möglichkeit einer Bedingung (Ursache oder Kraft), die es hervorzubringen vermag.} Auch diese Wahrheit wird wohl Jeder einleuchtend finden. Wenn nämlich etwas nicht unbedingt nothwendig ist; so bedarf es zu seinem Daseyn des Daseyns einer Bedingung, \RWbet{zur Möglichkeit} seines Daseyns also der Möglichkeit des Daseyns dieser Bedingung. So wird \zB\ zur Möglichkeit des Daseyns eines Hauses nicht zwar das wirkliche Vorhandenseyn, wohl aber die Möglichkeit eines Baumeisters, der Baumaterialien \usw\ erfordert. Daß aber schon die Möglichkeit der Ursache oder Bedingung genüge, und nicht auch ihre Wirklichkeit erfordert werde, wie man sonst häufig gesagt hat; erhellet daraus, weil das wirkliche Daseyn der Ursache, der vollständigen nämlich, (nicht aber nur eines Theiles derselben) auch schon das Daseyn~\RWSeitenw{348}\ der Wirkung selbst zur Folge hat. Wenn ein Baumeister, die Materialien, die Werkzeuge, der Wille zum Baue \usw , kurz Alles, was zur vollständigen Ursache des Daseyns eines Hauses gehört, nicht bloß möglich, sondern wirklich vorhanden ist: so bleibt auch das Haus nicht bloß in der Möglichkeit, sondern gelangt zur Wirklichkeit.
\item Das Wesen, von dessen Kraft oder Wirksamkeit die Hervorbringung eines möglichen Gegenstandes abhängt, kann nur Eines von Beidem, entweder Gott oder irgend ein endliches Wesen seyn. \RWbet{Dinge, deren Wirklichmachung von Gott allein abhängt, sind, wenn sie vollkommen möglich sind, schon eben darum auch wirklich}. Denn um der Gottheit vollkommen möglich zu seyn, müssen sie unter Anderem auch mit Gottes Eigenschaften, mit seiner höchsten Weisheit, Heiligkeit \usw\ in keinem Widerspruch stehen; denn wenn dieß nicht der Fall wäre, so wäre es Gott eben darum unmöglich, sie hervorzubringen, da jene Freiheit, die wir in ihm annehmen, auf keinen Fall in der Möglichkeit einer Abweichung vom Sittengesetze bestehet. Stimmt aber die Hervorbringung des Dinges mit Gottes Weisheit, Heiligkeit \usw\ überein; so bringt er es auch wirklich hervor.
\item \RWbet{Nur dann also kann etwas vollkommen Mögliches gleichwohl nicht wirklich seyn, wenn dessen Wirklichmachung (oder Realisirung) von einem endlichen und zwar mit Freiheit begabten Wesen abhängt}.
\item Da es nun solche Wesen in der That gibt, so ist die \RWbet{Sphäre des Möglichen allerdings größer als die des Wirklichen}.
\begin{RWanm}
Kant hat behauptet, daß diese beiden Sphären einander gleich wären.
\end{RWanm}
\item \RWbet{Aus der bloß problematischen Möglichkeit einer Sache läßt sich sofort noch nicht auf ihre vollkommene Möglichkeit schließen.} Denn da wir nicht alle Dinge kennen; so sind wir nicht berechtigt, bloß daraus, daß eine gewisse Sache mit keinem \RWbet{uns} bekannten Gegenstande in einem von \RWbet{uns} bemerkten Widerspruche stehe, \di\ problematische Möglichkeit habe, sogleich zu schließen, daß diese Sache auch mit keinem~\RWSeitenw{349}\ uns \RWbet{unbekannten} Gegenstande in einem von uns \RWbet{nicht} bemerkten Widerspruche stehe, und also vollkommen möglich sey.
\item Hieraus ist zu ersehen, daß es überhaupt eine sehr schwierige Sache sey, die vollkommene Möglichkeit eines Gegenstandes zu behaupten, wenn man sie nicht etwa aus der bereits bemerkten Wirklichkeit desselben folgern kann. -- Inzwischen sind wir gleichwohl im Stande, die absolute Möglichkeit einiger Dinge zu erkennen, ohne sie erst aus ihrer Wirklichkeit zu folgern, und zwar in nachstehenden Fällen:
\begin{aufzb}
\item \RWbet{Daß es uns absolut möglich sey, einen gewissen Willensentschluß zu fassen}, erkennen wir mit \RWbet{völliger Gewißheit}, noch bevor wir uns wirklich entschieden haben, sobald nur einer von folgenden zwei Fällen Statt findet: entweder daß
\begin{aufzc}
\item unsere Vernunft diesen Willensentschluß als eine Pflicht von uns fordert, oder daß
\item unser Glückseligkeitstrieb die Vollziehung der betreffenden Handlung, als für uns vortheilhaft, wünscht.
\end{aufzc}
\item \RWbet{Daß es uns absolut möglich sey, eine gewisse Veränderung außerhalb unser}, \zB\ in den uns umgebenden Gegenständen hervorzubringen, erkennen wir zwar nicht mit völliger Gewißheit, aber doch \RWbet{mit hoher Wahrscheinlichkeit}, wenn wir (nach a) erkannt, daß der Entschluß dazu uns völlig möglich sey, und wir es aus Erfahrungen wissen, daß wir durch einen \RWbet{ähnlichen Entschluß} solche Veränderungen in der Außenwelt schon oft hervorgebracht haben. Denn hieraus können wir mit vieler Wahrscheinlichkeit folgern, daß unser Entschluß auch jetzt dieselbe Wirkung hervorbringen werde. So kann ich \zB\ mit dem höchsten Grade von Wahrscheinlichkeit erwarten, daß es mir möglich seyn werde, diesen Schuldbrief meines Freundes vor seinen Augen zu verbrennen, und ihm auf diese Art die Erlassung seiner Schuld anzukündigen. Denn weil diese Handlung sittlich gut ist, so ist die Möglichkeit des Entschlusses außer Zweifel. Die Möglichkeit der Vollziehung aber kann ich nur mit vieler Wahrscheinlichkeit, nicht aber mit Gewißheit annehmen; weil es ja seyn könnte, daß in dem Augenblicke,~\RWSeitenw{350}\ da ich die Hand mit dem Briefe zum Feuer ausstrecken will, ein Schlagfluß sie mir lähme.
\item Wiefern wir endlich mit dem höchsten Grade der Wahrscheinlichkeit erkennen, daß es auch außerhalb unser mehre uns ähnliche freie Wesen gebe, \zB\ unsere Mitmenschen auf Erden; in sofern können wir nach den Gesetzen der Aehnlichkeit und Analogie auch bei diesen mit Wahrscheinlichkeit verschiedene Möglichkeiten, ähnlich den a und b, voraus bestimmen.
\end{aufzb}
\item Weit leichter als auf die absolute \RWbet{Möglichkeit} einer Sache läßt sich auf ihre \RWbet{absolute Unmöglichkeit} schließen. Hier gilt nämlich der Satz: Was immer logisch oder physisch, oder psychologisch, oder überhaupt in irgend einer uns bekannten Rücksicht, nicht bloß bei einer willkürlich angenommenen Voraussetzung (\dh\ bloß hypothetisch) unmöglich ist, das ist auch absolut unmöglich.
\end{aufza}
   
\RWpar{138}{Nur eine problematische Möglichkeit der Offenbarung läßt sich ohne Voraussetzung ihrer Wirklichkeit darthun}
\begin{aufza}
\item Auf eine göttliche Offenbarung sind die im vorigen § angegebenen drei Arten, die vollkommene Möglichkeit einer Sache zu beweisen, nicht anwendbar. Denn eine göttliche Offenbarung ist eine Sache, deren Wirklichmachung gar nicht weder von unserer, noch von anderer endlicher Wesen Willkür, sondern von Gott allein abhängt. Daher denn, wenn man ihre \RWbet{völlige Möglichkeit} dargethan hätte, schon eben dadurch auch ihre \RWbet{Wirklichkeit} bewiesen wäre. Es bleibt uns also nichts Anderes übrig, als nur ihre \RWbet{problematische Möglichkeit} zu zeigen, \dh\ zu zeigen, daß ihr Vorhandenseyn mit keiner uns bekannten Wahrheit in einem uns bekannten Widerspruche stehe, oder (was eben so viel heißt), daß noch Niemand ihre Unmöglichkeit erwiesen habe.
\item Auch diese bloß \RWbet{problematische Möglichkeit} aber wird zu unserem Zwecke genügen; denn schon aus ihr wird folgen, daß es \RWbet{sichere Kennzeichen} einer göttlichen Offenbarung geben müsse; wir werden sie bestimmen, und~\RWSeitenw{351}\ dann aus dem Vorhandenseyn derselben an einer bestimmten Religion nicht nur die völlige Möglichkeit einer Offenbarung, sondern sogar ihre Wirklichkeit erkennen.
\end{aufza}
\begin{RWanm}
Diese Ansicht der Dinge stimmt auf das Beste mit dem überein, was der \RWbet{gesunde Menschenverstand} hierüber urtheilt. Sicher würde es dieser viel zu gewagt finden, wenn wir uns anheischig machten, das \RWbet{wirkliche Daseyn} einer Offenbarung aus Gründen \RWlat{a priori} zu beweisen; was wir doch, ohne es selbst zu wissen und zu wollen, thun müßten, wenn wir uns erböten, die \RWbet{absolute Möglichkeit} einer Offenbarung zu beweisen, weil ja (nach \RWparnr{137}\ \no\,3) aus dieser auch schon ihre Wirklichkeit folgt. Daß man an einem solchen Anerbieten gleichwohl nicht immer Anstoß genommen habe, rührt nur daher, weil man den Ausdruck: \RWbet{absolut möglich} nicht immer in diesem strengen Sinne genommen, sondern sich hiebei häufig nur das dachte, was ich die problematische Möglichkeit nenne.
\end{RWanm}

\RWpar{139}{Erfordernisse zur Möglichkeit einer materiellen Offenbarung}
Will man die problematische Möglichkeit einer Sache darthun; so muß man zeigen, daß sie mit keinem uns bekannten Gegenstande in einem uns bekannten Widerspruche stehe. Es versteht sich aber von selbst, daß man die Sache hier nur mit solchen Gegenständen in Vergleichung zu stellen brauche, von denen es nicht gleich auf den ersten Blick einleuchtend ist, daß sie mit ihr in keinem Widerspruche stehen, \dh\ nur bloß mit solchen, die ihr zu widersprechen \RWbet{scheinen}. Vornehmlich muß man also bei diesem Geschäfte \RWbet{auf etwa schon vorhandene oder noch zu erwartende Einwürfe Rücksicht nehmen}.\par
In unserm gegenwärtigen Falle müssen wir erst die Stücke näher bestimmen, welche zu einer göttlichen Offenbarung erfordert werden, um dann zu sehen, ob das eine oder das andere derselben nicht etwa irgend eine Unmöglichkeit enthalte. Nun wissen wir bereits, daß eine Offenbarung und zwar selbst \RWbet{eine materielle nichts Anderes sey, als eine Art Zeugenschaft, welche Gott selbst}~\RWSeitenw{352}\ \RWbet{für irgend eine durch unsere Vernunft nicht erkennbare Lehre ablegt.} (\RWparnr{30}) Zu einer jeden Zeugenschaft aber werden, wenn sie nicht bloß gegeben, sondern auch angenommen werden soll, folgende Bedingnisse erfordert:
\begin{aufza}
\item Von Seite \RWbet{des Zeugen}:
\begin{aufzb}
\item eine gewisse Kenntniß, die wir nicht haben, und die er eben uns mittheilen soll;
\item die Fähigkeit, gewisse Vorstellungen in uns hervorzubringen, und zwar
\begin{aufzc}
\item Vorstellungen von jener Wahrheit, über die uns der Zeuge belehren soll; dann auch noch ferner
\item die Vorstellung von seinem ernsten Willen, daß wir ihm glauben sollen, weil er die Sache selbst für wahr hält.
\end{aufzc}
\item Endlich der Wille, diese Vorstellungen wirklich in uns hervorzubringen, und das Vollziehen dieses Willens.
\end{aufzb}
\item Von Seite \RWbet{unser}, die wir das Zeugniß annehmen sollen, wird erfordert, daß wir uns von dem Vorhandenseyn dieser drei Stücke überzeugen, also uns überzeugen,
\begin{aufzb}
\item daß sich der Zeuge nicht etwa selbst irre;
\item daß er die Fähigkeit habe, uns seine Gedanken verständlich mitzutheilen;
\item daß er im gegenwärtigen Falle auch den Willen gehabt habe, dieses zu thun.
\end{aufzb}\par
Allenthalben, wo diese sechs Erfordernisse vorhanden sind, findet auch eine Zeugenschaft Statt. Da aber der Beweis für das Vorhandenseyn der drei letzten Stücke jenen für das Vorhandenseyn der drei ersteren schon einschließt: so bedarf man, um die Möglichkeit einer Zeugenschaft für irgend einen Fall zu beweisen, nur die Möglichkeit der drei letzteren Stücke für diesen Fall darzuthun. Untersuchen wir also, ob diese drei Stücke auch in Beziehung auf Gott Statt finden. Wenn sich dieß zeigen läßt, oder wenn wir nur zeigen, daß man uns wenigstens das Gegentheil nicht gehörig darthun könne; so haben wir schon die problematische Möglichkeit einer göttlichen Offenbarung erwiesen.~\RWSeitenw{353}
\end{aufza}

\RWpar{140}{Die meisten dieser Erfordernisse sind ohne Zweifel vorhanden}
\begin{aufza}
\item Das erste Erforderniß von Seite dessen, der uns etwas bezeugen soll, ist der \RWbet{Besitz einer gewissen Kenntniß}, die wir nicht haben. Dieses Erforderniß ist in Beziehung auf Gott ohne Zweifel vorhanden. Da der göttliche Verstand ein allumfassender ist, und da es der Wahrheiten unendlich viele gibt, unser menschliche Verstand dagegen nur eine endliche Menge derselben kennt: so folgt mit Nothwendigkeit, daß Gott auch unendlich viele Wahrheiten kenne, die uns noch unbekannt sind.
\item Auch das \RWbet{dritte} Erforderniß hat, wenn wir die Möglichkeit des zweiten einstweilen zugeben, keine Schwierigkeit. Wenn es Gott in Rücksicht seiner sowohl als unserer Kräfte möglich ist, sich uns zu offenbaren, so hat die Annahme, daß er es \RWbet{wolle}, problematische Möglichkeit, weil eine Offenbarung nach Allem, was wir darüber zu urtheilen vermögen, nützlich, ja selbst nothwendig für uns ist. Auch sind wir vorläufig überzeugt von Gottes Wahrhaftigkeit, zu Folge derer er weder selbst irren, noch Andere betrügen kann. Wissen wir also nur, daß ein gewisser Gedanke in unserem Bewußtseyn von ihm in der bestimmten Absicht erweckt worden sey, damit wir demselben Glauben beimessen, weil er ihn selbst für wahr erkennt: so sind wir auch schon überzeugt, daß dieser Gedanke Wahrheit enthalte, und werden sonach glauben.
\item Alles kommt also nur darauf an, ob das \RWbet{zweite} Erforderniß Statt finde, \dh\ ob Gott auch die Fähigkeit habe, \RWbet{Vorstellungen in uns zu erwecken}, und zwar
\begin{aufzb}
\item Vorstellungen von jenen Wahrheiten, worüber uns in einer Offenbarung Auskunft gegeben werden soll, und
\item die Vorstellung von seinem Willen, daß wir jenen in uns erweckten Vorstellungen Glauben beimessen, weil er sie selbst für wahr erkennt.
\end{aufzb}
\item Was nun zuvörderst die Kraft betrifft,\RWbet{Vorstellungen in uns hervorzubringen}; so wissen wir hinlänglich, daß Gott eine solche Kraft besitze. Denn in der That sind ja alle Vorstellungen, die wir erhalten, entweder~\RWSeitenw{354}\ mittelbar, oder unmittelbar von ihm in unserem Bewußtseyn hervorgebracht; weil Alles, was immer geschieht, unmittelbar oder mittelbar von ihm kömmt. Auch daß Gott \RWbet{Vorstellungen von solchen Wahrheiten}, über die uns eben Auskunft in einer Offenbarung gegeben werden soll, in uns hervorbringen könne, ist eine Annahme, welche mit nichts Bekanntem in einem Widerspruche steht. Wirklich entstehen ja gar oft Gedanken in uns, über deren Wahrheit wir weder bejahend noch verneinend zu entscheiden vermögen. So fällt uns \zB\ ein, daß Gott dem menschlichen Geschlechte vergeben wolle, wenn es sich bessert: nur ob dieß wahr sey, können wir nicht wissen, wenn uns Gott nicht zu erkennen gibt, ob wir es annehmen sollen.
\item Die einzige Schwierigkeit beruht also nur noch auf der Frage, \RWbet{ob uns Gott auch von seinem Willen überzeugen könne, daß wir gewissen Gedanken, welche in unserm Gemüthe so eben aufgestiegen sind, Glauben beimessen sollen, weil er sie selbst für wahr erkennt}? Dieß ist allerdings erst noch eine Frage. Denn obgleich alle Vorstellungen, die in uns entstehen, wenigstens mittelbarer Weise durch \RWbet{Gott} in uns hervorgebracht werden; so bringt er doch gewiß nicht alle in der Absicht hervor, damit wir sie \RWbet{glauben}. Wir müssen daher ein \RWbet{bestimmtes Kennzeichen} suchen, woran wir jene Vorstellungen, die Gott in dieser Absicht hervorbringt, von den übrigen zu unterscheiden vermögen. Ob nun ein solches möglich sey, und worin es bestehen müßte, wird die gleich folgende Untersuchung zeigen.
\end{aufza}

\RWpar{141}{Wie der gemeine Menschenverstand die Frage über die Kennzeichen einer Offenbarung entscheide?}
\begin{aufza}
\item Die Untersuchung über die Möglichkeit einer Offenbarung hat uns im vorigen § auf die Frage von \RWbet{ihren Kennzeichen} geführt. Bevor wir diese zu beantworten wagen, wollen wir erst vernehmen, ob und auf welche Art der gemeine Menschenverstand sie entscheide. Denn weil diese Frage einen Gegenstand angeht, über den durch bloße \RWbet{Ver}\RWSeitenw{355}\RWbet{nunft} ohne Erfahrungen, die etwa nur Wenigen zu Gebote stehen, entschieden werden kann, der auch zugleich für jeden Menschen von hoher Wichtigkeit ist: so würden wir, falls sich hierüber bei allen Menschen eine gleichförmige Meinung vorfände, die zugleich so beschaffen wäre, daß sie der Sinnlichkeit nicht schmeichelt, mit der größten Zuversicht annehmen können und müssen, daß diese Meinung wahr sey. Wir hätten da also das Resultat, auf welches auch unsere Untersuchung am Ende hinausführen müßte, wenn nicht der stärkste Verdacht eines Irrthums wider sie Platz greifen soll.
\item Wir finden nun allenthalben, daß sich die Menschen, wenigstens alle vernünftigen und besseren Menschen nur dann erst entschließen, eine gegebene Religion als eine wahre göttliche Offenbarung anzunehmen, wenn folgende zwei Bedingungen eintreten: 
\end{aufza}\par
\RWbet{Erstlich} muß diese Religion ihrem ganzen Inhalte nach so beschaffen seyn, \RWbet{daß sie von ihrer Annahme Beförderung ihrer Tugend und Glückseligkeit erwarten}; dann müssen sich aber noch\par
\RWbet{zweitens} gewisse \RWbet{außerordentliche Begebenheiten} finden, die zur Entstehung, Erhaltung oder Ausbreitung dieser Religion gedient, und keinen für uns Menschen bemerkbaren Nutzen hätten, sollte es nicht der seyn, daß sie uns eben ein Zeichen des Willen Gottes werden, an diese Lehren zu glauben, weil er sie selbst für wahr erkläret.
\begin{aufza}\setcounter{enumi}{2}
\item Daß dieses in der That die allgemein herrschende Ansicht der Menschen sey, davon können wir uns zuvörderst überzeugen, wenn wir den Ersten den Besten durch keinen Schulunterricht irre geleiteten Menschen auf eine kluge Weise hierüber ausforschen. Wir dürfen da freilich nicht erwarten, daß er die eben genannten Stücke uns von selbst angebe; wenn wir ihm aber die Frage vorlegen werden, ob eine Religion, welche der Tugend und Glückseligkeit der Menschen nachtheilig ist, eine wahre göttliche Offenbarung seyn könne: so wird er dieses gewiß \RWbet{verneinen}, und im Gegentheile behaupten, daß jede Offenbarung, die uns Gott mittheilt, einen gewissen sittlichen Nutzen für uns haben müsse. Wenn wir ihn weiter fragen,~\RWSeitenw{356}\ ob dieser Nutzen auch immer für uns sichtbar seyn müsse; so wird er vielleicht in einige Bedenklichkeit gerathen. Wenn wir ihn aber fragen, \RWbet{ob} und \RWbet{woraus} er mit Sicherheit abnehmen könne, daß Gott eine gewisse Lehre von ihm geglaubt wissen wollte, wenn er gar keinen Nutzen an ihr bemerkt: so wird er bald gestehen, daß die Annahme, Gott wolle eine gewisse Religion von uns geglaubt wissen, nicht begründet genug wäre, so lange wir nicht an ihrer Lehre selbst eine gewisse Zuträglichkeit für unsere Sittlichkeit bemerkten. Sollte er dieß gleichwohl nicht zugestehen; so wird er wenigstens nicht läugnen, daß schon das \RWbet{bloße Glauben} an eine uns von Gott geoffenbarte Wahrheit \RWbet{eine uns nützliche Uebung} der Unterwerfung unseres Verstandes unter den göttlichen sey; und so zeigt es sich denn, er setze auf jeden Fall voraus, daß eine Lehre, die wir als eine wahre göttliche Offenbarung annehmen sollen, wenigstens den für uns bemerkbaren Nutzen, welchen die Uebung im \RWbet{Glauben} erzeugt, besitzen müsse. -- Wenn wir ihn ferner fragen, ob es, um eine Lehre als eine göttliche Offenbarung anzunehmen, schon genug sei, wenn sie nur sittliche Zuträglichkeit für uns hat: so wird er bestimmt erwidern, dieß reiche nicht hin; sondern es müßten sich noch überdieß gewisse \RWbet{Ereignisse} zutragen, die sich als \RWbet{Zeichen} des göttlichen Willens, daß wir die Lehre annehmen sollen, ansehen lassen. Und wenn wir fragen, wie diese Ereignisse beschaffen seyn müßten: so wird er die Eine Eigenschaft derselben, daß es \RWbet{außerordentliche} Begebenheiten seyn müssen, gleich in Bereitschaft haben. Wenn wir ihn weiter fragen, ob diese Ereignisse nicht in einer gewissen \RWbet{Verbindung mit der Lehre}, zu deren Bestätigung sie dienen sollen, stehen, ob sie \zB\ nicht zu ihrer Entstehung, Erhaltung oder Ausbreitung etwas beitragen müssen: so wird er dieß abermals ohne Bedenken bejahen. Wenn wir ihn endlich fragen, warum er aus dem Daseyn solcher Ereignisse auf Gottes Willen schließe, daß wir die Lehre, die sie hervorgebracht haben, gläubig annehmen sollen: so wird er erwidern, \RWbet{weil jene Ereignisse sonst nutzlos da wären}. Wir sehen also, daß er nur solche Ereignisse für Zeichen einer göttlichen Offenbarung halte, die nicht nur \RWbet{außerordentlich} sind, sondern an denen sich auch \RWbet{kein für uns Menschen}~\RWSeitenw{357}\ \RWbet{bemerkbarer Nutzen fände}, wenn es nicht der seyn sollte, daß sie uns eben als Zeichen des göttlichen Willens dienen, daß wir an jene Lehren glauben.
\item Daß diese Antworten wirklich die Gesinnungen ausdrücken, welche die ganze auf Erden verbreitete Menschheit zu allen Zeiten gehabt hat, beweiset uns die \RWbet{Geschichte der Ausbreitung aller für geoffenbart angenommenen Religionen bei allen Völkern und zu allen Zeiten}. Gab Jemand vor, eine göttliche Offenbarung empfangen zu haben, welche die Erlaubniß zu etwas Unsittlichem enthielt: so widersetzten sich alle bessere Menschen, sofern sie anders den nachtheiligen Einfluß dieser Lehre auf ihre Sittlichkeit erkannten, einem solchen Vorgeben immer. Und wenn sie die außerordentlichen Ereignisse, welche sich zur Bestätigung jener Lehren sollten ergeben haben, auf keine andere Weise zu erklären wußten: so schrieben sie lieber sie einem \RWbet{Blendwerke des Teufels} zu, als daß sie zugegeben hätten, es seyen Ereignisse, durch welche Gott den Menschen zu erkennen geben wolle, daß sie diese Lehre als seine Offenbarung annehmen sollten. Von der andern Seite wird man uns eben so wenig ein Beispiel aufzuweisen vermögen, daß eine Religion \RWbet{ganz ohne Wunder} -- und hätten es auch nur \RWbet{vermeintliche} Wunder seyn sollen -- als Gottes Offenbarung wäre angenommen worden. Wo aber Beides in Vereinigung sich vorfand, oder -- was hier ganz einerlei ist, -- wo man doch glaubte, daß Beides vorhanden sey, da weigerte sich kein besserer Mensch, eine solche Lehre als eine ihm von Gott selbst gewordene Offenbarung anzunehmen.
\item Wir treffen also bei allen nur etwas gebildeten Menschen auf Erden eine gleichförmige Meinung darüber an, was zu den Kennzeichen einer Offenbarung erfordert werde. Und diese Meinung ist so beschaffen, daß sie der \RWbet{Sinnlichkeit der Menschen} nicht im Geringsten schmeichelt, ihr vielmehr Abbruch thut. Denn weil die Menschen gestehen, daß eine Lehre nur dann als eine wahre göttliche Offenbarung angesehen werden dürfe, wenn ihre Annahme der Tugend und Glückseligkeit zuträglich ist: so schneiden sie hiedurch sich selbst die Gelegenheit ab, je etwas Unsittliches unter dem Vor\RWSeitenw{358}wande zu thun, daß es durch eine göttliche Offenbarung erlaubt worden sey; sie legen im Gegentheil durch dieß Geständniß sich die Nothwendigkeit auf, so Manches, wogegen sich ihre Sinnlichkeit sträubt, als einen von \RWbet{Gott selbst} an sie ergangenen Befehl zu erkennen und zu befolgen. Dieß Urtheil des gemeinen Menschenverstandes hat also alle \RWparnr{14}\ \no\,7 aufgestellten Erfordernisse, die zur Glaubwürdigkeit eines solchen Urtheiles gehören.
\end{aufza}
   
\RWpar{142}{Anfang der wissenschaftlichen Erörterung dieser Frage, worin der Grund ihrer Schwierigkeit liege?}
Eine geoffenbarte Religion soll eine solche seyn, die wir auf \RWbet{Gottes Zeugniß}, \dh\ aus dem Grunde annehmen, weil wir auf irgend eine Art eingesehen haben, daß es der Wille Gottes sey, wir sollen sie glauben. Alles kommt also darauf an, \RWbet{auf welche Art wir diesen Willen Gottes erkennen}? Bei \RWbet{menschlichen} Zeugenschaften verursacht es meist gar keine Schwierigkeit, zu erkennen, ob es des Menschen Wille sey, daß wir die Vorstellungen, die er in uns hervorbringt, glauben sollen. Der Mensch nämlich wirkt in einem sichtbaren Leibe, an dem wir, vermöge der Analogie mit unserm eigenen Leibe, jede willkürliche Bewegung sehr leicht von einer unwillkürlichen zu unterscheiden wissen; und fast eben so leicht wird es bei jeder Bewegung der erstern Art, die bestimmte \RWbet{Absicht}, aus der sie hervorgeht, nach der Analogie mit uns selbst, \dh\ nach der Beschaffenheit der Absichten, welche wir selbst bei \RWbet{ähnlichen} Bewegungen haben, zu errathen. So wissen wir \zB , wenn Jemand Thränen in den Augen hat, daß er nicht im Scherze, sondern im Ernste zu uns spreche, und also wolle, daß wir das glauben, was er jetzt spricht. Nicht so verhält es sich mit Gott, der, weil er \RWbet{unsichtbar} ist, nur durch Veränderungen, die er in dieser \RWbet{sichtbaren} Welt -- unmittelbarer oder mittelbarer Weise -- hervorbringt, zu uns zu sprechen vermag. Diese Welt also ist gleichsam der Körper, durch den Gott zu uns spricht. Nun wissen wir zwar freilich, daß \RWbet{alle Ereignisse} in der Welt durch Gottes Veranlassung erfolgen, und daß er bei einem jeden gewisse~\RWSeitenw{359}\ \RWbet{Absichten} habe. Hier also sind gleichsam keine \RWbet{unwillkürlichen Bewegungen}. Aber desto schwerer fällt es uns zu bestimmen, worin jedesmal diese Absichten Gottes bestehen. Da nämlich mangelt es an einer \RWbet{Aehnlichkeit}, die zwischen den Bewegungen an unserm \RWbet{Leibe} und zwischen den Veränderungen in der \RWbet{Welt}, welche in diesem Gleichnisse den \RWbet{Körper} Gottes vorstellt, anzutreffen wäre. Aus dieser \RWbet{Schwierigkeit} der Sache folgt jedoch noch keine \RWbet{Unmöglichkeit}, zumal da wir oft selbst bei \RWbet{Menschen} die Absicht, die sie bei einer gewissen Unternehmung haben, nicht aus der Aehnlichkeit zwischen \RWbet{ihrem} und dem \RWbet{uns} gewöhnlichen Verfahren, sondern durch eine ganz andere Schlußart, die auch auf Gott angewandt werden kann, und die ich jetzt in dieser Anwendung auf Gott näher beschreiben will, entnehmen.

\RWpar{143}{Wie der Mensch im Stande sey, die Absichten Gottes bei einem Gegenstande aus der Betrachtung desselben zu erkennen?}
\begin{aufza}
\item Nach der Erklärung, die ich von dem Begriffe einer göttlichen Offenbarung (\RWparnr{30}) aufgestellt habe, ist sie eine Lehre, die wir nur darum gläubig annehmen, weil wir aus gewissen Wahrnehmungen nach unserer besten Einsicht schließen, es sey der Wille Gottes, daß wir an diese Lehre glauben, weil er sie selbst als wahr und zuträglich für uns erkennt. \RWbet{Aus gewissen Wahrnehmungen} muß also dieser Wille Gottes gefolgert werden. Es frägt sich daher, wie wir an irgend einer \RWbet{Erscheinung} in der Welt erkennen, was Gott durch sie erreichen wolle. Auch wurde bereits (\RWlat{ibid.}) erinnert, daß der Ausdruck: \RWbet{Wille Gottes} hier in seiner \RWbet{eigentlichen Bedeutung} genommen werde, in der er gleichgeltend mit dem Worte \RWbet{Zweck} oder \RWbet{Absicht} ist. Es handelt sich also darum, \RWbet{wie man die Absichten Gottes bei einem gewissen Ereignisse oder Dinge aus der Betrachtung desselben erkenne?}
\item Hierauf erwidere ich nun: um zu errathen, zu welchem Zwecke Gott ein gewisses Ereigniß oder Ding hervorgebracht habe, müssen wir erst alle Theile und Einrichtungen desselben genau untersuchen; dann überlegen, zu was für verschiedenen~\RWSeitenw{360}\ Wirkungen dieses Ereigniß oder Ding unter Hinzutritt verschiedener, uns möglich scheinender Umstände benützt werden könnte. Wir müssen ferner nachdenken, welche von diesen Wirkungen die zuträglichste für das Wohl des Ganzen wäre. Finden wir Eine, die unläugbar zuträglicher ist, als alle übrigen, oder sind diese andern sogar nutzlos, ja schädlich: so können wir mit hoher Wahrscheinlichkeit schließen, Gott habe jenes Ereigniß oder Ding wirklich zur Hervorbringung nur eben dieser wohlthätigen Wirkung bestimmt, und die noch mangelnden Umstände werde er unfehlbar ehestens herbeiführen, oder, falls ihre Herbeiführung etwa von uns abhängt, wir selbst sollen sie herbeiführen, und die erwartete Wirkung werde dann eintreten.
\item Die Richtigkeit dieser Regel läßt sich durch folgende Betrachtung einsehen. Wir müssen, sagte ich, überlegen, was für verschiedene Wirkungen das Ereigniß oder Ding \RWbet{unter Hinzutritt verschiedener, uns möglich scheinender Umstände} hervorbringen könnte. Verschiedene Wirkungen nämlich kann man von Einer und derselben Sache eigentlich nur erwarten, wenn man sich vorstellt, daß noch verschiedene andere Umstände hinzutreten; denn unter einerlei Umständen wird einerlei Ding auch nur einerlei Wirkung haben. Daß aber unter allen den Wirkungen, die sich unter Hinzutritt verschiedener, uns möglich scheinender Umstände erwarten lassen, gerade diejenige von Gott beschlossen und beabsichtiget werde, die uns die zuträglichste für das gemeine Wohl scheint: dieser Schluß beruht auf zwei Voraussetzungen, deren eine völlig gewiß, die andere aber nur wahrscheinlich ist, daher er selbst nur Wahrscheinlichkeit hat. Die \RWbet{völlig gewisse} Voraussetzung ist, daß \RWbet{Gott durch alle Dinge immer nur diejenige Wirkung hervorbringen lasse, die unter allen möglichen dem Wohle des Ganzen die zuträglichste ist}. Die bloß \RWbet{wahrscheinliche} Voraussetzung ist, daß jene Wirkung, die \RWbet{uns} die zuträglichste scheint, auch \RWbet{in der That möglich}, und die zuträglichste \RWbet{ist}. Diese letzte Voraussetzung, sage ich, hat nur Wahrscheinlichkeit; weil es nicht völlig gewiß ist, daß die Umstände, die zur Hervorbringung jener Wirkung noch hinzutreten müssen, und von uns für möglich angesehen werden, in Wahrheit~\RWSeitenw{361}\ möglich sind; ingleichen daß diese Wirkung in der That so, wie wir dafür halten, für das Wohl des Ganzen die allerzuträglichste sey. Mit je größerer Sicherheit wir aber in jedem gegebenen Falle diese zwei Stücke behaupten können, um desto sicherer ist auch die Voraussetzung, und unser ganzer Schluß.
\item Man wird zu dieser Art des Schließens ein um so größeres Zutrauen fassen, wenn man erwägt, daß sie ganz derjenigen ähnlich sey, nach der wir bei \RWbet{Menschen} vorgehen, wenn wir den \RWbet{Zweck}, den sie bei einer gewissen Handlung oder Hervorbringung hatten, beurtheilen wollen. Wir thun da gleichfalls nichts Anderes, als daß wir die Handlung oder das hervorgebrachte Werk nach allen seinen Theilen und Einrichtungen untersuchen; dann überlegen, was für verschiedene Wirkungen es unter Hinzutritt verschiedener möglicher Umstände hervorbringen könnte; und endlich uns fragen, welche von diesen Wirkungen ein Mensch von der Art, wie uns der Urheber jenes Werkes bekannt ist, beabsichtigen könnte, \dh\ welche von diesen Wirkungen er sich entweder aus Trieb zur Glückseligkeit oder aus irgend einem sittlichen Grunde zu erreichen vorgesetzt haben könnte. Finden wir am Ende nur eine einzige Wirkung, zu deren Hervorbringung alle Theile und Einrichtungen des Werkes passen, und die zugleich ganz so beschaffen ist, daß sie ein solcher Mensch beabsichtigen konnte: so schließen wir ohne Bedenken, daß er das Werk wirklich zu dieser Absicht hervorgebracht habe.
\item Wenden wir nun diese Schlußart auf Gott an; so wissen wir, daß er vermöge seiner \RWbet{Vollkommenheit} keinen Glückseligkeitstrieb habe, also nie etwas zu seinem eigenen Vortheile verlangen, sondern vermöge seiner Heiligkeit immer nur das wollen könne, was am Zuträglichsten für das Wohl der Geschaffenen ist. Hieraus ergibt sich also, daß wir bei der Beurtheilung der Absichten, die Gott bei allen Veränderungen in der Welt hat, wirklich nicht anders als nach der (\no\,2.) aufgestellten Regel verfahren dürfen.
\item Endlich ist noch zu bemerken, daß wir uns dieser Art, auf Gottes Absichten zu schließen, nicht etwa nur in der \RWbet{Religion}, sondern auch in verschiedenen \RWbet{andern} Fächern des menschlichen Wissens, namentlich in der \RWbet{Natur}\RWSeitenw{362}\RWbet{geschichte, Astronomie} \udgl\  ohne allen Anstand bedienen. Oder wie geht \zB\ ein \RWbet{Naturforscher} vor, wenn er an irgend einem Thiere einen organischen Theil entdeckt, dessen Gebrauch ihm noch unbekannt ist; wie gehet er vor, um den Zweck dieses Gliedes, \dh\ den Zweck, zu welchem Gott dem Thiere dieß Glied verliehen hat, oder dasjenige, was Gott dadurch, zwar nicht bei allen, aber doch bei den meisten Thieren der Art hervorgebracht wissen will, zu errathen? Er untersucht genau alle Theile und Einrichtungen des Gliedes, und überlegt, zu was für Wirkungen dasselbe unter Hinzutritt möglicher Umstände von Außen tauglich wäre. Findet er Eine, die der Glückseligkeit des Thieres oder dem Wohle des Ganzen überhaupt, zusagt, und weiß er sich sonst keine andere wohlthätige Wirkung zu denken, zu deren Hervorbringung das Organ nach seiner ganzen Einrichtung eben so tauglich wäre: so wartet er oft nicht erst die wirkliche Beobachtung, ob das Thier dieses Organ wirklich hiezu gebrauche, ab; sondern schließt alsogleich, daß es zu diesem Zwecke da sey.
\end{aufza}

\RWpar{144}{Anwendung hievon auf Gottes Offenbarung}
Wenn wir durch diese Art zu schließen auf das Vorhandenseyn einer göttlichen Offenbarung geleitet, \dh\ berechtiget werden sollen, von einem wahrgenommenen Ereignisse in der Welt zu sagen, es sey die Absicht Gottes, uns durch dasselbe zu bestimmen, daß wir an eine gewisse Lehre glauben: so müssen wir gefunden haben, daß der Glaube an diese Lehre eine Wirkung wäre, bei deren Annahme jenes Ereigniß einen entschiedenen Nutzen hätte, während im entgegengesetzten Falle, wenn wir die Lehre nicht glauben, kein Nutzen sichtbar seyn darf, den das Ereigniß nach seiner ganzen Einrichtung haben könnte. Es muß denn also
\begin{aufza}
\item \RWbet{die Lehre selbst} von einer solchen Beschaffenheit seyn, daß wir mit aller Zuversicht erwarten können, ihre gläubige Annahme werde unserer Tugend und Glückseligkeit zuträglich seyn, und zwar zuträglicher, als eine jede andere, die wir statt ihrer annehmen könnten. Denn wäre dieß nicht der Fall; so wäre die Wirkung, die das Ereigniß~\RWSeitenw{363}\ hervorbringt, wenn wir uns durch dasselbe zur gläubigen Annahme jener Lehre bestimmen lassen, keine dem Wohle des Ganzen zuträgliche Erscheinung. Das Ereigniß muß ferner
\item in \RWbet{einer solchen Verbindung mit jener Lehre stehen}, daß wir sie nicht auch eben so gut ohne dasselbe hätten erfahren, und von dem göttlichen Willen, an sie zu glauben, uns hätten überzeugen können. Denn im entgegengesetzten Falle würden wir unsern Glauben gar nicht als eine Wirkung des Ereignisses ansehen können; es wäre folglich auch der \RWbet{Nutzen}, den Gott durch die Hervorbringung desselben beabsichtiget hat, noch immer nicht gefunden.
\item Endlich wird noch erfordert, daß für den Fall, wo wir das Ereigniß die gläubige Annahme dieser Lehre nicht bewirken lassen, gar keine nützliche Wirkung als Zweck seines Daseyns bemerkbar sey. Dazu ist nöthig, daß das Ereigniß ein \RWbet{ungewöhnliches} sey; denn wenn es ein gewöhnliches Ereigniß wäre, das wir auch sonst oft angetroffen haben: so müßten wir, da Gott nie etwas Unnützes thut, voraussetzen, daß es einen gewissen in allen jenen Fällen, wo wir es angetroffen haben, gemeinschaftlich Statt findenden Nutzen habe; und wären daher nicht berechtiget, anzunehmen, daß es unnütz wäre, wenn es nicht dießmal die Wirkung hätte, uns zum Glauben an jene Lehre zu bestimmen.
\end{aufza}

\RWpar{145}{Nähere Beschreibung der Beschaffenheit, die eine Lehre haben muß, wenn sie als eine göttliche Offenbarung angesehen werden soll}
Es wurde so eben gesagt, eine Lehre, die wir als eine göttliche Offenbarung ansehen sollen, müsse von einer solchen Art seyn, \RWbet{daß wir mit aller Zuversicht erwarten können, die von unserer Seite zu erfolgende gläubige Annahme derselben werde der Tugend und Glückseligkeit des Ganzen zuträglich seyn, und zwar zuträglicher, als eine jede andere Meinung, die wir statt ihrer annehmen könnten}. Diese Beschaffenheit einer einzelnen Lehre, oder auch eines ganzen Inbegriffes von Lehren ist es, die ich künftig zur Abwechslung durch die verschiedenen Namen der \RWbet{sittlichen Zuträglichkeit,}~\RWSeitenw{364}\ \RWbet{der Erwünschlichkeit, der inneren Vortrefflichkeit, des sittlichen Nutzens, der Heiligkeit} derselben, \usw\ bezeichnen werde. Um aber diesen so wichtigen Begriff desto bestimmter zu fassen, will ich die eben gegebene Erklärung stückweise noch etwas umständlicher betrachten, und zugleich angeben, warum ich gerade dieß und nichts Anderes verlange.
\begin{aufza}
\item Ich sage, eine Lehre müsse von solcher Beschaffenheit seyn, \RWbet{daß wir mit aller Zuversicht erwarten können}, \usw , um anzudeuten, es sey nicht eben nöthig, daß die Lehre in der That so nützlich sey, als wir erwarten; sondern nur, daß wir es mit aller Zuversicht erwarten können. Ich verlange aber nur dieß Letztere, weil aus dem vorigen §\ und der nachfolgenden Auseinandersetzung erhellet, daß schon das genügt, um, wenn auch die übrigen Erfordernisse da sind, zu schließen, daß Gott diese Lehre von uns geglaubt wissen wolle.
\item Ich sage ferner, \RWbet{daß die von unserer Seite zu erfolgende gläubige Annahme der Lehre zuträgliche Folgen versprechen müsse}, um zu erkennen zu geben, es sey nur nöthig, daß \RWbet{unsere}, nicht aber, daß eine \RWbet{allgemeine} Anerkennung der Lehre zuträgliche Folgen verspreche. Ich verlange nur jenes und nicht auch dieses, weil es bei der verschiedenen Empfänglichkeit der Menschen gar wohl möglich wäre, daß wir von einer \RWbet{allgemeinen} Verbreitung einer Lehre gewisse nachtheilige Folgen zu besorgen hätten, während wir davon, daß nur \RWbet{wir selbst} sie annehmen, lauter ersprießliche Wirkungen erwarten. In einem solchen Falle dürften wir also die Lehre immerhin als eine \RWbet{an uns} geschehene göttliche Offenbarung ansehen, wenn wir gleich zweifeln könnten, ob sie es auch für alle andere Menschen schon gegenwärtig seyn mag.
\item Ich sage überdieß, daß unser Glaube an die Lehre gewisse zuträgliche Folgen für die Tugend und Glückseligkeit \RWbet{des Ganzen} haben müsse, nicht in der Bedeutung, daß das Ganze in allen seinen \RWbet{einzelnen Theilen} etwas gewinnen müsse; sondern nur in dem Sinne, in dem wir diese Redensart beim obersten Sittengesetze überhaupt nehmen, wo wir zu sagen pflegen, daß die Tugend und Glückseligkeit des Ganzen gewinne, wenn auch nur irgend Ein oder einige Theile des Ganzen gewinnen, während die übrigen nichts verlieren. So ist es \zB\ schon genug, wenn wir von einer Lehre erwarten können, daß~\RWSeitenw{365}\ unser Glaube an sie nur unsern \RWbet{eigenen Eifer} in der Erfüllung dieser oder jener Pflicht beleben, oder uns mit unserem Schicksale zufriedener machen werde, u.\,s.\,f. Denn auch schon hiedurch gewinnt das Wohl des Ganzen.
\item Ich sage endlich, daß die Annahme jener Lehre dem Wohle des Ganzen \RWbet{größere} Vortheile versprechen müsse, \RWbet{als jede andere Meinung, die wir statt ihrer annehmen könnten}; denn wenn dieß nicht wäre, wenn wir um eine gewisse Lehre anzunehmen -- eine andere, die noch zuträglicher ist, verlassen müßten: so könnten wir nicht sagen, daß die Annahme jener einen wahren Nutzen gewähre.
\end{aufza}

\RWpar{146}{Die sittliche Zuträglichkeit einer Lehre allein genügt noch nicht zum Beweise ihrer göttlichen Offenbarung}
Man könnte wohl auf die Frage verfallen, \RWbet{ob die so eben beschriebene sittliche Zuträglichkeit einer Lehre nicht schon für sich allein Grundes genug sey, eine solche Lehre als eine wahre göttliche Offenbarung anzuerkennen?} -- Diese Frage ist aber zu verneinen; denn alle Gründe, welche man etwa für ihre \RWbet{bejahende Beantwortung} vorbringen wollte, werden bei einer nähern Betrachtung als \RWbet{unstatthaft} befunden.
\begin{aufza}
\item Wollte man sagen, daß es, wenn eine Lehre sittliche Erwünschlichkeit hat, also von solcher Beschaffenheit ist, daß unsere Annahme derselben überwiegende Vortheile für das Wohl des Ganzen verspricht, eben darum schon unsere \RWbet{Pflicht} und \RWbet{Schuldigkeit} sey, sie anzunehmen; so erwidere ich hierauf:
\begin{aufzb}
\item Es steht nicht in unserer \RWbet{Willkür} zu glauben, was wir \RWbet{wollen}; es ist uns vielmehr nicht möglich, etwas zu glauben, so lange wir noch keinen, weder \RWbet{innern} noch \RWbet{äußern}, weder \RWbet{wahren} noch bloß \RWbet{eingebildeten} Grund dazu haben.
\item Wir müßten uns also nur durch die willkürliche Richtung unserer Aufmerksamkeit auf allerlei \RWbet{Scheingründe zu täuschen} und zu überreden suchen, daß jene Lehre \RWbet{wahr} sey. Nun mag es zwar seyn, daß wir zu einer so wohlthätigen Selbsttäuschung, sofern sie möglich ist,~\RWSeitenw{366}\ verbunden sind, (\RWparnr{38}\ \no\,2.); aber ein Glaube, der nur auf eine solche Art bei uns entstände, würde auf jeden Fall nichts weniger als den Namen eines \RWbet{Offenbarungsglaubens} verdienen, indem er sich gar nicht auf ein vorausgesetztes \RWbet{Zeugniß Gottes bezöge}.
\end{aufzb}
\item Allein man sagt vielleicht, daß ein solcher Glaube allerdings auch den Namen eines geoffenbarten verdienen würde, weil er doch in der That auf einen vorausgesetzten göttlichen Willen gegründet werden könnte; \RWbet{denn wir wissen ja, daß Gott selbst wolle, wir sollen Alles glauben, was unsere Tugend und Glückseligkeit befördert.} Unser Glaube also würde den Namen einer göttlichen Offenbarung verdienen, so bald wir ihn darum annehmen würden, weil wir erkannt haben, daß es der Wille Gottes sey.
\end{aufza}\par
Hierauf erwidere ich, daß jener Wille Gottes, auf den sich ein Offenbarungsglaube stützt, nach der \RWparnr{30}\ gegebenen Erklärung kein aus solchen bloß \RWlat{a priori} geschöpften Gründen, sondern nur ein aus irgend einer \RWbet{gemachten Wahrnehmung erkannter göttlicher Wille} seyn müsse.
\begin{aufza}\setcounter{enumi}{2}
\item Man könnte endlich sagen, aus der \RWbet{sittlichen Zuträglichkeit} einer Lehre lasse sich auf die Wahrheit derselben \RWbet{mittelbar} durch Gottes Heiligkeit schließen. Denn, weil Gott heilig ist, also Tugend und Glückseligkeit, so viel es nur immer möglich ist, zu befördern sucht: so muß er auch machen, daß eine Lehre, welche die höchste Zuträglichkeit besitzt, von uns geglaubt werden könne. Weil wir dieß aber nicht anders vermögen, als wenn sie wahr ist, so muß er auch alle die Anstalten treffen, welche von seiner Seite in dieser Lehre vorausgesetzt werden, \dh\ er muß sie selbst wahr machen. -- Hierauf erwidere ich, daß
\begin{aufzb}
\item \RWbet{ein solcher Schluß immer sehr unsicher bliebe}. Denn es ist wahrlich doch sehr unsicher geschlossen, daß Gott, weil er die Tugend und Glückseligkeit des Ganzen befördern muß, auch so weit gehen müsse, daß er alle Anstalten, welche in einer gewissen Lehre von seiner Seite vorausgesetzt werden, nur darum wirklich machen müsse, damit wir diese uns wohlthätig scheinende Lehre gläubig an\RWSeitenw{367}nehmen könnten. Wie? würde denn nicht aus einem gleichen Grunde folgen, daß jeder einzelne Mensch von jeder \RWbet{einzelnen Hoffnung}, die seiner Tugend und Glückseligkeit zuträglich scheint, erwarten dürfte, daß Gott diese Hoffnung wahr machen werde?
\item Ferner gilt auch hier wieder die schon bei \no\,2\ angebrachte Bemerkung, daß ein auf diese Art erzeugter Glaube in keinem Falle ein Offenbarungsglaube wäre, weil er sich auf kein eigentliches göttliches Zeugniß stützte.
\end{aufzb}
\end{aufza}

\RWpar{147}{Nähere Beschreibung der Beschaffenheit, die ein Ereigniß haben muß, um zum Beweise einer göttlichen Offenbarung zu dienen}
Wir sagten \RWparnr{144}, daß ein Ereigniß, das zum Beweise \RWbet{einer göttlichen Offenbarung} dienen soll, folgende zwei Beschaffenheiten haben müsse:
\begin{aufzb}
\item Es muß in einer solchen \RWbet{Verbindung} mit der Lehre stehen, daß wir sie nicht auch eben so gut \RWbet{ohne dasselbe} hätten erfahren, und uns von dem göttlichen Willen, an sie zu glauben, überzeugen können.
\item Es muß ein \RWbet{ungewöhnliches} Ereigniß seyn.
\end{aufzb}
Diese beiden Beschaffenheiten wollen wir nun etwas genauer kennen lernen.
\begin{aufza}
\item Daß die Begebenheit, die zur Bestätigung einer Lehre dienen soll, in einer \RWbet{gewissen Verbindung} mit ihr stehen müssen, ist Jedem einleuchtend. Aber nicht jede Art von Verbindung ist hiezu hinreichend. Denn zu Folge der kosmologischen Lehre vom Zusammenhange der Welt (\RWlat{de nexu cosmico}) stehen wohl alle Dinge der Welt in einer gewissen \RWbet{Verbindung} mit einander. Also die außerordentlichen Begebenheiten, die sich \zB\ zu Jesu Zeiten in Palästina zutrugen, stehen in einer gewissen Verbindung auch mit der Lehre, die \RWbet{Confucius} 500 Jahre früher in China aufgestellt hat. Gleichwohl wird Niemand sagen, daß jene zu einem Beweise für die Göttlichkeit dieser dienen; sie stehen nämlich, obwohl in einer gewissen, doch nicht in jener eigen\RWSeitenw{368}thümlichen Verbindung, die hier erfordert wird. -- Es frägt sich also, von welcher Beschaffenheit diese seyn müsse? Die allgemeine Antwort auf diese Frage, die ich schon \RWparnr{144}\ gab, ist diese: das Ereigniß muß mit der Lehre in einer Verbindung von der Art stehen, daß wir die letztere \RWbet{nicht auch eben so gut ganz ohne das erstere} hätten erfahren, und von ihrer Gewißheit uns überzeugen können. Versuchen wir aber die Art und Weise, \RWbet{wie} dieß geschehen könne, näher zu bestimmen; so zeigt sich, daß es der Arten \RWbet{mehre} gebe, die einen bald mehr bald minder \RWbet{innigen} Zusammenhang erzeugen.
\begin{aufzb}
\item Der \RWbet{innigste Zusammenhang} zwischen der Lehre und dem Ereignisse, das zu ihrer Bestätigung dienen soll, ist ohne Zweifel vorhanden, wenn das Ereigniß zur \RWbet{Entstehung} der Lehre selbst mitgewirkt hat; \zB\ wenn eine vom Himmel herab tönende Stimme uns die Worte zuriefe: Menschen vernehmet, ich, euer Gott, habe euch zur Unsterblichkeit geschaffen! Denn wenn die Lehre, deren Entstehung selbst schon ein ungewöhnliches Ereigniß ist, sittliche Zuträglichkeit hat; so ist der Nutzen, den das Ereigniß stiftet, wenn wir die Lehre annehmen, am unverkennbarsten, weil wir ja ohne dasselbe sie gar nicht kennen gelernt hätten. Im Gegentheile aber, wenn wir die Lehre nicht annehmen, dann hat das Ereigniß, welchem sie ihre Entstehung verdanket, nicht den geringsten, für uns erkennbaren Nutzen.
\item Ein anderer gleichfalls hinreichend inniger Zusammenhang zwischen Lehre und Ereigniß ist da, wenn wir uns zum Beweise, daß Gott eine gewisse Lehre geglaubt haben wolle, die Erscheinung eines ungewöhnlichen Ereignisses wünschen, Gott um dasselbe anrufen, und dasselbe nun wirklich so, wie wir gewünscht hatten, eintritt. Hier ist zu erklären, aus welcher Absicht Gott das ungewöhnliche Ereigniß gerade so herbeiführt, wie wir es eben von ihm verlangten. Und diese Erklärung wird durch die Annahme, daß er es zur Bestätigung der Lehre thue, auf das Vollständigste gegeben. Der Fall ist nämlich ganz einerlei mit demjenigen, wo wir einen Men\RWSeitenw{369}schen, der uns zwar sprechen hört und auch versteht, aber nur selbst nicht sprechen kann, etwa weil seine Zunge gelähmt ist, auffordern, eine ihm vorgelegte Frage durch irgend ein Zeichen mit der Hand so oder anders zu erwidern. Wenn er dieß Zeichen wirklich von sich gibt, so zweifeln wir nicht einen Augenblick, es sey die Antwort auf unsere Frage; vorausgesetzt, daß das Zeichen eine Bewegung ist, die wir sonst nicht an ihm bemerken. Denn wäre es \zB\ eine Zuckung, die wir schon ohnehin von Zeit zu Zeit eintreten sahen, auch wenn wir ihn um nichts befragten; so würde freilich unsere Erklärung nicht angehen. -- Wenden wir dieses auf Gott an; so ist auch er ein Wesen, das unsere Fragen und Bitten hört und versteht, allein weil er ein übersinnliches Wesen ist, uns nicht durch Sprache antworten kann. Wir fordern ihn also auf, uns durch ein Ereigniß, das sonst nicht einzutreten pflegt, durch irgend ein \RWbet{ungewöhnliches} Ereigniß zu erwidern, ob er diese Lehre von uns geglaubt wissen wolle. Erfolgt nun dieses Ereigniß wirklich, von dessen Erscheinung wir seiner Ungewöhnlichkeit wegen, wenn es nicht eine Antwort auf unsere Frage seyn sollte, sonst keinen Zweck anzugeben wüßten; den Zweck einer solchen Antwort aber darum wohl annehmen können, weil diese Antwort einen uns einleuchtenden Nutzen hat, die Bestätigung jener sittlich zuträglichen Lehre nämlich: so kann uns nichts hindern, anzunehmen, daß es wirklich zu diesem Zwecke veranstaltet sey.
\item Ein Gleiches ist der Fall, wenn das Ereigniß nicht \RWbet{von uns selbst}, sondern von \RWbet{demjenigen}, der die Lehre vortrug, gewünscht wurde, wenn er Gott laut aufforderte, daß er doch zur Bestätigung seiner Lehre ein außerordentliches Ereigniß wirken möge, und es erfolgte nun in der That ein solches. Auch auf diesen Fall läßt sich das vorhin gebrauchte Gleichniß mit einer kleinen Abänderung anwenden, und die Schlußart wiederholen.
\item Es ist nicht einmal nöthig, daß jene Aufforderung Gottes von Seite des Lehrers mit \RWbet{ausdrücklichen Worten} geschehe, oder daß das ungewöhnliche Ereigniß \RWbet{vorhergesagt} worden sey; sondern es genügt, wenn sich der Lehrer nur so benom\RWSeitenw{370}men hat, und Alles so erfolgt ist, daß wir am Ende sehen, er habe das Ereigniß im Stillen gewünscht und vorhergesehen; denn auf die \RWbet{ausdrücklichen Worte} kommt hier nichts an. Das Gleichniß läßt sich noch immer anwenden.
\item Ja schon genug, wenn das Ereigniß dem Lehrer oder der Lehre nur zu irgend einem sichtbaren Vortheile gereicht, er mag es übrigens bestimmt vorhergesehen oder nicht vorhergesehen haben. Denn wenn der Lehrer durch ein ungewöhnliches Ereigniß, \zB\ vom Tode errettet, oder in den Stand gesetzt wird, seine Lehre noch weiter auszubreiten, \usw: so hindert uns nichts, diesem Ereignisse den Zweck beizulegen, daß uns Gott durch dasselbe sein Wohlgefallen an jener Lehre habe bezeugen wollen, wenn sie anders eine für uns bemerkbare sittliche Zuträglichkeit hat. Ist uns nun keine andere Erklärung des Zweckes dieses Ereignisses bekannt, weil es ein ungewöhnliches ist; so handeln wir vernünftig, uns an jene zu halten.
\end{aufzb}
\begin{RWanm}
Wer sich die hier befolgte Art zu schließen gehörig eigen gemacht hat, wird einsehen, \RWbet{daß das bloße Ausbleiben eines verlangten außerordentlichen Ereignisses} zur Bestätigung einer Lehre noch gar \RWbet{kein Zeugniß wider sie sey}. Wohl aber würden wir, wenn die Lehre eine für uns bemerkbare \RWbet{Schädlichkeit} hat, und auf die Aufforderung desjenigen, der sie uns vorträgt, zwar eine ungewöhnliche, aber nicht die verlangte, sondern eine ihr vielmehr ganz \RWbet{entgegengesetzte} Begebenheit erfolgt; ja auch schon dann, wenn eine außerordentliche Begebenheit, welche dem Lehrer nachtheilig ist, ohne vorhergegangene Aufforderung des Himmels eintritt, berechtiget seyn, eine solche Begebenheit als einen Beweis des göttlichen \RWbet{Mißfallens} an dieser Lehre, und also den Satz, daß sie falsch ist, das contradictorische Gegentheil derselben, als eine von Gott bezeugte Wahrheit zu betrachten. Ein Beispiel gibt das Ereigniß, das uns \RWbibel{Apg}{Apostelgesch.}{12}{20\,ff}\ erzählt wird. 
\end{RWanm}
\item Die zweite Eigenschaft eines Ereignisses, das zum Beweise einer göttlichen Offenbarung dienen soll, ist die \RWbet{Ungewöhnlichkeit}. So bekannt es auch schon aus dem gemeinen Sprachgebrauche ist, was eine \RWbet{ungewöhnliche Erscheinung}~\RWSeitenw{371}\ heiße, so dürfte es doch in einem wissenschaftlichen Vortrage nicht am unrechten Orte seyn, eine \RWbet{Erklärung} dieses Begriffes zu versuchen. Wenn wir auf die verschiedene Beschaffenheit und Zeitfolge aller der einzelnen Wahrnehmungen, die wir unser Leben hindurch zu machen Gelegenheit haben, nur etwas aufmerksam sind: so entdecken wir bald verschiedene \RWbet{Regeln}, nach denen sich jene Wahrnehmungen in ihrer Zeitfolge richten. So entdecken wir \zB , daß der Anblick einer Rose meistens begleitet sey mit der Empfindung eines gewissen Geruches, welchen wir eben deßhalb den Geruch der Rose nennen. Man pflegt solche Regeln \RWbet{Erfahrungsregeln} zu nennen. Eine Erscheinung nun, die keiner einzigen dieser von uns gebildeten Erfahrungsregeln zuwiderläuft, nennen wir eine \RWbet{regelmäßige, ordentliche} oder \RWbet{gewöhnliche Erscheinung}. Eine Erscheinung dagegen, die irgend einer von diesen Regeln widerspricht, bei der \zB\ gewisse Wahrnehmungen \ensuremath{a, b, c, d}, auf welche sonst immer gewisse andere \ensuremath{\alpha, \beta, \gamma, \delta} 
%\RWgriech{a}, \RWgriech{b}, \RWgriech{g}, \RWgriech{d} 
zu folgen pflegten, vorkommen, ohne von diesen letztern begleitet zu werden, heißt eine \RWbet{unregelmäßige, außerordentliche, ungewöhnliche} Erscheinung. So pflegt die Erscheinung, die wir das Sterben eines Menschen nennen, insgemein begleitet zu seyn mit einer bald darauf folgenden Reihe von Wahrnehmungen, welche wir das \RWbet{Verwesen} des Leichnames nennen. Wenn wir nun einmal die erste Reihe von Wahrnehmungen antreffen sollten, ohne daß darauf die letztere folgte, \dh\ wenn wir einmal einen Menschen sterben sehen sollten, ohne daß sein Leib in Verwesung überginge, wenn er im Gegentheile wieder lebendig würde: so würden wir dieß eine ungewöhnliche Erscheinung nennen. Aus dieser Erklärung ergibt sich, daß die \RWbet{Ungewöhnlichkeit} einer Erscheinung eine Eigenschaft sey, bei der ein Mehr oder Weniger, \dh\ ein Grad Statt findet. Je größer nämlich die Anzahl der Fälle ist, aus deren Beobachtung wir eine gewisse Erfahrungsregel abgezogen haben, \dh\ je öfter wir bereits bemerkt, daß die Wahrnehmungen \ensuremath{a, b, c, d} immer begleitet waren von den Wahrnehmungen \ensuremath{\alpha, \beta, \gamma, \delta}, desto größer ist der \RWbet{Grad} der Ungewöhnlichkeit einer Erscheinung, welche von dieser Regel abweicht, \dh\ desto größer ist die \RWbet{Ungewöhnlichkeit} eines~\RWSeitenw{372}\ Ereignisses, bei dem die Wahrnehmungen \ensuremath{a, b, c, d} ohne Begleitung der Wahrnehmungen \ensuremath{\alpha, \beta, \gamma, \delta} vorkommen.
\end{aufza}
\begin{RWanm} 
Meist wird das \RWbet{Ungewöhnliche} erklärt als etwas, das \RWbet{noch nie war}, oder \RWbet{doch nur selten vorkommt}. Diese Erklärung aber däucht mir ganz unrichtig zu seyn. Denn ihr zu Folge würde ja eine jede aus mehren einfachen Wahrnehmungen \RWbet{zusammengesetzte Erscheinung} eine ungewöhnliche seyn; weil sich an einem jeden zusammengesetzten Gegenstande einzelne Merkmale vorfinden, die ganz in derselben Verbindung, in der sie hier stehen, an keinem zweiten vorkommen. Nach dem schon einmal erwähnten Grundsatze \RWbet{Leibnitzen's} gibt es, und kann es auch nicht zwei Dinge (also auch nicht zwei Ereignisse) geben, welche einander in allen Stücken gleich sind. Gewiß wäre es nichts Außerordentliches, eine Tulpe zu finden, der keine andere so vollkommen ähnlich ist, daß man nicht irgend einen Unterschied bemerkte. Wohl aber wäre es etwas Außerordentliches, wenn Jemand eine Tulpe, die ohne Staubfäden ist, fände. Schon von der ersteren könnte man sagen, daß sie ein Gegenstand sey, der noch nie war; nämlich vollkommen so, wie diese war noch keine andere Tulpe gebaut. Allein dieß macht das Wesen des Außerordentlichen nicht aus. Bei der zweiten Tulpe dagegen kommt eine Trennung von Wahrnehmungen vor, die man sonst immer beisammen anzutreffen pflegt. Wo nämlich die Blumenblätter und der Griffel einer Tulpe angetroffen werden, da trifft man insgemein auch ihre Staubfäden an. Das Außerordentliche also, oder das Ungewöhnliche besteht nicht in einer \RWbet{Verbindung} von Wahrnehmungen, wie sie noch nie beisammen waren, sondern in einer \RWbet{Trennung} solcher, die stets beisammen waren. Zwar dürfte Jemand sagen, daß auch die hier versuchte Erklärung unrichtig sey, weil es ja doch Erscheinungen geben könnte, deren Ungewöhnlichkeit nicht in dem \RWbet{Wegbleiben} gewisser Umstände, welche sonst einzutreten pflegen, sondern in der Beschaffenheit der Wahrnehmungen \RWbet{selbst} läge, nämlich, wenn wir nur eine einzige, bisher noch nie empfundene Wahrnehmung erhielten. So nennen wir \zB\ einen Schmerz, den wir in unserem ganzen Leben noch nie empfunden hatten, mit Recht einen \RWbet{ungewöhnlichen Schmerz}. Doch eine nähere Betrachtung zeigt, daß sich auch diese Art von Ungewöhnlichkeit einer Erscheinung unter die obige Erklärung bringen lasse. Die Erscheinung nämlich, \RWbet{daß wir jetzt eine Empfindung haben, die wir durch unser}~\RWSeitenw{373}\ \RWbet{ganzes bisheriges Leben noch nie gehabt hatten} -- heißt eine ungewöhnliche, nur in sofern als auch sie gleichfalls Einer von unsern Erfahrungsregeln widerspricht, der nämlich, daß wir, haben wir erst ein gewisses höheres Alter erreicht, fortwährend nur Empfindungen erfahren, die irgend einer schon früher gehabten mehr oder weniger gleichkommen.
\end{RWanm}
   
\RWpar{148}{Wenn eine Lehre keine sittliche Zuträglichkeit hat, so kann sie durch kein auch noch so ungewöhnliches Ereigniß als eine göttliche Offenbarung erwiesen werden}
Man könnte, und ist auch wirklich auf den Gedanken gerathen, \RWbet{daß vielleicht ungewöhnliche Ereignisse für sich allein schon hinreichend wären, uns eine Lehre als eine wahre göttliche Offenbarung zu erweisen}. Dieses muß aber eben so, wie die ähnliche Vermuthung (\RWparnr{146}), daß die innere Vortrefflichkeit der Lehre zum Beweise einer Offenbarung allein genüge, \RWbet{verneinet} werden. Denn auch hier werden alle Gründe, die zur Unterstützung dieses Gedankens angeführt werden könnten, bei einer näheren Prüfung als unhaltbar befunden.
\begin{aufza}
\item Der \RWbet{gewöhnlichste} Grund, den man hier angibt, lautet: Wenn es der Wille Gottes nicht wäre, daß wir eine gewisse Lehre glauben, so dürfte er auch keine außerordentlichen Begebenheiten mit ihr in Verbindung setzen; denn hiedurch führt er uns ja nothwendig irre. -- Aber wer sieht nicht ein, daß man in diesem Schlusse schon annehme, was erst bewiesen werden soll? Denn, wenn noch unentschieden ist, ob außerordentliche Begebenheiten allein zur Bestätigung einer Lehre genügen; so kann man auch nicht sagen, daß uns Gott irre führe, wenn er dergleichen Ereignisse bei einer Lehre zuläßt, die er von uns nicht geglaubt wissen will.
\item Scheinbarer wäre schon folgender Grund: Wenn Gott auf das \RWbet{Gebet} eines Lehrers außerordentliche Begebenheiten gerade so, wie dieser sie verlangt, erfolgen läßt; so bezeuget er hiedurch ein überaus großes \RWbet{Wohlgefallen} an diesem Manne, und hiedurch mittelbar auch an seiner Lehre; er bedeutet uns also hiedurch, daß wir die letztere annehmen sollen. -- Allein ich antworte hierauf, es sey kein sicherer Schluß, daß Gott an demjenigen, dessen Gebete er erhört, ein besonderes~\RWSeitenw{374}\ Wohlgefallen habe; denn Gott erhört ja auch zuweilen das Gebet der Sünder und der Thoren zu ihrer eigenen Bestrafung. Aus andern Umständen also muß erst entschieden werden, ob diese Erhöhrung der Bitte als ein Beweis des göttlichen Wohlgefallens anzusehen sey oder nicht. Bei einer nähern Betrachtung zeigt es sich nun, daß diese andern Umstände nur in der \RWbet{innern Vortrefflichkeit der Lehre selbst} bestehen können. Denn um mit Recht annehmen zu können, daß Jemand \RWbet{Gottes Wohlgefallen habe}, müssen wir erst einen sittlichen Charakter bei ihm wahrgenommen haben; um ferner annehmen zu dürfen, daß Gott aus Wohlgefallen zu diesem Manne auch so weit gehen werde, daß er ihm seine Bitte um die Bestätigung seines Lehrvortrages gewähre, müssen wir auch an diesem Lehrvortrage selbst die größte Zuträglichkeit bemerken.
\item Man könnte weiter sagen: Eine Lehre kann sittliche Zuträglichkeit für uns besitzen, ohne daß wir sie erkennen. Diese von uns nicht erkannte sittliche Zuträglichkeit kann Gott bestimmen, uns die Lehre zu offenbaren; und hiezu wirkt er jene außerordentlichen Begebenheiten, aus denen wir also, auch ohne den Nutzen der Lehre selbst bemerken zu wollen, auf seinen Willen, daß wir die Lehre annehmen, schließen sollten. -- Ich antworte, es sey eine längst bekannte Regel der Logik, daß \RWbet{eine jede Annahme oder Hypothese, die eben so unerklärlich ist, wie dasjenige, zu dessen Erklärung sie ausgedacht wurde, zu verwerfen sey}. Um zu erklären, warum Gott gewisse außerordentliche Begebenheiten veranlasset habe, dürfen wir also die Hypothese, daß er sie zur Bestätigung einer gewissen Lehre herbeigeführt habe, nicht machen, wenn sich an dieser Lehre keine sittliche Zuträglichkeit bemerken läßt. Denn sonst ist die Annahme eben so unerklärlich, wie die Begebenheit selbst, zu deren Erklärung sie gemacht wird. Nun entsteht nämlich die weitere Frage, \RWbet{warum Gott wolle, daß wir diese Lehre glauben}? Und die Antwort: Sie kann einen Nutzen haben, ohne daß wir ihn sehen, wird durch die Gegenantwort: Auch jene Ereignisse können noch einen andern Zweck haben, den wir nicht sehen, aufgehoben.
\item Man könnte endlich sagen: Es ist doch einmal der allgemeine Glaube der Menschen, daß außerordentliche Bege\RWSeitenw{375}benheiten zur Bestätigung einer Lehre hinreichen. Also schon um dieses Glaubens willen darf Gott dergleichen Begebenheiten nirgends als nur bei einer Lehre zulassen,
die er von uns wirklich geglaubt wissen will. -- Hierauf erwidere ich aber, es sey
\begin{aufzb}
\item nach Ausweis des \RWparnr{141} keineswegs der \RWbet{allgemeine} Glaube, daß außerordentliche Begebenheiten \RWbet{allein} schon zur Bestätigung einer Offenbarung hinreichen. Gesetzt aber, dieß wäre allgemeiner Glaube, so müßte es
\item wohl eben darum auch \RWbet{Wahrheit} seyn, weil der gemeine Menschenverstand in Dingen dieser Art nicht leicht zu irren pflegt; und die Sache des Weltweisen wäre es nun, den innern Grund dieser Wahrheit anzugeben. Allein so eben sahen wir, daß sich kein solcher Grund auffinden lasse.
\end{aufzb}
\end{aufza}

\RWpar{149}{Wie und wodurch eigentlich außerordentliche Ereignisse, wenn sie in der gehörigen Verbindung mit einer sittlich zuträglichen Lehre stehen, zu ihrer
Bestätigung dienen}
Wenn außerordentliche Ereignisse \RWbet{für sich allein}, wie wir so eben gesehen, zum Beweise einer göttlichen Offenbarung nicht zureichend sind: wie kommt es, möchte man fragen, daß sie zu diesem Zwecke zureichend werden, sobald die sittliche Zuträglichkeit der Lehre noch hinzukommt? Wie sollen überhaupt \RWbet{Ereignisse}, sie mögen nun gewöhnliche oder ungewöhnliche seyn, uns von dem Willen Gottes, daß wir eine gewisse Lehre gläubig annehmen sollen, unterrichten können? Sie besitzen ja doch keine \RWbet{Sprache}; wie und \RWbet{wodurch} also können sie uns die \RWbet{Nachricht} von Gottes Willen ertheilen? -- Meine \RWbet{Antwort} auf diese Frage lautet: Jedes Ereigniß, es sey gewöhnlich oder ungewöhnlich, rührt näherer oder entfernterer Weise von Gott her, dergestalt, daß er, wenn auch nicht der bestimmende Grund, doch eine Bedingung seines Daseyns ist. Bei jedem Ereignisse ferner hat Gott auch irgend einen bestimmten \RWbet{Zweck}, und uns Menschen ist es nicht nur erlaubt, sondern es ziemet uns sogar, diesen Zweck, so viel es unsere beschränkten Einsichten vermögen, mit Bescheidenheit zu erforschen. Wenn das Ereigniß~\RWSeitenw{376}\ ein \RWbet{ungewöhnliches} ist; so muß auch der Zweck desselben ein \RWbet{ungewöhnlicher} seyn, indem gewöhnliche Wirkungen auch nur gewöhnlicher Ursachen bedürfen. Es ist daher eine um so wichtigere Aufgabe für unser Nachdenken, den Zweck eines solchen Ereignisses zu erfahren. Aus Gottes Heiligkeit aber wissen wir, daß dieser Zweck irgend ein guter seyn müsse. Wenn wir nun durchaus keine gute, \dh\ gemeinnützige Wirkung entdecken können, die das Ereigniß sonst hervorbrächte: ist es nicht natürlich, auf den Gedanken zu kommen, Gott habe es \RWbet{zur Bestätigung jener Lehre}, mit der es in Verbindung steht, gewirkt? Wenn diese Lehre eine für uns bemerkbare sittliche Zuträglichkeit hat; so stiftet das Ereigniß in dem Falle, daß es uns zum Glauben an diese Lehre bestimmt, einen bemerkbaren Nutzen, und somit ist der Zweck seines Daseyns erklärt. Hat aber diese Lehre keine für uns bemerkbare sittliche Zuträglichkeit; so hätten wir immer noch keinen Nutzen als Zweck jenes Ereignisses angegeben, auch wenn wir angeben wollten, daß es zur Bestätigung der erwähnten Lehre gewirkt sey. Also sind außerordentliche Begebenheiten wohl dann, wenn sie mit einer sittlich zuträglichen Lehre in Verbindung stehen, nicht aber außerdem zu ihrer Bestätigung tauglich. Daß Ereignisse keine \RWbet{Sprache} besitzen, ist eine sehr falsche Behauptung, wenn sie den Sinn haben soll, daß Ereignisse uns nicht \RWbet{belehren}, nicht \RWbet{etwas kund geben könnten}. Sagen wir doch von uns \RWbet{Menschen}, daß unsere \RWbet{Thaten oft lauter} und deutlicher als unsere \RWbet{Worte} reden; und die Ereignisse der Welt können, weil Gott sie alle näherer oder entfernterer Weise hervorbringt, als Thaten Gottes angesehen werden. Kein Zweifel also, daß sie uns auch seinen Willen kund geben können, daß er durch sie gleichsam zu \RWbet{sprechen} vermöge. Was aber insbesondere die \RWbet{außerordentlichen} Ereignisse anlangt; so gibt es mehre Umstände, um derentwillen diese ganz vornehmlich geeignet sind, uns Aufschlüsse über den göttlichen Willen zu geben, und zur Bestätigung seiner Offenbarung zu dienen:
\begin{aufza}
\item \RWbet{einmal} schon darum, \RWbet{weil sie Ereignisse sind, welche sonst keinen bemerkbaren Zweck ihres Daseyns haben, während jedes alltägliche Ereigniß}, eben weil es \RWbet{alltäglich} ist, auch~\RWSeitenw{377}\ einen \RWbet{alltäglichen} Zweck voraussetzt, der folglich, er sey bekannt oder nicht, sicher doch nicht der Zweck der Beglaubigung einer Offenbarung seyn kann. Außerordentliche Ereignisse sind zu diesem letztern Zwecke auch um so \RWbet{tauglicher}, weil sie
\item \RWbet{die Aufmerksamkeit unseres Geistes auf sich ziehen}, und uns zur näheren Betrachtung und Prüfung, wie ihrer selbst, so auch der religiösen Lehre, mit der sie verbunden sind, einladen. Ein Vortheil, der um so wichtiger ist, je weniger die Menschen, wie ich schon \RWparnr{119} erinnerte, geneigt und aufgelegt sind, \RWbet{abstracte Wahrheiten} zu untersuchen, wenn ihre Aufmerksamkeit auf sie nicht erst durch etwas Auffallendes rege gemacht worden ist. Hiezu gesellet sich
\item der dritte Umstand, daß \RWbet{außerordentliche Begebenheiten uns unmittelbarer als andere} an \RWbet{Gott}, als ihren eigentlichen \RWbet{Urheber}, erinnern. Alltägliche Begebenheiten nämlich wissen wir eben um ihrer Alltäglichkeit willen fast alle aus ihren nächsten Ursachen (Menschen, Thieren, und leblosen Geschöpfen) nach gewissen, durch die Erfahrung uns bekannt gewordenen Gesetzen herzuleiten. Dieß macht, daß wir sie meistens auch nicht anders anzusehen pflegen, als ob der völlig hinreichende Grund ihres Daseyns bloß in jenen nächsten Ursachen läge; an ihren letzten Grund aber, an Gott, pflegen wir bei ihnen seltener zu denken. Außerordentliche Ereignisse dagegen wissen wir aus ihren nächsten Ursachen schwer oder gar nicht herzuleiten; daher erinnern sie uns unmittelbar an Gott, der, wie der letzte Grund von \RWbet{Allem}, so auch \RWbet{ihr} Urheber seyn muß. Die Richtigkeit dieser Bemerkung beweiset der Sprachgebrauch, der von dergleichen außerordentlichen Begebenheiten vorzugsweise die Redensart gebraucht, \RWbet{daß sie Gott selbst hervorgebracht habe}. Erinnern wir uns aber deutlicher daran, daß Gott ihr Urheber sey; so fällt uns auch leichter ein, daß er einen bestimmten Zweck bei ihnen habe, und wir entschließen uns zu seiner Aufsuchung.
\item Ein vierter Umstand, der die Tauglichkeit außerordentlicher Begebenheiten zu dem Zwecke der Beglaubigung~\RWSeitenw{378}\ einer Offenbarung noch erhöht, ist, \RWbet{daß ihre Herbeiführung eine besondere Sorgfalt von Seite Gottes}, besondere Anstalten und Zurüstungen \RWbet{zu verrathen scheint}. Stehen diese außerordentlichen Begebenheiten mit einem gewissen religiösen Lehrbegriffe in genauerer Verbindung, hat er ihnen \zB\ seine Entstehung, Erhaltung oder Ausbreitung zu danken; so ist wohl nichts natürlicher, als der Schluß, es müsse Gott sehr viel gelegen seyn an diesem Lehrbegriffe; er werde ihn wohl nicht umsonst mit so vielem Kraftaufwande, durch so viele Zurüstungen zum Vorschein gebracht, erhalten und ausgebreitet haben; er habe uns sicher nur darum in seine Kenntniß gesetzt, damit wir ihn gläubig annehmen sollen.
\item Zuletzt sind außerordentliche Begebenheiten auch darum sehr tauglich zum Zwecke der Beglaubigung einer Offenbarung, weil alle \RWbet{jene Schlüsse}, durch die wir bei der Wahrnehmung solcher Begebenheiten auf Gottes Absicht schließen, uns eine gewisse Lehre durch sie zu bestätigen, selbst jedem \RWbet{ungebildeten Verstande} vollkommen \RWbet{einleuchtend} und \RWbet{geläufig} sind. Man gehe nur die so eben aufgestellten Puncte noch einmal in dieser Rücksicht durch, und man wird finden, daß auch der gemeinste Menschenverstand täglich dergleichen Schlüsse bildet.
\end{aufza}

\RWpar{150}{Aufstellung der beiden Kennzeichen einer göttlichen Offenbarung}
Aus dem Bisherigen ergibt sich, daß eine Lehre nur dann für Gottes Offenbarung von uns erkannt werden könne, dann aber auch mit einem mehr oder weniger hohen Grade der Wahrscheinlichkeit dafür erkannt werden müsse, wenn sich bei ihr folgende zwei Stücke finden:
\begin{aufzb}
\item \RWbet{wenn diese Lehre die \RWparnr{145}\ erklärte sittliche Zuträglichkeit für uns besitzt}, und
\item \RWbet{wenn sich gewisse außerordentliche Begebenheiten in der \RWparnr{147}\ näher bestimmten Verbindung mit ihr befinden.}
\end{aufzb}
Für sich allein genommen ist jedes dieser zwei Stücke unzulänglich, uns zu dem Urtheile, daß eine gewisse~\RWSeitenw{379}\ Lehre eine wahre göttliche Offenbarung sey, zu berechtigen (\RWparnr{146} und \RWparnr{148}). Beide in Vereinigung aber sind allerdings hinreichend. So einleuchtend dieß Letztere auch schon aus dem \RWparnr{149}\ Gesagten folgt; so wird es gleichwohl durch einen noch zwingenderen Schluß bestätiget werden, bis wir jetzt alle Einwürfe, welche man gegen die Möglichkeit einer Offenbarung erhoben hat, vernommen, und ihre Nichtigkeit eingesehen haben werden. Diese zwei Stücke können wir also wohl die \RWbet{Kennzeichen} oder \RWbet{Merkmale} einer göttlichen Offenbarung nennen, indem es gewöhnlich ist, solche Beschaffenheiten einer Sache, deren Vorhandenseyn man erst bemerkt haben muß, bevor man berechtiget ist, ihr ein gewisses Prädicat beizulegen, \RWbet{Kennzeichen oder Merkmale dieses Prädicats} zu nennen. Bevor wir nun berechtiget sind, einer Religion das Prädicat einer göttlichen Offenbarung beizulegen, müssen wir erst bemerkt haben, daß die zwei eben genannten Stücke sich an ihr befinden; und wenn wir sie wahrgenommen, sind wir hiezu berechtigt. Billig verdienen sie also den Namen \RWbet{der Kennzeichen einer göttlichen Offenbarung}. Dem \RWbet{ersten} Stücke kann man, wiefern es eine sich \RWbet{an der Lehre selbst befindliche} Beschaffenheit ist, den Namen des \RWbet{innern}, dem zweiten den Namen des \RWbet{äußeren Kennzeichens} geben.
\begin{RWanm}
Es könnte sich wohl ereignen, daß zwei oder mehre Religionen zugleich innere Vortrefflichkeit in ihrer Lehre, und gewisse außerordentliche Begebenheiten zu ihrer Bestätigung aufzuweisen hätten. Was wäre nun da zu thun? Sie alle \RWbet{zugleich} für göttliche Offenbarungen an uns erklären, hieße eine Ungereimtheit begehen. Gleichwohl besitzen sie, wie es scheint, jede die von uns aufgestellten Kennzeichen. Sollte dieß also nicht verrathen, daß unsere Theorie noch einen Mangel habe? -- Wer sich die bisherigen Begriffe gehörig eigen gemacht hat, wird diese Schwierigkeit leicht zu beseitigen wissen. Allerdings kann es mehre Religionen geben, welche gewisse außerordentliche Begebenheiten aufzuweisen haben, von deren Vorhandenseyn wir keinen Zweck einsehen, wenn wir nicht annehmen wollen, daß sie zu ihrer Bestätigung \RWbet{für uns oder Andere}~\RWSeitenw{380} da sind. Wenn ferner diese Religionen in ihrer Lehre einander nicht \RWbet{widersprechen}, wenn Eine nur mehr als die andere lehret; so kann es auch wohl seyn, daß jede für sich allein betrachtet sittliche Zuträglichkeit für uns hat; verglichen mit einander aber kommt diese Beschaffenheit immer nur \RWbet{Einer} aus ihnen, nämlich derjenigen zu, welche die meisten für uns zuträglichen Lehren enthält. Kein Zweifel also, daß Gott nur diese Eine von uns geglaubt wissen wolle, \dh\ daß wir nur diese allein als eine wahre göttliche Offenbarung für uns ansehen dürfen. Der Zweck, den die außerordentlichen Ereignisse der \RWbet{anderen Religionen} haben, bleibt uns entweder unbekannt; was nichts Befremdendes wäre, da es so viele Ereignisse gibt, deren Zweck wir nicht kennen; oder er liegt in der Beglaubigung dieser Religionen für andere Menschen, für welche, weil sie auf einer andern Stufe der Bildung stehen, und sich in andern Verhältnissen befinden, gerade diese Religionen vielleicht zuträglicher sind als die unsrige. Und so bestätiget sich denn die eben vorgetragene Lehre von den Kennzeichen einer Offenbarung auch noch in diesem Falle; und die ganze Schwierigkeit verschwindet, sobald man sich nur erinnert, wie der Begriff der sittlichen \RWbet{Zuträglichkeit} einer Lehre schon \RWparnr{145}\ festgesetzt wurde. 
\end{RWanm}

\RWpar{151}{Erklärung der Begriffe eines Zeichens oder Wunders und einer Weissagung}
Eine Begebenheit, aus der sich entnehmen läßt, daß Gott eine gewisse Lehre von uns als seine Offenbarung geglaubt wissen wolle, pflegt man mit einem Worte ein \RWbet{Zeichen} dieser Lehre (\RWlat{signum}, \RWgriech{shme~ion}) zu nennen. Die \RWbet{Beschaffenheiten}, die eine solche Begebenheit haben muß, sind nach dem Vorhergehenden:
\begin{aufzb}
\item Sie muß ungewöhnlich seyn.
\item Mit der Lehre, zu deren Bestätigung sie dienen soll, in der \RWparnr{147}\ beschriebenen Verbindung stehen, \dh\ sie muß
\begin{aufzc}
\item zur Entstehung, Erhaltung oder Ausbreitung dieser Lehre etwas beigetragen haben, und
\item es muß sich \RWbet{kein Nutzen} derselben angeben lassen, wenn es nicht der seyn sollte, daß sie uns zur Bestätigung jener Lehre diene.~\RWSeitenw{381}
\end{aufzc} 
\end{aufzb}
Weil also jedes Zeichen irgend ein \RWbet{ungewöhnliches} Ereigniß seyn muß; ungewöhnliche Ereignisse aber Verwunderung erregen, und daher auch den Nahmen \RWbet{Wunder} führen: so ist es Sitte geworden, auch jedes Zeichen ein \RWbet{Wunder} zu nennen, ja diese letztere Benennung ist heut zu Tage bei Weitem gewöhnlicher als die erste.\par
Eine besondere Gattung von Wundern nennt man auch \RWbet{Weissagungen}. Dieß nämlich sind \RWbet{Vorhersagungen}, oder noch allgemeiner zu sprechen -- \RWbet{Vorherdeutungen zukünftiger Ereignisse}, aus deren Erfüllung sich entnehmen läßt, daß Gott eine gewisse Lehre von uns geglaubt wissen wolle. Aus dem Vorhergehenden ergeben sich die Beschaffenheiten, die eine solche Weissagung haben muß, von selbst:
\begin{aufzb}
\item entweder die Vorhersagung oder die Erfüllung muß eine ungewöhnliche Begebenheit seyn, und diese Begebenheit muß
\item mit der Lehre, zu deren Bestätigung sie dienen soll, in der \RWparnr{147}\ beschriebenen Verbindung stehen.
\end{aufzb}
Man pflegt zu sagen, daß \RWbet{Wunder} und \RWbet{Weissagungen} das Daseyn einer göttlichen Offenbarung beweisen. Nimmt man die Worte Wunder und Weissagungen in der so eben erklärten Bedeutung, so ist dieß freilich wahr; denn da versteht man ja unter Wundern und Weissagungen nur eben Ereignisse von einer solchen Art, aus denen sich auf das Daseyn einer göttlichen Offenbarung schließen läßt. Verstände man aber unter Wundern und Weissagungen bloß ungewöhnliche Ereignisse; dann wäre es falsch, daß jedes ungewöhnliche Ereigniß das Daseyn einer Offenbarung beweise.

\RWpar{152}{Prüfung der wichtigsten Einwürfe gegen die Möglichkeit einer Offenbarung. 1.~Einwurf. Geoffenbarte Geheimnisse sollen ein Widerspruch seyn.}
Was wir so eben gesagt, gilt Alles nur so lange, als uns noch Niemand die Unmöglichkeit einer Offenbarung erweiset. Da sich aber die Feinde der göttlichen Offenbarung gerühmt, ihre Unmöglichkeit uns schon erwiesen zu haben: so wollen wir jetzt die wichtigsten ihrer Einwürfe der~\RWSeitenw{382}\ Reihe nach betrachten. Es wird sich zeigen, daß sie alle auf falschen Voraussetzungen, zum Theile auf den elendesten Trugschlüssen beruhen; und so werden wir uns dann um so gewisser von der behaupteten problematischen Möglichkeit einer Offenbarung und von der Gültigkeit der eben aufgestellten Kennzeichen derselben überzeugen.\par
\RWbet{Einwurf.} Der Begriff eines \RWbet{geoffenbarten Geheimnisses} enthält einen Widerspruch schon in sich selbst, beiläufig eben so, wie der eines \RWbet{viereckigen Dreieckes}. Wenn etwas Geheimniß ist, so ist es nicht geoffenbaret, und wenn etwas geoffenbaret ist, so ist es kein Geheimniß. Nun lehren gleichwohl alle Vertheidiger der Offenbarung, daß wenigstens eine materielle Offenbarung Geheimnisse enthalte; sie ist also logisch unmöglich. (Freimüthige Betrachtungen über die dogmatische Lehre von Wundern und Offenbarung. Brief 4.\ S.\,67.)\par
\RWbet{Antwort}. Das elende Wortspiel, auf dem dieser Einwurf beruht, fällt Jedem von selbst in die Augen. Wenn wir gewisse Lehrsätze der Offenbarung \RWbet{Geheimnisse} nennen, so thun wir dieß aus dem doppelten Grunde:
\begin{aufzb}
\item um hiedurch anzuzeigen, daß diese Lehren uns \RWbet{unbekannt} gewesen, und \RWbet{Geheimnisse} geblieben wären, wenn sie uns Gott nicht mitgetheilt hätte; dann aber auch,
\item um zu erinnern, daß sich noch jetzt so manches \RWbet{Unbegreifliche, Unbekannte} und \RWbet{Unerforschliche} an diesen Lehren befinde, \zB\ ihr innerer Grund, ihr Zusammenhang mit anderen Wahrheiten, das Wie und Wodurch, \usw\ Ganz in derselben Bedeutung und mit demselben Rechte nennt man \zB\ entdeckte (\dh\ geoffenbarte) Geheimnisse in der \RWbet{Natur} manche erst unlängst bekannt gewordene Naturkräfte, deren letzten Grund wir nicht zu erklären wissen, \zB\ die Kräfte des Magnets.
\end{aufzb}

\RWpar{153}{2.~Einwurf. Gott soll auf unsere Sinne nicht einwirken können}
Alle Vorstellungen, die wir nur immer erhalten, entstehen in uns durch den Eindruck äußerer (sinnlicher) Gegen\RWSeitenw{383}stände auf unsere Sinneswerkzeuge. Daraus folgt, daß nur \RWbet{sinnliche Gegenstände} in uns Vorstellungen hervorbringen können. Bei einer Offenbarung aber müßte Gott selbst, ein völlig reiner Geist, Vorstellungen in uns erwecken, also auf unsere Sinne, auf unser Gehör, Gesicht \usw\ einwirken; was doch unmöglich ist. (Freimüthige Betrachtungen. Brief 4.\ S.\,72.)\par
\RWbet{Antwort.} Bei diesem Einwurfe liegt
\begin{aufzb}
\item zuvörderst ein \RWbet{falscher Begriff von Offenbarung} zu Grunde. Man stellt sich nämlich vor, als ob die Offenbarung ein \RWbet{unmittelbares Einwirken Gottes auf uns}, auf unsere Sinne, voraussetzte. Denn sollte die Rede von einer bloß mittelbaren Einwirkung seyn; so könnte man wohl die Frage, ob Gott auf unsern Geist einwirken könne, vernünftiger Weise nicht einmal aufwerfen, weil ja alle Einwirkungen auf unsern Geist, die wir nur immer erfahren, mittel- oder unmittelbarer Weise von Gott herrühren. Nun kommt es aber nach \RWparnr{34} bei einer Offenbarung gar nicht darauf an, ob Gott die Vorstellungen in uns mittel- oder unmittelbarer Weise hervorbringe; sondern nur darauf, daß er uns von seinem Willen hinlänglich überzeuge, wir sollen diese wie immer in uns erweckten Vorstellungen glauben; und zur Bezeugung dieses Willens gehört nach \RWparnr{150} nichts Anderes, als daß die Entstehungsart dieser Vorstellungen etwas Ungewöhnliches sey, oder daß doch sonst etwas Ungewöhnliches mit ihnen in die \RWparnr{147} beschriebene Verbindung trete; und endlich, daß jene Vorstellungen sittliche Zuträglichkeit haben.
\item Uebrigens ist nicht erwiesen, daß Gott nicht sogar \RWbet{unmittelbar} auf unsere Sinne, ja selbst auf unsern Geist einwirken könne. Daraus, weil Gott ein \RWbet{reiner Geist ist}, folgt einmal gar nicht, daß er auf die Materie nicht unmittelbar einwirken könne; denn auch \RWbet{unser} Geist wirkt auf gewisse Theile der Materie unsers eigenen Leibes, nämlich auf diejenigen, die man das \RWbet{Seelenorgan} nennt, unmittelbar ein. Eben so wenig läßt sich behaupten, daß ein Geist nicht unmittelbar auf einen \RWbet{andern} Geist einwirken, und gewisse Vorstellungen in ihm hervorbringen könne.~\RWSeitenw{384}
\end{aufzb}

\RWpar{154}{3.\,Einwurf. Jede höhere Offenbarung soll wegen der Gesetze unsers Denkens unmöglich seyn}
Unser Erkenntnißvermögen ist an gewisse \RWbet{nothwendige Gesetze} (die Kategorien) gebunden; was diesen widerspricht, ist für uns undenkbar. Die Lehren, die eine Offenbarung aufstellt, könnten nur Eines von Beidem, entweder jenen \RWbet{gemäß}, oder \RWbet{nicht gemäß seyn.} Im ersteren Falle müßten uns übersinnliche Gegenstände wie \RWbet{sinnliche} vorgestellt, und in die sinnliche Welt herabgezogen werden. Im zweiten wären uns diese Lehren ganz und gar unverständlich. -- (Kritik aller Offenbarung von Fichte. \RWparnr{8}. S.\,168.\RWlit{168}{Fichte1})\par
\RWbet{Antwort.} Es ist wahr, daß Alles, was den nothwendigen Gesetzen des Denkens widerspricht, undenkbar sey, und nicht nur undenkbar für uns, sondern auch an sich selbst unmöglich. Der Fehler dieses Einwurfes liegt nur in der Zweideutigkeit der Redensart, \RWbet{daß etwas den nothwendigen Gesetzen des Denkens gemäß sey}. Dieses kann nämlich einmal bedeuten: es läßt sich aus ihnen herleiten; dann wieder: es widerspricht ihnen nur nicht. Daß dieser Unterschied gegründet sey, wird Niemand in Abrede stellen; denn es gibt doch gewiß Sätze, die aus gegebenen andern nicht herleitbar sind, ohne denselben gleichwohl zu widersprechen. So darf ja keine Wahrheit der andern widersprechen, aber doch sicher läßt sich nicht jede Wahrheit aus jeder andern herleiten. Mathematische Wahrheiten \zB\ lassen sich sicher nicht aus praktischen, und diese nicht aus jenen ableiten, \usw\ Nimmt man nun die erwähnte Redensart in der ersten Bedeutung; so ist es allerdings wahr, daß die Lehren einer höhern (materiellen) Offenbarung den Denkgesetzen \RWbet{nicht gemäß} sind, \dh\ aus ihnen nicht hergeleitet werden können; aber daraus folgt noch gar nicht, daß sie denselben \RWbet{widersprechen}, und folglich uns durchaus \RWbet{unverständlich} seyn müßten. Nimmt man aber die Redensart in der zweiten Bedeutung; so können und müssen alle Offenbarungslehren den Denkgesetzen \RWbet{gemäß} seyn; und es folgt doch nicht, daß uns die übersinnlichen Gegenstände wie sinnliche dargestellt werden müßten. Denn in eben dieser Bedeutung des Wortes~\RWSeitenw{385}\ sind auch die \RWbet{praktischen Wahrheiten} den \RWbet{mathematischen} gemäß, und dennoch werden uns sittliche Handlungen nicht als \RWbet{geometrische} Gegenstände, als Linien, Flächen oder Körper dargestellt.

\RWpar{155}{4.~Einwurf. Eine höhere Offenbarung soll sich in menschlicher Sprache nicht vortragen lassen}
Die Wahrheiten einer materiellen Offenbarung sollen \RWbet{über unsere Vernunft} seyn. Die Begriffe, die zu solchen übervernünftigen Wahrheiten gehören, finden sich gar nicht in dem Gebiete menschlicher \RWbet{Vernunftbegriffe}, und noch viel weniger finden sich im Gebiete der menschlichen Sprache \RWbet{Worte} für solche Begriffe. Bedient sich also die Gottheit bei ihrer Offenbarung unserer \RWbet{gewöhnlichen} Worte; so denken wir auch nur an die gewöhnlichen Dinge, und lernen folglich nichts Neues. Gebraucht sie aber neue und uns unbekannte Zeichen; so wissen wir damit wieder keinen Begriff zu verbinden, und lernen also gar nichts.\par
\RWbet{Antwort.} Es ist falsch, daß die Begriffe, die zu übervernünftigen (\dh\ nicht durch Vernunft erweislichen) Wahrheiten gehören, in dem Gebiete menschlicher Vernunftbegriffe nie könnten angetroffen werden; denn aus den bekanntesten Begriffen lassen sich Wahrheiten zusammensetzen, welche durch unsere Vernunft nicht erweislich sind. So wäre es \zB\ eine durch die Vernunft gewiß nicht erweisliche Wahrheit, wenn uns eröffnet würde, wie viele Gattungen lebendiger Wesen es auf dem Monde gibt; und doch bestände diese Wahrheit aus lauter bekannten Begriffen. Die Anzahl dieser Gattungen sey noch so groß, so befindet sie sich in der Reihe der uns bekannten Zahlen. Ein Gleiches gilt von der Beantwortung der Frage, ob Besserung allein hinreichend sey zur Vergebung der Sünden oder nicht; \usf\ Kann nun die Offenbarung uns eine Menge neuer Wahrheiten lehren, ohne dazu neuer \RWbet{Begriffe} zu bedürfen, so bedarf sie auch nur unserer gewöhnlichen Worte. -- Aber selbst \RWbet{neue Begriffe} kann uns die Offenbarung mittheilen, wenn man hierunter nur \RWbet{Zusammensetzungen} aus schon bekannten ein\RWSeitenw{386}facheren versteht; und für diese neuen Begriffe könnte sie, wofern es nöthig seyn sollte, auch \RWbet{neue Zeichen} in die Sprache einführen, Worte, deren Bedeutung sie uns nicht einmal durch eine vorhergegangene schulgerechte Erklärung erst müßte bekannt gemacht haben, sondern die wir schon aus dem Zusammenhange selbst würden verstehen lernen, so wie wir tausenderlei Worte auf diese Art erlernen.

\RWpar{156}{5.~Einwurf. Der Lehrer einer Offenbarung müßte Unfehlbarkeit haben}
Nach dem bekannten \RWbet{Gesetze der Sparsamkeit} darf sich Gott, wenn er sich dem menschlichen Geschlechte offenbaren will, nicht jedem Einzelnen aus uns, sondern er muß sich nur Einem oder Einigen \RWbet{unmittelbar}, und dann durch diese den Uebrigen \RWbet{mittelbar} offenbaren. Allein wenn diese Letzteren völlig versichert seyn sollen, daß sie das unverfälschte Wort Gottes aus dem Munde der Ersteren hören: so müssen diese, \dh\ die ersten Lehrer und Verkündiger der göttlichen Offenbarung völlig \RWbet{unfehlbar} seyn, und dieß zwar in ihrem \RWbet{Verstande} sowohl als auch in ihrem \RWbet{Willen}. Das Erste, damit sie nie irren; das Zweite, damit sie nie irre führen wollen. Eine solche Unfehlbarkeit kann aber keinem Menschen, ja überhaupt keinem endlichen Wesen zu Theil werden. (Freimüthige Betrachtungen \usw\ Brief 4. S.\,68.)\par
\RWbet{Antwort.} 1.~Gesetzt, es wäre wahr, was man in diesem Einwurfe behauptet, daß die Verkündiger einer Offenbarung völlige Unfehlbarkeit besitzen müßten, und daß Gott dem Menschen diese nicht mittheilen könnte: so würde eben deßhalb mit Unrecht behauptet, daß es \RWbet{gegen das Gesetz der Sparsamkeit} wäre, wenn Gott sich jedem Einzelnen aus uns unmittelbar offenbarte.
\begin{aufza}\setcounter{enumi}{1}
\item Aber es ist nicht wahr, daß jene Menschen, deren sich Gott zur weitern Verbreitung seiner Offenbarung bedienen will, gänzlich unfehlbar seyn müßten, sofern wir Uebrigen völlig versichert seyn sollen, das unverfälschte Wort Gottes aus ihrem Munde zu hören. Dazu wäre höchstens~\RWSeitenw{387}\ nöthig, daß sie nur in denjenigen Stücken, welche ihr Predigtamt angehen, unfehlbar wären, daß sie nur in Betreff religiöser Gegenstände die richtigen Ansichten und den Willen hätten, diese richtigen Ansichten getreulich mitzutheilen. Ja es lassen sich Mittel denken, bei denen selbst diese Unfehlbarkeit entbehrlich ist. Die Art \zB\ wie die \RWbet{katholische Kirche} ihre Erleuchtung durch den Geist Gottes zu empfangen behauptet, setzt keine Unfehlbarkeit irgend eines einzelnen Gliedes voraus, wie später gezeigt werden soll.
\item Endlich ist es falsch, daß keinem Menschen, ja auch keinem endlichen Wesen überhaupt die Gabe der Fehllosigkeit mitgetheilt werden könne. Daß kein Mensch \RWbet{aus sich selbst} und ohne Gottes besondere Unterstützung von allen Fehlern des Verstandes sowohl als des Herzens frei bleibe, ist allerdings wahr; allein warum sollte es Gott nicht möglich seyn, durch eine besondere Leitung zu bewirken, daß Jemand durch einen gewissen Zeitraum seines Lebens, ja durch sein ganzes Leben hindurch von jedem Fehler des Verstandes sowohl als auch des Herzens frei bleibe?
\end{aufza}

\RWpar{157}{6.~Einwurf. Eine höhere Offenbarung soll wegen des Gesetzes der Stätigkeit unmöglich seyn}
Das \RWbet{Gesetz der Stätigkeit} fordert, daß jede Veränderung, also auch jeder Fortschritt nur stufenweise geschehe. Durch eine Offenbarung aber würde dieses Gesetz verletzt, und es geschähe ein Sprung sowohl in unserer Verstandes- als Herzensbildung:
\begin{aufzb}
\item in unserer \RWbet{Verstandesbildung}; weil wir auf einmal schon hier auf Erden Einsichten erhielten, die wir nach dem Gesetze der Stätigkeit erst durch allmähliche Entwickelung in einem andern Leben erlangen sollen.
\item in unserer \RWbet{Herzensbildung}; indem wir durch den Gebrauch jener Heiligungsmittel, die eine Offenbarung zu besitzen vorgibt, plötzlich entsündigt, und aus lasterhaften in heilige Wesen verwandelt werden sollten; was doch der Wahrheit nach erst durch lange Uebungen geschehen kann. (Freimüthige Betrachtungen. Brief 8. S.\,118.)~\RWSeitenw{388}
\end{aufzb}\par
\RWbet{Antwort.} 1.~Ob das Gesetz der Stätigkeit in irgend einem Sinne ausnahmslos gelte, ist eine Frage, worüber bisher noch gestritten wird.
\begin{aufza}\setcounter{enumi}{1}
\item Gewiß ist es aber, daß dieses Gesetz in \RWbet{dem} Sinne genommen, in dem es allein vertheidigt werden könnte, durch eine Offenbarung nicht verletzt wird. Denn hier wird keine Art eines Sprunges, der unmöglich wäre, gefordert; weder
\begin{aufzb}
\item in unserer \RWbet{Verstandesbildung}. Denn es werden uns hier keine Einsichten, die wir nach jenem Gesetze unmöglich früher als in einem andern Leben einsammeln können, versprochen; sondern nur, was uns schon hier auf Erden zu wissen nöthig und nützlich ist, und was wir auch sehr wohl verstehen (obgleich nach seinen innern Gründen nicht begreifen) können, wird uns nicht unmittelbar, sondern durch die Vermittlung einer göttlichen Zeugenschaft bekannt; noch
\item in unserer \RWbet{Herzensbildung}. Denn eine wahre Offenbarung wird gewiß nicht lehren, daß der Mensch plötzlich, in einem Augenblicke, aus einem Lasterknechte in einen Heiligen umwandelt werden könnte. Auch das katholische Christenthum lehrt nichts dergleichen; sondern es lehrt nur, daß derjenige, der seine Sünden bereuet, den aufrichtigen Vorsatz der Besserung faßt, und die von Gott eingesetzten Heiligungsmittel gebraucht, von nun an die heilsame Vorstellung ergreifen dürfe, daß alle Schuld der Sünde von ihm hinweggenommen sey.
\end{aufzb}
\end{aufza}

\RWpar{158}{Verschiedene Einwürfe, welche die Möglichkeit und die Erkennbarkeit der Wunder betreffen}
Da die Gelehrten bisher beinahe insgemein glaubten, daß \RWbet{Wunder} oder solche Begebenheiten, die zur Bestätigung einer göttlichen Offenbarung dienen, gewisse \RWbet{unmittelbare} und \RWbet{übernatürliche Wirkungen Gottes} seyn müßten: so gab dieß Veranlassung zu einer Menge von Einwürfen \RWbet{gegen die Möglichkeit} und \RWbet{Erkennbarkeit der Wunder}, die ich nur kurz zu beantworten brauche, weil es fast~\RWSeitenw{389}\ auf den ersten Blick einleuchtet, daß sie meine Ansicht der Sache nicht treffen.\par
\RWbet{1.~Einwurf.} Wer eine Begebenheit für ein \RWbet{Wunder} erklärt, will Eines von Beidem sagen: entweder die Begebenheit ist eine Wirkung, die das Maß ihrer Ursache in \RWbet{Wahrheit} übersteigt, oder sie \RWbet{scheint} dasselbe nur zu übersteigen. Im letztern Falle ist sie kein \RWbet{wahres} Wunder; im erstern ist sie unmöglich, weil es ein metaphysisches Naturgesetz ist, daß jede Wirkung ihrer Ursache proportionirt seyn müsse. (Freimüthige Betrachtungen über die dogmatische Lehre von Wundern.)\par
\RWbet{Antwort.} Wenn wir in Uebereinstimmung mit dem gemeinen Menschenverstande eine Begebenheit für ein Wunder erklären; so wollen wir weder das Eine noch das Andere, was man in diesem Einwurfe uns zumuthet, sagen; sondern wir geben zu, daß die Begebenheit allerdings eine proportionirte Ursache habe, und behaupten bloß, daß diese Ursache zuletzt, so wie bei allen zufälligen Dingen in dem \RWbet{Willen Gottes}, und zwar hier insbesondere in der bestimmten Absicht, uns eine Offenbarung mitzutheilen, liege.\par
\RWbet{2.~Einwurf.} Wenn jedes Wunder eine Wirkung \RWbet{Gottes} seyn soll, so muß das vorhin erwähnte Naturgesetz, daß jede Wirkung ihrer Ursache proportionirt seyn müsse, durch dasselbe verletzt werden; denn Gott ist eine \RWbet{unendliche Kraft}, das Wunder aber nur eine \RWbet{endliche Erscheinung}. Und selbst wenn man zugeben wollte, daß Gott der Unendliche etwas bloß Endliches hervorbringen könne; so könnten wir doch aus einer bloß endlichen Erscheinung niemals auf ihn, das unendliche Wesen, als ihren Urheber schließen. (Freimüthige Betrachtungen.)\par
\RWbet{Antwort}. Wäre dieser Grund richtig, so würde er zu viel beweisen; denn jede Erscheinung in der Welt ist eine \RWbet{Wirkung Gottes} (gleichviel ob mittelbar oder unmittelbar, gleichviel ob zu diesem oder jenem bestimmten Zwecke dienend.) Allein auch ein \RWbet{unendliches Wesen} kann eine \RWbet{endliche Wirkung} hervorbringen, wenn es nicht eben mit seiner ganzen Kraft, sondern mit einem bloß \RWbet{endlichen Theile} derselben wirket. Geständlich ist aber~\RWSeitenw{390}\ dieses oder jenes einzelne \RWbet{Wunder}, das Gott hervorbringt, nicht seine \RWbet{einzige} Wirkung; sondern nur ein unendlich kleiner Theil seiner gesammten Wirkungen, deren vollständiger Inbegriff das \RWbet{Weltall} selbst ausmacht, das meiner Meinung nach allerdings unendlich ist. Eben so falsch ist auch der zweite Theil dieses Einwurfes; denn wir können von jeder auch noch so geringen Erscheinung oder Einrichtung in der Welt auf \RWbet{Gott}, als ihren Urheber, schließen, weil er ja in der That als der letzte Grund von Allem anzusehen ist. Doch darum handelt es sich bei einem Wunder gar nicht; sondern nur darum, ob Gott diese Erscheinung zu dem bestimmten Zwecke einer Offenbarung veranstaltet habe oder nicht. Und darüber können wir auf die \RWparnr{147}\ gezeigte Weise urtheilen.\par
\RWbet{3.~Einwurf.} Wenn Wunder möglich seyn sollten, müßte Gott die Natur der Dinge ändern; denn wenn \zB\ ein Mensch die Luft durchfliegen sollte, so müßte Gott entweder die Gesetze der Schwere vernichten, oder den Leib des Menschen oder die Natur der Luft umschaffen. Nun ist die Natur der Dinge \RWbet{ewig und unveränderlich}, oder man kann wenigstens nie ihre Veränderlichkeit oder Zufälligkeit beweisen. Denn die Erfahrung kann uns nie lehren, daß etwas \RWbet{zufällig}, so wenig, als daß es \RWbet{nothwendig} sey. Ueberdieß widerspricht die Abänderung der Naturgesetze der \RWbet{Weisheit Gottes}; er müßte zugleich \RWbet{wollen} und auch \RWbet{nicht wollen}, daß etwas sey. (Freimüthige Betrachtungen.)\par
\RWbet{Antwort.} Es läßt sich nicht darthun, daß Gott \RWbet{die Natur der Dinge}, \dh\ ihre wesentlichen Kräfte und Wirkungsgesetze nur im Geringsten umändern müßte, sofern er eine Erscheinung hervorbringen will, welche wir in der durch den allgemeinen Sprachgebrauch festgesetzten Bedeutung für ein Wunder zu erklären berechtiget wären. Dazu bedarf es nämlich nichts mehr als eines \RWbet{ungewöhnlichen Ereignisses}, bei dem wir keinen Nutzen seiner Veranstaltung als den der Beglaubigung einer gewissen Lehre bemerken können. Daß aber \RWbet{ungewöhnliche Ereignisse} nicht unmöglich sind, lehrt uns ja die \RWbet{Erfahrung selbst}, da sich doch unläugbar gewisse an sich sehr \RWbet{ungewöhnliche}~\RWSeitenw{391}\  \RWbet{Ereignisse} von Zeit zu Zeit ereignen. Sollte es ferner bei \RWbet{gewissen Wundern} auch noch so unbekannt seyn, durch welche \RWbet{Naturkräfte} Gott sie gewirkt habe; oder sollte es auch noch so scheinbar seyn, daß sie gewissen von uns für allgemein gehaltenen Naturgesetzen widersprechen: so würden wir doch nie mit Gewißheit behaupten können, daß dieses wirklich der Fall sey, oder daß diese Erscheinungen durch keine Naturkräfte bewirkt worden seyen. Wir brauchen uns nur zu erinnern, daß wir noch lange nicht \RWbet{alle} Naturkräfte und Wirkungsgesetze derselben kennen. Wenn uns ferner ein gewisses Naturgesetz nur durch \RWbet{Erfahrung} bekannt ist; so können wir eben deßhalb, weil die Erfahrung allein noch nicht entscheidet, ob es ein \RWbet{nothwendiges} oder ein \RWbet{zufälliges} Gesetz sey, eine Abweichung von demselben nicht für unmöglich erklären. Daß endlich \RWbet{jede} Abweichung von einem an sich zufälligen Naturgesetze der Weisheit Gottes widersprechen, daß Gott \RWbet{veränderlich} seyn und etwas zugleich \RWbet{wollen} und auch \RWbet{nicht wollen} müßte, wenn er von einem solchen Gesetze abweichen wollte: ist eine ganz falsche Behauptung. Gott könnte die Beobachtung eines gewissen Gesetzes in allen andern Fällen wollen, und nur nicht wollen, daß es in diesem Einen Falle gelte. Hiezu könnte er einen vernünftigen Grund haben, und es müßte nicht eben eine \RWbet{Veränderlichkeit} von seiner Seite seyn, sondern auch eine Veränderung von Seite der geschaffenen Wesen, \zB\ der Menschen, könnte es seyn, die ihn zu dieser Abänderung für den bestimmten Fall veranlaßt.\par
\RWbet{4.~Einwurf.} Da Alles in der Welt in der genauesten Verbindung und Wechselwirkung steht; so müßte jedes auch noch so geringe Wunder als ein gewaltsamer Eingriff in den natürlichen Lauf der Dinge \RWbet{Unordnungen} erzeugen, die sich durch das Ganze ausbreiten. Beiläufig eben so wie man in einer wohl eingerichteten Maschine kein einziges Rad herausnehmen darf, ohne den ganzen Gang derselben in's Stocken zu bringen; so kann Gott auch in diesem Weltgebäude kein einziges Ereigniß wegnehmen oder abändern, ohne das Ganze zu zerstören. (Deutscher Merkur. November 1781.)\RWlit{}{Teutscher1}~\RWSeitenw{392}\par
\RWbet{Antwort.} Es ist nicht abzusehen, warum durch ungewöhnliche Begebenheiten Unordnungen in dem ganzen Weltgebäude entstehen müßten. So entstehen ja \zB\ durch einen Steinregen keine Unordnungen im ganzen Weltgebäude. Kann Gott ferner die Wirkungen, welche ein ungewöhnliches Ereigniß haben sollte, falls sie dem Ganzen nachtheilig wären, nicht durch gewisse andere entgegengesetzte Kräfte wieder aufheben? Gesetzt das Wunder vom Stillstande der Sonne, worauf man hier anspielt, ließe sich nicht anders als durch einen wirklichen Stillstand der Erde in ihrer Umdrehung um ihre eigene Achse erklären; so hätte es freilich Veränderungen auf der ganzen Erde hervorgebracht; aber wer mag behaupten, daß diese Veränderungen eine Unordnung im ganzen Weltall hätten nach sich ziehen müssen? Bringt nicht die Anziehung der Planeten Venus und Mars viel größere Veränderungen, nicht zwar in der Rotation der Erde, wohl aber, was wichtiger ist, in ihrem Umlaufe um die Sonne hervor? und gleichen sich diese nicht dennoch aus?\par
\RWbet{5.~Einwurf.} Der Wunderglaube schadet dem \RWbet{Interesse des Verstandes}. Dieses fordert, daß wir uns unausgesetzt bemühen, zu jeder Erscheinung in der Natur eine Ursache aufzusuchen, und nicht aus Trägheit (\RWlat{ratio ignava}\editorischeanmerkung{Die faule Vernunft.}) die Dazwischenkunft Gottes annehmen. (\RWbet{Kant}'s Kritik der reinen Vernunft. Transcendentale Methodenlehre. I.~Hauptstück, 3.~Abschnitt.\RWlit{}{Kant2})\par
\RWbet{Antwort.} Durch den hier angenommenen Begriff von einem Wunder wird dem Erforschen jener natürlichen Ursachen, die es hervorgebracht haben mögen, kein Damm entgegengesetzt.\par
\RWbet{6.~Einwurf.} Wunder sollen \RWbet{Wirkungen Gottes} seyn. Nun erlaubt es aber (nach den Ansichten der kritischen Philosophie) das wohlverstandene \RWbet{Princip der Causalität} schlechterdings nicht, daß man zu einer \RWbet{sinnlichen Erscheinung} (dem Wunder) eine Ursache außerhalb der \RWbet{Sinnenwelt} (Gott) annehme; sondern die Ursache muß abermals \RWbet{sinnlich} seyn. (Deutscher Merkur 1787. April\RWlit{}{Anonym2}, 8tes Stück\RWlit{}{Anonym3}. October\RWlit{}{Anonym4}, 2tes Stück\RWlit[. -- Vermutlich meint Bolzano mit \anf{2tes Stück} diese Nr.\,2 des Folgejahres 1888. Die Aufzählung dieser vier Artikel entspricht dann derjenigen in RW I 400.]{}{Anonym5}.)~\RWSeitenw{393}\par
\RWbet{Antwort.} Wenn wir es auch für jetzt dahin gestellt lassen, ob die kritische Philosophie das Princip der Causalität wohl verstanden habe; so ist doch so viel richtig, daß selbst diese Philosophie \RWbet{teleologische Erklärungen} der Natur, \dh\ Urtheile über Zwecke, die Gott bei den Ereignissen oder Einrichtungen der Welt gehabt haben mag, annehme und vertheidige. (Siehe \zB\ \RWbet{Kant's} eigene Aufsätze hierüber im deutschen Merkur: Ueber den Gebrauch teleologischer Principien in der Philosophie. 1788. Jänner\RWlit{}{Kant6}, 2tes Stück\RWlit{}{Kant7}. Ingleichen die Kritik der Urtheilskraft.\RWlit{}{Kant5}) Ein Wunder ist nun nichts Anderes als ein Ereigniß, bei dem man die teleologische Erklärung wagt, daß Gott die Absicht gehabt habe, uns durch dasselbe zur Annahme einer gewissen für uns sittlich zuträglichen Lehre zu bestimmen. Und daß wir diese teleologische Erklärung ganz nach denselben Regeln wagen, nach denen jede andere gewagt wird, wurde schon \RWparnr{143} gezeigt.\par
\RWbet{7.~Einwurf.} Außerordentliche Naturerscheinungen (sogenannte \RWbet{Naturwunder}) machen die Aufmerksamkeit rege und ermuntern, weil man durch sie neue Entdeckungen in der Natur zu machen hofft; \RWbet{wirkliche Wunder} aber schlagen das Gemüth nieder; denn sie erregen die Besorgniß, daß wir durch ihre häufige Erscheinung allmählich das Zutrauen auch zu den schon für bekannt angenommenen Naturgesetzen verlieren werden. Diesem Uebel konnte nur dadurch vorgebeugt werden, daß man voraussetzte, die Wunder werden sich äußerst selten ereignen. Aber zu dieser Voraussetzung hat man kein Recht. Und dann fragt sich wieder, \RWbet{wie selten}? Will man \RWbet{jede unerklärliche Begebenheit} für ein Wunder halten; so muß man \RWbet{tägliche} Wunder zugeben, und dieß gefährdet zuletzt selbst unsere \RWbet{Moralität}; daher denn auch alle vernünftigen Männer und selbst weise Regierungen dem Glauben an neue Wunder feind sind, und höchstens Wunder der Vorwelt zugeben. Wer endlich sagt, daß Seltenheit schon in dem \RWbet{Begriffe} des Wunders liege, begehet eine \RWbet{Sophisterei}, indem er eine objective Frage von dem, was die \RWbet{Sache} ist, in eine subjective davon, was das \RWbet{Wort} bedeutet, umändert. (\RWbet{Kant}'s Religion innerhalb der Grenzen der bloßen Vernunft. 2te Auflage. S.\,123.)\RWlit{123}{Kant4}~\RWSeitenw{394}\par
\RWbet{Antwort.} Wunder in unserer Bedeutung des Wortes schlagen das Gemüth keineswegs nieder, und machen auch unsere Kenntniß der Naturgesetze nicht schwankend; denn wir geben immerhin zu, daß sie aus Naturkräften und nach Naturgesetzen erfolgt seyn dürften. Wir können auch mit Recht voraussetzen, daß sie nur selten erfolgen werden, weil auch ihr Zweck, nämlich die Beglaubigung einer Offenbarung, nur selten eintritt. Daß wir die Frage: Wie selten? nicht bestimmt beantworten können, ist wahr; aber wozu wäre uns dieses auch nöthig? Uebrigens heißt uns nicht \RWbet{jede} unerklärliche Begebenheit ein Wunder; sondern nur jene, der wir vernünftiger Weise den Zweck, daß sie Gott zur Beglaubigung einer Lehre veranstaltet habe, zuschreiben können. Daher werden wir schwerlich tägliche Wunder aufzuweisen haben. Auf keinen Fall aber ist es möglich, daß unsere \RWbet{Moralität} durch solche Wunder gefährdet werde, da wir nur dann zugeben, daß ein außerordentliches Ereigniß ein Wunder sey, und zur Bestätigung einer Lehre diene, wenn diese Lehre selbst den Charakter sittlicher Zuträglichkeit hat. In wiefern es jedoch auch einen unvernünftigen Wunderglauben gibt, der zur Entschuldigung böser Handlungen gemißbraucht wird, können sich weise Männer und kluge Regierungen demselben allerdings widersetzen, und dieß zwar gleichviel, ob diese Wunder alt oder neu sind. Eine Regierung aber, die sich jedem Glauben an ein neues Wunder bloß darum, weil es neu ist, widersetzen wollte, würde schwerlich den Namen einer \RWbet{weisen} Regierung verdienen, denn sie würde dem Fortschritte des Volkes zum Bessern auf echt chinesische Weise entgegenstreben. Nach der hier aufgestellten Theorie von den Kennzeichen einer Offenbarung ist es nicht \RWbet{Sophisterei}, wenn wir die Seltenheit (nicht zwar als einen Bestandtheil in die Erklärung des Begriffes eines Wunders aufnehmen, wohl aber) als ein nothwendiges Merkmal desselben angeben; denn es ist oben (\RWparnr{146}\ u.\ 149) deutlich gezeigt worden, daß und warum ein Ereigniß, das zur Bestätigung einer göttlichen Offenbarung dienen soll, ein ungewöhnliches Ereigniß seyn müsse; wie auch, daß es nicht mehr als dieß, nämlich kein übernatürliches Ereigniß zu seyn brauche. Und also trifft der Vorwurf, daß wir die Frage nach der objectiven Beschaffenheit der Sache (der Uebernatürlichkeit) mit einer bloß subjectiven Betrachtung (der Seltenheit) verwechseln, unsere Ansicht gewiß nicht.\par
\RWbet{8.~Einwurf.} Wunder können \RWbet{kein Gegenstand der Erfahrung} werden; indem Erfahrung nur dort vor\RWSeitenw{395}handen ist, wo die Erscheinung eine andere voraussetzt, auf die sie nach einer allgemeinen Regel folget.\par
\RWbet{Antwort.} Auch das Wunder ist eine Erscheinung, die auf eine andere nach einer allgemeinen Regel folgt, jedoch nach einer solchen, die uns \RWbet{unbekannt} ist. Will man nun solche Erscheinungen nicht \RWbet{Erfahrungen} nennen, so sind wir es zufrieden; dann muß man aber auch dem Steinregen und einer jeden bisher noch unerklärten Erscheinung den Namen einer \RWbet{Erfahrung} absprechen.\par
\RWbet{9.~Einwurf.} \RWbet{Wunder können von uns nicht erkannt werden}. Weil wir nicht alle Naturkräfte kennen, so  dürfen wir niemals von einer vorhandenen Erscheinung, so sehr sie auch den \RWbet{uns} bekannten Naturkräfen widersprechen mag, behaupten, daß sie nicht gleichwohl durch gewisse uns \RWbet{unbekannte} Naturkräfte, \zB\ durch die Kraft höherer Geister \udgl\  bewirkt sey, und folglich den Namen eines \RWbet{Wunders} keineswegs verdiene.\par
\RWbet{Antwort.} Dieser Einwurf trifft offenbar nur diejenigen, die das Wunder für eine \RWbet{unmittelbare} oder \RWbet{übernatürliche Wirkung Gottes} im strengsten Sinne halten; ich aber gebe selbst zu, daß es durch allerlei Naturkräfte vermittelt seyn könne.\par
\RWbet{10.~Einwurf.} Auch der bloße \RWbet{Zufall} kann eine uns unerklärbare Erscheinung bewirken, die wir mit Unrecht für ein Wunder halten; \zB\ ein zufälliger Windstoß kann den Wolken die Gestalt eines Kreuzes geben \udgl \par
\RWbet{Antwort.} Versteht man unter Zufall einen Erfolg, der schlechterdings \RWbet{keinen Grund} hat, so gibt es und kann es \RWbet{nirgends} einen Zufall geben. Versteht man aber darunter einen Erfolg, der nicht \RWbet{beabsichtigt} war; so gibt es wohl in Beziehung auf \RWbet{menschliche Wirksamkeit} einen Zufall, nicht aber in Beziehung auf \RWbet{Gott}; \dh\ wir Menschen können wohl oft einen Erfolg hervorbringen, ohne \Abweichung{denselben\< beabsichtiget zu haben. Gott aber kann nie etwas in der Welt hervorbringen oder auch nur geschehen lassen, ohne davon zu wissen und es zu wollen, mit seinem entweder unbedingten oder bedingten Willen. Was immer gut ist, will er mit seinem unbedingten Willen; und wir sind also, wenn von irgend etwas, sicher von Allem, was an sich gut ist, berechtigt zu behaupten, daß es in Hinsicht auf Gott nicht zufällig, sondern \RWbet{nach seiner Absicht} \>erfolge. (\RWparnr{26})}{ihn zu beabsichtigen, nicht aber Gott}{ (dazu hat Bolzano den \Alabel -Text in \A1label zuerst geändert, bevor er die oben abgedruckte, längere Variante verfasst hat, ohne die erstere zu streichen)} Finden wir also, daß gewisse Begebenheiten völlig geeignet sind, uns zur Beglaubigung einer heilsamen Lehre zu dienen; so können wir nicht im Geringsten~\RWSeitenw{396}\ zweifeln, daß Gott diese \RWbet{mögliche Wirkung} derselben vorausgesehen, und können ferner mit aller Sicherheit schließen, daß er sie auch \RWbet{gewollt} und \RWbet{beabsichtigt habe}.
   
\RWpar{159}{Verschiedene Einwürfe, die man gegen die Beweiskraft der Wunder erhoben hat}
Nach meiner obigen Erklärung eines Wunders oder Zeichens, die, wie ich glaube, dem Sprachgebrauche des Wortes ganz gemäß ist, liegt es schon in dem \RWbet{Begriffe} eines Wunders, daß es ein Ereigniß sey, das \RWbet{Beweiskraft} hat; und sonach wäre es eine Ungereimtheit, ihm diese absprechen zu wollen. Da aber die meisten Gelehrten den Begriff eines Wunders bisher auf eine ganz andere Weise erklärten, bald nämlich äußerten, daß sie ein jedes ungewöhnliche Ereigniß, bald wieder, daß sie nur eine jede übernatürliche, bald endlich nur eine jede unmittelbare Wirkung Gottes ein Wunder nennen wollten: so ließe sich allerdings darüber streiten, ob eine solche Wirkung auch die hier nöthige \RWbet{Beweiskraft} habe. Obgleich uns nun die Einwürfe, welche man gegen die Beweiskraft der Wunder in einer von diesen letztern Bedeutungen vorgebracht hat, eigentlich gar nicht treffen: so will ich doch die wichtigsten derselben beibringen, weil auch dieß beitragen wird, uns von der Richtigkeit der oben aufgestellten Ansichten noch völliger zu überzeugen.\par
\RWbet{1.~Einwurf.} Wunder können dem, der sie wirkt, kein größeres \RWbet{Ansehen} verschaffen; denn eigentlich ist es nicht er, der sie hervorbringt, sondern \RWbet{Gott} selbst, und er ist nur das \RWbet{Werkzeug}, dessen sich Gott bedient; und nur, was der Mensch \RWbet{selbst} thut, kann seinen \RWbet{Werth erhöhen.}\par
\RWbet{Antwort.} Sey es, daß Wunder das Ansehen des Wunderthäters nicht im Geringsten erhöhen, wenn man hierunter sein \RWbet{sittliches Ansehen} oder den sittlichen Werth desselben verstehet; so ist doch gewiß, daß Wunder seine \RWbet{Glaubwürdigkeit} vermehren; ja sie allein sind es, die allen Aussagen desselben, welche zugleich sittliche Zuträglichkeit haben, das Ansehen göttlicher Aussprüche, also vollkommene Glaubwürdigkeit verschaffen.~\RWSeitenw{397}\par
\RWbet{2.~Einwurf.} Wunder sind nicht einmal tauglich, dem Lehrer neuer Wahrheiten \RWbet{Aufmerksamkeit}, selbst nicht bei der \RWbet{großen Volksmenge} zu verschaffen. Denn durch den Anblick der gewirkten Wunder werden die Menschen dergestalt \RWbet{betäubt}, daß sie nun für die \RWbet{Lehre}, die der Wunderthäter vorträgt, keine Empfänglichkeit mehr behalten.\par
\RWbet{Antwort.} Weder aus der Erfahrung, noch aus psychologischen Gründen ist das, was man hier vorgibt, erweislich, daß nämlich der Anblick eines Wunders den Menschen dergestalt betäube, daß er sich in der Folge nicht wieder fassen, und keine Empfänglichkeit mehr für jenen Unterricht, den ihm der Wunderthäter ertheilt, sollte beweisen können. Und wie? wenn der vorsichtige Wunderthäter seine Lehre früher, als er das Wunder gewirkt hat, vortrüge? oder noch besser, wenn er das Wunder zuerst, nur um die Aufmerksamkeit zu spannen, ankündigte, dann seinen Lehrvortrag hielte, und zuletzt das Wunder selbst erfolgen ließe?\par
\RWbet{3.~Einwurf.} Verständigen und weisen Männern vollends muß jeder Wunderthäter als ein vermuthlicher Betrüger oder Schwärmer nur verdächtig werden. \RWlat{Ôtez les miracles} -- ruft daher J.\,J.\,Rousseau nicht mit Unrecht aus -- \RWlat{et tout le monde se jettera aux genoux de Jésus Christ!}\par 
\RWbet{Antwort.} Verständige und weise Männer müssen zwar allerdings Jeden, der sich für einen Wunderthäter ausgibt, erst sorgfältig prüfen, ob er nicht etwa ein Schwärmer oder Betrüger sey, aber ihn \RWbet{ohne Prüfung} dafür erklären, ist nicht verständig und weise.\par
\RWbet{4.~Einwurf.} Der Menschenverstand fordert \RWbet{Vernunftbeweise}, nicht \RWbet{Wunderdinge.} Was haben die Wunder mit der Lehre zu thun? (\RWbet{Mendelssohn}. S.~auch Proben rabbinischer Weisheit in Engel's Philosophen für die Welt.)\par
\RWbet{Antwort.} Daß der Verstand \RWbet{Beweise} fordere, wird ganz richtig behauptet; daraus folgt aber noch gar nicht, daß er die \RWbet{Wunder zurückstoßen} müsse; denn diese liefern ja auch einen Beweis für die Wahrheit einer Lehre, sobald diesselbe nur den Charakter der sittlichen Zuträglichkeit hat. Wahr und treffend bemerkt ist nur, daß eine Frage, die mit der Sittlichkeit in keinem Zusammenhange stehet, durch keine auch noch \ergaenzt{so} große Wunder entschieden werden könne (\RWparnr{148}).~\RWSeitenw{398}\par
\RWbet{5.~Einwurf.} Versteht man unter Wundern bloß \RWbet{unerklärliche} oder \RWbet{ungewöhnliche Ereignisse}, so kann ein jeder gelehrte Naturforscher Wunder vor ungelehrten Leuten wirken, und sie auf diese Art zur Annahme beliebiger religiöser Ansichten bestimmen. Ferner dieselbe Begebenheit, die wir heute nicht anders zu erklären wissen, als durch Voraussetzung der göttlichen Absicht, uns zum Beweise einer Offenbarung zu dienen, werden wir nach Jahrhunderten vielleicht ganz anders erklären lernen. Dann würden wir also dieselbe Religion, die wir jetzt für eine göttliche Offenbarung annehmen wollten, nicht mehr dafür ansehen dürfen. Da aber die \RWbet{Wahrheit} sich unmöglich ändern kann, so dürfen wir dieser Religion unser Zutrauen auch jetzt nicht schenken.\par
\RWbet{Antwort}. Es wird dem gelehrten Naturforscher keineswegs gelingen, uns eine jede beliebige Lehre als eine uns von Gott kommende Offenbarung vorzuspiegeln; sondern dieß könnte ihm höchstens mit einer Lehre gelingen, die zugleich sittliche Zuträglichkeit für uns hat. Gehen wir ferner mit aller Behutsamkeit zu Werke, und prüfen wir nicht nur die vorgeblichen Wunder, sondern auch die Lehre selbst: so können wir wohl das festeste Vertrauen zu Gottes Vorsorge hegen, daß er uns nicht zu unserem Schaden werde getäuscht werden lassen. Gott wird nicht zugeben, daß ein Betrüger uns durch vorgespiegelte Wunder zur Annahme einer Lehre bewege, die falsch und nachtheilig für uns ist. Ist aber die Lehre wahr und wohlthätig; so sey es immerhin auch ein Betrüger, dessen sich Gott, wenn es seiner Weisheit angemessen ist, als eines Werkzeuges zu seinen Offenbarungen bediente. Sollte die Nachwelt den Betrug entdecken, und durch die Voraussetzung eines Betruges im Stande seyn, die Entstehung jener Ereignisse auf ganz gewöhnliche Art zu erklären: so würde die Religion von nun an aufhören, eine göttliche Offenbarung für uns zu seyn; allein Gott würde dieß nur erst dann zulassen, wenn wir schon etwas Besseres hätten, und jener älteren Religion bereits entbehren könnten. Uebrigens darf man nicht glauben, daß eine jede Entdeckung eines Betruges, der bei der \RWbet{Entstehung oder Verbreitung einer Religion mitgewirkt} hat, die Göttlichkeit derselben widerlegen~\RWSeitenw{399}\ würde. Wenn auch selbst bei Voraussetzung eines Betruges noch gewisse ungewöhnliche Ereignisse angenommen werden müßten, durch welche dieser Betrug zu Stande kommen oder Glauben finden konnte: so bliebe noch immer zu erklären, zu welchem Zwecke Gott diese zugelassen habe. Fänden wir nun, daß die Religion, die der Betrüger gelehrt, sittliche Zuträglichkeit für uns besitze, und daß die ungewöhnlichen Ereignisse, die ihr zu Statten kamen, keinen für uns bemerkbaren Nutzen hätten, sollten sie nicht zu ihrer Beglaubigung dienen: so wären wir auch jetzt noch berechtiget, oder vielmehr verpflichtet, anzunehmen, daß sie den Zweck dieser Beglaubigung haben, und daß somit jene Religion eine göttliche Offenbarung sey.
   
\RWpar{160}{Einwürfe gegen die Möglichkeit der historischen Beglaubigung eines Wunders}
Wenn es gleich in Beziehung auf Gott möglich wäre, Wunder zu wirken, so kann doch ihr wirkliches \RWbet{Geschehenseyn}, sagt man, nie mit hinlänglicher Gewißheit \RWbet{beglaubiget werden}; und dieß zwar weder durch das \RWbet{Zeugniß der eigenen Sinne}, noch weniger durch \RWbet{fremdes Zeugniß}.
\begin{aufzb}
\item \RWbet{Nicht durch das Zeugniß der eigenen Sinne}. Denn wenn es mir geschähe -- sagt Rousseau -- daß ich ein \RWbet{Wunder} sehen sollte: so müßte ich über dem Bestreben, dasselbe \RWbet{natürlich} zu erklären, eher die \RWbet{eigene} Vernunft einbüßen, als das Wunder glauben.
\item \RWbet{Um so weniger durch fremdes Zeugniß}, denn
\begin{aufzc}
\item erstlich ist es doch immer unendlich wahrscheinlicher, daß sich etwas natürlich Mögliches, als daß sich etwas natürlich Unmögliches ereigne. Allein daß der vorhandene Zeuge irre oder lüge, ist etwas natürlich Mögliches. Daher muß man Jenes immer eher als Dieses annehmen.
\item ferner ist es doch immer unendlich wahrscheinlicher, daß Ein Mensch, nämlich derjenige, der uns ein Wunder gesehen zu haben erzählet, irre oder lüge, als daß die Aussage von Millionen Menschen, welche die entgegengesetzte natürliche Erscheinung beobachtet haben, irrig oder lügenhaft sey.~\RWSeitenw{400}\ Wenn uns \zB\ der h.~Evangelist Johannes die Auferstehung des Lazarus erzählt, so widerspricht er Millionen Menschen, welche versichern, es nie beobachtet zu haben, daß ein Todter am vierten Tage wieder auferstehe. Eines von Beidem muß man also nothwendig thun, entweder dem \RWbet{Einen}, oder den \RWbet{Millionen} Recht geben. Offenbar ist nun das Letztere immer vernünftiger als das Erstere.
\item Gibt man endlich die Wahrheit \RWbet{Einer} Wundererzählung zu, so ist kein Grund vorhanden, warum man nicht eben so die Wahrheit \RWbet{aller} Erzählungen von Wundern, welche je niedergeschrieben worden sind, zugeben sollte; und da wird man doch gewiß irre geführt. (Deutscher Merkur 1787.\ April\RWlit{}{Anonym2}, Juli\RWlit{}{Anonym3}, October\RWlit{}{Anonym4}.\ 1788.\ Februar.\RWlit{}{Anonym5} Freimüthige Betrachtungen.\RWlit{}{Schmerler1} \RWlat{Rousseau Lettres de la Montagne}\RWlit{}{Rousseau2} \uA )
\end{aufzc}
\end{aufzb}\par
\RWbet{Antwort.} Wer zu einem Wunder nichts Anderes fordert, als \RWparnr{151}\ angegeben wurde, wird es sicher nicht unmöglich finden, von dem Geschehenseyn eines solchen durch seine eigenen Sinne so wohl als auch durch fremde Aussage überzeugt zu werden. Denn dazu ist keineswegs nöthig, sich ein bestimmtes Urtheil über den eigentlichen Hergang der Sache anzumaßen; sondern es ist genug, sich nur zu überzeugen, daß in einem jeden Falle, es mag sich nun mit der in Rede stehenden Begebenheit so oder so verhalten haben, etwas Ungewöhnliches mit unterlaufen seyn müsse. 
\begin{aufzb}
\item Wären die Wunder etwas erwiesen Unmögliches, etwas unserer Vernunft Widersprechendes, dann allerdings müßte man erst die Vernunft verlieren, bevor man sie glauben könnte. Aber dieß sind nun einmal die Wunder, von denen wir sprechen, erwiesener Maßen nicht; sie sind nicht nur \RWbet{problematisch möglich}, sondern weil sie zur Beglaubigung einer Offenbarung nothwendig sind, und weil uns diese nützlich und nothwendig scheint: so können wir sie sogar erwarten, und sind verpflichtet, sie zu suchen. Wenn man nun etwas erwartet und sucht, und es dann wirklich findet: wie sollte man hierüber so erstaunen, daß man den Verstand verlöre? -- 
\item Wäre es ferner auch wahr,
\begin{aufzc}
\item daß das natürlich Mögliche immer unendlich wahrscheinlicher, als das natürlich Unmögliche ist; so ist es, wie ich schon oft erinnert habe, nicht wahr, daß Wunder natürlich unmöglich seyn müssen; sondern sie können allerdings auch durch Naturkräfte gewirkt seyn.~\RWSeitenw{401}
\item Es ist unrichtig, daß der Wundererzähler den Millionen anderer Menschen, welche die entgegengesetzte natürliche Begebenheit beobachtet haben, widerspreche; denn die Begebenheiten, die sie beobachteten, sind nicht die nämlichen, sondern nur von der nämlichen Gattung mit der, von welcher er spricht. Der Erzähler von \RWbibel{Joh}{Joh.}{11}{}\ widerspricht nicht im Mindesten den Millionen, die uns bezeugen, daß die von ihnen beobachteten Todten nicht wieder auferstanden wären; er setzt dieß vielmehr als etwas Bekanntes voraus, und versichert nur, daß bei der Leiche des Lazarus, welche von jene Millionen nicht beobachtet wurde, ein Anderes eingetreten sey.
\item Nicht alle Wundererzählungen verdienen gleichen Glauben; nicht bei allen läßt sich darthun, daß die Annahme eines Irrthums oder einer Lüge von Seite des Erzählers eine Sache von noch viel größerer Unwahrscheinlichkeit wäre, als die Annahme des erzählten Wunders selbst. Und Wundererzählungen, bei denen dieß nicht der Fall ist, werden wir keineswegs glauben.
\end{aufzc}
\end{aufzb}

\RWpar{161}{Schlußfolgerung aus dieser Prüfung: Die aufgestellten Kennzeichen einer göttlichen Offenbarung gewähren Sicherheit}
\begin{aufza}
\item Aus Allem, was über die Erwünschlichkeit oder Nothwendigkeit einer Offenbarung (im III.\,Hptst.) vorkam, dann aus den Untersuchungen, die in den ersten §§ des gegenwärtigen Hauptstückes angestellt wurden, endlich auch aus der einleuchtenden Unstatthaftigkeit aller Einwürfe, die wir so eben betrachtet, ergibt sich nun mit der vollesten Ueberzeugung der Schluß, \RWbet{daß eine Offenbarung problematische Möglichkeit habe}, \dh\ daß durchaus nichts vorhanden sey, wodurch die Unmöglichkeit derselben erwiesen werden könnte.
\item Im Gegentheile gibt es so manche Umstände, die uns das Daseyn einer Offenbarung höchst \RWbet{wahrscheinlich} machen. Hieher gehören vornehmlich die \RWbet{vielen Vortheile}, die sie, so viel wir es nur zu beurtheilen vermögen, dem menschlichen Geschlechte leisten könnte.
\item Ja, was noch mehr ist, aus der Bemerkung, daß beinahe \RWbet{alle Menschen} auf Erden den Glauben an gewisse Offenbarungen haben, läßt sich auch ohne in eine besondere Prüfung derselben einzugehen, mit völliger Sicherheit schlie\RWSeitenw{402}ßen, es müsse in der That wenigstens für gewisse Menschen eine \RWbet{göttliche Offenbarung} geben. Denn aus \RWparnr{31}\ ist zu entnehmen, daß jeder Mensch, der eine Offenbarung nur zu besitzen \RWbet{meint}, falls er bei Annahme dieser Meinung nach seiner \RWbet{besten Einsicht} verfuhr, eine göttliche Offenbarung auch in der That besitze. Wer also behaupten wollte, \RWbet{es gebe gar keine wirkliche Offenbarung auf Erden}, der müßte nur behaupten, daß alle diejenigen Menschen, die eine solche zu besitzen glauben, aus einer selbst verschuldeten Unwissenheit, aus Trägheit oder Leidenschaft in diesen Irrthum geriethen. So unzulässig nun diese Behauptung wäre, so sicher ist es im Gegentheile, daß mehre der auf Erden für geoffenbart gehaltenen Religionen dieses auch wirklich für viele Tausende aus ihren Anhängern sind.
\item Doch zu demjenigen, was wir jetzt eigentlich beweisen wollen, genügt es, daß man uns nur \RWbet{die problematische Möglichkeit einer Offenbarung zugebe}. Wenn man nur zugibt, daß der Mensch durchaus keine hinlängliche Ursache habe, Gott die Möglichkeit, sich ihm zu offenbaren, abzusprechen; so folgt schon daraus allein, daß uns Gott einige \RWbet{ganz sichere Kennzeichen} einer Offenbarung gewähren müsse. Denn gäbe es keine ganz sichere Kennzeichen; so könnten wir es nie zur Gewißheit in der Frage bringen, ob eine göttliche Offenbarung für uns wirklich vorhanden sey oder nicht? Diese Ungewißheit aber wäre nicht nur beunruhigend für uns, sondern sie müßte auch auf unsere Sittlichkeit nachtheilig wirken. Denn hätten wir keine hinreichende Gewißheit über diese Frage; so würden wir uns, so oft uns Leidenschaft reizt, überreden, daß keine Offenbarung da sey, und eben deßhalb gar manche Vorschrift, die eine durch Wunder bestätigte Religion uns aufstellt, leichtsinnig übertreten, ohngeachtet unser Gewissen uns darüber Vorwürfe machen und deutlich genug zu erkennen geben würde, daß wir eine Lehre, an der wir sittliche Zuträglichkeit finden, selbst in dem Falle, wenn sie keine göttliche Offenbarung ist, für verbindlich ansehen sollten; um wie viel mehr, wenn es nicht unmöglich ist, daß sie das wirklich sey. Um nicht Gelegenheit zu diesem sittlichen Verderben~\RWSeitenw{403}\ selbst zu geben, muß uns Gott nothwendig aus dieser Ungewißheit reißen, und daher Eines von Beidem thun: entweder uns deutlich erkennen lassen, daß keine Offenbarung da sey, und da seyn könne, weil sie \zB\ uns durchaus schädlich wäre, oder weil keine Kennzeichen derselben möglich sind \udgl ; oder er muß uns sichere und unzweideutige Kennzeichen einer Offenbarung geben. Da er das Erstere nicht gethan, so folgt, daß er das Letztere thun müsse.
\item Nun kann es aber nach Ausweis der eben vorhergegangenen Untersuchung keine andere Kennzeichen einer Offenbarung geben, als jene zwei aufgestellten. Also sind diese \RWbet{sicher}, \dh\ Gott darf nicht zulassen, daß sich diese beiden Kennzeichen vereinigt an einem Lehrbegriffe vorfinden, wenn er denselben nicht wirklich von uns geglaubt wissen will.\par
\item[\RWbet{Einwurf.}]\ Was Gott thun oder nicht thun, zulassen oder nicht zulassen dürfe, das für \RWbet{einzelne Weltbegebenheiten} bestimmen zu wollen, wäre \RWbet{Vermessenheit}. (C.~Ch.~E.~\RWbet{Schmid's Moralphilosophie}. 2.\,Aufl., Jena, 1795.\ S.\,149.)\RWlit{149}{Schmid1}\par
\item[\RWbet{Antwort.}] Ohne Vermessenheit können wir einige Regeln, nach welchen sich Gott bei seiner Weltregierung richten muß, aufstellen. Denn es ist doch wohl nicht Vermessenheit, wenn wir sagen, daß Gott unter allen nicht an sich selbst unmöglichen Verfügungen immer diejenige treffen müsse, die dem Wohle des Ganzen am Meisten zusagt? Das ist nun schon \RWbet{Eine Regel} des göttlichen Verhaltens, die wir ohne Vermessenheit aufstellen können. Aus dieser aber ergeben sich noch viele andere. So oft wir nämlich mit hinlänglicher Sicherheit urtheilen können, daß die Befolgung einer gewissen Regel nichts an sich Unmögliches, wohl aber etwas dem Wohle des Ganzen Zuträgliches sey, so können wir mit eben dieser Sicherheit erwarten, daß Gott diese Regel beobachten werde. Und eben daher kommt es, daß \zB\ Niemand ein Bedenken trägt zu sagen, Gott müsse jede sittlich gute That einmal belohnen, und jede sittlich böse einmal bestrafen. -- Ganz eben so können wir aber auch die Regel aufstellen, \RWbet{Gott dürfe nie zulassen, daß die beiden Kennzeichen einer Offen}\RWSeitenw{404}\RWbet{barung vereinigt an einem Lehrbegriffe anzutreffen sind, den er nicht wirklich von uns geglaubt wissen will}. Diese letzt ausgesprochene Regel ist eben so sicher, als die beiden vorhergehenden; sie hat nur das Eigenthümliche, daß sie eine Anwendung auf einzelne Weltbegebenheiten zuläßt, oder daß man kraft ihrer bestimmen kann, was Gott in einem gewissen einzelnen Falle nicht zulassen darf, wenn er heilig seyn will.
\end{aufza}

\RWpar{162}{Möglichkeit einer formalen sowohl, als materialen göttlichen Offenbarung}
Durch die so eben erwiesene Sicherheit der Kennzeichen einer Offenbarung ist nun auch der Beweis für die \RWbet{Möglichkeit der letztern selbst} vollendet. Es könnte aber eben aus jenen Kennzeichen, die ich hier zur Beglaubigung einer Offenbarung fordere, in Manchem der Zweifel entstehen, als wäre auf diese Art nur eine formale, nicht aber auch eine materiale göttliche Offenbarung möglich. Denn da wir an jeder Lehre, die sich uns als eine göttliche Offenbarung darstellen soll, innere Vortrefflichkeit bemerken wollen; so muß sie, wie es scheint, mit der Vernunft ganz übereinstimmen, muß aus ihr selbst \RWbet{erweislich} seyn; widrigenfalls könnten wir über ihre Tauglichkeit zur Beförderung der Tugend und Glückseligkeit kein bestimmtes Urtheil fällen. Jede erweisliche göttliche Offenbarung kann also nur formal, nie aber material seyn. (So ohngefähr dachten \RWbet{Kant, Fichte} und \RWbet{Andere}, s.~\zB\ \RWbet{Fichte}'s Kritik aller Offenbarung S.\,169.\RWlit{169}{Fichte1}) Gewöhnlich bezeichnet man diese Denkart mit der Benennung: \RWbet{Rationalismus}; die entgegengesetzte aber, oder die Lehre, daß es auch eine materiale Offenbarung gebe (besonders wenn man sich vorstellt, daß diese durch übernatürliche Kräfte herbeygeführt sey), wird der \RWbet{Super-} oder \RWbet{Supranaturalismus} genannt.\par
Ich \RWbet{antworte} hierauf, daß wir die innere Vortrefflichkeit einer Lehre sehr wohl beurtheilen können, ohne im Stande zu seyn, auch über ihre Wahrheit selbst zu entscheiden. Denn zu dem Erstern wird nur erfordert, daß wir einen gewissen Nutzen bemerken, den ein von unserer Seite einzutretender Glaube an sie verspricht. Um dagegen behaupten zu können, daß eine Lehre wahr sey, und daß somit die Anstalten, die sie von Seite Gottes voraussetzt, wirklich von ihm getroffen worden seyen, ist es nöthig, uns zu versichern, daß jene Anstalten unter allen nicht bloß uns bekannten, sondern~\RWSeitenw{405}\ auch an sich selbst möglichen andern Einrichtungen zur Beförderung der Tugend und Glückseligkeit im Weltall am Meisten beitragen. Wer sieht nicht, daß diese letztere Untersuchung etwas ganz Anderes und ungleich mehr als die erstere erfordert?

\RWpar{163}{Auch eine materiale Offenbarung praktischer Wahrheiten ist möglich}
Die Nützlichkeit einer materialen Offenbarung über gewisse von Gott allein abhängige Anstalten und Einrichtungen ist durch das eben Gesagte wohl außer Zweifel gesetzt. Doch die Bekanntmachung mit solchen Anstalten enthält bloß theoretische Lehren. Wenn es sich aber frägt, ob eine Offenbarung auch neue \RWbet{praktische Lehren}, \dh\ auch neue Pflichten aufstellen könne; so dürfte man hiegegen noch eine eigene \RWbet{Bedenklichkeit} erheben. Bei praktischen Sätzen nämlich scheint die moralische Vollkommenheit, die dazu nothwendig ist, damit sie das Kennzeichen einer Offenbarung besitzen, mit ihrer Wahrheit selbst auf Eines hinauszulaufen. Denn soll ein praktischer Satz sittliche Zuträglichkeit haben, so müssen wir einsehen, daß die Befolgung desselben die Tugend und Glückseligkeit befördern, und zwar unter \RWbet{allen} Verhaltungsarten am meisten befördern werde. Sehen wir aber dieses ein, so sind wir ja eben darum auch ohne Offenbarung schon zur Befolgung dieser praktischen Vorschrift verpflichtet. Also vermag eine Offenbarung nie \RWbet{neue} Pflichten zu lehren.\par

Hierauf \RWbet{erwidere ich zweierlei}:
\begin{aufza}
\item Es kann praktische Sätze geben, deren sittliche Zuträglichkeit nur erst aus jenen \RWbet{neuen theoretischen Lehren}, mit denen die Offenbarung uns bekannt gemacht hat, ersichtlich wird. Solche praktische Sätze hätten wir also auf keine Art als unsere Pflichten erkannt, bevor uns die Offenbarung nicht mit jenen theoretischen Lehren bekannt gemacht hatte. Von dieser Art ist \zB\ die Pflicht der Dankbarkeit gegen Jesum als unseren Erlöser.
\item Ferner kann man die sittliche Zuträglichkeit eines Verhaltens erkennen, ohne die Annahme desselben für~\RWSeitenw{406}\ eine bestimmte \RWbet{Pflicht} und \RWbet{Schuldigkeit} zu halten. Eine Offenbarung aber kann dieß Verhalten zu einer solchen bestimmten Pflicht und Schuldigkeit erheben. So kann \zB\ jeder Unbefangene einsehen, daß die Vorschrift des katholischen Christenthumes, bei dem Geschäfte seiner sittlichen Vervollkommnung sich eines Gehülfen zu bedienen, diesem von Zeit zu Zeit seinen innern Zustand zu eröffnen, \usw , der Tugend sehr zuträglich sey. Gleichwohl dürfte man in der natürlichen Religion diese Vorschrift höchstens als einen Rath, gewiß nicht als eine bestimmte Pflicht und Schuldigkeit aufstellen; recht füglich aber könnte uns Gott in einer Offenbarung erklären, daß er die Vergebung unserer Sünden an die Erfüllung dieser Vorschrift als eine unerläßliche Bedingung binde.
\end{aufza}

\RWabs[\RWSeitenwohne{407}]{Anhang}{Prüfender Ueberblick der sonst gewöhnlichen Theorieen über die Möglichkeit und Kennzeichen einer Offenbarung}
\RWpar{164}{Nothwendigkeit dieses Anhanges}
Obgleich die bisher vorgetragene Lehre über die Möglichkeit und die Kennzeichen einer Offenbarung meinem Dafürhalten nach ganz damit übereinstimmt, was der \RWbet{gemeine Menschenverstand} von jeher über diesen Gegenstand anerkannt hat; so weicht diese Darstellung doch in so vielen und so wesentlichen Stücken von allen bisherigen Darstellungen, die in den Schriften der Gelehrten angetroffen werden, ab, daß es kaum zu verzeihen wäre, wenn ich bei einer so wichtigen Sache nicht auch die Ansichten Anderer mit ihren Gründen anführen wollte. Ich werde dieß in der gedrängtesten Kürze und in Begleitung mit meinen Gegenbemerkungen thun, wodurch denn ein Jeder in den Stand gesetzt werden soll, um so verlässiger zu beurtheilen, welche Darstellung der Wahrheit näher komme.

\RWpar{165}{Wie die bisherigen Vertheidiger der Offenbarung den Begriff ihrer Möglichkeit faßten?}
\begin{aufza}
\item Daß man schon den Begriff einer Offenbarung insgemein anders, als ich es im I.\,Hptst.\ \RWparnr{28} u.\ 30.\ gethan, erklärt habe, erwähnte ich schon dort, und führte \RWparnr{34} auch die wichtigsten, bisher gegebenen Erklärungen an. Man konnte hieraus ersehen, daß Alle, die das Wort Offenbarung in jener engsten Bedeutung nehmen, in der die geoffenbarte Religion der natürlichen entgegengesetzt wird, im Grunde eben~\RWSeitenw{408}\ denselben Begriff damit verbanden wie ich, und höchstens darin fehlten, daß sie sich die Bestandtheile, aus denen dieser zusammengesetzte Begriff besteht, zu keinem deutlichen Bewußtseyn brachten. Aber eben hieraus läßt sich im Voraus schon vermuthen, daß sie auch die \RWbet{Möglichkeit} und noch mehr die \RWbet{Kennzeichen} einer Offenbarung nicht richtig angeben werden.
\item Daß eine Offenbarung \RWbet{möglich} sey, haben nicht nur alle Vertheidiger derselben, sondern auch Mehre aus denjenigen, die ihre \RWbet{Wirklichkeit} bestreiten, angenommen.
\item Um aber diese \RWbet{Möglichkeit} gehörig zu beweisen, unterschieden die Meisten erst gewisse \RWbet{Arten der Möglichkeit}, größtentheils andere, als die ich \RWparnr{136}\ angegeben habe.
\begin{aufzb}
\item So unterscheiden Einige die \RWbet{innere} und die \RWbet{äußere Möglichkeit} einer Sache und sagen: die erstere wäre vorhanden, wenn der Begriff der Sache sich nicht selbst widerspräche; die letztere aber, wenn auch die \RWbet{Wirklichmachung} derselben in der Macht irgend eines Wesens, wenigstens in der Macht Gottes stehe.
\item Andere dagegen unterscheiden die \RWbet{logische} und \RWbet{reale Möglichkeit}. Jene, sagen sie, wäre vorhanden, wenn eine Sache \RWbet{gedacht}; diese, wenn sie auch in \RWbet{Wirklichkeit gesetzt} werden könne.
\item Noch Andere unterscheiden die \RWbet{physische} und \RWbet{moralische Möglichkeit}; bei welchem Gegensatze sie unter jener verstehen, daß die Wirklichwerdung der Sache, abgesehen von den Forderungen des Sittengesetzes möglich sey; unter dieser, daß sie auch mit den Forderungen des Sittengesetzes in keinem Widerspruche stehe. (So thut es \zB\ \RWbet{Bretschneider.})
\end{aufzb}
\item Von einer Offenbarung behaupten sie dann insgemein, daß sie in allen diesen Bedeutungen des Wortes möglich wäre.
\item Ich meines Theils glaube schon an der Art, wie der Begriff der Möglichkeit hier aufgefaßt wurde, verschiedene Mängel zu gewahren.
\begin{aufzb}
\item Meinem Dafürhalten nach gehört es in keiner Bedeutung des Wortes zur \RWbet{Möglichkeit} einer Sache, daß ein Wesen da sey, das dieser Sache Wirklichkeit geben könne. Nach dieser Erklärung wäre \zB\ Gott nicht mög\RWSeitenw{409}lich; denn es ist doch gewiß kein Wesen vorhanden, das ihn hervorbringen könnte. Und selbst, was die Möglichkeit solcher Dinge betrifft, die an sich zufällig sind: so bedürfen auch diese zu ihrer Möglichkeit nicht der \RWbet{Wirklichkeit}, sondern nur der \RWbet{Möglichkeit} ihrer Ursache, also auch nicht des wirklichen Daseyns, sondern der bloßen Möglichkeit eines Wesens, das sie hervorzubringen vermag. (\RWparnr{137}\ \no\,2)
\item Die bloße \RWbet{Denkbarkeit} einer Sache ist meiner Meinung nach etwas ganz Anderes als ihre Möglichkeit. Jene ist weder nothwendig zu dieser, noch zu ihr hinreichend. Die Denkbarkeit einer Sache fordert die Möglichkeit eines Wesens, das sie zu denken vermag; davon ist aber, wenn man die Möglichkeit einer Sache \RWbet{selbst} untersucht, gar nicht die Rede. Wenn man \zB\ frägt, ob Thierchen, die Millionenmal kleiner als ein Sandkörnchen sind, möglich wären? so fragt man gar nicht darnach, ob es auch Jemand gebe, der diese kleinen Thierchen sich \RWbet{vorzustellen} vermöchte? -- Von der andern Seite wieder läßt sich sehr Vieles denken, was doch in keiner Bedeutung des Wortes möglich genannt werden darf; denn auch das \RWbet{Widersprechende}, \zB\ einen viereckigen Kreis, oder $\sqrt{-1}$ \udgl , kann man sich denken.
\item Wenn die Wirklichmachung einer Sache von Gott allein abhängt: so ist sie \RWbet{unmöglich}, sobald sie dem Sittengesetze nicht gemäß ist. Handelt es sich aber um eine Sache, deren Wirklichmachung von einem endlichen \RWbet{freien Wesen} abhängt: so hat die Frage, ob diese Wirklichmachung dem Sittengesetze gemäß oder nicht gemäß wäre, mit der Frage, ob diese Sache möglich ist, abermals nichts zu schaffen. (\RWparnr{136}\ Anm.)
\end{aufzb}
\end{aufza}

\RWpar{166}{Wie sie die Möglichkeit einer Offenbarung erwiesen?}
\begin{aufza}
\item Um nun die Möglichkeit der Offenbarung in einer jeden der vorhin angegebenen Bedeutungen des Wortes darzuthun, beriefen sich ihre Vertheidiger insgemein nur auf die \RWbet{Allmacht Gottes.} Da Gott Alles vermag, so muß er~\RWSeitenw{410}\ sich wohl auch uns offenbaren können. Sie schloßen ferner von dem, was der \RWbet{Mensch} vermag, auf das, was \RWbet{Gott} vermag. Ein Mensch kann sich dem andern offenbaren: um wie viel mehr, sagten sie, wird dieß Gott vermögen. Die \RWbet{moralische} Möglichkeit der Offenbarung bewiesen sie insbesondere aus ihrer Nützlichkeit und Nothwendigkeit für uns.
\item Hiegegen erinnere ich aber:
\begin{aufzb}
\item Aus der \RWbet{Allmacht Gottes kann} man die Möglichkeit einer Offenbarung, wie überhaupt die Möglichkeit einer jeden Sache, deren Hervorbringung von Gott allein abhängt, immer nur durch einen \RWbet{Zirkel} beweisen. Denn jene Allmacht, die sich allein bei Gott beweisen läßt, bestehet bloß darin, daß er alles dasjenige, \RWbet{was an sich selbst möglich ist}, dem \RWbet{Wohle des Ganzen} entspricht, und einer Bedingung zu seinem Daseyn bedarf, hervorbringen könne und wirklich hervorbringe. Also muß immer erst bewiesen werden, daß etwas an sich selbst möglich, und dem Wohle des Ganzen zuträglich sey, bevor sich aus Gottes Allmacht der Schluß ableiten läßt, daß auch er es hervorbringen könne.
\item Von dem, was der \RWbet{Mensch} vermag, läßt sich nicht sicher auf das, was \RWbet{Gott} vermag, schließen. So vermag der Mensch Böses zu thun, Gott aber nicht.
\item Endlich können wir auch nicht strenge darthun, daß eine Offenbarung dem menschlichen Geschlechte in allem Betrachte nützlich und nothwendig sey, sondern bloß sagen, daß sie uns, so weit wir es einsehen, nützlich und nothwendig \RWbet{scheine}. Daher kann auch nicht sicher gefolgert werden, daß es der Heiligkeit Gottes gemäß, oder moralisch möglich sey, sich uns zu offenbaren.
\begin{RWanm}
Die Bemerkung, daß es überhaupt zu viel sey, die \RWbet{vollkommene Möglichkeit} einer göttlichen Offenbarung zu behaupten, daß man sich nur mit einer \RWbet{problematischen} bescheiden müsse, wie auch der ganze Begriff dieser problematischen Möglichkeit selbst, ist den bisherigen Bearbeitern dieser Lehre meistens entgangen. Nur \RWbet{Gottfried Leß} verfällt (\RWbet{Ueber Religion} \usw\ Bd.\,2.\ S.\,180.)\RWlit{}{Less3} einmal auf diesen Begriff, \RWSeitenw{411}\ ohne ihn jedoch gehörig festzuhalten und zu benützen. Er nennt Behauptungen, die ich problematisch möglich nenne, \RWbet{logisch indifferente} Behauptungen. \end{RWanm}
\end{aufzb}
\end{aufza}
   
\RWpar{167}{Wie dieses insbesondere einige Anhänger der kritischen Philosophie gethan?}
Die \RWbet{kritische Philosophie} spricht der theoretischen Vernunft gänzlich das Recht ab, in Betreff irgend eines \RWbet{übersinnlichen Gegenstandes} etwas zu urtheilen. (S.\ \RWparnr{61}.) Ein solches Urtheil wäre es aber auch, wenn man die \RWbet{Möglichkeit einer göttlichen Offenbarung} behaupten oder läugnen wollte. Keines von Beidem soll also die theoretische Vernunft vermögen. Nichts desto weniger hatte der Stifter dieser Philosophie selbst schon ein (\aaO\ bereits beschriebenes) Mittel gefunden, wodurch er uns von den \RWbet{drei wichtigsten Wahrheiten}, die einen übersinnlichen Gegenstand betreffen (nämlich von \RWbet{Gottes Daseyn}, von unserer \RWbet{Freiheit} und \RWbet{Unsterblichkeit}) auf einem andern Wege, nämlich durch die so genannte \RWbet{Methode des praktischen Postulirens}, \RWbet{Gewißheit} zu verschaffen glaubte. Diese Art zu schließen gefiel nicht nur den Anhängern seiner Philosophie, sondern auch Mehren, die seine übrigen Behauptungen verwarfen. Allmählich glaubte man zu bemerken, daß die drei Gegenstände: \RWbet{Gott, Freiheit, Unsterblichkeit} gar nicht die einzigen wären, die sich auf diese~\RWSeitenw{412}\ Art herleiten ließen; und Freunde der christlichen Offenbarung versuchten durch ähnliche Postulate insonderheit noch die \RWbet{Möglichkeit} einer göttlichen Offenbarung, endlich auch \RWbet{mehre einzelne Lehren des Christenthums}, \zB\ die Versöhnungslehre \ua\ herzuleiten. Dieß thaten namentlich \RWbet{Tieftrunk} (in seiner \RWbet{Censur des christlich-protestantischen Lehrbegriffes.} Th.\,3., Berlin, 1796.)\RWlit{}{Tieftrunk1}, \RWbet{Jakob Frint} (in seinem \RWbet{Leitfaden der Religionswissenschaft}. Wien, 1806.)\RWlit{}{Frint1}, \RWbet{\Ahat{Kroll}{Knoll}} (in seinem \RWbet{philosophisch-kritischen Entwurfe der Versöhnungslehre.} Halle, 1799.)\RWlit{}{Kroll1} \umA\  
Das Postulat für die \RWbet{Möglichkeit} einer Offenbarung wurde so abgefaßt: Was immer die praktische Vernunft fordert, muß möglich seyn; sie fordert aber die Realisirung des höchsten Gutes; also muß diese möglich seyn. Allein ohne Offenbarung ist die Realisirung des höchsten Gutes keineswegs möglich; denn nur aus einer Offenbarung lernen wir ein Genugthuungsmittel für unsere Sünden kennen, und was es sonst noch für Gründe gibt, welche die sittliche Nothwendigkeit einer Offenbarung beweisen. Also muß eine göttliche Offenbarung möglich seyn.


\RWpar{168}{Beurtheilung dieser Beweisart}
\begin{aufza}
\item So vieles Scheinbare auch diese Methode des Postulirens haben mag, so glaube ich ihr doch nicht beipflichten zu dürfen. Den Obersatz zwar, oder die Behauptung: \RWbet{Was die praktische Vernunft fordert, das muß auch möglich seyn}, gebe ich ohne Bedenken zu; allein den Untersatz, oder den Satz: \RWbet{die Vernunft fordert die Realisirung des höchsten Gutes}, finde ich \RWbet{zweideutig.} Es frägt sich hier nämlich, was man unter dem \RWbet{höchsten Gute} verstehe. Versteht man darunter, wie dieß auch wirklich Einige gethan, \RWbet{den möglich höchsten Grad der Tugend, vereinigt mit dem möglich höchsten Grade der Glückseligkeit}: so ist es zwar völlig wahr, daß die Vernunft die Realisirung des höchsten Gutes fordere; aber der Schlußsatz, den man jetzt aus der Vereinigung jener zwei Vordersätze erhält, ist eine nichtssagende Tautologie: \RWbet{der mög}\RWSeitenw{413}\RWbet{lich höchste Grad der Tugend und Glückseligkeit ist möglich}. Aus einem dergleichen identischen Satze läßt sich nun niemals etwas folgern, um wie viel weniger, daß sich das Daseyn Gottes, die Freiheit und Unsterblichkeit der Seele, und endlich die Möglichkeit einer Offenbarung aus ihm ableiten ließen. Versteht man dagegen, wie es \RWbet{Kant} selbst gethan hat, unter dem höchsten Gute \RWbet{Heiligkeit} (\dh\ vollkommene Uebereinstimmung des Willens mit dem Sittengesetze) und \RWbet{eine dem wirklich erreichten Grade der Tugend angemessene Glückseligkeit}: dann erinnere ich, daß jener Untersatz eigentlich aus zwei in einander geschobenen Sätzen bestehe, nämlich: 
\begin{inparaenum} 
\item aus dem Satze: \RWbet{Die Vernunft fordert Heiligkeit}, und 
\item aus dem Satze: \RWbet{Die Vernunft fordert eine dem wirklich erreichten Grade der Tugend angemessene Glückseligkeit}.
\end{inparaenum}
Der erste Satz ist, wie man weiß, \RWbet{identisch} (\RWparnr{88}, \no\,3,\ c.); der zweite Satz aber drückt eine Wahrheit aus, welche man keineswegs als eine Grundwahrheit, sondern als eine gefolgerte, \dh\ als eine Wahrheit ansehen muß, die einen Grund hat, der sich muß angeben lassen; denn die Begriffe, aus denen dieser Satz besteht, sind offenbar nicht so \RWbet{einfach}, wie es bei einem \RWbet{Grundsatze} seyn muß. Versuchen wir aber, einen \RWbet{Beweis} für diesen Satz zu liefern, der sich als eine objective \RWbet{Begründung} desselben ansehen ließe; so werden wir bald inne, daß er auf keine Art erwiesen werden könne, als eben unter Voraussetzung des Daseyns Gottes, der Freiheit, der Unsterblichkeit \usw\ Praktische Sätze nämlich, wie der vorliegende ist, können nie anders als aus dem obersten Sittengesetze (oder aus einer von diesem abgeleiteten Wahrheit) gefolgert werden. Das oberste Sittengesetz lautet jedoch keineswegs so, wie es \RWbet{Kant} ausdrückte (\RWparnr{90}\ \no\,16.); sondern es fordert, wie \RWparnr{88}\ gezeigt wurde, \RWbet{daß Tugend und Glückseligkeit, so viel es möglich ist, befördert werden sollen}. Es fordert also nichts Anderes, \RWbet{als was an sich möglich ist}. Seine Gebote ergehen immer nur an wirklich vorhandene Wesen, die mit Vernunft begabt sind, und eine Wollkraft haben, vermittelst deren sie gewisse Veränderungen in dem Zustande anderer lebendiger Wesen hervorbringen können; Veränderungen, welche die Tugend oder die~\RWSeitenw{414} Glückseligkeit derselben befördern oder stören. Und von solchen vernünftigen Wesen fordert das Sittengesetz nicht was immer für eine, sondern nur diejenige Beförderung der Tugend und Glückseligkeit, die eben in ihrer Macht stehet. Von einem Menschen \zB\ fordert das oberste Sittengesetz keineswegs, daß er die Tugend und Glückseligkeit der Bewohner des Mondes befördere, weil ihm dieß unmöglich ist. Hieraus ist denn ersichtlich, daß man niemals aus einer \RWbet{Forderung des Sittengesetzes} (aus einem praktischen Satze) beweisen könne, daß etwas \RWbet{möglich} oder gar \RWbet{wirklich} sey; sondern umgekehrt muß man erst dargethan haben, daß etwas möglich sey, bevor es sich als eine \RWbet{Forderung der praktischen Vernunft} darstellen läßt. So müßte man also aus theoretischen Gründen beweisen, \RWbet{die Realisirung des höchsten Gutes} sey möglich, \dh\ es sey möglich, daß jeder Tugendhafte eine ihm angemessene Belohnung finde \usw\ Dann erst ließe sich behaupten, daß die Vernunft dieß fordere. Nun kann man aber nur unter Voraussetzung des Daseyns Gottes darthun, es sey möglich, daß jeder Tugendhafte eine ihm angemessene Belohnung erhalte, daß eine Offenbarung uns zu Theil werde \udgl\  Also begeht man in dieser Beweisart offenbar einen \RWbet{Zirkel}.
\item Die \RWbet{Kant}'sche Methode des Postulirens ist daher, meiner Meinung nach, \RWbet{überhaupt} unstatthaft. Wenn man sie aber zum Beweise der Möglichkeit einer Offenbarung anwendet, begeht man noch einen besondern Fehler. Der Satz nämlich, der als Untersatz aufgestellt werden muß: \RWbet{Die Realisirung des höchsten Gutes ist ohne Offenbarung nicht möglich} -- läßt sich nicht strenge erweisen. Denn wie ich schon mehrmals gesagt: alles, was der menschliche Verstand hierüber mit Fug und Recht behaupten kann, ist nur, daß eine Offenbarung, so viel wir einsehen können, zur Beförderung der Tugend und Glückseligkeit beitragen würde; woraus noch nicht sicher zu schließen ist, daß sie auch in der That zuträglich sey, und daß der allwissende Gott nicht irgend ein etwas kenne, das noch weit zuträglicher ist, oder nicht irgend einen verborgenen Nachtheil bemerke, den eine Offenbarung für uns oder Andere nach sich ziehen würde, weßhalb die Mittheilung derselben unzweckmäßig wäre.~\RWSeitenw{415}
\item Wie fehlerhaft diese Beweisart sey, erhellet endlich noch daraus, daß sie \RWbet{zu viel} beweiset. Nach dieser Art zu schließen würde man nämlich nicht nur die \RWbet{Möglichkeit}, sondern das wirkliche \RWbet{Daseyn} einer Offenbarung aus bloßen Gründen \RWlat{a priori} zu behaupten berechtigt seyn; denn wenn eine Offenbarung zur Beförderung des höchsten Gutes erforderlich ist, so genügt ja nicht bloß ihre \RWbet{Möglichkeit}, sondern ihr wirkliches \RWbet{Daseyn} wird hiezu erfordert.
\end{aufza}

\RWpar{169}{Wie man sich über die Kennzeichen einer unmittelbaren Offenbarung erklärte?}
\begin{aufza}
\item In Betreff der \RWbet{Kennzeichen} einer Offenbarung unterscheidet man insgemein den Fall Desjenigen, dem die göttliche Offenbarung \RWbet{unmittelbar} zu Theil wird, den Fall des \RWbet{göttlichen Gesandten}, des \RWbet{Lehrers} einer Offenbarung, und den Fall der übrigen Menschen, die sie erst \RWbet{mittelbar} von ihm annehmen sollen.
\item Anlangend nun denjenigen, dem eine göttliche Offenbarung \RWbet{unmittelbar zu Theil} wird: so behaupteten Manche, dieser bedürfe eigentlich gar keines \RWbet{besonderen Zeichens}, um überzeugt zu werden, daß die Gedanken, die jetzt in ihm entstehen, \RWbet{unmittelbar} von Gott hervorgebracht würden; er werde dieß nicht erst durch \RWbet{Schlüsse} herzuleiten brauchen, sondern \RWbet{unmittelbar fühlen}. So lehrten \zB\ \RWbet{Lilienthal} (in seiner \RWbet{guten Sache der göttlichen Offenbarung}, B.\,1.\ S.\,66.)\RWlit{}{Lilienthal1}; \RWbet{Kleuker} (\RWbet{Neue Prüfung}, \usw\ B.\,1.\ S.\,185.)\RWlit{}{Kleuker1} 
\item Andere sagten, der göttliche Gesandte erkenne, daß sich ihm Gott geoffenbaret habe,
\begin{aufzb}
\item aus der Lebhaftigkeit und Bestimmtheit der in ihm entstehenden Gedanken;
\item aus dem Bewußtseyn, daß er sie nicht durch eigene Thätigkeit erzeuget;
\item aus ihrer Gotteswürdigkeit;
\item aus Wundern, die es ihm gelingt zu wirken;
\item aus der Erfüllung einiger Vorhersagungen, die er gethan, \udgl\ So erklärten sich \zB\ \RWbet{Buddeus} (in sei\RWSeitenw{416}ner \RWlat{Institutio theologiae dogmaticae})\RWlit{}{Buddeus1}; \RWbet{Ammon}; \RWbet{Stäudlin}; \uA 
\end{aufzb}
\item Andere endlich gestehen geradezu, hierüber ließe sich von uns gar nichts bestimmen, sondern nur Jene könnten hierüber urtheilen, die einer solchen unmittelbaren Offenbarung von Gott wären gewürdigt worden. So äußert sich \zB\ \RWbet{Bretschneider}.
\end{aufza}


\RWpar{170}{Beurtheilung dieser Aeußerungen} 
\begin{aufza}
\item Wenn man, wie es so viele thun, die göttliche Offenbarung als eine \RWbet{unmittelbare Belehrung} Gottes an die Menschen erklärt; so kann man sie hierauf, ohne sich selbst zu widersprechen, nicht füglich wieder in eine \RWbet{unmittelbare} und \RWbet{mittelbare Belehrung} eintheilen, indem eine \RWbet{unmittelbare Belehrung}, die gleichwohl \RWbet{mittelbar} wäre, doch wohl ein Widerspruch ist. Inzwischen entspringt dieser Widerspruch freilich nur aus der \RWbet{fehlerhaften Erklärung}, die man von dem Begriffe einer \RWbet{Offenbarung} gegeben. Bleibt man bei dem Begriffe stehen, den der allgemeine Sprachgebrauch mit dem Worte \RWbet{Offenbarung} verbindet; so kann man allerdings \RWbet{mittelbare} und \RWbet{unmittelbare Offenbarungen} und zwar in mehrerlei Bedeutungen unterscheiden. Nicht zweckmäßig wäre es jedoch meiner Meinung nach, wenn man sich hier an die buchstäbliche Bedeutung der Worte halten, und unter einer \RWbet{unmittelbaren Offenbarung} bloß eine solche verstehen wollte, deren Inhalt Gott Jemanden \RWbet{unmittelbar}, \dh\ ohne Anwendung gewisser Mittelkräfte beigebracht hätte. Dann nämlich würde diese Eintheilung ganz ohne Nutzen seyn, weil sich nie würde nachweisen lassen, daß irgend eine vorhandene Religion in Bezug auf irgend Jemand eine \RWbet{solche unmittelbare} göttliche Offenbarung gewesen sey; weil es überhaupt gar nicht wahrscheinlich ist, daß sich Gott jemals in dieser Bedeutung des Wortes einem Menschen \RWbet{unmittelbar} geoffenbart, \dh\ gewisse Vorstellungen in seinem Geiste ohne alle Benützung natürlicher Kräfte und Mittel erzeugt habe. Mir däucht es also, daß man jene Eintheilung nur in folgender Bedeutung der Worte gelten lassen könne. Eine Offenbarung gelangt~\RWSeitenw{417}\ an Jemand \RWbet{mittelbar, oder kann eine nur mittelbar an ihn ergangene Offenbarung} heißen, wenn unter andern Mitteln, wodurch sie ihm zu Theil wird, auch dieses ist, daß sie schon \RWbet{vor} ihm einem \RWbet{andern} Menschen zu Theil geworden war, \dh\ wenn er den Willen Gottes, daß er dieß oder jenes glauben soll, unter Anderem auch \RWbet{daraus} erkennt, weil schon ein Anderer diese Lehre für eine von Gott geoffenbarte ansah. Ist dieses nicht der Fall; so sagt man, die Offenbarung sey an ihn \RWbet{unmittelbar} gelangt, oder sey eine \RWbet{unmittelbar an ihn ergangene Offenbarung}.
\item Doch wie man auch immer den Unterschied zwischen unmittelbaren und mittelbaren Offenbarungen auffassen wollte: so würden die \RWbet{Kennzeichen}, an denen derjenige, dem eine unmittelbare Offenbarung zu Theil wird, sie zu erkennen hat, immer die nämlichen bleiben, die in der obigen Theorie aufgestellt wurden; denn wir entwickelten sie dort aus dem Begriffe einer \RWbet{Offenbarung überhaupt}. Was nun von \RWbet{jeder} überhaupt gilt, das muß auch von der unmittelbaren gelten. Daraus ergibt sich aber schon, wie unrichtig es sey, wenn Einige behaupten, daß der, dem eine unmittelbare Offenbarung zu Theil wird, \RWbet{gar keiner besonderen Zeichen bedürfe}, aus denen er das Daseyn dieser Offenbarung erst durch einen Schluß herleiten müßte; sondern daß er dieß durch ein unmittelbares Gefühl erkennen werde. Das Urtheil: Diese Lehre ist eine an mich von Gott ergangene Offenbarung; mit andern Worten: diese so eben in mir entstandene Vorstellung hat Gott in der Absicht erzeugt, damit ich sie als eine durch ihn selbst bekräftigte Wahrheit annehme, ist ja doch offenbar weder ein Grundsatz \RWlat{a priori}, noch ein unmittelbares Wahrnehmungsurtheil, sondern ein eigentliches \RWbet{Erfahrungsurtheil}, ein Urtheil ist es, das ich aus andern folgern, durch Schlüsse ableiten muß. Daß es kein Grundsatz \RWlat{a priori} sey, ist daraus einleuchtend, weil es kein Urtheil aus bloßen Begriffen ist. Daß es auch kein reines Wahrnehmungsurtheil ist, erhellet daraus, weil reine Wahrnehmungsurtheile bloß das Vorhandenseyn gewisser Vorstellungen in uns aussagen, nicht aber über die Ursache, woher diese Vorstellungen rühren, um so weniger über den Zweck, zu welchem sie in uns angeregt worden sind, etwas bestimmen. (\RWparnr{14})~\RWSeitenw{418}
\item Wer sich die \RWparnr{143}\,ff.\ aufgestellten Begriffe eigen gemacht hat; wird ohne Beweis erkennen, daß weder
\begin{aufzb}
\item aus der besondern Lebhaftigkeit eines Gedankens, noch
\item aus dem Umstande, daß wir ihn nicht durch unsere eigene Thätigkeit erzeugt, auf den Willen Gottes, daß wir diesem Gedanken Glauben beimessen sollen, geschlossen werden könne. Denn mit welcher Lebhaftigkeit, und ohne uns ihre Entstehung aus unserer eigenen Thätigkeit erklären zu können, erwachen zuweilen nicht auch sehr böse, \dh\ solche Gedanken in unserer Seele, von denen wir völlig versichert seyn können, daß Gott, der Heilige, ihre Annahme und Vollziehung uns verbiete?
\item Doch eben so wenig kann (nach \RWparnr{146}) auch aus der bloßen Gotteswürdigkeit eines Gedankens, \dh\ aus seiner innern Vortrefflichkeit, wenn sich mit dieser nicht noch gewisse außerordentliche Begebenheiten verbinden, gefolgert werden, daß wir an diesen Gedanken eine göttliche Offenbarung besitzen.
\item Von den Kennzeichen der \RWbet{Wunder} endlich, und
\item der \RWbet{Weissagungen} rede ich später, weil man diese Kennzeichen auch bei einer mittelbaren Offenbarung fordert.
\end{aufzb}
\item Jedenfalls wird man doch zugeben, daß sich derjenige, dem eine Offenbarung, es sey unmittel- oder mittelbar, zu Theil werden soll, von der \RWbet{Wirklichkeit} derselben überzeugen müsse. Nun ist es freilich wahr, daß man sich oft von einer Wahrheit überzeuge, ohne die Gründe, auf denen diese Ueberzeugung beruht, zu einem deutlichen Bewußtseyn bei sich erheben, und sie auch Andern verständlich \RWbet{mittheilen} zu können. Ja es gibt sogar Wahrheiten, bei denen eine solche Angabe der Gründe an sich selbst unmöglich ist, weil sie gar keine Gründe haben. Aber von dieser letzteren Art ist nicht das Urtheil, daß eine gewisse Lehre eine von Gott an uns ergangene Offenbarung sey; sondern dieß Urtheil muß, wie ich so eben bemerkte, durch eine eigene Reihe von Schlüssen aus gewissen Vordersätzen von uns selbst abgeleitet werden; und diese Vordersätze uns zu einem deutlichen Bewußtseyn zu bringen, kann wohl nichts Unmögliches seyn.
\end{aufza}

\RWpar{171}{Was man als Kennzeichen einer göttlichen Offenbarung für Jene, denen sie nicht unmittelbar zu Theil geworden ist, angibt?}
\begin{aufza}
\item Daß diejenigen, denen eine göttliche Offenbarung nicht unmittelbar zu Theil geworden ist, sich erst durch die Bemerkung gewisser \RWbet{Kennzeichen} an ihr, also durch Schlüsse, von ihrer Wirklichkeit überzeugen müßten, gibt man durchgängig zu.~\RWSeitenw{419}
\item Gewöhnlich aber unterscheidet man \RWbet{zwei Arten} dieser Kennzeichen:
\begin{aufzb}
\item solche, aus denen nur die \RWbet{Möglichkeit}, daß eine gewisse Religion eine göttliche Offenbarung sey, folgt (\RWbet{negative Kennzeichen});
\item solche, aus denen gefolgert werden kann, daß sie es \RWbet{wirklich} sey (\RWbet{positive Kennzeichen}). 
\end{aufzb}
\item In Angabe der erstern herrscht unter den Gelehrten wenig Uebereinstimmung; der Eine fordert mehre, der Andere weniger Bedingungen. Folgende sind indessen die gewöhnlichsten:
\begin{aufzb}
\item Die Lehre einer wahren göttlichen Offenbarung darf nichts \RWbet{der Vernunft Widersprechendes} enthalten; also nichts, was irgend einer erwiesenen theoretischen oder praktischen Wahrheit widerspräche.
\item Sie muß \RWbet{die sittlichen Bedürfnisse des Menschen befriedigen}; also die Zweifel in der natürlichen Religion, um derentwillen uns eine Offenbarung nothwendig ist, befriedigend lösen.
\item Und eben deßhalb mehre Lehren enthalten, deren Wahrheit \RWbet{nicht durch bloße Vernunft erweislich ist}.
\item Sie muß sich \RWbet{für alle Zeiten und für alle Völker schicken}.
\item Sie muß durch Personen bekannt gemacht worden seyn, die einen \RWbet{untadelhaften Charakter} hatten.
\item Sie muß \RWbet{sehr alt} seyn.
\item Ihr Inhalt muß \RWbet{in Schriften niedergelegt} worden seyn. (Dieß fordern insbesondere die Protestanten).
\item Sie muß \RWbet{sichtbar wohlthätige Wirkungen hervorgebracht} haben. 
\end{aufzb}
\item Als \RWbet{Kennzeichen}, durch deren Daseyn nicht mehr die bloße Möglichkeit, sondern die \RWbet{Wirklichkeit} einer göttlichen Offenbarung bewiesen würde, gab man insgemein nur \RWbet{Wunder und Weissagungen} an.
\end{aufza}

\RWpar{172}{Beurtheilung dieser Behauptungen}
\begin{aufza}
\item Der Begriff, den man mit dem Ausdrucke: \RWbet{negative Kennzeichen}, verbindet, wäre wohl richtiger durch die~\RWSeitenw{420}\ Benennung: \RWbet{Bedingungen} oder \RWbet{Erfordernisse} zu einer Offenbarung, bezeichnet worden; auch hätte man nicht sagen sollen, daß eine Religion, an der sich diese Bedingungen finden, eine göttliche Offenbarung seyn \RWbet{könne}, sondern nur umgekehrt, daß eine Religion, an der sie sich \RWbet{nicht} finden, eine göttliche Offenbarung \RWbet{nicht} seyn könne.
\item Uebrigens lassen sich auch gegen manche dieser Erfordernisse Einwendungen machen, \zB\ 
\begin{aufzb}
\item[gegen c).] Was berechtigt uns zu fordern, daß jede Offenbarung eben \RWbet{material} seyn müsse?
\item[Gegen d).] Noch weniger sind wir berechtigt zu verlangen, daß eine und dieselbe Religion für alle Zeiten und Völker passend sey. Dieses ist vielmehr wenn anders nicht durch eine eigene Anstalt dafür gesorgt ist, daß die geoffenbarte Lehre im Verlaufe der Zeiten eine gewisse Entwickelung erfahre, etwas Unmögliches.\RWfootnote{%
	S.~\RWparnr{14} des III.~Hauptstückes.}
Es ist auch kaum zu begreifen, wie diese Forderung von Theologen aufgestellt werden konnte, die der mosaischen Religion die Würde einer göttlichen Offenbarung nicht absprechen, obgleich der Apostel ausdrücklich sagt, daß sie die \RWbet{Kinderreligion} der Welt gewesen.
\item[Gegen e).] Warum könnte sich Gott zur weiteren Verbreitung, ja wohl auch zuweilen zur ersten Auffindung nützlicher Lehren nicht auch selbst solcher Menschen bedienen, deren Charakter nicht ganz untadelhaft ist? Warum sollte er nicht die von ihnen vorgetragenen Lehren durch außerordentliche Begebenheiten als Lehren, die auch er selbst von uns geglaubt wissen will, bestätigen dürfen?
\item[Gegen f).] Was man alt oder nicht alt heißen könne, ist sehr relativ. Wir müssen also fragen, \RWbet{wie alt} wird eine Religion zum Wenigsten seyn müssen, die man als eine göttliche Offenbarung annehmen kann? Und diese Frage wird sich gewiß nicht beantworten lassen; woraus die Unstatthaftigkeit der ganzen Forderung von selbst erhellt. Zumal da auch die älteste Religion einmal \RWbet{neu} war, und wenigstens nach den gewöhnlichen Ansichten darüber schon in ihrem Ursprunge eine göttliche Offenbarung gewesen seyn mußte, wenn sie es jetzt seyn soll.
\item[Gegen g).] Warum gerade in \RWbet{Schriften}? Wie wenn ein Anderer forderte, daß die Lehrsätze der Offenbarung nicht auf verweslichem Pergamente, sondern auf Asbest niedergeschrieben, oder wohl gar auf eine Felsenwand mit großen Schriftzügen eingehauen seyn sollten? Eben der Umstand, daß eine solche Forderung, könnte sie je mit Recht~\RWSeitenw{421}\ gemacht werden, immer noch höher gespannt werden dürfte, und daß es folglich gar keine Art der göttlichen Offenbarung geben könnte, die man nicht unter dem Vorwande, sie erfülle diese Bedingung nicht, verwerfen dürfte, beweiset, daß wir zu einer solchen Forderung überhaupt gar nicht berechtigt sind, sondern es Gott anheim stellen müssen, auf welche Art er dafür sorgen wolle, daß die einmal gegebene Offenbarung nicht durch die Länge der Zeit verfälschet werde; und dazu gibt es in der That bessere Mittel als schriftliche Urkunden, von welcher Art sie auch immer seyn mögen.
\item[Gegen h).] Es ist wohl freilich gewiß, daß eine Offenbarung, wenn es Gott erst für gut befunden hat, uns eine mitzutheilen, \RWbet{wohlthätige Wirkungen} hervorbringen werde. Aber daß diese Wirkungen auch immer für uns \RWbet{bemerkbar}, und mit ganz unbestrittener Sicherheit bemerkbar seyn müßten, ist etwas Anderes, das wir zu fordern nicht berechtigt sind. Oder mit welchem Rechte wollten wir eine Religion, deren Lehrsätze den höchsten Grad sittlicher Zuträglichkeit für uns besäßen, die ihre Entstehung überdieß den außerordentlichsten Begebenheiten zu verdanken hätte, bloß darum nicht für eine göttliche Offenbarung an uns erklären, weil es uns scheint, daß sie durch \RWbet{Mißverstand} oder durch \RWbet{Mißbrauch} bisher sehr vielen, und wäre es auch \RWbet{überwiegend vielen Schaden} angerichtet habe? -- Und wie, wenn eine Religion erst eben aufgekommen ist; kann man dieß Merkmal auch bei dieser verlangen? -- 
\end{aufzb}
\item Wenn es seine Richtigkeit hätte, daß die bisher geprüften Kennzeichen eigentlich nur die \RWbet{Möglichkeit} einer Offenbarung erweisen, während durch \RWbet{Wunder} und \RWbet{Weissagungen} schon ihre \RWbet{Wirklichkeit} dargethan wird: so wäre es eben deßhalb bei dem Beweise einer Offenbarung ganz überflüssig zu untersuchen, ob  sie auch jene ersteren besitze. Es würde genug seyn zu zeigen, daß Wunder und Weissagungen vorhanden sind; denn wo nur diese wären, da wäre ja schon die Wirklichkeit einer Offenbarung erwiesen, aus der sich dann von selbst ihre Möglichkeit ergäbe. Der Fall aber, daß die ersteren fehlen, während die letzteren da sind, könnte sich gar nicht ereignen. Allein alle Vertheidiger der Offenbarung gestehen, daß es \RWbet{nothwendig} sey, zu untersuchen, ob~\RWSeitenw{422}\ auch die sogenannten \RWbet{negativen Kennzeichen} da sind; und sagen, daß wir, wenn diese fehlen, gar nicht berechtigt wären, die bemerkten Wunder und Weissagungen für \RWbet{wahre Wunder} und \RWbet{Weissagungen} zu halten. Daraus ergibt sich, daß diese Gelehrten jene \RWbet{negativen Kennzeichen} eigentlich zu \RWbet{Kennzeichen der Wunder} selbst erheben. Denn eine Beschaffenheit, deren Vorhandenseyn man an einem Gegenstande nothwendig erst bemerkt haben muß, bevor man berechtigt ist, ihm ein gewisses Prädicat beizulegen, nennt man ein \RWbet{Merkmal} dieses Prädicates. Ob und wie fern nun jene Gelehrten mit Recht die vorhin angeführten Stücke als Kennzeichen der wahren Wunder und Weissagungen erklären, und ob es richtig sey, daß das Vorhandenseyn von solchen Wundern und Weissagungen die Wahrheit einer Offenbarung beweise, können wir erst beurtheilen, wenn wir vernommen haben werden, was sie unter diesen \RWbet{Wundern und Weissagungen verstehen}.
\item Im Voraus können wir jedoch schon bemerken, daß die \RWbet{Weissagungen} nach dem Begriffe, den Alle mit diesem Worte verbinden, nichts Anderes sind, als eine besondere \RWbet{Gattung von Wundern}, nämlich Wunder, die in der Vorhersagung eines zukünftigen Erfolges bestehen, oder wunderbare Vorherbedeutungen. Gerade deßhalb aber ist es nicht logisch richtig,\RWbet{Wunder und Weissagungen} als \RWbet{zwei verschiedene Kennzeichen} einer göttlichen Offenbarung anzuführen, indem die letztern schon in dem Begriffe der erstern enthalten sind. Aus diesem Grunde wird es auch nicht nöthig seyn, daß ich in der Folge neben den verschiedenen Erklärungen, die man von dem Begriffe eines \RWbet{Wunders} gegeben, jedesmal auch die dazu gehörigen Begriffserklärungen einer \RWbet{Weissagung} anführe, indem sich diese aus jenen immer von selbst entnehmen und beurtheilen lassen.
\end{aufza}

\RWpar{173}{Verschiedene Erklärungen der Wunder nebst beigefügter Beurtheilung}
Man hat so vielerlei Erklärungen von dem Begriffe eines Wunders gegeben, daß ich mich hier begnügen muß, nur die merkwürdigsten hervorzuheben, und nebenbei zugleich, so oft es~\RWSeitenw{423}\ nöthig ist, die Art und Weise beizufügen, wie man versucht hat, ihre \RWbet{Erkennbarkeit} sowohl als ihre \RWbet{Beweiskraft} darzuthun.
\begin{aufza}
\item \RWbet{Nitzsch} erklärte die Wunder als \RWbet{Begebenheiten, die dazu tauglich und von Gott bestimmt wären, um einen Lehrer als göttlichen Gesandten darzustellen}. (\RWlat{Eventus insigniendo legato divino non tantum aptus, sed etiam divinitus destinatus} (\RWlat{Dissert. quarta de Chr.\ trib.\ miracul.\ pag.\,10.}))\RWlit{}{Nitzsch2} Auf die Frage, wodurch diese Begebenheiten den erwähnten Zweck erfüllen, antwortet er bloß, es wäre dazu eben nicht erforderlich, daß sie durchaus nicht aus Naturkräften erklärbar sind; \RWlat{modo ita recedant a \RWbet{noto naturae} ordine, ut cogitationem nostram ad naturae Dominum convertant, atque ut possint referri ad ipsius consilium.} (\RWlat{De revelatione religionis externa eademque publica. Lipsiis 1808. p.\,157.})\RWlit{}{Nitzsch1} Uebrigens räumt \RWbet{Nitzsch} diesen Begebenheiten nur eine \RWbet{praktische Beweiskraft} (\RWlat{practicam probandi vim}) ein, \dh\ nur eine solche, die von dem \RWbet{guten Willen}, das Bezeugte zu glauben, abhängt; (\RWlat{quae proficiscatur a rei probandae desiderio studioque, p.\,268}).
\end{aufza}\par
Auch ich meine, daß der Begriff eines \RWbet{Zeichens} oder \RWbet{Wunders} kein anderer sey, als der Begriff einer Erscheinung, aus der wir den Willen Gottes erkennen, daß wir eine gewisse Lehre glauben sollen. Allein hier wäre anzugeben, welche \RWbet{charakteristische Beschaffenheiten} ein Wunder haben, \dh\ wie die Begebenheit beschaffen seyn muß, damit wir auf die erwähnte Absicht Gottes bei ihr zu schließen berechtigt sind.\par
   Sehr anstößig aber klingt die Behauptung, daß die Beweiskraft eines Wunders bloß \RWbet{von unserem guten Willen} abhänge. Dieses kann höchstens in dem Sinne zugegeben werden, in welchem man sagen kann, daß die Anerkennung einer jeden religiösen Wahrheit, ja auch unzähliger anderer Wahrheiten, welche auf einer längeren Reihe von Schlüssen beruhen, \zB\ selbst der meisten rein mathematischen Sätze von unserem guten Willen abhängt; nämlich in sofern, als es auf unseren Willen ankommt, ob wir die Aufmerksamkeit unseres Geistes auf die Gründe jener Wahrheiten hinrichten~\RWSeitenw{424}\ oder nicht. Verstände man aber jene Behauptung so, als ob die Beweiskraft der Wunder auf einer Art von sittlicher Selbsttäuschung beruhte, so wäre sie falsch. Durch Schlüsse, die eben so richtig und bündig sind, als sie in irgend einem mathematischen Beweise vorkommen, kann sich der Mensch überzeugen, daß eine Religion, welche die beiden oben beschriebenen Kennzeichen an sich hat, eine wahre göttliche Offenbarung sey.
\begin{aufza}\setcounter{enumi}{1}
\item \RWbet{Hugo Grotius} (\RWlat{de veritate religionis christianae})\RWlit{}{Grotius1}, \RWbet{Kleuker} (neue Prüfung und Erklärung der vorzüglichsten Beweise der Offenbarung. Königsberg 1787)\RWlit{}{Kleuker1} \umA\  erklärten die Wunder bloß als \RWbet{Wirkungen Gottes}, ohne zu unterscheiden, ob diese Wirkungen unmittelbar oder mittelbar von ihm herrühren sollen. Hieher scheint auch die Erklärung einiger \RWbet{kritischer Philosophen} zu gehören, wenn sie sagen, \RWbet{ein Wunder sey eine Erfahrung des Uebersinnlichen, oder eine Begebenheit in der Sinnenwelt, die eine übersinnliche Ursache hat}, indem sie unter dieser \RWbet{übersinnlichen Ursache} eigentlich \RWbet{Gott} verstehen. Hieher gehört auch \RWbet{J.~F.~Mayer}'s Erklärung (in Ewald's und Flatt's Magazin) ein Wunder sey eine Wirkung der \RWbet{wahren Natur \dh\ Gottes}.
\end{aufza}\par
   Diese Erklärungen sind offenbar zu weit; denn nach ihnen wäre ja jede Begebenheit ein Wunder, indem sich jede als eine Wirkung Gottes, wenigstens als eine mittelbare Wirkung Gottes, betrachten läßt. Jedes Ereigniß können und müssen wir zuletzt auf Gott zurückführen.
\begin{aufza}\setcounter{enumi}{2}
\item Andere, die diesen Fehler bemerkt zu haben scheinen, erklären die Wunder \RWbet{als unmittelbare Wirkungen Gottes}. So thut es \zB\ \RWbet{Jakob Frint}.
\end{aufza}\par
   Nach dieser Erklärung, wenn wir sie buchstäblich nehmen, wäre nicht eine einzige Begebenheit, die der Mensch wahrnehmen kann, ein Wunder zu nennen; denn zur Entstehung jedes Ereignisses, das für uns wahrnehmbar seyn soll, tragen \RWbet{zum Theil} wenigstens auch gewisse endliche Substanzen und Kräfte bei, namentlich die Organe unsers Körpers; und man kann folglich nie sagen, das ganze wunderbare Ereigniß sey nur von Gott allein unmittelbar hervorgebracht, gesetzt auch, daß Gott \RWbet{Einiges} dabei unmittelbar gethan hätte. Wo mag \zB\ bei~\RWSeitenw{425}\ der Auferstehung eines Todten immerhin Gott auch Einiges unmittelbar thun; die ganze Begebenheit selbst kann doch nicht Gottes unmittelbare Wirkung heißen, in wiefern Mehres dabei \zB\ der Leib des Todten \udgl\  gewiß nur durch Gottes mittelbare Wirkung da ist. Wollte man aber, um diesen Fehler der Erklärung zu verbessern, sagen, daß man schon jede Begebenheit, bei der \RWbet{nur etwas} von Gott unmittelbar bewirkt wird, ein Wunder nennen wolle, so würden nun wieder \RWbet{alle} Begebenheiten zu Wundern; denn bei allen findet sich etwas, und zwar das Daseyn der bei ihnen wirksamen Substanzen selbst, welches von Gottes unmittelbarer Wirkung herrührt, indem bekanntlich die Schöpferkraft nur Gott allein zukommt. Man müßte also nur sagen, daß man ein Wunder oder eine unmittelbare Wirkung Gottes eine solche Begebenheit nenne, bei der Gott \RWbet{nebst der Erschaffung und Erhaltung der Substanzen}, die bei ihr thätig sind, noch irgend etwas \RWbet{Anderes}, nämlich gewisse \RWbet{Veränderungen an diesen Substanzen unmittelbar bewirket}. Bei dieser Erklärung ließe sich wohl die \RWbet{Möglichkeit} der Wunder nicht bestreiten; ihre \RWbet{Erkennbarkeit} aber und ihre \RWbet{Beweiskraft}, \dh\ ihre Fähigkeit, das Daseyn einer Offenbarung zu beweisen, ließe sich auf keine Weise darthun, wie wir bald sehen werden. Uebrigens verräth sich die Fehlerhaftigkeit dieser Erklärung auch schon aus folgenden zwei Umständen:
\begin{aufzb}
\item Wenn man die Wunder als unmittelbare Wirkungen Gottes erklärt; so läßt sich gar nicht absehen, wie die \RWbet{Vortrefflichkeit der Lehre}, zu deren Bestätigung sie dienen sollen, ein \RWbet{Kennzeichen} derselben seyn könne. Denn wie will man daraus, daß eine gewisse Lehre \RWbet{zuträglich} ist, schließen, daß eine gewisse Begebenheit durch keine \RWbet{endliche Kräfte} hervorgebracht seyn könne? Das Eine steht mit dem Andern offenbar nicht in der geringsten Verbindung.
\item Aus dieser Erklärung läßt sich auch nicht begreifen, wie Wunder einen \RWbet{Grad} haben können, welches doch der gemeine Menschenverstand voraussetzt, wenn er Ein Wunder größer als ein anderes nennt.~\RWSeitenw{426}
\end{aufzb}

\RWpar{174}{Vergebliche Versuche, die Erkennbarkeit der Wunder als unmittelbare Wirkungen Gottes zu beweisen}
\begin{aufza}
\item Bestimmt man den Begriff eines Wunders so, wie es im vorigen §~\no\,3\ zuletzt geschehen ist: so erhebt sich die Frage, \RWbet{wie und woran man erkenne, daß eine gewisse Begebenheit ein Wunder sey}? Läßt sich diese Frage nicht beantworten, \dh\ gibt es kein sicheres Kennzeichen der Wunder; so ist die ganze Theorie, die wir hier prüfen, zu verwerfen, weil sie ein \RWbet{Kennzeichen} der Offenbarung aufstellt, welches selbst \RWbet{unerkennbar} ist.
\item Und das ist wirklich der Fall; denn begreiflich wird man nur dann berechtigt seyn, zu behaupten, daß bei einer gewissen Begebenheit eine \RWbet{unmittelbare Einwirkung} Gottes Statt habe, daß sie mithin ein Wunder sey, wenn sich an ihr etwas von der Art befindet, das schlechterdings nicht als Gottes \RWbet{mittelbare} Wirkung, \dh\ als Wirkung einer endlichen geschaffenen Substanz, angesehen werden kann. Nun zeigt es sich aber bei einer nähern Betrachtung, daß wir zu der Behauptung, eine gewisse beobachtete Erscheinung könne durch keine endliche, geschaffene Substanz hervorgebracht seyn, weder durch Gründe \RWlat{a priori}, noch durch Gründe aus der Erfahrung jemals berechtiget seyn können.
\item \RWbet{Nicht durch Gründe} \RWlat{\RWbet{a priori}}. Zwar gab man verschiedene Wirkungen an, aus denen sich \RWlat{a priori} beweisen lassen sollte, daß sie von einer \RWbet{endlichen Substanz} nicht hervorgebracht werden können. Eine genauere Prüfung zeigt aber, es sey entweder nicht erweislich, daß eine solche Wirkung, wie man im Sinne hat, von einem endlichen Wesen gar nicht hervorgebracht werden könne; oder es sey nicht erweislich, daß diese Wirkung in einem gegebenen Falle Statt gefunden habe. So hat man
\end{aufza}
\begin{aufzb}
\item gesagt, \RWbet{kein endliches Wesen könne eine Substanz erschaffen oder vernichten}. Nehmen wir also eine Erscheinung wahr, bei welcher eine Substanz, die vorhin nicht vorhanden war, plötzlich~\RWSeitenw{427}\ entstehet, oder im Gegentheile eine vorhandene vernichtet wird: so können wir schließen, dieses sey durch Gottes \RWbet{unmittelbare} Wirkung geschehen. --
\end{aufzb}\par
Meiner Meinung nach kann sich gar niemals zutragen, was hier vorausgesetzt wird. Substanzen nämlich können nicht in der Zeit entstehen oder verschwinden; und wenn man sagt, daß sie erschaffen worden sind, so heißt dieß nicht, daß sie zu einer gewissen Zeit entstanden, sondern, daß sie den Grund ihres Daseyns in dem (beständigen) Willen Gottes haben. Allein selbst, wenn es möglich wäre, daß eine Substanz entstehe oder vergehe; so könnten wir dieß gleichwohl nie beobachten. Wir können höchstens beobachten, daß eine Substanz, die vorhin nicht auf uns wirkte, nun auf uns zu wirken anfange, oder daß eine Substanz, die bisher auf uns wirkte, jetzt zu wirken aufhöre; aber daraus folgt nicht, daß sie im ersten Falle früher gar nicht vorhanden gewesen, im zweiten Falle später zu seyn aufgehört habe. Denn solche Erscheinungen ließen sich ja wohl auch durch eine bloße Ortsveränderung \udgl\  erklären.
\begin{aufzb}\setcounter{enumi}{1}
\item Kein \RWbet{endliches Wesen}, sagte man ferner, \RWbet{kann eine Substanz in eine andere verwandeln}, \dh\ die wesentlichen Kräfte derselben ändern; \zB\ eine \Ahat{materielle}{materiale} Substanz mit geistigen Kräften versehen, \udgl\  Bemerken wir also einmal, daß etwas Solches geschehen ist, so können wir sicher schließen, daß dieses eine unmittelbare Wirkung Gottes, \dh\ ein Wunder sey.
\end{aufzb}\par
Ich sage, wenn man eine jede Veränderung, die in den Kräften einer Substanz hervorgebracht wird, eine \RWbet{Verwandlung} derselben in eine andere nennen will (was eine sehr unschickliche Redensart wäre): so ist es falsch, daß eine endliche Substanz nie eine andere verwandeln könne. Denn da ist jeder chemische oder organische Vorgang eine Verwandlung, und doch werden diese Erscheinungen bekanntlich bloß durch endliche Substanzen bewirkt. -- Nennt man aber Verwandlung eine solche Aenderung in den Kräften einer Substanz, \RWbet{die nur das unendliche Wesen allein bewirken kann}; so ist es freilich wahr, daß kein endliches We\RWSeitenw{428}sen eine Substanz in eine andere verwandeln könne; aber um diesen Satz anzuwenden, müßte man erst angeben können, welche Veränderungen dieß sind, und darthun, daß dergleichen irgendwo Statt gefunden haben. Dieß wird man aber nie vermögen; denn sey es auch wahr, was man zu einem Beispiele anführt, daß die Ausrüstung einer {materiellen} Substanz mit \RWbet{geistigen} Kräften eine solche Veränderung wäre, die nur das unendliche Wesen bewirken kann: wie will man darthun, daß sich diese Veränderung irgendwo zugetragen habe? Eben deßhalb, weil sie durch keine endliche Substanz hervorgebracht werden kann, ist sie so unwahrscheinlich, daß man in einem jeden gegebenen Falle, wo es den Anschein hat, sie habe sich zugetragen, jede andere Voraussetzung \zB , daß man nicht gut beobachtet habe, oder daß eine geistige Substanz mit jener materiellen bloß in Vereinigung getreten sey, \udgl\  lieber, als eine wirkliche Verwandlung annehmen müßte.
\begin{aufzb}\setcounter{enumi}{2}
\item \RWbet{Kein endliches Wesen}, behauptet man letztlich, \RWbet{kann ein vorhandenes Naturgesetz aufheben}. Bemerken wir also eine Erscheinung, die einem hinlänglich erwiesenen Naturgesetze gerade zuwider ist; so ist sie ein Wunder.
\end{aufzb}\par
Ich erinnere, daß man allgemein und mit Recht Naturgesetze einer doppelten Art unterscheide: Gesetze \RWlat{a priori} und empirische Naturgesetze. Die erstern sind solche, deren Nothwendigkeit wir durch die Vernunft selbst einsehen können, \zB\ daß ein geworfener Körper, wenn er sich selbst überlassen bleibt, seine Bewegung in gerader Linie mit gleicher Geschwindigkeit ins Unendliche fortsetzen müsse. Solche Naturgesetze kann auch Gott selbst nicht aufheben, weil sie Nothwendigkeit haben. \RWbet{Empirische Naturgesetze} dagegen sind Regeln, die wir uns bloß aus der Erfahrung abgezogen haben, und eben deßhalb noch nicht mit völliger Gewißheit und in gänzlicher Allgemeinheit aufstellen können, \zB\ daß alle Körper durch Wärme ausgedehnt werden, \udgl\  Mit welchem Rechte wollte man nun behaupten, daß keine endliche Substanz eine Wirkung hervorbringen~\RWSeitenw{429}\ könne, die diesen letzteren Regeln widerspräche, da wir doch von der Allgemeingültigkeit derselben nicht überzeugt sind, und wenn wir es wären, nicht einmal Gott die Macht, von ihnen abzuweichen, beilegen dürften? Das Täuschende in diesem Satze liegt nur in dem falschen Begriffe, den sich Viele von einem \RWbet{Naturgesetze} machen, wenn sie sich vorstellen, es wäre ein Gesetz, dessen Beobachtung Gott der Natur, \dh\ den geschaffenen Wesen willkürlich \RWbet{vorgeschrieben}, von dem er sich aber selbst freisprechen könne.
\begin{aufza}\setcounter{enumi}{3}
\item \RWbet{Auch nicht aus Gründen der Erfahrung}; denn um aus Gründen der Erfahrung berechtigt zu seyn zu der Behauptung, daß eine gewisse Wirkung von keiner endlichen Substanz herrühren könne, müßten wir uns durch Erfahrung erst eine vollständige Kenntniß aller geschaffenen Wesen, die es im Weltall gibt, und aller Kräfte derselben verschafft haben, was unmöglich ist. Man hat zwar vorgeschützt:
\begin{aufzb}
\item Eine \RWbet{vollständige} Kenntniß aller geschaffenen Wesen und ihrer Kräfte wäre nur nöthig, um auf eine \RWbet{positive} Art zu behaupten: so viel und nicht mehr vermögen endliche Substanzen zu bewirken; dagegen zu dem bloß \RWbet{negativen} Urtheile: dieses und jenes vermögen endliche Substanzen nicht, sey keine vollständige Kenntniß aller geschaffenen Wesen und ihrer Kräfte erforderlich. So wäre es \zB\ nur dazu, um positiv bestimmen zu können, wie groß die Last sey, die ein gewisser Mensch forttragen kann, nöthig, eine vollständige Kenntniß aller seiner Muskel und ihrer Kräfte zu haben; um aber bloß das negative Urtheil, daß er 10000 Centner nicht zu ertragen vermöge, fällen zu können, bedürfe es keiner solchen Kenntniß.
\end{aufzb}
\end{aufza}\par
   Die Unterscheidung zwischen einem positiven und negativen Urtheile ist richtig; aber zu dem Zwecke, zu dem man sie hier benützen will, ist sie nicht zureichend. Das negative Urtheil, welches hier aushelfen soll, kann nur für Eines~\RWSeitenw{430}\ von Beidem, entweder für ein aus Gründen \RWlat{a priori}, oder für ein aus Gründen der Erfahrung erkanntes Urtheil ausgegeben werden. Im ersten Falle gestehe ich die Möglichkeit einiger solcher Urtheile zu; aber ich habe schon \no\,3.\ gezeigt, daß sich aus ihnen nichts für die gegenwärtige Behauptung folgern lasse. Soll aber dieß Urtheil sich auf Gründe der Erfahrung stützen; so kann es offenbar kein streng erwiesenes seyn, so lange wir nicht alle geschaffenen Wesen und ihre Kräfte untersucht haben. Wenn es \zB\ Jemand nicht anders woher, \dh\ aus Gründen \RWlat{\RWbet{a priori}} weiß, daß es in einem gewissen Kasten, welcher Kugeln von den verschiedensten Farben enthält, nicht auch eine schwarze gebe, sondern wenn er dieß erst aus der Erfahrung, nämlich aus der Beschaffenheit der Kugeln, die man allmählich aus demselben hervorholt, kennen lernen will; so kann er so lange nicht mit voller Sicherheit behaupten, es könne keine schwarze Kugel aus diesem Kasten zum Vorscheine kommen, so lange es nur eine einzige von diesen Kugeln gibt, die noch nicht ausgezogen wurde, deren Farbe er also noch nicht kennen gelernt hat. Wohl mag es sehr \RWbet{unwahrscheinlich} seyn, daß eine schwarze Kugel im Kasten sey, wenn wir bereits eine sehr große Menge von Kugeln herausgezogen, und alle von andern Farben befunden haben; aber Gewißheit ist doch nie vorhanden. Eben so ist es, auch ohne eben alle Muskel und Kräfte eines Menschen zu kennen, sehr unwahrscheinlich, daß er durch eigene Kraft 10000 Centner werde zu heben vermögen, aber mit völliger Gewißheit können wir dieß gleichwohl nicht behaupten, so lange wir die Beschaffenheit nur Eines seiner Muskel nicht kennen. Doch möchten auch negative Urtheile von einer solchen Art, wie dieses als Beispiel gebrauchte; Urtheile nämlich, worin bloß von einem einzelnen Wesen oder auch von einer ganzen Gattung natürlicher Wesen, \zB\ der Menschen, die Rede ist, einen noch so hohen Grad der Gewißheit ersteigen: zu dem Schlusse, daß eine gewisse von uns beobachtete Erscheinung eine unmittelbare Wirkung Gottes seyn müsse, werden sie nie berechtigen. Denn ist es sicher genug, daß ein Wesen von dieser Art die von uns wahrgenommene Erscheinung nicht hervorbringen könne; so folgt hieraus nur, daß wir sie nicht einem solchen zuschreiben sollen;~\RWSeitenw{431}\ daß wir sie aber auch nicht als Wirkung irgend eines \RWbet{andern} geschaffenen Wesens, eines solchen, dessen Kräfte uns unbekannt sind, und dessen Gegenwart wir gar nicht bemerkt haben, zuschreiben dürfen, das folgt auf keine Weise. Dieses würde erst folgen, wenn unser negatives Urtheil von der Form wäre: \RWbet{Kein endliches Wesen vermag eine Erscheinung von der Art, wie die beobachtete, hervorzubringen.} Aber wo wäre die Erfahrung, die uns zur Bildung eines Urtheils von dieser Art nur im Geringsten berechtigen könnte?
\begin{aufzb}\setcounter{enumi}{1}
\item Andere sagen, durch die Erfahrung lernen wir die Kräfte \RWbet{einiger} geschaffener Wesen, wenigstens solcher, die uns zunächst umgeben, \zB\ des Menschen selbst, hinlänglich kennen, wenn auch nicht ihrem \RWbet{Grade}, doch ihrer \RWbet{Art} nach. Sehen wir nun, daß eine Begebenheit sich zuträgt, ohne daß irgend eines der endlichen Wesen, die wir dabei als \RWbet{thätig} annehmen können, die hiezu nöthige Kraft auch nur der Art nach besitzt: so können wir mit aller Sicherheit behaupten, daß Gott selbst durch seine unmittelbare Einwirkung diese Begebenheit hervorgebracht habe.
\end{aufzb}\par
   Ich sage, wir werden nie berechtigt seyn zu behaupten, daß keines der endlichen Wesen, die einen Antheil an der Begebenheit haben, die Kraft sie hervorzubringen besitze. Denn erstlich kennen wir die Kräfte auch selbst derjenigen Wesen, die uns zunächst umgeben, nicht ganz genau; und zweitens können wir nie behaupten, daß nur diejenigen Wesen allein, deren Gegenwart wir bemerken, und sonst keine anderen an der Hervorbringung der Begebenheit Antheil genommen haben.
\begin{aufza}\setcounter{enumi}{4}
\item Einige Gelehrte glaubten, daß man die Schwierigkeit, der die Erkennbarkeit der Wunder als unmittelbarer Wirkungen Gottes unterliegt, am Leichtesten mittelst der \RWbet{Kant'schen Methode des Postulirens} hinwegräumen könne. Denn obgleich wir bei keinem Ereignisse, so außerordentlich es immer seyn mag, durch theoretische Vernunft berechtigt werden, dasselbe für Gottes unmittelbare Wirkung~\RWSeitenw{432}\ zu erklären: so haben wir, sagten sie, doch von der andern Seite auch keine hinlänglichen Gründe, dieß zu läugnen. In diesem Zweifel nun, in dem uns die theoretische Vernunft verläßt, kann unser \RWbet{praktisches}, \dh\ moralisches Interesse den Ausschlag geben, und weil eine Offenbarung uns zur Beförderung unserer Tugend und Glückseligkeit nothwendig ist; ohne Wunder aber, \dh\ ohne unmittelbare Wirkungen Gottes nicht geglaubt werden kann: so sind wir berechtigt, von jenen außerordentlichen Begebenheiten, die in Verbindung mit einem gewissen unserer Tugend und Glückseligkeit zuträglichen Lehrbegriffe erscheinen, \RWbet{praktisch dafür zu halten}, daß sie durch Gottes unmittelbare Wirkung hervorgebracht, und also wahre Wunder seyen.
\end{aufza}\par
   Ich glaube die Unzulässigkeit der Methode des praktischen Postulirens schon oben (\RWparnr{168}) dargethan zu haben. Wie fehlerhaft sie sey, zeigt sich hier noch aus dem besondern Umstande, daß aus ihr mehr folgen würde, als man behaupten will und kann. Man schließt nämlich aus dem Umstande, weil die theoretische Vernunft über keine Erscheinung, so außerordentlich sie auch sey, entscheiden kann, ob sie durch unmittelbare oder mittelbare Wirksamkeit Gottes erfolgt sey, und aus dem ferneren Umstande, weil eine Offenbarung ohne unmittelbare Wirkungen Gottes nicht geglaubt werden kann, daß jene außerordentlichen Begebenheiten, die mit einer für uns zuträglichen Lehre in Verbindung stehen, durch Gottes unmittelbare Wirksamkeit hervorgebracht seyn müßten. Dieser Schluß aber findet auch bei den \RWbet{alltäglichsten} Begebenheiten Statt; denn auch bei solchen kann -- wie dieß die kritischen Philosophen selbst zugeben (s.~\zB\ \RWbet{Kant's Religion innerhalb der Grenzen der bloßen Vernunft} S.\,124.\ 2te Auflage)\RWlit{124}{Kant4} -- die theoretische Vernunft nicht mit Gewißheit sagen, daß sie bloß durch natürliche Kräfte, und also nur mittelbar von Gott hervorgebracht werden. Daraus würde denn folgen, daß man, wenn eine Lehre nur sittliche Zuträglichkeit hat, auch bei den \RWbet{alltäglichsten Begebenheiten}, denen sie ihre Entstehung verdankt, berechtiget wäre, sie für eine wahre göttliche Offenbarung, jene Begebenheiten aber für Wunder zu erklären. Was nun zu viel beweiset, beweiset nichts.~\RWSeitenw{433}


\RWpar{175}{Vergeblicher Versuch, die Beweiskraft der Wunder darzuthun, wenn man darunter unmittelbare Wirkungen Gottes verstehet}
Die Frage, \RWbet{warum man aus Wundern, wenn man darunter unmittelbare Wirkungen Gottes verstehet, auf das Vorhandenseyn einer Offenbarung, \dh\ auf den Willen Gottes schließen dürfe, daß wir die Lehre glauben, mit der diese Begebenheiten verbunden sind?} ist bisher allgemein nur auf folgende Weise beantwortet worden. Es wäre Gott \RWbet{unanständig}, ja seiner Heiligkeit gänzlich zuwider, wenn er zur Bestätigung einer Lehre, die er doch in der That nicht von uns geglaubt wissen will, \RWbet{unmittelbar} mitwirken wollte.\par
In diesem Schlusse begeht man
\begin{aufza}
\item einen \RWbet{Zirkel}, so ferne man schon voraussetzt, daß unmittelbare Wirkungen Gottes zum \RWbet{Beweise} einer Lehre dienen. Denn wenn sie nicht dazu dienen, so kann man eben darum nicht sagen, daß Gott durch die Hervorbringung solcher unmittelbarer Wirkungen zur Bestätigung jener Lehre \RWbet{mitwirken} würde.
\item Auch läßt sich nicht absehen, warum das \RWbet{Mitwirken} für Gott nur dann unanständig, und seiner Heiligkeit zuwider seyn sollte, wenn es \RWbet{unmittelbar} geschieht. Ist es denn nicht gleich viel, ob man, wenn etwas böse ist, mittel- oder unmittelbar dazu beiträgt? Nicht also darauf kommt es an, ob was Gott zur Entstehung eines religiösen Lehrbegriffes beigetragen hat, unmittelbar oder mittelbar geschehen ist -- (trägt er ja doch im Grunde zu Allem, was geschieht, etwas \RWbet{unmittelbar} bei, wie wir schon \RWparnr{173}\ \no\,3 erinnert) --; sondern nur darauf kommt es an, ob dasjenige, was Gott dabei gethan, den Zweck hat, uns zum Glauben an jenen Lehrbegriff zu bestimmen oder nicht. Finden wir nun an diesem Lehrbegriffe keine sittliche Zuträglichkeit, so sind wir auch nicht berechtigt zu behaupten, Gott habe diesen Zweck.~\RWSeitenw{434} Finden wir aber sittliche Zuträglichkeit an einem Lehrbegriffe, und ist seine Entstehung zugleich bewirkt durch gewisse Ereignisse, welche sonst keinen andern sichtbaren Nutzen hätten, wenn sie nicht zur Bestätigung desselben dienen sollten: so dürfen wir schließen, sie haben diesen Zweck, gleichviel ob diese Ereignisse \RWbet{mittel-} oder \RWbet{unmittelbar} von Gott bewirkt worden sind.  
\end{aufza}
   
\RWpar{176}{Noch andere Erklärungen der Wunder}
Lasset uns nun in der \RWparnr{173}\ abgebrochenen Aufzählung der merkwürdigsten Erklärungen der Wunder fortfahren:
\begin{aufza}\setcounter{enumi}{3}
\item Eine der gewöhnlichsten lautet, die Wunder wären \RWbet{übernatürliche}, \dh\ nicht durch Naturkräfte bewirkte Begebenheiten. (\RWbet{Leibnitz, Le~Clerc, Leß, Lilienthal, Gräffe, Klüpfel, Huber, Stattler, Ildeph.\ Schwarz}, \uvA)
\end{aufza}\par
Diese Erklärung der Wunder, welche die größte Aehnlichkeit mit der \RWparnr{34}\ geprüften Erklärung einer göttlichen Offenbarung selbst hat, läßt sich in keiner der verschiedenen Bedeutungen, die man dem Worte: \RWbet{übernatürlich}, unterlegen mag, rechtfertigen.
\begin{aufzb}
\item Versteht man nämlich, wie am Gewöhnlichsten geschieht, unter Natur den Inbegriff aller geschaffenen Wesen und Kräfte; unter einer übernatürlichen Wirkung also eine solche, die durch keine geschaffene Kraft, sondern durch Gottes unmittelbare Thätigkeit allein hervorgebracht ist: so ist die gegenwärtige Erklärung der Wunder eine und dieselbe mit derjenigen, die wir so eben beurtheilt haben.
\item Versteht man unter übernatürlichen Wirkungen solche, die Gott durch Wesen und Kräfte hervorbringt, welche er erst \RWbet{in der Zeit} erschuf; so wird es begreiflicher Weise nicht möglich seyn, solche Wirkungen je zu erkennen; gesetzt auch, daß der Begriff einer erst in der Zeit vollzogenen Schöpfung nichts Widersprechendes enthielte.~\RWSeitenw{435} 
\item Ein Gleiches gilt, wenn man darunter Wirkungen solcher Art verstehen will, die Gott durch eigens \RWbet{für sie allein} geschaffene Wesen und Kräfte hervorbringt.
\item Verstände man unter übernatürlichen Wirkungen solche, die Gott durch Kräfte \RWbet{höherer Wesen}, die auf der Erde nicht einheimisch sind, hervorbringt; so thäte man wohl recht daran, daß man die \RWbet{Seltenheit} als ein wesentliches Kennzeichen eines jeden Wunders angäbe. Daß aber das Wesen, durch welches dieses seltene Ereigniß zunächst hervorgebracht wird, gerade ein \RWbet{überirdisches} seyn müßte, ist von der Einen Seite eine willkürliche Forderung, und von der andern doch nicht genügend, weil zu einem eigentlichen Wunder noch eine \RWbet{zweite} Beschaffenheit, die ich schon oft wiederholt habe, erfordert wird, nämlich, daß sich kein Nutzen des Ereignisses angeben lassen müsse, wenn wir nicht annehmen dürfen, daß es uns zur Bestätigung einer gewissen Lehre diene.
\item Verstünde man vollends unter dem \RWbet{überirdischen Wesen}, das die Erscheinung hervorgebracht haben soll, ein \RWbet{geistiges}; dann wären Wunder von dieser Art wohl eben so wenig, wie die, welche \RWbet{unmittelbare Wirkungen} Gottes seyn sollen, zu erkennen. Zwar haben Einige, \zB\ \RWbet{Ildephons Schwarz}, gemeint, man könne gewiß seyn, daß ein Ereigniß nicht durch \RWbet{leblose Naturkräfte} bewirkt sey, wenn, falls man annehmen wollte, daß es auch durch Naturkräfte gewirkt seyn könne, \RWbet{alle Möglichkeit der Erfahrung} aufhören würde. Von dieser Art wäre \zB\ die Verwandlung des Wassers in Wein; denn wer annehmen wollte, daß dieß durch Naturkräfte möglich sey, der könnte keine Erfahrung anstellen, könnte \zB\ nie sagen: Was ich hier in das Gefäß gieße, ist Wasser; weil er nicht wissen könnte, ob es sich nicht in dem Augenblicke, da er den Begriff Wasser denkt, in Wein verwandelt habe. Hierauf erwidere ich aber, die Erwartung, daß sich das Wasser nicht in Wein verwandeln werde, gründe sich nicht auf die Voraussetzung, daß dieses etwas an~\RWSeitenw{436}\ sich Unmögliches sey; sondern nur auf die Erfahrung, daß es bisher entweder noch nie, oder doch nur \RWbet{äußerst selten} geschehen, und daher überaus wenig Wahrscheinlichkeit hat. Mithin ist Erfahrung möglich, auch ohne die Voraussetzung, daß solche Erfolge an sich \RWbet{unmöglich} sind; so wie Erfahrung möglich ist trotz dem, daß die Verwandlung des Weines in \RWbet{Essig} nicht nur durch bloße Naturkräfte möglich ist, sondern sehr oft in Wirklichkeit eintritt.
\end{aufzb}
\begin{aufza}\setcounter{enumi}{4}
\item \RWbet{Döderlein, Beda Mayr}, und Andere sagten, ein Wunder \RWbet{sey eine Handlung, die über die Kräfte des Handelnden geht}.
\end{aufza}\par
Eigentlich ist es, däucht mir, ein Widerspruch, denjenigen den Handelnden zu nennen, durch dessen Kräfte gleichwohl eine Begebenheit gar nicht erfolgt seyn soll. Aber auch abgesehen von diesem Widerspruche, der sich durch eine gewisse Auslegung beheben ließe; so wären dieser Erklärung zu Folge die Wunder übernatürliche Begebenheiten, und folglich träten die so eben gerügten Schwierigkeiten auch hier ein.
\begin{aufza}\setcounter{enumi}{5}
\item Die größten katholischen Theologen, ein heil.\ \RWbet{Augustin}, ein heil.\ \RWbet{Thomas von Aquino} \uA , erkannten, daß es zu dem Begriffe eines Wunders keineswegs gehöre, daß es ein übernatürliches, durch Gottes unmittelbare Wirksamkeit erzeugtes Ereigniß sey. So sagt \zB\ der heil.\ \RWbet{Augustin}: \RWlat{Miraculum est eventus non contra omnem, sed contra eam, quae nobis nota est, naturam}. Sie begnügten sich also mit der Erklärung, daß ein Wunder nur eine \RWbet{Abweichung von den uns bekannten Naturgesetzen} wäre. So richtig nun auch diese Ansicht der Wunder war; so kann man sie doch nicht als eine \RWbet{Erklärung} in der schulgerechten Bedeutung des Wortes nehmen. Denn verstände man unter den Naturgesetzen hier Gesetze \RWlat{a priori}, so wäre keine Abweichung von denselben möglich; verstände man dagegen bloße empirische Naturgesetze, so ist eine Abweichung von denselben nichts Anderes, als was wir eine \RWbet{ungewöhnliche Begebenheit} nannten. Diese Erklärung fällt also mit der gleich folgenden zusammen.~\RWSeitenw{437}
\item Ein Wunder, sagt man, \RWbet{ist jedes ungewöhnliche Ereigniß}. -- Es ist wohl richtig, daß jedes Wunder eine ungewöhnliche Begebenheit seyn müsse; aber nur ist dieß Eine Kennzeichen derselben nicht hinreichend; daher sich denn diese Erklärung nicht umkehren läßt, indem nicht jedes ungewöhnliche Ereigniß auch schon ein \RWbet{Wunder} ist.
\item Weil es leicht zu bemerken war, daß diese Erklärung zu weit sey; so machten Andere den Zusatz: \RWbet{eine ungewöhnliche Begebenheit, die den besondern Zweck hat, zur Bestätigung einer Offenbarung zu dienen}. So beiläufig drückten sich \RWbet{Bonnet, Stephani}, \uA\ aus, die eben deßhalb jedes Wunder nur ein \RWbet{Wunder der Vorsehung} genannt wissen wollten; weil es, wie sie bemerkten, bei einem Wunder gar nicht auf jene \RWbet{Mittelursachen}, durch die es herbeigeführt wird, sondern nur auf den \RWbet{Zweck}, den Gottes \RWbet{Vorsehung} dabei hat, ankommt.
\end{aufza}\par
Dieß Alles wäre nun ganz richtig angemerkt; nur hätte man noch bestimmter angeben sollen, woran man erkenne, daß eine ungewöhnliche Begebenheit jenen besondern Zweck der Bestätigung einer Lehre habe.
\begin{aufza}\setcounter{enumi}{8}
\item \RWbet{Weland, Friedrich Seiler}, und einige Andere, welche begriffen, wie schwer es sey, natürliche und übernatürliche Begebenheiten zu unterscheiden, glaubten der Schwierigkeit abzuhelfen, indem sie sagten, ein Wunder sey eine ungewöhnliche -- übrigens immerhin durch Naturkräfte gewirkte -- Begebenheit, welche ein Mensch -- der sogenannte Wunderthäter -- \RWbet{bestimmt vorhergesagt hat}. Fragte man weiter, wodurch dieser Mensch in den Stand gesetzt worden sey, diese Begebenheit vorherzusagen; so antwortete \RWbet{Weland: durch ein auf ungewöhnliche aber doch natürliche Art erhöhtes Erkenntnißvermögen; Seiler} dagegen: \RWbet{durch eine übernatürliche Erleuchtung Gottes}.
\end{aufza}\par
Diese Erklärungen sind nun
\begin{aufzb}
\item \RWbet{zu enge}, weil nicht zu jedem Wunder ein \RWbet{Wunderthäter}, der es voraussagte, oder auf dessen Fürbitte es erfolgte, nothwendig ist.~\RWSeitenw{438}
\item Die \RWbet{Weland'sche Erklärung} stimmt mit \no\,7.\ überein, die \RWbet{Seiler'sche} dagegen weicht der Schwierigkeit, der sie entgehen will, nicht aus, weil auch \RWbet{sie} zuletzt eine \RWbet{übernatürliche} Wirkung, nicht in der Außenwelt zwar, aber doch im \RWbet{Gemüthe des Wunderthäters} fordert.
\end{aufzb}
\begin{aufza}\setcounter{enumi}{9}
\item \RWbet{Köppen} sagte: Wunder sind Begebenheiten, von denen \RWbet{kein hinreichender Aufschluß} gegeben werden kann. Fast eben so lehrten auch \RWbet{Morus, Bahrdt, Spinoza}. --
\end{aufza}\par
Diese Erklärung gibt die ursprüngliche Bedeutung des Wortes \RWbet{Wunder} (\RWlat{miraculum}) sehr richtig an, aber nicht die \RWbet{eigentliche}, die es als Kennzeichen einer göttlichen Offenbarung, also in dem Falle hat, wenn wir es gleichbedeutend mit dem Worte \RWbet{Zeichen} nehmen. Es ist wohl wahr, daß sich bei den \RWbet{meisten Zeichen} kein völliger Aufschluß über die Ursachen, die sie vermittelt haben, geben läßt; daher wir uns eben über sie wundern, und sie den Namen der \RWbet{Wunder} erhalten haben mögen. Gleichwohl gehört dieß nicht \RWbet{wesentlich} dazu, und eine Begebenheit bliebe ein Zeichen oder Wunder auch dann, wenn wir uns über die Art ihrer Entstehung den völligsten Aufschluß zu geben vermöchten.
\begin{aufza}\setcounter{enumi}{10}
\item \RWbet{Karl Bretschneider} in seinem Handbuche der Dogmatik\RWlit{}{Bretschneider1}, erklärt die Wunder als außerordentliche Begebenheiten, bei denen das, was der Wunderthäter dabei that, die \RWbet{menschlichen Kräfte} entweder schlechterdings oder doch unter den vorhandenen Umständen \RWbet{übersteigt}. Bei dieser Erklärung, meint er, lasse sich die \RWbet{Erkennbarkeit} der Wunder nicht bestreiten, weil wir ja unsere menschlichen Kräfte genau genug bestimmen könnten. Ihre \RWbet{Beweiskraft} aber leitet er daraus her, weil das, was die menschlichen Kräfte übersteigt, von \RWbet{Gott} entweder unmittelbarer oder mittelbarer Weise, \zB\ durch höhere Geister, veranlaßt worden seyn müßte; daher es denn der Heiligkeit Gottes widerspräche, wenn das nicht wahr wäre, was durch solche Wunder bezeugt wird.
\end{aufza}\par
Die Fehler dieser Theorie aufzudecken, muß für denjenigen, der das Bisherige verstanden hat, so leicht seyn, daß wir es zur Ersparung des Raumes unterlassen.~\RWSeitenw{439}
\begin{aufza}\setcounter{enumi}{11}
\item \RWbet{Dr.~Philipp Ludwig Mutzel} (über den Glauben an die im neuen Testamente erzählten Wunder. 1815. S.\,32.)\RWlit{32}{Muzel1} sagt, ein Wunder sey eine Begebenheit, von der alle Menschen gestehen müssen, daß sie dieselbe nach den Gesetzen der Natur \RWbet{zu erklären nicht vermögen}, ja nicht einmal hoffen, daß sie dieselbe je werden erklären können.
\end{aufza}\par
Nach dieser Erklärung wären zwar Wunder \RWbet{möglich}, indem es wohl freilich Ereignisse geben kann, die so abweichend von allem Bisherigen sind, daß besonders Leute, welche das stete Fortschreiten in der Naturwissenschaft nicht kennen, nicht nur für ihre eigene Person die Hoffnung aufgeben würden, sie jemals zu erklären, sondern dieß Glück auch aller Nachwelt absprechen wollten. Der \RWbet{Vernünftige} aber müßte sich sehr bedenken, irgend eine Begebenheit in dieser Bedeutung des Wortes für ein Wunder zu erklären; indem die Geschichte uns zeigt, daß man schon oft geglaubt hat, eine gewisse Erscheinung werde sich nie erklären lassen, die gleichwohl in späterer Zeit recht gut erklärt worden ist. Nach unserer Theorie ist es zum Wesen eines Wunders nicht im Geringsten nöthig, daß die Entstehung desselben nicht aus Naturkräften erklärbar sey.

\RWpar{177}{Einige Folgerungen aus der hier aufgestellten Theorie von den Kennzeichen einer Offenbarung}
Zum bessern Verständnisse der hier aufgestellten Theorie von den Kennzeichen einer Offenbarung wird es nicht undienlich seyn, schließlich noch einige der wichtigsten Folgerungen, die sich aus ihr, wenn nicht im Gegensatze der \RWbet{bisherigen} Theorien, doch viel eigenthümlicher als aus jeder andern ergeben, in Kürze anzuführen.
\begin{aufza}
\item Erklärt Gott eine gewisse Lehre für seine Offenbarung, so bezeugt er hiemit bloß seinen Willen, daß wir sie gläubig annehmen sollen; über die Frage aber, ob jene Lehre eine bloß bildliche sey, \dh\ bloß eine Vorstellung, an die wir um ihrer sittlichen Zuträglichkeit willen gewiesen sind, obgleich wir durch sie die Sache, nicht, wie sie an sich ist, erkennen, sondern nur erfahren, wie wir sie uns vorstellen~\RWSeitenw{440}\ sollen, ertheilet er uns hiemit noch keine Antwort. Denn wenn wir nur wissen, es sey Gottes Wille, daß wir eine gewisse Lehre gläubig annehmen, und uns den Wirkungen, die ihre Vorstellung in unserem Gemüthe hervorbringt, hingeben: so können und sollen wir dieß auch thun, ohne daß es uns nothwendig wäre, zu wissen, ob wir durch diese Vorstellung die Sache auch so, wie sie an sich ist, erkennen. Wenn nämlich die Lehre nur \RWbet{bildlich} ist; so kann die Wissenschaft, daß sie dieß ist, und daß sie uns also den Gegenstand nicht, wie er an sich ist, schildert, oft nur die heilsame Wirkung, welche die Vorstellung dieser Lehre für uns hervorgebracht hätte, schwächen, oder uns doch den Gebrauch derselben erschweren. Wo dieß nun wirklich ist; wo es uns also keinen Nutzen brächte, zu wissen, ob eine Lehre eigentlich oder bloß bildlich zu nehmen ist; wo wir bei reiflicher Ueberlegung es noch gar selbst einsähen, daß eine Antwort auf diese Frage uns in keinem Betrachte zuträglich, in mancher Rücksicht noch eher nachtheilig werden könnte: da dürfen wir nach einer richtigen Theorie von den Kennzeichen einer Offenbarung weder verlangen noch erwarten, daß uns Gott diese Frage beantworten werde. Denn weil die sittliche Zuträglichkeit der Lehre ein wesentliches Erforderniß zu ihrer Offenbarung ist; so darf kein Lehrsatz, dem diese sittliche Zuträglichkeit fehlt, der etwas an sich \RWbet{Gleichgültiges}, vollends etwas uns \RWbet{Schädliches} aussagt, in den Inhalt der göttlichen Offenbarung mit einbezogen werden.
\item \RWbet{Natürliche} und \RWbet{geoffenbarte Religion} stehen nicht, wie es die Gegner der letztern ihr bisher häufig vorgeworfen haben, in einem \RWbet{Widerspruche} mit einander, sondern die letztere ist vielmehr nur als eine Fortsetzung und Vollendung der erstern anzusehen. Weil nämlich alle Lehren, die eine Offenbarung enthalten kann, sittliche Zuträglichkeit haben müssen; so dürfen sie eben darum den Lehren der natürlichen Religion nicht nur nicht widersprechen, sondern sie müssen vielmehr auf das Beste mit ihnen zusammenstimmen. Man könnte sie eben deßhalb, weil ihre Annahme der Tugend und Glückseligkeit zuträglich ist, von~\RWSeitenw{441}\ jetzt an, nämlich, nachdem sie durch die Offenbarung bekannt geworden sind, selbst in die \RWbet{natürliche Religion} als Sätze aufnehmen, die auch ohne ein göttliches Zeugniß schon einige, bald größere, bald geringere Wahrscheinlichkeit haben, oder an die zu glauben wenigstens zuträglich ist.
\item \RWbet{Es gibt Grade des Wunderbaren, größere und kleinere Wunder.} Denn schon die erste \RWparnr{151}\ angegebene Beschaffenheit eines Wunders oder der Zusammenhang, in welchem dasselbe mit der Lehre, die es bestätigen soll, steht, hat einen \RWbet{Grad}, der verschieden seyn kann, wie wir \RWparnr{147}\ gezeigt. Noch offenbarer gilt dieses von jener \RWbet{zweiten} Beschaffenheit, die wir zu einem Wunder fordern, der \RWbet{Ungewöhnlichkeit} desselben. Je größer die Ungewöhnlichkeit eines Ereignisses ist, um desto größer ist bei übrigens gleichen Umständen das \RWbet{Wunder selbst}. Diese dem gemeinen Menschenverstande so einleuchtende Wahrheit, die sich aus unserer Theorie so leicht erklärt, ist von den meisten Gelehrten zu Gunsten ihrer Theorie geläugnet worden. Denn weil sie die \RWbet{Wunder} für \RWbet{unmittelbare} oder \RWbet{übernatürliche} Wirkungen Gottes erklärten, so glaubten sie auch kein \RWbet{Mehr} oder \RWbet{Weniger}, \dh\ keinen \RWbet{Grad} bei ihnen annehmen zu können, weil doch die \RWbet{Unmittelbarkeit} einer Wirkung oder die Uebernatürlichkeit derselben keine Beschaffenheit ist, die einen \RWbet{Grad} hat.
\item Bei einem Wunder kommt es, nach unserer Theorie, nicht darauf an, durch welche \RWbet{nächste Ursachen} es etwa hervorgebracht sey, ob durch einen seltenen Zusammenfluß bloß solcher Kräfte, die man \RWbet{natürliche} nennt, oder durch Gottes \RWbet{unmittelbare} Einwirkung \udgl\  Eben deßhalb ist es auch \RWbet{dem Ansehen der göttlichen Offenbarung nicht im Geringsten nachtheilig}, wenn man die Wunder derselben, so viel es thunlich ist, aus bloß \RWbet{natürlichen Kräften}, aus einem günstigen Zusammenflusse besonderer Umstände \udgl\  zu erklären sucht; vorausgesetzt, daß man bei allem diesem nie die göttliche Absicht der Begläubigung einer Lehre läugnet. Im Gegentheile, wenn diese Versuche mit der gehörigen Vorsicht geschehen, und wenn~\RWSeitenw{442}\ man sich keine dem sittlichen Charakter des Lehrers nachtheilige Voraussetzungen, \zB\ Täuschungen \udgl , ohne hinlängliche Gründe erlaubt: so können solche Versuche sogar verdienstlich seyn. Denn statt das Ansehen der göttlichen Offenbarung hiedurch zu schmälern, entstehen vielmehr folgende Vortheile:
\begin{aufzb}
\item Die Wunder, die sich zum Besten der göttlichen Offenbarung zugetragen haben sollen, gewinnen an \RWbet{innerer Glaubwürdigkeit}; denn einerseits wird der Anstoß entfernt, den, es sey nun mit Recht oder Unrecht, Viele an jeder Abweichung von einem Naturgesetze nehmen; während doch andererseits gewiß ein Jeder zugeben wird, daß er bei übrigens gleicher Beschaffenheit der Zeugen geneigter sey, ein Ereigniß zu glauben, dessen Entstehung er sich aus bloß natürlichen Kräften zu erklären vermag, als ein anderes, bei dem er dieß nicht vermag. Dann wird uns auch
\item \RWbet{Gottes Weisheit} anschaulicher, wenn man uns zeigt, wie Gott durch bloß natürliche Kräfte so übernatürlich scheinende Wirkungen hervorzubringen gewußt hat. Es dürfte auch
\item nicht ganz zu läugnen seyn, daß der Glaube an so viele übernatürliche Begebenheiten eine gewisse der Aufklärung des menschlichen Verstandes nicht sehr zuträgliche Erwartung ähnlicher Ereignisse für die Zukunft und dadurch mittelbar auch einen und den andern Aberglauben erzeuge. Ein Schade, der wegfällt, so bald man die Wunder als eine Wirkung bloß \RWbet{natürlicher} Kräfte betrachtet, die nur durch Gottes Vorsehung in eine so ungewöhnliche Verbindung gesetzt worden sind; zu geschweigen, daß am Ende jedes Erklären einer Sache, wenn es den Regeln der Logik gemäß ist, der Denkkraft des Menschen eine ersprießliche Uebung darbeut.
\end{aufzb}
   Das dunkle Gefühl dieser Vortheile scheint es denn auch gewesen zu seyn, was von jeher so viele mitunter sehr achtungswürdige Männer veranlasset hat zu versuchen, ob nicht auch einige der Begebenheiten, die uns in der heiligen Schrift erzählt werden, zum Theile wenigstens aus natürlichen Ursachen erklärbar wären. Daß aber diese Ver\RWSeitenw{443}suche bisher übel berüchtigt sind, rührt wohl nur daher, weil man nach der gewöhnlichen Lehre der Schule besorgte, das Ansehen der göttlichen Offenbarung würde durch sie gefährdet; oder weil auch diejenigen, die solche Versuche angestellt hatten, wirklich nachtheilige Folgerungen aus ihnen herleiteten.
\item Aus unserer Theorie wird auch begreiflich, daß und warum sich eine göttliche Offenbarung zu ihrer Beglaubigung meistens \RWbet{mehrer Wunder} bediene. Weil nämlich jener Schluß, worauf die Wahrheit einer göttlichen Offenbarung beruht, seiner Natur nach nur ein Wahrscheinlichkeitsschluß ist, dessen Verlässigkeit um so höher steigt, je größer die Anzahl der Wunder ist, die eine Lehre zu ihrer Bestätigung nachweisen kann: so wird Gott, der nichts zur Hälfte thut, nicht ermangeln, uns auch in der \RWbet{wichtigsten Angelegenheit} unsers Lebens, in der Frage, ob dieser oder jener Glaube uns wirklich von ihm \RWbet{geoffenbart} sey, einen recht hohen und völlig beruhigenden Grad der Gewißheit dadurch zu verschaffen, daß er der Wunder \RWbet{mehre}, ja recht viele wirket. Nach der gewöhnlichen Theorie erscheinen die vielen Wunder als eine Art von \RWbet{Verschwendung}.
\item Um zur Erkenntniß der wahren göttlichen Offenbarung zu gelangen, ist es nach unserer Theorie nicht nöthig, daß jeder einzelne Mensch \RWbet{alle Religionen auf Erden kennen lerne, alle Wundererzählungen derselben prüfe}, \usw ; sondern wenn Jeder nur thut, was er in seiner Lage und bei seinen Einsichten vermag; so ist diejenige Religion, die ihm am Ende seiner Prüfung als geoffenbart erscheint, auch in der That für ihn geoffenbart. Dieses ergibt sich schon aus einer richtigen Auffassung des bloßen Begriffes einer Offenbarung. (\RWparnr{30})
\item Nach dieser Theorie gibt es für jeden Menschen zu einer bestimmten Zeit seines Lebens immer nur eine einzige göttlich geoffenbarte Religion; zu verschiedenen Zeiten aber kann es für eben denselben, um so mehr für verschiedene Menschen, auch verschiedene göttlich geoffenbarte Religionen geben. Je nachdem sich nämlich der Grad der Kultur und andere Umstände bei einem ganzen Volke, oder auch bei~\RWSeitenw{444}\ einzelnen Menschen ändern, kann eben dieselbe religiöse Ansicht bald für sie zuträglich seyn bald nicht; und eben so können auch dieselben Begebenheiten, die früher den Charakter der Wunder für sich noch nicht hatten, ihn jetzt erhalten.
\item Nach dieser Theorie ist es kein Einwurf gegen die göttliche Offenbarung \RWbet{Einer} Religion, daß auch mehre \RWbet{andere} Religionen außerordentliche Begebenheiten ähnlicher Art zu ihrer Bestätigung aufzuweisen haben; wenn nur nicht dargethan werden kann, daß der Lehrbegriff der letztern jenen der erstern an \RWbet{sittlicher Zuträglichkeit für uns} übertrifft. Nach der gewöhnlichen Theorie dagegen legt man sich die Verbindlichkeit auf zu beweisen, daß alle Erzählungen von Wundern, die sich nur jemals zur Bestätigung anderer Religionen auf Erden zugetragen haben, erdichtet sind. Eine schwierige, und wenn man nicht leichtsinnig zu Werke gehen will, wirklich fast unauflösliche Aufgabe! --~\RWSeitenw{445}
\end{aufza}

\endinput