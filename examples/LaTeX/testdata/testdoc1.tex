\documentclass[12pt, english, a4paper]{article}
\usepackage[natbib=true]{biblatex}
\bibliography{testdoc1.bib}
\usepackage{graphicx}

\begin{document}

\title{LaTeX Test-Document}

\date{May 2016}
 
\maketitle

\begin{abstract}
Here comes the Abstract \ldots. This part is {\em emphasized}.
\end{abstract}

\newpage

\tableofcontents

\section{Simple text components}

This is the first section. It [contains] a reference to
Willibald \citet[32]{Willibald2015}. Plus, it also refers to
Kunigunde's completed works \citep{Kunigunde2010}.

% this is a comment
This is line that ends with a LaTeX-comment. % this is another comment
% this is a comment line within(!) a paragraph
This line continues the paragraph!

These are several \\
lines \\
of text.

One paragraph

Another paragraph

% This is {\em an environment that {is \bf spanning} two paragraphs

% In LaTeX it is allowed that environments continue} over several
% paragraphs.
  % multiline
  % comment
Continuation of paragraph.

\subsection{Quotations}

This is a subsction about quotations. And here is the quote:

\begin{quote}
    Gerade hierauf beruht jene Glückseligkeit des ersten Viertels unseres Lebens,
    in Folge welcher es nachher wie ein verolorenes Paradis hinter uns liegt.
    Wir haben in der Kindheit nur wenige Beziehungen und geringe Bedürfnisse, also wenig
    Anregung des Willens: der größere Theil unseres Wesens geht demnach im Erkennen auf.
    \cite[199]{Schopenhauer1851}
\end{quote}


\subsection{Bullet Lists}

Here comes a bullet list:
\begin{itemize}
\item one item
\item another item
\item An item with several paragraphs

      This ist the second paragraph of the item. It should be visible that a new
      paragraph has started, but it should also be obvious that the second
      paragraph belongs to the same item as the first paragraph.
\end{itemize}
After the list the text condinues.\footnote{This is a footnote.} Although there is no extra newline in the source,
it should be clear that this is a new paragraph.

\begin{enumerate}
\item This is the first item of an enumeration.
\item no the second item
\item now the third
      \begin{enumerate}
      \item first item of the nested enumeration
      \item second nested item
      \end{enumerate}
\item fourth item of the enumeration
\end{enumerate}


\section{Complex document components}

\subsection{Images}

 This is, how images can be embedded in a document.

\begin{figure}
\begin{center}
\includegraphics[width=\textwidth]{Graph.eps}
\caption{\label{Graph} Evolutionary simulation of the reiterated Prisoner's Dilemma.}
\end{center}
\end{figure}

\subsection{Tables}

Following comes a complex figure with a table.
\begin{figure}

\begin{center}
\begin{scriptsize}
\begin{tabular}{l|c|c|c|}
\multicolumn{1}{c}{ } & \multicolumn{1}{c}{ }     & \multicolumn{2}{c}{$\overbrace{\hspace{7cm}}^{Experiments}$} \\ \cline{2-4}
                      & {\bf computer simulation} & {\bf analog simulation} & {\bf plain experiment} \\ \hline
materiality of object
                      & semantic                  & \multicolumn{2}{c|}{material} \\ \hline
relation to target
                      & \multicolumn{2}{c|}{representation}                 & representative \\ \hline
\multicolumn{1}{c}{ } & \multicolumn{2}{c}{$\underbrace{\hspace{7cm}}_{Simulations}$} & \multicolumn{1}{c}{ } \\
\end{tabular}
\end{scriptsize}
\end{center}
\caption{Conceptual relation of simulations and experiments \citep[page needed]{Arnold2013}}\label{SimulationExperimentsScheme}

\end{figure}


\subsection{Mathematical Formulae}

\end{document}
