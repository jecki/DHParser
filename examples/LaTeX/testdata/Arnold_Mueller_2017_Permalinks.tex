\documentclass[12pt, a4paper]{article}

\usepackage{microtype}
\usepackage[czech, ngerman]{babel} 
\usepackage[utf8x]{inputenc}
\usepackage{ucs} 
%\usepackage{natbib}
\usepackage{eurosym}  

\usepackage{mathptmx}       % selects Times Roman as basic font
\usepackage{helvet}         % selects Helvetica as sans-serif font
\usepackage{courier}        % selects Courier as typewriter font
\usepackage{type1cm}        % activate if the above 3 fonts are
                            % not available on your system
\usepackage{makeidx}
\usepackage{setspace}

\usepackage{ifpdf}
\ifpdf
\usepackage{xmpincl}
\usepackage[pdftex,pagebackref=true]{hyperref}
\hypersetup{
    colorlinks,
    citecolor=black,
    filecolor=black,
    linkcolor=black,
    urlcolor=black,
    unicode=true,
    bookmarksopen=true,     % Gliederung öffnen im AR
    bookmarksnumbered=true, % Kapitel-Nummerierung im Inhaltsverzeichniss anzeigen
    bookmarksopenlevel=1,   % Tiefe der geöffneten Gliederung für den AR
    pdfstartview=FitV,       % Fit, FitH=breite, FitV=hoehe, FitBH
    pdfpagemode=UseOutlines, % FullScreen, UseNone, UseOutlines, UseThumbs 
}
\includexmp{Arnold_Mueller_2017_Permalinks}
\pdfinfo{
  /Author (Eckhart Arnold / Stefan Müller)
  /Title (Wie permanent sind Permalinks?)
  /Subject (Diskussion des Begriffs und korrekten Einsatzes von Permalinks bei der dauerhaften Bereitstellung elektronischer Publikationen im Internet)
  /Keywords (Permalinks, Zitierbarkeit von Internetquellen,
  Digitale Identifikatoren, Elektronisches Publizieren, DOI)
}
\fi

\begin{document} 

\title{Wie permanent sind Permalinks?}

\author{Eckhart Arnold / Stefan Müller}

\date{1. Januar 2017}

\maketitle

\sloppy

\begin{center}
{\em erschienen in: Informationspraxis, Bd. 3, Nr. 1, 2017, \href{http://journals.ub.uni-heidelberg.de/index.php/ip/article/view/33483}{http://journals.ub.uni-heidelberg.de/index.php/ip/article/view/33483}}
\end{center}

\begin{abstract}
  \singlespacing

In diesem Artikel versuchen wir einige Missverständnisse aufzuklären,
die das Konzept von "`Permalinks"' betreffen. Wir erörtern zunächst,
was Permalinks sind und welchem Zweck sie dienen. Unserer Ansicht nach
sind Permalinks weniger als eine technische Kategorie zu verstehen
denn als ein Versprechen der bereitstellenden Institution oder Person,
die als Permalinks ausgewiesenen Links künftig nicht zu löschen oder
zu ändern.

Nach einer begrifflichen Abgrenzung von anderen Linkarten
diskutieren wir das Verhältnis von Permalinks zu anderen digitalen
Identifikatoren, insbesondere DOIs.  Wir stellen keinen wesentlichen
Vorteil von DOIs gegenüber Permalinks fest. Anschließend diskutieren
wir den Einsatz von Permalinks in der Praxis: Erst Permalinks machen
Internetquellen zitierbar. Dennoch kann es in bestimmten Fällen
unerlässlich sein, Internetquellen auch ohne Permalink zu
verwenden. Permalinks auf lebende Dokumente und Datenbankten sollten
immer auf einen bestimmten zeitlichen Zustand des Objekts verweisen
und im Falle von Datenbanken auch auf die konkrete Suchanfrage
(anstelle nur des einzelnen Datensatzes oder, umgekehrt, der Datenbank
im Ganzen). Bei der Abwicklung von wissenschaftlichen Institutionen
ist darauf zu achten, dass im Rahmen des Abwicklungsplans auch für die
geregelte Übertragung der Permalinks an einen neuen Besitzer Sorge
getragen wird.

Wir schließen mit einem konkreten Praxisbeispiel, dem
Aufbau des Publikationsservers der Bayerischen Akademie der
Wissenschaften. Dabei haben wir gelernt, dass BV-Nummern nicht für
Permalinks zu verwenden sind, weil sie -\/- z.B. im Rahmen der Doublettenkorrektur -\/-
stillschweigend gelöscht werden können.

\begin{flushleft}
  {\bf Schlüsselwörter}: Permalink, Zitieren von Internetquellen,
  Digitale Identifikatoren, Elektronisches Publizieren, DOI
\end{flushleft}


{\bf English abstract:} In this essay we try to clarify certain
misunderstandings concerning the concept of ``permalinks''. We start
by explaining what permalinks are and what their purpose is. In our
opinion permalinks should not primarily be understood as a technical
category but as a promise of the supplying institution or person, not
delete or change the links that are marked as permalinks in the
future.

After a conceptual distinction from other link types, we discuss the
relation of permalinks to other digital identifiers, in particular
DOIs. We find that DOIs do not have any advantages over
permalinks. Then, we discuss the practical usage of permalinks:
Permalinks make internet sources quotable in the first place. None the
less it is in some cases unavoidable to use internet sources that do
not have a permalink. Permalinks for living documantes or databases
should always refer to particular state of the reffered object in
time. In the case of databases they should furthermore point to a
particular query or its result - rather than just a particular set of
data or the database as a whole. When shutting down a scientific
institution, care should be taken to ensure an orderly transfer of its
permalinks to a new owner.

We close with a practical example, concerning the buildup of a
publication server of the Bavarian Academy of Sciences. We have
learned that BV-numbers (i.e. the identificators of the combined
Bavarian library cataloge) cannot be used as part of a permalink,
because they may silently be removed -- for example in the course of
dublet-correction.

\begin{flushleft}
  {\bf Keywords}: permalinks, quotation of internet sources,
  digital identificator, electronic publishing, DOI
\end{flushleft}

\end{abstract}

\newpage

\tableofcontents

\onehalfspacing

\section{Danksagung}

Dieser Beitrag ist aus einem Blogpost
(\href{https://dhmuc.hypotheses.org/1179}{dhmuc.\-hypotheses\-.org/\-1179})
entstanden, der für diese Veröffentlichung gründlich
überarbeitet und stark ergänzt wurde, die Anregungen aus der
Diskussion auf dem Blog aufgreift und die Kritik der Gutachter der
Zeitschrift Informationspraxis berücksichtigt. Wir möchten den
Teilnehmern der Diskussion und den Gutachtern des Artikels an dieser
Stelle ausdrücklich für ihre wertvollen Hinweise danken.

\section{Das Wichtigste in Kürze}\label{das-wichtigste-in-kuerze}

\begin{itemize}
\item Permalinks sind dauerhafte Netzadressen von digitalen
  Ressourcen. Dabei sichert der Bereitsteller der digitalen Ressource
  zu, dass unter der zum Permalink erklärten Netzadresse
  dieselbe digitale Ressource auf unabsehbare Zeit verfügbar bleibt.

\item Permalinks erkennt man daran, dass der Bereitsteller ihre
  Permanenz durch öffentlich zugängliche Permalinkleitlinien
  ausdrücklich zusichert.

  Ein Beispiel für Permalinkleitlinien ist:
  \href{https://de.wikipedia.org/w/index.php?title=Hilfe:Permanentlink\&oldid=152277048}{de.\-wikipedia.\-org/\-w/\-index.php?\-title=\-Hilfe:\-Permanentlink\&\-oldid=\-152277048}. Hier
  heißt es: "`Dieser Verweis bleibt statisch auf stets dieselbe
  Version gerichtet, selbst wenn die Seite anschließend verändert wird
  und damit neue Versionen erzeugt werden."'

  Ein Beispiel für einen darunter fallenden Permalink:
  \href{https://de.wikipedia.org/w/index.php?title=Lola_Montez\&oldid=154537216}{de.\-wikipedia.\-org/w\-/\-index.php?\-title=\-Lola\_Montez\&oldid=\-154537216}.

  Kein Permalink ist dagegen die Adresse:
  \href{https://de.wikipedia.org/wiki/Lola_Montez}{de.\-wikipedia.\-org/\-wiki/\-Lola\_Montez},
  denn unter einer solchen Adresse verfügbare Artikel können sich mit der Zeit
  durch Überarbeitungen völlig verändern oder sogar verschwinden. -\/- Dieses Beispiel zeigt die
  Bedeutung der ausdrücklichen Erklärung zu den
  Permanenzeigenschaften. Ohne sie würde man gerade im ersten Link
  nicht den Permalink vermuten, weil er ein Implementierungsdetail
  enthält: "`.php"'. Das ist ungünstig, wenn die Seite einmal nicht
  mehr mit PHP, sondern einer anderen Programmiersprache läuft.

  Ein neueres Beispiel für Permalinkleitlinien und eine vorbildliche Versionierung
  findet sich unter \href{http://www.kit.gwi.uni-muenchen.de/?page\_id=2707}{kit.gwi.\-uni-muenchen.\-de/\-?page\_id=\-2707} (von 2016).

\item Für die Zitierbarkeit genügt ein Permalink. Digitale
  Identifikatoren, wie z.B. eine DOI-Nummer, sind als Alternative
  ebenfalls möglich, aber nicht zwingend erforderlich, da sie keine
  größere Permanenz der Ressource garantieren können als ein
  Permalink.

\item Bei der Angabe einer URL in einem Literaturverweis sollte immer
  der Permalink angegeben werden.
  Idealerweise gibt der Bereitsteller ein Erscheinungsdatum für die Ressource an
  -- wie bei einer Druckveröffentlichung. Dieses Erscheinungsdatum sollte
  ebenfalls angegeben werden. Falls keines verfügbar ist, kann als Notbehelf
  das Zugriffsdatum angegeben werden.

\item Vorsicht: Nicht alles, was als Permalink bezeichnet wird, ist in
  dem eben beschriebenen Sinne permanent. Entscheidend ist, welche
  Permanenz-Eigenschaften in den Permalinkleitlinien ausdrücklich
  zugesichert werden (und wie vertrauenswürdig die dahinter stehende
  Institution ist).

  Beispiel: Die in den Einträgen auf
  \href{http://gateway-bayern.de/}{gateway-bayern.\-de/}
  als solche bezeichneten Permalinks haben nur schwache
  Permanenzeigenschaften. Sie können im Rahmen der Überarbeitung des
  Katalogs spurlos verschwinden.
\end{itemize}


\section{Einleitung}
\label{einleitung}

Die Zitierbarkeit digitaler Publikationen hängt wesentlich davon ab,
dass sie über dauerhafte Netzadressen, sogenannte "`Permalinks"',
verfügen. Dauerhafte Netzadressen sind erforderlich, um die
Nachvollziehbarkeit einer wissenschaftlichen Argumentation
sicherzustellen, die sich auf digital vorliegende Referenzen stützt.
Nachvollziehbar ist eine sich auf Zitate und Referenzen stützende
Argumentation nur dann, wenn der Rezipient oder die Rezipientin die
referierte Sekundärliteratur, Quellen oder Forschungsdaten
wiederfinden und einsehen kann. Bei Druckwerken dienen dazu seit jeher
die bibliographischen Angaben, die es im Zusammenhang mit einem
ausgebauten Bibliothekssystem ermöglichen, die referierte Literatur zu
beschaffen. Bei elektronischen Publikationen ermöglicht dies die
dauerhafte Netzadresse, sprich der "`Permalink"'. Sind Permalinks
vorhanden, dann können digitale Referenzen alle ihre Vorteile
ausspielen, insbesondere den, dass sie -- zumindest bei Open Access
Publikationen -- unmittelbar verfügbar sind ("`nur einen Mausklick weit
entfernt"'). Das kann die Nachvollziehbarkeit und Überprüfbarkeit
einer Argumentation in der wissenschaftlichen Praxis erheblich
vereinfachen und beschleunigen.

Aber woher bekommen Wissenschaftlerinnen und Wissenschaftler, die
digitale Publikationen zitieren möchten, einen Permalink auf die
digitale Publikation? Und welchen Praxis-Regeln sollten
Informationsdienste wie Bibliotheken, Archive, Repositorien und
Digital-Humanities-Datenzentren folgen, die Permalinks auf die von ihnen
verwalteten Publikationen oder Daten anbieten möchten? Wie das bei einem
noch recht jungen Begriff wie dem des Permalinks, noch dazu aus dem sich
rasch entwickelnden und ständigem Wandel unterworfenen Bereich der
Digitaltechnik, vielleicht nicht anders zu erwarten ist, wird der
Begriff uneinheitlich verwendet. Daraus können sich Unsicherheiten und
im schlimmsten Fall auch eine falsche Handhabung von Permalinks ergeben
-- ganz zu schweigen von den Vorbehalten, die besonders in den
Geisteswissenschaften bezüglich der Zitierbarkeit von Internetquellen
zuweilen immer noch anzutreffen sind.

Im Folgenden möchten wir daher einige Überlegungen dazu anstellen, was
Permalinks sind, d.h. insbesondere, woher sie ihre Permanenzeigenschaft
beziehen und über welche weiteren Eigenschaften sie verfügen müssen,
um ihre Zitierbarkeit in wissenschaftlichen Kontexten zu
ermöglichen. Den Begriff Permalink verstehen wir dabei so, dass es
sich immer um eine URL handelt, auch wenn wir gelegentlich
Querverbindungen zu anderen Arten von permanenten Identifikatoren
ziehen. Wir grenzen den Begriff gegen andere Linktypen und atypische,
d.h. für den wissenschaftlichen Kontext untaugliche, Verwendungsweisen
ab. Weiterhin berichten wir über eigene Erfahrungen, die wir an der
Bayerischen Akademie der Wissenschaften mit der Einführung eines
Permalinksystems beim Aufbau unseres Publikationsservers gemacht
haben.

Bei diesem Artikel handelt es sich um eine überarbeitete und
erweiterte Fassung eines Blogbeitrags auf
\href{https://dhmuc.hypotheses.org/1179}{dhmuc.\-hypotheses.\-org/\-1179},
der die Anregungen und die Kritik aus der Diskussion aufgreift. Wir
möchten allen, die sich an der Diskussion auf dem Blog beteiligt
haben, an dieser Stelle für ihre konstruktiven und hilfreichen
Beiträge danken.

\section{Was sind Permalinks?}
\label{was-sind-permalinks}

\subsection{Worauf die Permanenzeigenschaft von Permalinks
  beruht}
\label{worauf-die-permanenzeigenschaft-von-permalinks-beruht}

Die Permanenz eines Permalinks beruht einzig und allein auf der
erkennbaren und im besten Falle öffentlich erklärten
Selbstverpflichtung der Institution oder Person, die die Links
ausgibt, unter ein- und demselben Permalink auch künftig stets
dasselbe digitale Dokument bereit zu stellen, und der konsequenten
Einhaltung dieser Selbstverpflichtung.

Es ist wichtig, sich klar zu machen, dass die Permanenzeigenschaft
eine Frage der Leitlinien der bereitstellenden Institution ist und
kein technisches Merkmal eines Links, so wie etwa die einer URL
zugeordnete IP-Adresse als ein technisches Merkmal aufgefasst werden
könnte. Anders ist es aber auch gar nicht möglich, denn es existiert
kein technisches Verfahren, durch das man einen Internetlink dauerhaft
machen könnte.
%, so wie man z.B. die Echtheit eines digitalen Dokuments
% durch eine elektronische Signatur garantieren kann.
Technische Systeme wie z.B. "`Handle-Server"' können die
Bereitstellung von Permalinks nur unterstützen, aber nicht
garantieren, dass deren Betreiber die Links beibehält. Das bedeutet
aber auch, dass die Nutzerinnen und Nutzer eines Permalinks der
bereitstellenden Institution vertrauen müssen und diese Institution
ihrerseits hinreichend vertrauenswürdig sein muss.

Da die Existenz und Dauerhaftigkeit von Permalinks von der
bereitstellenden Institution abhängt, können auch Permalinks
"`verwaisen"': wenn die Institution aufhört zu existieren und das
Weiterbestehen der Permalinks nicht im Abwicklungsprozess der
Institution sichergestellt wird. Das sollte aber kein Hinderungsgrund
für die wissenschaftliche Nutzung von Permalinks und insbesondere das
Zitieren elektronischer Dokumente unter Angabe des Permalinks sein.
Durch die richtige Handhabung von Permalinks seitens der
bereitstellenden Institutionen lässt sich dieses Restrisiko nämlich
hinreichend eingrenzen.

\label{wesentliche-eigenschaften-von-permalinks}
Aus dem Erfordernis, für die Referenzierung wissenschaftlicher Quellen
und Sekundärliteratur verwendbar zu sein, ergibt sich eine Reihe von
Eigenschaften, über die Permalinks mindestens verfügen sollten, um
ihren Zweck zu erfüllen. Diese Eigenschaften können zugleich als
Kriterien verstanden werden, an denen sich die Permalinkleitlinien
eines Informationsdienste-Anbieters messen lassen müssen.

\subsection{Permalinks sollten als Permalinks öffentlich
  gekennzeichnet sein}
\label{permalinks-sollten-als-permalinks-oeffentlich-gekennzeichnet-sein}

Damit diejenigen, die elektronische Dokumente zitieren möchten,
wissen, dass es sich um einen Permalink handelt, muss die
bereitstellende Institution in irgendeiner Form deutlich machen, dass
und welche Links sie laut Selbstverpflichtung nicht mehr ändern
wird. Zitierbar ist ein Permalink nur dann, wenn man einigermaßen
sicher sein kann, dass die als permanent erklärte URL auch in Zukunft
erreichbar ist und auf denselben Inhalt verweist.

In der Praxis existieren dabei Grauzonen, weil die
Permalinkleitlinien nicht immer hinreichend deutlich erkennbar
gemacht werden oder weil der Ausdruck "`Permalink"' mehrdeutig und
manchmal nur in einem eingeschränkteren technischen Sinn verwendet
wird, der wissenschaftlichen Ansprüchen nicht genügt. So haben wir
z.B. beim Internet Archive (\href{https://archive.org/}{archive.\-org})
keine deutlichen Permalinkleitlinien gefunden,\footnote{Nur etwas
  versteckte Hinweise darauf wie "`Whenever you refer to files at
  archive.org, use our permalink-style form for the file within the
  item. It is of the form
  http://archive.org/download/IDENTIFIER/FILE"' auf
  \href{https://archive.org/help/video.php}{archive.org/\-help/\-video.php}}
obwohl wir davon ausgehen, dass die Links, die archive.org für
archivierte Websites vergibt, hinreichend vertrauenswürdig sind, um
zitiert zu werden, da diese Institution sonst ihren erklärten
Zweck\footnote{Siehe
  \href{https://archive.org/about/}{archive.org/about} : ``Its
  purposes include offering permanent access [...] to historical
  collections that exist in digital format.''}  verfehlen würde.

Auch ein Link, der in einer Zitierempfehlung angegeben ist, sollte
unserer Ansicht nach ein Permalink sein, wie es z.B. beim Deutschen
Textarchiv der Fall ist.\footnote{Siehe z.B. die Zitationshilfe unten
  auf der Seite auf:
  \href{http://www.deutschestextarchiv.de/book/show/fontane\_stechlin\_1899}{www.\-deutschestextarchiv.\-de/\-book/\-show/\-fontane\_stechlin\_1899}}
Da man sich bislang allerdings nicht darauf verlassen
kann,\footnote{Vgl. Klaus Graf (2015).} ist es besser, wenn Permalinks
bis auf weiteres ausdrücklich als solche gekennzeichnet werden.

\subsection{Ein Permalink muss immer zu demselben Ziel führen}
\label{ein-permalink-muss-immer-zu-demselben-ziel-fuehren}

Im Sinne der Zitierbarkeit ist zu verhindern, dass jemand, der einem
Permalink in einer wissenschaftlichen Arbeit folgt, eine (für die
vorliegende Frage entscheidend) andere Fassung des Dokumentes sieht
als jene, die vom Autor der Arbeit verwendet wurde. Im Falle von
Ergänzungen oder Berichtigungen eines Dokumentes sind folgende
(einander nicht ausschließende) Lösungen denkbar:

\begin{itemize}
\item Neben der Angabe des Permalinks wird deutlich festgehalten, dass
  und in welchem Umfang sich das Verweisziel ändern kann: Welche
  Änderungen können stillschweigend geschehen? Bei welchen wird
  dagegen eine der folgenden Maßnahmen ergriffen?

\item Änderungen am Ziel werden so durchgeführt, dass man eine
  Änderung als solche erkennen, den Zeitpunkt der Änderung erfahren
  und möglichst auch den Ausgangszustand erschließen kann. Zum
  Beispiel, indem Ergänzungen oder Berichtigungen eines Regests unter
  dem ursprünglichen Text erscheinen, datiert und als Zusätze
  ausgezeichnet sind. Im besten Fall ist diese Datierung und Auszeichnung
  auch maschinenlesbar hinterlegt. Dieser Maßnahme entspricht in der
  Druckwelt die Beigabe von Addenda und Corrigenda zu einem sonst
  bereits gesetzten Werk. Wichtig ist, der gesamten Veröffentlichung
  ein Erscheinungsdatum beizugeben und bei Änderungen anzupassen.

\item Änderungen am Ziel führen zur Vergabe eines neuen
  Permalinks. Das heißt, alle Bearbeitungszustände werden vorgehalten
  und mit einem je eigenen Permalink versehen. Um die Permalinks
  verschiedener Auflagen zu unterscheiden, bietet sich eine Angabe des
  Ausgabedatums im Format ISO 8061 an,
  z.B. "`.../regesten/342/2016-02-03"'. Dieser Maßnahme entspricht in
  der Druckwelt das Veröffentlichen mehrerer Auflagen.
\end{itemize}

Es versteht sich von selbst, dass die älteren Versionen und die darauf
verweisenden Permalinks verfügbar bleiben müssen, will man nicht das
Vertrauen in die eigenen Permalinkleitlinien untergraben.

Besonders für Dokumente, die sich häufiger ändern, kann es sinnvoll
sein, auch Links anzubieten, die immer auf die jeweils neueste Version
eines Dokuments verweisen (sogenannte "`kanonische Links"', siehe
unten).  Aber diese Links sind dann keine Permalinks im strengen
Sinne. Ein gutes Beispiel für dieses zweigleisige Verfahren bietet
Wikipedia \href{https://de.wikipedia.org/}{de.\-wikipedia.\-org}.  Unter
dem Artikel-Link findet man immer die aktuellste Fassung eines
Artikels. Zum Zitieren verwendet man aber den Permalink, den man über
eine gesonderte Funktion im Werkzeugmenü aufrufen muss.

Es ist diskutabel, ob auch minimale Veränderungen (z.B. die Korrektur
einzelner Tippfehler) eines Dokuments immer einen neuen Permalink
erfordern. Ebenfalls diskussionswürdig ist, ob das Dokument unter ein-
und demselben Permanentlink Veränderungen des Erscheinungsbildes
erfahren darf oder nicht. Solange sich keine zitierrelevanten
Informationen (z.B. Seitenzahlen oder Absatznummerierungen) ändern,
könnte man das als unproblematisch ansehen.

Eine ähnliche Grauzone bilden Permalinks, die auf eine Vorschaltseite,
z.B. mit weiteren bibliographischen Informationen, verweisen, von der
man dann zu dem gemeinten digitalen Zielobjekt gelangt. Zumindest wenn
die -- am besten öffentlich zugänglichen -- Permalinkleitlinien
deutlich werden lassen, worauf genau sich die Permanenz bezieht,
könnten Änderungen an der Vorschaltseite als vertretbar angesehen
werden. Die Vorschaltseite wäre dann -- anders als der von ihr
beschriebene Gegenstand -- strenggenommen nicht unter dem Permalink
zitierbar.  Allerdings sollte sich die digitale Ressource, auf die die
Vorschaltseite verweist, nicht ändern.

Die Frage der Identität eines digitalen Dokuments ist insgesamt eine
einigermaßen komplexe Frage, da die Identitätsrelation sowohl von der
Art des Dokuments als auch dem (antizipierten) Verwendungszweck
abhängen kann. Bit-Identität als vermeintlich neutrales Identitätsmaß
dürfte in den meisten Fällen zu einschränkend sein, da sich die
Bit-Identität z.B. schon bei der Umwandlung von einer älteren auf
eine neuere PDF-Version ändern kann. Umgekehrt kann die Bit-Identität
nicht die Permanenz aller Eigenschaften garantieren, die für das
Zitieren wesentlich sein können. So würde der Verweis auf eine
bestimmte Zeile eines HTML-Dokuments bei anderem Umbruch (z.B. auf
einem kleineren Bildschirm) ungültig werden, obwohl die Bit-Identität
gegeben ist.

Darüber hinaus gibt es bestimmte
Eigenschaften digitaler Dokumente, die selbst jenseits der
Bit-Identität flüchtige Eigenschaften sind, wie z.B. der Zeilenumbruch
eines HTML-Dokuments.

Es ist auch denkbar, dass hier von unterschiedlichen Institutionen mit
jeweils voller subjektiver Berechtigung unterschiedliche Maßstäbe
angelegt werden: Eine Bibliothek, die wissenschaftliche
Sekundärliteratur wie z.B. Fachartikel in digitaler Form bereitstellt,
wird sich vermutlich keiner Kritik aussetzen, wenn sie das
Erscheinungsbild der im Internet bereit gestellten Artikel
stillschweigend an neue technische Gegebenheiten, etwa neue
Geräteklassen oder Bildschirmauflösungen, anpasst, ohne deshalb einen
neuen Permalink zu vergeben. Bei einem Archiv dagegen, das dieselben
Werke nun nicht mehr als Sekundärliteratur, sondern als Primärquellen
für die künftige historische Forschung speichert, könnte es schon eher
erwünscht sein, dass es auch das Erscheinungsbild der digitalen
Archivalien bewahrt. Explizite und öffentlich zugängliche
Permalinkleitlinien können hier auf jeden Fall Klarheit schaffen.

\subsection{Ein Permalink muss eindeutig sein}
\label{permalinks-muessen-mehr-eindeutig-aber-nicht-unbedingt-ein-eindeutig-sein}

Ein- und derselbe Permalink sollte immer auf ein- und dasselbe
Dokument verweisen. Aber umgekehrt schadet es nicht, wenn zu ein- und
demselben Dokument unterschiedliche Permalinks führen, denn die
Wiederauffindbarkeit referierter digitaler Dokumente wird dadurch
nicht eingeschränkt. Permalinks sind also nicht notwendigerweise
eineindeutig.

Mehrfache Permalinks auf die gleiche Ressource können auf ganz natürliche
Weise zustande kommen, wenn das gleiche Dokument bei
unterschiedlichen Institutionen gespeichert wird, die jeweils eigene
Permalinks ausgeben. Es kann auch dazu kommen, wenn Permalinks
"`geändert"' werden, z.B. um sie an ein neues Namensschema anzupassen.
Denn, wenn überhaupt, dann können Permalinks nur dadurch "`geändert"'
werden, dass ein neuer Permalink hinzukommt, da der alte Permalink zum
Erhalt der Permanenzeigenschaft unbedingt beibehalten werden
sollte. Man kann höchstens in Zukunft auf die weitere Verwendung und
Bekanntgabe des älteren Permalinks verzichten.

Für die Nutzerinnen und Nutzer von Permanent-URLs bedeutet das, dass
sie sich nur anhand der Permalinks nicht sicher sein können, ob zwei
verschiedene Permalinks auf unterschiedliche Dokumente oder auf ein-
und dasselbe Dokument verweisen. Für die ein-eindeutige
Identifizierung einer Publikation sind vielmehr die vollständigen
bibliographischen Angaben erforderlich (die man ggf. durch Aufrufen
eines der Permalinks auf das Dokument in Erfahrung bringen kann). Dass
auf die vollständigen bibliographischen Angaben (nicht nur aus diesem
Grund) in wissenschaftlichen Publikationen ohnehin nicht verzichtet
werden kann, bedeutet aber auch, dass es nicht erforderlich ist, dass
Permalinks in ihrer URL Hinweise auf den Urheber oder Herausgeber
codieren müssten, was manchmal als kritischer Einwand gegen die
Zitierbarkeit von Kurz-URLs angeführt wird. Die Verwendbarkeit von
Kurz-URLs wiederum hängt lediglich davon ab, ob auch der verwendete
URL-Kürzungsdienst ebenfalls die Permanenz der Kurz-URLs garantiert
(siehe unten).

Ob durch die Zwischenschaltung von Linkauflösungssystemen (wie
z.B. DOI oder PURL) eine relativ größere Eineindeutigkeit erreicht
werden kann, ist fraglich. Da es mehrere solcher Systeme gibt,
vervielfältigt sich die Mehrdeutigkeit nur auf einer anderen Ebene.
Zudem müssten die Linkauflöser Sorge tragen, dass keine Dubletten
vorkommen. Schon allein um die Fehlerkorrektur zu ermöglichen, wird
sich die Vergabe mehr-eindeutiger Permalinks nicht völlig
vermeiden lassen.

\section{Zur Vermeidung von Missverständnissen: Was sind keine
  Permalinks?}
\label{zur-vermeidung-von-missverstaendnissen-was-sind-keine-permalinks}

Nachdem wir eben erläutert haben, durch welche wesentlichen
Eigenschaften sich Permalinks auszeichnen müssen, wollen wir im
Folgenden Permalinks gegen andere Arten von Links und andere
Verwendungsweisen des Ausdrucks "`Permalink"' abgrenzen, mit denen sie
leicht verwechselt werden können.

\subsection{Direktlinks}\label{direktlinks}

Als Direktlinks verstehen wir Netzadressen (URLs), die unmittelbar auf
eine bestimmte Ressource, etwa einen Blogeintrag oder Datensatz,
innerhalb eines Blogs oder Content-Management-Systems oder einer
Webapplikation führen. "`Unmittelbar"' bedeutet hier ohne den
nochmaligen Umweg über die Einstiegsseite oder eine Suchmaske. Leider
werden Direktlinks manchmal auch als "`Permalinks"' bezeichnet, obwohl
sie das höchstens in einem sehr eingeschränkten technischen Sinne
sind. So generiert beispielsweise das Blogsystem "`Wordpress"' unter
der Bezeichnung "`Permalink"' Direktlinks auf die einzelnen
Blogbeiträge. Um echte Permalinks handelt es sich aber nur dann, wenn
diejenigen, die den Blog betreiben, die Links selbst und die
darunter veröffentlichten Beiträge nicht nachträglich ändern und sich
dazu in ihren Leitlinien verpflichten. Ein Blog- oder
Content-Management-System, das eine Permanentlink-Funktion anbietet,
stellt damit also lediglich Direktlinks als technische Voraussetzung
für echte Permalinks bereit.

Taucht der Begriff in der Dokumentation eines Blogsystems auf, so ist
das vielleicht auch nicht anders zu erwarten. Aber auch in anderen
Kontexten wird der Ausdruck "`Permalink"' manchmal lediglich als
Synonym für einen Direktlink verwendet. So sind beispielsweise die als
solche bezeichneten Permalinks im bayerischen Verbundkatalog wie
\href{http://gateway-bayern.de/BV003390811}{gateway-bayern.\-de/\-BV003390811}
keine Permalinks in unserem Sinne, denn ihre Permanenz ist nicht
garantiert. Vielmehr können sie im Rahmen der Konsolidierung des
Bibliothekskatalogs, namentlich der Zusammenführung von Dubletten,
verschwinden und damit ins Leere führen (dazu unten mehr).

\subsection{Kurzlinks}
\label{kurzlinks}

Relativ offensichtlich ist es, dass Permalinks von Kurzlinks zu
unterscheiden sind, wie sie von Diensten wie tinyurl.com, goo.gl oder
bit.ly bereit gestellt werden. Technisch gesehen beruhen Kurzlinks auf
ganz ähnlichen Mechanismen wie die Linkauflöser für Permalinks, aber
ihr Sinn und Zweck ist ein anderer. So sind Kurzlinks ihrer Kürze
wegen vor allem für Twitter- oder Handy-Nachrichten hilfreich. Zum
Zitieren eignen sich Kurz-URLs von kommerziellen Diensteanbietern eher
nicht, da -- selbst wenn sie eine Langlebigkeitsgarantie abgeben --
nicht sicher ist, wie lange der Dienst sich am Markt halten kann. Auch
Großunternehmen, bei denen ein schnelles Verschwinden vom Markt nicht
zu erwarten ist, neigen in dieser Branche dazu, als unrentabel
empfundene Geschäftsbereiche vom einen auf den anderen Tag aufzugeben.

Etwas Anderes wäre es, wenn wissenschaftliche Informationsdienstleister
eigene Kurzlink-Dienste mit Permalinkleitlinien anbieten. Dabei
müsste allerdings sichergestellt werden, dass die zu kürzenden
Netzadressen Permalinks repräsentieren. Am einfachsten lässt sich das
realisieren, wenn der Dienst nur für solche Links Kürzungen anbietet, die permanente URLs
innerhalb der eigenen Domäne der bereitstellenden Institution
sind.

Unabhängig von diesen Überlegungen ist es in jedem Fall
sinnvoll, die URLs von Permalinks eher kurz zu halten, damit sie nicht
die Zeilenlänge in einem Buch oder Aufsatz überschreiten und
bei Bedarf bequem abgetippt werden können.

\subsection{Semantische Links}
\label{semantische-links}

Permalinks sind nicht dasselbe wie semantische Links. Unter
"`semantischen Links"' versteht man aussagekräftige Netzadressen, die
in der URL bereits für Menschen sinnvolle und verständliche Angaben zu
der damit adressierten digitalen Ressource enthalten (z.B.
\href{https://de.wikipedia.org/wiki/Der_Schmied_von_Kochel}{de.\-wikipedia.\-org/\-wiki/\-Der\_Schmied\_von\_Kochel}).
Zu der Verwechselung kommt es deshalb so leicht, da semantische Links
sich, gerade weil sie so gut lesbar sind, zum Zitieren besonders
anzubieten scheinen. Es ist zwar möglich und auch sinnvoll, für
Permalinks aussagekräftige URLs vorzusehen. Im Zweifelsfall ist die
Permanenz aber wichtiger als die Aussagekräftigkeit -- und wird von
ihr leider weit öfter gefährdet, als zunächst scheinen mag. Beispiele
für Angaben, deren Verwendung in einem Permalink Schwierigkeiten
machen kann:

\begin{itemize}
\item Eine historische Datierung (z.B. einer Urkunde) kann sich im Laufe der
Forschung ändern.

\item Titel sind nicht sicher eindeutig; und es kann sich der für ein
titelloses historisches Werk eingeführte Titel ändern.

\item Ein entsprechender Fall: Der einem historischen Werk zugeschriebene
Autor oder der für ihn übliche Name kann schwanken (wie bei Johannes von Saaz / Tepl).

\item Personennamen-GNDs sind nicht sicher eindeutig. Personen-GNDs wiederum
gefährden die Beständigkeit insofern, als sie falsch zugeordnet sein
können und dann geändert werden müssten.

\item Lemmaansätze sind nicht sicher eindeutig (wegen möglicher Homonymie)
und müssen möglicherweise geändert werden (z.B. bei Vereinheitlichungen
in der Lautwiedergabe oder der Behandlung von Fugenelementen).
Vorzuziehen sind eigene Kennungen, die man seinen Lemmaansätzen
zuordnet.

\item Sogar Bibliothekssignaturen können sich ändern oder verschwinden.
Verwendet man die früheren Signaturen weiter, entsteht eine Mischung aus
semantischen und scheinbar semantischen Kennungen (veralteten
Signaturen). Noch ungünstiger ist es, wenn die Bibliothek eine veraltete
Signatur neu verwendet.
\end{itemize}

Grundsätzlich sind Angaben, die möglicherweise einmal geändert werden
müssten, kaum permalinkgeeignet, es sei denn, dass man bei solchen
Änderungen eine neue Auflage veröffentlicht und dieser geänderten
Auflage ohnehin einen neuen Permalink zuteilt. Man kann auch dulden,
dass ein Permalink veraltete Angaben enthält; allerdings ist er dann
nicht länger aussagekräftig, sondern sogar geradezu irreführend
geworden. Daher ist es unter Umständen vorzuziehen, statt
bedeutungstragender Angaben von vornherein völlig bedeutungslose und
daher nie veraltende IDs zu verwenden.

\subsection{Kanonische Links}
\label{kanonische-links}

Mindestens ebenso leicht kann es zur Verwechselung von Permalinks mit
kanonischen Links kommen. Ein kanonischer Link ist die im
Zweifelsfalle vorzuziehende URL eines elektronischen Dokuments. Die
anderen Fassungen bieten z.B. denselben Inhalt an (nur unter einer
anderen URL) oder denselben Hauptinhalt mit z.B. einem Menü in anderer
Sprache.\footnote{Für Suchmaschinen sind als kanonisch ausgezeichnete
  Links der Verweis auf eine Seite, die sie für ihre Indizierung der
  vorliegenden Seite vorziehen sollen. Vgl. M. Ohye (2012).}

Häufig, aber nicht immer, verweist ein kanonischer Link auf die
aktuellste Fassung einer elektronischen Publikation. (Manchmal ist die
aktuellste Fassung noch eine beta-Version oder eine
Diskussionsvorlage, die erst später finalisiert wird und deshalb
gerade nicht kanonisch ist.) Kanonische Links spielen vor allem bei
Nachschlagewerken eine Rolle, deren Beiträge mit der Zeit überarbeitet
werden und bei denen die Nutzer so gut wie immer an der aktuellsten
Fassung interessiert sind.

Außer der schon erwähnten Wikipedia liefert die Stanford Encyclopedia
of Philosophy ein gutes Beispiel. Unter dem kanonischen Link (z.B.:
\href{http://plato.stanford.edu/entries/aristotle/}{plato.stanford.edu/\-entries/\-aristotle/})
ist die jeweils aktuellste Fassung eines Artikels abrufbar. Für jeden
Artikel existiert aber auch ein Permalink (z.B.:
\href{http://plato.stanford.edu/archives/fall2015/entries/aristotle/}{plato.stanford.edu/\-archives/\-fall2015/\-entries/\-aristotle/}),
der durch die Datumsangabe in der URL bereits als solcher erkennbar
ist. Eine Zitationshilfe liefert die korrekten bibliographischen
Angaben mit Permalink. Die Permalinkleitlinien werden auf einer
Dokumentationsseite
(\href{http://plato.stanford.edu/cite.html}{plato.stanford.edu/\-cite.html})
in angemessen knapper Form (eine Bildschirmseite) erläutert.

Eine ebenfalls geradezu vorbildliche Permalinkerklärung bietet die
Herzog-August-Bibliothek Wolfenbüttel. Siehe
\href{http://www.hab.de/de/home/bibliothek/digitale-bibliothek-wdb/garantieerklaerung.html}{www.hab.de/\-de/\-home/\-bibliothek/\-digitale-bibliothek-wdb/\-garantieerklaerung.html}
.\footnote{Wir möchten Klaus Graf für den Hinweis auf dieses
  schöne Beispiel danken.}

Die klare Unterscheidung zwischen Permalinks und kanonischen Links
wird auch dadurch erschwert, dass der Ausdruck "`kanonischer Link"'
manchmal fälschlich im Sinne von "`Permalink"' verwendet
wird.\footnote{So bei Van de Sompel et al. (2016a, 2), wenn sie den
  Attributwert "`canonical"' für das "`rel"'-Attribut des
  Verknüpfungstags vorschlagen, obwohl sie eigentlich einen Permalink
  kennzeichnen wollen.}

\subsection{Andere permanente Identifikatoren}
\label{andere-permanente-identifikatoren}

Schließlich sollten Permalinks nicht mit anderen Arten von permanenten
Identifikatoren verwechselt werden, die nicht durch eine Netzadresse
dargestellt werden. Anders als bei den zuvor besprochenen Linktypen
ist die Verwechslung in diesem Fall aber weniger schädlich, da andere
permanente Identifikatoren weitgehend dieselben Funktionen erfüllen
können wie Permalinks. Insbesondere können sie ebenso wie Permalinks
mittels des Identifikators die Zitierbarkeit von elektronischen
Dokumenten ermöglichen -- sofern sich dem Identifikator jederzeit ohne
große Mühe (also z.B. bereits durch Eingabe des Identifikators in eine
Suchmaschine) eine Netzadresse zuordnen lässt. Darin sehen wir einen,
wenn auch nur geringfügigen Nachteil von permanenten Identifikatoren,
die keine Permalinks sind. Ein Permalink kann immer so, wie er ist, in
die Adresszeile eines Internetbrowsers eingegeben werden und führt
dann zu der verknüpften Ressource. Ein permanenter Identifikator wie
z.B. eine DOI (falls sie nicht als URL mit doi.org angegeben wird)
erfordert einen Webservice, der einem zu dem Identifikator eine
Netzadresse liefert oder auch gleich auf die dem Identifikator
zugeordnete Netzseite weiterleitet. Inzwischen gilt es aber als gute
Praxis DOIs als URL, d.h. mit vorangestelltem ``doi.org'' anzugeben,
z.B. als ``http://doi.org/z9d''.

\section{Exkurs: Sind permanente Identifikatoren (z.B. DOI) eine bessere
  Alternative?}
\label{sind-permanente-identifikatoren-z.b.-doi-eine-bessere-alternative}

{\em Kurzfassung: Bevor Ihr Geld für DOIs ausgebt, backt Euch lieber 
Eure eigenen Permalinks. Es funktioniert genauso gut und kostet nichts!}

Interessanterweise erfreuen sich einige der eben beschriebenen
alternativen Identifikatoren, insbesondere die DOI-Kennungen
("`Digital Object Identifier"'), nach unseren Erfahrungen eines
gewissen Prestiges, so, als würde ein digitales Dokument erst durch
die Verleihung einer DOI zu einer respektablen elektronischen
Publikation. Diese Prestigezuschreibung ist unserer Ansicht nach in
der Sache nicht begründet und wohl eher durch die Analogie zur ISBN
von Druckpublikationen zu erklären, deren Prestige als vermeintliches
Mindestqualitätsmerkmal einer wissenschaftlichen Buchpublikationen
allerdings auf ähnlichen Missverständnissen beruht. Denn
Linkauflösersysteme (DOI, PURL, etc.) sind im Prinzip nichts anderes
als Verzeichnisse, in denen einem digitalen Objekt-Identifikator (wie
einer DOI) ein Link auf das Objekt zugeordnet wird. Sie garantieren
nicht und können auch nicht garantieren, dass dieser Link gültig
bleibt. Dies hängt wiederum allein von der Institution ab, die die
Seite hinter dem Link bereitgestellt hat.
%% Der Inhalt des hier folgenden Absatzes passt besser zum inzwischen
%% ergänzten Teil unten, wo sich auch ein entsprechender findet ("Weiterhin sind DOIs "...)
Anders verhält es sich mit dem verdienstvollen Angebot von archive.org,
Netzseiten nicht allein mit einem Permalink zu versehen, sondern auch
bei sich zu archivieren und damit selbst für die Verfügbarkeit der Ressource
zu sorgen.

Der Nutzen von Linkauflösersystem besteht darin, dass die Adresse des
Zielobjekts stillschweigend geändert werden kann und der Link immer
noch funktioniert. Sofern man annimmt, dass das Linkauflösersystem von
größerer Dauerhaftigkeit ist als die Permalinks der bereitstellenden
Institutionen, bietet das einen Vorteil. Das könnte z.B. dann der Fall
sein, wenn die Permalinks der bereitstellenden Institution einen
Domain-Namen enthalten, der durch Verkauf oder Abtretung an eine
andere Institution übergeht und dann nicht mehr verwendet werden
darf. Es ist aber auch nur dann der Fall, wenn der neue digitale Ort
des Zielobjekts an das Linkauflösersystem zurückgemeldet
wird. Entsprechende Leitlinien dürften sich aber nur schwer durch den
Betreiber des Linkauflösersystems erzwingen lassen, weil gerade der
häufigste Fall, in dem dies nötig wäre, die Auflösung einer
Institution ist. Es hängt nicht vom Betreiber des Linkauflösers,
sondern von dessen Kunden ab, ob durch die Entkoppelungsfunktion eine
größere Permanenz der (entkoppelten) digitalen Identifikatoren
gegenüber selbsterstellten Permalinks der Kunden möglich
ist. Möglicherweise wäre der Permanenz eher gedient, wenn man von
vornherein versuchen würde, Repositorien möglichst bei solchen
Institutionen anzusiedeln, denen man eine große Dauerhaftigkeit
zutrauen kann, also etwa Staatsarchive, Nationalbibliotheken, aber
auch solidere Webarchive wie das Internet Archive
(\href{https://archive.org}{archive.org}) oder CLOCKSS
(\href{https://clocks.org}{clocks.org}).\footnote{Wir sind Klaus Graf
  für den Hinweis auf diesen wichtigen Punkt dankbar.}

Wir haben daher Zweifel, ob die digitalen Identifikatoren einer
zentralen Organisation wie der DOI-Foundation tatsächlich eine größere
Permanenz verbürgen können als Permalinks. Theoretisch
wird die Permanenz durch die Einführung eines zusätzlichen
Zwischenschrittes zunächst einmal geschwächt, da sie nun von
mindestens zwei Institutionen, dem Linkauflöser und dem Bereitsteller
der digitalen Ressource, abhängig ist. Sollte dieser Nachteil durch
die Entkoppelungsfunktion hinreichend kompensiert werden können, dann
würden wir, schon um Bürokratie und Abhängigkeiten zu vermeiden, ihre
Realisierung durch eine Form von peer-to-peer-System
bevorzugen.\footnote{So strebt das beispielsweise auch die "`Permanent
  Identifier Community Group"' unter dem Dach des
  World-Wide-Web-Konsortiums an
  \href{http://www.w3.org/community/perma-id/2013/02/16/launch/}{www.w3.org/\-community/\-perma-id/\-2013/\-02/\-16/\-launch/}.}

Ein weiterer Grund ist, dass das (geschäftliche) Eigeninteresse
entsprechender Institutionen nicht immer mit den Interessen der Nutzer
harmonisieren muss.\footnote{Vgl. zu Kosten und Geschäftsmodellen etwa
  den Abschnitt über die "`Registration Agencies"' im DOI-Handbuch
  (DOI-Foundation 2016). Vgl. auch den Clarin-Report zu "`Persistent
  and Unique Identifiers"' von 2008 (Broeder et al. 2008), der das
  durchaus kritisch sieht.} Die Situation scheint sich zwar
etwas gebessert zu haben, seit Registrierungsagenturen nicht mehr unbedingt
eine Gebühr pro vergebener DOI verlangen. Trotzdem erscheinen uns Passagen, wie
wir sie in der Policy der Registrierungsagentur für Wirtschafts- und
Sozialdaten da\textbar ra finden, nicht gerade vertrauenerweckend:

\begin{quote}
  Der DOI-Bezug wird von da\textbar ra für akademische Einrichtungen
  kostenneutral angeboten. Es kann später notwendig werden, bei
  da\textbar ra anfallende Kosten für den DOI-Bezug an die
  Publikationsagenten weiterzugeben. (da\textbar ra Policy Version
  3.0, URL:
  \href{http://www.da-ra.de/de/ueber-uns/da-ra-policy/}{www.da-ra.de/\-de/\-ueber-uns/\-da-ra-policy/})
\end{quote}

Und selbst wenn eine Registrierungsagentur ihren Service für einige
ihrer Kunden, etwa akademische Einrichtungen, kostenfrei anbietet:
Irgendwer, und sei es am Ende der Steuerzahler, wird die Kosten tragen
müssen. Wenn unsere Argumentationslinie stimmt, dass der bürokratische
Aufwand im Falle der DOIs für die bloße Bereitstellung permanenter
Identifikatoren unnötig groß ausfällt, dann zahlt derjenige, der die
Kosten trägt, zu viel.

Dieser Ansicht sind wir auch deshalb, weil wir -- anders als z.B.
Hausstein und Grunow (2012, 48) -- Zweifel daran haben, dass durch
verbindliche Verträge zwischen einer DOI-Registrierungsagentur und
ihren Klienten, d.h. den Anbietern von digitalen Publikationsdiensten
oder Repositorien, eine größere Langlebigkeit erzielt werden kann.
Denn einerseits liegt es ja schon im Eigeninteresse der
Klienten die Langlebigkeit der Permalinks sicherzustellen, unter denen
sie ihr Material anbieten. Wozu bedarf es aber noch eines Vertrages,
wenn gar kein Interessenkonflikt entstehen kann?

Zum anderen würden Verträge gerade in den Situationen kein wirksames
Mittel mehr darstellen, in denen die Bereitstellung permanenter
Netzadressen trotz des Eigeninteresses scheitert, also etwa bei der
zuvor erwähnten institutionellen Auflösung des Klienten, oder falls
technische Inkompetenz auf Seiten des Klienten dazu führt, dass die
hinterlegten Netzadressen ihre Gültigkeit verlieren. Da die Erzeugung
permanenter Netzadressen nur einen sehr geringen technischen Aufwand
verursacht, erwarten wir auch nicht, dass ``daran zuerst gespart''
wird, sollte ein Publikationsdienstanbieter wirtschaftlich unter Druck
geraten, so dass wir unter diesem Gesichtspunkt ebenfalls kaum eine
Notwendigkeit für die explizite vertragliche Regelung mit einer
Registrierungsagentur sehen.

Abgesehen davon könnte sich der Publikationsdienstanbieter statt
gegenüber einer Registrierungsagentur auch ebensogut gegenüber seinen
Kunden vertraglich zur Bereitstellung dauerhafter Netzadressen oder
Identifikatoren verpflichten. Das würde, soweit es um eine rechtliche
Sicherstellung der Dauerhaftigkeit von Identifikatoren geht, die
Registrierungsagentur erübrigen. Auch unter administrativen
Gesichtspunkten sehen wir daher keine prinzipiellen Vorteile von DOIs.

Trotzdem kann man beobachten, dass DOIs sich mehr und mehr
durchsetzen.  Wir glauben, dass es dafür mehrere Gründe gibt, und
räumen ein, dass diese Gründe tatsächlich für DOIs sprechen, wenn
auch im Wesentlichen nur als Folge einer selbsterfüllenden
Prophezeiung.

Der wichtigste Grund dürfte der sein, dass bei DOIs
inzwischen der Netzwerkeffekt greift, da sie als permanente digitale
Identifikatoren mittlerweile schon so bekannt sind, dass sie manchmal
geradezu als Synonym dafür verwendet werden. Eine Folge dieser
Bekanntheit ist, dass in vielen bibliographischen Datensätzen bereits
ein Feld für die DOI vorgesehen ist, während es meist kein explizites
Feld für einen Permalink gibt, sondern lediglich ein URL-Feld, von dem
zweifelhaft bleibt, ob es ausschließlich für Permalinks zu verwenden
ist. (Siehe dazu den Abschnitt zur Kenntlichmachung von Permalinks
weiter unten.)

Weiterhin sind DOIs -- nicht zuletzt auch wegen ihrer Bekanntheit --
als solche sofort erkennbar, während man Permalinks nicht unbedingt
ansieht, dass sie als permanente Links gedacht sind. Werden nicht schon in der URL
entsprechende symbolische Merkmale (z.B. ``purl'' oder das Wort
``permanent'') eingebaut sind, dann gibt ja erst die
Permalinkerklärung des Herausgebers darüber Auskunft.

Schließlich hat
der Abschluss eines Vertrages mit einer Registrierungsagentur selbst
dann, wenn man ihn -- wie wir -- für verzichtbar hält, die Wirkung eines
Rituals, das allen Beteiligten den Sinn der Sache wieder ins Gedächnis
ruft bzw. sie zuallererst darauf aufmerksam macht. Wir denken jedoch,
dass man alle diese Vorzüge, da sie nicht auf intrinsischen
Eigenschaften von DOIs beruhen, auch billiger haben könnte.

In jedem Fall sind wir der Überzeugung, dass Identifikatoren wie DOI,
PURL nicht mehr leisten als jeder andere Permalink und daher unserer
Ansicht nach auch nicht als besser oder zitierfähiger beurteilt werden
sollten. Dazu abschließend ein Zitat von Tim Berners-Lee (dem Vater
des WWW):

"`This is {[}\ldots{}{]} probably one of the worst side-effects of the
URN discussions. Some seem to think that because there is research
about namespaces which will be more persistent, that they can be as
lax about dangling links as they like as "URNs will fix all that". If
you are one of these folks, then allow me to disillusion you.

Most URN schemes I have seen look something like an authority ID
followed by either a date and a string you choose, or just a string
you choose. This looks very like an HTTP URI. In other words, if you
think your organization will be capable of creating URNs which will
last, then prove it by doing it now and using them for your HTTP
URIs. There is nothing about HTTP which makes your URIs unstable. It
is your organization. Make a database which maps document URN to
current filename, and let the web server use that to actually retrieve
files."'  (Berners-Lee 1998)

\section{Permalinks in der Praxis: Worauf zu achten ist}
\label{permalinks-in-der-praxis-worauf-zu-achten-ist}

\subsection{Permalinks und Zitierbarkeit}
\label{permalinks-und-zitierbarkeit}

Die Frage der Zitierbarkeit von digitalen Ressourcen hängt von ihrer
Identifizierbarkeit und Wiederauffindbarkeit ab. Permalinks sind ein
Mittel, um digitale Ressourcen zu identifizieren und wiederzufinden,
und damit den Nachvollzug einer wissenschaftlichen Argumentation, die
sich auf andere Quellen stützt, nachzuvollziehen. Weitere Mittel sind
aber zumindest denkbar.\footnote{Zur Frage, wie man Online-Quellen
  zitiert, vgl. die ausführliche Zusammenstellung von Graf
  (2011).}  Insofern können digitale Ressourcen auch ohne Permalink
als zitierbar gelten, wenn die im zitierenden Werk gegebenen Angaben
dazu ausreichen, um sie ohne zu großen Aufwand zu identifizieren und
wieder zu finden.  Ist aber ein Permalink zu einer digitalen Ressource
vorhanden, dann wäre es im Sinne einer wissenschaftlich sorgfältigen
Zitierweise eine Nachlässigkeit, keinen Permalink oder einen anderen
permanenten Identifikator anzugeben.

Ein Permalink sollte mit Erscheinungsdatum zitiert werden.
Das erfordert, dass der Bereitsteller entweder Permalinks anbietet,
die das Erscheinungsdatum schon enthalten (am besten im Format ISO 8061,
z.B. ``http://example.org/\-id8738/\-2016-04-06''), oder ein Erscheinungsdatum
angibt und bei etwaigen Änderungen auch anpasst.

Ein Notbehelf ist demgegenüber die noch weithin anzutreffende Praxis,
beim Verweis auf Netzseiten ein Zugriffsdatum anzugeben. Denn es lässt
keinen Schluss darauf zu, ob die damals abgerufene Fassung dieselbe
ist wie die nun von einem Leser der Zitation abgerufene. Zudem: Wenn eine
digitale Ressource nicht mehr existiert, dann hilft auch der Hinweis auf einen
in der Vergangenheit möglichen Zugriff Lesern nicht weiter.

Eine aufwändigere, zur Sicherung zitierter Fassungen aber auch viel wertvollere
Maßnahme ist das Anlegen eigener Kopien von
Internetquellen. Zumindest, wenn es um wissenschaftliche Primärquellen
geht, kann es manchmal unvermeidlich sein, flüchtige Internetquellen
zu zitieren. Das wäre z.B. bei einer Forschungsarbeit über die
Bewertung der Flüchtingskrise in der Blogosphäre der Fall, denn nicht
jedes Internetblog bietet Permalinks an. Darüber hinaus erfüllt die
Permalinkfunktion gängiger Blogsysteme nicht automatisch schon
wissenschaftliche Ansprüche an die Persistenz von Inhalten -- von der
Frage ganz zu schweigen, wie sehr man darauf vertrauen kann, dass
Blogeinträge oder Beiträge zu Diskussionsforen nicht nachträglich
verändert werden. In solchen Fällen ist es nach wie vor hilfreich, ein
eigenes Forschungsarchiv anzulegen, in dem man Kopien der nicht mit
(vertrauenswürdigem) Permalink versehenen digitalen Ressourcen
sammelt, oder auf einen Archiv-Anbieter wie
z.B. \href{http://archive.org}{archive.org} oder
\href{http://webcitation.org}{webcitation.org}\footnote{Wir möchten
  Klaus Graf für den Hinweis auf WebCite (URL:
  \href{http://webcitation.org}{webcitation.org}) danken.}
zurückzugreifen. Wie oben bereits erwähnt, ist es bei letzteren leider
nicht immer leicht, sich über die Permalinkleitlinien und relevante
Details der technischen Umsetzung (z.B., inwieweit Unterseiten und
verlinkte Inhalte mitarchiviert werden) Klarheit zu
verschaffen. Sollte der Bedarf groß genug sein, erscheint es auch
denkbar, dass die Informationsdienste wissenschaftlicher Institutionen
in Zukunft entsprechende Archivierungsservices für digitale
Quellensammlungen als Cloud-Service anzubieten oder -- vielleicht noch
besser -- schon existierende Anbieter zu unterstützen.

Wie weiter oben beschrieben wurde, kann es je nach Striktheit der
Permalinkleitlinien vorkommen, dass unter einem Permalink
erreichbare digitale Ressourcen immer noch bestimmte flüchtige
Eigenschaften aufweisen können. Für die Zitierbarkeit hängt es dann
davon ab, ob die wissenschaftliche Argumentation, in deren
Zusammenhang auf die digitalen Ressource verwiesen wird, sich
maßgeblich auf diese flüchtigen Eigenschaften stützt. Für diesen
Ausnahmefall bietet sich als Ausweg ebenfalls die Eigenarchivierung
an.

Ebenfalls relevant für die Zitierbarkeit von digitalen ebenso wie
analogen Primärquellen ist die Frage ihrer Echtheit. Bei digitalen
Quellen verschärft sich die Frage insofern noch, als digitale Quellen
sich im Prinzip sehr leicht nachträglich ändern lassen. Zumindest bei
sensiblem Material, etwa archivierten Regierungsakten, könnte es daher
in Zukunft notwendig werden, die Permalinks mit den verlinkten
Dokumenten kryptographisch abzusichern, um die Vertrauenswürdigkeit zu
ermöglichen.

Von der Zitierbarkeit im weiteren Sinne ist die Zitierfähigkeit im
engeren Sinne zu unterscheiden. Die Zitierfähigkeit bezieht sich nur
auf wissenschaftliche Sekundärliteratur. Gemeint ist damit, dass die
eigene wissenschaftliche Argumentation sich nicht auf
Forschungsergebnisse aus dubiosen Quellen stützen sollte. Als
zitierfähig gilt daher in der Regel nur diejenige Sekundärliteratur,
die einen den wissenschaftlichen Gepflogenheiten entsprechenden
Qualitätssicherungsprozess wie z.B. das bei Fachjournalen übliche
Begutachtungsverfahren ("`peer review"') durchlaufen hat. Die
Zitierbarkeit im Sinne der Identifizierbarkeit und
Wiederauffindbarkeit ist eine notwendige, aber keine hinreichende
Bedingung für die Zitierfähigkeit. Insofern sollten digitale
Publikationen, um zitierfähig zu sein, am Besten über einen Permalink
verfügen. Zitierfähig als Sekundärliteratur wird die verlinkte
Publikation aber nur durch ihre wissenschaftliche Qualität, nicht
durch das Vorhandensein eines Permalinks.

\subsection{Permalinks auf Datenbanken und lebende Dokumente}
\label{permalinks-auf-datenbanken-und-lebende-dokumente}

Während gedruckte Publikationen nach den Fahnenkorrekturen finalisiert
sind, besteht bei digitalen Dokumenten und Daten die Möglichkeit, dass
sie -- solange sie verwendet werden -- auch kontinuierlich
geändert werden. Man kann sogar soweit gehen zu behaupten, dass gerade
die Auflösung der scharfen Trennung zwischen Erstellungs- und
Nutzungsphase neue Möglichkeiten für den wissenschaftlichen
Forschungsprozess eröffnet. Neben der von Hubertus Kohle prägnant
beschriebenen "`publish first -- filter later"'-Strategie wären hier
vor allem Wissenschaftsblogs zu nennen, die nicht nur den Austausch
und die Wissenschaftkommunikation im Vorfeld späterer
Fachveröffentlichungen verändern, sondern sich -- wie das
Polymath-Projekt (Gowers et al.  2016; Boulton 2012) vor Augen führt
-- tatsächlich auch schon als Mittel der massiv kollaborativen
Forschung bei der Lösung schwieriger wissenschaftlicher Probleme
bewährt haben, die im Rahmen der traditionellen Organisationsform der
Wissenschaft in derselben Zeit kaum hätten gelöst werden können.

Für digitale Dokumente, die nach ihrer Veröffentlichung noch verändert
werden, hat sich die Bezeichnung "`lebende Dokumente"'
eingebürgert. Wenn lebende Dokumente -- wie das Beispiel des
Polymath-Projekts zeigt -- bei der Genese wissenschaftlicher
Erkenntnis eine tragende Rolle spielen können, dann ist es wichtig,
sie zitierbar zu gestalten. Ein wissenschaftspolitisches Eigentor wäre
es dagegen, wenn man umgekehrt lebenden Dokumenten wegen ihres
veränderlichen Charakters die Zitierfähigkeit kategorisch absprechen
wollte. Für die Zitierbarkeit lebender Dokumente ist es erforderlich,
Permalinks bereit zu stellen, die auf den Zustand eines lebenden
Dokuments zu einem bestimmten Zeitpunkt verweisen. Eine Erläuterung,
wie das technisch realisiert werden kann, würde an dieser Stelle zu
weit führen. Wir möchten deshalb nur darauf hinweisen, dass dafür in
Form von versionierenden Datenbanken, die die Geschichte der
Änderungen an der Datenbank mitspeichern, und den aus der
Softwareentwicklung bekannten Versionsverwaltungssystemen wohlbewährte
Lösungen existieren.

Ähnliches gilt für die zunehmend wichtiger werdenden
Forschungsdatenbanken. In Zukunft werden auch in den
Geisteswissenschaften selbst die Forschungsergebnisse nicht mehr nur
Textform haben, sondern teilweise auch in Form von Daten vorliegen.
Damit wird es aber auch erforderlich, die entsprechenden Datenbanken
zitierbar zu gestalten. Da Zitierungen eine der Währungen sind, in denen
wissenschaftliche Anerkennung ausgetauscht wird, besteht ein sehr
erwünschter Nebeneffekt darin, dass auf diese Weise auch die Arbeit der
Recherche, Sammlung und Erarbeitung von wissenschaftlichen
Datenbeständen als wissenschaftliche Leistung gewürdigt und ihren
Urhebern zugeschrieben werden kann.

Die Bereitstellung von Permalinks für Datenbanken ist eine etwas
komplexere Angelegenheit als die für lebende Dokumente, da hier nicht
nur der zeitliche Verlauf zu berücksichtigen ist, sondern auch, dass
sich die für eine wissenschaftliche Argumentation relevante
"`Datenbankstelle"' (in Analogie zur zitierten Textstelle) in der
Regel aus bestimmten Datenbankabfragen ergibt. Ein Permalink auf die
gesamte Datenbank wäre zu grobkörnig, ein Link auf einen einzelnen
Datensatz zu feinkörnig, da in einer Abfrage ja in der Regel Daten aus
mehreren Datensätzen (bei relationalen Datenbanken: mehreren Zeilen
aus unterschiedlichen Tabellen) zusammengefasst werden. Inzwischen
existieren aber auch hierfür durchdachte Konzepte, wie man die
Zitierbarkeit von Datenbanken gestalten kann (Rauber et al. 2015;
Rauber et al. 2016). So ist es im Prinzip möglich, Permalinks für
konkrete Datenbankabfragen dynamisch zu erzeugen. Die technische
Voraussetzung dafür ist, dass sich eine Abfrage über den Zustand der
Datenbank zu einem bestimmten Zeitpunkt jederzeit rekonstruieren
lässt. Das ist mit versionierenden Datenbanken leicht möglich, sofern
neben der Datenbank noch ein Verzeichnis der Abfragen und der ihnen
zugeordneten Permalinks geführt wird.

Die Frage der "`Granularität"' von Permalinks stellt sich nicht nur
bei Datenbanken, sondern bereits bei Texten. Bei Texten kann es jedoch
notfalls genügen, wenn der Permalink auf ein Dokument von vielen
hundert Seiten verweist und die Seitenzahl lediglich im Zitat genannt
wird -- wobei eine Verknüpfung, die direkt auf eine bestimmte Seite
führt, natürlich die Bequemlichkeit erhöht. Bei Datenbanken erscheint
es dagegen unpraktikabel, die konkrete Abfrage, die die Gestalt von
vielen Zeilen Programmcode in einer Abfragesprache wie SQL haben kann,
im Text eines wissenschaftlichen Aufsatzes anzuführen. Durch die
Verwendung von Permalinks auf Datenbank{\em abfragen}, wie von Rauber
et al. (2016) vorgeschlagen, erübrigt sich das, da sich das Ergebnis
der zitierten Datenbankabfrage durch den Link wiederherstellen lässt.

Des Unterschiedes zwischen Permalinks auf elektronische Dokumente und
Permalinks auf Datenbankabfragen sollte man sich bewusst sein. Während
bei der Zitation von Permalinks auf Dokumente die bibliographischen
Angaben das Dokument beschreiben und Permalinks lediglich den Ort des
Dokuments im Internet angeben, ist bei Permalinks auf Datenbanksuchen
nach dem eben beschriebenen Konzept die Suchanfrage und damit die
Beschreibung des referenzierten Objekts im Permalink versteckt. Ein
zerbrochener Permalink würde es daher schwerer machen, die Referenz zu
rekonstruieren als im Falle eines elektronischen Dokuments. Die Praxis
muss zeigen, ob das zu Problemen führt. Möglicherweise wird sich dann
die Konvention herausbilden, die konkreten Suchanfragen im Anhang
zu dokumentieren. Anhänge mit reichem Zusatzmaterial werden bei
elektronisch publizierten Artikeln ohnehin zunehmend üblich.

\subsection{Die Abwicklung von Permalinks}
\label{die-abwicklung-von-permalinks}

Die eigentlich etwas paradox erscheinende Abwicklung von Permalinks
kann in zwei Fällen erforderlich werden. Der erste Fall ist der, dass
das digitale Objekt, auf das der Permalink verweist, einen
Lebenszyklus hat und dessen Ende erreicht ist. Ein Beispiel dafür wäre
ein Permalink auf einen Katalogeintrag, der im Rahmen einer
Dublettenkorrektur gelöscht wird. In diesem Fall könnte die Abwicklung
darin bestehen, dass der Permalink des gelöschten Eintrags nun auch
auf den nicht gelöschten verweist, insoweit die Dublette in zwei
Einträgen bestand, die beide dasselbe Buch verzeichnen. Wenn aber ein
Katalogeintrag als falsch herauskorrigiert werden muss, könnte der
Permalink auf einen Nachfolger oder Stellvertreter führen, z.B. auf
eine Seite, die die Korrektur transparent macht.

Der zweite Fall ist derjenige, bei dem eine wissenschaftliche
Institution, die Permalinks ausgegeben hat, aufhört zu existieren. In
diesem Fall sollte ein neuer Besitzer für die Permalinks gesucht
werden. Ein Problem stellt dann noch der in den Permalinks enthaltene
Domänenname dar. (Das ist, wie oben beschrieben, unserer Ansicht nach
der einzige Nachteil von Permalinks gegenüber anderen Arten von
digitalen Identifikatoren, die Identifikator und Netzadresse
entkoppeln.) Sofern der Domänenname frei wird, kann er vom neuen
Besitzer übernommen und weitergeführt werden. Wichtig ist nur, dass
dies im Rahmen der Abwicklung auch mitberücksichtigt wird.

Nur wenn der Domänenname an einen neuen Besitzer übergeht, der nicht
bereit ist, die Permalinks weiterzuführen bzw. eine Weiterleitung
einzurichten, ist das Zerbrechen der Permalinks bei der Abwicklung
einer wissenschaftlichen Einrichtung unvermeidlich. Wir gehen davon
aus, dass dieser Fall nur selten eintreten wird, denn in der Regel
dürfte der neue Domänenbesitzer, schon um lästige Anfragen zu
vermeiden, über eine Weiterleitung mit sich reden lassen. Ausschließen
lässt sich dieser Fall aber nicht.

\subsection{\texorpdfstring{Zwei offene Fragen:\\
Vorschaltseiten und Maschinenlesbarkeit von Permalinks}{Zwei
    offene Fragen: Vorschaltseiten und Maschinenlesbarkeit von
    Permalinks}}
\label{zwei-offene-fragen-vorschaltseiten-und-maschinenlesbarkeit-von-permalinks}

\subsubsection{Sollen Permalinks auf Vorschaltseiten oder direkt
  auf Dokumente verweisen?}
\label{sollen-permalinks-auf-vorschaltseiten-oder-direkt-auf-dokumente-verweisen}

Eine öfters diskutierte Frage ist die, ob Permalinks besser auf
Vorschaltseiten oder auf das gemeinte digitale Objekt selbst verweisen
sollten.\footnote{Auf die weitergehenden Implikationen dieses Themas,
  insbesondere auch, was den automatischen Abruf von Dokumenten in
  großer Menge angeht ("`harvesting"') können wir hier nicht eingehen.
  Vgl. dazu Van de Sompel et al. (2016b).} Wir haben den Eindruck,
dass es zunehmend üblich wird, Permalinks nicht direkt auf die
verlinkten digitalen Objekte verweisen zu lassen, sondern auf eine
Vorschaltseite, die Informationen (sprich "`Metadaten"') über das
Objekt und den Link auf das eigentliche Objekt enthält oder mehrere
Links auf unterschiedliche Repräsentationen des digitalen Objekts
z.B. in unterschiedlichen Dateiformaten. Vorschaltseiten sind für die
Nutzerinnen und Nutzer zunächst ein Nachteil, weil sie einen weiteren
Klick und die Suche nach dem endgültigen Link erfordern. Aber die
Vorteile, unterschiedliche Dateiformate, Ausprägungen
(z.B. hochauflösend oder komprimiert) anbieten oder auch Metadaten
oder Lizenzinformationen kenntlich machen zu können, wiegen in unseren
Augen die Nachteile auf. Hinzu kommt: Führt ein Permalink unmittelbar
auf ein PDF, so fehlt das für eine weitere Recherche vielleicht
erwünschte Menü der das PDF bereitstellenden Seite.  Besonders bei
retrodigitalisiertem Material erscheint uns überdies die Einbettung
einer Vorschaltseite in die Dateien selbst als noch störender. Im
Übrigen kann man der inzwischen schon häufig anzutreffenden Konvention
folgen, dass der Permalink keine Dateiendung enthält und auf eine
Vorschaltseite verweist, während der gleiche Link mit Dateiendung direkt
zum digitalen Objekt führt.\footnote{So handhabt es zum Beispiel der
  Publikationsserver der Bayerischen Akademie der Wissenschaften:
  \href{https://publikationen.badw.de}{publikationen.badw.de}}

\subsubsection{Maschinenlesbarkeit von Permalinks}
\label{maschinenlesbarkeit-von-permalinks}

Maschinenlesbare Metadaten sind ein Thema für sich, das hier nicht
ausführlich zu besprechen ist. Wir beschränken uns auf die Frage: Wie
kann ein Programm einen Permalink als solchen erkennen? Bisher gibt es
dafür leider keine standardisierte Lösung. Dabei ließe sich leicht
eine denken. So könnte man z.B. ein Signalwort wie
"`permalink"' vereinbaren, das als Wert des rel-Attributes von
Link-Elementen dienen könnte.\footnote{Die Idee kommt von Van de
  Sompel et al. (2016a, 2). Unglücklicherweise schlagen sie den
  Attributwert "`canonical"' vor. Dieser Wert wird aber schon
  verwendet, und zwar um einen kanonischen Link in dem Sinne zu
  bezeichnen, der oben unter "`Kanonische Links"' erläutert
  wurde. Vgl.  M. Ohye (2012)} Auf jener Seite selbst, zu der der
Permalink führen soll, könnte man den Permalink dann einfach mit
\texttt{<link href="...Permalink..."\ rel="permalink\dq />} im Kopf der
Seite vermerken (wobei für "`...Permalink..."' der Permalink zu schreiben wäre).

Auf anderen Seiten, von denen der Permalink zur Zielseite des
Permalinks führt, müsste man deutlich machen, dass es nicht die Seite
ist, auf der er steht, sondern die, zu der er führt. Dafür müsste man
das Subjekt der Permalink-Aussage ausdrücklich machen, z.B. im Rahmen
von RDFa mit dem about-Attribut:

\begin{quote}
\texttt{<a
  about="...Permalink..."\ href="...Permalink..."\ rel="permalink"\ ...>}
\end{quote}

(Im Rahmen von RDFa nähme man übrigens statt \texttt{rel} besser
\texttt{property}, worauf hier nicht weiter einzugehen ist.)

Unseres Wissens fehlt es in den großen Vokabularen wie
\href{http://schema.org/docs/full.html}{schema.org/\-docs/\-full.html},
worin Signalwörter für die maschinelle Verarbeitung vereinbart sind,
noch an einem Ausdruck für "`Permalink"'.

Auch in den verbreiteten Definitionen bibliographischer Metadaten gibt
es bisher nur einen Kennungstyp für Links (z.B. die Kennung "`URL"' im
Falle von BibLaTeX) und keine Möglichkeit, permanente und flüchtige
Links zu unterscheiden. Insgesamt bleibt die maschinenlesbare
Kennzeichnung von Permalinks noch ein Desiderat.

\section{Ein Praxis-Beispiel: Warum BV-Nummern nicht für Permalinks
  taugen}
\label{ein-praxis-beispiel-warum-bv-nummern-nicht-fuer-permalinks-taugen}

Nach den bis hierher weitgehend theoretischen Überlegungen möchten wir
nun noch kurz einige praktische Erfahrungen mitteilen, die wir mit der
Bereitstellung von Permalinks gemacht haben. Beim Aufbau des
Publikationsservers der Bayerischen Akademie der Wissenschaften
bestand unser erster Ansatz darin, für die Permalinks Kennungen zu
benutzen, die identisch mit den BV-Nummern des BV-Katalogs
sind. Mehrere Gründe waren dafür ausschlaggebend:

\begin{enumerate}
\item Alle Werke der Bayerischen Akademie der Wissenschaften werden im
  BV-Katalog eingetragen. Also hat jedes Werk, das auf dem
  Publikationsserver erscheinen kann, eine BV-Nummer.

\item Wir benutzen diese BV-Nummer ohnehin schon, um dem BV-Katalog
  die bibliographischen Angaben für Publikationen auf dem
  Publikationsserver zu entnehmen. Denn die Einträge im BV-Katalog
  wurden von bibliothekarischen Fachkräften vorgenommen, was eine
  bedeutende Qualitätssicherung darstellt, und können über die vom
  Katalog dankenswerterweise bereitgestellte SRU-Schnittstelle
  automatisch entnommen werden, so dass die Angaben nur einmal
  menschliche Arbeitskraft erfordern.

\item Wir gingen davon aus, dass eine einmal vergebene BV-Nummer auf
  unabsehbare Zeit zu einer Beschreibung der Publikation führt, für
  die sie vergeben wurde.

\item Durch die Nutzung von BV-Nummern wären unsere Permalinks
  indirekt in einen größeren Kontext eingeordnet gewesen, nämlich den
  des BV-Katalogs und hätten, indem sie die Katalognummer enthalten,
  zugleich eine gewisse Semantik gehabt.
\end{enumerate}

Aus zwei Gründen ließ sich dieses Vorhaben nicht ganz durchhalten:

\begin{enumerate}
\item Die BV-Nummern sind primär Identifikatoren für die
  Datenbankeinträge im BV-Katalog, nicht für die in den
  Datenbankeinträgen beschriebenen Werke. Wird nun ein
  Datenbankeintrag auf einen anderen umgelenkt -\/- was nötig sein
  kann, um Dubletten zusammenzuführen -\/-, dann verschwindet die
  BV-Nummer des umgelenkten Eintrags und mit ihr auch der Permalink
  des BV-Kataloges. (Dies nach einer brieflichen Auskunft von M.
  Kratzer von der BSB, für die wir hier herzlich danken.) Zum Beispiel
  hatte das Werk BV020097138 (siehe
  \href{http://gateway-bayern.de/BV020097138}{gateway-bayern.de/\-BV020097138})
  früher auch die BV-Nummer BV003899554, die der Katalog jetzt nicht
  mehr kennt; weitere Beispiele für verschwundene BV-Nummern sind
  BV023335674 und BV023335670.

\item Gelegentlich ist es nur für ausgebildete Bibliothekare und
  Bibliothekarinnen sicher erkennbar, welchem Katalogeintrag und damit
  welcher BV-Nummer eine Publikation zuzuordnen ist. Das gilt
  besonders im Fall von Dubletten, die der BV-Katalog schon deswegen
  nicht so selten enthält, weil er aus mehreren, früher unabhängigen
  Katalogen zusammengeführt wurde. Die Situation, wie sie ist, führt
  dazu, dass ein Rückstau von Publikationen entsteht, die bloß deshalb
  nicht auf dem Publikationsserver veröffentlicht werden können, weil
  ihre BV-Nummern noch nicht kontrolliert und gegebenenfalls
  korrigiert sind.
\end{enumerate}

Wir haben daraus gelernt und generieren für den Publikationsserver der
Akademie (\href{http://publikationen.badw.de/}{publikationen.badw.de})
inzwischen unsere eigenen Permalinks, deren Permanenz bis auf
Mehr-Eindeutigkeit (siehe oben) von uns garantiert wird.


\section{Schlussbemerkung}
\label{schlussbemerkung}

Die vorhergehende Diskussion von Permalinks könnte auf Grund ihrer
Ausführlichkeit den Eindruck erwecken, dass Permalinks eine
komplizierte Angelegenheit sind. Das ist nicht unsere
Ansicht. Vielmehr glauben wir, dass sich Permalinks, wenn man ihren
Zweck richtig verstanden hat, sehr einfach handhaben lassen --
sowohl was deren Bereitstellung als auch deren Nutzung angeht. Das
Wesentliche dazu haben wir am Anfang in ein paar Stichpunkten
zusammengefasst.

Die Ausführlichkeit der Diskussion liegt darin begründet, dass es auf
Grund der Neuheit des Konzepts wie bei so vielen Begriffen der digitalen
Welt leicht zu Missverständnissen kommen kann, die sich -- wie uns
unsere eigenen Erfahrungen mit den BV-Nummern gezeigt haben -- auch
durchaus in der Praxis auswirken können. Dieser Unsicherheiten wegen
lohnt sich eine ausführliche Diskussion.

Hinzu kommt, dass mit neuen Publikationsformen wie lebenden Dokumenten
und Datenbanken als Publikationen Permalinks eine zunehmende Bedeutung
im Kontext der "`reproduzierbaren Wissenschaft"' gewinnen. Dabei
erfahren Permalinks auch eine Funktionserweiterung. Sie dienen nicht
mehr nur als Ortsangabe für ein elektronisches Dokument, sondern als
Stellenangabe innerhalb eines digitalen Objekts. Umso größer dürften
in Zukunft daher auch die Anforderungen an die Verlässlichkeit von
Permalinks werden. Wir sind aber zuversichtlich, dass sich im Laufe
der Zeit und mit zunehmender Sensibilisierung für das Thema auf Seiten
der Anbieter Usancen heraus bilden, die dazu führen, dass man --
anders als in den Anfangstagen des Internets -- dauerhafte von
flüchtigen Links leichter unterscheiden und sich auf die
Dauerhaftigkeit von Permalinks zunehmend verlassen kann.

\section{Autoren}

Eckhart Arnold, Bayerische Akademie der Wissenschaften, Referat für IT
/ Digital Humanities, Alfons-Goppel-Str. 11, 80539 München, Email: arnold@badw.de, Web: \href{https://eckhartarnold.de}{eckhartarnold.de}\\[0.25cm]

Stefan Müller, Bayerische Akademie der Wissenschaften, Referat für IT
/ Digital Humanities, Alfons-Goppel-Str. 11, 80539 München, Email: mueller@badw.de

% \section{Quellen}\label{quellen}

\begin{thebibliography}{99}

\bibitem{berners-lee:1998}Berners-Lee, Tim (1998): Cool URIs don't
  change, {[}online{]}, verfügbar unter:
  \href{https://www.w3.org/Provider/Style/URI.html}{www.w3.org/\-Provider/\-Style/\-URI.html}

\bibitem{boulton:2012}Boulton, Geoffrey (Hrsg.) (2012): Final report -
  Science as an open enterprise, hrsg. Von der Royal Society,
  {[}online{]}, verfügbar unter:
  \href{https://royalsociety.org/topics-policy/projects/science-public-enterprise/report/}{royalsociety.org/\-topics-policy/\-projects/\-science-public-enterprise/\-report/}

\bibitem{broeder-et-al:2008}Broeder, Daan / Dreyer, Malte / Kemps-Snijders, Marc / Witt, Andreas /
Kupietz, Marc / Wittenburg, Peter (2008): Persistent and Unique
Identifiers, {[}online{]}, verfügbar unter:
\href{https://www.clarin.eu/sites/default/files/wg2-2-pid-doc-v4.pdf}{www.clarin.eu/\-sites/\-default/\-files/\-wg2-2-pid-doc-v4.pdf}

\bibitem{doi-foundation:2016}DOI-Foundation (2016): DOI Handbook,
  {[}online{]}, verfügbar unter:
  \href{http://www.doi.org/doi_handbook/8_Registration_Agencies.html}{www.doi.org/\-doi\_handbook/\-8\_Registration\_Agencies.html}

\bibitem{gowers-et-al:2016}Gowers, Tim / Kalai, Gil / Nielsen, Michael
  / Tao, Terry (2016): The polymath blog, {[}online{]}, verfügbar
  unter: \href{https://polymathprojects.org/}{polymathprojects.org/}

\bibitem{graf:2015}Graf, Klaus: ``Bitte ändern Sie Ihr Lesezeichen''. Wie die Uni Frankfurt Open Access beschädigt, Archivalia Blog, 22. Mai 2015, URL: \href{https://archivalia.hypotheses.org/1666}{archivalia.hypotheses.org/\-1666}

\bibitem{graf:2011}Graf, Klaus (2011): Wie zitiere ich \ldots{}
  Online-Quellen?, {[}online{]}, verfügbar unter:
  \href{https://archivalia.hypotheses.org/13353}{archivalia.hypotheses.org/\-13353}

\bibitem{hausstein-grunow:2012}: Hausstein, Brigitte; Grunow, Stefanie
  (2012) : DOI Registrierung sozial- und wirtschaftswissenschaftlicher
  Forschungsdaten mit dara, In: Vernetztes Wissen – Daten, Menschen,
  Systeme. 6. Konferenz der Zentralbibliothek, Forschungszentrum
  Jülich, 5. - 7.  November 2012. Proceedingsband, S. 45-59, URL:
  \href{http://hdl.handle.net/11108/81}{hdl.handle.net/\-11108/\-81}

\bibitem{Ohye-kupke:2012}Ohye, Maile / Kupke, Joachim (2012): The Canonical Link Relation,
{[}online{]}, verfügbar unter:
\href{https://tools.ietf.org/html/rfc6596}{tools.ietf.org/\-html/\-rfc6596}

\bibitem{rauber-asmi:2016}Rauber, Andreas / Asmi, Ari (2016): Identification of Reproducible
Subsets for Data Citation, Sharing and Re-Use. Draft, {[}online{]},
verfügbar unter:
\href{https://rd-alliance.org/system/files/documents/RDA-Guidelines_TCDL_draft.pdf}{rd-alliance.org/\-system/\-files/\-documents/\-RDA-Guidelines\_TCDL\_draft.pdf}

\bibitem{rauber-asmi:2015}Rauber, Andreas / Asmi, Ari / van Uytvanck, Dieter / Pröll, Stefan
(2015): Data Citation of Evolving Data. Recommendations of the Working
Group on Data Citation - Revision Oct. 20 2015, {[}online{]}, verfügbar
unter:
\href{https://rd-alliance.org/recommendations-working-group-data-citation-revision-oct-20-2015.html}{rd-alliance.org/\-recommendations-working-group-data-citation-revision-oct-20-2015.html}

\bibitem{van-de-sompel:2016a}Van de Sompel, Herbert / Rosenthal, David S. H. / Nelson, Michael
L.(2016a): Web Infrastructure to Support e-Journal Preservation (and
More), in: arXiv:1605.06154 {[}cs{]}, {[}online{]}, verfügbar unter:
\href{http://arxiv.org/abs/1605.06154}{arxiv.org/\-abs/\-1605.06154}

\bibitem{van-de-sompel:2016b}Van de Sompel, Herbert / Klein, Martin / Jones, Shawn M. (2016b):
Persistent URIs Must Be Used To Be Persistent, in: arXiv:1602.09102
{[}cs{]}, {[}online{]}, verfügbar unter:
\href{http://arxiv.org/abs/1602.09102}{arxiv.org/\-abs/\-1602.09102}

\bibitem{dara:2014} da\textbar ra Policy (Version 3.0), Registrierungsagentur
  für Wirtschafts- und Sozialdaten, 2014, URL: \href{http://www.da-ra.de/de/ueber-uns/da-ra-policy/}{www.da-ra.de/\-de/\-ueber-uns/\-da-ra-policy/}

\end{thebibliography}

\end{document}
