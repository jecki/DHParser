
%%% Local Variables:
%%% mode: latex
%%% TeX-master: t
%%% End:

\documentclass[12pt,a4paper,ngerman]{article}
\usepackage{microtype}
\usepackage{ae}
\usepackage[german,ngerman]{babel}
\usepackage{natbib}
%\usepackage[latin1]{inputenc}
\usepackage[utf8x]{inputenc}
\usepackage{ucs}
\usepackage[T1]{fontenc}
\usepackage{t1enc}
\usepackage{type1cm}
%\usepackage{graphicx}

\usepackage{ifpdf}
\ifpdf
\usepackage{xmpincl}
\usepackage[pdftex]{hyperref}
\hypersetup{
    colorlinks,
    citecolor=black,
    filecolor=black,
    linkcolor=black,
    urlcolor=black,
    bookmarksopen=true,     % Gliederung öffnen im AR
    bookmarksnumbered=true, % Kapitel-Nummerierung im Inhaltsverzeichniss anzeigen
    bookmarksopenlevel=1,   % Tiefe der geöffneten Gliederung für den AR
    pdfstartview=FitV,       % Fit, FitH=breite, FitV=hoehe, FitBH
    pdfpagemode=UseOutlines, % FullScreen, UseNone, UseOutlines, UseThumbs 
}
\includexmp{Adorno}
\pdfinfo{
  /Author (Eckhart Arnold)
  /Title (Aufklärungskritik als metaphysische Denunziation. Über den Begriff der Aufklärung bei Horkheimer und Adorno)
  /Subject (Eine eingehende Kritik der "Dialektik der Aufklärung" von Theodor W. Adorno und Max Horkheimer)
  /Keywords (Adorno, Horkheimer, Dialektik der Aufklärung, Instrumentelle Vernunft)
}
\fi

\sloppy

\begin{document}
\selectlanguage{ngerman}
%\linespread{1.3}

\title{Aufklärungskritik als metaphysische Denunziation.\\ 
       Über den Begriff der Aufklärung bei Horkheimer und Adorno}

\author{Eckhart Arnold}
\date{July 27th 2005}
%\maketitle

\begin{titlepage}

\begin{center}
\noindent
{\bf Philosophische Fakultät der Heinrich-Heine-Universität Düsseldorf}

\vspace{5cm}

\noindent
{\Large Aufklärungskritik als metaphysische Denunziation.\\
Über den Begriff der Aufklärung bei Horkheimer und Adorno}

\vspace{7cm}
\end{center}

\begin{flushleft}

\noindent
Hausarbeit zum\\
Hauptseminar: Mythos, Ritus, Religion\\
Sommer 2005\\
Leitung: Prof. Dr. Simone Dietz

\vspace{1cm}
\noindent
vorgelegt von:\\
Eckhart Arnold

\vspace{1cm}
\noindent
Düsseldorf, 4. August 2005
\end{flushleft}

\end{titlepage}

\fontsize{12}{18}
\selectfont

\pagenumbering{roman}
\tableofcontents

\newpage

\setcounter{page}{1}
\pagenumbering{arabic}

\section{Einleitung}

Die "`Dialektik der Aufklärung"' von Max Horkheimer und Theodor Adorno ist
zweifellos ein sehr vielschichtiges Werk. Nimmt man die von den Autoren der
"`Dialektik der Aufklärung"' im Vorwort geäußerte Zielsetzung ernst, nach der
sie sich die Erkenntnis "`vorgesetzt hatten ... , warum die Menschheit anstatt
in einen wahrhaft menschlichen Zustand einzutreten, in eine neue Art von
Barbarei versinkt"' \cite[S. 1]{adorno-horkheimer:1947}, dann könnte man
erwarten, es handele sich bei dem Buch um eine wissenschaftliche Analyse der
Ursachen der Entstehung von Faschismus und Kommunismus, eine Analyse, die
durch eine geschärfte Kultur- und Gesellschaftskritik ergänzt wird, welche
diejenigen Faktoren dingfest macht, die in den (noch) liberalen Gesellschaften
dem Übergang in den totalitären Staat vorarbeiten. In der Tat ist die
"`Dialektik der Aufklärung"' in diesem Sinne als Gesellschaftsanalyse und
-kritik verstanden worden, sonst hätte das Buch kaum eine der Bibeln der 68er
Bewegung werden können.

Aber es gibt noch eine andere Ebene, auf der das Werk verstanden werden
kann. Auf dieser Ebene handelt es sich nicht um eine wissenschaftliche
Analyse, sondern um den Ausdruck einer radikal pessimistischen
Weltanschauung. Diese Verständnisebene vermittelt sich nicht so sehr durch die
Feststellungen, die in dem Werk über Aufklärung, Mythologie, Kulturindustrie
oder Positivismus mitgeteilt werden, als vielmehr durch die Sprache, den
Duktus, sowie bestimmte Denkfiguren, die eher in eine religiös metaphysische
Richtung weisen. Hier wird das Urteil über eine Welt gesprochen, die nicht
mehr zu retten ist, in der das Heil ahnungsweise
vorstellbar,\footnote{Vgl. dazu besonders die Kapitelenden in der "`Dialektik
der Aufklärung"'.} seine Verwirklichung dabei aber so vollkommen
ausgeschlossen ist, dass man es sich eigentlich besser gleich aus dem Kopf
schlagen müsste. 

Zur besseren Unterscheidung werde ich diese beiden Ebenen {\em die Sachebene}
und {\em die metaphysische Ebene} nennen. Diese Unterscheidung erscheint mir
deshalb wichtig, weil das Werk auf der Sachebene fast vollkommen scheitert: Es
gelingt den Autoren nicht in glaubwürdiger Weise darzulegen, dass dem Prozess
der Aufklärung eine "`Dialektik"' eigen ist, durch die die Aufklärung am Ende
in "`Barbarei"' ausartet. Ebensowenig gelingt ihnen der Nachweis, dass die
neuzeitliche "`Barbarei"', mit der die totalitären Herrschaftsformen gemeint
sind, ein (und sei es auch nur ungewolltes) Erbe der Aufklärung ist. Dass das
Werk auf der Sachebene vollkommen scheitert, bedeutet nicht, dass es auch auf
der metaphysischen Ebene zum Scheitern verurteilt wäre. Dementsprechend
beanspruche ich nicht, dass mit der folgenden Kritik das letzte Wort darüber
gesprochen ist. Dennoch möchte ich mich in dieser Arbeit vor allem mit der
Sachebene des Werkes beschäftigen. Dies aus zwei Gründen:

1. Für eine Untersuchung des metaphysischen Pessmismus von Adorno und
Horkheimer ist die "`Dialektik der Aufklärung"' kaum die einschlägige Quelle.
Hierfür findet man in den Spätschriften beider sehr viel reicheres Material,
zumal sich die Autoren dort auch schon stärker von ihrem doktrinären Marxismus
gelöst haben, der in der "`Dialektik der Aufklärung"' noch mehr als spührbar
ist.\footnote{Z.B. in der plumpen Weise in der erkenntnistheoretische
  Standpunkte als Ausdruck gesellschaftlicher Herrschaftsverhältnisse gedeutet
  werden.}

2. Bevor man an die Deutung der metaphysischen Ebene heran geht, muss man sich
über die Sachebene zunächst einmal Klarheit verschaffen. Die Metaphysik
erscheint in anderem Licht, je nachdem, ob den metaphysischen Problemen
tatsächliche gesellschaftliche Gefahren, wie z.B. die Ausbreitung totalitärer
Herrschaftsformen, entsprechen, denen man entgegen treten kann und sollte,
oder ob sich die Metaphysik auf der Ebene eines Erlösungsutopismus
bewegt,\footnote{Zum Erlösungsutopismus vgl. die folgende Passage: "`Heute, da
  Bacons Utopie, daß wir `der Natur in der Praxis gebieten' in tellurischem
  Maßstab sich erfüllt hat, wird das Wesen des Zwanges offenbar, den er der
  unbeherrschten zuschrieb. Es war Herrschaft selbst. In ihre Auflösung vermag
  das Wissen, in dem nach Bacon die `Überlegenheit des Menschen' ohne Zweifel
  bestand, nun überzugehen. Angesichts solcher Möglichkeit aber wandelt im
  Dienst der Gegenwart Aufklärung sich zum Betrug der Massen um."' \cite [S.
  49]{adorno-horkheimer:1947} Adorno und Horkheimer scheinen also tatsächlich
  an so etwas wie eine herrschaftsfreie Gesellschaft geglaubt zu haben!}  der
ebensowenig befriedigt werden kann, wie der Wunsch wieder ins Paradies
zurückzukehren.

Hinzu kommt ein weiterer Grund: Die Gesellschaftskritik der Frankfurter Schule
scheint heutzutage nicht mehr eben übermäßig populär zu sein. Dennoch halte
ich eine kritische Auseinandersetzung mit der "`Dialektik der Aufklärung"' auf
der Sachebene immer noch für geboten. Denn ähnliche Auffassungen werden auch
von anderen zur Zeit vieldiskutierten Philosophen vertreten. Es ist schon
verblüffend: Wir leben in einer Zeit, die vom materiellen Überfluss gesegnet
ist, in der jeder beinahe mit Sicherheit damit rechnen darf, siebzig oder
achtzig Jahre alt zu werden. Zugleich leben wir (im Westen) in der freiesten
Gesellschaft, die es je gegeben hat, in der die Grundrechte des Menschen in
einem bisher nicht gekannten Maß verwirklicht worden sind und durch eine
wachsame Gerichtsbarkeit gegen Übergriffe der Regierung geschützt werden. Was
aber erfahren wir, wenn wir die Werke von Max Horkheimer, Theodor Adorno,
Michel Foucault oder, neuerlich, Giorgio Agamben zu diesem Thema konsultieren?
Dass das alles Essig ist, dass die Freiheit ein einziger Schwindel ist, dass
die Herrschaft über den Menschen in Wirklichkeit viel intensiver geworden ist,
und sich nur ihre Methoden als subtiler aber dafür umso durchgreifender
herausstellen, dass die Bürger von der Regierung fest im biopolitischen Griff
gehalten werden usw. usw.\footnote{So schreibt beispielsweise Foucault über
die Strafrechtsreformen Ende des 18.Jahrhunderts: "`Was sich abzeichnet ist
weniger ein neuer Respekt vor dem Menschen im Verurteilten - die Martern sind
auch für leichte Verbrechen noch häufig, sondern vielmehr eine Tendenz zu
einer sorgfältigeren und verfeinerten Justiz, zu einem lückenloseren
Durchkämmen des Gesellschaftskörpers"' \cite[S. 99]{foucault:1975}. Was auch
immer die Intentionen der Reformer gewesen sein mögen, dass es einen
objektiven Humanisierungsfortschritt bedeutet, wenn Schwerverbrecher heute
nicht mehr gevierteilt werden (Foucault liefert selbst die wenig erbauliche
Beschreibung am Anfang seines Buches), bleibt doch wohl unbestreitbar. Man
muss Foucault jedoch zu Gute halten, dass er trotz seines Hanges zur
einseitigen Überzeichnung immerhin eine bemerkenswerte Menge an empirischen
Material zur Untersützung seiner Thesen heranzieht. Dasselbe kann man nicht
von Giorgio Agamben behaupten, der mit Hilfe einer ganz ähnlichen
Montagetechnik, wie Adorno und Horkheimer sie gebrauchen (Vgl. Kapitel
\ref{Kritik} dieser Arbeit), einen Zusammenhang zwischen Menschenrechten und
Diktatur suggeriert: "`Faschismus und Nazismus sind vor allem eine
Redefinition des Verhältnisses zwischen Mensch und Bürger und werden ... nur
vor dem biopolitischen Hintergrund, den die nationale Souveränität und die
Menschenrechte eröffnet haben, ganz verstehbar."'
\cite[S. 139]{agamben:1995}}

Solche Enthüllungen über das wahre Wesen der vermeintlich freien
Gesellschaften mögen intellektuell sehr prickelnd sein, vielleicht haben sie
sogar den objektiven Wert, durch ihre maßlosen Übertreibungen auf tatsächliche
Gefahren aufmerksam zu machen, die der Freiheit auch in den modernen
Demokratien drohen könnten. Aber die Verkennung des vorhandenen und erreichten
Maßes von Freiheit birgt auch Gefahren. Es könnte sein, dass wir im
entscheidenden Augenblick versäumen, sie zu verteidigen, weil wir nicht
wissen, was sie uns wert sein sollte. 
 
Aus diesem Grund scheint mir eine kritische Auseinandersetzung mit den
Thesen der "`Dialektik der Aufklärung"' trotz der nachgelassenen Aktualität
des Werkes sinnvoll. Dabei stütze ich mich vor allem auf das erste Kapitel des
Werkes über den "`Begriff der Aufklärung"', auch wenn meine Kritik vor dem
Hintergrund des Gesamtwerkes formuliert ist, und teilweise Passagen aus den
späteren Kapiteln als Beleg angeführt werden.

\section{Die Grundthese der "`Dialektik der Aufklärung"'}

Im Vorwort der "`Dialektik der Aufklärung"' erklären die Autoren ihre
Intention damit, dass sie sich die Erkenntnis vorgesetzt hätten,
"`warum die Menschheit anstatt in einen wahrhaft menschlichen Zustand
einzutreten, in eine neue Art von Barbarei versinkt."'  \cite[S.
1]{adorno-horkheimer:1947} Und gleich auf der ersten Seite stellen sie
fest: "`.. die vollständig aufgeklärte Erde strahlt im Zeichen
triumphalen Unheils."' \cite[S. 9]{adorno-horkheimer:1947} Nirgendwo
in dem Werk wird eindeutig erklärt, was sie mit dem "`triumphalen
Unheil"' und der "`Barbarei"' meinen, aber aus zahlreichen
Anspielungen und dem zeitlichen Kontext der Entstehung des Werkes geht
hervor, dass mit der Barbarei offenbar totalitäre Herrschaftsformen
und in erster Linie der Faschismus gemeint sind.\footnote{Es dürfte im
  Hinblick auf den späteren Rezeptionserfolg des Werke nicht ohne
  Bedeutung gewesen sein, dass man auf Grund der Unspezifiziertheit
  des Begriffs die "`Barbarei"' auch als irgendeine Form von
  sinnentleerter Gesellschaft verstehen kann. In diesem weiten Sinne
  verstanden, würden allerdings der Krisenpessimismus und die
  Untergangsstimmung, die das Werk atmet, unangemessen und geradezu
  grotesk erscheinen. Deshalb gehe ich in meiner Interpretation davon
  aus, dass Adorno und Horkheimer ihre Leser und Leserinnen
  tatsächlich vor dem Totalitarismus warnen wollen und nicht bloß vor
  dem Konsumterror oder der Umweltzerstörung.}

Die Ursache für die "`Barbarei"' suchen Adorno und Horkheimer, wie
der Titel ihres Werkes andeutet, in den Verwerfungen eines
missglückten Aufklärungsprozesses. Weshalb führt aber nach Adornos und
Horkheimers Ansicht gerade die Aufklärung in den Totalitarismus?
Unstrittig ist (auch für Adorno und Horkheimer), dass dies keineswegs
den Intentionen der Aufklärung entspricht. Um diese Frage zu
beantworten, muss man auf den Ursprung der Aufklärung zurück
gehen. Die Aufklärung ist nämlich, glaubt man Adorno und Horkheimer,
mit einem Geburtsfehler belastet: Sie ist von Anfang an
mit Herrschaft verquickt. Für Adorno und Horkheimer fängt die
Aufklärung dabei schon in der Mythologie an. Aufklärung kann man im
allerweitesten Sinne als die rationale Kritik überkommener Vorurteile
verstehen. Und etwa in diesem Sinne meinen Adorno und Horkheimer, dass
bereits in der Mythologie solche Elemente rationaler Kritik
eingeschlossen sind, durch die frühere Mythologien verdrängt
werden \cite[S.14/15]{adorno-horkheimer:1947}. So stellen die Epen
Homers gegen über früheren Mythen einen Aufklärungsfortschritt dar,
doch es gibt etwas, "`was Epos und Mythos in der Tat gemein haben:
Herrschaft und Ausbeutung."' \cite[S.52]{adorno-horkheimer:1947}
Dieser Geburtsfehler zieht sich durch den gesamten
Zivilisationsprozess, gleichsam als habe sich schon sehr früh eine
Unwucht in das Getriebe der Menschheitsgeschichte eingeschlichen,
die mit der Zeit immer größere Zerstörungen anrichtet. So gesehen wird
im Totalitarismus des 20.Jahrhunderts nur manifest, was schon in der
Mythologie angelegt war: "`Schon der originale Mythos enthält das
Moment der Lüge, das im Schwindelhaften des Faschismus triumphiert."'
\cite[S. 52]{adorno-horkheimer:1947}

Die Anfänge des Aufklärungsprozesses als Verhängnisgeschichte
untersuchen Adorno und Horkheimer am Beispiel der Odyssee Homers. In
der Art und Weise, wie in der Odyssee auf die älteren Mythen
symbolisch Bezug genommen wird, zeigt sich für Adorno und Horkheimer,
dass dem Prozess der Aufklärung im Sinne der Überwindung weniger
aufgeklärter Stadien immer auch etwas Gewalttätiges anhaftet. Das
vorläufige Ende der Aufklärung im 20.Jahrhundert ist markiert durch
die positivstische Philosophie, die für die Autoren so etwas wie die
Vollendungsgestalt des aufklärerischen Denkens darstellt \cite[S. 22,
S. 24]{adorno-horkheimer:1947}, und durch die moderne
Unterhaltungskultur ("`Kulturindustrie"'), die nach Ansicht von Adorno
und Horkheimer "`Aufklärung als Massenbetrug"' \cite[S.
128]{adorno-horkheimer:1947} inszeniert. Es mag auf den ersten Blick
verblüffend erscheinen, dass Adorno und Horkheimer dabei den
Positivismus ebenso wie die moderne Unterhaltungskultur in einen
Zusammenhang mit dem Faschismus bzw. Totalitarismus stellen. Und die
kritische Untersuchung der Thesen der "`Dialektik der Aufklärung"' in
den folgenden Kapiteln dieser Arbeit wird zeigen, dass der
Zusammenhang zwischen Positivismus und Totalitarismus rein fingiert
und der zwischen der modernen Unterhaltungskultur und dem
Totalitarismus zumindest wesentlich schwächer ist, als Adorno und
Horkheimer vermuten. Aber Adorno und Horkheimer glaubten offenbar,
dass tatsächlich ein Zusammenhang zwischen dem Positivismus, der
Kulturindustrie und dem Faschismus bestand. Denn während sie ihre
radikale Aufklärungskritik in dieser Form nirgendwo noch einmal
wiederholt haben, finden sich ähnliche Vorwürfe gegen den Positivismus
auch in anderen Schriften der Autoren.\footnote{Für die
  Positivismuskritik siehe etwa Horkheimers Aufsatz "`Der neuste
  Angriff auf die Metaphysik"' von 1937. Dort schreibt Horkheimer:
  "`Und doch ist sie [die neupositivistische Philosophie, E.A.] in
  ihrer gegenwärtigen Gestalt nicht weniger fest als die Metaphysik
  mit den herrschenden Zuständen verknüpft. Wenn ihr Zusammenhang mit
  den totalitären Staaten [sic!] nicht offen zutage liegt, so ist er
  doch unschwer zu entdecken. Neuromantische Metaphysik und radikaler
  Positivismus gründen beide in der traurigen Verfassung eines großen
  Teils des Bürgertums, das die Zuversicht, durch eigene Tüchtigkeit
  die Verhältnisse zu verbessern, restlos aufgegeben hat und aus Angst
  vor einer entscheidenden Änderung des Gesellschaftssystems sich
  willenlos der Herrschaft seiner kapitalkräftigsten Gruppe
  unterwirft."' \cite[S. 116]{horkheimer:1937} -- Der Zusammenhang von
  Faschismus und "`Kulturindustrie"' liest sich in der "`Dialektik der
  Aufklärung"' z.B. so: "`In der totalen Hereinziehung der
  Kulturprodukte in die Warensphäre verzichtet das Radio überhaupt
  darauf, seine Kulturprodukte selber als Waren an den Mann zu
  bringen. Es erhebt in Amerika keine Gebühren vom Publikum. Dadurch
  gewinnt es die trügerische Form desinteressierter, überparteilicher
  Autorität, die für den Faschismus wie gegossen ist. Dort wird das
  Radio zum universalen Maul des Führers; in den Straßenlautsprechern
  geht seine Stimme über ins Geheul der Panik verkündenden Sirenen,
  von denen moderne Propaganda ohnehin schwer zu unterscheiden ist.
  Die Nationalsozialisten selber wußten, daß der Rundfunk ihrer Sache
  Gestalt verlieh wie die Druckerpresse der Reformation."'  \cite[S.
  168]{adorno-horkheimer:1947} } Positivismus und Kulturindustrie sind
für Adorno und Horkheimer wichtige Faktoren in einem Prozess, in dem
sich die bürgerliche Gesellschaft mit innerer Folgerichtigkeit zu
einer totalitären Gesellschaft wandelt.

Insgesamt lässt sich die Grundthese der "`Dialektik der Aufklärung"' also in
etwa folgendermaßen rekonstruieren: {\em Aufklärung ist von Anfang an mit dem
Fehler der Herrschaftsdienlichkeit und Gewaltsamkeit belastet. Eine
unkontrollierte ("`unreflektierte"') Aufklärung hat deshalb
selbstzerstörerischen Charakter und führt in den Totalitarismus.}

Diese These soll im Folgenden in zwei Schritten kritisiert werden. Als
Erstes versuche ich klar zu stellen, dass die These, dass Aufklärung
in den Totalitarismus führt, historisch betrachtet schlicht falsch
ist, und dass die von Adorno und Horkheimer behaupteten Zusammenhänge
von Positivismus, Kulturindustrie und Faschismus m.E. nicht gegeben
sind. Ich werde diese Fragen lösgelöst vom Text der "`Dialektik der
Aufklärung"' als historische bzw. philosophiehistorische Fragen
behandeln. Mir scheint nämlich, dass Adornos und Horkheimers
Begründung für ihre These so schwach ist, dass sie eine ernsthafte
Berücksichtigung kaum verdient, so dass es wenig sinnvoll ist, für die
Untersuchung der historischen Zusammenhänge auf den Text der
"`Dialektik der Aufklärung"' einzugehen. Die Rechtfertigung für diese
Vorgehensweise werde ich im folgenden Kapitel (Kapitel \ref{Kritik})
nachliefern, wo die "`Argumente"', die Adorno und Horkheimer für ihre
These anführen, näher betrachtet werden, und wo ich zu zeigen
versuche, dass man diese "`Argumente"', sofern man wissenschaftliche
Maßstäbe anlegt, nicht einmal ansatzweise ernst nehmen kann.

\section{Eine Klarstellung: Aufklärung ist nicht totalitär}
\label{Klarstellung}

Zunächst ist also zu untersuchen, ob es den von Adorno und Horkheimer
behaupteten "`dialektischen"' Zusammenhang zwischen Aufklärung und
Totalitarismus tatsächlich gibt. Dazu werde ich erstens die Frage untersuchen,
ob die von Adorno und Horkheimer suggerierten engen Zusammenhänge von
Positivismus und Totalitarismus sowie Kulturindustrie und Totalitarismus
existieren, und anschließend die grundsätzlichere Frage, ob der Totalitarismus
eine historische Folge der Aufklärung ist.

\subsection{Der Zusammenhang von Positivismus, Kulturindustrie und
  Totalitarismus ist eine Legende}

Die Frage, ob es einen Zusammenhang zwischen Positivismus und
Totalitarismus gibt, lässt sich sehr leicht beantworten, denn für
einen solchen Zusammenhang fehlt jeder innere und äußere Anhaltspunkt.
Fast nirgendwo werden in den Schriften der Neupositivisten politische
oder moralische Auffassungen vertreten, die denen eines totalitären
Regimes entsprechen. Die Ausnahme bilden gewisse Affinitäten zum
Marxismus am linken Rand des Wiener Kreises, z.B. bei Otto Neurath
\cite[S. 43ff.]{cartwright:1996}.\footnote{Als Beispiel für einen
  naturwissenschaftsorientierten Philosophen, der sich dem
  Nationalsozialismus zuwandte, könnte man {\em Hugo Dingler} nennen.
  Dingler kann dem Positivismus allerdings nur in einem sehr
  weitläufigen Sinn zugerechnet werden, da sein
  erkenntnistheoretischer Konventionalismus im Gegensatz zu den
  Auffassungen des logischen Empirismus stand. Im biographischen Teil
  des Standardwerkes von Friedrich Stadler über den Wiener Kreis,
  findet man keinen einzigen positivistischen Philosophen mit
  faschistischen Affinitäten \cite[]{stadler:1997}} Ganz im Gegenteil
hat sich der Positivismus durch seine metaphysikkritische Grundhaltung
sogar in einem Maße als ideologieresistent erwiesen wie kaum eine
andere philosophische Schule. So ziemlich alle prominenten Vertreter
des Wiener Kreises mussten ins Exil gehen \cite[]{stadler:1997}. Und
auch umgekehrt ist das Verhältnis der totalitären Staatsphilosophien
gegenüber dem Positivismus durch feindselige Ablehnung bestimmt. Das
gilt sowohl für den Faschismus, dessen Rückgriff auf mythologisierende
Ideologeme offensichtlich nicht mit der "`wissenschaftlichen
Weltauffassung"' der Positivisten vereinbar ist, als auch für den sich
auf einen vermeintlich wissenschaftlichen Marxismus stützenden
Kommunismus, für den Lenin die Linie gegenüber dem Positivismus, der
als bürgerliche Philosophie abgestempelt wird, schon frühzeitig in
seiner Schrift über "`Materialismus und Empiriokritizismus"'
\cite[]{lenin:1909} vorgegeben hat. Alles in allem ist es so gut wie
unmöglich zwischen der philosophischen Bewegung des Neupositivismus
und den totalitären Herrschaftsformen irgendeine Verbindung
herzustellen, auch wenn dies besonders in der Nachkriegszeit und nicht
nur von linker Seite öfters versucht worden ist.\footnote{Vgl. dazu
  die Vorwürfe Eric Voegelins gegen den "`destruktiven Positivismus"'
  \cite[S. 3ff.]{voegelin:1952}. Voegelin ist ziemlich bedeutungslos,
  aber seine Positivismuskritik lag in der Strömung des christlichen
  Humanismus durchaus im Trend.} Dass Adorno und Horkheimer
historische Tatsachen, wie die hier angeführten hartnäckig nicht zur
Kenntnis nehmen,\footnote{Noch in der Vorrede zur Neuausgabe der
  "`Dialektik der Aufklärung"' heißt es: "`Die in dem Buch erkannte
  Entwicklung zur totalen Integration ist unterbrochen, nicht
  abgebrochen; sie droht, über Diktaturen und Kriege sich zu
  vollziehen. Die Prognose des {\em damit verbundenen} Umschlags von
  Aufklärung in Positivismus, den Mythos dessen, was der Fall ist,
  schließlich die Identität von Intelligenz und Geistfeindschaft hat
  überwältigend sich bestätigt."'  \cite[S.
  IX/X]{adorno-horkheimer:1947} (Hervorhebung von mir, E.A.)} ist
leider nur zu bezeichnend für ihre philosophische Herangehensweise.
Der Positivismus ist nicht der einzige Leidtragende davon. Wie weiter
unten noch dargestellt wird, ergeht es der Aufklärung keineswegs
besser. (Und selbst die Mythologie wird in sehr drastischer
Vereinfachung weitgehend auf Mimesis reduziert, obwohl das Bild der
mythischen Vorstellungswelt als einer mimetischen Vorstellungswelt für
viele Mythen gar nicht besonders gut passt. Inwiefern wäre denn z.B.
der Schöpfungsmythos mimetisch?\footnote{Im Einklang mit der These,
  dass bereits der Mythos Aufklärung ist, könnte man die "`Dialektik
  der Aufklärung"' auch so interpretieren, dass höchstens der Ritus
  und eventuell die Urmythen mimetisch sind. Aber auch eine solche
  These bliebe weitgehend Spekulation.})

Gibt es zwischen dem Positivismus und den totalitären
Herrschaftsformen also keinerlei Zusammenhang, so ist die Annahme
eines irgendwie gearteten Zusammenhangs zwischen "`Kulturindustrie"',
d.h. derjenigen Art von Kunstwerken und Kulturgütern, die von
wirtschaftlich arbeitetenden Unternehmen nach dem Prinzip der
Gewinnmaximierung hergestellt und verbreitet werden, und totalitärer
Herrschaft schon sehr viel weniger abwegig, denn die totalitäre
Propaganda setzt häufig Stilmittel und Gestaltungsformen ein, wie sie
für die Populärkultur typisch und daher auch in freien Gesellschaften
geläufig sind. Insbesondere setzt die totalitäre Kulturpolitik in
hohem Maße auf Kitsch-Ästhetik \cite[S.
63ff.]{benz:2000}.\footnote{Wobei aber sowohl im italienischen
  Faschismus wie in der Sowjetunion vor Stalin eine künstlerische
  Avantgarde durchaus ihren Platz hatte.} Zudem kann man Adorno und
Horkheimer zugestehen, dass es Ihnen mit subtilen Analysen gelingt, in
manchen Kitschprodukten freier Gesellschaften Denkfiguren und
Verlogenheiten zu identifizieren, die -- in wesentlich massiverer Form
-- auch im totalitären Kontext auftreten. Dies ist aber nicht
verwunderlich, denn auch freie Gesellschaften sind niemals vollkommen
vorurteilslos, und in Kunst und Kultur spiegeln sich die Vorurteile
und Verlogenheiten einer Gesellschaft natürlich wieder. Aber bedeutet
das, dass ein Zusammenhang zwischen Kulturindustrie und Totalitarismus
besteht, etwa -- wie man sich vorstellen könnte -- dergestalt, dass
die Kulturindustrie den Bürgern bestimmte Denk- und Gefühlsformen
einhämmert, die von der totalitären Propaganda dann nur noch mit der
entsprechenden politischen Ideologie aufgefüllt werden müssen? Das mag
plausibel klingen, aber es ist sicherlich falsch.  Zwar stimmt es,
dass es eine Kulturindustrie und ihr Produkt, die Massenkultur, sowohl
in totalitären als auch in demokratischen Staaten gibt. Doch das zeigt
nur, das Kulturindustrie ein Merkmal moderner Gesellschaften ist, und
zugleich dass sie gerade nicht ein Spezifikum totalitärer Staaten ist.
Charakteristisch für totalitäre Staaten ist vielmehr, dass die
Kulturindustrie monopolisiert ist. Es gibt bestenfalls einen Zusammenhang
zwischen der Monopolisierung der Medien und des Kulturwesens und dem
Totalitarismus, aber nicht zwischen Kulturindustrie also solcher und
Totalitarismus. Adorno und Horkheimer können diese Suggestion bloß
aufrecht erhalten, indem sie die faktische Diversität des
Kulturangebots in einer nicht monopolisierten Kultur- und
Medienlandschaft leugnen. (In Amerika wird niemand gezwungen, sich
Jazz-Musik anzuhören, wenn er sie nicht mag, man kann dort ebensogut
einige der weltbesten Symphonieorchester besuchen!) Wenn
Kulturkritiker wie Adorno und Horkheimer die Verlogenheit der
Massenkultur und das gewisse Maß an Konformitätsdruck anprangern, das
es auch in den freien Gesellschaften noch gibt,\footnote{Darauf
  verweist das Tocqueville-Zitat \cite[S.
  141]{adorno-horkheimer:1947}, welches allerdings zu denken gibt.}
dann ist das an sich legitim und sogar begrüßenswert, aber man muss
sich im Klaren drüber bleiben, dass sie dann von Problemen handeln,
die weit entfernt sind von der Gefahr des Totalitarismus.

Es zeigt sich also sehr deutlich: Positivismus, Kulturindustrie und Faschismus
bzw. Totalitarismus sind drei sehr unterschiedliche Dinge, von denen der
Positivismus und der Totalitarismus überhaupt nichts miteinander gemein haben,
und die Kulturindustrie nur, wenn sie monopolisiert ist, eine Gefahr für die
Demokratie darstellt. Bei Adorno und Horkheimer fließen Positivismus,
Kapitalismus, bürgerliche Gesellschaft, Kulturindustrie, Technik, Faschismus
dagegen in ein und demselben Unheils- und Untergangssyndrom zusammen, das
irgendwie ein Ergebnis und eine Folge von Aufklärung sein soll. Bevor nun die
Frage untersucht wird, {\em weshalb} nach Adornos und Horkheimers Ansicht die
"`Barbarei"' eine Folge der Aufklärung ist, möchte ich, wie oben angekündigt,
zunächst klären, {\em ob} die "`Barbarei"' (worunter ich hier vor allem den
Totalitarismus verstehe, d.h. die Jazzmusik und das miese Filmangebot in
amerikanischen Kinos bleiben vorerst außen vor) überhaupt eine Folge der
Aufklärung ist.

\subsection{Faschismus und Kommunismus sind keine Folgen der Aufklärung}

Dass totalitäre Herrschaftsformen, also Faschismus und Kommunismus,
keine Folge der Aufklärung sind, gilt ziemlich eindeutig für den
Faschismus und etwas weniger eindeutig für den Kommunismus. Beim
Faschismus fällt die Beurteilung so eindeutig aus, weil sich die
faschistischen Bewegungen explizit gegen die Prinzipien der Aufklärung
gestellt haben. Unter den Prinzipien der Aufklärung verstehe ich dabei
1) auf intellektueller Ebene das Prinzip rationaler Begründung und
Kritik (d.h. Ansichten sollten durch Argumente begründet werden und
dürfen durch Argumente kritisiert werden, die Berufung auf Autorität
zählt nicht) 2) auf ethisch pratkischer Ebene die Grundidee von der
Autonomie des einzelnen Menschen, auf die wiederum die Menschenwürde
und das Selbstbestimmungsrecht jedes einzelnen Menschen gestützt sind
3) auf politischem Gebiet die Forderungen der Freiheit, Gleichheit und
Rechtstaatlichkeit.\footnote{Diese Prinzipien sind zumindest für die
  Aufklärer des 17. und 18. Jahrhunderts zentral. Bis auf den dritten
  Punkt, der die republikanischen politischen Forderungen beschreibt,
  trifft man sie aber auch bei älteren aufklärerischen Bewegungen,
  etwa in der griechischen Sophistik an. Adorno und Horkheimer
  verwenden freilich einen sehr weit gefassten und nicht mehr
  historisch verankerten Aufklärungsbegriff, der sich nur noch auf das
  Prinzip der rationalen Kritik zu beschränken scheint. Aber auch dann
  wäre es vollkommen absurd, den Faschismus oder Kommunismus als
  Ausfluss der Aufklärung, d.h. des Prinzips der rationalen Kritik,
  aufzufassen, denn beide totalitären Herrschaftsformen beruhen auf
  der Unterdrückung rationaler Kritik, auch wenn der Kommunismus
  dieses Mittel im Kampf gegen die "`bürgerliche Herrschaft"'
  einsetzt. Der entscheidende Punkt in diesem Fall ist, dass sich der
  Kommunismus, wenn er Kritik zwar als Mittel einsetzt, dennoch nicht
  zum {\em Prinzip} der rationalen Kritik bekennt, d.h. dem Prinzip
  dass die Politik der Regierung jederzeit mit Argumenten kritisiert
  werden kann.} All diese Prinzipien wurden von den faschistischen
Bewegungen {\em expressis verbis} und auf das schärfste
bekämpft.\footnote{Zur Ideologie des Nationalsozialisus bzw.
  Faschismus vgl. stellvertretend für zahlreiche andere Darstellungen
  die Beschreibung von Friedrich Pohlmann \cite[S.
  229ff.]{pohlmann:1992} oder gleich eine der Orginalquellen
  \cite[]{moussolini}, aus der hervorgeht, dass der faschistische
  "`Duce"' Moussolini mit dem aufklärerischen Menschenbild nicht
  einverstanden war (S. 5ff.).}  Der Faschismus war eine Gegenbewegung
zur Aufklärung, aber keine Folge der Aufklärung. Man kann der
Aufklärung unmöglich einen Vorwurf daraus machen wollen, dass die
gegen sie gerichtete Gegenbewegung zeitweise recht erfolgreich war.

Etwas weniger eindeutig liegt der Fall beim Kommunismus, denn der Kommunismus
beruht auf einem humanistischen Wertekanon, indem er die Gleichheit der
Menschen verficht, Freiheit von Unterdrückung und Menschenrechte
einfordert. Dazu -- und darin geht er über die Aufklärung hinaus, aber in
einer Weise, die nicht im Widerspruch zu den Prinzipien der Aufklärung steht
und an sich nur begrüßenswert ist -- tritt der Kommunismus sehr entschieden
für die soziale und ökonomische Gleichheit ein und nicht nur wie der
Liberalismus für die Gleichheit vor dem Gesetz. Während sich die
faschistischen Bewegungen außerdem gerne auf eine Art Pseudo-Mythologie
berufen, nimmt der Kommunismus für sich in Anspruch mit der marxistischen
Gesellschaftsanalyse über eine rationale, wissenschaftliche Grundlage zu
verfügen. 

% In der Tat hatte das planwirtschaftiche System nach dem Ersten
% Weltkrieg sogar bis weit ins bürgerliche Lager hinein überzeugte Anhänger,
% denn gegenüber einem krisenanfälligen, chaotischen und ungerechten
% Kapitalismus konnte eine Planwirtschaft, deren praktische Durchführbarkeit die
% Kriegswirtschaft bewiesen zu haben schien, als das rationalere und modernere
% Wirtschaftssystem vorkommen.

Ander\-erseits besteht kein Zweifel daran, dass die
auf\-klär\-er\-isch\--\-human\-ist\-isch\-en Werte, die dem
Kommunismus ursprünglich zu Grunde lagen, im real existierenden
Sozialismus größtenteils\footnote{Diese Einschränkung ist notwendig,
  weil auf bestimmten Sektoren, wie z.B. der Gleichberechtigung von
  Mann und Frau, der real existierende Sozialismus in der Tat
  zeitweise fortschrittlicher war.}  hemmungslos pervertiert worden
sind. So konnte (und kann) von politischer Freiheit in kommunistischen
Staaten keine Rede sein, Gleichheit galt allenfalls für die normalen
Bürger unterhalb der Bonzenklasse, und die Menschenrechte wurden und
werden in kommunistischen Ländern mit Füßen getreten, wobei einige
kommunistische Regime sogar millionfache Massenmorde angezettelt
haben. All das widerspricht so offensichtlich den Prinzipien der
Aufklärung, dass es unmöglich ist, den Kommunismus in irgend einer
Weise als verwirklichte oder vollendete Aufklärung aufzufassen. Die
Frage kann also nur noch lauten, ob die Art von Perversion von
Aufklärung, die wir im Kommunismus vorfinden, bereits in der
Aufklärung angelegt ist. Dies entspräche in etwa der These von Adorno
und Horkheimer, dass der Aufklärung eine "`Dialektik"' innewohnt, die
dazu führt, dass sich die Aufklärung am Ende selbst zerstört.  Um die
Aufklärung von diesem Verdacht freizusprechen, genügt es nicht, zu
zeigen, dass ihre Prinzipien pervertiert worden sind. Vielmehr muss
man zeigen, dass eine Verwirklichung der aufklärerischen Prinzipien
ohne den Bruch mit wesentlichen dieser Prinzipien möglich ist. Lässt
sich dies zeigen, dann ist der Beweis erbracht, dass sich die
Perversion der Aufklärung nicht mit innerer Notwendigkeit aus der
Aufklärung selbst ergibt. Nun ist dieser Beweis aber längst in der
politischen Praxis erbracht worden: In den liberalen Demokratien sind
die wesentlichen Prinzipien der Aufklärung sehr erfolgreich
verwirklicht worden, ohne dass sich Anzeichen von Barabarei zeigen,
die dem Faschismus oder Kommunismus auch nur annährend vergleichbar
wären. Damit ist nicht gesagt, dass in den liberalen Demokratien alles
zum Besten steht, und insbesondere kann man der politischen
Philosophie der Aufklärung vorwerfen, dass sie die sozialen Fragen
vernachlässigt.\footnote{Radikaldemokratische und soziale Bewegungen
  wie die der "`Levellers"' während der englischen Revolutionsepoche
  im 17.Jahrhundert oder der "`Sansculotten"' in der Französischen
  Revolution waren eine vorübergehende Erscheinung, und ihre Ideen
  fanden nicht Eingang in die Hauptströmungen aufklärerischen
  Denkens.}  Aber das berührt eine ganz andere Diskussion, nämlich
die, wie die Aufklärung noch verbessert werden kann. Auf keinen Fall
rechtfertigen die Schwächen, die die Aufklärung auf diesem oder
anderen Gebieten noch haben mag, die Diagnose Adornos und Horkheimers,
dass die Aufklärung mit innerer Logik der Barbarei zutreibt.
Insbesondere kann nicht behauptet werden, dass der Kommunismus die
logische Konsequenz oder die politische Vollendung der Aufklärung
sei.\footnote{Ich bin mir nicht sicher, ob Adorno oder Horkheimer das
  ernsthaft behaupten wollen. Aber wenn ihre These, dass der Prozess
  der Aufkärung in die Barbarei führt bzw. dass "`die vollständig
  aufgeklärte Erde .. im Zeichen triumphalen Unheils [strahlt]"'
  irgendeinen nicht trivialen Sinn haben soll, dann müssen sie
  entweder behaupten, dass Aufklärung zum Faschismus oder zum
  Kommunismus oder zu beidem führt. Alles andere wäre angesichts des
  Pathos ihres Buches einfach lächerlich.}  Dies gilt umso mehr als
die kommunistische Ideologie in Form ihres chiliastischen
Geschichtsbildes, der Funktion der Partei als Avantgarde der
Arbeiterklasse, aus der eine natürliche Herrschaftsberechtigung
abgeleitet wird, der kollektivistischen Rechts- und Moralauffassung
und des vielfach auftretenden Personenkultes Elemente
enthält,\footnote{Die Lehren des entwickelten, d.h.  leninistischen
  Kommunismus werden in kurzer Form bei Donald Busky dargestellt
  \cite[S. 163ff.]{busky:2002}.} die sich mit aufklärerischem Denken
kaum vereinbaren lassen. Zusammengenommen bedeutet dies: Die
Prinzipien der Aufklärung lassen sich ohne Bruch verwirklichen. Der
Kommnismus verwirklicht dagegen nicht die Prinzipien der Aufklärung,
sondern teilweise pervertiert er sie und teilweise bricht er mit
ihnen.

Das Gesamtergebnis all dieser Überlegungen lautet also, dass weder der
Faschismus noch der Kommunismus in irgendeiner Weise Produkte der Aufklärung
sind. Der Faschismus ist eine Gegenbewegung gegen die Aufklärung, der
Kommunismus ist eine Perversion der Aufklärung. Wenn diese Überlegungen
stimmen, dann kann die zentrale These der "`Dialektik der Aufklärung"', dass
die Aufklärung sich mit innerer Logik selbst zerstört, und dadurch zum
Totalitarismus führt, nur noch falsch sein. Im folgenden soll gezeigt werden,
dass die Begründung, die Adorno und Horkheimer liefern, wissenschaftlich
betrachtet dermaßen niveaulos ist, dass ihre zentrale These eigentlich auch
nur falsch sein konnte.

\section{Die Begründungsdefizite der "`Dialektik der Aufklärung"'}
\label{Kritik}

Jüngere Besprechungen der "`Dialektik der Aufklärung"' kommen kaum ohne
erhebliche Vorbehalte gegenüber den radikalen Thesen des Werkes aus
\cite[S. 130-157]{habermas:1985} \cite[]{schnaedelbach:1989}. Nur selten
werden die Schwächen der Argumentationsstrategie von Adorno und Horkheimer
jedoch im Detail untersucht. Im folgenden soll etwas ausführlicher als üblich
auf die Begründung eingangen werden, die Horkheimer und Adorno für ihre Thesen
liefern.  Dazu werden einige der wesentlichen "`Argumente"' und Beispiele von
Adorno und Horkheimer herausgegriffen und näher untersucht, um zu zeigen, wie
wenig diese "`Argumente"' und Beispiele in Wirklichkeit dazu taugen, irgend
etwas zu begründen.

Die Grundidee des Werkes ist, dass die Aufklärung ihrer Intention nach
Befreiung ist, dass sie in Wirklichkeit aber nur zur Ausweitung von
Herrschaft führt. Herrschaft ist dabei in einem mehrfachen Sinn zu
verstehen als Herrschaft des Menschen über die äußere Natur, als
Herrschaft des Menschen über seine eigene innere Natur (d.h. als
Triebbeherrschung im Freudschen Sinne), und als Herrschaft im
politischen Sinne, also als Herrschaft des Menschen über den Menschen.
In jedem Falle ist Herrschaft dabei etwas Schlechtes. Das gilt sogar
für die Herrschaft über die Natur, denn nach Adorno und Horkheimer
sollte die Aufklärung zwar dem Menschen die Furcht vor der Natur
nehmen \cite[S. 9]{adorno-horkheimer:1947}, aber dass sie dabei zur
Herrschaft über die Natur gerät, gehört schon zur "`Dialektik"' der
Aufklärung, d.h. zu jenen unerwünschten Nebenfolgen und inneren
Widersprüchen der Aufklärung, die, wie Adorno und Horkheimer glauben,
schließlich ihre Selbstzerstörung herbeiführt.

\subsection{Naturwissenschaftliches Denken als "`disponierendes Denken"'}

Die Aufklärung ist bei Adorno und Horkheimer im Wesentlichen als ein
Prozess der zunehmenden Rationalisierung zu verstehen, der sich auf
allen Ebenen durchsetzt, im Denken, in den gesellschaftlichen
Institutionen, und auch in den zwischenmenschlichen Beziehungen. Ihre
vollendete Gestalt erreicht die aufklärerische Rationalität im
mathe\-ma\-tisch\--\-na\-tur\-wis\-sen\-schaft\-lich\-en Denken, das
von der philosophischen Schule des Neupositivismus in gewisser Weise
zum Maßstab wissenschaftlichen und rationalen Denkens überhaupt
erhoben worden ist. Entsprechend ihrer Vorstellung von Aufklärung als
Prozess zunehmender Herrschaftsintensivierung behaupten Adorno und
Horkheimer denn auch, dass im naturwissenschaftlichen Denken und in
der positivistischen Philosophie dieser Zusammenhang besonders eng
ausfällt. Das liest sich bei Adorno und Horkheimer folgendermaßen:

\begin{quotation}
Noch die deduktive Form der Wissenschaft spiegelt Hierarchie
und Zwang. Wie die ersten Kategorien den organisierten Stamm und seine Macht
über den Einzelnen repräsentieren, gründet die gesamte logische Ordnung,
Abhängigkeit, Verkettung, Umgreifen und Zusammenschluß der Begriffe in den
entsprechenden Verhältnissen der sozialen Wirklichkeit, der
Arbeitsteilung. \cite[S. 27/28]{adorno-horkheimer:1947} 
\end{quotation}

Bereits die "`deduktive Form der Wissenschaft"' ist für Adorno und
Horkheimer also Ausdruck von Herrschaftsverhältnissen, von
"`Hierarchie und Zwang"'. Noch deutlicher drücken die Autoren dies an
einer anderen Stelle aus: 

\begin{quotation}
Die Allgemeinheit der Gedanken, wie die
diskursive Logik sie entwickelt, die Herrschaft in der Sphäre des
Begriffs, erhebt sich auf dem Fundament der Herrschaft in der
Wirklichkeit. \cite[S. 20]{adorno-horkheimer:1947} 
\end{quotation}

Wie begründen die Autoren aber ihre These, dass die "`diskursive
Logik"' und "`die deduktive Form"' der Wissenschaft Ausdruck von
gesellschaftlichen Herrschaftsverhältnissen sind? Die Antwort auf
diese Frage ist, dass sie es überhaupt nicht begründen. Die ganze
These ruht allein auf den persönlichen Assoziationen von Adorno und
Horkheimer. Und selbstverständlich ist die These falsch. Die
"`diskursive Logik"' ist ein Mittel, das demjenigen, der die
Herrschaft kritisieren will, ganz ebenso zu Gebote steht, wie den
Ideologen, die sie rechtfertigen. Die "`diskursive Logik"' ist ein
Werkzeug des Geistes, das schlechterdings jeder nutzen kann, zu
unterschiedlichsten Zwecken. Allenfalls kann man die Behauptung wagen,
dass Aufklärung, Klarheit und offene Diskussion (bei der wiederum die
"`diskursive Logik"' zum Tragen kommt) immer die bevorzugten Waffen
der Unterdrückten sein werden, da sie in der Regel zwar anzuklagen,
aber ihrerseits nichts zu verbergen haben.

Weil Adorno und Horkheimer aber glauben, dass das
naturwissenschaftliche Denken so eng mit Herrschaft und Unterwerfung
verbunden ist, verwundert es nicht, dass sie dem wissenschaftlichen
Denken nicht mehr die Fähigkeit zur Wahrheitserkenntnis zutrauen.
Ihrer Ansicht nach dient das wissenschaftliche Denken nicht der
Naturerkenntnis, sondern allein der Unterwerfung der Natur.
Dementsprechend ist es "`disponierendes Denken"' \cite[S.
20]{adorno-horkheimer:1947} und vom aufgeklärten
Menschen,\footnote{"`Das Selbst, das die Ordnung und Unterordnung an
  der Unterwerfung der Welt lernte"' \cite[S.
  20]{adorno-horkheimer:1947}.} der sich dieses disponierenden Denkens
bedient,\footnote{"`Die Aufklärung verhält sich zu den Dingen wie der
  Diktator zu den Menschen. Er kennt sie, insofern er sie manipulieren
  kann."' \cite[S. 15]{adorno-horkheimer:1947}} wird nach Adornos und
Horkheimers Meinung "`die Erkenntnis tabuiert, die den Gegenstand
wirklich trifft."' \cite[S. 20]{adorno-horkheimer:1947} Adorno und
Horkheimer hängen offenbar einer Vorstellung von Naturwissenschaft an,
nach der die Naturwissenschaft vor allem durch den Zweck ihrer
technischen Anwendung motiviert ist, und damit gewissermaßen rein
instrumentell auf Manipulation der Natur und nicht so sehr auf
Erkenntnis zielt. Daraus erklärt sich möglicherweise der Vorwurf, dass
angeblich "`Erkenntnis tabuiert"' wird, "`die den Gegenstand wirklich
trifft"'. Dieser Vorwurf ist etwas schwierig nachzuvollziehen, denn
aller Erfahrung nach pflegen doch die Naturwissenschaften ihren
Gegenstand ziemlich gut zu treffen. Und spätestens hier stellt sich
auch die Frage, wie die Erkenntnis, "`die den Gegenstand wirklich
trifft"', denn dann beschaffen sein soll. Offenbar denken die Autoren
an eine Art von dialektischer Erkenntnis, wie folgende Kritik am
Erkenntnismodus der Aufklärung deutlich macht, aus der sich indirekt
Hinweise darauf ergeben, wie sich Adorno und Horkheimer die richtige
Erkenntnis vorstellen:

\begin{quotation}

Das Vorfindliche als solches zu begreifen, den Gegebenheiten nicht
bloß ihre abstrakten raumzeitlichen Beziehungen abzumerken, bei denen
man sie dann packen kann, sondern sie im Gegenteil als die Oberfläche,
als vermittelte Begriffsmomente zu denken, die sich erst in der
Entfaltung ihres gesellschaftlichen, historischen, menschlichen Sinnes
erfüllen -- der ganze Anspruch der Erkenntnis wird
preisgegeben. \cite[S. 33]{adorno-horkheimer:1947}

\end{quotation} 

Das Erkenntnismodell, das Adorno und Horkheimer hier mit dem "`Anspruch der
Erkenntnis"' an sich gleichsetzen, ist offensichtlich das des deutschen
Idealismus,\footnote{An anderer Stelle schreiben die Autoren: "`Aufklärung hat
die klassische Forderung, das Denken zu denken -- Fichtes Philosophie ist ihre
radikale Entfaltung -- beiseitegeschoben"'
\cite[S. 31]{adorno-horkheimer:1947}. Das Zitat zeigt deutlich, wie sehr
Adorno und Horkheimer die Reflexionsphilosophie des deutschen Idealismus zur
Norm des philosophischen Denkens schlechthin verklären ("`klassische
Forderung"'). Durch diesen Trick fällt es ihnen dann sehr leicht ein
philosophisches Denken, das dieser Norm nicht folgt, als eine Art Verrat an
den Idealen der Philosophie erscheinen zu lassen.} wie es besonders in Hegels
dialektischer Philosophie zu einer Art Vollendung geführt worden ist. Aber
dann entstehen sogleich zwei Probleme: Erstens ist die Dialektik für die
Naturerkenntnis vollkommen unbrauchbar, wie Hegels Naturphilosophie ungewollt,
aber dafür umso eindrucksvoller bewiesen hat.\footnote{Als Beispiel können die
geradezu mittleiderregenden Passagen über das Fallgesetz aus Hegels
"`Enzyklopädie der Philsophischen Wissenschaften im Grundrisse"' dienen
\cite[§ 267, 268]{hegel:1817}. Und dabei zitiert er sogar Lagrange!} Zweitens
ist zu befürchten, dass die Dialektik noch viel stärker als Adorno und
Horkheimer es der "`diskursiven Logik"' und der "`deduktiven Form der
Wissenschaft"' unterstellen, durch den gesellschaftlichen Kontext von
"`Hierarchie und Zwang"', dem "`Fundament der Herrschaft in der Wirklichkeit"'
belastet ist, gelang es ihrem Meister und Erfinder doch mit Hilfe der
Dialektik alle möglichen politischen und sozialen Vorurteile seiner Zeit
philosophisch zu beweisen. So konnte Hegel z.B. dialektisch erklären, weshalb
Frauen und Männern in der Ehe unterschiedliche Rollen zukommen \cite[§
165,166]{hegel:1821}, und nicht weniger raffiniert konnte er mit Hilfe der
Dialektik die Notwendigkeit der monarchischen Staatsform \cite[§
279,280]{hegel:1821}, die Unmöglichkeit (in dem Sinne, dass es auch gar nicht
wünschbar wäre) des Weltfriedens \cite[§ 333, 337]{hegel:1821} und die
Nützlichkeit der Pressezensur \cite[§ 319]{hegel:1821}\footnote{Wobei Hegel in
diesem Falle nicht einmal die Dialektik bemüht, sondern lapidar erklärt: "`Die
Freiheit der öffentlichen Mitteilung ... hat ihre direkte Sicherung in den
ihre Ausschweifungen teils verhindernden, teils bestrafenden polizeilichen und
Rechtsgesetzen"'.} beweisen.

Der Vorwurf des "`disponierenden Denkens"' ist nicht der einzige Vorwurf, den
Adorno und Horkheimer gegen das aufklärerische und besonders das
naturwissenschaftliche Denken erheben. Sie erheben noch mindestens zwei
weitere Vorwürfe, die sich mit diesem Vorwurf gar nicht unbedingt leicht
vereinbaren lassen. Diese beiden Vorwürfe sind erstens der Vorwurf der
Verarmung des Weltbildes durch das mathematisch-naturwissenschaftliche Denken
und zweitens der Vorwurf, die Natur bloß abzubilden.

\subsection{Der Zusammenhang von Aufklärung und Mythologie}

Der zweite Vorwurf ist mit der Behauptung, dass das
naturwissenschaftliche Denken "`disponierendes Denken"' sei, d.h. ein
Denken, dass vor allem auf die Nutzbarmachung der Natur für technische
und industrielle Anwendungszwecke zielt, nicht ohne Weiteres
vereinbar, denn wenn man die Natur nutzbar machen will, dann genügt es
nicht, sie bloß abzubilden, sondern man muss auch einen Sinn für die
Möglichkeiten entwickeln, die in den Dingen stecken. Dann stimmt aber
gerade nicht mehr, was Adorno und Horkheimer der mathematischen
Naturwissenschaft unterstellen, dass "`die Unterwerfung alles Seienden
unter den logischen Formalismus, .. mit der gehorsamen Unterordnung
der Vernunft unters unmittelbar Vorfindliche erkauft"' wird
\cite[S. 33]{adorno-horkheimer:1947}. Wieso insistieren Adorno und
Horkheimer darauf, dass das naturwissenschaftliche Denken "`das
Tatsächliche"' nur wiederholt bzw. sich blind bei dessen
"`Reproduktion"' bescheidet und riskieren dabei den Widerspruch zu
ihrer anderen Behauptung, dass das naturwissenschaftliche Denken
"`disponierendes Denken"' ist? Der Grund könnte darin liegen, dass sie
nur so eine der zentralen Thesen ihres Werkes rechtfertigen können,
nämlich die These, dass Aufklärung im Grunde bloß Mythologie sei. Denn
auch in der Mythologie wird nach Adornos und Horkheimers Auffassung
die Wirklichkeit wiedergespiegelt, indem in der Mythologie die
Zusammenhänge in der Natur als unabänderliche Schicksalsgesetze
verstanden werden. Wenn die Naturwissenschaft die Natur ebenfalls nur
abbildet und dabei ebenso zu unabänderlichen Gesetzmäßigkeiten kommt,
dann muss nach der überaus fragwürdigen Logik Adornos und Horkheimers
die Aufklärung (die sich auf das naturwissenschaftliche Denken stützt)
genau dasselbe sein wie die Mythologie. Diese reichlich simple
Begründung ihrer These wird von Adorno und Horkheimer mit großem
rhetorischen Pomp in Szene gesetzt:

\begin{quotation}

Der mathematische Formalismus aber, dessen Medium die Zahl, die
abstrakte Gestalt des Unmittelbaren ist, hält statt dessen den
Gedanken bei der bloßen Unmittelbarkeit fest. Das Tatsächliche behält
recht, die Erkenntnis beschränkt sich auf seine Wiederholung, der
Gedanke macht sich zur bloßen Tautologie. Je mehr die Denkmaschinerie
sich das Seiende unterwirft, umso blinder bescheidet sie sich bei
dessen Reproduktion. Damit schlägt Aufklärung in Mythologie zurück,
der sie nie zu entrinnen wußte. Denn Mythologie hatte in ihren
Gestalten die Essenz des Bestehenden: Kreislauf, Schicksal, Herrschaft
der Welt als die Wahrheit zurückgespiegelt und der Hoffnung
entsagt. In der Prägnanz des mythischen Bildes wie in der Klarheit der
wissenschaftlichen Formel wird die Ewigkeit des Tatsächlichen
bestätigt und das bloße Dasein als der Sinn ausgesprochen, den es
versperrt. Die Welt als gigantisches analytisches Urteil, der einzige,
der von allen Träumen der Wissenschaft übrig blieb, ist vom gleichen
Schlage wie der kosmische Mythos, der den Wechsel von Frühling und
Herbst an den Raub Persephones knüpfte. \cite[S. 33]{adorno-horkheimer:1947}

\end{quotation} 

Lässt man sich von der aufwendigen Rhetorik der Autoren nicht beeindrucken,
dann fällt an diesem Zitat auf, wie ausgesprochen schwach die Begründung der
These bleibt, dass Aufklärung selbst bloß Mythologie ist. Die Begründung
dieser These ruht genauso wie die Behauptung eines Zusammenhangs zwischen dem
naturwissenschaftlichen Denken und Herrschaftsverhältnissen allein auf
Assoziationen und Suggestionen, wie z.B. der Suggestion, dass sich durch die
Verwendung mathematischer Formalismen "`der Gedanke ... zur bloßen
Tautologie"' macht, oder etwas später, in der Unterstellung die aufgeklärte
Wissenschaft betrachte die "`Welt als gigantisches analytisches
Urteil"'. Beide Behauptungen, dass die mathematischen Formalismen in der
Wissenschaft Tautologien seien, und das die Wissenschaft die Welt zu einem
gigantischen analytischen Urteil macht, sind übrigens schlicht und einfach
falsch und niemals von irgendeinem positivistischen Philosophen vertreten
worden. Wenn man die Mathematik wissenschaftlich anwendet, dann sind die
mathematischen Formeln gerade keine Tautologien mehr, und sofern man die
Unterscheidung zwischen analytischen und synthetischen Urteilen trifft (was
gar nicht einmal alle im weiten Sinn positivistischen Philosophen tun), sind
Urteile, d.h. Aussagen über die Welt, wie sie die Naturwissenschaften
aufstellen, als empirische Urteile selbstverständlich immer synthetische
Urteile. 

\label{Kritik_naturwissenschaftlichen_Denkens}
Das Empörenswerte an der Art und Weise, wie Adorno und Horkheimer hier
"`argumentieren"', besteht darin, dass sie all das, was
Naturwissenschaftler, Aufklärer und positivistische Philosophen
ausdrücklich zu diesem Thema geäußert haben, komplett übergehen, und
der Aufklärung statt dessen teilweise Ansichten und Tendenzen
unterstellen, die den explizit geäußerten Meinungen aufklärerischer
und positivistischer Philosophen vollkommen widersprechen.
Empörenswert ist diese unfaire und unseriöse Vorgehensweise auch
deshalb, weil in dem gesamten Werk gegenüber dem
naturwissenschaftlichen und aufklärerischen Denken der ständige
latente Vorwurf mitschwingt, dass dieses Denken die Dinge nicht für
sich sprechen lässt, sondern sie nach eigenen von instrumentellen
Verwertungsabsichten geleiteten Interessen zurichtet.\footnote{Vgl.
  dazu den Anfang des Kapitels über "`Juliette oder Aufklärung und
  Moral"' \cite[S. 90/91]{adorno-horkheimer:1947}.}  Wie das oben
angeführte Zitat zeigt, muss man den Vorwurf, den Erkenntnisgegenstand
nicht für sich sprechen zu lassen, sondern ihn nach eigenem Belieben
zuzurichten, viel eher wohl gegen Adornos und Horkheimers Behandlung
der Aufklärung richten. Wie ignorant Adorno und Horkheimer dabei
häufig vorgehen, mag noch das folgende Zitat vor Augen führen, in
welchem Adorno und Horkheimer ihre recht eigentümliche Deutung der
Newtonschen Physik geben: "`Die Lehre von der Gleichheit von Aktion
und Reaktion behauptete die Macht der Wiederholung übers Dasein, lange
nachdem die Menschen der Illusion sich entäußert hatten, durch
Wiederholung mit dem wiederholten Dasein sich zu identifizieren und so
seiner Macht sich zu entziehen."' \cite[S. 18]{adorno-horkheimer:1947}
Darf man darauf hinweisen, dass dieses tiefe Wort der beiden großen
Philosophen blanker Unfug ist?  Ich glaube, man muss es tun, denn die
Lehre von "`Aktion und Reaktion"' behauptet nichts, aber auch wirklich
gar nichts hinsichtlich einer "`Macht der Wiederholung"' über das
Dasein. Sie besagt vielmehr, dass es zu jeder Kraft, die ein Körper
auf einen anderen ausübt eine Gegenkraft vom gleichen Betrag aber
entgegengesetzter Richtung gibt, die der zweite Körper wiederum auf
den ersten ausübt. Beispiel: Die Sonne übt durch die Gravitation eine
bestimmte Kraft auf die Erde aus, und nach dem Gesetz von Aktion und
Reaktion übt die Erde deshalb in umgekehrter Richtung eine ebenso
große Kraft auf die Sonne aus: Erde und Sonne ziehen sich gegenseitig
an (womit sich unter Berücksichtigung der unterschiedlichen Masse von
Sonne und Erde und der Initialgeschwindigkeit der Erde erklären lässt,
warum die Erde um die Sonne kreist). Was in aller Welt hat das mit der
"`Macht der Wiederholung übers Dasein"' zu tun?

Wie sich gezeigt hat, kann man die Begründung, die Adorno und
Horkheimer für ihre These, dass Aufklärung in Mythologie zurück
schlägt, liefern, kaum ernst nehmen, aber ist deswegen auch die These
auch falsch? Es wäre ja auch denkbar, dass Adorno und Horkheimer eine
an sich vernünftige These bloß etwas ungeschickt begründet haben. Aber
leider ist nicht nur die Begründung ungeschickt, sondern die These ist
auch im Wesentlichen falsch. Die Einschränkung "`im Wesentlichen"' ist
notwendig, weil die These in einem weiten und in einem engen Sinne
verstanden werden kann. In einem weiten Sinne verstanden, könnte man
sie als richtig beurteilen, aber sie wäre völlig banal. Wenn
Aufklärung und Mythologie bloß deshalb ein und dasselbe sind, weil sie
beide in irgendeiner Weise die Welt abbilden oder "`das Tatsächliche
bestätigen"', dann ist das ungefähr so, als wenn jemand behauptet,
Regen und Sonnenschein seien ein und dasselbe, weil beides bloß
Wetter ist. 

In einem engeren Sinne aufgefasst ist sie allerdings
falsch, denn zwischen Aufklärung und Mythologie bestehen sehr
gravierende Unterschiede. So beruht die Naturwissenschaft sehr
wesentlich auf den beiden Prinzipien der rationeln, intersubjektiven
Kritisierbarkeit ihrer Theorien und der empirischen Überprüfung an
Hand von Beobachtung und Experiment. Mythen sind aber in der Regel
nicht empirisch überprüfbar und die rationale Kritik von Mython gerät
fast zwangsläufig mit gesellschaftlichen Tabus in Konflikt. Die
Mythologie und die rationale Naturerkenntnis der Naturwissenschaften
sind also schon vom Prinzip her sehr verschiedene Dinge, und wollte
man einen ernsthaften Vergleich zwischen Mythen und
naturwissenschaftlichen Theorien anstellen, so würde man auch im
Einzelnen auf jede Menge bedeutender Unterschiede, und im Ganzen
höchstwahrscheinlich auf sehr viel mehr Unterschiede als
Gemeinsamkeiten treffen. Alles in allem ist die These von Adorno und
Horkheimer also nicht nur schlecht begründet, sondern tatsächlich auch
falsch.

Aber die These von der tieferen Identität von Aufklärung und
Mythologie ist nicht nur unbegründet und falsch. Selbst wenn sie
richtig wäre, wäre sie darüber hinaus ziemlich irrelevant. Denn
angenommen es stimmte, dass Aufklärung im Grunde auch nur Mythologie
ist. Was wäre damit erklärt? Kann das etwa erklären, warum die
"`Menschheit ... in ... Barbarei versinkt"' \cite[S.
1]{adorno-horkheimer:1947}? Das könnte es nur, wenn man ungefragt das
durchaus aufklärerische Vorurteil voraussetzt, dass Mythologie immer
etwas Barbarisches ist. Ansonsten erklärt die mit wichtiger Miene
vorgetragene Feststellung, dass "`Aufklärung in Mythologie
zurückschlägt"' überhaupt nichts. Warum rücken Adorno und Horkheimer
diese These dann aber so sehr ins Zentrum ihrer Abhandlung?  Eine
Erklärung dafür könnte im religiös-metaphysischen
Vorstellungshintergrund ihrer Philosophie zu finden sein. Die Ursünde
des Mythos beruht für Adorno und Horkheimer nämlich darauf, dass er
mit Herrschaft verquickt ist. Und, wie wir gesehen haben, gelingt es
nach Adornos und Horkheimers Interpretation der Aufklärung nicht, sich
von dieser Verstrickung zu lösen. Vom Standpunkt ihres chiliastischen
Marxismus aus gesehen laufen Mythos und Aufklärung daher tatsächlich
auf ein- und dasselbe hinaus, denn es kann der Aufklärung naturgemäß
nicht gelingen, uptopische Versprechungen einzulösen, die sie in
Wirklichkeit nie gemacht hat, um die Horkheimer und Adorno die
Menschheit deswegen aber nicht weniger betrogen glauben. Auf ganz
ähnliche Weise fließen auch Faschismus und bürgerliche, d.h.
demokratische Herrschaft bei Adorno und Horkheimer immer wieder in
ein- und dieselbe undifferenzierte Negativvision der durch und durch
von Herrschaft, Lüge und Unterdrückung geprägten Welt zusammen.

\subsection{Das verarmte Weltbild der positivistischen Philosophie}

Der Vorwurf, die Welt bloß abzubilden und dadurch eine Art Kult des
Tatsächlichen zu betreiben,\footnote{"`Die mathematische
  Verfahrensweise wurde gleichsam zum Ritual des Gedankens. ... Mit
  solcher Mimesis aber, in der das Denken der Welt sich gleichmacht,
  ist nun das Tatsächliche so sehr zum Einzigen geworden, daß noch die
  Gottesleugnung dem Urteil über die Metaphysik verfällt."'  \cite[S.
  31/32]{adorno-horkheimer:1947}} stützt sich auf eine Kritik an der
Verarmung des Weltbildes durch eine einseitig naturwissenschaftliche
Betratungsweise, wie man sie auch bei anderen Philosophen antreffen
kann. Adorno und Horkheimer führen Edmund Husserl als Gewährsmann an,
der in seiner Schrift über die "`Krisis der europäischen
Wissenschaften"' \cite[]{husserl:1936} die Fragwürdigkeit einer allein
an der mathematischen Theoriebildung der Naturwissenschaften
orientierten Weltauffassung deutlich herausstellt. Anders als Adorno
und Horkheimer argumentiert Husserl jedoch vorwiegend
erkenntnistheoretisch (auch wenn er die erkenntnistheoretischen Thesen
seiner Krisis-Schrift um eine etwas fragwürdige Geschichtsphilosophie
ergänzt). Er kritisiert, dass die naturwissenschaftlichen Theorien,
wenn man sie ontologisch (und nicht, wie man wohl soll,
instrumentalistisch) auslegt, an die Stelle der konkreten Phänomene,
abstrakte mathematische geometrische Gestalten setzen. So werden
insbesondere die "`Sinnesfüllen"' wie Farbe, Klang, Geschmack durch
geometrische Größen wie z.B. die Wellenlänge einer Schwingung ersetzt
\cite[§9 c)]{husserl:1936}. Ein anderer Kritikpunkt Husserls besteht
darin, dass der Anspruch der Naturwissenschaften, speziell der Physik,
die Welt vollständig zu beschreiben, nur idealiter in unendlichem
Fortschreiten realisiert wird, womit er in Wirklichkeit natürlich
niemals ganz eingelöst werden kann \cite[§9 e)]{husserl:1936}.

Bei Adorno und Horkheimer wird diese Kritik entschieden
radikalisiert. Der Vorwurf, den sie erheben, besteht nicht mehr -- wie
bei Husserl -- nur darin, dass durch das naturwissenschaftliche
Weltbild bestimmte, für uns bedeutsame Aspekte der Wirklichkeit
ausgeblendet werden, sondern sie unterstellen, dass durch das
mathematisch-naturwissenschaftliche Weltbild die Welt auf eine
unausweichliche Weise vorherbestimmt ist. ("`Wenn im mathematischen
Verfahren das Unbekannte zum Unbekannten einer Gleichung wird, ist es
damit zum Altbekannten gestempelt, ehe noch ein Wert eingesetzt
wird."' \cite[S. 31]{adorno-horkheimer:1947}) In diesem Sinne ist auch
der berüchtigten Satz zu verstehen: "`Aufklärung ist totalitär wie nur
irgendein System."' \cite[S. 31 (Vgl. auch
S.12)]{adorno-horkheimer:1947} Diese Äußerung fällt in einem
erkenntnistheoretischen Kontext. Allerdings dürfte der politische
Beiklang sehr wohl von den Autoren beabsichtigt sein, denn sie stellen
-- entsprechend der Marxschen Basis-Überbau-Theorie -- während des
ganzen Kapitels Analogien zwischen der Erkenntnistheorie und den
gesellschaftlichen Verhältnissen her.

Wie schon an anderen Stellen wird hier deutlich, dass mit Adorno und
Horkheimer zwei Philosophen sprechen, denen das naturwissenschaftliche
Denken durch und durch fremd ist. Sonst könnte ihnen die Welt nicht
schon durch die Annahme in unausweichlicher Weise festgelegt
erscheinen, dass alle Vorgänge in der Welt vollständig mathematisch
beschreibbar sind. Denn solange nicht auch die Naturgesetze
selbst genannt werden, ist mit der Annahme der mathematischen
Beschreibbarkeit allein noch so gut wie gar nichts über die Welt
gesagt. Die Geschichte der Naturwissenschaften ist deshalb auch eine
Geschichte von immer wieder neuen überraschenden Einsichten; nur eben
nicht für Leute, denen es genügt, dass viele dieser Einsichten
mathematisch formuliert sind, um sich nicht weiter dafür zu
interessieren.

Hätten Adorno und Horkheimer zwischen den Erkenntnissen der
Naturwissenschaften einerseits und der Ausweitung des
naturwissenschaftlichen Weltbildes auf alle Bereiche der Philosophie
unterschieden, wie sie in den materialistischen und positivistischen
Philosophien stattfindet, und ihre Kritik vornehmlich gegen die
materialistische Philosophie anstatt gegen die mathematische
Naturbetrachtung schlechthin gerichtet, dann wäre sie noch diskutabel
gewesen. Als Naturwissenschaftskritik wirken ihre Ausführung jedoch
schlicht inkompetent.

\subsection{Die Fragwürdigkeit der Beispiele von Horkheimer und Adorno}

Soweit die "`Argumente"', die Adorno und Horkheimer ins Feld führen. Wie
verhält es sich mit Beispielen? Wenn jemand in einer philosophischen
Abhandlung behauptet, dass Aufklärung dasselbe ist wie Mythologie, und dass
sie (deshalb?) in die Barbarei führt, dann liegt es nahe, Beispiele von
aufklärerischen Philosophen anzuführen, bei denen sich die Mythologie und die
barbarischen Tendenzen besonders deutlich zeigen. Und in der Tat führen Adorno
und Horkheimer in ihrem Buch auch einige Philosophen an. Das Sonderbare ist
nur, dass die Philosphen, die sie anführen, entweder keine Aufklärer sind,
oder dass sie, wenn es Aufklärer sind, von Adorno und Horkheimer in genau
derselben fragwürdigen und unfairen Weise fehlinterpretiert werden, die weiter
oben schon an Adornos und Horkheimer Darstellung des Zusammenhangs von
Aufklärung und Mythologie (Seite
\pageref{Kritik_naturwissenschaftlichen_Denkens}) kritisiert wurde.

Die beiden Philosophen, die von Adorno und Horkheimer am
ausführlichsten diskutiert werden, sind der Marquis de Sade und
Friedrich Nietzsche. Daneben wird auch Immanuel Kant recht häufig
erwähnt. (Auf den Neupositivismus gibt es zahlreiche Anspielungen,
aber kaum namentliche Erwähnungen.) Nun sind allerdings weder der
Marquis de Sade noch Friedrich Nietzsche besonders aufklärerische
Philosophen. Beiden ist höchstens gemeinsam -- und hier könnte man
allenfalls eine schwache Verbindung zur Aufklärung herstellen -- dass
sie die Emanzipation von den tradierten sittlichen Normen predigen.
Nur setzen sie an die Stelle der tradierten Normen eine Moral, die
kein Aufklärer jemals vertreten würde. So fordert der Marquis de Sade
vollkommene Straflosigkeit für jede Art von Sexualverbrechen (und auch
noch für einige andere Verbrechen), während Nietzsche die bekannte
Herrenmenschenethik vertritt.\footnote{Nietzsches moraliphilosophische
  Vorstellungen sind besonders in den Schriften "`Jenseits von Gut und
  Böse"' \cite[z.B. Neuntes Hauptstück, 260.Abschnitt]{nietzsche:1885}
  und "`Zur Genealogie der Moral"' \cite[z.B. Erste Abhandlung,
  5.Abschnitt]{nietzsche:1887} ausgedrückt. Die moralischen
  Überzeugungen des Marquis de Sade sind größtenteils in die Dialoge
  seiner Romane eingeflochten. In knapper Form sind sie in dem in die
  "`Philosophie im Budoir"' eingeschalteten Essay dargestellt
  \cite[S.191-268]{sade:1795}. Vgl. zu Nietzsche auch den Kommentar
  Russells \cite[Kapitel über Nietzsche (3.Buch,
  XXV.Kapitel)]{russell:1946}, der gut als Beispiel dafür stehen kann,
  wie aufgeklärte Philosophen auf Nietzsches Morallehre reagieren.
  Eine Reaktion, der Adorno und Horkheimer freilich sogleich
  unterstellen, sie rühre bloß vom Hass her, weil Nietzsche und de
  Sade nicht (wie nach ihrer Ansicht die Aufklärer) die
  Unbegründbarkeit der Moral durch Vernunft vertuscht hätten.}  Wie
können Nietzsche und der Marquis de Sade dann aber mit der Aufklärung
in Verbindung gebracht werden?  Die Antwort lautet, dass sie nach der
Ansicht Adornos und Horkheimers in geradezu idealtypischer Weise einen
der inneren Widersprüche der Aufklärung zum Ausdruck bringen. Dieser
Widerspruch besteht darin, dass sich durch aufklärerisches Denken das
Letztbegründungsproblem der Ethik nicht lösen lässt. Während die
Aufklärer dieses Problem aber in einer unehrlichen Weise vertuschen
(z.B. Kant, der sich hinsichtlich seines kategorischen Imperativs auf
ein "`Faktum der Vernunft"' beruft), haben der Marquis de Sade und
Nietzsche offen die Konsequenzen daraus gezogen.:

\begin{quotation}

Die Unmöglichkeit, aus der Vernunft ein grundsätzliches Argument gegen Mord
vorzubringen, nicht vertuscht, sondern in alle Welt hinaus geschrieen zu
haben, hat den Haß entzündet, mit dem gerade die Progressiven Sade und
Nietzsche heute noch verfolgen. Anders als der logische Positivismus nehmen
beide die Wissenschaft beim Wort. \cite[S. 127]{adorno-horkheimer:1947}

\end{quotation} 

Auffällig ist, nebenbei bemerkt, an dieser Passage, dass Adorno und
Horkheimer den "`Progressiven"' (die wieder einmal anonym bleiben)
keine Chance lassen: Empören sich die "`Progressiven"' über Nietzsche
und de Sade, dann wissen Aodrno und Horkheimer (woher eigentlich?),
dass es sich dabei um pure Heuchelei handelt. Täten sie es nicht, dann
würden sie erst recht die These von einer "`Dialektik der Aufklärung"'
bestätigen. 

Aber, abgesehen davon, haben nicht Adorno und Horkheimer
vielleicht recht? Ist es nicht tatsächlich eine Schwäche der
Aufklärung, dass sie das Letztbegründungsproblem der Ethik mit der
Vernunft nicht lösen kann? Dagegen ist Folgendes einzuwenden: Wie auch
immer man die Lösbarkeit und Unlösbarkeit des Letztbegründungsproblems
beurteilen mag, wenn es nicht mit der Vernunft gelöst werden kann,
dann gibt es auch keine andere Möglichkeit es zu lösen. Insbesondere
ist es nicht möglich, das Letztbegründungsproblem religiös (oder
mythisch) zu lösen, denn mit Berufung auf die Religion lässt sich jede
Moral, eine so gut wie die andere rechtfertigen, ohne dass
irgendwelche bestimmten moralischen Normen auf diese Weise als die
einzig gültigen ausgezeichnet werden könnnten.\footnote{Die Religion
mag den Vorzug haben, dass sie in sehr viel stärker, als das auf
andere Weise möglich ist, moralisches Handeln motivieren kann,
begründen kann sie es nicht bzw. nur in der Weise, dass das
Begründungsproblem auf die Religion verschoben wird, was es eher noch
komplizierter werden lässt.}  Schlechterdings jede Moralphilosophie
hat das Problem, dass sie ihre Normen nicht letztbegründen kann. Nur
tritt diese Tatsache in manchen Philosophien offener zu Tage als bei
anderen. Dann kann die Unlösbarkeit des Letztbegründungsproblems kein
Aspekt des Prozesses sein, den Adorno und Horkheimer als "`Dialektik
der Aufklärung"' bezeichnen. Ist dies aber einmal zugestanden, so
erscheint es höchst zweifelhaft, Nietzsche und den Marquis de Sade zu
konsequenten Vollendern aufklärerischer Moralphilosophie zu
stilisieren. Dass sind sie beileibe nicht, denn diejenigen moralischen
Normen, die sie vorgeschlagen haben, stehen in schärfstem Widerspruch
zu den Werten, die die Philosophen der Aufklärung vertraten. 

Man kann mit Einschränkungen behaupten, dass die Unlösbarkeit des
Letztbegründungsproblems von vielen Vertretern des modernen
Positivismus zumindest stillschweigend zugestanden wird.\footnote{Von
  den Philosophen der Aufklärungsepoche ließe sich dies mit der
  möglichen Ausnahme David Humes allerdings nicht behaupten. Dass auch
  Philosophen, die sich -- anders als fast alle Denker der
  Aufklärungsepoche -- der Unlösbarkeit des Letztbegründungsproblems
  bewusst waren, keineswegs dazu neigten aus diesem Faktum amoralische
  Konsequenzen zu ziehen, beweist, dass Friedrich Nietzsche und der
  Marquis de Sade sich von den Aufklärern nicht durch größere
  Konsequenz unterscheiden, wie Adorno und Horkheimer unterstellen,
  sondern ganz einfach durch ihre sittliche Verkommenheit. Diese
  sittliche Verkommenheit beruht dann aber auf einer bestimmten Wahl
  der ethischen Werte seitens dieser Autoren und kann nicht als
  Ausfluss des Prozesses der Aufklärung interpretiert werden.} Aber
wenn Adorno und Horkheimer den Positivisten daraus einen Strick drehen
wollen, dann müssen sie sich fragen lassen, ob sie denn ihrerseits das
Letztbegründungsproblem der Ethik lösen können. Können sie es lösen,
dann bräuchten sie ihre Lösung bloß mitzuteilen und könnten sich ihre
Vorwürfe sparen. Können sie es nicht, dann haben sie den Philosophen,
die das Letztbegründungsproblem für unlösbar halten, auch nichts
vorzuwerfen. 

Der grundlegende Denkfehler Adornos und Horkheimers
besteht darin, dass sie dem aufklärerischen Denken, sofern
man den Neupositivismus dazu rechnet, eine Tatsache vorwerfen, die es
nicht erfunden sondern bloß festgestellt hat.  Ähnlich, wie ja auch
das "`Gesetz von Aktion und Reaktion"', wie bereits angemerkt wurde,
keine Erfindung zur Rechtfertigung unveränderlicher gesellschaftlicher
Ordnung ist, sondern ein Naturgesetz, dass experimentell festgestellt
werden kann.  Auf denselben Denkfehler gründet sich Adornos und
Horkheimers Interpretation von Kants Philosophie. Sie interpretieren
Kants Erkenntnistheorie als Ausdruck einer Weltauffassung, wie sie für
die spätbürgerliche Epoche charakteristisch ist:

\begin{quotation}

Die Sinne sind vom Begriffsapparat je schon bestimmt, bevor die Wahrnehmung
erfolgt, der Bürger sieht a priori die Welt als den Stoff, aus dem er sie sich
herstellt. Kant hat intuitiv vorweggenommen, was erst Hollywood bewußt
verwirklichte: die Bilder werden schon bei ihrer Produktion nach den Standards
des Verstandes vorzensiert, dem gemäß sie nachher angesehen werden
sollen. \cite[S. 91]{adorno-horkheimer:1947}

\end{quotation}

Abgesehen davon, dass der Analogieschluss zwischen der Formung der
Erscheinungen durch den Erkenntnisapparat bei Kant und den
Gestaltungsprinzipien von Filmproduktionen doch eher weitläufig ist, so dass
es fragwürdig erscheint, hier von einer intuitiven Vorwegnahme zu sprechen,
verkennen Adorno und Horkheimer vollkommen, dass Kants erkenntnistheoretische
Konstruktion durch sachliche Erklärungsabsichten motiviert ist. Man kann den
Kantschen Schematismus der Wahrnehmung aus vielen Gründen kritisieren, aber
ihn bloß als Ausdruck bestimmter gesellschaftlicher Tendenzen der bürgerlichen
Epoche zu verstehen, wird der Kantschen Philosophie nicht gerecht. Dies gilt
umso mehr als gerade Kant sich in seiner Moralphilosophie und seiner
politischen Philosophie als ein (auch im Sinne Adornos und Horkheimers) höchst
reflektierter Philosoph offenbart, für den keineswegs die "`Vernunft die
Instanz des kalkulierenden Denkens, das die Welt für die Zwecke der
Selbsterhaltung zurichtet"' \cite[S. 90]{adorno-horkheimer:1947} ist.

Es zeigt sich also, dass nicht nur die Argumentation Adornos und Horkheimers
in der "`Dialektik der Aufklärung"' auf sehr schwachen Füßen steht, auch ihre
Beispiele sind, um es vorsichtig zu formulieren, sehr unglücklich gewählt und
kaum geeignet ihre These zu stützen. Dabei hätten sich vielleicht sogar
entsprechende Beispiele finden lassen. Ein mögliches Beispiel wäre Jeremy
Bentham, dessen theoretische Beschreibung des "`Panopticums"', einer Art von
perfektem Gefängnis, schon eher dazu einlädt über die Zusammenhänge von
aufgeklärter Philosophie, Sadismus und pathologischem Machbarkeitswahn
nachzudenken.\footnote{Die Bedeutung von Benthams "`Panopticum"' ist besonders
von Michel Foucault hervorgehoben (und vielleicht wiederum etwas übertrieben)
worden \cite[S. 256ff.]{foucault:1975}.} Und es dürfte, auch wenn das hier
nicht vertieft werden kann, noch viele weitere Beispiele geben, denn
zweifellos hat die Aufklärung auch ihre Schattenseiten. Zu den Schattenseiten
der Aufklärung zählt beispielsweise die häufig im aufklärerischen Denken
anzutreffende Konstruktion der Geschichte als einer Geschichte des
zivilisatorischen Fortschritts, die beinahe notwendigerweise mit einer
Abwertung vermeintlich primitiverer Gesellschaftszustände einhergeht. Adorno
hat dieses Problem einmal in einer im Vergleich zur Argumentation der
"`Dialektik der Aufklärung"' sehr viel überzeugenderen Weise an Goethes
"`Iphigenie"' exemplifiziert: Anders als das antike Vorbild von Euripides
endet Goethes "`Iphigenie"' scheinbar versöhnlich: Iphigenie und ihr Bruder
Orest werden von Thoas, dem Herrscher der Insel Tauris, schließlich freiwillig
entlassen. Doch dieser humanistische Schluss hat einen Haken: Während die
"`zivilisierten"' Griechen Iphigenie, Orest, Pylades ihr Recht bekommen, muss
der "`rohe Skythe"' Thoas alles dafür geben, ohne irgend eine Gegenleistung zu
erhalten. Dieser moralische Konstruktionsfehler des Dramas ist ein Ausfluss der
humanistischen Ethik, nach der die primitive Lebensform mitsamt ihren
barbarischen Bräuchen schlechterdings keine Existenzberechtigung hat
\cite[S. 509-510]{adorno:1969}. Natürlich kann man bei diesem Beispiel die
methodische Frage aufwerfen, inwieweit ein Theaterstück eines bestimmten
Dichters repräsentativ für eine ganze Epoche oder Geistesströmung wie den
Humanismus ist, aber unplausibel ist das Vorgehen Adornos in diesem Fall
nicht, und im Gegensatz zu dem, was Adorno und Horkheimer in der "`Dialektik
der Aufklärung"' in das naturwissenschaftliche Denken hineininterpretieren,
fällt die Interpretation der "`Iphigenie"' sehr stringent und überzeugend aus.

Welche Beispiele man aber zur Untersütztung der Thesen der "`Dialektik der
Aufklärung"' auch anführen mag, bestätigen lässt sich allenfalls die eher
banale These, dass die Aufklärung auch Schattenseiten hat. Die starke These,
die Adorno und Horkheimer vertreten, dass die Aufklärung mit innerer Logik
ihrer eigenen Zerstörung zutreibt, ist unter keinen Umständen haltbar. Weder
gelingt es ihnen, ihre These durch eine auch nur halbwegs überzeugende
Argumentation zu stüzten, noch führen sie glaubwürdige Beispiele an. 

\section{Ergebnis}

Alles in allem hat sich gezeigt, dass die zentralen Thesen der "`Dialektik der
Aufklärung"' unter wissenschaftlichen Gesichtspunkten betrachtet schlichtweg
falsch sind. Die Aufklärung führt weder direkt noch indirekt in die
Barbarei. Zwischen dem aufgeklärten bzw. naturwissenschaftlichen Denken und
der Mythologie besteht ein himmelweiter Unterschied. Man kann nicht wirklich
behaupten, dass schon der Mythos Aufklärung ist, und es stimmt nicht, dass die
Aufklärung in Mythologie zurückschlägt. Es kann zwar vorkommen, dass Menschen
ihre aufgeklärten Grundsätze vergessen und sich in Denken und Handeln wieder
an Mythen orientieren, aber dann ist die Aufklärung gescheitert und nicht in
Mythologie zurück geschlagen. Als ein solches Scheitern von Aufklärung kann
man den Faschismus deuten, aber damit geht der Faschismus gerade nicht auf das
Konto der Aufklärung. 

Dieses kritische Fazit bezieht sich vor allem auf den sachlich
wissenschaftlichen Gehalt der "`Dialektik der Aufklärung"'. Es wurde
Eingangs darauf hingewiesen, dass man sich die "`Dialektik der
Aufklärung"' auch auf anderen Verständnisebenen aneignen
kann. Insbesondere der letzte Teil des Buches, die "`Aufzeichnungen
und Entwürfe"' \cite[S. 218 - 275]{adorno-horkheimer:1947} verweisen
das Werk eher in ein literarisches Genre. Aber andererseits wollten
die Autoren ja keinen Roman verfassen, sondern ein philosophisches
Werk. Und dann interessiert eben nicht in erster Linie die Frage "`Was
wollte der Autor uns damit sagen?"' sondern vielmehr "`Ist das, was
die Autoren behaupten, wahr oder ist es falsch?"' Bedauerlicherweise
ist das, was Adorno und Horkheimer in dem Buch behaupten, größtenteils
falsch. 

Nun könnte man natürlich fragen, warum man sich überhaupt an den
sachlichen Fehlern des Buches stören sollte, und es nicht gleich als
einen Ausdruck metaphysischer Weltablehung auffassen sollte, ähnlich
wie die Philosophie Schopenhauers, die auch von kaum jemanden wörtlich
für wahr gehalten, aber doch von vielen geschätzt wird. Ein Grund,
warum es viel leichter fällt, die Irrtümer in der Philosophie
Schopenhauers zu akzeptieren, besteht darin, dass Schopenhauer
wesentlich sorgfältiger zwischen der metaphysischen Weltdeutung und
der wissenschaftlichen Welterklärung unterscheidet. Begibt er sich auf
die Ebene der Erklärung, so argumentiert er stets sehr umsichtig und
genau und respektiert die Tatsachen (so beispielsweise in seiner
ausgezeichneten Analyse des Ehrbegriffes in den "`Aphorismen zur
Lebensweisheit"'). Adorno und Horkheimer leiten viel unmittelbarer aus
ihren metaphysischen Voraussetzungen (d.h. aus ihrem chiliastischen
Marxismus und der dialektischen Methode) eine Welterklärung ab, die
sie hermetisch gegen Einwände abriegeln. Das verleiht der "`Dialektik
der Aufklärung"' ebenso wie manchen ihrer anderen Schriften einen
entschieden ideologischen Zug, von dem man schwerlich absehen kann,
selbst wenn man Sympathie für den Pessimismus und die Weltverachtung
Adornos und Horkheimers empfindet. Erst im Spätwerk dieser beiden
Philosophen klingt das ideologische Moment ab.

Möglicherweise wäre die "`Dialektik der Aufklärung"' ein sehr viel
glaubwürdigeres Buch geworden, wenn Adorno und Horkheimer nicht gerade die
Aufklärung ins Zentrum ihrer Zivilisationskritik gestellt hätten. Denn trotz
aller Übertreibung sind manche Aspekte ihrer Kritik an der Massenkultur oder
der Verdinglichung oder auch des instrumentellen Denkens im Umgang mit der
Natur durchaus plausibel. Aber die Verbindung zur Aufklärung und zur
positivistischen Philosophie ist nicht nachvollziehbar und wohl bloß aus den
Ressentiments der Autoren zu erklären.

Der überaus negative Befund mag bei einem Buch wie der "`Dialektik der
Aufklärung"' verwundern, dem heute von manchen Gelehrten Klassikerstatus
zugebilligt wird \cite[]{beck:1998}.\footnote{Im Ganzen äußert sich Beck
jedoch sehr kritisch zur "`Dialektik der Aufklärung"'.} Immerhin hat die
Dialektik der Aufklärung in ihrer Rezeptionsgeschichte mehrmals Phasen
ausgesprochener Popularität erlebt. In der 68er Bewegung hatte sie wohl den
Status eines Kultbuches und auch in den 80er Jahren gewann sie aufgrund ihrer
radikal pessimistischen Zivilisationskritik im Zusammenhang mit der
Umweltbewegung erneut Popularität. Aber selbst wenn man die
zivilisationskritische Grundhaltung der "`Dialektik der Aufklärung"' teilt,
dann ist es zumindest in wissenschaftlicher Hinsicht wenig lohnend, sich mit
dem Buch auseinanderzusetzen, denn in der "`Dialektik der Aufklärung"' werden
die Gefahren des zivilisatorischen Prozesses falsch dargestellt, und die
Verantwortung dafür mit der Aufklärung dem falschen Schuldigen
zugewiesen.\footnote{In der Sekundärliteratur wird häufig kolportiert, dass
Adorno und Horkheimer an den Zielen der Aufklärung hätten festhalten wollen
\cite[S. 31]{schnaedelbach:1989} und in der Vorrede beteuern Adorno und
Horkheimer: "`Wir hegen keinen Zweifel ... dass die Freiheit in der
Gesellschaft vom aufklärenden Denken unabtrennbar ist."'
\cite[S. 3]{adorno-horkheimer:1947}. Wenn es tatsächlich ihr Ziel war, an
der Aufklärung festzuhalten, dann haben Adorno und Horkheimer ihr Ziel
allerdings gründlich verfehlt, denn im gesamten Buch findet sich kaum ein
gutes Wort über die Aufklärung oder aufklärerische Philosophen.} In dem ganzen
Buch findet sich kein einziges Argument, das geeignet wäre, jemanden, der
nicht sowieso schon die Meinung der Autoren teilt, von den Gefahren einer
ungezügelten technischen Zivilisation zu überzeugen. Lohnend könnte die
Auseinandersetzung allerdings zur Selbstbestätigung der eigenen
pessimistischen Stimmungslage sein, denn in dieser Hinsicht zahlen sich die in
kraftvoller Sprache dargebotenen kompromisslosen Verdammungsurteile der
Autoren zugegebenermaßen voll aus.

Heute scheint die Popularität des Werkes allerdings stark nachgelassen zu
haben. Philosophische Kongresse zur "`Dialektik der Aufklärung"' finden
anscheinend nicht mehr häufig statt. Diskutiert wird sie noch in politischen
Zirkeln wie der kommunistischen Arbeitsgruppe der PDS, die sich in dem Werk
Schützenhilfe dafür erhofft, "`wie der Einfluss der Kriegspartei im
öffentlichen Leben der BRD verstanden und bekämpft werden kann; ... wie die
Ausbreitung des Strebens nach faschistischer Gewaltherrschaft erklärt und
gebrochen bzw. verhindert werden kann."' \cite[]{kag} Dass die "`Dialektik
der Aufklärung"' obskuren politischen Zirkeln zur geistigen Grundlage dient,
entspricht nicht ihrem intellektuellen Niveau, ist aber angesichts der
wissenschaftlichen Schwächen des Werkes ein nicht ganz unverschuldetes
Schicksal.

% www.kag-berlin.de/adorno.htm

%\section{Die "`Dialektik der Aufklärung"' als metaphysische Lebenskritik}
%\label{DDAInterpretation}
%\subsection{Weltgeschichte und Unheilsgeschehen}
%\subsection{Das Ausbleiben der Parusieerwartung}
%\subsection{Erlösung als Geste: Die Versöhnung}

%\section{Der Pessimismus der "`Dialektik der Aufklärung"' im Vergleich}
%\label{DDAVergleich}
%\subsection{Huxley}
%\subsection{Beckett}
%\subsection{Schopenhauer}

%\section{Was hat uns die "`Dialektik der Aufklärung"' heute noch zu sagen}

\newpage

\bibliographystyle{dinat}

\bibliography{Adorno}

\end{document}
