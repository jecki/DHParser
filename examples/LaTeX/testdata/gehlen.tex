\documentclass[12pt,a4paper]{article}
\usepackage{microtype}
\usepackage{ae}
\usepackage[german, ngerman]{babel}
\usepackage[utf8x]{inputenc}
\usepackage{ucs}
\usepackage[T1]{fontenc}
\usepackage{t1enc}
\usepackage{type1cm}

\usepackage{ifpdf}
\ifpdf
\usepackage{xmpincl}
\usepackage[pdftex]{hyperref}
\hypersetup{
    colorlinks,
    citecolor=black,
    filecolor=black,
    linkcolor=black,
    urlcolor=black,
    bookmarksopen=true,     % Gliederung öffnen im AR
    bookmarksnumbered=true, % Kapitel-Nummerierung im Inhaltsverzeichniss anzeigen
    bookmarksopenlevel=1,   % Tiefe der geöffneten Gliederung für den AR
    pdfstartview=FitV,       % Fit, FitH=breite, FitV=hoehe, FitBH
    pdfpagemode=UseOutlines, % FullScreen, UseNone, UseOutlines, UseThumbs 
}
\includexmp{gehlen}
\pdfinfo{
  /Author (Eckhart Arnold)
  /Title (Die Humanismuskritik Arnold Gehlens in seinem Spätwerk "Moral und Hypermoral")
  /Subject (Eine Auseinandersetzung mit Gehlens pluralistischer Ethik)
  /Keywords (Arnold Gehlen, Humanitarismus, Konservativismus, Humanismus)
}
\fi

%\renewcommand{\baselinestretch}{1.5}

\newcommand{\high}[1]{\raisebox{0.7ex}{#1}}
\newcommand{\low}[1]{\raisebox{-0.7ex}{#1}}

\sloppy

\begin{document}

\begin{titlepage}

\title{Die Humanismuskritik Arnold Gehlens in seinem
    Spätwerk "`Moral und Hypermoral"'}
\author{Eckhart Arnold}
\date{15.Mai 1998}

\setlength{\parindent}{0em}

\begin{center} {\large\bf Philosophische Fakultät der
    Rheinischen-Friedrich-Wilhelms Universität Bonn} \end{center}

\setlength{\parskip}{2cm}

\begin{center} {\Large Die Humanismuskritik Arnold Gehlens in seinem
    Spätwerk "`Moral und Hypermoral"'} \end{center}

\setlength{\parskip}{3cm}

Hausarbeit zum\\ Hauptseminar:\\ Die Humanismusdebatte im 
20.Jahrhundert\\ WS 1997/98\\[0.5cm]Leitung: Dr. habil. G. Seubold

\setlength{\parskip}{1cm}

vorgelegt von:\\[0.5cm]Eckhart Arnold\\6.Fachsemester, Magister
% \\Kaiserstr.  57, 53113 Bonn\\Tel.: 0228 / 26 44 01

\setlength{\parskip}{1cm}

Bonn, 1.Mai, 1998

\end{titlepage}

\pagenumbering{roman} \tableofcontents

\newpage

\pagenumbering{arabic}

\setcounter{page}{1}

\section{Einleitung}

In dieser Arbeit soll die Kritik, die Arnold Gehlen in seinem Werk
"`Moral und Hypermoral"'\footnote{Arnold Gehlen: Moral und Hypermoral.
  Eine pluralistische Ethik, Wiesbaden, 5.Aufl., 1986, im folgenden
  zitiert als Gehlen: Hypermoral.} am Humanismus übt, dargestellt und
kritisch durchleuchtet werden. Gehlens Kritik am Humanismus zielt vor
allem auf dessen ethische Seite, wenn er auch die andere Seite des
Humanismus, das humane Ideal als Ziel der Selbsterziehung, als
übermäßige menschliche Selbstbezogenheit ebenfalls
ablehnt.\footnote{Späterer Zusatz (6.2.2006): Die Behauptung, dass
  Gehlen sich gegen die "`humanistische"' Ethik wendet ist
  mißverständlich, da Gehlen selbst wörtlich nur vom
  "`Humanitarismus"' und nicht vom "`Humanismus"' spricht. Zwar greift
  Gehlen in seinem Werk eindeutig die Prinzipen einer humanistischen
  Ethik in dem weiter unten von mir definierten Sinn an. Trotzdem ist
  es zumindest ungenau ihn als Kritiker des "`Humanismus"'
  darzustellen, wie ich es in diser Arbeit getan habe. (Neben einem
  aufmerksamen Leser hatte mich auch mein
  Dozent Günther Seubold damals auf diesen Fehler hingewiesen.) Die
  Gültigkeit meiner Argumente gegen Gehlens Standpunkt bleibt von
  dieser terminologischen Ungenauigkeit abgsehen aber unberührt.}
Deshalb konzentriert sich diese Arbeit auf die Erörterung der
ethischen Fragen, zumal die Diskussion eines Ideals andere Methoden
und Fragestellungen erfordern würde als die Klärung ethischer
Streitfragen.

In einem gewissen Rahmen ist es dabei notwendig, unmittelbar auf einige
Fragen der Ethik einzugehen. Hierbei weicht diese Arbeit von den
"`Vorgaben"' Gehlens ab: Während Gehlen in seinem Werk die Fragen der
Ethik vornehmlich auf einer metatheoretischen Ebene behandelt, werden in
dieser Arbeit die ethischen Fragen unmittelbar angegangen, d.h. im
Vordergrund steht (beispielsweise) die Frage "`Was ist Gerechtigkeit?"'
und {\em nicht} "`Wie entsteht der Begriff der Gerechtigkeit?"'. Diese
Herangehensweise wird zu Beginn des kritischen Teils dieser Arbeit
(Kapitel 3.2) gerechtfertigt.  Ihr liegt die Vorstellung zu Grunde, daß
es nicht die Aufgabe der Philosophie ist, ihre Zeit in Gedanken zu
fassen oder ihre eigene Geschichte zu reflektieren, sondern daß sie die
Antworten auf ganz bestimmte Fragen suchen soll ("`Was kann ich
wissen?"', "`Was soll ich tun?"', "`Was ist der Sinn des Lebens?"'
etc.). Zur Beantwortung dieser Fragen braucht die Philosophie ihre
eigene Geschichte nicht zu kennen; höchstens kann eine Kenntnis der im
Laufe der Geschichte zu diesen Fragen gegebenen Antworten hilfreich
sein.  Deshalb liegt in dieser Arbeit der Akzent auch nicht auf der
historischen Erarbeitung des Ursprungs und der Herkunft von Gehlens
Gedanken sondern auf der Behandlung der ethischen Sachprobleme, die
Gehlens "`ethischer Pluralismus"' aufwirft. Dies spiegelt sich auch in
der Verwendung der Sekundärliteratur wieder. Wenig Gebrauch wurde von
Sekundärliteratur gemacht, die lediglich den Inhalt von Gehlens
Philosophie darstellt oder die geistesgeschichtliche Position von
Gehlens Philosophie aufhellt. Statt dessen wurde häufiger auf Werke
zurückgegriffen, die sich den einzelnen Sachbereichen widmen, die von
Gehlens Ausführungen miterfaßt sind.

Die Arbeit ist so aufgebaut, daß nach einer kurzen Skizze der
Grundpositionen von Gehlens Philosophie zunächst Gehlens kritische
Einwände gegen die humanistische Ethik zusammenhängend dargestellt
werden. Darauf folgt eine eingehende Kritik der einzelnen Argumente
Gehlens. Ausgehend von dieser Kritik wird Gehlens Ansatz einer
pluralistischen Ethik grundsätzlich in Frage gestellt. Da die reine
Kritik niemals ganz überzeugend bleibt, sofern nicht auch positive
Möglichkeiten aufgezeigt werden, wird zum Abschluß der Arbeit umrißhaft
eine humanistische Ethik konstruiert, die die Einwände Gehlens gegen den
Humanismus berücksichtigt, soweit diese berechtigt sind.

\section{Die philosophische Entwicklung Arnold Gehlens}

Arnold Gehlen (geb. 1904 in Leipzig, gest. 1976 in Hamburg) studierte
Philosophie in Köln und Leipzig. Dort promovierte er 1927 bei Hans
Driesch und wurde 1934 Nachfolger auf dessen Lehrstuhl. In seinen
früheren Schriften zeigt sich Gehlen vor allem der Lebensphilosophie und
dem Deutschen Idealismus, insbesondere Fichte, zugeneigt.\footnote{Als
  Beispiel sei hier aus den frühen Schriften (willkürlich)
  herausgegriffen: Arnold Gehlen: Wirklichkeitsbegriff des Idealismus
  (1933), in: Arnold Gehlen: Philosophische Schriften II.  (1933-1938),
  Frankfurt am Main 1980, S.181-198.} Weisen diese Schriften, bei denen,
wie im Bereich der akademischen Philosophie üblich, vor allem andere
Philosophen (und weniger bestimmte philosophische Probleme) im Zentrum
stehen, noch eine relativ große Spannbreite auf, ohne allzu deutlich
schon irgend eine Festlegung erkennen zu lassen, so hat Gehlen in seinem
1940 erschienen Hauptwerk "`Der Mensch"' sein eigentliches Thema mit der
philosophischen Anthropologie gefunden. Gehlens Werk "`Der Mensch. Seine
Natur und seine Stellung in der Welt"' stellt einen vorläufigen
Höhepunkt der im 20.Jahrhundert mit den bahnbrechenden Werken Schelers
und Plessners neu erwachten philosophischen Anthropologie
dar.\footnote{Arnold Gehlen: Der Mensch. Seine Natur und seine Stellung
  in der Welt, Berlin 1940.}  Während Schelers Werk sowohl hinsichtlich
der überwiegend geisteswissenschaftlichen Methode als auch in der
Denkweise und Begriffswahl (Geist-Seele Dualismus, metaphysische
Sonderstellung des Menschen) noch durchaus traditionell gehalten ist,
löst sich Gehlen von solchen überkommenen Bindungen, indem er unter
wissenschaftspragmatischer Umgehung eingefahrener Fragestellungen und
systematischer Heranziehung der wissenschaftlichen Erkenntnisse aus
Biologie und Zoologie ein integrierendes Gesamtbild des Menschen zu
entwerfen versucht, welches die Ergebnisse der verschiedenen humanen
Einzelwissenschaften zusammenführt und philosophisch ausdeutet. Gehlen
legt dabei ein Konzept von Philosophie zu Grunde, das er als
"`empirische Philosophie"' bezeichnet, ein Ansatz für den Schopenhauers
System das Vorbild abgibt. Nach dieser Vorstellung von Philosophie kann
das Allermeiste, was wir über das Wesen des Menschen oder auch das Wesen
der Welt wissen können, nur aus der Erfahrung bzw. einer philosophischen
Deutung der Erfahrung entnommen werden. Ein Rückgriff auf die
Transzendentalphilosophie erschiene demgegenüber wenig hilfreich, denn
was lehrt schon die Transzendentalphilosophie beispielsweise über die
Triebstruktur des Menschen? Was die Philosophie dabei von den
Erfahrungswissenschaften, auf die sie sich stützt, unterscheidet, ist
das Ziel, ein umfassendes Bild zu gewinnen, und dieses Bild
philosophisch zu deuten.\footnote{Trotz der von Gehlen öfters
  wiederholten Beteuerung des empirischen Charakters seiner Theorie, muß
  bezweifelt werden, ob Gehlens Theorie strengeren
  wissenschaftstheoretischen Anforderungen (insbesondere der
  Falsifizierbarkeit) genügen kann. Besonders deutlich wird dies etwa in
  seinem Werk "`Urmensch und Spätkultur"' (Arnold Gehlen: Urmensch und
  Spätkultur. Philosophische Ergebnisse und Aussagen, Wiesbaden 5.Aufl.,
  1986, im folgenden zitiert als Gehlen: Urmensch.), wo Gehlen eine
  Reihe von Einzelbeispielen für "`institutionelles"' und
  "`darstellendes"' Verhalten liefert, aus denen sich jedoch kaum eine
  systematische und an objektiven Kriterien überprüfbare Theorie
  zusammensetzen läßt. (Besonders schwer wiegt hier, daß Gehlen wichtige
  Grundbegriffe wie {\em Institution} und {\em anthropologische
    Kategorie} undefiniert läßt) Vgl. zu den begrifflichen
  Schwierigkeiten dieses Werkes: Alfred Heuß: Gehlens Anthropologie und
  der "`Ursprung"' der Geschichte, in: Helmut Klages / Helmut Quaritsch
  (Hrsg.): Zur geisteswissenschaftlichen Bedeutung Arnold Gehlens,
  Berlin 1994, S.235-353. - Gehlens Anspruch, Empiriker zu sein,
  entspringt daher vielleicht eher dem Zweck der Abgrenzung gegenüber
  metaphysischen Bestimmungen des Menschen (Scheler) und gegenüber der
  Kritischen Theorie (Frankfurter Schule) als eingelösten
  erkenntniskritischen Ansprüchen.  Vgl.  auch Nevil Johnson: Das
  Gehlensche Denken in der angelsächsischen Welt: Überlegungen zu den
  Hindernissen auf dem Wege einer Rezeption, in: H.Klages / H.Quaritsch
  (Hrsg.), a.a.O., S.747-771.} In "`Der Mensch"' deutet Gehlen den
Menschen mit einem von Herder übernommenen Begriff als "`Mängelwesen"',
welches anders als die Tiere nicht an eine bestimmte Umwelt angepaßt
ist, durch die es von seinen Instinkten sicher geleitet wird, sondern
das sich durch Kulturbildung erst seine Umwelt und seine eigene Natur
schaffen muß. In seinen späteren Schriften schreitet Gehlen auf dem Wege
der empirischen Philosophie und Anthropologie fort, wobei er sich
vermehrt auch der Soziologie zuwendet. Er erweitert seine
anthropologische Theorie dabei um die Lehre von den Institutionen, mit
der er die Struktur und Funktion der Kultur als gesellschaftliches
Ordnungssystem zu ergründen versucht.

Der Mensch ist nach Gehlen im Gegensatz zum Tier wesentlich "`handelndes
Wesen"'. Anders als beim Tier nämlich ist das menschliche Verhalten
nicht bereits durch eine artspezifische Instinktstruktur festgelegt. Dem
Mangel an festgelegten Instinkten entspricht körperlich das Fehlen von
spezialisierten Organen, wie Klauen, Pelz oder Giftstacheln, die den
Tieren das Überleben in ihrer Umwelt möglich machen. Wie kann aber der
Mensch als Naturwesen überleben, wenn er weder bei der Nahrungssuche von
Instinkten geleitet wird noch dabei wie die Tiere von einer geeigneten
Organausstattung unterstützt wird? Dies erreicht der Mensch, indem er im
Prozeß der Kulturbildung sich eine Lebensweise schafft und auf Dauer
stellt, die ihm das Bestehen in der Welt ermöglicht. Die Kultur muß beim
Menschen für den Zusammenhang von gattungsmäßig festgelegter
Instinktstruktur und artspezifischer Umwelt, wie er bei den Tieren
besteht, Ersatz schaffen. Dies geschieht dadurch, daß im Prozeß der
Kulturbildung aus der Welt eine faßbare Menge von Bedeutungsgehalten
ausgesondert wird, mit denen sich der Mensch die Welt als nunmehr
verständliche Umwelt zu eigen macht. Zugleich erfahren die flüssigen
Instinkte des Menschen eine Verfestigung zu Verhaltensregulationen.
Diese Verhaltensregulationen funktionieren ähnlich wie die Instinkte
beim Tier nach einem Reiz-Reaktions-Schema, nur können beim Menschen die
Reiz-Reaktionsketten entkoppelt und durch Einschiebung von
"`Phantasmen"' fast beliebig verlängert werden. Dies erklärt auch, warum
der Mensch über Geist verfügt. Der Geist umfaßt die Symbolgehalte, mit
denen der Mensch sich in seiner Umwelt orientiert, und die Phantasmen,
die dem Menschen anstellte äußerer Reize als Handlungsziele vor das
(innere) Auge treten.\footnote{Zu Gehlens Anthropologie vgl. Arnold
  Gehlen: Der Mensch (1.Teilband), Frankfurt am Main 1993, S.3ff.}

Solcherart verfestigte Verhaltensregulationen und Bedeutungsgehalte
we\-rden nach Gehlen in den gesellschaftlichen Institutionen
gespeichert. Dies gilt sowohl für die allergrundlegensten Institutionen,
wie Sprache und Sitten, als auch für darauf aufbauende Institutionen,
wie z.B. die Herrschaftsinstitutionen. Die Institutionen erfüllen eine
mehrfache Funktion für den Menschen. Sie geben Orientierungen vor und
entlasten damit den Einzelnen vom ständigen Improvisationsdruck, sie
leisten eine Abstimmungsfunktion innerhalb der Gesellschaft, d.h. sie
ermöglichen ihren Mitgliedern ihr Verhalten gegenseitig zu deuten und
geben Reaktionsmöglichkeiten vor, von denen jeder Handelnde sicher sein
kann, daß sie vom Anderen verstanden werden. Schließlich speichern die
Institutionen das gesamte Wissen bzw. die Weisheit einer Kultur. Hierzu
gehört nicht nur theoretisches Wissen von der Welt, sondern jede
funktionierende Institution enthält in sich ein wertvolles Wissen
darüber, wie Leben möglich ist - ein Leben, daß sich der von Natur aus
"`unfestgelegte"' Mensch erst erfinden muß, wobei es unzählige Gefahren
des Scheiterns gibt. Der hohe Wert, der jeder Institution schon als
solcher zukommt, wird besonders deutlich, wenn man sich vor Augen hält,
daß die meisten Institution in einem überaus langwierigen und mühsamen
Prozeß der Kulturbildung entstanden sind.\footnote{Zu Gehlens
  Institutionenlehre vgl. Gehlen: Urmensch, a.a.O., S.7-121. - Vgl.
  Friedrich Jonas: Die Institutionenlehre Arnold Gehlens, Tübingen 1966,
  S.43ff. - Vgl.  Gehlen: Hypermoral, S.95-102.}

Welche Konsequenzen ergeben sich aus dieser Lehre?

In Bezug auf die Freiheit des Menschen ergibt sich die Konsequenz, daß
nur die Gattung Mensch frei (oder genauer: {\em unbestimmt}) ist. Für
das Individuum kann es Freiheit nur innerhalb aber niemals jenseits der
Institutionen oder gar gegen die Institutionen geben. Der Versuch,
Freiheit außerhalb der Institutionen zu verwirklichen, wäre so absurd,
als wollte man eine Geschichte erzählen, ohne sich der Sprache zu
bedienen. Ein sinnvolles und erfülltes Leben kann es nur im Konformismus
zu den Institutionen geben. Emanzipatorische Bestrebungen sind ebenso
wie die aufklärerische Kritik an den Institutionen ein riskantes
Unterfangen, denn leicht (und gerne) werden Institutionen zerstört, aber
schwer sind sie zu errichten. Gehlens Philosophie mündet so in einen
formalen Konservativismus, der das Bestehende um seiner selbst willen
heiligt.

Dieser Konservativismus ist auch für Gehlens politischen Standpunkt
charakteristisch. Es gibt in seinen späteren Veröffentlichungen keine
Hinweise darauf, daß Gehlen die Demokratie direkt abgelehnt hätte.
Allerdings sah er Ende der 60er Jahre die Bundesrepublik einem
gefährlichen Verfall der öffentlichen Autorität entgegen gehen. Darüber
hinaus ist das, was Gehlen in "`Moral und Hypermoral"' zu dem Thema
Staat äußert, nur unter Einschränkungen mit den Grundsätzen der
liberalen Demokratie zu vereinbaren. Während der Zeit des Hitlerregimes
war Gehlen ein begeisterter Befürworter des Nationalsozialismus. Er war
seit dem 1.Mai 1933 Mitglied der NSDAP, seine aktive Mitarbeit
beschränkte sich abgesehen von der philosophischen und ideologischen
Unterstützung, die er dem Regime leistete, auf eine zwei Semester
andauernde Tätigkeit als Dozentenbundführer. Allerdings hatte Gehlen
auch gute Gründe die Nazis hochzuschätzen, denn seine steile Karriere in
dieser Zeit, die ihn auf glänzende Lehrstühle in Königsberg (1936) und
Wien (1940) trug, verdankte er der Protegierung durch die
Nazi-Administration.\footnote{Vgl. das Nachwort des Herausgebers
  Karl-Siegbert Rehberg, in: Gehlen: Der Mensch (2.Teilband),
  Frankfurt/M 1993, S.751-786 (S.753-755).} Überhaupt eröffneten die
Nazis damals aufstrebenden jungen Akademikern viele attraktive Chancen,
indem sie verdiente jüdische und oppositionelle Wissenschaftler von den
Universitäten jagten. In seinen Schriften aus dieser Zeit
(einschließlich der ersten Auflage von "`Der Mensch"'\footnote{Vgl.
  Gehlen: Mensch 1940, a.a.O., S.364ff., S.427ff.})  finden sich
vielfältige Anklänge an den Nationalsozialismus.\footnote{Vgl. Arnold
  Gehlen: Der Idealismus und die Gegenwart (1935), in: Arnold Gehlen:
  Gesamtausgabe. Band 2.  Philosophische Schriften II. (1933-1938),
  Frankfurt am Main 1980, S.347-357. oder Arnold Gehlen: Rede über
  Fichte (1938), in: Ebda., S.385-396.} Eine Zeitlang hat Gehlen wohl
sogar mit dem Gedanken gespielt, eine "`nationalsozialistische
Philosophie"' zu schaffen, doch ein 1935 entstandenes Fragment, in
welchem Gehlen sich in einer Art philosophischem Rassismus versucht,
blieb in seiner Schublade liegen.\footnote{Vgl. den Text dieses
  Fragmentes in den Anmerkungen des Herausgebers zu Gehlen: Der Mensch
  (2.Teilband), Frankfurt/M 1993, S.790-795.} Gehlens Anthropologie ist
mit dem Rassismus unvereinbar, seine Institutionenlehre enthält jedoch
implizit eine starke Option für die geschlossene Gesellschaft. Und auch
wenn Gehlen später dem totalitären Anspruch, dem er unter dem
Nationalsozialismus so zugeneigt war, grundsätzlich abhold geworden ist,
was seinen Ausdruck unter anderem in dem in "`Moral und Hypermoral"'
konstruierten ethischen Pluralismus findet, so behielt Gehlen viele
seiner nationalsozialistisch imprägnierten Grundüberzeugungen bei.

Vor allem hat Gehlen sich nie von der Überzeugung trennen können, daß
ein Leben des Menschen ohne Mythos möglich sei. Man könnte Gehlens
Haltung in dieser Frage als einen etwas mutwilligen Reflex auf den
Nihilismus bezeichnen.  Ausgehend von Hegel und Nietzsche hält Gehlen
die Religion in ihrer ursprünglichen Form für überlebt, aber
gleichzeitig zweifelt er nicht daran, daß der Mensch solcher absoluten
und absolut verpflichtenden Weltdeutungen, wie Religion oder Mythos sie
bieten, unbedingt bedarf.\footnote{In explizitem Bezug auf die Religion
  schreibt Gehlen in der ersten Auflage von "`Der Mensch"', "`daß wir
  über den Sinn des Daseins oder des Lebens nichts aussagen können, daß
  aber einen solchen Sinn zu unterstellen notwendig, nicht nur erlaubt
  ist, weil das Leben zur Lösung seiner uns unbekannten Aufgabe des
  Bewußtseins, des Sinnbereichs selber, bedarf."' (Gehlen: Mensch 1940,
  S.466.) - Hier taucht auch der - auch aus dem völkischen Denken und
  ebenso der Nazi-Ideologie vertraute - Topos auf, daß dem Menschen eine
  biologische Lebensaufgabe von ethisch verpflichtendem Charakter
  vorgeschrieben ist. (Vgl. dazu auch das Kapitel "`Urphantasie"' in
  demselben Werk.) - Gegenüber den in diesem Kapitel geäußerten
  Auffassungen wird man wohl festhalten dürfen, daß die Natur (und
  ebenso der liebe Gott) ihre tieferen (biologischen) Absichten schon
  selber verfolgen werden und können, daß aber der Mensch sich zu nichts
  anderem verpflichtet fühlen muß, als was er auch (bei klarem
  Bewußtsein) einsehen kann. (Dieses Prinzip markiert nebenbei bemerkt
  den Unterschied zwischen dumpfer Mythologie und
  Offenbarungsreligion.)} In diesem Punkt unterscheidet sich Gehlens
anthropologischer Pessimismus scharf von dem ansonsten verwandten
Pessimismus Sigmund Freuds oder Max Webers. Während Freud und Max Weber
den Wunsch nach religiöser Lebensdeutung für ein starkes menschliches
Bedürfnis halten, sind sie dennoch fest davon überzeugt, daß ein
rational geleitetes Leben möglich und auch wünschenswerter
ist.\footnote{Vgl Max Weber: Wissenschaft als Beruf, Stuttgart 1996,
  S.16ff. - Vgl. Sigmund Freud: Massenpsychologie und Ich-Analyse / Das
  Ende einer Illusion, Frankfurt /M 1993, S.107ff.}  Gehlen fürchtet
dagegen, daß der Mensch ohne Mythos auf sich selbst zurückgeworfen und
so zu einem haltlosen und sinnentleerten Leben verdammt wird. Von einem
{\em mutwilligen} Reflex auf den Nihilismus kann man im Falle Gehlens
deshalb reden, weil Gehlens Theorie die Falschheit (oder zumindest die
Relativität) aller Mythen implizit voraussetzt, aber Gehlen trotzdem die
Unterwerfung unter den Mythos fordert.

Man kann genaugenommen nur dann behaupten, eine Theorie verstanden zu
haben, wenn man, außer daß man sie mit eigenen Worten wiedergeben kann,
in der Lage ist, entweder Kritik an ihr zu üben, oder zu begründen,
warum diese Theorie richtig ist. Zur Darstellung einer Theorie gehört
daher immer auch die Berücksichtigung möglicher Einwände. Mögliche
kritische Einwände sollen an dieser Stelle wenigstens kurz angerissen
werden.

Gegen Gehlens anthropologische Theorie könnte man geneigt sein, den
Vorwurf des biologischen Reduktionismus zu erheben: Wenn Gehlen den
Menschen gegenüber dem Tier als Mängelwesen bestimmt und gleichzeitig
das spezifisch Menschliche mit möglichst ähnlichen Begriffen erklärt,
wie sie die Biologie zur Erklärung des Verhaltens von Tieren gebraucht
(z.B.  "`flüssige Instinkte"' statt "`Instinkte"'), so macht Gehlen
damit das Tier zum Maßstab von Lebewesen bzw. von in der Welt möglichem
Leben überhaupt, d.h.  letztenendes wird das Tier zum Maßstab des
Menschen, denn der Mensch kann bei Gehlen nur dadurch (Über-)leben, daß
er sich künstlich Lebensbedingungen, d.h. einen Zusammenhang von Umwelt
und darauf abgestimmten "`Instinkten"', schafft, die denen des Tieres
ähneln. Diesem möglichen Einwand gegenüber ist jedoch festzuhalten, daß
das Verfahren der Reduktion in der Wissenschaft geläufig und solange
legitim ist, wie Kriterien für die Überprüfbarkeit dieser Reduktion
angebbar sind. Es besteht natürlich die Gefahr, daß z.B. die Annahme,
die Motive menschlichen Handelns seien im instinktnahen Bereich
anzusiedeln, zu einem Dogma erstarrt, welches in irgendeiner Weise immer
recht behält.

Bei Gehlens Institutionenlehre stellt sich das Problem, daß sie für
archaische Gesellschaften gut nachvollziehbar ist, aber nicht ohne
weiteres in modernen Gesellschaften. Übt man nun aus der
Institutionenlehre heraus eine Kritik an der modernen Gesellschaft, so
müßte zuvor die Frage untersucht werden, ob die Institutionenlehre auf
die moderne Gesellschaft, deren Funktionieren vielleicht schon auf
neuen, theoretisch noch nicht erfaßten anthropologischen Prinzipien
beruht, überhaupt anwendbar ist. Es ist im Zusammenhang mit Gehlens
Gesellschaftskritik auch zu erwähnen, daß ein gewisser Zug der
vereinfachenden Popularisierung in Gehlens Philosophie bereits angelegt
ist. Dies wird daran deutlich, daß viele der Schlüsselbegriffe Gehlens
("`Entlastung"', "`Reizüberflutung"', "`Institutionen"' usw.) schon bei
ihm selbst eine Doppelbedeutung annehmen, einmal als Terminus technicus
innerhalb der Theorie und zugleich als Begriffe, die sich auch auf die
Alltagserfahrung beziehen lassen, und denen dann mancherlei Sinn
beigelegt werden kann.

\section{Gehlens Humanismuskritik in "`Moral und Hypermoral"'}

Im Jahre 1968 erschien Gehlens Schrift "`Moral und Hypermoral. Eine
pluralistische Ethik"'.\footnote{Arnold Gehlen: Moral und Hypermoral.
  Eine pluralistische Ethik, Wiesbaden 5.Aufl., 1986.} Gehlen
beabsichtigt darin, die anthropologischen Grundlagen der Ethik als eines
spezifisch menschlichen Phänomens darzulegen und mehrere typische
Ethosformen, darunter insbesondere die des "`Humanitarismus"' als der
"`zur ethischen Pflicht gemachte[n] unterschiedslose[n]
Menschenliebe"'\footnote{Ebda., S.79.}, sowohl anthropologisch
abzuleiten, als auch in ihrer kulturellen Funktion und Wirkung zu
deuten. Das Resultat dieser Bemühungen ist, wie der Untertitel seines
Werkes sagt, eine "`pluralistische Ethik"'. Mit dieser Bezeichnung ist
weder eine Ethik der besonderen Toleranz gegenüber unterschiedlichen
Lebensauf\/fassungen gemeint, noch versteht Gehlen darunter, daß es etwa
unterschiedliche aber gleichermaßen legitime Moralauf\/fassungen gibt.
Vielmehr bedeutet "`ethischer Pluralismus"'\footnote{Vgl. ebda., S.10.},
daß es verschiedene Ethosformen mit unterschiedlichem anthropologischen
Ursprung gibt, die jedoch nur jeweils für einen bestimmten Bereich des
menschlichen Handelns gültig sind. So ist etwa das "`Familienethos"' nur
für die Familie gültig, keineswegs jedoch für den Staat und die Politik,
denn dort gelten die "`Staatstugenden"'. Die Grenzen zwischen den
Gültigkeitsbereichen dieser Ethosformen sind unscharf, so daß es
zwischen verschiedenen Ethosformen immer wieder zu Konflikten kommen
kann, die theoretisch unauflösbar sind, da nach Gehlens Auffassung kein
Ethos grundsätzlich die Vorherrschaft über ein anderes beanspruchen
darf. Im Alltagsleben werden diese Gegensätze in der Regel im dort
herrschenden "`Durcheinander, mittlerer
Tugendhaftigkeit"'\footnote{Ebda., S.26.} durch moralische Inkonsequenz
ausgeglichen. Für problematisch, um nicht zu sagen gemeingefährlich,
Gehlen, wenn ein bestimmtes Ethos mit dem Anspruch der universellen
Gültigkeit auf alle Bereiche des menschlichen Lebens ausgedehnt wird, so
wie sich dies seiner Ansicht nach in der Gegenwart mit dem
"`Humanitarismus"' vollzieht. In dieser Ausweitung sieht Gehlen nicht
nur eine Radikalisierung, die Aggressionen freisetzt, sondern auch eine
Gefahr für die Überlebensfähigkeit einer Gesellschaft, da nun ein Ethos
gesellschaftliche Bereiche reguliert, für die es sachlich unpassend
ist.\footnote{Vgl. ebda., S.70ff.}

Abgesehen von dieser philosophisch-anthropologischen Zielsetzung
verfolgt Gehlen mit diesem Werk aber auch die Absicht einer
polemischen Zeitkritik.  Der Leser bekommt dabei sehr bald den
Eindruck, daß gegenüber dieser Absicht die anthropologische
Untersuchung nur beiläufig durchgeführt wird und eher als Vorwand
dient zu einem gewaltigen Rundumschlag, in welchem Gehlen von der
Antibabypille\footnote{Vgl. ebda., S.83.} bis zum Zensurverbot des
Grundgesetzes\footnote{Vgl. ebda., S.151. - Gehlen nimmt hier
irrtümlich an, daß das Zensurverbot die Verantwortlichkeit des Autors
ausschließt. Dies gilt jedoch noch nicht einmal im rechtlichen Sinne,
da der Autor nachträglich immer noch zur Verantwortung gezogen werden
kann, etwa durch eine Verleumdungs- oder Beleidigungsklage. Unter
Zensur versteht man im rechtlichen Sinne ausschließlich die Kontrolle
einer Schrift {\em vor} ihrem Erscheinen mit der möglichen Folge eines
Erscheinungsverbotes. Dies allein wird durch Art. 5 GG Abs. 1 Satz 3
ausgeschlossen. Abgesehen von der rechtlichen Verantwortung muß auch
der Journalist oder Buchautor ähnlich wie der Fabrikant auf dem Markt
bestehen. Sein Handeln ist also nicht folgenlos, wie Gehlen dies
notorisch unterstellt. Einleuchtend ist jedoch, daß ein politischer
Autor in der Regel nicht in die Situation gerät, die Folgen der
Verwirklichung seiner Empfehlungen tragen zu müssen. In diesem Sinne
wird in der Tat eine möglicherweise vorhandene Leichtfertigkeit
seinerseits folgenlos bleiben. Dasselbe gilt auch für Philosophen, die
Bücher über Ethik verfassen.} alle möglichen Gegenwartserscheinungen
der politischen und gesellschaftlichen Kultur der Bundesrepublik in
den ausgehenden 60er Jahren geißelt.

\subsection{Der Begriff des "`Humanitarismus"' und Gehlens Kritik an der humanistischen Ethik}

In dieser Arbeit wird, wie bereits im Vorwort erwähnt, Gehlens Kritik am
"`Humanitarismus"' als ein Angriff auf die Ethik des Humanismus verstanden.
Dazu muß zunächst untersucht werden, ob sich Gehlens Begriff des
"`Humanitarismus"' überhaupt mit der humanistischen Ethik deckt, oder ob
Gehlen nicht eine Übersteigerungsform des Humanismus kritisiert, die auch
kein überzeugter Humanist ernsthaft vertreten würde.

Gehlen definiert im 5.Kapitel seines Werkes "`Humanitarismus"' als "`die zur
ethischen Pflicht gemachte unterschiedslose Menschenliebe"'.\footnote{Ebda.,
S.79.} Mit "`unterschiedslos"' ist hierbei gemeint, daß die Menschenliebe
sich auf die gesamte Menschheit erstrecken soll und damit "`unterschiedslos"'
sowohl Angehörigen des eigenen Stammes, der eigenen Familie oder der eigenen
Nation als auch "`fremden"' Menschen, die nicht zu diesen Nahgruppen gehören,
zukommt. Weiterhin soll sich die "`unterschiedslose Menschenliebe"' auf die
Angehörigen aller sozialer Klassen, auf Sklaven gleichermaßen wie auf
"`Patrizier"' beziehen.\footnote{Vgl. dazu besonders Gehlens Ausführungen zum
antiken Ursprung des "`Humanitarismus"' in den ersten beiden Kapiteln des
Werkes, worin diese Bedeutungselemente von "`Humanitarismus"' deutlich
auszumachen sind, ebda., S.13-36. Weiterhin geht der universalistische
Charakter, den Gehlen im "`Humanitarismus"' feststellt, aus seiner
anthropologischen Erklärung des Humanitarismus als eines "`elargierten"'
Familienethos hervor. - Vgl. ebda., S.83ff.} Nun gehört aber gerade der
Universalismus wesentlich zur humanistischen Ethik. Unter humanistischer
Ethik verstehe ich dabei die Auffassung, daß {\em jedem} Menschen als
Individuum ein besonderer Wert (Menschenwürde) zukommt, und daß es keinen
Wert gibt, der über dem Wert des Individuums steht. Solche Werte, die mit der
Menschenwürde konkurrieren könnten, wären beispielsweise Nation, Geschichte,
Ehre etc. Diese Werte sind nach der humanistischen Ethik bloß sekundär in dem
Sinne, daß sie entweder mittelbare Werte sind, denen gegenüber das Wohl der
einzelnen Menschen als Endzweck zu betrachten ist, oder daß sie
untergeordnete Werte sind, denen im Falle eines Wertekonfliktes die
Menschlichkeit unbedingt vorhergeht. Dies ist der Kern jeder humanistischen
Ethik. Ein Humanismus, der sich nicht auf die gesamte Menschheit bezieht,
sondern an den Grenzen der Nation haltmacht oder nur für die eigene Sippe
gilt, wäre dagegen eine {\em contradictio in adjecto}. Genau dieser
universalistische Zug der humanistischen Ethik und die Überordnung der
Menschenliebe über alle anderen Werte sind es jedoch, die Gehlen unter der
Bezeichnung "`Humanitarismus"' als eine Form von "`Moralhypertrophie"'
ausdrücklich ablehnt. Gehlens Kritik des "`Humanitarismus"' muß daher in der
Tat als ein Angriff auf die humanistische Ethik verstanden werden.

Welches sind nun die Argumente mit denen Arnold Gehlen gegen den Humanismus
zu Felde zieht? Es lassen sich in Gehlens Werk drei
Hauptargumentationsstränge ausmachen:

\begin{enumerate}
\item Ein {\em historischer} Argumentationsstrang: Gehlen zufolge zeigt sich
  der "`Humanitarismus"' zuerst in der griechischen und römischen Antike und
  zwar als typische Erscheinung der Verfallsperioden dieser Epochen. Die
  humanistische Ethik erscheint ihm daher als ein Symptom und gleichzeitig
  eine Ursache von Dekadenz.
\item Ein {\em politischer} Argumentationsstrang: Die humanistische Ethik ist
  nach Gehlen mit machtpolitischen Notwendigkeiten unvereinbar. Dringt sie in
  die Politik ein, so gefährdet sie den Selbsterhalt von Staat und Nation.
\item Ein {\em anthropologischer} Argumentationsstrang: Das Ethos der
  Menschenliebe leitet Gehlen aus den biologischen Schonungs- und
  Liebesinstinkten innerhalb der Familie als natürlicher Lebensgemeinschaft
  her. Durch "`Instinktelargierung"', einer spezifisch menschlichen
  biologischen Eigenschaft, wird dieser Instinkt nach und nach auf größere
  Gruppen ausgedehnt, bis hin zu abstrakten Gruppen, deren Mitglieder
  untereinander nicht mehr persönlich bekannt sind, wie der Nation und
  schließlich der gesamten Menschheit. Da es, anthropologisch gesehen, noch
  andere Quellen der Moral gibt (Gegenseitigkeit, vitale Werte,
  Institutionen\footnote{Vgl. ebda., S.47.}), die als "`Sozialregulationen"'
  nicht minder lebensnotwendig sind, darf die humanistische Ethik keine
  Alleingültigkeit beanspruchen.
\end{enumerate}

Die oben kurz skizzierten Argumente Gehlens sollen nun im einzelnen
ausgeführt werden.

\subsubsection{Das historische Argument: Humanismus als Symptom der Dekadenz}

Historisch betrachtet hängt für Gehlen die Entstehung und Ausbreitung des
"`Humanitarismus"' mit dem Zusammenbruch der griechischen Stadtstaaten in der
Spätantike und der Entstehung von zentralistisch regierten,
multiethnischen Großreichen wie dem Alexanderreich zusammen. Später wurde
er dann in das römische Kaiserreich exportiert.\footnote{Vgl. ebda., S.80f.}
Der "`Humanitarismus"' wurde dabei nach Gehlens Auffassung von einem genau
"`angebbaren"' Personenkreis, nämlich den kynischen und stoischen
Philosophen, erfunden und erfüllte innerhalb dieses historischen Prozesses
ganz bestimmte ideologische Funktionen. Auch haften dem
"`Humanitarismus"' auf Grund seiner Entstehungsumstände gewisse typische
Charakterzüge an.

Die Funktionen oder die Zwecke des "`Humanitarismus"' sind dabei sowohl
individueller als auch politischer Natur. Für das Individuum dient der
"`Humanitarismus"' der kompensatorischen Auffüllung einer durch den
Zusammenbruch des Staates aufgerissenen emotionalen Lücke mit
"`verallgemeinerten Tugenden privater Herkunft, wie Wohlwollen,
Hilfsbereitschaft usw."'.\footnote{Ebda., S.24.} Einher damit geht
insbesondere bei den Kynikern ein gehöriges Maß an Primitivisierung, welches
gegenüber den Anforderungen, die die gesellschaftlichen Institutionen im
intakten Staatswesen an den Einzelnen stellen, zunächst als entlastend
empfunden wird.\footnote{Vgl. ebda., S.15.} Gehlens Deutung dieser
Erscheinungen ist aus seiner Institutionenlehre abgeleitet, nach der die
Institutionen tief ins Verhalten und Gemütsleben des Einzelnen eingreifen und
zugleich, ungeachtet ihrer primären Entlastungsfunktion, ihrerseits eine
Belastung darstellen, von der sich der Mensch wiederum entlasten möchte. Da
weiterhin höhere Leistungen des Menschen nur auf Grundlage und im Rahmen von
Institutionen erbracht werden können, die dafür die Symbolgehalte vorgeben,
bedeutet der Verlust oder der Zusammenbruch von Institutionen (egal welcher
Institutionen) stets einen Rückfall in die Primitivität. Abgesehen von den
bisher erwähnten Funktionen erfüllt der "`Humanitarismus"' speziell für seine
philosophischen Vertreter die Aufgabe ihnen Ansehen, Einfluß und Geltung zu
verschaffen. Sie verfolgen damit, wie Gehlen unterstellt, eine Strategie der
"`Eroberung der Eroberer"', indem sie - gewissermaßen unter Verrat ihres
geschlagen Vaterlandes - sich den neuen Herrschern der Großreiche andienen
und durch die Predigt von Güte und Menschenliebe versuchen, Schonung für sich
selbst zu erwirken. Gehlen versäumt es nicht, in diesem Zusammenhang darauf
hinzuweisen, daß einige führende Kyniker davongelaufene Sklaven oder Menschen
von einer ähnlich verachteten sozialen Herkunft sind.\footnote{Vgl. Ebda.,
S.30f.}

Neben diesen für einzelne Individuen nützlichen Eigenschaften erfüllt der
"`Humanitarismus"' nach Gehlen aber auch eine eminent politische Funktion: Er
stützt mit seinem Weltbürgerethos die Bildung von nationenübergreifenden,
zentralistisch regierten Vielvölkerstaaten ideologisch ab. Dies erreicht der
"`Humanitarismus"' in dem er die patriotischen Werte verdrängt und zugleich
durch seinen eher der Privatsphäre entstammenden Wertekanon die Folgsamkeit
der Untertanen sichert. Letzteres hängt damit zusammen, daß eine politische
(Widerstands-)Bewegung, um zugkräftig zu sein, stets als Kollektivbewegung
auftreten muß, wobei ein rein individualistischer Wertekanon hinderlich ist.
Dennoch war das humanitäre Ideal in der Antike durchaus in der Lage
Sklavenaufstände zu motivieren.\footnote{Vgl. ebda., S.35.}

Obwohl das Ethos der Humanität solcherart für bestimmte Systeme einen
stärkenden und stabilisierenden Einfluß gewinnt, ist und bleibt es nach der
Auffassung Gehlens andererseits ein Merkmal der Dekadenz - eine Deutung, die
Gehlen von George Sorel übernimmt.\footnote{Vgl. ebda., S.81f. - Gehlen
erklärt zwar nicht ausdrücklich sein Einverständnis mit Sorels These, doch
Gehlen legt dem Leser mit seiner suggestiven Beschreibung der Antike genau
diese These ziemlich nahe.} Als Dekadenzmoral lähmt der Humanitätsgedanke den
Willen und hemmt die Bereitschaft zur Ausübung notwendiger Gewalt. Diesen
Vorgang skizziert Gehlen am Beispiel des römischen Kaiserreiches, das seiner
Meinung nach durch seine humanitäre Großzügigkeit in eine innere und äußere
Krise geriet: "`Die Eroberer und Feldherren waren Verirrungen auf dem Wege
zum wahrhaft Guten - hundert Jahre später allerdings mußte Aurelian gegen die
Barbaren schon die Hauptstadt selber ummauern, und von geordneten Finanzen
war keine Rede mehr..."'.\footnote{Vgl. ebda., S.81.}

Alle diese Einsichten, die Gehlen an einer Betrachtung der antiken Geschichte
über den Charakter des humanitären Ethos gewonnen hat, überträgt Gehlen in
einem zweiten Schritt auf die Gegenwart. Auch heute "`sind nun bei uns unter
dem Einfluß der beispiellosen Niederlage und nach der Zerstörung aller inneren
Reserven die Individuen auf ihre Privatinteressen und deren kurzfristige
Horizonte zurückgefallen. Was sie dort finden, ist die egalitäre Moral der
Familie..."'.\footnote{Ebda., S.143.} Vorangetrieben wurde der Prozeß der
Zersetzung des Staatsethos durch den "`Humanitarismus"' bereits einige Jahre
vor der Kriegsniederlage durch so skrupellose Intellektuelle wie Karl Barth,
dessen 1938 erschienene Schrift "`Rechtfertigung und Recht"' Gehlen jedoch zu
durchschauen meint: "`Wer hier mit dem Anspruch der höheren Autorität spricht,
ist klar, hier will das Ethos des Humanitarismus das des Patriotismus
verschlingen..."'.\footnote{Ebda., S.134.} In den 60er Jahren sind es wiederum
evangelische Theologen, die anstatt sich um das Jenseits zu kümmern,
zersetzende humanistische Lehren für das Diesseits in die Welt
setzen.\footnote{Vgl. ebda., S.121-139.}

Insgesamt erscheint also das humanitäre Ethos in Gehlens historischer
Perspektive als: 1. das Resultat eines politischen Zusammenbruchs mit
einhergehendem Kulturverlust, 2. ein Mittel ideologisch zweckmäßiger
Fremdtäuschung zum Vorteil bestimmter (Herrschafts-)Gruppen mit
gleichzeitiger öffentlicher Geltungssteigerung der Intellektuellen, 3. als
Merkmal und Ursache von Dekadenz.

\subsubsection{Das politische Argument: Humanismus als Gefahr für die Staatstugenden}

Bereits bei der Darstellung der historischen Argumentation Gehlens trat zu
Tage, daß, nach Gehlens Auf\/fassung, die humanistische Ethik in einem
Gegensatz zum Patriotismus steht. Dies ist einer der schärfsten Vorwürfe
Gehlens gegen das humanitäre Ethos, daß es die vaterländischen Tugenden
bzw. "`Staatstugenden"' unterminiere und damit die Existenz des Staates als
solchen gefährde. Was ist nun Gehlens Auf\/fassung vom Wesen der Politik
und weshalb gerät die Vaterlandsliebe mit der Menschenliebe in Konflikt?

Gehlens Vorstellung vom Wesen der Politik kann als ein übersteigerter
politischer Realismus bezeichnet werden. Der politische Realismus behauptet,
daß ein Staat Sicherheit nur durch Macht gewinnen kann. Da der Kampf um Macht
ein Nullsummenspiel ist, bei dem der Gewinn des Einen stets nur durch
Verluste eines Anderen erzielt werden kann, darf das Machtstreben um der
Sicherheit willen nie aufhören, und es findet seine theoretische Grenze erst
in der Weltherrschaft.\footnote{Klassiker des politischen Realismus ist nach
wie vor: Hans J. Morgenthau: Macht und Frieden. Grundlegung einer Theorie der
internationalen Politik, Gütersloh 1963. Vgl. ebda., S.49ff. - Etwas
vermittelnder, aber grundsätzlich in derselben Richtung: John H. Herz:
Politischer Realismus und politischer Idealismus. Eine Untersuchung von
Theorie und Wirklichkeit, Meisenheim am Glan 1959. Vgl. ebda., S.32ff.} Diese
Überzeugungen teilt auch Arnold Gehlen.\footnote{Vgl. Gehlen: Hypermoral,
a.a.O., S.103ff.} Was Gehlens Auf\/fassungen zu einer Übersteigerung des
politischen Realismus macht ist, daß die äußere und innere Souveränität des
Staates jenseits jener Zweckbestimmung für Gehlen einen Selbstwert
ausmachen. Hieraus erwächst dem Staat eine eigene ethische Würde, die sich
nur durch eine besonders hohe Autorität des Staates sowohl nach außen als
auch nach innen hin aufrecht erhalten läßt. Dementsprechend müssen auch
sozialstaatliche Forderungen nicht bloß der begrenzten Ressourcen wegen für
problematisch gelten, sondern sie erscheinen als ein Angriff auf die
Autorität des Staates an sich.\footnote{Vgl. ebda., S.110.} Hierfür ist nun
nach Gehlens Auffassung das Humanitätsethos deshalb mitverantwortlich, weil
es seit der Aufklärung eine unzertrennliche Verbindung mit dem
"`Masseneudaimonismus"' eingegangen ist. Unter "`Masseneudaimonismus"'
versteht Gehlen die Forderung, daß jeder Mensch ein Anrecht auf ein Minimum
an materiellem Wohlstand haben solle, wobei sich von selbst versteht,
daß diese Minimalforderung mit zunehmendem Gesamtwohlstand auch immer weiter
nach oben geschraubt wird. "`Masseneudaimonismus"' und "`Humanitarismus"'
gehen in Gehlens Augen eine geradezu teuflische Verbindung ein, worüber sich
seine Empörung in Passagen wie der folgenden Luft macht: "`Im Bunde mit dem
Masseneudaimonismus wird die Unwiderstehlichkeit dieses Ethos [des
,Humanitarismus', E.A.] verständlich, das mit der Hebung des Lebensstandards
aller Menschen und ihrer gegenseitigen friedlichen Anerkennung zugleich auf
eine globale Endogamie zusteuert, so daß man zu der Überzeugung kommt, wir
hätten hier den Ausdruck oder die Ideologie der steilen Zunahme der
Weltbevölkerung vor uns - die rasende Multiplikation des Vermehrungsprozesses
gibt sich damit moralisch grünes Licht."'.\footnote{Ebda., S.83.}

Doch nicht nur durch die entstehende Anspruchsmentalität und die damit
verbundene Aufweichung der Autorität des Staates geht nach Gehlens
Überzeugung eine politische Gefahr vom "`Humanitarismus"' aus. Ein
weiteres Problem besteht darin, daß die staatliche Politik ein Ethos
erfordert, in dem die Kategorien Sicherheit und Ehre eine hervorgehobene
Rolle spielen.\footnote{Vgl. ebda., S. 111.} Da diese Kategorien dem
Humanitätsethos wesensfremd sind, gefährden humanitäre Forderungen in
der Politik die machtpolitische Schlagkräftigkeit des Staates. Die
unweigerliche Folge davon ist ein Verlust der
Selbstbestimmungsmöglichkeiten eines Volkes und ein Herabsinken des
Nationalstolzes, das auch jeden einzelnen Bürger schmerzlich berühren
muß. Letzteres ist für Gehlen insbesondere deshalb bedauerlich, da die
Ehre der Nation in seinen Augen einen unmittelbaren ethischen Wert
darstellt.\footnote{Vgl. ebda., S.116}

Es stellt sich die Frage, wie Gehlen zu der Auffassung kommt, daß der Staat
nicht nur bestimmte Tugenden induziere, sondern auch einen eigenen ethischen
Wert hat. Dies wird wiederum verständlich aus Gehlens Institutionenlehre.
Institutionen vermitteln danach eine eigene Sollgeltung. Da sich
Institutionen nach Gehlens Auffassung niemals ganz allein zweckrational
begründen lassen (dies anzunehmen käme für Gehlen einem naiven Glauben in
die primäre Vernünftigkeit der menschlichen Natur gleich), so treten
Institutionen dem Menschen als etwas Absolutes, Gültigkeit aus eigenem Recht
beanspruchendes gegenüber. Die Berechtigung des Staates geht deshalb über
seinen rationalen Existenzzweck, nach innen und nach außen Sicherheit zu
gewähren, hinaus.

\subsubsection{Das anthropologische Argument: Humanismus als überdehntes Familienethos}

Die Ethik zerfällt nach der Grundthese von Gehlens ethischem Pluralismus in
verschiedene Ethosformen, die er als "`Sozial-Regulationen"' bezeichnet, da
sie dem Zweck dienen, das Zusammenleben des Menschen in arterhaltender Weise
zu regulieren. Die verschiedenen Ethosformen alias "`Sozial-Regulationen"'
gehen auf unterschiedliche anthropologische "`Instinktresiduen"' zurück.
Gehlen zählt davon vier verschiedene auf: Gegenseitigkeit, physiologische
Tugenden, Familienethos, Institutionenethos.\footnote{Vgl. ebda., S.47.}
Der "`Humanitarismus"' hat sich nach Gehlen aus dem Familienethos heraus
entwickelt. Das Familien- und Sippenethos kommt als Grundlage des
"`Humanitarismus"' deshalb allein in Frage, weil nur dieses Ethos die
entsprechenden Liebes- und Agressionshemmungsinstinkte bereitstellt.
Ursprünglich bezogen sich diese Instinkte nur auf die (sinnlich-anschaulich
bekannten) Familienmitglieder. Allerdings beobachtet man schon in
vergleichsweise "`primitiven"' Gesellschaften die Ausdehnung dieses Ethos auf
die weitere Verwandtschaft, wobei "`Verwandtschaft"' nicht unbedingt
Blutsverwandtschaft bedeuteten muß, sondern durch kulturell sehr
verschiedene Verwandtschaftssysteme festgelegt ist. Bereits auf dieser Stufe
ist also eine Erweiterung ("`Elargierung"') des Ethos auf eine schließlich
nur noch abstrakt gegebene Gemeinschaft angelegt, bei der nicht mehr jedes
Mitglied alle anderen Gemeinschaftsmitglieder kennt. Der nächste Schritt
der Elargierung des Familienethos' besteht in seiner Ausdehnung auf
Königreiche, wobei der König als Vaterfigur die noch immer nötige
familiäre Bezugsperson darstellt. Auf dieser Ebene tritt bereits jener
klassische Konflikt zwischen Familie bzw. Sippe und Staat auf, der in einigen
griechischen Tragödien, wie z.B. der "`Antigone"' des Sophokles, seinen
Niederschlag gefunden hat. In den nicht mehr archaischen Monarchien -
Gehlen wählt hier als Beispiel den französischen Absolutismus - kommt noch
eine weitere Konfliktlinie hinzu, nämlich die zwischen dem nunmehr auf den
Staat bezogenen Familienethos, das im Ideal auf Liebe und Treue zum Monarchen
einerseits und väterlicher Fürsorge des Monarchen andererseits
beruht, und dem neuen Institutionenethos der Institutionen rationaler
Herrschaft wie der Verwaltung und dem Heer.\footnote{Vgl. ebda., S.86-92.}

Der "`Humanitarismus"' stellt nun nach Gehlen den letzten möglichen Schritt
einer Erweiterung dar, nämlich die Erweiterung des Familienethos auf die
ganze Menschheit. Gehlen sieht dies überaus kritisch und kann sich diesen
letzten Schritt nur noch aus der bereits besprochenen
politisch-instrumentellen Verwertung durch eine interessierte Schicht von
Intellektuellen erklären: "`...so sollte eine scheinbar unpolitische
Binnenmoral der ,Menschheit' von einer überdehnten Hausmoral geliefert
werden... Da wird doch der verdeckt politische Inhalt erkennbar, denn mit
dieser Priorität würde man den Staatstugenden die Wurzeln abgraben, dem
Behauptungswillen, der Treue zur eigenen Gründung, der wachsenden Sorgfalt
und dem Willen, Grenze und Identität zu behaupten - mit einem Wort: dem
Patriotismus. Auch wird klar, wer das Wort führt: der Intellektuelle der
Großstadt, der Konformist der Negation, dessen ganze Geltungschance von
einer Kritik abhängt, die schmerzend trifft."'\footnote{Ebda., S.92.}

Ein weiterer Grund, aus dem Gehlen das Familienethos - wenn auch nicht
kategorisch so doch für die hohen Aufgaben von Politik und Kultur - ablehnt,
ist es, daß seiner Meinung nach das Familienethos noch nie eine große
kulturelle Leistung hervorgebracht hat: "`alles was Größe hat: `Staat,
Religion, Künste, Wissenschaften wurde außerhalb ihres Bereiches [des
Bereiches der Familie, E.A.] hochgezogen"'.\footnote{Ebda., S.93}

\subsection{Kritik der antihumanistischen Argumente Gehlens}

\subsubsection{Vorüberlegung zu Gehlens Methode: Entlarvungstechnik und empirische Ethik}

Bevor Gehlens Argumente im Einzelnen untersucht werden, bedarf Gehlens
Argumentationsweise einer näheren Betrachtung. Es stellt sich dabei
nämlich heraus, daß einige der von ihm angewandten Methoden zur Klärung
von Problemen der philosophischen Ethik von vornherein untauglich sind.

Unter philosophischer Ethik verstehe ich dabei die Antwort auf die Frage:
"`Was sollen wir tun?"' Diese Frage unterscheidet sich sowohl von dem
Fragenkomplex: "`Was glauben die Menschen, was sie tun sollen? Welches sind
die ethischen Normen einer bestimmten Kultur, eines bestimmten Volkes oder
einer bestimmten Epoche?"', als auch von der Frage: "`Welche Interessen
verfolgt ein Mensch, wenn er diese oder jene Auffassung zu Fragen der Moral
äußert?"'. Die Antwort auf letzteres könnte man als eine Interessentheorie
des moralischen Denkens bezeichnen, die Antwort auf die vorhergehende Frage
gibt die empirische Ethik. Betrachtet man nun Gehlens Argumente im Hinblick
auf diese Unterscheidung, so ist festzustellen, daß nur Gehlens im
vorhergehenden als "`politisch"' eingeordnete Argumentation unmittelbar
philosophisch-ethischer Art ist, denn nur hier argumentiert Gehlen auf der
normativen Ebene, indem er dem Humanitätsethos den der Wert der
Selbsterhaltung des Staates gegenüberstellt. Es stellt sich nun die Frage,
ob und inwieweit interessentheoretische Erwägungen und empirische
Untersuchungen für die philosophische Ethik fruchtbar gemacht werden
können.

\paragraph{Die Entlarvungstechnik in der Ethik} Ein Teil der Vorwürfe, die
Gehlen gegen das Humanitätsethos vorbringt, besteht darin, daß Gehlen bei den
Vertretern des Humanitätsethos unredliche Motive wie übermäßige Geltungssucht,
sinnlose Zerstörungswut oder opportunistisches politisches Kalkül aufdecken zu
können glaubt. Gehlen folgt damit einem in der Philosophie des öfteren
verwendeten Verfahren der Moralkritik, welches man als {\em
  Entlarvungstechnik} bezeichnen kann. Das Vorbild der Entlarvungstechnik als
Methode der Moralkritik liefert Friedrich Nietzsche in seiner Schrift "`Zur
Genealogie der Moral"', worin Nietzsche die christliche Nächstenliebe als
"`Sklavenmoral"' denunziert, die von Schwächlingen zu dem eigennützigen Zweck
erfunden wird, die Starken und Schönen, die "`blonden Bestien"', wie Nietzsche
sie, wohlmöglich in Anspielung auf die (blonde ?) Mähne von Löwen,\footnote{So
  zumindest die wohlwollende Deutung dieses Ausdrucks durch Volker Gerhardt
  unter Verweis auf D.Brennecke im Nachwort zu: Friedrich Nietzsche: Zur
  Genealogie der Moral. Eine Streitschrift, Stuttgart 1993, S.171-187
  (S.184).} nennt, im genüßlichen Gebrauch ihrer brutalen Kräfte zu
hemmen.\footnote{Vgl. Friedrich Nietzsche: Zur Genealogie der Moral, a.a.O,
  S.26ff. (Erste Abhandlung, 10.Abschnitt.)}  Grundsätzlich wird bei der
Entlarvungstechnik eine bestimmte Moral nicht dadurch kritisiert, daß gezeigt
wird, daß sie zu anderen, höherrangigen moralischen Werten in Widerspruch
steht oder bei ihrer Anwendung zu (moralisch) untragbaren Zuständen führt,
sondern es wird versucht, den Nachweis zu erbringen, daß ihre philosophischen
Vertreter ein bestimmtes, meist egoistisches, in jedem Falle aber moralfremdes
Interesse verfolgen, wenn sie diese Moral einfordern, wobei es sich bei diesen
Interessen dann in der Regel nicht um offen liegende sondern um vermutete
latente Interessen handelt. Eine andere Variante der Entlarvungstechnik ist
die von manchen Marxisten gebrauchte, Moral generell als ideologischen Überbau
über ökonomischen Verhältnissen zu betrachten.\footnote{Vgl. die Stichworte
  Moral und Moralphilosophie in: Georg Klaus/Manfred Buhr (Hrsg.):
  Philosophisches Wörterbuch. Band 2, Leipzig 1975, S.823-826.} Es genügt
dann, den gesellschaftlichen Standpunkt (bürgerlich oder proletarisch) eines
Philosophen festzustellen, und man weiß, was man von dessen Ethik zu halten
hat. Gehlen selbst übernimmt sogar bestimmte Nietzeanische Denkfiguren, wenn
er das Vorgehen der kynischen oder stoischen Philosophen als die "`Eroberung
der Eroberer"' deutet.

Die Entlarvungstechnik hat jedoch einen entscheidenden Nachteil. Selbst
wenn wir sicher wissen, welche unredlichen Motive einen Philosophen zur
Aufstellung bestimmter moralischer Grundsätze bewegt haben, so sagt dies
noch längst nichts darüber aus, ob diese Grundsätze gut oder schlecht
begründet sind, und ob sie nicht vielleicht trotz der tadelnswerten
Absichten des Philosophen dasjenige wiedergeben, was objektiv moralisch
richtig ist. Man kann sogar noch einen Schritt weiter gehen: Wenn die
Entlarvungstechnik uns etwas über die moralische Richtigkeit einer
ethischen Auffassung mitteilen könnte, dann würde ein und derselbe
moralische Imperativ gültig oder ungültig sein je nachdem, von wem er
gerade geäußert wird, was offensichtlich absurd ist. Mit Hilfe der
Entlarvungstechnik kann man daher keinerlei Feststellungen über die
Gültigkeit sittlicher Urteile treffen.  Darüber hinaus schleichen sich
bei der exzessiven Verwendung dieser Technik oft noch andere Fehler ein.
So werden recht häufig die moralfremden Interessen mehr unterstellt als
schlüssig nachgewiesen, was auch schwierig genug wäre. Bei Gehlen tritt
letzteres besonders deutlich in der Ungerechtigkeit hervor, mit der er
den Theologen Karl Barth behandelt.

Trotz aller Kritik gegen die Entlarvungstechnik muß dennoch eingeräumt
werden, daß sie für bestimmte Zwecke durchaus von Nutzen sein kann. So kann
uns die Entlarvungstechnik oft auf die die richtige Fährte führen, wo die
Schwächen einer ethischen Theorie zu suchen sind. Auch kann sie, wenn und
nachdem wir festgestellt haben, daß ein Philosoph ziemlich irrige
Moralvorstellungen entwickelt hat, helfen zu erklären, wie diese Irrtümer
zustande gekommen sind.

Insgesamt bleibt jedoch festzustellen, daß Gehlen, soweit er nur die
Vertreter des Humanitätsethos in schlechtem Licht erscheinen läßt, noch
keinen Grund gegen die Gültigkeit dieses Ethos vorgebracht hat. Ebensowenig
sagt die Feststellung, daß das Humanitätsethos in geschichtlichen
Dekadenzperioden entstanden ist, etwas über die moralische Richtigkeit oder
Falschheit dieses Ethos aus.

\paragraph{Empirische Ethik} Einen nicht geringen
Teil von Gehlens Werk bilden Überlegungen zur historischen Entstehung, zur
kulturellen Funktion und zur anthropologischen Ableitung des Humanitätsethos.
Es soll hier nicht bestritten werden, daß die empirische Ethik als ein
bestimmtes Forschungsgebiet ihre Berechtigung hat. Es stellt sich allerdings
die Frage, in welcher Beziehung empirische und philosophische Ethik zueinander
stehen, d.h. ob und unter welchen Bedingungen aus den Ergebnissen der
empirischen Ethik normative Schlußfolgerungen gezogen werden können.

Es scheint zunächst so zu sein, daß aus der empirischen Ethik keinerlei
normative Schlußfolgerungen gezogen werden können. Aus der bloßen
Tatsache, daß zu einer bestimmten Zeit bei einem bestimmten Volk eine
bestimmte Norm anerkannt wurde, läßt sich nicht ableiten, daß diese Norm
tatsächlich gilt. Dies ist auch dann nicht möglich, wenn über eine
bestimmte Norm alle Völker und alle Zeiten übereinstimmen, denn dies
schließt einen Irrtum noch nicht aus. Auch daß sich die Anwendung einer
Norm innerhalb einer Gesellschaft im Sinne einer "`Sozial-Regulation"'
als nützlich erwiesen hat, beweist noch nichts über ihre normative
Gültigkeit, denn sie könnte auch dann immer noch ungerecht oder ethisch
verwerflich sein. So ist zum Beispiel die Norm, daß Frauen "`hinter den
Herd"' gehören, für das Familienleben und damit auch für die
Gesellschaft keineswegs unnütz, aber sie ist dennoch in höchsten Maße
ungerecht. Eine gesellschaftlich nützliche Förderung der öffentlichen
Sicherheit wäre zu erwarten, wenn die Polizei Verbrecher zu dem Zweck
foltern dürfte, die Namen ihrer Komplizen herauszufinden. Dennoch wird
dies aus ethischen Gründen für gewöhnlich abgelehnt.

Ebensowenig können aus einer Theorie der anthropologischen Quellen der Ethik,
sei diese nun mehr anthropologischer oder historischer oder anderer Art,
irgendwelche Schlußfolgerungen in Bezug auf die normative Gültigkeit einer
moralischen Forderung abgeleitet werden. Denn wie sollte sich ein Mensch
verhalten, der wissen möchte, ob eine bestimmte Norm richtig ist, und nun -
anthropologische Untersuchungen anstellend - bemerkt, daß diese Norm über
keine anthropologische Quelle verfügt?  Soll er nun darauf verzichten, sich
nach dieser Norm zu verhalten, oder müßte nicht schon allein die Tatsache, daß
er sich dieser Norm gemäß verhalten könnte, eine Theorie Lügen strafen, die
diese Handlungsautonomie leugnet? Es scheint sich nämlich so zu verhalten, daß
eine bewußte Überzeugung ein hinreichender Grund für einen Menschen ist, um
gemäß der moralischen Norm, von deren Richtigkeit er überzeugt ist, zu
handeln. Zwar könnte es sein, daß das Bewußtsein des Menschen so beschaffen
ist, daß es nur von ganz bestimmten Normtypen, um deren Ermittlung die
Anthropologie sich bemüht, überzeugt sein kann. Doch sollte irgendwann einmal
ein Mensch trotzdem von der Richtigkeit einer Norm überzeugt sein, die sich
nicht unter diesen Normtypen einordnen läßt, dann wäre bereits damit die
anthropologische Theorie widerlegt. Für den handelnden Menschen bedeutet dies,
daß er sich in der Wahl seiner Maximen zwangsläufig für frei halten
muß.\footnote{In diesem Sinne hat das Kantische "`Du kannst, denn du sollst!"'
  durchaus seine Berechtigung, wenn es auch nicht als Beweis der menschlichen
  Willensfreiheit taugt. Vgl. Immanuel Kant: Kritik der praktischen Vernunft,
  Hamburg 1990, S.33-35.} Oder anders formuliert: Die Anthropologie kann
Einschränkungen der menschlichen Freiheit bloß feststellen (und sich dabei
irren) aber nicht vorschreiben.

Wie verhält es sich aber nun, wenn ein Mensch feststellt, daß die
sittliche Norm, nach der er glaubt, handeln zu müssen, einer
anthropologischen Quelle entspringt, die nicht identisch ist mit dem
Bereich, in dem die Handlung stattfindet? Man könnte sich in dieser
Situation etwa einen Politiker vorstellen, den - vielleicht unter dem
Einfluß der seichten Lektüre stoischer Philosophen - das Gefühl
beschleicht, er müsse auch im Bereich der Politik die ursprünglich (d.h.
vor hunderten oder tausenden von Jahren!)  der Familienmoral
entstammenden Prinzipien der Humanität stärker zur Geltung bringen. Wäre
nun der Hinweis auf den familiären Ursprung der Humanität ein
schlagendes Argument gegen ihre Berücksichtigung in der Politik?
Keineswegs, denn man würde hierbei einen ähnlichen Fehler begehen, wie
bei dem Gebrauch der Entlarvungstechnik, da aus den Entstehungsumständen
einer Ethosform weder etwas über ihre Brauchbarkeit als
"`Sozial-Regulation"' noch über ihre normative Gültigkeit abgeleitet
werden kann. Dies gilt insbesondere dann, wenn es sich bei dem in Frage
stehenden Ethos um einen "`elargierten Instinkt"' handelt, der ja im
Laufe seiner Elargierung auch eine Zweckanpassung an die neue
Gebrauchssituation erfahren haben könnte. Man muß dabei beachten, daß
nach der Gehlenschen Anthropologie der Mensch nicht wie das Tier über
festgelegte Instinkte verfügt, sondern daß der Funktionswandel seiner
Instinkte für ihn geradezu wesenstypisch ist. Wollte man nun behaupten,
die menschlichen Instinkte könnten keine anderen Funktionen übernehmen
als die ursprünglichen, so hieße dies, die Überlebensfähigkeit des
"`weltoffenen"' Tieres Mensch überhaupt zu bestreiten. Die Frage, ob und
inwieweit das Humanitätsethos in der Politik zur Geltung gebracht werden
kann, ist daher selbst dann keine Frage seines Ursprungs, wenn wir es
mit Gehlen als erweitertes Familienethos betrachten, sondern sie hängt
davon ab, welche Spielräume und Handlungsmöglichkeiten dem Menschen im
Bereich der Politik eröffnet sind. Für die Beantwortung dieser Frage,
die eine Frage der Politik bzw. der Politischen Wissenschaft darstellt, ist
die Anthropologie jedoch ein viel zu grobes Werkzeug.

Es besteht allerdings ein gewisser Unterschied zwischen der Frage nach der
Verbindlichkeit moralischer Normen für einen einzelnen Menschen, der an sich
selbst beliebig rigorose Ansprüche stellen kann, und der Frage, welche
moralischen Normen für alle Mitglieder einer Gesellschaft verbindlich gelten
sollten. Bei der zweiten dieser Fragen muß das unter anthropologischen
Gesichtspunkten realistischerweise Mögliche berücksichtigt werden, denn es hat
keinen Zweck moralische Normen einzufordern, die die meisten Menschen
chronisch überfordern, weil die Folge höchstens eine generelle Mißachtung der
Moral sein würde, wie ja auch das Zivilrecht und das Strafrecht in gewissem
Maße das vielleicht krude Rechtsempfinden der Bürger berücksichtigen müssen
und nicht bloß eine abstrakte Moral, da sonst die Bürger dazu neigen könnten,
zur Selbstjustiz zu greifen. (Dies ist z.B. bei der Diskussion der Frage, ob
der Strafe eine Rache-Funktion zukommen soll, zu berücksichtigen.) Im Bereich
des öffentlichen Lebens und der Politik sind daher anthropologische und andere
empirische Informationen zur Klärung der Frage der Gebräuchlichkeit und damit
- will man nicht ganz weltfremd bleiben - auch der Gültigkeit moralischer
Normen notwendig. Allerdings spielt hier nicht die Frage nach dem Ursprung
von Normen eine Rolle, sondern es steht die schwierige und - sollte die
Freiheit des Menschen denn eine Tatsache sein - nie ganz zu klärende Frage
nach dem menschlich Möglichen im Vordergrund.

Es hat sich also gezeigt, daß die empirische Ethik kaum zur Begründung
moralischer Normen herangezogen werden kann. Abgesehen davon besteht
die Gefahr, daß philosophisch-ethisch relevante Unterscheidungen
unterschlagen werden, da sie bei einer empirischen Untersuchung
möglicherweise gar nicht zum Tragen kommen. Geht man etwa von einer
psychologischen Untersuchung aus, so würde man moralische Imperative
nach der Art ihrer Repräsentation in der Psyche des Menschen
einteilen. Es ist nun jedoch gut möglich, daß ethisch relevante
Unterschiede wie z.B. der zwischen hypothetischen und kategorischen
Imperativen sich in der Intensität der Sollensgefühle gar nicht
wiederspiegeln. Gleichzeitig entsteht aus dem unvorsichtigen Rückgriff
auf die empirische Ethik ein Risiko, dem Gehlen, wie es scheint, nicht
ganz entgangen ist: Gehlen diskutiert an keiner Stelle seines Werkes
die ethischen Theorien und Gedanken, die von den philosophischen
Befürwortern des Humanismus im Laufe der Jahrtausende entwickelt
worden sind, vielmehr beschränkt er sich bei der Bildung seines
Begriff des "`Humanitarismus"' weitgehend auf die Merkmale, die er aus
der historischen Betrachtung, bei der er obendrein recht wählerisch
verfährt, aus seiner anthropologischen Ableitung und aus der
hermeneutischen, d.h.  lediglich die Intention "`entlarvenden"' und
nicht die Aussage berücksichtigenden, Deutung der Auffassungen von
Befürworten des Humanitätsethos gewinnt. So erhält das Humanitätsethos
nach und nach Merkmale wie: Dekadenzmoral, Privatmoral,
Gesinnungsethik. Das dabei entstehende Zerrbild der humanistischen
Ethik zu kritisieren und in beinahe jeder Hinsicht in Verruf zu
bringen fällt Gehlen dann natürlich nicht mehr schwer. Dennoch zielt,
wie zu Anfang dieser Arbeit bereits festgestellt, Gehlens Angriff
durchaus auch auf den Kerngehalt der humanistischen Ethik, nämlich auf
die Gleichheit aller Menschen und auf die Menschlichkeit als oberstes
Prinzip der Moral.

\subsubsection{Kritik des historischen Argumentes}

Von allem, was Gehlen gegen den "`Humanitarismus"' vorbringt, sind seine
historischen Ausführungen am wenigsten überzeugend. Zwar stimmt es, wenn
Gehlen die Entstehung des antiken Humanismus der Verfallszeit der griechischen
Stadtstaaten zurechnet, doch schon wenn Gehlen daraus gegenüber den Kynikern
und den Stoikern den Vorwurf der opportunistischen Anbiederung an die neuen
Herrscher ableitet, urteilt Gehlen ungerecht. Schließlich müßte man Zenon und
Anthistenes ihre politische Meinung doch lassen, wenn sie die Großreiche für
die bessere politische Ordnung halten (falls sie dies taten). Auch verwundert
es, daß Gehlen sie der Entpolitisierung der Bevölkerung zeiht ("`für die
Vielen die Lämmerweide"'), wo sich Gehlen die Politisierung des Bürgers doch
ohnehin nur als Inpflichtnahme durch den Staat vorzustellen weiß, an der es
wohl auch Großreiche nicht gänzlich fehlen lassen werden. Weiterhin verwickelt
Gehlen sich in Widersprüche, wenn er der kynisch-stoischen Ethik einerseits
eine herrschaftsabstützende Absicht unterstellt und ihr gleichzeitig ihre
vermeintlich dekadent-subversive Wirkung vorwirft.  Historisch sehr fragwürdig
ist auch Gehlens Andeutung, daß dem römischen Kaiserreich sein Humanitätsethos
zu schaffen gemacht habe. Trajan, der, weil er das Feldlager liebte, auch
"`der Soldatenkaiser"' genannt wurde, war gewiß nicht humanitär verweichlicht.
Auch sein tüchtiger Nachfolger Hadrian wußte, dadurch daß er immer wieder die
Provinzen bereiste und die Grenzen befestigte, die Sicherheit des römischen
Reiches zu gewähren, obwohl er anders als sein Vorgänger auf Kriegszüge und
(letztenendes doch nicht zu haltende) kriegerische Erwerbungen verzichtete.
Antoninus Pius wirft man vor, er habe sich zu wenig um die Provinzen gekümmert
und Kriege um jeden Preis vermieden, doch mag dies vielleicht nicht weniger
mit den Handlungs- und Entscheidungsgewohnheiten des ehemaligen
Verwaltungsbeamten Antoninus als mit humanitärer Verweichlichung zu tun haben.
Seinen Nachfolger Marc Aurel hinderten die stoischen Überzeugungen jedenfalls
nicht daran, sich als tatkräftigen Kriegsherren zu beweisen.\footnote{Vgl.
  Hans-Georg Pflaum: Das römische Kaiserreich, in: Hans-Georg Pflaum /
  Berthold Rubin / Carl Schneider / William Seston: Rom. Die römische Welt,
  Frankfurt /M / Berlin 1963, S.317-428 (S.360-382).} In der Blütezeit des
römischen Kaiserreiches war die Außenpolitik also keineswegs durch humanitäres
Abschlaffen geprägt, wenn sie in ihren Zielsetzungen auch maßvoller blieb als
die vorhergehenden Phasen imperialistischer Expansion. Im Innern jedoch hat
das Humanitätsideal tatsächlich zu einer gewissen Humanisierung beigetragen,
die Gehlen unverständlicherweise mit einem ziemlich süffisanten Unterton
beschreibt.\footnote{Kahrstedt ist zwar auch der Ansicht, daß die Philosophie
  im 2.Jh. einen zunehmend "`pietistischen"' Einschlag bekommen habe, aber er
  hält dies nicht für eine Konsequenz des stoischen Humanitätsideales, sondern
  eher für eine Folge des Vordringens der orientalischen Religionen. Vgl.
  Ulrich Kahrstedt: Kulturgeschichte der römischen Kaiserzeit, Bern 1958,
  S.305ff.} Daß die politisch-ökonomische Doppelkrise im 3.Jahrhundert, so wie
Gehlen durch seinen Hinweis auf die notwendig gewordenen Schutzmaßnahmen
Aurelians suggeriert, eine kausale Folge der Verweichlichung durch das
Humanitätsethos war, dürfte sich historisch ebenfalls kaum belegen lassen.
Wendet man den Blick schließlich der Endphase des römischen Reiches zu, so ist
der Verfall des Reiches nach Kaiser Konstantin vor allem durch die innere
Zerrissenheit und die blutigen Rivalitätskämpfe der Nachfolger Konstantins
bedingt, was dann wohl doch mehr auf Kosten des "`Machtethos"' als des
Humanitätsethos geht.

Der Zusammenhang von Humanismus und Dekadenz bzw. Kulturverlust scheint noch
weniger vorhanden zu sein, wenn man den Renaissance-Humanismus oder den
Humanismus der Aufklärung als Beispiel wählt. Gehlen schützt sich freilich vor
dieser Einsicht, indem er die Aufklärung pauschal als zersetzend
ablehnt.\footnote{Vgl. Gehlen: Hypermoral, a.a.O., S.102.} In dieselbe
Richtung geht auch seine Bemerkung, die Familie habe niemals etwas Großes
hervorgebracht, die man sich dahingehend zu übersetzen hat, daß das
Humanitätsethos, als dem Familiengeist entsprungen, kulturellen
Höchstleistungen im Wege stehe. Hierzu ist zu sagen, daß die Ethik gar nicht
die Aufgabe hat, die Menschen zu kulturellen Leistungen zu stimulieren. Große
kulturelle Fortschritte sind, abgesehen davon, oft in Umbruchsperioden zustande
gekommen, in denen eine ursprünglich vergleichsweise starre und geschlossene
Gesellschaft sich liberalisierte und neuen Einflüssen öffnete (z.B. Rußland im
19.Jahrhundert, Deutschland in den 20er Jahren).

\subsubsection{Kritik des politischen Argumentes}

Wie bereits dargelegt wurde, hält Gehlen das Übergreifen des humanitären
Ethos auf die Politik aus mehreren Gründen für verhängnisvoll: Es gefährdet
die äußere Souveränität des Staates, indem es die Sicherheitspolitik
moralisch delegitimiert. Es gefährdet die innere Handlungsfähigkeit des
Staates, indem es im Verbund mit dem Masseneudaimonismus eine
Anspruchsmentalität beim Bürger entstehen läßt. Und es unterminiert den
Patriotismus, was die nationale Identität gefährdet.

Bevor diese Einwände auf ihre Überzeugungskraft hin untersucht werden,
muß jedoch auf Gehlens Ansicht eingegangen werden, der Staat induziere
ein eigenes Ethos, mit Gehlens Worten: "`die necessitas rerum im
Staate als letzte, auch ethische Berufungsinstanz, als Sichbeugen
unter den Sachzwang in Ehre und Disziplin"'.\footnote{Ebda., S.106.}
Es gibt nun allerdings einen sehr einfachen Grund, warum der Sachzwang
niemals eine letzte und schon gar keine ethische Berufungsinstanz sein
kann: Sachzwänge entstehen erst in Bezug auf beabsichtigte Ziele. Nur
wenn jemand die Absicht hat dieses oder jenes Ziel zu verwirklichen,
werden die äußeren Umstände bzw. die Grenzen der
Handlungsmöglichkeiten dieses Menschen zu Sachzwängen. Dies bedeutet
jedoch, daß die Berufungsinstanz nicht der Sachzwang, sondern das zu
verwirklichende Ziel ist. Um eine {\em ethische} Berufungsinstanz
handelt es sich dann, wenn dieses Ziel ein ethischer Wert ist. Von
einer letzten ethischen Berufungsinstanz kann sinnvollerweise nur dann
die Rede sein, wenn es sich bei diesem ethischen Wert um einen
obersten ethischen Wert handelt. Der Irrtum, der darin besteht, den
Sachzwang selbst als Berufungsinstanz zu betrachten, zieht die fatale
Folge nach sich, daß das verfolgte Ziel unter dem Namen "`Sachzwang"'
verborgen und so insgeheim sanktioniert wird, so daß eine rationale
Abwägung dieses Ziels gegenüber anderen Zielen unterbleibt.  Dies kann
Politikern, die ihre Absichten manchmal gern verdunkeln und die
Güterabwägung zwischen ihren eigenen und möglichen anderen Zielen
vermeiden möchten, nur recht sein. Der Bürger, der die Berufung auf
den Sachzwang als "`letzte, auch ethische Berufungsinstanz"'
blindlings abkauft, läuft Gefahr, bei der Beurteilung von Politik
hilflos zu werden, da er politische Entscheidungen nicht mehr als die
schwierige rationale Abwägung zwischen konkurrierenden Zielen unter
Berücksichtigung der beschränkten Handlungsmöglichkeiten sowie der
Kosten-Nutzen-Relation versteht, sondern ihm statt dessen Politik und
Geschichte durch schicksalhafte Notwendigkeiten bestimmt zu sein
scheinen, deren Unausweichlichkeit er finster billigen muß.  Dieser
Blindheit scheint auch Arnold Gehlen nicht immer ganz zu entgehen,
wenn er die Auffassung vertritt, daß Kriege in der Regel ungewollt aus
unvermeidlichen machtpolitischen Zwangslagen entstehen\footnote{Vgl.
  ebda., S.113. - Vgl. dazu auch Gehlens in ihrer Arglosigkeit und
  Naivität kaum zu überbietende Behauptung, das Handeln von Menschen
  in Führungspositionen sei nicht durch Machttrieb zu erklären,
  sondern müsse vielmehr als eine Art Sachwaltung der objektiven
  Ansprüche der Institution verstanden werden, Gehlen: Urmensch,
  a.a.O., S.68f. - Durchaus treffend erscheint daher auch Adornos
  Einschätzung, daß Gehlen die Unterwerfung unter die Institutionen
  auf Grund einer tiefenpsychologisch erklärbaren Identifikation mit
  den angstgebietenden Mächten fordert. Vgl. dazu das Streitgespräch
  zwischen Adorno und Gehlen in: Grenz, Friedemann: Adornos
  Philosophie in Grundbegriffen. Auflösung einiger Deutungsprobleme,
  Frankfurt am Main 1974, S.225-251 (S.245f.).}, oder wenn er, in
kritikloser Übernahme der damaligen ideologischen
Selbstrechtfertigung, die Einschätzung wiedergibt, der
(Kolonial-)Imperialismus werde vorangetrieben durch den
"`biologische[n] Druck wachsender Massen"' und drücke so "`die
furchtbare Wahrheit aus, daß Leben von Leben zehrt"'.\footnote{Vgl.
  Gehlen: Hypermoral, a.a.O., S.108.}

Wenn auch Sachzwänge als solche als Berufungsinstanz ausfallen, so kann doch
andererseits kein Zweifel darüber bestehen, daß die Herstellung äußerer
Sicherheit ein wichtiges Ziel staatlicher Politik darstellt. Gehlen vertritt
im Einklang mit der Schule des politischen Realismus die Auffassung, daß dazu
die Maximierung von Macht erforderlich ist.\footnote{Vgl. ebda., S.115/116.}
Er deutet dabei nur vage an, wo für ihn die Grenzen des noch sinnvollen
Machtgebrauchs liegen.\footnote{Vgl. ebda., S.113/114., S.119.} Über das vom
politischen Realismus als sinnvoll betrachtete Maß des Machtgebrauchs geht
Gehlen deutlich hinaus, wenn er die äußere Aggression auch zur Bewältigung
innerer Krisen für notwendig und sinnvoll erachtet, wie er das offenbar bei
seiner Deutung des Imperialismus tut.\footnote{Vgl. ebda., S.106-109. -
Vgl. dazu auch: Arnold Gehlen: Die Gesellschaftsordnung im Widerstreit der
Interessengruppen und der gesellschaftlichen Mächte, in: Arnold Gehlen:
Gesamtausgabe. Band 7.  Einblicke, Frankfurt am Main 1978, S.209-222 (S.212,
S.217f.).} Auch muß es aus Sicht des politischen Realismus, der den
Machtgebrauch streng zweckrational auf das sicherheitspolitische Ziel bezieht
und dadurch in gewisser Weise limitiert, als eine gefährliche Übertreibung
angesehen werden, ein Eigenethos der Macht zu postulieren, und den
Machtgebrauch als einen Genuß hochzuschätzen.\footnote{Vgl. ebda., S.116.} Da
der politische Realismus die denkbar pessimistischsten anthropologischen
Vorstellungen in der Politik zu Grunde legt\footnote{Vgl. Hans J. Morgenthau,
a.a.O., S.21-24.} und darüber hinaus keinerlei moralische Einschränkungen
vornimmt, markiert er die oberste Grenze des überhaupt zu rechtfertigenden
Machtgebrauchs. Es stellt sich jedoch immer noch die Frage, ob Gehlen nicht
im Grundsätzlichen damit Recht behält, daß eine effiziente Außen- und
Sicherheitspolitik mit der humanistischen Ethik unvereinbar ist. Legt man die
humanistische Ethik jedoch verantwortungsethisch aus, so läßt sich mit ihr
auch der Einsatz militärischer Gewalt rechtfertigen, sofern er dem Schutz von
Leben, Freiheit und Menschenwürde der Bürger dient. Damit ist allerdings auch
schon die Grenze gezogen, bis zu der nach humanistischem Verständnis
Sicherheitspolitik sinnvoll und notwendig ist. Weder wäre es sinnvoll, einen
despotischen Staat um seiner selbst will zu verteidigen, noch ließe sich ein
unbegrenztes Machtstreben rechtfertigen, wie dies der politische Realismus
verlangt. Da nach der humanistischen Ethik das Lebensrecht der Bürger anderer
Staaten nicht weniger wiegt als das der Bürger des eigenen Staates, so kann
die äußere Machterweiterung - im Gegensatz auch zu den Forderungen eines
normativ verstandenen politischen Realismus - höchstens bis zur Herstellung
eines Gleichgewichtszustandes gerechtfertigt werden.

Bis zu einem gewissen Grade läßt sich also auch eine harte Sicherheitspolitik
mit der humanistischen Ethik rechtfertigen. Damit wird Gehlens Vorwurf
hinfällig, der "`Humanitarismus"' gefährde die äußere Selbstbehauptung des
Staates. Daß Gehlen bei seiner "`Apologie der Macht"'\footnote{Gehlen:
Hypermoral, a.a.O., S.116.} weit über das vernünftige Maß hinausgeht hängt
unter anderen damit zusammen, daß er den Staat und die Souveränität des
Staates als Selbstzweck betrachtet. Auf das Staatsverständnis Gehlens wird
weiter unten noch eingegangen werden. Zuvor soll noch kurz beleuchtet werden,
welche sicherheitspolitische Bedeutung der Souveränität zukommt.
Souveränität im Sinne weitgehend uneingeschränkter außenpolitischer
Handlungsmöglichkeiten ist für Gehlen eine unerläßliche Voraussetzung dafür,
die Sicherheit des Staates und die Selbstbestimmungsmöglichkeiten einer
Nation zu gewährleisten. Dies ist jedoch nur teilweise richtig. Die
Sicherheit des Staates läßt sich im Rahmen kollektiver Sicherheitssysteme
unter teilweiser Aufgabe der Souveränität oftmals besser erreichen als im
nationalstaatlichen Alleingang. Was die Selbstbestimmungmöglichkeiten angeht,
so ist das unter den Besatzungsstatuten stehende Westdeutschland in der Zeit
von 1948-1990 das beste Beispiel dafür, daß der Mangel an Souveränität noch
keinen Nachteil für die Bürger bedeuten muß. Es ist wenig einleuchtend, wenn
Gehlen behauptet, diese Situation habe "`nur für sehr kleine
Interessentenkreise Vorteile, so für Intellektuelle, Fabrikanten und den
kleinen Kreis der Erkennenden"'.\footnote{Ebda., S.112.} Auch ist kaum
anzunehmen, daß es mehr Menschen als ein Häuflein national gesinnter
Intellektueller sind, die "`das entgangene stolze Bewußtsein, einem großen,
starken, geachteten und gefürchteten Volk anzugehören"'\footnote{Ebda.,
S.113.} grämt.

Noch weniger überzeugend als seine außenpolitischen Argumente sind Gehlens
innenpolitische Bedenken - zumindest dann, wenn man sie als Einwände gegen
den Humanismus auffaßt. Gehlen befürchtet, daß die Ausweitung der
Ansprüche im Sozialstaat zum Immobilismus führt, derart daß der Staat nur
noch Umverteilungsaufgaben und keine eigentlich politischen Aufgaben mehr
wahrnimmt.\footnote{Vgl. ebda., S.117f.} Es ist nicht ganz klar, was in
Gehlens Augen die eigentlich politischen Aufgaben sind. Vermutlich läuft
seine Befürchtung darauf hinaus, daß zugunsten des Sozialetats die anderen
Haushalte zu stark geschröpft werden könnten. In diesem Punkt, und auch was
seine Kritik der zunehmenden Anspruchsmentalität und des Verhaltens der
Interessengruppen angeht, ist Gehlen durchaus zuzustimmen. Soweit seine
Kritik am Sozialstaat jedoch grundsätzlicher Natur ist, scheint sie eher mit
Gehlens autoritärem Staatsverständnis zusammen zu hängen, nach welchem es
sich einfach nicht gehört, wenn die Bürger Ansprüche an den Staat stellen.
Als Kritik an der humanistischen Ethik sind Gehlens Einwände jedoch
verfehlt, denn die humanistische Ethik fordert keineswegs eine Ausweitung des
Sozialstaates über seine Leistungsgrenzen hinweg. Auch führt das
Humanitätsethos keineswegs mit innerer Logik über den von Gehlen so
genannten "`Masseneudaimonismus"' zu den ständig sich überschlagenden
Ansprüchen aggressiv auftretender Interessengruppen. Selbst wenn es eine
Tendenz des Humanitätsethos gäbe, in dieser Weise exzessiv auszuarten, so
hieße es immer noch das Kind mit dem Bade ausschütten, wollte man deswegen
die humanistische Ethik aufgeben oder auf den familiären Bereich
beschränken.

Weder im innen- noch im außenpolitischen Bereich gefährdet die humanistische
Ethik also das Funktionieren des Staates. Im Gegenteil lassen sich sogar
bestimmte sicherheitspolitische Forderungen mit dem Humanitätsethos normativ
begründen. Wenn Gehlen die Gültigkeit des Humanitätsethos im Bereich der
Politik leugnet, dann geschieht dies zu einem Teil auch deshalb, weil das
Humanitätsethos seinem Staatsverständnis zuwider läuft. Gehlens Ideal des
Staates ist das eines nationalen autoritären Machtstaates. Als solcher stellt
er für Arnold Gehlen einen Selbstzweck dar. Und nur als solcher hat der Staat
für Gehlen überhaupt einen Wert. Anders ist Gehlens flammende Empörung
darüber, daß "`der Leviathan mehr und mehr die Züge einer
Milchkuh"'\footnote{Ebda., S.110.} annehme und dem "`Anprall der
Gesellschaft"'\footnote{Ebda., S.109. - Vgl. ebda., S.117ff.}  geradezu
hilflos ausgeliefert sei, kaum nachzuvollziehen. Der Wert des Staates
erschöpft sich für Gehlen keineswegs in seiner Nutzenfunktion für die
Gesellschaft, sondern dem Staat selbst kommt als Institution eine
wertkonstituierende und sinngebende Wirkung zu, durch die der Bürger eine
"`Daseinssteigerung"' erfährt.

Zur kritischen Erörterung dieses Staatsverständnisses empfiehlt es sich,
dessen verschiedene Aspekte in Form von Fragen zu formulieren und einzeln zu
untersuchen: 1.Ist die Existenz des Staates wenigstens teilweise ein
Selbstzweck bzw. ein eigener ethischer Wert? 2.Ist die Zugehörigkeit zu
einem starken Staat oder einer souveränen Nation für den Bürger sinngebend
und bedarf der Bürger einer solchen Sinngebung? 3.Ist es wünschenswert,
daß der Staat Sinngebungsfunktionen übernimmt?

Die erste dieser Fragen ist eine unmittelbare Wertfrage und deshalb etwas
schwierig zu entscheiden. Denn obwohl klar ist, daß von verschiedenen
ethischen Standpunkten nur einer gültig sein kann, da in der Ethik anders
als etwa in der Ästhetik eine Toleranz widersprechender Werte nicht
hinnehmbar ist, ist es der Philosophie bisher noch nicht gelungen die
Letztgültigkeit irgend eines ethischen Wertes oder eines ethischen Satzes zu
beweisen. Alles, was sich bisher erreichen läßt, ist die Rückführung
ethischer Werte auf wieder andere ethische Werte. Die Kritik moralischer
Überzeugungen müßte sich so gesehen auf die Aufdeckung logischer
Inkonsistenzen beschränken. Allenfalls wäre es noch möglich aufzuzeigen,
welche Konsequenzen sich aus der Verwirklichung bestimmter Werte ergeben.
Stellt sich nun jemand auf den Standpunkt, daß ein bestimmter ethischer Wert
eine absolute und damit keiner weiteren Rückführung auf andere Werte mehr
bedürftigen Gültigkeit besitzt und hält er die Konsequenzen, die sich aus
der Verwirklichung dieses Wertes ergeben, für hinnehmbar, ja vielleicht sogar
auf Grund dieses Wertes für wünschenswert, so gibt es keine Möglichkeit ihn
von der Falschheit seines Standpunktes zu überzeugen. Was bleibt, ist es,
diesem Standpunkt eine andere Überzeugung entgegenzustellen. 

Vom humanistischen Standpunkt aus stellt nun die Existenz ein\-es Staates (und
besonders eines Rechtsstaates) zweifel\/los einen hohen Wert dar, denn sie
geht mit einem Maß der Verwirklichung humanistischer Werte im Inneren einher,
welches sich im staatsfreien Naturzustand niemals erreichen läßt. Der Wert des
Staates bleibt aber ausschließlich der eines überaus nützlichen Mittels zu dem
Zweck, die Sicherheit und Freiheit der Bürger zu gewährleisten. Der Staat
erhält somit seinen Wert nur durch den Bürger und nicht umgekehrt. Daß dem
Staat keine eigenen ethischen Rechte oder ein selbständiger ethischer Wert
zukommt, läßt sich auch damit begründen, daß der Staat als ein abstraktes
Gebilde niemals Leiden empfinden kann und deshalb des Schutzes durch die Moral
höchstens insofern bedarf, als seine Existenz die Bürger vor Leiden schützt.
Selbst wenn dem Staat ein eigener Geist unterstellt wird, was unter
heuristischen Gesichtspunkten mehr oder weniger zweckmäßig seien mag, so kann
der Staat daher doch niemals über ein sittliches Wesen verfügen.

Was die zweite Frage betrifft, so kann nicht bestritten werden, daß für den
Bürger die Identifikation mit dem Nationalstaat eine emotionale Bereicherung
darstellen kann. Es handelt sich ohne Zweifel um echte und natürliche,
vielleicht sogar edle Gefühle, die keinen Spott verdienen, wenn die Bürger mit
dem Geschick ihrer Nation mitfühlen und z.B. über einen militärischen Sieg
ihres Landes in Begeisterung geraten und bei einer Kriegsniederlage
Zerknirschung empfinden. Dennoch ist es sehr fragwürdig, ob für die
Ausgeglichenheit des Seelenlebens der Bürger eine Notwendigkeit besteht, daß
der Staat außerhalb seiner rationalen Funktionen auch noch solche emotionalen
Bedürfnisse der Bürger befriedigt, wie den Durst nach Ruhm und Ehre oder
vielleicht sogar religiöse Bedürfnisse. Schließlich lassen sich diese
Bedürfnisse auch auf andere Weise befriedigen. Für die Befriedigung der
religiösen Bedürfnisse halten die christlichen Kirchen attraktive Angebote
bereit, und wem wie Gehlen die Kirchen in der heutigen Zeit etwas zu lau
geworden sind, der kann rigoroseren Sekten beitreten. Dem Wunsch, sich Ruhm zu
erwerben und die eigene Ehre zu verteidigen, läßt sich sehr gut im Sport,
insbesondere im Mannschaftssport, nachgehen. Der Sport eignet sich hierfür
sogar viel besser, da der einzelne Mitspieler viel unmittelbarer an den
strategischen Entscheidungen beteiligt ist als etwa der
kriegsdienstverpflichtete Soldat auf dem Schlachtfeld. Wenn auch eine gewisse
affektive Bindung der Bürger an den Staat durchaus wertvoll und nützlich sein
kann, so leidet andererseits das Lebensglück der Bürger nicht weiter darunter,
wenn der Staat mangels Macht den Bürgern das "` 'stolze Bewußtsein, einem
großen, starken, geachteten und gefürchteten Volk anzugehören '
"'\footnote{Ebda., S. 113.} vorenthält.\footnote{Eine etwas andere Frage ist
  es, ob eine starke emotionale Bindung an den Staat nicht der
  Einsatzbereitschaft der Bürger für den Staat förderlich ist. Joachim Fest
  vertritt hier die Auffassung, daß der liberale Staat dem totalitären Staat,
  der sich auf weltanschaulichen Fanatismus stützt, keine gleichwertigen
  Mobilisierungsreserven aufbieten kann. Dagegen läßt sich jedoch einwenden,
  daß liberale Staaten in fast allen anderen Bereichen deutliche
  Effizienz-Vorteile aufbieten können. Im Übrigen weisen auch die scheinbar in
  der Konsum-Lethargie versunkenen Bevölkerungen liberaler Staaten im Zeichen
  äußerer Bedrohung sehr lebendige Abwehrreflexe auf. - Vgl. Joachim Fest: Die
  schwierige Freiheit. Über die offene Flanke der offenen Gesellschaft, Berlin
  1993, S.31ff.}

Es ist darüber hinaus nicht nur nicht notwendig, daß der Staat eine
Sinngebungsfunktion erfüllt, sondern sogar von Nachteil, wenn er dies tut.
Der Grund hierfür liegt darin, daß die staatliche Politik in der Regel
stets Entscheidungen für die gesamte Bürgerschaft trifft. Dies bedeutet
aber, daß nur dasjenige in den Bereich staatlicher Politik fallen sollte,
was unbedingt allgemeinverbindlich entschieden werden muß. Auf die Frage
jedoch, was im Leben Wert und Sinn hat, werden subjektiv stark differierende
Antworten gegeben. Da hier eine allgemeinverbindliche Antwort zu finden
auch gar nicht notwendig ist, würde es dem Grundsatz weltanschaulicher
Toleranz zuwiderlaufen und im übrigen den Staat überfordern, wollte der
Staat dem Leben des Bürgers Sinn und Orientierung verleihen.

Gehlens Staatsverständnis ist also weder mit dem Begriff von Staat an sich
identisch, noch erscheint es überhaupt wünschenswert. Da das
Humanitätsethos darüber hinaus eine hinreichende normative Grundlage sowohl
für die Existenz des Staates als auch für die zur Erfüllung der
staatlichen Aufgaben notwendigen Mittel abgibt, bleibt kein Grund mehr die
Existenz eines autonomen Staatsethos anzunehmen, es sei denn man wünschte
dieses Staatsethos aus moralischer Überzeugung um seiner selbst willen.

Noch nicht restlos geklärt ist bisher die Frage, in welcher Beziehung
das Humanitätsethos zu den patriotischen Tugenden und zur nationalen
Identität steht. Die Beantwortung dieser Frage wird sich aus dem folgenden
Abschnitt ergeben, wo der Antagonismus von Humanitätsethos und Institutionen
betrachtet wird.

\subsubsection{Kritik des anthropologischen Argumentes}

Daß Gehlens Vorwurf gegen den "`Humanitarismus"', das Humanitätsethos
entstamme einer Familienmoral selbst dann nicht trifft, wenn diese Aussage
historisch wahr sein sollte, wurde bereits in den Vorüberlegungen dargelegt.
Es bleiben jedoch noch einige andere Vorbehalte Gehlens zu klären. Gehlens
Ansicht nach kollidiert der "`Humanitarismus"' (insbesondere in seiner
nochmals gesteigerten Form als Moralhypertrophie) zwangsläufig mit dem
Eigenethos von Institutionen und wirkt, wenn er nicht eingegrenzt wird,
kulturzerstörend. Diesen Vorwurf der Zersetzung richtet Gehlen nicht nur gegen
den "`Humanitarismus"' sondern gegen die Aufklärung und aufklärerisches
Bemühen überhaupt. Gehlen ist der Ansicht, daß Institutionen niemals
vollständig rational begründet werden können.\footnote{Gehlen erfaßt in seiner
  Institutionenlehre mit den Begriffen der "`sekundären Zweckmäßigkeit"' und
  der "`Stabilisierung nach rückwärts"' zwar auch die Möglichkeit der
  (zweck-)rationalen Begründung von Institutionen, aber diese erscheint ihm
  immer noch als höchst prekär. Vgl. F. Jonas, a.a.O., S.69ff.}
Dementsprechend lehnt Gehlen auch die rationale Kritik an Institutionen ab, da
diese nur zu dem Ergebnis führen kann, daß die Institutionen unnötig
sind.\footnote{Vgl. ebda., S.102.}

Zunächst zur Frage der rationalen Begründ- und Kritisierbarkeit von
Institutionen und der Rolle der Aufklärung:\footnote{Einer
ausgearbeiteten und teilweise auch empirisch abgestützten Theorie
etwas gleichwertiges entgegenzuhalten ist im Grunde unmöglich. (Man
müßte schon einige Jahre Forschung dazu betreiben.) Aber es ist
immerhin möglich anzudeuten, wie eine solche Theorie aufgebaut werden
könnte.} Es ist zu unterscheiden zwischen der Begründung und der
Entstehung einer Institution. Die meisten der heutigen
gesellschaftlichen und staatlichen Institutionen sind in irgend einer
Weise historisch gewachsen. Dies bedeutet jedoch noch nicht, daß sie
rational unbegründet sind. Eine historisch gewachsene Institution kann
ebenso rational begründet sein wie eine bewußt konstruierte
Institution, wenn sie einem bewußten Zweck dient, und wenn sie dazu
geeignet ist, diesen Zweck zu erfüllen. In dieser Weise können ehemals
irrational begründete Institutionen nachträglich eine rationale
Begründung erhalten. So war die Ehe früher ein heiliges Sakrament,
während die Ehe heute zweckmäßig ist, weil die Familie eine geeignete
Form der "`Brutpflege"' darstellt. Umgekehrt können Institutionen
rational kritisiert werden, indem festgestellt wird, daß sie dem
Zweck, den sie erfüllen sollen, nicht dienlich sind, oder daß ihr
Zweck nicht mehr als sinnvoll anzusehen ist. Nun könnte eingewandt
werden, daß doch die Frage, ob ein Zweck sinnvoll sei, sich nicht
rational entscheiden lasse. Dies ist zwar richtig, doch selbst wenn
der Mensch nicht im Geringsten autonom ist, sondern ihm Zwecke nur
dadurch sinnvoll erscheinen können, daß sie von einer mächtigen
Institution verkörpert werden, ist es immer noch möglich, bestimmte
Institutionen im Namen anderer zu kritisieren, ja es ist sogar
prinzipiell möglich, alle Institutionen auf diese Weise in Frage zu
stellen - nur nicht alle auf einmal. Probleme entstehen lediglich,
wenn sich die Institutionen allzu rasch wandeln, da die menschliche
Gefühlswelt immer eine gewisse Zeit braucht, sich an neue
Institutionen anzupassen. Außerdem müssen Institutionen, welche auch
eine Abstimmungsfunktion leisten, sich gesamtgesellschaftlich
durchsetzen, was wiederum Zeit in Anspruch nimmt. Aber abgesehen
davon, daß es schwierig ist, diese Leistungsgrenze zu ermitteln, ist
der seelische Anpassungsaufwand für rational begründete Institutionen
wesentlich geringer, wodurch die Gefahr allzu rascher Veränderung
wenigstens teilweise wieder ausgeglichen wird. Als historisch
unrichtig muß Gehlens Behauptung angesehen werden, daß Aufklärung
stets nur eine destruktive aber niemals eine konstruktive Wirkung
habe.\footnote{Vgl. Gehlen, Hypermoral, S.102.} Aus der
Aufklärungsepoche Ende des 18.Jahrhunderts stammen viele großartige
Institutionen, z.B. der moderne Verfassungsstaat. Und mit einer durch
und durch auf aufklärerischem Gedankengut beruhenden Verfassung haben
es die Vereinigten Staaten von Amerika zur Weltmacht gebracht.

Wie ist nun die Kollisionsgefahr zwischen humanitärer Moral und dem Eigenethos
von Institutionen zu bewerten? In der Tat wäre es sehr problematisch, wenn die
Menschen in ihrem alltäglichen Leben jederzeit und bei jeder Handlung zu
berücksichtigen hätten, ob diese Handlung sich mit den hohen Forderungen der
Humanität im Einklang befindet. Bei näherem Hinsehen stellt sich jedoch
heraus, daß die Forderung, die Menschlichkeit müsse dem Institutionenethos,
d.h. den Sekundärtugenden täglicher Pflichterfüllung, vorhergehen, kaum noch
Probleme bereitet und im Gegenteil sogar geradezu natürlich erscheint, wenn
sie auf die unmittelbaren Handlungen sowie den Zweck der Institution
beschränkt wird aber nicht mehr die für den einzelnen meist unabsehbaren
Folgewirkungen des eigenen Handelns umfaßt. In der Regel treten solche
Kollisionen im Alltagsleben eher selten auf. Wenn sie jedoch auftreten, so ist
es üblich der Menschlichkeit vor dem Institutionenethos den Vorrang zu geben:
Ein Angestellter, der zum Nutzen seiner Firma illegale Geschäfte abwickelt
wird ebenso bestraft wie der Eifersuchtsmörder, der sich durch die Institution
der Ehe zu seiner Schandtat berechtigt fühlt.  Soldaten oder
Verwaltungsbeamte, die auf Anweisung Verbrechen begehen, können sich nicht mit
dem Befehlsnotstand herausreden, sie müssen sich für ihr Handeln unmittelbar
rechtlich verantworten. Man kann ohne allzu große Übertreibung sogar
behaupten, daß der Vorrang des Humanitätsethos in gewisser Weise zu den
normativen Grundlagen der Bundesrepublik gehört, stehen doch die Grund- und
Menschenrechte in der rechtlichen Normenhierarchie an der Spitze.

Die bisherigen Beispiele waren insoweit unproblematisch, als sie nur
mögliches Fehlverhalten innerhalb von Institutionen betrafen, die als
solche moralisch einwandfrei waren. Schwerer ist es von Menschen zu
verlangen, daß sie es erkennen, wenn die Institution, der sie dienen,
unmenschlich ist. Sich gegen eine gesellschaftlich anerkannte und
vielleicht sogar prestigeträchtige Institution zu stellen erfordert
ein gehöriges Maß an Mut und selbständigem Denkvermögen. Dagegen, so
weitreichende Forderungen an den Einzelnen zu stellen, spricht
zweierlei: Einmal könnte man darin eine moralische Überforderung des
Einzelnen sehen, die ihn mit Aufmerksamkeitsansprüchen und der
möglichen Pflicht, den Helden zu spielen, überlastet. Zweitens könnte
diese Forderung die Grundlagen von Staat und Gesellschaft aushöhlen,
da das Funktionieren unübersehbar komplexer Gesellschaftssysteme ganz
wesentlich darauf beruht, daß Autorität anerkannt wird. Hätte nun
jeder Einzelne das Recht und sogar die Pflicht, über die moralische
Legitimität dieser Autoritäten zu entscheiden, so könnte dies fatale
Folgen bis hin zum Abgleiten in die Anarchie haben.

Diese möglichen Einwände sollen nun im Einzelnen diskutiert werden.

\paragraph{1.Die Überforderung des Einzelnen durch das Humanitätsethos}

Nimmt man an, daß eine bestimmte Institution - z.B. der nationalsozialistische
Staat - unmenschlichen Zwecken dient, so kann man davon ausgehen, daß auch ein
an abgelegener Stelle in dieser Institution tätiger Mensch irgendwann einmal
mit dieser Unmenschlichkeit in Berührung kommen kann: Der Schreibtischtäter
kennt die Anweisungen, die über seinen Schreibtisch laufen, der Bahnbeamte,
der mit den Deportationszügen zu tun hat, weiß unter welchen Bedingungen die
Deportationen stattfinden, der Hilfspolizist, der zu einer Massenerschießung
befohlen wird, kennt die Verbrechen aus seinen eigenen Handlungen. Um die
Unmenschlichkeit der Institution zu erkennen, muß ein Mensch in einer solchen
Situation also keine weitläufigen Untersuchungen anstellen sondern lediglich
dem mehr trauen, was er mit eigenen Augen sieht, als der Deutung, die er "`von
oben"' bekommt. Daß dies leider selten genug der Fall ist\footnote{Vgl.
  Christopher Browning: Ganz normale Män\-ner. Das Re\-serve-Polizeibatallion
  101 und die "`Endlösung"' der Judenfrage in Polen, Hamburg 1996, S.208ff.}
spricht, da es im Prinzip leicht möglich ist, nicht dagegen es als Forderung
aufzustellen. Nun könnte jedoch mit dem Hinweis geantwortet werden, daß es in
den oben skizzierten Fällen, unter den Bedingungen einer totalitären Diktatur
nur unter Lebensgefahr möglich gewesen wäre, sich dem Gehorsam zu entziehen.
Darauf ist zu antworten, daß es eben Situationen gibt in denen ein Mensch nur
die Wahl hat, entweder ein Held oder ein Schurke zu sein. Es ist nicht
unbillig, zuweilen von Menschen Heldentaten zu fordern. Üblicherweise wird ja
auch von jedem Soldaten, der in die Schlacht geschickt wird, erwartet, daß er
für das Kriegsziel sein Leben riskiert. Bezogen auf den Zweiten Weltkrieg
könnte man daher zugespitzt sagen: Mit dem selben Recht, mit dem von einem
Soldaten der Alliierten gefordert wurde, sein Leben im Kampf gegen Hitler aufs
Spiel zu setzen, hätte man von einem deutschen Soldaten fordern können, sein
Leben beim Desertieren zu riskieren.\footnote{Denkt man diesen Gedanken zu
  Ende, so gelangt man zu dem allgemeinen Grundsatz, daß jeder Soldat in jedem
  Krieg das Recht {\em und} die Pflicht hat, sich selbst die Seite
  auszusuchen, auf der er kämpft.} Wird also erwartet, daß der Einzelne auch
für seine Institution verantwortlich ist, so übersteigt die Anforderung an das
Heldentum des Einzelnen folglich nicht das übliche und von allen
Nicht-Pazifisten ohne Widerspruch anerkannte Maß. Übrigens vertritt auch
Gehlen die Auffassung, daß der Einzelne für das Agieren seiner Institutionen
haftbar ist.\footnote{Vgl. Gehlen, Hypermoral, S.98f.} Konsequenterweise müßte
der Einzelne dann auch die Pflicht (mindestens aber das Recht) haben, sich aus
moralischen Gründen gegen das Ethos der Institution zu entscheiden.

\paragraph{2.Der Gehorsam und das Gewissen}

In den Situationen, in denen es für den Einzelnen geboten wäre, sich aus
moralischen Gründen gegen die Institutionen zu entscheiden, ist er in der
Regel auf sein eigenes Gewissen verwiesen, denn die gesellschaftliche Moral
heiligt für gewöhnlich die herrschenden Institutionen. Eine Institution bei
der dies nicht der Fall wäre, würde vermutlich schnell abgeschafft werden,
oder sie würde von selbst verschwinden. Wird nun grundsätzlich vom Einzelnen
Verantwortung für die Institutionen gefordert, so hieße dies der freien
Gewissensentscheidung Tür und Tor zu öffnen, einer Gewissensentscheidung, die
dann oft genug falsch oder verlogen ausfallen wird und so bestenfalls zur
einer erhöhten Zahl von Deserteuren, untreuen Mitarbeitern oder abtrünnigen
Verwaltungsbeamten führt, die schlimmstenfalls aber auch Terroristen die
moralische Selbstermächtigung zu beliebigen Verbrechen möglich macht.
Andererseits kann die gegenteilige Forderung, unter allen Umständen die
(institutionelle) Pflicht zu tun, nicht weniger fatale Folgen haben. Mit
dieser Moral lassen sich die Menschen zu dienlichen Werkzeugen für die
schlimmsten Verbrechen vom Weltkrieg bis zum Völkermord machen. Denn, mag das
Gewissen auch eine tyrannische Macht sein, so ist es der blinde Gehorsam
nicht weniger.

Angesichts dieser Situation scheint es immer noch besser,
grundsätzlich von der Gültigkeit des Institutionenethos auszugehen, aber
dabei klarzustellen, daß die Tugenden der Disziplin und Pflichterfüllung
gegenüber den Institutionen neben der Menschlichkeit sekundär bleiben.

Von diesem Standpunkt aus läßt sich nun die Frage nach der Geltung der
"`vaterländischen Tugenden"' beantworten. Das Humanitätsethos steht weder
der Heimatliebe im Sinne einer persönlichen Präferenz noch dem
Patriotismus im Sinne eines maßvoll gehandhabten moralischen Gebotes
entgegen. Allerdings bleibt der Patriotismus der Humanität untergeordnet, und
die Grenze liegt dort, wo der Patriotismus in Chauvinismus, d.i. der
moralischen Abwertung anderer Nationen oder Völker, übergeht, oder wo der
Patriotismus aus falsch verstandenem vaterländischem Stolz den Staat und die
Souveränität als Selbstzweck betrachtet und dadurch eine
maßvoll-vernünftige Sicherheitspolitik erschwert.

\section{Gehlens Programm der pluralistischen Ethik und der Vorwurf der Moralhypertrophie}

\subsection{Die Unzulänglichkeit von Gehlens pluralistischer Ethik}

Zum Abschluß der Kritik von Gehlens Auffassungen soll auf einer mehr
grundsätzlichen Ebene Gehlens Programm einer pluralistischen Ethik
betrachtet werden. Prinzipiell muß von einer philosophischen Ethik verlangt
werden können, daß sie in jeder (moralisch fragwürdigen)
Handlungssituation eine eindeutige Lösung vorgibt, wie in dieser Situation
moralisch richtig zu handeln ist. Eine Philosophie, die es zuläßt, daß in
ein und der derselben Situation gegensätzliche Handlungsweisen
gleichermaßen moralisch geboten sind ({\em tragische Situation}), stellt
sich selbst ein Armutszeugnis aus, denn sie hat die Aufgaben der
philosophischen Ethik nicht lösen können. Zu behaupten, tragische
Situationen seien in der Ethik unvermeidlich, würde von ebensowenig
Originalität zeugen, wie die Behauptungen, alles sei erlaubt, oder jedes
Wissen sei relativ. Gehlens ethischer Pluralismus geht davon aus, daß für
verschiedene Lebens- bzw. Handlungsbereiche unterschiedliche Ethosformen
zuständig sind. Dennoch vermeidet Gehlens ethischer Pluralismus tragische
Situationen nicht gänzlich, denn die Grenzen der Gültigkeitsbereiche der
verschiedenen Ethosformen sind unscharf. Dieses Ergebnis, so dürftig es für
den Anspruch aller philosophischen Ethik, Antwort auf die Frage "`Was soll
ich tun?"' zu geben, ist, müßte immer noch dann akzeptiert werden, wenn es
sich durchaus nicht vermeiden ließe. So sind wir ja auch in der Frage der
Letztbegründung der Ethik praktisch gezwungen, den ethischen Dezisionismus
hinzunehmen, da bisher noch keine Ethik hat bewiesen werden können, obwohl
der Dezisionismus eigentlich nicht weniger skandalös ist. Für Gehlen ist
dieses Ergebnis aus zwei Gründen unvermeidbar. Einmal ist es die
Folge seines empirischen Ansatzes, denn faktisch sind die Grenzen zwischen
den Ethosformen in der Tat fließend. Daß dies jedoch nicht dem
philosophischen Versuch im Wege steht, hier eine sinnvolle Grenzbestimmung zu
treffen, sofern man nur zwischen dem faktisch Vorhandenen und dem normativ
Geltenden unterscheidet, ergibt sich aus der bereits ausgeführten Kritik an
dem empirischen Ansatz in der Ethik. Der zweite Grund besteht darin, daß
jeder Versuch, die tragischen Situationen durch einen ethischen Monismus zu
vermeiden, nach Gehlens Ansicht zu einer Enthemmung von Aggressivität und zur
Moralhypertrophie führen muß.

Den Vorwurf der Aggressivität, der Nivellierung und des
Gesinnungsterrors gegen den "`Humanitarismus"' zu erheben, wird Gehlen
vom ersten bis zum letzten Kapitel seines Werkes nicht müde. Zum
Schluß des Buches versteigt Gehlen sich sogar zu dem waghalsigen
Vergleich des innerhalb der Bundesrepublik laut werdenden
"`Humanitarismus"' mit der Unterdrückung von nationaler und ethnischer
Identität, ja sogar mit Völkermord.\footnote{Vgl. Gehlen: Hypermoral,
  S.185.} Dabei betreffen die Beispiele von "`humanitaristischer"'
Aggressivität, die Gehlen wiedergibt, mit Ausnahme des historischen
Beispiels der französischen Revolution bestenfalls Fälle verbaler
Aggressivität.\footnote{Späterer Zusatz (6.2.2006): Ein kritischer Leser hat 
% (Stefan Hernold, der nicht genannt werden möchte)
  mich zurecht darauf hingewiesen, dass man die erhebliche geistige
  Intoleranz, die zeitweise zu den Auswüchsen der 68er Bewegung
  gehörte, in diesem Zusammenhang nicht unberücksichtigt lassen
  sollte. Bezieht man den zeitgeschichtlichen Kontext mit ein, dann
  erscheint Gehlens Polemik, wiewohl übertrieben, so doch um einiges
  verständlicher, als das in meiner Charakterisierung in diesem Absatz
  zum Ausdruck kommt.} Gehlens Vorwurf der Aggressivität ist um so
weniger berechtigt, als er sich selbst streckenweise zum Anwalt kaum
gehemmter Machtausübung aufwirft und die Freuden kriegerischer
Gewaltausübung gar als "`Daseinsprämien"'\footnote{Gehlen: Hypermoral,
  S.116.}  preist, ohne daß ihn dies daran hindert, sich
selbstmitleidig über die Gewalt und Intoleranz des "`Humanitarismus"'
zu empören. Wenn man freilich so argumentiert, dann wird Christus in
die Rolle des Tyrannen gedrängt, und der Großinquisitor darf sich als
der Gekreuzigte fühlen.

\subsection{Die bedingte Berechtigung von Gehlens Vorwurf der Moralhypertrophie}

Muß also der Vorwurf der Aggressivität des übersteigerten "`Humanitarismus"'
wegen seiner maßlosen Überzogenheit zurückgewiesen werden, so ist Gehlens
Diagnose der Moralhypertrophie andererseits im Kern zutreffend.  Mit
"`Moralhypertrophie"' kennzeichnet Gehlen eine Einstellung, die alle Probleme
in Staat und Gesellschaft als primär moralische Probleme auffaßt. Jede
politische Entscheidung erscheint dann als eine moralische Frage ersten
Ranges, bei der die Humanität selbst auf dem Spiel steht, wobei darüber
hinweggegangen wird, daß politische Probleme zu einem großen Teil aus
Sachfragen bestehen oder aus der Abwägung von Interessen (d.h nicht von
moralischen Gütern), und daß selbst dann, wenn eine politische Entscheidung
moralische Fragen aufwirft, in der Regel dennoch eine pragmatische
Entscheidungsfindung geboten ist. Die Einstellung der Moralhypertrophie geht
einher mit der Auffassung, daß die menschlichen Handlungen im wesentlichen
durch gute oder böse Absichten bzw. das Gute und das Böse an den Absichten
motiviert sind. Psychologisch ist die Einstellung der Moralhypertrophie bei
ihren Vertretern gekennzeichnet durch eine Art latenter Empörung, durch eine
Verbissenheit, die nur darauf wartet, sich in giftigen Lamentos über die
Schlechtigkeit "`der Politiker"' oder des Menschen ganz allgemein Luft zu
machen. Eine Ursache des Auftretens der Moralhypertrophie in modernen
Gesellschaften liegt vermutlich in der Undurchschaubarkeit ihrer
Lebensgrundlagen und der Komplexität ihres Gefüges. Die Moralhypertrophie
kompensiert diese Undurchschaubarkeit, indem sie durch ihre simple
Gut-Böse-Logik scheinbar jeden Vorgang verständlich macht, wobei sie sich auf
den psychologischen Mechanismus stützt, daß dem Menschen schwierige Dinge
verständlich erscheinen, wenn sie auf etwas Vertrautes (in diesem Fall das
moralische Beurteilungsschema) zurückgeführt werden, wobei die sachliche
Stimmigkeit oder Unstimmigkeit dieser Rückführung keine Rolle spielt. Nach
Gehlen wird die Moralhypertrophie durch die Anspruchsmentalität der Bürger im
Sozialstaat erheblich gefördert. Dies läßt sich besonders an dem Agieren der
Interessengruppen im Staat aufweisen. Üblicherweise treten die
Interessengruppen so auf, als ob das Interesse ihrer Mitglieder geradezu deren
gutes Recht sei. Daß dies in Wirklichkeit sogar zu Verzerrungen der
Gerechtigkeit führen kann, führte in jüngster Zeit die sehr unterschiedliche
öffentliche Resonanz vor Augen, die die Schließung von Kohlengruben und
Stahlwerken in Westdeutschland mit gewerkschaftlich gut organisierter
Belegschaft und die Abwicklung der ostdeutschen Industrie
begleitete.\footnote{Es wäre jedoch ein Fehler, das Vorhandensein von
  Interessengruppen im Staat grundsätzlich zu kritisieren, denn die
  Interessengruppen erfüllen für den Staat lebenswichtige Funktionen: 1.Sie
  erkennen, bündeln und artikulieren Interessen, Probleme und Bedürfnisse der
  Bürger. 2.Sie organisieren faktisch vorhandene gesellschaftliche Macht und
  üben sie in einigermaßen geregelten und legalen Bahnen aus. 3.Sie beziehen
  als zivilgesellschaftliche Institutionen den Bürger in den politischen
  Prozeß ein und vermitteln so zwischen Staat und Bürger. Was konservative
  Kritiker von Interessengruppen dabei häufig übersehen ist, daß die
  Interessengruppen egoistische Einzelinteressen und substaatliche
  Machtzentren weniger schaffen als (in legaler und geregelter Form) zum
  Ausdruck bringen.}

Anders, als Gehlen meint, resultiert das Phänomen der Moralhypertrophie nicht
schon daraus, daß "`ein Ethos die Herrschaft über die anderen beansprucht"',
d.h. aus einer monistischen Konstruktion der Ethik. Die Moralhypertrophie
oder, um einen gebräuchlicheren Ausdruck zu verwenden, der
"`Moralismus"'\footnote{Ausführlich und wesentlich überzeugender als Gehlen
  hat dieses Phänomen Hermann Lübbe in einem Essay analysiert. - Hermann
  Lübbe: Politischer Moralismus. Der Triumph der Gesinnung über die
  Urteilskraft, Berlin 1987. - Lübbes historische These, daß der Moralismus
  eine notwendige psychische Voraussetzung für die Verbrechen im Dritten Reich
  ist, trifft, wie mir scheint, jedoch nur für bestimmte Tätergruppen zu.},
ist ein Phänomen, das nicht an eine bestimmte Moral gebunden ist, und das
unter verschiedenen Bedingungen auftreten kann.  Geradezu wesenstypisch ist
der Moralismus für totalitäre Staaten. Dann erscheint Nachlässigkeit bei der
Arbeit als Sabotage, die Unterschlagung von "`Volkseigentum"' wird zum
politischen Delikt, oder die Bekanntschaft mit Juden gilt als eine öffentliche
Schande. Aber auch in Demokratien tritt der Moralismus als Mode und
Zeiterscheinung immer wieder auf. Dabei ist der Moralismus nicht an eine
bestimmte politische Richtung gebunden. Die nationalistischen Agitatoren
beispielsweise, die sich in der Weimarer Republik über Gustav Stresemanns sehr
machtbewußte, aber pragmatische Außenpolitik empörten, verhielten sich nicht
weniger moralistisch, als jene evangelischen Theologen in der Bundesrepublik,
die Arnold Gehlen im Visier hat. Ein Hauch von Moralismus scheint sich auch in
Gehlens Politikvorstellung einzuschleichen, wenn er die Ehre als politische
Kategorie betrachtet, denn die Ehre ist eine hochmoralische und zugleich
radikal antipragmatische Kategorie.

\subsection{Die Grenzen des Vorwurfs der Moralhypertrophie}

Auch wenn eine pluralistische Ethik für den Moralismus vermutlich weniger
anfällig ist, so stellt eine monistische Ethik wie die humanistische Ethik
weder selbst eine Form von Moralhypertrophie dar, noch führt sie
zwangsläufig oder mit innerer Logik dorthin. Gegenüber Gehlens
pluralistischer Ethik hat die humanistische Ethik jedoch den Vorteil, daß
sie tragische Situationen nicht zustande kommen läßt. Zusätzlich hat der
konkrete ethische Pluralismus, den Gehlen vorschlägt, den Nachtteil, daß
durch die Auffassung ethischer Imperative als "`Sozialregulationen"' der
Unterschied zwischen gerechten und ungerechten "`Sozialregulationen"' leicht
verloren geht. Auch unterscheidet Gehlen nicht besonders sorgfältig zwischen
ethischen Werten und anderen Werten (ein Irrtum der durch die
"`Wert"'-Terminologie anstelle der Rede von ethischen Imperativen stark
begünstigt wird). So wird dann beispielsweise ein vitaler Wert, wie die
Gesundheit, in die Nähe ethischer Werte gerückt. Dabei ist die Gesundheit
keineswegs ein ethischer Wert. Höchstens kann der Schutz der Gesundheit ein
ethischer Imperativ sein, was ein feiner aber - in Bezug auf diesen und noch
mehr auf andere vitale Werte - wichtiger Unterschied ist. Diesen Unterschied
vernachlässigt Gehlen auch, wenn er der humanistischen Ethik "`Formalismus"'
vorwirft.\footnote{Vgl. ebda., S.83., Vgl. ebda., S.143.} Denn wenn die
Würde des einzelnen Menschen für den Humanismus der höchste ethische Wert
ist, so impliziert dies noch längst nicht, daß jedes Tun und jede
Eigenschaft von Menschen als ethisch wertvoll angesehen wird. Es besteht
durchaus kein Widerspruch, zwischen der Anerkennung der Menschenwürde eines
Straftäters bzw. seines Wertes als Mensch und der Mißbilligung seiner Tat.

\section{Gegenentwurf: Hierarchische Ethik und Humanität als Primärtugend}

Abschließend soll wenigstens kurz skizziert werden, wie eine humanistische
Ethik unter sinnvoller Berücksichtigung der Gehlenschen Kritik konstruiert
werden kann, ohne daß dabei die Grundprinzipien
der humanistischen Ethik aufgegeben werden. Dafür muß berücksichtigt
werden, daß eine unmittelbare und unbegrenzte Umsetzung humanitärer
Prinzipien nicht in jeder Situation oder in jedem Bereich des Handelns ohne
weiteres möglich ist. Es heißt jedoch weder das Prinzip der Humanität
aufgeben noch den kollektiven Selbstmord riskieren, wenn man in diesen
Fällen immer noch eine maximale Beachtung humanitärer Prinzipien fordert.
Weiterhin ist in Rechnung zu stellen, daß die Ethik einer Gesellschaft, wenn
man die ethischen Normen als Sozialregulationen auffaßt, nicht
ausschließlich auf humanitäre Tugenden gegründet werden kann.
Sekundärtugenden wie Fleiß, Disziplin, Pflichterfüllung und Gehorsam im
Sinne der Respektierung legitimer und wohlbegründeter Autorität sind
unerläßlich. Der Humanismus kann diesen Notwendigkeiten Rechnung tragen,
wenn er die Sekundärtugenden zuläßt (was er für gewöhnlich tut), soweit
ihre Erfüllung nicht den Forderungen der Menschlichkeit widerspricht.
Schließlich muß die Gefahr des Moralismus berücksichtigt werden. Dieser
Gefahr, der nicht allein die humanistische Ethik ausgesetzt ist, läßt sich
mit etwas intellektueller Disziplin dadurch begegnen, daß man nicht jeden
Bagatellfall gleich als Angriff auf die Menschenwürde interpretiert. Hierbei
können Sekundärtugenden im Übrigen sogar hilfreich sein, da sie solche
weniger gravierenden Problemfälle gewissermaßen abfangen. Eine Beleidigung
beispielsweise wäre dann in erster Linie ein Verstoß gegen die Höflichkeit
und nicht schon gegen die Menschenwürde, auf die sie nur unter einiger
Überstrapazierung dieses Begriffs bezogen werden könnte.

Im Ergebnis erhalten wir auf diese Weise eine {\em hierarchische Ethik}. An
der Spitze dieser Ethik steht die Würde und Gleichheit des Menschen. Andere
Werte, seien sie ethischer oder anderer Natur, spielen die Rolle mittlerer
Prinzipien oder sekundärer Tugenden, was bedeutet, daß sie höchstens
insoweit Gültigkeit beanspruchen können, als sie zu dem Humanitätsethos
nicht im Widerspruch stehen. Im Konfliktfall hat immer die Menschlichkeit das
letzte Wort, sie ist in diesem Sinne {\em Primärtugend}. Die Humanität muß
jedoch, um unter realistischen Bedingungen anwendbar zu sein,
verantwortungsethisch, d.h. als prinzipiell einer utilitaristischen Abwägung
gegenüber sich selbst fähig gedacht werden. Die hierarchische Ethik macht
sich so die Vorteile von Gehlens ethischem Pluralismus (relativer Schutz vor
Moralismus, eine der Differenziertheit des gesellschaftlichen Lebens
angemessene Vielfalt von Prinzipien) zu eigen, aber sie vermeidet seine
Nachteile (Möglichkeit "`tragischer Situationen"', Verwirrung in Bezug auf
die unterschiedliche Wichtigkeit verschiedenartiger Werte, Gefahr der
Verselbständigung bestimmter gesellschaftlicher Funktionen wie der
Sicherheitspolitik).

Es ist dabei zu trennen zwischen Humanismus und humanistischer Ethik. Zwar
geht die humanistische Ethik aus dem Humanismus als einem durch
Selbsterziehung anzustrebenden und durch Bildung vermittelten Ideal
harmonisch hervor, doch auch wenn man, wie Gehlen, im Humanismus die Gefahr
eines in die Sinnleere menschlicher Selbstbezogenheit führenden Ideals
sieht, so muß man deswegen noch nicht die humanistische Ethik verwerfen.

\section{Schluß}

Zum Abschluß soll die Frage beleuchtet werden, ob unter dem Eindruck
von Gehlens Kritik des Humanitarismus der Begriff des Humanismus
gewandelt werden muß, da sich der Humanismus vielleicht in einigen
Punkten als ein nicht mehr haltbares Ideal erwiesen hat.

Meiner Meinung nach übersteht zumindest die ethische Seite des
Humanismus die Kritik Gehlens weitgehend unbeschadet. Weder Gehlens
anthropologische Ableitung noch seine historische Entlarvung noch
seine durchaus reaktionären Ausführungen zur Politik enthalten
schlüssige Gründe gegen die humanistische Ethik in dem von mir
definierten Sinne (Kapitel 3.1). Im wesentlichen hängt dies mit
technischen Mängeln von Gehlens Argumentationsweise zusammen. So wie
Gehlen in seinem Werk Moral und Hypermoral die Probleme der Ethik
angeht, lassen sich ethische Fragen eben nicht entscheiden. Lediglich
in zwei Punkten scheinen mir die Mahnungen Gehlens
berücksichtigenswert: 1.Die humanistische Ethik darf nicht
gesinnungsethisch (miß-)verstanden werden. (Als ein solches
Mißverständnis aus der Sicht der humanistischen Ethik könnte man etwa
den Pazifismus der Friedensbewegung in den 80er Jahren ansehen, soweit
er moralisch begründet worden ist.) 2.Die humanistische Ethik darf nicht in
Moralismus ausarten: Weder dient sie in irgend einer Weise der
verständnismäßigen Erschließung der Welt (keine Ethik leistet dies
bzw. kann dies leisten), noch können alle menschlichen Lebensbereiche
in unmittelbarem Bezug auf die Prinzipien der humanistischen Ethik
ethisch geregelt werden. Insofern ist eine Vielfalt moralischer
Prinzipien erforderlich, die jedoch nicht pluralistisch nebeneinander
stehen, sondern hierarchisch einander über- und untergeordnet sind.

Abgesehen davon enthält Gehlens Werk Moral und Hypermoral nur eher
wenig, was von philosophischem Interesse ist. Zu denken wäre hier an
das Askeseideal und - wenn auch weniger in ethischer als in
anthropologischer Hinsicht - an Gehlens anthropologische Ableitung des
Humanitarismus. Ansonsten wirkt dieses Buch eher wie ein haßerfülltes
Pamphlet, in welchem ein verbitterter Konservativer seinem Frust über
die Gesellschaft und politische Kultur der zweiten deutschen
Demokratie Luft macht, und das sich streckenweise liest wie ein warmes
Plädoyer für ein bißchen mehr Faschismus in userer Zeit.

Damit, daß die humanistische Ethik weiterhin befürwortenswert ist, ist
allerdings die Frage noch nicht beantwortet, ob der Humanismus als Ideal noch
aktuell ist oder sein kann. In dieser Hinsicht ist es jedenfalls
bemerkenswert, daß Gehlen den Humanismus nicht - wie es offenbar manche
Postmodernisten tun - deshalb ablehnt, weil Humanismus in seinen Augen etwa
bedeutete, ein bestimmtes Wesen des Menschen oder eine bestimmte Form
menschlichen Lebens als Ideal tyrannisch zu verabsolutieren und damit alle
anderen Möglichkeiten des Menschseins in intoleranter Weise auszuschließen.
Vielmehr wirft Gehlen - soweit sich das aus den vereinzelten Bemerkungen in
Moral und Hypermoral zu dieser Frage schließen läßt - ganz im Gegenteil dem
Humanismus seinen Formalismus vor, der darin besteht, den Menschen, wie auch
immer und was auch immer er ist, also gerade ohne den Vorbehalt, daß der
Mensch einem bestimmten Wesensideal von Menschsein genügen muß, zu
verherrlichen.

\newpage

\begin{thebibliography}{99}

\bibitem{browning} Brow\-ning, Christopher: Ganz normale Män\-ner. Das
Re\-ser\-ve-\-Polizei\-batallion 101 und die "`Endlösung"' der Judenfrage in
Polen, Hamburg 1996.

\bibitem{fest} Fest, Joachim: Die schwierige Freiheit. Über
die offene Flanke der offenen Gesellschaft, Berlin 1993.

\bibitem{freud} Freud, Sigmund: Massenpsychologie und Ich-Analyse /
Das Ende einer Illusion, Frankfurt am Main 1993.

\bibitem{gehlenHM} Gehlen, Arnold: Moral und Hypermoral. Eine
  pluralistische Ethik, Wiesbaden, 5.Aufl., 1986.

\bibitem{gehlenM1940} Gehlen, Arnold: Der Mensch. Seine Natur und seine
  Stellung in der Welt, Berlin 1940.

\bibitem{gehlenG2} Gehlen, Arnold: Gesamtausgabe. Band 2. Philosophische
Schriften II. (1933-1938), Frankfurt am Main 1980.

\bibitem{gehlenG31} Gehlen, Arnold: Gesamtausgabe. Band 3. Der Mensch. Seine
Natur und seine Stellung in der Welt. Textkritische Edition unter
Einbeziehung des gesamten Textes der 1.Auflage von 1940. Teilband 1,
Frankfurt am Main 1993.

\bibitem{gehlenG32} Gehlen, Arnold: Gesamtausgabe. Band 3. Der Mensch. Seine
Natur und seine Stellung in der Welt. Textkritische Edition unter
Einbeziehung des gesamten Textes der 1.Auflage von 1940. Teilband 2,
Frankfurt am Main 1993.

\bibitem{gehlenG7} Gehlen, Arnold: Gesamtausgabe. Band 7. Einblicke, Frankfurt
am Main 1978. %Sicherheitsrisiken(1973),S.279-285, Was ist deutsch?(1971),S.413-426.

\bibitem{gehlenUS} Gehlen, Arnold: Urmensch und Spätkultur. Philosophische
Ergebnisse und Aussagen, Wiesbaden 5.Aufl., 1986.

\bibitem{grenz} Grenz, Friedemann: Adornos Philosophie in
Grundbegriffen. Auflösung einiger Deutungsprobleme, Frankfurt am Main
1974. %S.251 (225-251).

\bibitem{herz} Herz, John H.: Politischer Realismus und politischer
Idealismus. Eine Untersuchung von Theorie und Wirklichkeit, Meisenheim am
Glan 1959.

\bibitem{jonas} Jonas, Friedrich: Die Institutionenlehre Arnold Gehlens,
Tübingen 1966.

\bibitem{kahrstedt} Kahrstedt, Ulrich: Kulturgeschichte der römischen
Kaiserzeit, Bern 1958.

\bibitem{kant} Kant, Immanuel: Kritik der praktischen Vernunft, Hamburg 1990.

\bibitem{klages} Klages,Helmus / Quaritsch, Helmut (Hrsg.): Zur
geisteswissenschaftlichen Bedeutung Arnold Gehlens, Berlin 1994.

\bibitem{klaus} Klaus,Georg / Buhr,Manfred (Hrsg.): Philo\-soph\-isch\-es
Wör\-ter\-buch. 2. Band, Leip\-zig 1975.

\bibitem{luebbe} Lübbe, Hermann: Politischer Moralismus. Der Triumph der
  Gesinnung über die Urteilskraft, Berlin 1987.

\bibitem{morgenthau} Morgenthau, Hans J.: Macht und Frieden. Grundlegung
einer Theorie der internationalen Politik, Gütersloh 1963.

\bibitem{nietzsche} Nietzsche, Friedrich: Zur Genealogie der Moral. Eine
Streitschrift, Stuttgart 1993.

\bibitem{pflaum} Pflaum, Hans-Georg / Rubin, Berthold / Schneider,
Carl / Seston, William: Rom. Die römische Welt. Frankfurt/M / Berlin
1963.

\bibitem{weber} Weber, Max: Wissenschaft als Beruf, Stuttgart 1996.

\end{thebibliography}

\end{document}


% ,daß wir über den Sinn des Daseins oder des Lebens nichts aussagen können,
% daß aber einen solchen Sinn zu unterstellen notwendig, nicht nur erlaubt
% ist, weil das Leben zur Lösung seiner uns unbekannten Aufgabe des
% Bewußtseins, des Sinnbereichs selber, bedarf. Mensch 1940, S.466.

