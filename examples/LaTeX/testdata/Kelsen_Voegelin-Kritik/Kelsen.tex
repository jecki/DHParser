% Errata
% Zeile 3891: "against the Holy" fehlt dort ein "the"?

\chapter{Hans Kelsen: A New Science of Politics}
\selectlanguage{\USenglish}

\label{M1}
\section{A Crusade against Positivism}

It is an undeniable fact that the extraordinary progress science has
achieved in modern times is, in the first place, the result of its
emancipation from the bonds in which theology had held it during the
Middle Ages. The principle of truthfully describing reality and
explaining it on a strictly empirical basis, without having recourse
to theology or any other metaphysical speculation, is called
positivism. It is another fact that a positivistic social science is
not in a position to justify an established social order as the
realization of absolute values. For it can evaluate a social
institution only as a means appropriate to achieve a presupposed end,
but inappropriate if another end is presupposed. That is to say, it
can evaluate a social institution only conditionally, or, what amounts
to the same, it can attribute to it only a relative value, ``value''
-- positive or negative -- meaning the relationship of a means to an
end. This is a relationship of cause and effect, and can be
ascertained in a scientific way on the basis of human
experience. Consequently, a positivistic social science cannot
evaluate an end which is not itself a means for another end, but an
ultimate end. It cannot evaluate a social institution unconditionally,
or, what amounts to the same, it cannot attribute to it an absolute
value. The absolute in general, and absolute values in particular,
belong to a transcendental sphere which is beyond scientific
experience, the field of theology and other metaphysical
speculations. Hence scientific positivism goes hand in hand with
relativism.

When the foundations of the established social order are shaken by wars
and revolutionary movements and the need for an absolute, not merely
relative, justification of that order becomes urgent, religion, and with
religion theology \label{M2}and other metaphysical speculations are brought to
the front of intellectual life and become ideological instruments of
politics. In view of the great importance science assumes in modern
society, the -- always existing but in periods of social equilibrium
repressed -- tendency of using social science for the same purpose increases.
And this tendency manifests itself in a passionate opposition against
relativistic positivism and the attempt to bring science again under the
sway of theology and other metaphysical speculations.

A characteristic and very serious symptom of this tendency is a recently
published book which has created widespread comment: Eric Voegelin's The
New Science of Politics.\footnote{Eric Voegelin, \emph{The New
    Science of Politics}. An Introduction. Charles Walgreen Lectures.
  The University of Chicago Press, 1952.} It undertakes not more and not
less than a complete restoration of political science, which is
necessary because -- as Professor Voegelin asserts -- this science has
been destroyed by positivism. Voegelin does not underestimate the
gigantic import of his enterprise. He says: ``When science is as
thoroughly ruined as it was around 1900, the mere recovery of
theoretical craftsmanship is a considerable task, to say nothing of the
amounts of materials that must be reworked in order to reconstruct the
order of relevance in facts and problems.''\footnote{L.c., p. 23.} In
opposition to ``destructive positivism'' which shirked its task to
``penetrate to a theoretical understanding of the source of order and
its validity'' that is, the idea of justice, the new science of
politics is to be established with respect to this task on the basis of
``metaphysical speculation and theological symbolization''; that is to
say, placed under the spiritual authority of Plato and Thomas Aquinas,
the main but not the only representatives of this type of
thinking.\footnote{L.c., p. 6.} \label{M3}Voegelin accuses positivism of
having destroyed science, but does not give any approximately clear
definition of that school of thought against which he pleads his grave
indictment. The collective term ``positivism'' in general, and the term
``positivistic'' social or political science in particular, comprises
many different types of theoretical systems which have only a negative
criterion in common: the refusal to have recourse to metaphysical -- and
that implies religious-theological -- speculation. Voegelin seems to be
conscious of this fact, for he speaks of ``the variety of positivistic
phenomena'' and considers it inappropriate to define positivism ``as the
doctrine of this or that outstanding positivistic
thinker.''\footnote{L.c., pp. 6f.} Hence the decisive trend in his fight
against positivism can be only the reaction against the
anti-metaphysical attitude prevailing in modern social philosophy and
science.  The emancipation of political science from metaphysics and
especially from theology does not go back as far in time as the
emancipation of natural science. Until the end of the 18th century,
theology kept political science under its strict control. The doctrine
that the state is a divine institution and the ruler an authority
ordained by God was almost generally accepted. Hence it is not exactly
a ``new'' science of politics at which Voegelin, according to the title
of his book, is aiming. It is a very old one, which has been abandoned
because it has been proved to be a pseudo-science, the instrument of
definite political powers. Voegelin sets forth against positivism
as a whole two arguments of a most general character. \label{M4}Both arguments
can easily be rejected. The first is ``the destruction worked by
positivism'', due to the assumption that the only scientific method --
which, consequently, is to be applied also by the social sciences -- is
the ``mathematizing'' method successfully applied by the natural
sciences. This argument is utterly wrong. For there is a school of
thought of outspoken positivistic, that is anti-metaphysical and
anti-theological, social science which expressly and emphatically
distinguishes between the problems of the social sciences to which the
methods of natural science may be applied with more or less
modifications, and the problems to which a wholly different method must
be applied. Since I consider myself as a typical representative of
positivism, I may refer to my essay ``Causality and
Imputation''\footnote{\emph{Ethics}, Vol. 61 (1950) pp. 1-11.}, in
which I summarize the results of the methodological doctrine
distinguishing between social sciences applying -- as do the natural
sciences -- the principle of causality, such as sociology, and social
sciences applying a totally different principle, that of imputation,
social sciences dealing with norms, such as ethics and
jurisprudence. These are the sciences which Voegelin has in mind when he
accuses positivistic social science of destructive effects, the sciences
dealing with the problem of right and wrong, of justice and
injustice. There can be no doubt that a scholar of such extraordinary
knowledge of literature is not unaware of this school of thought within
positivistic social science.

The second argument set forth against positivism is in truth identical
with the first one, in which it is implied. Positivism, Voegelin
asserts, makes \label{M5}the use of a method the criterion of science, instead
of measuring the adequacy of a method by its usefulness to the purpose
of science. He does not make any attempt at proving this criticism by
quoting writers guilty of this error. He reminds the destructive
positivists of the truth that ``different objects require different
methods''.\footnote{L.c., p. 5.} Voegelin certainly knows that nobody
else has insisted on this truth, by stigmatizing the logical fallacy of
``syncretism of methods'', so energetically as the above mentioned
positivist, whose main concern was, and still is, to show that the
object of certain social sciences is totally different from that of the
natural sciences, and that consequently a method other than the one
applied by the latter is adequate to the former.\footnote{Cf. Hans
  Kelsen, ``Was ist die Reine Rechtslehre.'' In: \emph{Demokratie und
    Rechtsstaat}. Festschrift zum 60. Geburtstag von Prof. Z.
  Giacometti, Zuerich 1953, S. 147.}

But he does not stick very consistently to the principle the violation
of which he lays to the door of positivism: that different objects
require different methods. He says -- as an objection against positivism
in social science -- that if we are lead by positivistic social science,
``to the notion that social order is motivated by will to power and
fear, we know that we have lost the essence of the problem somewhere in
the course of our inquiry -- however valuable the results may be in
clarifying other essential aspects of social order.''\footnote{Voegelin,
  l.c., p. 6.} That a positivistic social science may have valuable
results with respect to essential aspects of social order, is hardly
compatible with its utterly ``destructive'' character. But it is not
this inconsistency which counts in this connection. Voegelin asserts
that by being led to the notion that social order is motivated by will
to power and fear, we have lost the essence of the problem. Of which
problem? The one he mentioned before, namely the question of ``right and
wrong, of justice and injustice'': a question to which, as he suggests,
we may find an adequate answer ``in the Platonic Agathon, or the
Aristotelian Nous, or the \label{M6}Stoic Logos, or the Thomistic \emph{ratio
  aeterna}''\footnote{L.c., p. 6.} But this question concerns an object
wholly different from the object of the quest on to which destructive
positivism answers by referring to the ``will to power and fear.'' The
one is a problem of value and -- as Voegelin's reference to Plato shows --
the problem of the absolute value; the other a problem of facts, the
motives of human behavior by which social orders are established,
without regard to the question as to whether these orders do or do not
correspond to the absolute value of justice. It is a positivistic school
of social science that emphasizes the difference between those two
problems, as Voegelin does know very well. For it is just this
distinction which he later makes responsible for the destructive
effect of positivistic science which insists on this distinction. It is
therefore contrary to the principle according to which different objects
require different methods, that Voegelin reproaches positivistic science
for having lost its problem. The problem he has in mind is simply a
problem different from that at which positivistic social science is
directed by referring to the will to power and fear. Only by confusing
the two problems, he can say, as an argument against positivism, that
``the methods of a psychology of motivations are not adequate for the
exploration of the problem''\footnote{Ibid.}, namely, the problem of
absolute justice, which is not the problem of the positivistic science
against which he argues.

It is in particular for the solution of the problem of absolute justice
that Voegelin undertakes his restoration of the social science,
destroyed by positivism. What he suggests is in principle nothing but a
return to the metaphysical and theological speculation of Plato,
Aristotle and Thomas Aquinas. This is rather strange, for it is a
``science'' which Voegelin intends to restore and if the history of
science shows anything, it is the fact that true science, as an
objective cognition of reality independent from the wish and fear of the
subject of cognition must be separated from metaphysical-theological
speculation, that is to say, from the \label{M7}products of man's wishful or
fearful imagination of a transcendental sphere lying beyond his sensual
apperception controlled by his reason. If only that intellectual
attitude had prevailed which is manifested in the
metaphysical-theological speculation of Plato, Aristotle and Thomas
Aquinas, modern science could not have been developed. This exclusion of
metaphysics and theology from science does not mean that the use of a
certain method of cognition is made the criterion of science; it means
that the adequacy of metaphysical-theological speculation is measured by
its usefulness to the purpose of science. That this kind of speculations
is not only useless to the purpose of science but constitutes a serious
obstacle to its progress, is an undeniable fact, shown by the
intellectual history of mankind.

It may be argued that the exclusion of metaphysical and theological
speculations from science is justified only as far as natural science
is concerned; that in the field of social science the recourse to
metaphysics and theology -- and that means Religion -- is admissible
and even necessary, because this is the only way to arrive at a
solution of the most important problem of that science, the absolute
value, implied in the question of what is right and wrong, that is,
the question of justice. And indeed it cannot be denied -- as pointed
out -- that on the basis of a social science which abstains from such
a recourse no definite answer to the question of justice, excluding
any other, can be reached. However, it can be and has been shown that
the innumerable attempts which have been made from the earliest times
of antiquity until to-day to solve the problem of justice as an
absolute value by metaphysical-religious speculation have completely
failed. The results of these speculations are of two types
only.\footnote{Cf. Kelsen, \emph{Was ist Gerechtigkeit?}, Wien, 1953.}
If the values proclaimed are so substantial that they can be applied
to real social relations, they prove to be principles at the basis of
a positive social order established under \label{M8}definite economic,
political and other cultural conditions of a certain time and a
certain space, as, e.g., the absolute values maintained by Christian
theology.\footnote{Cf. Kelsen, ``Die Idee der Gerechtigkeit nach den
Lehren der christlichen Theologie.''  \emph{Studia philosophiae},
Cf. Kelsen, ``The Platonic Justice.'' \emph{Ethics}, 48, (1937/38),
pp. 367ff.; as to Aristotle's \label{change1} philosophy of justice:
Kelsen, ``The Metamorphoses of the Idea of Justice.''
\emph{Interpretations of Modern Legal Philosophies}. Essays in honor
of Roscoe Pound. 1947, pp. 390ff.} If, however, they are not of this
type, they are empty formulae which by their very nature as
transcendental truth exclude any definition that could confer on them
a content concrete enough to make them applicable in an unambiguous
way to social reality. Hence they can be, and actually are, used to
justify any positive social system whatever. This is exactly the case
with the Platonic Agathon, the Aristotelian Nous, the Stoic Logos and
the Thomistic \emph{ratio aeterna}, to which Voegelin wants to lead
back political science.

After having attributed the destruction worked by ``positivism'' ``in
the first place'' to the fact that positivistic social
science tried to apply the methods of mathematizing natural sciences to
social problems, Voegelin admits that ``a transfer of methods of
mathematical physics in any strict sense of the word to the social
sciences has hardly ever been attempted'' and declares: ``if positivism
should be construed in a strict sense as meaning the development of
social science through the use of mathematizing methods, one might
arrive at the conclusion that positivism has never
existed.''\footnote{Voegelin, l.c., p. 4} How could a positivism which --
as characterized by Voegelin -- never existed, destroy science? Hence
Voegelin must divert his attack from an existing positivistic social
science to ``the intention of making the social sciences 'scientific'
through the use of methods which as closely as possible resemble the
methods employed in sciences of the external world.''\footnote{L.c.,
  p. 7.} If, as Voegelin assumes, this intention has never been realized
-- otherwise it would be more than a mere ``intention'' -- it can hardly
have the destructive effect which justifies the heroic attempt \label{M9}of a
restoration of the science of politics. A mere ``intention'' can have no
effect at all. However, the attempt to approach certain problems of
social science by using methods similar to those applied in natural
science, that is to say, the attempt to find out a causal nexus among
social phenomena had led to quite satisfactory results. To mention only
two characteristic examples: the relationship which exists between
economic facts and political and legal organization, shown by
sociologists who follow, with reservations, the Marxian interpretation
of society; and the influence of certain religious ideas on forms of
economics, demonstrated by Max Weber. No objective critique can deny
these achievements.

The use of method as the criterion of science -- which, according to
Voegelin is one of the fundamental errors of positivism -- ``abolishes
theoretical relevance''. Hence positivistic social science is guilty of
the ``accumulation of irrelevant knowledge.'' This ``is the first of the
manifestations of positivism.''\footnote{L.c., pp. 8,9.} But this
manifestation has nothing to do with positivism. The accumulation of
irrelevant knowledge is not necessarily a characteristic of an
anti-metaphysical or religiously indifferent science. The accumulation
of irrelevant knowledge is avoided by the establishment of a definite
criterion of relevance, by determining a certain point of view from
where a distinction can be made between relevant and irrelevant facts.
This is possible without any recourse to metaphysics or religion or the
assumption of a ``transcendent truth''; and Voegelin does not make any
attempt to prove the contrary.

But, as a matter of fact, he does not maintain his accusation that the
first manifestation of the destruction of science by positivism is
accumulation of irrelevant knowledge. After having ridiculed ``the
fantastic accumulation of irrelevant knowledge through huge 'research
projects' whose most interesting feature is the quantifiable expense
that has gone into their production,'' he admits: ``Major research
enterprises which contain nothing but \label{M10}irrelevant materials are
rare, indeed, if they exist at all. ...~Even the staunchest positivist
will find it difficult to write a completely worthless book about
American constitutional law as long as with any conscientiousness he
follows the lines of reasoning and precedents indicated by the
decisions of the Supreme Court.''\footnote{L.c., p. 8, 9.} This can
only mean that in the field of constitutional law -- and the same is
true as far as all the other fields of political science are concerned
-- the destructive positivism, against which Voegelin is fighting,
simply does not exist.

One of his main objections against this positivism refers even to
writings of which he expressly declares that they operate ``on relevant
materials'', that the damage they have done ``is not due to an
accumulation of worthless materials''; that they, on the contrary,
furnish ``reliable informations concerning facts.''\footnote{L.c., pp.
  9, 10.} How, then, can the accumulations of irrelevant knowledge be
the first manifestation of the destruction of science by positivism?  If
it is not the accumulation of irrelevant knowledge, what else is wrong
with these positivists who, in spite of their destructive effect,
operate ``on relevant materials'' and furnish ``reliable informations''?
``Their principles of selection and interpretation had no proper
theoretical foundation but derived from the \emph{Zeitgeist}, political
preferences, or personal idiosyncrasies.''\footnote{L.c., pp. 9,10.} How
writers who are biased in this way can furnish ``reliable informations
concerning facts,'' is difficult to understand. Besides, none of the
defects referred to: \emph{Zeitgeist}, political preferences, personal
idiosyncrasies, have anything to do with positivism. In his choice of a
metaphysical assumption or a theological dogma an anti-positivistic
writer may be affected by the Zeitgeist, or political preferences, or
personal idiosyncrasies, just as a positivist in his selection or
interpretation of the material. As to examples of a positivistic social
science built on such improper foundations, Voegelin refers to ``the
treatises on Plato which discovered in him a precursor of Neo-Kantian
logic or according to the political fashions of the time, a
constitutionalist, a utopian, a socialist, or a Fascist
...''\footnote{L.c., p. 10.} These treatises have \label{M11}been written by
authors who belong to the most different schools and some of them were
anything else but ``positivists.'' The interpretations of Plato's
political philosophy, which Voegelin rejects, are possible from a
positivistic as well as from a non-positivistic point of view, and the
distinction between a realistic and a utopian or a democratic and
autocratic political doctrine may legitimately be applied to a
philosophy which Voegelin considers to be of the utmost importance for
our time. Besides, the fundamental categories of political thought
according to which Plato's political system is interpreted by allegedly
destructive positivists are taken from Plato himself, whom Voegelin
regards as the founder of a political science\footnote{L.c., p. 1.} to
the principles of which we should return. That any of the
interpretations of Plato with which Mr. Voegelin does not agree is the
result of an undue influence of the \emph{Zeitgeist}, political
preferences, or personal idiosyncrasies, is an assertion for which
Voegelin does not give the slightest proof.

In the same way, that is without any documentation, he asserts that
``histories of political ideas'' were ``unable to discover much
political theory in the Middle Ages'' because they ``defined politics in
terms of Western constitutionalism.'' Voegelin does not specify these
histories and does not show that the historians belonged to the
positivistic school of thought; nor does he indicate which positivists
``completely ignored the block of political sectarian movements which
culminated in the reformation''\footnote{L.c., p. 10.}; which ignorance
can certainly not be attributed to an anti-metaphysical attitude of the
historian. He then jumps, without any sufficient reason, to Gierke's
\emph{Genossenschaftsrecht}, to which he objects that it advocates the
``theory of the \emph{Realperson},'' which is certainly just the
opposite of a positivistic doctrine. After having thus stigmatized the
second manifestation by which ``science has been destroyed'', Voegelin
proceeds to the third manifestation. And \label{M12}this is the queerest
argument set forth against destructive positivism. It is ``the
development of methodology, especially in the half-century from 1870 to
1920.''\footnote{L.c., p. 10.} Voegelin emphasizes that ``the movement
was distinctly a phase of positivism'' because ``the perversion of
relevance, through the shift from theory to method, was the very
principle by which it lived.'' But, at the same time he admits: ``it was
instrumental in overcoming positivism.'' And how did destructive
positivism ``overcome'' positivism and thus perform a highly
constructive function? By insisting on methodological clarification it
achieved just that understanding the lack of which Voegelin declared as
one of the ``two fundamental assumptions'' of ``the destruction worked
by positivism:'' the understanding of the ``specific adequacy of
different methods for different sciences.''\footnote{L.c., p. 11.} If
positivism is at the same time destructive and constructive, lacking and
gaining the understanding the new science of politics considers as
essential, this science is fighting against an imaginary opponent. As
representatives of the destructive positivism Voegelin denounces two of
the most prominent philosophers of the 20th century: Husserl and
Cassirer. But at the same time he recognizes that their works constitute
``important steps towards the restoration of theoretical relevance.''
Nevertheless he maintains, precisely in this connection, his accusation
of destruction of science, although he concedes that ``the movement as a
whole, therefore, is far too complex to admit of generalizations ...''
Yet it is just the movement as a whole which Voegelin accuses of having
destructed science.

The destruction of science by positivism is -- according to Voegelin --
due above all to its ``attempt at making political science (and the
social sciences in general) 'objective' through methodologically
rigorous exclusion of all 'value-judgments'.''\footnote{L.c., p. 11.}
But, on the other hand, he admits that this attempt did ``awaken
the consciousness of critical standards'' and ``insofar as the attack
on value-judgments was an attack on uncritical opinion under the guise
of political science, it had the wholesome effect of \label{M13}theoretical
purification.''\footnote{L.c., p. 12.} How an intellectual attempt
can have the effect of ``theoretical purification'' and at the same time
that of a destruction of science, is difficult to understand. Whatever
the effect of this attempt might be, it presupposes the distinction
between objective, i.e. verifiable propositions concerning facts and
judgments concerning values which, by their very nature, are subjective
and hence not scientific. This distinction, Voegelin asserts, is an
error due to the fact that the positivistic thinkers ``did not master
the classic Christian science of man. For neither classic nor Christian
ethics and politics contain 'value-judgments' but elaborate, empirically
and critically, the problems of order which derive from philosophical
anthropology as part of a general ontology. Only when ontology as a
science was lost, and when consequently ethics and politics could no
longer be understood as sciences of the order in which human nature
reaches its maximal actualization, was it possible for this realm of
knowledge to become suspect as a field of subjective, uncritical
opinion.''\footnote{L.c., pp. 11f.}

The statement that positivistic social science excludes ``all'' value
judgments is a gross misinterpretation of the theory concerned. The term
``value judgment'' has -- as positivistic writers have pointed out -- many
meanings. A judgment which most frequently is characterized as a value
judgment is the proposition that something is an appropriate means for
the realization of a presupposed end. Since the relationship of means to
end, as pointed out, coincides with the relation of cause and effect,
the proposition in question is objectively verifiable; and if it is -- as
usually -- considered a value judgment, no positivist excludes this value
judgment from a scientific theory, because of its ``subjectivity.'' It
is a specifically positivistic view that value judgments concerning
appropriate means are a special type of propositions concerning facts,
and that only judgments to the effect that something ought to be
considered as an ultimate end are the value judgments which in the last
analysis are based on emotional factors and for this reason subjective
and hence relative only. Other judgments which usually are
characterized as ``value judgments'' are propositions by which positive
legal and moral orders prescribing a definite human behavior are
described in terms of \label{M14}statements about what ought to be done, and
propositions by which conformity or non-conformity of actual human
behavior with positive law or morality is ascertained. Propositions of
this type are the essence of scientific jurisprudence and ethics, which
have nothing to do with metaphysics or theology. The methodological
postulate that scientific jurisprudence and ethics (including political
theory) are to be value-free means only that the description, analysis
and explanation of a positive system of law and morality -- and only a
positive system of norms, that is, a normative order established by acts
of human beings and, by and large, applied and obeyed, can be the object
of scientific knowledge -- should not be influenced by ``political
preferences or personal idiosyncrasies'' of the writer: a principle
which -- as we have seen -- Voegelin himself maintains in his criticism of
positivism. Hence it is quite astonishing that the same author, who
condemns positivism as destructive because its interpretation of Plato
is biased by subjective value judgments on the part of representatives
of this school of thought, rejects the distinction between objective
propositions concerning facts and subjective value judgments. And even
more astonishing is the argument that ``neither classic nor
Christian ethics and politics'' -- that is, the metaphysical speculation
of Thomas Aquinas -- ``contain 'value-judgments' ''. This is indeed a
metaphysical-theological way to argue. A statement is true if in
conformity, and false if not in conformity with that what is written in
Plato's dialogues or in the Bible. Even if one places the term value
judgment between quotation marks, one cannot deny that the statement
that a certain human behavior is just or unjust, that is to say, that it
ought or ought not to take place, is a value judgment and as such
different from the \label{M15}statement that a certain human behavior actually
takes place or has taken place; and one cannot deny that the works of
Plato as well as those of Christian ethics and politics are full of
statements about what is just, that the problem of the value we call
justice is their very center. It is precisely the insufficiency of the
answer positivism can give to the question of justice by which Voegelin
justifies his condemnation of this school of thought.\footnote{L.c.,
  p. 6.} It is true that the value judgments of classic and Christian
ethics and politics referring to ultimate ends claim to be objective;
but it is just this claim which, examined by a scientific theory of
values, proves to be unfounded.

Voegelin seems to assume that Plato's mystic philosophy of the good
and the speculations of Christian ethics and politics about divine
justice have nothing to do with value judgments because they have the
character of ``ontology,'' that is, cognition of the being. But
metaphysical ontology is the typical way of presenting subjective
values as objective truths.  This is confirmed by Voegelin's
definition of ontology as a science ``of the order in which human
nature reaches its maximal actualization.''\footnote{L.c., p. 12.} If
this phrase has any meaning at all, it can only refer to a normative
order; the statement that human nature reaches ``maximal
actualization'' can only mean that if human behavior is in conformity
with this order it realizes the highest possible value; and that means
the absolute value. That it is the absolute value Voegelin has in mind
- although he does not admit it expressly -- when he appeals to an
ontology, based on metaphysics and theology, results from the fact
that he says of the positivistic social science which has ``lost''
this ontology: ``Neither the most scrupulous care in keeping the
concrete work 'value-free' nor the most conscientious observation of
critical method in establishing facts and causal relations could
prevent the sinking of historical and political sciences into a morass
of relativism.''\footnote{L.c., p. 13.} It is against relativism -- that
is the view according to which only relative values are accessible to
human reason and that, consequently, no \label{M16}scientific decision is
possible between economic security as the ultimate end or highest
value of Marxism, and individual freedom as the ultimate end or
highest value of liberalism -- it is just against this philosophy that
the ``ontology'' based on metaphysics and theology is directed. For
metaphysical and religious speculations aim by their very nature at
the absolute in general and the absolute value in particular.

It is a characteristic tendency of metaphysical-religious speculation to
efface the difference between reality and value, between the ``is'' and
the ``ought.'' For reality is according to the fundamental assumption of
such speculation the realization of the absolute value: the will of a
transcendent authority; which is the assumption that the world is
created by God. Only under this presupposition reality and value
coincide, is there -- in the last analysis -- no difference between ``is''
and ``ought.'' Then, and only then, ontology, the cognition of being,
can pretend to be at the same time the cognition of the absolute value.
\label{M17}If ontology, as Voegelin asserts, is a science of the order in which
human nature reaches its maximal actualization, the question arises
which is the order under which such actualization takes place? Marxists
assert that it is the social order of communism; their opponents assert
that it is the social order of capitalism. Who is right and who is
wrong? Is Voegelin really so naive as to believe that a scientific
answer to this question can be derived from the Platonic Agathon or the
Thomistic \emph{ratio aeterna}? Although everybody agrees with the
ideal of a maximal actualization of human nature -- since everybody can
interpret this vague formula according to his fancy -- and although the
Platonic as well as the Thomistic formula has long been known, there is
still a passionate fight about the right way to its realization. If
there were an answer to this question as demonstrable and convincing as
a scientific answer has to be, the great conflict of our time would
disappear, just as there is no conflict with respect to the question how
to build a steam engine or to treat syphilis. If the new science of
politics is in the possession of the answer, what is it waiting for? The
effect which it would have on the social life of our time would attest
its scientific truth. Until the new science of politics discloses its
secret, its appeal to the Platonic Agathon or the Thomistic \emph{ratio
  aeterna} must be considered as idle talk. It will probably object that
the question does not allow an answer as clear and unambiguous as
natural science can give. Then any of the highly contradictory answers
that may be, and actually have been, deduced from the empty formulae
of the Platonic Agathon or the Thomistic \emph{ratio aeterna} must be
recognized as equally valid. Which means that the metaphysical-religious
speculation leads exactly to the same situation which Voegelin so
critically characterizes as the ``morass of relativism.'' He thinks that
he can strike a deadly blow at positivism by ascertaining that the
exclusion of subjective value judgments, and consequently the rejection
of the ``whole body \label{M18}of classic Christian metaphysics ...~could
result in nothing less than a confession that a science of human and
social order did not exist.''\footnote{L.c., p. 12.} If Voegelin
understands by a ``science of human and social order'' the science of
social order ``in which human nature reaches its maximal
actualization,'' that is to say, the establishment, by science, of a
social order guaranteeing the realization of the absolute value, he need
not extort a ``confession'' that such science does not exist. That it
does not exist is no secret, and the positivistic science never
pretended to be such a science. If it ever existed, it has been
destroyed, as Voegelin again and again asserts, otherwise, he could not
ask for its ``restoration.'' But as long as this miraculous science of a
social order in which human nature reaches its maximal actualization is
not yet definitely established -- and it seems that the new science of
politics has the ambition to be or to become such a science -- Voegelin
must not expect that somebody who has that ``consciousness of critical
standards'' that he considers as desirable can believe in its existence.

The methodological postulate of value-free description and explanation
of social phenomena is one of the main elements of Max Weber's
positivistic sociology. In order to demonstrate the futility of this
postulate, Voegelin tries to show that, if Weber's work is not
completely without importance, if it constitutes, in spite of its
positivism, at least to a certain extent an ``ascent toward
essence''\footnote{L.c., p. 15.}, it is so because Weber actually --
although unintentionally and unconsciously -- attributes to science the
function of determining values. That means that Weber's work is
self-contradictory. The way in which Voegelin achieves this result is
significant. He states quite correctly: ``A value-free science meant to
Weber the exploration of causes and effects, the construction of ideal
types that would permit distinguishing regularities of institution as
well as deviations from them, and especially the construction of typical
causal relations. \label{M19}Such a science would not be in a position to tell
anybody whether he should be an economic liberal or a socialist, a
democratic constitutionalist or a Marxist revolutionary, but it could
tell him what the consequences would be if he tried to translate the
values of his preference into political practice.''\footnote{L.c.,
  p. 14.} But then he continues: ``On the one side, there were the
'values' of political order beyond critical evaluation; on the other
side, there was a science of the structure of social reality that might
be used as technical knowledge by a politician. ...~In the intellectual
climate of the methodological debate the 'values' had to be accepted as
unquestionable, and the search could not advance to the contemplation of
order.'' For, as Voegelin asserts, ``a 'value-free' political science is
not a science of order.''\footnote{L.c., p. 16.} By ``a science of
order'' he understands a science establishing a normative order
constituting absolute values.  Weber's sociology certainly does not
claim to be the science of an order in which human nature reaches its
maximal actualization, that is to say, the science of a normative order
constituting an absolute value. But a science which -- as the sociology of
Weber -- has for its object the causal relations in social reality is
also a science of order because it is the order of nature, a causal
order according to which such science interprets reality. Voegelin's
identification of ``order'' with a normative order of absolute values is
unjustified and misleading because it produces the idea -- and this is
probably the intention of this identification -- that outside of this
normative order there is no order, but chaos; which, of course, is not
true.

It is incorrect and a misleading interpretation of the postulate of a
value-free political science to maintain, as Voegelin does, that from
the point of view of such science the values of a political order are
``beyond critical evaluation,'' that these values have ``to be accepted
as unquestionable.'' A value-free political science only maintains that
the values which a political system tries to realize cannot be confirmed
by science as absolute values. That does not mean that a critical
evaluation of the political system is impossible; it means only that the
recognition of an absolute value is not \label{M20}possible on the basis of a
political ``science.'' A value-free political science does not exclude
the possibility of scientific judgments concerning the appropriateness
of social measures as means for presupposed ends, that is to say,
judgments about relative values in the sense explained above; it is far
from asserting that political values have to be accepted as
unquestionable.  Just the contrary is true. It is the ``science of
order'' postulated by Voegelin which insists upon the unquestionable
acceptance of values, because this science pretends to prove their
absolute validity.

Referring to the teaching of political science at universities, Voegelin
states that ``the science of Weber'' only ``supposedly left the
political values of the students untouched, since the values were beyond
science.'' In truth, these values are touched, for the political
science, in spite of its tendency not to ``extend the principles of
order'' may ``have the indirect effect of inviting the students to
revise their values when they realized what unsuspected, and perhaps
undesired, consequences their political ideas would have in practice.''
And from this Voegelin concludes: ``An appeal to judgment would be
possible, and what could be a judgment that resulted in reasoned
preference of value over value be but a value judgment? Were reasoned
value-judgments possible after all?''\footnote{L.c., p. 16.} The answer
has to be in the affirmative. ``Reasoned value judgments'', that is,
value judgments determined by reason and hence scientific value
judgments, are possible, even according to Weber's supposedly value-free
science.

The situation to which Voegelin refers is that which is correctly
described as conflict of values. It is the consequence of the fact that
usually not one but two or more values are presupposed as ultimate ends,
as for instance, individual freedom and social security, and that the
realization of the one proves to be incompatible with the realization of
the other. Then a choice between these values is necessary, a decision
must be made about which one is preferred to the other.

\label{M21}Science can demonstrate that the means by which a certain end 
-- that is to say a value -- is to be realized are inappropriate for
the realization of the other value. This judgment is a reasoned
judgment in the sense of a judgment determined by reason, a scientific
judgment. For it is a judgment about the relation of cause and effect,
and that means a judgment concerning facts. But the judgment according
to which one value is to be preferred to another value, is a pure
value judgment and it is not at all possible on the basis of
scientific reason, as is the judgment concerning the appropriate
means. Voegelin concludes from the fact that the judgment according to
which the realization of one value is incompatible with the
realization of another value is a reasoned judgment, that the judgment
according to which one value is to be preferred to another value, too,
is a reasoned, that is, scientific judgment. This is a false
conclusion. On the basis of this false conclusion he arrives at the
following thesis, which has considerable significance for his fight
against positivism in general and a value-free political science in
particular because it shows the goal at which this new political
science is driving: ``The teaching of a value-free science of politics
in a university would be a senseless enterprise unless it were
calculated to influence the values of the students by putting at their
disposition an objective knowledge of political
reality.''\footnote{L.c., p. 16.}

By putting at the disposition of the students an objective knowledge
of political reality it is impossible to influence the values of the
students. And putting at the disposition of the students an objective
knowledge of reality is a highly meaningful enterprise, even if, nay,
just because, the choice of the value is left to the students, that
is, just because science does not restrict the freedom of this
choice. If the student realizes that in his choice of political value,
in his decision to support a socialist or a capitalist, a democratic
or an autocratic system, he cannot rely on the authority of science
that science has not and cannot restrict the freedom of his choice, he
will become aware of the fact that he has to make this choice under
his own \label{M22}responsibility; which is a highly moral consequence of the
value-free science. It is the fear of this responsibility that leads
to the tendency to shift the responsibility for the political decision
from the subject to an objective authority, to science. And it is a
misuse of this weakness of the individual if in totalitarian states
the universities have to assume the task of political indoctrination
of the students, or, as Voegelin formulates it: ``to influence the
values of the students.'' Since this cannot be achieved by an
``objective knowledge of political reality,'' the task of influencing
the political values of the students can be fulfilled only by an
ideologically distorted knowledge of political reality, that is to
say, by a doctrine, which pretends that just that value which the
political power, directly or indirectly controlling the university,
prefers, is immanent in reality, and hence the only true, the absolute
value. That means that the universities become the instruments of
politics, and, where they are under the exclusive control of the
government, a kind of intellectual police. That is exactly what the
universities of Nazi-Germany and Fascist Italy have been and the
universities of communist Russia still are -- in complete conformity
with the principle ``to influence the values of the students.''

Voegelin does not content himself to ascertain the regrettable fact that
Max Weber did not ``take the decisive step toward a science of
order''\footnote{L.c., p. 19.}; he has the ambition to explain why Weber
was unable of such an achievement; which amounts to the rather naive
question why Weber remained a positivist and did not turn to
metaphysics. Among the many astonishing statements, he makes in his
fight against positivism, his answer to this question deserves
particular notice. Although he must admit that the amount of material
Weber mustered in his sociology of religion, ``is indeed
awe-inspiring,'' Voegelin considers himself competent to ascertain a gap
in Weber's knowledge, ``a scientific omission,''\footnote{L.c., p. 20.}
as he puts it. If Weber had filled this gap, he would have taken ``the
decisive step toward a science of order,''\footnote{L.c., p. 19.} that
is to say, he had become a \label{M23}metaphysician. This, it is true, Voegelin
does not say directly, but it is implied in his following statements. He
says immediately after blaming Weber for his omission: ``Weber's
readiness to introduce verities about order as historical facts stopped
short of Greek and medieval metaphysics. In order to degrade the
politics of Plato, Aristotle, or St.~Thomas to the rank of 'values'
among others, a conscientious scholar would first have to show that
their claim to be science was unfounded.''\footnote{L.c., p. 20.} With
the same right one could say that a conscientious scholar who, like
Voegelin, summons political science to return to the speculations of
metaphysicians, has first to show that the claim of metaphysics to be
science is founded, which, of course, he is far from doing. He only
asserts that the ``attempt [to show that the metaphysical speculations
of Plato, Aristotle and Thomas are not science,] is self-defeating. By
the time the would-be critic has penetrated the meaning of metaphysical
speculation with sufficient thoroughness to make his criticism weighty,
he will have become a metaphysician himself.'' That implies that Weber
would have become a metaphysician if he had penetrated the meaning of
metaphysics. Voegelin continues: ``The attack on metaphysics can be
undertaken with a good conscience only from the safe distance of
imperfect knowledge.'' That means: the only reason for not being a
metaphysician is imperfect knowledge of metaphysics. This statement, if
made without proving that all positivists had only imperfect knowledge
of metaphysics -- and such proof is of course impossible -- has no more
weight than the statement that the only reason for not being a
positivist is imperfect knowledge of positivism.

Now, what is the gap which Voegelin discovered in Max Weber's
awe-inspiring knowledge of the various religions, the omission which
prevented this positivist from becoming a metaphysician? Lo and
behold, the knowledge ``of pre-Reformatic \label{M24}
Christianity.''\footnote{L.c., p. 20.} Voegelin has no right to
maintain that Weber was ignorant of medieval Christianity the wisdom
of which is concentrated in the work of Thomas Aquinas: For his only
sole argument could be the fact that Weber did not take into
particular consideration pre-Reformation Christianity, which might
have many other reasons. Besides, the metaphysics of medieval
Christianity is not so different from classic and other Christian
metaphysics -- Thomas Aquinas' metaphysics is essentially influenced
by Aristotle's speculations -- that knowledge of classic metaphysics
or metaphysics of post-medieval Christianity could not have the same
effect of converting a positivist into a metaphysician. It seems that
Voegelin anticipated this objection; for he reproaches Weber not only
with ignorance of pre-Reformation Christianity but also with the
above-mentioned lack of knowledge of Greek metaphysics. Voegelin
seriously maintains that Weber has not ``seriously occupied himself
with Greek philosophy.''\footnote{L.c., p. 20.} To maintain that Weber
was a positivist because he had no sufficient knowledge of
pre-Reformation Christianity and of Plato's and Aristotle's
metaphysics is an inadmissible statement, not only because it cannot
be proved, but because it implies -- as a dogma -- the view that
classic and Christian metaphysics represent an absolute truth.

Voegelin has not only discovered the omission of pre-Reformation
Christianity in Weber's sociology of religion, he knows also the reason
of this omission. He says: ``The reason of the omission seems to be
obvious. One can hardly engage in a serious study of medieval
Christianity without discovering among its 'values' the belief in a
rational science of human and social order and especially of natural
law. Moreover, this science was not simply a belief, but it was actually
elaborated as a work of reason. Here Weber would have run into the fact
of a science of order ...~''\footnote{L.c., p. 20.} How can the reason
of the omission be the specific content of the metaphysical speculations
concerned if Weber had no knowledge of this content? And if he had the
knowledge, he is then guilty of having intentionally omitted dealing
with these metaphysical speculations, in order to maintain his
positivistic view. If Voegelin does not \label{M25}accept the first-mentioned
interpretation of his attack against Weber because it reveals his
argument as illogical, he exposes himself to the suspicion of imputing
to a great scholar scientifically improper motives. Perhaps Voegelin did
not mean what he actually said, namely, that Weber omitted to take into
consideration pre-Reformation Christianity because this material would
have shown to him the existence of a science of order and consequently
would have forced him to give up his positivistic negation of such a
science. Voegelin's idea probably was that if Weber had studied
pre-Reformation Christianity, he would have changed his view concerning
a science of order. But this assumption would be as inadmissible as the % im Original: ``as inadmissible at'' !
above-mentioned conclusion Voegelin draws from Weber's alleged ignorance
of classic metaphysics.

Voegelin asserts that medieval Christianity has elaborated ``a rational
science of human and social order and especially of natural law'', the
``science of order'' to which he wants to drive back the political
science of our time. The core of this science is indeed the natural law.
This is nothing particular to the metaphysics of pre-Reformation
Christianity. On the contrary. The natural-law doctrine flourished in
post-Reformation Christianity, and was very well known to Max Weber. But
the question is whether this doctrine, and the entire metaphysical
speculation of which it was an essential part, is really a ``science'',
as Voegelin asserts. According to the standard he adopts with respect to
the anti-metaphysical attitude of a ``conscientious scholar'' he is
obliged to show that the claim of this metaphysical speculation to be a
science is founded; and this all the more as conscientious scholars have
submitted the natural-law doctrine again and again to the tribunal of
science and the claim has always been dismissed, especially because of
the highly contradictory results of this doctrine.\footnote{Cf. Hans
  Kelsen, ``The Natural-Law Doctrine before the Tribunal of Science.'',
  \emph{The Western Political Quarterly}, II., (1949), pp. 481ff.}  But
Voegelin is far from complying with this standard.

He quite correctly states that for Max Weber the evolution of mankind
toward the rationality of positive science ``was a process of
disenchantment (\emph{Entzauberung}) and de-divinization
(\emph{Entg�ttlichung}).''\footnote{Voegelin, l.c., p. 22.} But Voegelin
believes that he can hear in Weber's theory ``overtones'' of a ``regret
that divine enchantment had seeped out of the world''; that Weber's
rationalism was a mere ``resignation.'' I have known Max Weber
personally and studied his works very carefully, and on the basis of
this knowledge I may say that the ``overtones'' and the ``resignation''
exist only in the metaphysical imagination of Voegelin. His imagination
was probably stimulated by the laudable wish to mitigate somehow his
criticism of a great master and to be able to say finally of Weber: ``He
saw the promised land but he \label{M25a}was not permitted to enter
it''\footnote{L.c., p. 22.}, which, of course, is the land to which
political science will be lead by a new Moses, about whose identity no
reader of the \emph{New Science of Politics} can have the slightest
doubt.

\label{M26}
\section{A new Theory of Representation}

\subsection{1.}

In the Introduction to his book \emph{The New Science of Politics} Prof.
Voegelin expresses his ``intention of introducing the reader to a
development of political science which as yet is practically unknown to
the general public ...~''\footnote{L.c., p. 3.} It seems to be rather
strange that the development of a science which has not remained the
secret of an esoteric sect but presented to the public in printed books
and articles, could be unknown to those to whom these publications are
addressed. And we can hardly believe that Voegelin understands by
general public readers who have no scientific background, that he simply
intends to popularize the results of a new science of politics, which
already exists in form of monographs.  His book is just the contrary of
a popular presentation of political theory.  Even for an expert in this
field it is difficult to understand. For one of its peculiarities is
that the author describes relatively simple and by no means unknown
facts in a complicated language overloaded with superfluous foreign
words, especially Greek terms, which are out of place if their use is
not necessary to reproduce faithfully the content of classic writings.
Since there are English words which perfectly express Voegelins ideas,
the embellishment of the new science by words as \emph{agathon},
\emph{Kosmion}, \emph{xynon}, \emph{eidos}, and the like might well be
mistaken as an attempt to impress the reader with the great erudition
of the author, a device that a scholar of so high a scientific standard
as Voegelin does not need.

From the very first chapter, entitled ``Representation and Existence'',
the new science demonstrates its skill in complicating, to the degree of
almost complete obscurantism, a problem familiar to every political
scientist: that \label{27}of political representation. In order to explain
what this term means, Voegelin thinks it necessary first to deal with a
peculiarity of the object of social science, the well-known fact that
men living in society interpret their mutual behavior and the
relationships constituted by it, and that social science, in describing,
and that implies interpreting, the social phenomena, has to take this
primary interpretation, the ``self-interpretation of society,'' as
Voegelin calls it,\footnote{L.c., p. 27.} into consideration.  Natural
science, the interpretation of natural phenomena, does not encounter
such primary interpretation. A stone does not say to the mineralogist: I
am a plant.  But the head of a state may say: I am authorized by God to
exercise power.  Political science in describing the function of this
head of state may confirm or reject this primary interpretation.
Positivistic political science, e.g., rejects it as an ideological
misinterpretation of political reality, whereas Christian metaphysics --
to the principles of which the new political science wishes to return --
confirms it according to the teaching of the first theologian of
Christianity, St.~ Paul: ``There is no authority except from God, and
those that exist have been instituted by God.''\footnote{Rom. 13, 1.} A
critical analysis of primary interpretation is an important task of
political science.

What Voegelin calls self-interpretation of society is nothing new; but
the new science of politics seems to have made a discovery by telling us
that ``human society is ...~a cosmion, illuminated with meaning from
within by the human beings who continuously create and bear it as the
mode and condition of their self-realization.''\footnote{Voegelin, l.c.,
  p. 27.}  This description is certainly very poetic, but scientifically
not correct. For the self-interpretation is not always illuminating but,
on the contrary, obscuring. Voegelin admits that a ``critical
clarification'' of the self-interpretation which social science
encounters is a function of this science.\footnote{L.c., p. 28.} And
the primary interpretation by man of their social behavior has nothing
to do with man's ``self-realization,'' provided that this term has any
meaning at all.

\label{M28}Voegelin quite correctly states that when political science begins
to interpret social phenomena it starts from the self-interpretation
which it finds in its object. He characterizes this method as the
Aristotelian procedure.\footnote{L.c., p. 28, 31.} Since, as Voegelin
says, political science ``inevitably'' starts from the
self-interpretation of society and proceeds by critical clarification
to theoretical concepts\footnote{L.c., p. 28.} and since there was a
political science prior to Aristotle's {\em Ethics} and {\em
Politics}, Aristotle could not have been the first to apply this
method. Voegelin refers to Politics 1280a7ff.  This is a passage
within Aristotle's analysis of the three forms of government:
monarchy, aristocracy, polity, and their forms of degeneration:
tyranny, oligarchy, democracy.  In discussing the limits of oligarchy
and democracy ``and what is just in each of these states,'' Aristotle
says: ``All men have some natural inclination to justice; but they
proceed therein only to a certain degree; nor can they universally
point out what is absolutely just, as for instance what is equal
appears just, and is so; but not to all, only among those who are
equals; and what is unequal appears just, and is so; but not to all,
only among those who are unequals; which circumstances some people
neglect and therefore judge ill; the reason for which is they judge
for themselves, and everyone almost is the worst judge in his own
cause.''  This is the distinction between a merely relative and the
absolute justice.  Aristotle explains the fact that men do not agree
about equality as justice: ``because they judge ill in their own
cause, and also because each party thinks that if they admit what is
right in some particulars, they have done justice on the whole ...~but
what is absolutely just they omit.'' Then, Aristotle, in order to
answer the question as to what is not merely relative but absolute
justice, examines the purpose for which civil society was founded. His
answer is that civil society is ``not founded in the purpose of men's
merely living together, but for their living as men ought.'' This is
the usual empty formula, from which Aristotle concludes that ``those
who contribute most to this end deserve to have greater power in the
city than those who are their \label{M29}equals in family and freedom, but
their inferiors in civil virtue, or those who excel them in wealth but
are below them in worth.''\footnote{1281a.}  This statement amounts to
the truism that the rulers shall be virtuous, or that only the
virtuous shall be the ruler. On the basis of this principle Aristotle
later justifies the hereditary monarchy.\footnote{1284b, 1288a.  Cf.
Hans Kelsen, ``The Philosophy of Aristotle and the Hellenic-Macedonian
Policy.''  Ethics, 48 (1937), pp. 1ff.} The only statement of
Aristotle on which Voegelin could base his assumption of a specific
``Aristotelian procedure'', consisting in a careful distinction
between ``theoretical concepts and the symbols that are part of
reality''\footnote{Voegelin, l.c., p. 31.} is the remark that ``some
people'' neglect that equality is just only in a relative, not in an
absolute sense, that they ``judge ill'' because ``they judge for
themselves and everyone almost is the worst judge in his own cause.''
It is very doubtful whether Aristotle's criticism of the doctrine that
equality is justice -- which is the basis of democracy -- was meant as a
``critical clarification of socially pre-existing symbols''
\footnote{L.c., p. 28.}; and as an attack on the political
theory prevailing in democratic Athens.  But even if it is admitted that
it was a clarification of a self-interpretation of society, it cannot be
denied that Aristotle applied this method only in this connection.  In
defining his concepts of the three forms of government and their
degenerations, he says nothing that could be interpreted as the
``Aristotelian method'' in the sense of Voegelin. Aristotle states: ``We
usually call a state which is governed by one person for the common
good, a kingdom, one that is governed by more than one, but by a few
only, an aristocracy; either because the government is in the hands of
the most worthy citizens, or because it is the best form for the city
and its inhabitants. When the citizens at large govern for the public
good, it is called a polity; which is also a common name for all other
governments, and these distinctions are consonant to reason; for it will
not be difficult to find a person, or a very few, of distinguished
abilities, but almost impossible to meet \label{M30}with the majority of a
people eminent for every virtue.''  Further, he says: ``the
corruptions attending each of these governments are these; a kingdom may
degenerate into a tyranny, an aristocracy into an oligarchy, and a state
into a democracy.  Now a tyranny is a monarchy where the good of one man
only is the object of government, an oligarchy considers only the rich,
and a democracy only the poor; but neither of them have a common good in
view.''\footnote{1279a, b.} Here Aristotle does evidently not start
from the judgment people render in their own cause -- if that means
self-interpretation of society. On the contrary.  He starts from a
concept of democracy which is evidently not the symbol used in political
reality, the self-interpretation of a democratic society. He begins his
definition with the formula ``We usually call ...~'', pretending that
the following definitions are those used in political debates of daily
life.  But the governments which presented themselves as democracies did
not at all use this term to designate a government that has no common
good in view, a degenerated corrupted government. Aristotle imputes to
the term democracy a meaning which it certainly had not in political
reality, and he uses this device not for the scientific purpose of an
objective analysis, but for the political purpose the tendency of which
was directed against democracy; for aristocracy and especially for
monarchy. If there is a specifically ``Aristotelian procedure'' at all,
it is the one which manifests itself in his definition of democracy.

Voegelin formulates as a methodological postulate of the new science
of politics: ``theoretical concepts and the symbols that are part of
reality must be carefully distinguished.''\footnote{L.c., p. 31.}
Confusion which consists in taking ``symbols used in political
reality'' for ``theoretical concepts''\footnote{L.c., p. 29.} must be
avoided. It is, however, hardly possible to separate completely a
scientific interpretation of social phenomena from the primary or
self-interpretation which is implied in the object of political
science.  For the primary interpretation is more or less influenced by
the existing political science, and political science -- as a social
phenomenon, especially when it intends to serve as a political
instrument -- to a certain degree becomes the object of political
science. \label{M31}Voegelins presentation is itself an example of the
confusion of the two kinds of interpretation, and thus shows how
easily the distinction he postulates may be omitted. He refers to the
``Marxian idea of the realm of freedom, to be established by a
communist revolution''.\footnote{L.c., p. 29, ibid.} He has probably in
mind the doctrine set forth by Marx, and especially by Engels in his
\emph{Anti-Duehring}\footnote{Friedrich Engels, \emph{Herrn Eugen
    Duehrings Umw�lzung der Wissenschaft}, 10th ed., 1919, p. 306.}
that the progress to communism is ``der Sprung der Menschheit aus
dem Reich der Notwendigkeit in das Reich der Freiheit'' (the leap
of mankind from the realm of necessity into the realm of
freedom). This is a concept of the theory developed by Marx and
Engels, a theoretical concept. This theory and its description of
the communist society as a realm of freedom may be rejected as
erroneous theory by another social theory; but even a wrong theory
is a theory and its concepts are theoretical concepts, because
intended to serve as a description and explanation of reality.
Voegelin deals with the Marxian concept of the realm of freedom as
with a symbol used in political reality because it is part of the
Marxian movement. But it is so only because the Marxian theory is
used in the reality of the Marxian movement and it is as such the
object of a critical theory; just as the theory of Thomas Aquinas
is used by the Church in the political reality of a social
movement which -- like the socialist movement -- may be the object
of a critical theory. This is the so-called critique of
ideology. The fact that a theoretical concept is used in political
reality and thus is a symbol of self-interpretation of society
does not and cannot deprive it of its character as theoretical
concept. Voegelin takes a concept of the Marxian theory
\label{M32}as a symbol of self-interpretation of society; and he
can do so because the idea of a ``realm of freedom'' is both. On
the basis of his critique of the Marxian theory as an ideology of
the socialist movement, Voegelin states that ``the symbol 'realm
of freedom' is useless in critical science''.\footnote{Voegelin,
l.c., p. 29.}  But that does not mean that it is not a theoretical
concept; unless Voegelin assumes that only a correct doctrine is a
``theory'', and that means that only a doctrine which expresses
the absolute truth is a ``theory'' in the solely admissible use of
the term. Under this assumption no theory at all has come into
existence until now. The history of science is the history of a
permanent change and transformation of theories, of a process in
which one theory is replaced as erroneous by another theory, which
inevitably will have the same fate. Nevertheless, Voegelin seems
to proceed from the assumption that only a correct theory is a
theory. He exhorts political science so passionately to return to
the principles of the classic and Christian metaphysics because he
believes that there we may find the absolute truth. He should not
ignore that the doctrine of the ideas, which is the core of
Plato's metaphysics and which culminates in the idea of the
absolute good (the \emph{agathon}), was refuted in the metaphysics
of Aristotle as a superfluous reduplication of the object of
cognition, and, hence, may be considered -- from this point of
view -- just as the Marxian realm of freedom from the point of
view of Voegelin's metaphysics -- as ``useless in critical
science.''

The danger which results from the parallelism of the interpretation of
social reality used in this reality, and the interpretation of social
reality by political science is not that the self-interpretation of
society is taken to be a ``theory''. For it may indeed be a theory
although a theory to be rejected by political science. The decisive
difference between the two interpretations is not that the one is,
whereas the other is not, a theory, but the fact that the interpretation
of social reality by political science ought to be, and can be,
objective, \label{M33}whereas the self-interpretation of society, although it,
too, pretends to be objective, is necessarily more or less subjective,
that is to say, determined by the social interests of the interpreting
subject. This difference, to be sure, is only a relative one. There is
no other science in which the fulfillment of the requirement of
objectivity is so difficult as in political science. The greater
therefore is the danger that political science will uncritically take
over the political theory used in political reality by those who
exercise political power, that is to say by governments or by groups
opposed to the government, as a political instrument, and thus becomes
itself a political instrument. This is the real danger of political
science: that it gives up the attempt to be objective. And the danger is
unavoidable if political science refuses to be ``value-free.'' For, if a
political science identifies itself with a definite political value, and
that means with a definite political system, it inevitably is degraded
to a handmaid of politics. Then there can be not one political science
as there is only one science of biology, but there must be always at
least two sciences of politics, advocating opposite political values,
the results of the one being as ``true'' as those of the other.

Since the new science of politics expressly refuses to be a value-free
interpretation of political reality, it is quite understandable that it
is not at all this danger which, according to Voegelin, is of importance.
Instead, he emphasizes the distinction between theoretical concepts and
symbols used in political reality. But just as he ignores this
distinction by taking a theoretical concept -- the Marxian realm of
freedom -- as a symbol used in political reality, he also takes symbols
used in political reality as theoretical concepts.

\label{M34}In his analysis of the development of the concept of representation
he refers to the Magna Carta, the writs of summons of the 13th and 14th
century, the address of Henry VIII to parliament in Ferrer's case, two
treatises of Sir John Fortescue, the \emph{History of the Lombards} of
Paulus Diaconus, and professor Haurious's \emph{Precis de Droit
  constitutionnel}.\footnote{L.c., pp. 38ff.} The concepts used in the Magna
Charta, the writs of summons, the address of Henry VIII are symbols of
self-interpretation of society; the concepts presented in the works of
Fortescue, Paulus Diaconus, and Hauriou are symbols of political
science.  There is not the slightest difference in the treatment of
these sources by Voegelin's new political science. He takes the concept
``the king in Parliament''\footnote{L.c., p. 39} used in the address
of Henry VIII just as Fortescue's doctrine that ``all that the king does
ought to be referred to his kingdom''\footnote{L.c., p. 45}  as a
contribution to the theory of representation; and he considers the
definition of Parliament as ``the \emph{commune consilium regni
  nostri}'' in the Magna Carta just as important to political science as
Prof. Haurious's definition of the concept of
representation.\footnote{L.c., pp. 48f.}

Voegelin first discusses the ``elemental aspects'' of
representation.\footnote{L.c., pp. 31ff.} The aspect is elemental but it
is ``theoretically'' elemental. Voegelin states expressly that he speaks
``of the theoretically elemental aspect''\footnote{L.c., p. 31.} of his
topic, and refers to ``the theoretization of representative
institutions'' on the elemental level.\footnote{L.c., p. 33.} We must
remember that Voegelin just before has insisted on the sharp distinction
between theoretical concepts and symbols used in political reality. But
now he characterizes the concept of representation used as ``a symbol in
political reality''\footnote{L.c., p. 32.} as the theoretically
elemental aspect of representation, and by use ``in political reality''
he understands here the use of the term representation ``in political
debate, in the press, and in the publicist literature.''\footnote{L.c.,
  pp. 31,32.} Since publicist literature implies books on political
science, the distinction between symbols used in political reality, i.e.
symbols of self-interpretation of society and theoretical concepts is
abandoned.  And the way in which Voegelin describes the meaning
``representation'' has in political reality is precisely the \label{M35}way in
which traditional political science defines the concept of
representation or, more exactly, of democratic representative
institutions such as election of the legislative and the main executive
organs of the state on the basis of a universal and free
suffrage.\footnote{L.c., p. 32.} After declaring this aspect of the
problem as the theoretically elemental aspect, he asks: ``What can the
theorist do with an answer of this type in science?  Does it have any
cognitive value?'' Voegelin's answer is not directly in the negative.
But he attributes to the elemental concept of representation only little
cognitive value. This is the reason why he thinks it necessary to
replace it by a concept of greater cognitive value, the ``existential''
type of representation.

But why is the definition of representation as a system of government
according to which the organs of the state are elected on the basis of
universal and free suffrage ``elemental'' and why has it only little
cognitive value? Because it refers only to the ``external existence of
society''\footnote{L.c., p. 31.}, ``to simple data of the external
world''\footnote{L.c., p. 33.}, because it ``casts light only on an area
of institutions within an existential framework'', whereas ``the
framework itself remains in the shadow.''\footnote{L.c., pp. 34, 33.} It
is evidently the ``existential framework'' that Voegelin considers as
essential, not the ``external existence of society.'' But what does he
understand by the existential framework and where does representation
take place unless within the external existence of society? If -- as
Voegelin seems to assume -- the ``external existence of society'' is its
existence in ``the external world'', where else can society, as an
aggregate of inter-human relations, exist but in the external world?
External existence of society can be only the existence from the point
of view of the social scientist. Society as object of social science
does not exist -- as his ideas or feelings do -- within the scientist, but
in the external world, that is to say, in a world which from the
viewpoint of an objective science is supposed to exist outside the
scientist. Hence only the external existence of society comes into the
consideration of a social science. If there is, in contradistinction to
the external existence, also an internal existence of society, it could
be only society in its primary interpretation by men living in society
or -- as Voegelin puts it -- society in its self-interpretation, society
as \label{M35a}``a cosmion illuminated from within'', a cosmion which ``has its
inner realm of meaning''\footnote{L.c., p. 31.} But, as Voegelin admits,
``this realm exists tangibly in the external world in human beings who
have bodies and through their bodies participate in the organic and
inorganic externality of the world.''\footnote{Ibid.} There is no
sufficient reason to disparage a concept of representation as elemental
because it refers to ``the external existence of society'', for a
scientific concept of representation can refer only to the external
existence of society; and, as a matter of fact, the final definition of
representation presented by Voegelin -- the ``existential'', not the
``elemental'' aspect of the problem, as he calls it -- refers, as we
shall see, to exactly the same external existence of society as the
elemental definition.

What, then, is the real reason for presenting the definition of a
representative form of government as a government elected by the people
on the basis of universal suffrage as merely elemental, to be replaced
by a more appropriate one, the ``existential'' definition?

The allegedly elemental definition of representation is the definition
of a certain type of representation: representation of a community
organized by a democratic constitution. It is of importance to note that
the type of representation which the new science of politics depreciates
as elemental is that sort of representation which is the essence of
democracy. This is not the only possible type of representation. The
statement that an individual ``represents'' a community means that the
individual is acting as an organ of the community, and he is acting as
an organ of the community when he fulfills certain functions determined
by the social order constituting the community. If the order, as in the
case of the state, is a legal order, the functions determined by this
order are the creation and application of the order. It stands to reason
that the legal order must be a valid order; \label{M36}and it is valid if it
is, by and large, effective. Only if an individual acts as an organ of
the state can his action be imputed to the state, that is, to the
community constituted by the legal order; and that means that his action
can be interpreted as an action of the state, and the acting individual
as a representative of the state. The legal order determines not only
the function but also the individual who has to fulfill the function,
the organ. There are different methods of determining the organ. If the
organ is to be an assembly of the individuals subjected to the legal
order, or individuals elected by these individuals, a democracy, and
that means a democratic type of representation, is established. But the
community, especially the state, is represented not only if it is
organized as a democracy. An autocratic state, too, is represented by
organs, although they are not determined in a democratic way. Since any
organized community has organs, there is representation whenever there
exists an organized community, especially a state. It seems that the new
science accepts this view. It calls the ``process in which human beings
form themselves into a society for action'' ``the articulation of
society''.\footnote{L.c., p. 37.} Existential representation is ``the
result of political articulation.'' To represent a society means to
``act for the society'' and that means that the acts of the
representative are imputed to the society as a whole and not to the
acting individual. ``When his acts are effectively imputed in this
manner, a person is the representative of a society''.\footnote{Ibid.}

\label{M37}However, in a modern so-called representative democracy the organs
are considered by traditional political theory to represent the state by
representing the people of the state. The statement that the parliament
or the president in a democratic state represent the people means
nothing else but that the individuals subjected to the legal order
constituting the state have a decisive influence on the creation of the
legislative and executive organs in question, insofar as the constitution
authorizes them to elect these organs. It is true that representation of
the state and representation of the people of the community are two
different concepts, which traditional political theory does not always
distinguish clearly enough. But there can be no doubt about the meaning
of the statements concerned when traditional political theory refers to
representative institutions. As is so frequently the case, one and the
same term is used in a wider as well as in a narrower sense. Just as
``constitutional'' monarchy indicates a monarchy which has a specific ,
namely, a more or less democratic constitution, although an absolute
monarchy, too, has a constitution and thus is, in this sense, also a
constitutional monarchy. The term ``representative institutions''
signifies a democratic type of representation, although there exists
also a non-democratic type of representation. Just as there is no state
without a constitution, although the term constitution is used also in a
narrower sense, only for a special type of constitution, there is no
state without representation, although the term representation is used
also in a narrower sense, for a specific type of representation. To use
a term in a wider and in a narrower sense is not the best terminological
practice, but there is nothing ``elemental'' in it.

Much more important than the double meaning of representation out of
which hardly any misunderstanding can arise, is the fact that the term
representation can claim to mean not only representation of the state
but at the same time representation of the people of the state: only
and exclusively if it refers to representation by organs elected in a
democratic way. For if the \label{M38}statement that a state organ
represents the people is not to imply a gross fiction, it can mean
nothing else but that the individuals subjected to the legal order
constituting the state are entitled to exercise decisive influence on
the creation of the organs. It is already a fictitious interpretation to
say that the will of the organ is identical with the will of the
people, especially if the organ is not bound in the discharge of his
function by instructions given to him by his electorate, not to speak
of the highly problematical character of the so-called will of the
people where the people, that is, the electorate is divided in two or
more antagonistic parties. It is certainly a still more inadmissible
fiction to say of an autocratically established organ, that is to say,
an organ on the creation of which the subjects of the state have no
influence at all, that it is a representative, not only of the state,
but also of the people of the state. The new science of politics seems
not to be interested in avoiding this fiction.

In order to proceed from the elemental to the existential type of
representation Voegelin maintains that ``the elemental type of
representative institutions'' -- that is representation by organs
elected on the basis of universal and free suffrage -- ``does not
exhaust the problem of representation.'' This is certainly true. As
pointed out, there exists not only a democratic but also a
non-democratic type of representation.  But this does not justify
considering the former as elemental. The new science itself
characterizes the democratic type of representation as representation
in a ``constitutional'' sense\footnote{For instance, l.c., p. 49.},
though the non-democratic type is also constitutional, since any type
of representation must be established by the constitution.  Thus the
new science of politics uses the term constitution in the same way as
the old political science uses the term representation: in a narrower
sense, although the term has also a wider sense. It is for another
reason that the new science of politics considers the democratic type
of representation as elemental. It is elemental because it is --
according to the new science -- meaningless.  The way in which the
democratic process is described by Voegelin is quite
significant. \label{M39}''In the theoretization of representative
institutions on this [elemental] level'', says Voegelin, ``the
concepts which enter into the construction of the descriptive type
refer to simple data of the external world. They refer to geographical
districts, to human beings who are resident in them, to men and women,
to their age, to their voting which consists in placing check marks on
pieces of paper by the side of names printed on them, to operations of
counting and calculation that will result in the designation of other
human beings as representatives, to the behavior of representatives
that will result in formal acts recognizable as such through external
data, etc.''\footnote{L.c., p. 33.} The tendency of the description is
evident. The democratic process is presented as something that has no
bearing on the essence of the phenomenon in question. It has only a
formal character; it is of secondary importance.  ``The procedure of
representation is meaningful only when certain requirements concerning
its substance are fulfilled.''; ``the establishment of the procedure
does not automatically provide the desired substance.''\footnote{L.c.,
p. 35.} By ``the establishment of the procedure'' only the election
procedure can be meant. And if it is not the democratic procedure
which by itself provides the desired ``substance,'' then, perhaps, a
non-democratic procedure may provide it.  Thus everything depends on
the meaning of the ``substance.'' But what is the meaning of the
substance?  Since the elemental concept of representation is to be
replaced by the existential concept, probably something like
``existence.'' When Voegelin rejects the elemental concept of
representation as having only little ``cognitive
value,''\footnote{L.c., p. 32.} he says that the existence of the
democratic countries, the representative institutions of which are
described in this elementary way by referring to the fact that their
organs are elected by the people, ``must be taken for granted without
too many questions about what makes them exist or what existence
means.''\footnote{Ibid.} This statement can only express the idea that
the definition of democratic representation as representation by
elected organs is worthless because election of the organs by the
people does not -- in itself -- guarantee the existence, or a
satisfactory existence, of the state. Voegelin's critique of the
so-called elemental concept of representation confuses two different
questions: the question as to what is democratic representation, and
the question as to whether democratic representation assures the
existence or satisfactory existence of the state. It is \label{M40}the
confusion of the essence of a political phenomenon with its value; and
this confusion is a serious methodological error.  With respect to the
``substance'' of representation, Voegelin says that ``certain
mediatory institutions, the parties, have something to do with
securing or corrupting this substance'' and that ``the substance in
question is vaguely associated with the will of the people, but what
is meant by the symbol 'people' does not become clear. This symbol
must be stored away for later examination.''\footnote{L.c., p. 35.}
However -- as we shall see -- the symbol ``people'' does not become
clearer by Voegelin's later examination. On the contrary. It becomes
``mysterious'', a ``mystical substance.''\footnote{L.c., p. 43, 44.}
Yet this, perhaps, seems to be clear: the existential representation
at which the new science of politics is driving, claims to be a
representation of the people, although the symbol ``people'' may mean
something different from what it means in the elemental concept of
representation, i.e., the electorate.  As far as the ``mediatory
institutions, the parties ...~securing or corrupting this substance''
are concerned, we learn that ``the disagreement on the number of
parties that will, or will not, guarantee the flow of the substance
suggests an insufficiently analyzed ulterior issue that will not come
into grasp by counting parties.''\footnote{L.c., p. 35.}  Voegelin
refers to the fact that there exists a variety of opinion concerning
the effect of political parties on the working of a representative
system that he summarizes as follows: ``a representative system is
truly representative when there are no parties, when there is one
party, when there are two or more parties, when the two parties can be
considered factions of one party ...~a representative system will not
work if there are two or more parties who disagree on points of
principle.''\footnote{Ibid.} By a ``representative system'' he
understands in this connection a system of democratic representation.
Here, again, he confuses the question as to the essence of democratic
representation with the question under what conditions a democratic
system works satisfactorily. That political parties are possible in a
democracy, and that a constitution which does not allow the free
formation of political parties, which allows either no party at all or
only one party, is not democratic, cannot \label{M41}be, and has never been
denied by those who are of the opinion that parties are, or are not,
advantageous for the working of a democratic system. The view that
only one party is to be allowed in order to guarantee the workability
of the government is a common element of the anti-democratic
ideologies of fascism, national-socialism and communism; fascist
Italy, national-socialist Germany have been, and communist Russia
still is, typical ``one-party states.'' This term can have no other
meaning.  For if the constitution, as in a democracy, guarantees free
formation of political parties, the coming into existence of more than
one party is inevitable. A democracy cannot be a one-party state.
Until now, we were of the opinion that there is a vital difference
between a political system which allows only one party, the one
supporting the government, and a political system under which the
formation of parties is free; and that in a one-party state where
there are no free elections because the citizens can vote only for the
candidate of one party, the government cannot be considered as
representing the people. But the new science of politics informs us
that ``a type concept like the 'one-party state' must be considered as
theoretically of dubious value; it may have some practical use for
brief reference in current political debate, but it is obviously not
sufficiently clarified to be of relevance in science. It belongs to
the elemental class like the elemental type concept of representative
institutions.''\footnote{L.c., p. 36.} But a one-party state, as we
shall see, may offer an ideal case of ``existential'' representation.

The most characteristic type of a one-party state is the Soviet Union.
Voegelin says of this state: ``While there may be radical disagreement
on the question whether the Soviet government represents the people,
there can be no doubt whatsoever that the Soviet government represents
the \label{M41a}Soviet society as a political society in form for action in
history''.\footnote{Ibid.} He does not state in an unambiguous way that
the Soviet government does not represent the Soviet people; he does
not say that it represents the Soviet state, and not the Soviet
people. The only thing he decidedly asserts is that the Soviet
government represents the Soviet ``society.'' But by ``Soviet
society'' the soviet people may be understood. For, in order to show
that the Soviet government represents the Soviet society, he
refers to the fact that the ``legislative and administrative acts of
the Soviet government are domestically effective in the sense that the
governmental commands find obedience with the people ...~'' and that
``the Soviet government can effectively operate an enormous military
machine fed by the human and material resources of the Soviet
society.'' The soviet government represents the Soviet society because
it effectively controls the Soviet people. In this connection Voegelin
says: ``under the title of political societies in form for action the
clearly distinguishable power units in history come into
view.''\footnote{Ibid.} These ``power units'' are usually called
states. But why does the new science avoid this term? Why does it not
clearly distinguish representation of the state from representation of
the people? ``Political societies,'' says Voegelin ``in order to be in
form for action, must have an internal structure that will enable some
of its members -- the ruler ...~ to find habitual obedience for their
acts of command; and these acts must serve the existential necessities
of a society, such as the defense of the realm and administration of
justice.''\footnote{L.c., p. 36f.}

It is a generally recognized principle that a body of individuals, in
order to be considered as the government of a state, must be
independent of other state governments and able to obtain for the
legal order under which it is acting as government the permanent
obedience of the subjects. This principle applies to any government,
whether democratic or autocratic. The principle is only a particular
application of the more general principle that the legal order
constituting the state is valid only if it is by and large
effective, that is to say, obeyed by the individuals whose behavior it
regulates. It seems that the new science of politics presents this
principle, taken for granted by the old political and legal science,
under the term of ``existential'' representation. For it declares
``defense'' and ``administration of justice'' as ``the existential
necessities of a society'' and states: the ``process in which human
beings form themselves into a society for action shall be called the
articulation of a society. As the result of political articulation we
find human beings, the rulers, who can act for the \label{M41b}society, men
whose acts are not imputed to their own persons but to the society as
a whole -- with the consequence that, for instance, the pronunciation
of a general rule regulating an area of human life will not be
understood as an exercise in moral philosophy but will be experienced
by the members of the society as the declaration of a rule with
obligatory force for themselves. When his acts are effectively imputed
in this manner, a person is the representative of a
society.''\footnote{L.c., p. 37.} In this context ``representation''
presupposes effective imputation, which can only mean that the
imputation to the state of the acts of the ruler takes place only if
the rule is effective.

It is evident that the principle according to which the legal order
constituting the state is valid only if it is to a certain extent
effective, has no direct relation to the question of representation,
that is to say, to the determination by the legal order of the organs
of the community constituted by this order, the individuals competent
to represent the state. Only a valid legal order can determine the
representatives, and only a relatively effective legal order is valid.
The principle of effectiveness refers to the legal order constituting
the state, not the organs of the state. It is not the organs who are
effective, it is the norms created and applied by them in conformity
with a valid legal order which are to be effective. That the
government is effective means that the norms which are issued by this
organ and which form part of the legal order constituting the state
are effective.  The acts performed by an organ of the state and
especially by the government are acts of the state, that is to say,
imputable to the state, and, hence the individual performing the acts
represents the state, not because the organ is effective, but because
the individual and his act is determined by a valid, and that means by
a relatively effective legal order. Since only a valid, that is,
relatively effective legal order constitutes the community called
state, only on the basis of such a legal order organs of a state and,
therefore, representation is \label{M41c}possible, whether it is democratic
or non-democratic representation, representation of the state which is
or is not at the same time representation of the people. Effectiveness
- as a quality of the constituent order -- is a condition of any type
of representation, because a condition of the existence of the state.
Whether a body of individuals, as the government of a state,
represents the state, and at the same time the people of the state,
does not depend on the effectiveness of the commands, that is, the
norms, which it issues; for any body of individuals is the government
of a state only if it acts in conformity with an effective legal order
constituting the state, whether democratic or autocratic, and if the
norms issued by this body, forming an essential part of the legal
order, are by and large obeyed. Whether a government, which always
represents the state, represents also the people of the state, that is
to say, whether it is a democratic government, depends only and
exclusively on the answer to the question whether or not it is
established in a democratic way, that is to say, elected on the basis
of universal and free suffrage. Hence it is impossible to
differentiate the democratic type of representation from any other
type of representation by the criterion of effectiveness. But this is
just what the new science of politics endeavors. It is just the
differentiation at which the new science aims when it deprecates the
democratic type of representation as elemental because it does not --
as does the existential type -- imply the element of
effectiveness. Only by obliterating the difference between
representation of the state and representation of the people can the
new science maintain that there exists a difference of cognitive value
between democratic representation, as a merely ``elemental''
representation, and representation of the state, as an ``existential''
representation. By obliterating this difference, by avoiding the term
``representation of the state'', by using the ambiguous formula
``representation of society'', the new science creates the impression
that only that concept of representation which includes the element of
effectiveness is the correct one, and that this type of representation
always implies, in some way, representation of the
people. ``Obviously'', says the author of the new science, ``the
representative ruler of an articulated society cannot represent it as
a whole without standing in some sort of relationship to the other
\label{M42}members of society..'' By ``the other members of society'' only
the people can be understood.  ``...~under pressure of the democratic
symbolism, the resistance to distinguishing between the two relations
terminologically has become so strong that it has also affected
political theory ...~. The government represents the people, and the
symbol 'people' has absorbed the two meanings which, in medieval
language, for instance, could be distinguished without emotional
resistance as the 'realm' and the 'subjects'.''\footnote{L.c., p. 38.}
The ``two relations'' which under the pressure of democratic symbolism
are not distinguished, are the relationship of the ruler to society as
a whole and the relationship of the ruler to ``the other members of
society.'' The statement that the government in a democracy represents
the people as subjected to the government means that the government,
by representing the people as the society not including the members of
the government -- ``the other members of society'' -- represents the
society as a whole, because the members of the government belong to
the people as subjected to the government. They are at the same time
governing and subjected to the government. As members of the
government they are not -- as is the ruler in an autocracy -- exempt
from the government, they are at the same time governing and subjected
to the government. It is just for this reason that only in a democracy
the government represents the society as a whole, because it
represents the society including the members of the government. But it
is very likely that the new science of politics understands by
``society as a whole'' the state. For this term is supposed to mean
about the same as, in medieval language, the term ``realm'', in
contradistinction to the ``subjects.'' This terminology corresponds to
the modern distinction between ``state'' and ``people.''  The
statement that a democratic government represents the people does
indeed mean that the government representing the people represents the
state. Again we ask: Why does the new science refrain from using the
modern term ``state,'' which is much less ambiguous than the medieval
``realm'' literally meaning ``kingdom''? Why does it speak of
``society as a whole'' when it really means state? Evidently because
representation \label{M43}of society as a whole implies necessarily
representation of ``the members of society'', because the existential
representative of the state has to be considered as representing also
the people. ``The representative ruler of an articulated society'' can
only be a ruler who effectively represents society; and if he
effectively represents society, he represents it ``as a whole'',
especially if ``society as a whole'' means ``state.'' It can only be
the society as a whole which a ruler in the existential sense -- an
``existential'' ruler -- represents; and by the representative ruler of
an articulate society, referred to in the above-quoted statement,
obviously an existential ruler is meant.  But every government --
whether democratic or autocratic -- is a ruler in the existential
sense, an existential ruler. And, now, the new science of politics
declares that the representative ruler of an articulated society
cannot represent it as a whole -- and that probably means: cannot
represent the state -- without standing in some sort of relationship to
the other members of the society, that is to say, to the people. That
he stands in a relationship to the people can only mean that he
represents the people; For representation of the people is one of the
two relations terminologically not distinguished ``under pressure of
the democratic symbolism.'' The ruler must stand ``in some sort'' of
relationship to the other members of the society, that is to the
people, but not necessarily in that sort which is constituted by
elections on the basis of universal and free suffrage. For this sort
of relationship is only elemental, not existential. The Soviet
government, as the new science asserts, ``represents the Soviet
society as a political society'' in the most effective way because the
``legislative and administrative acts of the Soviet government are
domestically effective in the sense that the governmental commands
find obedience with the people'' and it ``can effectively operate an
enormous military machine fed by the human and material resources of
the Soviet society.''\footnote{L.c., p. 36.} That can only signify
that the Soviet government represents the Soviet society ``as a
whole,'' especially if ``society as a whole'' means state. Hence
\label{M44}the Soviet government is the ideal type of an existential ruler, a
``representative'' ruler of an articulated society, represented ``as a
whole'' by the ruler. If a representative ruler of an articulated
society ``cannot represent it as a whole without standing in some sort
of relationship to the other members of the society,''\footnote{L.c.,
p. 38.} that is to say, without representing in some way the people,
then the Soviet government, which is certainly not a democratic
government, represents the Soviet people. This, of course, is not
expressly maintained by the new science of politics. But it is clearly
implied in its doctrine of representation, with its tendency to
belittle the importance of the democratic type of representation as
merely elemental and to put in the foreground the existential type, in
which the principle of effectiveness is emphasized.

Voegelin refers to the fact that in the Magna Carta the Parliament is
designated as ``the common council of our realm,'' which probably
means -- expressed in modern terminology -- as an organ of the state,
and, as he emphasizes, ``not perhaps as a representation of the
people;''\footnote{Ibid.} which is not astonishing since the Magna
Carta was not the constitution of a democracy. Then Voegelin refers to
the writs of summons of the 13th and 14th centuries and ascertains
that they draw ``the new participants of representation into the royal
representation itself. Not only is the realm the king's but the
prelates, the magnates, and the cities are also his. ...~The
symbol 'people' does not appear as signifying a rank in articulation
and representation; it is only used, on occasion, as a synonym for
realm in a phrase like the 'common welfare of this
realm'.''\footnote{L.c., p. 39.} Then he quotes the following sentence
from the address of Henry VIII to Parliament in Ferrers' case: ``We be
informed by our Judges that we at no time stand so highly in our
estate royal as in the time of Parliament, wherein we as head and you
as members are conjoined and knit together into one body politic, so
as whatsoever offense or injury (during that time) is offered to the
meanest member of the House is to be judged as done against our
person \label{M45}and the whole Court of Parliament.''\footnote{L.c., p. 40.}
That means -- expressed in modern language -- that the king and each
member of the Parliament is to be considered as representing the
Parliament. There is no question of a representation of the people.

The ``representation of the people'' appears only when Voegelin quotes
Lincoln's formula of a ``government of the people, by the people, for
the people.''\footnote{Ibid.} Voegelin says that with this formula
``the limit is reached where the membership of the society has become
politically articulate down to the last individual, and,
correspondingly, the society becomes representative of itself.'' The
``limit'' of articulation -- meaning organization -- is reached when the
legislative and the main executive organs are to be elected on the
basis of free and universal suffrage, that is to say, when
representation in the elemental sense is established. This is, without
the slightest doubt, the tenor of Lincoln's formula. And if this
formula is -- as Voegelin here admits -- a ``masterful, dialectical
concentration, an unsurpassable fusion of democratic symbolism with
theoretical content,'' why did Voegelin, some pages before say, that
the ``theorist'' cannot do very much with the elemental concept of
representation, that it has no or not enough ``cognitive value''? And,
indeed, he is not satisfied with this masterful and unsurpassed
formula. For he thinks further clarification is necessary. This
clarification he finds in the works of Sir John Fortescue who, in the
15th century published \emph{The Governance of England} and his famous
\emph{De laudibus legum Anglie}. What are the important contributions
made by this writer to the theory of representation, to the
existential, not the elemental, representation?  First, his creation
of the concepts of ``eruption'' and ``proruption''.\footnote{L.c.,
pp. 42f.} Fortescue \label{M45a}simply says: ``{\em ex populo erumpit
regnum}''\footnote{Fortescue, \emph{De laudibus legum Anglie}, chap.
XIII.}, the kingdom emerged from the people.  But Voegelin says:
``Fortescue coined the term 'eruption' as a technical term for
designating the initial articulation of a
society.''\footnote{Voegelin, l.c., p. 42.}, which is a rather
imaginative interpretation of Fortescue's simple statement. The term
``proruption'' is used by Fortescue in the following sentence:
``...~\emph{regnum Anglie} ...~\emph{in dominium politicum et regale
prorupit} ...~'' (the kingdom of England was transformed into a
dominium regal and political). The terms ``eruption'' and
``proruption'' have no specifically technical significance. They do
not allow us ``to distinguish'' -- as Voegelin asserts -- ``the
component in representation that is almost forgotten wherever the
legal symbolism of the following centuries came to predominate in the
interpretation of political reality,''\footnote{L.c., p. 43.} namely,
the principle of effectiveness.  Voegelin interprets Fortescue's
statement in which the term ``proruption'' is used to designate the
transformation of a dominion royal into a dominion royal and political
to mean: ``a realm will be achieved only when a head is erected,
\emph{rex erectus est}, that will rule the body.'' This, however, is
not what Fortescue means by this statement. He says: ``A king of
England cannot, at his pleasure, make any alterations in the laws of
the land, for the nature of his government is not only regal, but
political. Had it been merely regal, he would have a power to make
what innovations and alterations he pleased, in the laws of the
kingdom, impose tollages and other hardships upon the people, whether
they would or no, without their consent ...~but it is much otherwise
with a king, whose government is political, because he can neither
make any alteration, or change in the laws of the realm without the
consent of the subject, nor burthen them, against their wills, with
strange impositions ...~''\footnote{Fortescue, \emph{Commendation of
the Laws of England}. Translated by Francis Grigor. London, 1917,
chap. IX, p. 17.} That the government of England is transformed from a
\emph{dominium regale} into a \emph{dominium politicum et regale}
means that the government of England is transformed from an absolute
into a limited monarchy. This is the main point \label{M46}in the work that
Fortescue wrote for the instruction of a prince. The terminological
distinction of eruption and proruption is of no importance at all. It
is certainly not -- as Voegelin maintains -- a ``theoretical
achievement.''\footnote{Voegelin, l.c., p. 43.}

Another achievement of Fortescue is, according to Voegelin, that ``he
transferred the Christian symbol of the \emph{corpus mysticum} to the
realm.''\footnote{Ibid.} Voegelin admits that Fortescue calls the
realm, that is, the state a \emph{corpus mysticum} ``only
analogically.''  But he adds: ``The \emph{tertium comparationis} would
be the sacramental bond of the community.'' This sacramental bond, he
says, ``would be neither the Logos of Christ ...~nor a perverted
Logos as it lives in modern totalitarian communities.'' Nevertheless
Fortescue was on the ``search for an immanent Logos of society''; ``he
was not clear about the implications'' of his search but ``he found a
name for it; he called it the \emph{intencio populi}. This
\emph{intencio populi} is the center of the mystical body of the
realm; ...~he described it as the heart from which is transmitted
into the head and members of the body as its nourishing blood stream the
political provision for the well-being of the people.'' But Voegelin
emphasizes that ``the animating center for a social body is not to be
found in any of its human members. The \emph{intencio populi} is
located neither in the royal representative nor in the people as a
multitude of subjects but is the intangible living center of the realm
as a whole. The word 'people' in this formula does not signify an
external multitude of human beings but the mystical substance erupting
in articulation; and the word 'intention' signifies the urge or drive
of this substance to erupt and to maintain itself in articulate
existence as an entity which, by means of its articulation can provide
for its well-being.''\footnote{L.c., p. 44.}

\label{M46a}This interpretation gives rise to the impression that Fortescue was
a mystic. This, however, is not the case. In his \emph{Commendations of
  the Laws of England}, to which Voegelin's interpretation refers,
Fortescue says expressly to the prince for whom he wrote this work, as a
reply to the argument that a kingdom ought to be governed by the best of
laws but that ``nature always covets what is best'': ''Sir! there is no
such mystery in these things..''\footnote{Fortescue, l.c., chapt. VII,
  p. 13.} Fortescue simply compares the state as a body politic -- as so
many others have done before and after him -- without any metaphysical or
mystical intention, with the body of man. The purpose of this comparison
is in the first place to show -- a point not mentioned by Voegelin -- that
``it is absolutely necessary, where a company of men combine and form
themselves into a \emph{body politic}, that some one should preside as
the governing principal, who goes usually under the name of
\emph{King}.''\footnote{L.c., chapt. XIII, p. 21.} The decisive argument
is: just as the human body must have a head, a state must choose a king.
``In this order, as out of an embrio, is formed an human body, with one
head to govern and control it; so, from a confused multitude is formed a
regular kingdom, which is a sort of a mystical body, with one person, as
the head, to guide and govern. And, as in the natural body (according to
the \emph{philosopher}) the heart is the first thing which lives, having
in it the blood, which it transmits to all the other members, thereby
importing life, and growth and vigour; so in the body politic, the first
thing which lives and moves is the intention of the people, having it in
the blood, that is the prudential care and provision for the public
good, which it transmits and communicates to the head, as the principal
part ...~''.\footnote{Ibid.}  From this passage it follows that Fortescue
employs the term ``a sort of mystic body'' only in order to stress that
the political body is not a ``natural body'', that only a comparison is
meant. He uses the term political ''body'' in no other sense than it is
used in ordinary language. \label{M47}As to the meaning of the ``intention of
the people'', his comparison of this element with the heart shows
plainly his view of the essential part the people plays in a limited
monarchy. What he is driving at in this respect becomes clear by his
comparison of the laws of the state with the ``nerves or sinews of the
body natural ...~.  And as the head of the body cannot change its nerves
or sinews, ...~\emph{neither can a king, who is the head of the body
  politic, change the laws thereof, nor take from the people what is
  their's, by right}, against their consents''.\footnote{L.c., chapt.
  XIII, p. 22.}  There can be no doubt that ``people'' means the
``company of men'' of which he says in the above-quoted passage that
they ``combine and form themselves into a body politic.'' What he has in
mind when he refers to the people is indicated in one of the most
important statements of his political theory, which Voegelin does not
quote: ``\emph{For he [the king] is appointed to protect his subjects in
  their lives, properties and laws; for this very end and purpose he has
  the delegation of power from the people; and has no just claim to any
  other power but this}.''\footnote{Ibid.} There is nothing of a
``mystical substance erupting in articulation,'' nothing of a
``sacramental bond of the community,'' nothing of an ``immanent
Logos''\footnote{Voegelin, l.c., pp. 43f.} in the people from which the
king, according to the theory of Fortescue, derives his power.

The final result of Fortescue's doctrine of representation is --
according to Voegelin -- the statement: Since ``the king is in his realm
what the pope is in the church, a \emph{servus servorum Dei}'', ``all
that the king does ought to be referred to his kingdom.''\footnote{L.c.,
  p. 45.} Voegelin calls this doctrine ``the most concentrated
formulation of the problem of representation.'' We may assume that he
considers Fortescue's doctrine as the perfect formulation of the
existential, not merely elemental, representation. But the statement
that the acts of the king are to be imputed to the kingdom means only
that the king is the organ of the state, and the statement that the king
is a \emph{servus servorum Dei} nothing but a metaphysical expression of
the idea that it is his duty to serve his subjects. The statements
are taken from Fortescue, \emph{The Governance of England}, and read as
follows: ``Wherefore all he doth owith to be referred to his kingdom.
For though his estate be the highest estate temporal in the earth,
yet it is an office, in which the mynestrith to his reaume
defence and justice. And therefore he may say of him selff and off his
reaum, as the pope saith off him selff and the church, in that he
writithe, servus servorum Dei.'' Consequently the state must financially
sustain the king as the church sustains the pope. The statements are
made in chapter VIII dealing with the question of the finances of the
king. Neither the statement that all the king does ought to be referred
to his kingdom, nor the statement that the king is in his realm what the
pope is in the church can be regarded as a contribution to the theory of
existential representation.  Interesting is only the statement which
Fortescue makes in his \emph{De laudibus legum Anglie}, and which
Voegelin does not quote: that the king ``has the delegation of power
from the people.'' For it may be interpreted to mean that the king
represents the people although he is not elected, nor can be dismissed
by the people, nor are his actions determined by the people. The only
basis of the idea that he represents the people is that he is obliged to
protect the people and that he cannot change the law against the will of
the people represented in Parliament. His government is interpreted as a
government by the people only and exclusively because it is a government
for, or not against, the people. This, of course, is a fiction. And the
purpose of the fiction is not only to emphasize the moral obligation of
the king to govern in the interest of the people but also -- and not at
the least -- to strengthen the authority of the king by the idea
that he represents the people, that the people governs through him. No
government can claim to represent the people or have its power from the
people if the establishment of the government is legally independent of
the will of its subjects, organized in popular assemblies\label{M48} or
electorates. It is the same fiction as used by Soviet political theory
which presents communist party dictatorship as a democratic government
only because it claims to govern in the interest of the people. Is it
the intention of the theory of existential, and not merely elemental,
representation to legitimize this fiction?

In order to demonstrate ``articulation'' as condition of not merely
elemental but existential representation, Voegelin goes back to a writer
of the 8th century, Paulus Diaconus, author of the \emph{History of the
  Lombards}. However, this display of profound knowledge of historical
literature is hardly in proportion to its result. It is the fact
reported by Paulus Diaconus that the Lombards after having been ruled by
two dukes, set themselves a king, and, then, ``the victorious wars
began.''\footnote{L.c., p. 47.} The moral of the story -- and perhaps of
all history -- according to Voegelin is: ``To be articulate for action
meant to have a king; to lose the king meant to lose fitness for action,
when the group did not act, it did not need a king.''\footnote{Ibid.}
\label{M49}Action, of course, means war. Is it the platitude that democracy is
not the best form of articulation for the purpose of waging war, that
Voegelin wants to prove? Or is it rather the thought, scantily hidden
behind the platitude: that an autocratic military leader is in a more
profound sense of the term a ``representative'' of the people than a
democratic government, because the former does, whereas the latter does
not ``articulate'' the people for action and thus provides the ``desired
substance''?

The last source from which Voegelin tries to extricate the idea of an
existential and not merely elemental representation is Maurice Hauriou's
\emph{Precis de Droit constitutionel}.\footnote{Maurice Hauriou,
  \emph{Precis de Droit constitutionel}, 2nd ed., 1929.} Hauriou is a
typical representative of the traditional theory of public law
prevailing in France during the first half of the 20th century.
Therefore it would be surprising to find in his writings a new doctrine
of representation. Voegelin asserts that according to Hauriou ``To be a
representative means to guide, in a ruling position, the work of
realizing the idea through institutional embodiment, and the power of a
ruler has authority in so far as he is able to make his factual power
representative of the idea.''\footnote{Voegelin, l.c., p. 48.} The
``idea'' is what Hauriou calls \emph{id\'{e}e directrice}. By
``realizing the idea of the institution'' the government becomes
``representative in the existential sense'', not in the merely
``constitutional'', that is, ``elemental sense.''\footnote{L.c., p.
  49.} If we examine what Hauriou has to say about representation we
find that he uses this term -- as do most of the writers on
constitutional law and political theory -- in different meanings. He
distinguishes between the representation of an individual by an
individual (``\emph{l'homme qui agit au nom d'un autre homme est son
  repr�sentant}'') and representation of an institution, especially the
state, by an individual (\emph{l'homme qui agit au nom d'une institution
  est �galement son repr�sentant ou son organe}'').\footnote{Hauriou, p.
  19.}  He emphasizes that in the field of public law the term
``repr\'{e}sentant'' means the \label{M50}same as ``organe''; he says that it
is quite impossible to define the organs of a juristic person without
making it apparent that they are in some way
``representatives.''\footnote{L.c., p. 209.} As far as the organs of the
state are concerned, he distinguishes them from simple agents or
employees (``des simple agents ou commis'').  An organ representing the
state is characterized by the fact that the individual having this
capacity has the power of initiative and is politically
responsible.\footnote{L.c., p. 212/13.} Finally, he uses the term
``representation'' to designate the specific democratic form of
representation. His definition of representative government runs as
follows: ``The representative government may be defined as a government
in which one or two assemblies elected by the people represent the
people vis-a-vis the central power and participate in the government,
first, by voting taxes, then, by voting laws.''\footnote{L.c., p. 147.} 
Hence, ``to be a representative'' may mean
many things quite different from what Voegelin presents as Hauriou's
doctrine concerning the meaning of being a representative. The
\emph{id\'{e}e directrice}, which, according to Voegelin plays so
important a role in Hauriou's doctrine that Voegelin attributes to its
realization the transformation of a merely elemental into an existential
representation, is according to Hauriou simply the idea of organization.
Examining the problem of social organization from a sociological point
of view, Hauriou says in opposition to other doctrines especially the
organism theory -- that the primary phenomena of social organization are
of political character: the coming into existence of a directing or
founding center (\emph{centre directeur ou fondateur}) of governmental
organs, governmental equilibrium, and, finally consentments. ``The role
of the directing or founding center is to implant an idea in the social
milieu; this idea is that of the organization to be created, considered
as an enterprise to be realized; it implies a plan of the organization
and, consequently, contains potentially the form of the
latter.''\footnote{L.c., p. 72.} This idea is the \emph{id\'{e}e
  directrice}. \label{M51}``A social organization,'' says Hauriou, ``becomes
durable ...~when, owing to the balance of organs and power, ``the
government can be subordinated to the \emph{id\'{e}e directrice} which
is within the organization from the moment of its foundation, and when
this system of ideas and balance of powers has been confirmed, in its
form, by the consentment of the members of the institution as well as
the social millieu.''\footnote{L.c., p. 73.} The theory that a social
organization comes into existence when its idea is implanted in the
social milieu is highly problematical, and the statements concerning the
durability of social organizations are truisms. Neither the one nor the
other is in essential connection with Hauriou's doctrine of
representation, which does not deviate in any respect from the
traditional view. There is nothing in this doctrine from which the
``lesson'' could be drawn which Voegelin formulates as the result of his
analysis: ``In order to be representative, it is not enough for a
government to be representative in a constitutional sense (our elemental
type of representative institutions); it must also be representative in
the existential sense of realizing the idea of the
institution.''\footnote{Voegelin, l.c., p. 49.} Hauriou's above-quoted
definition of representative government shows not the slightest tendency
toward Voegelin's existentialism. And there is no basis of Voegelin's
assertion that Hauriou's theory of representation implies the ``warning
...: If a government is nothing but representative in the constitutional
sense a representative ruler in the existential sense will sooner or
later make an end of it; and quite possible the new existential ruler
will not be too representative in the constitutional
sense.''\footnote{Ibid.} It is not Hauriou, it is the new science of
politics which conveys this warning, the meaning of which can only be an
appeal for a political reform by which the merely elemental
representation, that is, the democratic form of government, is
transformed into a \label{M52}regime capable of organizing the mass of the
people into a body fit for action.

\subsection{2.}

After having dealt under the -- not quite appropriate -- heading
``representation and existence'' with the problem of representation
of the state by its organs, Voegelin pretends to deal with another
aspect of the problem of representation under the heading
``representation and truth.'' But the topic treated in this chapter has
nothing to do with the representation of the state by its organs and
even less with the problem of truth.

The social phenomenon to which Voegelin calls the attention of the
reader is the well known fact that at a relatively primitive state of
social and intellectual evolution the social order is interpreted as
being in conformity with a divine will, the will of the gods or of a
special god, or of a god supposed to be the only existing God. A
characteristic version of this interpretation is the idea that the
social order, especially the legal order of a state, is the imitation
of the order of the universe, a microcosm, a little world as the image
of the macrocosm, the big world. It is a theological ideology of
society and a typical attempt at presenting the given social order as
the realization of divine justice on earth. Its purpose is obviously
to justify this order and thus to confirm and strengthen the authority
of the ruler, who -- \label{M53}this is the most important part of
this ideology -- is presented as a descendant or deputy of God. If the
alleged relationship between the divine order of the world and the
social order in question is termed ``representation,'' if one says, as
Voegelin does, that according to this idea the social order
``represents'' the divine order of the cosmos, the term representation
has a totally different meaning than when it is used to express the
relationship of the state to its organs. Representation in this case
means organship, representation in the other case means likeness,
image picture, reproduction, and the like. These, by the way, are not
the only meanings of the term, which, e.g., is used also to designate
a dramatic production or performance. Under the cover of this term
with many different meanings, Voegelin treats two totally different
problems which have nothing to do with each other, claiming that the
analysis of the one adds something to the understanding of the other.

The interpretation of an established social order as being in conformity
with the divine order of the world is not primarily a
self-interpretation of society, as Voegelin asserts. It is the result of
religious speculation by priests, a theological theory of society. It is
quite understandable that this theological theory is adopted by the
ruler of the state, in whose interests the ideology evidently is
elaborated, and thus becomes secondarily a self-interpretation of
society. It is a typical case where the symbols of self-interpretation
of society cannot be separated from theoretical concepts. But
Voegelin speaks of these ideologies as of ``symbols in which a society
interprets the meaning of its existence ...~''\footnote{L.c., p. 53.}

More serious, because utterly misleading, is the interpretation of the
ideology in question as ``representation of truth.'' Voegelin asserts
that the meaning of the ideology according to which a state order is
the imitation of the divine order of the cosmos is that society
interprets itself as \label{M54}the ``representation of
truth''\footnote{Ibid.}; and the new science of politics accepts this
interpretation. Voegelin confronts the ``truth represented by
society'' with the ``truth represented by the theorist'' which is the
truth of the statements a social theorist makes about society. In the
phrase ``representation of a truth by a theorist'' the term
representation has a third meaning, different from that which it has
in the phrases ``representation of a state by its organ'' and
``representation of the divine order by the state order.'' Voegelin
raises the question: ``If the truth represented by the theorist should
be different from the truth represented by society, how can the one be
developed out of the other by something that looks as innocuous as a
critical clarification?''\footnote{Ibid.} That shows that Voegelin
considers the truth represented by society as being of the same kind
as the truth claimed by a \label{M55}scientific statement. He refers
to the case where a ``truth represented by the theorist was opposed to
another truth represented by society,'' and asks: ``is such language
empty, or is there really something like a representation of truth to
be found in political societies in history?'' It is of the greatest
importance to note that Voegelin does not formulate the problem which
he calls ``representation of truth'' as the problem of a political
ideology falsely pretending to express a truth, but as the question
whether there is ``really something like a representation of truth to
be found.'' He presupposes an affirmative answer to this question. He
insists upon the view that the problem of representation is not
exhausted ``by representation in the existential sense,'' that it is
``necessary to distinguish between representation of society by its
articulated representatives and a second relation in which society
itself becomes the representative of something beyond itself, of a
transcendent reality.''\footnote{L.c., p. 54.} This transcendent
reality is the truth ``represented'' by society. And because this
truth is ``truth'' in the specific sense of the term, in the same
sense as the truth of a scientific statement, truth in a logical and
epistemological sense, Voegelin is dealing with an ideology according
to which a social order is the imitation of the cosmic order.

It may be that theological ideologists, and rulers adopting the
ideologies produced by these ideologists working in the service of the
rulers, speak of a divine truth, realized in a society constituted by a
given social order. But what they mean by this truth is not a truth in
the logical and epistemological sense, not a truth of science, but a
moral-political value: justice. The confusion of truth with justice is a
characteristic element of unscientific, religious speculation. Truth in
a scientific sense is the quality of a proposition, and a proposition
is true if it is in conformity with reality; justice is the quality of
human behavior or of a normative order established by acts of human
behavior. It means conformity with a supreme norm presupposed to be
valid. If, as it is assumed in theological speculation, reality is
created by the will of God, who is absolutely just, and hence
justice is immanent in reality, truth and justice seem to coincide. When
Jesus stands before Pilatus, the Gospel of St.~John -- essentially
influenced by theological speculations -- lets him say: ``I was born to
bear witness to the truth. Every one who is of the truth hears my
voice.'' But what he meant was: he was born to bear witness to justice,
the justice that he was about to establish in the kingdom of God, this
realization of divine justice on earth; or -- as Voegelin, confirming the
language of the Gospel without an attempt at ``critical clarification'',
would say, -- the representation of a transcendent truth.

As an example of a ``cosmological empire'', that is to say, a state
which by its theological ideologists and, then, by its rulers has been
interpreted as the representation of truth, Voegelin refers to Persia
under Achaemenides. He quotes an inscription ``celebrating the feats
of Darius I,'' according to which ``the king was victorious because he
was the righteous tool of Ahuramazda; he 'was not wicked, nor a liar';
neither he nor his family were servants of Ahriman, of the Lie, but
\label{M56}'ruled according to righteousness.' ''\footnote{L.c., p. 55.} It is
evident that here truth means the same as righteousness; that is,
justice. And if the enemies of the king are called liars, if it is said
that ``lies made them revolt,'' nothing else is expressed but the
judgment of the king that revolution is a crime, because against
justice. The new science of politics, accepting without objection this
confusion of truth with justice, considers the king as the
representative of the truth and the revolutionaries as ``representatives
of the Lie''. The inscription contains the passage: ``By the grace of
Ahuramazda there is also much else that has been done by me which is not
graven in this inscription; it has not been inscribed lest he who should
read this inscription hereafter should then hold that which has been
done by me to be too much and should not believe it, but should take it
to be lies.''\footnote{L.c., pp. 55f.} This means -- expressed in a
typical oriental exaggeration: the deeds of the king are incredibly
great. But Voegelin's interpretation is: ``No fibs for a representative
of the truth; he must even lean over backward.''\footnote{L.c., p. 56.}

Another ``cosmic'' empire, that is, a state that interprets itself as
representative of a transcendent truth -- the truth which the new
political science tries to find in political societies in history -- is
the empire of the Mongols. To prove the cosmic character of this empire
Voegelin quotes some passages from a letter of the ruler of the Mongols
to the Pope in which the former rejects the request of the latter that
the Mongols should receive baptism and submit to the authority of the
Church. In the letter the Khan of the Mongols asserts, as dozens of
other kings of much smaller kingdoms have asserted and still assert,
that he has his authority from his God: ``By the virtue of God, from the
rising of the sun to its setting, all realms have been granted to us'';
and of course, that his god is the true god: ``By order of the living
God Genghis Khan, the sweet and venerable Son of God, says: God is high
above all, He, Himself, the immortal God, and on earth, Genghis Khan is
the only Lord.''\footnote{L.c., pp. 57, 58.} If such a statement is
sufficient for the science of \label{M57}politics to interpret the realm of the
Mongols as representation of truth, it was not necessary to go back as
far as that in history. The Kingdom of Prussia under William II., who
again and again asserted to be by the grace of God the ruler of men, or
modern Japan the emperor of which is worshiped as a son of God, were as
good examples of cosmic empires and representation of truth as the
Mongol empire. If a political science has to say about this ideology
anything else but that it is a lie spread by agents of the respective
governments for the purpose of bolstering their authority, it exposes
itself to the suspicion of having purposes other than that to bear
witness to the truth. Plato, to be sure, whose metaphysics the new
science considers as its foundation, has another idea of truth. He
distinguishes not only between truth and lies but also between two kinds
of lies: lies which are and lies which are not politically useful. If
they are politically useful, the government -- but only the government --
is allowed to use them. They are a kind of truth, a political truth. But
taking a lie, for political reasons, as a truth, is the typical device
of an illusive ideology. It has no place in a scientific description of
political reality. 

If the Khan of the Mongols who rejects baptism and submission to the
authority of the Pope is a representative of truth because he asserts
himself to be the son of God, then the Pope, who asserts to be the
deputy of God on earth, is also a representative of truth, and, hence,
there exists ``a conflict of truth,'' a situation which Voegelin
seriously considers to be possible. But if ``truth'' is taken in its
scientific, and not in its political sense, no conflict of truth is
possible, because of two contradictory statements only one can be true,
the other being rejected as untrue. This is especially the case to which
Voegelin refers when he speaks of a conflict of truth. It is the case
where ``a truth represented by the theorist was opposed to another truth
represented by society.''\footnote{L.c., p. 53.} If what the theorist
asserts is true, and, is opposed to \label{M58}the self-interpretation of
society, then this self-interpretation is not ``another truth'', but a
lie. It is another truth only in the Platonic sense according to which a
lie, if it is useful to the government, is a truth. If a lie can be a
truth, no ``critical clarification'' of the self-interpretation of
society -- a task of political science recognized by Voegelin -- is
possible.

Critical clarification of self-interpretation of society is just that
function of a social science which is called critique of ideology. It is
self-evident that the new science of politics cannot renounce this
function. Hence Voegelin is looking for a truth ``that is apt to
challenge the truth of the cosmological empires.''\footnote{L.c., p. 60.}
Why only the truth of the cosmological empires, why not the truth of any
self-interpretation of society -- or social ideology -- is not quite
understandable. For the truth ``apt to challenge the truth of the
cosmological empires'' must be apt to challenge any truth which is
challengeable, and that means which is not as it pretends to be a truth,
but a lie. It can be only the truth of science. This truth is not a
secret treasure hidden somewhere to be discovered by a visionary, but is
a rational method, the result of the development of science in general
and social science in particular. It can by its very nature, not be the
exclusive possession of a particular school of thought. But according to
Voegelin's new science of politics, there is a ``discovery'' of this
truth, and this discovery is ``a historical event of major
dimensions.''\footnote{Ibid.} Which is this historical event?

The discovery of the truth, Voegelin asserts, has taken place roughly
in ``the period from 800 to 300 B.C.'', and occurred ``simultaneously
in the various civilizations but without apparent mutual influence'';
in China, India, Persia, Israel, and Hellas. The discoverers of the
truth are Confucius and Lao-tse, the writers of the \emph{Upanishads}
and Buddha, Zoroaster, the Jewish prophets, some philosophers and
poets of ancient Greece. Hence it is in religious speculations, that
is just there where the truth of the cosmological empires, the truths
of the self-interpretation of societies have their origin, that the
truth is to be found by which \label{M59}these self-interpretations are to be
challenged.  How the truth of which Darius I. pretended to be the
representative by asserting that he was victorious ``because he was
the righteous tool of Ahuramazda'' and not a servant of Ahriman, could
be ``challenged'' by the truth discovered by Zoroaster, who founded
the religion of Ahura Mazda and Ahriman which is the source of King
Darius' truth, is indiscoverable. And even more mysterious is how it
is possible to find one and the same truth in religions which are so
different as that of Zoroaster and that of the Jewish prophets, or
between the thought of the Buddha and the doctrine of ideas of Plato;
or even between the ethics of Confucius, which did not refer to a
particular god, and the metaphysics of Aristotle, which backed the
polytheistic religion of Greece. Voegelin speaks of a ``simultaneous
outbreak of the truth of the mystic philosophers and
prophets''\footnote{Ibid.}. But if it was an ``outbreak'' at all, and
not the result of gradual evolutions, it was the outbreak of very
different truths, and insofar as it was a ``mystic'' truth, it is
hardly apt to challenge the truth of the cosmological empires, or any
similar ``truths,'' as a ``theoretical truth''\footnote{Ibid.},
``opposed to another truth represented by society''\footnote{L.c.,
p. 53.}. And it is just for the purpose to find a basis for the
opposition of a theoretical interpretation of society to the
self-interpretation of society that Voegelin is directing the new
political science toward the truth of mystic philosophers.

The difficulty of explaining how so many and partly contradictory truths
are to be considered as ``the'' truth does not exist for Voegelin. For,
after having referred to at least five totally different religious and
philosophical systems as the sources of the truth for which the new
political science is looking, he resolutely restricts the field of the
search for this truth by turning ``to the more special form which this
outbreak, the outbreak of the truth, has assumed in the West.'' He
justifies this restriction by the fact that ``only in the West ...~has
the outbreak culminated in the establishment of philosophy in the Greek
sense and in particular of a theory of politics.''\footnote{L.c., p.
  61.} But why is only philosophy in the Greek sense, and not philosophy
\label{M60}or religion, in the Chinese, Indian, Persian, Jewish sense exemplary
for the new political science, if the representatives of these
philosophies and religions are likewise in the possession of ``the''
truth, apt to challenge the ``truth'' of a self-interpretation of
society? Why should Greek philosophy, and not Persian religion be used
to challenge the truth of the Persian empire, if this truth, too, is an
object of the new science of politics? And, above all: why is philosophy
in the Greek sense only the metaphysics of Plato and Aristotle and not
the philosophy of the Sophists, of Leucippus and Democritus, the
founders of the theory of atoms?

That the new science of politics has no answer to these questions could
be forgiven as a pardonable sin, if it only would answer that question
which is decisive in its fight against positivism because this answer
alone could justify its claim to substitute for a destructive
pseudo-science a constructive, true science of politics. It is the
answer to the question: What is this ``challenging theoretical
truth''\footnote{L.c., p. 60.} which Voegelin promises to disclose when
he says that its establishment is ``a rather complex affair requiring a
more detailed examination.''\footnote{L.c., p. 60.}

It has been pointed out that Voegelin considers himself entitled to
restrict these examinations to ``philosophy in the Greek sense.''
First, he examines Plato's formula ``that a polis is man written
large.''\footnote{L.c., p. 61.} Voegelin does not mention the decisive
point: that Plato uses this analogy between man and state only in
order to answer the question, what is justice, as a virtue of man.
The result at which Plato aims with this analogy is that justice is
achieved when in the soul of man the rational element (reason) rules
over the emotional elements (feelings and will). The postulate to act
reasonably is a commonplace and in itself no answer to the question of
justice. But Plato uses this empty formula in order to justify his
postulate that in the ideal state a small group of so-called
philosophers shall rule -- with the help of the army (the warriors) as
their \label{M61}instrument -- over the working people, completely
excluded from any direct or indirect participation in the
government. This extremely autocratic regime is the just order of the
ideal state, presented by Plato not as a definite, only as a
provisional answer to the question of justice; but it is the only
answer we can find in his \emph{Republic}. There is no reference to
the constitution of the ideal state in Voegelin's analysis of Plato's
analogy between man and state.

Voegelin formulates the result of his analysis as follows: ``Plato was
engaged concretely in the exploration of the human soul, and the true
order of the soul turned out to be dependent on philosophy in the strict
sense of the love of the divine \emph{sophon}.''\footnote{L.c., p. 63.}
\emph{Sophon} means wisdom. What wisdom? Wisdom in general or a special
wisdom? And what is the truth thus achieved? Voegelin continues: ``The
true order of man, thus, is a constitution of the soul, to be defined in
terms of certain experiences which have become predominant to the point
of forming a character. The true order of the soul in this sense
furnishes the standard for measuring and classifying the empirical
variety of human types as well as of the social order in which they find
their expression.'' But which is the ``true'' and that means according
to Plato, the just -- ``order of the soul''? Without a precise
determination of the constitution, no standard of measuring social
orders is possible. Why does Voegelin not reproduce Plato's answer to
this question, why does he not say, that -- according to Plato -- the only
just order of the state is the autocratic constitution of his ideal
republic?

Voegelin proceeds to Aristotle's Ethics, which, as is well-known, is an
ethics of virtues. The question as to what is morally good, which are
the norms of the moral order, is answered by the enumeration of certain
virtues, such as courage, truthfulness, and the like. Again, Voegelin
does not mention the essential point of the Aristotelian ethics, the
principle according to which the question can be answered: How to define
the virtues a man must have in order to be considered as morally good.
It is the famous doctrine that in ethics virtue can be defined just as
in geometry the point dividing a given line into two equal parts can be
\label{M62}determined. For virtue is the middle between two extremes, which
are vices. This is the so called \emph{mesotes} doctrine (\emph{mesos}
meaning the middle). Sine the extremes, i.e., the vices, must be known
in order to find out what is between them, just as the two endpoints of
a line must be given in order to determine the center, Aristotle's
ethics presupposes as self-evident what is to be considered as a vice;
and since a vice implies -- as its opposite -- the virtue, the ethics of
Aristotle amounts to a systematization of the traditional morals of
Greece at his time. This is Aristotle's positive contribution to the
science of ethics.\footnote{Cf. Kelsen, ``Metamorphoses of the Idea of
  Justice'', pp. 190ff.} But there is not a word of this in Voegelin's
analysis of Aristotle's ethics. He only gives a ``brief catalogue'' of
certain virtues, namely, ``love of the \emph{sophon},'' that is, love of
wisdom; ``the variants of the Platonic Eros toward the \emph{kalon} and
the \emph{agathon}'', that is, the desire for the beautiful and the
good; ``the Platonic Dike,'' that is, justice; the ``right
superordination and subordination of the forces in the soul,''; ``and,
above all, ...~ the experience of death, as the cathartic experience of
the soul which purifies conduct by placing it ...~ into the perspective
of death.''\footnote{Voegelin, l.c., p. 65.} Yet the virtues to love
wisdom, the beautiful and the good and justice, and to place conduct
into the perspective of death are insignificant generalities as long as
we do not learn what is the content of wisdom, what is beautiful, good,
just, which conduct should be placed in the perspective of death, and
what is the result of such placing. These are the questions to be
answered by a political science which pretends to be in the possession
of the ``truth.'' What Voegelin has to say in this respect is not an
answer to these questions. He does not convey even ``brief and
incomplete ...~hints'', as he modestly asserts.\footnote{L.c., p. 66.}
He presents nothing but a catalogue of questions.

\label{M63}After having asserted that the truth of the soul according to Plato
can be achieved through the love of wisdom, without telling us anything
about this wisdom, and leaving us completely in the dark about the
content of this truth of the soul, Voegelin increases the mystery by the
statement: ``The discovery of the new truth [which is the truth of the
soul] is not [-- as it could be assumed --] an advancement of
psychological knowledge in the -- immanent sense.'' If psychology, the
science, the object of which is the soul, cannot describe the truth of
the soul, the suspicion arises that it is not at all the soul the truth
of which is in question. But Voegelin is still dealing only with the
soul. The truth of the soul, he says, cannot be discovered by
psychology: ``One would rather have to say that the psyche'' -- for some
unknown reason he calls the soul now by its Greek form -- ``itself is
found as a new center in man at which he experiences himself as open
toward transcendental reality.''\footnote{L.c., p. 67.} An empirical
science like psychology cannot find in the soul an open door through
which ``transcendental reality'' may enter. Such discovery is denied to
any science, but reserved to the metaphysics of mystic philosophers.

``The psyche as the region in which transcendence is experienced must be
differentiated out of a more compact structure of the soul, ...~''
Then comes the tautological statement: ``the openness of the soul is
experienced through the opening of the soul itself''; and, finally, the
admission: ``This opening, which is as much action as it is passion, we
owe to the genius of the mystic philosophers.''\footnote{Ibid.} The new
political science thus enters the nebulous region of mysticism. Since
there are many men -- and \label{M64}among them certainly many political
scientists -- who will confess that they never experienced themselves as
open toward transcendent reality, and since Voegelin is of the opinion
that this experience is essential with respect to that truth toward
which the new political science is to be orientated, he must describe
this experience, must tell us what he knows about the object of that
experience, namely, the transcendental reality. As long as, for some
reason or another, he does not comply with this requirement, he cannot
expect that his statement about the truth of the soul is considered as
anything but a meaningless agglomeration of words. And he has indeed not
the slightest intention to satisfy our curiosity. Instead he adds to the
mysterious ``truth of the soul'' a new, even more mysterious truth,
which evidently is the truth at which the new science is aiming. He
says: ``Through the opening of the soul the philosopher finds himself in
a new relation with God; he not only discovers his own psyche as the
instrument for experiencing transcendence but at the same time discovers
the divinity in its radically nonhuman transcendence. Hence, the
differentiation of the psyche is inseparable from a new truth about
God.''\footnote{Ibid.} If somebody boasts of having found a new truth he
has to show the difference which exists between the new and the old
truth. Voegelin of course is not in a position to show this difference
because he knows as little about the new as he knows about the old
``truth of God.'' If the truth of God is the truth of the new political
science, then this science is theology, and that means the hopeless
attempt to achieve human knowledge of something which, by definition, is
not accessible to human knowledge. Besides, if theology is a science at
all, it is certainly not a ``new'' science, and Voegelin's effort to
establish it is completely superfluous. He has the choice among a
remarkable number of quite different theologies, of which each claims to
have its own ``truth of God.'' And, indeed, the truth of God in Plato's
theology is quite different from the truth of God in the protestant or
the catholic \label{M65}theologies. If Voegelin demands that political science
turn to theology, he has to say precisely which theology he has in mind,
or, what amounts to the same, he has to indicate the God whose truth the
new science of politics should accept. Is it really the God of Plato?
What about the theology of Thomas Aquinas, for whom Voegelin, in the
Introduction to his book, showed almost the same enthusiasm as for
Plato? Catholic theologians -- certainly more competent in this respect
than a professor of political science -- will probably not recognize the
speculation of a heathen philosopher about the impersonal idea of the
agathon as the truth of the God of the Bible.

The result of Voegelin's examination concerning the ``theoretical
truth'' with the aid of which the new science of politics has to fulfill
its task: to challenge the truth of the self-interpretation of society
or -- to formulate this task in the usual way -- to perform a critique of
social ideologies, are the statements that there exists a truth of the
soul and a truth of God, and that the one is connected with the other.
By these statements nothing more is affirmed than that there exists a
God and that the soul is able to achieve some experience of God. Even if
these unproved and unprovable statements are accepted, they are of no
use at all to a political science which -- as Voegelin postulates -- has
to solve the problems of social values, that is to say, has to decide
the questions as to whether democracy or autocracy is the best form of
government, capitalism or communism the best economic organization,
pacifism or imperialism, nationalism or internationalism the best
policy. No value judgment concerning social institutions or political
actions can properly be deduced from these completely empty statements.
It seems that Voegelin anticipated this objection. For he says: ``The
theorist is the representative of a new truth in rivalry with the truth
represented by society. So much is secured. But there seems to be left
the difficulty of the impasse that the new truth has little chance of
becoming socially effective, of forming a society in its
image.''\footnote{L.c., p. 70.}  Indeed, \label{M66}how could a society be
formed in the image of a truth the content of which is unknown? But
Voegelin immediately adds: ``This impasse, in fact, did never exist.''
We may expect that now Voegelin will show how a society can be formed in
the image of the truth of God in connection with the truth of the soul,
in spite of the fact that the content of these truths remains unknown.
In order to avoid any uncertainty about the status of the problem, we
have to remember that when Voegelin refers to the truth of the soul and
the truth of God as signposts for the new science of politics he does
not refer to the constitution of Plato's ideal state or Aristotle's
political suggestions. As pointed out, he does not even mention the
former and takes only Aristotle's ethics of virtues into consideration.
The problem, as he formulates it, is a problem of ``the theorist'' --
that is, political scientist --, as ``the representative of a new
truth,''\footnote{Ibid.} i.e., the truth of the soul and of God, facing
the truth represented by society -- any society whatsoever --; it is the
problem of ``forming society'' -- any society whatsoever, not only Greek
society -- in the image of the truth of the soul and the truth of God,
and not in the image of a concrete political program developed by Plato
or Aristotle or any other of the mystic philosophers or prophets to whom
he attributes the discovery of ``the challenging theoretical truth.''

The way in which Voegelin tries to convince the reader that a society
can be formed in conformity with the empty formulas ``truth of the
soul'' and ``truth of God'' is, indeed, more than astonishing. He refers
to ``the Athens of Marathon and the tragedy'' and says: ``Here, for a
golden hour in history, the miracle had happened of a political society
articulated down to the individual citizen as a representable unit, the
miracle of a generation which individually experienced the
responsibility of representing the truth of the soul and expressed this
experience through the tragedy as a public cult.''\footnote{L.c., p.
  71.} In the battle of Marathon the Athenian Army distinguished itself
by a heroic attitude. This is \label{M67}the historic fact on which Voegelin
bases his interpretation, that the Athenian society represented at that
time the truth of the soul. But until now, he had not told us that the
truth of the soul or the truth of God means the moral norm prescribing
an heroic attitude in war. His statement that at the time of the battle
of Marathon the state of Athens was ``a political society articulated
down to the individual citizen as a representable unit which
individually experienced the responsibility of representing the truth of
the soul'' is without any foundation if it means more than that in the
battle of Marathon the Athenian army exhibited an heroic attitude. No
accumulation of poetical phrases like the ``golden hour in history'' or
the ``miracle of a generation'' can make the critical reader forget that
behind all these words there is nothing but the fact of a heroically
fought battle, as has been fought by many other armies of many other
nations, and that, if valiant fighting is sufficient for a political
science to interpret the society concerned as a representative of the
soul, this interpretation does not add anything essential to the
statement that the army of that society distinguished itself in a heroic
battle. But it is rather significant that Voegelin's truth of the soul
when it assumes some concrete content refers to militaristic ideals,
just as his concept of existential representation refers, if not
exclusively then primarily, to warlike action.

In order to demonstrate that the Athenian society -- at least for a short
time -- represented the truth of the soul, Voegelin presents an analysis
of Aeschylus' tragedy the \emph{Suppliants}. In this drama the highly
religiously-minded poet tried to show that in a conflict between
positive law and the natural or divine law, i.e., the Dike of Zeus, the
latter prevails. As usual in Greek tragedy, the chorus expresses the
moral of the play: ``It is Zeus who brings the end to pass.'' The
purpose of the play is -- as that of many other plays of Aeschylus and
Sophocles -- to strengthen the belief in the religion of the state and
thus authority of the government, using religion as one of its most
effective instruments. This is, from the point of a political science
which \label{M68}considers as its task a critique of ideology, the significance
of this as of many other Greek tragedies. But what is, according to the
new science of politics presented by Voegelin, the political importance
of the \emph{Suppliants}? The fact that the king in the play decides to
apply the divine law is interpreted to mean: ``the royal descent into
the depth of the soul'' and that this ``decision represents the truth of
the God.''\footnote{L.c., p. 73.} But this decision of the king does not
root deeper in his soul than any other decision; and if it
``represents'' anything divine, it does not represent the truth, but the
justice of God, ``the Dike of Zeus,'' as Voegelin himself asserts. This
is exactly the idea expressed in the tragedy; and if, as Voegelin
emphasizes, the ``tragedy was a public cult,'' this interpretation of
the tragedy is also a self-interpretation of society. The new political
science does not at all challenge this self-interpretation, it does not
denounce the tragedy as a political ideology. On the contrary, it takes
over this ideological self-interpretation without any critique.

By referring to the play of Aeschylus, Voegelin wants to show -- as he
expressly declares -- that the tragedy intended to make the people
``understand the Athenian \emph{prostasia} as the organization of a
people under a leader -- in which the leader tries to represent the
Jovian Dike and uses his power of persuasion to create the same state of
the soul in the people on occasion of concrete decisions, while the
people are willing to follow such persuasive leadership into the
representation of truth ...~''\footnote{Ibid.} That means that the
tragedy tried to strengthen the belief of the people in Zeus and the
other gods of the Greek religion and in the authority of their leaders
as the executives of the will of the gods. But if a political science,
as the new science of politics, approves of this ideological function of
the tragedy, and by approving it presents itself as the representative
of a truth of God, it proves to be itself a political ideology. Voegelin
refers to the \emph{Suppliants} of Aeschylus, a pro-religious play, as
representative of the Greek tragedy. But only the tragedies \label{M69}of
Aeschylus and Sophocles have the tendency to support the state religion
as an instrument of the state policy. The plays of the third great poet,
Euripides, assumed a highly critical attitude toward religion.
Euripides, the poet of the Greek enlightenment, is as much a
representative of the Greek tragedy as the conservative Aeschylus. It is
significant that Voegelin ignores completely this aspect of the Greek
tragedy, just as he ignores the intellectual movement of the sophists,
against which Plato's philosophy was the reaction. He mentions only the
\emph{Troades} of Euripides and this only to say that ``the issue is the
mass of filth, abuse, vulgarity and atrocity displayed by the Greeks on
occasion of the fall of Troy.''\footnote{L.c., p. 74.} By referring only
to the issue of this play, Voegelin does not do justice to the great
political importance of the work of Euripides.

The fact that a head of a state, like Darius or a Mongol Khan, pretends
to realize the will of a god is no sufficient reason for a political
science to interpret this state -- as Voegelin does -- as a
``cosmological'' empire representing a ``truth''. It stands to reason
that the battle of Marathon is no proof of the assertion that the
Athenian society, even only temporarily, was the representation of the
truth of the soul and that the Greek tragedy is no proof of the
assertion that the Athenian society was at any time the representation
of the truth of God. But Voegelin seriously affirms to have shown that
``society as a whole proved to represent a transcendent
truth.''\footnote{L.c., p. 76.} A critical analysis of his attempt to
demonstrate this thesis can not help revealing the complete lack of
foundation.

\label{M70}
\subsection{3.}

Under the heading ``the struggle for representation in the Roman
empire,'' Voegelin is dealing in the main, with four quite different
topics: the ``beginnings of a theocratic conception of
rulership''\footnote{L.c., p. 85.}; the contrast between the doctrine of
Varro that the gods were instituted by political society, and the
philosophy of Cicero, backing the traditional view that the auspices of
Romulus and the rites of Numa laid the foundations of the state, that is
to say, the contrast between a religious ideology and an attempt at a
critical theory; the transformation of republican Rome into a
monarchical empire; and the change from the old polytheistic religion of
ancient Rome to Christianity as the state religion of the later empire.
All these problems -- as we shall see -- have nothing to do with the
theory of representation as the relationship of the state to its organs,
and very little connection with one another. 

As an introduction to the discussion of these subjects, Voegelin
displays remarkable knowledge of theological problems, such as
``\emph{philia} [friendship] between God and men'',\footnote{L.c.,
p. 77.} and the revelation of God's grace ``through the incarnation of
the Logos in Christ.''\footnote{L.c., p.  78.} He presents his view
concerning the ``substance of history'' which, as he asserts,
``consists in the experience in which man gains the understanding of
his humanity and together with it the understanding of its limits,''
and which ``entails consequences for a theory of human existence in
society which, under the pressure of a secularized civilization, even
philosophers of rank sometimes hesitate to accept without
reservation.''\footnote{L.c., pp. 78, 79.} He formulates one of these
consequences ``as the principle that a theory of human existence in
society must operate within the medium of experiences which have
differentiated historically.'' And, applying this principle, he
arrives at the conclusion: ``Since the maximum of differentiation was
achieved through Greek philosophy and Christianity, this means
concretely that theory is bound to move within the historical horizon
of \label{M71}classic and Christian experiences.''\footnote{Ibid.} One
can only hope that Voegelin means by ``theory'' only political theory
and thus does not include natural sciences in his prohibition against
going beyond Plato and St.~Thomas Aquinas.

To argue against the highly problematical pronunciamentos of Voegelin's
theology and theological philosophy of history is superfluous, because
they have nothing to do with his treatment of the problems mentioned
above.

In order to show the beginnings of a theocratic conception of rulership,
Voegelin quite arbitrarily picks out a historic event of minor
importance known only to specialists of Byzantine history, but dealt
with in a recently published monograph\footnote{Hendrik Berkhof:
  \emph{Kirche und Kaiser: Eine Untersuchung der Entstehung der
    byzantinischen und der theokratischen Staatsauffassung im vierten
    Jahrhundert}, trans. Gottfried W. Locher (Zollikon-Z�rich,
  1947).}: the affair of the Roman goddess Victoria, in 384. The
restoration of this altar was disputed between pagans and the Christian
party, and in the course of this dispute St.~Ambrose, speaker of the
Christian party, used as an argument against the restoration the
following sentence: ``While all men who are subject to Roman rule serve
(\emph{militare}) you emperors and princes of the earth, you yourselves
serve (\emph{militare}) the omnipotent God and holy
faith.''\footnote{Voegelin, l.c., pp. 84f.} This sentences Voegelin
interprets to mean: ``The truth of Christ cannot be represented by the
\emph{imperium mundi} but only by the service of God.''\footnote{L.c.,
  p. 85.} It is easy to show that this is a misinterpretation. The
sentence in question is a commonplace in Christian political theory. It
expresses nothing but the idea that, just as the citizens are subject to
their ruler and obliged to obey his orders, so the ruler is subject to
God and obliged to obey his orders. It is the ordinary interpretation of
the relation between the ruler and his subjects as a relationship of
superordination and subordination. As men ``serve'' the rulers, the
ruler ``serves'' God; to ``serve'' means to be subjected. There is a
correlation between serving and ruling. The relationship of serving and
ruling is a relationship different from the one which is characterized
as ``representation'', that is, \label{M72}the relationship between organ and
community. The recognition of the ruler as a servant of God is the
Christian justification of his rulership, and has exactly the same
ideological function as the formula of the Mongol Khan who presented
himself as the Son of God, although it is not the service-relationship,
but the father-son relationship which is used for the purpose of
justification. Hence Voegelin is wrong when he says that the formula of
Ambrose is the inversion of the Mongol formula, because the
``formulation of St.~Ambrose does not justify the imperial
monarchy ...~.  It does not speak of any rule at all but of service.''
But it speaks decidedly of the ``Roman rule'', and ``service'' is only
one side of the relationship of which the other is ruling. Voegelin
asserts: ``The appeal of St.~Ambrose does not go to the imperial ruler
but to the Christian who happens to be the incumbent of the
office.''\footnote{Ibid.} This is an open contradiction to the wording
of St.~Ambrose's appeal which is addressed to the ruler of the Roman
empire, to the Christian individual in his capacity as ruler of men.
That this individual as a Christian is subjected to God would be of no
importance at all if this individual were not the emperor and
consequently had the power to forbid the restoration of the Altar of
Victoria.

If the statement that something, especially a community, ``represents''
the truth of Christ has any meaning at all, it can only mean that the
order of the community is in conformity with the prescriptions of the
Christian religion. If the emperor serves the Christian God, the empire
which is under his rule is in conformity with the prescriptions of the
Christian religion. It is the very meaning of St.~Ambrose's appeal that
the \emph{imperium mundi}, that is, the Roman empire, should be in
correspondence with the prescriptions of the Christian religion, or, to
use the terminology of Voegelin, should represent the truth of Christ.
There is nothing in the sentence of St.~Ambrose that could justify
Voegelin's interpretation that ``the truth of Christ cannot be
represented by the \emph{imperium mundi}'';\footnote{Ibid.} and even
less foundation \label{M73}for the interpretation that the truth of Christ can
be represented ``only by the service of God.''\footnote{Ibid.} The
representation to which Voegelin refers when he speaks of representation
of truth, is representation of truth by a community in the sense he
refers to the cosmological empires of the Persians and Mongols as
representatives of truth. It is always a ``society'' which he interprets
as the representative of a transcendent truth.\footnote{Cf. l.c., p.
  76.} How, then, can the service of God represent the truth of Christ?
Only if by the service of God Voegelin means the Christian church, the
Christian priesthood. There is, of course, no reference to the church in
St.~Ambrose's sentence. How, then, can Voegelin interpret the sentence
as the ``beginnings of a theocratic conception of rulership'', since
theocracy means the rule by priesthood? The answer is clear: By using
the term theocracy, well established in political science, to express
something that has nothing to do with it. He says: ``These are the
beginnings of a theocratic conception of rulership in the strict sense,
theocracy not meaning a rule by the priesthood but the recognition by
the ruler of the truth of God.''\footnote{L.c., pp. 85f.} According to
this definition of theocracy a state is a theocracy if the ruler
recognizes the truth of God. The ruler recognizes the truth of God if he
``serves the omnipotent God and holy faith'', as St.~Ambrose bids the
emperor; and if the emperor does serve the Christian God and the
Christian faith, the truth of Christ is indeed represented by the
service of God, performed by the emperor representing the
\emph{imperium} and, hence, by the \emph{imperium}. If the truth of
Christ is not represented by the church, it can be represented only by
that society which is called the state, especially when the service of
God is to be performed by the head of the state. By his new definition
of theocracy, Voegelin disowns his interpretation of St.~Ambrose's
sentence.

To date the beginnings of a theocratic conception of rulership -- meaning
the recognition by the ruler of the truth of God -- with the
insignificant statement made by a saint, but not exceptionally learned
theologian, on occasion of a not very important event in Byzantine
history, is strange enough. But even stranger is the fact that
\label{M74}Voegelin sees the theocratic conception of rulership ``unfolded
fully'' in the statement made by St.~Augustine in his \emph{Civitas
  Dei}: ``the true happiness of the emperor can be measured only by his
conduct as a Christian on the throne.''\footnote{L.c., p. 86.} This
statement is evidently an application of Plato's thesis that only a just
man is happy. For St.~Augustine identifies justice with Christianity,
which identification is one of the characteristic features of the
political theory developed in the \emph{Civitas Dei}. St.~Augustine's
statement has as little to do with a theocratic concept of rulership as
the sentence of St.~Ambrose.

The second topic treated in the chapter on the struggle for
representation in the Roman Empire is the contrast between Varro's and
Cicero's attitudes toward the Roman religion. According to Augustine,
Varro advocated the view that ``the gods were instituted by political
society''\footnote{L.c., p. 88.}: which, naturally, Augustine could not
accept. Varro's conception is similar to a doctrine advocated by the
sophists and rejected by Plato because of its effect of undermining the
authority of the established religion, the maintenance of which was an
important concern of Plato's political theory. The same tendency
prevails in Cicero's philosophy. Cicero accepted the official opinion
that the auspices of Romulus and the rites of Numa laid the foundations
of the state (in the words of Voegelin).\footnote{L.c., p. 89.} The
opposition between the Varronic and the Ciceronian conception is the old
conflict between a rationalistic critical and an anti-rationalistic
dogmatic, hence, politically conservative attitude toward religion as an
ideology of the state. Voegelin asserts: ``The Varronic and Ciceronian
expositions are precious documents for the theorist.''\footnote{Ibid.}
The theorist can find in the conflict between Plato and the sophists
much more material concerning the opposition between a pro-religious and
an anti-religious political theory. But since Voegelin is interested
here in Roman history, the antagonism between Roman authors may be of
more importance than a conflict in the intellectual history of Greece.
\label{M75}Granted this, why does Voegelin consider the conflict so
``precious'' for the theorist? Because he wants to show that Cicero's
view is correct, and Varro's view incorrect. He does not agree with the
``conventional treatment of Cicero'', which is not very favorable to
this trivial philosopher. This treatment, says Voegelin, coming to the
assistance of the defender of the traditional religion, ``is apt to
overlook that in his [Cicero's] work something considerably more
interesting is to be found,'' which -- according to Voegelin is evidently
important enough to make us revise our opinion of Cicero. This is ``the
archaic experience of social order before its dissolution through the
experience of the mystic philosophers.''\footnote{Ibid.} ``The archaic
experience of social order'' is an obscuring phrase referring to the
fact that in early times the social order was considered to be of divine
origin. It is an ``experience'' which has by no means been dissolved
but, on the contrary, confirmed by the ``experience of the mystic
philosophers,'' whose merit is according to Voegelin that they
discovered the truth of God. What else is the archaic experience of
social order but the belief that a society represents the truth of God,
of course the truth of their own national God or gods. But this
inconsistency is irrelevant in this connection. What counts is that the
new science of politics confirms Cicero in his opposition to Varro:
``Romans like Cicero understood the problem quite well.''\footnote{L.c.,
  p. 90.} Which problem? The question as to whether gods are the
creation of society or society the creation of gods? And Cicero
understood this problem quite well by advocating the second view?

The only point in which Cicero's philosophy does not get the unreserved
approval of the new science of politics is his refusal to follow Greek
philosophy in matters of religion. Voegelin explains this attitude as
follows: ``Rome was the Rome of its gods into every detail of daily
routine; to participate experientially in the spiritual revolution of
philosophy would have implied the recognition that the Rome of the
ancestors was finished and that a new order was in the making into which
the Romans would have to merge -- as the Greeks had to merge, whether
they liked it \label{M76}or not, into the imperial constructions of Alexander
and the Diadochi and finally of Rome.''\footnote{L.c., p. 91.} By
``revolution of philosophy'' Voegelin means in the first place the
metaphysics of Plato and Aristotle. But both metaphysicians were most
anxious not to make dubious by their speculations the authority of the
established religion of the state. For Plato and Aristotle Athens was
the Athens of its gods, just as for Cicero Rome was the Rome of its
gods. To participate in this ``revolution of philosophy'' would
certainly not have the consequence of recognizing that ``the Rome of the
ancestors was finished,'' that is to say, that the traditional religion
of Rome could not be maintained. The metaphysics of Plato and Aristotle
was quite compatible with this religion, as it was compatible with the --
not very different -- religion of Athens. There may be indeed a certain
relationship between the transition of a city-state to a world empire
and a change of religious ideology. But the non-participation in the
``revolution of philosophy'' could not prevent or retard such change.

In this connection Voegelin makes an interesting remark. When Rome
merged into the empire, he says, and ``the struggle between the various
types of alternative truth, among philosophies, oriental cults, and
Christianity'' entered into a crucial phase, ``the existential
representative, the emperor, had to decide which transcendental truth he
would represent now that the myth of Rome had lost its ordering force.
For a Cicero such problems did not exist ...~.''\footnote{Ibid.}  But
it seems that the problem of deciding which of the different
``transcendental truths'' proclaimed by the various religions of our
time the modern state has to represent according to the teaching of
the new science of politics, does not exist for this science either,
which boasts of having discovered that society in general and the state
in particular represents a transcendental truth, but does not bother
with the question which of the many transcendental truths is the right
one. Or is perhaps one of these transcendental truths as good as another
for the new science of politics? This can hardly be the case, for, then,
the new science of politics would be guilty of positivistic, and hence
destructive, relativism.

\label{M77}The third problem Voegelin examines under the heading ``The struggle
for representation in the Roman empire'' is the question how the
republican city-state could develop into a monarchical world power. This
is primarily a question of the change of form of government in its
connection with the territorial extension of state power. It is only
secondarily a question of the change of form of representation, insofar
as the relationship between the organs of the state which are considered
to be the representatives of the state and the mass of the subjects is
constitutionally different in a republic from that in a monarchy.
Voegelin formulates the problem as ``the question how the institutions
of republican Rome ...~could be adapted in such a manner that an
emperor would emerge from them as the existential representative of the
Mediterranean \emph{orbis terrarum}.''\footnote{L.c., p. 92.} Voegelin's
analysis contains nothing essential or new that could be considered as a
contribution to the theory of representation. Its main purpose is to
show that the glorious rise of republican Rome to a world power is due
to the \emph{Fuehrer}-principle. Following a recently published
monograph, \emph{Vom Werden und Wesen des Prinzipats},\footnote{Anton
  von Premerstein, \emph{Vom Werden und Wesen des Prinzipats}, ed. Hans
  Volkmann (``Abhandlung der Bayrischen Akademie d. Wissenschaften,
  Phil.-hist. Abt., Neue Folge,'' Heft 15, Munich 1937).} Voegelin
points out that the principal institution which developed into the
imperial office was that of the \emph{princeps civis} or \emph{princeps
  civitatis}, which he translates: ``the social and political
leader.''\footnote{Voegelin, l.c., p. 92.} ``At the core of the
institution was the patronate, a relationship created through the fact
of various favors ...~'', the relationship between the patron and his
clients. ``The emergence of the principate, thus, may be described as an
evolution of the patronate.''\footnote{L.c., p. 95.} The result of
this evolution is summarized as follows: ``The patrocinial
articulation of a group into leader and followers had expanded into the
form of imperial representation.''\footnote{L.c., p. 97.} Whether the
doctrine that the principate had its origin in the institution of the
patronate and, hence, whether the articulation was patrocinial, is here
of no importance. Decisive is only that the new science of politics
ascribes the emergence of Rome as a world power to the articulation into
leader and followers. In view of this it is not \label{M78}superfluous to
note that the phrase ``articulation into leader and followers'' is the
literal translation of the German \emph{Gliederung in Fuehrer und
  Gefolgschaft}, the fundamental concept of the political ideology of
National Socialism culminating in the \emph{Fuehrer-Prinzip}. That Rome
on its way to a world power turned from a republic to a monarchy is a
well known fact, what is new is only the description of this fact in
terms of Nazi ideology.

It stands to reason that a monotheistic religion is a more appropriate
ideology for a monarchical form of government than a polytheistic
religion. Hence, the movement in imperial Rome toward the belief in a
supreme God and, finally, the acceptance of Christianity as the most
outspoken monotheistic religion, does not offer a particularly
difficult problem. It is another topic dealt with in the chapter on
representation in the Roman empire. Following, in this respect, the
Dutch theologian Hendrik Berkhof\footnote{\label{change2} Cf. Berkhof, l.c.,
pp. 41ff.}, Voegelin ascribes the ``surprising turn of events which in
311-13 gave freedom to Christianity''\footnote{Voegelin, l.c., p. 99.}
to the fear that the powerful God of the Christians might take revenge
by making trouble for the rulers if they prevented his worship. This
may be true; but the God of the Christians would have not been
considered as powerful if the number of his followers had not
remarkably increased among the subjects of the empire, and if the
Christian religion had not been recognized as an ideology useful to
the government. To be sure, the attitude of the Christian movement in
its beginnings was not very much in favor of the state in general and
of the Roman empire in particular. The original teaching of Christ had
certainly an anarchistic -- and, in this sense, revolutionary --
tendency. But it was against this tendency that St.~Paul's spiritual
efforts were directed, of which his letter to the Romans is the most
striking symptom. In the fourth century A.D., St.~Paul's doctrine that
all state authority \label{M79}originates in God was an accepted element of
Christian religion; and such a religion could be used as an ideology
of the Roman empire. Its incompatibility with polytheism proved to be
irrelevant within a civilization characterized by an almost unlimited
tolerance in the field of religion. Hence it is hardly possible to
consider -- as Voegelin does -- the anti-polytheistic tendency of
Christianity as a ``revolutionary substance.''\footnote{L.c., p. 100.}
This revolutionary substance consists according to Voegelin in the
fact that by rejecting polytheism Christianity de-divinized the world.
``What made Christianity so dangerous was its uncompromising, radical
de-divinization of the world.''\footnote{L.c., p. 100.} Voegelin
approves Celsus' critique of Christianity: ``He understood the
existential problem of polytheism; and he knew that the Christian
de-divinization of the world spelled the end of a civilizational epoch
...~''.\footnote{L.c., p. 101.} But how is such revolutionary
de-divinization, manifest already in the 2nd century, compatible with
the fact, referred to by Voegelin, that the Christian theologian
Eusebius, in the forth century, praised the emperor Constantine
``because in his imperial he had imitated the divine monarchy''; and
that this Christian theologian taught that ``the one \emph{basileus}
on earth represents the one God, the one King in Heaven, the one Nomos
and Logos,'' and saw in the \emph{pax Romana} the fulfillment of
eschatological prophecies?\footnote{L.c., p. 104.}  That means that
Eusebius recognized the Roman empire as a divine monarchy, the
realization of God's will on earth, which is just the contrary of a
de-divinization of the world. Voegelin thinks that the doctrine of the
three divine personalities made it impossible to maintain the idea of
the emperor representing the triune godhead. He refers to Gregory of
Nazianzus' view that the Christians ``do not believe in the monarchy
of a single person in the godhead, for such a godhead would be a
source of discord; Christians believe in the triunity -- and this
triunity of God has no analogue in creation.'' But Voegelin must admit
that Gregory ``declared the Christians to believe in the divine
monarchy.''\footnote{L.c., p. 105.} Hence the belief in the three
divine personalities -- not so different from a polytheistic religion
-- had not at all the effect of de-divining the world of the Roman
\label{M80}empire, which, in spite of this belief, was recognized as a divine
institution.  Voegelin refers also to St.~Augustine, who did not share
Eusebius' recognition of the \emph{pax Romana} as the fulfillment of
eschatological prophecies of eternal peace, but, on the contrary
pointed to the fact that within the Roman empire wars were going on.
But Voegelin cannot ignore that Augustine did not exclude the
possibility of the prophecy being fulfilled in the future; he quotes
Augustine's statement; ``but perhaps, we hope, 'it will be
fulfilled'.''\footnote{L.c., p. 106.} If this sentence is to be
understood in connection with Augustine's re-interpretation of the
Messianic idea, his doctrine that the \emph{kingdom} of God will not
be realized in this but in another world, this doctrine does not mean
that the will of God cannot be realized in this world. For Augustine's
teaching was not and could not be, in opposition to Jesus' prayer
``Thy kingdom come; Thy will be done on earth as it is in heaven'',
which is not addressed to a God whose existence is restricted to
another world.  Augustine's theology, as any Christian theology, does
not and cannot furnish evidence of an ``uncompromising, radical
de-divinization of the world'' by Christianity. There is, especially,
no reason for Voegelin's assumption that Augustine's counterposition
``is the end of political theology in orthodox Christianity. The
spiritual destiny of man in the Christian sense cannot be represented
on earth by the power organization of a political society; It can be
represented only by the church. The sphere of power is radically
de-divinized; it has become temporal.''\footnote{L.c., p. 106.} There
was never a radical de-divinization of the sphere of power, for never
St.~Paul's teaching has been abandoned that ``there is no governing
authority except from God, and those that exist have been instituted
by God.''\footnote{Roman, 13.} It is just the divinization of
governmental power which is a specific achievement of Paulinian
Christianity. ``The double representation of man in society through
church and empire'' -- as Voegelin characterizes the dualistic
organization of society during the Middle Ages -- has nothing to do
with a de-divinization of a society of which the church was an
essential part and the empire a divine institution.

How impossible Voegelin's doctrine is that Christianity means
de-divinization manifests itself in the fact that he is very ambiguous
with respect to the object of this de-divinization. He speaks one time
of a ``de-divinization of the world'',\footnote{Voegelin, l.c., p. 100.}
\label{M81}another time of a ``re-divinization of society''\footnote{L.c., p.
  106.}, and occasionally he restricts his doctrine to the assertion of
a ``de-divinization of the temporal sphere of power''\footnote{L.c., p.
  107.}. But that this doctrine, even in this restricted version, is
untenable, becomes evident in Voegelin's final definition. He says: ``by
de-divinization shall be meant the historical process in which the
culture of polytheism dies from experiential atrophy, and human
existence in society became reordered through the experience of man's
destination, by the grace of the world-transcendent God, toward eternal
life in beatific vision.''\footnote{L.c., p. 107.} It can only be
Christianity by which ``human existence in society became reordered'',
and only human existence ``in society'' can be the object of this
reordering. And if this reordering consisted in making man aware of his
destination, by the grace of God, toward eternal life, then society was
not de-divinized, but, on the contrary, radically divinized by
Christianity. Voegelin's definition of de-divinization says the contrary
of what the term to be defined says: the emancipation of man in society
from the divine in general and from religion in particular. If there is
such a thing as divinization of the world, it is the view that the world
is created and governed by the almighty God and that consequently
nothing can happen in nature or history without or against his will,
that nature as well as society exists only through and with Him; which
view is the essence of Christian religion in all its varieties. There is
only one point in Voegelin's second definition of de-divinization that
seems to justify it: the reference to the transcendence of God.

\label{M82}However, the transcendence of God has never been
interpreted as incompatible with God's immanence. That the almighty and
absolutely just creator and ruler of the world is at the same time
transcendent to and immanent \emph{in} the world, is one of the most
essential elements of any kind of Christian metaphysics. To infer from
God's transcendence that there is no God in the world amounts to a
theology of atheism. And just as in Christian theology the transcendence
of God is necessarily combined with its immanence, thus the immanence of
God -- which becomes most intensive in the mystic experience of the union
with God, the so-called \emph{unio-mystica} -- was never interpreted in
the speculations of the mystics as a negation of the transcendence of
God. It is precisely the belief in a transcendent God which is the
basis of the mystic experience, created by the passionate desire to
bring God into the individual existence of man. That there is a logical
contradiction between the transcendence and the immanence of God is, of
course, no objection to irrational metaphysical, theological or mystic
speculation. It is true that the existence of evil induced some
theological speculation to the admission of a non-divine element in the
world and to the construction of a kind of counter-God in the person of
Satan or the Anti-Christ. But even the most radical dualism of a
theological good-evil speculation could not lead to a de-divinization of
the world. Just as the presence of God could not exclude the presence of
the counter-God, the presence of the latter could not exclude the
presence of the former. If there is such a thing as de-divinization, it
is the tendency to interpret nature and society without referring to an
unknown transcendental entity which, by definition, is inaccessible to
human cognition. This is the essence of that anti-religious,
anti-metaphysical \label{M83}science which Voegelin condemns as destructive
positivism. If there is a characteristic by which so-called modernity
can be distinguished from earlier periods of history, it is the
development of this science which, with its tendency of emancipating the
explanation of the world from religion, has de-divinized the world. In
the introduction to his book, Voegelin says that for Max Weber ``the
evolution of mankind toward the rationality of positive science'',
postulated by Comte, was ``a process of disenchantment
(\emph{Entzauberung}) and de-divinization
(\emph{Entg�ttlichung}).''\footnote{L.c., p. 22.}  Voegelin does not
reject this definition, which indeed is the only possible one of this
term. But, now, Voegelin uses it to designate just the opposite of an
evolution toward rationality. For he embarks on the more than
paradoxical enterprise to show that the nature of modernity is the
re-divinization of a world allegedly de-divinized by Christianity. It
seems that Voegelin has completely forgotten what he has said about the
de-divinization of the world through Comte's positivistic philosophy.
For among those who, according to Voegelin, are responsible for the
re-divinization of society in modern times he mentions in the first
place: Comte.\footnote{L.c., p. 124.}

\label{M83a}
\section{Gnosticism a new category of political science}

\subsection{1.}

In the history of religion by the term gnosticism a religious movement
is understood which flourished during the 2nd and 3rd centuries, and was
finally replaced by Manichaeism. Its characteristic feature is a strong
tendency toward mysticism, which manifests itself by the conviction that
the initiates possess a secret and strictly esoteric knowledge based on
a mysterious revelation derived from Jesus himself. Gnosticism is one of
the many mystic religions that came into existence at the end of
-antiquity and, like all of them, it aims at individual salvation by
means of certain rules, secret formulas, names and symbols, which have a
more or less magic character. That gnosticism could be replaced by
Manichaeism is due to the fact that both are fundamentally dualistic.
Gnostic speculation, just as the religion of Mani, refers to the
opposition between the two realms of good and evil, light and darkness,
\label{M84}the spiritual and the material world.\footnote{Cf. the article
  ``Gnosticism'' in the \emph{Encyclopedia Britannica}, Vol. X. (1945),
  pp. 462ff.} It is in principle the same dualism as that of the
spiritual and the temporal sphere, accepted in the doctrine of the
medieval church, with the difference that in the gnostic-Manichaean
speculation there was a strong tendency toward identifying the material
sphere of human life so completely with the realm of evil, and of
opposing it so radically to the spiritual as the only divine sphere,
that the former could be considered somehow as de-divinized. If that
movement is in question which in the history of religion is generally
called gnosticism, its aim is just the contrary to the one to which
Voegelin attributes what he calls gnosticism. Historic gnosticism did
not divinize but rather de-divinize -- of course not the world or
society, but -- the material sphere of human life.

Although Voegelin devotes a great part of his study to the allegedly
decisive influence of gnosticism on modern civilization, he is very
vague concerning the meaning of this term as used by him. He gives
nowhere a clear definition or precise characterization of that spiritual
movement which he calls gnosticism. He does not refer to Corinthus,
Carpocrates, Basilides, Valentinus, Bardesanes, Marcion, or any other
leader of the gnostic sects, all belonging to the first centuries of the
Christian era. He designates as ``the first clear and comprehensive
expression of the idea''\footnote{Voegelin, l.c., p. 110.}, which he
considers as the essential function of gnosticism, namely, the
re-divinization of society, the work of Joachim of Flora, a monk and
mystic theologian who lived during the second part of the 12th century
(1145-1202), about a thousand years after historic gnosticism
flourished. Voegelin does not offer any literal quotation of the work to
which he attributes the decisive influence on the formation of
modernity. He contends himself to give the titles of some recently
published monographs dealing with this author and a general
characterization of his main idea. What he reports as the main
contribution of \label{M85}Joachim of Flora has nothing specifically gnostic in
it, provided this term means what is usually understood by it in the
history of philosophy and religion. Voegelin refers to Joachim's
division of the history of mankind in three periods, corresponding to
the three persons forming the one God according to the doctrine of
Christian theology: an age of the Father, the leader of which is
Abraham; an age of the Son, the leader of which is Christ; and a third
age, the age of the future, which, Joachim predicted, will be
inaugurated precisely in the year 1260 by a mysterious personality, the
\emph{Dux e Babylone}, a free invention of Joachim. The first age,
although it was the age of the God Father himself, was the lowest in
rank for it was only the age of the lay man; the second age is the age
of the priest; but the third, the most perfect age will be the age of
the monk. During this age of monachism the entire world will become a
vast monastery and mankind will wholly be directed toward ecstasy, a
feature that Voegelin does not mention. This is ``the clear and
comprehensive expression of the idea'' which, born in the mind of an
eccentric monk whose writings were condemned by the Church in the 13th
century (1260, at the Council of Arles), has formed modernity,
because it aims at a re-divinization of the world.  ``In his trinitarian
eschatology'', says Voegelin, ``Joachim created the aggregate of symbols
which govern the self-interpretation of modern political society to this
day.''\footnote{L.c., p. 111.} But why does Voegelin call Joachim's
theology of history ``gnosticism''? The reader will find no direct and
explicit answer to this question. He may only guess that it is implied
in the following statement: ``In order to lend validity and
cognition to the idea of a final Third Realm, the course of
history as an \label{M86}intelligible, meaningful whole must be assumed 
accessible to human knowledge, either through a direct revelation or
through speculative gnosis.''\footnote{L.c., p. 112.}  Joachim's
speculation is ``gnosis'' because Joachim conceives of the course of
history as an intelligible, meaningful whole. Why the view that
history has a meaning is possible only on the basis of ``a direct
revelation'' or ``speculative gnosis'', is not understandable. This view
has been advocated by extremely rationalistic, anti-metaphysical and
anti-religious, positivist thinkers, who did not refer to revelation or
gnosis; and were far from any attempt to divinize history, because they
intended just the contrary, namely, to de-divinize history, Comte and
Marx. And the opposite view: that history has no meaning at all, too,
has been advocated in modern civilization. But Voegelin asserts:
``Hence, the Gnostic prophet or, in the later stages of secularization,
the Gnostic intellectual becomes an appurtenance of modern civilization.
Joachim himself is the first instance of the species.''\footnote{Ibid.}

Without making an attempt at showing that the Humanists and
Encyclopedists were influenced by the work of Joachim, Voegelin simply
affirms -- what before him already Jacob Taubes in his
\emph{Abendl�ndische Eschatologie} has affirmed\footnote{Jacob Taubes,
  \emph{Abendl�ndische Eschatologie}, Bern, 1947 (quoted by Voegelin,
  pp. 108, 111), p. 81: ``The scheme: Antiquity -- Middle Ages -- Modern
  Times, is nothing else but a secularization of Joachim's prophecy of
  the three ages.''} -- that in their periodization
of history into ancient, medieval and modern history, the three ages of
Joachim are recognizable.\footnote{Voegelin, l.c., p. 111.} In another
connection he says that the ``conception of a modern age succeeding the
Middle Ages is itself one of the symbols created by the Gnostic
movement.''\footnote{L.c., p. 133.} But when he starts his
interpretation of modern age as gnostic revolution, and the Reformation,
with which the modern age begins, as a ``revolutionary eruption of
the Gnostic movements'', he declares that the problem when a modern
period of history begins, ``cannot be solved on the level of Gnostic
symbolism.''\footnote{L.c., p. 134.} That means that the gnostic symbol
of history as a sequence of three ages and the humanist and
encyclopedist periodization of history into ancient, medieval, and
modern history have nothing else in common but \label{M87}the division of a
whole into three parts, which is a general scheme of articulation or
systematization, as old as human thinking. To recognize Joachim's three
ages in our periodization of history is as justified as to recognize the
mystic trinity in the distinction between childhood, manhood, and old
age. Our completely rationalized periodization of history can have
nothing to do with Joachim's trinitarian speculation projected into
history, not only because there is no provable connection between the
two, but because their meaning is totally different. The three stages of
Joachim represent an order of rank, the third stage being understood as
a definite stage of perfection. Its purpose is evidently the
glorification of monachism, and not at all a scientific analysis of
history. Our periodization of history into three stages has never been
understood as a definite articulation; only as a systematization from
the point of view of our present knowledge of history, which may be
replaced at any time and especially in the future on the basis of a more
extensive and profounder knowledge. The concept of the third stage:
modern times, is far from implying the idea of perfection and compatible
with any value judgment whatsoever. It certainly does not convey the
idea of a definite status of human civilization, not capable of further
evolution.


As Dr.~Faustus in Goethe's famous play, after having drunk the magic
potion sees Helene in every woman, Voegelin sees Joachim's trinitarian
eschatology whenever he finds a partition into three periods, in
``Turgot's and Comte's theory of a sequence of theological,
metaphysical, and scientific phases; in Hegel's dialectic of the three
stages of freedom and self-reflective Spiritual fulfillment'' and, above
all, in ``the Marxian dialectic of the three stages of primitive
communism, class society, and final communism''\footnote{L.c., pp.
  111/2.} \label{M88}Joachim's age of the monk, he seriously contends, ``has
become a formidable component in the contemporary democratic creed, and
it is the dynamic core in the Marxian mysticism of the realm of freedom
and the withering away of the state.''\footnote{L.c., p. 113.} It is not
worth while to deal with the fantastic and in no way specified view that
a prophecy made at the end of the 12th century to the effect that in
1260 an age of monachism under the leadership of a duke of Babylon -- the
product of the imagination of a mystic -- has anything to do with the
belief that democracy, that is, a government on which the governed
subjects have direct or indirect influence, is a good government. But it
is perhaps not quite superfluous to analyze Voegelin's -- no less
fantastic -- interpretation of Marxism as gnosticism. For this
interpretation plays a decisive part in the justification of his thesis
that gnosticism is the very nature of modernity.

It is true that there exists a certain similarity -- frequently pointed
out\footnote{For instance, by Fritz Gerlich, \emph{Kommunismus als Lehre
    vom tausendj�rigem Reich}, 1920.  Cf. also Taubes, l.c., p. 136.}
between the Marxian interpretation of history as sequence of a happy
status of mankind during the period of primitive communism, followed by
the unhappy period of society split into classes, and a stage of
happiness in the communist society of the future, on the one hand, and a
certain religious scheme on the other hand. But this religious scheme
has nothing to do with gnosticism, nor with Joachim's trinity
speculation. It is the messianic belief in the existence of a paradise
at the beginning of time, which has been lost by the fall of man, but
which will return with the kingdom of God predicted by the prophets.
Although the similarity is \emph{prima facie} striking, and although
Marx might have been unconsciously influenced by messianic ideas, it is
nevertheless not more than a surface analogy. First of all, because the
correspondence between the paradise of the past and that of the future --
essential for the \label{M89}messianic scheme -- is of secondary importance in
the Marxian construction. As a matter of fact, it is only Engels, not
Marx himself, who is responsible for the doctrine that communism was the
original stage of mankind and that this stage was one of perfect
freedom, because a stateless and lawless anarchy. Engels accepted this
doctrine probably only for the purpose of showing that a stateless and
lawless communist society, predicted by the economic interpretation of
society, was not a utopian imagination but has already existed in the
history of mankind. The communist society of the future is not -- as the
kingdom of God is the return of the first paradise -- the
re-establishment of early communism; it is not a technically primitive,
but a highly developed social organization. And, above all, the
prediction of a state- and lawless communist society of the future is
the result of a rationalistic, anti-metaphysical, critical analysis of
social reality. In this respect Marx' philosophy of history is just the
contrary of the messianic belief in a paradise, lost as a punishment
inflicted by God and to be regained by the grace of God. It is an
essential feature of revolutionary Marxism that the paradise of the
future will be the work of man, in a world completely de-divinized by
the most radical and most reckless critique of religion ever undertaken.
There is nothing mystical in this social philosophy; and to speak of
``Marxian mysticism'', the supposed intention of which is a
re-divinization of society, is to fly in the face of historical
truth.\footnote{The situation is different with respect to Hegel's
  theology of history to which Marx expressly opposed his
  economic-materialistic interpretation. Taubes (l.c., pp. 90ff.)
  shows that there is indeed much more than a surface similarity between
  Joachim's trinitarian eschatology and Hegel's dialectic of history by
  referring to Hegel's view that the divine trinity is the essence of
  the history of the world.}

By his visionary prediction of an imaginary duke of Babylon, Joachim --
according to Voegelin's interpretation, has created the symbol of the
leader, although the age of the monk of which this duke is supposed to
be the ``leader'' represents the symbol ``of the brotherhood of
autonomous persons,''\footnote{Voegelin, l.c., p. 112.} which is
incompatible with leadership. Hence it is not too astonishing that
Voegelin can discern the \label{M90}duke of Babylon, the leader of the age of
the monk, in Machiavelli's \emph{principe} and ``in the supermen of
Condorcet, Comte, and Marx.''\footnote{Ibid.} The symbol of the leader
is one of the oldest elements of social consciousness of man and did not
have to wait for its creation by Joachim in the 12th century. If the
nebulous duke of Babylon was imagined by Joachim as a leader at all, he
was the leader in the sense of a patron saint of an age, the age of the
monk, like Abraham was the patron saint of the age of the layman, and
Jesus Christ the patron saint of the age of the priest. The patron saint
of an age is something totally different from the \emph{principe} of
Machiavelli, the head of a small state. The ``supermen'' of Condorcet,
Comte and especially Marx are visionary creations of Voegelin, no less
fantastic than Joachim's \emph{Dux e Babylone}.

Another symbol created by Joachim in the 12th century -- though an
essential element of the Jewish religion long before Jesus Christ -- is,
we learn from Voegelin, ``that of the prophet.''\footnote{Ibid.} It is
``sometimes blending into'' that of the leader. Hence it seems that
Voegelin interprets the Joachitic vision of the three leaders to mean
that these leaders are at the same time prophets. The personality of a
prophet-leader is not an invention of Joachim. Long before him, and
certainly well known by him, Mohammed entered the history of the
world.

According to Voegelin, not only the most outstanding philosophers of
modern times but also political movements, as e.g. National Socialism,
can be understood only as manifestations of gnosticism or -- what
seems to Voegelin to be the same -- as Joachitic mysticism. He says:
``Hitler's millenial prophecy authentically derives from Joachitic
speculation ...~''\footnote{L.c., p. 113.} The idea of a realm of a
thousand years has by no means its origin in Joachim's prophecy
of an age of monachism; and even if it were possible to prove that
Hitler, or those who furnished his political ideology, have taken
over from Joachitic speculation the propaganda phrase of the
\emph{Dritte Reich} that will last a thousand years -- which supposition
Voegelin does not prove at all -- ``Hitler's millennial prophecy'' has
\label{M91}turned out to be a tragicomical joke, a political slogan that even
the Nazi ideologists did not take seriously. But this is one of
Voegelin's two examples of a modern political society whose
self-interpretation is governed by the symbols created by Joachim's
trinitarian eschatology. It seems, however, that Voegelin is not quite
sure about Nazism as a gnostic-Joachitic movement. For later on he
characterizes ``the \emph{Dritte Reich} of the National Socialist
movement'' as a merely ``nationalist, accidental touch ...~due to the
fact that the symbol of the \emph{Dritte Reich} did not stem from the
speculative effort of a philosopher of rank but rather from dubious
literary transfers''.\footnote{Ibid.} But just a few lines before we are
taught that this symbol ``authentically'' derives from Joachitic
speculation.'' Now we learn -- what we already knew -- that the ``National
Socialist propagandists picked it up from Moeller van den Bruck's tract
of that Name.'' Moeller van den Bruck, who was not a Nazi, found the
formula in the course of his work on a German edition of Dostoevski,
who, as a fervent Russian nationalist, had accepted the ideology of
Russian imperialism: that Russia was the successor of the
Roman-Byzantine empire and as such the Third Rome.

This latter ideology is Voegelin's second example of a
self-interpretation of a modern political society governed by the
symbols created by Joachim's trinitarian eschatology. It is one of the
arguments for his doctrine that gnosticism is the nature of modernity
and that the re-divnization of the world is the essential function of
gnosticism. ``The third Rome'', Voegelin asserts, ``is characterized by
the same blend of an eschatology of the spiritual realm with its
realization by a political society as the National Socialist idea of the
\emph{Dritte Reich}.''\footnote{L.c., p. 114.} But he says that the
Third Rome is an ``other branch of political re-divinization'' and
emphasizes that ``Russia developed a type \emph{sui generis} of
re-presentation, in both the transcendental and the existential
respects.''\footnote{L.c., p. 116.} Nevertheless, he speaks of a
``blending'' of ``later variants'' of the Joachitic symbols \label{M92}``with
the political apocalypse of the Third Rome''\footnote{L.c., p. 117.} and
thus vaguely hints at some connection between the political ideology of
Russian imperialism and Joachim's mystic theology. But the only document
to which he refers as a source of the Moscovite formula of the Third
Rome shows not the slightest symptom that could allow the conjecture
that it has been influenced by the mystic speculation of the Italian
monk. Voegelin's analysis of ``the political apocalypse of the Third
Rome'' contains nothing that would make such a conjecture plausible. If,
by the way, Moeller is the source of the slogan of the \emph{Dritte
  Reich} and the source of Moeller is Dostoevski, then the Nazi ideology
does not derive ``authentically'' from Joachim's speculation but from
the Russian ideology of the Third Rome.

``After the fall of Constantinople to the Turks,'' says Voegelin ``the
idea of Moscow as the successor to the Orthodox empire gained ground in
Russian clerical circles.'' Then he quotes a letter of a Russian
theologian to Ivan the Great. The decisive passage runs as follows:
``Know you, ...~O pious Tsar that all empires of the orthodox
Christians have converged into thine own. You are the sole autocrat of
the universe, the only tsar of all Christians ...~According to the
prophetic books all Christian empires have an end and will converge into
one empire, that of our gossudar, that is, into the Empire of Russia.
Two Romes have fallen, but the third will last, and there will not be a
fourth one.''\footnote{L.c., pp. 114f.} A submissive servant of an
autocrat expresses in a way for which the term ``Byzantinism'' has been
coined, the opinion that his master has the right to subjugate to his
rule all other countries and that his rule will last forever. In order
to justify the imperialistic policy of the ``autocrat of the Universe,''
he furnishes a religious ideology; which is his professional function.
There is nothing mystic in this manifestation of theologian servility.
The idea of the Third Rome as the never ending rule of the Russian
gossudars is based on ``the prophetic books,'' which, \label{M93}if
indicating a definite source, can mean only that part of the Holy
Scripture which is called the Books of the Prophets, and on no gnostic
or other mystic source whatsoever.

Voegelin's attempt to use the Russian Caesaro-papism as an argument for the
political re-divinization of the world, as the essential meaning of
modernity, is particularly unfortunate. For, in contradistinction to the
doctrine prevailing in the West that the Pope as the head of the Church
is, if not superior to, at least independent of the emperor as the head
of the state, the Russian Caesaro-papism, ``with its tendency toward
transforming the church into a civil institution,''\footnote{L.c.,
  p. 159.} means that the emperor as the head of the state,
representing the temporal sphere, is at the same time the head of the
church, representing the spiritual sphere; which is much nearer to a
\emph{de}- than to a \emph{re}-divinization of society. 

\subsection{2.}

The new science of politics does not restrict itself to the bold
assertion that the Joachitic eschatology has positively affected modern
politics and that Western political societies interpret the meaning of
their existence through symbols produced by this eschatology; it
undertakes also a ``critical analysis of its principal
aspects.''\footnote{L.c., p. 118.} But what it criticizes is not the
principal  aspects of the specific form this eschatology has assumed in
the speculation of Joachim of Floris -- the prediction of an age of
monachism -- but simply Joachim's attempt to find a meaning or -- as
Voegelin prefers to say -- an ``eidos'' in history. This attempt is not
specific to the mystic theology of Joachim of Floris; it is essential to
any theological interpretation of history. For if, as theology must
presuppose, mankind is created by a God endowed with absolute reason, it
must have been created for some purpose, and consequently its existence
in time, governed by the all powerful and absolutely just God, must have
some meaning. Joachim was certainly not \label{M94}the first who interpreted
history in this way. His idea that history aims at a definite, perfect
state of mankind is evidently modeled after the messianic scheme.

Karl L�with, whose \emph{Meaning in History}\footnote{Karl L�with,
  \emph{Meaning in History}, 1949, p. 151.} is -- besides Taubes'
\emph{Abend\-l�nd\-ische Eschatologie} -- the main source of Voegelin's
view of Joachim's trinitarian eschatology and its influence on modern
philosophies of history, says that Joachim's ``interpretation of the
angel of the apocalypse (Rev. 7:2) as the \emph{novus dux} entitled to
'renovate the Christian religion' '' meant ``that a messianic leader was
to appear, 'whosoever it will be', bringing about a spiritual renovation
for the sake of the Kingdom of Christ ...~'' But Voegelin maintains
that ``the problem of an eidos of history,'' that is, the question
whether history has a meaning, did not occur ``in orthodox
Christianity''\footnote{Voegelin, l.c., p. 119.}, it arose in Joachim's
eschatology, which was ``a speculation on the meaning of history. In
order to determine its specific difference, it must be set off against
the Christian philosophy of history that was traditional at the time,
that is, against Augustinian speculation.''\footnote{L.c., p. 118.}
However, according to the traditional Christian philosophy of history
prevailing in the 12th century history had a definite meaning. For
Augustine, just as for Joachim, the meaning of history was salvation. In
the chapter on Augustine's Theology of History L�with says: ``What
really matters in history, according to Augustine, is not the transitory
greatness of empires, but salvation or damnation in a world to come. His
fixed viewpoint for the understanding of the present and past events is
the final consummation in the future: last judgment and resurrection.
This final goal is the counterpart of the first beginning of human
history in creation and original sin.''\footnote{L�with, l.c., p. 168.}
``The whole of Augustine's work serves the purpose of vindicating God in
history.''\footnote{L.c., p. 170.} Long before Joachim, Augustine
distinguished several epochs in history, not, as Joachim did, three
periods according to the three personalities of \label{M95}God, but -- more
``historically'' -- six periods according to the six days of creation.
``The first extends from Adam to the Great Flood, the second from Noah
to Abraham, and the third from Abraham to David, with Nimrod and Nimus
as their wicked counterparts. The fourth epoch extends from David to the
Babylonian Exile, the fifth from there to the birth of Christ. The sixth
and last epoch, finally, extends from the first to the second coming of
Christ at the end of the world.''\footnote{L.c., pp. 170f.} Orosius,
the disciple of Augustine, too, recognized salvation as the meaning of
history; guided by the conviction ``that God governs the course of human
history'' and ``all power derives ultimately from God'', he
distinguished four periods, represented by four kingdoms: first, the
Babylonian, then the Macedonian, later the African, and finally the
Roman kingdom. ``This meaningful succession, culminating in Christian
Rome, indicates that 'one God has directed the course of history in the
beginning for the Babylonians, and in the end for the
Romans.''\footnote{L.c., p. 176f.} In view of these facts it is hardly
possible to maintain that the problem of an eidos in history did not
occur in orthodox Christianity and that this problem first arose in
Joachim's trinitarian eschatology, and, in particular, that it arose
``from the Joachitic immanentization.''\footnote{Voegelin, l.c., p.
  119.}

By ``immanentization'' Voegelin means that whereas Augustine projected
the fulfillment of the Christian expectation of the kingdom of God into
another world, Joachim -- in conformity with the original messianic idea
of the kingdom of God -- predicted its realization on earth, in the
future of monachism. This tendency of Joachim's eschatology is
characterized by Voegelin as follows: ``The age of Joachim would bring
an increase of fulfillment within history, but the increase would not be
due to an immanent eruption; it would come through a new
transcendental irruption of the spirit.''\footnote{Ibid.} By the
bombastic term ``transcendental irruption of the spirit'' -- in
contradistinction to ``immanent eruption'' -- Voegelin evidently wants to
express the re-divinization of the world. But neither the Messianic
kingdom of God on earth nor Joachim's age of monachism mean that God,
the divine spirit, will enter a world completely forsaken by God; just
as Augustine's transfer of the kingdom of God into another world did not
mean that this world is completely separated from God. Such an idea is
incompatible with Christian religion. Hence ``Joachitic
immanentization'' cannot be interpreted as re-divinization. \label{M96}Anyway,
this immanentization is certainly not the only way to find a meaning in
history. Voegelin's statement: ``The problem of an eidos in history,
hence, arises only when Christian transcendental fulfillment becomes
immanentized''\footnote{L.c., p. 120.} is without foundation. But he is
right when he emphasizes: ``Such an immanentist hypostasis of the
eschaton ...~is a theoretical fallacy. Things are not things, nor do
they have essences, by arbitrary declaration.'' And then he arrives at
the highly rationalistic truth: ``The course of history as a whole is no
object of experience; history has no eidos, because the course of
history extends into the unknown future. The meaning of history, thus,
is an illusion.''\footnote{Ibid.} This truth has been found, long before
Voegelin undertook his crusade against the destructive positivism, by a
representative of the positivistic science of history, Theodor Lessing,
who, at the beginning of the 20th century, published a book under the
title \emph{Geschichte als Sinngebung des Sinnlosen} (History as
Attribution of Meaning to the Meaningless).\footnote{Theodor Lessing,
  Geschichte als Sinngebung des Sinnlosen, 3rd ed., Muenchen 1921.} In
this book Lessing calls ``the view that history reflects reason and
meaning, progress and justice'', a ``pious delusion.'' He refutes the
opinion ``that history is to be written on the basis of a science
attributing meaning to its object,'' that ``historic reality is a chain
of causes having meaningful effects, revealing in the course of events
a natural or even divine reason.''\footnote{L.c., pp. 3f.} Voegelin
does not mention this predecessor in the discovery that history has no
meaning. And this is quite understandable, for Theodor Lessing belongs
to a school of historians which is far from theologico-metaphysical
speculations. Less understandable is how Voegelin can try to make the
followers of the new science of politics believe that his rationalistic
sceptical view that we cannot find any meaning in history, that history
has no meaning because it extends into the unknown future, is compatible
with his anti-positivistic postulate that the new science of politics
has to be based on metaphysical speculation and theological
symbolization. For it \label{M97}is just metaphysics and theology that are
guilty of the fallacious ``illusion'' of finding a meaning in history,
because the fact that the future is unknown to men does not at all
prevent them from speculating about the unknown in general and the
unknown future in particular, whether it be the unknown future of all
mankind or the unknown future of the individual man, his fate after
death, the ``truth of the soul'' discovered by Plato and taught by
Christian theology, and just for this reason highly praised by Voegelin.
Is gnosticism or what Voegelin designates by this term not of the same
flesh as Plato's metaphysics and Christian theology? And if there is a
difference, it is because gnosticism is still more intensive in its
drive toward the unknown transcendental sphere than Plato's metaphysics
and Christian theology, to the principles of which according to Voegelin
the science of politics has to return in order to become again
constructive. If such a political science has anything to object against
the positivists Comte and Marx, it cannot be that they tried to find a
meaning in history and that the meaning they thought they had found is --
in the opinion of Voegelin -- somehow similar to that which a
theological-metaphysical speculation like Joachim's trinitarian
eschatology has discovered.

As for the rest, to find a meaning in history does not necessarily
presuppose a metaphysical hypothesis, that is to say, the recourse to a
transcendental sphere. By the meaning of history nothing else may be
understood but that the social life of men and its evolution, just as
nature, is determined by laws; and to find out these socio-historical
laws is to find out the meaning of history. Whether this view is correct
or not, whether it is possible to find out laws of evolution in history,
is another question. But just as natural science in its attempt at
describing natural phenomena in accordance with laws or a law of
evolution formulated on the basis of facts, an analogous interpretation
of history is, in principle, possible without any metaphysical
hypothesis. This was certainly the intention of Marx who predicted the
communist society of the future as \label{M98}the outcome of a law of social
evolution, just as an astronomer predicts an eclipse of the sun. The law
he believed he had found might have been a product of his wishful
thinking of his desire for the realization of socialism. That he
presented the realization of socialism as the outcome of a law of
evolution is due to the influence that the evolutionary,
anti-theological natural science of his time had on his thinking, and
certainly not to any mystic speculation. 

After denouncing the gnostic belief in a meaning of history as a
fallacious illusion, Voegelin tries to explain why men deceive
themselves and others by such an illusion. The reason is not, he
asserts, ``that the thinkers who indulged in it were not intelligent
enough to penetrate it. Or that they penetrated it but propagated it
nevertheless for some obscure evil reason.''\footnote{Voegelin, l.c., p.
  121.} The true reason is: the feeling of uncertainty. By ``their
fallacious construction'' the thinkers ``achieved a certainty about the
meaning of history, and about their own place in it, which otherwise
they would not have had.''\footnote{L.c., p. 122.} Then Voegelin asks:
``What specific uncertainty was so disturbing that it had to be
overcome'' by the fallacious illusion of gnostic speculation about the
meaning of history? His answer to this question is one of the most
original paradoxes of a study so rich in paradoxical statements. The
feeling of uncertainty that had to be overcome by the fallacious
illusion of gnosticism is that feeling of uncertainty which is the
result of Christian belief. ``Uncertainty is the very essence of
Christianity.''\footnote{Ibid.} Until now we were of the opinion that
the essence of Christianity is just the feeling of certainty which an
all-powerful, absolutely just and at the same time all-merciful God,
whose will is done in heaven as well as on earth, gives to the believer;
and that nobody can be so firmly convinced that history has a meaning as
a Christian who believes that an omniscient, infinitely wise spirit
directs its course for the best of mankind. To give to his incredible
statement about \label{M99}uncertainty as ``essence of Christianity'' some
appearance of credibility Voegelin repeats his unfounded assertion that
Christianity by its victory over paganism has de-divinized the world.
``The feeling of security in a 'world full of gods is lost with the gods
themselves; when the world is de-divinized, communication with the
world-transcendent God is reduced to the tenuous bond of faith, in the
sense of Heb.11:1 ...~''\footnote{L.c., p. 122.}  Faith, according to
this letter of St.~Paul, is ``the assurance of things hoped for, the
conviction of things not seen.'' Any belief in God is a conviction of
things not seen. If this belief is by its very nature ``tenuous'' and,
as Voegelin says, ``may snap easily'', there never has been a firm
belief in God, which, of course, is in open contradiction to facts. The
view that polytheism, with its inevitable consequence of human-like gods
acting one against another, as e.g.~Zeus and Hera, gives man a greater
feeling of certainty than Christian monotheism, is refutable by the
undeniable fact that the victory of Christianity over the religion of
Rome is due to a great extent just to the feeling of absolute certainty
the believers in Christ gained by this belief.

The meaning of history is according to Christianity as well as to
gnosticism (or what Voegelin calls gnosticism) salvation in a coming
realm; and the coming is in both cases absolutely certain. Hence the
question where and when it will take place can make no difference with
respect to the feeling of certainty of the individual believer. And,
indeed, in order to explain how through the gnostic immanentization
the certainty can be achieved which the traditional Christian religion
cannot guarantee, Voegelin applies the term ``immanentization'' with a
new meaning. Now it does mean a ``fallacious construction'', a wrong
theory about the meaning of history, the ``fallacious
immanentization of the Christian eschaton,''\footnote{L.c., p. 121.}
that is to \label{M100}say, the transfer of the realm to come from heaven to
earth, a historical period. It designates a religious experience in
the soul of the individual.  Voegelin assumes, it seems, that in the
12th century ``a fall from faith in the Christian sense ...~as a mass
phenomenon'' occurred: for he maintains that those who lost their
Christian faith fell into the gnostic movement of this time. ``The
fall could be caught only by experiential alternatives, sufficiently
close to the experience of faith that only a discerning eye would see
the difference, but receding far enough from it to remedy the
uncertainty of faith in the strict sense. Such alternative experiences
were at hand in the gnosis which had accompanied Christianity from its
very beginnings.''\footnote{L.c., pp. 123, 124.} The religious
experience, very close to the experience of Christian faith but
nevertheless different from it, is the ``immanentization'' through
which that certainty is achieved which Christian faith cannot bring
about. It is characterized as follows: ``The attempt at immanentizing
the meaning of existence is fundamentally an attempt at bringing our
knowledge of transcendence into a firmer grip than the \emph{cognitio
fidei}, the cognition of faith, will afford; and Gnostic experiences
offer this firmer grip in so far as they are an expansion of the soul
to the point where God is drawn into the existence of
man.''\footnote{L.c., p. 124.}  The ``expansion of the soul to the
point where God is drawn into the existence of man'' is the \emph{unio
mystica}, the typical experience achieved by a mystic in a state of
ecstasy, the feeling of being united with God. ``...~the men who fall
into these experiences'', says Voegelin ``divinize themselves by
substituting more massive modes of participation in divinity for faith
in the Christian sense.''\footnote{Ibid.} This is exactly the
deification of man and the assimilation of the creature to the
Creator, which is the essence of a mystic experience as described by
one of the great mystics, Meister Eckehart (1260-1327), in his
\emph{Opus Tripartitum}. If the immanentization as described by
Voegelin in the just-quoted statements is gnosticism, gnosticism is
pure mysticism; and then it is not understandable how gnosticism can
become a category of a social science.  For the \label{M101}self-divinization
of man through his union with God is a most individual experience
which has no social implication. The mystic is a-social. For his union
with God isolates him from others.  Self-divinization is no basis of
social cooperation with others as equals. It stands to reason that
such an experience is possible only on the basis of a strong belief in
the existence of a transcendent God, and that the transcendence of God
- as pointed out -- is not only perfectly compatible with the immanence
of God in the experience of the mystic but is an indispensable
condition of this experience. Only because the mystic believes in a
transcendent God has he the desire to draw him into his individual
existence. Besides, the mystic union with God is a very rare
experience in the life of the mystic, and outside of this experience
God exists for him only in his transcendence. All this is nothing new,
but it must be mentioned because Voegelin seriously maintains that the
mystic experience of immanentization, the ``operation of getting his
grip on God'' along with its self-divinization of man, is ``the core
of the redivinization of society.''\footnote{L.c., p. 124.}  Among
those who are responsible for this redivinization, who fall off
through their doctrines, induce men to fall into these gnostic-mystic
experiences, he mentions Hegel and Schelling, some ``paraclectic
sectarian leaders'' whom he does not name, and, above all, Comte, Marx
and Hitler.  In view of the fact that the gnosticism of Hitler and the
gnosticism of Marx are note quite the same, Voegelin is forced to
concede: ``The intellectual symbols developed by the various types of
immanentists will frequently be in conflict with one another, and the
various types of Gnostics will oppose one another.''\footnote{L.c.,
p. 125.} But, gnostics they are, even the atheist Marx. How can
Marxism be an attempt at drawing God into the existence of man, at
getting a firmer grip on God than the Christian religion affords, if
Marx, following Feuerbach, declares the belief in God the most harmful
illusion of mankind and religion the opium of the people; how can an
atheist be a mystic; how can he have that experience which
\label{M102}presupposes the passionate belief in the existence of God and
which, from the viewpoint of an atheist like Marx, is nothing but the
hallucination of an hysteric? Very simple: By taking an inappropriate
metaphor for the expression of reality. Of Feuerbach and Marx Voegelin
says that they ``interpreted the transcendent God as the projection of
what is best in man into a hypostatic beyond; for them the great
turning point of history, therefore, would come when man draws his
projection back into himself, when he becomes conscious that he
himself is God, when as a consequence man is transfigured into
superman.''\footnote{Ibid.} The interpretation of Feuerbach's critique
of religion -- followed by Marx -- that man himself is God, obscures the
essence of his teaching, namely, that there is no transcendent power
above man, that it is not God who created man in his image, that it is
man who has created in his image this imaginary entity. Man cannot
himself be God because there is no God. Hence, neither Feuerbach nor
Marx exalted man to a superman. On the contrary. Marx calls the
emancipation of man from religion, as an ideology of the existing
social order humiliating man, the ``return of man to himself,'' the
``reconquest'' or ``restoration of man.''  Identifying man with God as
an interpretation of Feuerbach's and Marx' critical destruction of the
belief in God is fundamentally wrong, because if man is supposed not
to believe in the existence of God, he cannot become conscious that he
himself is God. To interpret the rationalistic, outspoken
anti-religious, anti-metaphysical philosophy of Feuerbach and Marx as
mystic gnosticism, to speak of a ``Marxian transfiguration'' of man
into God, and to say of the atheistic theory of Marx that it carries
``to its extreme a less radical medieval experience which draws the
spirit of God into man, while leaving God himself in his
transcendence,''\footnote{Ibid.} is, to formulate it as politely as
possible, a gross misinterpretation.

According to the teachings of Marx -- as according to any
rationalistic, non-religious political doctrine -- the welfare of man
can be achieved only \label{M103}through man's own work, performed in
confidence on man's own capacities and not by the grace of a
transcendental authority. It is the principle: help yourself and do
not rely on the help of God. This is the principle that Voegelin
interprets as the Marxian transfiguration, as Marx' attempt at drawing
the spirit of God into man in a more radical way than medieval
theology, which left God himself in his transcendence.  And on the
basis of this interpretation of atheism as the most intensive belief
in God he recognizes ``the essence of modernity as the growth of
gnosticism''\footnote{L.c., p. 126.} during which ``civilizational
activity became a mystical work of self-salvation'', the ``miracle of
self-salvation.''\footnote{L.c., p. 129.} As an example of modern
philosophy which aims at such mystic-miraculous self-salvation by
drawing the spirit of God into man, Voegelin refers not only to
Marxism but also to another atheistic enemy of religion, Friedrich
Nietzsche. He ``raised the question,'' says Voegelin, ``why anyone
should live in the embarrassing condition of a being in need of the
love and grace of God.  'Love yourself through grace -- was his
solution -- then you are no longer in need of your God, and you can
act the whole drama of Fall and Redemption to its end in
yourself'.''\footnote{Ibid.} Nobody, with the exception of the founder
of the new science of politics, can see in these words of Nietzsche
anything else but the unambiguous expression of the most
anti-religious, anti-metaphysical, and consequently
anti-mystic-gnostic, attitude. It is true that Nietzsche, in
contradistinction to Feuerbach and Marx, spoke of a superman. But
Nietzsche's hero was not a superman because he was able to produce the
mystic experience of a union with God; he was, not a God himself, but
a man above the ordinary men precisely for the contrary reason:
because he was able to separate himself from God, that is to say, to
emancipate himself from the belief in God, because for him God did not
exist, ``God was dead'', God was ``murdered'' by him. The emancipation
from the belief in God may be poetically called a ``murder of God.''
But it can certainly not -- even not with a poetical license be called
a ``gnostic murder''\footnote{L.c., p. 131.}, as Voegelin does in
order to maintain his impossible attempt to interpret atheism as
gnosticism. \label{M104}If we are asked to believe that the atheism of
Marx and Nietzsche is gnosticism, then we should not be astonished to
learn from the new science of politics that Comte -- ``the founder of
positivism''\footnote{L.c., p. 130.}, of the destructive positivism
that we have to abandon because it does not ``rely on the methods of
metaphysical speculation and theological
symbolization''\footnote{L.c., p. 6.} -- is a gnostic too, that his
rationalistic philosophy is gnosticism, and that means mysticism. How
does Voegelin work out this re-interpretation of Comte?  Very simple,
again by taking a metaphor for the real thing. Comte's quite
insignificant remark that the memory of ``those who contribute to
civilization will be preserved, whereas those who do not will be
forgotten, is interpreted by Voegelin to mean that Comte guaranteed,
as ``a premium on civilizational contributions,'' ``immortality
through preservation of the contributor and his dead in the memory of
mankind'' and ``the reception of the meritorious contributor into the
calendar of positivistic saints,'' whereas those ``who would rather
follow God than the new Augustus Comte'' ``would simply be committed
to the hell of social oblivion.''\footnote{L.c., pp. 130, 131.} Thus
Voegelin finds in Comte's philosophy the metaphysical belief in
immortality, the recognition of saints, and last but not least, the
threat of hell: ``Here is a gnostic paraclete setting himself up as
the world-immanent Last Judgment of mankind, deciding on immortality
or annihilation for every human being.'' A more arbitrary
misinterpretation of Comte's rationalistic positivism is hardly
possible. Voegelin continues: ``The materialistic civilization of the
West, to be sure, is still advancing; but on this rising plane of
civilization the progressive symbolism of contributions,
commemoration, and oblivion draws the contours of those 'holes of
oblivion' into which the divine redeemers of the Gnostic empires drop
their victims with a bullet in the neck. This end of progress was not
contemplated in the halcyon days of Gnostic exuberance
...~''\footnote{L.c., pp. 130, 131.} Does this mean that Comte's view
that only those who contribute to civilization will not be forgotten,
implies the idea that progress will come to an end? This was certainly
not the \label{M105}idea of Comte. Or does it mean that the murders
committed by totalitarian dictators have anything to do with Comte's
harmless prediction? If it has not this meaning -- as we hope out of
respect for an American professor of political science -- it has no
meaning at all. If Marx and Nietzsche, Comte and Hitler are gnostics,
then, of course, liberalism\footnote{L.c., p. 130.}  as well as
totalitarianism\footnote{L.c., p. 132.} are manifestations of gnostic
mysticism. That the one restricts the competence of the state to a
minimum whereas the other expands it to a maximum, is of minor
importance as compared with the fact that both represent gnosticism.
Only one thing essential to modern civilization is left of which we
may hope that it cannot be subjected to this forcible
re-interpretation: modern science, the advancement of which is due to
everything else but to mystic speculation. But the new political
science is not inclined to justify our hope. For we read on page 127
of Voegelin's book: ``Finally, with the prodigious advancement of
science since the seventeenth century, the new instrument of cognition
would become, one is inclined to say inevitably, the symbolic vehicle
of Gnostic truth.''  ``Scientism,'' that is, the appreciation of
science, the readiness to rely on science, is according to Voegelin
``one of the strongest Gnostic movements in Western society; and the
immanentist pride in science is so strong that even the special
sciences have each left a distinguishable sediment in the variants of
salvation through physics, economics, sociology, biology, and
psychology.'' By distorting appreciation of science into ``immanentist
pride'' in science and reliance on science into mystic ``salvation''
through science, the re-interpretation of modern science as a
``symbolic vehicle of Gnostic truth'' is achieved.

% [Von Kelsen wieder gestrichen : ???Voegelin expected that his re-interpretation or, as he calls it,
% his ``inversion of the socially accepted meaning of terms would arouse a
% certain hostility.''\footnote{L.c., p. 23.} In this point, too, he is
% mistaken. For what he is doing is not inverting, but perverting the
% meaning of terms. And this cannot arouse hostility but only regret.]

\label{M106}
\subsection{3.}

According to the new science of politics it is not only the nature of
modernity which is to be interpreted as gnosticism; but also the
Reformation, with which the modern age begins, is to be ``understood
as the successful invasion of Western institutions by Gnostic
movements.''  How does the new science of politics justify this
revolutionary re-interpretation of the Reformation? Voegelin declares:
``The event is so vast in dimensions that no survey even of its
general characteristics can be attempted in the present lecture.''
Consequently he restricts his task to an analysis of ``Certain aspects
of the Puritan impact on the English public order.''\footnote{L.c.,
pp. 134, 135.} These aspects are the religious ideologies produced
within the left wing of the Puritan movement for the purpose of
legitimizing the English revolution of the 16th century. Voegelin
admits that ``Puritanism as a whole cannot be identified with its left
wing''\footnote{L.c., p. 151.}, but he justifies his selection of
materials by the statement that he does not intend ``to give a
historical account of Puritanism,'' that he is concerned only ``with
the structure of Gnostic experiences and ideas''\footnote{Ibid.}; and
this structure can be found in the material he has chosen. But even if
he had demonstrated the gnostic character of the religious ideology of
left wing Puritanism -- which he did not -- his amazing interpretation
of the Reformation, of which the Puritan movement was only one of many
components and not the most decisive one, as the revolutionary
eruption of gnostic movements would remain completely
unfounded. \label{M107}Voegelin's interpretation of Puritanism is not based
on an analysis of the movement itself but on the famous description
Hooker gave of this movement. From this description Voegelin gathers
the fact that Puritans in their criticism of the existing social
conditions insisted on having a ``cause'', that the term ``cause'' was
of recent usage and that probably the Puritans had invented
it.\footnote{L.c., p. 135.} To have a ``cause'' in order to start a
movement is interpreted by Voegelin as a ``formidable weapon of the
Gnostic revolutionaries'' How does this having a cause manifest
itself? ``In order to advance his 'cause', the man who has it will,
'in the hearing of the multitude,' indulge in severe criticisms of
social evils and in particular of the conduct of the upper
classes. ...~The next step will be the concentration of popular
ill-will on the established government. This task can be
psychologically performed by attributing all fault and corruption, as
it exists in the world because of human frailty, to the action or
inaction of the government. ...~After such preparation, the time will
be ripe for recommending a new form of government as the 'sovereign
remedy of all evils'.''\footnote{L.c., pp. 135, 136.} It is absolutely
undiscoverable what there is gnostic in this having a cause and in
advancing the cause in this way, which is the way of any political
movement directed against the established government and especially of
a revolutionary movement. It is not specifically Puritan, and not in
the least mystic.

Another symptom of the gnostic character of the Puritan movement is that
it ``relies on the authority of a literary source'', namely, the Holy
Scripture, and that ``the leaders will then have to fashion the very
notions and conceits of men's minds in such a sort' that the followers
will automatically associate scriptural passages and terms with their
doctrine, however ill founded the association may be.''\footnote{L.c.,
  p. 136.} This is the attitude of every political movement the ideology
of which is furnished by Christian theology, without any gnostic or any
other mystic implication.

\label{M108}The ``decisive step in consolidating a Gnostic attitude'' is
according to Voegelin described by Hooker when he accuses the Puritans of
``the persuading of men credulous and over-capable of such pleasing
errors, that it is the special illumination of the Holy Ghost, whereby
they discern those things in the word, which others reading yet discern
them not.''\footnote{Ibid.} Again there is nothing gnostic in the
attempt of a religious leader to make his followers believe that he is
illuminated by a transcendent authority. The ``special illumination of
the Holy Ghost'' on which Puritan interpreters of the Scripture tried to
base their authority has nothing to do with the mystic experience of a
union with God. Only if Voegelin could prove that such mystic experience
played an essential role in the Puritan movement, would his
interpretation of this movement as gnostic in the sense of mystic be
justifiable. But such a proof is impossible for the simple reason that
no social movement can have such a mystic character, because the mystic
experience has -- as pointed out -- by its very nature an anti-social or
at least an a-social character. A true mystic is far from being even
interested in a criticism or a reform of society, which was the main
concern of Puritanism. A revolutionary mystic is a contradiction in
terms.

In order to show that the movement which he calls the ``Puritan
Revolution''\footnote{L.c., p. 135.} is a Gnostic revolution, Voegelin
refers also to ``two technical devices ...~which to this day have
remained the great instruments of Gnostic revolution.''\footnote{L.c.,
  p. 138.} The first device is a ``Gnostic koran'', and the Gnostic
koran of the Puritan movement was Calvin's \emph{Institutes}. Why does
Voegelin invent the strange term of ``Gnostic koran'' in order to
characterize Calvin's work? Because a ``work of this type would serve
the double purpose of a guide to the right reading of Scripture and of
an authentic formulation of truth that would make recourse to earlier
literature unnecessary. For the designation of this genus of Gnostic
literature a technical term is needed; since the study of Gnostic
phenomena is too recent to have developed one, the Arabic term
\emph{koran} will have to do for the present.''\footnote{L.c., p. 138f.}
But the Arab term \emph{Qur'an} means nothing but ``recitation''
\label{M109}and the Koran is the Holy Scripture of the Moslems, considered by
them as the word of God communicated to the Prophet Mohammed through the
intermediation of an angel. Hence the Koran is itself a Scripture and
not a guide to the right reading of Scripture as Calvin's
\emph{Institutes}; and hence it is by no means appropriate to call this
book a koran. Nor is there any reason to consider an interpretation of
the Scripture which claims to be authentic as ``gnostic.'' The fact that
Calvin -- according to Hooker -- claimed to owe his divine knowledge ``to
none but only to God'' is certainly not a sufficient basis for such an
interpretation. To be the mouthpiece of God was the claim of Moses,
Jesus and Mohammed. If this claim is the criterion of gnosticism, the
three religions founded by these prophets are gnostic and, then, the
term ``gnosticism'' means simply religion. But Voegelin does not
maintain the direct inspiration by God as the criterion of gnosticism.
For he considers also the French Encyclopedia as a gnostic koran: ``In
the eighteenth century, Diderot and D'Alembert claimed koranic function
for the \emph{Encyclop�die} \emph{francaise} ...~'' Why? Did Diderot
and D'Alembert claim to owe their knowledge directly to God? This is
impossible because they were rationalists. But Voegelin says that they
claimed ``koranic function'' for the \emph{Encyclop�die francaise}
because they considered the Encyclop�die ``as the comprehensive
presentation of all human knowledge worth preserving.''\footnote{L.c.,
  p. 139.} Did they pretend that every word written in this dictionary
had the value of an eternal unchangeable truth? They expressly declared
to present in the encyclopedia only \emph{l'etat present des sciences et
  des arts}\footnote{D'Alembert, \emph{Discours preliminaire de
    l'Encyclop�die}. Publi� par F. Picavet, Paris, 1929, p. 75.}  (the
present status of the sciences and arts), which implies that this status
is not at all to be regarded as definitive. Even Voegelin can attribute
to the encyclopedists only the opinion that ``nobody would have to use
any work antedating the \emph{Encyclop�die}, and all future sciences
would assume the form of supplements to the great collection of
knowledge.''\footnote{Voegelin, l.c., p. 140.} This statement can only
refer to a passage in D'Alembert's \emph{Discours preliminaire de
  l'Encyclop�die} which runs as \label{M110}follows: ``De tout ce qui pr�c�de,
il s'ensuit que dans l'ouvrage que nous annoncons, on a trait� des
sciences et des arts, de mani�re qu'on n'en suppose aucune connaissance
pr�liminaire; qu'on y expose ce qu'il importe de savoir sur chaque
mati�re, que les articles s'expliquent les uns par les autres, et que
par consequent la difficult� de la nomenclature n'embarrasse nulle
part.''\footnote{D'Alembert, l.c., p. 150.} That means nothing else
but that the reader can understand the content of the various articles
without looking for explanation in other books, since for every term
used in one article, but not explained in it, there is an explanation to
be found in another article. D'Alembert continues: ``D'o� nous inf�rons
que cet ouvrage pourra, du moins un jour, tenir lieu de biblioth�que
dans tous les genres � un homme du monde; et dans tout les genres,
except� le sien, � un savant de profession, qu'il d�veloppera les vrais
principes des choses; qu'il en marquera les rapports; qu'il contribuera
� la certitude et au progr�s des connaissances humaines ...~'' If that
could be interpreted to mean that ``nobody would have to use any work
antedating the \emph{Encyclop�die}'', it should be added that it is said
only with respect to the \emph{homme du monde}, the educated layman,
that the \emph{savant}, the professional scholar, is expressly excepted;
and that the reference to the progress of knowledge evidently implies
that the \emph{Encyclop�die} does not pretend to be the end of this
progress. D'Alembert's \emph{Discours pr�liminaire de l'Encyclop�die}
shows clearly that by the publication of this work everything was
intended but imposing upon the world a ``gnostic koran.''
 
In order to ridicule the uncritical way in which Marxists ascribe an
undisputed authority to the founders of scientific socialism, it is
usual to say that the works of Marx and Engels are the ``bible'' of
their followers. This, of course, is an exaggerating
metaphor. Substituting for the bible the koran, and taking the metaphor
for the expression of reality, Voegelin arrives at his last example of a
gnostic koran: ``the works of Karl Marx have become the koran of the
faithful, supplemented by the patristic literature of
Leninism-Stalinism.''\footnote{Voegelin, l.c., p. 140.} 

\label{M111}The second device of gnostic revolution is ``putting a taboo on the
instruments of critique''.\footnote{Ibid.} This is a device used by any
religious movement which pretends to be in possession of absolute truth.
To be of another but the authentic opinion laid down by the established
authority is a punishable crime. The concept of heresy, which plays such
a fateful part in the history of the Christian Church, implies this
idea. To punish heretics and thus prevent a critique of the authentic
doctrine is not practiced only by ``gnosticism''.  The example given by
Voegelin to illustrate a Puritan taboo on critique is significant.
Hooker was blamed in the anonymous \emph{Christian Letter} of 1599 -- a
document of Puritan origin -- for having used Aristotle against Holy
Scripture.\footnote{L.c., p. 141.} Does Voegelin seriously assert that
the church, the representative of the ``Christian tradition'' which he
opposes to gnosticism,\footnote{L.c., p. 137.} does not use this same
device, that this Church allowed or allows free critique of the Holy
Scripture? Voegelin can not deny that the Reformation was directed
against the established Church as a movement which, at least at its
earlier stages, set forth among its aims a free interpretation of the
Bible. How, then can he characterize the Reformation as a whole as a
revolutionary eruption of gnostic movements if he at the same time
declares the suppression of this freedom as a specific gnostic device?
If the nature of modernity is gnosticism, and if gnosticism means the
suppression of the freedom of critique, how can the undeniable fact be
explained that a characteristic element of modern civilization is the
political movement toward democracy, which implies the principle of free
critique, and that it was just with the Puritans of the Left that
democratic theories originated?\footnote{Cf. A.D.Lindsay, \emph{The
    Modern Democratic State}, 5th ed. 1951, p. 117.} The new science of
politics does not face this question, unless the following statement is
taken for an answer: ``However well the constitutional freedoms of
speech and press may be protected, however well theoretical debate may
flourish in small circles, and however well it may be carried on in the
practically private publications of a handful of scholars, debate in the
politically relevant public sphere will be in substance the game with
loaded dice which it has become in contemporary progressive societies --
to say nothing of the quality of debate in totalitarian empires.
Theoretical debate can be protected by constitutional guarantees, but it
can be established only by the willingness to use and accept theoretical
argument.''\footnote{Voegelin, L.c., pp. 141f.} \label{M112}Does this mean
that there is no freedom of critique in the Western democracies? If so,
the statement is simply not true. And if Voegelin does not, and can not,
deny that there is no taboo put on the instrument of critique -- how
could, otherwise, Feuerbach's and Marx' critique of religion and other
ideologies have been possible in this civilization -- and if it cannot be
denied that the democratic creed is not based on a gnostic koran, like
Puritanism on Calvin's \emph{Institutes}, then there is no answer to the
question how within a civilization the nature of which is supposed to be
gnosticism, important societies do not use the specific gnostic devices.

As a specifically gnostic element of the Puritan revolution Voegelin
considers the fact that the Puritan revolutionaries interpreted the
kind of society which they were fighting for as the realm to come and
as an event that required their military cooperation. On the basis of
the most disputable statement that ``there is no passage in the New
Testament from which advice for revolutionary political action could
be extracted,''\footnote{L.c., p. 145.} Voegelin assumes that the
Puritans falsely justified their revolutionary actions by referring to
the Revelation of St.~John and arrogated to themselves the function of
the angel who ``comes down from heaven and throws Satan into the
bottomless pit ...~'' To document this view, he refers to a pamphlet,
\emph{A Glimpse of Sion's Glory}, published in 1641, attributed to a
Puritan.  But the passages quoted by Voegelin\footnote{L.c., pp.
145ff.}  contain nothing that could justify the interpretation of the
document as a product of gnosticism. With respect to the relation
between the revolutionaries and God, the only relation relevant to the
interpretation of the Puritan revolution as gnostic, the pamphlet --
according to the quotations of Voegelin -- says that the omnipotent
God will come to the aid of the Saints, that is, the Puritan
revolutionaries. God ``shall do these things, by that power, whereby
he is able to subdue all things unto himself. Mountains shall be made
plain, and he shall come skipping over mountains and over
difficulties.  Nothing shall hinder him.'' Voegelin who first
emphasized as a specific gnostic element that ``a Gnostic who will not
leave the transfiguration of the world to the grace of God beyond
history but will do the work of God himself, right here and now, in
history,''\footnote{L.c., p. 147.}  must now admit that ``the author
of the pamphlet knows that not ordinary human powers will establish
the realm but that human efforts will be subsidiary to the action of
God''; whereby he drops this ``gnostic'' element in his picture of the
Puritan revolution. Now he is no longer interested in the gnostic
character of the Puritan revolution but in the fact that the ideology
of the English revolution shows certain similarities with the ideology
of the Russian revolution: ``in this God who comes skipping over the
mountains we recognize the dialectics of history that comes skipping
over thesis and antithesis, until it lands its believers in the plain
of the Communist synthesis.''\footnote{Ibid.}  That there are certain
similarities between the ideology of the Puritan and that of the
Soviet revolution is true; and the results of Voegelin's comparison
are indeed very interesting. But the ideologies of all revolutions
exhibit these similarities. To recognize the God who came skipping
over the mountains in the Marxian dialectics, is one of the
metaphysical exaggerations which play a not very fortunate part in
Voegelin's interpretation of social phenomena. But even if the Marxian
dialectics were identical with the God of the Puritans and the Russian
revolution only a repetition of the English revolution, although the
former resulted in a liberal parliamentarianism whereas the latter in
a totalitarian dictatorship, there would not be the slightest reason
to speak of both as of gnostic revolutions and to designate the
dictatorship of the proletariat as a concept of ``later
Gnostics.''\footnote{L.c., p. 149.} How arbitrarily Voegelin uses the
term ``gnostic'' becomes evident in his statements concerning
``gnostic wars.'' He says: ``The revolution of the Gnostics has for
its aim the monopoly of existential representation. The Saints can
foresee that the universalism of their claim will not be accepted
without a struggle by the world of darkness but that it will produce
an equally universal alliance of the world against
them.''\footnote{L.c., p. 151.} It is obvious that this applies only
to the Russian, not to the English revolution, since only the former
and not the latter -- in spite of its apocalyptic ideology -- has
aimed at a world revolution, i.e. at the revolutionary establishment
of a new social organization comprising the whole of mankind. The
fundamental difference between the two revolutions is obscured by
terming them both as gnostic. Only on the basis of the unjustified
assumption that the Russian revolution is a ``gnostic'' revolution,
that is to say on the basis of an unfounded terminology, can Voegelin
speak of the split into two worlds as of ``the Gnostic mysticism of
the two worlds.'' But the two worlds of the Joachitic speculation, the
alleged model of the Puritan and Russian revolutions, the existing and
the coming world, are not at all bent on mutual
destruction. \label{change3}The split into two worlds is the result of
the Russian revolution but was not at all the result of the
``gnostic'' Puritan revolution. Voegelin says: ''The two worlds which
are supposed to follow each other chronologically will, thus, become
in historical reality two universal armed camps engaged in a death
struggle against each other.''\footnote{Ibid.} How two worlds which
follow each other in time can become two camps existing at the same
time one beside the other is indeed a mystery; but that ``two
universal armed camps'' are engaged in a struggle against each other
is no mystery at all.  Nevertheless, Voegelin continues: ``From the
Gnostic mysticism of the two worlds emerges the pattern of the
universal wars that has come to dominate the twentieth century.'' But
certainly not the ``gnostic'' revolution of the seventeenth century!
Since both are gnostic, Voegelin, referring to both, says: ``The
universalism of the Gnostic revolutionary produces the universal
alliance against him'', and thus arrives at the ``Gnostic wars'' of
our time: ``The real danger of contemporary wars does not lie in the
technologically determined global extend of the theater of war; their
true fatality stems from their character as Gnostic wars, that is, of
wars between worlds that are bent on mutual
destruction.''\footnote{Ibid.} The gnostic character of these wars
consists in the fact that they take place between two worlds ``that
are bent on mutual destruction.'' But the two worlds of the mystic
speculation -- the world of Satan and the world of God -- are not at
all bent on mutual destruction. On the \label{M115}contrary, only the
one will be replaced by the other, and the other will last
eternally. This is the case of the gnostic wars. They have just as
little to do with ``gnosticism'' as the ``gnostic'' revolutions and
the ``gnostic'' nature of modern civilization.

If the Puritan religious ideology of the English revolution is
considered to be an essential element of modern civilization, then
Hobbes' highly rationalistic philosophy, which much more than
Puritanism has influenced the thinking of modern man, cannot be
ignored in an analysis of modern civilization. Hence Voegelin quite
correctly considers it his duty to deal also with Hobbes'
\emph{Leviathan}. Although this work is one of the first and most
remarkable attempts to establish, at a time when theological
speculation and natural-law doctrine were still prevailing, a
positivistic political theory, Voegelin tries to locate also Hobbes
within his all-comprising category of gnosticism. He says that the
theory ``which Hobbes developed in the \emph{Leviathan}, to be sure,
purchased its impressive consistency at the price of a simplification
which itself belongs in the class of Gnostic misdeeds''\footnote{L.c.,
p. 152.}; as if simplification were a specific gnostic misdeed. He
further maintains that the essential intention of Hobbes was to
establish ``Christianity (understood as identical in substance with
the law of nature) as an English \emph{theologia civilis} in the
Varronic sense.''\footnote{L.c., p. 155.} By \emph{theologia civilis}
in the Varronic sense, Voegelin understands -- as we may suppose on the
basis of earlier statements\footnote{L.c., pp. 81ff.} -- a theology
that is a doctrine about God imposed by the authority of the state
upon the citizens. Since according to Hobbes' political doctrine all
teaching should be placed under the control of the state, theology
too could be taught only with the permission and by the authorization
of the government. But this extension of the competence of the state
to matters of religion has nothing to do with gnosticism or any other
kind of mysticism. It is \label{M116}the rationalistic attempt to use
religion as an instrument of politics. Finally, Voegelin maintains
that Hobbes' \emph{Leviathan} was ``an instance of the general class
of Gnostic attempts at freezing history into an everlasting final
realm on this earth''; it shows Hobbes' ``own Gnostic
intentions.''\footnote{L.c., pp. 160f.}

To confirm this interpretation, Voegelin refers to a passage in the
XXXth chapter of \emph{Leviathan}, which he reproduces as follows: ``He,
therefore, declared it the duty of the sovereign to repair the ignorance
of the people by appropriate information. If that were done, there might
be hope that his principles would 'make their constitution, excepting by
external violence, everlasting'.''\footnote{L.c., p. 160.} Since
Voegelin quotes literally only a few words, it is not superfluous to
quote the whole passage: ``So, long time after men have begun to
constitute commonwealths, imperfect, and apt to relapse into disorder,
there may, Principles of Reason be found out, by industrious meditation,
to make their constitution (excepting by external violence) everlasting.
And such are those which I have in this discourse set fourth: Which,
whether they come not into the sight of those that have Power to make
use of them, or be neglected by them, or not, concerneth my particular
interest, at this day very little.'' This passage
must not be interpreted without taking into consideration some preceding
statements. At the end of chapter XXVIII, Hobes emphasizes that the
commonwealth is mortal, and says that he will speak in the following
chapter of its ``diseases.''  Chapter XXIX begins with the statement:
``Though nothing can be immortal, which mortals make, yet, if men had
the use of reason they pretend to, their Commonwealth might be secured,
at least, from perishing by internal disease.'' If interpreted in the
light of these statements, it is evident that Hobbes did not predict the
inevitable, because God-sent coming of an everlasting realm -- the
essential content of ``gnostic'' eschatology. He only referred to the
possibility of constituting a commonwealth \label{M117}which would not perish
by internal disease. He did not exclude the possibility that such a
commonwealth will never be established, and he expressly admitted that
even if it would be established, it might perish by ``external
violence.'' Hence the term ``everlasting'' used in the passage quoted by
Voegelin is evidently a typical hyperbole, a slightly exaggerated
expression of the idea of an internally stabilized regime. And, last but
not least: Hobbes did not refer to a final stage of mankind, even not to
a world-wide commonwealth, but to commonwealths in the plural form.
There is not a shadow of a similarity between the realistic picture of
Hobbes' Leviathan and the utopian stage of perfection mankind will reach
in the realm to come according to the messianic prophecy.

\subsection{4.}

In the last chapter of his work on the new science of politics Voegelin
deals with ``The End of Modernity.'' %% \footnote{L.c., pp. 162ff.} 
Since the nature of modernity is gnosticism, we may expect to learn how
gnosticism, and hence modern civilization will come to an end, and
perhaps also what kind of civilization will come next. And indeed,
Voegelin identifies the end of modernity with ``the end of the Gnostic
dream'' which, as he says, ``is perhaps closer at hand than one
ordinarily would assume.''\footnote{L.c., p. 173.} What does he mean by
the ``Gnostic dream''? The idea of an everlasting state of perfection.
Gnosticism, he says, destroys the oldest wisdom that ``what comes into
being will have an end,'' and that ``the mystery of this stream of being
is impenetrable.'' Gnosticism is a ``counterexistential dream
world.''\footnote{L.c., p. 167.} As a consequence of ``its disregard for
the structure of reality,'' gnostic politics ``is self-defeating,''
because this disregard ``leads to continuous warfare.''\footnote{L.c.,
  p. 173.} This is a rather problematical statement, since continuous
warfare may have many different causes and is a phenomenon by no means
restricted to ``gnostic'' modernity. But this statement is not the most
questionable present in Voegelin's theory of the end of modernity. The
``self-defeating factor'' \label{M118}is only one of the two dangers which are
inherent in ``gnosticism as a civil theology.''\footnote{L.c., p. 166.}
The other danger is ``the destruction of the truth of the
soul.''\footnote{Ibid.} That gnosticism is, or can become, a civil
theology, is something new. Until now, we learned only that gnosticism
has the tendency to establish a civil theology. Because Hobbes intended
to establish Christianity as the English civil theology, he was
classified as a gnostic, in spite of the fact that this attempt was
precisely directed against gnosticism, the gnostic revolution of the
Puritans. But how can gnosticism itself become a civil theology or, what
amounts to the same, a civil religion, a state religion? What is the
content of the gnostic religion? As far as this question is concerned,
the new science of politics, which refuses to give a clear definition of
gnosticism -- as also in many other respects -- is not very clear. We can
only conjecture that Voegelin, when speaking of gnosticism as a civil
theology, has in mind the political doctrines of Marxian communism and
Hitlerite Nazism, which he characterized as gnostic ideologies and to
which, indeed, applies that characteristic of the gnostic search for a
civil theology that Voegelin formulates as putting a taboo on the
instrument of critique. On page 187 we read: ``As far as our experience
with totalitarian empires goes, their characteristic feature is the
elimination of debate concerning the Gnostic truth which they themselves
profess to represent. The National Socialists suppressed the debate of
the race question, once they had come to power; the Soviet government
prohibits the debate and development of Marxism. The Hobbesian principle
that the validity of Scripture derives from governmental sanction and
that its public teaching should be supervised by the sovereign is
carried out by the Soviet government in the reduction of communism to
the 'party line'.'' But what has gnosticism as civil theology to do with
the \label{M119}destruction of the ``truth of the open soul''?\footnote{L.c.,
  p. 163.} The truth of the open soul is -- so we have learned before --
the truth that the soul is open to God, that the soul is ``the region in
which transcendence is experienced'' and that the truth of the open soul
``is inseparable'' from the ``truth about God.''\footnote{L.c., p. 67.}
And, further, we were informed by the new science of politics that
gnosticism is an ``attempt at immanentizing the meaning of existence,''
the mystic attempt of ``an expansion of the soul to the point where God
is drawn into the existence of man''\footnote{L.c., p. 124.}, which is
nothing contrary to, but rather a still more intensive attempt at
opening the soul to God, the most radical recognition of the truth of
the soul. But Voegelin, speaking of the end of modernity, says: ``The
immanentization of the Christian eschaton made it possible to endow
society in its natural existence with a meaning which Christianity
denied to it. And the totalitarianism of our time must be understood as
journey's end of the Gnostic search for a civil theology.''\footnote{L.c.,
  p. 163.} By the ``totalitarianism of our time he'' he can only mean
Hitlerite Nazism and Marxian communism. That these political systems are
``journey's end of the Gnostic search for a civil theology'' or -- as he
formulates it later -- gnosticism as civil theology, results also from
the statements that gnostic movements, in their origin and development
from Joachim to the Puritans, closely connected with Christianity,
``tended to abolish Christianity,''\footnote{Ibid.} that the destruction
of this truth of the soul ``is the cause of the bleak atrocity of
totalitarian governments in their dealings with individual human
beings,''\footnote{L.c., p. 164.} and that it is difficult to foresee
``the probable reaction of a living Christian tradition against
gnosticism in the Soviet empire.''\footnote{L.c., p. 165.} Since the
National Socialist empire has already disappeared, the end of the
gnostic dream and thereby the end of modernity seems to coincide in the
main with the end of the Soviet empire.

\label{M120}It would be a mistake to interpret the ``gnostic dream'' of
totalitarianism to refer only to the political ideology of National
Socialist Germany and the Soviet state. It refers also to political
reality. Voegelin speaks of a ``Gnostic dream world.'' He admits, it
is true, that ``the nonrecognition of reality'' which in gnosticism
``is a matter of principle'' does not prevent the gnostic politicians
from remaining aware ``of the hazard of existence in spite of the fact
that it is not admitted as a problem in the Gnostic dream world; nor
does the dream impair civic responsibility or the readiness to fight
valiantly in case of an emergency. The attitude toward reality remains
energetic and active, but neither reality nor action in reality can be
brought into focus; the vision is blurred by the Gnostic dream. The
result is a very complex pneumopathological state of
mind.''\footnote{L.c., pp. 168, 169.} This may be true to a certain
extent with respect to the Nazi regime, but there is no sufficient
reason for the opinion that the vision of reality of the Soviet
government is ``blurred by the Gnostic dream,'' that is to say, by
their political ideology. The leaders of this Gnostic empire are
anything but dreamers and their state of mind is not at all
pneumopathological, whatever that term may mean. But later we read:
Whereas in ``the sixteenth century the dream world and the real world
were still held apart terminologically through the Christian symbolism
of the two worlds,'' later, with ``radical immanentization the dream
world has blended into the real world
terminologically.''\footnote{L.c., p. 169.} This ``identification of
dream and reality as a matter of principle has practical results which
may appear strange but can hardly be considered surprising.'' But,
surprising indeed, they are: ``The critical exploration of cause and
effect in history is prohibited: and consequently the rational
co-ordination of means and ends in politics is impossible. Gnostic
societies and their leaders will recognize dangers to their existence
when they develop, but such dangers will not be met by appropriate
actions in the world of reality. They will rather be met by magic
operations in the dream world, such as disapproval, moral
condemnation, declarations of intention, resolutions, appeals to
\label{M121}the opinion of mankind, branding of enemies as aggressors,
outlawing of war, propaganda for world peace and world government,
etc. The intellectual and moral corruption which expresses itself in
the aggregate of such magic operations may pervade a society with the
weird, ghostly atmosphere of a lunatic asylum, as we experience it in
our time in the Western crisis.''\footnote{L.c., p. 170.} It is hardly
possible that Voegelin considers the imperialistic policy of the Nazi
and the Soviet governments as magic operations in a dream world,
performed by nothing but moral condemnation, declarations of
intention, outlawing of war, propaganda for world government. It is
evident that now the gnostic dream world to which the just-quoted
statements refer is not the world of the totalitarian states the
political ideology of which is the civic theology of an everlasting
realm of perfection; it is the politics of the Western powers directed
against the totalitarian states which he castigates as ``the
manifestations of Gnostic insanity in the practice of contemporary
politics.'' Now it is not totalitarianism that Voegelin accuses of
identifying dream and reality, of prohibiting critical exploration of
cause and effect in history, establishing its political ideologies as
civil theology, putting a taboo on the instrument of critique. Now he
does not criticize gnosticism as a civil theology of the Soviet
society but wants to expose ``the dangers of gnosticism as a civil
theology of Western society.''\footnote{L.c., p. 173.} Can he, under
his responsibility as a truth-loving scholar, maintain that there is
anything in the political system of the Western democracies that can
be compared with that system which he calls the civil theology of the
totalitarian states? If the most radical contrasts, such as the
contrast between the liberal-democratic regimes of the United States,
Great Britain, France, on the one hand, and the
totalitarian-autocratic regime of the Nazi and Soviet states, can be
comprised in the concept of gnosticism, this concept is an empty
shell, and the combination of the term ``gnostic'' with ``politics,''
``revolution,'' ``civilization,'' and the like does not add any
meaning to the second part of the combination. Still more serious than
this misuse of terminology is the tendency behind it: to obscure the
difference between antagonistic regimes. Voegelin \label{M122}says that the
``nonrecognition of reality'' in the gnostic dream world has as a
consequence that ``types of action which in the real world would be
considered as morally insane because of the real effects which they
have will be considered moral in the dream world because they intended
an entirely different effect. The gap between intended and real effect
will be imputed not to the Gnostic immorality of ignoring the
structure of reality but to the immorality of some other person or
society that does not behave as it should behave according to the
dream conception of cause and effect. The interpretation of moral
insanity as morality, and of the virtues of \emph{sophia} and
\emph{prudentia} as immorality, is a confusion difficult to
unravel. And the task is not facilitated by the readiness of the
dreamers to stigmatize the attempt at critical clarification as an
immoral enterprise.''\footnote{L.c., pp. 169f.} The reader might think
that these statements refer to the gnostic dream world of the
totalitarian states. But the following passage shows that Voegelin is
speaking of the ``liberals'' of Western civilization: ``As a matter of
fact, practically every great political thinker who recognized the
structure of reality, from Machiavelli to the present, has been
branded as immoralist by Gnostic intellectuals -- to say nothing of the
parlor game, so much beloved among liberals, of panning Plato and
Aristotle as Fascists''.\footnote{L.c., p. 170.} It is not by Nazi
gnostics that Machiavelli -- whose \emph{principe} is according to
Voegelin the gnostic symbol of the leader -- is branded as immoral, and
Plato as a Fascist; and it is not the Nazi or Soviet theory that
Voegelin accuses of establishing a ``continuous Gnostic barrage of
vituperation against political science in the critical
sense.''\footnote{Ibid.} Does he seriously maintain that outside the
totalitarian states there is a barrier against any kind of criticism
in the field of political science and does his statement imply that in
the Western democracies a ``political science in the critical sense''
- which he evidently identifies with an anti-positivistic and
anti-relativistic school of thought -- is not allowed to develop
freely?  Can he seriously deny that this science, of which his
\label{M123}new science of politics is a significant product, is no less, and
even more \emph{en vogue} in Western society than its opponent, and
that to stigmatize its opponents as destructive, is not a
vituperation? By publishing his anti-gnostic book in a gnostic society
of the West he makes ample use of the essential difference which
exists between this society and the gnostic society of the East, which
difference he tries to eclipse by calling both ``gnostic''. As a
``manifestation of gnostic insanity in the practice of contemporary
politics'' of ``Western society'', Voegelin points to the attitude
toward the National Socialist movement as to ``the Gnostic chorus
wailing its moral indignation at such barbarian and reactionary doings
in a progressive world -- without however raising a finger to repress
the rising force by a minor political effort in proper
time.''\footnote{L.c., pp. 171ff.} Has he forgotten that by the
gnostic insanity of the Western societies the Nazi movement has been
destroyed after a very short existence? ``Gnostic politicians,'' he
says with indignation, ``have put the Soviet army on the Elbe,
surrendered China to the Communists, at the same time demilitarized
Germany and Japan, and in addition demobilized our own
army.''\footnote{L.c., p. 172.} That means that the new science of
politics disapproves of the policy of the Roosevelt and Truman
administrations, and thus shows that it is indeed a ``political''
science, not a science of politics. Voegelin considers the ``evolution
of mankind toward peace and world order'' as ``mysterious'' and does
not believe in ``the possibility of establishing an international
order in the abstract without relation to the structure of the field
of existential forces.''\footnote{Ibid.} He is against disarmament as
based on the erroneous view that armies are ``the cause of war and not
the forces and constellations which build them and set them into
motion''\footnote{Ibid.}. It would be easy to show that the democratic
politicians and liberal intellectuals he tries to ridicule as gnostic
dreamers do not believe in a predestined evolution of mankind, but try
to bring about peace and world order, without having any illusion
about the difficulties involved in such policy, because they are fully
aware of the forces operating against it, \label{M124}that they suggest
disarmament or reduction of armament not because they are so stupid to
assume that armies are ``the'' cause of war, but because they know --
what nobody can deny -- that the existence of an army makes war
possible and that the militaristic mentality which is inevitably
connected with the maintenance of armies is an additional motive for
a policy which uses war as its instrument. Voegelin cannot obscure the
fact that he distorts the ideas of his political opponents by
referring to them as ``gnostics.''  This term, applied to the politics
of Western society in general and to the Democratic Party of the
United States and the liberal intellectuals in particular, is degraded
to an invective. Its true meaning becomes evident when he deals with
the thesis ``that Western society is ripe to fall for
communism.''\footnote{L.c., p. 176.} This pessimistic opinion -- as
Voegelin certainly knows -- has been pronounced by the great economist
Joseph Schumpeter whose anti-communist attitude is beyond
doubt.\footnote{C.f. Joseph A. Schumpeter, \emph{Capitalism, Socialism
and Democracy}, 1942, and the same author, paper: ``The March into
Socialism'', American Economic Review, Vol. XL (No. 2), 1950, pp. 446ff.} 
Without referring to this author, Voegelin says of the opinion
advocated by him that it ``is an impertinent piece of Gnostic
propaganda''\footnote{Voegelin, l.c., p. 176.}, that means: communist
propaganda. Now we understand what it really means when, on the basis
of the new science of politics, the Roosevelt and Truman policy, the
movement for peace, world order and disarmament are called gnostic. It
means exactly the same as what is meant when on the lowest level of
propaganda those who do not conform with one's own opinion are smeared
as communists.

One of the main concerns of Voegelin's theory of modernity as gnosticism
is to accuse not only the Soviet state but also the alleged civil
theology of Western societies of having destroyed ``the truth of the
soul'', and to condemn the policy of these democracies as ``Gnostic
insanity''\footnote{L.c., p. 170.}. Among the Western societies ``the
American and English democracies'' represent, according to his own
testimony, ``the oldest, most firmly consolidated stratum of civilized
tradition.''\footnote{L.c., p. 170f.} \label{M125}England and America are
certainly the most prominent, the most representative among the Western
societies; and if the Western societies are guilty of gnosticism, this
accusation must be directed in the first place at these two democracies.
As a matter of fact it has been directed expressly at least against the
United States under the Roosevelt and Truman administration. But,
finally, Voegelin admits: ``Of the major European political societies,
however, England has proved herself most resistant against Gnostic
totalitarianism; and the same must be said for the America that was
founded by the very Puritans who aroused the fears of
Hobbes.''\footnote{L.c., p. 187.} In the catastrophic situation into
which gnosticism with its destruction of the truth of the soul has
driven modern civilization, he sees -- at the end of his book -- ``a
glimmer of hope, for the American and English democracies which most
solidly in their institutions represent the truth of the soul are, at
the same time, existentially the strongest powers.''\footnote{L.c., p.
  189.}  This is the -- quite contradictory -- truth of gnosticism as
to the nature of modernity. It is in the end Voegelin's gnostic dream.

\selectlanguage{\ngerman}









%%% Local Variables: 
%%% mode: latex
%%% TeX-master: "Kelsen_A_New_Science_of_Politics"
%%% End: 
