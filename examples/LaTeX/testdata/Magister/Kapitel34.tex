
%%% Local Variables: 
%%% mode: latex
%%% TeX-master: "Main"
%%% End: 

\subsection{Die Stufen des Ordnungswissens}

Im vierten, mit "`Die Spannungen in der Wissensrealität"'\footnote{Voegelin,
  Anamnesis, S.323.} betitelten Abschnitt seines Aufsatzes untersucht Voegelin
die Beziehung zwischen den verschiedenen Stufen des Ordnungswissens, wozu die
pränoetische, die noetische und die Verfallsstufe des Wissens von der
richtigen Ordnung zählen.

Nachdem Voegelin in einer kurzen Einleitung noch einmal auf die Weisen des
Ordnungswissens und ihre gegenseitige Beziehung im Prozess der noetischen
Differenzierung eingegangen ist, stellt er zunächst in groben Zügen die
historische Entwicklung von prä-noetischem über das noetische Ordnungswissen
bis hin zur Entgleisung des Ordnungswissens in der Gegenwart dar. Daraufhin
erörtert Voegelin ausführlich, wie seiner Ansicht nach das verlorengegangene
Ordnungswissen in der Gegenwart wieder hergestellt werden kann. Als Letztes
geht Voegelin mit dem mystischen Denken Jean Bodins und Henri Bergsons auf
zwei mögliche historische Anknüpfungspunkte jüngeren Datums zur
Wiedergewinnung des Ordnungswissens ein.

Wie bereits ausgeführt, tritt Ordnungswissen Voegelin zufolge zunächst in
einer prä-noetischen, mythischen Form auf. Durch die Noese wird das
prä-noetische Ordnungswissen ergänzt und vertieft. Erst auf der Stufe des
noetischen Ordnungswissens wird sich der Mensch seiner Existenz in der
"`Spannung zum Grund"' bewußt, so daß das Ordnungswissen nun einer expliziten
Kontrolle aus dem Wissen um die "`Spannung zum Grund"' heraus unterliegt
("`explizit-rationale Kontrolle"'\footnote{Voegelin, Anamnesis, S.325. - Es
  ist ziemlich wahrscheinlich, daß Voegelin in dieser Passage (S.323-325.) den
  Ausdruck "`rational"' doppeldeutig gebraucht: einmal im Sinne seiner
  Definition von "Ratio"' als "`der Spannung des Bewußtseins zum Grund"'
  (S.289.), dann aber auch in dem - dem gewöhnlichen Wortgebrauch
  näherkommenden - Sinn von "`begriffliches Wissen"' bzw. "`begriffliches
  Denken"'.}). Unter Rückgriff auf das bewußt gewordene Wissen von der
"`Spannung zum Grund"' ist es zwar möglich, das prä-noetische oder auch das
entgleiste Ordnungswissen in Frage zu stellen. Dennoch bleibt das noetische
Ordnungswissen auf die prä-noetischen Wissensbestände, die es erweitert aber
nicht ersetzt, sachlich angewiesen. Auch auf der gesellschaftlichen Ebene, wo
sich das noetische zum kompakten Ordnungswissen ähnlich wie die Theologie zum
Volksglauben verhält, kann es das prä-noetische Wissen niemals gänzlich
verdrängen.\footnote{Vgl. Voegelin, Anamnesis, S.323-325.}

Die historische Entwicklung des Ordnungswissens, die Voegelin im folgenden
skizziert, kann als eine Bewegung in dialektischen Dreischritten gedeutet
werden: Auf eine Phase heilen Ordnungswissens prä-noetischer oder noetischer
Art folgt als deren Antithese eine dogmatische Entgleisung oder, wie Voegelin
auch sagt, eine "`Parekbasis"' falschen Ordnungswissens, welche dann durch die
Noese aufgehoben wird. Aber auch das noetische Ordnungswissen entgleist zum
Dogmatismus, so daß eine weitere Noese - denn ein höheres als das noetische
Ordnungswissen gibt es nicht - vonnöten ist, die wiederum diese Entgleisung 
aufhebt.

In dieser Weise folgt nach Voegelins Geschichtsbild auf die griechische
Mythologie und die "`Parekbasis der Sophistik"' die "`klassische
Noese"'\footnote{Voegelin, Anamnesis, S.325.} der nach-sokratischen
Philosophie. Nicht zuletzt, weil sie politisch mit der Polisgesellschaft für
das in den Eroberungszügen Alexanders des Großen unterlegene Modell optierte,
war der klassischen Noese kein langfristiger Erfolg beschieden, und sie
entgleiste ihrerseits "`zur philosophischen Dogmatik der
Schulen"'.\footnote{Voegelin, Anamnesis, S.326.} Obwohl zur "`Dogmatik der
Schulen"' herabgekommen, taugte, wenn man Voegelins Worten Glauben schenken
darf, das als "`Parekbasis einer Noese ...  charakteriesierte
Phänomen"'\footnote{Voegelin, Anamnesis, S.326.} dennoch dazu, gegenüber der
jüdisch-christlichen Offenbarungsreligion als "`Repräsentant der
Noese"'\footnote{Voegelin, Anamnesis, S.326.} zu fungieren. Da im
jüdisch-christlichen Kontext der Übergang von der vor-noetischen
Offenbarungsweisheit zum noetischen Ordnungswissen weniger schroff verlief,
kam es dabei nicht wie im antiken Griechenland zum radikalen Bruch mit dem
traditionellen Ordnungswissen. Vielmehr verschmolz die Philosophie mit dem
traditionellen Ordnungswissen der Offenbarung zur Theologie. In dieser Form
hat die Noese zwar bei der Bekämpfung der Häresien (zur der Voegelin schon in
früheren Schriften der katholischen Kirche das volle historische Recht
zuerkennt\footnote{Vgl. Eric Voegelin: Das Volk Gottes. Sektenbewegungen
  und der Geist der Moderne (Hrsg. von Peter J. Opitz), München 1994, S.31-34.
  - Voegelin setzt in dieser Schrift recht kritiklos voraus, daß die
  katholische Kirche objektiv die Wahrheit des Geistes verkörpert. Eine
  legitime Auflehnung gegen die katholische Kirche ist dann kaum noch denkbar.
  Allerdings hätte die katholische Kirche seiner Ansicht nach besser daran
  getan, sich die Häresien einzuverleiben, als sie gewaltsam zu
  bekämpfen.}) Großes geleistet, aber zugleich stand sie der Entwicklung der
Naturwissenschaften und einer von der religiösen Orthodoxie unabhängigen
Geschichtsdeutung ("`Freilegung der Realitätsbereiche von Welt und
Geschichte"'\footnote{Voegelin, Anamnesis, S.327.}) im Wege. Dadurch hat sie
nicht wenig dazu beigetragen, jene Revolte gegen den Grund heraufzubeschwören,
welche in Voegelins Augen das charakteristische Merkmal der Neuzeit ist. Da
bisher die neuzeitliche Auseinandersetzung mit der Theologie als "`Spiel von
dogmatischer Position und Opposition"'\footnote{Voegelin, Anamnesis, S.327.}
und somit auf einer reinen Parekbasis ohne mystischen Erfahrungsgehalt
stattgefunden hat, ist nun der nächste historisch-dialektische Schritt zu
leisten, durch welchen das noetische Ordnungswissen wiederhergestellt wird.

Die Wiederherstellung des noetischen Ordnungswissens gestaltet sich
deshalb so schwierig, weil sie nicht auf der Ebene der Dogmatik in den
Streit mit der "`Revolte gegen den Grund"' eintreten darf, sondern sich
durch die Neu-Erschließung des Bewußtseins als dem Ordnungszentrum über
alle Dogmatik erheben muß. Zur "`Revolte gegen den Grund"' rechnet
Voegelin die verschiedensten philosophisch-weltanschaulichen Strömungen
der Gegenwart. Wörtlich zählt er "`die Ideologien des Positivismus,
Marxismus, Historismus, Szientismus, Behaviorismus, ..  Psychologisieren
und Soziologisieren, .. welt-intentionale Methodologien und
Phänomenologien"'\footnote{Voegelin, Anamnesis, S.328.} auf.  Alle diese
geistigen Strömungen dienen seiner Ansicht nach vor allem einem Zweck:
Das Empfinden für die existenzielle Spannung zum Grund im eigenen
Inneren abzutöten, und durch die Entwicklung einer "`Obsessivsprache"'
zu verhindern, daß die Frage nach dem Grund überhaupt aufkommen kann.
Nach Voegelins Überzeugung gibt es in diesen geistigen Strömungen
nichts, woran es sich im Interesse des noetischen Ordnungsdenkens lohnt
anzuknüpfen. Immerhin räumt Voegelin später ein, daß der
"`ideologische[n] Revolte"' historisch gesehen eine wertvolle Funktion
im Kampf gegen den "`Sozialterror der [theologischen]
Orthodoxie"'\footnote{Voegelin, Anamnesis, S.329.} zukam.

Die Abkehr von der "`Revolte gegen den Grund"' wird nach Voegelins
Einschätzung bislang überwiegend von diversen Traditionalismen und
Konservativismen getragen. In einem weiteren großen Rundumschlag zählt
Voegelin hier einen ganzen Reigen von rechts-liberalen bis konservativen
Erneuerungsbewegungen seiner Zeit auf. Alle diese Erneuerungsbewegungen leiden
nach Voegelins Ansicht allerdings darunter, daß sie versuchen, die "`Revolte
gegen den Grund"' auf der gleichen Ebene zu bekämpfen, weshalb Voegelin sie
auch als "`Sekundärideologien"' (zu den Primärideologien der "`Revolte gegen
den Grund"') bezeichnet. Ihr oberflächliches Vorgehen führt dazu, daß die
Sekundärideologien lediglich zu älteren Dogmatiken zurückkehren. Aber nicht
zuletzt deshalb, weil die Revolte sich teilweise zu Recht gegen die älteren
Dogmatiken aufgelehnt hat, darf die Umkehr nicht bloß eine Rückkehr sein,
sondern sie muß zur "`Wissensrealität"' der bewußt gemachten existentiellen
"`Spannung zum Grund"' vordringen.\footnote{Vgl. Voegelin, Anamnesis, S.329.}

Woran soll man aber dann anknüpfen, wenn der Rückbezug auf die Tradition nur
wieder in eine "`Sekundärideologie"' mündet? Eine Möglichkeit besteht für
Voegelin darin, sich am Vorgehen des schon einmal als Vorbild
angeführten Albert Camus zu orientieren, der auf die griechische Mythologie
zurückgeht.\footnote{Vgl. Voegelin, Anamnesis, S.312-313.} Nach Voegelins
Interpretation bedient sich Camus deshalb des griechischen Mythos, weil im "`
`Blödsinn' der Zeit .. keine Heimat für den Menschen"'\footnote{Voegelin,
  Anamnesis, S.330.} zu finden ist. Camus' Lebensweg, der sich in Voegelins
Deutung ein wenig wie das Gleichnis vom verlorenen Sohn ausnimmt, führt
beispielhaft die Entwicklung von der Revolte "`gegen die Präsenz des Lebens in
der Spannung zum göttlichen Grund"'\footnote{Voegelin, Anamnesis, S.330.} bis
zur demütigen Wiedereinkehr in ein Leben, "`geordnet durch die liebende
Spannung der Existenz zum göttlichen Grund"'\footnote{Voegelin, Anamnesis,
  S.313.} vor. Zwar hat Camus das letzte Stadium nicht mehr erreichen können
(da er vorher durch einen Autounfall ums Leben kam), aber Voegelin glaubt
dennoch den Tagebüchern von Camus mit einiger Sicherheit entnehmen zu
können, welchen Fortgang dessen geistige Entwicklung genommen
hätte.\footnote{Vgl. Voegelin, Anamnesis, S.312., S.330.}

Nicht weniger schwierig als die individuelle Umkehr stellt sich das Problem
dar, wie die Politische Wissenschaft wieder auf den richtigen Weg gebracht
werden kann. Von der Politischen Wissenschaft, wie sie an den Universitäten
überwiegend betrieben wird, ist nach Voegelins Auffassung wenig zu erwarten,
da diese Politische Wissenschaft sich vorwiegend mit den verschiedenen
politischen Institutionen der Gegenwart beschäftigt, welche nach Voegelins
Ansicht sämtlich vom falschen Ordnungsverständnis durchseucht sind. Von
welcher Seite dürfen dann aber Antworten auf die Fundamentalfragen politischer
Ordnung erhofft werden? Voegelin glaubt, daß Hinweise zur Beantwortung dieser
Fragen noch am ehesten in der schönen Literatur bei Autoren wie z.B. Robert
Musil und Hermann Broch oder, als Alternative und zur Ergänzung der schönen
Literatur, in den Altertumswissenschaften zu finden sind, da sich die
Altertumswissenschaften mit den noch unverdorbenen prä-dogmatischen
Wissensrealitäten beschäftigen. Voegelin vertritt die Ansicht, daß die
wissenschaftliche Untersuchung prä-dogmatischer Ordnungssymbole "`eine
Bewegung auf die Noese hin"'\footnote{Voegelin, Anamnesis, S.332.}
herbeiführt. Die florierenden Altertumswissenschaften künden daher nach seiner
optimistischen Überzeugung von einer Gegenbewegung historischen Ausmaßes gegen
die dogmatischen Scheinrealitäten.  Voegelin räumt allerdings
ein, daß "`die Bewegung noch nicht bewußt zentriert ist."'\footnote{Voegelin,
  Anamnesis, S.332.} Nichts desto trotz erwartet Voegelin für die Zukunft
bedeutsame "`Durchbrüche"' zur Noese, wenn er auch verständlicherweise keine
genauen Vorhersagen darüber riskiert, wann genau und wo sich diese
Durchbrüche ereignen werden.

Zu guter Letzt geht Voegelin noch auf die wichtigsten historischen Residuen
der Noese ein. Diese sind seiner Ansicht nach im klassischen Altertum und in
der mittelalterlichen und neuzeitlichen Mystik zu finden.\footnote{Vgl.
  Voegelin, Anamnesis, S.333.} Hinsichtlich der klassischen Noese entwickelt
Voegelin nur noch einmal die Genese ihres Mißverständnisses als dogmatische
Metaphysik, ohne die klassische Noese selbst ein weiteres Mal zu behandeln.
Als bedeutsame Vertreter der Mystik erwähnt Voegelin Jean Bodin und Henri
Bergson, wobei Voegelin jedoch nur auf Bodin ausführlich eingeht.

Das Verständnis der klassischen Philosophie wird nach Voegelins Ansicht
erheblich durch die anti-metaphysische Einstellung der neuzeitlichen
Philosophie getrübt. Die Anwendung der neuzeitlichen Metaphysikkritik
auf die klassische Philosophie beruht jedoch auf einem Mißverständnis.
Nicht nur wurde das unter dem Titel "`Metaphysik"' bekannte Werk des
Aristoteles erst sehr viel später so genannt,\footnote{Der Titel
  "`Metaphysik"' stammt von Andronikus von Rhodos, der im 1.Jh. v. Chr.
  die Werke des Aristotels herausgab. Er betitelte das Werk, in dem
  Aristoteles seine prima philosophia entwickelt, als "`Metaphysik"',
  weil es auf das Buch der "`Physik"' folgte.  Vgl. den Artikel über
  Metaphysik von Th. Kobusch in: Joachim Ritter / Karlfried Gründer:
  Historisches Wörterbuch der Philosophie. Band 5: L-Mn, Basel /
  Stuttgart 1980, S.1186-1279 (S.1188).}  auch die Entwicklung des
Terminus "`Metaphysik"' als eines philosophischen Fachbegriffes findet,
Voegelins Darstellung zufolge, erst im Mittelalter statt. Den
bedeutsamsten Beitrag zum dogmatischen Mißverständnis der klassischen
Philosophie hat Thomas von Aquin geleistet, der, obwohl er nach
Voegelins Urteil ebenfalls von größter Offenheit der Seele war, in
Aristoteles' "`Metaphysik"' offenbar nicht die Noese eines Ko-Noetikers
erkannte\footnote{Dies ist meine Interpretation.  Voegelin geht nicht
  darauf ein, wie es zu diesem erstaunlichen Mißverständnis kommen
  konnte, wo doch Thomas von Aquin erstens ein großer Kenner der
  aristotelischen Philosophie und zweitens, so wie Voegelin ihn sonst
  beurteilt, ein Denker von der gleichen noetischen Höhe wie Aristoteles
  war.} und sie rein dogmatisch als "`eine Wissenschaft von {\it primae
  causae}, von {\it principia maxime universalia}, und von Substanzen
{\it quae sunt maxime a materia speratae}"'\footnote{Voegelin,
  Anamnesis, S.333.} verstand, wofür Thomas von Aquin von Voegelin,
ganz entgegen dessen sonstiger Hochschätzung für diesen
mittelalterlichen Gelehrten, scharf kritisiert wird. Nachdem die
Metaphysik einmal in dieser Form mißverstanden worden war, wurde sie in
der Folge als rein dogmatisches Geschäft weiterbetrieben. Es ist daher
nach Voegelins Ansicht den aufgeklärten Kritikern der Metaphysik auch
gar kein Vorwurf dafür zu machen, daß sie diese Form der Metaphysik
angreifen. Nur muß bei der Rezeption der Metaphysikkritik darauf
geachtet werden, daß sich die Metaphysikkritik des 18.Jahrhunderts
lediglich gegen die dogmatische Metaphysik richten kann nicht aber gegen
die klassische Philosophie. So hatte Kant in erster Linie die Metaphysik
Christian Wolfs im Visier, weshalb, wie Voegelin nahelegt, die
klassische Philosophie von Kant eher irrtümlicherweise vernachlässigt
wird und von seiner Metaphysikkritik in Wirklichkeit unberührt
bleibt.\footnote{Vgl. Voegelin, Anamnesis, S.335.}

Die mystische Religionsauffassung Jean Bodins geht, Voegelins Deutung zufolge,
auf die Neoplatonisten der Renaissance und auf Dionysius Areopagita,
insbesondere auf dessen Begriff der {\it conversio}, der Hinwendung zu Gott
zurück. Im Kern besagt Bodins mystische Religionsauffassung, daß die wahre
Religion nicht irgendeine bestimmte religiöse Lehre sei, sondern allein in der
Hinwendung zu Gott bestehe. Die Mystik kommt nach Voegelins Interpretation bei
Bodin auf zweierlei Weise zum Tragen. Zum einen erlaubt sie ihm, die
Geschichte zu verstehen. Zum anderen bildet die mystische Religionsauffassung
den Ausganspunkt für Bodins Bekenntnis zur Toleranz. Nach dem Geschichtsbild,
das Voegelin in Bodins "`Lettre à Jean Bautru"' ausgesprochen findet,
verläuft die Geschichte einzelner Gesellschaften zyklisch, indem zunächst ein
von Gott auserwählter Prophet sein Volk durch die Vermittlung von
Offenbarungswissen zu neuen Höhen geistiger Wahrheit führt, worauf jedoch,
sofern sich das störrische Volk für die prophetische Botschaft überhaupt
empfänglich gezeigt hat, im Laufe der Zeit die wahre Religion, die es gelehrt
wurde, mehr und mehr zu einem Buchstabenglauben erstarrt. Ein ganz analoges
Geschichtsbild findet Voegelin in Bergsons "`Deux Sources de la Morale et de
la Religion"' wieder.\footnote{Vgl. Voegelin, Anamnesis, S.336-337.} Freilich
darf bezweifelt werden, ob jenes ein wenig romantische Geschichtsbild, nach
welchem das Schicksal eines Volkes oder einer Zivilisation im wesentlichen von
der Frische des religiösen Glaubens und der Führung durch erleuchtete
Propheten abhängt, wirklich ein starkes Argument für die
erkenntnisaufschließende Kraft der Noese darstellt.

Überzeugender wirken dagegen die Konsequenzen, die sich aus der
mystischen Religionsauffassung für die Toleranz ergeben. Für Bodin
stellt nach der Interpretation, die Voegelin aus dem "`Lettre à Jean
Bautru"' in Verbindung mit Bodins "`Colloquium Heptaplomeres"' gewinnt,
der Rückgang auf die Mystik eine Möglichkeit dar, religiöse Toleranz zu
begründen. Die Wahrheit der Religion liegt in der mystischen Hinwendung
zu Gott selbst. Die göttliche Wahrheit als solche ist unaussprechlich,
und jeder sprachliche Ausdruck dieser Wahrheit trägt lediglich
behelfsmäßigen Charakter. Da es unsinnig ist, sich im Namen religiöser
Dogmen, die bloß sprachliche Notbehelfe sind, zu bekämpfen, ergibt sich
aus der Unaussprechlichkeit der Wahrheit Gottes das Gebot der
Toleranz.\footnote{Vgl. Voegelin, Anamnesis, S.337. - Voegelins Text
  bedarf hier ein wenig der Interpretation, da nicht ganz deutlich wird,
  wie sich das "`Wesen der Toleranz"' aus "`einer Balance zwischen den
  Bereichen des Schweigens und des Ausdrucks in der Wissensrealität"'
  ergibt.} Voegelin führt Bodins Gedanken des Unfaßbaren (das
"`Ineffabile"') noch weiter aus, wobei er eine Übereinstimmung mit
Thomas von Aquins {\it tetragrammaton} (nach Thomas der höchste und
umfassendste Gottesname) und mit seiner eigenen Vorstellung von der
"`existenziellen Spannung zum Grund"' feststellt.\footnote{Vgl.
  Voegelin, Anamnesis, S.337-338.} Dieses Ergebnis verblüfft ein wenig,
denn im Hinblick auf Voegelins im ersten Abschnitt seines Aufsatzes
entwickelten Realitätsbegriff erscheint die Deckungsgleichtheit von
Voegelins und Bodins Vorstellung der mystischen Realität eher
fragwürdig: Während sich bei Bodin (nach Voegelins eigener Deutung) das
"`Ineffabile"' als das eigentlich Wahre von seinem sprachlichen Ausdruck
als einem nur unbeholfenen Hinweis deutlich unterscheidet, hat Voegelin
zuvor noch mit größtem Nachdruck darauf bestanden, daß die Symbole, die
die Realität ausdrücken, den unabtrennbaren Teil eines
Gesamtzusammenhanges von Realität bilden, der die Termini des
Partizipierens, das Partizipieren selbst und ausdrücklich auch die
Symbole umfaßt.\footnote{Vgl. Voegelin, Anamnesis, S.305-307.}

Weiterhin führt Voegelin aus, daß die "`Einsicht in das Wissen vom Ineffabile
.. erhebliche Bedeutung für das Verständnis einer großen Klasse von
Ordnungsphänomenen"'\footnote{Voegelin, Anamnesis, S.338.} habe, die über das
"`Wesen der Toleranz"' noch hinausgehen. So sei es für die kompakte Erfahrung
des "`Ineffabilen"' charakteristisch, daß die symbolische Ausdrucksform
Sakralcharakter gewinne, so daß weitere Differenzierungen der Erfahrungen nur
noch als Kommentare zum Sakraltext auftreten könnten. Inwiefern die
Feststellung dieses "`Ordnungsphänomens"' aus der "`Einsicht in das Wissen vom
Ineffabile"' fließt, bleibt ein Rätsel, zumal Voegelin schon
wenig später feststellt, daß es auch im Bereich der Ideologie das Phänomen
ideologischer Klassiker mit "`anschließender kommentatorischer und
apologetischer Literatur"' gibt.\footnote{Vgl. Voegelin, S.338-340.}

\subsection{Kritik von Voegelins Bodin- und Camus-Deu\-tung}

Da dies bereits im ersten Teil dieser Arbeit geschehen ist, erübrigt es sich,
an dieser Stelle noch einmal ausführlich auf Voegelins Geschichtsbild und
seine polemische Zeitkritik einzugehen. Auch Voegelins Empfehlung an die
Politikwissenschaft, das Wissen von politischer Ordnung lieber durch die
Lektüre der modernen Klassiker der schönen Literatur oder durch Vertiefung in
das Altertum zu erweitern als durch das Studium der politischen Institutionen
der Gegenwart, soll an dieser Stelle nicht weiter nachgegangen werden.
Ohnenhin ist der Eintritt des von Voegelin prophezeiten großen noetischen
Durchbruchs in den dreißig Jahren, die seit der Publikation seines Aufsatzes
verflossen sind, bisher ausgeblieben, so daß diese Empfehlungen, die in
Antizipation eines solchen Durchbruches ausgesprochen wurden, nicht
mehr allzu aktuell sind.

Ebenfalls keine besonders weittragende Bedeutung kommt Voegelins These zu, daß
die Metaphysikkritik der aufklärerischen Philosophie auf einem Mißverständnis
beruht, soweit sie sich nicht nur gegen die neuzeitliche Metaphysik richtet,
sondern auch gegen die klassische Philosophie gewendet wird. Es ist kaum
anzunehmen, daß die antike Philosophie, wenn sie, wie Voegelin dies fordert,
als Kryptomystik interpretiert worden wäre, in der aufklärerischen Kritik
besser abgeschnitten hätte als die Leibniz-Wolffsche Metaphysik oder die
Einsichten des "`Geistersehers"' Swedenborg.\footnote{Vgl. Immanuel Kant:
  Träume eines Geistersehers, erläutert durch Träume der Metaphysik, in:
  Frank-Peter Hansen (Hrsg.): Philosophie von Platon bis Nietzsche, CD-ROM,
  Berlin 1998, S.23599ff. / S.70ff. (Zweiter Teil, Zweites Hauptstück:
  Ekstatische Reise eines Schwärmers durch die Geisterwelt.) (Konkordanz:
  Immanuel Kant: Werke in zwölf Bänden. Herausgegeben von Wilhelm Weischedel.
  Frankfurt am Main 1977. Band 2, S.970ff.). - Daß die
  aufklärerisch-positivistische Religionskritik nicht notwendigerweise auf
  dogmatischen Mißverständnissen beruht und daher den Hinweis auf die
  religiöse Erfahrung keineswegs zu fürchten braucht, verdeutlicht Ayers
  erkenntnistheoretische Kritik der Berufung auf die religiöse Erfahrung. Vgl.
  Alfred J. Ayer: Language, Truth and Logic, New York [u.a.]  1982, S.157-158.
  - Zur Ersetzung der Religion durch die reine Religiösität in der modernen
  Religionsapologetik seit Schleiermacher: Vgl. Hans Albert: Kritischer
  Rationalismus. Vier Kapitel zur Kritik illusionären Denkens, Tübingen 2000,
  S.147ff.}

Lohnender erscheint es, auf Voegelins Deutung der Vorbilder Camus und Bodin
einzugehen. Führt Camus' Lebensweg tatsächlich jene Entwicklung von der
Auflehnung gegen Gott bis zur demütigen Unterwerfung unter Gott vor, die
Voegelin mit solchem Entzücken registriert? Liefert Bodins mystische
Religiosität wirklich eine optimale Begründung der Toleranz?

Legt man für die Interpretation von Camus' geistiger Entwicklung seine beiden
großen Essays "`Der Mythos von Sisyphos"'\footnote{Albert Camus: Der Mythos
  von Sisyphos. Ein Versuch über das Absurde, Hamburg 1998 (zuerst 1942), im
  folgenden zitiert als: Camus, Mythos von Sisyphos.} und der "`Mensch in der
Revolte"'\footnote{Albert Camus: Der Mensch in der Revolte. Essays, Hamburg
  1997 (zuerst 1951), im folgenden zitiert als: Camus, Mensch in der Revolte.}
zu Grunde, so ergibt sich das Bild einer Entwicklung, die weniger dramatisch
verläuft, als sie bei Voegelin erscheint. Die metaphysische Auflehnung, von
der der "`Mythos von Sisiphos"' handelt, richtet sich gegen die Absurdität der
Welt, sie richtet sich nicht primär gegen Gott, und die Absurdität ist auch
kein Resultat der Abwesendheit Gottes, die durch den Glauben behoben werden
könnte. Gott ist ebenso eine Lüge wie all die falschen Tröstungen, bei denen
die verschiedenen existentialistischen Philosophien am Ende doch wieder
herauskommen.\footnote{Vgl. Camus, Mythos von Sisyphos, S.39ff.} Der Glaube,
so könnte man sagen, ist ein Beruhigungsmittel für den Geist, welches der am
Leben Leidende stolz verschmäht. Nicht umsonst erwählt Camus den Empörer
Sisyphus zum Helden seines Essays.

Welche Entwicklung findet nun im Übergang zu "`Der Mensch in der Revolte
statt"'? Der wesentliche Unterschied zwischen diesen beiden Essays
besteht darin, daß der "`Mythos von Sisyphos"' ein rein metaphysisches
Thema behandelt: den Menschen in der Auseinandersetzung mit der absurden
Welt. "`Der Mensch in der Revolte"' erweitert diese Thematik ins
Politische, was natürlich Konsequenzen für die Deutung des Absurden nach
sich zieht. Angesichts des philosophisch gerechtfertigten Massenmordes,
wie er sich in den totalitären Staaten vollzog, war Camus seine
Philosophie des Absurden, deren stolzer Inidivualismus auch eine
Gleichgültigkeit gegen die Moral implizierte, offenbar nicht mehr
geheuer.\footnote{Vgl. Camus, Mensch in der Revolte, S.9-18 (Einleitung:
  Das Absurde und der Mord).} Es kommt zwar nicht zu einer Revision aber
zu einer Präzisierung seines früheren Standpunktes. Zu der Auflehnung
tritt das Bewußtsein der humanen Verantwortlichkeit hinzu. Die
Auflehnung ist nun nicht mehr eine Auflehnung gegen die metaphysische
Unsinnigkeit der Welt, die sich aus Kontingenzerfahrungen speist,
sondern sie wandelt sich zur Revolte gegen die Unsinnigkeit menschlichen
Leidens, deren Ursprung recht konkrete moralische und politische
Erfahrungen sind.

"`Der Mensch in der Revolte"' stellt daher auch in erster Linie eine
philosophische Auseinandersetzung mit der linksgerichteten Variante des
Totalitarismus und eine scharfe Abrechnung mit dessen intellektuellen
Verherrlichern im Westen dar. Den Kommunismus deutet Camus als Ausfluß
von Nihilismus und Gottesmord.\footnote{Vgl. Camus, Mensch in der
  Revolte, S154ff.} In diesen Punkten berührt sich Camus' Deutung am
stärksten mit der Theorie Voegelins und anderen christlichen-religiösen
Erklärungen des Totalitarismus. Allerdings handelt es sich dabei um eine
der weniger überzeugenden Passagen von Camus' Essay. Kein Kommunist oder
Sozialist muß sich getroffen fühlen, wenn Camus der kommunistischen
Revolution eine Genealogie von Nihilisten und Dandys unterschiebt. Der
Kommunismus war kein Nihilismus. Er trat mit handfesten materiellen
Glücksversprechungen an, und es können kaum die Motive einer exklusiven
künstlerischen Bohème gewesen sein, die ihm seine Massenunterstützung
sicherten. Hier entsteht bei Camus ein politisches Weltbild, das
zusammengesetzt ist aus literarischen Referenzen - ähnlich, wie es nicht
selten bei Voegelin geschieht. Trotz dieser Schwächen behält Camus in
der Sache recht. Im Jahre 1951 konnte man unmöglich noch den Kommunismus
(und gar den Kommunismus stalinscher Prägung) unterstützen, der seinen
Kredit durch das Scheitern seiner Prophezeiung wie durch seine
Verbrechen längst verspielt hatte, auch wenn es noch nicht jeder
wahrhaben wollte.\footnote{Vgl. Camus, Mensch in der Revolte, S.238ff. -
  Als Beispiel einer sehr weitgehenden intellektuellen Apologie des
  Kommunismus aus dieser Zeit: Vgl. Maurice Merlau-Ponty: Humanismus und
  Terror, Frankfurt am Main 1990 (entstanden 1946/47), S68ff.}

Zu welchem Ergebnis gelangt nun Camus? Bleibt, wenn der Gottesmord mit
innerer Logik zum Terror führt, als die einzig akzeptable Haltung nur
noch die "`die liebende Spannung der Existenz zum göttlichen
Grund"'\footnote{Voegelin, Anamnesis, S.313.} übrig? War die
metaphysische Auflehnung des "`Mythos von Sysiphus"' nur ein dummer
Jungenstreich eines unreifen Philosophen? Dies läßt sich kaum behaupten.
"`Der Mensch in der Revolte"' schließt mit dem Gegensatz von Revolte und
Revolution. Beide sind Ausdruck einer Auflehnung, aber die Revolte
anerkennt ein Gesetz des "`Maßes"' und damit Grenzen der eigenen
Unfehlbarkeit, des eigenen Rechtes, über andere zu verfügen, und
besonders des eigenen Rechtes, für politische Zwecke Gewalt auszuüben.
Die Revolution dagegen bedeutet die Entgleisung der Revolte in einen
schrankenlosen Terror, der der Anmaßung entspringt, mit einer abstrakten
Philosophie das menschliche Leben in seiner Totalität erfassen zu
können, so daß keine humanen Vorbehalte mehr möglich sind, da durch die
abstrakte Totalität schon alles berücksichtigt ist. Der Gedanke des
Maßes, welcher, repräsentiert durch den Mythos von Nemesis, den dritten
Teil des von Voegelin als Indiz für die Entwicklung Camus'
herangezogenen Werkprogrammes bildet,\footnote{Vgl. Voegelin, Anamnesis,
  S.330. - Vgl.  Albert Camus: Tagebücher 1935-1951, Hamburg 1997,
  S.465.}  verweist zwar auf eine Realität, die sich - ganz im Sinne
Voegelins - nicht uneingeschränkt und nicht ungestraft manipulieren
läßt, aber dieser Gedanke hebt die Revolte nicht auf.\footnote{Vgl.
  Albert Camus: Tagebuch März 1951 - Dezember 1959, Hamburg 1997, S.32.
  Dort schreibt Camus: "`Maß. Sie halten es für die Lösung des
  Widerspruchs. Es kann nichts anderes sein als die Bestätigung des
  Widerspruchs und der heroische Entschluß, sich daran zu halten und ihn
  zu überleben."'} Die Annerkennung des christlichen Gottes, jenes
Herren, dem man "`Ja!"' und "`Danke!"' sagen muß, ganz gleich, welches
Leid den Menschen geschieht, hätte für Camus ebenso die Aufkündigung der
Solidarität mit den Menschen bedeutet, wie das unkritische Lob der
Revolution von Seiten der Intellektuellen.\footnote{Ebd., S.263:
  "`Nemesis.  Wesentliche Komplizität von Marxismus und Christianismus
  (weiterentwickeln).  Deswegen bin ich gegen beide."'} Hier liegt ein
fundamentaler Gegensatz zu Voegelin vor. Während Voegelin die
Menschenliebe ohne die Partizipation an ein und demselben göttlichen
Grund für unmöglich hält,\footnote{Vgl. Eric Voegelin: In Search of the
  Ground, in: Conversations with Eric Voegelin.  (ed. R. Eric O'Connor),
  Montreal 1980, S.1-20 (S.10).}  schließt für Camus gerade die
Menschenliebe die Liebe zum transzendenten Sein aus.

Hinsichtlich Voegelins Bodin-Interpretation ist zunächst einmal festzuhalten,
daß Bodin die religiöse Toleranz fast ausschließlich mit
politisch-pragmatischen Gründen rechtfertigt.  In seinen "`Six Livres de la
République"' rät Bodin von der gewaltsamen Unterdrückung etablierter
Religionen und Sekten ab, da dies kriegerische Auseinandersetzungen
heraufbeschwören kann. Allenfalls durch sein eigenes Vorbild und durch Anreize
soll der Herrscher die Untertanen zum Übertritt zur wahren Religion bewegen,
mit welcher Bodin eine der konkreten Religionen meint, ohne sie jedoch zu
nennen.\footnote{Vgl. Jean Bodin: Sechs Bücher über den Staat. Buch IV-VI.
  (Hrsg. von P.C. Mayer-Tasch), München 1986, S.150-151.} Auch im "`Colloquium
Heptaplomeres"' ändert sich an der wesentlich pragmatischen Begründung der
Toleranz nicht viel. Die Frage der Wahrheit der Religion bleibt im Streit der
Sieben gänzlich unentschieden.  Einigkeit herrscht am Ende der Diskussion
lediglich darüber, daß jeder Mensch in eine religiöse Gemeinschaft eingebunden
sein muß, und daß es das Beste ist, wenn die Existenz jeder bestehenden
Glaubensgemeinschaft geduldet wird, sofern sie dazu bereit ist, den anderen
ihren Glauben nicht streitig zu machen.\footnote{Vgl. Jean Bodin: Colloquium
  of the Seven about Secrets of the Sublime.  Colloquium Heptaplomeres de
  Rerum Sublimium Arcanis Abditis, Princton 1975 (im folgenden zitiert als:
  Bodin, Heptaplomeres), S.473.} Kaum Anhaltspunkte bietet Bodin für
eine Interpretation, nach welcher die Toleranz durch die Differenz zwischen
der unaussprechlichen Wahrheit und dem symbolischen Ausdruck der Religion
begründet wird. Zwar taucht im Gespräch der Sieben einmal der Vorschlag auf,
auf die natürliche Religion als der ursprünglichen Religion zurückzugehen,
doch wird dieser Vorschlag als nicht praktikabel abgelehnt.\footnote{Bodin,
  Heptaplomeres, S.462f.} Mit der natürlichen Religion meinte Bodin dabei den
alt-israelischen Glauben, dem die unterschiedlichen monotheistischen
Glaubensrichtungen entsprungen sind, die bei Bodin so schwer vermittelbar
gegeneinander stehn. Auch hier ist also von einer konkreten Religion die Rede
und nicht von einem mystischen Wahrheitskern als Grundlage aller
Religionen.\footnote{Wollte man den Hinweis Bodins auf die ursprüngliche
  Religion weiter ausbauen, so käme man wohl zu einer deistischen Begründung
  der Toleranz, wie sie im Denken der Aufklärung (z.B.  in Lessings "`Nathan
  der Weise"') zu finden ist. Vgl. Voegelin, Anamnesis, S.380 (Anmerkung 19).}
Daß bei Bodin die Forderung der Toleranz weit eher einer Einsicht in die
politischen Notwendigkeiten als einer tiefen Überzeugung entspringt, geht auch
aus dem Schönheitsfehler hervor, daß Bodin nur die bestehenden Konfessionen
einbezieht und von einer Glaubens- und Gewissensfreiheit im heutigen Sinne bei
ihm keine Rede sein kann.\footnote{Vgl. Georg Roellenbleck: Der Schluß des
  "`Heptaplomeres"' und die Begründung der Toleranz bei Bodin, in: Horst
  Denzer (Hrsg.): Jean Bodin.  Verhandlungen der internationalen Bodin Tagung
  in München, München 1973, S.53-67 (S.66.). - Nach Roellenblecks Deutung ist
  die Toleranzbegründung im "`Heptaplomeres"' allerdings nicht nur Ausdruck
  politischen Kalküls, sondern sie steht auch in Zusammenhang mit einem
  tieferen Harmoniegedanken. Dennoch fällt es angesichts der Grenzen von
  Bodins Toleranzbegriff und angesichts der rein politisch begründeten
  Toleranz in den "`Six Livres de la République"' schwer, hierin mehr als
  nur eine nachträgliche ideologische Verschönung (die subjektiv ehrlich
  gemeint sein mag) zu sehen.}

Unabhängig davon, wie treffend Voegelins Interpretation die Überlegungen
Bodins wiedergibt, kann die Frage aufgeworfen werden, ob Voegelins
Bodin-Interpretation ihrerseits einen gangbaren Weg zur Begründung der
Toleranz aufzeigt. Der Hinweis auf die unaussprechliche Wahrheit, die
hinter jedem symbolischen Ausdruck liegt, begründet einerseits eine
überaus starke Form religiöser Toleranz, indem jede andere religiöse
Überzeugung als gleichwertvoll wie die eigene anzusehen ist, da sie sich
auf dieselbe Wahrheit bezieht. Andererseits enthält Voegelins Begründung
der Toleranz zum Teil ähnliche Schönheitsfehler wie die Überlegungen von
Bodin, indem Atheisten von der Toleranz auch bei Voegelin ausgenommen zu
sein scheinen.\footnote{Ähnlich ernüchternd wirken Voegelins an anderer
  Stelle geäußerte Bemerkungen darüber, daß ein fruchtbarer Dialog mit
  Indern oder Chinesen nur denkbar wäre, wenn diese Völker sich zuvor
  bereit finden, fleißig die Philosophie von Platon und Aristoteles zu
  studieren, damit eine Grundlage für das Gespräch vorhanden ist. Vgl.
  Conversations with Eric Voegelin. (ed. R. Eric O'Connor), Montreal
  1980., S.70/71.} Abgesehen davon stellt sich das Problem der Toleranz
in vollem Maße erst dann, wenn es nicht gelingt, in irgend einer Weise
eine Übereinstimmung zu einer fremden Auffassung herzustellen.
Beispielsweise stellt es sich dann, wenn man sich mit einer Anschauung
oder Lebensweise konfrontiert sieht, die einem, ohne daß sie einen
selber tangieren müßte, in jeder Hinsicht verkehrt und widerwärtig
vorkommt. Gerade in einer solchen Situation ist Toleranz gefragt,
während es ein Leichtes ist, tolerant zu sein, wenn man feststellt, daß
man im Grunde einer Meinung ist.

\subsection{Der Leib-Geist-Dualismus in der Theorie der Politik}

Im vorletzten Abschnitt seines umfangreichen Aufsatzes untersucht Voegelin
verschiedene Grundsatzfragen einer Theorie der Geschichte und der Politk.
Zunächst klärt Voegelin, daß ein vernüftiger Ansatz in der politischen Theorie
beide Seiten der menschlichen Natur, die leibliche und die geistige
berücksichtigen muß. Im Anschluß daran entwickelt Voegelin den Begriff des
Sozialfeldes, welcher im Wesentlichen ein etwas allgemeinerer Begriff von
Gemeinschaft ist, und stellt schließlich einige grundsätzliche Überlegungen zu
dem Zusammenhang von Sozialfeldern und geschichtlicher Entwicklung an.

Der Mensch ist ein Wesen, das ein Bewußtsein und einen Leib hat. Dies gilt
natürlich nicht nur für den Menschen als Einzelwesen, sondern auch für die
"`soziale Existenz"'\footnote{Voegelin, Anamnesis, S.340.} des Menschen. Nach
Voegelins Ansicht ist es die Leiblichkeit des Menschen, die bedingt, daß jede
Gesellschaft über Herrschaftsinstitutionen verfügen muß, die Ordnung im
Inneren und Sicherheit nach Außen schaffen. Die Untersuchung der pragmatischen
Probleme muß daher in der Politischen Wissenschaft einen breiten Raum
einnehmen. Aber Ordnung kann zugleich nur vom Bewußtsein ausgehen, weshalb
diese Seite der menschlichen Natur in der Politischen Wissenschaft nicht
vernachlässigt werden darf.

Voegelin glaubt nun spezifische Irrtümer verschiedener Ansätze des politischen
Denkens aus der Vernachlässigung jeweils eines Teils der menschlichen Natur
erklären zu können. Diese Vernachlässigung scheint für Voegelin nicht bloß ein
theoretisch-wissenschaftlicher Fehler zu sein, sondern er erblickt darin 
"`Krankheitsbilder, an denen das pneumopathische Phänomen des
Realitätsverlustes, der Verdunkelung von Sektoren der
Realität"'\footnote{Voegelin, Anamnesis, S.341.} zu erkennen ist.

Aus der Vernachlässigung der Leiblichkeit resultieren nach Voegelins
Überzeugung alle Formen von Utopien, wobei Voegelin mit feinen
Unterscheidungen nicht allzu kleinlich verfährt, so daß in diese Kategorie
alles von der aufklärerischen Fortschrittsidee bis zum Dritten Reicht fällt,
und auch Karl Marx sich anscheinend wieder einmal für Nietzsches Übermenschen
verantworten muß.\footnote{Vgl. Voegelin, Anamnesis, S.341. - Zwar spricht
  Voegelin nur allgemein vom Übermenschen ("`sei es der von Marx oder von
  Nietzsche"'), aber die Frage stellt sich, wo Marx denn jemals den
  Übermenschen gepredigt hat.}

Auf die Vernachlässigung der Geistnatur des Menschen sind nach Voegelins
Ansicht die Gesellschaftsvertragstheorien zurückzuführen. Voegelin unternimmt
jedoch nicht den geringsten Versuch, zu erklären, inwiefern sich die
Vertragstheorien allein auf die Leiblichkeit des Menschen beschränken. Sein
lapidarer Verweis auf Platons Staat hilft nicht weiter, da Platon im Staat
zwar die sophistische Vertragstheorie skizziert aber nicht eigens
widerlegt.\footnote{Vgl. Platon: Der Staat, Stuttgart 1997, S.126 (359a). -
  Die Vertragstheorie wird dort im Zusammenhang einer umfassenden Wiedergabe
  sophistischer Lehren über die Gerechtigkeit aufgeführt. Allerdings widerlegt
  Sokrates diese Lehren nicht im Einzelnen, sondern er geht statt dessen
  sogleich zur Konstruktion des idealen Staates über. Wir erfahren daher nicht
  die Gründe, die gegen die Vetragstheorie sprechen. Allenfalls könnte man aus
  dem Bau des idealen Staates indirekt solche Gründe ableiten.}

Schon zuvor hat Voegelin darauf hingewiesen, daß es kein Kollektivbewußtsein
gibt. Sofern man sich jedoch im Klaren darüber bleibt, daß sich
gesellschaftliches Handeln oder Denken stets aus den Handlungen und Gedanken
einzelner Individuen zusammensetzt, ist es im Sinne einer abkürzenden
Ausdrucksweise legitim, davon zu sprechen, daß "`jede Gesellschaft die Symbole
hervorbringt, durch die sie ihre Erfahrung von Ordnung
ausdrückt."'\footnote{Voegelin, Anamnesis, S.342.} Werden bestimmte
Ordnungserfahrungen und -symbole von einer Gruppe von Menschen geteilt und zur
Grundlage ihrer Handlungsweisen gemacht, so spricht Voegelin von einem
"`sozialen Feld"'. Soziale Felder können dauerhaft und institutionell
verfestigt sein, dann handelt es sich um "`Gesellschaften"', sie können aber
auch rein ideeller Natur sein, wie z.B. die "`ideologischen
Sozialfelder"'.\footnote{Voegelin, Anamnesis, S.342.} Darüber hinaus schließen
sich die Zugehörigkeiten zu manchen Sozialfeldern gegenseitig aus, andere
nicht. Voegelin bringt die Exklusivität von Sozialfeldern irrtümlich mit der
Frage in Zusammenhang, ob die Sozialfelder in der Leiblichkeit oder nur im
Bewußtsein fundiert sind. So glaubt Voegelin, daß die organisierten
Gesellschaften (Staaten) auf Grund ihres Fundamentes in der Leiblichkeit
wechselseitig exklusiv sein müssen. Aber in Wirklichkeit hängt dies von der
Gestaltung des Staatsbürgerschaftrechtes ab, das eine Doppelstaatsbürgerschaft
zulassen kann oder nicht. Umgekehrt schließen viele Religionen die
gleichzeitige Zugehörigkeit zu einer anderen Religion aus, obwohl eine
Religion doch gewiß eher ein "`Feld des Bewußtseins"' ist. Das Leibfundament
spielt für die Exklusivität also keine Rolle.\footnote{Vgl. Voegelin,
  Anamnesis, S.342-343.}  Probleme befürchtet Voegelin, wenn es innerhalb
einer Gesellschaft mehr als nur das eine tragende Sozialfeld gibt. Die
pluralistische Demokratie erscheint ihm daher als ein "`prekäre[r]
Kompromiß"'.\footnote{Voegelin, Anamnesis, S.342.}

Ein weiteres wissenschaftliches Problem, das Voegelin in diesem Zusammenhang
aufwirft, ist die Frage, welches die Sozialfelder sind, innerhalb derer
historische Prozesse stattfinden. Die Nationalstaaten sind als Einheiten
offenbar zu klein, da die geschichtlichen Entwicklungsprozesse meist weit über
den Rahmen einzelner Nationalstaaten hinausreichen. Arnold Toynbee betrachtete
die Zivilisationen als diejenigen Einheiten, welche eine umfassende
geschichtliche Betrachtung in den Blick nehmen muß. Dagegen wendet Voegelin
jedoch ein, daß es auch multizivilisatorische Reiche gibt, die auf diese Weise
nicht erfaßt werden können. Voegelin bezeichnet solche
zivilisationsübergreifenden Sozialfelder mit dem von Herodot übernommenen
Begriff der Oikoumene. Weiterhin vertritt Voegelin die These, daß (in der
Zeit, als er den Aufsatz verfaßte) die Oikoumene global geworden sei, und er
befürchtet, daß diese global gewordene Oikoumene das "`potentielle
Organisationsfeld für ideologische Imperien"' sei.\footnote{Voegelin,
  Anamnesis, S. 344.}

Von dem Thema "`Oikoumene"' dazu angeregt, kommt Voegelin noch einmal auf die
Themen Menschheit und Geschichte zu sprechen. Im Wesentlichen wiederholt Voegelin
allerdings nur, was er zu diesen Themen bereits zuvor geäußert hat. Mensch und
Menschheit sind "`nicht Gegenstände der Außenwelt, über die man
selbst-gewisse, empirische Aussagen machen könnte; vielmehr sind sie Symbole,
die von konkreten Menschen ... als Ausdruck für den menschlich-repräsentativen
Charakter ihrer Erfahrung vom Grund gefunden wurden"'.\footnote{Voegelin,
  Anamnesis, S.344.} Die Symbole "`Mensch"' und "`Menschheit"' legen das
"`Wissen von der Menschenwesentlichkeit"' aus, welches erfahren wird, "`Wenn
das Bewußtsein von Ordnung durch die existentielle Spannung zum Grund die
Helle der noetischen und pneumatischen Erfahrungen
erreicht"'.\footnote{Voegelin, Anamnesis, S.344.} Es fällt nicht leicht, dies
mit Voegelins vorheriger Behauptung zu vereinbaren, daß die noetische
Erfahrung ihren repräsentativen Charakter nur unter der Voraussetzung einer
ausschließlich durch die kosmische Primärerfahrung begründbaren
Wesensgleichheit aller Menschen erhält.\footnote{Vgl. Voegelin, Anamnesis,
  S.291. Eine konsistente Interpretation beider Textstellen ist nur möglich,
  wenn man annimmt, daß der "`menschlich-repräsentative Charakter"' der
  Erfahrung allein durch Rekurs auf die kosmische Primärerfahrung gegeben ist,
  daß er aber, sobald sich die noetische Erfahrung eingestellt hat, durch
  dieselben Symbole ("`Mensch"' und "`Menschheit"') zum Ausdruck gebracht
  wird, wie die "`Menschenwesentlichkeit"' (Existenz des Menschen in der
  "`Spannung zum Grund"' als dem Wesen des Menschen), ohne daß jedoch die
  "`Menschenwesentlichkeit"' die Primärerfahrung in ihrer Funktion der
  Begründung des repräsentativen Charakters der Erfahrung ablösen könnte
  (weshalb es sich in diesem Falle auch nicht um eine Differenzierung von
  Erfahrung handelt). - Wie auch immer man die beiden Textstellen (S.290/291,
  S.344/345) in ihrer Beziehung zueinander deuten mag, falsch ist Voegelins
  Ansicht aus den bereits genannten Gründen in jedem Falle.} Geschichte ist
für Voegelin die Geschichte der auf den transzendenten Seinsgrund bezogenen
Menschheit. Weiterhin unterscheidet Voegelin die "`universale
Menschheit"'\footnote{Voegelin, Anamnesis, S.345.}, zu der auch alle schon
gestorbenen und noch zukünftig lebenden Menschen gehören, von der
"`kontemporanen Kulturmenschheit"'\footnote{Voegelin, Anamnesis, S.344.}, die
nur die in der Gegenwart lebenden Menschen umfaßt. Trivialerweise ist nur die
kontemporane Kulturmenschheit ein Feld möglicher Organisation, während die
universale Menschheit lediglich im "`Interpretationsfeld"' Geschichte
auftaucht. Voegelin weist außerdem noch einmal nachdrücklich daraufhin, daß
das "`Bewußtsein von der existentiellen Spannung zum Grund ... ontisch über
alle immanent-zeitlichen Prozesse der Geschichte
hinaus[ragt]."'\footnote{Voegelin, Anmanesis, S.345} Vermutlich, weil die
Spannung zum Grund zur Selbsterfahrung des Menschen gehört, schließt Voegelin
im folgenden, daß das Symbol "`Geschichte"' (so wie Voegelin es versteht) aus
dem Wissen von der Spannung zum Grund stammt,\footnote{Voegelin spricht an der
  entsprechenden Stelle (S.345 unter (8)) von "`Menschwesentlichkeit"', doch
  scheint dies lediglich ein weiteres Synonym Voegelins für die Spannung zum
  Grund zu sein, welches sich dadurch erklärt, daß für Voegelin die
  existentielle Spannung zum Grund das Wesens des Menschen ausmacht.} und daß
eine Interpretation der Ordnung der Existenz einer Gesellschaft zunächst auf
die "`Akte des Selbstverständnisses"' eingehen muß, "`um von diesem Zentrum
her die Ramifikationen in die Ordnung der Gesamtexistenz zu
verfolgen."'\footnote{Voegelin, Anamnesis, S.345.} Weiterhin stellt Voegelin
fest, daß nur die vergangene Geschichte interpretiert werden kann, daß es aber
unmöglich ist, den "`Sinn der Geschichte"' in die unvorhersagbare Zukunft
hinein zu verfolgen. Eschatologien können daher auch nur als
mythisch-symbolischer Ausdruck der "`Spannung zur Ewigkeit des
Grundes"'\footnote{Voegelin, Anamnesis, S.346.} einen Sinn beanspruchen. Sie
dürfen nicht als noetische Analyse oder empirische Aussage mißverstanden
werden.

\subsection{Kritik: Die Unerheblichkeit des Leib-Geist Dualismus}

Es ist ziemlich offensichtlich und bedarf daher keiner ausführlichen
Erörterungen, daß der Versuch, mit Hilfe des Leib-Geist Dualismus politisches
Denken und soziale Gebilde zu charakterisieren, nicht gerade einen Glücksgriff
theoretisch-wissenschaftlicher Erklärungskunst darstellt. Voegelin scheint
sich nicht recht im Klaren darüber zu sein, daß die Ausdrücke "`Leiblichkeit"'
und "`Leibfundament"' in den Zusammenhängen, in denen er sie verwendet, kaum
mehr als Metaphern sind, die er zudem in einer recht willkürlichen und kaum
nachvollziehbaren Weise mit verschiedenen politischen Theorieansätzen und
Denkweisen assoziert. Zwar mag es sein, daß Voegelin die politischen Utopien
zu Recht für bedenklich erachtet. Aber mit Hinweis darauf, daß in der Utopie
das Bewußtsein von seiner Leiblichkeit freigesetzt werde, läßt sich das
utopische Denken ebensowenig erledigen, wie man dem Faschismus einen
entscheidenden Schlag versetzt hat, wenn es einem gelingt, die Gnosis
philosophisch zu widerlegen. Daß andererseits die
Gesellschaftsvertragstheorien als Folge der Reduktion des Menschen auf seine
Leiblichkeit und ihre Begierden zu betrachten sind, leuchtet schon deshalb
nicht ein, weil der Abschluß eines Vertrages Vernunft und damit Geist und
Bewußtsein voraussetzt. Voegelin hoffte wohl, mit Hilfe der ontologischen
Unterscheidung Theorieansätze und Ideologien, die ihm mißfielen, bereits bei
einem besonders elementaren Fehlschluß ertappen zu können (so wie man manchmal
insgeheim hofft, bei einem Philosophen, dessen Ansichten man verabscheut,
irgendwo einen offensichtlichen Widerspruch anzutreffen). Aber aus dem
Leib-Geist Dualismus lassen sich keinerlei gültige Kriterien zur Beurteilung
politischer Theorien ableiten.

Auf Voegelins wissenschaftliche Programmatik, die er in diesem Abschnitt
seines Aufsatzes entwickelt, und auf die Begriffe Menschheit und Geschichte
braucht an dieser Stelle kein weiteres Mal eingegangen zu werden, da Voegelin
kaum etwas Neues zu seinen bisherigen Begriffsbestimmungen hinzufügt.
Lediglich in bezug auf den Geschichtsbegriff ist festzuhalten, daß Voegelin
offenbar versucht, sein methodisches Prinzip, daß "`sich jede Untersuchung von
Ordnung auf die Akte des Selbsverständnisses [einer Gesellschaft] zu
konzentrieren"'\footnote{Voegelin, Anamnesis, S.345.} habe, bereits aus dem
Begriff der Geschichte als eines universalen Interpretationsfeldes, welches
die gesamte vergangene und zukünftige Menschheit in ihrer Beziehung zum
Seinsgrund umfaßt, herzuleiten. Wenn jener methodische Grundsatz auch als
regulatives Prinzip der Forschung durchaus sinnvoll sein kann, so erscheint
dennoch der Versuch, diesen Grundsatz a priori abzuleiten und damit zum Dogma
zu erheben, als überaus fragwürdig.

%  Anzumerken ist nur,
% daß Voegelin, der an anderer Stelle darauf insistiert hat, daß die Bedeutung
% von Wörtern keinesfalls beliebig gewählt werden darf, selbst einen recht
% freizügigen Gebrauch von der nominalistischen Befugnis macht, die Wortbedeutung
% nach Gutdünken festzulegen, wenn er darauf besteht, daß "`Mensch"' und
% "`Menschheit"' Symbole für den "`menschlich-repräsentativen Charakter"' der
% "`Erfahrung vom Grund"'\footnote{Voegelin, Anamnesis, S.344.} sein sollen. Im
% Vergleich dazu erscheint Platons Definition des Menschen als "`federloser
% Zweibeiner"' denn doch einleuchtender und durch ihre Vorbehaltslosigkeit
% auch humaner. 

%  Angemerkt
% sei nur, daß Voegelins Prinzip, daß "`sich jede Untersuchung von Ordnung auf
% die Akte des Selbsverständnisses [einer Gesellschaft] zu
% konzentrieren"'\footnote{Voegelin, Anamnesis, S.345.} habe, als regulatives
% Prinzip durchaus sinnvoll sein kann (und in Voegelins umfangreichen
% historischen Studien auch ihren Erfolg bewiesen hat), daß Voegelin jedoch den
% Fehler begeht, diesen Grundsatz a priori aus seinem abstrusen
% Geschichtsbegriff abzuleiten, womit er ihn zu einem Dogma verhärtet.

\subsection{"`Common Sense"' als kompaktes Ordnungswissen} 

Im Schlußteil seines Aufsatzes faßt Voegelin nach einer weiteren
ausführlichen Wiederholung seiner bisherigen Ergebnisse seine Ansichten von
politischer Ordnung und politischer Ordnungswissenschaft in Form eines knappen
Modelles zusammen. Außerdem knüpft Voegelin noch einmal an sein
Ausgangsproblem des Unterschiedes zwischen politikwissenschaftlicher und
naturwissenschaftlicher Methode an, wobei er zu dem Ergebnis kommt, daß die
Politikwissenschaft, soweit es nicht um die noetische Ergründung der
Seinsrealität geht, lediglich aus "`Common Sense"'-Einsichten bestehen kann.
In diesem Zusammenhang stellt Voegelin auch die These auf, daß der "`Common
Sense"' eine kompakte Form noetischen Ordnungswissens darstelle.

Voegelins Modell politischer Ordnung gestaltet sich folgendermaßen: Ordnung
existiert auf den drei Ebenen der Ordnung des Bewußtseins, der Ordnung der
Gesellschaft und der Ordnung der Geschichte.\footnote{Als vierte Ebene spielt
  bei Voegelin häufig noch eine mehr oder weniger eigenständige Ebene der
  Ordnung des Seins eine Rolle, die hier jedoch nicht eigens aufgeführt wird.
  Es besteht jedoch (trotz der vergleichsweise idealistischen Tendenz dieses
  Aufsatzes) kein Zweifel daran, daß für Voegelin die Ordnung des Bewußtseins
  auf die Erfahrung einer Ordnung des Seins zurückgeht.} Diese drei Ebenen
bilden eine Reihenfolge, die nicht verändert werden kann. So geht zwar die
Geschichte aus der Abfolge gesellschaftlicher Ordnungen hervor, aber es wäre
ein schwerer Fehler, die gesellschaftliche Ordnung nach Maßgabe einer
Geschichtsphilosophie gestalten zu wollen. Als einen weiteren Grundsatz stellt
Voegelin das Prinzip auf, daß keine dieser Ebenen unabhängig von den anderen
ist (was genaugenommen so verstanden werden muß, daß die nachfolgenden Ebenen
abhängig von den vorhergehenden sind, da andernfalls dieses Prinzip der
Forderung der Unumkehrbarkeit der Reihenfolge der Ordnungsebenen
widerspräche). Abgesehen von diesen Ordnungsebenen und ihrer Abfolge
untereinander muß eine politische Theorie die Hierarchie der Seinsstufen vom
materiellen Sein bis zum Bewußtsein berücksichtigen, nach welcher die höheren
Seinsstufen in den niedrigeren fundiert sind, während die niedrigeren
Seinsstufen durch die höheren organisiert werden.\footnote{Vgl. Voegelin,
  Anamnesis, S.349-350. - Vgl. auch Eric Voegelin: Vernunft: Die klassische
  Erfahrung, in: Eric Voegelin: Ordnung, Bewußtsein, Geschichte. Späte
  Schriften.  (Hrsg. von Peter J. Optiz), Stuttgart 1988, S.127-164 (S.162).}

Welchem Zweck dient dieses Modell, außer daß es in knapper Form die Essenz von
Voegelins Ordnungsphilosophie zusammenfaßt? Voegelin glaubt, daß irrige
Theorien politischer Ordnung an Verstößen gegen die Prinzipien dieses Modells
leicht erkannt werden können. Dies zeigt sich für Voegelin an manchen
Geschichtsphilosophien, was Voegelin jedoch nicht an einem Beispiel, sondern
nur unter Berufung auf einen anderen Kritiker der Geschichtsspekulationen
demonstriert.\footnote{Vgl. Voegelin, Anamnesis, S.350-351.} Auch in der
Auffassung, man könne in der Politikwissenschaft so wie in der
Naturwissenschaft allgemeine Gesetze finden, erblickt Voegelin einen Verstoß
gegen sein Modell, ohne daß er jedoch deutlich erklärt, weshalb hier ein
Verstoß vorliegt. Jedenfalls vertritt Voegelin die Ansicht, daß die Politische
Wissenschaft, sofern sie sich mit politischen Institutionen
und nicht mit der Noese beschäftigt, lediglich "`Common Sense"'-Einsichten zu
Tage fördern kann. Zu derartigen "`Common Sense"'-Einsichten gehört
beispielsweise die Feststellung, "`daß Macht die Tendenz hat, von ihrem
Besitzer mißbraucht zu werden"', oder auch die Einsicht, "`daß Kabinette nicht
mehr als eine gewisse Zahl von Mitgliedern haben sollten, weil über eine Zahl
von etwa zwanzig hinaus Beratungen und Entscheidungen schwierig
werden."'\footnote{Voegelin, Anamnesis, S.351.} 

Dabei stellt der "`Common Sense"' für Voegelin weit mehr dar als bloß eine
Ansammlung von Faustregeln des politischen Alltagsverstandes. Voegelin
vertritt vielmehr die These, daß der "`Common Sense"' eine kompakte Form
des noetischen Ordnungswissens ist - nicht anders als dies auf seine Weise der
Mythos ist, nur daß der "`Common Sense"' die Noese voraussetzt, während der
Mythos ihre Vorstufe bildet. Voegelin begründet seine These im Rückgriff auf
den schottischen Philosophen Thomas Reid. Reid verstand unter "`Common Sense"'
den Teil der Vernunft ({\it reason}), der, ausschließlich aus selbst-evidenten
Elementarwahrheiten zusammengesetzt, auch den ungebildetesten Menschen noch zu
Gebote steht. Voegelin interpretiert dies dahingehend, daß für Thomas Reid der
"`Common Sense"' ein kompakter Typus von Rationalität (im Sinne der Spannung
des Bewußtseins zum transzendenten Seinsgrund) sei. Demnach unterscheiden sich
der "`Common Sense"' und die aristotelische Noese nur durch die
"`Bewußtseinshelle"' (das Wissen darum, daß das Bewußtsein sich in einer
Spannung zum Grund befindet).

Voegelin glaubt weiterhin, daß der "`Common Sense"', dessen Philosophie im
18.Jahrhundert "`eben noch rechtzeitig, um nicht von der ideologischen
Dogmatik gebrochen zu werden"',\footnote{Voegelin, Anamnesis, S.353.}
entstanden ist, ein "`echtes Residuum der Noese"'\footnote{Voegelin,
  Anamnesis, S.354.} darstellt. Daraus erklärt sich für Voegelin die
"`Widerstandskraft des anglo-amerikanischen Kulturbereiches gegen die
Ideologien"'. Langfristig jedoch reicht in Voegelins Augen die
Widerstandskraft des "`Common Sense"' gegen die Ideologien nicht aus, so daß
die Wiedergewinnung der "`Helle des noetischen
Bewußtseins"'\footnote{Voegelin, Anamnesis, S.354.} unerläßlich ist.
  
\subsection{Kritik: "`Common Sense"' ist kein Ordnungswissen}

Da die Kritik an Voegelins Modell der Entstehung politischer Ordnung
größtenteils bereits vorweggenommen wurde, genügt es, an dieser Stelle
die wichtigsten Einwände kurz zusammenzufassen: Gegen Voegelins
Forderung, daß die Ordnung der Gesellschaft stets aus einer durch
besondere mystische Erfahrungen bestimmten Ordnung des Bewußtseins
hervorzugehen hat, ist einzuwenden, daß es keinerlei Anhaltspunkte dafür
gibt, daß dieser Weg (und nur dieser Weg) zu einer guten und
erfolgreichen politischen Ordnung führt. Überzeugend erscheint dagegen
Voegelins Ablehnung einer geschichtsphilosophischen Ableitung des
gesellschaftlichen Ordnungsentwurfes, auch wenn Voegelins Argumentation
in diesem Punkt ein wenig umständlich wirkt.\footnote{Wesentlich klarer
  argumentiert im Vergleich zu Voegelin sein Zeitgenosse Karl Popper
  gegen die Geschichtsphilosophien und insbesondere gegen die
  Vorhersagbarkeit der Geschichte: Vgl. Karl Popper: Das Elend der
  Historizismus, Tübingen 1987 (6.Aufl.), S.XI-XII.} Als recht
dogmatisch und theoretisch überaus fragwürdig erweist sich Voegelins
Forderung nach der Berücksichtigung der ontologischen Stufenhierarchie
in der politischen Theorie. Abgesehen davon, daß die Rechtfertigung
ontologischer Stufentheorien viele schwierige philosophische Probleme
aufwirft,\footnote{Eine wesentliche Schwierigkeit ontologischer
  Stufentheorien besteht darin, daß häufig angenommen wird, daß die
  Gesetze, die auf den niederen Ebenen gelten, auf den höheren Ebenen
  aus Kraft gesetzt sind, so daß beispielsweise die körperliche Ebene
  deterministisch, der Geist aber frei gedacht wird. Gerade dies ist
  jedoch nicht möglich, denn wenn alle physischen Vorgänge
  deterministisch ablaufen, kann auch der Geist, dessen Wirken auf
  physischen Vorgängen beruht (bzw. der im Körper "`fundiert"' ist),
  auch dann nicht frei sein, wenn sich das Geistige nicht im Physischen
  erschöpft: Grundsätzlich können auf den höhreren Ebenen nur weitere
  Gesetze hinzukommen, aber nicht die auf den niederen Stufen
  bestehenden Gesetze aufgehoben werden.} und die ontologischen
Stufentheorien darüber hinaus jederzeit Gefahr laufen, durch neue
naturwissenschaftliche Erkenntnisse umgestoßen zu werden, ist es kaum
ersichtlich (und auch Voegelin demonstriert es nirgendwo in einer
einleuchtenden Weise), wie aus der ontologischen Stufenhierarchie heraus
eine Kritik an einer politischen Theorie geübt werden kann. Da die
ontologischen Zusammenhänge auf der Ebene der Politik keine unmittelbare
Wirksamkeit entfalten, kann wahrscheinlich jede politische Theorie ohne
Aufgabe ihrer zentralen Forderungen an irgendeine vorgegebene Ontologie
angepaßt werden. So dürfte es vermutlich keinem Marxisten Probleme
bereiten, gegebenenfalls zuzugeben, daß der Mensch ein
leiblich-geistiges Wesen ist, dessen Geist im Leib fundiert ist. Im
Übrigen können gegen politische Ideologien wie den Faschismus oder den
Kommunismus wahrlich bessere Vorwürfe erhoben werden als die
Nicht-Berücksichtigung des Aufbaus des Seinsbereiches Mensch. Alles in
allem erscheint Voegelins Modell zur Beurteilung politischer
Ordnungsentwürfe und theoretischer Erklärungsansätze in der
Politikwissenschaft wenig tauglich.

Ernstzunehmender als Voegelins Modell politischer Ordnung ist seine
Be\-haup\-tung, daß die Politische Wissenschaft lediglich "`Common
Sense"'-Einsichten zu Tage fördern kann. Es kann in der Tat nicht
bestritten werden, daß die politischen Wissenschaften in dem Bestreben,
möglichst allgemeine Grundsätze zu finden, und prognosefähige Theorien
aufzustellen, längst nicht dieselben Erfolge feiern können wie die
Naturwissenschaften. Dies mag damit zusammenhängen, daß insbesondere bei
historisch-politischen Untersuchungen die Allgemeinheit fast immer durch
einen Verlust an Genauigkeit erkauft wird.\footnote{So verläuft
  beispielsweise die Transition von einer Diktatur zur einer Demokratie
  in unterschiedlichen Staaten trotz gewisser Ähnlichkeiten
  unterschiedlich, so daß eine allgemeine Transitionstheorie, je mehr
  Transitionen in unterschiedlichen Ländern sie beschreiben soll, desto
  größere Abweichungen zu den einzelnen Transitionen aufweisen wird. In
  den Naturwissenschaften verhält sich dies anders: Ein Stein, der in
  Moskau zu Boden fällt, fällt auf die gleiche Weise wie ein Stein in
  Budapest. Und aus der allgemeinen Gravitationstheorie lassen sich die
  Planetenbewegungen ebenso exakt ableiten wie die Fallgesetze.} Die
Gründe, die Voegelin gegen die Möglichkeit größerer Verallgemeinerungen
in der Politikwissenschaft anführt, sind jedoch alles andere als
überzeugend. Weshalb der "`Bau des Seinsbereich Mensch"' es verbietet
nach allgemeinen Gesetzen des menschlichen Verhaltens zu suchen, bleibt
völlig im Dunkeln.\footnote{Abgesehen davon verwechselt Voegelin an der
  entsprechenden Stelle (Voegelin, Anamnesis, S.352: "`Jeder
  Versuch...Mensch"') offenbar Axiome mit Naturgesetzen.  (Von der
  sauberen Unterscheidung zwischen normativen und deskriptiven Theorien
  und einer fairen Beurteilung der Absichten szientistischer Ansätze
  ganz zu schweigen.)} Im übrigen kann nicht von vornherein behauptet
werden, daß die Schwierigkeiten, auf die die Formulierung
allgemeingültiger Theorien in den Gesellschaftswissenschaften stoßen,
auch in aller Zukunft unbehebbar bleiben werden. Es gibt daher auch
keinen zwingenden Grund, weshalb die Naturwissenschaften nicht als
Vorbild betrachtet werden sollten. Umso mehr gilt dies, als der Versuch,
naturwissenschaftliche Denkweisen für die Politikwissenschaft fruchtbar
zu machen, völlig ungefährlich ist. Denn wenn er gelingt, ist die
Wissenschaft ein großes Stück weiter, und wenn er scheitert, haben
lediglich einige Forscher ihre Zeit verschwendet. Gewiß ist ein solcher
Versuch jedoch keine "`potentielle Quelle gesellschaftlicher Unordnung,
insofern er Störungen des rationalen Bewußtseins in anderen Menschen
bewirken kann"'.\footnote{Voegelin, Anamnesis, S.352.} Aber auch wenn
man zugibt, daß die Gesetzmäßigkeiten, die die Politikwissenschaft
bisher gefunden hat, von einer recht unspektakulären Art sind, so muß
dies nicht unbedingt bedeuten, daß Gesetzmäßigkeiten, wie etwa jene
vielzitierte Feststellung Lord Actons, daß Macht korrumpiert, nur
Erkenntnisse des gesunden Menschenverstandes darstellen. Dies wörtlich
zu behaupten, hieße zu verkennen, daß sich derartige Einsichten nicht
selten erst als Folge des gedanklichen Bewußtwerdens langer und
schmerzlicher historischer Erfahrungen durchsetzen.

Mehr als fragwürdig muß Voegelins Versuch genannt werden, den "`Common Sense"'
unter Rückgriff auf Thomas Reid als eine kompakte Form noetischen
Ordnungswissens zu interpretieren. Als Thomas Reid den "`Common Sense"' als
einen Teil der Vernunft bestimmte, dachte er natürlich nicht im Traum an
irgendetwas, was der Voegelinschen Ratio (als "`Sachstruktur"' des in der
"`existentiellen Spannung zum Grund"' stehenden Bewußtseins) auch nur in
irgendeiner Weise nahegekommen wäre.  Für Thomas Reid besteht der "`Common
Sense"' aus denjenigen elementaren Prinzipien des menschlichen
Urteilsvermögens, die es erlauben, selbst-evidente Zusammenhänge zu
beurteilen.  Die Vernunft (reason) umfaßt darüber alle Schlußfolgerungen, die
mit Hilfe dieser selbstverständlichen Wahrheiten gezogen werden
können.\footnote{Vgl.  Thomas Reid: Essays on the intellectual powers of man.
  (Ed. A.D.  Wooz\-ley), London 1941, S.329ff. - Zur Schule der Schottischen
  Philosophen vgl. James Mc\-Cosh: The Schottish Philosophy, 1875 (ed.  1995
  by James Fieser) auf: ""http:""//""socserv2"".""socsci"".""mcmaster"".""ca""/""\~{ }econ""/""ugcm""/""3ll3""/""mccosh""/""scottishphilosophy.pdf"" 
  (Archive for the history of
  economic thought, McMaster University, Hamilton, Canada; letzter Zugriff am:
  30.3.2005).  Zu Thomas Reid: Ebd., S.178-209.}  Mystische Erfahrungen oder
noetisches Ordnungswissen spielen bei Thomas Reid keine Rolle. Dementsprechend
erscheint es auch recht fragwürdig, den "`Common Sense"' Thomas Reids als eine
kompakte Form der aristotelischen Noese einzustufen.\footnote{Mit seiner
  These, daß der "`Common Sense"' eine kompakte Form der aristotelischen Noese
  sei, landet Voegelin allerdings insofern noch einen Zufallstreffer, als der
  Philosophie des Aristoteles tatsächlich eine gewisse Nähe zum "`Common
  Sense"' eigen ist, was sich am deutlichsten in der Lehre von der goldenen
  Mitte darstellt.  (Vgl. auch Russel, History of Western Philosophy, S.176.)
  Mit der Noese hat dies jedoch nichts zu tun, denn sonst müßte eine ähnliche
  Nähe zum "`Common Sense"' auch bei Platon, den Voegelin als Noetiker dem
  Aristoteles an die Seite stellt, zum Vorschein kommen.  Eine Verwandtschaft
  mit dem "`Common Sense"' kann der Philosophie Platons im Gegensatz zu der
  des Aristoteles jedoch kaum nachgesagt werden.}  Es stellt sich natürlich
die Frage, wie Voegelin ein derartig grober Interpretationsfehler unterlaufen
konnte.  Möglicherweise hat Voegelin Gedanken, die ihm bei der Lektüre von
Thomas Reid in den Sinn kamen, mit der Aussage dieses Philosophen verwechselt.
Vielleicht ist Voegelin an dieser Stelle aber auch das Opfer seiner eigenen
Definitionswillkür geworden, indem er zuerst den Ausdruck "`Ratio"' ganz
entgegen der üblichen Wortbedeutung definiert hat, und er dann offenbar
vergißt, daß Thomas Reid als ein Philosoph des 18.Jahrhunderts den Ausdruck
"`reason"' natürlich noch in einem ganz gewöhnlichen Sinne gebraucht.

% (Mißverständnisse
% dieser Art sind in der Philosophiegeschichte nicht selten und führen manchmal
% zu fruchtbaren neuen Entdeckungen.)

Als historischer Mythos muß auf Grund dieses Interpretationsfehlers die These
Voegelins betrachtet werden, daß sich die Noese im 18.Jahrhundert in den
"`Common Sense"' geflüchtet habe, um auf diese Weise das Heraufkommen der
Ideologien zu überwintern und den anglo-amerikanischen Kulturbereich gegen
totalitäre Versuchungen zu immunisieren. Ohnehin hätte - um diese These
einigermaßen glaubhaft zu begründen - der "`Common Sense"' als soziales
Phänomen näher eingegrenzt werden müssen, wofür der Hinweis auf eine
philosophische Schule des 18. und frühen 19.Jahrhunderts nicht ausreicht.

\subsection{Fazit}

Insgesamt stellt Voegelins Aufsatz "`Was ist politische Realität?"' zwar
einen grandiosen Versuch dar, unter Rückgriff auf unterschiedliche
philosophische Disziplinen von der politischen Philosphie bis zur
Sprachphilosophie und unter breitem Einbezug der Tradition
abendländischen politischen Denkens das Wesen der politischen
Wirklichkeit zu bestimmen. Zugleich ist jedoch festzuhalten, daß dieser
ambitionierte Versuch fast gänzlich gescheitert ist.  Verantwortlich
dafür sind vorwiegend die kaum zu übersehenden argumentativen Schwächen
in Voegelins Aufsatz. Darüber hinaus entsteht im Vergleich zu den frühen
Schriften aus dem ersten Teil von "`Anamnesis"' der Eindruck, daß
Voegelin sich auf dem besten Wege zu einer zunehmenden dogmatischen
Verhärtung seiner Position befindet. Während Voegelins Aufsatz "`Zur
Theorie des Bewußtseins"' noch mit einem Reichtum an originellen
Einfällen und Gedankenblitzen glänzt, wiederholt Voegelin in seinem
Grundsatzreferat "`Was ist politische Realität"' nurmehr
gebetsmühlenartig dieselben Dogmen. Dabei steht die Schärfe von
Voegelins immer wieder durchbrechender Polemik in einem eigentümlichen
Kontrast zu dem Mangel an einer stichhaltigen Begründung seiner eigenen
Auffassungen.

% \footnote{Gegenüber dem Vorwurf argumentativer
%   Schwächen könnte eingewandt werden, daß dieser Vorwurf fehlginge, da für
%   Voegelin politische Theorie weniger argumentativ begründet als vielmehr
%   erzählerisch glaubhaft vermittelt werden sollte.  (Darauf deuten Voegelins
%   eigene Interpretationen politischer Theorien sowie seine Ablehnung des
%   "`dogmatischen"' Spiels von theoretischen Positionen und Gegenpositionen
%   hin.) Ich lasse dahingestellt, ob Voegelins philosophische Schriften als
%   Erzählungen mehr hergeben, aber bevor eine Erzählung zur Begründung einer
%   politischen Ordnungsvorstellung dienlich sein kann, müßte erst einmal
%   geklärt werden, wie überhaupt irgend etwas durch eine Erzählung schlüssig
%   begründet werden kann. Zwar hat Voegelin dieses Problem gelegentlich
%   angesprochen (Vgl. Voegelin, Order and History V, S.26.), aber er hat
%   niemals eine auch nur halbwegs überzeugende Antwort darauf gegeben.}

% Dabei gibt sich
% Voegelin jedoch nicht die geringste Mühe zu erklären, weshalb dies der Fall
% ist.  wo doch das Schließen eines Vertrages Vernunft (und damit Geist)
% voraussetzt. Würde die Leiblichkeit genügen, dann müßte man Vertragschlüsse
% zumindest bei Säugetieren beobachten können. Voegelin stellt lediglich die
% klinische Diagnose, daß dergleichen 

% , wobei sie sich
% der dogmatischen Schulphilosophie - diese, wie man annehmen muß, mit neuem
% Sinn erfüllend\footnote{Voegelin erklärt an dieser Stelle nicht ausführlich,
%   wie das als "`Dogmatik"' und "`Parekbasis einer Noese"' charakterisierte
%   Phänomen auf einmal wieder zum "`Repräsentant der Noese"' (S.326.) werden
%   kann.} - als Ausdrucksmittel bediente.

% Statt dessen stellt er sogleich die klinische Diagnose, daß das "`Auftreten
% der sogennanten Vertragstheorien ... auf das umfassendere Syndrom geistiger
% Störung einer Gesellschaft aufmerksam mach."'\footnote{Voegelin, Anamnesis,
%   S.341.} Im Übrigen verweist Voegelin auf das zweite Buch von Platons {\it
%   Politeia}, wo seiner Ansicht nach schon "`alles Wesentliche zur Sache ...
% gesagt wurde."'\footnote{Voegelin, Anamnesis, S.342.} Der Verweis führt jedoch
% in die Leere, da im zweiten Buch der Politeia zwar die sophistische
% Vertragstheorie skizziert aber nicht eigens wiederlegt wird.\footnote{Vgl.
%   Platon: Der Staat, Stuttgart 1997, S.126. (359a).}


% \footnote{Voegelins
%   Haltung erinnert unweigerlich an seinen einmal gegen Marx geäußerten Vorwurf
%   des "`intellektuellen Schwindels"', welchen zu begründen er sich ebensowenig
%   Mühe gegeben hat, wie seine professorale Kritik an einem abgewandelten
%   Schillerzitat bei Hegel den Vorwurf der "`Fälschung"' rechtfertigt. (Vgl.
%   Wissenschaft, Politik und Gnosis) - Auf dem selben Niveau seine
%   Wagner-Kritik in: Eric Voegelin: Die deutsche Universität und die Ordnung
%   der deutschen Gesellschaft, in: Die deutsche Universität im Dritten Reich.
%   Eine Vortragsreihe der Universität München, München 1966, S.241-282
%   (S.252-253). In ästhetischer Hinsicht mag Voegelins Kritik ja berechtigt
%   sein, aber was ist an dem Gebrauch von Aliterationen in einem Opern-Libretto
%   politisch so verdächtig?}


%%% Local Variables: 
%%% mode: latex 
%%% TeX-master: "Main" 
%%% End:






















