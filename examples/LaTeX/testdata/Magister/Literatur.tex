 
%%% Local Variables: 
%%% mode: latex
%%% TeX-master: Main.tex
%%% End: 

\newpage

\chapter{Literatur}

\setlength{\parindent}{0ex}


\setlength{\parskip}{5ex}

{\large I. Schriften von Eric Voegelin}

\setlength{\parskip}{3ex}

{\bf Voegelin, Eric}: Anamnesis. Zur Theorie der Geschichte und Politik,
München 1966.  % UB 66/5523

\setlength{\parskip}{1.5ex}

{\bf Voegelin, Eric}: Autobiographical Reflections (ed. Ellis Sandoz), Baton
Rouge and London 1996.

{\bf Voegelin, Eric / Schütz, Alfred / Strauss, Leo / Gurwitsch, Aron}:
Briefwechsel über "`Die Neue Wissenschaft der Politik"'. (Hrsg. von Peter
J. Opitz), Freiburg/München 1993.  % 94/3274

{\bf Voegelin, Eric}: On the Form of the american Mind, Baton Rouge / London
1995.

{\bf Voegelin, Eric}: "`Die spielerische Grausamkeit der Humanisten"'. Eric
Voegelins Studien zu Niccolò Machiavelli und Thomas Morus. (Hrsg. von
D. Herz), München 1995.

{\bf Voegelin, Eric}: Die Größe Max Webers. (Hrsg. von Peter J. Opitz),
München 1995.  % UB 96/7988

{\bf Voegelin, Eric}: What is history and other late unpublished writings,
Baton Rouge and London 1989.

{\bf Voegelin, Eric}: Der Liberalismus und seine Geschichte, in: Karl Forster
(Hrsg.): Christentum und Liberalismus, München 1960, S.11-42.

{\bf Voegelin, Eric}: Ordnung, Bewußtsein, Geschichte. Späte Schriften.
(Hrsg. von Peter J. Optiz), Stuttgart 1988.

{\bf Voegelin, Eric}: Order and History. Volume One. Israel and Revelation,
Baton Rouge / London 1986 (zuerst: 1956).

{\bf Voegelin, Eric}: Order and History. Volume Two. The World of the Polis,
Baton Rouge / London 1986 (zuerst: 1957).

{\bf Voegelin, Eric}: Order and History. Volume Three. Plato and Aristotle,
Baton Rouge / London 1986 (zuerst: 1957).

{\bf Voegelin, Eric}: Order and History. Volume Four. The Ecumenic Age, Baton
Rouge / London 1986 (zuerst: 1974).

{\bf Voegelin, Eric}: Order and History. Volume Five. In Search of Order,
Baton Rouge / London 1987.

{\bf Voegelin, Eric}: Die politischen Religionen, München 1996 (zuerst 1938).

{\bf Voegelin, Eric}: Der autoritäre Staat. Ein Versuch über das
österreichische Staatsproblem, Wien / New York 1997 (zuerst 1936).

{\bf Voegelin, Eric}: "`Structures of Consciousness"' (ed. Zdravko Planinc),
in: Voegelin-Research News Volume II, No 3, September 1996, auf:
http:""//""vax2.concordia.ca/\~{ }vorenews/v-rnII3.html (Host: Eric Voegelin
Institute, Lousiana State University).

{\bf Voegelin, Eric}: Die deutsche Universität und die Ordnung der deutschen
Gesellschaft, in: Die deutsche Universität im Dritten Reich. Eine
Vortragsreihe der Universität München, München 1966, S.241-282.

{\bf Voegelin, Eric}: Das Volk Gottes. Sektenbewegungen und der Geist der
Moderne (Hrsg. von Peter J. Opitz), München 1994.

{\bf Voegelin, Eric}: Die Neue Wissenschaft der Politik. Eine Einführung,
München 1959.  % UB 59/4508

{\bf Voegelin, Eric}: Wissenschaft, Politik und Gnosis, München 1959.

{\bf Voegelin, Eric}: Die geistige und politische Zukunft der westlichen
Welt. (Hrsg. von Peter J. Opitz und Dietmar Herz), München 1996.

Conversations with Eric Voegelin. (ed. R. Eric O'Connor), Montreal 1980.


\setlength{\parskip}{5ex}

{\large II. Über Eric Voegelin}

\setlength{\parskip}{3ex}

{\bf Cooper, Barry}: Eric Voegelin and the Foundations of Modern Political
Science, Columbia and London 1999.

\setlength{\parskip}{1.5ex}

{\bf Dahl, Robert A.}: The Science of politics: New and Old, in: World
Politics Vol. VII (April 1955), S.484-489.

{\bf Faber, Richard}: Der Prometheus-Komplex. Zur Kritik der Politotheologie
Eric Voegelins und Hans Blumenbergs, Königshausen 1984.

{\bf Farrell, Thomas J.}: The Key Question. A critique of professor Eugene
Webbs recently published review essay on Michael Franz's work entitled "'Eric
Voegelin and the Politics of Spiritual Revolt: The Roots of Modern Ideology"',
in: Voegelin Research News, Volume III, No.2, April 1997, auf:
http://vax2.concordia.ca/\~{ }vorenews/v-rnIII2.html (Host: Eric Voegelin
Institute, Lousiana State University. Zugriff am: 5.3.2000).

{\bf Germino, Dante}: Eric Voegelin on the Gnostic Roots of Violence, München
1998.

{\bf Henkel, Michael}: Eric Voegelin zur Einführung, Hamburg 1998.

{\bf Hughes, Glenn} (Ed.) The Politics of the Soul. Eric Voegelin on Religious
Experience, Lanham / Boulder / New York / Oxford 1999.

{\bf Kiel, Albrecht}: Gottesstaat und Pax Americana. Zur Politischen Theologie
von Carl Schmitt und Eric Voegelin, Cuxhaven / Dartford 1998.

{\bf McAllister, Ted V.}: Revolt against modernity. Leo Strauss, Eric Voegelin
\& the Search For a Postliberal Order, Kansas 1995.

{\bf McKnight, Stephen A. / Geoffrey L. Price} (Hrsg.): International and
Interdisciplinary Perspectives on Eric Voegelin, Missouri 1997.

{\bf Morrissey, Michael P.}: Consciousness and Transcendence. The Theology of
Eric Voegelin, Notre Dame 1994.

{\bf Nida-Rümelin, Julian}: Das Begründungsproblem bei Eric
Voegelin. (Typoskript eines Vortrages beim Internationeln Eric-Voegelin
Symposium in München August 1998, Eric Voegelin-Archiv in München)

{\bf Opitz, Peter J. / Sebba, Gregor} (Hrsg.): The Philosophy of Order. Essays
on History, Consciousness and Politics, Stuttgart 1981.

{\bf Petropulos, William}: The Person as `Imago Dei'. Augustine and Max
Scheler in Eric Voegelins `Herrschaftslehre' and `The Political Religions',
München 1997.

{\bf Poirier, Maben W.}: VOEGELIN-- A Voice of the Cold War Era ...? A COMMENT
on a Eugene Webb review, in: Voegelin Research News, Volume III, No.5, October
1997, auf: http://vax2.concordia.ca/\~{ }vorenews/V-RNIII5.HTML (Host:
Eric Voegelin Institute, Lousiana State University. Zugriff am: 5.3.2000).

{\bf Sandoz, Ellis} (Hrsg.): Eric Voegelin's significance for the modern
mind, Lousiana 1991.

{\bf Sandoz, Ellis} (Hrsg.): Eric Voegelin's Thought. A critical appraisal,
Durham N.C. 1982.

{\bf Webb, Eugene}: Eric Voegelin. Philosopher of History, Seattle and London
1981.

{\bf Webb, Eugene}: Review of Michael Franz, Eric Voegelin and the
Po\-li\-tics of Spiritual Revolt: The Roots of Modern Ideology, in:
Voe\-ge\-lin Research News, Volume III, No. 1, February 1997, auf:
http:""//""vax2"".""concordia"".ca/\~{ }vorenews/v-rnIII2.html (Host: Eric
Voegelin Institute, Lousiana State University. Zugriff am: 5.3.2000).

{\bf Weiss, Gilbert}: Theorie, Relevanz und Wahrheit. Zum Briefwechsel
zwischen Eric Voegelin und Alfred Schütz (1938-1959), München 1997.



\setlength{\parskip}{5ex}

{\large III. Weitere Sekundärliteratur}

\setlength{\parskip}{3ex}

{\bf Albert, Hans}: Kritischer Rationalismus. Vier Kapitel zur Kritik
illusionären Denkens, Tübingen 2000.

\setlength{\parskip}{1.5ex}

{\bf Albert, Hans}: Kritische Vernunft und menschliche Praxis, Stuttgart 1984.

{\bf Aristoteles}: Metaphysik. Schriften zur Ersten Philosophie (Hrsg. und
übersetzt von Franz F. Schwarz), Stuttgart 1984.

{\bf Augustinus, Aurelius}: Bekenntnisse, Stuttgart 1998.

{\bf Ayer, Alfred J.}: Language, Truth and Logic, New York [u.a.] 1982.

{\bf Baumanns, Peter}: Kants Philosophie der Erkenntnis. Durchgehender
Kommentar zu den Hauptkapiteln der "`Kritik der reinen Vernunft"', Würzburg
1997.

{\bf Bergson, Henri}: Materie und Gedächnis, Hamburg 1991.

{\bf Bergson, Henri}: Die beiden Quellen der Moral und Religion, Olten 1980.

{\bf Blumenberg, Hans}: Die Legitimität der Neuzeit. Erneuerte Ausgabe,
Frankfurt am Main 1996.

{\bf Bodin, Jean}: Colloquium of the Seven about Secrets of the
Sublime. Colloquium Heptaplomeres de Rerum Sublimium Arcanis Abditis, Princton
1975.

{\bf Bodin, Jean}: Sechs Bücher über den Staat. Buch IV - VI. (Hrsg. von
P.C. Mayer-Tasch), München 1986.

{\bf Brisson, Luc}: Einführung in die Philosophie des Mythos. Antike,
Mittelalter und Renaissance. Band I, Darmstadt 1996.

{\bf Buber, Martin} (Hrsg.): Ekstatische Konfessionen, Leipzig 1921. 

{\bf Camus, Albert}: Der Mensch in der Revolte. Essays, Hamburg 1997 (zuerst
1951).

{\bf Camus, Albert}: Der Mythos von Sisyphos. Ein Versuch über das Absurde,
Hamburg 1998 (zuerst 1942).

{\bf Camus, Albert}: Tagebücher 1935-1951, Hamburg 1997.

{\bf Camus, Albert}: Tagebuch März 1951 - Dezember 1959, Hamburg 1997.

{\bf Cassirer, Ernst}: Versuch über den Menschen. Einführung in eine
Philosophie der Kultur, Hamburg 1996.

{\bf Denzer, Horst} (Hrsg.): Jean Bodin. Verhandlungen der internationalen
Bodin Tagung in München, München 1973.

{\bf Descartes, René}: Meditationen über die Grundlagen der Philosophie,
Hamburg 1993.

{\bf Fest, Joachim}: Die schwierige Freiheit. Über die offene Flanke der
offenen Gesellschaft, Berlin 1993.

{\bf Forster, Karl} (Hrsg.): Christentum und Liberalismus, München
1960. (Studien und Berichte der katholischen Akademie in Bayern.)

{\bf James, William}: The Varieties of religious Experience, Cambridge,
Massachusetts / London, England 1985 (zuerst 1902).

{\bf Hersch, Jeanne}: Karl Jaspers. Eine Einführung in sein Werk, 4. Aufl.,
München 1990.

{\bf Herz, John H.}: Politischer Realismus und politischer Idealismus.  Eine
Untersuchung von Theorie und Wirklichkeit, Meisenheim am Glan 1959.

{\bf Hobbes, Thomas}: Leviathan oder Stoff, Form und Gewalt eines kirchlichen
und bürgerlichen Staates, Frankfurt am Main 1998 (erste Auflage 1984).

{\bf Husserl, Edmund}: Die Krisis der europäischen Wissenschaften und die
transzendentale Phänomenologie, Hamburg 1996.

{\bf Husserl, Edmund}: Cartesianische Meditationen, Hamburg 1987.

{\bf Husserl, Edmund}: Die phänomenologische Methode. Ausgewählte Texte
I. (Hrsg. von Klaus Held), Stuttgart 1985.

{\bf Kuhn, Helmut} (Hrsg.): Die Philosophie und die Frage nach dem
Fortschritt, München 1964.

{\bf Kant, Immanuel}: Kritik der praktischen Vernunft, Hamburg 1990.

{\bf Kant, Immanuel}: Kririk der reinen Vernunft, Hamburg 1976.

{\bf Kant, Immanuel}: Schriften zur Geschichtsphilosophie, Stuttgart 1985.

{\bf Kant, Immanuel}: Träume eines Geistersehers, erläutert durch Träume der
Metaphysik, in: Frank-Peter Hansen (Hrsg.): Philosophie von Platon bis
Nietzsche, CD-ROM, Berlin 1998. (folgt: Immanuel Kant: Werke in zwölf Bänden.
Herausgegeben von Wilhelm Weischedel, Frankfurt am Main 1977. Band 2).

{\bf Landgrebe, Ludwig}: Phänomenologie und Geschichte, Gütersloh 1967.

{\bf Lübbe, Hermann}: Säkularisierung. Geschichte eines ideenpolitischen
Begriffs, München 1965.

{\bf Mann, Thomas}: Essays. Band 5: Deutschland und die Deutschen 1938-1945.
(Hrsg. von Hermann Kurzke und Stephan Stachorski), Frankfurt am Main 1996.

{\bf McCosh, James}: The Schottish Philosophy, 1875, (ed. 1995 by James
Fieser) auf: ""http:""//""socserv2"".""socsci"".""mcmaster"".""ca""/\~{ }econ/""ugcm""/""3ll3""/""mccosh""/""scottishphilosophy.pdf"" 
(Archive for
the history of economic thought, McMaster University, Hamilton, Canada;
letzter Zugriff am: 30.3.2005).

{\bf Merlau-Ponty, Maurice}: Humanismus und Terror, Frankfurt am Main 1990
(entstanden 1946/47).

{\bf Platon}: Der Staat, Stuttgart 1997

{\bf Popper, Karl R.}: Das Elend des Historizismus, 6.Aufl., Tübingen 1987.

{\bf Popper, Karl R.}: Die offene Gesellschaft und ihre Feinde. Band I. Der
Zauber Platons, 7.Aufl., Tübingen 1992.

{\bf Popper, Karl R.}: Die offene Gesellschaft und ihre Feinde. Band II.
Falsche Propheten: Hegel, Marx und die Folgen, 7.Aufl., Tübingen 1992.

{\bf Reid, Thomas}: Essays on the intellectual powers of
man. (Ed. A.D. Woozley), London 1941.

{\bf Reid, Thomas}: An Inquiry into the human mind on the principles of common
sense, Edinburgh 1997. 

{\bf Ritter, Joachim / Gründer, Karlfried}: Historisches Wörterbuch der
Philosophie. Band 5: L-Mn, Basel / Stuttgart 1980.

{\bf Russel, Bertrand}: A History of Western Philosophy, London / Sydney /
Wellington 1990.

{\bf Schelling, Friedrich Wilhelm}: Philosophie der Offenbarung, in:
Frank-Peter Hansen (Hrsg.): Philosophie von Platon bis Nietzsche, CD-ROM,
Berlin 1998. (folgt der Ausgabe: Friedrich Wilhelm Joseph von Schelling:
Werke. Auswahl in drei Bänden. Herausgegeben und eingeleitet von Otto Weiß.
Leipzig 1907. Band 3.)

{\bf Topitsch, Ernst} (Hrsg.): Werturteilsstreit, Darmstadt 1971.

{\bf Weber, Max}: Gesammelte Aufsätze zur Wissenschaftslehre. (Hrsg. von
Johannes Winckelmann), Tübingen 1988. 

A Book of Contemplation wich is called the Cloud of Unknowing, in which a Soul
is oned with God. (ed. Evelyn Underhill, 2nd ed.  John M. Watkins), London
1922, auf: http://www.ccel.org/u/unknowing/cloud.htm (Host: Christian Classics
Ethereal Library at Calvin College. Zugriff am: 5.4.2000).


% McCosh http:///www.utm.edu/research/iep/text/mccosh/mccosh.htm

%%% Local Variables: 
%%% mode: latex
%%% TeX-master: "Main"
%%% End: 







