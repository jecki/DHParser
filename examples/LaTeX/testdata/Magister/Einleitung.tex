
%%% Local Variables: 
%%% mode: latex
%%% TeX-master: "Main"
%%% End: 

\chapter{Einleitung}

% Thema: Bewußtseinsphilosophie als Grundlage politischer Ordnung/über Eric Voegelin

\section{Thema}

Das Thema dieser Arbeit ist die Bewußtseinsphilosophie Eric
Voegelins.\footnote{Zur Biographie: Eric Voegelin wurde 1901 in Köln geboren.
  1922 Promotion bei Hans Kelsen und Othmar Spann. 1929-38 Privatdozent für
  Staatslehre und Soziologie in Wien. 1938 Flucht vor den Nazis in die USA.
  1942-58 Professor of Gouvernment an der Lousiana State University in Baton
  Rouge.  1958 Professor für Politikwissenschaft in München. 1969 Rückkehr in
  die USA.  1974 Senior Research Fellow an der Hoover Institution on War,
  Revolution and Peace in Stanford. 1985 Tod. (Angaben aus: Michael Henkel:
  Eric Voegelin zur Einführung, Hamburg 1998, S.13-35., S.198-199.)} Sie wird
untersucht unter dem besonderen Aspekt der Begründung politischer Ordnung
durch religiöse Bewußtseinserfahrungen.

Eric Voegelin vertrat eine höchst eigentümliche und in der heutigen Zeit im
westlichen Kulturkreis geradezu befremdlich wirkende Auf\/fassung von den
religiösen Erfahrungen des Menschen als der notwendigen Grundlage politischer
Ordnung. Damit ein politisches Gemeinwesen über eine stabile und im ethischen
Sinne gute politische Ordnung verfügt, genügt es nach Voegelins Ansicht
keineswegs, wenn sich diese Ordnung auf ein ausgeklügeltes System von
Institutionen und auf eine wohldurchdachte Verfassung stützt. Für Voegelin muß
die politische Ordnung darüber hinaus tief im religiösen Empfinden der Bürger
verwurzelt sein. Nur dann kann sie eine ausreichende Resistenz gegenüber
inneren und äußeren Anfechtungen entwickeln, und nur dann kann ihr eine
ethische Qualität zugesprochen werden. In dieser Arbeit soll kritisch
hinterfragt werden, ob die religiöse Erfahrung tatsächlich eine notwendige
Voraussetzung politischer Ordnung bildet und ob eine solche Grundlegung der
politischen Ordnung überhaupt wünschenswert ist.

Wenn für Voegelin die politische Ordnung im religiösen Empfinden oder,
um es in seiner eigenen Terminologie zu formulieren, in den
existentiellen "`Erfahrungen"' der Bürger verwurzelt sein muß, so ist
dies natürlich nicht in der Weise zu verstehen, daß der Staat den
religiösen Bereich der menschlichen Natur für seine Zwecke einspannen
soll, wie dies die totalitären Staaten anstreben. Das religiöse
Empfinden geht nicht vom Staat oder vom gesellschaftlichen Kollektiv
aus, sondern es entspringt dem existentiellen Erleben des Einzelnen, und
nach Maßgabe dieses im individuellen Erleben verankerten religiösen
Empfindens muß die politische Ordnung gestaltet werden.  Damit dies
funktioniert, ist natürlich die Intaktheit des religiösen Empfindens von
größter Bedeutung. Die intakte "`Ordnungserfahrung"' bildet für Voegelin
nicht nur eine notwendige Voraussetzung (guter) politischer Ordnung, sie
stellt auch eine, zwar nicht allein hinreichende, aber doch stark
begünstigende Bedingung dar, gegenüber der alle pragmatischen Probleme
politischer Ordnung, wie z.B. die Einzelheiten der Verfassungsordnung,
vergleichsweise sekundär sind.

Kommt dem Unterschied zwischen intaktem und nicht intaktem
religiös-existentiellen Empfinden eine derartig große Bedeutung zu wie
bei Voegelin, so ist klar, daß eine rein funktionalistische Definition
des Religiösen (z.B. durch die gesellschaftliche Funktion, die die
Religion übernimmt) nicht ausreicht. Vielmehr ist es erforderlich, sich
auf die inhaltliche Ebene der religiösen Dogmen und
Erfahrungen\footnote{Voegelin beschränkt sich auf die Erfahrungen, da
  seinem mystischen Religionsverständnis gemäß auch die Dogmen nur
  Ausdruck von religiösen Erfahrungen (und nicht von geoffenbartem
  Wissen) seien können.} zu begeben.  Wie kann aber hier zwischen echt
und unecht, zwischen richtig und falsch unterschieden werden?  Voegelin
verfolgt in dieser Frage einen zweifachen Ansatz. Zum einen geht er
historisch vor, indem er sich bemüht, die geschichtlichen
Differenzierungsprozesse religiöser Erfahrung nachzuzeichnen und dabei
die differenziertesten Stufen religiös-existentiellen Welterlebens
ausfindig zu machen. Zum anderen versucht Voegelin, auf
bewußtseinsphilosophischem Wege das Wesen der religiösen bzw.
existentiellen Erfahrungen zu ergründen und in unmittelbarer
Selbsterfahrung nachzuvollziehen. Da letzten Endes auch die historische
Beurteilung religiöser Erfahrungen nur am Maßstab der
bewußtseinsphilosophisch ermittelten Wesensauf\/fassung möglich ist, muß
der bewußtseinsphilosophische Ansatz als der grundlegendere dieser
beiden Ansätze angesehen werden. Dieser Arbeit liegt daher die
Interpretationsannahme zu Grunde, daß die Bewußtseinsphilosophie
Voegelins innerhalb der Systematik seines Gedankengebäudes das Zentrum
einnimmt.\footnote{Diese Annahme entspricht Voegelins Selbstdeutung.
  Vgl. Eric Voegelin: Anamnesis. Zur Theorie der Geschichte und Politik,
  München 1996, im folgenden zitiert als: Voegelin, Anamnesis, S.7.}

Voegelin geht es nicht nur darum, empirisch den Zusammenhang zwischen
vorfindlichen politischen Ordnungsgefügen und den sie fundierenden
religiösen Erfahrungen aufzuweisen. Auch wenn derartige Untersuchungen
in seinem zum größten Teil geschichtlichen Oevre bei weitem überwiegen,
so verfolgt Voegelin ebensosehr die normative Absicht, durch die
bewußtseinsphilosophische Aufdeckung der religiösen Erfahrungsquellen
die verbindliche Grundlage einer humanen und totalitarismusresistenten
politischen Ordnung für die Gegenwart zu finden, welche für Voegelin in
den westlichen Demokratien, die ihm in Ermangelung religiöser Grundlagen
auf Sand gebaut schienen, noch unzureichend verwirklicht war. In dieser
Arbeit steht die Untersuchung des normativen Aspektes im Vordergrund. Es
geht mir nicht um die Frage, ob Voegelins Modellvorstellung von
politischer Ordnung auf das alte Ägypten oder das römische Kaiserreich
anwendbar ist, sondern es soll versucht werden herauszufinden, ob
Voegelins Vorstellungen in der heutigen Zeit unter den Bedingungen
pluralistischer und sich entwickelnder multikultureller Gesellschaften
noch (oder vielleicht gerade) tragfähig sind und normative Gültigkeit
beanspruchen dürfen. Letzteres ist natürlich nicht nur eine Frage von
Zeitumständen, sondern vor allem eine Frage der Begründungsqualität.

\section{Methode}

Die Untersuchungsmethode, die in dieser Arbeit angewandt wird, ist die einer
rationalen Rekonstruktion, d.h. es wird versucht, anhand einzelner Texte
Voegelins seine Thesen zu rekonstruieren und ihre Begründung kritisch zu
prüfen. Nur am Rande wird dagegen auf philologische und historische Fragen
eingegangen wie die, welche Entwicklung Voegelins Begriffe innerhalb seines
Werkes durchgemacht haben, durch welche Philosophen er beeinflußt wurde oder
welche zeitgeschichtlichen Umstände auf sein Denken Einfluß genommen haben. Im
Vordergrund steht statt dessen die Frage der Gültigkeit von Voegelins
Theorie.

% Für eine
% rationale Rekonstruktion, die auf die Frage der Gültigkeit einer Theorie
% abzielt, ist die Klärung werk- und zeitgeschichtliche Zusammenhänge der
% Theorie zwar eine Verständnisvoraussetzung, aber ihr kommt kein vordringliches
% thematisches Interesse zu.

% \footnote{Damit soll nicht gesagt werden, daß eine
%   philologische Analyse des Voegelinschen Werkes nicht höchst aufschlußreich
%   könnte, würde eine genaue Untersuchung der Herkunft von Voegelins
%   Denkfiguren und philosophischen Stichworten doch zweifellos zeigen, wie fest
%   Voegelin von seiner geistigen Prägung her an bestimmten philosophischen
%   Strömungen des 19. und 20. Jahrhunderts haftet, und daß er weit weniger aus
%   der Philosophie der Antike schöpft, als daß das seinem auch in der
%   Sekundärliteratur häufig kolportierten Selbstbild entspricht.}

% würde sie doch zweifellos zu Tage fördern, wie sehr Voegelin .
% Wollte man
% einmal versuchen, ausführlich und kritisch nachzuvollziehen, woher Voegelin
% die Denkfiguren und Stichwörter seiner Philosophie bezieht, so würde sich
% zweifellos ein anderes Bild ergeben als das gelegentlich noch in der
% Voegelin-Literatur kolportierte und im wesentlichen seinem Selbstverständnis
% entsprechende Bild des großen Gelehrten, der in gnostisch verwirrter Zeit auf
% dem mühsamen Wege anamnetischer Wiedererinnerung die beinahe verschollenen
% Schätze noetischen Ordnungswissen aus der klassischen Literatur der Antike
% hebt. Eher würde sich das Bild eines Geschichtsphilosophen ergeben, dessen
% intellektueller Horizont zwar einige Jahrtausende der Menschheitsgeschichte
% umfaßt, der von seiner geistigen Prägung her jedoch fest in bestimmten
% philosophischen Strömungen des 19. und 20. Jahrhunderts verwurzelt ist.

Gegen eine derartige Herangehensweise sind von zwei gegensätzlichen Richtungen
her Einwände denkbar. Einerseits könnte eingewandt werden, daß Voegelin
heutzutage keineswegs mehr aktuell und eine theoretische Auseinandersetzung
mit seinen Gedanken daher nicht mehr von Interesse sei. Andererseits könnte
gegen die Methode der rationalen Rekonstruktion und Kritik der Vorwurf erhoben
werden, daß sie, da einem positivistischen Wissenschaftsideal verpflichtet,
dem Denken Voegelins nicht gerecht werden könne.

Der erste Einwand ließe sich dahingehend weiter ausführen, daß Voegelin als
ein typischer Vertreter der Epoche des kalten Krieges inzwischen nurmehr eine
historische Erscheinung sei.\footnote{Dies deutet mit Vorsicht Eugene Webb an.
  Vgl. Eugene Webb: Review of Michael Franz, Eric Voegelin and the Politics of
  Spiritual Revolt: The Roots of Modern Ideology, in: Voegelin Research News,
  Volume III, No. 1, February 1997, auf:
  http://vax2.concordia.ca/\~{ }vorenews/v-rnIII2.html (Host: Eric Voegelin
  Institute, Lousiana State University. Zugriff am: 5.3.2000).} Wenn man heute
einen politischen Romantiker wie, um ein beliebiges Beispiel zu wählen,
Konstantin Frantz analysierte, so würde man auch keine Zeit damit
verschwenden, seine weltfremden Träumereien von einem christlichen Europa zu
widerlegen, sondern ihn von vornherein nur unter einer rein geistes- oder
zeitgeschichtlichen Perspektive, also gewissermaßen als ein historisches
Kuriosum betrachten. Werden derartige Vorbehalte gegen Voegelin auch selten
offen geäußert, so liegen sie doch in der Luft des wissenschaftlichen
Zeitgeistes und bilden auch unausgesprochen einen der Gründe, weshalb Voegelin
heutzutage weitgehend in Vergessenheit geraten ist. Sollte sich aber Voegelins
Theorie auch als gänzlich unhaltbar erweisen, so scheint mir eine
Auseinandersetzung mit Voegelin auf der Sachebene dennoch lohnend, weil
Voegelins Theorie als ein bestimmter Ansatz quasi-religiöser Politikbegründung
eine geistige Möglichkeit repräsentiert, die unabhängig davon, ob sie gerade
in Mode ist oder nicht, aus grundsätzlichem Interesse der Untersuchung wert
ist. Im übrigen können auch bei politikphilosophischen Grundsatzdiskussionen
Stimmungsumschwünge eintreten, die das, was noch wenige Jahrzehnte zuvor als
abwegig galt, auf einmal wieder naheliegend und vertretbar erscheinen lassen.
Dies gilt umso mehr, als auch die abstruseste Philosophie zur Grundlage
politischen Handelns und politischer Ordnung gemacht werden kann. Und wenn
einmal eine obskure Philosophie gesellschaftlich wirksam geworden ist, so
bleibt der bloße Hinweis auf ihre Abstrusität ohnmächtig, da diese Philosophie
dem Empfinden der meisten Menschen dann ganz natürlich erscheint.

Dem zweiten Einwand liegt die Frage zu Grunde, ob die Methode der rationalen
Rekonstruktion für eine Untersuchung von Voegelins Werk angemessen ist.
Voegelin wünschte sich von seinen Lesern eine ganz bestimmte
Lesehaltung, die weniger durch eine kritisch-rationale Einstellung als durch
den meditativen Nachvollzug seiner Gedanken bestimmt sein sollte, denn er
glaubte, eine besondere Art von Wissenschaft zu verfertigen, bei der es gerade
nicht auf das Aufstellen von Thesen und das kritische Abwägen von Argumenten
ankommt. Aber zugleich beanspruchte Voegelin, mit seinen Schriften die
theoretischen Grundlagen politischer Ordnung zu bestimmen. Ob diese Grundlagen
tragfähig sind, läßt sich jedoch nur überprüfen, indem man sie rational
analysiert. Die Rechtfertigung für meine, dem Denken Voegelins vielleicht
etwas fremde, analytische Herangehensweise, liegt also in Voegelins eigener
Zielvorgabe, die geistigen Grundlagen guter politischer Ordnung zu finden. Da
eine politische Ordnung für jeden, der in ihr lebt, verbindliche Geltung haben
soll, so muß ihre Begründung auch intersubjektiv nachvollziehbar sein.
Übrigens nahm Voegelin für seine Art von Politikwissenschaft in Anspruch,
daß sie rationale Wissenschaft sei. Aber dies beruht, wie noch zu zeigen sein
wird, auf einer willkürlichen Umdeutung des Begriffes der Rationalität.

Anders, als sich dies für die Methode der rationalen Rekonstruktion eigentlich
empfiehlt, erfolgt die Darstellung von Voegelins bewußtseinsphilosophischen
Schriften nicht durch eine Zuspitzung von Voegelins Aussagen auf einzelne
Thesen, sondern in der Form einer Wiedergabe seines Gedankenganges. Der Grund
hierfür besteht darin, daß Voegelins Texte in hohem Maße einem erzählerischen
Stilprinzip verpflichtet sind und sich daher gegen eine Zuspitzung auf
einzelne klare Thesen sträuben. Eine Zusammenfassung in Thesen würde deshalb
bereits ein sehr hohes Maß von Interpretation in Voegelins Texte hineintragen,
so daß nicht mehr leicht zu erkennen wäre, wie die Thesen aus Voegelins Worten
entnommen worden sind. Aus diesem Grund wird der Inhalt eines jeden
untersuchten Textes zunächst ausführlich mit eigenen Worten wiedergegeben, so
daß sich meine Interpretation leicht nachvollziehen läßt. Unmittelbar an die
Darstellung eines jeden Textes oder auch einzelner Textpassagen schließt sich
eine eingehende Kritik dieser Textpassagen an. Mag dieses Verfahren der
intermittierenden Kritik auch einen Eindruck von Voreiligkeit und
Nicht-ausreden-lassen-wollen erwecken, so ist es doch dadurch gerechtfertigt,
daß die untersuchten Texte bezüglich ihrer Entstehungszeit teilweise recht
weit auseinanderliegen und dementsprechend unterschiedliche Fragen aufwerfen.
Außerdem läßt sich eine ins Einzelne gehende Kritik nur schwer an eine
umfassende Darstellung anschließen, nach welcher dem Leser nur noch die groben
Züge des Gedankenganges im Gedächnis geblieben sind. Eine Detail-Untersuchung
ist aber beabsichtigt, denn der Wert einer Philosophie enscheidet sich weniger
an den großen Linien der ihr zu Grunde liegenden metaphyischen Weltauf\/fassung
als an der Qualität ihrer Durchführung im Detail. 

% Schließlich soll nicht
% verhehlt werden, daß in dieser Darstellungsweise meine sehr kritische Meinung
% zu Voegelin zum Ausdruck kommt. Es würde gewiß ein wenig sonderbar erscheinen,
% zunächst in aller Seelenruhe über fünfzig oder sechzig Seiten Voegelins
% Gedankengänge auszubreiten, nur um im Anschluß daran mit der Eröffnung
% aufzuwarten, daß all diese Überlegungen im Übrigen samt und sonders verkehrt
% seien.
 
\section{Quellen und Sekundärliteratur}

Eine umfassende Darstellung von Voegelins Bewußtseinsphilosophie würde,
soll es nicht bei einer bloßen Übersicht bleiben, den Umfang einer
Magisterarbeit sprengen. Bewußtseinsphilosophische Überlegungen
begleiten Voegelins Schaffen von seinen frühesten Schriften bis zu den
spätesten Werken,\footnote{Vgl. etwa das Kapitel über "`Time and
  Existence"', in: Eric Voegelin: On the Form of the american Mind,
  Baton Rouge / London 1995, S.23ff.} wobei die Bedeutung der
Bewußtseinsphilosophie in Voegelins Werk im Laufe der Zeit immer mehr
zunimmt.  Dabei geht Voegelins Bewußtseinsphilosophie fließend in seine
Geschichtsdeutung und seine politische Theorie über.\footnote{Besonders
  deutlich wird dies in der Einleitung zu Order and History I. Vgl. Eric
  Voegelin: Order and History. Volume One. Israel and Revelation, Baton
  Rouge / London 1986 (zuerst: 1956), im folgenden zitiert als:
  Voegelin, Order and History I, S.1-11.} Eine Vollständigkeit
beanspruchende Untersuchung von Voegelins Bewußtseinsphilosophie müßte
all diese Zusammenhänge mitberücksichtigen und in hohem Maße auch solche
Schriften Voegelins einbeziehen, die nicht im engeren Sinne
bewußtseinsphilosophisch genannt werden können.

Aus pragmatischen Gründen beschränkt sich diese Arbeit daher auf die
Untersuchung von "`Anamnesis"', dem einzigen ausdrücklich als
bewußtseinsphilosophisch ausgewiesenen größeren Werk, welches Voegelin zu
dieser Thematik selbst veröffentlicht hat. Weiterhin werden aus dem Werk
"`Anamnesis"', das eine Reihe von bewußtseinsphilosophischen und historischen
Aufsätzen Voegelins versammelt, nur die im engeren Sinne
bewußtseinsphilosophischen Schriften berücksichtigt, welche den ersten und
dritten Teil dieses Werkes bilden, während der zweite Teil von
"`Anamnesis"' überwiegend historische Probleme behandelt. Durch die
Beschränkung auf "`Anamnesis"' bleiben die späteren Entwicklungen von
Voegelins Bewußtseinsphilosophie außen vor. So wird Voegelins
Auseinandersetzung mit dem Thema "`Egophanie"' (Selbstbezogenheit des modernen
Menschen im Gegensatz zur Gottbezogenheit), welches in "`Order and History
IV"' einen so großen Raum einnimmt,\footnote{Vgl. Voegelin, Order and History
  IV, S.260ff.} nicht näher behandelt. Auch der Komplex der
"`consciousness-reality-language"' und das "`paradox of consciousness"', zwei
zentrale Begriffe der letzten, in "`Order and History V"' erreichten
Entwicklungsstufe seiner Bewußtseinsphilosophie, treten in "`Anamnesis"'
lediglich in der noch vergleichsweise kruden Form des dort entwickelten
vielschichtigen und paradoxen Realitätsbegriffes auf.\footnote{Vgl. Eric
  Voegelin: Order and History. Volume Five. In Search of Order, Baton Rouge /
  London 1987, im folgenden zitiert als: Voegelin, Order and History V,
  S14-18.  - Vgl. Voegelin, Anamnesis, S.304-305.} Trotz dieser
Einschränkungen umfaßt "`Anamnesis"', besonders durch den zeitlichen Abstand
der darin aufgenommenen Texte, eine große Spannbreite von Voegelins
bewußtseinsphilosophischem Denken und kann daher als durchaus repräsentativ
für Voegelins gesamte Bewußtseinsphilosophie angesehen werden.  Die Kritik an
Voegelin, die in dieser Arbeit anhand einzelner bewußtseinsphilosophischer
Schriften entwickelt wird, ist zu einem großen Teil von grundsätzlicher Art,
so daß sie sich leicht auf andere Schriften Voegelins übertragen läßt. Die
Beschränkung der Untersuchung auf einige wenige Texte ist nicht zuletzt
dadurch begründet, daß es eher durch eine eingehende Darstellung möglich ist,
Voegelins Schriften gerecht zu werden, die sich durch eine Fülle des
verarbeiteten Materials und einen verblüffenden Reichtum an interessanten
Nebengedanken und beiläufigen Überlegungen auszeichnen, als durch einen
notwendigerweise oberflächlich bleibenden Gesamtüberblick. Einer allzu großen
Fixierung auf bloße Einzelaspekte von Voegelins Bewußtseinsphilosophie wird
dadurch entgegengewirkt, daß im ersten Teil der Arbeit ein
Gesamtüberblick über die politische und historische Philosophie Voegelins
gegeben wird, in welche die Bewußtseinsphilosophie eingebettet ist.

Die Lage der Sekundärliteratur zu Eric Voegelin und zu seiner
Bewußtseinsphilosophie ist nicht in jeder Hinsicht günstig. Zwar gibt es über
Eric Voegelin und besonders zu seinem Hauptwerk "`Order and History"' schon
ein beachtliches Schrifttum,\footnote{Vgl. Geoffrey L. Price: Recent
  International Scholarship on Voegelin and Voegelinian Themes. A Brief
  Topical Bibliography, in: Stephen A. McKnight / Geoffry L. Price (Hrsg.):
  International and Interdisciplinary Perspectives on Eric Voegelin, Missouri
  1997, S.189-214. - Eine regelmäßig aktualisierte Bibliographie enthalten die
  Voegelin-Research News des Eric Voegelin Insitute der Louisiana State
  University, http://vax2.concordia.ca/\~{ }vorenews/} aber gerade zu Voegelins
Bewußtseinsphilosophie sind Einzeluntersuchungen noch recht dünn
gesät.\footnote{Eine erschöpfende Darstellung der mittleren Schaffensperiode,
  einschließlich der Bewußtseinsphilosophie des ersten Teils von Anamnesis
  liefert Barry Cooper. Vgl. Barry Cooper: Eric Voegelin and the Foundations
  of Modern Political Science, Columbia and London 1999, S.161ff. - Für die
  spätere Schaffensperiode, insbesondere "`Order and History V"': Vgl. Michael
  P. Morrissey: Consciousness and Transcendence. The Theology of Eric
  Voegelin, Notre Dame 1994, S.117ff. - Meist wird die Bewußtseinsphilosophie
  jedoch nur im Rahmen einer anderen Thematik mitbehandelt. Vgl.
  beispielsweise: Petropulos, William: The Person as `Imago Dei'. Augustine
  and Max Scheler in Eric Voegelins `Herrschaftslehre' and `The Political
  Religions', München 1997, S.35-38.} Hinsichtlich dieser Seite von Voegelins
Werk herrscht noch ein erhebliches Forschungsdefizit, zu dessen Behebung auch
diese Arbeit einen Beitrag leisten möchte. Darüber hinaus leidet die
Sekundärliteratur zuweilen an einer gewissen Einseitigkeit, die, wie es
scheint, dadurch zustande kommt, daß sie zu einem großen Teil von überzeugten
Anhängern Voegelins bestritten wird, während die vorhandenden und möglichen
Gegner Voegelins ihn offenbar mehr oder weniger ignorieren. Nicht selten wird
recht unkritisch das Selbstbild Voegelins, des großen Gelehrten, der in
gottvergessener Zeit in den Tiefen der Geschichte auf Wahrheitssuche geht,
kolportiert und geradezu eifersüchtig gegen Einwände
verteidigt.\footnote{Deutlich wird dies etwa an den heftigen Reaktionen auf
  Eugene Webbs maßvolle Voegelin-Kritik.  - Vgl. Thomas J.  Farrell: The Key
  Question. A critique of professor Eugene Webbs recently published review
  essay on Michael Franz's work entitled "'Eric Voegelin and the Politics of
  Spiritual Revolt: The Roots of Modern Ideology"', in: Voegelin Research
  News, Volume III, No.2, April 1997, auf:
  http://vax2.concordia.ca/\~{ }vorenews/v-rnIII2.html - Maben W. Poirier:
  VOEGELIN-- A Voice of the Cold War Era ...? A COMMENT on a Eugene Webb
  review, in: Voegelin Research News, Volume III, No.5, October 1997, auf:
  http://vax2.concordia.ca/\~{ }vorenews/V-RNIII5.HTML (Host jeweils: Eric
  Voegelin Institute, Lousiana State University. Zugriff am: 5.3.2000).}
Freilich ist Voegelin nicht ganz unschuldig daran, daß sein Werk unter die
Zeloten gefallen ist, sah er doch selbst in Ansichten, die zu seiner Denkweise
im Gegensatz standen, die "`Rhetorik deformierter Existenz"' am Werk, und
empfahl er einmal sogar, dem "`verführerischen Zwang [für den modernen
Menschen, E.A.], sich selbst zu deformieren"', mit den einem altägyptischen
Dichter entnommenen Worten entgegenzutreten: "`Siehe, mein Name wird übel
riechen durch dich // mehr als der Gestank von Voegelmist // an Sommertagen,
wenn der Himmel heiß ist"'.\footnote{Vgl. Eric Voegelin: Äquivalenz von
  Erfahrungen und Symbolen in der Geschichte, in: Eric Voegelin: Ordnung,
  Bewußtsein, Geschichte, Späte Schriften (Hrsg. von Peter J. Optiz),
  Stuttgart 1988, S.99-126 (S.105).} Insgesamt scheint ein gewisser Mangel
zwar nicht an einzelnen kritischen Tönen aber an kritischer Auseinandersetzung
mit Voegelin zu bestehen.\footnote{Als ein durchaus typisches Beispiel für
  diese Art von Sekundärliteratur, die fast nur aus Bestandsaufnahme, aber so
  gut wie gar nicht aus kritischer Diskussion besteht sein hier nur das
  folgende herausgegriffen: Glenn Hughes (Ed.): The Politics of the Soul. Eric
  Voegelin on Religious Experience, Lanham / Boulder / New York / Oxford 1999.
  - Als Beispiele der Voegelin-Kritik seien herausgegriffen: Mit
  gesellschaftskritischem Akzent: Richard Faber: Der Prometheus-Komplex.  Zur
  Kritik der Politotheologie Eric Voegelins und Hans Blumenbergs, Königshausen
  1984. - Ideologiekritisch vor allem gegenüber Voegelins Gnosis-Begriff:
  Albrecht Kiel: Säkularisierung als Geschichte des Unheils.  Die
  Gleichsetzung von Rationalität und Ordnung mit Katholizität in der
  Geschichtsphilosophie Eric Voegelins, in: Albrecht Kiel: Gottesstaat und Pax
  Americana. Zur Politischen Theologie von Carl Schmitt und Eric Voegelin,
  Cuxhaven und Dartford 1998, S.95-118. - Erhebliche Zweifel an der
  philologischen Genauigkeit Voegelins meldet Zdravko Planinc an: Zdravko
  Planinc: The Uses of Plato in Voegelin's Philosophy of Consciousness:
  Reflections prompted by Voegelin's Lecture, "`Structures of Consciousness"',
  in: Voegelin-Research News, Volume II, No.  3, September 1996, auf:
  http://vax2.concordia.ca/\~{ }vorenews/v-rnII3.html (Host: Eric Voegelin
  Institute, Lousiana State University. Zugriff am: 5.3.2000).}

Außer der Sekundärliteratur zu Eric Voegelin wird auch philosophische
Literatur zu den Themen, die Voegelin in seinen
bewußtseinsphilosophischen Texten anspricht, herangezogen. Hier besteht
allerdings die Schwierigkeit, daß es in der Philosophie kein Expertentum
gibt und daß man daher je nachdem, auf welche Schule man zurückgreift,
zu einer sehr unterschiedlichen Ansicht des Gegenstandes gelangen kann.
In dieser Arbeit wurden vor allem die Autoren zu Rate gezogen, die auch
Voegelin in seinen Schriften anspricht. Dies bereitet für die
Untersuchung des ersten Teils von Anamnesis keine Probleme, da klar ist,
daß Voegelin sich hier vornehmlich mit der Phänomenologie
auseinandersetzt.  Schwieriger ist dies jedoch für den dritten Teil von
"`Anamnesis"', da Voegelin hier bereits wesentlich selbständiger
vorgeht.  Weiterhin werden solche Autoren miteinbezogen, die von
Voegelin zwar nicht immer ausdrücklich erwähnt werden, auf die er sich
jedoch stillschweigend zu beziehen scheint.

\section{Aufbau}

Die Arbeit ist in drei Teile untergliedert. Der erste Teil gibt einen Grundriß
von Voegelins politischer Philosophie. Ziel ist es, die Hauptthesen von
Voegelins politischer Philosophie darzustellen sowie seinen methodischen
Ansatz zu bestimmen. Insbesondere soll gezeigt werden, wie und an welcher
Stelle bewußtseinsphilosophische Voraussetzungen in sein politisches Denken
eingehen. In diesem Teil beziehe ich mich überwiegend auf Voegelins "`Neue
Wissenschaft der Politik"',\footnote{Eric Voegelin: Die Neue Wissenschaft der
  Politik. Eine Einführung, München 1959, im folgenden zitiert als: Voegelin,
  Neue Wissenschaft der Politik.} da dieser Schrift unter Voegelins Werken am
ehesten der Charakter einer Programmschrift eigen ist. Dabei werden von
vornherein auch kritische Einwände gegen Voegelins Auf\/fassungen diskutiert.
Die Kritik dient nicht zuletzt dazu, den Problemhorizont abzustecken, der bei
der Untersuchung von Voegelins Bewußtseinsphilosophie berücksichtigt werden
muß.

Im zweiten Teil werden ausführlich Voegelins bewußtseinsphilosophische
Schriften dargestellt und einer eingehenden Detail-Kritik unterzogen. Den
Abschluß des zweiten Teils bildet die Diskussion einiger Grundprobleme von
Voegelins Bewußtseinsphilosophie, wobei die kritische Betrachtung von
Voegelins Begriff der (religiösen) Erfahrung im Zentrum steht. Es gilt dabei
kritisch Bilanz zu ziehen, ob der in Voegelins Denken zentrale Begriff der
Erfahrung hinreichend durch die bewußtseinsphilosophischen Überlegungen
Voegelins begründet und erläutert ist, um für das Verständnis und die
Gestaltung politischer Ordnung fruchtbar gemacht werden zu können.

Im letzten, mehr essayistisch gehaltenen Teil der Arbeit wird schließlich auf
einer etwas allgemeineren Ebene die Frage angesprochen, ob gute politische
Ordnung einer religiösen Grundlage bedarf. Dabei wird zu zeigen versucht, daß
eine religiös-spirituelle Grundlegung der Politik, wie sie Voegelin
vorschwebte, sowohl aus grundsätzlichen Überlegungen als auch insbesondere
unter den Bedingungen einer pluralistischen und zunehmend multikulturellen
Gesellschaft vor erheblichen Schwierigkeiten steht. Zugleich wird die Frage
aufgeworfen, ob eine rein säkulare, durch Konsens bestimmte Grundlegung
politischer Ordnung auf Basis eines Gesellschaftsvertrages denkbar ist, und ob
daher politische Ordnung des transzendenten Bezuges nicht ohnehin gänzlich
entraten kann.

% Wenn man so will ist dies nichts weiter als eine liberale
% Selbstvergewisserung. Da Theorie der liberalen Demokratie hierzulande zur Zeit
% sowieso die herrschende Meinung wiedergibt, kann dieser Teil eher knapp
% ausfallen. Die Grundthesen des dritten Teils lauten kurz gefaßt:

% \begin{itemize}
% \item Wenn die Bewußtseinsphilosophie kein objektives Wissen über die
%   Ordnung des Seins vermitteln kann, so kann sie auch auch keine Grundlage
%   politischer Ordnung bilden.
% \item Wenn Politik auf Transzendenz gegründet wird, dann wird die Religiosität
%   zu einer Angelegenheit der politischen Öffentlichkeit. Dies wirft Probleme
%   für die Religionsfreiheit und Toleranz auf.
% \item Es ist (insbesondere in einer multikulturellen Gesellschaft)
%   aussichtsreicher Konsens auf der Ebene der Werte als auf der Ebene der
%   Wertbegründung zu suchen.
% \item Die Notwendigkeit politischer Ordnung entsteht aus dem Umstand, daß
%   Menschen einander in die Quere kommen können, und deshalb Abmachungen
%   treffen müssen, damit dies nicht geschieht. Politik hat daher ihrem Wesen
%   nach mehr mit der niederen, materiellen Sphäre des unumgehbaren
%   Notwendigkeiten zu tun als mit der geistigen Sphäre. Es ist daher ein
%   Fehler, von der Politik den Ausdruck spiritueller Wahrheit zu
%   verlangen.
% \item Die Trennung von Religion und Politik zu fordern, bedeutet weder die
%   Religion zu leugnen noch sie auf die Privatsphäre zu begrenzen, denn
%   zwischen der politischen Öffentlichkeit und der Privatsphäre gibt es eine
%   Reihe weiterer Öffentlichkeiten (etwa die der religiösen
%   Glaubensgemeinschaften) in denen der Ausdruck und die Pflege der
%   Spiritualität in kollektiver Form möglich ist.
% \item Solange die Mehrheit der Bürger von der politischen Ordnung nicht den
%   Ausdruck ihrer religiösen Überzeugungen erwartet, kann eine nicht
%   spirituelle Grundlegung der Politik Legitimität entfalten.
% \item Da das Letztbegründungsproblem in der Ethik ohnehin noch nicht gelöst
%   ist, steht hinsichtlich der ethischen Qualität der politischen Ordnung die
%   Vertragstheorie, welche sich diesem Problem entzieht, nicht schlechter da,
%   als eine religiöse Grundlegung politischer Ordnung, welche dieses Problem
%   verschiebt. 
% \end{itemize}
 
%  Abgesehen von "`Anamnesis"' ist Voegelins
% Bewußtseinsphilosophie eher über sein gesamtes Werk verteilt als in bestimmten
% Schriften zusammengefaßt.\footnote{{\bf wichtigste bewußtseinsphilosophische
%     Passagen aufzählen aus: Form d. am.  Geistes, Rasse U. Staat,
%     Briefwechsel, Einleitung OH I,II, Anamnesis OH IV, OH V, Sammelband
%     Ordnung, Bewußtsein Geschichte, Aufsätze?}} Bei den zu untersuchenden
% Primärtexten beschränke ich mich auf den ersten und dritten Teil von
% "`Anamnesis"' sowie den Anfang von "`Order and History V"'. Diese Beschränkung
% ist teils inhaltlich und teils pragmatisch begründet. Inhaltlich habe ich
% versucht, mich auf solche Texte zu beschränken, in denen vorwiegend der
% normative Aspekt der Grundlagen politischer Ordnung zur Geltung kommt.

% \footnote{Dies wirft selbst
%   wiederum eine philosophische Grundsatzfrage auf. Wahrscheinlich besteht
%   hierin der wesentliche Unterschied zwischen Platon und den Neu-Platonisten,
%   daß für Platon der Weg zur Erkenntnis des Höchsten, der über alle
%   wissenschaftlichen Profanerkenntnise führt, einen eigenen Wert hat und eine
%   notwendige Voraussetzung zur Erkenntnis des höchsten bildet, während sich
%   die Neu-Platonisten nur noch auf das Höchste konzentrieren und dem
%   vermeintlich niederen kein Interesse mehr entgegen bringen. Ich halte es
%   hier mit dem Wort aus Goethes Faust: "`Willst du das Unendliche erreichen,
%   so schreite nur im Endlichen nach allen Seiten."' Oder anders gesagt: In
%   jeder Wahrheit steckt die höchste Wahrheit. Voegelin war zweifellos eher
%   Neu-Platonist.}

%%% Local Variables: 
%%% mode: latex
%%% TeX-master: "Main"
%%% End: 

















