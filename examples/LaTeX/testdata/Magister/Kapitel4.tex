\chapter{Braucht Politik spirituelle Grundlagen?}

Nachdem im zweiten Teil dieser Arbeit Voegelins Bewußtseinsphilosophie fast
ausschließlich unter einer philosophischen Perspektive betrachtet wurde, soll
nun wieder die Beziehung zur Politik hergestellt werden. Voegelin ging es mit
seinen bewußtseinsphilosophischen Untersuchungen nicht bloß um die seelische
Ordnung der menschlichen Einzelexistenz, sondern vor allem auch um die
politische Ordnung der Gesellschaft. Nicht umsonst trägt die große
bewußtseinsphilosophische Abhandlung, die den Schlußteil seines Werkes
"`Anamnesis"' bildet, die Frage nach der politischen Realität im Titel. Im
folgenden wird zunächst versucht zu klären, inwiefern für Voegelin spirituelle
Erfahrungen die Voraussetzung guter politischer Ordnung bilden und von
welcher Gestalt eine politische Ordnung ist, die die spirituelle Erfahrung
nach Voegelins Maßstäben in angemessener Weise berücksichtigt. Anschließend
wird, losgelöst von Voegelins Theorie, in Bezug auf einige Grundfragen
politischer Ordnung untersucht, ob politische Ordnung ohne eine spirituelle
Grundlage auskommen kann. 

\section{Spirituelle Wahrheit und politische Ordnung bei Voegelin}

Bereits bei der Untersuchung von Voegelins Bewußtseinsphilosophie fiel auf,
daß die Zusammenhänge zwischen den sprituellen Erfahrungsgrundlagen
politischer Ordnung und der politischen Ordnung selbst, die sich als
rechtliche und institutionelle Ordnung einer Gesellschaft konkretisiert,
merkwürdig im Dunkeln bleiben. Zwar läßt Voegelin keine Gelegenheit aus, um
vor den verhängnisvollen Folgen zu warnen, die ein Verlust des
Erfahrungskontaktes zum transzendenten Seinsgrund nach seiner Überzeugung
unweigerlich nach sich zieht, aber diese Warnungen sind wissenschaftlich kaum
präziser, als es die pauschale Behauptung wäre, daß das Unheil der Zeit eine
Folge der menschlichen Gottlosigkeit sei. Alles läuft bei Voegelin letzlich
auf die anthropologische These hinaus, daß der Mensch des Kontaktes zum
Seinsgrund bedarf, um seine Existenz zu ordnen, und daß keine politische
Ordnung erzielt werden kann, wenn nicht sowohl auf der Seite der Herrschenden
als auch auf der Seite der Beherrschten der in genau dieser Weise existentiell
geordnete Menschentypus dominiert. Man mag einwenden, daß Voegelin im Rahmen
seiner bewußtseinsphilosophischen Untersuchungen aus Gründen der thematischen
Beschränkung diese Punkte nur habe andeuten können. Doch auch in seinen
anderen Schriften beschäftigt sich Voegelin fast ausschließlich mit der
geistigen Seite politischer Ordnung und fast nie mit dem Zusammenhang der
geistigen Grundlagen und der konkreten Machtordnung, wiewohl er an der
Feststellung, daß zwischen beiden Bereichen eine enge Beziehung besteht,
keinerlei Zweifel duldet.

Wie wenig erklärende Kraft Voegelins Theorie hat, wenn er tatsächlich einmal
versucht, mit ihr die Ursachen gesellschaftlicher Unordnung zu beschreiben,
läßt sich am besten an einem Beispiel nachvollziehen: In seinem 1959
gehaltenen Vortrag "`Die geistige und politische Zukunft der westlichen Welt"'
stellt Voegelin ein "`Gesetz der westlichen Ordnung"' auf, welches besagt, daß
es drei "`Autoritätsquellen"' der Ordnung gibt, erstens die herrscherliche
Macht, zweitens die Vernunftphilosophie und drittens die religiöse
Offenbarung, und daß Ordnung herrscht, solange diese Autoritätsquellen relativ
autonom voneinander bleiben, Unordnung aber dann, wenn sie
zusammenfallen.\footnote{Vgl. ebd., S.21-23.} Verfolgt man nun Voegelins
weitere Ausführungen zu diesem Gesetz, so springen einige Merkwürdigkeiten ins
Auge: Zunächst einmal unternimmt Voegelin keinen Versuch, zu erklären, weshalb
aus dem Zusammenfallen der drei Autoritätsquellen gesellschaftliche oder
politische Unordnung resultiert. Solange Voegelin dies
nicht demonstriert, liefert sein Gesetz nur eine willkürliche Definition von
"`Unordnung"', die mit innerem Unfrieden, chaotischen Zuständen oder
tyrannischen Übergriffen des Staates gar nichts zu tun haben
muß.\footnote{Dies gilt umso mehr, als nach Voegelins Gesetz auch im Reich des
  Kaisers Justinian, anhand von dessen {\it constitutio imperatoria majestas}
  Voegelin sein Gesetz entwickelt, größte Unordnung herrschen müßte, da ja der
  Kaiser alle drei Autoritätsquellen in seiner Person vereint.} Des weiteren
stützt sich Voeglins folgende Argumentation nur marginal auf das gerade erst
aufgestellte Gesetz. Es zeigt sich, daß es Voegelin keineswegs auf die
Autonomie der Autoritätsquellen ankommt, - sonst müßte er ja auch die Trennung
von Staat und Kirche und die Abspaltung der Naturwissenschaft von der
Philosophie befürworten\footnote{Vgl. ebd., S.31, S.34.} - sondern darauf, daß
christlicher Glaube und Philosophie (mit der selbstverständlich nur die von
Voegelin favorisierten Richtungen der Philosophie gemeint sind\footnote{Vgl.
  ebd., S.35.}) einen bestimmenden Einfluß auf Gesellschaft und Politik
erlangen. Dafür ist Voegelin sogar bereit, bemerkenswerte Einschränkungen der
demokratischen Rechte hinzunehmen. So fordert er "`sehr energisch mit
Parteiverboten"'\footnote{Ebd., S.33.} gegen Parteien "`antichristlicher oder
antiphilosophischer Art"'\footnote{Ebd.} vorzugehen. Der Grund für diese
radikale Forderung liegt dabei einzig in Voegelins vorgefaßter Meinung, daß
die westlichen Demokratien sich nur halten können, wenn die Regierung im
christlichen Geiste über eine weitgehend christliche Bevölkerung
regiert.\footnote{Vgl. ebd., S.32-33.} Warum sie nur unter dieser Bedingung
funktionieren können, dafür gibt Voegelin trotz seiner historisch
weitausholenden Erörterungen keinerlei Begründung.

Ob die Berücksichtigung der spirituellen Erfahrung für die Herstellung
politischer Ordnung überhaupt irgendwelche Vorteile erbringt, läßt sich nicht
zuletzt deshalb nur schwer klären, weil Voegelin niemals deutlich mitteilt,
von welcher Gestalt eine optimal erfahrungsbegründete politische Ordnung sein
würde. Versucht man sich hilfsweise an Voegelins historische Beispiele zu
halten, dann erhält man ein recht irritierendes Bild.  So ist für Voegelin
beispielsweise im christlichen Mittelalter vor der Reformation mit der
Trennung von geistlicher und weltlicher Autorität bei der gleichzeitigen
Legitimation und Gestaltung der weltlichen Ordnung nach religiösen Prinzipien
eine optimale Verwirklichung spirituell erfahrungsbegründeter politischer
Ordnung gegeben. Dies gilt umso mehr, als nach Voegelins Einschätzung im
mittelalterlichen Christentum die bislang größte Erfahrungshelle des
Ordnungswissens erreicht worden ist. Gleichzeitig herrscht jedoch mit der
hierarchischen Gesellschaftsform und dem feudalen Herrschaftssystem im
Mittelalter eine politische Ordnung, die alles andere als human und gerecht
ist. Ein weiteres irritierendes Beispiel stellt die politische Philosophie
Platons dar. Für Voegelin war Platon ein Philosoph von größter Offenheit der
Seele und höchster spiritueller Empfindsamkeit. Aber die
politisch-institutionelle Ordnung, die Platon im "`Staat"' entworfen hat,
bildet geradezu das Musterbeispiel einer totalitären
Schreckensutopie.\footnote{Vgl. dazu die bekannte Kritik in: Karl Popper: Die
  offene Gesellschaft und ihre Feinde. Band I. Der Zauber Platons, 7.Aufl.,
  Tübingen 1992, S.104ff. - Poppers Deutung ist freilich nicht unumstritten.
  Außer einem in der Tat verfälschenden Platon-Zitat auf dem Umschlag (in der
  Taschenbuchausgabe: S.9.) wird ihm unter anderem vorgeworfen, sich bei
  seiner Kritik an dem Personenzentrierten Ansatz Platons zu sehr auf den
  "`Staat"' zu konzentrieren, und den stärker institutionellen Ansatz der
  "`Gesetze"' zu vernachlässigen. Von dieser Kritik unberührt bleibt
  allerdings Poppers massiver Vorwurf der Inhumanität gegen Platon.}
Hält man sich diese Beispiele vor Augen, so erscheint es geradezu absurd, daß
Voeglin der Wiedererlangung einer spirituellen Realitätserfahrung vermittels
der Öffnung der Seele eine so große Bedeutung bemißt. Eher müßte man den
Schluß ziehen, daß für gute politische Ordnung ein niedriges spirituelles
Niveau von Vorteil ist. Gewiß, die soeben gegebenen Beispiele sind
Extrembeispiele, denn Voegelin befürwortete auch die amerikanische Demokratie,
die in der Tat eine sehr erfolgreiche Verwirklichung humaner und gerechter
politischer Ordnung darstellt, und nach Voegelins Ansicht steht der
politischen Philosophie Platons die des Aristoteles, welche wesentlich
vernünftiger ist, in nichts nach. Aber immer noch stellt sich dann die Frage,
ob überhaupt eine Korrelation zwischen dem Niveau der spirituellen Erfahrung
und der Güte der politischen Ordnung besteht.

Dieser Frage, ob politische Ordnung überhaupt eine spirituelle Grundlage
benötigt, um eine erfolgreiche und gerechte politische Ordnung zu sein, soll
im folgenden nachgegangen werden.

%  Nun stellt sich
% jedoch die Frage, was an der hierarchischen Gesellschaftsordnung und dem
% feudalen Herrschaftssystem so vorteilhaft gewesen ist, daß diese Epoche
% Zeugnis von der Notwendigkeit oder wenigstens der Nützlichkeit der religiöser
% Politikbegründung ablegen könnte. Auf diese Frage wird man nur sehr schwer
% eine überzeugende Antwort finden. Am Beispiel des vorreformatorischen
% Christentums zeigt sich also, daß die spirituelle Erfahrung in einer anderen
% Epoche eine politische Ordnung von einer gänzlich anderen Gestalt
% hervorbringt, und daß ein hohes spirituelles Niveau keineswegs eine politische
% Ordnung hervorbringen muß, die human und gerecht ist. Besonders, wenn man sich
% Letzteres vor Augen 

% Ebenso schwierig wie die Frage nach dem Zusammenhang zwischen der spirituellen
% Erfahrung und der eigentlichen Ordnung, ist die Frage zu beantworten, welche
% Gestalt eine politische Ordnung hat, die Voegelins Maßstäben von einer in der
% Erfahrung der Spannung zum Grund verwurzelten Ordnung entspricht. Da Voegelin
% selbst nur selten Ratschläge zur Gestaltung der institutionellen Ordnung
% gegeben hat, könnte man versuchen zur Beantwortung dieser Frage auf die
% Beispiele von politischer Ordnung und auf diejenigen Theoretiker der Politik
% zurückzugreifen, über die sich Voegelin in seinen Werken lobend geäußert hat.
% Aber schon wenige Beispiele zeigen, daß sich dabei ein recht uneindeutiges
% Bild ergibt.

% Ein Optimum von Ordnung ist für Voegelin in der Gegenwart in den Demokratien
% angelsächsischer Prägung verwirklicht. Und in der Tat stellt beispielse die
% amerikanische Demokratie eine politische Ordnung dar, die höchst erfolgreich
% und von dauerhafter Beständigkeit ist, und in der darüber hinaus in hohem Maße
% die die Prinzipien der individuellen Freiheit und der Gerechtigkeit
% verwirklicht sind. Zwar beruht, wie bereits gezeigt wurde, Voegelins
% Einschätzung, daß das angelsächsische Zivilregime mit dem Common Sense von
% einer kompakten Form spirituellen Ordnungswissens getragen ist, auf einem
% Mißverständnis. Aber wenn man von diesem Problem einmal absieht und mit
% Voegelin davon ausgeht, daß die angelsächsische Demokratie ihre Gestalt einer
% spirituellen Ordnungserfahrung verdankt, auf die sie gegründet ist, dann ist
% das Beispiel der angelsächsischen Demokratie ein klarer Pluspunkt für die
% spirituelle Ordnungserfahrung.

% Dieser positive Eindruck ändert sich jedoch, wenn man einmal eines der
% historischen Beispiele für spirituell wohlbegründetete politische Ordnung
% betrachtet, wie etwa das christliche Mittelalter vor der Reformation. 

% Dieser Eindruck verstärkt sich noch, wenn man einen Blick auf die 
% politischen Philosophen wirft, denen Voegelins ganz besondere Hochschätzung
% gilt. 

\section{Gibt es spirituelle Sachzwänge?}

Wenn eine politische Ordnung zusammenbricht, so kann das daran liegen, daß sie
an sich selbst gescheitert ist, oder an äußeren Bedingungen, die ihren
Zusammenbruch herbeigeführt haben. In ähnlicher Weise ist hinsichtlich des
spirituellen Bereiches zu unterscheiden, ob die Vernachlässigung der
Spiritualität den Menschen das Zusammenleben unmöglich macht, oder ob die Welt
von mythischen Gesetzmäßigkeiten durchwaltet ist, deren Nicht-Berücksichtigung
sich an den Menschen rächt. Deutlicher noch könnte in theologischer Sprache
formuliert werden: Eine gottlose Gesellschaft kann entweder an ihrer eigenen
Gottlosigkeit zu grunde gehen oder vom Zorn der Götter vertilgt werden. Die
Frage, ob eine gottlose Gesellschaft lebensfähig ist, wird in den folgenden
Abschnitten untersucht werden. An dieser Stelle steht die Frage im
Vordergrund, ob die Menschen die Rache der Götter fürchten müssen.

Nun glauben heutzutage sicherlich nur noch wenige Menschen ernsthaft an die
Rache der Götter, andererseits ist nicht zu bestreiten, daß sich das Denken in
Kategorien mythischer Gesetzmäßigkeiten noch immer einer erstaunlichen
Lebendigkeit erfreut. Dies macht sich beispielsweise im ökologischen Bereich
bemerkbar, wenn versucht wird, die Unvermeidbarkeit ökologischer Katastrophen
bereits aus einer falschen Einstellung zur Natur abzuleiten, einer
Einstellung, die durch eine menschliche Herrschafts- und
Bemächtigungsbeziehung zur Natur geprägt sei, wobei dann Heidegger oder gar
Ludwig Klages als Propheten eines Unheils angerufen werden, von dem sie noch
nichts haben ahnen können.\footnote{Vgl. Franz Tenigl: Einführende Worte zur
  Bedeutung von Ludwig Klages, in: Steffi Hammer (Hrsg.): Widersacher oder
  Wegbereiter? Ludwig Klages und die Moderne, Berlin 1922, S.9-13 (S.12).}

Im politischen Bereich kann sich ein ähnlicher Aberglaube in der Ansicht
äußern, daß es ein geheimnisvolles Gesetz des Maßes gäbe, das ungestraft kein
Mensch und besonders kein Politiker überschreiten dürfe. Ein solches Gesetz
des Maßes der Dinge stellt Camus am Schluß seines großen politischen Essays
"`Der Mensch in der Revolte"' auf.\footnote{Albert Camus: Der Mensch in der
  Revolte. Essays, Hamburg 1997, S.331-335.} Er glaubt, daß sich dieses Maß
auf allen Ebenen des Seins, in den jüngsten Entdeckungen der Physik ebenso wie
in der Politik, widerspiegelt. Die Verletzung dieser Grenze ist es, die die
legitime Revolte in den Wahnsinn von Revolution und Massenmord überführt. Aber
Camus' Ausführungen zum Gesetz des Maßes beschränken sich nicht auf die
mythische Ebene. Sie enthalten einen rationalen Kern, den Camus zuvor
ausführlich darlegt und nun im Gedanken des Maßes in einer Formel
zusammenfaßt. Dieser rationale Kern beinhaltet unter anderem die Anerkennung
der Würde eines jeden Menschen, insbesondere auch des Feindes, sowie das
Eingeständnis der eigenen Fehlbarkeit und Irrtumsmöglichkeit. Letzteres ist
wichtig, da bei einer bloß wahrscheinlichen Prophezeiung die revolutionäre
Rechnung nicht mehr aufgeht, welche im Namen der Prophezeiung den Massenmord
rechtfertigt. Bei Voegelin bleibt es oft weniger deutlich, ob und welchen
rationalen Kern er den von ihm interpretierten Mythologien und Philosophien
bzw. seinen eigenen, gelegentlich mythologisierenden Ausführungen zu grunde
legt.\footnote{Mythologisierend verhält sich Voegelin beispielsweise, wenn er
  die Beachtung des Leib-Geist-Dualismus und der Stufenhierachie des Seins in
  der politischen Theorie einfordert, und in der Mißachtung dieser Prinzipien
  eine Ursache theoretischer und in der Folge auch gesellschaftlicher
  Verwirrung sieht. Was das historische Verständnis betrifft, so ist Voegelin
  zwar zuzustimmen, wenn er die allegorische und andere rationalisierende
  Interpretationsweisen entschieden ablehnt. Aber wenn die Einsichten der
  Alten für das neuzeitliche Bewußtsein fruchtbar gemacht werden und gar eine
  purifizierende Wirkung entfalten sollen, dann ist es unerläßlich, ihren
  rationalen Gehalt herauszuschälen, da die mythische Denk- und Sprechweise
  uns nur in der Ratlosigkeit zurückläßt, und auf mythischer Ebene nicht das
  Richtige vom Falschen geschieden werden kann.}

Nun kann natürlich die Frage aufgeworfen werden, ob in der Welt nicht
vielleicht tatsächlich irgendwelche mythischen Gesetze wirksam sind, und
ob die Menschen nicht besser daran täten, sie genau zu beachten. Aber
hier stellt sich das Problem, daß sehr viele konkurrierende
Gesetzmäßigkeiten dieser Art denkbar sind, über deren Gültigkeit nicht
entschieden werden kann, solange sie nicht auf möglicherweise
vorhandene, rational erkennbare Zusammenhänge zurückgeführt werden.
Einige Philosophen glauben beispielsweise an das Gesetz des Maßes,
andere eher an die Gesetze der Dialektik.\footnote{Ich bezweifele, daß
  Hegels Dialektik etwas anderes ist als eine Art abstarkter Mythologie.
  Ganz sicher ist sie keine Logik, denn dazu fehlt ihr die
  Folgerichtigkeit, und ebensowenig ist sie eine Wissenschaft, denn sie
  erlaubt keinerlei Prognosen; sie erlaubt es überhaupt nicht,
  irgendwelche Sacherkenntnis zu gewinnen, die man nicht schon vorher
  hatte.} Eine Entscheidung, wer recht hat, kann nur getroffen werden,
wenn man weiß, worauf sie ihre Gesetze mit welchen Konsequenzen
beziehen. Und diese Entscheidung kann je nach dem behandelten
Einzelproblem verschieden ausfallen. Wenig aussichtsreich erscheint es
jedoch, bereits auf der Ebene mythischer Allgemeinheit eine Entscheidung
zu treffen, wie Voegelin dies anzustreben scheint, wenn er die
Grundbegriffe philosophischer Denker als Symbole versteht, die
spirituelle Erfahrungen artikulieren, deren Echtheit und Tiefe es zu
bestimmen gilt.

Insgesamt kann festgehalten werden, daß es keine mythischen Gesetzmäßigkeiten
gibt, denen die politische Ordnung Rechnung tragen müßte. Die äußeren
Bedingungen, denen eine politische Ordnung Genüge leisten muß, sind die
"`Naturgesetze"' der Politik und der Gesellschaft sowie die nicht zu
vernachlässigende Tatsache, daß die menschliche Kenntnis dieser Zusammenhänge
äußerst beschränkt ist. Die Nicht-Beachtung dieser Zusammenhänge kann zu
politischen Torheiten führen. Aber dies ist etwas anderes als der Frevel einer
Übertretung mythischer Gesetze, vor dem man sich fürchten müßte, weil er die
Rache der gestörten Weltordnung heraufbeschwören würde. Da sich eine
Spiritualität in der Weltordnung nicht bemerkbar macht, stellt sie keine
äußere Bedinung des politischen Handelns sondern höchstens noch einen
psychologischen Faktor dar.

\section{Bedarf die Legitimation der politischen Ordnung einer religiösen Komponente?} 

Ein entscheidendes Problem einer jeden politischen Ordnung, bei welchem der
Rückgriff auf regligiöse Wahrheiten naheligend erscheinen könnte, ist das
Problem der Legitimation der politischen Ordnung. Die Legitimation erfüllt
eine zweifache Funktion. Zum einen soll sie die grundsätzliche Zustimmung der
Herrschaftsunterworfenen zur Herrschaftsordnung sicherstellen. Zum anderen
dient sie der Motivation von Einsatzbereitschaft für den eigenen
Herrschaftsverband, was besonders im Kriegsfall von großer Bedeutung ist. Es
stellt sich nun die Frage, ob eine Legitimation politischer Ordnung ohne
welttranszendente Bezugspunkte möglich ist, und ob sie genügend Intensität
erreicht, um die Stabilität des politischen Systems auch in Krisenzeiten zu
gewährleisten.

Die heutzutage in der westlichen Welt übliche Form der Legitimation ist die
einer Gesellschaftsvertragstheorie. Die Gesellschaftsvertragstheorie
legitimiert dabei sowohl die Existenz eines Staates überhaupt als auch im
besonderen die demokratische Herrschaftsform. Die Existenz des Staates wird
dadurch legitimiert, daß ohne Staat der Einzelne vor Übergriffen von
seinesgleichen auf sein Leben und Vermögen keinen Augenblick sicher ist, so
daß die Menschen ohne Staat ohnehin nichts besseres tun könnten, als durch
Vertrag einen Staat zu gründen, der sie voreinander beschützt. Die
demokratische Herrschaftsform wird dadurch legitimiert, daß sie diejenige
Herrschaftsform ist, zu der die Menschen in einem auf Basis freier Zustimmung
geschlossenen Vertrag am ehesten ihre Zustimmung geben könnten, da sie ihnen
nicht nur vor den Übergriffen der Mitbürger sondern auch vor dem
Machtmißbrauch des Herrschers die größte Sicherheit bietet.\footnote{Ich
  beziehe mich hier in erster Linie auf die Hobbessche
  Gesellschaftsvertragstheorie unter Berücksichtigung der Lockeschen Kritik
  dieses Modells.}

Die Gesellschaftsvertragstheorien rechtfertigen die Existenz des Staates und
die demokratische Herrschaftsform, indem sie sich rational einleuchtender
Argumente bedienen. Der Sinn des Staates wird dabei hinreichend durch den
Zweck der Schaffung innerer Sicherheit erklärt, ein Zweck der, so sollte man
meinen, im Eigeninteresse eines jeden Menschen liegt. Eine zusätzliche
Legtitimation, etwa durch göttliche Autorität, könnte innerhalb dieses
Gedankenganges sogar problematisch erscheinen, denn, wenn es nicht schon
genügend rationale Gründe gäbe, um die Existenz des Staates zu legitimieren,
dann hätte der Staat ohnehin kein Existenzrecht, und alle weiteren
Rechtfertigungen seiner Existenz müßten als Ideologie verworfen werden.

Eine Legitimation politischer Ordnung ohne welttranszendente Bezugspunkte
scheint also grundsätzlich möglich zu sein, wenn man voraussetzt, daß die
Menschen vernünftig genug sind, um ihre eigenen Interessen zu erkennen.
Erweist sich diese Art der Legitimation aber auch als krisenfest, wenn die
politische Ordnung vor besonderen Herausforderungen steht? Sind die liberalen
Demokratien im Falle eines Krieges in der Lage, ohne die Mobilisierung
religiöser Energien in genügendem Maße Opferbereitschaft für sich zu
motivieren? Und handeln sie sich bei der Auseinandersetzung mit ideologischen
Bewegungen im Inneren nicht einen entscheidenden Nachteil dadurch ein, daß sie
die religiösen Gefühle der Bürger unangestastet lassen müssen (und
wollen)?\footnote{Vgl. Joachim Fest: Die schwierige Freiheit. Über die offene
  Flanke der offenen Gesellschaft, Berlin 1993, S.38ff.}

Gegen die erste dieser Befürchtungen kann eingewandt werden, daß auch in den
liberalen Demokratien angesichts äußerer Herausforderungen gesellschaftliche
Mechanismen wirksam werden, die die Abwehrbereitschaft der demokratischen
Gesellschaft erheblich stärken. So macht sich im Falle eines Krieges oft eine
Art von innerem Zusammenrücken der Gesellschaft bemerkbar, das sich
beispielsweise in einer schlagartigen Zunahme der Beliebtheitswerte der
jeweiligen Regierung äußern kann. Auch haben sich beispielsweise im Zweiten
Weltkrieg die Soldaten, die auf Seiten der liberalen Demokratien kämpften,
nicht weniger tapfer geschlagen als die Armeen der totalitären Regime, was
beweist, daß im Ernstfall durch quasi-religiöse Sinnversprechungen keine
wesentlichen Vorteile zu erzielen sind. Weit entfernt davon, eine
Schwachstelle der liberalen Ordnung zu offenbaren, können äußere
Herausforderungen diese Ordnung zeitweise sogar erheblich stärken.

Ebensowenig zwingend ist das Argument, daß ein rein rational legitimiertes
System keine ausreichende Immunität gegen die verführerische Kraft
chiliastischer politischer Bewegungen im Inneren entwickeln könnte.  Zumindest
ist nicht unmittelbar ersichtlich, wie eine spirituelle oder religiöse
Legitimationskomponente hier Abhilfe schaffen könnte. Jede Form der
Legitimation kann zusammenbrechen, wenn das politische System, das sie
legitimiert, sich als erfolglos erweist oder wenn sie durch eine vom Geist der
Zeit als überzeugender empfundene Legitimation herausgefordert wird. Dies
würde auch für eine Legitimation auf Basis der existentiellen Spannung zum
transzendenten Seinsgrund gelten, ganz gleich, welches Maß philosophischer
Wahrheit diese Legitimation für sich beanspruchen dürfte. Dem heutigen
Zeitgeist würde eine derartige Legitimation sicherlich recht unglaubwürdig
erscheinen. Dies widerspricht zwar nicht ihrer philosophischen Richtigkeit,
läßt es aber unangebracht erscheinen, sie in der Gegenwart als vorbeugende
Arznei gegen Ideologien zu empfehlen. Darüber hinaus ließe sich die Überlegung
anstellen, daß gerade spirituelle Legitimationskomponenten ein Einfallstor für
Ideologien darstellen könnten, da durch sie dem Irrationalismus bereits
öffentlicher Glaubwürdigkeitskredit eingeräumt wird.

% Andererseits gibt es durchaus gravierende Einwände, die gegen religiöse oder
% spirituelle Legitimationskomponenten sprechen. So stellt sich in einer
% pluralistischen Gesellschaft die nicht unerhebliche Frage, woher die
% spirituelle Wahrheit zur Legitimation der politischen Ordnung bezogen werden
% soll. Und unabhängig davon, kann eine spirituelle Legitimation
% ernsthaft nur dann gefordert werden, wenn auch irgend eine spirituelle
% Wahrheit vorweisbar ist, in deren Namen die Legitimation vorgenommen wird.
% Ohne spirituelle Wahrheit kann es keine spirituelle Legitimation geben, auch
% wenn sie noch so nützlich wäre.

Auch wenn das Problem der hinreichenden Legitimation politischer Ordnung mit
großen Unsicherheiten behaftet ist (da sich nicht bloß die Frage stellt,
wodurch eine politische Ordnung philosophisch gerechtfertigt ist, sondern vor
allem, wann eine politische Ordnung als gerechtfertigt empfunden wird) scheint
der Rückgriff auf religiöse, spirituelle oder existentielle Wahrheiten für die
Legitimation politischer Ordnung nicht zwingend erforderlich zu sein.

\section{Wertbegründung und -konsens in der pluralistischen
  Gesellschaft} 

Ein wichtiges Argument, welches für die Religion und besonders für eine
stärkere Geltung der Religion im gesellschaftlichen Leben angeführt
werden könnte, beruht auf dem philosophischen Problem der
Letztbegründung ethischer Wertbe. Dieses Argument lautet in etwa wie
folgt: Keine Gesellschaft, so könnte argumentiert werden, kann ohne
einen Satz verbindlicher ethischer Grundwerte existieren. Es wäre aber
absurd, diese Grundwerte, die absolut gelten müssen, zur Disposition
eines Konsensfindungsverfahrens zu stellen, sei dies nun eine
verfassungsgebende Versammlung oder auch nur ein gedachter
Gesellschaftsvertrag, zumal dann immer noch die Gültigkeit des
Verfahrens als Wertvoraussetzung übrig bliebe. Zugleich zeigt die
Philosophiegeschichte, daß alle Versuche einer rein säkularen
Letztbegründung ethischer Werte zum Scheitern verurteilt sind. Mit
anderen Worten: Wenn es Gott nicht gäbe, dann wäre alles erlaubt. Also
muß das religiöse Bewußtsein in der Gesellschaft mindestens noch so wach
sein, daß die Verbindlichkeit der Grundwerte anerkannt wird.

Stimmt dieses Argument, und ist die Religiösität damit tatsächlich
unverzichtbar?  An diesem Ergebnis scheint kein Weg vorbeizuführen, denn wenn
eine philosophische Letztbegründung der Ethik nicht möglich ist, dann bleibt
als einzige Form der Wertbegründung ein ethischer Dezisionismus übrig, d.h.
jeder wählt sich seine Werte selbst aus, und wenn jemand die Wahl trifft,
überhaupt keine Werte zu beachten, dann ist dies genauso möglich. Diese
theoretische Konsequenz ist gewiß sehr ernüchternd. Aber kann die Religion
überhaupt Abhilfe schaffen? Das ist überaus zweifelhaft, denn durch eine
religiöse Wertbegründung würde das Begründungsproblem nicht gelöst, sondern
nur auf die Religion verschoben werden. Dadurch dürfte das Begründungsproblem
aber eher noch komplizierter werden, da außer den Werten nun auch die Wahrheit
von religiösen Dogmen, Erfahrungen, Mysterien und anderen heiligen
Gegebenheiten auf dem Prüfstand steht. Zwischen konkurriernden religiösen
Glaubensüberzeugungen objektiv zu entscheiden ist aber so gut wie unmöglich.
Die Anerkennung einer Religion beruht letzten Endes auf einem Glaubensakt und
damit nicht weniger auf einer persönlichen Entscheidung als die sittlichen
Werte nach der Theorie des ethischen Dezisionismus.

Eine Letztbegründung oder gar ein regelrechter Beweis ethischer Werte
scheint also unmöglich zu sein. Die universelle Verbindlichkeit
bestimmter Werte läßt sich daher nur noch auf Basis eines Konsenses
erreichen, auch wenn dies dem Charakter ethischer Werte als
unverfügbarer Werte zu widersprechen scheint.  Dabei dürfte es
höchstwahrscheinlich sogar aussichtsreicher sein, den Konsens auf der
Ebene der Werte als auf der Ebene der philosophischen oder religiösen
Begründung der Werte zu suchen. Denn darüber, daß töten oder stehlen
verwerflich ist, läßt sich gewiß leichter eine Einigung erzielen als
über die Frage, ob Allah oder der liebe Gott oder die philosophische
Vernunft der legitime moralische Gesetzgeber ist. Und dort, wo
unversöhnliche Wertauffassungen aufeinanderprallen, würde es eine
Einigung erst recht erschweren, wenn der Streit zuerst auf der
metaphysischen bzw. existentiellen Ebene entschieden werden soll. Von
großer Bedeutung ist dabei, daß die Akzeptanz von Werten nicht zwingend
durch die existentielle Haltung eines Menschen bedingt ist, sondern daß
sie auch auf Grund der Einsicht in die Nützlichkeit eines Wertes für das
gesellschaftliche Zusammenleben erfolgen oder im Dialog mit anderen
vereinbart werden kann. Ein Wertkonsens ist daher grundsätzlich auch
ohne einen einheitlichen spirituellen Erfahrungshintergrund der
Beteiligten denkbar.

Auch in der Frage des Wertbegründung und des gesellschaftlichen
Konsenses über bestimmte Grundwerte lautet daher das Ergebnis, daß der
Rückgriff auf die Spiritualität eher hinderlich als förderlich und in
jedem Falle unnötig ist.

% Daher versuchen auch verfeindete
% Parteien zur Herstellung eines Dialoges zunächst irgendeine gemeinsamen Basis
% zu finden, während die Differenz in den Grundüberzeugungen noch lange bestehen
% bleiben kann. Gewiß, die Situation bleibt prekär, solange der gemeinsame
% Standpunkt nur aus politischem Kalkül oder aus Furcht vor dem anderen
% anerkannt wird. Aber Vertrauen kann bereits dann entstehen, wenn der
% gemeinsame Standpunkt oder die gemeinsamen Werte von den unterschiedlichen
% Gruppen aus je eigener Überzeugung anerkannt wird. Denkbar ist im übrigen, daß
% eine Gruppierung eine andere zur Anerkennung bestimmter Werte auf der Basis
% utilitarischer Argumente bewegen kann, auch wenn für sie selbst diese Werte
% aus viel tieferen Gründen notwendig erscheinen.

\section{Sinngebung durch die politische Ordnung?}

Aus Voegelins Sicht müßte ein ethischer Wertkonsens jedoch als ein höchst
brüchiges Fundament der gesellschaftlichen Ordnung beurteilt werden, sofern er
sich nicht auf einen einheitlichen spirituellen Erfahrungshintergrund stützen
kann. Dies hängt unter anderem damit zusammen, daß Voegelin die
Dialogmöglichkeiten zwischen Menschen mit unterschiedlichem spirituellem
Erfahrungshintergrund überaus skeptisch beurteilt, was sogar soweit führt, daß
Menschen ohne spirituelle Erfahrungen von Voegelin als potentielle
Ordnungsstörer eingestuft werden. Ähnliche Auffassungen kehren auch bei
manchen Anhängern Voegelins wieder. So wurde die Ansicht, daß die Gegensätze
zwischen Menschen, die an die Existenz eines transzendenten Seins glauben, und
Menschen, die sie bestreiten, weitgehend unversöhnlich bleiben müssen, solange
über diese Schlüsselfrage nicht Einigkeit erzielt worden ist, unlängst von
Thomas J. Farell bekräftigt, der in diesem Zusammenhang die Leugnung der
Existenz eines transzendenten Seins unter Berufung auf prominente Psychologen
wie C.G.Jung und ganz auf der Linie Voegelins als eine Art Geisteskrankheit
deutet. Allerdings räumt auch Farell ein, daß es Profanbereiche gibt,
innerhalb derer ein fruchtbarer Dialog zwischen Menschen, die jene
Schlüsselfrage unterschiedlich beantworten, möglich ist.\footnote{Vgl. Thomas
  J.  Farrell: The Key Question. A critique of professor Eugene Webbs recently
  published review essay on Michael Franz's work entitled "'Eric Voegelin and
  the Politics of Spiritual Revolt: The Roots of Modern Ideology"', in:
  Voegelin Research News, Volume III, No.2, April 1997, auf:
  http://vax2.concordia.ca/\~{ }vorenews/v-rnIII2.html (Host: Eric Voegelin
  Institute, Lousiana State University. Zugriff am: 5.3.2000).} Die Frage
stellt sich nun, ob die politische Ordnung zu diesen Profanbereichen des
menschlichen Lebens gehört.

Damit ist zugleich eine Grundfrage des Wesens politischer Ordnung
angeschnitten: Ist die (in der Neuzeit stets durch den Staat
repräsentierte) politische Ordnung nur ein Mittel zu bestimmten Zwecken
wie etwa der Schaffung innerer und äußerer Sicherheit, oder ist sie
darüber hinaus Ausdruck einer historischen Suche nach Ordnung, die mit
dem Sinn der Welt und dem Sinn des Lebens in Zusammenhang steht? Im
ersteren Fall kann die politische Ordnung voll und ganz dem
Profanbereich zugeordnet werden, so daß eine Einigung über alle
wesentlichen Prinzipien der politischen Ordnung auch zwischen Menschen
mit unterschiedlicher Offenheit der Seele im Bereich des Möglichen
liegt. Nur im letzteren Fall müßte zunächst eine gesellschaftlich
verbindliche Entscheidung über die metaphysische Schlüsselfrage der
Existenz transzendenten Seins getroffen werden.

Welche dieser beiden grundverschiedenen Wesensauffassungen politischer Ordnung
ist nun aber die richtigere? Um diese Frage zu beantworten, empfiehlt es sich,
von unterschiedlichen Funktionen des Politischen auszugehen, einer
Friedenssicherungsfunktion und einer spirituellen Funktion, und dann zu
klären, in welcher Beziehung sich diese Funktionen zueinander befinden, d.h. 
insbesondere, ob die politische Ordnung die Friedenssicherungsfunktion nur
erfüllen kann, wenn sie auch spirituelle Funktionen erfüllt. Sollte sich
herausstellen, daß sich beide Funktionen trennen lassen, dann kann als
nächstes die Frage gestellt werden, welche der beiden Funktionen die für
die politische Ordnung wesentlichere ist, und ob es nicht günstiger
wäre, die andere Funktion innerhalb eines anderen Rahmens zu erfüllen,
also etwa die spirituellen Ziele nicht auf der Ebene der politischen Ordnung
sondern im Rahmen privater religiöser Vereinigungen zu verfolgen.  

Geht man zunächst einmal davon aus, daß die Stiftung inneren Friedens die
Kernfunktion politischer Ordnung ist, so kann man überlegen, was mindestens zu
einer politischen Ordnung gehören muß, damit sie diese Kernfunktion erfüllen
kann. Sicherlich sind für die Erfüllung der Kernfunktion der Friedenssicherung
Herrschaftsinstitutionen notwendig, die die Einhaltung des Friedens
garantieren. Weiterhin müssen sich die Herschaftsinstitutionen auf die
Loyalität oder wenigstens den regelmäßigen Gehorsam der Bürger stützen können.
Eine politische Ordnung, die die Kernfunktion der Friedenssicherung erfüllen
soll, bedarf daher auch einer Legitimation, wozu mindestens eine politische
Philosophie oder Herrschaftsideologie vorhanden sein muß, die den Bürgern den
Zweck der politischen Ordnung erklärt. Dann könnte eingewandt werden, daß ein
echter Frieden noch gar nicht vorhanden ist, solange nicht auch Gerechtigkeit
herrscht. Es wären also auch noch Vorkehrungen für die Gerechtigkeit zu
treffen usw. Führt man diese Überlegungen weiter fort, so gelangt man
irgendwann einmal zu einer politischen Mindestordnung, die alles umfaßt, was
notwendig ist, um die Kernfunktion der Friedenssicherung zu erfüllen. Gehört
zu dieser Mindestordnung bereits die Funktion der
Sinnvermittlung?\footnote{Unter Sinnvermittlung ist zu verstehen, daß die
  politische Ordnung in ihrer Gestalt Ausdruck der in spiritueller Erfahrung
  erlebten sinnhaften Seinsordnung ist, die sie zugleich ihren Mitgliedern
  weitervermittelt. Dies trifft die Intention Voegelins besser als der (an
  sich verständlichere) Ausdruck Sinngebung, da nach Voegelins Verständnis die
  politische Ordnung keinesfalls die Quelle des Sinns ist, sondern idealiter
  in die sinnhafte Gesamtordnung der Welt eingebettet ist.} Nach den
Überlegungen der vorhergehenden Abschnitte ist dies wahrscheinlich nicht der
Fall, denn die politische Ordnung bedarf des Rückgriffes auf die spirituelle
Erfahrung weder zur Legitimation noch um der Begründung verbindlicher Werte
willen, noch ist die spirituelle Erfahrung bei der Bewältigung politischer
Probleme von Vorteil. Also ist die Sinnvermittlungsfunktion, wenn überhaupt,
eine rein optionale Funktion politischer Ordnung, soweit unter politischer
Ordnung die eben angedeutete Mindestordnung zu verstehen ist. Neben der
Sinnvermittlungsfunktion sind noch weitere solcher optionaler Funktionen
politischer Ordnung denkbar (z.B. Sozialstaatlichkeit\footnote{Historisch
  hatte die Entwicklung des Sozialstaates natürlich durchaus einiges mit der
  Sicherung des inneren Friedens zu tun, aber für die theoretische Frage, ob
  und warum der Staat sozialstaatliche Aufgaben übernehmen soll spielen
  historisch-kontingente Tatsachen nur bedingt eine Rolle.}). Solche
Funktionen der politischen Ordnung zuzurechnen, ist dann empfehlenswert, wenn
ihre Erfüllung am ehesten oder sogar einzig und allein auf der Ebene der
politischen Ordnung möglich ist, und wenn dabei keine gravierenden Nachteile
entstehen. Nun kann die Sinnvermittlungsfunktion aber zweifellos auch anders
als im Rahmen der politischen Ordnung erfüllt werden. Die Vermittlung von
Lebensinn, die sinnhafte Deutung der Welt und die Erfüllung menschlicher
Transzendenzbedürfnisse kann, wenn schon nicht individuell, so doch auf jeden
Fall im Rahmen von Kirchen und Religionsgemeinschaften geleistet werden. Es
tut der Spiritualität also keinerlei Abbruch, wenn ihr nur ein Platz außerhalb
der politischen Ordnung angewiesen wird, während andererseits nicht einzusehen
ist, welche Vorteile entstehen, wenn sie der politischen Ordnung aufgebürdet
wird.

Die Tatsache, daß die Spiritualität keineswegs darunter leiden muß, wenn
sie nicht als Bestandteil der politischen Ordnung betrachtet wird,
scheint Voegelin zu übersehen, wenn er es den
Gesellschaftsvertragstheorien zum Vorwurf macht, daß sie sich nur auf
die leibliche Seite des Menschen konzentrieren und die geistige Seite
des Menschen vernachlässigen.\footnote{Vgl. Voegelin, Anamnesis,
  S.341/342.} Seinem Vorwurf liegt ein fundamentales Mißverständnis des
Zweckes politischer Ordnung zu Grunde. Die Notwendigkeit politischer
Ordnung entsteht letztlich aus dem Umstand, daß Menschen einander in die
Quere kommen können und deshalb Abmachungen treffen müssen, damit dies
nicht geschieht. Politik hat daher ihrem Wesen nach mehr mit der
niederen, materiellen Sphäre der unumgehbaren Notwendigkeiten zu tun als
mit der geistigen Sphäre. Es ist deshalb ein Irrtum, von der politischen
Ordnung den Ausdruck spiritueller Wahrheit zu verlangen.  Und der
Verzicht darauf bedeutet keinesfalls eine Leugnung des Geistes, da
gerade nach den liberalen Gesellschaftvertragstheorien die politische
Ordnung gar nicht beansprucht, das ganze Wesen des Menschen zu umfassen.

Umgekehrt wäre es höchst prekär, religiöse Erfahrungen zu einer Angelegenheit
von politischer Bedeutung zu erklären. Denn wenn die politische Ordnung auf
eine Erfahrung der Transzendenz gegründet wird, dann wird die Religiosität zu
einer Frage der politischen Ordnung. Sie dürfte dann nicht mehr im Belieben
des Einzelnen stehen, was erhebliche Probleme für die Religionsfreiheit und
Toleranz aufwirft. Dann wäre es in der Tat nur konsequent, nach dem Irrenarzt
zu rufen, wenn es Menschen geben sollte, die es wagen, die Transzendenz zu
leugnen.  Ansonsten ist die Leugnung der Transzendenz eine sehr harmlose
"`Krankheit"', denn sie beeinträchtigt weder das Lebensglück der Befallenen,
noch hindert sie sie daran, die Rechte ihrer Mitbürger zu respektieren.

Als Gesamtergebnis läßt sich festhalten, daß weder die politische
Ordnung der Transzendenzerfahrungen bedarf, noch die Realisierung bzw.
der Ausdruck der Transzendenzerfahrungen durch die politische Ordnung
geschehen muß. Da andererseits die Forderung der Berücksichtigung
spiritueller Erfahrungen bei der Gestaltung politischer Ordnung
erhebliche ethische Bedenken hinsichtlich der Toleranz aufwirft, so
ergibt sich, daß Transzendenzerfahrungen bei der Gestaltung der
politischen Ordnung besser keine Rolle spielen sollten. Mit einem Wort:
Wenn es Gott gäbe, müßte man ihn ignorieren - wenigstens in der Politik.

% Auch der
% Ausdruck transzendenter Sinn- und Ordnungserfahrungen läßt sich weit besser
% durch die Schaffung von Sakralkunst verwirklichen als durch die
% transzendenzerfahrungsangemessene Gestaltung politischer Ordnung, bei man ja
% noch alles mögliche andere berücksichtigen müßte.

% Handelt es sich nicht um
% unversöhnliche Ansichten, zwischen denen nur eine willkürliche Wahl die
% Entscheidung treffen kann? Eine Möglichkeit, um zu einer rational begründeten
% Entscheidung zu gelangen, besteht darin, zunächst von zwei unterschiedlichen
% Funktionen des Politischen auszugehen, die diesen Auffassungen zu Grunde
% liegen, und dann zu klären, ob diese Funktionen notwendigerweise miteinander
% verbunden sein müssen. So würde nach der ersten Auffassung die Kernfunktion
% der politischen Ordnung eine Friedensstiftungsfunktion sein. Diese Funktion
% des Politischen zieht weitere Funktionen zwingend nach sich. So muß eine
% politische Ordnung, die längerfristig Frieden herstellen soll, auch in der
% Lage sein, Legitimität zu erzeugen. Die Friedensstiftungsfunktion zieht also
% beispielsweise eine Legitimierungsfunktion nach sich. Nun ist die Frage zu
% stellen, ob die Friedensstiftungsfunktion auch iregendwann einmal so etwas wie
% eine spirituelle Funktion des Politischen, von welcher die zweite der oben
% beschriebenen Auffassungen ausgeht, nach sich zieht. Nach allen bisherigen
% Überlegungen bedurfte die politische Ordnung jedoch weder zur Wertbegründung,
% noch zur Legitimation noch zur Bewältigung politischer Probleme des
% Rückgriffes auf die Spiritualität. Die Funktion der Friedenssicherung
% erfordert also nicht die Erfüllung spiritueller Funktionen. Betrachtet man die
% Friedenssicherung als Kernfunktion des Politischen so stellt die spirituelle
% Funktion höchstens noch eine optionale Funktion des Politischen dar - ähnlich,
% wie dies aus derselben Perspektive für die Sozialstaatlichkeit gelten würde.
% Umgekehrt kann nun die Frage gestellt werden, ob die spirituelle Funktion nur
% erfüllt werden kann, wenn sie als optionale Funktion dem Politischen
% zugeschlagen wird. Dies ist jedoch nicht der Fall, da sich die spirituellen
% Funktionen wie etwa die Sinngebung, oder die Erfüllung menschlicher
% Transzendenzbedürfnisse wenn schon nicht individuell so doch auf jeden Fall im
% Rahmen von Kirchen und Religionsgemeinschaften erfüllen lassen. Es tut der
% Spiritualität also keinerlei Abbruch, wenn ihr nur ein Platz außerhalb der
% politischen Ordnung angeweisen wird. 

% - wie es Anfang der 80er Jahre aus den Reihen der CDU gefordert
%   wurde -

% \footnote{Denkbar 
%   natürlich Rambo-Naturen, die den Thrill eines Lebens im kriegerischen
%   Naturzustand der Langeweile einer bürgerlichen Existenz vorziehen.}

\chapter{Schlußwort: Was bleibt von Eric Voegelin?}

Nach der scharfen Kritik, die in dieser Arbeit an Eric Voegelins
Bewußtseinsphilosophie geübt wurde, ist es angebracht, sich einen Überblick
darüber zu verschaffen, welche Rolle Eric Voegelin in der heutigen
wissenschaftlichen und politischen Diskussion noch spielen kann, und in
welcher Richtung die Auseinandersetzung über sein Werk fortzuführen wäre. Dazu
werde ich im folgenden drei Aspekte der Frage der Aktualität von Voegelins
Werk ansprechen: 1.Welche Bedeutung kommt Voegelins Philosophie zu? 2.Wie
aktuell sind seine Vorstellungen politischer Ordnung und politischer
Unordnung? 3.Gibt es dennoch ein politikwissenschaftliches Vermächtnis Eric
Voegelins, das fortzuführen sich lohnt.

\section{Zum Charakter von Voegelins Philosophie}

Die Philosophie Eric Voegelins halte ich, wie aus den bisherigen
Ausführungen sicherlich hervorgegangen ist, nicht für sonderlich
geglückt. Dabei halten nicht nur die Ergebnisse seiner Philosophie einer
kritischen Prüfung nicht stand, auch die Art seines Philosophierens ist
keinesfalls nachahmenswert.  Voegelins Philosophie, und dies gilt sowohl
für seine Geschichtsphilosophie als auch für seine
Bewußtseinsphilosopihe, ist eine überaus dogmatische Philosophie, sie
stellt außerdem eine hochgradig monologische Philosophie dar, und
darüber hinaus erscheint sie über weite Strecken als das, was Karl
Popper sehr treffend "`orakelnde Philosophien"' genannt
hat.\footnote{Vgl. Karl Popper: Die offene Gesellschaft und ihre Feine.
  Band II. Falsche Propheten: Hegel, Marx und die Folgen, 7.Aufl.,
  Tübingen 1992, S.262ff.}

Eine dogmatische Philosophie ist eine Philosphie, die ein Weltbild
artikuliert, ohne es zu begründen. Während eine kritische Philosophie
versucht, ihre Thesen rational zu begründen,\footnote{In Anlehnung an Kant
  wird unter einer kritischen Philosophie in der Regel eine
  erkenntniskritische Philosophie verstanden, aber diese Bedeutung ist etwas
  zu eng für eine praktikable Abgrenzung, zumal sie die Existenz einer
  verbindlichen Erkenntnistheorie als Appelationsinstanz voraussetzt.} findet
bei einer dogmatischen Philosophie gar keine oder nur eine tautologische
Begründung statt, oder eine Begründung durch Voraussetzungen, die ihrerseits
nicht weniger begründungsbedürftig sind als die begründeten Thesen. Damit ist
nicht gesagt, daß dogmatische Philosophien notwendigerweise schlechte
Philosophien sind, denn die Bildung und Ausgestaltung eines Weltbildes (oder
auch nur einer Unternehmensphilosophie) ist weder eine triviale noch eine
unbedeutende Aufgabe, aber dogmatische Philosophien können nur in begrenztem
Maße Objektivität für sich in Anspruch nehmen. Und genau in diesem Sinne ist
Voegelins Philosophie eine hochdogmatische Philosophie. Deutlich wird dies
immer wieder an metaphysischen Voraussetzungen wie der Existenz eines
transzendenten Seinsgrundes, der Ontologie der Seinsstufen, der Auffassung der
Geschichte als eines theogonischen Prozesses usw. Voegelins Philosophie wird
dadurch nicht uninteressanter, und die Darstellung seines Weltbildes ist ihm
einigemale auch in einer ästhetisch und rhetorisch sehr ansprechenden Weise
gelungen.\footnote{Dies gilt besonders für die Einleitungen von Order and
  History I und II. (Vgl. Voegelin, Order and History I, S.1ff. - Vgl.
  Voegelin, Order and History II, S.1-20.) - Fast noch schöner hat es Thomas
  Hollweck gesagt: Vgl. Thomas Hollweck: Truth and Relativity: On the
  Historical Emergence of Truth, in: Opitz, Peter J. / Sebba, Gregor (Hrsg.):
  The Philosophy of Order. Essays on History, Consciousness and Politics,
  Stuttgart 1981, S.125-136. - Für meinen Geschmack eher mißlungen und an eine
  schlechte Predigt erinnernd: Eric Voegelin: Ewiges Sein in der Zeit, in:
  Voegelin, Anamnesis, S.254-280.} Aber Voegelins Philosophie ist eben auch
nicht mehr als eine Philosophie. Sie ist nicht {\it die} Philosophie, und es
steht jedem Menschen frei, sich zu ihr zu bekennen oder sie abzulehnen.

Als problematischer stellt sich der monologische Charakter von Voegelins
Philosophie dar. Auch diese Eigenschaft hat Voegelins Philosophie mit
der Philosophie anderer Denker gemeinsam. Der monologische Charakter
findet sich bei Voegelin sowohl auf der Ebene des Philosophierens als
auch auf der Ebene seiner philosophischen Doktrin. Auf der Ebene des
Philosophierens äußert sich der monologische Charakter in Voegelins
heftigen polemischen Ausfällen, in seiner Weigerung, mit jedem, der
seine Grundüberzeugungen nicht teilt, auch nur ein Wort zu
reden,\footnote{Vgl. Conversations with Eric Voegelin. (ed. R.  Eric
  O'Connor), Montreal 1980, S.58ff.} und in der fast paranoiden
Vorstellung einer Ansteckungsgefahr, die von der vermeintlichen
Krankheit deformierter Existenz ausgeht, welche er hinter den von ihm
unerwünschten Philosophien allzeit vermutete. Wichtiger noch als diese
etwas schrulligen Äußerungen eines leidenschaftlichen intellektuellen
Temperamentes ist die Rolle des monologischen Prinzips innerhalb von
Voegelins Doktrin.  Philosophische Wahrheit wird für Voegelin immer von
Einzelnen erfahren und dann sprachlich an andere weitervermittelt, was
ein überaus schwieriger Prozeß ist, da die Erfahrung, die in gewisser
Weise auch eine Verständnisvoraussetzung bildet, im Anderen durch die
sprachliche Vermittlung erst angeregt werden muß.\footnote{Vgl. auch
  William C.  Haravard, Jr.: Notes on Voegelin's contributions to
  political theory, in: in: Ellis Sandoz (Hrsg.): Eric Voegelins
  Thought. A critical appraisal, Durham N.C. 1982, S.87-124
  (S.112-113).} Nach dieser Vorstellung von philosophischer Wahrheit ist
es unmöglich, daß Wahrheit im philosophischen Dialog gefunden wird, denn
die Erfahrung eines Menschen kann logischerweise nicht durch Argumente
eines anderen Menschen korrigiert werden. Die typische
Gesprächssituation, die Voegelins Philosophie zu Grunde liegt, ist daher
nicht der Dialog unter Gleichgestellten, sondern stets das belehrende
Gespräch, in welchem die Rollen von Lehrer und Schülern, von Führer und
Gefolgsleuten, von Prophet und Jüngern klar verteilt sind. Problematisch
erscheint am monologischen Charakter von Voegelins Philosophie, daß eine
legitime Pluralität von Weltanschauungen dadurch theoretisch ebenso
ausgeschlossen ist, wie die gegenseitige Befruchtung gegensätzlicher
Standpunkte. Pluralismus war für Voegelin beinahe gleichbedeutend mit
Verwirrung, und ein Philosoph, der die Wahrheit existentiell erfahren
hat, kann sich durch andere Standpunkte höchstens noch beirren lassen.

Für den heikelsten Punkt halte ich allerdings die philosophische
Geheimniskrämerei, zu der Voegelin nicht immer aber in seinen späteren
Schriften immer häufiger neigt. Ein philosophischer Geheimniskrämer ist
jemand, der das Rätsel und das Gefühl des Geheimnisvollen mehr liebt als
die Lösung der Rätsel. Voegelin hat sich in mehrfacher Weise der
philosophischen Geheimniskrämerei befleißigt. Dies beginnt mit Voegelins
oft unklarer und vieldeutiger Ausdrucksweise, es geht fort über die
nicht wenigen technischen Mängel seiner Philosophie, unter denen
insbesondere die Schlußfehler der {\it petitio principii} und des {\it
  non sequitur} einen prominenten Platz einnehmen, und der Höhepunkt ist
erreicht, wenn Voegelin sich auf Paradoxien und Mysterien beruft. Ich
gebe zu, daß dies eine höchst subjektive Kritik ist, und wer in Hegel
einen großen Philosophen sieht, der wird Voegelin wegen seiner
Denkfehler gewiß nicht tadeln wollen. Aber mir scheint, daß ein
Philosoph, der sich auf ein Mysterium beruft, mit demselben Mißtrauen
betrachtet werden sollte, wie ein Politiker, der sich auf sein Ehrenwort
beruft. Nicht daß von vornherein ausgeschlossen werden kann, daß es in
der Welt Mysterien gibt. Aber bei einem Mysterium hat alles Denken ein
Ende, und unter der Berufung auf Mysterien läßt sich jede beliebige
Behauptung aufstellen. Deshalb sollte zuerst eine genaue Prüfung
stattfinden, bevor die Annahme akzeptiert wird, daß ein Mysterium
vorliegt. In dieser Hinsicht scheint mir Voegelin in der Tat sehr
voreilig gewesen zu sein, wenn er etwa von einem Paradox des Bewußtseins
spricht, obwohl die Tatsache, daß das Bewußtsein die Welt wahrnehmen
kann, von der es selbst zugleich ein Teil ist, doch bestenfalls eine
staunenswerte Besonderheit aber gewiß kein Paradoxon ist.\footnote{Vgl.
  Voegelin, Order and History V, S.14-15.} Ebensowenig kann ich mich zu
der Ansicht durchringen, daß, wie Voegelin uns im letzten Band von
"`Order and History"' weismachen will, das Wort "`Es"' in dem Satz "`Es
regnet"' auf eine geheimnisvolle "`Es-Realität"' verweist, die die
Partner im Seien, Gott, Welt, Mensch und Gesellschaft
umgreift.\footnote{Vgl. Voegelin, Order and History V, S.16.} Vielleicht
gibt es Menschen, die in solchen Philosophemen den tiefsten Ausdruck
ihres ureigensten Welterlebens finden können. Für meinen Teil scheint
mir jedoch, daß Voegelin hier alle guten Grundsätze des klaren Denkens
in den Wind schlägt.

Was bleibt aber von Voegelins Philosophie, wenn sie tatsächlich so sehr von
Irrtümern und Denkfehlern gespickt sein sollte? Eines kann auch die schärfste
positivistische Kritik nicht verhindern; selbst wenn es ihr gelingt zu
beweisen, daß eine Philosophie durch und durch logischer Unfug ist, so kann
sie doch nicht verhindern, daß der Philosoph etwas damit gemeint
hat. Voegelins Philosophie ist daher mindestens der Ausdruck einer
Weltanschauung - einer reichen und tiefen Weltanschauung, wenn man sich an die
besseren seiner Schriften hält. Mehr ist sie nicht.

\section{Zur Frage der Aktualität von Voegelins Ordnungsentwurf}

Die Frage der Aktualität von Voegelins Ordnungsentwurf bedarf keiner
langen Erörterungen, da die Antwort hierauf eindeutig ausfällt, und sie
sich auch in der wissenschaftlichen Voegelin-Debatte mehr und mehr
durchzusetzen scheint.\footnote{Vgl. Eugene Webb: Review of Michael
  Franz, Eric Voegelin and the Politics of Spiritual Revolt: The Roots
  of Modern Ideology, in: Voegelin Research News, Volume III, No. 1,
  February 1997, auf: http://vax2.concordia.ca/\~{
    }vorenews/v-rnIII2.html (Host: Eric Voegelin Institute, Lousiana
  State University. Zugriff am: 5.3.2000).} Voegelins Vorstellung von
politischer Ordnung ist in hohem Maße bedingt und beeinflußt durch das
Zeitalter der Ideologien und des Totalitarismus, in welchem sie
entstanden ist. Voegelin hatte den raschen Zusammenbruch einer fragilen
Demokratie noch ebenso vor Augen, wie die unfaßbare Greuel des
Nationalsozialismus und das geisterhafte Schweigen über die Verbrechen
in der deutschen Nachkriegsgesellschaft. Nicht minder gegenwärtig waren
ihm die Verbrechen der kommunistischen Regime und die
Menschheitsbedrohung durch das atomare Wettrüsten. Unter solchen
Bedingungen kann ein Gefühl von Sicherheit nur schwer aufkommen, und
dies erklärt zum Teil Voegelins polemischen Eifer, welcher sich
wohlmöglich einem Gefühl der Dringlichkeit verdankt, welches nach dem
Ende des kalten Krieges nicht mehr unmittelbar verständlich wirkt. Die
Zeitumstände erklären auch einiges von dem, was man die metaphysische
Überhöhung des Politischen bei Voegelin nennen könnte. Aus heutiger
Sicht mag es sehr befremdlich und unwissenschaftlich wirken, die
metaphysische Kategorie des Bösen in die Politikwissenschaft
einzuführen.\footnote{Vgl. Voegelin, Eric: Die politischen Religionen,
  München 1996 (zuerst 1938).} Aber um mit einer Erscheinung wie dem
Nationalsozialismus fertigzuwerden erscheint dieser Versuch, wiewohl
wissenschaftlich fragwürdig, doch nicht ganz unverständlich. Vor dem
zeithistorischen Hintergrund ist es daher sehr wohl nachvollziehbar, daß
Voegelin sich nicht auf die Frage beschränkte, welches die geeignetesten
politischen Insititutionen für einen guten Staat sind, sondern dem Übel
an die Wurzel gehen wollte und nach den metaphysischen Bedingungen
wahrer politischer Ordnung fragte.

Indes leben wir heute mit einer liberalen politischen Ordnung, die seit
fünfzig Jahren stabil ist, und die auch keine Risse aufzuweisen scheint,
obwohl sich die Gesellschaft gegenüber den Fünfziger Jahren gewiß noch
weiter säkularisiert hat, was Voegelins Grundthesen über die Ursachen
politischer Unordnung doch sehr zweifelhaft erscheinen läßt. Dazu
vermitteln Voegelins Äußerungen über politische Ordnung nicht selten den
Eindruck, daß es Voegelin weit eher darauf ankam, eine wahre politische
Ordnung (nach den Maßstäben seiner privatreligiösen Überzeugungen) zu
finden, als eine im moralischen und pragmatischen Sinne gute politische
Ordnung. Als recht gravierend fällt dabei ins Gewicht, daß Voegelin in
seinem metaphysischen Eifer oft hart an der Grenze zum religiösen
Fanatismus operiert. Seine Suche nach der wahren Ordnung beschwört
dadurch die "`entgegengesetzte Gefahr"' (John H.  Herz\footnote{Vgl.
  John H. Herz: Politischer Realismus und politischer Idealismus.  Eine
  Untersuchung von Theorie und Wirklichkeit, Meisenheim am Glan 1959.
  Mit der "`entgegengestzten Gefahr"' bezeichnet Herz die besonders dem
  politischen Idealismus inhärente Gefahr bei der Bekämpfung politischer
  Mißstände durch das Mittel der Bekämpfung genau den entgegengesetzten
  Mißstand herbeizuführen. (Beispiel: Die Bekämpfung des
  kapitalistischen Ausbeutungssystems mündet in die kommunistische
  Diktatur.)}) der religiösen und weltanschaulichen Intoleranz herauf.
Nicht zuletzt aus diesem Grund ist uns heutzutage bei der Suche nach
guter politischer Ordnung mit etwas altmodischem Liberalismus weitaus
besser gedient als mit Voegelins metaphysischen Rezepturen.

\section{Das Vermächtnis Eric Voegelins}

Wenn Voegelins Philosophie nichts hergibt, und seine politische
Ordnungsvorstellung nichts taugt, wäre es dann nicht besser, Eric Voegelin
ganz zu vergessen? Zwei wichtige Gründe lassen es, trotz aller Kritik,
wünschenswert erscheinen, Eric Voegelin dem drohenden Vergessen zu entreißen.
Zum einen ist da Voegelins imposante Gelehrsamkeit. Voegelins Interpretationen
der Klassiker des politischen Denkens fallen zwar häufig sehr eigenwillig aus
 - nicht zuletzt deshalb, weil sich Voegelin meist nur auf ganz bestimmte und
scheinbar willkürlich ausgewählte Textpassagen bezieht. Aber Voegelins Auswahl
vollzieht sich fast immer vor dem Hintergrund einer profunden Kenntnis des
Gesamtwerkes. Wenn Voegelin daher auch die falsche Quelle ist, um sich über
die Klassiker des politischen Denkens zu informieren, so dürften Kenner eines
Denkers, den Voegelin behandelt hat, bei Voegelin fast stets einen
ausgefallenen Kommentar auf dem allerhöchsten Niveau finden. 

Als zweiter Grund, und als das eigentliche Vermächtnis Voegelins, gilt
es die bedeutende kulturwissenschaftliche Horizonterweiterung
festzuhalten, die die Politologie durch Eric Voegelin erfahren hat. In
dieser Hinsicht ist Voegelin gerade in der heutigen Zeit von großer
Aktualität, denn der Umgang mit fremden Kulturen erfordert auch für die
Politik und die Politische Theorie nicht nur eine Kenntnis der Gesetze
und Spielregeln von Diplomatie und Außenpolitik, sondern auch ein
Verständnis dieser Kulturen.  Wenn es um Fragen der Weltpolitik und der
zukünftigen Weltordnung geht, hat Voegelin daher noch immer ein
gewichtiges Wort mitzureden. Dabei ist zu hoffen, daß Voegelins Denken
als Vorbild für ein einfühlendes Verständnis fremder Kulturen dient, und
nicht im Fahrwasser von Samuel Huntingtons "`Zusammenprall der
Kulturen"' zur düsteren Prophetie unüberbrückbarer Gegensätze mißbraucht
wird.\footnote{Zur Aktualität Voegelins im Zusammenhang mit Huntingtons
  Theorie: Vgl.  Michael Henkel: Eric Voegelin zur Einführung, Hamburg
  1998, S.167.}

Von entscheidender Bedeutung für die Nachwirkung Eric Voegelins dürfte es
jedoch sein, daß die Diskussion um Voegelins Werk mit der notwendigen
kritischen Distanz geführt wird, was die bisherige Sekundärliteratur zu Eric
Voegelin eher vermissen läßt. Der sicherste Weg, Eric Voegelin zu einem
Nischendasein in den Zirkeln religiöser Sektierer zu verdammen, besteht
darin, seine Ressentiments, besonders seinen an Don Quichotte gemahnden Kampf
gegen echte und vermeintliche Gnostiker in allen Formen und Farben, zu einem
unverzichtbaren Wesensbestandteil seiner Wissenschaft zu
erklären.\footnote{Vgl.  Maben W. Poirier: VOEGELIN-- A Voice of the Cold War
  Era ...? A COMMENT on a Eugene Webb review, in: Voegelin Research News,
  Volume III, No.5, October 1997, auf: http://vax2.concordia.ca/\~{
    }vorenews/V-RNIII5.HTML (Host: Eric Voegelin Institute, Lousiana State
  University. Zugriff am: 5.3.2000).}  Gerade hier wäre es notwendig, eine
kritische Sonderung vorzunehmen, was bei Voegelin Wissenschaft und was
Vorurteil ist. Ein weiterer wichtiger Punkt, der zur Entmystifikation
Voegelins beitragen könnte, wäre sicherlich die Erforschung von Voegelins
Biographie, insbesondere seiner frühen Jahre. Voegelins geistige Entwicklung
ist zweifellos viel spannungsgeladener als seine Autobiographie dies vermuten
läßt, denn Voegelin war gerade in jungen Jahren von jenen irrationalistischen
Strömungen der Geisteskultur der Zwanziger und Dreißiger Jahre nicht wenig
beeinflußt, gegen deren politische Auswüchse er dann wissenschaftlich zu Felde
zog. Es liegen also noch reichlich Felder brach, welche die zukünftige
Voegelin-Forschung, wenn es eine geben sollte, bearbeiten könnte.


%  Auch steht
%   die Kritik an Poppers Platon-Interpretation nicht der Bedeutung von Poppers
%   Werk als ein Grundbuch des modernen Liberalismus entgegen, welches durch das
%   Schlagwort der "`offenen Gesellschaft"' ein höchst erfolgreiches Symbol für
%   das Selbstverständnis der pluralistischen Gesellschaft geschaffen hat.

% Dies dürfte auch
% einer der Gründe dafür sein, daß sich bei den Lesern Voegelins so häufig das
% Gefühl einstellt, Steine statt Brot gereicht zu bekommen, und daß durch die
% Lektüre eines Textes von Voegelin nicht selten im Leser ein Fragen in
% Verwirrung daüber angeregt wird, was Voegelin nun eigentlich hat mitteilen
% wollen. Schließlich wünscht der Leser eines wissenschaftlichen Buches über
% Politik nicht an einem meditativen Prozeß teilzunehmen, sondern Antworten auf
% konkrete Fragen zu erhalten. 

%%% Local Variables: 
%%% mode: latex 
%%% TeX-master: "Main" 
%%% End: 










