
%%% Local Variables: 
%%% mode: latex
%%% TeX-master: "Main"
%%% End: 


\section{"`Was ist politische Realität?"'}  

Der Aufsatz "`Was ist politische Realität?"', mit dem Voegelin sein Werk
"`Anamnesis"' beschließt, stellt eine umfassende Grundsatzarbeitet über das
Wesen politischer Realität und die Grundlagen einer diese Realität adäquat
beschreibenden Politikwissenschaft dar. Der Aufbau und die Argumentation des
Aufsatzes sind einigermaßen verwickelt, denn obwohl Voegelin im Vorwort zu
"`Anamnesis"' diesem Aufsatz "`eine umfassende und vorerst befriedigende
Neuformulierung der Philosophie des Bewußtseins"'\footnote{Voegelin,
  Anamnesis, S.8.} attestiert, verraten häufige Wiederholungen, begriffliche
Unklarheiten und gelegentliche Selbstkorrekturen innerhalb des Aufsatzes, daß
sich Voegelin seiner Sache keineswegs sicher war.  Daher gebe ich zunächt eine
kurze Übersicht über die wichtigsten Themenkomplexe, die sich aus Voegelins
Aufsatz extrahieren lassen, bevor dessen Inhalt im Einzelnen dargestellt und
kritisiert wird.

Der wohl wichtigste Themenkomplex dieses Aufsatzes bezieht sich auf den
Begriff der Realität. "`Realität"' ist bei Voegelin ein Inbegriff
absoluter metaphysischer Wahrheiten, die die Welt im Ganzen und die Stellung
des Menschen in der Welt betreffen.  Das Wissen um diese metaphyischen
Wahrheiten ("`Ordnungswissen"') wird dem Menschen durch ein inneres Gefühl
("`Ordnungserfahrung"') vermittelt. Eine politische Ordnung kann nur dann eine
gute politische Ordnung sein, wenn sie sich auf dieses Ordnungswissen
gründet.

Der zweite Themenkomplex betrifft Voegelins sprachphilosophische
Ausführungen.  Voegelin war der Ansicht, daß die Wörter, mit denen die
Ordnungserfahrung artikuliert wird, sich nicht wie gewöhnliche Wörter
auf etwas Gegebenes beziehen, das ihre Bedeutung ist, sondern daß sie
"`Indizes"' sind, die etwas über die innere Verfassung und über
besondere Erfahrungen des Bewußtseins vermelden.

Der dritte Themenkomplex behandelt die Beziehungen, die zwischen
unterschiedlich niveauvollen Formen des Ordnungswissens bestehen. Voegelin
zufolge können die Ordnungserfahrungen in einzelnen Fällen klarer oder
weniger klar und damit das ihnen korrespondierende Ordnungswissen niveauvoller
("`differenzierter"') oder weniger niveauvoll ("`kompakter"') sein. Dennoch
betreffen sie stets dieselbe Realität. Voegelin glaubt, daß es eine
geschichtliche Entwicklung von kompakterem zu immer differenzierterem
"`Ordnungswissen"' gibt.

Der vierte Themenkomplex bezieht sich auf den Verlust und das tragische
In-Vergessenheit-Geraten des Ordnungswissens. Voegelin unterscheidet
nicht nur zwischen kompaktem und differenziertem Ordnungswissen, sondern auch
zwischen Philosophien, die überhaupt Ausdruck von Ordnungserfahrungen sind,
und solchen Philosophien, die lediglich aus dogmatischer Begriffsklauberei
und leerer Spekulation bestehen. Zwar bleibt die Realität immer dieselbe,
aber sie kann in Vergessenheit geraten und das Ordnungswissen
schlimmstenfalls durch Ideologien verdrängt werden. Voegelin bezeichnet dieses
Phänomen als "`Realitätsverlust"', und er hält es für die Ursache von
politischen Katastrophen wie z.B. den Totalitarismus.


\subsection{Naturwissenschaft und Politikwissenschaft} 

Im einleitenden Teil seines Aufsatzes stellt Voegelin die Behauptung
auf, daß die Politische Wissenschaft von einer fundamental anderen Art
sei als die Naturwissenschaften, so daß die Politikwissenschaft nach
Voegelins Ansicht nicht zu einem durchgängig logisch zusammenhängenden
System von Aussagen ausgebaut werden kann.  Die Gründe hierfür sind für
Voegelin prinzipieller Natur: 1. Der Gegenstandsbereich der
Politikwissenschaft ist bereits durch nicht wissenschaftliche
Interpretationen besetzt. 2. Der Gegenstand (Politik) wird durch
Interpretationen des Gegenstandes selbst geformt. 3.  Unterschiedliche
Interpretationen der Politik, seien sie nun wissenschaftlicher oder
unwissenschaftlicher Art, streiten einander ihren Wahrheitsanspruch ab
und betrachten sich gegenseitig nur als Störfaktor innerhalb des
Gegenstandsbereiches, indem sie beispielsweise gegen die jeweils andere
Interpretation den Ideologievorwurf erheben.\footnote{Vgl. Voegelin,
  Anamnesis, S.284-285.}

Aus all dem schließt Voegelin, daß die Beziehung von Wissen und Gegenstand in
der Politikwissenschaft von grundsätzlich anderer Art ist als in den
Naturwissenschaften und daß daher die Politikwissenschaft auch eine besondere
Art von Wissen hervorbringen muß, welches Voegelin als "`noetische
Interpretation"' bezeichnet.\footnote{Vgl. Voegelin, Anamnesis, S.287.}

Die Gründe, die Voegelin andeutet, legen jedoch nur sehr bedingt die
Konsequenz der Wesensverschiedenheit von Politikwissenschaft und
Naturwissenschaft nahe.\footnote{Ich untersuche hier nur die Gründe, die
  Voegelin für diese These anführt. Eine Untersuchung, ob diese These, für die
  gewiß bessere Argumente ins Feld geführt werden können, grundsätzlich
  richtig ist, würde an dieser Stelle zu weit führen.} Der erste Grund gibt
eine Bedingung wieder, die in genau derselben Weise auch für die
Naturwissenschaft gilt, trifft sie doch ebenfalls auf religiöse,
abergläubische und esoterische Deutungen der Natur. Der dritte Grund besagt
lediglich, daß es bei den Deutungen der Politik anders als innerhalb der
Naturwissenschaften nicht nur eine kooperative, sondern auch eine feindliche
Konkurrenz gibt. Es folgt daraus jedoch nicht, daß die Politikwissenschaft
anders als eine Naturwissenschaft aufgebaut sein muß. Allein der zweite Grund
könnte diese Konsequenz rechtfertigen. Allerdings erläutert Voegelin weder, ob
und wie infolge dieser Selbstbezüglichkeit die konventionelle Theoriebildung
Gefahr läuft zu scheitern, noch zeigt er, wie die "`noetische Interpretation"'
derartige Probleme vermeidet. Aus dem ersten Teil von Voegelins Aufsatz
ergeben sich also keine stichhaltigen Gründe für die Vorteile oder
die Notwendigkeit des noetischen Verfahrens.

% Die noetische Interpretation könnte immer noch dann sinnvoll sein, wenn sich
% nachweisen ließe, daß die Gestalt politischer Ordnung kausal durch eben jene
% Bewußtseinerfahrungen bestimmt ist, deren angemessene Auslegung - wie aus den
% weitern Ausführung Voegelins hervorgeht - das Ziel der noetischen
% Interpretation bildet. In der Tat setzt Voegelin eine solche kausale
% Bedingtheit der politischen Ordnung durch eine bestimmte Klasse von
% Bewußtseinserfahrungen voraus. Allerdings gibt er keine Gründe dafür an. 

\subsection{Voegelins Begriff der Realität}
 
Im zweiten Teil seines Aufsatzes beschäftigt Voegelin sich mit dem Wesen und
der Rolle der noetischen Interpretation. Voegelin beginnt zunächst mit einigen
dogmatischen Voraussetzungen über den Ursprung politischer Ordnung.  Dann
entwickelt er am Beispiel des Aristoteles den Begriff der "`noetischen
Exegese"' der Realitätserfahrung und versucht die komplizierte Beziehung
zwischen der noetischen Exegese und der vergleichsweise primitiveren
mythischen Auslegung zu bestimmen. Darauf geht Voegelin auf die Schwächen der
aristotelischen Philosphie ein und leitet zu seiner eigenen Fortführung der
aristotelischen Exegese über, in deren Zentrum ein höchst eigentümlicher
Begriff der "`Realität"' steht. Schließlich geht Voegelin auf das Thema des
"`Realitätsverlustes"' und der seiner Ansicht nach daraus resultierenden
politischen Unordnung ein.
 
\subsubsection{Die "`Spannung zum Grund"' als Ursprung der Ordnung}

Politische Ordnung entspringt Voegelin zufolge in letzter Instanz einer
inneren Erfahrung des Menschen, der Erfahrung, geordnet zu sein "`durch
die Spannung zum göttlichen Grund seiner Existenz"'.\footnote{Voegelin,
  Anamnesis, S.287.} Von dieser Erfahrung "`strahlen"' in einer nicht
näher spezifizierten Weise "`die Interpretationen gesellschaftlicher
Ordnung aus"'.\footnote{Voegelin, Anamnesis, S.287.} Da diese Erfahrung
nicht gegenständlich ist, (ähnlich, vermutlich, wie auch eine Stimmung
oder das Lebensgefühl eines Menschen nicht gegenständlich sind) kann es
kein "`intersubjektives Wissen"' von der richtigen Ordnung geben, was
Voegelin später jedoch nicht daran hindert, auf der intersubjektiven
Verbindlichkeit der existenziellen Ordnungserfahrung zu bestehen. Im
Ringen um einen angemessenen Ausdruck für diese innere Erfahrung,
welches Anlaß für die verschiedensten Interpretationen der richtigen
Ordnung gibt, erblickt Voegelin den Ursprung von "`Spannungen in der
politischen Realität"'.\footnote{Voegelin, Anamnesis, S.288.} So
vielfältig die Interpretationen der Ordnung auch sind, so ist ihnen doch
gemeinsam, daß sie alle nur von einem Grund der Ordnung ausgehen, selbst
dann, wenn, wie zur Zeit des Aristoteles, das Faktum einer
Interpretationsvielfalt schon bekannt ist.  Daraus schließt Voegelin,
daß es auch tatsächlich nur einen Ordnungsgrund gibt. Dieser Schluß ist
jedoch aus mehreren Gründen fragwürdig: Erstens läßt sich der Befund des
Glaubens an einen einzigen Grund schwer mit polytheistischen Religionen
oder mit naturphilosophischen Elementelehren, die mehr als ein Element
annehmen (z.B. die vier Elemente Feuer, Wasser, Erde, Luft bei
Empedokles), vereinbaren.\footnote{Es kann berechtigterweise in Zweifel
  gezogen werden, ob es bei den Elementelehren der Vorsokratiker um die
  Bestimmung eines {\it Ordnungs-}grundes geht. Aber im Zusammenhang der
  Voegelinschen Interpretation der Philosophiegeschichte wäre diese
  Annahme konsequent.} Zweitens unterscheiden sich die Interpretationen,
die einen einzigen Grund annehmen, zum Teil sehr stark voneinander
hinsichtlich der Eigenschaften dieses Grundes. Es daher sehr fraglich,
ob in den unterschiedlichen Interpretionen derselbe Grund gemeint ist.
Drittens folgt daraus, daß es den Glauben an einen einzigen Grund gibt,
weder daß dieser Grund existiert, noch daß es auch in Wirklichkeit nur
ein einziger ist.

\subsubsection{Die "`noetische"' Exegese bei Aristoteles} 

Voegelin geht nun in einiger Ausführlichkeit auf die Metaphysik des
Aristoteles ein. Aristoteles hat nach Voegelins Auffassung als einer der
ersten Philosophen eine umfassende "`noetische Exegese"' des Bewußtseins
geliefert. Die "`noetische Exegese"' folgt historisch auf die rein mythische
Deutung der Ordnung. Sie entsteht, wenn das Bewußtsein des Menschen entdeckt
und infolge dessen der Grund der Ordnung in der inneren Erfahrung und nicht
mehr im Kosmos gesucht wird. Die Auslegung der Bewußtseinserfahrung ist es,
was Voegelin "`noetische Exegese"' nennt.\footnote{Vgl. Voegelin, Anamnesis,
  S.288.  Wörtlich spricht Voegelin davon, daß die "`noetische Exegese"' den
  "`Logos"' des Bewußtseins auslegt.}
 
Woraus entspringt das Bedürfnis nach einer noetischen Exegese? Voegelins
Aristoteles-Interpretation zufolge lebt der Mensch, der den Grund seiner
Existenz nicht kennt, in einem Zustand der Angst.\footnote{Daß
  Aristoteles nicht eigentlich von "`Angst"' spricht, erklärt Voegelin
  kurzerhand damit, daß es in der griechischen Sprache kein
  entsprechendes Wort gegeben habe.  (Vgl. Voegelin, Anamnesis, S.288.)
  An dieser Stelle wird sehr deutlich, wie anachronistisch Voegelin
  einen Schlüsselbegriff des modernen Existentialismus in die Deutung
  der klassischen Philosophie hineinträgt.}  Diese Angst ist zugleich
eine metaphysisch sehr informative Angst, denn sie enthält "`das Wissen
des Menschen um seine Existenz aus einem Seinsgrund, der nicht der
Mensch selbst ist."'\footnote{Voegelin, Anamnesis, S.289.} Nun möchte
der Mensch diesen Seinsgrund verständlicherweise näher kennenlernen.
Deshalb strebt er nach Wissen. Dieses Streben hat die Form eines
suchenden Begehrens, es hat die Richtung auf den Seinsgrund hin, und es
wird am anderen Ende vom Seinsgrund durch eine eigenständige
Anziehungskraft - über die dieser gemäß Aristoteles verfügt -
unterstützt. Die Richtung dieser Suche bezeichnet Voegelin als
"`Ratio"'. Unter "`rational"' versteht Voeglin daher völlig abweichend
vom üblichen Wortgebrauch in etwa das, was Bergson (nach Voegelins
Interpretation) mit der "`Offenheit der Seele"' meint, also eine
besonders ausgeprägte spirituelle Sensibilität.\footnote{Vgl.
  Vogelin, Anamnesis, S.289.  - Vgl. auch Eric Voegelin: In Search of
  the Ground, in: Conversations with Eric Voegelin. (ed. R. Eric
  O'Connor), Montreal 1980, S.1-20 (S.4-5). - Hier führt Voegelin anhand
  von Aristoteles aus, daß von Rationalität nur die Rede sein kann, wenn
  nicht bloß das Mittel in bezug auf den Zweck sondern auch der Zweck
  selbst rational ist, wozu die Zweck-Mittel-Ketten irgendwann einmal
  zum Nous als dem höchsten Zweck führen müssen. Dieses Argument liefert
  zwar eine Definition von Nous, beweist aber weder dessen Existenz noch
  die Identität des so definierten Nous mit dem transzendenten
  Seinsgrund, der sich (mutmaßlich) in mystischen Erfahrungen zeigt. -
  Auf das Grundproblem, welches die legitime Bedeutung umstrittener
  Ausdrücke wie z.B.  "`Ratio"' ist, kann an dieser Stelle nicht
  eingegangen werden. Zwei Anmerkungen erscheinen mir jedoch angesichts
  der von Voegelin verfolgten semantischen Strategie notwendig: 1.Die
  legitime Worbedeutung ist nicht notwendigerweise die historisch
  ursprünglichste Bedeutung dieses Wortes, da sich auch Wörter und
  Begriffe entwickeln können, und die Entwicklung nicht notwendigerweise
  eine Degeneration sein muß. Daher wäre es falsch zu sagen: Aristotels
  hat als erster von "`Ratio"' gesprochen, also müssen wir uns an das
  halten, was Aristoteles damit gemeint hat. 2.Wenn man ein Wort in
  einer anderen als der üblichen Bedeutung verwenden will, so muß man
  entweder darauf achten, die neue Bedeutung so zu wählen, daß das
  semantische Feld des Wortes erhalten bleibt (z.B. rational ist immer
  etwas, was jedermann durch nachdenken einsichtig werden kann), oder
  man muß das gesamte semantische Feld abändern, was möglicherweise eine
  Lawine von Redefinitionen nach sich zieht. Bei beiden Punkten spielt
  es keine Rolle, wie fehlgeleitet der herrschende Sprachgebrauch ist.
  Im übrigen ist immer Abhilfe durch die Einführung neuer Begriffe
  möglich.} Voegelin führt nun noch weiter aus, wie sich bei
Aristoteles die Beziehung zwischen menschlichem Wissen und göttlichem
Seinsgrund als eine Form von "`Partizipation"', d.i. der Teilhabe des
Menschen am göttlichen Seinsgrund, darstellt. Obwohl Voegelin den
Begriff der Partizipation im folgenden für seine eigenen Überlegungen
übernimmt, werden weder die genaue Bedeutung dieses Begriffes noch die
Bedingungen der Möglichkeit eines derartigen Vorgangs von Voegelin näher
bestimmt.  Der Verzicht auf die Klärung dieses Begriffes ist um so
verwunderlicher, als Voegelin feststellt, daß in Aristoteles
Überlegungen an dieser Stelle noch sehr massiv mythische Denkweisen Eingang
gefunden haben.
 
Diese Feststellung führt Voegelin zu einem neuen Thema, nämlich der
grundsätzlichen Frage nach der Beziehung von Mythos und noetischer Exegese.
Voegelin zufolge beruht der Mythos auf einem eigenen Typ von Welterfahrung,
den er im Gegensatz zur noetischen Erfahrung als "`Primärerfahrung"'
bezeichnet. Für gewöhnlich ersetzt bzw. "`differenziert"' die noetische
Erfahrung die Primärerfahrung.  Aber es gibt eine Ausnahme, bei der dies nicht
möglich ist: Die Erfahrung der Wesensgleichheit aller Menschen. Diese Ausnahme
berührt zugleich eines der Fundamentalprobleme der gesamten philosophischen
Konzeption Voegelins, nämlich das Problem, wie die noetischen Erfahrungen,
obwohl sie kein intersubjektives Wissen zulassen,\footnote{Dies behauptet
  Voegelin auf S.287. Vgl. Vogelin, Anamnesis, S.287.} dennoch für alle
Menschen gültig sein können. Nach Voegelins Ansicht geht die universelle
Gültigkeit noetischer Erfahrung aus der Wesensgleichheit aller Menschen
hervor, welche Gegenstand der mythischen Primärerfahrung ist. Offensichtlich
ist die Ersetzung dieser Primärerfahrung durch eine noetische Erfahrung nicht
möglich, denn dies würde zu einem Begründungszirkel führen. Voegelin übersieht
jedoch mehrere Schwierigkeiten: Erstens würde das Problem der Universalität
der noetischen Erfahrung nur auf das Problem der Universalität des Mythos
verschoben werden, so daß sich auf einer anderen Ebene genau dasselbe
Gültigkeitsproblem wieder stellt. Zweitens folgt aus der grundsätzlichen
Wesengleichheit aller Menschen nicht, daß die Menschen auch hinsichtlich ihrer
religiösen Erfahrungen gleich sind, oder daß die religiöse Erfahrung eines
Menschen verbindlich für einen anderen Menschen sein kann. Drittens läßt sich
die Wesensgleichheit aller Menschen prinzipiell nicht mythisch begründen, denn
Mythen können höchstens etwas veranschaulichen aber niemals
begründen.\footnote{Vgl. zum letzten Punkt auch die Ausführungen zu Voegelins
  Aufsatz "`Zur Struktur des Bewußtseins"' weiter vorne in dieser Arbeit.}
 
Doch damit ist noch nicht alles über den komplizierten Zusammenhang von
noetischer Exegese und Mythos gesagt. Wird versucht, die tieferen
Beziehungen dieser beiden Auslegungsweisen zu einander und zur
Wirklichkeit zu ergründen, so findet man sich Voegelin zufolge zunächst
vor einer Reihe von Aporien wieder, die aufgelöst werden müssen: Die
erste Aporie beruht darauf, daß sowohl die noetische Erfahrung als auch
andere Auslegungsweisen, seien sie nun mythischer oder dichterischer
oder philosophischer Art, Formen der Partizipation darstellen.
Gleichzeitig wird das Wort "`Partizipation"' aber auch als
Selbstbezeichnung allein der noetischen Erfahrung verwendet.  Voegelin
übersieht, daß hier offenbar ein Wort in zweierlei Bedeutung gebraucht
wird. Anstatt durch die Einführung eines neuen Wortes oder durch ein
qualifizierendes Adjektiv Klarheit zu schaffen,\footnote{Dazu ist es
  keineswegs notwendig, wie Voegelin unter (2) (Anamnesis, S.292.) sagt,
  "`das Partizipieren des Philosophen ... von den anderen Fällen zu
  dissoziieren und ihm kognitive Qualität zuzuschreiben"'.  Zum
  "`Dissoziieren"' genügt es hinsichtlich des von Voegelin aufgeworfenen
  logischen Problemes, daß die Fälle überhaupt unterschieden werden
  können, was offenbar gegeben ist, denn wenn zwischen noetischer und
  nicht-noetischer Auslegung unterschieden werden kann, dann kann auch
  zwischen noetischer Auslegung und der Klasse unterschieden werden, die
  die noetische und nicht-noetische Auslegung (und möglicherweise noch
  weitere Übergangsformen) enthält.} zieht Voegelin die falsche
Schlußfolgerung, daß die Partizipation als Spezies unter sich selbst als
Genus fiele. Voegelin krönt seinen logischen Fehler durch die
kategorische Feststellung, daß "`die Logik der Gegenstände und ihrer
Klassifikation"'\footnote{Voegelin, Anamnesis, S.293.}  nicht auf den
Realitätsbereich des Partizipierens anwendbar sei. Übrigens glaubte
Voegelin recht häufig, vor einem Rätsel zu stehen, wenn er auf
mehrdeutige Ausdrücke traf. Als ein Opfer seiner anti-nominalistischen
Vorurteile erkannte er in diesen Vieldeutigkeiten nicht eine sprachliche
Ungenauigkeit, wie sie durch eine saubere begriffliche Unterscheidung
leicht bereinigt werden kann, sondern er vermutete in derartigen
Vieldeutigkeiten oftmals einen tiefern Sinn und damit ein schwieriges
philosophisches Problem.\footnote{An prominenter Stelle liefert dafür
  die Diskussion des Begriffes der Geschichte in "`Order and History I"'
  ein Beispiel. (Vgl. Voegelin, Order and History I, S.126-133.)
  Voegelin hätte sich einen Großteil seiner mühevollen Erörterungen
  sparen können, wenn er von vornherein klar zwischen Geschichte und
  Geschichtsbewußtsein bzw. zwischen Geschichte und Heilsgeschichte
  unterschieden hätte. Denn es ist durchaus nichts Absurdes daran zu
  sagen, daß die alten Ägypter, wie jedes Volk, eine Geschichte hatten
  aber keine Heilsgeschichte wie das Volk Israel, während es in der Tat
  falsch wäre zu behaupten, Israel habe eine Geschichte, Ägypten aber
  nicht.} In Wirklichkeit handelt es sich jedoch nur um den klassischen
Fall eines philosophischen Scheinproblems.

Trotz der völlig mißlungenen Herleitung seines Gedankens läßt sich aus
Voegelins Worten dennoch entnehmen, worauf er hinaus will. Im Folgenden
versteht Voegelin das Wort "`vergegenständlichen"' nicht mehr im Sinne von
"`klassifizieren"', sondern im Sinne von "`zum Gegenstand einer Untersuchung
machen"'. Diese Form von Vergegenständlichung ist eine Voraussetzung
wissenschaftlicher Erkenntnis, aber sie schneidet gleichzeitig die Möglichkeit
eines existentiellen Verstehens im Sinne von Jaspers ab.\footnote{Jaspers,
  Begriff der existentiellen Kommunikation, Philosophie II, S.51, S.58 - Vgl.
  Jeanne Hersch: Karl Jaspers. Eine Einführung in sein Werk, 4. Aufl., München
  1990, S.31-35. - Es gibt zahlreiche Berührungspunkte zwischen dem Denken
  Voegelins und der Philosophie Jaspers', auf die hier jedoch nicht
  ausführlich eingegangen werden kann. Einige Bemerkungen über die nicht
  weniger gravierenden Unterschiede sind jedoch dringend angebracht: Bei
  Voegelin wird die Existenzphilosophie um eine politische Militanz
  verschärft, die geeignet ist, einige ihrer Botschaften geradezu ins
  Gegenteil zu verkehren. So glaubt Voegelin, die Öffnung zur Transzendenz
  ebenso einfordern zu können wie die existentielle Kommunikation, die zudem
  auf Basis von Bedingungen zu erfolgen hat, welche Voegelin vorschreibt
  (Anerkennung der Existenz des und einer liebenden Beziehung zum
  transzendenten Sein). Das Scheitern der existentiellen Kommunikation auf
  Basis der geöffneten Seele bedeutet für Voegelin nicht bloß ein
  existentielles Mißlingen von individueller Tragik, sondern es begründet -
  wenn man Voegelins Polemik ernst nimmt - den Vorwurf eines schuldhaften
  Vergehens, welches in letzter Instanz die politische Untragbarkeit des
  Scheiternden nach sich zieht.} Hieraus ergibt sich für Voegelin, daß die
noetische und die nicht-noetische Auslegung sich nicht gegenseitig
"`vergegenständlichen"' können, ohne daß etwas dabei verloren ginge, weil
beide Formen des Partizipierens und damit derselben existentiellen
Betroffenheit sind, die durch die Berührung mit der Transzendenz entsteht. Aus
dem Blickwinkel der noetischen Auslegung darf also anderen Auslegungsformen
der Rang der Partizipation nicht abgesprochen werden.
 
Aber auch wenn der Mythos daher nicht gänzlich der Unwahrheit verfällt, so
wird doch, wie Voegelin meint, aus der noetischen Exegese heraus ein
Wahrheitsgefälle sichtbar. Die Entwicklung von niederer Wahrheit zu höherer
Wahrheit nennt Voegelin das "`Feld der Geschichte"'. Diese Entwicklung findet
zunächst im Bewußtsein einzelner Menschen statt, die eine vollkommenere
Ausdrucksform für die Partizipation und damit eine höhere Wahrheit finden. Da
diese neue Ausdrucksform jedoch zur Infragestellung nicht bloß der bisherigen
persönlichen Überzeugungen des Denkers, sondern auch der gesellschaftlich
tradierten Ausdrucksformen führt, erlangt sie gesellschaftliche
Bedeutung.\footnote{Vgl. Voegelin, S.294.}
 
Diesen logisch entwickelten Beziehungen zwischen noetischer Exegese und
anderen Auslegungsformen versucht Voegelin nun bei Aristoteles nachzuspüren.
Aristoteles nimmt in seiner Metaphysik auf zwei geistige Traditionen Bezug:
Auf die Mythologie und auf die Philosophie von den Vorsokratikern bis Platon.
Üblicherweise werden diese beiden Traditionen als zwei unterschiedliche, ja
gegensätzliche Diskurstypen innerhalb der hellenischen Geisteskultur
betrachtet. Dieser Sichtweise entspricht ebenfalls weitgehend das Selbstbild
der antiken griechischen Philosophen einschließlich des
Aristoteles.\footnote{Vgl. Luc Brisson: Einführung in die Philosophie des
  Mythos. Antike, Mittelalter und Renaissance. Band I, Darmstadt 1996, S.13-19
  / S.52-53.} Voegelin leugnet auch keineswegs, daß Aristoteles sich mit den
Vorsokratikern auf gleicher Ebene auseinandersetzt.\footnote{Darauf, daß
  Aristotels sich in der "`Metaphysik"' ebenfalls kritisch auf Platon bezieht,
  geht Voegelin nicht näher ein. Dies hätte ihn vermutlich auch zu längeren
  Ausführungen gezwungen, da Voegelin in Platon einen Philosophen desselben
  noetischen Erfahrungsniveaus wie Aristoteles sieht, Aristoteles jedoch
  Platon nicht anders als die Vorsokratiker behandelt, indem für ihn Platon
  nur eine weitere (und besonders falsche) Theorie vom Grund des Seins
  entwickelt hat. (Vgl. Aristoteles: Metaphysik. Schriften zur Ersten
  Philosophie (Hrsg. und übersetzt von Franz F. Schwarz), Stuttgart 1984,
  S.34-36 (987b-988a).)} Allerdings hält Voegelin, der glaubt, in diesem Punkt
Aristoteles besser als dieser sich selbst zu verstehen, ihm dies als eine
Ungenauigkeit vor, denn da Aristoteles sich nach Voegelins Verständnis bereits
auf einem höherem Erfahrungsniveau als die Vorsokratiker befindet, so dürfte
Aristoteles nach Voegelins Ansicht eigentlich nicht mehr auf gleicher Ebene
mit den Vorsokratikern diskutieren.
 
Während für Voegelin also Aristoteles (und Platon) von den
Vorsokratikern durch eine Erfahrungsstufe getrennt sind, scheint ihm der
Unterschied zwischen Philosophie und Mythos andererseits weniger
fundamental. Auf allen Stufen, vom Mythos über die Vorsokratiker bis zu
Aristoteles, geht es, Voegelin zufolge, um eine Erfahrung der
"`Partizipation"' und um deren Artikulation in Symbolen.  Aufgrund
dieses Philosophiebegriffes ist Voegelin gezwungen, den argumentativen
Charakter der hellenischen Philosophie zu leugnen. Daß Aristoteles gegen
die Vorsokratiker argumentiert, stellt für Voegelin lediglich einen
Überredungstrick dar, der dazu dient, die soziale Dominanz seines
eigenen Auslegungsmodells in einem Umfeld zu sichern, in welchem
philosophisches Argumentieren den üblichen Diskursmodus
bildete.\footnote{Vgl. Voegelin, Anamnesis, S.296 (oben).} Was sich von
Stufe zu Stufe ändert, ist die Erfahrung, die von Mal zu Mal
"`differenzierter"' wird. Durch diese Steigerung entsteht die
Geschichte. Und zwar entsteht dabei nicht, wie man denken könnte, irgend
eine bestimmte Geschichte, etwa die Geschichte der religiösen oder
philosophischen Erfahrungen, sondern es entsteht die Geschichte
schlechthin, denn Geschichte wird "`durch das Bewußtsein konstituiert,
so daß der Logos des Bewußtseins darüber entscheidet, was geschichtlich
relevant ist, und was nicht."'\footnote{Voegelin, Anamnesis, S.299.}
Voegelin fügt hinzu, daß die Zeit, in der sich die Geschichte abspielt,
keineswegs "`die der Außenwelt ist, ... sondern die dem Bewußtsein
immanente Dimension des Begehrens und Suchens nach dem
Grund."'\footnote{Voegelin, Anamnesis, S.299.}  Da ferner alle Menschen
nach dem Grund suchen, ist die solcherart durch das Bewußtsein
konstituierte Geschichte "`universell-menschlich"'\footnote{Voegelin,
  Anamnesis, S.299.}.
 
Es fällt schwer, diese Äußerungen über die Geschichte nachzuvollziehen. Denn
entweder man versteht sie als Aussagen über das, was konventionellerweise als
Geschichte bezeichnet wird, also etwa über die politische Geschichte. Dann
sind Voegelins Aussagen schlicht falsch, denn die politische Geschichte spielt
sich natürlich in der äußeren Zeit ab, und der "`Logos des Bewußtseins"' kann
so wenig über das entscheiden, was geschichtlich relevant ist, wie er über das
entscheiden kann, was geschehen ist. Oder man versteht Voegelins Äußerungen
als Definition von "`Geschichte"'. Dann bleibt die so definierte Geschichte
jedoch für alle, die nicht Anhänger der Voegelinschen oder einer ähnlichen
Philosophie sind, völlig irrelevant.  "`Universell-menschlich"' ist diese
Geschichte höchstens ihrem eigenen Anspruch nach, ähnlich, wie auch manche
Religionen sich selbst als "`universell-menschlich"' verstehen, ohne es
jedoch, da es ihrer eine Vielzahl gibt, jemals zu sein.

\subsubsection{Der Begriff der politischen Realität}
 
Voegelin leitet nun mit einer Kritik an Aristoteles über zu seiner eigenen
noetischen Exegese. Die größte Schwäche von Aristoteles' noetischer Exegese
erblickt Voegelin darin, daß Aristoteles an zentraler Stelle immer wieder auf
den sehr mißverständlichen Ausdruck "`Ousia"' zurückgreift. Voegelin zufolge
ist dieser Ausdruck noch der mythischen Primärerfahrung verhaftet und bezieht
sich auf die "`fraglos, selbstverständlich und überzeugend uns
entgegentretende Wirklichkeit der `Dinge' "'.\footnote{Voegelin, Anamnesis,
  S.301.} Dieser mythische Überhang, den Voegelin in diesem Falle offenbar
nicht wie im Falle der mythisch begründeten Wesensgleichheit aller Menschen für
sachlich notwendig hält, rächte sich historisch, indem spätere Philosophen,
bei welchen die noetische Erfahrung so weit in den Vordergrund gerückt war,
daß die mythische Primärerfahrung fast völlig verblaßt war, die aristotelische
"`Ousia"' als Gegenstandsbezeichnung mißverstanden und
begrifflich-philosophische Spekulationen daran knüpften. Das Mißverständnis
des Aristoteles ist Voegelin zufolge die Ursache für den theologischen und
philosophischen Dogmenstreit über Fragen wie die der Unsterblichkeit der
Seele, der Beweisbarkeit der Existenz Gottes oder der Endlichkeit oder
Unendlichkeit der Welt. Das historische Unheil vollendet sich für Voegelin mit
der Aufklärung und dem Positivismus, die nicht nur, was noch zu rechtfertigen
wäre, die dogmatischen Argumente der mittelalterlichen Philosophie angreifen,
sondern die auch die höhere Realität des Partizipierens des Menschen am
Seinsgrund leugnen. Dies zieht nach Voegelins Überzeugung auf individueller
Ebene die psychopathologische Erscheinung des "`realitätslosen Existierens"'
und auf gesellschaftlicher Ebene den Totalitarismus nach sich. Voegelin
illustriert diese Zusammenhänge mit einigen Beispielen aus der schönen
Literatur, worin Wirklichkeitsverlust und Sprachlosigkeit thematisiert
werden.\footnote{Vgl. Voegelin, Anamnesis, S.302-303.}

Wenn die noetische Exegese des Aristoteles also in einigen Punkten noch
unvollkommen oder wenigstens mißverständlich ist, dann stellt sich
natürlich die Frage, wie sie besser durchgeführt werden kann. Voegelin
versucht dies, indem er statt der problematischen "`Ousia"' des
Aristoteles den Begriff der Realität in den Mittelpunkt seiner
Überlegungen stellt. "`Realität"' wird gewöhnlicherweise als der
Inbegriff all dessen verstanden, was tatsächlich vorhanden ist, im
Gegensatz zu dem, was bloß in der Vorstellung oder der Phantasie
existiert.  Voegelin gebraucht dieses Wort in einem anderen Sinne.  Für
ihn ist "`Realität"' ein Inbegriff bestimmter metaphysischer
Seinszusammenhänge, die er in dem Satz zusammenfaßt: "`Eine Realität,
genannt Mensch, bezieht sich, innerhalb eines umgreifend Realen, durch
die Realität des Partizipierens, genannt Bewußtsein, erfahrungs- und
bildhaft auf die Termini des Partizipierens als
Realitäten"'.\footnote{Voegelin, Anamnesis, S.304.} Der wesentliche Teil
dieser Aussage liegt in dem Wort "`Partizipieren"' und darin, daß zu den
"`Termini des Partizipierens"' (denjenigen Dingen, die aneinander
partizipieren) auch der "`göttliche Grund"' gehört, dessen Existenz
Voegelin, wie üblich, ohne weitere Begründung als vermeintliches
Erfahrungsfaktum voraussetzt. Weiterhin spielt für Voegelin eine große
Rolle, daß der Vorgang der Partizipation und die partizipierenden
Bestandteile ("`Termini des Partizipierens"') einen untrennbaren
Gesamtzusammenhang bilden. Wollte man also beispielsweise nur von Gott
bzw.  dem göttlichen Grund reden, ohne auch auf die Beziehung des
Menschen zu Gott einzugehen, so würde man sich nach Voegelins Ansicht
wohl eines gedanklichen Fehlers oder wenigstens einer Ungenauigkeit
schuldig machen.  Voegelin ist um die Wahrung dieses Gesamtzusammenhangs
so ängstlich besorgt, daß er es sogar für unumgänglich hält, das Wort
"`Realität"' vieldeutig zu gebrauchen, derart daß es zugleich sowohl den
Gesamtzusammenhang als auch jeden einzelnen Bestandteil des
Zusammenhanges und darüber hinaus auch noch die "`Symbole"' bezeichnet,
die zur Artikulation des Gesamtzusammenhanges oder seiner Bestandteile
gebraucht werden.\footnote{Vgl.  Voegelin, Anamnesis, S.305, S.307. -
  Daß Voegelin die Vieldeutigkeit des Wortes "`Realität"' für notwendig
  erklärt, verwundert umso mehr, als er sie selber durch den Gebrauch
  unterschiedlicher und sich auf jeweils andere Aspekte beziehende
  Ausdrücke ("`Realität"', "`Partizipation"', "`Termini des
  Partizipierens"') zu umgehen weiß. (Vgl. auch: Voegelin, Order and
  History V, S.16-18. Hier tritt an die Stelle des vieldeutigen
  Realitätsbegriffes der Komplex von Bewußtsein-Realität-Sprache, dessen
  einzelne Elemente ebenfalls terminologisch eindeutig gekennzeichnet
  sind.)  Vermutlich haben wir es hier wieder mit dem sprachlogischen
  Problem der vieldeutigen Ausdrücke zu tun, welches Voegelin so viel
  Kopfzerbrechen bereitete.} Die "`Realität"' des Partizipierens und
seiner "`Termini"' ist immer und in gleichbleibender Weise vorhanden,
unabhängig davon, auf welchem Niveau (noetisch oder prä-noetisch) sie
erlebt und artikuliert wird. Sie bleibt als Realität selbst dann noch
gegenwärtig, wenn sie geleugnet wird.  Etwas irritierend wirkt es auf
den ersten Blick, daß Voegelin trotz dieser ausdrücklichen Erklärung
wenige Zeilen weiter nicht mehr von der Konstanz der Realität ausgeht,
sondern davon spricht, daß die "`Realität"' zugleich konstant und
veränderlich ist.\footnote{Vgl. Voegelin, Anamnesis, S.306.}  Vielleicht
muß man sich das Partizipieren ähnlich der Beziehung der Verwandtschaft
zwischen verwandten Menschen vorstellen, die auch dann noch vorhanden
ist, wenn die Verwandten kein Wort miteinander reden, die aber dadurch
stark intensiviert werden kann, daß die Verwandten wieder anfangen,
miteinander zu kommunizieren, indem sie beispielsweise Geburtstagsgrüße
oder Weihnachtskarten austauschen. Auch die Partizipation kann
intensiviert werden, wenn sich die Menschen ihrer bewußt werden und sie
auf das Niveau "`noetischer Erfahrung"' heben.  Einen derartigen
Zusammenhang scheint Voegelin im Auge zu haben, wenn er von der
gleichzeitigen Konstanz und Veränderlichkeit der Partizipation spricht.
Im ganzen repräsentiert der Begriff der Realität in Voegelins
Gedankengebäude jedoch das Unveränderliche gegenüber den sich wandelnden
Erfahrungen und ihren unterschiedlichen Artikulationen.\footnote{Vgl.
  auch Vgl. Eric Voegelin: Äquivalenz von Erfahrungen und Symbolen in
  der Geschichte, in: Eric Voegelin, Ordnung, Bewußtsein, Geschichte,
  Späte Schriften (Hrsg. von Peter J. Opitz), Stuttgart 1988, S.99-126
  (S.107-108 / S.111-112.).}
 
Ein schwerwiegendes Mißverständnis ist es Voegelin zufolge, wenn auf Grund
einer plötzlichen und sehr intensiven Steigerung der Partizipationserfahrung
irrtümlich geglaubt wird, der Mensch und die Welt selbst hätten sich nun in
ihrem Wesen verwandelt. In diesem Mißverständnis glaubt Voegelin die Ursache
sowohl der aufklärerischen Fortschrittsidee als auch von apokalyptischen
Visionen und Endzeithoffnungen entdecken zu können.\footnote{Vgl. Voegelin,
  Anamnesis, S.307.} Die Behauptung, daß eine plötzlich intensivierte
Partizipationserfahrung die Ursache dieser Phänomene sei, verblüfft ein wenig,
da Voegelin unmittelbar zu vor noch das politische Unheil aus der Leugnung der
metaphysischen Seinsrealität abgeleitet hat.\footnote{Vgl. Anamnesis,
  S.302/303.} Besonders deutlich wird diese Unsitmmigkeit bei Voegelins
Deutung der aufklärerischen Fortschrittsidee: Wenn die Aufklärung die Leugnung
der metaphysischen Realitätserfahrung par exellence verkörpert, wie kann dann
die aufklärerische Fortschrittsidee zugleich Ausdruck des Überschießens dieser
Realitätserfahrung sein?
 
Der Überschwang durchbrechender neuer Realitätserfahrung kann weiterhin dazu
führen, daß Bewußtsein und Realität, die nach Voegelins Auffassung im
Verhältnis eines Teils zum Ganzen stehen, irrtümlich für vollidentisch gehalten
werden. Diese Gefahr deutet sich schon bei Aristoteles an, wenn er, an
Parmenides anknüpfend, Denken und Gedachtes miteinander identifiziert. Bei
Hegel, der wiederum auf Aristoteles zurückgreift, wird dann der göttliche
Grund in das Bewußtsein hineingezogen, womit für Voegelin der schwerwiegende
Tatbestand gnostischer Spekulation erfüllt ist.\footnote{Vgl. Voegelin,
  Anamnesis, S.307-309.}
 
Ausgehend von seiner Vorstellung davon, was Realität in Wahrheit ist, stellt
Voegelin nun einige methodologische Grundsätze hinsichtlich der Interpretation
von unterschiedlichen Deutungen der Realität ("`Realitätsbildern"') auf.
Selbstredend scheint Voegelin auch hier wieder vorauszusetzen, daß Mythologie,
Religion und Philosophie samt und sonders solche "`Realitätsbilder"'
verkörpern. Zunächst müssen daher die "`Realitätsbilder"' als Ausdruck jener
von Voegelin als wahr und gültig erkannten "`Realitätsform des
Partizipierens"'\footnote{Voegelin, Anamnesis, S.309.} verstanden werden. Wenn
alle "`Realitätsbilder"' als Ausdruck jener einen "`Realitätsform"' verstanden
werden, so hat dies Voegelin zufolge den wissenschaftsökonomischen Vorteil,
daß sich daraus unmittelbar eine Erklärung für die oft überraschende
Übereinstimmung räumlich und zeitlich unabhängig voneinander entstandener
"`Realitätsbilder"' ergibt, ohne daß "`okkasionelle Theorien"' zur Deutung
solcher Übereinstimmungen gefunden werden müssen. Dies ist ein für Voegelins
Verhältnisse überraschend einleuchtendes Argument. Voegelin unterschlägt dabei
jedoch, daß jener wissenschaftsökonomische Vorteil dadurch wieder aufgehoben
wird, daß nun "`okkasionelle Theorien"' zur Erklärung von Abweichungen zwischen
"`Realitätsbildern"', die es ja auch gibt, erfunden werden müssen. Ein
weiterer methodologischer Grundsatz, den Voegelin in diesem Zusammenhang
aufstellt, besteht darin, daß "`Realitätsentwürfe, die sich als Systeme
geben"'\footnote{Voegelin, Anamnesis, S.310.} am Maßstab der Realität, mit
welcher selbstredend die von Voegelin als wahr und richtig erkannte Realität
des Partizipierens gemeint ist, untersucht werden müssen. Es genügt nicht, sie
nur auf Grundlage ihrer eigenen Voraussetzungen zu verstehen. Voegelin
vertritt also wenigstens in Bezug auf bestimmte "`Realitätsentwürfe"'
inzwischen genau den gegenteiligen Grundsatz zu der im Husserl-Brief
aufgestellten Forderung, das Selbstverständnis eines Denkers zur Grundlage der
Interpretation seiner Philosophie zu nehmen.
 
Nachdem Voegelin noch einmal kurz das Thema des "`Realitätsverlustes"'
gestreift hat, kommt er auf auf die Möglichkeit der "`periagogé"', der
inneren Umkehr zu sprechen, durch die sich jeder Mensch auch in
realitätsverlassener Zeit von falschen "`Ersatzrealitäten"' reinigen kann. Als
Beispiel zieht Voegelin hier die Entwicklung von Albert Camus heran, in dessen
intellektuellem Werdegang er vorbildhaft die inneren Kämpfe verkörpert sieht,
die nach Voegelins Ansicht ein Mensch in der heutigen Zeit durchleben muß,
"`der im Widerstand gegen die Zeit seine Wirklichkeit als Mensch gewinnen
will."'\footnote{Voegelin, Anamnesis, S.313.}

\subsection{Kritik von Voegelins Realitätsbegriff}
 
Wie überzeugend ist nun Voegelins Vorstellung von Realität, von der
Notwendigkeit ihrer Anerkennung und von den Gefahren ihres Verlustes? Hier
stellt sich erstens die Frage der metaphysischen Wahrheit von Voegelins
Realitätsvorstellung: Gibt es wirklich ein transzendentes Sein, und beugt es
sich tatsächlich gnädig zum liebend hingerissenen Menschen hinab? Zweitens
stellt sich die Frage der Begründbarkeit von Voegelins Realitätsbild: Woher
wissen wir, daß die Realität so beschaffen ist, wie Voegelin es sagt? Kann die
Übereinstimmung mit der inneren Erfahrung auch dann noch eine hinreichendes
Kriterium für die Wahrheit des Voegelinschen Realitätsbildes sein, wenn man
wie Voegelin zugibt, daß es in bezug auf diesen Gegenstand von einander
abweichende innere Erfahrungen gibt? Drittens stellt sich die Frage, ob der
Realitätsverlust, so wie ihn Voegelin versteht, in der Tat mit Notwendigkeit
oder wenigstens Wahrscheinlichkeit politische Unordnung nach sich zieht, und
ob umgekehrt die Anerkennung der Voegelinschen Seinsrealität für die
Errichtung politischer Ordnung in irgend einer Weise vorteilhaft ist. Es
empfiehlt sich, die letzte dieser Fragen zuerst zu untersuchen, denn von der
Beantwortung dieser Frage hängt es ab, ob den anderen Fragen nur eine
theoretische Bedeutung oder auch eine praktisch-politische Dringlichkeit
zukommt.

\subsubsection{Voegelins Verwechselung von gewöhnlichem und spirituellem Realitätsverlust}

Auf die Unklarheiten, die sich durch die unterschiedlichen Formen von
Realitätsverlust, von denen Voegelin spricht, ergeben, wurde bereits
hingewiesen. An dieser Stelle ist daher vor allem die grundsätzliche
Frage zu stellen, ob Realitätsverlust das politische Chaos nach sich
zieht?  Bei oberflächlicher Betrachtung könnte man geneigt sein, diese
Frage ohne jedes Zögern zu bejahen. Wenn die Bürger und insbesondere die
Politiker das Gefühl für die Grenzen ihrer Möglichkeiten verlieren,
vollkommen unrealistische Wünsche hegen oder gar utopisch-weltfremde
Vorstellungen davon haben, was überhaupt möglich ist, dann steht
allerdings zu befürchten, daß eine chaotische Politik dabei herauskommt.
Nach genauerer Untersuchung von Voegelins Äußerungen stellt sich jedoch
heraus, daß es gar nicht dies ist, was er mit Realitätsverlust meint.
Unter Realitätsverlust versteht Voegelin vielmehr die Nicht-Anerkennung
einer bestimmten metaphysischen Seinsrealität und insbesondere des
"`Partizipierens"' des Menschen am transzendenten göttlichen Seinsgrund.
Zur besseren Unterscheidung kann das, was Voegelin unter
Realitätsverlust versteht, als spiritueller Realitätsverlust bezeichnet
werden. Wie vehält sich nun der spirituelle Realitätsverlust zum
gewöhnlichen Realitätsverlust? Zieht ein spiritueller Realitätsverlust
auch einen Realitätsverlust auf pragmatischer Ebene nach sich? Diese
Annahme ist wenig einleuchtend. Warum sollte denn beispielsweise ein
Mensch, der nicht an die Existenz eines transzendenten Seins glaubt,
weniger als andere Menschen dazu in der Lage sein, die Grenzen des
Möglichen zutreffend einzuschätzen?  Umgekehrt spricht ebensowenig
dafür, daß jemand, der über ein hohes Maß an spirituellem Realitätssinn
verfügt, bessere Voraussetzungen für das Verständnis oder die Gestaltung
der politischen Wirklichkeit mitbringt.  Voegelin leugnet auch
keineswegs, daß die richtige Gesinnung und eine erfolgreiche
pragmatische Politik zweierlei sind. Wozu ist dann aber der richtige
spirituelle Realitätssinn überhaupt wichtig? Wenn Voegelin im
Zusammenhang mit dem Thema "`Realitätsverlust"' immer wieder auf die
totalitären Herrschaften anspielt, so liegt der Grund wohl darin, daß
Voegelin sich von einer Verbreitung des Empfindens für die spirituelle
Realität eine besondere immunisierende Wirkung gegen den Totalitarismus
und totalitäre Demagogie erhoffte. Wohlmöglich ging Voegelin davon aus,
daß die in der "`Spannung zum Grund"' lebenden Menschen schon deshalb
nicht auf den Totalitarismus hereinfallen würden, weil die totalitäre
Propaganda, der durch politische Bildung und Aufklärung auf der
Sachebene so schwer beizukommen ist, dann ihrem innersten Lebensgefühl
widersprechen würde. Eine gewisse Plausibilität kann man Voegelins
Überlegung daher nicht absprechen. Nur vernachlässigt Voegelin völlig,
daß auch andere Existenzweisen als nur die Existenz in der "`Spannung
zum Grund"' oder ihre kompakten Vorstufen dies leisten können. Hier wäre
etwa an die Existenzweise eines Atheisten mit humanen moralischen
Grundsätzen zu denken. Daß Voegelins Menschenkenntnis und psychologische
Phantasie in dieser Hinsicht von einer solchen Engstirnigkeit sind,
könnte damit zusammenhängen, daß die "`Realität"' Voegelins auch den
Kern seiner Seinsmetaphysik und seiner theogonischen Geschichtsdeutung
bildet. In diesen Bereichen kann es natürlich nur eine Wahrheit geben.
Aber warum muß das, was für die Metaphysik und die Geschichtsphilosophie
gut ist, auch für die Abwehr des Totalitarismus nützlich und das
Alleinseligmachende sein? (Noch etwas tiefgehender könnte man versuchen,
Voegelins Engstirnigkeit bei der Beurteilung unterschiedlicher
Weltanschauungen als Ausdruck eines Manichäismus Voegelins zu deuten,
d.h. als Folge einer seinen Überlegungen implizit zugrunde liegenden
Ansicht, daß alles Wahre nur einer Quelle entspringen kann, und alles
Falsche ebenfalls einer einzigen Quelle entspringen muß.)
 
Die Schwierigkeiten, die bei Voegelin entstehen, wenn er die Notwendigkeit und
Geeignetheit spirituellen Wahrheitsbesitzes zur Bewältigung der
pragmatisch-politischen Realität begründen will, können auch als ein
theologisches Problem seines mystischen Gottesverständnisses gedeutet werden.
Denn daß das Leben nach den Gesetzen Gottes auch das pragmatisch klügste bzw.
richtigste ist, ergibt sich aus der konventionellen christlichen
Gottesauffassung zwanglos dadurch, daß Gott als allmächtiges Wesen die
Unterwerfung des Menschen honorieren kann, oder daß er als gütiges und
allwissendes Wesen von vornherein vom Menschen nur fordert, was gut für ihn
ist.\footnote{Das Problem wird hier natürlich nur auf einer rein theoretischen
  Ebene besprochen. Daß der Glaube an den gütigen, allwissenden und
  allmächtigen Gott die Menschen in der Realität oft zu den größten Dummheiten
  und den unheiligsten Grausamkeiten angestiftet hat, kann natürlich nicht
  geleugnet werden.} In Voegelins mystisch ausgedünntem Gottesverständnis
bleibt von Gott jedoch nur ein transzendentes Sein übrig (welches zudem bloß
uneigenständiger Pol einer Beziehung ist). Die Attribute der Allmacht und
Allwissenheit scheinen dadurch so gut wie ausgeschlossen zu sein. Lediglich
die Güte ist - der von Voegelin beschriebenen Erfahrung des Hingezogenseins
nach zu urteilen - noch vorhanden (wenn sie sich auch als Sirenengesang
erweisen kann, wie es die transzendente Variante der Gnosis vor Augen führt,
die sich bei Voegelin nicht auf einen falschen Gott sondern auf das richtige
transzendente Sein in der falschen Weise bezieht). Es fehlt bei diesem
ohnmächtigen transzendenten Sein aber jede Gewähr, daß die spirituell
richtige, nach der "`Spannung zum Seinsgrund"' ausgerichtete Existenzweise
auch in pragmatischer Hinsicht die richtige ist. Sie könnte auch genau das
Gegenteil davon sein.

\subsubsection{Die Zirkularität der Begründung von Voegelins Realitätsbegriff}

Als nicht weniger problematisch als der Zusammenhang von spirituellem
Realitätsverlust und politischem Chaos erweist sich die
Begründungsproblematik von Voegelins Realitätsbegriff. Woher kann man
wissen, daß das, was Voegelin über die metaphysische Seinsrealität sagt,
wahr ist? Aus Voegelins Gedankengang heraus müßte darauf die Antwort
gegeben werden, daß sich diese Wahrheit aus der Erfahrung ergibt, wobei
unter Erfahrung nicht die Sinneserfahrung sondern entweder jenes innere
Erleben der "`noetischen"' Erfahrung oder die mythische
"`Primärerfahrung"' zu verstehen ist. Hier stellt sich jedoch ein
unlösbares Problem: Indem Voegelin zugibt, daß es unterschiedliche
Erfahrungen gibt, denen unterschiedliche Realitätsbilder entsprechen,
wie kann dann die Erfahrung noch ein Kriterium für die Wahrheit einer
bestimmten Auffassung der "`Realität"' abgeben? Auf diese Frage gibt
Voegelins Bewußtseinsphilosophie keine Antwort. Auch der Begriff der
Differenziertheit kann zur Beantwortung dieser Frage nicht herangezogen
werden, denn dazu müßte er, soll ein Zirkelschluß vermieden werden,
unabhängig von den Begriffen der Realität und der Erfahrung definiert
werden. Damit bleibt aber nur noch ein rein formaler
Differenzierungsbegriff übrig, der, wie im ersten Teil dieser Arbeit
bereits ausgeführt, kaum zu wertenden Vergleichen herangezogen werden
kann.\footnote{Zumindest liefert Voegelin keinerlei Anhaltspunkte dafür,
  wie dies geschehen könnte. (Rein theoretisch ist es natürlich denkbar,
  daß der Begründungsregreß an dieser Stelle oder an irgend einer
  späteren mit einem sinnvollen Kriterium abbricht, nur muß dieses
  Kriterium dann auch angegeben werden und als sinnvoll oder evident
  ausgewiesen sein.)}

An anderer Stelle, in seinem Aufsatz "`Äquivalenz von Erfahrungen und Symbolen
in der Geschichte"', behauptet Voegelin, daß sich seine Aussagen über das
Wesen der Realität geschichtlich überprüfen lassen.\footnote{Eric Voegelin:
  Äquivalenz von Erfahrungen und Symbolen in der Geschichte, in: Eric
  Voegelin, Ordnung, Bewußtsein, Geschichte, Späte Schriften (Hrsg. von Peter
  J. Optiz), Stuttgart 1988, S.99-126 (S.109).} Die Aussagen dürfen nach
Voegelins Ansicht dann als gültig angesehen werden, wenn sie sich auf die
Geschichte beziehen, ohne "`einen erheblichen Teil des geschichtlichen Feldes
ignorieren oder im Dunkeln lassen"'\footnote{Ebd.} zu müssen, und wenn sie
"`erkennbar äquivalent mit den Symbolen [sind], die unsere Vorgänger in der
Suche nach der Wahrheit der menschlichen Existenz geschaffen
haben"'.\footnote{Ebd.} Dieses Prüfungskriterium ist offensichtlich zirkulär,
weil bereits zuvor bekannt sein müßte, welche Symbole der "`Vorgänger"'
echte Erfahrungssymbole sind, welche allein in die Prüfung einbezogen
werden dürfen.\footnote{Zur Zirkularität von Voegelins Begründung der Wahrheit
  bestimmter Symbolismen besonders deutlich: Vgl. Eugene Webb: Philosophers of
  Consciousness. Polanyi, Lonergan, Voegelin, Ricoeur, Girard, Kierkegaard,
  Seattle and London 1988, S.126ff.} Dieser Zirkelschluß läßt sich auch nicht
zu einem hermeneutischen Verstehenszirkel erweitern, denn abgesehen davon,
daß der hermeneutische Zirkel höchstens die innere Folgerichtigkeit
der schrittweise verfeinerten Deutung gewährleistet, treten in Voegelins
Geschichtsbild zwei Symboltraditionen auf (die Tradition der echten Symbole
und die Tradition der Entgleisungen), die höchstwahrscheinlich beide die
Grundlage eines hermeneutischen Zirkels mit jeweils symmetrischen Stärken und
Schwächen bilden können. Darüber hinaus sind die Kriterien, die Voegelin
anführt, nur dann ihrem Zweck angemessen, wenn bereits zuvor als
metaphysisches Postulat vorausgesetzt wird, daß die Geschichte der Ausdruck
des Prozesses der Realität des Partizipierens ist, und daß die Symbole
Ausdruck der menschlichen Erfahrung des Partizipierens sind. Am Schluß des
Aufsatzes über die "`Äquivalenz von Erfahrungen und Symbolen in der
Geschichte"' gibt Voegelins dies auch ganz ungeniert zu.\footnote{Ebd.,
  S.126.} Damit kann aber von einer historischen Prüfbarkeit seiner Aussagen
über die Realität keine Rede mehr sein.

Im Ergebnis stellt sich also heraus, daß es bereits {\it innerhalb} der
Voegelinschen Theorie weder möglich ist, die Realitätsadäquatheit von
Erfahrungen festzustellen, noch die Richtigkeit von Realitätsauffassungen,
einschließlich der Realitätsauffassung, die Voegelin selbst vertritt, zu
beurteilen.

% denn wollte man diese Frage damit zu
% beantworten, daß stets die differenzierteste Erfahrung als die
% realitätsadäquateste zu gelten habe, dann würde das Problem auftreten, wie die
% Differenziertheit einer Erfahrung bestimmt werden kann. Offensichtlich kann die
% Differenziertheit nicht durch die Realitätsadäquatheit definiert werden, denn
% dann könnte sie nicht als Kriterium derselben dienen. Wird der Begriff der
% Differenziertheit aber in irgend einer anderen Weise definiert (z.B. durch das
% Auseinandertreten von Immanenz und Transzendenz), so bleibt, sofern das Wissen
% um die Realität nicht schon vorausgesetzt wird, völlig schleierhaft, warum die
% differenziertere Erfahrung auch die realitätsadäquatere Erfahrung ist. Man
% könnte nun vielleicht auf den Ausweg verfallen, den Begriff der
% Differenziertheit aus der Erfahrung selbst entnehmen zu wollen. Doch ein
% erfahrungsübergeifendes Bewertungskriterium für Erfahrungen würde sich nur
% dann ergeben, wenn allen Erfahrungstypen ein- und derselbe Begriff von
% Differenziertheit entnommen werden könnte, und wenn zugleich innerhalb aller
% Erfahrungstypen dieser selbe Begriff von Differenziertheit als relevantes
% Merkmal für die Realitätsadäquatheit erfahren würde. Dies ist jedoch sehr
% unwahrscheinlich und Voegelin leugnet auch keineswegs, daß aus Sicht der
% kompakteren Erfahrung die differenziertere falsch und illegitim erscheint. Es
% ist also berteits {\it innerhalb} der Voegelinschen Theorie weder möglich, die
% Realitätsadäquatheit von Erfahrungen festzustellen, noch ist es möglich die
% Richtigkeit von Realitätsauffassungen einschließlich der Realitätsauffassung,
% die Voegelin selbst vertritt, zu beurteilen.

\subsubsection{Die Fragwürdigkeit von Voegelins Seinserfahrung} 

Es bleibt schließlich zu überlegen, ob Voegelins Vorstellung von Realität
überhaupt der Wahrheit entspricht. Die richtige Art, diese Frage anzugehen,
bestünde zweifellos darin, zunächst zu untersuchen, ob ein transzendentes Sein
überhaupt existiert, und dann zu klären, ob es sich in der von Voegelin
behaupteten Beziehung zum Menschen befindet. Dieses Vorgehen würde jedoch
genau auf das hinauslaufen, was Voegelin als dogmatisches Mißverständnis von
Symbolen, die Erfahrungen beschreiben, kritisiert. Da nun aber, unabhängig von
der Berechtigung eines solchen Vorwurfes, die Frage von Interesse ist, ob
Voegelin wenigstens nach seinen eigenen Maßstäben Recht behält, so empfiehlt
sich der Versuch, Voegelins Ansatz einmal naiv nachzuvollziehen, und über die
Frage zu meditieren, ob die Realität tatsächlich so erfahren wird, wie
Voegelin sie beschreibt. Auf diese Weise läßt sich außerdem klären, ob die
recht kritische Sicht von Voegelins Philosophie nur der in dieser Arbeit
verwendeten rationalistischen Methode zuzuschreiben ist, oder ob auch eine dem
Ideal der immanenten Kritik verpflichtete Herangehensweise zu kritischen
Resultaten kommten könnte. Im folgenden erlaube ich mir daher das Protokoll
einer philosophischen Meditation über eine der Schlüsselpassagen aus Voegelins
Werk "`Anamnesis"' wiederzugeben. 

Voegelin beschreibt die Erfahrung der Realität an einer Stelle seines
Vortrages "`Ewiges Sein in der Zeit"' mit den folgenden Worten:
 \begin{quote}
   Wie immer es um den Menschen als das Subjekt der Erfahrung bestellt sein
   möge, so erfährt er seelisch eine Spannung zwischen zwei Seinspolen, deren
   einer, genannt der zeitliche, in ihm selbst liegt, während der andere
   außerhalb seiner selbst liegt, jedoch nicht als Gegenstand im zeitlichen
   Sein identifiziert werden kann, sondern als ein Sein jenseits alles
   zeitlichen Seins der Welt erfahren wird. Vom zeitlichen Pol her wird die
   Spannung als ein liebendes und hoffendes Drängen zur Ewigkeit des
   Göttlichen erfahren; vom Pol des ewigen Seins her als ein gnadenhaftes
   Anrufen und Eindringen. Im Verlauf der Erfahrung wird weder das ewige Sein
   als ein Objekt in der Zeit gegenständlich, noch wird die erfahrende Seele
   aus ihrem zeitlichen in ewiges Sein transfiguriert; vielmehr ist der
   Verlauf zu charakterisieren als ein Sich-Ordnen und Sich-Ordnen-Lassen der
   Seele durch ihr liebendes Sich-Öffnen für das Eindringen des ewigen
   Seins.\footnote{Vgl. Voegelin, Anamnesis, S.265. - Vgl. Peter J. Opitz:
     Rücker zur Realität: Grundzüge der politischen Philosophie Eric
     Voegelins, in: Peter J.  Opitz / Gregor Sebba (Hrsg.): The Philosophy of
     Order. Essays on History, Consciousness and Politics, Stuttgart 1981,
     S.57/58.}
 \end{quote}
 Wird die Realität tatsächlich in dieser Weise erfahren? Auf diese Frage
 ist natürlich nur eine subjektive Antwort möglich, aber für meinen Teil
 kann ich diese Frage doch ziemlich klar verneinen: Die Realität wird
 nicht als ein Partizipieren erfahren, in dessen Verlauf ein sich gnädig
 herabbeugendes transzendentes Sein in die liebend sich
 entgegendrängende Seele des Mensch eindringt. Die Welt fühlt sich
 einfach nicht so an, wie Voegelin es beschreibt. Vor allem fühlt sich
 die Welt nicht dermaßen melodramatisch an.  Dementsprechend schwer
 fällt es, diesen Worten den Ausdruck der tiefsten Wahrheiten der Seele
 und des höchsten Sinns der Welt zu entnehmen. Schon die
 Zusammenstellung von Lieben und "`Sich-Ordnen-Lassen"' mutet grotesk
 an, und in der Rede vom "`Eindringen des ewigen Seins"' in die sich
 öffnende Seele kann ich ebensowenig eine Wahrheit finden, die sich
 leidenschaftlich bejahen ließe. Wollte Voegelin mit dieser Passage etwa
 jene geheimnisvolle "`Spannung zum Grund"' beschreiben, die seiner
 Ansicht nach für das menschliche Leben von so entscheidender Bedeutung
 ist? Aber was dabei herauskommt, ist bloß schlechter Geschmack, hart an
 der Grenze zum religiösen Erbauungskitsch. Und wollte Voegelin allen
 Ernstes den Menschen, die derartige Empfindungen nicht teilen, eine
 geschlossene Seele und eine existentielle Deformation ihrer selbst
 vorwerfen?  Das kann, das darf nicht wahr sein! Aber die Frage darf
 wohl aufgeworfen werden, ob nicht eine gehörige Portion menschlicher
 Eitelkeit dazu gehört, sich einzubilden, die eigene sterbliche Seele
 sei Schauplatz solch kosmisch bedeutsamer Vorgänge wie des Eindringens
 der Transzendenz in die Immanenz.  Und ebenso stellt sich die Frage, ob
 nicht auch ein wenig intellektuelle Einfalt zu dem Glauben gehört,
 dergleichen könne so ohne weiteres möglich und wirklich und obendrein
 uns Menschen in der mystischen Schau gegenwärtig sein.\footnote{Dem
   psychologischen Scharfblick Tolstojs ist die Einsicht zu verdanken,
   daß das mystische Denken nicht, wie man voreilig vermuten möchte,
   eine besondere Tiefe und Empfänglichkeit des Geistes und der
   Vorstellungskraft voraussetzt, sondern im Gegenteil auch auf einer
   ausgeprägten Oberflächlichkeit derselben beruhen kann. So
   charakterisiert Tolstoj in "`Anna Karenina"' die Hinwendung des
   betrogenen Alexej Karenin zu einer gerade in Mode gekommenen
   mystischen Richtung des Christentums mit folgenden Worten: "`Es
   fehlte ihm, gleich Lydia Iwanowna und den anderen Leuten, die
   derselben neuen Auffassung huldigten, jegliche Tiefe der
   Vorstellungskraft, jener geistigen Fähigkeit, dank welcher die durch
   die Phantasie hervorgerufenen Bilder mit dem Vorstellungskomplex und
   zugleich mit der Wirklichkeit im Einklang bleiben. Er sah nichts
   unmögliches und Absurdes in dem Gedanken, daß der Tod, der nur für
   die Ungläubigen existierte, für ihn nicht vorhanden sei und daß, da
   er den vollkommenen Glauben besaß, dessen Maß er im übrigen selbst
   bestimmte, auch für die Sünde in seiner Seele kein Raum sei und er
   daher schon hier auf Erden des Heils teilhaftig werde."'  (Leo N.
   Tolstoi: Anna Karenina, München 1992, S.511.)}

% Auch für den
%    noetischen Denker existiert der Tod nicht mehr: "`Auf Grund der göttlichen
%    Präsenz, die der Unruhe ihre Richtung gibt, wird die Entfaltung des
%    noetischen Bewußtseins als ein Prozeß des Unsterblich-Werdens erfahren."'
%    (Eric Voegelin: Vernunft: Die Erfahrung der klassischen Philosophen, in:
%    Eric Voegelin: Ordnung, Bewußtsein, Geschichte. Späte Schriften.  (Hrsg.
%    von Peter J.  Optiz), Stuttgart 1988, S{\bf ???}.)

% Und schließlich: Wer braucht,
%  wer wünscht sich das Eindringen des ewigen Seins überhaupt? Würde nicht ein
%  sterblicher Mensch viel eher abwinken und sagen: "`Danke, zu viel der Ehre!
%  Laß nur die Götter Götter sein. Mein Leben war vorher elend oder schön und wird es
%  nacher sein, ob ich sie geschaut habe oder nicht!"' Und täte er als
%  Sterblicher nicht recht daran. Handelte er nicht viel klüger, wenn er vor die
%  Wahl gestellt zwischen der Schau des Ewigen und dem Genuß zeitlicher Güter
%  von ganzem Herzen ausriefe: "`Gern glaube ich, was ich mit Händen
% greife!"'\footnote{Hinweis auf den Rechenfehler in Pascals va-banque-Spiel}  
 
 Selbstverständlich müssen deratig hochmetaphysische Überlegungen
 letzten Endes als eine Angelegenheit des persönlichen weltanschaulichen
 Geschmackes angesehen werden, da sie sich weder durch eine
 (intersubjektive) Erfahrung überprüfen noch argumentativ entscheiden
 lassen. Voegelin ist freilich nicht bereit, die subjektive Relativität
 seiner Vorstellung von der höchsten Realität zuzugeben. Vielmehr setzt
 er die Wahrheit seiner Realitätsauf\/fassung absolut und zieht sie ohne
 Umstände als Verständnisgrundlage und Bewertungsmaßstab aller anderen
 Weltanschauungen heran. Deutlich wird dies immer wieder an Urteilen wie
 diesem: "`Unter den Erfahrungen des Partizipierens schließlich hat die
 noetische dadurch ihren besonderen Rang, daß sie die Spannung zum
 göttlichen Grund nicht nur als Sachstruktur des Bewußtseins, sondern
 als die Grundspannung aller Realität, die nicht selbst der göttliche
 Grund ist, zur Klarheit bringt."'\footnote{Voegelin, Anamnesis, S.304.}
 Sinnvoll ist ein solches Urteil nur, wenn als gegeben vorausgesetzt
 wird, daß "`die Spannung zum göttlichen Grund"' in der Tat "`die
 Grundspannung aller Realität"' ist, was aber, gerade weil es
 unterschiedlich erfahren wird, niemand mit Sicherheit behaupten kann.
 Unter der Hand gerät Voegelin daher auch seine eigene Philosophie zu
 einem jener geschlossenen Dogmensysteme, die sich mit Hilfe
 intellektueller Tricks gegen jede Kritik abschirmen. Zwar beschreibt
 Voegelin die Realität als offen, aber seine Beschreibung der Realität
 ist ihrerseits ganz und gar nicht offen. Zu den intellektuellen Tricks,
 mit denen Voegelin seine Philosophie zu einem geschlossenen System
 abriegelt, gehört unter anderem die im folgenden zu beschreibende
 Theorie der sprachlichen Indizes, mit der er seinen eigenwilligen
 Sprachgebrauch rechtfertigt.

\subsection{Voegelins Theorie der sprachlichen Indizes}

Die Theorie der sprachlichen Indizes beschreibt die sprachlichen
Eigentümlichkeiten der verbalen Wiedergabe noetischer Erfahrungen. Es geht
dabei um das Problem, die Besonderheit noetischer Beschreibungen zu erfassen,
denn rein äußerlich unterscheidet sich die sprachliche Wiedergabe echter
noetischer Erfahrungen durch nichts von der Sprache dogmatischer Metaphysik.
Außerdem versucht Voegelin, mit seiner Theorie der sprachlichen Indizes seinen
eigenen philosophischen Sprachgebrauch zu rechtfertigen und insbesondere das
Definitionsrecht bestimmter Begriffe (Welt, Mensch, Geschichte etc.) für sich
zu reklamieren.

Voegelin beginnt zunächst mit einer knappen Zusammenfassung der wichtigsten
Züge seines Realitätsbegriffes. Daran anknüpfend stellt er seine Theorie der
sprachlichen Indizes als eines Ausdruckes der (noetischen) Erfahrungen dieser
Realität vor. Schließlich zieht Voegelin aus dieser Theorie eine Reihe von
Schlußfolgerungen in Bezug auf die Politikwissenschaft, die menschliche Natur
und die Deutung der Geschichte.

Realität ist eine komplexe Beziehung, von Mensch, Dingen und Seinsgrund. Diese
Beziehung wird vom Menschen nicht beobachtet, sondern "` `von innen'
"'\footnote{Vgl. Voegelin, Anamnesis, S.316.} erfahren. Ungeachtet dessen
bleiben der Mensch und sein Leben jedoch in äußere Zusammenhäge eingordnet.
Der Seinsgrund ragt durch das Bewußtsein des Menschen in die Welt hinein, aber
der Mensch kann sich nicht durch das bewußte Partizipieren am Seinsgrund über
die Welt hinausheben. (Voegelin baut hier dem gnostischen Mißverständnis der
Möglichkeit einer Erlösung durch Wissen vor.)  In der Erfahrung des
Partizipierens gewinnen wir, Voegelin zufolge, gültige "`Einsichten"' nicht
nur in das Partizipieren selbst, sondern auch in die Termini des
Partizipierens, also beispielsweise in das Wesen des Menschen und den
Seinsgrund. Noetisches Wissen ist der unmittelbar den "`Bewegungen"' des
Partizipierens entspringende Ausdruck dieser Einsichten.\footnote{Vgl.
  Voegelin, Anamnesis, S.315-316.}

An diesem Punkt führt Voegelin seine Theorie der sprachlichen Indizes ein.
Voegelin greift für diese Theorie eine Denkfigur auf, die er bereits in seinem
Aufsatz über die Struktur des Bewußtseins entwickelt hat, in welchem er
die These vertritt, daß das Bewußtsein nicht zeitlich sondern durch
Erhellungsdimensionen strukturiert sei, die dann als "`Zukunft"' und
"`Vergangenheit"' sprachlich gekennzeichnet oder, wie Voegelin nun sagen
würde, {\it indiziert} werden.\footnote{Vgl. Voegelin, Anamnesis, S.44.} Die
Theorie der Indizes besagt, daß die sprachlichen Ausdrücke, mit denen die
noetischen Erfahrungen artikuliert werden, nicht gegenständlich als Aussagen
über etwas sondern als Kennzeichnung von inneren Erfahrungen bzw. Erlebnissen
verstanden werden müssen. Dies gilt, obwohl diese sprachlichen Ausdrücke ihrer
äußeren Form nach gegenstandsförmlich sind. So wäre also etwa der Satz: "`In
der noetischen Erfahrung dringt der transzendente Seinsgrund in das Bewußtsein
ein"' nicht als Aussage über das transzendente Sein und das menschliche
Bewußtsein zu verstehen, sondern als Kennzeichnung einer inneren Erfahrung des
Eindringens, die offenbar von solcher Intensität und Eigenart ist, daß zu
ihrem angemessenen Ausdruck vom "`Eindringen des transzendenten Seins"'
gesprochen werden muß. Warum aber müssen die noetischen Erfahrungen überhaupt
gegenständlich ausgedrückt werden, wenn dies doch so mißverständlich ist?
Voegelin glaubt, daß es zum gegenständlichen Ausdruck keine Alternative gibt,
"`weil das Bewußtsein gegenstandsförmlich ist"'.\footnote{Voegelin, Anamnesis,
  S.316. - Unter Gegenstandsförmlichkeit versteht Voegelin, daß "`Bewußtsein
  [..] immer Bewußtsein-von-Etwas ist"' (Anamnesis, S.307.). - (Sicherlich
  würde Voegelin nicht ausschließen, daß man auch in die mythische
  Ausdrucksweise zurückverfallen könnte, aber dabei würde ein weniger
  differenzierteres Ausdrucksniveau in Kauf genommen werden müssen.)}

Schwere Fehler und Mißverständnisse ergeben sich nach Voegelins Ansicht, wenn
Ausdrücke, die Indizes von Bewußtseinserfahrungen sind, unabhängig von diesen
Erfahrungen als Begriffe für etwas eigenständig Seiendes verwendet werden.
Voegelin illustriert dies an einer Reihe von Beispielen. So gibt es für
Voegelin "`weder eine immanente Welt noch ein transzendentes Sein als
Entitäten"',\footnote{Voegelin, Anamnesis, S.316.} vielmehr sind die Ausdrücke
"`immanent und "`transzendent"' Indizes, welche Bereichen der Erfahrung
zugeteilt werden. Nach Voegelins Überzeugung ist es daher unsinnig, über die
Existenz von transzendentem oder immanentem Sein zu streiten. Weiterhin ist
Voegelin der Ansicht, daß der Ausdruck Mensch wenigstens in bestimmter
Hinsicht einen Index der Erfahrung darstellt, denn unter "`Mensch"' ist auch
"`der immanente Pol der existenziellen Spannung zum Grund zu
verstehen"'.\footnote{Voegelin, Anamnesis, S.317.} Da außerdem nach Voegelins
Ansicht auch der Ausdruck "`Philosophie"' ein Index der Erfahrung ist, so
glaubt Voegelin folgern zu können, daß es unmöglich ist, den Menschen im
Rahmen einer philosophischen Anthropologie ausschließlich als welt-immanentes
Wesen zu verstehen. In der Vernachlässigung dieses Grundsatzes in der
Anthropologie erblickt Voegelin nicht bloß einen philosophischen
Irrtum, wie er beim Nachdenken schon einmal unterlaufen könnte, sondern eine
Form von Realitätsverlust.\footnote{Vgl. Voegelin, Anamnesis, S.316-317.}

Aus der Theorie der sprachlichen Indizes folgt für Voegelin eine Reihe
von Konsequenzen, die überwiegend bereits gewonnene Einsichten bekräftigen und
vertiefen. Die erste Konsequenz ergibt sich hinsichtlich des
Begriffes der Wissenschaft. "`Wissenschaft"' ist für Voegelin ebenfalls ein
Index. Sie entdeckt "`sich selbst als das Strukturwissen von Realität, wenn
die Selbsterhellung des Bewußtseins und seiner Ratio sich historisch
ereignet"',\footnote{Voegelin, Anamnesis, S.318. - Daß Wissenschaft ebenfalls
  ein Index sein soll, verblüfft auf den ersten Blick, denn Wissenschaft ist
  primär eine menschliche Tätigkeit und nicht etwas, das erfahren wird, so daß
  man zunächst geneigt ist, an dieser Stelle an einen Kategorienfehler
  Voegelins zu glauben. Aber Voegelin scheint offenbar ernsthaft die Ansicht
  vertreten zu wollen, daß Wissenschaft in erster Linie aus der
  Selbsterfahrung des wissenschaftlichen Denkens entsteht.} wobei in
Erinnerung zu rufen ist, daß Voegelin unter "`Ratio"' die zum Seinsgrund hin
geöffnete Seele versteht und nicht etwa Vernunft oder Verstand im gewöhnlichen
Sinne. Dieses historische Ereignis hat, Voegelin zufolge, bei Platon und
Aristoteles stattgefunden, deren Noese "`die Indizes Wissenschaft ({\it
  episteme}) und Theorie ({\it theoria}) entwickelt hat."'\footnote{Voegelin,
  Anamnesis, S.318.} Selbst die moderne Naturwissenschaft verdankt nach
Voegelins Ansicht ihren Wissenschaftscharakter weniger dem Erfolg ihrer
Methoden als vielmehr der Tatsache, daß ihre Methoden mit der "`Ratio der
Noese verträglich sind."'\footnote{Voegelin, Anamnesis, S.318.} Erst die Noese
legt nämlich die "`Welt"', welche wiederum ein sprachlicher Index des
Bewußtseins ist, als ein von mythischen und anderen Glaubenselementen
gereinigtes Feld für die Bearbeitung durch die Naturwissenschaft
frei.\footnote{Vgl. Voegelin, Anamnesis, S.318.}

Um über Partizipationserfahrungen angemessen reden zu können, genügen
allerdings nicht allein die sprachlichen Indizes, welche diese Erfahrungen
selbst ausdrücken. Es ist darüber hinaus eine Art von Begriffen notwendig, mit
denen {\it über} diese Erfahrungen gesprochen werden kann. Diese Begriffe
bezeichnet Voegelin als Typenbegriffe. Als historische Beispiele für
Typenbegriffe führt Voegelin die Ausdrücke "`philodoxos"' und "`sophistes"'
von Platon und die Ausdrücke "`philosophos"' und "`philomythos"' von
Aristoteles an. Unter seinen eigenen Begriffen rechnet Voegelin unter anderem
die Begriffe der "`kompakten und differenzierten Erfahrungen"' und der
"`noetischen und revelatorischen Transzendenzerfahrungen"' zu den
Typenbegriffen.\footnote{Vgl. Voegelin, Anamnesis, S.319.} Die
Erforderlichkeit von Typenbegriffen wird besonders dann akut, wenn infolge
geistesgeschichtlicher Differenzierungsprozesse die kompakteren
Partizipationserfahrungen in eine Rolle realitiver Unwahrheit gedrängt werden,
so daß ihr symbolischer Selbstausdruck nicht mehr zählt, und Begriffe gefunden
werden müssen, um die kompakten Erfahrungen angemessen bezeichnen zu können.

Im Zusammenhang mit der geschichtlichen Entwicklung von
Partizipationserfahrungen kommt Voegelin auf das Problem der Beziehung
des überindividuellen Prozesses der Geschichte zum individuellen
Bewußtsein zu sprechen, welches nach Voegelins Auffassung durch seine
Transzendenzerfahrungen der Träger dieses Prozesses ist. Für Voegelin
gibt es Bewußtsein ausschließlich in der Form des konkreten Bewußtseins
einzelner Individuen. Es ist "`diskret real"'.\footnote{Voegelin,
  Anamnesis, S.320.}  Wie können aber die individuellen
Transzendenzerfahrungen der vielen diskret realen Bewußtseine innerhalb
eines sinnhaften historischen Prozesses oder Feldes der Geschichte
verortet werden, von dessen Existenz Voegelin nach wie vor überzeugt
ist?\footnote{Auch nachdem Voegelin die Auffassung einer linearen
  Geschichtsentwicklung aufgegeben hat (Vgl. Eric Voegelin:
  Historiogenesis, in: Voegelin, Anamnesis, S.79-116.), hält er dennoch
  daran fest, daß das "`Feld der Geschichte"' prozeßhaft geordnet ist.
  (Vgl. Eric Voegelin: Ewiges Sein in der Zeit, in: Voegelin, Anamnesis,
  S.254-280.)}  Voegelin beantwortet diese Frage damit, daß in den
Transzendenzerfahrungen der vielen Bewußtseine stets ein und derselbe
transzendente Seinsgrund erfahren wird: "`Geschichte wird zu einem
strukturell verstehbaren Feld der Realität durch die Präsenz des einen
Grundes, an dem alle Menschen partizipieren..."'.\footnote{Voegelin,
  Anamnesis, S.320.} Keinesfalls kann dagegen die Geschichte (wie etwa
bei Hegel) als die Entfaltung eines kollektiv-überindividuellen oder gar
absoluten Bewußtseins verstanden werden, da hierbei vollkommen ignoriert
wird, daß Bewußtsein nur als das Bewußtsein einzelner Menschen
vorkommt.\footnote{Vgl.  Voegelin, Anamnesis, S.320-321.}

Schließlich weist Voegelin noch auf die problematischen Folgen hin, die
aus dem unsachgemäßen Gebrauch von Typenbegriffen entstehen.
Typenbegriffe dürfen, so scheint es Voegelin aufzufassen, legitimerweise
nur dann eingesetzt werden, wenn ihr Gebrauch durch eine eigene
noetische Erfahrung gedeckt ist, durch welche allein die weniger
differenzierten Erfahrungen richtigerweise als Typen von relativ
geringerem Wahrheitsgrad erkannt werden können. Dazu muß außerdem hinter
jedem Typus die je eigene Erfahrungsgrundlage dieses Typus erkannt
werden. (Voegelin greift hier auf das bereits in der "`Neuen
Wissenschaft der Politik"' entwickelte Prinzip zurück, daß die
Erfahrungen und nicht die Ideen "`die Substanz der Geschichte"' bilden.)
Die Vernachlässigung dieser Prinzipien führt nach Voegelins Ansicht zu
unersprießlichen Dogmenstreitereien zwischen sich gegenseitig
typisierend einordnenden Meinungen, die bis zum allgemeinen
Ideologieverdacht ausarten können, ohne daß jemals die entscheidende
Ebene der Transzendenzerfahrungen auch nur in den Blick
gerät.\footnote{Vgl. Voegelin, S.321-323.} Voegelin gesteht sich nicht
ein, daß seine Theorie auch nur eine weitere Position in der
wissenschaftlichen Auseinandersetzung der Theorien darstellt (was auch
dann der Fall wäre, wenn sie tatsächlich und als einzige von allen
Theorien wahr wäre), und daß er durch seine polemischen Ausfälle selbst
nicht wenig zum allgemeinen Ideologieverdacht beiträgt.

\subsection{Kritik von Voegelins Sprachtheorie}

Die Theorie der sprachlichen Indizes erweist sich in vielerlei Hinsicht
als höchst unglaubwürdig und zweifelhaft. Dies beginnt schon mit den
Voraussetzungen der Theorie: Voegelins Theorie der sprachlichen Indizies
stellt eine Theorie über die Bedeutung bestimmter sprachlicher Ausdrücke
dar.  Sie besagt, daß bestimmte sprachliche Äußerungen, obwohl sie von
ihrer Form her Aussagen über Gegenstände sind, dennoch eine andere
Bedeutung haben, die Bedeutung eines reinen Ausdruckes von inneren
Bewußtseinserfahrungen. Warum wird aber der Ausdruck von inneren
Erlebnissen in die Form gegenständlicher Aussagen gepreßt? Voegelins
Antwort lautet: Das Bewußtsein ist gegenständlich, und weil das
Bewußtsein gegenständlich ist, können Bewußtseinserfahrungen nicht
anders als in der uneigentlichen Form gegenständlicher Aussagen
artikuliert werden. Gegen diese Antwort liegen die Einwände jedoch auf
der Hand: Ungeachtet der Gegenständlichkeit oder
Nicht-Gegenständlichkeit des Bewußtseins, ist es ohne weiteres möglich,
den Ausdruck von Erfahrungen als solchen sprachlich kenntlich zu machen
und von Aussagen über Dinge zu unterscheiden, indem man z.B. Sätze von
der Form "`Ich hatte die Erfahrung, daß..."' oder "`Ich hatte ein
Gefühl, als ob..."'  bildet. So könnte jemand, dem eine
Transzendenzerfahrung widerfahren ist, beispielsweise äußern: "`Ich
hatte das Gefühl, daß das transzendente Sein in meine Seele eindrang."'
Wie dieses Beispiel gleichfalls vor Augen führt, wird die Möglichkeit,
Erfahrungen als Erfahrungen sprachlich zu artikulieren, auch nicht durch
die "`gegenständliche"' Subjekt-Prädikats-Form eingeschränkt, welche die
Grammatik den Sätzen unserer Sprache vorschreibt.\footnote{Derartiges
  deutet Voegelin in seinem Aufsatz "`Ewiges Sein in der Zeit"' an,
  worin die Theorie der sprachlichen Indizes ebenfalls angesprochen
  wird. Vgl. Voegelin, Anamnesis, S.266.} Abgesehen von diesen
Einwänden kann es gar nicht ohne weiteres als ausgemacht gelten, daß das
Bewußtsein in jeder Hinsicht als gegenstandsförmlich aufzufassen ist.
Zwar haben die meisten Bewußtseinsvorgänge (z.B. Wahrnehmen, Denken,
Fühlen) die Form intentionaler Akte, indem sich in ihnen ein Subjekt
durch einen Bewußtseinsakt auf einen Gegenstand des Bewußtseins bezieht.
Aber wie verhält es sich beispielsweise mit Stimmungen? Zudem wäre es
auch gar nicht ausdenklich, wie ein rein gegenstandsförmliches
Bewußtsein Transzendenzerfahrungen haben könnte, sofern diese
Erfahrungen ungegenständlich sind.
% Wenn sie aber doch gegenständlich
% sind\footnote{Was Voegelin jedoch eindeutig leugnet vgl. Voegelin, Anamnesis,
%   S.287.}, warum darf dann das in ihnen Erfahrene nicht wie ein Gegenstand
% diskutiert werden?

Doch Voegelin geht nicht nur von falschen Voraussetzungen aus. Seine
Theorie wirkt auch deshalb unglaubwürdig, weil er sich selbst nicht an die von
ihm gezogenen Grenzen hält. So bestreitet Voegelin zwar entschieden, daß eine
Diskussion über die Existenz von immanenter Welt und transzendentem Sein als
Entitäten sinnvoll ist, aber er setzt selbst die Existenz eines transzendenten
Seins als Entität voraus, wenn er beispielsweise an den Anfang seines
Aufsatzes "`Ewiges Sein in der Zeit"' den Satz stellt: "`Ewiges Sein
verwirklicht sich in der Zeit."'\footnote{Vgl. Voegelin, Anamnesis, S.254. -
  Der Aufsatz "`Ewiges Sein in der Zeit"' geht der Abhandlung "`Was ist
  politische Realität?"' in Voegelins Werk "`Anamnesis"' unmittelbar vorher.
  Beide Aufsätze stehen inhaltlich in enger Beziehung zu einander.} Wenn
"`Ewiges Sein"' in diesem Satz nicht als Entität, sondern bloß als Index des
Bewußtseins zu verstehen wäre, so könnte es sich gar nicht in der Zeit
verwirklichen, da es ja ohne das menschliche Bewußtsein noch gar nicht
existieren würde. Auch Voegelins Realitätsbegriff wäre nicht mehr haltbar,
wenn das tranzendente Sein als reiner Index des Bewußtseins verstanden werden
müßte. Voegelin behauptet ja gerade, daß die Realität des Partizipierens (am
transzendenten Seinsgrund) auch dann noch bestehen bleibt, wenn die Erfahrung
des Partizipierens verlorengegangen ist oder geleugnet wird. Wenn aber der
Seinsgrund nur Index des Bewußtseins wäre, dann würde es auch kein
Partizipieren ohne die Bewußtseinserfahrung des Partizipierens geben können. 

Schon von vornherein ließe sich gegen die Theorie der sprachlichen
Indizes eben jener ontologische Vorbehalt geltend machen, den Voegelin
am Ende seines Aufsatzes "`Zur Struktur des Bewußtseins"' gegenüber der
reinen Bewußtseinsphilosophie vertritt, daß es in erster Linie auf das
Sein und nicht auf das Bewußtsein ankommt.\footnote{Vgl. Voegelin,
  Anamnesis, S.56.} Wie leicht sich Voegelins Theorie der sprachlichen
Indizes aushebeln läßt, wenn man die Indizes als Kennzeichnung reiner
Erfahrungsbereiche auffasst, kann an Voegelins Behauptung demonstriert
werden, daß man den Menschen innerhalb einer philosophischen
Anthropologie nicht angemessen als welt-immanentes Wesen verstehen
könne. Diese Behauptung stellt sich bei genauerem Hinsehen als weit
anspruchsloser heraus als sie auf den ersten Blick erscheint. Denn da
"`welt-immanent"' für Voegelin lediglich ein Index der Erfahrung ist, so
beinhaltet diese Behauptung nur, daß nicht geleugnet werden darf, daß es
Menschen gibt, die innere Erlebnisse haben, zu deren Ausdruck sie sich
genötigt fühlen, Worte wie "`immanent"' und "`transzendent"' zu
verwenden.\footnote{Für den Fall, daß Voegelin so interpretiert werden
  müßte, daß nach seiner Theorie alle Menschen Transzendenzerlebnisse
  hätten, kann statt "`daß es Menschen gibt, die innere Erlebnisse
  haben,..."'  genausogut "`daß alle Menschen innere Erlebnisse
  haben,..."' eingesetzt werden. Das nachfolgende Argument bleibt dann
  immer noch gültig. Allerdings hätte dann auch der Materialist, der das
  transzendente Sein leugnet Transzendenzerlebnisse, was ihn jedoch
  nicht hindern muß ihre Wirklichkeit zu leugnen.} Dies nicht zu leugnen
dürfte allerdings auch dem hartgesottensten Materialisten keinerlei
Sorgen bereiten, da er dadurch ja noch längst nicht genötigt ist
zuzugeben, daß es ein transzendentes Sein tatsächlich gibt. Ja er könnte
unter Berufung auf Voegelins Theorie der sprachlichen Indizes sogar
ausdrücklich darauf verweisen, daß es illegitim sei, von einer inneren
Erfahrung, die als Erfahrung von Transzendenz sprachlich indiziert wird,
auf die Existenz eines transzendenten Seins zu schließen. Voegelins
Theorie gleicht daher - um ein Bild von Schopenhauer zu entlehnen -
einer Grenzfeste, die zwar uneinnehmbar ist, deren Besatzung aber auch
nicht in der Lage ist auszubrechen, so daß man sie getrost im Hinterland
zurücklassen kann.

Wenn die sprachlichen Indizes überhaupt irgendeinem Zweck dienen sollen,
so sind wir also gezwungen, hinter ihnen die Existenz von Entitäten
anzunehmen, auf welche sie verweisen. Voegelins Theorie der sprachlichen
Indizes hätte dann immer noch dadurch ihren guten Sinn, daß sie es verbietet,
sich bei der Diskussion über die Indizes von den Erfahrungen zu lösen, in
denen diese Entitäten mutmaßlich zum Vorschein kommen. Zur
erkenntnistheoretischen Rechtfertigung von Aussagen über die Transzendenz taugt
die Theorie der sprachlichen Indizes dann allerdings nicht mehr.

Kaum noch rechtfertigen läßt sich Voegelins Theorie der sprachlichen Indizes
jedoch dort, wo seine Indizes mit herkömmlichen Begriffen konkurrieren, wie
dies bei dem Begriff der Wissenschaft der Fall ist. Zwar ist es erfreulich zu
hören, daß die Methoden der modernen Naturwissenschaft "`mit der Ratio der
Noese verträglich sind"'.\footnote{Voegelin, Anamnesis, S.318.} Aber da das
Gelingen der Naturwissenschaft selbstverständlich in keiner Weise davon
abhängt, ob ihre Methoden mit der bewußt gewordenen existentiellen Spannung
zum Grund vereinbar sind, so ist es - jedenfalls soweit es um die
Naturwissenschaften geht - wenig sinnvoll, die Definition des Begriffes
Wissenschaft an die "`Platonisch-Aristotelische Noese"' zu knüpfen, zumal sich
die experimentelle Naturwissenschaft von der platonischen und aristotelischen
{\it episteme} sehr erheblich unterscheidet. Nicht ganz unzweifelhaft
erscheint auch die These, daß die Beseitigung "`mythische[r],
revelatorische[r] oder ideologische[r]
Wahrheitshypotheken"'\footnote{Voegelin, Anamnesis, S.318.}  durch die Noese
eine historische Ermöglichungsbedingung der Naturwissenschaft darstellt. Die
Anfänge der Naturwissenschaft fallen bereits in prä-noetische Zeit. So wurde
die Entwicklung der Astronomie durch das kosmologische Weltbild nicht etwa
behindert, sondern eher noch gefördert. Und bereits am Beispiel des Thales
läßt sich veranschaulichen, daß die Entgöttlichung der Welt, anders als dies
gelegentlich zu hören ist,\footnote{Vgl. Eric Voegelin: Die geistige und
  politische Zukunft der westlichen Welt (Hrsg. von Peter J.  Opitz und
  Dietmar Herz), München 1996, S.26/27.} keine notwendige Voraussetzung für
die Entfaltung unbefangenen naturwissenschaftlichen Forschergeistes darstellt,
denn Thales hinderte die Überzeugung, daß alles von Göttern erfüllt sei, nicht
daran, dieser Deutung die materialistische Erklärung hinzuzufügen, daß alles
auf und aus Wasser sei.

Aber auch wenn man sich nicht auf die Naturwissenschaften beschränkt, so kann
Voegelins Definitionsversuch, nach welchem Wissenschaft dasjenige ist, was
sich als "`das Strukturwissen von Realität"' infolge der sich historisch
ereignenden "`Selbsterhellung des Bewußtseins"' selbst
entdeckt,\footnote{Voegelin, Anamnesis, S.318.} nicht ohne weiteres
überzeugen. Ob irgendeine menschliche Erkenntnisaktitivität als Wissenschaft
eingestuft werden kann oder nicht, hängt weder von dem Selbstverständnis
derjenigen ab, die diese Aktivität ausüben (auch Alchemisten, Astrologen und
Naturheiler hielten und halten sich schließlich für Wissenschaftler), noch
hängt es von den historischen Rahmenbedingungen ab, unter denen diese
Erkenntnisaktivität entstanden ist. Entscheidend ist einzig und allein die
Frage, ob bei dieser Erkenntnisaktivität eine Welterkenntnis von objektiver
und nachprüfbarer Gültigkeit herauskommt. Eine Vorentscheidung über ein
bestimmtes, etwa mathematisch-naturwissenschaftliches Wissenschaftsmodell ist
mit diesem Kriterium noch nicht getroffen, so daß Voegelins Vorbehalten
gegenüber einer zu engen Wissenschaftsauffassung Rechnung getragen wird. Nur
wenn die historische {\it episteme} des Aristoteles nicht schon per
definitionem mit Wissenschaft gleichgesetzt wird, läßt sich außerdem die
wichtige Frage aufwerfen, ob und in welchem Maße die aristotelische {\it
  episteme} tatsächlich Wissenschaft ist.

Auf die Fragwürdigkeit der noetischen Definition von Voegelins Begriff der
Geschichte wurde bereits im vorhergehenden Abschnitt hingewiesen. Voegelins
Vorwurf gegen die hegelianischen Geschichtskonstruktionen, daß sie
fälschlicherweise ein reales Kollektivbewußtseins zu Grunde legen würden,
während Bewußtsein in Wirklichkeit nur "`diskret real"' vorkomme, ist dagegen
vollkommen berechtigt. Nur stellt sich die Frage, ob Voegelin nicht
seinerseits auf einer anderen Ebene den hegelianischen
Geschichtskonstruktionen nahekommt, wenn er darauf besteht, daß "`Geschichte
.. zu einem strukturell verstehbaren Feld der Realität durch die Präsenz des
einen Grundes"'\footnote{Voegelin, Anamnesis, S.320. - Vgl. auch Voegelin,
  Order and History IV, S.305.} wird, welches sich nicht in
Einzelvorstellungen auflösen ließe. Hier ist anzumerken, daß erstens nach wie
vor jeder Anhaltspunkt für die Richtigkeit der Annahme fehlt, daß es einen
transzendenten Seinsgrund gibt, und daß dieser ein einziger ist. Zweitens
operiert Voegelin mit der falschen Alternative, daß entweder ein gemeinsamer
Grund existieren müsse, oder nur "`jeder ein privates - im klassischen Sinne
von `idiotisches' - Bewußtsein für sich selbst"'\footnote{Voegelin, Anamnesis,
  S.320.} hätte. Auch wenn es keinen gemeinsamen transzendenten Seinsgrund
gibt, so können doch die diskret realen Bewußtseine durch miteinander Reden,
durch Einfühlung, Mitleid und teilnehmende Freude, durch gemeinsame Erlebnisse
und gemeinsames Handeln auf das Schönste zu einander in Kontakt treten. Und
drittens bleibt hinsichtlich der Bedeutung der Geschichte für den Menschen und
die Menschheit anzumerken, daß, solange in der Geschichte nicht irgend eine
Form religiöser Erbauung gesucht wird,\footnote{Vgl. dazu Poppers an Karl
  Barth anknüpfende Kritik der theogonischen Geschichtsdeutung, in: Karl
  Popper: Die offene Gesellschaft und ihre Feine. Band II. Falsche Propheten:
  Hegel, Marx und die Folgen, 7.Aufl., Tübingen 1992, S.316-328.} niemandem
etwas entgeht, wenn sich herausstellt, daß Geschichte kein "`Feld der Realität
[ist] ..., an dem alle Menschen partizipieren"'.\footnote{Voegelin, Anamnesis,
  S.320.}

Im Ganzen stellt Voegelins Theorie der Indizes ein sehr fragwürdiges
Unterfangen dar. Sie geht nicht nur von falschen Voraussetzungen bezüglich der
Natur des menschlichen Bewußtseins und der Sprache aus, sondern sie führt als
ein Verfahren der Begriffsklärung oft zu recht willkürlichen Definitionen
zentraler Begriffe wie z.B. Wissenschaft, Geschichte, Rationalität, Realität.
Es fällt nicht leicht, sich hierbei des Eindruckes zu erwehren, daß Voegelin
versucht, höchst strittige Sachfragen (wie z.B. ob Rationalität in der
Spannung zum Grund besteht, ob Realität in erster Linie spirituelle Realität
ist, ob Geschichte sich nicht in der Zeit sondern in der Bewußtseinsdimension
des Begehrens und der Suche nach dem Grund abspielt...)  durch Definitionen
vorzuentscheiden und ihre Diskussion dadurch zu verhindern, daß er von
vornherein alle zentralen Begriffe für sich reklamiert, so daß die
Formulierung von Kritik erheblich erschwert wird.

% Diese Deutung Voegelins ist deshalb ein wenig erstaunlich, weil Voegelin
% unmittelbar zuvor noch das politische Unheil aus der Leugnung der
% metaphysischen Seinsrealität abgeleitet hat.\footnote{Vgl. Anamnesis,
%   S.302/303.} Nun soll es auf einmal Ausdruck einer Übersteigerung der
% Erfahrung von eben dieser Seinsrealität sein. Es handelt sich um zwei
% unterschiedliche und sogar gegensätzliche Formen von Realitätsverlust, so daß
% sie sich als Erklärung für dieselbe politische Unordnung offenbar
% ausschließen.\footnote{Gelegentlich (aber nicht: Anamansis, S.302/303.)
%   stellt Voegelin es so dar, daß Beseitigung der Realitätserfahrung durch
%   Aufklärung und Positivismus lediglich die Voraussetzung für die Auffüllung
%   des menschlichen Denkens und Empfindens durch falsche Realitätsbilder
%   schafft, welche dann erst das eigentliche politische Chaos herbeiführt. Aber
%   auch dann haben wir zwei widersprüchliche Erklärungen: Die eine nimmt an,
%   daß falsche Realitätsbilder die Ursache des Chaos sind. Die andere nimmt an,
%   daß eine allzu intensive Erfahrung der wahren Realität das Chaos verursacht.
%   Da die historischen Voraussetzungen der ersten Erklärung (Abstumpfung jedes
%   spirituellen Realitätsempfindens durch zersetzende intellektualistische
%   Kritik) das Eintreten des zweiten Phänomens eher ausschließen, müßte man
%   folgern, daß beide Formen von Realitätsverlust nur historisch sehr weit
%   voneinander entfernt auftreten können. Geht es nur um den Totalitarismus im
%   20.Jahrhundert, so hätte sich Voegelin für eine Erklärung entscheiden
%   müssen.} Besonders deutlich wird dieser Widerspruch bei Voegelins Deutung
% der aufklärerischen Fortschrittsidee. Wenn die Aufklärung die Leugnung der
% metaphysischen Realitätserfahrung par exellence verkörpert, wie kann dann die
% aufklärerische Fortschrittsidee zugleich Ausdruck des Überschießens dieser
% Realitätserfahrung sein?
 
%Für den Leser bleibt dadurch
%ein etwas fader Beigeschmack von intellektueller Unredlichkeit zurück.

% Warum können die als "`transzendentes Sein"' indizierten Erfahrungen
% bzw. Erfahrungsbestandteile nicht ebensogut mit den Worten "`geistige
% Verwirrung"' indizieren. 

%  Sie kann in einer starken Form interpretiert
% werden, so daß die sprachlichen Indizes ausschließlich als Ausdruck von
% Partizipationserfahrungen zu verstehen sind, ohne daß in jedem Falle die
% Existenz der Termini des Partizipation außerhalb der Erfahrung vorausgesetzt
% wird. Nach einer schwächen Interpretation von Voegelins Theorie würde dagegen
% auch eine erfahrungsunabhängige Existenz der Termini des Partizpierens
% angenommen. Dann wäre beispielsweise die Existenz des transzendenten
% Seinsgrundes immer noch noch gegeben, wenn kein Bewußtsein vorhanden wäre, um
% ihn zu erfahren. Voegelins Theorie würde in diesem Fall lediglich fordern, daß
% die Diskussion über die "`Termini des Partizipierens"' sich eng an das in der
% Partizipationserfahrung gegebene halten muß.

% Für die strenge Interpretation von Voegelins Theorie spricht, daß Voegelin
% selbst ausdrücklich sagt, daß es immanentes und transzendentes Sein nicht als
% Entitäten gebe, und daß eine Diskussion über ihre Existenz nicht möglich
% sei.\footnote{Vgl. Voegelin, Anamnesis, S.316.} Aber diese Interpretation
% führt, wenn man konsequent an ihr festhält, zu Ergebnissen, von denen sehr
% fraglich ist, ob Voegelin sie ernsthaft hätte vertreten wollen. 

%%% Local Variables: 
%%% mode: latex 
%%% TeX-master: "Main" 
%%% End:





