
%%% Local Variables: 
%%% mode: latex
%%% TeX-master: "Main"
%%% End: 

\chapter{Die Grundzüge von Voegelins Philosophie}

\section{Voegelins theoretischer Ansatz}

\subsection{Die Kritik des Positivismus}

Eric Voegelin entwickelte seinen eigenen wissenschaftlichen Ansatz in
ausdrücklicher Opposition zu den herkömmlichen Vorgehensweisen in den
Sozialwissenschaften, wobei er sich insbesondere gegen die szientistischen
Ansätze in den Gesellschaftswissenschaften wandte. Diese kritische Seite der
wissenschaftlichen Neuorientierung, die Voegelin in der "`Neue[n] Wissenschaft
der Politik"' vornimmt, betrifft das an den Naturwissenschaften orientierte
Methodenideal des Positivismus sowie die von Max Weber aufgestellte Forderung
der Wertfreiheit der Wissenschaft.

Voegelin unternimmt in der "`Neuen Wissenschaft der Politik"' nicht die
Auseinandersetzung mit einer bestimmten, elaborierten positivistischen
Wissenschaftstheorie. Es geht ihm vielmehr um die Charakterisierung der
geistesgeschichtlichen Strömung des Positivismus und um die Kritik des
geistigen Klimas, welches diese Strömung in den Gesellschaftswissenschaften
hervorgerufen hat. Den Begriff des Positivismus faßt Voegelin dabei
vergleichsweise weit. Seiner skizzenhaften historischen Darstellung zufolge
ging der Positivismus aus von der Rezeption der Newtonschen Physik durch die
Aufklärer und lief von diesem Ausgangspunkt über Auguste Comte als seinem
ersten vorläufigen Höhepunkt fort bis zur Entwicklung der Methodologie am Ende
des 19. und Anfang des 20.Jahrhunderts. Die Methodologie trägt als eine
skeptische und ideologiekritische Erscheinung allerdings auch schon den Keim
der Gegenbewegung in sich.\footnote{Vgl. Voegelin, Neue Wissenschaft der
  Politik, S.24-31. - Eine ausführliche Darstellung von Voegelins
  Positivismuskritik in: Barry Cooper: Eric Voegelin and the Foundations of
  Modern Political Science, Columbia and London 1999, S.67ff.}
  
Das positivistische Denken führt nach Voegelins Ansicht dazu, daß als
Untersuchungsgegenstand der Wissenschaft nur noch dasjenige zugelassen wird,
was sich mit einem bestimmten Kanon quasi-naturwissenschaftlicher Methoden
erfassen läßt. Hierdurch wird in Voegelins Augen die Relevanzordnung der
Wissenschaft geradezu umgekehrt. Denn anstatt daß das Thema bzw. die
wissenschaftliche Fragestellung vorgegeben ist und der Wissenschaftler sich
nun nach den geeigneten Methoden zur Bearbeitung dieses Themas umsieht, gibt
nach dem positivistischen Wissenschaftsverständnis der Methodenkatalog vor,
welche Fragen überhaupt gestellt werden können.\footnote{Vgl. Voegelin, Neue
  Wissenschaft der Politik, S.24.} Nun gibt es aber Fragen, die nicht nur für
das Leben des Einzelnen von größter Bedeutung sind, sondern deren Beantwortung
auch die Gestalt der Gesellschaft und die Form der politischen Ordnung
entscheidend prägt, welche aber unter den methodologischen Vorgaben des
Positivismus kaum angemessen untersucht werden können. Hierzu gehören
beispielsweise die Frage nach dem, was moralisch gut und richtig ist, oder
auch die Frage nach dem Sinn des Lebens oder dem Sinn der Welt im Ganzen. Es
ist offensichtlich, daß jede Gesellschaft vor der Herausforderung steht, auf
die erste dieser Fragen eine gemeinverbindliche Antwort zu geben. Und die
Antwort auf die zweite Frage ist ersichtlich wenigstens dort von öffentlicher
Bedeutung, wo die Religion noch einen großen Einfluß auf die Politik ausübt.
%\footnote{Aber auch dort, wo die Religion keinen Einfluß mehr auf die Politik
%  ausübt oder ausüben soll, wie z.B. in den laizistischen Staatswesen, ist
%  implizit zu mindest eine negative Teilantwort auf diese Frage gegeben
%  worden, nämlich die, daß der Sinn des Lebens oder die Welt im Ganzen nicht
%  so beschaffen ist, daß für alle Menschen dieselbe Antwort und zugleich die
%  Pflicht zu deren verbindlicher Anerkennung gilt.}  
Nach dem positivistischen Verständnis ist es jedoch unmöglich, auf irgend eine
dieser Fragen eine objektive Antwort zu geben.  Gegenstand der Wissenschaft
kann nach positivistischer Auf\/fassung daher bestenfalls sein, welche Antworten
Gesellschaften oder einzelne Menschen auf diese Fragen geben oder unter
welchen Bedingungen Menschen dazu neigen, derartige Fragen aufzuwerfen und
dann auf diese oder jene Weise zu beantworten. Ausgeschlossen ist jedoch die
Erörterung moralischer oder spiritueller Fragen durch die Wissenschaft selbst.
Solche Fragen dürfen vom Forscher, wenn überhaupt, dann höchstens privat und
nach Feierabend gestellt werden.\footnote{Gerade dies ist es, was Max Weber
  mit gar nicht schlechten Gründen fordert. Vgl. Max Weber: Der
  Sinn der "`Wertfreiheit"' der soziologischen und ökonomischen
  Wissenschaften, in: Max Weber: Gesammelte Aufsätze zur Wissenschaftslehre,
  Tübingen 1988, im folgenden zitiert als: Weber, Wissenschaftslehre,
  S.489-540 (S.492/493).}

Genau dies hält Voegelin aber für einen untragbaren Zustand. Wichtiger
vielleicht noch als das Problem der massenhaften Anhäufung irrelevanter
Nebensächlichkeiten, welches Voegelin dem Positivismus ebenfalls
vorwirft,\footnote{Vgl. Voegelin, Neue Wissenschaft der Politik, S.27. - Vgl.
  auch die Diskussion des Relevanzproblems im Briefwechsel zwischen Voegelin
  und Alfred Schütz, der in dieser Frage eine klarere Auf\/fassung vertritt, in:
  Eric Voegelin / Alfred Schütz / Leo Strauss / Aron Gurwitsch: Briefwechsel
  über "`Die Neue Wissenschaft der Politik"' (Hrsg. von Peter J. Opitz),
  München 1993, 55ff. Schütz geht von einer bloß relativen Relevanz in Bezug
  auf bestimmte Fragestellungen aus. Voegelin gerät dagegen in
  Schwierigkeiten, wenn er den Anspruch absoluter Relevanz von bestimmten
  Fragestellungen begründen will.} ist es daher, daß einige der relevantesten
Fragen der Menschheit vom Positivismus unter ein wissenschaftstheoretisches
Tabu gestellt werden. Wie soll, so könnte man im Sinne Voegelins fragen,
Politikwissenschaft möglich sein, wenn sie auf die Frage, ob der Faschismus
dem Kommunismus vorzuziehen sei oder beiden vielleicht der Liberalismus, nur
mit einem Achselzucken oder allenfalls mit dem (natürlich wertfrei zu
haltenden) Hinweis auf die möglichen Folgen der Entscheidung antworten kann?
Das Problem der Möglichkeit von Werturteilen in der Wissenschaft wird von
Voegelin besonders detailliert an Max Weber herausgearbeitet.

Max Weber vertrat die Ansicht, daß Wertfragen nicht objektiv beantwortet
werden können. Daher kann auch nicht wissenschaftlich über die Richtigkeit und
Falschheit von Werten befunden werden. Wertfragen müssen vielmehr entschieden
werden. Weber hat diese Art von Entscheidungen, die letztlich ohne rationale
Anhaltspunkte getroffen werden müssen, auch des öfteren als "`dämonisch"'
charakterisiert.\footnote{Vgl. Voegelin, Neue Wissenschaft der Politik,
  S.33/34. - Zu der recht komplexen Beziehung Voegelins zu Webers Werk, auf
  die hier nicht ausführlich eingegangen werden kann, vgl. Peter J. Opitz: Max
  Weber und Eric Voegelin, in: Eric Voegelin: Die Größe Max Webers. (Hrsg. von
  Peter J.  Opitz), München 1995, S.105-133.} Dennoch spielen Werte in der
Wissenschaft in einem anderen Zusammenhang bei Max Weber durchaus eine Rolle.
Die Auswahl des Gegenstandes der Wissenschaft erfolgt nämlich wertgesteuert
durch die wissenschaftliche Interessenrichtung des Forschers.\footnote{Vgl.
  Max Weber: Die "`Objektivität"' sozialwissenschaftlicher und
  sozialpolitischer Erkenntnis, in: Weber, Wissenschaftslehre, S.146-214.
  (S.175-185).} Dies ist einer der Punkte, an denen Voegelins Kritik am
Wertfreiheitsdogma ansetzt.  Wenn Werte nicht wissenschaftlich begründet
werden können, sie aber gleichzeitig die Voraussetzung zur "`Konstitution des
Gegenstandes der Wissenschaft"'\footnote{Voegelin, Neue Wissenschaft der
  Politik, S.37.} bilden, dann gibt es ebenso viele Wissenschaften, wie es
Werte gibt. Das Ergebnis wäre ein Relativismus als Konsequenz der
Objektivitätsforderung.\footnote{Vgl.  Voegelin, Neue Wissenschaft der
  Politik, S.37.} Für Voegelin hat Max Weber damit das Prinzip der wertfreien
Wissenschaft ad absurdum geführt. Voegelin anerkennt durchaus das historische
Verdienst Max Webers, welches für ihn darin besteht, daß Max Weber das
Problematische des Positivismus reflektiert hat, wenn er auch immer noch in
den positivistischen Tabus seiner Zeit befangen blieb. Daß Max Weber der
Durchbruch zu einer umfassenden, d.h. auch Werte mit einschließenden
politischen Ordnungswissenschaft nicht gelungen ist, erklärt sich Voegelin mit
eben dieser Befangenheit Webers und damit, daß Weber bei seinen historischen
Studien genau die Epochen und Denker ausließ, bei denen er auf eine solche
Ordnungswissenschaft hätte stoßen können, nämlich die Epochen der griechischen
Antike und des vorreformatorischen Christentums, in denen Denker wie Platon
und Aristoteles oder, im anderen Fall, Thomas von Aquin über eine politische
Ordnungswissenschaft verfügten.\footnote{Vgl.  Voegelin, Neue Wissenschaft der
  Politik, S.41.}

Es gibt jedoch einen Punkt, über den Voegelin bei seiner Auseinandersetzung
mit Max Weber mit einer gewissen Ungeduld hinweggeht. Daß Max Weber Werte für
rational unbegründbar hält, hängt nicht bloß mit dem ungünstigen historischen
Umstand zusammen, daß er in einer positivistisch geprägten Epoche lebte, noch
kann es restlos dadurch erklärt werden, daß Max Weber die Auseinandersetzung
mit dem christlichen Mittelalter und Vertretern der Ordnungswissenschaft wie
Thomas von Aquin, Platon und Aristoteles peinlichst vermieden hätte. Vielmehr
besaß Max Weber sachliche Gründe für die Annahme, daß sich Werte nicht
rational begründen lassen. Bisher, und dies gilt auch noch 80 Jahre nach Max
Weber, ist es noch niemandem gelungen, die Gültigkeit irgendeiner moralischen
Norm vollständig zu beweisen. Das Einzige, was erreicht worden ist, ist die
Rückführung moralischer Normen auf andere Normen. Irgendwann einmal gelangt
man auf diese Weise jedoch zu einer obersten Norm (z.B. Menschenliebe,
kategorischer Imperativ oder dergleichen), die sich nicht auf weitere Normen
zurückführen läßt. Es kann nun behauptet werden, daß diese Norm ein "`Faktum
der Vernunft"'\footnote{Immanuel Kant: Kritik der praktischen Vernunft,
  Hamburg 1990, S.36.} oder ein Befehl Gottes ist oder daß sie von Natur aus
gilt. Aber all das sind lediglich eloquente Beteuerungen ihrer Gültigkeit und
keine rationalen Begründungen. Max Weber hielt Wertfragen deshalb
wissenschaftlich nicht für entscheidbar. Aus diesem Grund sprach er sich
dagegen aus, in der Wissenschaft Werturteile zu fällen, und nicht bloß, weil
er ein Opfer des positivistischen Zeitgeistes gewesen wäre.\footnote{Vgl. Max
  Weber: Die "`Objektivität"' sozialwissenschaftlicher und sozialpolitischer
  Erkenntnis, in: Weber, Wissenschaftslehre, S.146-214 (S.151-157). - Vgl. Max
  Weber: Der Sinn der "`Wertfreiheit"' der soziologischen und ökonomischen
  Wissenschaften, in: Weber, Wissenschaftslehre, S.489-540 (S.508). - Zur
  Diskussion über die Wertfreiheit in den Sozialwissenschaften: Hans Albert /
  Ernst Topitsch (Hrsg.): Werturteilsstreit, Darmstadt 1971, im folgenden
  zitiert als: Albert/Topitsch, Werturteilsstreit. Darin besonders deutlich
  gegen die Möglichkeit wissenschaftlicher Wertbegründung: Walter Dubislav:
  Zur Unbegründbarkeit der Forderungssätze, S.439-454. Noch am ehesten mit
  Voegelins Auf\/fassung vergleichbar: Jürgen Habermas: Erkenntnis und
  Interesse, S.334-364 (S.337).} Daher ist es auch kaum anzunehmen, daß Max
Weber seine Auf\/fassungen zur Werturteilsproblematik hätte revidieren müssen,
wenn er die Zeitalter von Aristoteles oder von Thomas von Aquin in seinen
Forschungen stärker berücksichtigt hätte, denn weder Aristoteles noch Thomas
von Aquin haben einen Methode gefunden, mit der Wertfragen wissenschaftlich
entschieden werden können.

Ebensowenig stimmt es, daß die interessengeleitete bzw. wertbezogene Auswahl
der Gegenstände wissenschaftlicher Forschung - Voegelin spricht hier mit einer
sehr unklaren phänomenologischen Terminologie von der "`Konstitution des
Gegenstandes der Wissenschaft"'\footnote{Voegelin, Neue Wissenschaft der
  Politik, S.37.} - zum Relativismus führt. Von Relativismus kann nur dann die
Rede sein, wenn es sich um widersprüchliche Aussagen zu ein und demselben
Gegenstand handelt, die alle vom jeweiligen Standpunkt aus gleichermaßen
berechtigt erscheinen. Da sich die Wertbezogenheit aber nur auf die Auswahl
der Untersuchungsgegenstände bezieht und nicht die Aussagen über diese
Gegenstände selbst und die Verfahren zur Prüfung der Richtigkeit der Aussagen
betrifft, wird die Gefahr des Relativismus vermieden.\footnote{Vgl. Ernest
  Nagel: Der Einfluß von Wertorientierungen auf die Sozialforschung, in:
  Albert/Topitsch, Werturteilsstreit, S.237-250 (S.237-239). Eine m.E.
  Voegelins Sichtweise ähnelnde Auf\/fassung vertritt dagegen Hans Albert, in:
  Hans Albert: Kritische Vernunft und menschliche Praxis, Stuttgart 1977,
  S.71f. Albert scheint jedoch zu übersehen, daß die Normierungen, die den
  Erkenntnisprozeß durchsetzen, ausschließlich von der Art der hypothetischen
  Imperative Kants sind, welche kein eigentliches Wertbegründungsproblem
  aufwerfen.} Max Weber gerät auch nicht in einen Widerspruch, wenn er
ungeachtet der Wertfreiheit der Wissenschaft den Marxismus wissenschaftlich
kritisiert, denn das Wertesystem des Marxismus bleibt von dieser Kritik
unberührt. Lediglich die Sachaussagen des Marxismus, also etwa Aussagen über
den vermuteten Verlauf der künftigen Geschichte, können wissenschaftlich
kritisiert werden.

Die Positivismuskritik Voegelins bedarf ebenfalls einer gewissen
Differenzierung: Das zentrale Motiv wenigstens des Neupositivismus ist
nicht so sehr die Forderung nach Imitation der naturwissenschaftlichen
Methoden in allen Wissensbereichen. Dem Neupositivismus geht es
vielmehr darum, daß Erkenntnis nur möglich ist, wenn sich Kriterien
anführen lassen, die es erlauben, die Richtigkeit oder Falschheit von
Behauptungen festzustellen.\footnote{Vgl. Richard von Mises: Kleines
  Lehrbuch des Positivismus. Einführung in die empiristische
  Wissenschaftsauf\/fassung, Frankfurt am Main 1990 (zuerst: Den Haag
  1939), S.135ff. - Ob Voegelins Kritik tatsächlich auch auf den
  Neupositivismus, wie er vom Wiener Kreis um Moritz Schlick
  entwickelt wurde, abzielt, läßt sich der Darstellung in der "`Neuen
  Wissenschaft der Politik"' nicht unmittelbar entnehmen. Aber
  schwerlich kann Voegelin nur Auguste Comte im Auge haben, der zu der
  Zeit, als die "`Neue Wissenschaft der Politik"' erschien, in der
  philosophischen und wissenschaftlichen Diskussion keine
  Rolle mehr spielte. (Vgl. Robert A. Dahl: The Science of politics: New
  and Old, in: World Politics Vol. VII (April 1955), S.484-489.)} Nur
dann läßt sich überhaupt zwischen echter Erkenntnis und bloßer Meinung
unterscheiden.  Zumindest bezogen auf wissenschaftliche Erkenntnis ist
kaum zu bestreiten, daß diese Forderung berechtigt ist, wobei
allerdings über die erforderliche Strenge der Prüfungskriterien
unterschiedliche Auf\/fassungen bestehen können.  Die Bemerkungen, die
Voegelin zu dem Problem der Überprüfung wissenschaftlicher
Erkenntnisse in der "`Neue[n] Wissenschaft der Politik"' fallen läßt,
sind wenig erhellend. Sie besagen kaum mehr, als daß die Ergebnisse
einer wissenschaftlichen Untersuchung die Erwartungen des
Wissenschaftlers erfüllen müssen.\footnote{Vgl. Voegelin, Neue
  Wissenschaft der Politik, S.23. Voegelin äußert sich dort mit Worten
  wie diesen: "`Wenn die Methode das anfangs nur trübe Geschaute zu
  wesenhafter Klarheit gebracht hat, dann war sie adäquat; ..."'. -
  Interessanterweise kritisiert Voegelin eben diesen Grundsatz, daß
  die Wahrheit der Prämissen durch das Ergebnis der Untersuchung
  gerechtfertigt wird, einige Jahre später bei Hegel auf das
  Schärfste. Vgl. Eric Voegelin: Wissenschaft, Politik und Gnosis,
  München 1959, im folgenden zitiert als: Voegelin, Wissenschaft,
  Politik und Gnosis, S.55.} Dies garantiert jedoch noch keine
Erkenntnis und könnte schlimmstenfalls sogar auf die bloße Bestätigung
der Vorurteile des Forschers hinauslaufen.

Ist Voegelins Kritik des Positivismus daher zwar in mancher Hinsicht
unzulänglich, so bleibt seine Grundintention, die auf die Schaffung der
Politikwissenschaft als einer umfassenden Ordnungswissenschaft zielt, welche
sich auch den Wert- und Sinnfragen nicht verschließt, dennoch nachvollziehbar.
Wie sieht nun diese umfassende Ordnungswissenschaft aus?

\subsection{Politikwissenschaft als Ordnungswissenschaft}

Voegelin verfolgt mit seiner Politikwissenschaft sowohl eine rein
theoretische als auch eine normative Absicht. Zum einen stellt er Prinzipien
zur Analyse bestehender politischer Ordnungen auf, zum anderen glaubt er,
Kriterien angeben zu können, mit denen über den Wert einer politischen Ordnung
objektiv entschieden werden kann. Beidem liegt jedoch ein und dieselbe
dogmatische Vorstellung vom Wesen politischer Ordnung zu grunde: Politische
Ordnung ist Voegelin zufolge stets ein Abbild der Seinsordnung, wie sie von
der jeweiligen Gesellschaft in spiritueller Erfahrung erlebt wird.

\subsubsection{"`Artikulation"' und "`Repräsentation"' als Grundfunktionen
  politischer Ordnung}

Im Zentrum des nicht-normativen Teils des politikwissenschaftlichen
Programmes der "`Neuen Wissenschaft der Politik"' stehen die Begriffe
der Artikulation, der Repräsentation und der Erfahrung. Unter
Artikulation versteht Voegelin den Prozeß der Entstehung einer
politischen Gesellschaft. Voegelin spricht auch davon, daß sich eine
Gesellschaft "`zur historischen Existenz"' artikuliere.\footnote{Vgl.
  Voegelin, Neue Wissenschaft der Politik, S.61., S.67.} Den Ausdruck
"`Artikulation"' gebraucht Voegelin deshalb, weil die Symbole, mit denen
eine Gesellschaft ihr Selbstverständnis ausdrückt, in seinen Augen
bereits einen wesentlichen Teil der gesellschaftlichen Wirklichkeit
ausmachen und dadurch die politische Gemeinschaft recht eigentlich erst
hervorbringen.\footnote{Vgl. Voegelin, Neue Wissenschaft der Politik,
  S.50.}  Voegelin bezeichnet diesen Komplex von Symbolen, die das
Selbstverständnis einer Gesellschaft ausdrücken, auch als
"`Symbolismus"'.\footnote{Der Ausdruck "`Symbolismus"' dürfte an Ernst
  Cassirers "`Philosophie der symbolischen Formen"' angelehnt sein. Auch
  Cassirer hält die Fähigkeit, Symbole zu bilden, für eine menschliche
  Leistung {\it sui generis} und betrachtet den Begriff der
  "`symbolischen Form"' daher als einen irreduziblen Grundbegriff der
  Humanwissenschaften, ohne allerdings so weitreichende Konsequenzen zu
  ziehen wie Voegelin.  Vgl. Ernst Cassirer: Versuch über den Menschen.
  Einführung in eine Philosophie der Kultur, Hamburg 1996, S.49.} In dem
Ausdruck "`Artikulation"' klingt darüber hinaus etwas von Voegelins
spezifischem Verständnis von politischer Ordnung an. Voegelin zufolge
äußern die politischen Gesellschaften nicht bloß irgendein beliebiges
Selbstverständnis, sondern durch ihre Ordnung "`artikulieren"' sie
zugleich ihr Verständnis der Ordnung des Seins.\footnote{In der
  "`History of Political Ideas"' spricht Voegelin noch etwas plastischer
  von "`Evokation"'.  (Vgl.  Voegelin, "`Introduction"' zur "`History of
  Political Ideas"', S.23ff.)  Möglicherweise erschien Voegelin der
  Ausdruck "`Evokation"' später zu relativistisch, indem dieses Wort
  suggeriert, daß die "`evozierte"' Realität ein gesellschaftliches
  Artefakt ist und nicht Ausdruck von spirituellen Erfahrungen.  - Etwas
  mysteriös erscheint es dabei, daß bloß durch die Bildung und den
  Gebrauch von Symbolen eine Wirklichkeit soll hervorgebracht werden
  können.  Die Möglichkeit eines solchen Vorgangs und der
  Wirklichkeitscharakter dieser Symbolwelten - für Voegelin offenbar
  fraglose Selbstverständlichkeiten - bedürften eigentlich noch der
  Klärung.}

Zur Artikulation, d.h. zur Entstehung und Erhaltung einer politischen
Gesellschaft gehört aber auch eine Form herrschaftlicher Organisation dieser
Gesellschaft. Diesen Aspekt beschreibt Voegelin mit dem Begriff der
Repräsentation. Unter Repräsentation versteht Voegelin, abweichend vom üblichen
Begriff demokratischer Volksvertretung, die herrschaftliche Vertretung
beliebiger Art im politischen Handeln der Gesellschaft.\footnote{Vgl.
  Voegelin, Neue Wissenschaft der Politik, S.60 unten, S.61 oben, wo sehr
  deutlich wird, daß Voegelin mit "`Repräsentation"' eigentlich eher
  Herrschaft als Repräsentation im Sinne einer ganz bestimmten und eben nicht
  beliebigen Form der Herrschaftsbestellung meint.} Voegelin unterscheidet drei
Ebenen der Repräsentation: deskriptive Repräsentation, existenzielle
Repräsentation und transzendente Repräsentation. Unter deskriptiver
Repräsentation versteht Voegelin ein beliebiges System institutioneller
Regelungen, welches handlungsbevollmächtigte Vertreter einer politischen
Gemeinschaft hervorbringt. Ausgeschlossen bleiben auf dieser Ebene noch Fragen
wie die nach der Legitimität, Effizienz und auch der tieferen Wahrheit eines
solchen Systems.\footnote{Vgl. Voegelin, Neue Wissenschaft der Politik, S.57.
  - Auch wenn Voegelin es anfangs so erscheinen läßt, deckt sich sein Begriff
  deskriptiver Repräsentation nicht mit dem in der (westlichen)
  Politikwissenschaft üblichen Begriff von Repräsentation, denn das
  herkömmliche Verständnis von Repräsentation umfaßt auch den
  Legitimitätsaspekt und beschränkt den Begriff andererseits auf die
  demokratische Repräsentation, so daß es ein klarer Mißbrauch dieses
  Ausdruckes wäre, im Falle des Sowjetsystems von repräsentativer Regierung zu
  reden.} Von existenzieller Repräsentation spricht Voegelin, wenn eine
deskriptive Repräsentation vorliegt, durch die eine Herrschaft hervorgebracht
wird, deren Anordnungen Gehorsam finden und die in der Lage ist, die vitalen
Bedürfnisse einer politischen Gesellschaft (Schutz nach außen, Sicherheit im
Inneren) zu garantieren. Der Begriff entspricht weitgehend dem, was man
üblicherweise eine legitime Herrschaft nennt.\footnote{Vgl. Voegelin, Neue
  Wissenschaft der Politik, S.77.}

Bis zu diesem Punkt bietet Voegelin, abgesehen von seiner eigenwilligen
Terminologie, nichts Ungewöhnliches. Ein grundlegend neuer Aspekt tritt jedoch
mit der dritten Bedeutungsebene von Voegelins Repräsentationsbegriff, der
transzendenten Repräsentation, hinzu. Die transzendente Repräsentation bezieht
sich nicht mehr nur auf Regierung und Herrschaft, sondern auf die politische
Ordnung einer Gesellschaft im Ganzen. Alle politischen Gesellschaften halten
ihrem Selbstverständnis nach ihre eigene politische Ordnung für die wahre
Ordnung. Voegelin faßt dies so auf, daß die politischen Gesellschaften durch
ihre politische Ordnung eine höhere Wahrheit repräsentieren.\footnote{Vgl.
  Voegelin, S.81ff.} So glaubten etwa die Menschen in Mesopotamien oder im
alten Ägypten, daß sich in ihrer politischen Ordnung die Ordnung des Kosmos
widerspiegele bzw. fortsetze. Doch ist dies nicht die einzige Möglichkeit der
Wahrheitsrepräsentation. In Platons idealem Staat etwa repräsentiert die
politische Ordnung eine Wahrheit, derer der Philosoph im Inneren seiner Seele
gewahr wird. Voegelin sieht deshalb im Auftreten Platons den Durchbruch zu
einem neuen Typus von transzendenter Repräsentation.\footnote{Vgl. Voegelin,
  Neue Wissenschaft der Politik, S.93.}

\subsubsection{Der Begriff der "`Erfahrung"' als Zentralbegriff von Voegelins
  Theorie politischer Ordnung}

Eine zentrale Stellung kommt in diesem Zusammenhang dem Begriff der Erfahrung
zu. Die Wahrheit, die die politischen Gesellschaften repräsentieren, beruht
nach Voegelins Ansicht auf einer spirituellen Erfahrung der Ordnung des Seins.
Dieser Begriff der Erfahrung verlangt auf Grund seiner großen Bedeutung für
Voegelins Verständnis von politischer Ordnung eine etwas eingehendere
Untersuchung.\footnote{Zur Genese des Begriffes der Erfahrung im Werk Eric
  Voegelins: Vgl. Peter J. Opitz: Rücker zur Realität: Grundzüge der
  politischen Philosophie Eric Voegelins, in: Peter J.  Opitz /
  Gregor Sebba (Hrsg.): The Philosophy of Order. Essays on History,
  Consciousness and Politics, Stuttgart 1981, S.21-73.}

Erfahrung spielt bereits in Voegelins frühesten Schriften eine Rolle, in
welchen der Nachvollzug der seelischen Hintergründe und motivierenden
Erfahrungen zu den grundlegenden Methoden des Verständnisses philosophischer
Texte gehört.\footnote{Deutlich wird dies etwa in: Eric Voegelin: On the Form
  of the american Mind, Baton Rouge / London 1995, S.23ff.} Voll entfaltet und
für das Verständnis politischer Ordnungen fruchtbar gemacht wird dieser
Begriff jedoch erst mit "`Order and History"'. Der Begriff der Erfahrung ist
nicht nur einer der wichtigsten Begriffe bei Voegelin, sondern, von den in ihn
eingehenden theoretischen Voraussetzungen her, zugleich auch einer der
anspruchsvollsten Begriffe Voegelins.

Um Verwechselungen zu vermeiden, soll zunächst geklärt werden, was
"`Erfahrung"' bei Voegelin nicht bedeutet: "`Erfahrung"' bedeutet bei Voegelin
nicht wissenschaftliche Empirie. Die wissenschaftliche Empirie bezieht sich
auf deutlich abgrenzbare und klar beschreibbare Sinneserfahrungen. Die
Erfahrung Voegelins meint dagegen eher ein schwer faßbares inneres Erleben.
Zwar spricht Voegelin an einer Stelle davon, daß man sich zur empirischen
Überprüfung auf die Erfahrung zu beziehen habe, aber dies geschieht wohl
vornehmlich aus dem Wunsch, seinen eigenen Begriff von Erfahrung an die Stelle
der wissenschaftlichen Empirie treten zu lassen, und nicht weil diese beiden
Begriffe irgend etwas gemeinsam hätten.\footnote{Vgl. Voegelin, Neue
  Wissenschaft der Politik, S.96. - Voegelins Fehler, um nicht zu sagen sagen,
  seine Unredlichkeit besteht darin, daß er einerseits von einer Überprüfung
  an der Erfahrung spricht, daß dann aber, wenn sich bei bestimmten Menschen
  diese Erfahrung nicht einstellt, die Betreffenden kurzerhand für verstockt
  und ihre Erfahrungen für deformiert erklärt werden. Unter solchen Bedingungen
  ist eine Überprüfung anhand der Erfahrung natürlich ausgeschlossen. Vgl.
  auch Ted V. McAllister: Revolt against modernity. Leo Strauss, Eric Voegelin
  \& the Search For a Postliberal Order, Kansas 1995, S.172. (McAllister
  scheint das Problematische daran freilich nicht recht zu sehen.)} Weiterhin
meint Voegelin, wenn er von "`Ordnungserfahrung"' spricht, niemals den Komplex
täglicher Lebenserfahrungen, in einer geordneten gesellschaftlichen Umwelt zu
leben.  Dies würde Voegelins Konzeption geradezu auf den Kopf stellen, denn
für Voegelin resultiert die gesellschaftliche Ordnung aus der
Ordnungserfahrung und nicht umgekehrt.

Was bedeutet aber nun "`Erfahrung"', wenn es sich nicht um
Sinneserfahrungen handelt? Mit "`Erfahrung"' meint Voegelin
hauptsächlich ein bestimmtes inneres Erleben mystisch-religiöser Art.
Diese Erfahrung existiert in unterschiedlichen Varianten. So
unterscheidet Voegelin die Erfahrungen hinsichtlich ihres Niveaus nach
kompakten und differenzierten Erfahrungen. Auf dem kompakten
Erfahrungsniveau, welches vor allem für die älteren kosmologischen
Gesellschaften charakteristisch ist, ist die Erfahrung als inneres
Erlebnis noch gar nicht bewußt und unauflöslich mit dem allgemeinen
Daseinsgefühl verwoben. Der Gegensatz "`kompakt-differenziert"' wird bei
der Darstellung von Voegelins Geschichtsphilosophie noch genauer
erörtert werden. Es sei nur soviel vorweggenommen, daß Voegelin an eine
historische Entwicklungstendenz hin zu immer differenzierteren
Erfahrungen glaubte. Obwohl nämlich die Erfahrung als inneres Erlebnis
eine höchst individuelle, ja geradezu intim persönliche Angelegenheit
darstellt, ist sie dennoch durch ein erstaunliches Maß von
gesellschaftsinterner Einförmigkeit gekennzeichnet. Alle Individuen
einer Gesellschaft haben bei Voegelin offenbar die gleichen seelischen
Erlebnisse, solange bis ein Prophet oder Philosoph kommt und ihnen eine
neue Art des seelischen Empfindens nahe bringt.\footnote{Nicht eindeutig
  läßt sich übrigens die Frage klären, ob und wodurch sich bei Voegelin
  die Erfahrung des Propheten von den Erfahrungen seiner Anhänger
  unterscheidet: Gibt es (1.) keinen Unterschied oder besteht (2.) bloß
  ein Unterschied der Intensität oder liegt (3.) auch hier ein
  Unterschied der Differenziertheit vor? Für das Letztere spricht, daß
  es bei Voegelin gelegentlich den Anschein hat, als sei nur eine
  gesellschaftliche Elite starker Seelen der differenziertesten
  Erfahrung fähig. Vgl. Voegelin, Neue Wissenschaft der Politik,
  S.172-174. (Die Theorie, die Voegelin an dieser Stelle vertritt, ist
  übrigens in jeder Hinsicht unglaubwürdig: 1.Unsicherheit ist gewiß
  nicht das Wesen des Christentums. Jeder christliche Priester wird uns
  ganz im Gegenteil bestätigen, daß gerade der Glaube der unsicheren,
  zufälligen und schutzlosen Existenz des Menschen Halt und Sicherheit
  zu geben vermag. 2.Aus der Annahme, daß die gnostischen Turbulenzen
  des Mittelalters und der Neuzeit ursprünglich dadurch verursacht
  wurden, daß mit der Ausbreitung und Zunahme höherer Bildung im
  Spätmittelalter zunehmend auch Unberufene mit der vollen und, wie
  Voegelin glaubt, nur für ganz starke Seelen erträglichen Wahrheit des
  Christentum in Berührung kommen, folgt im Umkehrschluß, daß in den
  Jahrhunderten davor die gesellschaftliche Elite des Klerus
  zufälligerweise mit der natürlichen Elite der Seelenstarken
  zusammenfiel, was auch dann sehr unwahrscheinlich erscheint, wenn man
  Voegelins fragwürdige Prämissen bezüglich der spirituellen
  Überforderung durch das Christentum akzeptieren würde.)} Dennoch
zweifelte Voegelin nicht daran, daß die Erfahrung ursprünglich und
authentisch ist.

Wenn die Erfahrung ein inneres Erleben ist, so stellt sich die Frage,
was dort eigentlich erlebt wird. Was ist der Inhalt der Erfahrung? Dies
ist eine Frage, bei deren Beantwortung auch Voegelin vor großen
Schwierigkeiten stand.  Oberflächlich könnte der Eindruck entstehen, daß
je nach historischem Erfahrungsniveau unterschiedliche Dinge erfahren
werden: In der kosmischen Erfahrung wird der Kosmos erfahren, auf
differenzierterem Erfahrungsniveau dagegen wird die Transzendenz
erfahren. Aber Voegelin wollte es nicht bei einem unvermittelbaren
Gegensatz zwischen den verschiedenen Erfahrungstypen bewenden lassen. In
seinen Augen ist die Erfahrung der Transzendenz schon auf kompaktem
Niveau unbewußt mitgegenwärtig. Will man die übergreifenden
Gemeinsamkeiten der unterschiedlichen Erfahrungstypen herausstellen, so
läßt sich in etwa festhalten, daß nach Voegelins Vorstellung in jedem
Falle das Sein im Ganzen und die Stellung des Menschen im Sein erfahren
wird, nur daß auf kompaktem Erfahrungsniveau das Sein als sinnhaft
geordneter Kosmos erlebt wird, während es auf differenziertem Niveau als
Stufenfolge immanenter und transzendenter Seinsstufen erfaßt wird.
Entscheidend ist, daß in jedem Falle die ontologische Ordnung ein und
desselben Seins erfahren wird.\footnote{Vgl. Voegelin, Anamnesis,
  S.305.}

\subsubsection{Von der Ordnungserfahrung zur politischen Ordnung}

Die Beziehung zwischen Erfahrung und Ordnung stellt sich bei Voegelin in der
Weise dar, daß die spirituelle Erfahrung die Quelle der politischen Ordnung
ist. Man könnte etwas überspitzt sagen, daß bei Voegelin die Erfahrung die
Basis bildet, während die politischen Ideen und Institutionen den Überbau
verkörpern, wobei Voegelin allerdings nicht gänzlich leugnet, daß es auf der
Ebene pragmatischer Politik auch Probleme gibt, deren Lösung weitgehend
unabhängig von dem ist, was sich auf der spirituellen Basis-Ebene abspielt.
Ordnung existiert dabei auf insgesamt drei Ebenen: als Ordnung des Seins, als
Ordnung der Seele und als Ordnung der Gesellschaft.\footnote{Als viertes müßte
  noch die "`Ordnung der Geschichte"' hinzugefügt werden, die in der Regel
  allerdings keine Voraussetzung der politischen Ordnung einer Gesellschaft
  darstellt (außer m.E. im Falle sich selbst primär geschichtlich
  legitimierender Gesellschaften), sondern sich umgekehrt aus der Abfolge
  politischer Ordnungen in der Geschichte ergibt.} Die Reihenfolge dieser
Ordnungen ist nicht umkehrbar: Zunächst existiert die Ordnung des Seins. Diese
wird vom Menschen erfahren und verleiht ihm dadurch eine Ordnung der Seele,
welche sich wiederum auf die Ordnung der Gesellschaft auswirkt. Voegelin
scheint es für ausgeschlossen zu halten, daß es zu diesem Weg, eine Ordnung
der Seele und eine Ordnung der Gesellschaft zu erlangen, eine legitime
Alternative gibt.\footnote{Vgl. Voegelin, Anamnesis, S.349. - In der "`Neuen
  Wissenschaft der Politik"' geht Voegelin noch nicht ganz so weit. Seine
  überwiegend positive Bewertung der Trennung von Staat und Kirche sowie
  seine weitgehend befürwortende Besprechung von Hobbes legen nahe, daß sich
  Voegelin des möglichen Gegensatzes zwischen spiritueller Wahrheit und
  pragmatischer Notwendigkeit hier noch einigermaßen bewußt war, so daß als
  Ideallösung auch ein Kompromiß zwischen beiden angesehen werden könnte. Vgl.
  Voegelin, Neue Wissenschaft der Politik, S.211-223.} Es handelt sich dabei
um einen der vielen dogmatischen Grundsätze von Voegelins Theorie, die er
voraussetzt aber niemals begründet.

Die Ordnungserfahrungen unterschiedlicher Kulturen sind nun allerdings häufig
höchst gegensätzlich beschaffen und stehen oft in unvereinbarem Gegensatz
zueinander.\footnote{Vgl. Voegelin, Neue Wissenschaft der Politik, S.86-90.}
Voegelin hält es dennoch für möglich, politische Ordnungen nach ihrer
Wertigkeit zu unterscheiden.\footnote{Vgl.  Voegelin, Neue Wissenschaft der
  Politik, S.91.} Maßstab hierfür sind nicht irgendwelche moralischen Ideale,
etwa Humanität oder Gerechtigkeit, sondern die spirituelle Erfahrung selbst.
Hieraus ergibt sich die normative Zielsetzung. Um zur normativ richtigen
politischen Ordnung zu gelangen, ist es erforderlich, eine spirituelle
Empfindsamkeit auf höchstem Niveau zu kultivieren, was Voegelin in Anlehnung
an Henri Bergson als das "`Öffnen der Seele"' bezeichnet.\footnote{Vgl. Henri
  Bergson: Die beiden Quellen der Moral und der Religion, Olten 1980, S.33-36.
  Bei Voegelin wird das "`Öffnen der Seele"' im Gegensatz zu Bergson jedoch
  eher sensitiv als schöpferisch gedacht.} Dieses "`Öffnen der Seele"'
ermöglicht es, die Ordnung des Seins angemessen zu erfahren und dadurch einen
"`autoritativ"' gültigen Maßstab für die richtige politische Ordnung zu
gewinnen.\footnote{Vgl. Voegelin, Order and History II, S.6/7.}

\subsubsection{Probleme der Voegelinschen Konzeption politischer Ordnung}

An dieser Stelle steht die normative Konzeption Voegelins vor einem
schwierigen Problem: Wie kann gültig zwischen richtiger und falscher Erfahrung
von der Ordnung des Seins unterschieden werden? In der "`Neuen Wissenschaft
der Politik"' gibt Voegelin auf diese Frage keine zufriedenstellende Antwort.
Der Hinweis auf das unterschiedliche, kompakte oder differenzierte Niveau von
Erfahrungen hilft kaum weiter, da gerade das Niveau aus der jeweiligen Sicht
der unterschiedlichen Erfahrungen gegensätzlich beurteilt werden
dürfte.\footnote{Das Problem der Relativität der Wertungen wird von Voegelin
  zwar öfters angesprochen und manchmal auch eine Weile verfolgt (z.B.
  anläßlich der Interpretation von Aristoteles in: Eric Voegelin: Order and
  History. Volume Three. Plato and Aristotle, Baton Rouge / London 1986
  (zuerst: 1957), im folgenden zitiert als: Voegelin, Order and History III,
  S.299-302.), aber niemals glaubhaft gelöst.}  Zweifelhaft ist auch Voegelins
Voraussetzung, daß es eine Ordnung des Seins gibt, die jenseits
naturgesetzlicher Bestimmtheit einen sinnhaften Zusammenhang der Dinge
herstellt. Selbst wenn eine solche Ordnung des Seins objektiv vorhanden wäre,
so ist noch längst nicht geklärt, ob diese Ordnung des Seins auch dazu
geeignet ist, die Grundlage der politischen Ordnung einer Gesellschaft
abzugeben. Woher können wir die Sicherheit nehmen, daß die Normen, die aus der
Ordnung des Seins abgeleitet sind, moralisch akzeptabel und mit den
praktischen Erfordernissen der Politik verträglich sind? Ohne eine schlüssige
Antwort auf diese Frage dürfte Voegelins normatives Programm, welches die
Revitalisierung eines spirituell-religiös eingebundenen Politikverständnisses
in der heutigen Zeit anstrebt, kaum bei der Gestaltung der gesellschaftlichen
und politischen Praxis hilfreich sein.

Aber nicht nur die normative Seite von Voegelins politikwissenschaftlichem
Ansatz bereitet Schwierigkeiten. Auch Voegelins analytische Konzeption ruht
auf einer Reihe von sehr anspruchsvollen und längst nicht restlos geklärten
Voraussetzungen. Schon die Annahme, daß alle politischen Ordnungen auf einer
metaphysischen Erfahrung der Ordnung des Seins beruhen müssen, fordert
Widerspruch heraus, denn offensichtlich kommen die liberalen Demokratien ohne
eine derartige Grundlage aus. Voegelin versucht diese Tatsache zu leugnen,
indem er entweder die Situation des Liberalismus als höchst gefährdet und
prekär darstellt, weil ihm eine solche Grundlage fehlt,\footnote{Vgl. Eric
  Voegelin: Der Liberalismus und seine Geschichte, in: Karl Forster (Hrsg.):
  Christentum und Liberalismus, München 1960, S.13-42 (S.35-42).} oder den
liberalen Demokratien unterstellt, in verschleierter Form (als "`common
sense"') doch ein solches metaphysisches Ordnungswissen zu
konservieren.\footnote{Vgl. Voegelin, Anamnesis, S.352-354. Vgl. Voegelin,
  Neue Wissenschaft der Politik, S.259.}  Hier zeigt sich übrigens eine
erkenntnistheoretische Gefahr des normativ-ontologischen Ansatzes. Wird
nämlich ein und dieselbe Theorie zur Sacherklärung wie zur normativen Kritik
verwandt, so liegt es nahe, diejenigen Fälle, die der Theorie widersprechen
könnten, statt als falsifizierende Gegenbeispiele in Betracht zu ziehen, als
illegitime Fälle der normativen Kritik zu unterwerfen.\footnote{Dies wird auch
  besonders an Voegelins Behandlung der Philosophiegeschichte deutlich.
  Philosophien, die Voegelin nicht als Artikulation von Seinserfahrungen
  deuten kann, verwirft er in der Regel als törichte oder gefährliche
  Abirrungen. Dabei hätte Voegelin sich eigentlich eingestehen müssen, daß es
  in der Philosophie nur in Einzelfällen um die Artikulation von religiösen
  Erfahrungen geht.}

Darüber hinaus wirft Voegelins Erfahrungsbegriff als methodisches
Analysewerkzeug einige Probleme auf. Die Aufgabe des Verstehens politischer
Ordnung besteht für Voegelin darin, in die Erfahrungen der entsprechenden
Gesellschaft oder, wenn es sich um die Untersuchung einer politischen Theorie
handelt, in die Erfahrung des entsprechenden Theoretikers einzudringen.
Voegelin unterscheidet dieses Eindringen in die motivierenden Erfahrungen klar
von der bloßen Rekonstruktion einer Theorie auf der Ebene der politischen
Ideen.\footnote{Vgl.  Voegelin, Neue Wissenschaft der Politik, S.115-117,
  S.176. - Auf Seite 176 schreibt Voegelin: "`Es wird .. nicht überflüssig
  sein, sich des Prinzips zu erinnern, daß die Substanz der Geschichte auf der
  Ebene der Erlebnisse, nicht auf der Ebene der Ideen zu finden ist"'.} Die
Ideen sind nur die Oberfläche, während die Erfahrungen das seelische Innere
repräsentieren, auf das es eigentlich ankommt. Um also beispielsweise die
politische Theorie Platons zu verstehen, genügt es nicht zu untersuchen,
welche Institutionen und Gesetze Platon vorschlägt. Vielmehr ist danach zu
fragen, welche motivierenden inneren Erlebnisse Platons Denken zugrunde
liegen. Aber wie kann man den seelischen Erfahrungshintergrund einer Theorie
sicher rekonstruieren? Das Verfahren, welches Voegelin zur Ergründung der
Erfahrungen anwendet, scheint eines der Innervation mit anschließender
Selbstauslegung zu sein. Naturgemäß sind daher die Ergebnisse, zu denen
Voegelin gelangt, stark subjektiv gefärbt. So findet Voegelin beispielsweise
bei Thomas von Aquin die tiefste und reinste Transzendenzerfahrung, während er
der Reformation eine echte Erfahrungsgrundlage offenbar nicht in gleichem Maße
zubilligen will.\footnote{Vgl. Voegelin, Neue Wissenschaft der Politik,
  S.188.} Oft läßt sich bereits die Auswahl der von Voegelin als relevant
eingestuften und zur Deutung herangezogenen Quellen nicht ohne weiteres
nachvollziehen. Indessen muß eingeräumt werden, daß die Beweggründe des
Handelns und Denkens von Menschen in der Tat häufig in inneren seelischen
Regungen bestehen, die als solche niemals nach außen dringen, so daß man in
derartigen Fällen entweder auf jede Chance des Verstehens ganz verzichten oder
einen Versuch auf dem unsicheren Wege psychologischer Einfühlung wagen muß.
Nur bleibt es dann immer noch grundsätzlich fragwürdig, ob für das Verständnis
einer Theorie das Verständnis des Theoretikers und seiner Motive überhaupt
eine notwendige Voraussetzung bildet. Die Wahrheit oder Falschheit einer
Theorie entscheidet sich schließlich nicht an den Motiven des Erfinders der
Theorie.
% \footnote{Für
%   Voegelin gelten als vollwertige politische Theorien von vornherein nur
%   diejenigen Theorien, die als Ausdruck von Transzendenzerlebnissen verstanden
%   werden können. Unter dieser Vorgabe wären die seelischen Hintergründe in der
%   Tat relevant.}

Im Ganzen beruht Voegelins politikwissenschaftliches Paradigma mit seiner
starken Betonung der religiös-mystischen Erfahrung in hohem Maße auf
bewußtseinsphilosophischen oder sogar theologischen Voraussetzungen. Ob es
Voegelin gelingt, diese Voraussetzungen hinreichend plausibel zu begründen,
wird daher anhand seiner Bewußtseinsphilosophie zu prüfen sein. 

\section{Voegelins Geschichtsdeutung}

Eine sehr große Bedeutung mißt Voegelin der Geschichte und der
Geschichtlichkeit der menschlichen Existenz bei. Voegelin folgt damit nicht
nur einer Mode seiner Zeit, sondern die Untersuchung der historischen
Entwicklung der politischen Ordnungsvorstellungen bildet neben der
Erfahrungsanalyse auch einen wichtigen Bestandteil von Voegelins Methode
des Verständnisses politischer Ordnungsvorstellungen. Eine politische
Ordnungsvorstellung verstehen heißt bei Voegelin, ihre motivierenden
Erfahrungen aufzudecken und sie auf ihre historische Urform zurückzuführen. Um
ein vollständiges Bild von Voegelins politischem Denken zu geben, ist es daher
unerläßlich, auch auf seine Geschichtsdeutung einzugehen. Dabei soll die
Darstellung von Voegelins Geschichtsphilosophie, d.h. seiner Grundvorstellung
vom Ablauf und der Bedeutung der Geschichte im Vordergrund stehen.

Voegelin hat seine Geschichtsdeutung neben der "`History of Political
Ideas"' vor allem in den fünf Bänden von "`Order and History"', seinem
eigentlichen wissenschaftlichen Hauptwerk, entfaltet.\footnote{Angaben
  im Literaturverzeichnis. Die "`History of Political Ideas"' ist
  bisher nur einzeln und in Auszügen veröffentlicht.} Die
philosophischen Grundlagen dieser Geschichtsdeutung legt Voegelin
dabei insbesondere in den Einleitungen zu den Einzelbänden dar.
"`Order and History"' läßt sich am ehesten als eine Geschichte der
spirituellen Entwicklung der Menschheit in theologischer Absicht
charakterisieren. Um eine Geschichte der spirituellen Entwicklung
handelt es sich, weil Voegelin sich darin fast ausschließlich der
Deutung von religiösen und (stets spirituell interpretierten)
philosophischen Weltauf\/fassungen widmet, während der Zusammenhang
dieser Weltauf\/fassungen mit der gesellschaftlichen und
institutionellen Ordnung nur am Rande behandelt wird.  Von einer
theologischen Absicht bei diesem Unternehmen kann man deshalb
sprechen, weil Voegelin sich nicht darauf beschränkt, die
unterschiedlichen religiösen Vorstellungswelten darzustellen, sondern
das Ziel verfolgt aufzuzeigen, wie sich aus der geistigen Entwicklung
der Menschheit nach und nach so etwas wie spirituelle Wahrheit
herausschält. Diese Geschichtsdeutung wird von Voegelin durch eine zum
Teil an anderer Stelle dargelegte historische Metaphysik überwölbt,
nach welcher die Geschichte als ein theogonischer Prozeß aufzufassen
ist, in dessen Verlauf sich ein transzendentes ewiges Sein in der Zeit
verwirklicht, indem es über das Medium des menschlichen Bewußtseins in
die Immanenz eindringt.\footnote{Zu Voegelins Geschichtsphilosophie:
  Vgl.  Eugene Webb: Eric Voegelin. Philosopher of History, Seattle
  and London 1981.  - Vgl.  Jürgen Gebhardt: Toward the process of
  universal mankind: The formation of Voegelin's philosophy of
  history, in: Ellis Sandoz (Hrsg.): Eric Voegelins Thought. A
  critical appraisal, Durham N.C. 1982, S.67-86.}

\subsubsection{Geschichte als Geschichte der spirituellen Entwicklung der Menschheit}

Voegelins Geschichtsphilosophie ist im wesentlichen die einer Entwicklungs- und
Fortschrittsgeschichte. Anders als die Fortschrittsgeschichten beispielsweise
der Aufklärer ist Voegelins Fortschrittsgeschichte jedoch eine Geschichte des
spirituellen und nicht des moralischen oder technischen Fortschritts. Dieser
Geschichte des spirituellen Fortschritts liegt allerdings ein ahistorischer
Kern in Form einer religiös-existenzialistischen Metaphysik zu grunde, die
Voegelins Auf\/fassung von der existenziellen Situation des Menschen
widerspiegelt: Der Mensch findet sich in einer Welt wieder, deren Sinn er
nicht kennt. Er ist sich zwar dunkel bewußt, daß er in dieser Welt eine Rolle
zu spielen hat, die er nicht selbst bestimmen darf,\footnote{Voegelin ist
  deutlich von der religiösen Variante des Existentialismus beeinflußt, die
  der Freiheit ein transzendentes Ziel vorgibt - anders als der atheistische
  Existentialismus, der die Freiheit absolut setzt.} dennoch weiß er nicht,
was für eine Rolle dies ist. Anfänglich kann er diese Rolle nur schwach ahnen,
doch er glaubt, daß sie etwas mit der Ordnung des Seins zu tun
hat.\footnote{Vgl. Voegelin, Order and History I, S.1/2.} Das Trachten des
Menschen zielt nun darauf ab, diese Ordnung zu finden und seine eigene
Existenz sowie die Ordnung der Gesellschaft in Einklang mit ihr zu bringen,
weil er hofft, dadurch seiner flüchtigen Existenz etwas mehr Dauerhaftigkeit
zu verleihen. Die Suche nach der Ordnung des Seins bestimmt nun die
Entwicklung der Geschichte. Die Dynamik dieser Entwicklung entspringt den
wechselnden Auf\/fassungen davon, was die richtige Ordnung des Seins ist. Da der
Mensch die wahre Ordnung des Seins nur ahndungsvoll spüren kann, ist es ihm
nicht möglich, seiner Auf\/fassung von der richtigen Ordnung anders Ausdruck zu
verleihen als dadurch, daß er sein Gefühl der Ordnung bzw. seine
Ordnungserfahrung durch Analogien in Symbole faßt. Aus den solcherart
artikulierten Ordnungsauf\/fassungen bestehen die bereits erwähnten Symbolismen.
In der geschichtlichen Abfolge der Symbolismen glaubt Voegelin nun einen
Fortschritt erkennen zu können, der, wie Voegelin es nennt, von
"`kompakteren"' zu "`differenzierteren"' Symbolismen führt.\footnote{Die
  Einleitung von Order and History I legt die Auf\/fassung nahe, daß es stets
  dieselbe Seinserfahrung ist, die nur unterschiedlich vollkommen artikuliert
  wird. (Vgl. Voegelin, Order and History I, S.1-11.) Im Schlußkapitel von
  "`Anamnesis"' unterscheidet Voegelin dann unterschiedlich differenzierte
  Ordnungserfahrungen. Nur das Sein (bzw. die Realität) bleibt dasselbe, und
  sogar dies gilt nur unter Einschränkungen.  (Vgl. Voegelin, Anamnesis,
  S.286ff.)}

Der Fortgang von einem Symbolismus zum nächsten tritt oft plötzlich und
sprunghaft infolge neuer spiritueller Erlebnisse einzelner Personen ein, die
Voegelin als "`spirituelle Ausbrüche"' bezeichnet und die sich von den
Menschen, denen sie widerfahren, auf den Rest der Gesellschaft übertragen
(sofern dieser nicht gerade an einer törichten Verstocktheit leidet). Wenn der
Übergang sehr plötzlich eintritt und der Unterschied zwischen dem alten und
dem neuen Symbolismus besonders groß ist, dann spricht Voegelin von einem
"`Sprung im Sein"'.\footnote{Vgl. Voegelin, Order and History I, S. 123. Dort
  definiert Voegelin den Begriff "`Sprung im Sein"' als die "`Entdeckung des
  tranzendenten Seins als die Quelle der Ordnung im Menschen und der
  Gesellschaft"' (meine Übersetzung, E.A.). Nach dieser Definition dürfte es
  (wenigstens innerhalb der Geschichte einer Zivilisation) eigentlich nur
  einen einzigen "`Sprung im Sein"' geben. Allerdings gebraucht Voegelin den
  Ausdruck auch häufig als Synonym für "`spiritueller Ausbruch"'.} Ein solcher
"`Sprung im Sein"' fand zum Beispiel statt, als Moses das Volk Israel aus
Ägypten führte, denn dabei wurde - abgesehen davon, daß dieses Ereignis die
Geburtsstunde des Monotheismus war - der kosmische Symbolismus, welcher
typischerweise mit einer zyklischen Geschichtsauf\/fassung verbunden ist, durch
den völlig neuartigen historischen Symbolismus ersetzt.\footnote{Vgl.
  Voegelin, Order and History I, S.116ff.}  Später hat Voegelin diese
Auf\/fassung allerdings teilweise revidiert, nachdem er festgestellt hatte, daß
unabhängig von diesem Ereignis auch andere Völker auf die Idee gekommen waren,
die Geschichte nicht nur zyklisch zu betrachten.\footnote{Vgl.  Voegelin,
  Anamnesis, S.79ff. - Vgl. Eric Voegelin: Order and History. Volume Four. The
  Ecumenic Age, Baton Rouge / London 1986 (zuerst: 1974), im folgenden zitiert
  als: Voegelin, Order and History IV, S.7-13.} Weitere wichtige Übergänge
sind unter anderem die Entwicklung vom Judentum zum Christentum und der
Übergang von der Mythologie zur Philosophie im antiken Griechenland. Wenn
Voegelin in diesen Übergängen einen Fortschritt sieht, so stellt sich
natürlich die Frage, was die nachfolgenden Symbolismen gegenüber den
vorhergehenden als überlegen auszeichnet.  Warum ist die mosaische
Religion der ägyptischen überlegen, und was verleiht dem Christentum vor dem
Judentum den Vorzug, fortschrittlicher zu sein?  Voegelin versucht meist,
solche Wertungen mit der Behauptung der größeren Differenziertheit des seiner
Ansicht nach besseren Symbolismus zu begründen.  Da dem Konzept der
Differenzierung für Voegelins Vorstellung vom Fortschritt (oder auch
gelegentlichem Rückschritt) der Geschichte eine zentrale Bedeutung zukommt,
soll es etwas ausführlicher untersucht werden.

\subsubsection{Exkurs: Die Begriffe "`Kompaktheit"' und "`Differenzierung"'}

Das Begriffpaar "`kompakt-differenziert"' ist neben dem Begriff der
(spirituellen) Erfahrung ein weiteres großes Feigenblatt der Voegelinschen
Geschichtsphilosophie, denn mit diesem Begriffspaar kaschiert Voegelin eine
Reihe von Begründungsproblemen, Unklarheiten und fragwürdigen Voraussetzungen.
Der Gegensatz "`kompakt-differenziert"' kann zunächst auf einer rein formalen
Ebene verstanden werden. Aber es zeigt sich rasch, daß die formale Bedeutung
nicht ausreicht, um alle Funktionen dieses Gegensatzpaares zu rechtfertigen.

Auf der formalen Ebene bedeutet Differenzierung das Auseinandertreten von
zuvor wesensmäßig nicht unterschiedenem Sein in unterschiedliche Seinsklassen.
Insbesondere im Auseinandertreten von immanentem weltlichen und transzendentem
göttlichen Sein sieht Voegelin einen bedeutenden Differenzierungsfortschritt.
In etwas stärkerer Anlehnung an die Bewußtseinsphilosophie kann die
Essenz der formalen Bedeutungsebene dieses Begriffspaares in etwa
folgendermaßen wiedergegeben werden: Zunächst findet sich der Mensch vor einer
verwirrenden Vielfalt von Bewußtseinserlebnissen wieder. Erst nach und nach
und unter großen Unsicherheiten lernt der Mensch das Innere vom Äußeren, das
Transzendente vom Immanenten und die immanenten Dinge voneinander zu
unterscheiden.\footnote{Vgl. Order and History I, S.3.} Soweit beschreibt der
Begriff der Differenziertheit lediglich gewisse phänomenale Eigenschaften von
Weltauf\/fassungen.

Unzulänglich bleibt der rein formale Differenzierungsbegriff, weil es auf der
phänomenalen Ebene oft der Willkür überlassen bleibt, was als kompakt und was
als differenziert bezeichnet wird. So vertritt Voegelin beispielsweise die
Ansicht, daß der Monotheismus wegen der deutlicheren Erkenntnis des
welttranszendenten Charakters des Göttlichen differenzierter ist als der
Polytheismus.\footnote{Vgl. Eric Voegelin: Die geistige und politische Zukunft
  der westlichen Welt (Hrsg. von Peter J. Opitz und Dietmar Herz), München
  1996, S.25.} Aber ebensogut könnte man behaupten, daß der Polytheismus
differenzierter ist, weil im Polytheismus die vielfältigen Funktionen des
undifferenziert Göttlichen des Monotheismus deutlich auf eine Vielzahl von
Göttern verteilt sind. Das Begriffspaar "`kompakt-differenziert"' drückt auf
dieser Bedeutungsebene ähnlich wie viele andere abstrakte Begriffspaare (z.B.
"`formal-material"') lediglich einen Unterschied aus, ohne diesen zu
qualifizieren.

Damit die Unterscheidung zwischen kompakten und differenzierten Symbolismen zu
wertenden Vergleichen, wie Voegelin sie anstellt, herangezogen werden kann,
müssen noch weitere Voraussetzungen erfüllt sein: Die verglichenen Symbolismen
müssen sich auf denselben Gegenstand beziehen, damit ein Vergleich
stattfinden kann, und es muß ein gültiger Bewertungsmaßstab vorhanden sein, um
die Beurteilung der Symbolismen durchzuführen.

Die Voraussetzung, daß sich die verglichenen Symbolismen auf den gleichen
Gegenstand beziehen müssen, ist nicht schon dann erfüllt, wenn beide
Symbolismen eine Antwort auf die selbe Herausforderung geben, etwa auf die
Frage nach dem Sinn der Welt oder dem Wesen Gottes. An einem Beispiel läßt
sich dies verdeutlichen: Voegelin ist der Ansicht, daß die christliche
Gnadenlehre gegenüber dem bei Platon und Aristoteles vorherrschenden
Gottesverständnis eine Differenzierung darstellt, da bei den griechischen
Philosophen Gott bloß Ziel menschlicher Sehnsucht ist, während nach
christlichem Verständnis Gott dieser Sehnsucht durch die Gnade auch
entgegenkommt.\footnote{Vgl. Voegelin, Neue Wissenschaft der Politik,
  S.113-114.} Gegen dieses Argument liegt freilich der Einwand nahe, daß es
sich hier um unterschiedliche Gottesvorstellungen handelt, und daß nach der
griechischen Vorstellung Gott nun einmal nicht die Eigenschaft der Gnade
besitzt. Das christliche Verständnis scheint also eher eine Modifikation als
eine Differenzierung darzustellen. Der Gebrauch des Ausdruckes
"`Differenzierung"' kann genaugenommen nur dann als voll gerechtfertigt
betrachtet werden, wenn der differenziertere Symbolismus nichts anderes
ausdrückt als das, was im kompakteren Symbolismus bereits gemeint aber noch
unvollkommen ausgedrückt ist. Es braucht wohl kaum dargelegt werden, daß dies
im Einzelfall äußerst schwierig nachzuweisen sein dürfte, sofern man nicht
dogmatisch unterstellt, daß ohnehin alle Symbolismen nur denselben
vorgegebenen Bestand von Erfahrungen ausdrücken, die genau zu kennen sich der
Interpret zudem anmaßen muß.\footnote{Vgl. auch Eugene Webb: Philosophers of
  Consciousness. Polanyi, Lonergan, Voegelin, Ricoeur, Girard, Kierkegaard,
  Seatlle and London 1988, S.126ff.}

Nicht weniger dunkel bleibt, woher Voegelin die Bewertungsmaßstäbe nimmt, nach
denen er die verschiedenen Symbolismen beurteilt. Selbst wenn man bei dem eben
angeführten Beispiel einmal annimmt, daß sich beide Gottesvorstellungen
nachweisbar auf dieselbe religiöse Erfahrung stützen, so fehlt immer noch
jeder Anhaltspunkt, aus dem heraus die christliche Gnadenlehre im
inhaltlich-wertendenden Sinne als differenzierterer Ausdruck der zugrunde
liegenden religiösen Erfahrung beurteilt werden kann als die Gottesvorstellung
der griechischen Philosophen. Wie auch bei anderen methodischen Problemen hat
es den Anschein, daß Voegelin glaubt, diese Frage im Einzelfall ad-hoc, durch
genaues Hinschauen und ein wenig Genialität in evidenter Weise beantworten zu
können.

Ob das Problem, den Begriff der Differenzierung zu rechtfertigen, gelöst
werden kann, wird noch anhand von Voegelins Bewußtseinsphilosophie zu
untersuchen sein, da es sich schließlich um die Differenzierung von
Bewußtseinserfahrungen handelt.

\subsubsection{Der Sinn der Geschichte}

Wenn Voegelin Religionen, Philosophien und auch die konkreten politischen
Ordnungen als Ausdruck einer Suche nach der Ordnung des Seins deutet, so ist
dies nicht bloß das heuristische Prinzip eines besonders einfühlsamen
Geistesgeschichtlers. Voegelin ist vielmehr selbst fest davon überzeugt, daß
es eine objektive, sinngebende und wertvermittelnde höhere Ordnung des Seins
gibt. Die Suche nach dieser Ordnung betrachtet Voegelin als das historische
Projekt der Menschheit. Durch dieses menschheitliche Projekt der Suche nach
Ordnung wird für Voegelin zu allererst die Einheit der Menschheit und der Sinn
der Geschichte hergestellt.\footnote{Vgl. Eric Voegelin: Order and History.
  Volume Two. The World of the Polis, Baton Rouge / London 1986 (zuerst:
  1957), im folgenden zitiert als: Voegelin, Order and History II, S.1-7. Auch
  wenn Voegelin leugnet, daß es einen erkennbaren Sinn der Geschichte geben
  kann, so scheint Voegelins Geschichtsphilosophie dennoch wenigstens so etwas
  wie den vorläufigen Sinn der Geschichte beschreiben zu wollen. Anders als
  auf den Sinn der Geschichte bezogen lassen sich Äußerungen wie die, daß die
  menschliche Existenz in Gesellschaft eine Geschichte habe, weil sie eine
  Dimension der Spiritualität habe (Vgl. ebd., S.2), kaum verstehen. Denn
  Geschichte im Sinne einer Abfolge wechselnder Gesellschaftszustände gäbe es
  ja auch ohne die Spiritualität, wie auch die Geschichtswissenschaft
  üblicherweise die Dimension der Spiritualität der menschlichen Existenz
  weitgehend beiseite läßt.}  Voegelin leugnet allerdings entschieden, daß dem
Menschen das Ziel der Geschichte bekannt werden kann und daß ihr Ausgang
vorhersagbar wäre.\footnote{Implizit gibt es jedoch auch in Voegelins
  Geschichtsphilosophie ein Ende der Geschichte, denn über das optimal
  differenzierte Ordnungswissen hinaus ist keine weitere Steigerung von
  Ordnungswissen mehr denkbar (und alle anderen geschichtlichen Entwicklungen
  sind politische Profangeschichte, für die Voegelin sich nicht interessiert).
  Nur fällt bei Voegelin das logische (wenn auch nicht zeitliche) Ende der
  Geschichte nicht wie bei Hegel auf das frühe 19.Jahrhundert, sondern auf das
  christliche Mittelalter.} Hierin setzt sich Voegelin in ausdrücklichen
Gegensatz zu Geschichtsphilosophien der Hegelschen Machart, durch die er
ansonsten durchaus beeinflußt ist. Weiterhin vertritt Voegelin einen
konsequenten individualistischen Vorbehalt, was die Verkörperung des Geistes
in der Geschichte angeht. Geist verkörpert sich bei Voegelin in der Geschichte
niemals durch kollektive Gebilde wie den Staat oder die Nation. Vielmehr
dringt der Geist ausschließlich über das konkrete Bewußtsein des menschlichen
Individuums in die Geschichte ein.\footnote{Vgl. Voegelins gegen Hegel
  gerichtete Bemerkungen, in: Eric Voegelin: "`Structures of Consciousness"',
  in: Voegelin-Research News Volume II, No 3, September 1996, auf:
  http://vax2.concordia.ca/\~{ }vorenews/v-rnII3.html (Host: Eric Voegelin
  Institute, Lousiana State University), im folgenden zitiert als: Voegelin,
  Structures of Consciousness, Abschnitt I (1).} Partikularismen gegenüber,
seien sie nationaler oder anderer Art, ist Voegelin eher abgeneigt.  Ohne die
Realität partikulärer Gebilde zu leugnen, bleibt für Voegelin zumindest vor
der Geschichte die höchste Gemeinschaft stets die ganze
Menschheit.\footnote{Vgl. Voegelin, Order and History II, S.1-20.} In diesem
Bezug auf die Menschheit, und zwar nicht nur auf die gegenwärtige Menschheit,
sondern auf die Menschheit in ihrer gesamten Geschichte, kommt ein moralisches
Anliegen Voegelins zum Ausdruck, welches auch seine Kritik an der
aufklärerischen Fortschrittsgeschichte motiviert. Die Fortschrittsgeschichte
entwertet in Voegelins Augen die vergangene Menschheit, indem sie in der
Vergangenheit nur die Vorstufe und das Mittel zum Zweck der Gegenwart
erblickt. Sie vergißt dabei, daß die vergangenen Menschen auch einmal eine
Gegenwart hatten, die sie genauso durchleben mußten, wie die heute lebenden
Menschen ihre Gegenwart bewältigen müssen.\footnote{Vgl. Voegelin, Order and
  History II, S.3.}  Freilich stellt sich die Frage, wie Voegelin nun
seinerseits derartige Entwertungen vermeiden will, und an Voegelins
Gnosiskritik wird deutlich, daß auch Voegelin ganze historische Epochen
verdammen konnte.

Mit dem vierten Band von "`Order and History"' tritt ein Bruch in Voegelins
historischem Programm ein. Dieser Bruch resultiert zu einem Teil aus der
Feststellung, daß die Geschichte nicht linear, sondern in vielfältigen
Verzweigungen, Verästelungen und unabhängig nebeneinander herlaufenden
Entwicklungssträngen verläuft.\footnote{Vgl. Voegelin, Order and History IV,
  S.1-6.} Dieses Faktum war Voegelin schon zuvor bewußt, wenn er auch dessen
Ausmaß unterschätzte, und er es daher in der Konzeption von "`Order and
History"' zunächst eher vernachlässigt hat. Zum anderen Teil kommt der Bruch
durch die Entdeckung zustande, daß die Idee einer fortschreitenden, nicht
zyklischen geschichtlichen Entwicklung keineswegs einzigartig mit dem "`Sprung
im Sein"' zur Zeit von Moses verbunden ist, sondern sehr häufig bereits im
Rahmen kosmischer Symbolismen auftritt.  Voegelin geht nun noch stärker als
zuvor davon aus, daß es in der Praxis zu einer Verschränkung kosmischer und
nach-kosmischer Symbolismen und weniger zu einer deutlichen Ablösung des einen
durch den anderen kommt.\footnote{Vgl.  Order and History IV, S.7-12.} Seine
Grundvorstellung von der Geschichte als Prozeß der zunehmenden Differenzierung
spiritueller Erfahrungen behält Voegelin jedoch bei. In dieser Hinsicht bleibt
der Bruch weniger dramatisch, als es die Einleitung von "`Order and History
IV"' zunächst vermuten läßt.

Voegelins Geschichtsphilosophie ist eingebettet in eine kosmische
Geschichtsmetaphysik, der zufolge die Geschichte die Verwirklichung eines
ewigen Seins in der Zeit darstellt.\footnote{Vgl. Voegelin, Anamnesis,
  S.254ff.} Dieser Prozeß, den Voegelin gelegentlich bis in die
Naturgeschichte und die Evolution zurückverlängert,\footnote{Vgl. Voegelin,
  Structures of Consciousness, Abschnitt I (2).} verursacht und bestimmt auch
die Menschheitsgeschichte, indem das ewige Sein im menschlichen Bewußtsein als
anziehender transzendenter Pol wirkt. Da die Menschheitsgeschichte jedoch
nicht erkennbar auf einen Punkt zuläuft, sondern sich vielfältig verzweigt,
wäre es freilich plausibler, im ewigen Sein eine dem menschlichen Bewußtsein
entspringende Vorstellung und nicht ein in das Bewußtsein eindringendes
transzendentes Sein zu vermuten. Ohnehin verwundert es ein wenig, daß sich das
transzendente Sein zu seiner immanenten Verwirklichung als Ort gerade die Erde
- ein Staubkorn im All, wie die Astronomen versichern - ausgesucht hat, wo
doch das ganze Universum zur Verfügung gestanden hätte. (Oder hängt die
Verwirklichung der Transzendenz in der Immanenz von Wasser, Kohlenstoff und
günstigen Temperaturbedingungen ab?)  Voegelin wandelt mit seiner
Geschichtsmetaphysik ersichtlich auf den Spuren der Geschichtsphilosophien des
deutschen Idealismus, wonach die Geschichte ein Prozeß ist, in welchem der
Geist zu sich selbst kommt. Bei Voegelin ist es nicht der Geist sondern die
Realität, die sich im Menschen selbst erhellt.\footnote{Voegelin begründet
  dies damit, daß der Mensch Teil der Realität ist, und daß folglich, wenn der
  Mensch die Realität erkennt, diese sich selbst zwar nicht geradewegs
  erkennt, aber doch erhellt. Vgl.  Voegelin, Structures in Consciousness,
  Abschnitt I (2)-(3).  Die Logik dieser Begründung ist ungefähr die folgende:
  Wenn ich durch meine Heimatstadt spazieren gehe und die Häuser anschaue,
  dann erblickt, da ich ja ein Teil meiner Heimatstadt bin, die Stadt sich
  selbst.  (Ein wahrhaft fürstliches Gefühl!)} Nicht anders als die
Philosophen des Deutschen Idealismus verliert sich Voegelin dabei in
metaphysische Spekulationen ohne Maß und Zügel.

So sehr Voegelin im übrigen auch die Historizität der menschlichen
Existenz betont, es bleibt hier eine schwer überbrückbare Spannung zu
den mehr ahistorischen Zügen seiner Philosophie bestehen. Sowohl
Voegelins existenzialistisches Menschenbild als auch seine
Seinsmetaphysik und seine Bewußtseinsphilosophie sind wesentlich
ahistorisch.\footnote{Vgl. zur unreflektierten Ahistorizität von
  Voegelins prozeßtheologischer Geschichtsdeutung die Diskussion über
  Voegelins Vortrag über "`Ewiges Sein in der Zeit"'. Dort
  insbesondere Baumgartners treffende Einwürfe, in: Helmut Kuhn /
  Franz Wiedmann (Hrsg.): Die Philosophie und die Frage nach dem
  Fortschritt, München 1964, S.340.} Hinzu kommt Voegelins Neigung,
sich gelegentlich recht unbekümmert über die Jahrhunderte hinweg mit
Philosophen und Propheten auseinanderzusetzen. Nicht selten trägt er
dabei in anachronistischer Weise Konzepte in die Interpretation der
Klassiker hinein, die der Philosophie des 19. und 20.  Jahrhunderts
entnommen sind.\footnote{Vgl. Zdravko Planinc: The Uses of Plato in
  Voegelin's Philosophy of Con\-s\-cious\-ness: Reflections prompted by
  Voegelin's Lecture, "`Structures of Con\-s\-cious\-ness"', in:
  Voegelin-Re\-search News Volume II, No 3, September 1996, auf:
  http:\-//vax2.concordia.ca/\~{ }vorenews/v-rnII3.html (Host: Eric
  Voegelin Institute, Lousiana State University).} Wenn dies nicht
immer sogleich auf\/fällt, so hängt dies auch damit zusammen, daß
Voegelin ein wenig zögerlich war, die moderne Herkunft seiner Ideen
auch stets anzuerkennen. Wird versucht, die Beziehung zwischen den
historischen und ahistorischen Zügen von Voegelins Philosophie näher
zu bestimmten, so läßt sich feststellen, daß, ebenso wie in Bezug auf
politische Ordnung, auch hinsichtlich der Geschichte die
Bewußtseinsphilosophie und die Seinsmetaphysik die theoretische
Grundlage bilden, auf der die geschichtlichen Prozesse von Voegelin
gedeutet werden.

\section{Gnosisbegriff und Zeitkritik}

Da sich für Voegelin gute politische Ordnung auf ein richtiges Verständnis der
höheren Ordnung des Seins und auf eine wohlausgebildete Ordnung der Seele
gründet, so liegt es für ihn natürlich nahe, den Ursprung politischer
Unordnung in spiritueller Desorientierung zu suchen. Der Gnosisbegriff, mit
dem Voegelin lange Zeit die Formen spiritueller Desorientierung beschrieb,
bildet zugleich das Hauptinstrument seiner politischen Gegenwartskritik, einer
Kritik, die sich auf die gesamte Neuzeit, insbesondere aber auf das 20.
Jahrhundert bezieht. Die Ursprünge des Gnosisbegriffes reichen zurück bis zu
Voegelins Schrift über die "`Politischen Religionen"' von 1938,\footnote{Eric
  Voegelin: Die politischen Religionen, München 1996 (zuerst 1938).} einer
entschiedenen Kampfschrift gegen den Nationalsozialismus. Ausgehend von dem
dort noch verwendeten Begriff der "`politischen Religion"' bildet Voegelin
später seine Theorie von der Neuzeit als einem Zeitalter der wiedererwachten
Gnosis, welche ihren prägnantesten Ausdruck in den totalitären
Herrschaftsformen des 20.Jahrhunderts gefunden hat. Voegelins Gnosistheorie
kann daher auch als seine Form der Auseinandersetzung mit dem Phänomen des
Totalitarismus verstanden werden, zumal sie merklich durch den
zeitgeschichtlichen Kontext ihrer Entstehung geprägt ist.

Unter Gnosis versteht Voegelin eine spirituelle Desorientierung im
Zusammenhang mit der Erfahrung der Transzendenz, die unter bestimmten
Bedingungen den Glauben mit sich führen kann, daß ein geschichtlicher
Endzustand von vollendeter Glückseligkeit innerhalb einer absehbaren Zeit die
gegenwärtige schlechte Welt ablösen kann.\footnote{Zur Definition des
  Begriffes bei Voegelin: Vgl. Dante Germino: Eric Voegelin on the Gnostic
  Roots of Violence, München 1998, im folgenden zitiert als: Germino, Voegelin
  on the Gnostic Roots of Violence, S.26. - Vgl. Voegelin, Neue Wissenschaft
  der Politik, S.169-171. - Vgl. Voegelin, Order and History IV, S.18-27. -
  Vgl. Voegelin, Wissenschaft, Politik und Gnosis, S.17-19.} Diese recht
allgemeine Bedeutung erlaubt es Voegelin, den Begriff der Gnosis von den
üblicherweise darunter gefaßten abweichenden Glaubensströmungen des Vor- und
Frühchristentums und des Mittelalters auf die chiliastischen politischen
Bewegungen der Neuzeit zu übertragen. Die Gnosis stellt in Voegelins Augen
insofern ein spirituelles Mißverständnis dar, als sie auf einer seiner Ansicht
nach falschen Vorstellung vom Wesen der Transzendenz beruht. Zwar liegt der
Gnosis dieselbe religiöse Erfahrungssubstanz zugrunde wie dem Christentum,
nämlich die differenzierende Erfahrung eines transzendenten göttlichen Seins,
aber in der Gnosis wird das transzendente Sein als so überwältigend erlebt,
daß die Immanenz in die Sinnlosigkeit absinkt und dem Welthaß verfällt. Die
Spannung zwischen Immanenz und Transzendenz, welche für Voegelin recht
eigentlich die Realität des menschlichen Existierens ausmacht, wird dadurch
aufgelöst zu Gunsten einer einseitigen, geradezu tagträumerischen Fixierung
auf das transzendente Ziel.\footnote{Vgl. Voegelin, Order and History IV,
  S.19/20.}

Ist das Gefühl für die Spannung zwischen Transzendenz und Immanenz aber erst
einmal verlorengegangen, so kann an der Stelle des transzendenten Zieles
leicht auch irgend ein immanenter Weltgehalt untergeschoben werden, der dann
alle Attribute des göttlichen Seins erbt und zum Gegenstand eines äußerst
unheiligen Götzenkultes erhoben wird. Dieser Fall von "`gnostischem
Immanentismus"' ist in Voegelins Augen höchst charakteristisch für die gesamte
Neuzeit.\footnote{Vgl. Voegelin, Neue Wissenschaft der Politik, S.175-180,
  S.229ff.} Voegelin trifft innerhalb dessen, was er als Gnosis bezeichnet,
noch allerlei Einzelunterscheidungen, die ihm aber letzten Endes nur dazu
dienen, recht wahllos alle politischen und geistigen Strömungen der Neuzeit,
denen in irgend einer Weise nachgesagt werden kann, daß sie ein Ideal
vertreten, unter dem Begriff des gnostischen Immanentismus zu versammeln. So
nennt Voegelin als gnostische Bewegungen Progressivismus, Liberalismus,
Humanismus, Marxismus, Kommunismus, Faschismus, Psychoanalyse und je nach
Bedarf noch einige mehr.\footnote{Vgl. Voegelin, Neue Wissenschaft der
  Politik, S.176. - Vgl. Germino, Voegelin on the Gnostic Roots of Violence,
  S.27.} Angesichts dieser breit gefächerten Auswahl neuzeitlicher gnostischer
Bewegungen verwundert es nicht, daß Voegelin im Gnostizismus das Wesen der
Moderne überhaupt erblickt.

Der gnostische Immanentismus führt, da dem Menschen das transzendente
Ordnungskorrektiv verlorengeht, in letzter Instanz zur Selbstvergottung des
Menschen. Neben Auguste Comte ist Friedrich Nietzsche Voegelins Hauptbeispiel
und zugleich sein wichtigster Gewährsmann für diese These.\footnote{Vgl.
  Voegelin, Neue Wissenschaft der Politik, S.182-184.} Voegelin läßt sich
nicht dadurch irritieren, daß es säkularistische Philosophien gibt, die nicht
die Konsequenz der Selbstvergottung ziehen. Im Zweifelsfall interpretiert
Voegelin solche scheinbar harmlosen Säkularismen als Schritte auf dem Wege,
der unvermeidlich zur Selbstvergottung des Menschen führt. So ist Voegelin
beispielsweise überzeugt, daß der Liberalismus mit innerer Logik zum
Kommunismus führt.\footnote{Vgl. Eric Voegelin: Der Liberalismus und seine
  Geschichte, in: Karl Forster (Hrsg.): Christentum und Liberalismus, München
  1960, S.28-31 (S.11-42).} Plausibel wird diese Auf\/fassung freilich nur, wenn
man das Dogma zugrunde legt, daß der Mensch nicht nicht-religiös sein kann,
und daß dementsprechend die Säkularisierung nicht zum Verschwinden religiöser
Absolutheitsansprüche, sondern bloß zu deren Fehlbesetzung führen kann.

Die Vollendung der menschlichen Selbstvergottung und Herrschsucht erreicht der
gnostische Immanentismus im politischen Bereich in den totalitären
Herrschaftsformen. Der Totalitarismus ist für Voegelin ein unmittelbares
Resultat der antichristlichen Unterdrückung der "`Wahrheit der Seele"' sowie
des Irrglaubens, daß in der Geschichte mit Hilfe politischer Aktion ein
Endzustand glückseliger Verklärtheit herbeigeführt werden kann. Voegelin hat
allerdings nie ernsthaft versucht, den Zusammenhang von religiösen Abirrungen
und gewalttätiger Politik auf der Ebene des konkreten Handelns detailliert
nachzuzeichnen. Seine Erklärung verbleibt auf einer rein geistigen Ebene und
ihre Glaubwürdigkeit hängt von einer empirisch nicht überprüften Einschätzung
der Bedeutsamkeit des religiösen Faktors im politischen Geschehen ab.

Auch aus anderen Gründen ist der Gnosisbegriff zur Beschreibung chiliastischer
politischer Bewegungen ungünstig gewählt: Der historischen Gnosis, deren
Gewaltpotential sich kaum mit dem ihrer rechtgläubigen Verfolger messen kann,
wird Voegelin am allerwenigsten gerecht. Es genügt eben nicht, den
vermeintlichen Welthaß oder irgendwelche spirituellen Mißverständnisse (aus
Sicht der eigenen Religiosität!) festzustellen, um daraus auf die latente
Gewalttätigkeit der Gnosis zu schließen.\footnote{Vgl. Eric Voegelin: Das Volk
  Gottes. Sektenbewegungen und der Geist der Moderne (Hrsg. von Peter
  J.Opitz), München 1994, S.74. Dort schreibt Voegelin über die häretischen
  Bewegungen des späten Mittelalters: "`Da die eschatologische Gewalt jenseits
  von Gut und Böse liegt, und da der Krieg für die Welt des Lichtes eine
  transzendentale geistige Operation ist, in der die Mächte der Finsternis aus
  dem Kosmos entfernt werden, werden sich die Gläubigen zwangsläufig [sic!]
  in einer Gründlichkeit der Vernichtung ergehen, die von der Warte der
  Realität aus als Bestialität und Grausamkeit erscheint."' - Viel zu sehr
  vereinfacht wird dies auch von Dante Germino. Vgl. Germino, Eric Voegelin on
  the Gnostic Roots of Violence, S. 28. Darüber hinaus läßt sich Germinos
  These, daß in expressiver Gewalt, also in einer Form von Hooliganism, das
  Wesen der totalitären Gewalt besteht, nicht leicht mit der administrativen
  und planvollen Form der Durchführung totalitärer Massenmorde vereinbaren.
  Bei bestimmten Tätergruppen - etwa Gestalten wie Eichmann oder jenen "`ganz
  normalen Männern"' (C.Browning), die als Befehlsempfänger die Morde
  durchführten - kann man ein expressives Moment ihrer Gewalttaten nur schwer
  auf\/finden.} Daß es gerade die Vergöttlichung eines Teilinhaltes der Welt
sein soll, die die exzessive Gewalttätigkeit nach sich zieht, ist nicht
besonders einleuchtend, da ja auch im Namen des transzendenten Gottes in Form
von Kreuzzügen, Progromen oder Hexenverfolgungen so mancher Exzess der Gewalt
stattfand, wobei sich das vorreformatorische Christentum - von Voegelin
romantisch als letzter Hort intakten Ordnungswissens verklärt - nicht wenig
hervorgetan hat. Die Gefahr dürfte wohl eher vom Absolutheitswahn, dem
Irrglauben, über eine absolute und für alle Menschen verpflichtende Wahrheit
zu verfügen, ausgehen als vom Säkularismus oder der Gnosis. Voegelins Konzept
der politischen Religion ist ebenso wie sein Begriff der Gnosis nicht nur
allzu undifferenziert, sondern er verfehlt darüber hinaus auch das
Wesentliche, indem er die Neigung zur Gewalttätigkeit nicht primär als eine
Frage der Form (fanatisch oder tolerant) sondern als eine Frage des Inhalts
des Glaubens auf\/faßt, womit er sich auf die Ebene religiös-konfessioneller
Polemik begibt.\footnote{Es verwundert daher auch nicht, daß Voegelin zum
  Verständnis der politischen Bewegungen der Neuzeit die Lektüre des Werkes
  "`Adversus Haereses"' des Kirchenvaters Iraeneus (2.Jh. nach Christus!)
  empfiehlt.  Vgl. Voegelin, Neue Wissenschaft der Politik, S.178. - Nicht
  ganz zu unrecht wird Voegelin von Albrecht Kiel als "`katholischer
  Fundamentalist"' gesehen. Vgl.  Albrecht Kiel: Gottesstaat und Pax
  Americana. Zur Politischen Theologie von Carl Schmitt und Eric Voegelin,
  Cuxhaven und Dartford 1998, S.3, S.95ff.}  Als Analysewerkzeug zum
Verständnis der chiliastischen politischen Bewegungen der Neuzeit entwertet
Voegelin seinen Gnosis-Begriff außerdem dadurch, daß er zuweilen auf ihn
zurückgreift, um einer wenig qualifizierten politischen Polemik Ausdruck zu
verleihen.\footnote{Vgl. beispielsweise Voegelin, Neue Wissenschaft der
  Politik, 6.Kapitel, S.224-259. - Als eine Form von politischem Moralismus
  weisen sich Voegelins Äußerungen an dieser Stelle dadurch aus, daß er den
  Sachproblemcharakter schwieriger politischer Entscheidungsdilemmata leugnet,
  indem er ihre Lösung zu einer jedem Einsichtigen völlig selbstverständlichen
  Banalität stilisiert (Ebda.  S.236-238), wodurch er im zweiten Schritt die
  Verfehlung jener vermeintlich eindeutig und offensichtlich richtigen
  Lösungen auf billige Weise moralischen Makeln der Entscheidungsträger
  anlasten kann.}

% Voegelin
%   geht auch bei der Untersuchung von fremden Theorien des öfteren so vor, daß
%   er die Theorie zunächst als völlig abwegig darstellt, um dann ihre Erfindung
%   für "`aufklärungsbedürftig"' zu erklären und ungesäumt zur klinischen
%   Diagnose des Falles überzugehen.}

Als Totalitarismustheorie gehört Voegelins Gnosistheorie insgesamt noch eher
einer Phase der (natürlich legitimen) emotionalen Auseinandersetzung und
inneren Abwehr des Phänomens an. Sie erscheint als eine Erklärung des
Totalitarismus, wie sie in den fünfziger Jahren nicht untypisch war: Der
Totalitarismus wird mit einer historisch weitausholenden Fundamentalerklärung,
als deren Haupterklärungsmoment ideologische Faktoren fungieren, erfaßt und
als die Folge des Abfalls von der Religion interpretiert. Besonders in dieser
Hinsicht ähnelt Voegelins Gnosiskonzept jenen gerade in der Nachkriegszeit
populären Säkularisierungstheorien, wie sie Hermann Lübbe eingehend untersucht
hat.\footnote{Vgl. Hermann Lübbe: Säkularisierung.  Geschichte eines
  ideenpolitischen Begriffs, München 1965, S.108ff. - Stärker philosophisch
  als zeitgeschichtlich orientiert: Hans Blumenberg: Die Legitimität der
  Neuzeit. Erneuerte Ausgabe, Frankfurt am Main 1996. (Vgl. S.138.)} Die
Erklärung Lübbes für das Auftreten der Säkularisierungstheorien läßt sich
allerdings nur in Teilen auf Voegelin übertragen, da Voegelin als Emigrant
gewiß keine apologetischen Absichten hatte. Für Voegelin stellte im Gegenteil
gerade die Ignoranz der Nachkriegsgesellschaft, die sich in Verdrängung,
Verklärung und darin äußerte, daß gestandene Nazi-Schufte als angesehene
Bürger gelten und in einzelnen Fällen sogar einflußreiche Posten besetzen
konnten, einen Hinweis darauf dar, daß die tieferen existenziellen Bedingungen
der politischen Katastrophe noch fortdauerten.\footnote{Vgl.  Voegelins
  Vorlesung über "`Hitler und die Deutschen"', S.1ff. (Typoskript im
  Eric-Voegelin-Archiv in München.) - Vgl. Eric Voegelin: Die deutsche
  Universität und die Ordnung der deutschen Gesellschaft, in: Die deutsche
  Universität im Dritten Reich.  Eine Vortragsreihe der Universität München,
  München 1966, S.241-282 (S.241ff.).}

Wenn Voegelin die verschiedensten geistigen Strömungen der Neuzeit mehr oder
weniger unterschiedslos als Gnosis identifiziert, so ist auch dies wohlmöglich
die Folge eines Methodenmißbrauchs. Voegelin verwendet bei der Untersuchung
der Geistesgeschichte häufig die Technik der Suche nach
Strukturverwandtschaften. Eine Strukturverwandtschaft scheint dabei nicht viel
mehr als eine sich in irgend einer Weise aufdrängende Analogie zwischen zwei
oder mehreren Theorien zu sein. So erkennt Voegelin zwischen dem in die Phasen
des Vaters, des Sohnes und des heiligen Geistes unterteilten Geschichtsbild
von Joachim Fiori und dem Dreistadiengesetz des Auguste Comte eine
Strukturverwandtschaft.  Zu diesen beiden Theorien steht wieder die
nationalsozialistische Ideologie vom Dritten Reich in der Beziehung einer
Strukturverwandtschaft.\footnote{Vgl. Voegelin, Neue Wissenschaft der Politik,
  S.157-162.} Die Feststellung von Strukturverwandtschaften kann, wie dieses
Beispiel zeigt, durchaus erhellend sein, aber sie erlaubt es allein noch
nicht, auf kausale Zusammenhänge oder auch nur auf eine historische
Traditionslinie zu schließen. Verfährt man mit den Strukturverwandtschaften
allzu großzügig, dann ist es ein Leichtes zu beweisen, daß ein öffentliches
Tennismatch seinem Wesen nach dasselbe ist wie ein Gladiatorenkampf im alten
Rom.\footnote{Ein extremes Beispiel einer solchen Simplifizierung liefert
  Voegelin in einem Brief an Alfred Schütz, wo er den Abwurf der Atombombe zu
  einem Ausfluß des Phänomenalismus (nach Voegelins Wortgebrauch die
  Auf\/fassung, daß nur den (natur-)wissenschaftlich erfassbaren Phänmenen
  substantielle Wirklichkeit zukommt) stilisiert.
  Vgl. Gilbert Weiss: Theorie, Relevanz und Wahrheit. Zum Briefwechsel
  zwischen Eric Voegelin und Alfred Schütz, München 1997, S.46-49. - Ähnlich
  willkürlich wie die Strukturverwandtschaften scheint das Kriterium der
  Äquivalenz von Erfahrungen zu sein. Vgl. dazu Eric Voegelin: Äquivalenz von
  Erfahrungen und Symbolen in der Geschichte, in: Eric Voegelin, Ordnung,
  Bewußtsein, Geschichte, Späte Schriften (Hrsg. von Peter J. Optiz),
  Stuttgart 1988, S.99-126 (S.105-106 / S.110).}

Gehört Voegelins Gnosistheorie auch ohne Zweifel zu den schwächeren Seiten
seiner Politikwissenschaft, so liegt in der grundsätzlichen Frage, ob der
Verlust der Spiritualität nicht auch die Gefahr eines Wertverfalls nach sich
zieht, indem mit der Spiritualität auch das innere Empfinden für den Sinn und
die Bedeutung des Lebens verlorengeht, ein Vorbehalt, wie er sich auch heute
noch manchem religiösen Menschen aufdrängen mag. Unter diesem Aspekt ist daher
die Frage an Voegelins Bewußtseinsphilosophie zu richten, ob sie diesen
Vorbehalt rechtfertigen kann.  Gelingt es Voegelin zu zeigen, daß die
Spiritualität im menschlichen Bewußtsein verankert ist, und daß ihre Leugnung
oder ihr Verlust zu existenzieller Unsicherheit und (gefährlichen)
kompensatorischen Gegenreaktionen führt?

% Etwa drei verschiedende Formen geistiger Abirrung sind es, die Voegelin unter
% den Begriff der Gnosis faßt. Zwei davon beruhen auf spirituellen
% Mißverständnissen, die dritte besteht in der Leugnung einer spirituellen
% Realität schlechthin.\footnote{Zuweilen greift Voegelin jedoch auch auf den
%   Gnosis-Begriff zurück, um einem wenig qualifizierten politischem
%   Moralismus Ausdruck zu verleihen. (Vgl.  beispielsweise Voegelin, Neue
%   Wissenschaft der Politik, 6.Kapitel, S.224-259. - Als politischer Moralismus
%   weisen sich Voegelins Äußerungen an dieser Stelle dadurch aus, daß er den
%   Sachproblemcharakter schwieriger politischer Entscheidungsdilemmata leugnet,
%   indem er ihre Lösung zu einer jedem Einsichtigen völlig selbstverständlichen
%   Banalität stilisiert (Ebda. S.236-238), wodurch er im zweiten Schritt die
%   Verfehlung jener vermeintlich eindeutig und offensichtlich richtigen
%   Lösungen auf billige Weise moralisch Makeln der Entscheidungsträger anlasten
%   kann.)}  Die auf spirituellen Mißverständnissen beruhenden Formen der Gnosis
% können grob in eine transzendente und eine immanente Gnosis unterteilt werden.
% Beiden liegt eine falsche Auffassung vom Wesen der Transzendenz zu Grunde.
% Überwältigt vom Eindruck des religiösen Erlebnisses der Transzendenz glaubt
% der Gnostiker, daß sich dieses als höchst beseligend erfahrene Sein in der
% Zeit verwirklichen lassen müsse. Während die historische Gnosis 

%  Es wird noch zu untersuchen sein, ob Voegelins
% Bewußtseinsphilosophie dieses Problem lösen kann.\footnote{Zwar spricht
%   Voegelin gelegentlich dieses Problem an, aber er vermeidet dann aber in der
%   Regel durch Weitschweifigkeit sowohl die Lösung des Problems als auch das
%   Eingeständnis, daß er das Problem nicht lösen kann. (Vgl. beispielsweise
%   Voegelin, Order and History III, S.299-303.) - }

% \section{Geistesgeschichtliche Einordnung Voegelins}

% \section{Die Stellung der Bewußtseinsphilosophie innerhalb von Voegelins
%   Philosophie}


%%% Local Variables: 
%%% mode: latex
%%% TeX-master: "Main"
%%% End: 































