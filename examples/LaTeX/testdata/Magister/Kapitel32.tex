
%%% Local Variables: 
%%% mode: latex
%%% TeX-master: "Main"
%%% End: 


\section{"`Zur Theorie des Bewußtseins"'}

\subsection{Voegelins Schrift "`Zur Theorie des Bewußtseins"'}

In seinem Aufsatz "`Zur Theorie des Bewußtseins"'\footnote{Eric Voegelin:
  "`Zur Theorie des Bewußtseins"', in: Voegelin, Anamnesis, S.37-60.} legt
sich Voegelin Rechenschaft über seinen eigenen philosophische Standpunkt ab.
Einleitend erklärt Voegelin, daß seine Aufzeichnungen die Ergebnisse
anamnetischer Experimente enthielten, doch bezieht sich dies wohl eher auf die
auf diesen Aufsatz folgenden Berichte von Kindheitserinnerungen, der Aufsatz
selbst zumindest besteht weit überwiegend aus theoretischer Diskussion.

Es fällt nicht leicht, die Thematik dieses Aufsatzes zu beschreiben,
denn Voegelin reißt darin viele verschiedene Themen an. Den
Hauptthemenschwerpunkt bildet eine Beschreibung der Struktur des
Bewußtseins, seiner Beziehung zur Welt und eine Diskussion der
Möglichkeiten des Bewußtseins dasjenige, was außerhalb des Bewußtseins
liegt, zu erfahren und zu beschreiben. Daneben skizziert Voegelin
umrißhaft eine ontologische Theorie und schließlich versucht Voegelin,
nachdem er der Ontologie das Primat vor der Bewußtseinsphilosophie
eingeräumt hat, das Auftreten der von ihm für falsch oder zu einseitig
gehaltenen Bewußtseinsphilosophien historisch und wissenssoziologisch zu
erklären.

Den Ausgangpunkt für Voegelins Überlegungen bildet eine Kritik der Theorien
des Bewußtseins, die die Bewußtseinsstrommetapher in den Mittelpunkt ihrer
Beschreibung stellen. Voegelin hält dies für eine falsche Akzentuierung: Zwar
strömt das Bewußtsein auch, aber das Strömen ist weder der wesentliche noch
ein alle anderen Bewußtseinsleistungen bedingender Faktor im Bewußtsein. Vor
allem tritt das Strömen nur bei bestimmten Bewußtseinsvorgängen zu Tage wie
etwa beim Hören von Tönen. Bei anderen Bewußtseinstätigkeiten - Voegelin
beschreibt als Beispiel die inneren Vorgänge beim Betrachten eines Gemäldes -
läßt sich das Strömen des Bewußtseins nur erfassen, wenn die Aufmerksamkeit
von der Hauptsache abgelenkt wird. Voegelin betrachtet es daher als eine
Spekulation, wenn der Bewußtseinsstrom als die Grundform aller
Bewußtseinsvorgänge aufgefaßt wird.\footnote{Vgl. Voegelin, Anamnesis,
  S.37-43.}

Nach Voegelins eigener Vorstellung vom Aufbau des Bewußtseins ist das
Bewußtsein nicht durch zeitliches Fließen sondern thematisch durch
Aufmerksamkeitszuwendung und -abwendung strukturiert. Voegelin nimmt an,
daß es im Bewußtsein ein Aufmerksamkeitsquantum von nicht genau
bestimmter aber beschränkter Größe gibt, welches verschiedenen Bereichen
des Bewußtseins in mehr oder weniger starker Konzentration zugewandt
werden kann. Im Bewußtsein gibt es nun zwei besonders ausgezeichnete
Bereiche, Voegelin nennt sie "`Erhellungsdimensionen"', denen bestimmte
Formen der Aufmerksamkeitszuwendung, "`Erinnerung"' und "`Projektion"',
entsprechen.  Diese beiden Erhellungsdimensionen bezeichnet Voegelin
daher passenderweise als "`Vergangenheit"' und "`Zukunft"'. Aus dem
Zusammenspiel von Aufmerksamkeitszuwendung und den Erhellungsdimensionen
"`Vergangenheit"' und "`Zukunft"' leitet sich die Vorstellung der
(inneren wie äußeren) Zeit ab.  Voegelin vertritt also, wie es scheint,
eine idealistische Auffassung der Zeit, die derjenigen nicht unähnlich
ist, die Augustinus als erster in den "`Bekenntnissen"' dargelegt
hat.\footnote{Aurelius Augustinus: Bekenntnisse, Stuttgart 1998,
  S.312-330 (Elftes Buch. XIII.15 - XXVIII.38). Man könnte geneigt sein,
  gegen die Logik dieser Art von Zeittheorien einzuwenden, daß die
  Vorgänge innerhalb des Bewußtseins, aus denen die Zeit hervorgeht,
  doch schon die Zeit als solche voraussetzen. Aber in der Tat setzen
  sie höchstens bestimmte (zeitliche) Relationen voraus.  (Voegelin
  scheint diese Kritik jedoch ernst zu nehmen, denn er bringt sie etwas
  später als Selbsteinwand vor; vgl. Voegelin, Anamnesis, S.54/55.) Die
  idealistischen Zeittheorien scheitern aus einem anderen Grund: Wenn
  die Zeit ausschließlich eine Form oder Leistung des Bewußtseins ist,
  so ist nicht erklärlich, wie die Kommunikation zwischen Bewußtseinen
  (alias Menschen) zeitlich aufeinander abgestimmt erfolgen kann, da die
  Botschaften des einen an das andere Bewußtsein doch durch eine äußere
  Welt hindurch müssen, in der die Zeitrelationen, die der
  idealistischen Annahme zufolge reine Bewußtseinsprodukte sind,
  verloren gehen müßten. (Das Argument stammt aus: Russel, History of
  Western Philosophy, S.689.)} Da das Bewußtsein insgesamt endlich ist,
so ist auch die Zeitvorstellung aus einem endlichen Vorgang abgeleitet.
Dieser Vorgang ist der einzige Prozeß, von dem wir eine innere Erfahrung
haben. Voegelin behauptet, daß dadurch der Bewußtseinsprozeß "`zum
Modell des Prozesses überhaupt"'\footnote{Voegelin, Anamnesis, S.44.}
wird, und daß all unsere Begriffe von Prozessen nur von diesem einzigen
uns zur Verfügung stehenden Modell abgezogen sind.\footnote{Später
  scheint Voegelin genau das Gegenteil zu behaupten: "`... die {\it
    Ordnung} des Augenblicksbildes in der Dimension, die durch die
  Erhellung geschaffen wird, zur Sukzession eines Prozesses erfordert
  Erfahrungen von bewußtseinstranszendenten Prozessen."' (Voegelin,
  Anamnesis, S.55.).} Infolge der Endlichkeit dieses Modells entstehen
Ausdruckskonflikte bei der Beschreibung unendlicher Prozesse, wie sie
außerhalb des Bewußtseins gelegentlich vorkommen können. Von diesen
Ausdruckskonflikten rühren nach Voegelins Überzeugung auch die
Kantischen Antinomien und die Paradoxe der Mengenlehre
her.\footnote{Vgl.  Voegelin, Anamnesis, S.44/45. - Welche Paradoxe der
  Mengenlehere Voegelin meint, geht aus dem Text leider nicht hervor.
  Wahrscheinlich denkt Voegelin dabei an die Russelsche Antinomie, die
  als Variante des Lügnerparadoxons in der naiven, d.h nicht
  axiomatischen, Mengenlehre auftritt, wenn man z.B. versucht die Menge
  aller Mengen, die sich nicht selbst als Element enthalten, zu bilden.
  (Vgl. auch die Kritik dieser Passage im folgenden Abschnitt.)}  Um
unendliche Prozesse, die zwar erfahren bzw. erahnt aber nicht
widerspruchsfrei beschrieben werden können, überhaupt in irgendeiner
Weise zu artikulieren, ist es erforderlich, sich der geheimnisvollen
Ausdrucksweise der Mythensymbolik zu bedienen. Den Begriff
"`Mythensymbol"' definiert Voegelin als ein "`finites Symbol, das für
einen transfiniten Prozeß `transparent' sein soll."'\footnote{Voegelin,
  Anamnesis, S.45.} Eine genauere Eingrenzung, welches die unendlichen
Prozesse sind, die des Ausdruckes durch die Mythensymbolik bedürfen,
gibt Voegelin nicht an.  Seine Beispiele legen nahe, daß es sich dabei
um die Gegenstände handelt, die die Religion zum welttranszendenten
Bereich rechnet. Als Beispiele dieses Gebrauchs der Mythensymbolik führt
Voegelin einige allegorische Deutungen bekannter Mythensymbole an. So
vermittelt etwa die unbefleckte Empfängnis "`die Erfahrung eines
transfiniten geistigen Anfangs"'\footnote{Voegelin, Anamnesis, S.45.}.
Voegelin versäumt es leider zu klären, wie ein transfiniter geistiger
Anfang Gegenstand der Erfahrung werden kann, und ob es jemals einen
Menschen gegeben hat, der etwas derartiges tatsächlich erfahren hat.

Ein größeres Problem im Zusammenhang mit der Mythensymbolik stellt die Frage
der Adäquatheit des Mythos dar. Wenn der Mythos die Erfahrung von etwas
Unendlichem ausdrückt, so kann er diese Erfahrung natürlich auch unangemessen
oder falsch ausdrücken. Voegelin erläutert dieses Problem am Beispiel zweier
von Platon erzählter Mythen. Hier fällt die Entscheidung leicht, da Platon
selbst mitgeteilt hat, welchen Mythos er für den "`richtigeren"' hält. Ein
allgemeines Kriterium für die Adäquatheit gibt Voegelin nicht
an.\footnote{Vgl. Voegelin, Anamnesis, S.46/47.}

Im Zusammenhang mit der Mythensymbolik kommt Voegelin auch auf das
Husserlsche Problem der "`Konstitution der Intersubjektivität"' zu
sprechen.\footnote{Vgl.  Husserl, Cartesianische Meditationen, S.91ff.
  (§42ff.).} Dahinter verbirgt sich die Frage, woraus hervorgeht, daß
die Menschen außerhalb des eigenen Bewußtseins eigene Wesen mit einem
eigenen Bewußtsein sind. Im philosophischen Gedankenexperiment ist es
möglich, sich vorzustellen, daß die anderen Menschen - so wie Gestalten
im Traume - nur Hirngespinste des eigenen Bewußtseins sind, oder daß sie
"`Zombies"' sind, die körperlich und von ihrem Verhalten her Menschen
gleichen, in deren Gehirnen jedoch kein Bewußtsein lebt. Edmund Husserl
hat zu zeigen versucht, daß das Ich bestimmte Objekte des
Erfahrungsfeldes als "`alter ego"' konstituieren kann.\footnote{Vgl.
  Husserl, Cartesianische Meditationen, S.112-116 (§ 50/51).} Voegelin
hält dies für ein reines Verwirrspiel. Seiner Ansicht nach existiert
dieses philosophische Problem gar nicht, sondern es ist ein Faktum, daß
das Bewußtsein die anderen Menschen als
"`Nebenbewußtsein"'\footnote{Voegelin, Anamnesis, S.47.} erfährt. Übrig
bleibt nur ein eher moralphilosophisches Problem, welches darin besteht,
dieses "`Erfahrungsfaktum"'\footnote{Voegelin, Anamneis, S.47.} so zum
Ausdruck zu bringen, daß die Mitmenschen als gleichartig anerkannt
werden können. Zur Behandlung dieses Problems greift Voegelin auf die
Mythengeschichte zurück. Es ist nicht ganz klar, warum Voegelin es für
notwendig erachtet, die Lösung im Rahmen der Mythensymbolik zu suchen.
Bei den anderen Menschen handelt es sich schließlich auch nur um
endliche Wesen, weshalb der zuvor beschriebene Ausdruckskonflikt nicht
zum Tragen kommt. Zudem sind die anderen Menschen nicht
welttranszendent, sondern nur in dem trivialen Sinne
bewußtseinstranszendent, in dem auch Tiere und tote Gegenstände
bewußtseinstranszendent sind. Vielleicht hofft Voegelin, daß die alten
Mythenweisheiten sich in dieser Frage als besonders überzeugend
erweisen. Die Unerläßlichkeit der Mythensymbolik beim Ausdruck
unendlicher oder welttranszendenter Vorgänge schließt ja noch nicht aus,
daß die Mythensymbolik auch für andere Zwecke brauchbar ist. Der
Mythengeschichte entnimmt Voegelin nun, daß alle bisherigen
Gleichheitsideen historisch auf die beiden Mythen der Abstammung aller
Menschen von einer Mutter oder der geistigen Prägung durch ein und
denselben Vater (Gottesebenbildlichkeit) zurückgeführt werden
können.\footnote{Vgl. Voegelin, Anamnesis, S.47/48.} In diesem
Zusammenhang stellt Voegelin die kühne Behauptung auf, daß die
erkenntnistheoretischen Probleme der Intersubjektivität nur innerhalb
dieses mythischen Rahmens behandelbar sind.\footnote{Vgl. Voegelin,
  Anamnesis, S.48.} Im folgenden geht Voegelin dann zu einem kleinen
Exkurs über einige mythengeschichtliche Einzelprobleme des Konfliktes
zwischen Gleichheits- und Gemeinschaftsmythen über.\footnote{Vgl.
  Voegelin, Anamnesis, S.48-50.} Diese Ausführungen sind für seine
philosophische Argumentation allerdings nicht mehr von wesentlicher
Bedeutung.

Um angemessen über Dinge und Zusammenhänge reden zu können, die mehr
sind als bloß Gegenstände endlicher, innerweltlicher Erfahrung,
existiert für Voegelin neben der Mythensymbolik noch eine
philosophisch-begriffliche Alternative in Form der Prozeßtheologie. Sie
beschreibt "`die Beziehungen zwischen dem Bewußtsein, den
bewußtseinstranszendenten innerweltlichen Seinsklassen und dem
welttranszendenten Seinsgrund"'\footnote{Voegelin, Anamnesis, S.50.}.
Dieser Aufgabe ist die Prozeßtheologie im Gegensatz zu anderen Ansätzen
innerhalb der Metaphysik deshalb gewachsen, "`weil in ihr zumindest der
Versuch gemacht wird, die bewußtseinstranszendente Weltordnung in einer
`verstehbaren' Sprache zu interpretieren"', nämlich in einer Sprache,
die an "`der einzig `von innen' zugänglichen Erfahrung des
Bewußtseinsprozesses"'\footnote{Voegelin, Anamnesis, S.51.} orientiert
ist. Wie dies mit der vorherigen Behauptung zu vereinbaren ist, daß
gerade dieses Modell zum Ausdruck der Erfahrung transfiniter
Wirklichkeit eher untauglich sei, enthüllt Voegelin
nicht.\footnote{Gerade aufgrund dieser Ungeeignetheit entstanden
  schließlich die Ausdruckskonflikte (Vgl. Voegelin, Anamnesis,
  S.44/45.), welche für die Prozeßtheologie nun auf einmal nicht mehr
  zählen. - Es hilft hier nichts anzunehmen, daß die Prozeßtheologie
  sich vielleicht nur auf die Untersuchung der Beziehungen zwischen den
  Seinsklassen Bewußtsein, immanentes Sein und transzendentes Sein
  beschränkt, und daß diese Beziehungen vielleicht im Gegensatz zu den
  Prozessen innerhalb des transzendenten Seins noch endlich sind, denn
  dann müßte wenigstens vom Ansatz her auch jede andere Metaphysik
  gleichermaßen zur Beschreibung dieser Zusammenhänge geeignet sein.
  Voegelin beabsichtigt aber gerade, die Prozeßtheologie als die einzig
  mögliche und gültige Metaphysik hinzustellen.} Die Prozeßtheologie
stützt sich auf zwei "`Erfahrungskomplexe"': Zum einen stützt sie sich
auf die "`Erfahrung"', daß die Welt aus mehreren wesensverschiedenen
aber dennoch voneinander abhängigen Seinsstufen aufgebaut ist, und zum
anderen basiert sie auf der in der Meditation zugänglichen Erfahrung des
"`welttranszendenten Seinsgrundes"'.  Werden diese beiden Erfahrungen
kombiniert, so ergibt sich aus der Erfahrung der Abhängigkeit der
Seinsstufen voneinander die "`Nötigung"', sie als Phasen eines Prozesses
der Entfaltung einer identischen Substanz zu betrachten, welcher - hier
kommt die meditative Erfahrung ins Spiel - im welttranszendenten
Seinsgrund seinen Ursprung hat. Da die Prozeßtheologie unmittelbar auf
"`ontologischen Erfahrungen"' beruht, entzieht sie sich auch den
ansonsten naheliegenden erkenntnistheoretischen Einwänden Kantischer
Provenienz, wonach es unzulässig ist, Kategorien der innerweltlichen
Erfahrung auf das anzuwenden, was außerhalb aller möglichen Erfahrung
liegt.\footnote{Vgl. Voegelin, Anamnesis, S.50-54.}

Auf der Grundlage dieser ontologischen Stufentheorie vollzieht Voegelin nun
den Übergang vom Primat der Bewußtseinsphilosophie zum Primat der
Ontologie.\footnote{Dieser Übergang ist übrigens durchaus typisch. In
  ähnlicher Weise ging auch Heidegger, ausgehend von Husserls
  phänomenologischer Bewußtseinsphilosophie, zur Ontologie über. Etwas vom
  Heideggerschen Pathos läßt sich bei Voegelin ebenfalls verspüren, wenn er
  vor dem möglichen Mißverständnis warnt, man sei wieder in den "`friedlichen
  Gewässern der Erkenntnistheorie"' (Voegelin, Anamnesis, S.56.) angelangt.}
Zunächst geht Voegelin jedoch zum Ausgangspunkt seiner
bewußtseinsphilosophischen Überlegungen zurück. Wenn das Bewußtsein durch die
"`Erhellungsdimensionen"' der "`Vergangenheit"' und "`Zukunft"' strukturiert
ist und Zeit als solche dem Bewußtsein nicht unmittelbar gegeben ist, so kann
bezweifelt werden, daß zwischen diesen Erhellungsdimensionen die zeitliche
Beziehung der Sukzession besteht. Das Element der Zeitlichkeit läßt sich aus
diesen Erhellungsdimensionen deshalb nicht ableiten,\footnote{Zuvor scheint
  Voegelin jedoch gerade dies versucht zu haben. (Vgl. Voegelin, Anamnesis,
  S.44.)} weil es auch vorstellbar ist, daß Erinnerungen und Projektionen nur
Phantasien eines im Augenblickspunkt der Gegenwart verharrenden Bewußtseins
sind. Wie kann man aber einem solchen "`Solipsismus des
Augenblickes"'\footnote{Voegelin, Anamnesis, S.55.} entgehen? Der einzige
Ausweg besteht für Voegelin in der "`Einsicht, daß das menschliche Bewußtsein
nicht eine Monade ist, welche die Existenzform des Augenblicksbildes hat,
sondern daß es menschliches Bewußtsein ist, d.h. Bewußtsein im Fundament des
Leibes und der Außenwelt."'\footnote{Voegelin, Anamnesis, S.55.} Voegelin
spricht hier zwar von einer "`Einsicht"', aber diese Einsicht hat eher den
Charakter eines Postulates, denn, nachdem Voegelin einmal beim Solipsismus des
Augenblickes angelangt ist, gibt es nichts mehr, woraus das Sein der Zeit und
der Welt mit Gewißheit oder auch nur Wahrscheinlichkeit erschlossen werden
könnte.

Unter dem Gesichtspunkt dieser ontologischen Einsicht ist auch der Begriff des
Bewußtseinsprozesses neu zu deuten. Damit wir die Bewußtseinsvorgänge als
zeitlichen Prozeß auffassen können, sind "`Erfahrungen von
bewußtseinstranszendenten Prozessen"'\footnote{Voegelin, Anamnesis, S.55.}
erforderlich. Wie dies möglich ist, wenn, wie Voegelin zuvor kategorisch
behauptet hat, der innere Bewußtseinsprozeß seinerseits das einzige Modell
darstellt, mit dem wir bewußtseinstranszendente Prozesse verstehen
können,\footnote{Vgl. Voegelin, Anamnesis, S.44.} bleibt etwas im Dunkeln.
Voegelin scheint von einer Art wechselseitiger Abhängigkeit zwischen Sein und
Bewußtsein auszugehen, wenn er im folgenden einerseits die physische
Bedingtheit des Bewußtseins betont, zugleich aber der idealistischen Ansicht
Raum gibt, daß das Sein der Dinge von der Beziehung auf ein Bewußtsein
abhängig ist. Es läßt sich nicht leicht feststellen, auf welche Weise Voegelin
bei dieser Argumentation einem Zirkelschluß entgehen will. Wahrscheinlich zu
Recht weist Voegelin jedenfalls darauf hin, daß daraus, daß das Bewußtsein uns
in innerer Erfahrung nur als reines Bewußtsein gegeben ist, nicht folgt, daß
es nichts anderes als reines Bewußtsein ist. Vielmehr liefert nach Voegelins
Überzeugung die innere Erfahrung nur eine Teilansicht eines untrennbaren
materiell-geistigen Seinskomplexes. In der inneren wie der äußeren Erfahrung
bekommt der Mensch jeweils nur den äußersten Zipfel eines Seins zu fassen, das
sich weit über das in der Erfahrung Gegebene hinaus erstreckt.\footnote{Vgl.
  Voegelin, Anamnesis, S.55-57.}

Aus all diesen Überlegungen zieht Voegelin die Schlußfolgerung, daß die
Bewußtseinsphilosophie keinen geeigneten Anfangspunkt der Philosophie
darstellt. Das Bewußtsein setzt vielmehr das Sein voraus und die Frage des
Anfanges kann nun immer weiter zurückgeschoben werden bis zur Frage des
Anfangs der Geschichte des Kosmos. Offenbar trennt Voegelin nicht zwischen der
Frage des erkenntnistheoretischen Ausgangspunktes und der Frage der
historischen Seinsvoraussetzungen des Erkenntnisvermögens. Das
Problem eines absoluten Anfanges der Philosophie wird Voegelin noch in "`Order
and History V"' beschäftigen.\footnote{Vgl. Voegelin, Order and History V,
  S.13f.}  Vorerst gelangt Voegelin zu dem Ergebnis, daß das Bewußtsein auf
Grund dieser nie vollständig aufklärbaren Anfangsvoraussetzungen nicht wie
äußere Gegenstände erkannt und beschrieben werden kann, sondern daß es
lediglich durch Besinnung sich selbst und sein eigenes Sein erhellen
kann.\footnote{Vgl.  Voegelin, Anamnesis, S.57/58.}

Als letztes Thema dieses Aufsatzes behandelt Voegelin die wissenssoziologische
Frage, wie es zu dem Auftreten der seiner Ansicht nach verfehlten
Bewußtseinsphilosophien kommen konnte. Voegelin liefert eine solche Erklärung
an zwei Stellen seines Aufsatzes. Die erste Erklärung bezieht sich auf den
speziellen Fall der Bewußtseinsstromtheorien, die zweite Erklärung betrifft
die Bewußtseinsphilosophie im Allgemeinen.

In den Bewußtseinsstromtheorien glaubt Voegelin ein "`laizistisches
Residuum der christlichen Existenzvergewisserung in der
Meditation"'\footnote{Voegelin, Anamnesis, S.37.} wiederentdecken zu
können. Vermutlich auf Grund von einfühlendem Nachvollzug gelangt
Voegelin zu der Auffassung, daß im Bewußtseinserlebnis des "`Strömens"'
der "`Engpaß des Leibes spürbar wird"'.\footnote{Voegelin, Anamnesis, S.
  40. Außer seinen eigenen Assoziationen, für die sich in den
  zeitphilosophischen Texten, auf die Voegelin sich bezieht, durchaus
  einzelne Hinweise finden lassen, führt Voegelin noch einen eher aus
  dem Zusammenhang gegriffenen Gedanken von William James (Vgl. William
  James: Essays in Radical Empiricism, Cambridge, Massachusetts /
  London, England 1976, S.19.) und etwas später (Anamnesis, S.42)
  Bergsons Behandlung der eleatischen Paradoxe an, welche jedoch, wie es
  scheint auf einem Mißverständnis der physikalischen Begriffe von Ort
  und Bewegung zu beruhen scheint: Daß ein Körper sich zu einem
  bestimmten Zeitpunkt an einem ganz bestimmten Punkt im Raum befindet
  schließt nämlich nicht aus, daß er in diesem Punkt einen
  Bewegungszustand hat.  (Vgl. Henri Bergson: Materie und Gedächnis,
  Hamburg 1991, S.184-190.)}  Voegelin schließt daraus, daß die
Bewußtseinsstromtheorien ebenso wie die christliche Meditation auf eine
Form der Transzendenz zielen.  Während die Meditierenden in der
christlichen Meditation jedoch Welttranszendenz suchen, zielen die
Bewußtseinsstromtheorien lediglich auf die bloße Bewußtseinstranszendenz
in Richtung der Leibsphäre hin.\footnote{Vgl. Voegelin, Anamnesis,
  S.41-42.}

In einem etwas allgemeineren Rahmen stellt das Auftreten der
Bewußtseinsphilosophie für Voegelin die Reaktion auf eine Krise der Symbole
dar, wie sie alle Kulturen von Zeit zu Zeit heimsucht. Die Symbole, mit denen
die Menschen ihre Transzendenzerfahrungen ausdrücken, tendieren dazu, im Laufe
der Zeit schal und inhaltsleer zu werden. Die daraus resultierende Kulturkrise
kann nur durch die Beseitigung der alten und die Bildung neuer Symbole zum
Ausdruck der Transzendenzerfahrungen behoben werden. Platon war dies als
Antwort auf die Krisis der hellenischen Kultur im 5.Jahrhundert vor Christus
in vorbildlicher Weise gelungen. Die neuzeitliche Philosophie, die mit
Descartes ihren Anfang nimmt, stand nach Voegelins Ansicht vor einer ähnlichen
Aufgabe, doch hat sie ihr Ziel verfehlt, indem sie zwar mit der Tradition
gründlich aufräumte aber zugleich auch die Transzendenzerfahrungen aus dem
Themenkanon der Philosophie ausschloß.\footnote{Vgl. Voegelin, Anamnesis,
  S.58-60.}

\subsection{Kritik von Voegelins Theorie des Bewußtseins}

Sind Voegelins Überlegungen "`Zur Theorie des Bewußtseins"' überzeugend, geben
sie die Beziehungen zwischen Sein und Bewußtsein richtig wieder, dürfen
Voegelins Argumente als stichhaltig angesehen werden? Da es kaum möglich ist,
auf alle Einzelheiten der sehr reichhaltigen Ausführungen Voegelins
einzugehen, sollen zur genaueren kritischen Untersuchung nur einige Punkte
herausgegriffen werden, die für Voegelins Argumentation von wesentlicher
Bedeutung sind.

Innerhalb von Voegelins eigener Darstellung der Bewußtseinsstruktur findet
sich an zentraler Stelle das Argument, daß sich das Bewußtsein bei dem Versuch,
transfinite Prozesse deskriptiv zu beschreiben, auf Grund seiner eigenen
Endlichkeit unvermeidlich in Widersprüche verwickelt. Das Argument spielt
deshalb eine wesentliche Rolle, weil diese Widersprüche es erforderlich werden
lassen, zur Deutung bestimmter Wirklichkeitsbereiche auf die zwar subtilen und
seelisch sensiblen aber an Klarheit und Objektivität hinter einer deskriptiven
Beschreibung zurückstehenden Instrumente der Mythensymbolik und der
Prozeßtheologie zurückzugreifen. Unglücklicherweise steckt gerade in dieser
Passage von Voegelins Darstellung eine Reihe von Fragwürdigkeiten, die sich
nicht ohne weiteres auflösen lassen.

Zunächst einmal ist es zweifelhaft, ob, wie Voegelin es behauptet, der
Bewußtseinsprozeß das einzige erfahrene Modell eines Prozesses
darstellt.  Prozesse oder, mit anderen Worten, zeitlich ablaufende
Vorgänge im weitesten Sinne erleben wir tagtäglich in der äußeren
Erfahrung, z.B. wenn wir ein fahrendes Auto beobachten. Die äußere
Erfahrung eignet sich dabei mindestens ebenso gut, wenn nicht besser,
als die innere Erfahrung, um den Begriff eines Prozesses zu bilden.
Abgesehen davon ist es aber auch überhaupt nicht erforderlich, zur
Bildung eines Begriffes diesen von irgend einer Erfahrung abzuziehen.
Ebenso wie ein großer Teil unseres Wissens nicht aus unmittelbarer
Erfahrung stammt, gibt es auch viele Begriffe, die rein abstrakt sind.
Alle mathematischen Begriffe gehören zu dieser Klasse. Insbesondere ist
es ohne Probleme möglich, widerspruchsfreie Begriffe von Unendlichkeit
zu bilden. Die Mengenlehre verfügt über mehrere solcher Begriffe.
Freilich decken diese Begriffe nicht alle Wortbedeutungen von
"`unendlich"' ab, und die "`unendliche Sehnsucht"', von der ein
romantischer Dichter schwärmen mag, wird von der Mengenlehre nicht
erfaßt, aber es ist nun nicht mehr einleuchtend, weshalb die Finitheit
des Bewußtseinsprozesses bei der Beschreibung von unendlichen Prozessen
zu Ausdruckskonflikten führen muß. Außerdem scheint sich Voegelin auch
hinsichtlich der Bedeutung der Kantischen Antinomien geirrt zu haben.
Kants Antinomien beruhen letztlich auf unterschiedlichen
Voraussetzungen, die den einander gegenübergestellten Beweisen und
Gegenbeweisen zu Grunde liegen.  Um Antinomien könnte es sich nur noch
dann handeln, wenn diese Voraussetzungen gleichermaßen notwendig wären.
Aber dies - und hierin irrt Kant und mit ihm viele seiner Interpreten
und, wie es scheint, leider auch Voegelin - ist nicht der
Fall.\footnote{Vgl. Immanuel Kant: Kritik der reinen Vernunft, Hamburg
  1976, S.454-469. Für die Kant-Apologetik stellvertretend: Peter
  Baumanns: Kants Philosophie der Erkenntnis. Durchgehender Kommentar zu
  den Hauptkapiteln der "`Kritik der reinen Vernunft"', Würzburg 1997,
  S.742ff. - Daß Kant irrt, kann man sich leicht überlegen, wenn man bei
  den Antinomien genau darauf achtet, von welchen expliziten und
  impliziten Voraussetzungen Kant bei seinen Beweisen jeweils ausgeht.
  Es würde zu weit führen, dies hier im einzelnen auszuführen.}  Was
Voegelin schließlich mit den Paradoxen der Mengenlehre meint, geht aus
dem Text leider nicht hervor.  Möglicherweise meint Voegelin die
Russelsche Antinomie, die in der naiven Mengenlehre auftritt.  Aber
erstens handelt es sich nicht um ein Paradox der Unendlichkeit, und
zweitens läßt sie sich mühelos durch eine axiomatische Fassung der
Mengenlehre beseitigen.\footnote{Vgl. Jürgen Schmidt: Mengenlehre
  (Einführung in die axiomatische Mengenlehre). I.  Grundbegriffe,
  Mannheim 1966, S.22-24.}

Voegelins Argument ließe sich im Grundsätzlichen immer noch dann
rechtfertigen, wenn es gelänge zu zeigen, daß bestimmte Wirklichkeitsbereiche
aus anderen Gründen als dem ihrer "`Infinitheit"' einer deskriptiven
Beschreibung unzugänglich sind. Dann müßte die Diskussion um die Fragen
geführt werden, ob es diese Wirklichkeitsbereiche tatsächlich gibt, und wenn
es sie gibt, ob Mythensymbolik oder Prozeßtheologie sie erfassen können. So wie
Voegelin argumentiert, bleibt die Notwendigkeit des Gebrauches dieser
Symbolformen jedoch unbegründet. 

Einen weiteren wichtigen Abschnitt, der zwar weniger für Voegelins folgende
Argumentation von Bedeutung ist, aber dafür seine grundsätzliche philosophische
Einstellung widerspiegelt, bildet Voegelins Versuch, das Problem der
Anerkennung der Mitmenschen als gleichartige und gleichwertige Wesen (in
Voegelins Terminologie: das Problem der "`Erfahrung vom Nebenmenschen"') mit
Hilfe der Mythengeschichte zu lösen. Voegelins Argumentation enthält eine
Reihe von Schwachpunkten. Die erste Schwierigkeit bildet der Begriff des
"`Erfahrungsfaktums"'. Obwohl wir, wie auch Voegelin einräumt, von unseren
Mitmenschen keine innere Erfahrung haben, sollen wir dennoch durch ein
Erfahrungsfaktum unmittelbar davon in Kenntnis gesetzt sein, daß sie ein
Innenleben haben. Nun mögen wir zwar intuitiv den Eindruck haben, daß in
unseren Mitmenschen auch ein denkendes und fühlendes Bewußtsein steckt, aber
die Berufung auf die Intuition ist auch dann noch ein schwaches Argument in
der Erkenntnistheorie, wenn sie hochtönend als "`Fundamentalcharakter"' der
"`Transzendenzfähigkeit"'\footnote{Voegelin, Anamnesis, S.47.} des Bewußtseins
bezeichnet wird. Der Einwand gegen Husserl, daß sich das Du nicht im Ich
konstituiert, ist dagegen durchaus angebracht, denn das Bewußtsein kann
unmöglich durch Konstitution etwas hervorbringen, was außerhalb seiner selbst
existiert. Als geradezu abwegig erscheint allerdings Voegelins Behauptung, daß
dieses erkenntnistheoretische Problem nur im Rahmen der altertümlichen
Gleichheitsmythen behandelt werden kann. Weder für die Formulierung dieses
Problems noch erst recht zu seiner Lösung ist der Rückgriff auf die
Mythengeschichte notwendig oder auch nur hilfreich.

Der zweite Schwachpunkt von Voegelins Argumentation liegt in seiner Annahme,
daß die moralische Gleichheit aller Menschen nur im Rückgriff auf alte Mythen
artikuliert werden kann. Bei der Behandlung dieser moralphilosophischen
Problematik müssen drei unterschiedliche Ebenen klar voneinander getrennt
werden: Die Ebene der Begründung von Werten, die Ebene der Artikulation bzw.
Formulierung der Werte und die Ebene der Vermittlung und Verbreitung der
Werte. Für die Begründung des Gleichheitswertes kann die Mythengeschichte
offensichtlich nicht herangezogen werden. Wenn die moralische Gleichheit der
Menschen nämlich im Sinne einer moralischen Intuition auf einem
"`Erfahrungsfaktum"' beruht,\footnote{Im Bereich der Ethik ist anders als in
  der Erkenntnistheorie die Berufung auf die Intuition unter Umständen
  legitim. Es stellt sich dann nur die Frage, ob intuitiv begründete Werte
  intersubjektive Verbindlichkeit beanspruchen dürfen.} dann besteht die
einzig ehrliche Weise, diesen Wert zu begründen, darin, auf diese Intuition
bzw.  Erfahrung hinzuweisen, und gegebenenfalls die Begleitumstände (Stimmung,
äußere Situation, evtl. eingenommene bewußtseinsstimulierende Substanzen) zu
beschreiben, unter denen sie zustande kommt oder in besonders deutlicher Weise
hervortritt.

Die Formulierung des Gleichheitswertes ist mit und ohne Rückgriff auf
Mythen möglich. Ohne Rückgriff auf die Mythologie kann sie
beispielsweise durch die Worte erfolgen: "`Alle Menschen sind gleich"'.
Bereits mit diesen schlichten Worten ist der Inhalt der Gleichheitsidee
vollständig und ohne jede Mythologie ausgedrückt. Eine Artikulation
unter Rückgriff auf die Mythologie könnte durch Erzählung der Geschichte
von Adam und Eva erfolgen. Der mythische Ausdruck ist nun sicherlich die
schlechtere Art der Artikulation, denn da es sich schwer verbergen läßt,
daß die Geschichte von Adam und Eva nicht der historischen Wahrheit
entspricht, riskiert man heutzutage sehr leicht, für unseriös oder
lächerlich gehalten zu werden, wenn man mit einer solchen Geschichte
aufwartet. Analoges gilt für das Problem der Vermittlung des
Gleichheitswertes. Zu der Zeit, als Voegelin den Aufsatz "`Zur Theorie
des Bewußtseins"' niederschrieb, war alles Mythologische sehr in Mode.
Es war daher ein naheliegender Gedanke und konnte gar als eine
Überlebensnotwendigkeit erscheinen, den totalitären Mythen aufgeklärte
Mythen entgegenzustellen, um das Verständnis für die Urwahrheiten des
menschlichen Zusammenlebens wiederzuerwecken.\footnote{Eine ähnliche
  Absicht lag bekanntlich Thomas Manns Josephs-Romanen zu grunde, deren
  zeitgeschichtlichen Bezug er in einer späteren Selbstdeutung auf die
  Formel gebracht hat, daß der "`Mythos .. in diesem Buch dem Fascismus
  aus den Händen genommen"' wurde.  (Thomas Mann: Joseph und Seine
  Brüder (Vortrag in der Library of Congress am 17.11.1942), in: Thomas
  Mann: Essays. Band 5: Deutschland und die Deutschen 1938-1945. (Hrsg.
  v. Hermann Kurzke und Stephan Stachorski), Frankfurt am Main 1996,
  S.185-200 (S.189).)}  In der heutigen Zeit würde eine solche
Werbestrategie jedoch recht altfränkisch herauskommen. Daher ist
heutzutage auch nicht mehr zu erwarten, daß der Ordnungswille "`nur
aktiv sein kann, wo er seinen Sinn in der Ordnung des
Gemeinschaftsmythos hat"'\footnote{Voegelin, Anamnesis, S.50.} Im
Gegenteil ist es eher zu befürchten, daß der Ordnungswille unglaubwürdig
wird, wenn er sich nur auf eine mythische Grundlage berufen kann.
Erfolgversprechender könnte es sein, das Gefühl für die Gleichheit aller
Menschen zu wecken, indem man darauf aufmerksam macht, daß alle
Menschen, gleichgültig an welchem Ort der Welt, von denselben
menschlichen Nöten und Freuden, Hoffnungen und Sorgen bewegt werden.

Wie man sieht, ist also der Rückgriff auf die Mythengeschichte für das
Gleichheitsideal nicht erforderlich und nur begrenzt nützlich. Voegelins
Argument dafür, daß er unerläßlich sei, ist historischer Art und besteht in
der Behauptung, daß alle Gleichheitsideen Derivate jener beiden erwähnten
Urmythen sind.\footnote{Daß sich aus dem von Voegelin behaupteten
  Ausdruckskonflikt bei der Artikulation unendlicher Prozesse für diesen Fall
  kein notwendiger Grund für den Gebrauch der Mythensymbolik ableiten läßt,
  wurde bereits erwähnt.} Inwieweit dies historisch richtig und zwingend ist,
sei dahingestellt. Für Voegelin war diese Vorstellung wohlmöglich deswegen
attraktiv, weil sie ihm erlaubte, eine Analogie zum Leib-Geist-Dualismus
herzustellen, indem der eine Mythos ein leiblicher und der andere ein
geistiger ist. Aber selbst wenn Voegelins historische These richtig sein
sollte, so folgt daraus nicht, daß die Gleichheitsidee niemals etwas anderes
sein kann als ein Derivat dieser Urmythen. Insbesondere kommt es bei der
moralphilosophischen Diskussion der Gleichheitsidee nur auf den
Inhalt und die Begründung dieser Idee an. Diese sind aber von der
Entstehungsgeschichte unabhängig, so daß die Diskussion darüber unbekümmert um
historische Zusammenhänge geführt werden kann.

Nicht nur Voegelins Ausführungen zur Mythensymbolik sondern auch seine
Interpretation der Prozeßtheologie wirft einige Fragen auf. Vor allem
Voegelins Annahme, daß die Prozeßtheologie einen Bereich von
"`ontologischen Erfahrungen"'\footnote{Voegelin, Anamnesis, S.54.}
auslegt, bedarf der Klärung. Denn der Begriff der Erfahrung wird mit
dieser Annahme stark überstrapaziert. Das Wissen um die Stufen des Seins
ist deskriptives Wissen, das sich bestenfalls auf die Erfahrung stützt,
das aber über die unmittelbare Erfahrung weit hinaus geht. Auch daß die
meditative Erfahrung ein Wissen vom Seinsgrund vermittelt, muß als
höchst zweifelhaft angesehen werden,\footnote{Vgl. dazu die Ausführungen
  zu Voegelins Descartes-Interpretation weiter oben in dieser Arbeit.}
sofern "`Seinsgrund"' ein ontologischer Begriff ist und nicht nur ein
Name für die meditative Erfahrung selbst, wie Voegelins spätere Theorie
der "`Indizes"' des Bewußtseins dies nahelegt. Im letzteren Fall
bestünde dann allerdings auch keine "`Nötigung"' mehr (und es wäre im
Gegenteil sogar ganz und gar unmöglich), den ontologischen Seinsprozeß
im "`Seinsgrund"' entspringen zu lassen. Wird jedoch auf diese Weise in
Zweifel gezogen, daß es eine privilegierte Klasse ontologischer
Erfahrungen gibt, dann bleibt auch die Prozeßtheologie, die Voegelin
skizziert, als eine bestimmte ontologische Theorie in vollem Umfange
angreifbar durch die Erkenntniskritik.

Abgesehen von diesen Schwierigkeiten bleibt auch der Sinn und Zweck der
Prozeßtheologie etwas im Unklaren. Voegelin zufolge geht die
Prozeßtheologie aus von der Frage: "` `Warum ist etwas, warum ist nicht
Nichts?'  "'\footnote{Voegelin, Anamnesis, S.51. - Vgl. Friedrich
  Schelling: Philosophie der Offenbarung, Zwölfte Vorlesung, in:
  Frank-Peter Hansen (Hrsg.): Philosophie von Platon bis Nietzsche,
  CD-ROM, Berlin 1998, S.37855 / S.72 (Konkordanz: Friedrich Wilhelm
  Joseph von Schelling: Werke. Auswahl in drei Bänden. Herausgegeben und
  eingeleitet von Otto Weiß. Leipzig 1907.  Band 3, S.781).} Allerdings
unternimmt die Prozeßtheologie dann keinen ernsthaften Versuch, diese
Frage zu beantworten. Eher scheint sie darauf hinauszulaufen, das Gefühl
der Verblüffung, das in jener Frage liegt, zu artikulieren. Wenn sich
aber die überwiegende Mehrzahl der Menschen nicht mit dem
erkenntnistheoretischen Befund der Unbeantwortbarkeit dieser Frage
zufrieden geben will, wie Voegelin durchaus plausibel
vermutet,\footnote{Vgl.  Voegelin, Anamnesis, S.51.} warum sollte sie
sich dann mit der Prozeßtheologie, die diese Frage auch nicht
beantworten kann, abspeisen lassen?

Hinsichtlich der Einbettung der Bewußtseinsphilosophie in die Ontologie,
wie sie Voegelin im letzten Abschnitt seines Aufsatzes vollzieht, sind
vor allem die zwei Thesen zu prüfen, daß die ontologische Problematik
die Voraussetzung der Erkenntnistheorie bzw. Bewußtseinsphilosophie
bildet, und daß der Mensch sich auf sein Bewußtsein und sein Wesen nur
orientierend besinnen aber es niemals zu einem Gegenstand äußerer
Beschreibung machen kann.

Die erste dieser Thesen ist auch für Voegelins wissenssoziologische
Erklärungen von Bedeutung, denn nur, wenn sie bejaht wird, kann der "`Versuch
einer `radikalen' Bewußtseinsphilosophie aufklärungsbedürftig
"'\footnote{Voegelin, Anamnesis, S.58.} erscheinen. In einer bestimmten
Hinsicht kann die Triftigkeit von Voegelins Einwand gegen die reine
Bewußtseinsphilosophie kaum bestritten werden. Die Erklärung der meisten
Bewußtseinsvorgänge dürfte nur schwer möglich sein, ohne auf die Tatsache
zurückzugreifen, daß es sich um das Bewußtsein eines Menschen handelt, der in
einer materiellen Außenwelt lebt. So ist etwa das gelegentliche Auftreten des
Bewußtseinsphänomens "`Hunger"' nur verständlich, wenn man weiß, daß das
Bewußtsein, in dem es auftritt, das Bewußtsein eines Lebewesens ist, welches
von Zeit zu Zeit der Speise und des Trankes bedarf. Auch darf wohl behauptet
werden, daß ontologische Fragen insgesamt relevanter sind als nur rein
bewußtseinsphilosophische Probleme, denn der Erhalt und die Wohlfahrt unseres
Lebens hängt von dem ab, was in der Welt geschieht und nicht von dem, was sich
davon im Bewußtsein spiegelt. Insofern spricht für Voegelins kritische
Einstellung gegenüber der reinen Bewußtseinsphilosophie auch eine starke
intuitive Plausibilität.

Aber ist damit auch die Möglichkeit einer reinen, d.h. ausschließlich
introspektiven Beschreibung der Bewußtseinsvorgänge ausgeschlossen? Und muß
die Erkenntnistheorie nun doch, trotz der drohenden Gefahr von
Begründungszirkeln, ein Wissen um die Außenwelt voraussetzen? In dieser
Hinsicht scheint Voegelins These unzureichend begründet zu sein. Auch wenn
viele Bewußtseinsvorgänge losgelöst von der Außenwelt nur schwer zu deuten
sein dürften, so bleibt doch die Möglichkeit, das Bewußtsein als reines
Bewußtsein introspektiv zu beschreiben, immer noch bestehen. Sollte zur
Beschreibung des reinen Bewußtseins als Prozeß eine Form von Zeitlichkeit
vorausgesetzt werden müssen, die nicht introspektiv erfahrbar ist, so genügt
es, allein die Existenz dieser Form von Zeitlichkeit zu postulieren, ohne
zugleich auch den Leib und die Geschichte vorauszusetzen. Eine solche
Beschreibung des reinen Bewußtseins würde auch dann keine weiteren
ontologischen Hypothesen voraussetzen, wenn es faktisch substanzidentisch mit
seinem leiblichen Fundament (Gehirn) sein sollte. Dabei ist übrigens die
Hypothese der Substanzidentität zum Verständnis "`der Fundierung von
Bewußtsein in Leib und Materie"'\footnote{Voegelin, Anamnesis, S. 55.} nicht
einmal zwingend erforderlich, denn diese Fundierung könnte auch durch die
Hypothese der kausalen Verursachung von Bewußtseinsphänomenen durch mit diesen
nicht substanzidentische physische Phänomene erklärt werden. Die
Erkenntnistheorie schließlich setzt schon deshalb nicht die Ontologie voraus,
weil die erkenntnistheoretischen Probleme auf einer anderen Ebene, auf der
Ebene der Gültigkeit, liegen als die ontologischen Probleme. Zwar ist das
Faktum, daß es Erkenntnis und Wahrheit gibt, davon abhängig, daß es Lebewesen
gibt, die erkennen können, aber die Gültigkeit von Erkenntnis und die Antwort
auf die Frage, worin Wahrheit besteht, und nach welchen Kriterien sie
festgestellt werden kann, hängen nicht von diesen ontischen Voraussetzungen
ab. Am leichtesten läßt sich dies an einem Beispiel verdeutlichen: Damit der
Satz "`Zwei mal zwei ist vier."' existiert, muß es wenigstens ein intelligentes
Wesen geben, welches ihn denkt oder äußert,\footnote{Manche Philosophen
  glauben auch, daß Sätze wie dieser in einem platonischen Ideenhimmel
  existieren. Die Sätze würden dann auch existieren, wenn es keine Menschen
  oder nicht einmal eine Welt gäbe.}  und damit dieses Wesen existiert, müssen
weitere ontische Voraussetzungen erfüllt sein. Die Wahrheit dieses Satzes
hängt jedoch von keiner dieser Voraussetzungen ab.\footnote{Man könnte nun
  vermuten, daß nicht die Wahrheit aber die Bedeutung eines Satzes von
  ontischen Voraussetzungen, z.B. von der Bedeutungsgeschichte der in ihm
  verwendeten Wörter abhängt. Aber die Art und Weise, wie die Bedeutung eines
  Wortes entstanden ist, stellt keine Bedeutungsvoraussetzung des Wortes dar,
  auch wenn die Etymologie eines Wortes hilfreich sein kann, um die
  Bedeutung herauszufinden, wenn sie nicht bekannt ist.}

Wie verhält es sich mit Voegelins zweiter These, daß das Bewußtsein sich nicht
selbst wie einen Gegenstand betrachten kann? Voegelin führt als Grund für
diese These an, daß die Bewußtseinsphilosophie "`ein spätes Ereignis in der
Biographie des Philosophen ist"', welches wiederum ein Ereignis in der
Geschichte seiner Gemeinschaft, in der Geschichte der Menschheit und in der
Geschichte des Kosmos ist. Aber diese Begründung ist wenig stichhaltig und
dürfte eher auf Kosten einer holistischen Überzeugung Voegelins gehen als auf
rationaler Überlegung beruhen. Jeder noch so profane Gegenstand hat auch seine
Vorgeschichte im Kosmos, und das Wissen über ihn hat eine Vorgeschichte in der
Geschichte des menschlichen Wissens. Dennoch wird niemand bestreiten, daß es
Gegenstände gibt, die vollständig erkannt werden können. Wenn irgend etwas nur
historisch verstanden werden kann, so muß es dafür speziellere Gründe geben.
Und außer dem Kosmos selbst gibt es vermutlich nichts, dessen Erkenntnis die
Geschichte des gesamten Kosmos voraussetzt.

Problematisch ist auch jener Teil von Voegelins Aufsatz, in welchem er das
Auftreten der Bewußtseinsphilosophie in der Neuzeit historisch zu deuten
versucht. Wie bereits dargelegt, sind Erkenntnistheorie und mit gewissen
Einschränkungen auch die Bewußtseinsphilosophie legitime Einzeldisziplinen der
Philosophie, die nicht als Teilgebiet einer allgemeinen Ontologie behandelt
werden müssen. Ihr Auftreten ist daher bereits wesentlich weniger
"`aufklärungsbedürftig"' als Voegelin dies meint. Abgesehen davon bleibt es
schleierhaft, woher Voegelin überhaupt die historische Aufgabe der
Bewußtseinsphilosophie nimmt, eine neue Symbolik für religiöse
Transzendenzerfahrungen zu suchen. Sofern Voegelin nicht wie Husserl die
Existenz eines historischen Telos voraussetzen will, das jeden Philosophen
verpflichtet, sich mit diesem Problem zu beschäftigen, kann er den Philosophen
kein Versäumnis vorwerfen, wenn sie sich für die Frage der Symbolisierung
von Transzendenzerfahrungen nicht interessieren. Freilich hat Voegelin das
Recht, den Ausschluß der Besinnung auf Transzendenzerfahrungen aus dem
Themenkanon der Philosophie zu tadeln, wenn er selbst der Ansicht ist, daß
dieses Thema in der Philosophie einen Platz haben sollte. Allerdings ist zu
berücksichtigen, daß andere Philosophen dies explizit ablehnen, und daß sie
dazu mindestens ein ebensogutes Recht haben. Abgesehen davon existierten auch
in der Neuzeit mit Religion und Theologie durchgängig geistige Disziplinen, die
sich mit der Transzendenz beschäftigten und immer noch beschäftigen. Insofern
ist es ein wenig voreilig, eine historische Krise der Symbole zu suggerieren.
Schließlich ist anzumerken, daß gerade die Husserlsche Philosophie, welche
sich noch am ehesten angeschickt hat, die Bewußtseinsphilosophie zum
allumfassenden Universalparadigma auszuweiten, sich gegenüber dem religiösen
Denken als sehr aufgeschlossen erwiesen hat. Daß die Zuwendung zum
Bewußtseinsstrom ein Substitut für die christliche Existenzvergewisserung in
der Meditation darstellt, scheint lediglich einer Wortassoziation des Wortes
"`transzendent"' geschuldet zu sein, welches welttranszendent oder
bewußtseinstranszendent bedeuten kann.\footnote{Dieses gilt zumindest, soweit
  von einem "`Substitut"' die Rede ist. Daß im Bewußtseinsstrom auch
  Existenzversicherung gesucht werden kann, läßt sich jedoch durchaus
  nachvollziehen. Aber kann man beispielsweise William James, der bekennender
  Christ war, unterstellen, der Bewußtseinsstrom habe ihm als Substitut der
  christlichen Existenzvergewisserung gedient?}

% (unter dem heutzutage
% vorherrschenden spirituell stumpfen und religiös ungebildeten Menschenschlag)

% in welchen er die Haltung der
%   blinden Unterwerfung unter den Mythos (verkörpert u.a. durch die Gestalt des
%   Laban, der seinen eigenen Säugling im Fundament seines Hauses einmauert in
%   dem blinden Aberglauben, damit Unheil abwenden zu können) als
%   "`Gottesdummheit"' der fortschrittlichen Gestaltung des Mythos durch Jakob
%   und Joseph als "`Gottesklugheit"' entgegenstellt.

% Da Voegelin diese These auch an anderer Stelle häufig
% wiederholt, und sie darüber hinaus zu den Kerngedanken seines
% historischen Ansatzes zu gehören scheint, lohnt es sich diese
% These etwas ausführlicher zu untersuchen. Leider wird bei
% Voegelin nicht recht deutlich, worin die Begründung dieser These
% bestehen soll. Wird versucht, aus Voegelins etwas zusammenhanglosen
% Bemerkungen\footnote{Vgl.  Voegelin, Anamnesis, S.57/58.} eine
% Argumentation zu rekonstruieren, so könnte sie in Form einzelner
% Thesen möglicherweise wie folgt lauten: 
% \begin{enumerate}
% \item Das Bewußtsein kann sich nicht wie einen Gegenstand selbst
%   erkennen, weil das Erkennende und das Erkannte ein und dasselbe
%   sind. 
% \item Das Bewußtsein kann sich nicht zum Gegenstand machen, weil
%   das konkrete Einzelbewußtsein in Seinsvoraussetzungen
%   eingebettet ist, die zu umfangreich sind, als daß sie alle
%   jemals ergründet werden könnten. (Das Bewußtsein ist das
%   Bewußtsein eines Menschen, dieser Mensch hat eine
%   Lebensgeschichte, diese Lebensgeschichte ist Teil der
%   Geschichte der Gemeinschaft in der er lebt, diese wiederum Teil
%   der Menschheitsgeschichte usw. bis zum Anfang des Kosmos.)
% \item Das Bewußtsein kann sich nicht zum Gegenstand machen, weil
%   die Mittel des Erkennens ebenfalls eine nicht vollständig
%   ergründbare Vorgeschichte haben. (Um Erkenntnis zu artikulieren
%   ist es notwendig, Symbole zu gebrauchen. Diese Symbole gehören
%   zur Symbolsprache einer Gemeinschaft von Menschen usw.)
% \item Eine gangbare Alternative zur gegenständlichen Erkenntnis
%   des Bewußtseins durch sich selbst bildet jedoch eine mystische
%   Operation orientierender Besinnung innerhalb des
%   Bewußtseinsraumes.
% \end{enumerate}
% Man sieht sofort, daß Punkt 3 sehr stark der zuvor besprochenen
% These der ontologischen Voraussetzungen der Erkenntnistheorie
% ähnelt. Das dort gesagte läßt sich daher leicht übertragen. Daß
% die Existenz der Erkenntnismittel von Seinsvoraussetzungen
% abhängt, hat keinen Einfluß auf ihre Tauglichkeit als
% Erkenntnismittel. Zwar wäre es ebenso kühn zu behaupten, man
% könne alle Erkenntnisvoraussetzungen klären. Aber in dieser
% Hinsicht besteht in Bezug auf die gegenständliche Erkenntnis und
% die Erkenntnis vom Bewußtsein keinerlei Unterschied. Und wenn die
% Erkenntnis von Gegenständen ohne Klärung sämtlicher
% Erkenntnisvoraussetzungen möglich ist, dann läßt sich aus Punkt 3
% auch kein Argument gegen die Möglichkeit der Erkenntnis vom
% Bewußtsein ableiten.

% Zwischen Punkt 2 und Punkt 3 besteht die Gemeinsamkeit, daß auch
% die in Punkt 2 erwähnte Bedingung nicht zwingenderweise
% bewußtseinstypisch ist sondern ebenfalls für andere Gegenstände
% gelten kann. Sofern nicht gerade ein dogmatischer Holismus
% vertreten wird, schließt allerdings die Tatsache, daß ein
% Gegenstand in ontische oder historische Zusammenhänge eingebunden
% ist, die nicht vollständig erkannt werden können, noch nicht aus,
% daß dieser Gegenstand selbst vollständig erkannt werden kann. So
% ist beispielsweise ein so banaler Gegenstand wie ein Löffel in
% vielfältige historische Zusammenhänge eingebunden: Er muß in
% einer Besteckfabrik hergestellt worden sein. Diese Besteckfabrik
% wird (wenn es sich um billiges Besteck handelt) mit Aluminium
% beliefert, daß in einem Hüttenwerk aus dem Erz herausgelöst
% worden ist. Die Erze sind auf Grund komplizierter geologischer
% Vorgänge in die Erde eingelagert worden, und alles zusammen ist
% mit dem Urknall entstanden. Nun braucht man aber, um zu erkennen,
% was ein Löffel ist, offensichtlich von alldem überhaupt nichts zu
% wissen. Aus genau dem selben Grund besagt auch der Hinweis, daß
% das Bewußtsein ein spätes Ereignis in einer Kette von Ereignissen
% ist, nichts bezüglich der Grenzen seiner Erkennbarkeit.

% Im Gegensatz ist es jedoch denkbar daß der Erkenntnis vom
% Bewußtsein dadurch Beschränkungen auferlegt sind, daß in diesem
% Falle das Erkennende und das Erkannte identisch sind. Voegelin
% führt jedoch nicht aus, in welcher Weise hierdurch der
% Bewußtseinserkenntnis Grenzen gezogen werden. 

% Aber diese These scheint wenn
% nicht falsch so doch wenigstens unzureichend begründet zu sein.
% Daß die Bewußtseinsphilosophie und folglich auch die
% Erkenntnistheorie auf bewußtseinsphilosophischem Wege das
% klassische Problem der Existenz einer Außenwelt nicht lösen kann,
% bedeutet insbesondere dann nicht, daß sie die Ontologie
% voraussetzt, wenn die Existenz der Außenwelt nur durch ein
% Postulat - der Gebrauch des Wortes "`Einsicht"' bei Voegelin ist
% ungedeckt - behauptet wird. Erst mit Hilfe dieses Postulats kann
% gefolgert werden, daß das Bewußtsein in anderen Seinsphären
% "`fundiert"' ist. Aber auch wenn die mit gewissen Einschränkungen
% plausible These der Fundiertheit stimmen sollte, dann schließt
% dies nicht aus, daß eine rein introspektive
% Bewußtseinsphilosophie durchgeführt werden kann, solange sie
% nicht die Behauptung aufstellt, daß das Bewußtsein auch ontisch
% nichts anderes als das introspektiv Erfaßte sei. Ebenso ist es
% möglich das "`reine"' Bewußtsein deskriptiv als Prozeß zu fassen.
% Selbst wenn der Begriff des Prozesses so verstanden werden soll,
% daß er Zeitlichkeit (und nicht bloß jene Relationen, die
% introspektiv zwischen den Erhellungsdimensionen erfahrbar sind)
% einschließen soll, so würde es genügen, nur die Existenz der Zeit
% für das Bewußtsein zu postulieren, ohne irgendwelche
% weitergehenden ontologischen "`Einsichten"' vorauszusetzen. Damit
% würde man aber innerhalb des Bereiches des reinen Bewußtseins
% bleiben. 

% Scheinen also die Begründungen Voegelins für seine These
% unzureichend zu sein, so bedeutet dies noch nicht, daß die These
% selbst falsch ist. Es stellt sich daher die Frage, ob
% möglicherweise andere Gründe dafür sprechen, daß die
% Erkenntnistheorie die Ontologie voraussetzt. 

\section{Die "`anamnetischen Experimente"' Voegelins}

Den ersten Teils seines Werkes "`Anamnesis"' schließt Voegelin mit der
Wiedergabe einiger Kindheitserinnerungen ab. Es handelt sich um
Schilderungen intellektueller Erlebnisse seiner Kindheit, in welchen zum
erstenmal, in einer freilich dem zarten Alter entsprechenden Weise, die
Fragen auftauchten, welche Voegelin sich später als Bewußtseinsphilosoph
erneut stellte. Da diese Erinnerungen teilweise erst durch den Versuch
wieder zu Tage traten, sich Rechenschaft über die ersten Anfänge jener
Bewußtseinserlebnisse und Stimmungen abzulegen, die Voegelin als
Philosoph untersuchte, spricht Voegelin von "`anamnetischen
Experimente[n]"',\footnote{Voegelin, Anamnesis, S.61.}  deren Resultate
diese Erinnerungen sind. Unter den recht reizvoll und oft mit
Augenzwinkern erzählten Episoden, die Voegelin aus seiner Kindheit
mitteilt, finden sich Stücke wie jenes von dem Karnevalszug, der in dem
Kind eine dunkle Angst erregte, weil sich der Zug, da ihn einzelne
Jecken immer wieder verließen, um in den Seitenstraßen zu verschwinden,
am Ende aufzulösen schien.\footnote{Vgl. Voegelin, Anamnesis, S.64 (Nr.
  3).}  In einer anderen Episode berichtet Eric Voegelin, der einen Teil
seiner Kindheit in Königswinter nahe dem Siebengebirge verbrachte, von
den drei Breibergen, die man vom Ölberg aus sehen kann. Dem Märchen
zufolge muß man sich durch diese Breiberge hindurchfressen, um in das
dahinter liegende Schlaraffenland zu gelangen. Die Angst des Kindes,
dabei im Brei stecken zu bleiben, trübte sehr die Hoffnung auf das
Schlaraffenland.\footnote{Vgl. Voegelin, Anamnesis, S.65-66 (Nr. 5).}
Andere Episoden teilen ähnliche Gefühle der Zweifelhaftigkeit des
vollkommenen Glückes mit.\footnote{Vgl. Voegelin, Anamnesis, S.66
  (Nr.6), S.73-64 (Nr.16).} Wohlmöglich betrachtete Voegelin diese
frühen Erfahrungen als Vorboten der späteren Skepsis des
Politikwissenschaftlers gegenüber der Utopie.\footnote{Voegelin gibt
  keine Erläuterungen zu den einzelnen Episoden, die ihre Bedeutung für
  sein späteres Denken erklären könnten. Einige vorsichtige
  Deutungsversuche unternimmt Barry Cooper. Vgl. Barry Cooper: Eric
  Voegelin and the Foundations of Modern Political Science, Columbia and
  London 1999, S.204-207.} In einer weiteren Episode schildert Voegelin
den starken emotionalen Eindruck, den das Märchen vom Kaiser und der
Nachtigall, die durch ihren Gesang den Tod dazu erweicht, vom Kaiser
abzulassen, in ihm hinterlassen hat. Später fand Voegelin diese
Stimmung aus seiner Kindheit im Märchen nicht mehr, wohl aber beim
Anhören mancher Musikstücke wieder: "`Die Bedeutung, die ein Musikwerk
für mich hat, ist bestimmt durch den Grad, in dem es diese süße
Beklemmung zwischen Tod und Leben wieder erregt."'\footnote{Voegelin,
  Anamnesis, S.75 (Nr.18).}
 
Solcher und ähnlicher Art sind die von Voegelin wiedergegebenen
Kindheitserinnerungen. Doch was soll mit ihrer Mitteilung bewiesen werden? In
den einleitenden Vorbemerkungen zu seinen "`anamnetischen Experimenten"' führt
Voegelin die Thesen aus dem vorangehenden Aufsatz noch einmal auf: Das
Bewußtsein ist kein Strom, es verfügt über vielfältige Transzendenzfähigkeiten
und die Besinnung über das Bewußtsein greift Bewußtseinserlebnisse des
Philosophen auf, die bereits sehr viel früher in seinem Leben erstmals zu
Tage getreten sind. Weiterhin sieht Voegelin in den frühen
Bewußtseinserlebnissen "`Erfahrungseinbrüche"' und "`Erregungsquellen"', "`aus
denen es zu weiterer philosophischer Besinnung treibt"'. Die Intensität und
Emotionalität\footnote{Voegelin spricht wörtlich von der "`Natur der
  Erfahrungseinbrüche"', der "`Art der Erregungen"' und der "` `Stimmung' "'
  des Bewußtseins. (Vgl. Voegelin, Anamnesis, S.61.)} solcher
"`Erfahrungseinbrüche"' bilden für Voegelin den Maßstab der Radikalität, d.i.
der Breite und Tiefe einer philosophischen Besinnung.

Sind aber solche Erfahrungen, wie Voegelin sie erzählt, für die Behandlung
bewußtseinsphilosophischer Probleme überhaupt relevant? Sicherlich ist nicht
für jedes philosophische Problem der Rückgang auf die Erfahrung seines ersten
Auftretens erforderlich. Für die Lösung des zenonschen Problems etwa, wie aus
unendlich vielen Einzelschritten ein kontinuierlicher Übergang entstehen
kann,\footnote{Vgl. Voegelin, Anamnesis, S.71/72 (Nr. 14: Der Laib Brot).}
spielt es sicherlich keine Rolle, wann und wie es zum erstenmal dem
Philosophen, der es behandelt, begegnet ist.  Auch wenn er es erst im
Erwachsenenalter in einem Buch gelesen hat, hindert ihn nichts daran, dieses
Problem angemessen zu erörtern. In welchem Falle ist es dann aber notwendig,
auf die Problemerfahrungen zurückzugehen, und in welchem Fall nicht?
Offensichtlich ist dies dann nicht erforderlich, wenn die Erfahrung nur den
Anlaß gibt, über ein philosophisches Problem nachzudenken. Außer wenn es um
Fragen der Selbsterkenntnis geht, hat das Erlebnis eines philosophischen
Problems jedoch vermutlich nie eine andere Bedeutung als die, ein Anlaß des
Nachdenkens zu sein. Dies gilt auch für die Probleme der
Bewußtseinsphilosophie. Für die Lösung beispielsweise des Problems der
Konstitution von Gegenständen ist zwar möglicherweise der Bezug auf innere
Erfahrungen, nicht aber der Rückgang auf die ersten Erfahrungen dieser Art
oder auf das erstmalige Erlebnis, daß es sich hier um ein Problem handelt,
erforderlich. Im übrigen stünde eine Philosophie, die sich nur auf die eigenen
inneren Erlebnisse des Philosophen stützt, vor dem Problem, daß sie bloß
subjektiv gültige Ergebnisse liefern könnte.

Es scheint also, daß Voegelin die Bedeutung von "`Erfahrungseinbrüchen"' für
die Philosophie erheblich überschätzt hat. Deshalb ist es auch ein
zweifelhaftes Unterfangen, die Philosophie an den vermeintlich zu grunde
liegenden inneren Erlebnissen messen zu wollen, zumal dies die Gefahr birgt,
daß dann in letzter Konsequenz die Heftigkeit und Leidenschaftlichkeit
des Denkens mehr wiegen als die Qualität der Argumente. Eingeräumt werden muß
allerdings, daß die Tiefe einer philosophischen Untersuchung wahrscheinlich
auch durch die Intensität der zur Philosophie motivierenden Erfahrungen
mitbestimmt ist, nur ist die Intensität der Motivation nicht der
Bewertungsmaßstab für philosophische Werke. Unter den zur Philosophie
motivierenden Erlebnissen dürften für die meisten Menschen dabei wohl die
"`Erfahrungseinbrüche"' der Jugend eine größere Rolle spielen als die der
Kindheit. Aber Voegelins "`anamnetische Experimente"' sind vermutlich eher als
Beispiele zu verstehen, denn als eine vollzählige Auflistung.

Abgesehen davon bleiben Voegelins "`anamnetische Experimente"' ein wenig
hinter den Erwartungen zurück, die durch seine vorangegangenen Ausführungen
geweckt werden. In den vorangegangenen beiden Abhandlungen kommt der
mystischen Erfahrung des welttranszendenten Seinsgrundes eine zentrale
Bedeutung zu. Für Voegelins These etwa, daß wir das Sein nur verstehen könnten
als einen im welttranszendenten Seinsgrund entspringenden Prozeß, ist diese
Erfahrung eine unabdingbare Voraussetzung. Damit diese These schlüssig wird,
müßte es auch tatsächlich der Seinsgrund sein, der sich in dieser Erfahrung
zeigt. Sollte es sich bloß um irgend ein überwältigendes Meditationserlebnis
handeln, welches dann aus lauter Begeisterung eine Erfahrung vom Seinsgrund
genannt wird, so wäre die These noch unzureichend begründet, denn der Prozeß
des Seins kann sicherlich nicht einer Erfahrung im Bewußtsein entspringen.
Voegelins "`anamnetische Experimente"' bilden einen der wenigen Anlässe für
Voegelin, von eigenen Erfahrungen zu berichten.  Ein glaubhaftes und
unzweideutiges Transzendenzerlebnis fördern Voegelins "`anamnetische
Experimente"' jedoch nicht zu Tage. Zwar deutet Voegelin in seinen
Einleitenden Bemerkungen zu den anamnetischen Experimenten noch an, daß die
Bewußtseinstranszendenz, die in "`finiter Erfahrung"' in die Welt hinein
führt, nur eine Art von Transzendenz sei, und er führt unter den verschiedenen
Transzendenzerfahrungen, die in der Biographie des Bewußtseins schon lange vor
dem Einsetzen der philosophischen Besinnung vorgegeben sind, auch die
Erfahrung der Transzendenz in den Seinsgrund auf,\footnote{Vgl.  Voegelin,
  Anamnesis, S.61.} aber in den Kindheitserlebnissen läßt sich dann nichts
mehr davon wiederfinden. Dort ist zwar von dem Schlaraffenland und auch von
einer Wolkenburg die Rede, aber einen Hinweis auf irgend etwas, was auch nur
annähernd als transzendente Seinsphäre oder gar als der Grund allen Seins
gelten könnte, sucht man vergebens.

Endlich gibt es noch einen weiteren, mehr psychologischen Grund, der das
Unterfangen, in Kindheitserinngerungen die "`Erregunsquellen"' ausfindig zu
machen, aus denen es im Erwachsenenleben "`zu weiterer philosophischer
Besinnung treibt"',\footnote{Voegelin, Anamnesis, S.62.} fragwürdig erscheinen
läßt. Wenn wir im Erwachsenenalter rückblickend unsere Kindheit betrachten und
uns dabei außerdem noch auf der Suche nach unseren eigenen geistigen Ursprüngen
befinden, dann läßt es sich nicht immer vermeiden, daß wir in unsere Erinnerung
etwas hineininterpretieren, was ursprünglich gar nicht vorhanden war. Die
Gefahr einer Selbstmystifikation ist bei "`anamnetischen Experimenten"' nur
schwer zu umgehen. Es besteht hier übrigens eine Analogie zu jener größeren
historischen "`Anamnese"' der Wiedererweckung verschollenen Ordnungswissens
aus den Quellen der antiken Philosophie und mittelalterlichen Theologie, wo
bei Voegelin ebenfalls gelegentlich der Eindruck entstehen könnte, daß eine
durch und durch moderne existentialistische Philosophie dem Denken der Alten
aufgestülpt wird.

Voegelins Programm der "`Anamnese"' scheitert im Ganzen also aus drei Gründen:
Erstens ist die Genealogie eines Gedankens für den Gedanken selbst, d.h. für
seinen Inhalt, seine Richtigkeit oder Falschheit, bedeutungslos.
Zweitens führt das Verfahren der "`Anamnese"' mit großer Wahrscheinlichkeit zu
einer Verfälschung der Genealogie. Drittens gelangt man auf diesem Wege
ebensowenig zu jener vermeintlich vorhandenen Transzendenz wie durch die
philosophische Meditation.


%%% Local Variables: 
%%% mode: latex
%%% TeX-master: "Main"
%%% End: 












