%%This is a very basic article template.
%%There is just one section and two subsections.
\documentclass[12pt, a4paper, german]{article}
\usepackage[utf8x]{inputenc}
\usepackage{ucs} % unicode
\usepackage[T1]{fontenc}
\usepackage{t1enc}
\usepackage{type1cm}
\usepackage[german]{babel}

\usepackage{eurosym}
\usepackage{amsmath, amssymb}
\usepackage{graphicx}
\usepackage{natbib}
\usepackage{rotating}

\sloppy
 
\begin{document}
\title{Übungsaufgaben zum Satz von Arrow und zu Lotterien}

\author{Eckhart Arnold}
\date{16. Juli 2009}

\maketitle

\section{Aufgabe 1}

Gegeben sei eine Menge von Individuen $A, B, C, \ldots$ und eine Menge von
Alternativen $x, y, z, \ldots$. Angenommen es gibt zu jedem Paar von Alternativen
ein einzelnes Individuum, das in beide Richtungen {\em vollständig
entscheidend} für dieses Paar von Alternativen ist (d.h. wenn $x,y$ ein Paar von
Alternativen und das $i$-te Individuum dasjenige ist, welches für dieses Paar
vollständig entscheidend ist, dann gilt wann immer $x \succ_i y$ (Präferenz des
$i$-ten Individuums) so auch $x \succ_K y$ (kollektive Präferenz) und wann immer
$x \prec_i y$ dann auch $x \prec_K y$).

{\bf Zeige}, dass dann gilt, dass es ein Individuum gibt, dass vollständig
entscheidend für alle Alternativen ist. Vorausgesetzt werden dürfen außer der
Annahme die Voraussetzungen für den Satz von Arrow (Pareto-Effizienz,
unbeschränkter Bereich, paarweise Unabhängigkeit) sowie die Eigenschaften
wohlgeordneter Präferenzen, insbesondere die {\em Transitivität}.

\section{Aufgabe 2}

Angenommen, dass $i$-te Individuum sei {\em beinahe entscheidend} für $x$
über $y$, d.h. wann immer $x \succ_i y$ und für alle $n \neq i$ umgekehrt $y
\succ_n x$ gilt, dann gilt für die kollektiven Präferenzen $x \succ_K y$.

{\bf Zeige}, dass dann das $i$-te Individuum {\em vollständig entscheidend} für 
$x$ über $z$ ist, wobei $z$ eine beliebige Alternative außer $x$ und $y$ ist.

\section{Aufgabe 3}

Gegeben sei folgendes System zur Bestimmung kollektiver Präferenzen über einer
Menge von Alternativen: Jedes Individuum schreibt die beiden bevorzugten
Alternativen auf einen Zettel. Die kollektive Präferenzordnung der Alternativen
ergibt sich dann aus der Häufigkeit der Nennungen, d.h. die Alternative, die am
häufigsten genannt wurde, kommt an die erste Stelle, die am zweithäufigsten
genannte an die zweite usf. 

{\bf Zeige}, dass dieses Wahlsystem einer der drei folgenden Bedingungen nicht
genügt:
\begin{enumerate}
  \item Transitivität der kollektiven Präferenzen
  \item Pareto-Effizienz
  \item Paarweise Unabhängigkeit
\end{enumerate}

{\em Diskussionsfrage (nicht für die Klausur): Wie könnte jemand das Wissen um
diese Eigenschaft des gegebenen Wahlverfahrens nutzen, um z.B. die
Jahrgangsstufensprecherwahl zu manipulieren? Handelt es sich dabei um ein
ernstzunehmendes Problem? Was meinen Sie?}

\section{Aufgabe 4}

Angenommen $B$ sei ein bestes Gut einer Menge von Gütern und $W$ ein
schlechtestes Gut, und auf den Lotterien über diese Menge von Gütern sei eine
Nutzenfunktion $u$ so definiert, dass $u(x) := a$ für diejenige Zahl $a$, für
die gilt $x \sim L(a,B,W)$. 

{\bf Zeige}: Die so definierte Nutzenfunktion hat die
Erwartungsnutzeneigenschaft, d.h $u(L(a,x,y)) = a\cdot u(x) + (1-a)\cdot u(y)$.
(Die Reduzierbarkeitsbedingung und das
Substitutionsgesetz für Lotterien dürfen Sie voraussetzen.)


\end{document}

