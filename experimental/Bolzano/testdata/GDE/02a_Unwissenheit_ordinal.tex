\section{Entscheidungen unter Unwissenheit I}

In dieser und der folgenden Woche werden wir uns mit Entscheidungen unter
Unwissen beschäftigen. Entscheidungen unter Unwissen sind
Entscheidungen, bei denen wir nicht wissen mit welcher Wahrscheinlichkeit
bestimmte Ereignisse (bzw. "`Welt-Zustände"') eintreten können, bei denen wir
aber immer noch eine klare Vorstellung davon haben, mit welchen Ereignissen
als Bedingungen unserer Entscheidungen und mit welchen Ergebnissen als
Resultaten der Entscheidungen überhaupt zu rechnen ist. Entscheidungen unter
Unwissen sind zu unterscheiden von Entscheidungen unter "`vollständiger
Unwissenheit"' einerseits, bei denen wir nicht einmal mehr mit Sicherheit
angeben können, zu welchen möglichen Resultaten unsere Handlungen führen
können, und von "`Entscheidungen unter Risiko"' andererseits, bei denen wir
zusätzlich Aussagen über Wahrscheinlichkeit der in Betracht zu ziehenden
Ereignisse machen können. 

Naturgemäß bieten Entscheidungen unter Risiko, bei denen wir
Wahr\-schein\-lich\-keiten angeben können, die besten Angriffspunkte für eine
formale Theorie des Entscheidens. Aber auch Entscheidungen unter Unwissenheit
sind bis zu einem gewissen Grade einer formalen Behandlung zugänglich, und weil
dabei die Wahrscheinlichkeitstheorie nicht erforderlich ist, handelt es sich
technisch gesehen sogar um den einfacheren Teil der Entscheidungstheorie, weshalb
wir diesen Teil auch zuerst besprechen.

\subsection{Die einfachste Entscheidungsregel: Das Prinzip der Dominanz}
\label{DominanzPrinzip}
Bisher haben wir nur über die Darstellung von Entscheidungsproblemen in Form von
Entscheidungsbäumen und -tabellen gesprochen. Wie kann man aber nun (mit Hilfe
von Bäumen oder Tabellen) Entscheidungsprobleme lösen? Ein besonders
offensichtliches Prinzip, das bei der Lösung von Entscheidungsproblemen eine
Rolle spielt, ist das {\em Prinzip der Dominanz}. Betrachten wir dazu noch
einmal die eingangs vorgestellte Entscheidungstabelle:

\begin{center}
\begin{tabular}{cc|c|c|}
& \multicolumn{1}{c}{} & \multicolumn{2}{c}{{\bf Zustand}} \\
&           & schwere Klausur   & leichte Klausur    \\ \cline{2-4}
& lernen    & {\em bestehen}    &  {\em bestehen}    \\ \cline{2-4}
\raisebox{1.5ex}[-1.5ex]{{\bf Handlung}} 
& faulenzen & {\em durchfallen} &  {\em bestehen} \\
\cline{2-4}
\end{tabular}
\end{center}

Man sieht anhand der Tabelle sofort, dass es auf jeden Fall besser wäre zu
lernen als zu faulenzen, denn in dem Fall, dass die Klausur schwer ist,
erzielt man durch Lernen ein besseres Ergebnis und in dem Fall, dass sie leicht
wird, ist das Ergebnis wenigstens nicht schlechter als wenn man nicht lernt.
Das bei dieser Überlegung implizit zu Grunde gelegte Entscheidungsprinzip kann
man folgendermaßen formulieren.

\begin{quotation}
\marginline{schwache Dominanz}
{\em Prinzip der schwachen Dominanz}: Wenn eine Handlung unter
allen Umständen zu einem mindestens gleichguten Ergebnis führt wie alle anderen
Alternativen und in mindestens einem möglichen Fall zu einem besseren Ergebnis,
dann wähle diese Handlung.
\end{quotation}

\marginline{starke Dominanz}
Analog zu dem Prinzip der schwachen Dominanz kann man auch ein Prinzip der
starken Dominanz aufstellen, bei dem gefordert wird, dass die zu wählende
Handlung unter allen Umständen zu einem eindeutig besseren Ergebnis führt als
sämtliche verfügbaren Alternativen. An dieser Stelle ist die Unterscheidung
zwischen schwacher Dominanz und starker Dominanz noch nicht besonders
wichtig. Der Begriff der starken Dominanz könnte sogar verzichtbar
erscheinen. Allerdings spielt diese Unterscheidung spätestens bei der Suche nach
geeigneten Lösungsstrategien in der Spieltheorie wieder eine wichtige Rolle und
wird uns dort noch beschäftigen.

\marginline{Mögliche Fehlschlüsse}
Das Prinzip der schwachen Dominanz erscheint so einfach und eindeutig, dass man
nicht vermuten sollte, dass es bei seiner Anwendung irgendwelche
Schwierigkeiten auftreten könnten. Dass das nicht unbedingt stimmen muss, kann
das folgende Beispiel verdeutlichen: Angenommen, Sie betreten ein Wettbüro, in
dem Sportwetten für die Sportarten Fussball und Tennis angeboten werden. Der
Einsatz beträgt in jedem Fall 2 Euro, aber da sehr viel weniger Leute an
Tennis interessiert sind als an Fussball, können Sie bei einer Tenniswette
höchstens € 10.000 gewinnen, während bei einer Fußballwette satte 
€ 50.000 drin sind. Ihre Entscheidungstabelle würde als folgendermaßen
aussehen:

\begin{center}
\begin{tabular}{c|c|c|}
\multicolumn{1}{c}{ } & \multicolumn{1}{c}{Wette gewinnt} &
\multicolumn{1}{c}{Wette verliert} \\ \cline{2-3}
Tenniswette   & €  9.998    & € -2         \\ \cline{2-3}
Fussballwette & € 49.998    & € -2         \\ \cline{2-3}
\end{tabular}
\end{center}

Wollte man in dieser Situation auf das Prinzip der Dominanz zurückgreifen, dann
müsste man sich eigentlich ganz klar für die Fussballwette entscheiden. Warum
könnte das aber ein Trugschluss sein? Der Grund ist folgender: Es ist höchst
wahrscheinlich, dass die Gewinnchancen bei beiden Wetten sehr unterschiedlich
verteilt sind. Werden einem zwei solche Wetten angeboten, dann ist davon
auszugehen, dass die Gewinnchancen bei der Fussballwette sehr viel geringer
sind als bei der Tenniswette. Je nachdem, um wieviel sie geringer sind, könnte
es sein, dass die Tenniswette sogar aussichtsreicher ist als die Fussballwette.
(Was "`aussichtsreicher"' dabei exakt heisst, werden wir noch genau definieren,
wenn wir Entscheidungen unter Risiko besprechen.) Wenn man so will, besteht der
"`Denkfehler"' bei diesem Beispiel also darin, dass die Problemspezifikation
unvollkommen war, indem wichtige Hintergrundinformationen über die Natur dieses
speziellen Entscheidungsproblem, \marginline{Handlungs\-ab\-hängige
Wahr\-schein\-lich\-keiten} nämlich die Handlungs\-abhängigkeit der
Eintrittswahrscheinlichkeiten der Ereignisse, bei der Formalisierung in
Tabellenform "`vergessen"' wurden.

% An diesem Beispiel wird zugleich deutlich, dass das Prinzip der Dominanz nur
% dann sinnvoll angewandt werden kann, wenn die Chancen für das Eintreten der
% Zufallsereignisse entweder gleichverteilt sind, oder wenn wir zumindest
% keinerlei Wissen darüber haben, wie sie verteilt sein könnten.\footnote{Die
% Anwendung des Dominanzprinzips im letzteren Fall ist aber umstritten. Siehe dazu auch,
% was weiter unten zum "`Prinzip der Indifferenz"' (Abschnitt
% \ref{Indifferenzprinzip}) gesagt wird.}

Daneben gibt es aber noch ein weiteres denkbares Problem, wie das folgende, mit
leichten Abwandlungen aus Resniks Buch \cite[S.9 ]{resnik:1987} übernommene
Beispiel verdeutlicht. Das Beispiel gibt stark vereinfacht die strategische
Problematik der Aufrüstung im kalten Krieg wieder:

\begin{center}
\begin{tabular}{c|c|c|}
\multicolumn{1}{c}{ } & \multicolumn{1}{c}{Krieg}   & 
\multicolumn{1}{c}{Frieden}  \\ \cline{2-3} 
Aufrüsten & "`Tot"' & hohe Militärausgaben  \\ \cline{2-3} 
Abrüsten  & "`Rot"' & "`Friedensdividende"' \\ \cline{2-3}
\end{tabular}
\end{center}

Nimmt man einmal an, dass es besser ist, sich zum Kommunismus bekehren zu lassen
als zu sterben, dann müsste man nach dem Prinzip der Dominanz eigentlich
Handlungsalternative "`Abrüsten"' eindeutig vorziehen, denn unabhängig davon, ob
es Krieg oder Frieden gibt, erzielt man mit der Entscheidung zugunsten der
Abrüstung in beiden Fällen das jeweils bessere Ergebnis. Wo ist der Haken an
dieser Argumentation? Der "`Haken"' besteht darin, dass das Eintreten der
Zustände "`Krieg"' oder "`Frieden"' nicht unabhängig davon ist, welche Handlung
gewählt wird.\marginline{Strategische Interaktion} Zumindest nach Ansicht von
Aufrüstungsbefürwortern hätte damals eine zu weit gehende Abrüstung die Gefahr
eines Überfalls durch die Ostblockstaaten drastisch erhöht. Stimmt man dem zu,
dann ist es keineswegs mehr so eindeutig, dass Abrüsten die bessere Wahl ist.

Dieses Beispiel zeigt, dass es noch eine weitere stillschweigende Voraussetzungen
für die Anwendung des Prinzips der Dominanz (wie sowie übrigens auch anderer
Entscheidungsregeln) gibt, nämlich die Unabhängigkeit der "`Zufallsereignisse"'
bzw. der Weltzustände von den getroffenen Entscheidungen. In dem angeführten
Beispiel ist eine solche Unabhängigkeit nicht gegeben, da wir es mit einem
Gegenspieler zu tun haben, der auf unsere Entscheidungen reagiert. Strengenommen
haben wir es daher gar nicht mehr mit einem reinen Entscheidungsproblem zu tun,
sondern mit einem Problem strategischer Interaktion, das bereits in das Gebiet
der Spieltheorie fällt.

\subsection{Präferenzen}
\label{Praeferenzen}
In der letzten Vorlesungsstunde wurde als Beispiel für ein mögliches
Entscheidungsproblem, bei dem uns die Entscheidungstheorie {\em nicht}
weiterhelfen kann, die Frage angeführt, ob der nächste Urlaub lieber in den
Bergen oder an der See gebucht werden sollte. Der Grund, weshalb uns die
Entscheidungstheorie hier nicht weiterhelfen kann, besteht darin, dass es bei
diesem Entscheidungsproblem noch darum geht, wie die verschiedenen Ergebnisse
der Entscheidung zu bewerten sind. Grundsätzlich setzt die Entscheidungstheorie
voraus, dass wir uns über die Bewertung der möglichen Ergebnisse, sprich über
unsere {\em Präferenzen} schon im Klaren sind. Im Folgenden ist daher zunächst
einiges über Präferenzen zu sagen, insbesondere welche Anforderungen an die
Präferenzen gestellt werden müssen, damit sie im Sinne der Entscheidungstheorie
wohlgeformt sind. 

Unter {\em Präferenz} ist im Zusammenhang der Entscheidungstheorie eine Relation
zu verstehen, die festlegt, wann ein mögliches Resultat\footnote{Die {\em
Resultate} eines Entscheidungsprzesses sind nicht zu verwechseln mit der
Entscheidung selbst. Das Resultat ist vielmehr das, was bei einer Entscheidung
heraus kommt, die Entscheidung selbst ist die Wahl, die man trifft, um dann ggf.
ein bestimmtes Resultat zu erzielen. Die Präferenzen, von denen hier die Rede ist
beziehen sich zunächst auf die Resultate, auch wenn man im übertragenen Sinne
ebenfalls davon sprechen könnten, dass eine Entscheidung einer anderen {\em
vorgezogen} wird, weil man sich von ihr ein besseres Resultat erhofft.} eines
Entscheidungsprozesses einem anderen vorgezogen wird. (Da wir es mit
Entscheidungsproblemen zu tun haben, bezieht sich unsere Präferenzrelation auf
die möglichen Resultate von Entscheidungsprozessen. In der Ökonomie würde man die
Präferenzrelation dagegen eher auf der Menge möglicher "`Güterbündel"' oder
dergleichen definieren. Der Einfachtheit halber wird daher im Folgenden auch oft
von "`Gütern"' anstelle von "`Resultaten"' oder "`Ergebnissen"' die Rede sein.)
Wenn $x$ und $y$ zwei mögliche Resultate eines Entscheidungsprozesses sind, dann
schreiben wir $x \succ y$, um auszudrücken, dass $x$ gegenüber $y$ vorgezogen
wird. Und wir schreiben $x \sim y$, wenn $x$ und $y$ gleich gut bewertet werden
bzw. wenn diejenige Person, die die Entscheidung trifft, zwischen $x$ und $y$
{\em indifferent} ist. Eine wohlgeformte Präferenzrelation muss folgende
fundamentale Eigenschaften erfüllen:

\marginline{Eigenschaften der Präferenzrelation}
\begin{enumerate}
\label{Ordnungsaxiome}
\item {\em Antisymmetrie:} Wenn $x \succ y$, dann nicht $y \succ x$ und auch
nicht $x \sim y$
\item {\em Zusammenhang:} Für jedes Paar $x, y$ aus der
Menge der möglichen Resultate gilt entweder $x \succ y$ oder $y \succ x$ 
oder $x \sim y$
\item {\em Transitivität:} Wenn $x \succ y$ und $y \succ z$, dann auch $x
\succ z$. (In analoger Weise gilt: $x \sim y \wedge y \sim z \Rightarrow x
\sim z$, sowie weiterhin: $x \sim y \wedge y \succ z \Rightarrow x \succ z$ und:
$x \succ y \wedge y \sim z \Rightarrow x \succ z$)
\end{enumerate}

\marginline{Recht\-fertigungs\-problem des Präferenz\-konzepts}
Mit welchem Recht können wir fordern, dass eine Präferenzrelation diese
Eigenschaften erfüllen muss? Man kann diese Frage von zwei Seiten aus betrachten:
1) von der Seite des entscheidungstheoretischen Formalismus aus und 2) von der
empirischen und normativen Seite aus. Von der Seite des
entscheidungstheoretischen Formalismus stellt sich die Situation so dar, dass
z.B. bestimmte Lösungsverfahren nur dann tatsächlich richtige (d.h. die
Präferenzen optimal erfüllende) Entscheidungen liefern, wenn die
Präferenzrelation in dem oben beschriebenen Sinne wohlgeformt ist; und zwar schon
deshalb, weil die entsprechenden Lösungsverfahren unter genau dieser
Voraussetzung entwickelt worden sind. Anderseits gilt aber auch, dass die
Entscheidungstheorie beansprucht unser Handeln beschreiben (empirische Anwendung
der Entscheidungstheorie) und richtig anleiten (normative Anwendung der
Entscheidungstheorie) zu können. Dann sollten diese Eigenschaften auch den
Eigenschaften von Präferenzen von Menschen in empirischen
Entscheidungssituationen mehr oder weniger entsprechen.\footnote{Insgesamt haben
wir es hier mit drei Perspektiven auf die Entscheidungstheorie zu tun: 1. der
logischen; 2. der empirischen; 3. der normativen. Häufig wird nur zwischen den
letzteren beiden unterschieden. Dabei wird dann in der Regel eingeräumt, dass die
Entscheidungstheorie zwar das empirisch beobachtbare Verhalten von Menschen nicht
richtig beschreibt. Aber meistens wird dennoch darauf bestanden, dass sie in
normativer Hinsicht dennoch zu richtigen Entscheidungen anleitet. Das stimmt
insofern, als die normative Anwendung vergelichsweise schwächere
erkenntnistheoretische Rechtfertigungsprobleme aufwirft als die empirische, aber
auch die normative Anwendung beruht immer noch auf bestimmten empirischen
Voraussetzungen, wie z.B. der, dass wohlgeformte Präferenzrelationen die
empirischen Phänomene der Präferenz (d.i. des Vorziehens, des Beabsichtigens, des
Wertschätzens etc.) halbwegs richtig erfassen. Vgl. dazu die klassische
Darstellung von Savage \cite[S. 7ff.]{savage:1954}, der hinsichtlich solcher
subtiler Unterscheidungen im Übrigen sehr umsichtig und genau verfährt.} Kann
man das ungeprüft voraussetzen? Wenigstens bei den Eigenschaften der {\em
Transitivität} und des {\em Zusammenhangs} sind in dieser Hinsicht erhebliche
Abstriche zu machen.

Zur {\em Tansitivität}: Wie könnte man zunächst einmal die Eigenschaft der
Transitivität rechtfertigen?
\marginline{Geldpumpen\-argument}\label{Geldpumpenargument} Ein beliebtes
Argument zur Rechtfertigung dieser Eigenschaft ist das sogenannte {\em
Geldpumpenargument}. Angenommen, es gibt jemanden, dessen Präferenzen nicht
transitiv sind. Dann gibt es drei Weltzustände (bzw. "`Resultate"' oder
"`Güterbündel"') $a, b, c$, für die für diese Person gilt: $a \prec b \prec c
\prec a$. Wenn diese Person aber b gegen über a vorzieht, so bedeutet dass (wie
die Ökonomen glauben), dass sie gegebenenfalls bereit wäre, für den Übergang von
a zu b einen bestimmten Geldbetrag zu zahlen. Dann wäre sie aber wiederum bereit
einen Geldbetrag für den Übergang von b zu c bezahlen. Ist sie aber erst einmal
bei c angekommen, dann würde sie wegen $c \prec a$ nochmals bereit sein für den
Übergang zu a in die Tasche zu greifen, und das ganze Spiel fängt von vorne an
und könnte beliebig oft wiederholt werden. Die Überlegung zeigt, dass
intransitive Präferenzen in gewisser Weise unplausibel bzw. inkonsequent sind.

\marginline{Beispiel für sinnvolle intransitive Präferenzen}
\label{intransitivePraeferenzen}
Allerdings gibt es ebenso Beispiele dafür, dass Präferenzen auf ganz natürliche
Wese transitiv sein können, z.B. das folgende \cite[S. 20]{delong:1991}: Frau
Schmidt möchte einen Schachcomputer kaufen. Es gibt drei Modelle, A, B und C.
Einem Testbericht kann sie entnehmen, dass Modell A in einem Probespiel Modell B
geschlagen hat. Modell B hat wiederum Modell C geschlagen, aber Modell C hat
Modell A geschlagen. (Man kann sich überlegen, dass diese Situation sehr wohl
möglich ist, denn es ist denkbar, dass der Algorithmus von Modell A mit dem von
Modell B sehr gut "`klarkommt"', aber nicht mit dem von Modell C, auch wenn
Modell C schlechter als Modell B ist.) Die einzig sinnvollen Präferenzen, die
Frau Schmidt in Bezug auf die Schachcomputer haben kann, sind in diesem Fall
intransitiv, nämlich $A \succ B \succ C \succ A$. Man kann leicht andere
Beispiele dieser Art konstruieren. Der Grund für die, in diesem Fall, sinnvolle
Intransitivität von Präferenzen liegt darin, dass sich unsere Präferenzen
häufig an objektive Beziehungen wie "`stärker als"', "`besser als"',
"`Sieger über"' etc. knüpfen, die ihrerseits oftmals nicht transitiv sind. (So
ist ja auch z.B. von der Fussballmanschaft an der Spitze der Liga keineswegs
gesagt, dass sie alle anderen Mannschaften besiegt oder mindestens
unentschieden gespielt hat.) 

Wenn es aber sinnvolle transitive Präferen gibt, was wird dann aus dem
Geldpumpenargument, könnte man nun fragen. Die Antwort darauf ist, dass man dann,
wenn intransitive Präferenzen auftreten, verschiedene Mechanismus anwenden kann,
um mit den möglicherweise daraus resultierenden Problemen fertig zu werden. In
dem Beispiel von eben könnte Frau Schmidt sich einfach beliebig für irgendeinen
der Schachcomputer entscheiden oder ein Los werfen. (Dass es einem raffinierten
Verkäufer tatsächlich gelingen könnte, eine Geldpumpe aus ihr zu machen, ist
wohl eher unrealistisch\ldots\footnote{Rabin und Thaler formulieren es sehr treffend:
"`It does not seem to us obvious that if you can take some of a fool’s money from
him some of the time then you can take all of his money all of the
time"'.\cite[S. 227]{rabin-thaler:2001} Den Hinweis auf den Artikel von Rabin und
Thaler verdanke ich Matthias Brinkmann.})

\marginline{Inkonsistenz bei kollektiven Präferenzen}
Ganz besonders stellt sich das Problem intransitiver oder, ganz allgemein
gesprochen, inkonsistenter Präferenzn im Zusammenhang von Kollektivpräferenzen
(d.h. den gemeinsamen Präferenzen eines Kollektivs von Menschen). Wie wir gesehen
haben, sind schon die Präferenzen einzelner Menschen nicht immer transitiv
geordnet (und zwar nicht bloß auf Grund von Inkonsequenz oder menschlicher
Unvollkommenheit, sondern weil es manchmal durchaus Sinn hat, wenn Präferenzen
nicht transitiv sind!). Diese Situation tritt nocht viel leichter auf, wenn wir
vor dem Problem stehen, aus den Einzelpräferenzen einer Vielzahl von Individuuen
eine sinnvolle kollektive Präferenz abzuleiten. Denn dazu müsste irgendein
geeigneter Abstimmungsmechanismus vorhanden sein, der es erlaubt aus den
vielfältigen und möglicherweise höchst disparaten Interessen der Einzelnen eine
gemeinsame Zielvorstellung zu bilden. Es gehört nun aber zu den interessantesten
Theoremen der Social-Choice Theory, die unter Stichworten wie "`Paradox des
Liberalismus"' und "`Satz von Arrow"' bekannt geworden sind, dass einen solchen
Abstimmungsmechanismus zu finden nicht immer leicht und manchmal sogar unmöglich
ist, sofern bestimmte Anforderungen an die Fairness und die Vernunft eines
solchen Abstimmungsmechanismus gestellt werden. (Inwiefern diese Anforderungen
notwendig sind oder variiert werden können, so dass die "`Probleme"' nicht mehr
auftreten, ist dann Gegenstand der Diskussion.) Wir werden auf diese Theoreme im
Laufe dieses Semesters noch ausführlich eingehen (Kapitel
\ref{LiberalismusParadox} und \ref{SatzVonArrow} dieses Skripts).
 
Eine weitere Einschränkung der Gültigkeit der Annahme transitiver Präferenzen
ergibt sich aus folgender Überlegung \cite[p. 23/24]{resnik:1987}:
\marginline{Grenzen der Transitivität bei marginalen Präferenzunterschieden}
 Man stelle sich zwei Tassen Kaffe vor, eine ohne Zucker und eine, die eine sehr
 kleine Menge Zucker enthält, gerade so viel, dass man den Zucker beim Trinken
 noch nicht
bemerkt. Jemand, der entscheiden sollte, welche Tasse Kaffee er vorzieht, würde
also indifferent zwischen diesen beiden Kaffeetassen sein, auch wenn er
vielleicht gezuckerten Kaffee bevorzugt. Nun denken wir uns eine dritte
Kaffeetasse, die wiederum ein klein wenig mehr Zucker enthält als die zweite,
aber nicht so viel mehr, als dass man den Unterschied bemerken könnte. Dann, eine
vierte Kaffeetasse, die sich wiederum von der dritten durch einen nur marginal
größeren Zuckergehalt unterscheidet usw. Irgendwann haben wir dann eine
Kaffeetasse, die soviel Zucker enthält, dass sich der Geschmack von dem der
allerersten Kaffeetasse deutlich unterscheidet. Dann würde jemand, der
gezuckerten Kaffee bevorzugt, diese letzte Tasse unseres Gedankenexperiments der
ersten Tasse sicherlich vorziehen, was aber im Widerspruch zur Transitivität der
Indifferenzbeziehung steht. Das Gedankenexperiment ist zudem so konstruiert, dass
es sich in diesem Fall nicht um ein Beispiel von Inkonsequenz oder Irrationalität
handelt, sondern dass sich die Transitivität der Präferenzrelation "`beim besten
Willen"' nicht aufrecht erhalten lässt. Wenn wir das Gedankenexperiment als
glaubhaft ansehen, dann bleibt uns nichts weiter übrig als zuzugestehen, dass wir
in der Wirklichkeit nicht immer von transitiven Präferenzen ausgehen können, und
dass die Präferenzrelation, so wie sie hier definiert ist, lediglich eine bessere
oder manchmal auch schlechtere Annhährung an die Wirklichkeit darstellt. Man kann
bereits an dieser Stelle antizipieren, dass unsere Modelle und Theorien
spätestens dann in Schwierigkeiten geraten, wenn sie irgendwann einmal, und
möglicherweise völlig unbemerkt (!), innerhalb einer komplizierten mathematischen
Beweisführung allzu starke Anforderungen an die Gültigkeit von
Indifferenzbeziehungen stellen.\footnote{An diesem Problem leidet ganz wesentlich
die mathematische Rückführung kardinaler auf ordinale Präferenzen, die in Kapitel
\ref{NeumannMorgenstern} und
\ref{DiskussionNeumannMorgenstern} vorgestellt und diskutiert wird.}
 
Der tiefere Grund für das eben beschriebene Problem besteht darin, dass
Relationen vom Typ "`{\em ungefähr} gleich wie"' im Gegensatz zu Relationen vom
Typ "`gleich wie"' nicht (vollkommen) transitiv sind. Da wir es in der Empirie
aber schon auf Grund von Messungenauigkeiten fast immer mit dem ersteren Typ zu
tun haben, kann das zu Problemen führen, wenn man vollständige (d.h. über eine
beliebig große Anzahl von Zwischengliedern erhalten bleibende) Transitivität
voraussetzt.

\marginline{Grenzen des Zusammenhangs von Präferenzen}
Neben der Transitivität, lässt sich aber auch in Zweifel ziehen, ob man stets
davon ausgehen kann, dass unsere Präferenzen {\em zusammenhängend} sind.
Zumindest wenn wir eine größere und nicht mehr ohne Weiteres überschaubare Menge
von Gütern (oder möglichen Entscheidungsresultaten) betrachten, kann man sich
leicht vorstellen, dass es nicht mehr so ohne Weiteres möglich ist, von jedem
Paar aus dieser Menge eindeutig zu sagen, welche der Relationen $\succ$, $\prec$
oder $\sim$ zwischen den beiden Gliedern des Paars besteht. Einige Autoren wie
z.B. \cite{kaplan:1996}, die die Voraussetzung durchgehend zusammenhängender
Präferenzen für allzu artifiziell halten, führen deshalb neben der Beziehung der
{\em Indifferenz}, die besteht, wenn wir zwei Güter gleich hoch schätzen,
eine davon deutlich zu unterscheidende Beziehung der {\em Unentschiedenheit} oder
auch "`Unentschlossenheit"' ein, die dann besteht, wenn wir nicht sicher sind, ob
wir eine Sache einer anderen vorziehen oder nicht, was ja etwas anderes ist, als
wenn wir eine Sache als genauso gut bewerten wie eine andere.
\marginline{Unterschied von Indifferenz und Unentschiedenheit}
 Dieser Unterschied
ist recht subtil, denn man kann sowohl hinsichtlich der Indifferenz als auch
hinsichtlich der Unentschiedenheit mit Recht sagen, dass wir {\em weder} den
einen {\em noch} den anderen der beiden Gegenstände, zwischen denen wir
indifferent bzw. unentschieden sind, dem anderen vorziehen. Trotzdem ist es noch
etwas anderes, wenn wir es deshalb nicht tun, weil sie uns beide gleich lieb
sind, oder deshalb, weil wir unentschieden zwischen beiden sind.

Die Annahme, dass es so etwas wie Unentscheidenheit gibt, erscheint besonders bei
unüberschaubar großen Gegenstandsmengen oder bei solchen Gegenstandsmengen, die
Güter von sehr unterschiedlicher Art enthalten, sehr viel realistischer, denn
anderenfalls würde man voraussetzen, dass die Frage, welches von zwei Gütern man
vorzieht, oder ob man sie beide als gleichwertig beurteilt, immer schon
entschieden ist, selbst wenn wir sie uns im konkreten Fall noch gar nicht
vorgelegt haben. Aber es ist immerhin möglich, eine Entscheidungstheorie auch auf
der Grundlage zu konstruieren, dass es neben Bevorzugung und Indifferenz auch so
etwas wie Untschlossenheit gibt. In diesem Fall muss man die Forderung, dass die
Präferenzen "`zusammenhängend"' sind, zu der Eigenschaft des {\em beschränkten
Zusammenhangs} abschwächen \cite[S. 13, 24]{kaplan:1996}. Noch weiter geht der
Ansatz, die Entscheidungstheorie nicht "`präferenzbasiert"', sondern
"`wahlbasiert"' aufzubauen \cite[SEITE???]{mascolell-whinston-green:1995}.
\marginline{Präferenz\-basierter und wahlbasierter Ansatz}
Dabei wird statt einer Präferenzrelation über einer Menge
von Alternativen ({\em präferenzbasierter Ansatz}) eine Wahlfunktion definiert,
die aus Teilmengen einer Menge von Alternative die bevorzugte Alternative
innerhalb dieser Teilmenge auswählt ({\em wahlbasierter Ansatz}). Die
Formulierung der Entscheidungstheorie gestaltet sich dadurch technisch etwas
komplizierter. Wir werden im Folgenden daher nur den präferenzzentrierten Ansatz
zu Grunde legen und der Einfachheit halber davon ausgehen, dass es keine
Unentschiedenheit gibt bzw. dass alle denkbaren Unentscheidenheiten im Vorfeld
der Entscheidungsfindung geklärt worden sind. Rechtfertigen lässt sich das auf
jeden Fall solange, wie wir uns auf Anwendungsfälle nur mit sehr begrenzten und
überschaubaren Zielmengen beschränken. Zudem setzen wir eine gültige
Präferenzrelation nur jeweils {\em lokal} für das in Frage stehende
Entscheidungsproblem voraus. Wir unterstellen nicht, dass irgendjemand "`global"'
(d.h. bezüglich aller Ziele und Wünsche, die man im Leben haben kann) über
wohlgeordnete (d.h. transitive und durchgängig zusammenhängende) Präferenzen
verfügt.

\subsection{Ordinale Nutzenfunktionen}

Mit Hilfe einer Präferenzrelation kann man die Gütermenge, auf die sich die
Relation bezieht, in eine Menge von Indifferenzklassen {\em partionieren}, indem
man jeder Indifferenzklasse alle diejenigen Güter zuordnet, zwischen denen
Indifferenz herrscht. \marginline{Indifferenz\-klassen} Ist die
Präferenzlrelation wohlgeformt, dann schöpfen die Indifferenzklassen die gesamte
Gütermenge aus, und jedes Gut ist Element genau einer
Indifferenzklasse.\footnote{Ökonomen sprechen statt "`Indifferenzklassen"' auch
gerne von "`Indifferenzkurven"'. Die Indifferenzkurven erhält man, wenn man die
Indifferenzklassen grafisch darstellt.} Weiterhin induziert die Ordnung der Güter
durch die Präferenzrelation eine Ordnung auf der Menge der Indifferenzklassen.
Wir können schreiben, $I_x \succ I_y$ genau dann wenn $x \succ y$ für $x \in I_x,
y \in I_y$, wobei mit $I_x$ bzw. $I_y$ jeweils die Indifferenzklasse gemeint sein
soll, der $x$ bzw. $y$ angehört.\footnote{Man beachte, dass, wenn man die
Indifferenzklassen in dieser Weise durch die in ihnen enthaltenen Güter
identifiziert, unterschiedlich idizierte Indifferenzklassen, z.B. $I_a$,$I_b$
durchaus ein- und diesselbe Indifferenzklasse darstellen können, nämlich dann,
wenn zwischen den Gütern im Index Indifferenz herrscht, also wenn $a \sim b$.}
Aus der Konstruktion der Indifferenzklassen ergibt sich dabei, dass wenn $x \succ
y$ für ein irgend ein beliebiges $x \in I_x$ und ein beliebieges $y \in I_y$ dann
gilt $x' \succ y'$ für jedes $x' \in I_{x'}$ und jedes $y' \in I_{y'}$. Wir
können nun den Indifferenzklassen bzw. ihren Elementen Zahlen zuordnen, deren
Ordnung der Ordnung der Indifferenzklassen entspricht.
\marginline{Nutzen\-funktionen und Nutzenskalen}
Diese Zuordnung bezeichnen wir als {\em Nutzenfunktion} oder auch als {\em
Nutzenskala}, wobei die Nutzenskala jedoch strenggenommen die Zielmenge der
Nutzenfunktion ist. Eine Nutzenfunktion $u: G \mapsto \mathbb{R}$ ist also eine
Abbildung der Gütermenge $G$ auf die reellen Zahlen, für die Folgendes gelten
muss:
\begin{eqnarray}
u(x) > u(y)  \quad\mbox{genau dann wenn}\quad  x \succ y \\
u(x) = u(y)  \quad\mbox{genau dann wenn}\quad  x \sim y
\end{eqnarray}
Wichtig ist dabei, dass bei dieser Art von Nutzenfunktionen, den zugeordneten
Zahlenwerten keine andere Bedeutung zukommt als diejenige, das
Ordnungsverhältnis zwischen den Gütern auszudrücken. Man kann also z.B. sagen,
dass ein Gut x, dem eine Nutzenfunktion den Wert 4 zuordnet, nützlicher ist als
ein Gut y, dem sie den Wert 1 zuordnet. Aber es wäre falsch zu sagen, dass das
Gut x viermal so nützlich ist, wie das Gut y. Die beiden folgenden
Nutzenfunktionen drücken dementsprechend denselben Nutzen aus:

\begin{center}
\begin{tabular}{c|c|c|ccc|c|c|c}
G & x & y & z & & G & x  & y & z \\ \cline{1-4} \cline{6-9}
u & 1 & 2 & 3 & & v & -1 & 2 & 7 \\
\end{tabular}
\end{center}

Man nennt die so interpretierten Nutzenfunktionen auch {\em ordinale
Nutzenfunktionen}. Zwei ordinale Nutzenfunktionen beschreiben genau dann
denselben Nutzen, wenn sie sich durch "`ordnungserhaltende Transformationen"'
ineinander überführen lassen. Eine ordnungserhaltende oder auch "`{\em ordinale
Transformation}"' ist eine Transformation, die die Bedingung erfüllt:
\marginline{ordinale Transformationen}
\begin{eqnarray}
t(a) > t(b) \quad\mbox{genau dann wenn}\quad a > b \quad\mbox{für alle}\quad a,
c \in \mathbb{R}
\end{eqnarray}
wobei $G$ die Gütermenge und $t: \{ u(x) | x \in G\} \mapsto \mathbb{R}$ die
Transformation der Nutzenskala $u$ in eine andere Nutzenskala ist.

Mit Hilfe ordinaler Nutzenskalen lassen sich unsere Entscheidungstabellen (oder
unsere Entscheidungsbäume) in einer noch einfacheren und übersichlicheren Form
darstellen, indem wir die möglichen Resultate des Entscheidungsprozesse durch
ihre Zahlenwerte auf einer (beliebigen) Nutzenskala widergeben. Die
Entscheidungstabellen sehen dann noch einmal etwas schematischer aus, z.B. so:

\begin{center}
\begin{tabular}{c|c|c|c|c|}
\multicolumn{1}{c}{ } & \multicolumn{1}{c}{$S_1$} &
\multicolumn{1}{c}{$S_2$} & \multicolumn{1}{c}{$S_3$} & 
\multicolumn{1}{c}{$S_4$} \\ \cline{2-5} 
$A_1$ &    3  &    7  &    2  &    0 \\ \cline{2-5} 
$A_2$ &    2  &    1  &    2  &   -1  \\ \cline{2-5}
$A_3$ &    4  &    6  &    5  &    0  \\ \cline{2-5}
\end{tabular}
\end{center}

Ein Vorteil dieser Darstellung besteht darin, dass sich Entscheidungs\-regeln
besonders leicht anwenden lassen, da sich die Präferenzordnung unmittelbar an
der Größe der Zahlen ablesen lässt. In diesem Beispiel kann man beinahe
sofort "`sehen"', dass die Entscheidung $A_2$ durch beide anderen
Handlungsalternativen dominiert wird und damit sicherlich ausscheidet. Welche
der verbleibenden Alternativen gewählt werden solte, lässt sich anhand der
Dominanz allein nicht mehr entscheiden. Dafür benötigt man weitergehende
Entscheidungsregeln, denen wir uns nun zuwenden.

\subsection{Entscheidungs\-regeln auf Basis des ordinalen\\Nutzens}

Mit dem {\em ordinalen Nutzen} haben wir das Rüstzeug um einige einfache
Entscheidungsregeln zu formulieren. Für kompliziertere Entscheidungsregeln
benötigen wir stärkere Nutzenkonzepte, wie das des kardinalen Nutzens bzw. der
"`Neumann-Morgensternschen Nutzenfunktion"', die weiter unten besprochen wird
(Kapitel \ref{NeumannMorgenstern}). Im folgenden werden wir mehrere
unterschiedliche Entscheidungsregeln besprechen, die alle auf ihre Weise sinnvoll
sind, deren Anwendung aber interessanterweise zu jeweils anderen
Entscheidungsempfehlungen führt. Wenn diese Regeln aber jeweils unterschiedliche
Entscheidungsempfehlungen nahelegen, dann wirft das die Frage auf, welche dieser
Regeln denn nun eigentlich die "`richtige"' Entscheidung empfiehlt. Dazu ist zu
sagen, dass es im Bereich der "`Entscheidungen unter Unwissen"' keine unter allen
Umständen beste Regel gibt. Alle der in dieser und der nächsten Woche
besprochenen Regeln haben ihre relative Berechtigung, je nach der Situation in
der sich das Entscheidungsproblem stellt. Anders sieht die Sache erst aus, wenn
wir Entscheidungen unter Risiko betrachten. Denn dort kann man zeigen, dass mit
der {\em Erwartungsnutzenhypothese} unter wenigen Einschränkungen in der Tat so
etwas wie eine eindeutig beste Entscheidungregel vorhanden ist.

Bei den Entscheidungen unter Unwissenheit gibt es aber keine solche beste oder
einzig richtige Regel. Daher stellt sich bei jeder der folgenden
Regeln die Frage: Wann sollte man sie anwenden? Oder auch:
Warum sollte man gerade diese Regel anwenden? Die Antwort auf diese Fragen muss
zwangsläufig von der Situation und/oder von subjektiven Voraussetzungen
wie Vorlieben oder Abneigungen abhängig sein. Denn gäbe es eine generelle
Antwort, dann hätte man damit auch eine beste Regel.

\subsubsection{Die Maximin-Regel}
\label{maximinRegel}
Die erste Entscheidungsregel, die wir besprechen wollen, ist die sogenannte {\em
Maximin-Regel}, die besagt, dass man die Verluste minimieren soll, oder, was
dasselbe ist, dass man das minimale Ergebnis maximieren soll. (Eben deshalb heißt
sie "`Maximin-Regel"'.) Mit Hilfe von Entscheidungstabellen kann man die Regel
folgendermaßen anwenden: \marginline{einfache Maximinregel}
Zunächst markiert man in jeder Zeile (also für jede
Handlungsalternative) den kleinsten Nutzenwert. Und anschließend wählt man
diejenige Handlung aus, bei der markierte Wert von allen am größten ist. Das
sieht dann folgendermaßen aus:

\begin{center}
\begin{tabular}{l|c|c|c|c|}
\multicolumn{1}{c}{ } & \multicolumn{1}{c}{$S_1$} &
\multicolumn{1}{c}{$S_2$} & \multicolumn{1}{c}{$S_3$} & 
\multicolumn{1}{c}{$S_4$} \\ \cline{2-5}
$A_1$   &    3  &    4  &    7  &   1*  \\ \cline{2-5}
$A_2$   &   -6* &   12  &    2  &    2  \\ \cline{2-5}
$A_3$   &    5  &    0* &    3  &    1  \\ \cline{2-5}
$A_4$** &    2* &    4  &    3  &    2* \\ \cline{2-5}
$A_5$   &    3  &    5  &    5  &    1*  \\ \cline{2-5}
\end{tabular}
\end{center}

Die nach der Maximin-Regel beste Entscheidung ist in diesem Fall also die
Entscheidung $A_4$, weil das schlechteste mögliche Ergebnis bei dieser
Entscheidung mit einer 2 bewertet ist, während es bei allen anderen
Entscheidungen einen noch niedrigeren Wert hat. (Dass der Wert 2 dabei bei
dieser Entscheidung zweimal vorkommt, schadet nicht.)

Führt diese Entscheidungsregel immer zu einem eindeutigen Ergebnis? Nicht
unbedingt, denn es ist ja möglich dass das schlechteste mögliche Ergebnis
mehrerer Handlungsalternativen den gleichen Nutzenwert hat. Wie sollte man nun
vorgehen? Eine naheliegende Erweiterung der Maximin-Regel besagt, dass man in
diesem Fall unter den verbleibenden Handlungsalternativen nach dem
zweitschlechtesten Ergebnis auswählen soll, dann nach dem drittschlechtesten
usf. \marginline{lexikalische Maximinregel} 
Diese Erweiterung der Maximin-Regel nennt man auch die {\em lexikalische
Maximin-Regel}. Auf ein Beispiel angewandt, funktioniert das folgendermaßen:

\begin{center}
\begin{tabular}{l|c|c|c|c|}
\multicolumn{1}{c}{ } & \multicolumn{1}{c}{$S_1$} &
\multicolumn{1}{c}{$S_2$} & \multicolumn{1}{c}{$S_3$} & 
\multicolumn{1}{c}{$S_4$} \\ \cline{2-5}
$A_1$   &    2  &    4  &    1* &   6   \\ \cline{2-5}
$A_2$   &    0* &    3  &    12 &   7   \\ \cline{2-5}
$A_3$*  &    5  &    2* &    3  &   4   \\ \cline{2-5}
$A_4$   &    2  &   -1* &    7  &   1   \\ \cline{2-5}
$A_5$*  &    2* &    6  &    4  &   5   \\ \cline{2-5}
\end{tabular}
\end{center}
 
Nach dem ersten Schritt bleiben also nur noch die Entscheidungen $A_3$ und
$A_5$ übrig. Im zweiten Schritt reduzieren wir die Tabelle auf diese beiden
Strategien und ignorieren das jeweils schlechteste Ergebnis, um uns nun nach dem
zweitschlechtesten zu richten:

\begin{center}
\begin{tabular}{l|c|c|c|c|}
\multicolumn{1}{c}{ } & \multicolumn{1}{c}{$S_1$} &
\multicolumn{1}{c}{$S_2$} & \multicolumn{1}{c}{$S_3$} & 
\multicolumn{1}{c}{$S_4$} \\ \cline{2-5}
$A_3$   &    5  &    x &    3*  &   4   \\ \cline{2-5}
$A_5$** &    x  &   6  &    4*  &   5   \\ \cline{2-5}
\end{tabular}
\end{center}

Die beste Entscheidung nach der lexikalischen Minimax-Regel besteht
also in der Wahl der Handlung $A_5$. (Und wenn selbst die lexikalische
Minimax-Regel kein eindeutiges Ergebnis zu Tage fördert, dann ist es wirklich
egal, welche der verbleibenden Handlungen man wählt, oder?)

Sollte der kleineste Wert, wie in der folgenden Tabelle, mehrmals vorkommen,
dann darf er nur einmal gestrichen werden, wobei es beliebig ist, an welcher
Stelle er gestrichen wird:

\begin{center}
\begin{tabular}{l|c|c|c|}
\multicolumn{1}{c}{ } & \multicolumn{1}{c}{$S_1$} &
\multicolumn{1}{c}{$S_2$} & \multicolumn{1}{c}{$S_3$}
\\ \cline{2-4}
$A_1$   &    -1  &   2 &  100  \\ \cline{2-4}
$A_2$   &    -1  &  -1 &   3   \\ \cline{2-4}
\end{tabular}
\end{center}

Beispielsweise könnte man im ersten Schritt den Wert -1 in der zweiten Zeile
in der zweiten Spalte streichen:

\begin{center}
\begin{tabular}{l|c|c|c|}
\multicolumn{1}{c}{ } & \multicolumn{1}{c}{$S_1$} &
\multicolumn{1}{c}{$S_2$} & \multicolumn{1}{c}{$S_3$}
\\ \cline{2-4}
$A_1$   &    x  &  2  &  100  \\ \cline{2-4}
$A_2$   &    -1 &  x  &   3   \\ \cline{2-4}
\end{tabular}
\end{center}

Damit ist klar, dass die Handlung $A_1$ gewählt werden sollte, denn in der
reduzierten Tabelle ist der minimale Gewinn bei Handlung $A_1$ mit 2 größer als
bei Handlung $A_1$ mit -1.

In welchen Situationen bietet sich die Verwendung der Minimax-Regel an?
Sicherlich wird man dann auf diese Regel zurückgreifen, wenn es bei irgendeiner
Entscheidungssituation vor allem darum geht, Schäden zu vermeiden, also z.B. wenn
Leib und Leben in Gefahr geraten könnten.
\marginline{Anwendung der lexikalischen Maximin-Regel}
 Ein sehr berühmtes Beispiel für
die Anwendung der Maximin-Regel in der Philosophie hat John Rawls geliefert, der
in seiner "`Theorie der Gerechtigkeit"' fordert, dass man die Gerechtigkeit der
Gesellschaftordnung nach dem Maximin-Prinzip beurteilen soll: Diejenige
Gesellschaftsordnung ist die Gerechteste, in der es den am schlechtesten
Gestellten im Vergleich zu allen anderen möglichen und, so eine weitere
Forderung von Rawls, {\em freien} Gesellschaftsordnungen am Besten geht
\cite[S. 96ff.]{rawls:1971}. Damit setzt sich Rawls bewusst vom Utilitarismus ab, der
bekanntlich fordert, den Gesamtnutzen aller zu maximieren. Wir werden in der
nächsten Vorlesungsstunde auf diese Diskussion noch ausführlicher eingehen
(Kapitel \ref{RawlsHarsanyiDebatte}).

\subsubsection{Die Maximax-Regel}

Analog zur Maximin-Regel könnte man, wenn man wollte, auch eine {\em
Max\-imax}-\-Regel formulieren. Nach der Maximax-Regel müsste dann diejenige
Handlung gewählt werden, bei der der maximale Erfolg am größten ist. Diese Regel
ist eher etwas für ausgeprägte Optimisten oder sehr risikobereite Menschen oder
für Situationen, in denen es mehr darauf ankommt, Kühnheit und Sportsgeist zu
zeigen als Vorsicht und Besonnenheit. Genauso wie sich zur
Maximin-Regel eine lexikalischen Maximin-Regel bilden lässt, ließe sich
ebenfalls eine lexikalische Maximax-Regel zur Maximax-Regel formulieren.
 
\subsubsection{Die Rangordnungsregel}
\label{Rangordnungsregel}

Wie würde man die Lösung zu beurteilen haben, die die Maximin-Regel für
folgendes Beispiel liefert:

\begin{center}
\begin{tabular}{c|c|c|c|c|c|}
\multicolumn{1}{c}{ } & \multicolumn{1}{c}{$S_1$} &
\multicolumn{1}{c}{$S_2$} & \multicolumn{1}{c}{$S_3$} & 
\multicolumn{1}{c}{\ldots} & \multicolumn{1}{c}{$S_{100}$} \\ \cline{2-6}
$A_1$ & 0 & 2 & 2 & $\cdots$ & 2 \\ \cline{2-6}
$A_2$ & 1 & 1 & 1 & $\cdots$ & 1 \\ \cline{2-6}
\end{tabular}
\end{center}

Nach der Maximin-Regel müsste man $A_2$ wählen. Das bedeutet aber, man zieht
$A_2$ der Handlung $A_1$ vor, obwohl von 100 Fällen $A_2$ nur in einem einzigen
nicht schlechter ist als $A_1$. Das könnte -- je nach Situation -- wenig 
sinvoll erscheinen und verdeutlicht, dass die Eigenschaft der Maximin-Regel
jeweils nur ein einzelnes Spaltenelement in die Prüfung einzubeziehen unter
Umständen eine Schwäche sein kann. Könnte man eine Regel formulieren, die
dieser Schwierigkeit entgeht? 

Denkbar wäre z.B. folgende Regel:\marginline{Rang\-ord\-nungs\-regel} Man
bestimme für jedes Element innerhalb jeder Zeile, welchen Rang es innerhalb
seiner Spalte hat. Dann summiere man die gefundenen Werte zeilenweise auf und
wähle die Handlung, deren Zeile die kleinste Summe hat. (Bei dieser Regel
bestimmen wir erst den Rang statt unmittelbar mit den Zahlen in der Tabelle zu
rechnen, weil es wenig Sinn hat, mit ordinalen Nutzenwerten zu rechnen, die ja
nur dazu dienen sollen, eine Rangfolge wiederzugeben.) Nach diesem Verfahren
würde die Handlung $A_1$ eine Rangzahl von 101 erhalten, da ihr Ergebnis in 99
von hundert möglichen Fällen auf den ersten Rang kommt und in einem Fall auf den
zweiten ($99 \cdot 1 + 2 = 101$). Die Handlung $A_2$ würde eine Rangzahl von 199
erhalten ($1 \cdot 1 + 99 \cdot 2 = 199$). Damit müsste nach dieser Regel $A_1$
gewählt werden.

Natürlich ist auch die Rangordnungsregel nicht vollkommen. So kann es Fälle
geben, in denen die Rangzahlen mehrerer oder gar aller
Handlungsalternativen genau gleich sind. Aber in diesen Fällen kann man dann
immer noch unbedenklich auf die Maximin-Regel zurückgreifen, da dann praktisch
ausgeschlossen ist, dass es sich um eine für die Maximin-Regel problematische
Situation wie die in der Tabelle weiter oben dargestellte handelt.

Mit der Maximin-, der Maximax- und der Rangordnungsregel haben wir drei
Entscheidungsregeln vorgestellt, die sich bei Entscheidungen unter Unwissen und
bei bloß ordinalen Nutzenwerten anwenden lassen, wobei die wichtigste dieser
Regeln die Maximin-Regel ist. Stellt sich die Frage:\marginline{weitere
Entscheidungsregeln?} Könnte es noch weitere Regeln für diese Art von
Entscheidungsproblemen geben? Das ist allerdings anzunehmen. Vielleicht fällt
Ihnen selbst eine weitere Regel ein. Dabei ist zu beachten, dass eine gute
Entscheidungsregel folgenden Bedingungen genügen muss:

\begin{enumerate}
  \item Sie muss stabil bezüglich ordinaler Transformationen der Nutzenwerte
  sein, d.h. wenn man die Nutzenwerte in der Entscheidungstabelle durch ordinal
  transformierte ersetzt, sollte die Entscheidungsregel immer noch dieselbe
  Entscheidung empfehlen.
  \item Es sollte irgendwelche plausiblen Gründe geben, die für diese
  Entscheidungsregel sprechen, z.B. besondere Entscheidungssituationen, in
  denen sie intuitiv sinnvoll erscheint.
  \item Es sollte möglichst wenig Gegenbeispiele in Form von denkbaren
  Entscheidungsproblemen geben, bei denen die Anwendung der Regel abwegig
  erscheint.
\end{enumerate}


