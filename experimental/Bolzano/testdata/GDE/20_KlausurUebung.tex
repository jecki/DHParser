\subsection{Aufgaben zur Klausurvorbereitung}

Hier sind ein par Aufgaben von der Art, wie sie in der Klausur vorkommen werden.

\subsubsection{Entscheidungen unter Unwissenheit}

\begin{enumerate}
  \item Betrachte folgende Entscheidungstabellen:
\begin{center}
\begin{tabular}{c|c|c|c|c|cc|c|c|c|c|}
\multicolumn{1}{c}{} & \multicolumn{4}{c}{Tabelle 1:} &
\multicolumn{2}{c}{} & \multicolumn{4}{c}{Tabelle 2:}
\\
\cline{2-5} \cline{8-11}
$A_1$ & 4 &  8 & 12  & 0 & & $A_1$ & 0 & -1& 2 & 5 \\ 
\cline{2-5} \cline{8-11} 
$A_2$ & 3 &  2 & 3   & 3 & & $A_2$ & -3& 12& 2 & 4 \\ 
\cline{2-5} \cline{8-11}
$A_3$ & 1 &  5 & 14  & 6 & & $A_3$ & 1 & 8 & -2& 6 \\ 
\cline{2-5} \cline{8-11}
$A_4$ & 2 &  3 & 1   & 7 & & $A_4$ & 2 & 5 & 1 & 0 \\ 
\cline{2-5} \cline{8-11}
\end{tabular}
\end{center}
Löse beide Entscheidungstabellen:
\begin{enumerate}
  \item nach der (lexikalischen) Maximin-Regel
  \item nach der (lexikalischen) Minimax-Bedauerns-Regel
  \item nach dem Indifferenzprinzip
  \item nach der Optimismus-Pessimismus-Regel mit einem Optimismus-Index von 3/4
\end{enumerate}

\item Welche der folgenden Nutzenfunktionen beschreiben jeweils denselben
{\em ordinalen} Nutzen und welche denselben {\em kardinalen} Nutzen:
\begin{enumerate}
  \item \begin{tabular}{l|c|c|c|c|c|c|}
\cline{2-7}
Gut:                      & A & B & C & D & E & F \\
\cline{2-7}
 $u_1$:     & 3 & 2 & 5 & 8 & 1 & 4 \\
\cline{2-7}
 $u_2$:     & 6 & 4 & 8 & 16 & 2 & 7 \\
\cline{2-7}
 $u_3$:     & 7 & 4 & 13 & 22 & 1 & 10 \\
\cline{2-7}
\end{tabular}
 
  \item $u_1(x) = 2x \qquad u_2(x) = -x \qquad u_3(x)=x^2 \qquad 
  u_4(x) = 5x^2-3$
\end{enumerate}

\end{enumerate}

\subsubsection{Wahrscheinlichkeitsrechnung}

\begin{enumerate}
  \item Ein Patient, der kürzlich einen Urlaub in Zentralafrika verbracht hat,
  wird mit Verdacht auf Malaria in die Klinik eingeliefert. Es ist bekannt, dass etwa
  bei 0.5\% derartiger Verdachtsfälle tatsächlich eine Malariaerkrankung
  auftritt. Die behandelnde Ärztin führt zunächst einen Antigen-Schnelltest
  durch. Dieser Schnelltest hat eine
  positiv-positiv Rate von 80\% und eine positiv-negativ Rate von 0.01\%.
  Der Test fällt {\em negativ} aus.
  
  Da der Schnelltest nicht besonders sensitiv ist (wie man an der niedrigen
  positiv-positiv Rate sieht), führt die Ärztin noch einen zweiten Test auf
  Basis einer Polymerase-Kettenreaktion durch. Dieser Test, der mit einer 
  positiv-positiv Rate von 99,5\% und einer positiv-negativ Rate
  von 0.3\% sehr viel zuverlässiger ist, fällt positiv aus.
  
  Mit welcher Wahrscheinlichkeit muss die Ärztin davon ausgehen, dass der
  Patient an Malaria erkrankt ist?

  \item Die Laplace'sche Wahrscheinlichkeit wird wie folgt definiert:
  \begin{enumerate}
    \item Es gibt eine endliche Menge von Elementarereignissen: $\Omega$.
    (Beispiel: Beim Würfeln $\Omega = \{1,2,3,4,5,6\}$)
    \item Jedes Ereignis ist durch eine Menge $E$ charakterisiert, die
    Teilmenge von $\Omega$ ist: $E \subseteq \Omega$. (Beispiel: Das Ereignis,
    eine gerade Zahl zu würfeln, wird durch die Menge
    $E=\{2,4,6\}$ beschrieben.)
    \item Die Wahrscheinlichkeit eines Ereignisses ist definiert als die Anzahl
    der Elemente der Ereignismenge ("`günstige Fälle"') geteilt
    durch die Anzahl der Elementarereignisse ("`mögliche Fälle"'). Wenn $|M|$
    die Anzahl der Elemente der Menge $M$ beschreibt, dann ist die
    Wahrscheinlichkeit $p$ also definiert durch: $p(E) := \frac{|E|}{|\Omega|}$.
  \end{enumerate}
  {\em Beweise}, dass die Laplac'sche Wahrscheinlichkeit die kolmogorowschen
  Axiome erfüllt:
  \begin{enumerate}
    \item Axiom: $\forall_{E \subset \Omega} \qquad p(E) \in \mathbb{R} \qquad
    \mbox{und} \qquad p(E) \geq 0$
    \item Axiom: $p(\Omega) = 1$
    \item Axiom: $\forall_{E,F \subset \Omega} \qquad E \cap F \neq
    \emptyset \Rightarrow p(E \cup F) = p(E) + p(F)$
  \end{enumerate}   
  
  \item Zeige, dass aus den drei kolmogorwschen Axiomen, die {\em Monotonie}
  von Wahrscheinlichkeiten folgt: 
  \[\forall_{E,F \subset \Omega} \qquad E \subset F \Rightarrow p(E)
 \leq p(F) \]
\end{enumerate}

\subsubsection{Entscheidungen unter Risiko}

\begin{enumerate}
  \item In Amerika ist eine Grippewelle ausgebrochen. Experten rechnen damit,
  dass die Grippewelle mit einer Wahrscheinlichkeit von 60\% auch Deutschland
  erreicht. Wenn sie Deutschland erreicht, dann erkrankt ein Anteil von 15\%
  der Bevölkerung. Wird die Grippe nicht behandelt, so sterben 3\% der
  Erkrankten.
 
  Die Gesundheitsministerin erwägt nun, ein breit angelegtes Impfprogramm für
  die gesamte Bevölkerung durchführen zu lassen. Wird die Impfung frühzeitig
  verabreicht, so senkt sie das Erkrankungsrisiko auf 2\%. Allerdings ist die
  Impfung nicht ganz ohne Risiko, denn es kommt -- geheim gehaltenen Zahlen
  zufolge -- bei 0.2\% der geimpften Personen zu schweren Komplikationen, die
  zum Tod führen.
  
  Wenn die Grippe bereits ausgebrochen ist, kann die Gesundheitsministerin
  immer noch die Entscheidung treffen, eine Impfung durchführen zu lassen,
  falls das nicht schon vorher geschehen ist. Allerdings ist die Impfung zu
  diesem späteren Zeitpunkt nicht mehr so effektiv. Sie senkt das
  Erkrankungsrisiko dann nur noch auf 10\% bei gleichem Risiko von
  Komplikationen.

  Aufgaben:
  \begin{enumerate}
  \item Stelle das Entscheidungsproblem als Entscheidungsbaum dar.
  \item Sollte die Gesundheitsministerin eine frühzeitige Durchführung des
  Impfprogramms anstreben?
  \item Angenommen es hätte im Vorfeld eine öffentliche Diskussion über die
  Risiken des Impfprogramms gegeben, so dass die Durchführung des Impfprogramms
  zu einem frühen Zeitpunkt, als noch nicht klar war, ob sie Deutschland
  überhaupt erreicht, politisch nicht durchsetzbar war. Angenommen weiterhin,
  die Grippewelle hat Deutschland schließlich dennoch erreicht und der Ruf nach
  einer schleunigen Massenimpfung wird laut. Sollte die Gesundheitsministerin jetzt 
  doch noch das Impfprogramm durchführen?
  \end{enumerate}
  
  \item Für eine auf einer Menge von Lotterien definierte Präferenzrelation
  gilt neben den üblichen Ordnungsgesetzen von Präferenzrelationen u.a.:
  \begin{enumerate}
    \item {\em Bedingung der höheren Gewinne}: Für beliebige Lotterien $x$,$y$
  und $L^*$ und jede beliebige Wahrscheinlichkeit $a$ gilt: 
  \begin{enumerate}
     \item $L^* \succ x$ genau dann wenn $L(a, L^*, y) \succ L(a, x, y)$.
     \item $L^* \succ y$ genau dann wenn $L(a, x, L^*) \succ L(a, x, y)$.
  \end{enumerate}
  \item {\em Reduzierbarkeit zusammengesetzter Lotterien}:
  Für jede zusammengesetzte Lotterie der Form
  $L(a, L(b,x,y), L(c,x,y))$ gilt $L(a, L(b,x,y), L(c,x,y)) \sim L(d,x,y)$ mit $d:=ab+(1-a)c$. 
  \end{enumerate}
  {\em Zeige} allein mit Hilfe dieser beiden Bedingungen (und der
  Ordnungsgesetze für Präferenzrelationen):
  \begin{enumerate}
    \item Es kann {\em nicht} gelten: $L(a,x,x) \succ x$
    \item Es kann {\em nicht} gelten: $x \succ L(a,x,x)$ 
  \end{enumerate}  

  \item Nimm weiterhin folgende Bedingungen als gegeben an (ergibt sich aus
  der vorhergehenden Aufgabe): Für alle Wahrscheinlichkeiten $a$ und alle
  Lotterien $x$ gilt: $L(a,x,x) \sim x$

  {\em Zeige} allein mit dieser und den Bedingungen aus der vorhergehenden
  Aufgabe: Wenn $B$ ein bestes Grundgut ist, dann kann es keine Lotterie $L(a, x, y)$ geben
  für die gilt: $L(a, x, y) \succ B$
\end{enumerate}



\subsubsection{Spieltheorie}

\begin{enumerate}
  \item Löse das folgende Spiel durch sukkzessive Dominanz (Gib dazu in der
  richtigen Reihenfolge die zu streichenden Zeilen- bzw. Spaltenstrategien an):
\begin{center}
\begin{tabular}{c|c|c|c|c|}
\multicolumn{1}{c}{} & 
\multicolumn{1}{c}{$S_1$} &
\multicolumn{1}{c}{$S_2$} &
\multicolumn{1}{c}{$S_3$} &
\multicolumn{1}{c}{$S_4$} \\ \cline{2-5}
$Z_1$ & 4 & 2 & 0 & 14 \\ \cline{2-5}
$Z_2$ & 11& 7 & 1 & 12 \\ \cline{2-5}
$Z_3$ & 9 & 6 & 4 & 5  \\ \cline{2-5}
$Z_4$ & 3 & 4 & 2 & 8  \\ \cline{2-5}
\end{tabular}
\end{center}

  \item Gegeben seien diese beiden Spiele:

\begin{center}
\begin{tabular}{c|c|c|cc|c|c|}
\multicolumn{1}{c}{} & \multicolumn{2}{c}{{\bf Spiel A}} & 
\multicolumn{2}{c}{} & \multicolumn{2}{c}{{\bf Spiel B}}
\\ 
\multicolumn{1}{c}{} & \multicolumn{1}{c}{$S_1$} & \multicolumn{1}{c}{$S_2$} 
& &
\multicolumn{1}{c}{} & \multicolumn{1}{c}{$S_1$} & \multicolumn{1}{c}{$S_2$} 
\\ \cline{2-3} \cline{6-7}
$Z_1$ &  2, 1  & 0,0 &  &  $Z_1$ & 0,0  & -1,1  \\ \cline{2-3} \cline{6-7}
$Z_2$ & -1,-2  & 1,3 &  &  $Z_2$ & 1,-1 & -2,-2 \\  \cline{2-3} \cline{6-7}
\end{tabular}
\end{center}

  Aufgaben:
  \begin{enumerate}
    \item Bestimmte zu jedem Spiel:
      \begin{enumerate}
        \item die {\em reinen} Nash-Gleichgewichte (sofern vorhanden).
        \item die {\em gemischten} Nash-Gleichgewichte (sofern vorhanden).
      \end{enumerate} 
    \item Bestimme den Erwatungswert der Spiele für jeden Spieler in den
    gemischten Gleichgewichten.
  \end{enumerate} 

\end{enumerate}

