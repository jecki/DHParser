\section{Wissenschaftskritische Diskussion der Reichweite und Grenzen der formalen Entscheidungstheorie in der Philosophie}

{\em Nachtrag:} Als ich diese Vorlesung in den Jahren 2008 und 2009 in Bayreuth
gehalten habe und dazu dieses Skript geschrieben habe, hatte ich aus Zeitgründen
leider das letzte Kapitel nicht mehr niedergeschrieben. Das hole ich jetzt
(Januar 2020) endlich nach. Eigentlich ist es das wichtigste Kapitel, den es
liefert eine abschließende Bewertung der Entscheidungstheorie. Bewertet wird die
Entscheidungstheorie dabei nicht als betriebswirtschaftliche Lehre zur
Optimierung betrieblicher Abläufe, sondern als philosophischer Ansatz zur
Erklärung sozialen Verhaltens. Denn als solche wird die Entscheidungstheorie
(und die darauf aufbauende Spieltheorie) in der analytischen Philosophie
verkauft (und übrigens ebenso auch innerhalb des Rational-Choice-Lagers in den
Sozialwissenschaften und der Volkswirtschaftslehre). 


An vielen Stellen in dieser Vorlesung wurde
bereits auf die Fragwürdigkeiten und Schwächen der Entscheidungstheorie
hingewiesen. Diese Schwächen fallen ganz besonders dann ins Gewicht, wenn man
die Entscheidungstheorie nicht als eine Hilfsmittel betrachtet wird, mit dem man
Anhaltspunkte zur Lösung einer äußerst beschränkten Klasse von
Entscheidungsproblemen z.B. aus dem Bereich der Betriebswirtschaft gewinnen
kann, sondern als eine universale Logik des menschlichen Handelns, wie das in
der analytischen Philosophie gerne getan wird. Im folgenden sollen
zusammenfassend die bestehenden erheblichen Defizite der formalen
Entsscheidungstheorie für die Beschreibung und Erklärung menschlichen Handelns
betrachtet werden.



\subsection{Die drei zentralen Schwächen der formalen Entscheidugnstheorie}

Für das häufige Versagen der formalen Entscheidungs- und Spieltheorie bei der
Beschreibung und Erklärung menschlichen Handelns gibt es intrinsische und
extrinsische Gründe. Mit den intrinsischen Gründen meine ich hier Eigenschaften
der Theorie, die ihre innere Logik und ihre mögliche Beziehung zu den
empirischen Gegenständen, die in ihren Bereich fallen und mit ihr erklärt werden
sollen, betreffen. (Und ``möglich'' meint hier alle denkbaren Beziehungen zur
Empirie, nicht die tatsächlich schon von Forschern bisher erprobten und
erforschten.) Unter den extrinsischen Gründen verstehe ich die Gründe für das
Versagen der Entscheidungstheorie, die sich aus dem Umgang der Forscher mit
dieser Theorie ergeben. So verhalten sich beispielsweise gerade in der
analytischen Philosophie viele Entscheidungs- und Spieltheoretiker ausgesprochen
dogmatisch (hinsichtlich des Glaubens an die Theorie) und ignorant (gegenüber
alternativen Ansätzen, ebenso wie gegenüber der Empirie, sofern sie sich nicht
auf Modelle beruft). Dafür kann die Theorie als mathematisches Gebilde nun
nichts, aber mit einer Theorie lernt man immer auch Haltungen des Umgangs mit
ihr, und die machen diese Theorie zumindest in der analytischen Philosophie noch
schlimmer als sie ohnehin schon ist.

Aber zunnächst zu den intrinsischen Gründen. Es gibt drei wesentliche
Eigenschaften der Entscheidungs- und Spieltheorie, die ihre Reichweite
und Erklärungsfähigkeit drastisch einschränken und sie zu einer
umfassenden Theorie menschlichen Handelns untauglich werden lassen:

\begin{enumerate}

\item {\bf Unrealistische Voraussetzungen}: Dazu gehört insbesondere die Annahme
vollstängig geordneter und transitiver Präferenzen, die empirisch nicht
zutreffen und normativ nicht sinnvoll sind. Zum beruht ihre (rein theoretische)
Rechtfertigung auf einer Art Mythos (Geldpumpenargument), was sie auch nicht
vetrauenerweckender macht. 

\item {\bf Performative Selbstwidersprüchlichkeit}: Aus den Voraussetzungen der
Theorie folgt, dass die Bedingungen für ihre Anwendbarkeit im Normalfall nicht
gegeben sind. Dies ist eine unmittelbare aber wegen ihrer unerfreulichen
Botschaft nur selten (und in den Lehrbüchern schon gleich gar nicht)
thematisierte Konsequenz des Satzes, wenn man ihn auf multikriterielle
Entscheidungsprobleme anwendet.

\item {\bf Keine messbaren Größen}: Die zentrale Größte, auf der sich praktisch
alle "Gesetztmäßigkeiten" der Spiel- und Entscheidungstheorie beruhen, sind die
Nutzenwerte. Weder für den ordinalen noch für den kardinalen gibt es auch nur
halbwegs präzise Messmethoden zu ihrer Bestimmung. Da sie zudem (siehe Punkt 1)
auf unrealistischen Voraussetzungen beruhen, erscheint es mehr als zweifelhaft,
ob sie als reale Größen überhaupt existieren. Aber selbst wenn man sie als
versteckte Größen in der Natur unterstellt, so führt das fehlen präziser
Messverfahren dazu, dass weitaus die meisten Modelle der Entscheidungs- und
Spieltheorie empirisch nicht überprüfbar sind, und damit -- nach dem Popperschen Falsifikationskriterium -- strengenommen nicht einmal als wissenschaftlich gelten können. 

\end{enumerate}

\subsubsection{Unrealistisch}

vollständig geordnete und transitive Präferenzen


\subsubsection{Partiell selbstwidersprüchlich}

Der Satz von Arrow angewandt auf multikriterielle Entscheidungsprobleme


\subsubsection{Keine messbaren Größen}

weder kardinale noch ordinale Nutzenwerte sind messbar


\subsection{Relevanz}


\subsection{Der falsche Umgang mit der formalen Entscheidungstheorie in der analytischen Philosophie}
