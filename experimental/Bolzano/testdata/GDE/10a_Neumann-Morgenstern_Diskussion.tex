\section{Diskussion der Neu\-mann-Morgen\-stern\-schen Nut\-zen\-theorie}
\label{DiskussionNeumannMorgenstern}

Nachdem in der letzten Vorlesung die Neumann-Morgensternsche Nutzentheorie
mathematisch entwickelt worden ist, soll in dieser Vorlesung ihr Sinn und ihre
Bedeutung diskutiert werden. Bei formalen Beweisführungen wie dem Beweis aus der
letzten Vorlesung, der zeigt, dass man zu einer beliebigen Menge von Präferenzen
mit Hilfe des Konstruktionsmittels der Lotterien eine kardinale Nutzenfunktion
konstruieren kann, die dem Erwartungsnutzenprinzip genügt, tut man nämlich immer
gut daran sich Klarheit darüber zu verschaffen, was dabei inhaltlich bewiesen
wurde und unter welchen Voraussetzungen es bewiesen wurde. Um diese Frage zu
klären werden wir im Folgenden verschiedene Lesarten des Beweises diskutieren.

\subsection{Unterschiedliche Lesarten der Neu\-mann-Morgen\-stern\-schen
Nutzentheorie}

\subsubsection{NM als Beweis der Existenz kardinaler Nutzenfunktionen}
\label{LesartKardinalerNutzen}

Eine mögliche Lesart wäre die, dass uns die Neumann-Morgensternsche
Nutzentheorie zeigt, dass wir immer eine kardinale Nutzenskala verwenden
dürfen. In dieser Hinsicht scheint der Beweis ein ebenso verblüffendes wie
zwingendes Resultat zu liefern. Verblüffend erscheint das Resultat, weil wir ja
keineswegs von vornherein die Existenz von
"`Präferenzintervallen"' angenommen haben, wie Resnik das zu Anfang des 4.
Kapitels seines Buches in wenig plausibler Weise tut \cite[S.
82]{resnik:1987}. Vielmehr wurde für die Konstruktion der 
kardinalen Nutzenfunktion nach Neumann-Morgenstern zunächst
nur die Existenz einer wohlgeformten Präferenzrelation vorausgesetzt, sowie
die Gesetze der Wahrscheinlichkeitsrechnung, die als solche noch nichts 
darüber aussagen, wie man mit Nutzenwerten umgehen kann. Die Konstruktion 
der kardinalen Nutzenfunktion erfolgte dann allein durch 
Indifferenzvergleiche zwischen Gütern, wobei zu der Menge der Güter 
allerdings auch die besondere Art von gedachten Lotterien gehören muss, 
von der Neumann und Morgenstern in ihrer Theorie Gebrauch machen.

Allerdings ist darauf hinzuweisen, dass der Beweis nicht ausschließt, dass wir
auf der Menge der Lotterien eine Nutzenfunktion konstruieren können, die nicht
die Erwartungsnutzeneigenschaft hat und die sich nicht positiv linear in die auf
dieser Menge konstruierte Nutzenfunktionen mit Erwartungsnutzeneigenschaft
transformieren lässt.

Aber selbst, wenn sich dieses Problem noch irgenwie lösen ließe, kommt hinzu,
dass die Neumann-Morgensternsche Nutzentheorie so voraussetzungsarm eben
doch nicht ist. Wir können eine kardinale Nutzenfunktion konstruieren, aber nur
wenn die Präferenzrelation "`reich"' genug dafür ist, d.h. wenn ihr
\marginline{Voraussetzung einer "`reichen"' Präferenzstruktur}
Gegenstandsbereich alle diejenigen Lotterien umfasst, die nicht weiter
reduziert\footnote{Siehe die Bedingung der Reduzierbarkeit auf Seite
\pageref{Reduzierbarkeit}.} werden können.

Lehnt man kardinale Nutzenfunktionen mit dem Argument ab, dass die Zuweisung von
Nutzenwerten, die mehr ausdrücken als eine bloße Ordnungsrelation, willkürlich
und empirisch nicht zu rechtfertigen ist, dann kann auch die
Neuman-Morgensternsche Nutzentheorie kein wirklich überzeugendes Gegenargument
liefern, denn anstelle der willkürlichen Zuweisung von Zahlenwerten werden jetzt
nicht minder willkürlich Indifferenzbeziehungen zwischen einer neukonstruierten
Klasse gedachter Güter (den Lotterien) und den Grundgütern angenommen. Das
Rechtferigungsproblem der kardinalen Größen ist damit nur besser "`versteckt"'
aber nicht gelöst worden. Nach wie vor kann man also nur in solchen Kontexten von
der Existenz kardinaler Nutzenfunktionen ausgehen, in denen sich die Zuweisung
von Werten auf einer Intervallskala empirisch rechtfertigen lässt. Dies ist z.B.
dann der Fall, wenn wir es mit Geldwerten zu tun haben {\em und} wenn wir Grund
zu der Annahme haben, dass die Geldwerte in dem ensprechenden Kontext einen
konstanten Grenznutzen haben.\footnote{Vergleiche auch die Ausführungen auf Seite
\pageref{RisikoaversionGrenznutzen}.} 

In anderen Fällen, in denen kardinale
Nutzenwerte zwar nicht präzise messbar sind, aber in denen sich die
``Intensität'' von Präferenzen in irgendeinerweise bemerkbar macht, kann eine
kardinale Nutzunktion immmer noch komparativ gesehen die bessere Annährung
an die Wirklichkeit darstellen als eine ordinale Nutzenfunktion. Dies gilt
zumindest dann, wenn die Ungenauigkeit bei der Feststellung der kardinalen
Nutzenwerte in der entsprechenden Anwendungssituation eher vertretbar erscheint
als der Wegfall der Informationen über die Intensität der Präferenzen bei der
Verwendung ordinaler Nutzenfunktionen.

Dass die Neumann-Morgensternsche Nutzentheorie den Bereich der Anwendbarkeit des
kardinalen Nutzens nicht erweitern kann, sollte uns nicht verwundern. Es wäre im
Gegenteil sehr sonderbar, wenn man das empirische Problem der Metrisierung und
Messung von Präferenzen durch eine rein theoretisch-mathematische Konstruktion
lösen könnte.

\paragraph{Erwartungsnutzen statt Erwartungswert}

Im Zusammenhang mit der Neumann-Morgensternschen Nutzentheorie wird oft eine
Diskussion darüber geführt, wie sich Geldwerte zu Nutzenwerten verhalten \cite[S.
85ff.]{resnik:1987}. Der Vorteil von Geldwerten gegenüber bloß ordinalen
Nutzenwerten besteht darin, dass man mit Geldwerten rechnen kann, was mit
ordinalen Nutzenwerten nur sehr begrenzt möglich ist.
\marginline{Mögliche Diskrepanz von Geldwert und Nutzenwert}
 Das bekannte Problem, wenn
wir mit Geldwerten anstatt mit Nutzenwerten rechnen, besteht darin, dass Geldwert
und Nutzen einander keinesfalls immer entsprechen müssen, z.B. weil der
Grenznutzen des Geldes nicht konstant ist. Zudem sind viele
Entscheidungssituationen denkbar, in denen die Ergebnisse nicht sinnvoll als
monetäre Kosten oder Gewinne beziffert werden können. Sofern es überhaupt möglich
ist eine kardinale Nutzenfunktion anzugeben, erscheint daher der Rückgriff auf
Nutzenwerte anstelle von Geldwerten zunächst die sinnvollere Alternative zu sein.
Dieser scheinbare Vorteil des kardinalen Nutzens gegenüber dem Geldwert wird
jedoch in der Regel dadurch zunichte, dass sich kardinale Nutzenwerte sehr viel
schlechter präzise messen lassen als Geldwerte. (Die theoretische Konstruktion
des kardinalen Nutzens aus Lotterien, wie sie von von Neumann und Morgenstern
vorgenommen wird, kann kaum eine zuverlässige Grundlage für empirische Messungen
abgeben.) Zudem ist auch der kardinale Nutzen oft schlicht nicht vorhanden.
Auch wenn Geldwerte unter Umständen nur lose an den Nutzen geknüpft sind, den jemand aus
einem bestimmten Geldbetrag beziehen kann, ist das Rechnen mit Geldbeträgen, wo
dies möglich ist, daher in der Regel die sehr viel zuverlässigere Alternative.
Nicht nur aus didaktischen Gründen stützt beispielsweise Kaplan daher (anders als
Resnik) den Aufbau der Entscheidungstheorie von vornherein nur auf Lotterien über
Geldwerte \cite[]{kaplan:1996}. Alles in allem kann man festhalten: Welche
konzeptionellen Probleme auch immer mit dem Geldwert bzw. dem erwarteten Geldwert
verknüpft sind, sie können durch die Einführung von Nutzenwerten statt Geldwerten
auch nicht immer befriedigender gelöst werden.

\subsubsection{NM als Beweis des Erwartungsnutzens}

Eine weitere Lesart der Neumann-Morgensternschen Nutzentheorie besagt, dass die
Neumann-Mor\-gen\-sternsche Nutzentheorie uns die Gültigkeit des
Erwartungsnutzenprinzips auch bezogen auf Einzelfälle beweist. Sie liefert damit
eine stärkere Rechtfertigung des Erwartungsnutzens als der Hinweis auf das Gesetz
der großen Zahlen und empirisch-statistische Überlegungen (siehe Kapitel
\ref{RechtfertigungErwartungsnutzen}, Seite
\pageref{RechtfertigungErwartungsnutzen}ff.). Auch hier gilt die Einschränkung,
dass das Resultat nur unter den vorausgesetzten "`Bedingungen"' (siehe Seite
\pageref{LotterienBedingungen}) bewiesen wurde. Anders als bei der ersten Lesart
(Seite \pageref{LesartKardinalerNutzen}), die die Konstruktion kardinaler
Nutzenfunktionen hervorhebt, liefert die Neumann-Morgensternsche Nutzentheorie
bei dieser Lesart auch mit dieser Einschränkung noch ein gehaltvolles Resultat.
Denn die Rechtfertigung des Erwartungsnutzenprinzips (auch für den Einzelfall)
erübrigt sich keineswegs von selbst in den Kontexten, in denen wir mit Geldwerten
zu tun haben oder kardinalen Nutzen annehmen dürfen. 
\marginline{Verletzung des Erwartungs\-nutzen\-prinzips als Inkonsequenz} Was
die Neumann-Morgensternsche Nutzentheorie zeigt ist, dass die Verletzung des
Erwartungsnutzenprinzips nicht nur (auf lange Sicht) zu einer Minderung des
Gewinns führt, sondern auch Ausdruck inkonsequenten Verhaltens ist. Der Nachweis
dieser Inkonsequenz funktioniert aber nur dort, wo wir genügend "`reiche"'
Präferenzen annehmen dürfen. Ist das aber nicht der Fall, dann können wir
gegenüber Abweichungen vom Erwartungsnutzenprinzip auch nicht mit Hinweis auf
Neumann-Morgenstern den Vorwurf der Inkonsequenz erheben.\footnote{Vergleiche
dazu auch die frühere Diskussion zwischen Rawls und Harsanyi, Kapitel
\ref{RawlsHarsanyiDebatte}, Seite \pageref{RawlsHarsanyiDebatte}ff. .}

\subsubsection{Der Erwartungsnutzen in der Empirie}

Man kann die Entscheidungstheorie in zweierlei Weise verstehen: Als {\em
empirische Theorie}, die mehr oder weniger genau beschreibt, wie sich Menschen in
Entscheidungssituationen verhalten, und die zugleich erklärt, weshalb sie sich so
entscheiden, wie sie es tun, nämlich, weil sie ihren Nutzen maximieren wollen.
Oder als {\em normative Theorie} (im instrumentellen, nicht im moralischen
Sinne\footnote{Instrumentell-normative Theorien sind Theorien, die uns sagen, wie
wir ein {\em gegebenes Ziel} am besten erreichen können, die aber nichts darüber
aussagen, ob das Ziel es wert ist verfolgt zu werden. (In der Terminologie der
Moralphilosophie Immanuel Kants könnte man sagen, sie befassen sich
ausschließlich mit "`hypothetischen Imperativen"'.) Moralisch-normative Theorien
sind dagegen philosophische Theorien, die etwas darüber aussagen, welche Ziele
und Zwecke im Leben wertvoll sind oder welche Handlungen man ausführen bzw.
unterlassen muss unabhängig von irgendwelchen Zielen und Zwecken
(deontologischer Ansatz).}), die uns lehrt, wie wir richtige Entscheidungen
treffen sollen, um einen vorgegebenen Zweck so gut wie möglich zu erreichen.

Die Neumann-Morgensternsche Nutzentheorie konkretisiert die
Entscheidungstheorie in dem Sinne, dass sie uns zeigt, dass nutzenmaximierende
Entscheidungen unter der Voraussetzung vorgegebener und genügend reicher 
Präferenzen dem Prinzip des Erwartungsnutzen folgen (sollten). Wenn man diese
Theorie als empirisch-deskriptive Theorie auffassen will (oder größeren
empirisch-deskriptiv verstandenen ökonomischen Theoriegebilden zur Grundlage
geben will), dann stellt sich die Frage, ob sie menschliches
Entscheidungsverhalten richtig oder falsch beschreibt. 

Zu dieser Frage haben Daniel Kahneman und Amos Tversky eine Reihe von berühmten
\marginline{Kahnemann und Tversky}
Experimenten durchgeführt. Eins läuft so ab: Die Probanden sollen
ein Entscheidungsproblem mit folgender Hintergrundgeschichte lösen:

\begin{quote}
``Sie sind Gesundheitsminister
und wissen, dass eine unbekannte Grippewelle in unabsehbarer Zeit
Ihr Land heimsuchen wird, die voraussichtlich 600 Menschen das
Leben kosten wird. Gegen diese Krankheit sind zwei verschiedene
Präventionsprogramme entwickelt worden, über deren Anwendung
Sie entscheiden sollen. Ihnen werden folgende Präventionsprogramme
vorgeschlagen.'' \cite[S. 43]{fritz:2002}
\end{quote}

Die Probanden sind bei diesem Experiment in zwei Gruppen unterteilt. Die 
erste Gruppe
erhält folgende Information über die Wirksamkeit der Präventionsprogramme
\cite[S. 44]{fritz:2002}:\label{tverskyBeispiel}

\begin{quote}
\begin{itemize}
\item Bei Anwendung des Präventionsprogramms A werden 200 Personen gerettet.
\item Bei Anwendung von Programm B gibt es eine Wahrscheinlichkeit von 1/3, dass
600 Menschen gerettet werden und eine Wahrscheinlichkeit von 2/3, dass niemand
gerettet wird.
\end{itemize}
\end{quote}

Der zweiten Gruppe wird dagegen genau dieselbe Information in der folgenden
Form mitgeteilt:

\begin{quote}
\begin{itemize}
\item Bei Anwendung des Programms C werden 400 Menschen sterben.

\item Bei Anwendung des Programms D gibt es eine Wahrscheinlichkeit von 1/3,
dass niemand sterben muss und eine Wahrscheinlichkeit von 2/3, dass 600 Menschen
sterben müssen.
\end{itemize}
\end{quote}

Nicht nur die Informationen sind für beide Gruppen diesselben, sondern auch der
Erwartungsnutzen beider Programme ist derselbe, da sowohl bei Anwendung von
Programm A als auch bei der Anwendung von Programm B nach dem
Erwartungsnutzenprinzip der Tod von 200 Menschen zu erwarten ist. Würden sich
die Probanden im Sinne der Erwartungsnutzenhypothese verhalten, 
dann müssten sie erstens zwischen beiden beiden Programmen
indifferent sein, d.h. bei einer hinreichend großen Zahl von Probanden müssten
sich ca. 50\% für Programm A (bzw. C) und 50\% für Programm B (bzw.
D) entscheiden. Und zweitens dürfte es insbesondere keine Unterschiede zwischen
der ersten und der zweiten Gruppe von Probanden geben. 

\marginline{Empirische Verletzung des E'n-Prinzips}
Kahneman und Tversky stellten jedoch fest, dass in der ersten Gruppe von
Probanden 72\% das Programm A wählten, während sich in der zweiten Gruppe nur
22\% für das entsprechende Programm C entschieden. Das Erwartungsnutzenprinzip
ist damit als empirische Hypothese über menschliches Entscheidungsverhalten
widerlegt. Andere Experimente bestätigen diesen Befund. 

Man könnte einwenden, dass von diesem Experiment die Neumann-Morgensternsche
Nutzentheorie als empirische Theorie nicht widerlegt ist, weil in diesem Fall
eine der Bedingungen ihrer Anwendbarkeit (genügend reiche Präferenzstruktur)
möglicherweise nicht gegeben ist. Dennoch kommt sie durch dieses Experiment in
Schwierigkeiten, denn die Neumann-Morgensternsche Nutzentheorie setzt mit der
Reduzierbarkeitsbedingung (siehe Seite \pageref{Reduzierbarkeit}) implizit
voraus, dass Menschen indifferent gegenüber unterschiedlichen Repräsentationen
desselben Entscheidungsproblems sind. Genau das ist aber, wie Kahnman und
Tversky eindrucksvoll zeigen konnten, nicht der Fall. Vielmehr hängt das
menschliche Entscheidungsverhalten -- wie es übrigens auch die Alltagserfahrung
nahelegt -- sehr wesentlich davon ab, \marginline{"`Framing"'-Effekt} wie ein
Entscheidungsproblem dargestellt wird ("`Framing-Effekt"').

Es ist denkbar, dass das Experiment anders ausgefallen wäre, wenn man auf
Probanden zurückgegriffen hätte, die zuvor in der Entscheidungstheorie
instruiert worden sind. Aber dann hieße das immer noch,
\marginline{Grenzen der empirischen Anwendbarkeit} 
dass die Entscheidungstheorie empirisch-deskriptiv nur solche
Entscheidungssituationen richtig erfasst, in denen "`professionelle"'
Entscheider die Entscheidungen treffen, nicht aber generell alle
Entscheidungssituationen.

\subsubsection{NM als Rationalitätskriterium}

Wenn man die Neumann-Mor\-gen\-stern\-sche Nutzentheorie weniger als
empirisch-deskriptive denn als normative Theorie liest, dann besagt sie, dass
man, will man rationale Entscheidungen treffen, sich bei Entscheidungen unter 
Risiko an das
Erwartungsnutzenprinzip halten sollte. Rationalität wird dabei wie immer in
diesem Zusammenhang im Sinne der spärlichen Definition David Humes
als "`die Fähigkeit zu gegebenen Zwecken die geeigneten Mittel zu finden"'
verstanden. Dieser Rationalitätsbegriff ist nicht zu verwechseln mit dem in der
kontinentalen Tradition üblichen, vor allem durch Kant geprägten
umfassenden Vernunftbegriff, der auch eine Fähigkeit der Vernunft zur
Erkenntnis des moralisch Richtigen unterstellt. 

Aber auch im Sinne der rein instrumentell verstandenen Rationalität ist die Frage
zu stellen, ob rationale Entscheidungen stets dem Prinzip der Erwartungsnutzens
gehorchen müssen. In dieser Hinsicht ist es wichtig, sich darüber im Klaren zu
sein, dass die Neumann-Morgensternsche Nutzentheorie lediglich zeigt, dass {\em
wenn} genügend reichhaltige und wohlgeformte Präferenzen vorhanden sind,
rationale Entscheidungen nach Maßgabe des Erwartungsnutzens getroffen werden
müssen. Was sie aber {\em nicht} beweist \marginline{Grenzen der normativen
Anwendbarkeit} und auch nicht beweisen kann ist, dass man stets über eine
entsprechend reiche Präferenzrelation verfügen sollte bzw. dass es, wenn man
nicht darüber verfügt, rational wäre, sich gefälligst eine zuzulegen. Wenn die
Konstruktion kardinaler Präferenzen nach Neumann-Morgenstern daran scheitert,
dass die Präferenzen nicht reichhaltig genug sind (indem sie nicht auch alle
denkbaren Lotterien einbeziehen), dann kann man nicht mit Berufung auf den
Neumann-Morgensternschen Beweis den Vorwurf der Irrationalität erheben. Dieser
Beweis zeigt nur, dass unter bestimmten und bestenfalls teilweise
selbstverständlichen Voraussetzungen ein bestimmtes Verhalten rational ist. Er
zeigt nicht, dass die Erfüllung der Voraussetzungen des Beweis selbst eine
Forderung der Rationalität ist.\footnote{Dasselbe gilt nicht nur für
Neumann-Morgenstern, sondern für die Theorien des rationalen Handelns
überhaupt. Z.B. kann uns die Theorie sagen, wie wir wählen sollten, wenn wir
transitive Präferenzen haben, aber sie (d.h. zumindest die hier entwickelte
Theorie) kann uns nichts darüber sagen, wie wir uns entscheiden sollten, wenn
wir keine transitiven Präferenzen haben. Insbesondere kann sie nicht sagen,
dass wir transitive Präferenzen haben sollten, denn das ist eine Voraussetzung
nicht aber ein Ergebnis der Theorie.}

Nun könnte man aber fragen, ob es nicht andere Gründe dafür gibt, die
Voraussetzungen für den Beweis, insbesondere die Möglichkeit der Ausdehnung der
Präferenzordnung auf eine vollständige Menge von Lotterien (siehe Seite
\pageref{MengeVonLotterien}), als eine Forderung der Rationalität zu
akzeptieren. Man könnte sich z.B. darauf berufen, dass es immer möglich sein
muss, bei zwei Gütern zu entscheiden, welches man dem anderen vorzieht,
oder ob man beide Güter gleich hoch schätzt. Kann man sich zwischen zwei Gütern
nicht entscheiden, so bedeutet dies nichts anderes, als das man zwischen beiden
Gütern indifferent ist. Also enthält die Annahme der Ausdehnbarkeit einer
gegebenen Präferenzordnung auf die vollständige Menge der Lotterien über alle
in der Präferenzordnung vorkommenden Güter keine ungewöhnlichen oder
unzumutbaren Voraussetzungen.

Ein (noch relativ leicht ausräumbares) Problem kann jedoch dadurch entstehen,
dass wir uns unter Umständen nur deshalb nicht zwischen zwei Gütern entscheiden
können, weil wir nicht verstehen, was die Güter beinhalten. Wenn man diese Art
von \marginline{Indifferenz oder Unentschlossenheit?} Unsicherheit oder
Unentschlossenheit im Sinne des eben geführten Arguments als Indifferenz
interpretiert, dann kann das zur Folge haben, dass wir Indifferenz zwischen zwei
Gütern annehmen, die eindeutig unterschiedlichen Wert haben. Man könnte sich
folgendes Beispiel vorstellen: Jemand wird vor die Wahl gestellt entweder einen
Lottoschein auszufüllen, bei dem er eine Chance von ca. 1:14.000.000 hat, sechs
Richtige zu bekommen, oder sich mit demselben Einsatz an einer
Lotto-Tippgemeinschaft zu beteiligen, deren Gewinnchancen sich nach einem
hochkomplizierten und kaum durchschaubaren Schema richten, das von einer kundigen
Mathematikerin erfunden wurde, der die Tippgemeinschaft gehört. Angenommen unser
Lotto-Spieler hat keine klare Vorstellung davon, wie gut seine Gewinnchancen bei
der Beteiligung an der Tippgemeinschaft sind. Dann müssten wir nach der zuvor
geführten Argumentation annehmen, dass der Spieler indifferent zwischen einem
selbstausgefüllten Schein und der Tippgemeinschaft ist,\footnote{Vgl. dazu auch
die früheren Ausführungen zum sogenannten ``Indifferenzprinzip'' Kapitel
\ref{Indifferenzprinzip} Seite \pageref {Indifferenzprinzip}} auch wenn die
Gewinnchancen bei der Tippgemeinschaft objektiv niedriger sind (da auch die
Betreiber einer Tippgemeinschaft ja von irgendetwas leben müssen).

Das Beispiel führt auf schöne Weise vor Augen, dass Unsicherheit bzw.
Unentschlossenheit eben doch nicht dasselbe ist, wie Indifferenz. Im Zusammenhang
mit der Neumann-Morgensternschen Nutzentheorie stellt diese Art epistemischer
Unsicherheit jedoch nicht unbedingt ein gravierendes Problem dar, da man allzu
komplizierte Lotterien auf Grund der Reduzierbarkeit von Lotterien immer soweit
umformen und vereinfachen kann, bis man die Chance für jeden in einer
verschachtelten Lotterie vorkommenden Gewinn mit einer ganz bestimmten
Prozentzahl angeben kann, was verständlich genug sein dürfte.

Aber es gibt andere Beispiele, wo die Sache komplizierter wird. Nehmen wir an,
jemand bekomme die Gelegenheit an einem Fussballtippspiel zur EM 2008 zu wetten,
ob am 16. Juni Deutschland oder Österreich gewinnt. Gewinnt er die Wette, bekommt
er € 100 Euro, sonst nichts. Nun nehmen wir weiterhin an, der wettende Fußballfan
hat gute Gründe davon auszugehen, dass es wahrscheinlicher ist, dass Deutschland
gewinnt, als dass Österreich gewinnt. Er wird also in jedem Fall auf Deutschland
wetten. Wenn man die Wette als ein Gut betrachtet, dann stellt sich die Frage:
Welche Neumann-Morgensternschen Lotterie der Form L(a, 100 €, 0 €) ist
indifferent zu dieser Wette? Das Problem besteht darin, dass jede Lotterie mit $a
> 0.5$ in Frage käme. Aber sobald wir uns für irgend eine bestimmte Lottie
entscheiden, also z.B. $a = 0.8$ dann stellen wir implizit auch die Behauptung
auf, dass die Fussballwette mehr wert ist als die Lotterie mit $a = 0.75$, eine
Bahuptung für die jedoch keine hinreichenden Gründe vorhanden sind, da unser
Fussballfan nur Gründe für die vergleichsweise vage Annahme hat, dass Deutschland
besser als Österreich ist, aber nicht dafür, dass Deutschlands Gewinnchancen auch
mehr als 75\% betragen. Man könnte versuchen, dass Problem dadurch zu lösen, dass
man $a$ marginal größer als $0.5$ wählt, also $a = 0.5 + \epsilon $. Aber dann
haben wir implizit die Behauptung aufgestellt, dass die Fussballwette weniger
wert ist als die Lotterie mit $a = 0.55$, obwohl wir dafür ebensowenig
hinreichende Gründe haben. Mark Kaplan, von dem ich dieses Argument adaptiert
habe, bezeichnet die dogmatische Forderung, in jedem Fall
\marginline{Der Fehler der "`falschen Präzision"'}
irgendeinen bestimmten Wahrscheinlichkeitswert zuzuweisen, deshalb auch recht
treffend als "`{\em the sin of false precision}"' \cite[S.
23]{kaplan:1996}.\footnote{Das Problem ist ähnlich wie diejenigen, die das
Indifferenzprinzip aufwirft (siehe Kapitel \ref{IndifferenzPrinzipParadoxien},
Seite \pageref{IndifferenzPrinzipParadoxien}ff.).}

Akzeptiert man diese Einwände, dann bedeutet das, dass die Möglichkeit die
Entscheidungstheorie normativ, d.h. als Anleitung zum richtigen Ent"-scheiden
bei gegebener Zielsetzung, einzusetzen, wesentlich davon abhängt, ob bestimmte
empirische Voraussetzungen gegeben sind. Zu diesen Voraussetzungen gehört, dass
wir uns einigermaßen über den Wert der erzielbaren Gewinne (resp.
"`Ereignisse"' oder "`Güter"') im Klaren sind, und dass die vorkommenden
Unsicherheiten von solcher Art sind, dass wir einigermaßen präzise
Wahrscheinlichkeitswerte dafür angeben können. Dementsprechend gibt die formale
Entscheidungstheorie selbst dann nicht das Modell für Rationalität oder
rationales Handeln schlechthin an, wenn wir unter Rationalität allein die
"`instrumentelle Rationalität"' verstehen. Man kann lediglich sagen, dass die
formale Entscheidungstheorie den Begriff "`instrumenteller Rationalität"' in
denjenigen Fällen konkretisiert, in denen die Voraussetzungen für ihre
Anwendbarkeit gegeben sind.


\subsubsection{Mögliche Auswege?}

Soeben wurde noch einmal verdeutlicht, dass die Neumann-Morgen\-sternsche
Nutzen\-theorie ihr Resultat (Existenz einer kardinalen Nutzenfunktion, die dem
Erwartungsnutzenprinzip gehorcht) nicht bloß aus selbstverständlichen
Voraussetzungen ableitet von der Art, dass
man Lotterien mit höheren Gewinnen oder besseren Gewinnchancen bevorzugen soll,
sondern dass sie auch von recht anspruchsvollen empirischen Voraussetzungen
abhängt. Diese Feststellung ist insofern ernüchternd, als damit der
Anwendungsbereich der entsprechenden Entscheidungstheorie doch empfindlich
eingeschränkt wird, was umso bedauerlicher ist als die Techniken der formalen
Entscheidungstheorie dort, wo man sie anwenden kann, sehr leistungsfähig sind. 

Will man den Anwendungsbereich der Entscheidungstheorie ausweiten, so kann
man versuchen, die Entscheidungstheorie auf weniger
anspruchsvolle Voraussetzungen zu gründen. Wenn es gelingt ähnlich starke
Resultate aus vergleichsweise schwächeren Voraussetzungen abzuleiten, dann wäre
das in jeder Hinsicht ein Gewinn für die Entscheidungstheorie. In der Tat
ist ein großer Teil der wissenschaftlichen Diskussion der Konstruktion
von Erweiterungen und Alternativen gewidmet, die geeignet
sind, ihren Anwendungsbereich auszuweiten. Hier soll nur an einem
Einzelbeispiel angedeutet werden, wie das funktionieren kann. Das Beispiel
betrifft nicht die Neumann-Morgensternsche Nutzentheorie im Speziellen, sondern
den Präferenzbegriff als Grundlage der Entscheidungstheorie. 

Wir erinnern uns, dass eine der Bedingungen für wohlgeformte Präferenzen (siehe
Seite \pageref{Ordnungsaxiome}) darin bestand, dass die Präferenzen
{\em zusammenhängend} sein müssen, d.h. für jedes Paar $x, y$ aus
der Menge der möglichen Resultate einer Entscheidungssituation gilt entweder $x
\succ y$ oder $y \succ x$ oder $x \sim y$. Damit ist ausgeschlossen, dass es
jenseits der Indifferenz ($\sim$) so etwas wie Unentschlossenheit oder
Unsicherheit bei Präferenzen gibt, was im Umkehrschluss wiederum heisst: 
Die auf diesen Präferenzbegriff gegründete Entscheidungstheorie ist überhaupt
nur dort anwendbar, wo diese axiomatische Voraussetzung empirisch geben ist,
d.h. wo keine Unentschlossenheit in dem zuvor anhand einiger Beispiele
%(siehe oben die Beispiele der Lotto-Tippgemeinschaf und der Fussballwette)
diskutierten Sinn vorkommt. 
\marginline{Eine alternative Entscheidungstheorie}
Kaplan unternimmt nun einen Versuch eine
Präferenzrelation zu definieren, die die Möglichkeit dieser Art von
Unentschlossenheit mit einbezieht \cite[]{kaplan:1996}. Wie muss er
dabei vorgehen, und was muss er dafür leisten? Damit dieses Vorhaben gelingt,
muss zweierlei geleistet werden: Zunächst muss ein Axiomensystem aufgestellt
werden, in dem in irgendeiner Form auch so etwas wie "`Untentschlossenheit"'
enthalten ist. Dann muss gezeigt werden, dass man auch aus diesem Axiomensystem
möglichst gehaltvolle Gesetze einer Entscheidungstheorie ableiten kann. Wir
werden auf die Einzelheiten von Kaplans Konstruktion nicht eingehen, sondern
nur zeigen, wie er das {\em Zusammanhangsaxiom}, das wohlgeformte Präferenzen
erfüllen müssen, so abwandelt, dass es auch einen gewissen Grad von
Unentschlossenheit zulässt. Kaplan baut seine Entscheidungstheorie etwas anders
auf als Resnik, indem er -- teils aus didaktischen Gründen und der
Anschaulichkeit und Einfachheit halber -- von vornherein von der Zuweisung
von Geldwerten zu bestimmten Ergebnissen (die er "`well mannered states of
affaires"' nennt) ausgeht, aber dieses Detail ist in unserem Zusammenhang nicht
wesentlich. Er definiert den "`moderaten Zusammenhang"' von Präferenzen
nun folgendermaßen:

\begin{quote}
\marginline{"`moderater"' Zusammenhang}
{\em Moderater Zusammenhang} (vgl. \cite[S. 13]{kaplan:1996}): Die Präferenzen
sind charakterisiert durch eine nicht-leere Menge von Zuweisungen von 
Geldwerten zu {\em allen} Ergebnissen,
wobei gilt: 
\begin{enumerate}
  \item Es herrscht Indifferenz zwischen $A$ und $B$ ($A \sim B$), wenn jede der
  Zuweisungen $A$ denselben Wert zuweist wie $B$.
  \item $A$ wird $B$ vorgezogen ($A \succ B$), wenn keine der Zuweisungen 
  $B$ einen größeren Wert zuweist als $A$, und wenn wenigstens eine der
  Zuweisungen $A$ einen größeren Wert zuweist als $B$.
\end{enumerate}
\end{quote}

Zu Erläuterung: Die Menge der Zuweisungen ist eine Menge von Abbildungen von
Geldwerten zu Gütern. Jede dieser Abbildungen entspricht dabei einer
Nutzenfunktion im Sinne der orthodoxen Entscheidungstheorie, wie wir sie in
dieser Vorlesung kennen gelernt haben. Diese Konstruktion kann zunächst
verblüffend erscheinen. Denn wenn wir "`Unentschlossenheit"' modellieren wolllen,
dann -- so sollte man meinen -- müssten wir doch eigentlich versuchen mit
spärlicheren Präferenzrelationen anzusetzen, die nicht jedem Paar von Gütern bzw.
Ereignissen $A,B$ zwingend eine der Relationen $\sim, \succ, \prec$ zuweisen.
Aber darin besteht gerade der Trick: Anstatt (auf welche Weise auch immer) eine
spärlichere Präferenzrelation zu konstruieren, arbeit Kaplan mit einer Menge von
einer Nutzenfunktion vergleichbaren Abbildungen ("`Zuweisungen"'), die teilweise
miteinander übereinstimmen, teilweise aber auch voneinander abweichen können.
Diese Abweichungen zwischen den verschiedenen Quasi-Nutzenfunktionen erlauben es,
so etwas wie Unentschlossenheit zu erfassen. Wollte man etwa die Präferenzen des
Fussballfans erfassen, der überzeugt ist, dass Deutschland größere Gewinnchancen
hat als Österreich, aber unentschlossen ist, wenn es darum geht, um wieviel die
Gewinnchancen Deutschlands größer sind als die Österreichs, dann würde seine
Menge der Zuweisungen alle solchen Zuweisungen enthalten, die der Fussballwette
einen mindestens gleichgroßen Wert zuweisen, wie der Lotterie L(0.5, 100 €, 0 €).
Damit gilt nach dem Axiom des "`moderaten Zusammenhangs"', dass die Fußballwette
der Lotterie L(0.5, 100 €, 0€) vorgezogen wird, was zum Ausdruck bringt, dass
unser Fussballfan einen Gewinn seiner Wette für wahrscheinlicher hält als einen
Verlust. Zugleich gilt aber auch, dass die Fussballwette zu keiner bestimmten
Lotterie indifferent ist, was eben die Unsicherheit des Fans bezüglich der Frage
zum Ausdruck bringt, um wieviel die Gewinnchancen größer als die Verlustchancen
sind.

Wie Kaplan aus seinem Axiomensystem eine gehlatvolle Entscheidungstheorie
ableitet, kann hier nicht mehr ausgeführt werden. Soviel sollte jedoch
deutlich geworden sein, dass man dem Problem der eingeschränkten Anwendbarkeit
bis zu einem gewissen Grade durch andere, möglicherweise liberalere
Axiomatisierungen der Entscheidungstheorie begegnen kann. Allerdings bleibt
auch bei alternativen Axiomatisierungen die Anwendbarkeit der
Entscheidungstheorie immer auf diejenigen empirischen Entscheidungssituationen
begrenzt, in denen wir die Gültigkeit der Axiome voraussetzen können. Es gibt
keine Entscheidungstheorie, die schlechterdings alle Entscheidungssituationen
erfassen könnte, so wie z.B. in den Naturwissenschaften die Kinemathik {\em
alle} Bewegungen von Körpern im Raum erfassen kann. Es ist überhaupt einer der
Unterschiede von Natur- und Gesellschaftswissenschaften, dass die formalen
Theorien in den letzteren immer nur einer mehr oder weniger begrenzte
Reichweite haben, was vermutlich in der Natur des Gegenstandes liegt.

\subsection{"`Paradoxien"' der Nutzentheorie}

Wie wir eben gesehen haben, gibt es eine Reihe ernst zu nehmender Einwände gegen
Entscheidungs- und Nutzentheorie, die jedoch nicht dazu führen, dass diese
Theorie gänzlich verworfen werden müsste, die es aber sehr wohl erlauben ihren --
manchmal uneingestandenen -- Voraussetzungsreichtum herauszuarbeiten und ihren
Anwendungsbereich auf diejenigen Entscheidungsprobleme einzuschränken, zu deren
Behandlung sie sich tatsächlich eignet. Viel häufiger als deratige Einwände wird
in der Fachliteratur im Zusammenhang mit der Nutzen- und Entscheidungstheorie
eine Reihe sogenannter Paradoxien diskutiert. Eine Paradoxie im strengen Sinne
ist eine Aussage, aus deren Wahrheit ihre Falschheit folgt, und aus deren
Falschheit wiederum ihre Wahrheit folgt (wie z.B. das berühmte Lügnerparadox, das
entsteht, wenn ein Athener sagt: "`Alle Athener lügen"'). (Ein Paradox ist damit
zu unterscheiden von einem einfachen logischen Widerspruch, der nur zur Folge
hat, dass eine Theorie oder eine Aussage falsch ist. Wenn z.B. aus der Wahrheit
einer Aussage ihre Falschheit folge, aus ihrer Falschheit aber wieder nur ihre
Falschheit, dann handelt es sich um eine logisch falsche Aussage, aber nicht um
ein Paradox.) Ein Entscheidungsparadox ist eine Entscheidungssituation,
\marginline{Definition: "`Entscheid\-ungs\-paradox"'} in der man mit gleichem
Recht widersprüchliche Entscheidungen fordern muss. Eine Entscheidungstheorie,
die solche Paradoxien zulässt, hat, wie sich versteht, ein ernstes Problem. Fast
alle der im Folgenden diskutierten (vermeintlichen) Paradoxien lassen sich jedoch
auflösen. Sie beruhen zum größten Teil auf mehr oder weniger gewollten
Missverständnissen der Entscheidungs- und Nutzentheorie. Als Einwände gegen die
Neumann-Morgensternsche Nutzentheorie wiegen sie, meiner Meinung nach, daher sehr
viel weniger schwer als die zuvor erörterten Probleme. Ihre Diskussion kann aber
ebenso wie die Diskussion von Beispielen dabei helfen, die Entscheidungstheorie
besser zu verstehen. Zudem verdeutlichen sie Grenzen der Entscheidungstheorie und
mögliche Fallstricke bei ihrer Anwendung.

\subsubsection{Allais' Paradox}

Bei Allais's Paradox werden -- ähnlich wie in dem zuvor vorgestellten Experiment
von Kahneman und Tversky\footnote{Wobei das Allais-Paradox aber nicht den
"`Framing"'-Effekt erfassen kann!} -- zwei scheinbar unterschiedliche
Entscheidungssituationen mit einander verglichen, in denen eine Person zwischen
Alternativen mit unterschiedlichen Gewinnchancen wählen kann
\cite[]{myerson:1991}:

\begin{quote}
{\em Situation A:}
\begin{enumerate}
  \item Alternative: 12 Mio € mit 10\% Chance und 0 € mit 90\%
  \item Alternative: 1 Mio € mit 11\% Chance und 0 € mit 89\%
\end{enumerate}

{\em Situation B:}
\begin{enumerate}
  \item Alternative: 1 Mio € sicher
  \item Alternative: 12 Mio € mit 10\%, 1 Mio € mit 89\% und 0 € mit 1\% 
\end{enumerate}
\end{quote}

Viele Menschen werden sowohl in Situation A als auch in Situation B die erste
Alternative bevorzugen. Dabei erhöht in Wirklichkeit die Wahl der zweiten
Alternative in Situation B den Nutzen in demselben Maße gegenüber der ersten
Alternative wie die Wahl der ersten Alternative in Situation A gegenüber der
zweiten. Durch die Berechnung des Erwartungswertes kann man sich leicht davon
überzeugen, aber diese Feststellung gilt sogar unabhängig davon wie man die
Geldwerte auf Nutzenwerte abbildet, sofern man -- wie es die Nutzentheorie
voraussetzt -- denselben Geldwerten dieselben
Nutzenwerten zuordnet.

In der Tat handelt es sich hierbei aber nicht wirklich um ein Paradox, sondern
nur um das empirische Phänomen, welches schon in dem eben beschrieben Experiment
von Kahneman und Tversky zu Tage getreten ist, dass Menschen sich oft nicht
rational verhalten. Eine Entscheidungsregel nach Art des Sprichtworts: "`Lieber
den Spatz in der Hand als die Taube auf dem Dach"', wie sie im Alltagsleben
gebräuchlich ist, widerspricht schlicht den Regeln der rationalen
Entscheidungstheorie. Die Theorie gerät dadurch insofern nicht in Probleme als
sie eindeutig fordern, in Situation A die erste und in Situation B die zweite
Alternative zu wählen. Der Widerspruch zu alltagspraktischen
Entscheidungsverhalten, das ja oft auch seine guten Gründe hat, legt freilich die
Frage nahe, warum sich im Alltag Entscheidungsregeln herausgebildet haben, die zu
Entscheidungen führen, die der Theorie zufolge keineswegs optimal sind.
Möglicherweise existieren dafür besondere Gründe, die von der Theorie noch nicht
erfasst worden sind. Denkbar ist aber auch, dass die Alltagspraxis einfach
suboptimal ist oder dass in den meisten Alltagssituationen, die -- wie wir
gesehen haben -- recht anspruchsvollen Voraussetzungen für die Anwendung der
Entscheidungstheorie nicht gegeben sind, in welchem Fall gar keine
Inkompatibilität zwischen Alltagspraxis und Theorie vorliegt.

\paragraph{Exkurs: Eine evolutionäre Vermutung zur Erklärung vermeintlich
irrationalen Entscheidungsverhaltens}

Der empirische Befund, auf dem auch Allais' Paradox beruht, dass Menschen sich
häufig, wenn nicht gar typischerweise nicht risikoneutral, sondern am
ehesten riskoavers (manchmal auch risikofreudig) verhalten, wirft die Frage auf,
warum das so ist. Sollten die Menschen nicht langfristig durch Erfolg und
Misserfolg darüber belehrt worden sein, dass Risikoneutralität am ehesten dazu
angetan ist, den Erfolg zu maximieren? Hätte nicht schon die Evolution
risikoneutrales Verhalten prämieren müssen? 

Samir Okasha hat unlängst folgende hypothetische Erklärung für den
\marginline{Evolutionärer Vorteil von Risikoaversion?}
evolutionären Vorteil von risikoaversem Verhalten vorgeschlagen
\cite[]{okasha:2007}: Wir nehmen eine Population von zwei Typen einer Spezies an,
Typ A und Typ B. Von beiden Typen soll es 5 Individuen geben. Typ A geht für den
Nachwuchs große Risiken ein, so dass sich mit 50\%-iger Wahrscheinlichkeit die
Population von Typ A auf 10 erhöhen könnte, aber mit ebenso mit 50\%-iger
Wahrscheinlichkeit auch auf 0 absinken könnte. Typ B ist dagegen genetisch auf
ein Verhalten programmiert, dass dazu führt, dass Typ B unter normalen
Bedingungen in der nächsten Generation seine Populationsgröße erhält, also
wieder 5 Individuen stellt. Man sollte meinen, dass nach dem
Erwartungsnutzenprinzip beide Typen gleich erfolgreich sind (weil $10\cdot 0.5 +
0\cdot 0.5 = 5$). Samir Okasha macht nun darauf aufmerksam, dass, wenn wir
statt der absoltuen Bevölkerungszahl, die relativen Bevölkerungsanteile
betrachten, Typ B, der kein Risiko eingeht, erfolgreicher ist, denn Typ B wird
im Durchschnitt einen Bevölkerungsanteil von $0.5 \cdot \frac{1}{3} + 0.5\cdot 1
= \frac{2}{3}$  bekommen, während Typ A $0.5 \cdot \frac{2}{3} + 0.5 \cdot 0 =
\frac{1}{3}$ erhält. 

Heisst das, dass risikoaverses Verhalten evolutionär von
Vorteil ist? Das gilt höchstens vordergründig, denn wenn wir das Verhalten
korrektverweise auf den relativen Bevölkerungsanteil beziehen, dann zeigt sich,
dass nur Typ B sich risikoneutral verhält, während Typ A risikofreudig ist.
Ganz im Einklang mit der Theorie wird aber das risikoneutrale Verhalten
prämiert. Das Gedankenexperiment Okashas widerspricht also nicht dem
Erwartungsnutzenprinzip. 

Nun könnte man fragen, ob dann denn nicht auch ein beobachtetes risikofreudiges
Verhalten "`in Wirklichkeit"' bzw. auf einer höheren Ebene dem
Erwartungsnutzenprinzip entspricht, sofern man nur die evolutionäre Größe richtig
identifiziert, auf die sich das Verhalten bezieht. Dazu ist zweierlei zu sagen:
1) Solange die entsprechenden Größen nicht tatsächlich empirisch identifiziert
werden (können), so dass man diese Annahme überprüfen kann, muss die Theorie
durch die entsprechenden empirischen Befunde als widerlegt gelten. 
\marginline{Unauf\-lösbarkeit des "`Framing"'-Effekt}
2) Selbst wenn
dies gelingen würde, dann wäre damit noch nicht der "`Framing"'-Effekt erledigt,
d.h. wir könnten das Ergebnis von Kahnemann und Tverskys Experiment (siehe Seite
\pageref{tverskyBeispiel}) zwar noch in dem Punkt mit der Theorie vereinbaren,
dass der Unterschied in der Bewertung der Alternativen innerhalb jeder
Vergleichsgruppe erklärt wäre, nicht aber die Diskrepanz im Verhalten zwischen
den Vergleichsgruppen, die nicht mehr auf der unterschiedlichen Zusammensetzung
des Erwartungswertes, sondern nur auf der unterschiedlichen Formulierung des
Fallbeispiels beruht. Hierbei handelt es sich um ein genuin psychologisches
Phänomen, das mit dem Erwartungsnutzenprinzip auf keinen Fall mehr in Einklang
zu bringen ist.

\subsubsection{Ellsberg Paradox}

Ein anderes Paradox ist das Ellsberg Paradox. Es entsteht so: Jemand hat die Wahl
zwischen zwei Arten von Glückspielen. Bei dem ersten muss sie eine Kugel aus
einer Urne ziehen, die zur Hälfte rote und zur Hälfte schwarze Kugeln enthält,
wobei sie gewinnt, wenn sie eine rote Kugel zieht. Bei dem zweiten Spiel muss sie
wieder aus einer Urne mit roten und schwarzen Kugeln ziehen und gewinnt wieder,
wenn sie eine rote Kugel zieht. Nur weiss sie bei dem zweiten Spiel nicht
wieviele rote und schwarze Kugeln die Urne enthält.

Die meisten Menschen würden in einer solchen Situation angeblich das erste Spiel
mit bekannter Kugel-Verteilung vorziehen \cite[S.
24]{myerson:1991}.\footnote{Über den bedeutenden Entscheidungstheoretiker Savage
geht das Gerücht um, ``that the author of what is perhaps the most elegant
derivation of expected utility theory [...] reported after careful consideration
of the problem in the light of his theory, he would still want to choose I and
IV'' (wie bei einem echten Gerücht ausnahmsweise mal ohne Quellenangabe ;) ).}
 Ein "`Paradox"' entsteht dann, wenn man das
Indifferenzprinzip (siehe Kapitel \ref{Indifferenzprinzip}) voraussetzt, das
besagt, dass man bei unbekannten Wahrscheinlichkeiten eine Gleichverteilung
voraussetzen soll. Akzeptiert man das Indifferenzprinzip, dann handelt es sich
aber wiederum nicht um ein Paradox, sondern -- sofern die Behauptung über das,
was die meisten Menschen tun würden stimmt -- lediglich um einen Widerspruch
zwischen Theorie um Empirie, der zeigt, dass das Indifferenzprinzip empirisches
beobachtbares Entscheidungsverhalten bei Entscheidungen unter Unwissenheit nicht
richtig beschreibt. Lehnt man das Indifferenzprinzip überhaupt ab, so entsteht
von vornherein kein Paradox.

\subsubsection{St. Petersburg Paradox}

Das St. Petersburg Paradox setzt unbeschränkte Nutzenskalen voraus. Bei den
Beweisen der in der letzten Vorlesung vorgestellten Fassung der
Neumann-Morgensternschen Nutzentheorie wurde von der Voraussetzung
begrenzter Nutzenskalen Gebrauch gemacht (siehe Seite
\pageref{bestesSchlechtestesGut}). Man kann die
Nutzentheorie jedoch auch mit unbeschränkten Nutzenskalen konstruieren, 
nur fallen dann die mathematischen Beweise etwas komplizierter aus.

Das St. Petersburg Paradox beruht auf dem un\-be\-schränkten St.
Peters\-burg-Spiel, welches nachfolgenden Regeln gespielt wird: Es wird eine
Münze geworfen. Zeigt sie Kopf, dann erhält der Spieler 2 € und das Spiel ist
beendet. Andernfalls wird sie ein weiteres Mal geworfen. Zeigt sie diesmal Kopf,
so erhält der Spieler 4 €. Wenn nicht wird die Münze ein weiteres Mal geworfen
und bei Kopf 8 € ausgezahlt usw. Das Paradox besteht darin, das -- rein
theoretisch -- ein Akteur bereit sein müsste, jeden Preis dafür zu zahlen, um an
dem Spiel teilzunehmen, denn der Erwartungswert des St. Petersburgspiels
berechnet sich nach: \[ EW = \frac{1}{2}\cdot 2 + \frac{1}{4}\cdot 4 + \ldots +
\frac{1}{2^n} \cdot 2^n + \ldots = 1+1+1+\ldots = \infty \]

Nun ist aber nicht wirklich einzusehen, warum das ein Problem sein sollte. In
der Praxis gibt es keine unendlichen Spiele, so dass das Problem in der Praxis
auch nicht auftreten kann. Was die Theorie betrifft, so bleibt unverständlich,
was man dagegen einwenden sollte, dass irgendeine Option unendlich viel wert
ist, wenn man in der Theorie schon unbegrenzte und damit potentiell unendlich
große Nutzenwerte zulässt.
 
\subsubsection{Das Hellseherparadox} (auch bekannt als "`Newcomb's Paradox"')
\label{Hellseherparadox}

Das Hellseherparadox taucht des öfteren in phil"-osophischen Diskussionen
auf, wenn solche Fragen erörtert werden, wie die des Unterschieds zwischen
Korrelation und Kausalität oder der Mög"-lichkeit zeitlich rückwärts gerichteter
Kausalität. Für die Entscheidungstheorie hat das Hellseherparadox
vergleichsweise geringere Bedeutung, zumal es sich ebenso leicht wie die
anderen lösen lässt. Die Geschichte zu diesem Paradox ist zunächst die
Folgende: 

\begin{quotation}
Ein Hellseher hat in einem Raum zwei Schachteln aufgestellt, eine rote und eine blaue. In die rote
Schachtel legt er 1.000 €. Die blaue Schachtel ist zunächst leer. Nun wird
einer der Zuschauer gebeten, den Raum zu verlassen. Wenn er wiederkehrt, wird er
vor die Wahl gestellt entweder nur die blaue oder beide Schachteln zu nehmen.
Er bekommt dann den Inhalt derjenigen Schachteln, die er genommen hat. Damit
das Ganze interessanter wird, erklärt ihm der Hellseher, dass er inzwischen
vorhersagen wird, welche Entscheidung der Zuschauer treffen wird, und dass er,
wenn er vorhersagt, dass der Zuschauer nur die blaue Schachtel nimmt, 1.000.000
€ in die blaue Schachtel legen wird. Dem Zuschauer ist bekannt, dass der
Vorhersager bisher in 90\% der Fälle richtig vorhersagt hat. Welche Schachtel
sollte der Zuschauer wählen? \cite[S. 109]{resnik:1987}
\end{quotation}

Das Paradox entsteht nun dadurch, dass man mit Hilfe der
Entscheidungstheorie scheinbar genauso gut die eine wie die andere Lösung
rechtfertigen kann. 

{\em 1. Rechtfertigung der Wahl beider Schachteln}: Da der Hellseher seine
Vorhersage abgibt, bevor der Zuschauer eine Wahl trifft, sind die möglichen
Zustände (blaue Schachtel ist leer oder blaue Schachtel ist nicht leer)
unabhängig von der Wahl des Zuschauers. 
Als Tabelle dargestellt sieht das Entscheidungsproblem wie unten abgebildet aus,
wobei die Wahrscheinlichkeiten für das Eintreten der Ereignisse unbekannt sind,
aber wegen der Unabhängigkeit von den Handlungen dieselben sind:
\begin{center}
\begin{tabular}{c|c|c|}
\multicolumn{1}{c}{}   & \multicolumn{1}{c}{blaue Schachtel leer}   
                       & \multicolumn{1}{c}{nicht leer} \\ \cline{2-3} 
Nimm blaue Schachtel   &    0 €                 & 1M € \\ \cline{2-3} 
Nimm beide Schachteln  & 1.000 €                & 1M + 1.000 € \\ \cline{2-3}
\end{tabular}
\end{center}
Wie man sieht, ist die Handlung beide Schachteln zu nehmen streng dominant,
d.h. sie liefert, welches Ereignis auch immer eintritt, stets das bessere
Ergebnis. Also sollte der Zuschauer in jedem Fall beide Schachteln nehmen.

{\em 2. Rechtfertigung der Wahl der blauen Schachtel}: Der Hellseher verfügt
offenbar tatsächlich über die Gabe des Hellsehens, sonst würde er nicht zu 90\%
richtig vorhersagen. Also variiert die Wahrscheinlichkeit, mit der die blaue
Schachtel leer ist oder nicht, mit der Wahl, die der Zuschauer trifft. 
Die Entscheidungstabelle müsste korrekterweise so dargestellt werden:

\begin{center}
\begin{tabular}{c|c|c|}
\multicolumn{1}{c}{}   & \multicolumn{1}{c}{blaue Schachtel leer}   
                       & \multicolumn{1}{c}{nicht leer} \\ \cline{2-3} 
Nimm blaue Schachtel   &    0 € (p=0.1)         & 1M € (p=0.9) \\ \cline{2-3} 
Nimm beide Schachteln  & 1.000 € (p=0.9)        & 1M + 1.000 € (p=0.1)
\\ \cline{2-3}
\end{tabular}
\end{center}
Da es sich um eine Entscheidung unter Risiko handelt, bei der das
Erwartungsnutzenprinzip gilt, ist unser Zuschauer gut beraten, wenn er nur die
blaue Schachtel nimmt.

Handelt es sich hierbei tatsächlich um ein Paradox und leidet die
Entscheidungstheorie an Antinomien, d.h. an inneren Widersprüchen? Wie bei
sovielen philosophischen Antinomien\footnote{Die berühmten Antinomien aus Kants
"`Kritik der reinen Vernunft"' sind dafür das paradigmatische Beispiel, leider
auch hinsichtlich der Tatsache wie ein mangelndes Verständnis der logischen
Situation zu philosophischen Irrtümern führen kann.} entsteht der Schein eines
Widerspruchs nur dadurch, dass bei beiden Argumentationen jeweils von 
unterschiedlichen Voraussetzungen ausgegangen wird. In Wirklichkeit handelt es
sich nämlich gar nicht um einen Widerspruch, sondern darum, dass in dem einen
wie in dem anderen Fall aus unterschiedlichen Voraussetzungen Unterschiedliches
abgeleitet wird. Bei der ersten Rechtfertigung wird vorausgesetzt, dass
Hellseherei nicht möglich ist. Bei der zweiten dagegen, dass sie möglich ist.
Die beiden Argumentationen kommen also deshalb zu unterschiedlichen
Ergebnissen, weil sie von unterschiedlichen Problemspezifikationen ausgehen.
\marginline{Berück\-sichtigung der Problem\-spezikifaktion}
Dass die Entscheidungstheorie bei unterschiedlichen und einander
widersprechenden Problemspezifikationen unterschiedliche Lösungen liefert 
ist nur natürlich und
verweist nicht auf einen Widerspruch innerhalb der Entscheidungstheorie.

