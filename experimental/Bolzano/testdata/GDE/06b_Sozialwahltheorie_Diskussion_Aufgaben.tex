\subsection{Aufgaben}

\begin{enumerate}
  \item Eignet sich der Satz von Arrow -- wenn überhaupt -- eher zur Kritik
  individualistischer oder kollektivistischer Politikkonzeptionen? Warum?
  
  \item Ein Vorschlag, um die Probleme zu umgehen, die aus zyklischen
  kollektiven Präferenzen entstehen, besteht darin, zwischen den in einem
  Zyklus erfassten Gütern einfach Indifferenz anzunehmen (QUELLENANGABE).
  Wenn also $a \succ_K b \succ_K c \succ_K a$, dann setze man einfach $a \sim_K
  b \sim_K c$ fest und eliminiere dadurch den Zyklus. Warum wird durch dieses
  Verfahren das Prinzip der {\em paarweisen Unabhängigkeit} verletzt? 
  
  \item \label{BordaAufgabe} Bei der {\em Borda}-Zählung wird jedem Gut in den
  individuellen Präferenzen eine Rangzahl zugewiesen (ähnlich wie bei der Rangordnungsregel
  für Entscheidungen unter Unwissenheit, Seite \pageref{Rangordnungsregel}).
  Die Rangzahlen für jedes Gut werden dann zusammengezählt und aus der Summe
  die Platzierung in dern kollektiven Präferenzen bestimmt. Zeige:
  \begin{enumerate}
    \item Bei der Borda-Zählung können anders als beim Condorcet-Verfahren
    keine transitiven kollektiven Präferenzen entstehen.
    \item Die Borda-Zählung verletzt das Prinzip der paarweisen Unabhängigkeit
    (Seite \pageref{ArrowVoraussetzungen}f.).
  \end{enumerate}
  
  \item Man könnte sich für Rikers Sichtweise, dass demokratische
  Entscheidungsverfahren nicht das leisten (können), was sie leisten sollen,
  nämlich den Willen der Mehrheit, des Volkes etc. zum Ausdruck zu bringen,
  auch andere Argumente überlegen, die sich nicht auf den Public Choice Ansatz
  stützen. Diskutieren Sie die folgenden beiden:
  \begin{enumerate}
    \item Bei der Bundestagswahl können die Bürger gar nicht ihre Präferenzen
    zu den verschiedenen politischen Themen zum Ausdruck bringen, sondern
    müssen sich zwischen einer kleinen Zahl von Gesamtpaketen entscheiden.
    Insofern kann von echter Demokratie keine Rede sein.
    \item Eine häufig zu hörende Klage: Die Programme der großen Partein
    unterscheiden sich im Grunde kaum noch voneinander. Alle tendieren zur
    Mitte hin. Wo bleibt für den Wähler da noch die Möglichkeit sich zu
    entscheiden?
  \end{enumerate}
\end{enumerate}
