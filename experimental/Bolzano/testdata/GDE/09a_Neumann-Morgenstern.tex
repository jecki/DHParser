\section{Die Neumann-Morgensternsche Nutzentheorie}
\label{NeumannMorgenstern}
Schon zuvor (Kapitel \ref{Risiko} wurde gezeigt, wie man mit Hilfe des
Erwartungsnutzens Entscheidungen unter Risiko treffen kann. In diesem Kapitel
werden wir auf die theoretischen Grundlagen des Erwartungsnutzens und besonders
des sogenannten "`Neumann-Morgensternschen"' Nutzenbegriffs eingehen, der schon
früher als "`kardinale Nutzenfunktion"' eingeführt wurde (siehe Kapitel
\ref{KardinalerNutzen}).

Die Grundidee der Neumann-Morgensternschen Nutzentheorie besteht darin, neben
den bestehenden Gütern (bzw. den Ergebnissen von Ent\-schei\-dungs\-pro\-zessen)
"`Lotterien"' als gedachte Güter einzuführen und durch den Vergleich (hinsichtlich
der Präferenzrelation) von Lotterien und Gütern bzw. Lotterien untereinander
sowie mit Hilfe von als selbst-evident angesehenen Konsistenzbedingungen eine
kardinale Nutzenfunktion und das Prinzip des
Erwartungsnutzen abzuleiten.  Die folgende Darstellung lehnt sich vor allem
an Resnik an \cite[S. 88-98]{resnik:1987}. Wie sehen diese "`Lotterien"'
aus und wie kommen sie zu Stande?

\marginline{Begriff der "`Lotterie"'}
Grundsätzlich ist eine Lotterie immer eine Wahrscheinlichkeitsverteilung über
einer disjunkten, aber zugleich erschöpfenden Menge von Ereignissen.
Kompliziert und, wenn es sich nicht gerade um Geldwerte handelt,
zugegebenermaßen etwas unplausibel wird die Theorie dadurch, dass diese
Lotterien als mögliche Güter bzw. erzielbare Ergebnisse eines
Entscheidungsprozesses in die Präferenzrelation eigeordnet werden können
müssen. Das stellt sich dann etwa wie folgt dar:

Angenommen jemand ordnet seine Präferenzen bezüglich der drei Güter "`Eiscreme"',
"`Joghurt"' und "`Trockenes Brot"' auf diese Weise:
\marginline{Problematik der Präferenzbestimmung bei Lotterien}
\begin{center}
Eiscreme $\succ$ Joghurt $\succ$ Trockenes Brot
\end{center}
Dann postuliert die Theorie,\footnote{Siehe die "`Kontinuitätsbedingung"'
weiter unten auf Seite \pageref{Kontinuitaet}.}, dass es eine Lotterie mit zwei
möglichen Preisen, nämlich "`Eiscreme"' als Hauptgewinn und
"`Trockenes Brot"' als Niete gibt (wobei man den Hauptgewinn mit einer
bestimmten Wahrscheinlichkeit $a$ erhält und die Niete dementsprechend mit der
inversen Wahrscheinlichkeit $1-a$), so dass zwischen dieser Lotterie und dem in
der Mitte eingeordneten Gut "`Joghurt"' Indifferenz herrscht. Angenommen eine
Person speist gerne Jogurt, so dass dieser Indifferenzpunkt bei einer
Gewinnwahrscheinlichkeit von $a=80\% $ erreicht wird. Dann gilt, wenn wir
unsere gedachte Lotterie mit "`Lotterie (a=0.8, Eiscreme, Trockenes Brot)"'
bezeichnen:
\begin{center}
Lotterie (a=0.8, Eiscreme, Trockenes Brot) $\sim$ Joghurt
\end{center}
Wozu in aller Welt soll das gut sein? Und woher soll nun einer wissen, ob
er zwischen Jogurt und einer 80\%-igen Gewinnchance auf Eiscreme (bei Strafe von
trockenem Brot) indifferent ist und nicht etwa einer 70\%-igen oder 60\%-igen
etc.? Die Antwort auf die erste Frage ist, dass sich damit eine raffinierte,
und unter einer großen Gruppe von Ökonomen und einer kleinen Gruppe von
Philosophen überaus populäre Nutzentheorie aufbauen lässt, die wir gleich
kennen lernen werden. Die Antwort auf die zweite Frage stellt eine etwas
schwierige Angelegenheit dar, die man lange diskutieren müsste. So recht
überzeugend lässt sie sich, wenn es sich nicht gerade mal wieder um Geldwerte
handelt, offen gestanden nicht beantworten, so dass wir an dieser Stelle schon
eine gehörige Portion guten Willen mitbringen müssen, um die Theorie zu
akzeptieren. Zugleich wird an dieser Stelle deutlich, warum es mit dem 
Hilfsmittel der Lotterien immer möglich ist, aus beliebigen wohlgeformten
Präferenzen eine kardinale Nutzenfunktion zu erzeugen: Indem wir unserem Akteur 
nämlich eine definitive Wahrscheinlichkeitsangabe abnötigen, zwingen wir ihn zu
genau der Zahlenangabe, die wir brauchen, um eine Intervallskala zu konstruieren, 
und die uns beim bloß ordinalen Nutzen fehlt. 

Ist man dazu bereit, sich die Theorie trotz ihrer m.E. zweifelhaften
Voraussetzungen anzuhören, so wird man Lotterien der Einfachheit halber in der
Form darstellen:
\marginline{Formale Darstellung von Lotterien}
\[ L(a, x, y) \] 
Dabei sind $x$ und $y$ zwei beliebige Güter (bzw. Ergebnisse).
$a$ ist die Wahrscheinlichkeit, mit der der Gewinn $x$ herauskommt, und $1-a$ ist
dementsprechend die Wahrscheinlichkeit mit der die "`Niete"' $y$ gezogen wird. In
allgemeiner Form, d.h. bei mehr als zwei Gütern, werden Lotterien so dargestellt:
\[L((p_1,\ldots,p_n), (x_1,\ldots,x_n)) \qquad p_1 + \ldots + p_n = 1\] 
wobei
$x_1,\ldots, x_n$ ein Tupel von $n$ Gütern (oder Ergebniss) ist und $p_1,\ldots,
p_n$ die Wahrscheinlichkeiten mit der das jeweilige Gut "`gewonnen"' wird.
Im Folgenden werden wir uns aber auf zwei-stellige Lotterien beschränken, da
man mehrstellige Lotterien immer als verschachtelte zweistellige Lotterien
darstellen kann.

Wenn man schon zulässt, dass Güter mit dieser Art von Lotterien darauf hin
verglichen werden können, ob irgendein Akteur indifferent zwischen ihnen ist,
dann ist es nur ein kleiner Schritt, auch noch Lotterien mit Lotterien zu
vergleichen. D.h. wenn $L_1(a_1, x_1, y_1)$ eine Lotterie ist und 
$L(a_2, x_2, y_2)$ eine weitere, dann kann man für jedes Gut oder jede
Lotterie die bezüglich der Präferenzen des Akteurs zwischen $L_1$ und $L_2$
eingeordnet ist eine Lotterie $L(b, L_1, L_2)$ konstruieren, so dass der Akteur
zwischen dieser Lotterie und dem mittleren Gut (oder der mittleren Lotterie)
indifferent ist. 

Auf diese Weise kann man nach folgenden drei Regeln eine "`vollständige Menge"'
\label{MengeVonLotterien}
\footnote{Dieses Verfahren, aus einer Grundmenge mit Hilfe bestimmter
"`Produktionsregeln"' einen "`Abschluss"' zu erzeugen (wobei ein {\em Abschluss}
allgemein als die Menge aller derjenigen Objekte verstanden werden kann, die aus
einer Menge von Grundobjekten mit Hilfe gegebener Produktionsregeln erzeugt
werden können), ist uns schon bei dem {\em De Finetti-Abschluss} in der letzten
Vorlesung begegnet (siehe Seite \pageref{DeFinettiAbschluss}).} von Lotterien
konstruieren \cite[S. 91]{resnik:1987}:
\marginline{Definition der Menge aller Lotterien}
\begin{enumerate}
  \item Jedes {\em Grundgut} ("`basic prize"') ist eine Lotterie. (Im
  Zweifelsfall kann man für ein Gut $x$ ja immer die Lotterie $L(1, x, x)$
  nehmen.) \label{bestesSchlechtestesGut} Es wird
  weiterhin angenommen, dass es ein oder mehrere beste bzw. schlechteste
  Grundgüter gibt (was immer gegeben ist, wenn die Menge der Grundgüter endlich
  ist).
  \item Wenn $L_1$ und $L_2$ Lotterien sind, dann auch $L(a, L_1, L_2)$ für
  jedes beliebige $a$ mit $0 \leq a \leq 1$.
  \item Es gibt keine Lotterien außer den nach den ersten beiden Regeln
  konstruierten.
\end{enumerate}
 
Weiterhin wird verlangt, dass für die Lotterien folgende Bedingungen gelten
\cite[S. 90-92]{resnik:1987} \label{LotterienBedingungen}:
\marginline{"`Konsistenz"'-bedingungen von Lotterien}
\begin{enumerate}
  \item {\em Ordnungsbedingung}: Auf der vollständigen Menge der Lotterien ist
  eine Präferenzrelation definiert, (die bezüglich der ursprünglichen Güter 
  mit der auf der Menge dieser Güter definierten Präferenzrelation
  übereinstimmen sollte.)
  \item {\em Kontinuitätsbedingung}:\label{Kontinuitaet}
  \marginline{Kontinuitäts\-bedingung} 
  Für beliebige Lotterien $x$,$y$ und $z$
  gilt: Wenn $x \succ y \succ z$, dann gibt es eine Lotterie 
  $L(a, x, z)$, so dass $y \sim L(a, x, z)$.  
  \item {\em Bedingung der höheren Gewinne}:\label{BedHoehereGewinne}
  Für beliebige Lotterien $x$,$y$
  und $z$ und jede beliebige Wahrscheinlichkeit $a > 0$ gilt: $x \succ y$ genau
  dann wenn $L(a, x, z) \succ L(a, y, z)$.
  (Einfach gesagt: Eine Lotterie wir dann vorgezogen, wenn man "`höhere Preise"'
  gewinnen kann.)
  \item {\em Bedingung der besseren Chancen}: Für jedes Paar von Lotterien $x$
  und $y$ und beliebige Wahrscheinlichkeiten $a$ und $b$ gilt: 
  Wenn $x \succ y$ dann ist $L(a, x, y) \succ L(b, x, y)$ genau dann wenn $a > b$.
  (Einfach gesagt: Bei gleichen Preisen wird die Lotterie mit den besseren
  Chancen bevorzugt.)
  \item {\em Reduzierbarkeit zusammengesetzter Lotterien}:
  \label{Reduzierbarkeit}\marginline{Reduzier\-barkeit von Lotterien} 
  Für jede zusammengesetzte Lotterie der Form $L(a, L(b,x,y), L(c,x,y))$ gilt:
  \[L(a, L(b,x,y), L(c,x,y)) \sim L(ab+(1-a)\cdot c, x, y) \] 
  (Einfach ausgedrückt: Zusammengesetzte Lotterien, deren innere Lotterien
  dieselben Güter enthalten (!), können ent\-sprech\-end den Gesetzen der
  Wahrscheinlichkeitsrechnung auf einfachere reduziert werden.)
\end{enumerate}

Wenigstens die zweite und dritte dieser Bedingungen kann man als selbstevident
betrachten. Die anderen Bedingungen sind zumindest plausibel, wenn man sich
überhaupt auf das Gedankenexperiment mit den "`Lotterien"' einlässt.
Nun lässt sich beweisen, dass man, wenn diese
Bedingungen gegeben sind, eine Nutzenfunktion konstruieren kann, die die
Erwartungsnutzeneigenschaft hat, und die zugleich eine kardinale Nutzenfunktion
ist. Insgesamt muss die so konstruierte Nutzenfunktion $u$ also die folgenden
Eigenschaften haben:
\marginline{Bed. f. kardinale Nutzenfunktionen mit
Erwartungsnutzeneigenschaft}
\begin{enumerate}
  \item $u(x) > u(y)$ genau dann wenn $x \succ y$
  \item $u(x) = u(y)$ genau dann wenn $x \sim y$
  \item $u(L(a,x,y)) = au(x) + (1-a)u(y)$ ({\em Erwartungsnutzeneigenschaft})
  \item Jede Nutzenfunktion $u'$, welche die ersten drei Bedingungen erfüllt,
  kann durch positiv lineare Transformation in die Nutzenfunktion $u$ überführt
  werden.
\end{enumerate}
Wie kann man das beweisen? Resnik folgend kann der Beweis in zwei Schritten
geführt werden, indem zuerst die {\em Existenz} einer Nutzenfunktion bewiesen
wird, die die ersten drei Eigenschaften erfüllt, und dann die {\em
Eindeutigkeit} dieser Nutzenfunktion bis auf positive lineare Transformation.

\subsection{Vorbereitung des Beweises}
\label{Corrolarien}
Bevor wir diesen Beweis führen, sollen einige unmittelbare Corrolarien der
Bedingung der höheren Gewinne und der Bedingung der besseren Chancen vorgestellt
werden, die uns helfen, den folgenden Beweis leichter zu führen. Für den Beweis
dieser Corrolarien verwenden wir die Tatsache, dass die Lotterie $L(a,x,y)$
identisch ist mit der Lotterie $L(1-a,y,x)$ und daher entsprechnd ersetzt werden
kann.

\begin{enumerate}
  \item {\em Corrolar zur Bedingung der besseren Chancen}: 
  \[\forall_{x,y}\forall_{a,b} \qquad x \prec y \Rightarrow 
    L(a,x,y) \prec L(b,x,y) \quad \Leftrightarrow \quad a > b
  \]
  {\em Beweis}: Sei $x \prec y$, dann ist $y \succ x$, dann gilt aber nach der
  Bedingung der besseren Chancen:
  \begin{eqnarray*}
  L(1-b,y,x) \succ L(1-a,y,x) & \Leftrightarrow & 1-b > 1-a \\
  L(b,x,y) \succ L(a,x,y) & \Leftrightarrow & b < a \\
  L(a,x,y) \prec L(b,x,y) & \Leftrightarrow & a > b \\
  \end{eqnarray*}
  
  \item {\em Corrolar zur Bedingung der besseren Chancen}:
  \[\forall_{x,y}\forall_{a,b} \qquad x \not\sim y \Rightarrow
    L(a,x,y) \not\sim L(b,x,y) \quad \Leftrightarrow \quad a \not= b
  \] 
  {\em Beweis}: Wenn $x \not\sim y$, dann ist entweder $x \succ y$ oder
  $x \prec y$. Wenn $x \succ y$, dann ist nach der Bedingung der besseren
  Chancen entweder $L(a, x, y) \succ L(b, x, y) \Leftrightarrow a > b$
  oder $L(b, x, y) \succ L(a, x, y) \Leftrightarrow b > a$, also in jedem Fall
  $ L(a,x,y) \not\sim L(b,x,y) \Leftrightarrow a \not= b$. Wenn aber $x \prec
  y$, dann folgt aus dem vorherigen Korrolar auf dieselbe Weise, dass
  $ L(a,x,y) \not\sim L(b,x,y) \Leftrightarrow a \not= b$. Da dieser Ausdruck
  sowohl für $x \succ y$ als auch für $x \prec y$ folgt, folgt er in jedem Fall
  für $x \not\sim y$.
 
  \item {\em Corrolar zur Bedingung der höheren Gewinne}:
  \[\forall_{x,y,z}\forall_{a<1} \qquad 
  x \succ y \Leftrightarrow L(a, z, x) \succ L(a, z, y) \]
  Inhaltlich bedeutet dies, dass die Bedinung der höheren Gewinne auf der
  zweiten Stelle der Lotterie ebenso gilt wie auf der ersten.
  Beweis: Da nach der Bedingung der höhren Gewinne $x \succ y
  \Leftrightarrow L(b, x, z) \succ L(b, y, z)$ für alle $b>0$, gilt für alle 
  $(1-b) < 1$ auch $x \succ y \Leftrightarrow L(1-b,z,x) \succ L(1-b,z,y)$. 
  Mit $a := 1-b$ gilt dann aber auch ("`ohne Beschränknung der Allgemeinheit"'
  wie die Mathematiker sagen, da man für jedes $1-b$ ein entsprechendes
  $a := 1-b$ definieren kann) die Behauptung.  
  
  \item {\em Corrolar zur Bedingung der höheren Gewinne}:\\
  Für alle Lotterien $x,y,z$ und alle $a$ mit $0<a<1$ gilt:
  \[x \not\sim y \quad \Leftrightarrow \qquad L(a,z,x) \not\sim L(a,z,y)
  \quad \wedge \quad L(a,x,z) \not\sim L(a,y,z)
  \]
  Beweisskizze: Wenn $x \not\sim y$, dann ist entweder $x \succ y$ oder $x \prec
  y$. In beiden Fällen ergibt sich die Implikation, dass 
  $(L(a,z,x) \not\sim L(a,z,y)) \wedge (L(a,x,z) \not\sim L(a,y,z))$ unmittelbar
  aus der Bedingung selbst. Zu zeigen ist nun noch, dass auch die
  umgekehrte Implikation gilt: Wenn $(L(a,z,x) \not\sim
  L(a,z,y)) \wedge (L(a,x,z) \not\sim L(a,y,z))$ dann $x \not\sim y$. Wiederum
  sind zwei Fälle von $\not\sim$ zu unterscheiden, nämlich $\succ$ und $\prec$. 
  Die Implikation ergibt sich dann wiederum
  unmittelbar aus der Bedingung selbst.

  \item {\em Corrolar: Subsitutionsgesetz}:
  \label{Substitutionsgesetz}\marginline{Sub\-sti\-tu\-ier\-bar\-keit von
  Lotterien} \[\forall_{L^*}\forall_{b} \qquad L^* \sim L(a,x,y)
  \quad \Rightarrow \quad L(b,L^*,z) \sim L(b, L(a,x,y), z) \] 
  Beweis: Aus dem vorhergehenden Corrolar ergibt sich (bis auf die Sonderfälle
  $b=0$ und $b=1$), dass 
  \[L(b, L^*, z) \not\sim L(b,L(a,x,y), z) \quad \Rightarrow \quad L^*
  \not\sim L(a,x,y)\] Im Umkehrschluss muss daher gelten:
  \[ L^* \sim L(a,x,y) \quad \Rightarrow \quad L(b,L^*,z) \sim L(b, L(a,x,y), z)
  \] Für die Sonderfälle $b=0$ und $b=1$ gilt die Formel unmittelbar, wie man sich
  leicht überlegen kann. 
\end{enumerate}

\subsection{Existenz der Nutzenfunktion}

Um den Beweis der {\em Existenz} einer Nutzenfunktion mit der
Erwartungsnutzeneigenschaft zu führen, konstruieren wir eine solche Funktion $u$
und zeigen, dass sie eine Nutzenfunktion ist (Eigenschaften 1 und 2) und dass sie
die Erwartungsnutzeneigenschaft besitzt (Eigenschaft 3). Dazu bezeichnen wir
zunächst entsprechend Resniks Darstellung \cite[S. 94]{resnik:1987} das beste Gut
als $B$ ("`best"') und das schlechteste Gut als $W$ ("`worst"'). (In dem Fall,
dass es mehrere beste oder schlechteste Güter gibt, bezeichnet $B$ ein beliebiges
bestes Gut und $S$ ein beliebiges schlechtestes Gut.) Dann setzen wir
fest:\marginline{Obere und untere Begrenzung der Nutzenskala} \[ u(B) := 1
\qquad\mbox{und}\qquad u(x) := 1 \qquad
 \mbox{für jede Lotterie $x$ mit} \qquad x \sim B \] \[ u(W) := 0
\qquad\mbox{und}\qquad u(x) := 0 \qquad \mbox{für jede Lotterie $x$ mit} \qquad x
\sim W \] Nun betrachten wir eine beliebige Lotterie $x$, die hinsichtlich der
Präferenzrelation zwischen $B$ und $W$ eingeordnet ist (also: $B \succ x \succ
W$). Nach der {\em Kontinuitätsbedingung} gibt es dann auch eine Lotterie $L(a,
B, W) \sim x$ mit einer Wahrscheinlichkeit $a$, $0 \leq a \leq 1$. Wir können
nun\marginline{Definition einer Nutzen\-funktion~$u$} \[u(x) := a\] setzen, falls
die Wahrscheinlichkeit $a$ eindeutig bestimmt ist. Das ist aber der Fall, weil
für jedes $a' \neq a$ auf Grund der {\em Bedingung der besseren Chancen} gilt:
$L(a', B, W) \not\sim L(a, B, W)$ Da die Indifferenzrelation $\sim$ transitiv ist
("`wohlgeformte Präferenzen"'), muss dann auch gelten: $L(a', B, W) \not\sim x$.

Man beachte, dass aus der Definition $u(x) := a$ für alle Lotterien $x$
unmittelbar folgt:
\[x \sim L(u(x), B, W) \]
Genau dasselbe ist es zu sagen, dass für jede bliebige Lotterie $L(a,
x, y)$ gilt: \[L(a, x, y) \sim L(u(L(a,x,y)), B, W) \]
Von diesem Zusammenhang werden wir weiter unten noch Gebrauch machen.

Mit $u(x) = a$ haben wir dann aber bereits eine Funktion definiert, die jeder
Lotterie $x$ einen eindeutigen Wahrscheinlichkeitswert $a$ zuordnet. Zu zeigen
ist noch, dass es sich dabei um eine Nutzenfunktion mit der
Erwartungsnutzeneigenschaft handelt. Dazu müssen wir zunächst nachweisen, dass
die ersten drei der oben aufegführten Eigenschaften für die so definierte
Funktion $u$ gegeben sind. 

{\em Teilbeweis} der Eigenschaft $u(x) > u(y)$ genau dann wenn $x \succ y$:
\marginline{Monotonie\\von $u$}
Wenn $u(x) = a$ für dasjenige $a$, für welches gilt $L(a, B, W) \sim x$, dann
ergibt sich durch Einsetzen unmittelbar $x \sim L(u(x), B, W)$. Aufgrund der
{\em Bedingung der besseren Chancen} wissen wir, dass 
\[ L(u(x), B, W) \succ L(u(y), B, W) \qquad \Leftrightarrow \qquad u(x) > u(y)
\] 
Da jeweils gilt $x \sim L(u(x), B, W)$ und $y \sim L(u(y), B, W)$ können wir
die Lotterien in der vorkommenden Äquivalenzaussage durch $x$ und $y$ ersetzen
und erhalten das Gesuchte.

{\em Teilbeweis} der Eigenschaft $u(x) = u(y)$ genau dann wenn $x \sim y$:
Aus der {\em Bedingung der besseren Chancen} ergibt sich, dass 
\[ L(u(x), B, W) \sim L(u(y), B, W) \qquad \Leftrightarrow \qquad u(x) = u(y)
\] 
denn wäre $u(x) \neq u(y)$, dann wäre entweder $u(x) > u(y)$ oder $u(x) < u(y)$,
und in beiden Fällen besagt die Bedingung der besseren Chancen, dass dann auch
für die entsprechenden Lotterien $\succ$ oder $\prec$ gelten muss, so dass
$\sim$ nur noch gelten kann, wenn $u(x) = u(y)$. Durch Ersetzen analog zum
Vorigen erhalten wir wiederum das Gesuchte.

{\em Teilbeweis} der Eigenschaft $u(L(a,x,y)) = au(x) + (1-a)u(y)$.
\marginline{Erwartungs\-nutzen\-eigenschaft\\ von $u$}
Um den Beweis zu führen bedienen wir uns des zuvor als
Corollar bewiesenen Substitutionsgesetzes (siehe Seite
\pageref{Substitutionsgesetz}). Der Einfachheit halber soll dabei $L^*$ für die
Lotterie $L(a, x, y)$ stehen. Nach der Definition der Nutzenfunktion ($u(x) :=
b$ für dasjenige $b$, für welches $x \sim L(b, B, W)$), gilt: \[ x \sim L(u(x),
B, W) \] \[ y \sim L(u(y), B, W) \]
Durch Substitution von $x$ und $y$ in der Lotterie $L^*$ erhalten wir:
\[ L^* \sim L(a, L(u(x),B,W), L(u(y),B,W)) \]
Nach der {\em Reduzierbarkeitsbedingung} ergibt sich daraus:
\[ L^* \sim L(a, L(u(x),B,W), L(u(y),B,W)) \sim L(d, B, W) \]
mit $d = au(x) + (1-a)u(y)$. Da aber (nach unserer Definition von $u$) gilt:
$L^* \sim L(u(L^*), B, W)$, so erhalten wir daraus: 
\[ L(u(L^*),B,W) \sim L(d,B,W) \]
Da auf Grund der Bedingung der besseren Chancen, wie zuvor bewiesen, in diesem
Falle $u(L^*) = d$ sein muss, folgt das Gesuchte. 
Damit ist der Beweis der Existenz einer Nutzenfunktion, der die
Erwartungsnutzeneigenschaft zukommt, abgeschlossen.

\subsection{Eindeutigkeit der Nutzenfunktion}
\label{EindeutigkeitNMU}
Die {\em Eindeutigkeit} der eben definierten 
Nutzenfunktion ist so zu verstehen, dass wir keine Nutzenfunktion mit der
Erwartungsnutzeneigenschaft aus den Bedingungen für Lotterien herleiten können,
die sich nicht positiv linear in alle
anderen daraus ableitbaren Nutzenfunktionen mit
Erwartungsnutzeneigenschaft transformieren lässt.

Wir müssen also zeigen, dass jede beliebige Nutzenfunktion mit
Erwatungsnutzeneigenschft $u'$, die die auf der vollständigen Menge der
Lotterien definierte Präferenzrelation wiedergibt, 
eine positiv linear transformierte der eben
konstruierten Nutzenfunktion $u$ ist, dass also gilt: 
\[ u'(x) = au(x) + b \qquad \mbox{mit} \qquad a > 0\] 
Der Beweis nach Resnik geht wie folgt \cite[S.97/98]{resnik:1987}:

Angenommen, wir verfügen neben der oben konstruierten Nutzenfunktionen $u$ noch
über eine weitere Nutzenfunktion mit Erwartungsnutzeneigenschaft $u'$, die die
vollständige Menge der Lotterien auf eine andere Nutzenskala abbildet. Aus dem
Erwartungsnutzenprinzip ergibt
sich, dass beide Abbildungen {\em surjektiv}
\marginline{Surjetivität von $u$} sind (d.h. dass jeder Wert der
Nutzenskala innerhalb des Intervalls zwischen dem größten und dem kleinsten
Nutzenwert ein Nutzenwert irgendeiner Lotterie ist), denn (Beweisskizze) sei $x$
eine Lotterie, die den höchsten möglichen Nutzenwert hat, und $y$ eine Lotterie,
die den kleinsten möglichen Nutzenwert hat, und sei $j$ irgendein Nutzenwert
dazwischen, dann hat mit $a := (j-u(y))/(u(x)-u(y))$ die Lotterie $L(a, x, y)$
genau den Nutzenwert $j$. Da dies für jedes beliebige $j$ gilt, gehören alle
reellen Zahlen auf der Skala innerhalb des Bereiches vom kleinsten bis zum
größten Nutzenwert zum Wertebereich der Nutzenfunktion.

\marginline{Transformation der $u$-Skala in die $u'$-Skala durch die Abbildung
$I$} Wenn jede Zahl auf der Nutzenskala vom kleinsten bis zum größten
Nutzenwert der Nutzenwert einer Lotterie ist, dann können wir eine Abbildung
 $I$ definieren, die die Nutzenwerte der einen Skala auf die der anderen
abbildet. Dazu definieren wir zunächst $u^{-1}(e)$ als eine Funktion,
\footnote{Bei $u^{-1}$ handelt es sich nicht um eine Umkehrfunktion im strengen
Sinne, da die Funktion $u$ nicht umkehrbar ist, weil sie unterschiedlichen
Argumenten, nämlich verschiedenen Lotterien zwischen denen Indifferenz
herrscht, den gleichen Funktionswert zuordnet.} die jedem Wert $e$ der $u$-Skala 
eine (von möglicherweise mehreren) Lotterien $x$ zuordnet, für die gilt:
$u(x)=e$. Für jede Zahl $e$ auf der $u$-Skala gilt dann:

\[ I(e) := u'(u^{-1}(e)) \]

% Da wir davon ausgegangen sind, dass es ein bestes und ein schlechtestes Gut
% gibt (\label{BSGVerwendet} siehe Seite \pageref{bestesSchlechtestesGut}), so
% muss auch die Nutzenskala $u'$ eine obere und eine untere Schranke haben, 
% die wir mit $d$ und
% $c+d$ bezeichnen können. (Natürlich könnte man sie auch einfach mit zwei
% Variablen $c$ und $d$ bezeichnen statt durch eine Variable und die Summe
% derselben mit einer anderen Variablen, aber diese zunächst
% komplizierter erscheinende Bezeichnungsweise vereinfacht ein wenig das
% Folgende.) 

Im folgenden zeigen wir zunächst, dass für die Funktion $I$ eine der
Erwartungsnutzeneigenschaft von $u$ und $u'$ analoge Eigenschaft gilt, nämlich:
$I(ak+(1-a)m) = aI(k) + (1-a)I(m)$ für jedes $k$ und $m$ auf der
$u$-Skala. Daraus leiten wir dann das Gewünschte ab.

{\em Nachweis der erwartungsnutzenanalogen Eigenschaft von $I$}:
\marginline{Erwartungs\-nutzen\-anloge Eigenschaft von $I$}
 Zunächst
einmal gilt nach der Definition von $I$ und der Erwartungsnutzeneigenschaft von
$u$, dass:
\[u'(L(a,x,y)) = I(u(L(a,x,y))) = I(au(x) + (1-a)u(y)) \]
Nun gilt aber ebenso nach der Erwartungsnutzeneigenschaft von $u'$ und wiederum
nach der Definition von $I$, dass:
\[u'(L(a,x,y)) = au'(x) + (1-a)u'(y) = aI(u(x)) + (1-a)I(u(y))) \]
In beiden Gleichungen steht der Term $u'(L(a,x,y))$. Also kann man die
Gleichungen zusammensetzen, und erhält:
\[I(au(x) + (1-a)u(y)) = u'(L(a,x,y)) = aI(u(x)) + (1-a)I(u(y))) \]
Nun muss man sich nur noch folgendes klar machen: Aufgrund der zurvor
bewiesenen Surjetivität von $u$ gibt es zu jedem $k$ und $m$ auf der 
$u$-Skala mindestens je eine Lotterie $x$ und eine Lotterie $y$, so dass $u(x)
= k$ und $u(y) = m$. Dann gilt aber ohne Beschränkung der Allgemeinheit für
jedes $k$ und $m$ auf der $u$-Skala, dass
\begin{eqnarray*}
I(ak + (1-a)m) & =  & I(au(x) + (1-a)u(y))  \\ 
{ } & = & aI(u(x)) + (1-a)I(u(y))) \\
{ } & = & aI(k) + (1-a)I(m)
\end{eqnarray*}
was die erwartungsnutzenanaloge Eigenschaft von $I$ ist, die nachgewiesen werden
sollte.

% Für jedes beliebige Paar von Werten $k$ und $m$ auf der $u$-Skala gilt, dass
% sie erstens die Nutzenwerte irgendwelcher Lotterien sind, 
% d.h. $\exists_{x,y} u(x)=k \wedge u(y) =
% m$, und dass zweitens für $0 \leq a \leq 1$ der Wert $ak + (1-a)m$ ebenfalls auf der
% Nutzenskala liegt. Weiterhin ergibt sich aus $u(x)=k$ und $u(y)=m$, dass
% $I(k) = u'(x)$ und $I(m) = u'(y)$. 
% Schließlich folgt aus der Erwartungsnutzeneigenschaft für den
% Nutzen der Lotterie $L(a,x,y)$:
% \begin{eqnarray*}
% u(L(a,x,y)) & = & au(x) + (1-a)u(y) = ak+(1-a)m
% \end{eqnarray*}
% Auf beiden Seiten der Gleichung steht ein Nutzenwert der $u$-Skala. Daher können
% wir nun auf beiden Seiten der Gleichung die Funktion $I$ anwenden und erhalten:
% \begin{eqnarray*}
% I(u(L(a,x,y))) & = & I(ak+(1-a)m) \qquad \Leftrightarrow \\
% u'(L(a,x,y)) & = & I(ak+(1-a)m)
% \end{eqnarray*}
% Indem wir uns die Erwartungsnutzeneigenschaft von $u'$ zu Nutze machen und
% außerdem $I(k) = u'(x)$ und $I(m) = u'(y)$ (siehe oben) verwenden, erhalten
% wir:
% \[ I(ak+(1-a)m) = u'(L(a,x,y)) = au'(x) + (1-a)u'(y) = aI(k) + (1-a)I(m) \]

Mit diesem Wissen können wir folgende Rechnung aufstellen:
\marginline{positiv lineare Transformierbarkeit von $u$ in $u'$}
\begin{eqnarray*}
u'(x) & = & I(u(x)) \qquad \qquad \qquad \qquad \qquad \mbox{nach Definition von
$I$} \\ { }   & = & I(u(x)\cdot 1 + (1-u(x)) \cdot 0) \qquad \mbox{etwas
Algebra ;)} \\ { }   & = & u(x)I(1) + (1-u(x))I(0) \qquad \mbox{{\scriptsize
                     erwartungsnutzenanaloge Eigenschaft von $I$}} \\
{ }   & = & u(x)(I(1)-I(0)) + I(0)
\end{eqnarray*}
Wenn wir nun $a := I(1)-I(0)$ und $b := I(0)$ setzen, dann haben wir gezeigt,
dass $u'$ eine linear transformierte von $u$ ist:
\[ u'(x) = au(x)+b \]
Da $I(1) > I(0)$ sein muss (wg. der Monotonieeigenschaft von $u$ (und
damit auch von $u^{-1}$) und $u'$), ist $a > 0$, so dass es sich tatsächlich um
eine {\em positive} lineare Transformation handelt. {\em q.e.d.}

\subsection{Die Bedeutung der Neumann-Morgensternschen Nutzentheorie}

Was ist damit gezeigt? Wir haben gezeigt, dass sich das Erwartungsnutzenprinzip
(Seite \pageref{Erwartungsnutzen}) und die entsprechende Entscheidungsregel für
Entscheidungen unter Risiko (siehe Seite \pageref{Erwartungsnutzenregel}) aus
plausiblen Voraussetzungen von der Sorte "`Bevorzuge eine Lotterie mit höheren
Gewinnchancen gegenüber einer mit geringeren Gewinnchancen"' logisch ableiten
lässt. Oft werden diese Voraussetzungen als selbstevident angesehen, so dass eine
Person, die Entscheidungen rational trifft, immer von dem
Erwartungsnutzen ausgehen müsste. Ein anderes Entscheidungsverhalten müsste
dementsprechend als irrational eingestuft werden.

Interessanterweise verhalten sich die meisten Menschen in diesem Sinne aber
\label{RisikoaversionGrenznutzen}
irrational, indem sie je nach Situation, ihren Nutzen bei unsicheren Ereignissen
entweder oberhalb des rechnerischen Erwartungsnutzens ansetzen ("`Riskofreude"')
oder unterhalb ("`Risikoscheu"' bzw. "`Risikoaversion"'). Dieser Punkt kann
leicht missverstanden werden, da in der ökonomischen Literatur oft behauptet
wird, dass risikoscheues oder -freudiges Verhalten sehr wohl mit dem
Erwartungsnutzenprinzip vereinbar ist \cite[S. 103]{osborne:2004}, indem es sich
darin niederschlägt, dass riskante Ereignisse einfach entsprechend höhere oder
niedrigere Nutzenwerte zugewiesen bekommen. So würde eine risikoaverse Person den
Nutzen von 1000 Euro gleich hoch veranschlagen wie den Nutzen von einer 50\%
Chance auf 3000 Euro. Und umgekehrt würde eine risikofreudige Person vielleicht
den Nutzen von 1000 Euro so hoch veranschlagen wie den von einer 25\% Chance auf
3000 Euro.\footnote{Das Beispiel stammt von Matthias Brinkmann. Auf das Problem
hat mich außer Matthias Brinkmann auch Johannes Hemker hingewiesen
(Dankeschön!).} (Rechnerisch ist das weniger Geld, aber sie liebt das Risiko, so
dass der Nutzen derselbe bleibt. Und es wäre ja auch eine fragwürdige Theorie,
die vorschreiben wollte, welche Präferenzen jemand bezüglich eines Risikos haben
darf.) Diese Art der Risikobewertung ist jedoch nur dann mit dem
Erwartungsnutzenprinzip vereinbar, wenn für die risikoaverse Person Geldmengen
einen ihrer Risikoscheu entsprechenden abnehmenden Grenznutzen haben (konkave
Nutzenfunktion), und für die risikofreudige einen ensprechenden zunehmenden
Grenznutzen (konvexe Nutzenfunktion). Das bedeutet, wenn die risikoaverse Person
den Nutzen von 1000 Euro mit zwei Nutzeneinheiten bewertet und den Nutzen von
einer Lotterie, bei der sie mit einer 50\% Chance 3000 Euro gewinnen kann,
ebenfalls mit zwei Nutzeneinheiten, dann ist das nur dann mit dem
Erwartungsnutzenprinzip vereinbar, wenn sie 3000 Euro auch ohne Lotterie bloß
mit vier Nutzeneinheiten bewertet.

\marginline{Unterschied zwischen Risikoaversion und abnehmendem Grenznutzen}
Nun sind aber die Präferenzen hinsichtlich eines Risikos (Risikoaversion oder
Risikofreude oder Risikoneutralität) und die Präferenzen hinsichtlich einer mehr
oder weniger großen Menge von irgendetwas (abnehmender, zunehmender oder
gleichbleibender Grenznutzen) empirisch betrachtet zunächst einmal
unterschiedliche Dinge, und es wäre sehr riskant von vornherein eine Harmonie
zwischen beiden anzunehmen.\footnote{Beiläufig bemerkt führt dies eins der
Risiken abstrakter mathematischer Theoriebildung vor Augen, die oft mit einem
Verlust an empirischer Information einhergeht, denn mathematisch stellt sich die
Risikoaversion genauso dar wie der abnehmende Grenznutzen, nämlich durch eine
konkave Nutzenkurve.} Das einzige, was man sagen kann, ist dass risikofreudiges
oder risikoaverses Verhalten bezüglich irgendwelcher Güter oder Geldwerte noch
nicht zwangsläufig Ausdruck von Irrationalität (im Sinne einer Verletzung des
Erwartungsnutzenprinzips) sein muss. Es ist aber stets mit der Möglichkeit zu
rechnen, dass es das ist. Bezüglich von Nutzenwerten (im Unterschied zu Gütern
oder Geldwerten, die erst auf Nutzenwerte abgebildet werden müssen) ist eine
Verletzung des Erwartungsnutzenprinzip aber immer irrational.


Es ist daher Vorsicht geboten, wenn man die
Theorie rationaler Entscheidungen zur Erklärung von empirisch beobachtbarem
Entscheidungsverhalten heranziehen will. Das allein widerspäche aber noch nicht
ihrer normativen Anwendung z.B., wenn es darum geht, betriebswirtschaftliche
Entscheidungen an ihr zu orientieren. Doch auch in dieser Hinsicht gibt es eine
Reihe von Einwänden, die gegen die Theorie erhoben worden sind. Oft werden diese
Einwände in die Form (vermeintlicher) Paradoxien gekleidet, die sich aus der
Neumann-Morgensternschen Nutzentheorie ableiten lassen. Mit diesen Einwänden
werden wir uns im folgenden Kapitel beschäftigen.



