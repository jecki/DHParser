\subsection{Aufgaben}

\begin{enumerate}
  \item {\em Stelle folgendes Ent\-scheidungs\-problem als
  Ent\-scheidungs\-baum dar}: Paula und Fritz stoßen an einer vielbefahrenen
  Kreuzung mit ihren Autos zusammen. Vieles spricht dafür, dass Fritz schuld ist. Deshalb bietet Fritz'
  Versicherung Paula € 5.000 als Schadensersatz und Schmerzensgeld an.
  Paula glaubt jedoch, dass ihr mehr zusteht und überlegt vor Gericht
  zu ziehen. Wenn Sie klagt, dann könnte es sein, dass Fritz' Versicherung ihr
  Angebot im Falle einer außergerichtlichen Einigung auf € 10.000 erhöht. Es ist
  aber auch möglich, dass die Versicherung ihr ursprüngliches Angebot
  beibehält. Gewinnt Paula den Prozess, dann erhält sie € 20.000. Verliert Sie
  den Prozess, dann bekommt sie gar nichts. (Gerichts- und Anwaltskosten können
  zunächst vernachlässigt werden.)

  \item {\em Aufgabe}: Angenommen, bei einer Klage, die später zurückgezogen
  wird, fallen für die Klägerin immer noch Anwalts- und Gerichtskosten von €
 500 an. Angenommen weiterhin, der Prozess kostet den Verlierer oder die
 Verliererin € 2.500. {\em Wie sieht nun der Entscheidungsbaum aus?}

 \item Stelle das Entscheidungsproblem aus der vorhergehenden Aufgabe als
 Tabelle dar.

 \item {\em Aufgabe}: Angenommen, Paula würde eine
  Klage gar nicht erst in Erwägung ziehen, wenn die Versicherung von Fritz ihr
  gleich € 10.000 anbietet und sie würde ihre Klage wieder fallen lassen, wenn
  sich die Versicherung außergerichtlich auf € 15.000 mit ihr einigt. Für die
  Prozesskosten der Versicherung soll dasselbe gelten wie in Aufgabe 2. {\em Wie
  sieht der Entscheidungsbaum aus Sicht der Versicherung aus?}

 \item \label{ZeilenSpaltenPermutation} {\em Erkläre:} Wenn man bei einer
 Enscheidungstabelle beliebig oft ganz Spalten oder ganze Zeilen vertauscht,
 stellt sie immer noch ein- und dasselbe Entscheidungsproblem dar. Warum?
% \item  {\em Aufgabe zur Wahrscheinlichkeitsrechnung}: Auf einer
% Geburtstagsfeier treffen 40 Gäste ein. a) Wie groß ist die Wahrscheinlichkeit,
% dass sich unter den Gästen mindestens eine Person befindet, die am selben Tag
% Geburtstag hat, wie das "`Geburtstagskind"'? b) Ab welcher Anzahl von geladenen
% Gästen würde die Wahrscheinlichkeit mehr als 50\% betragen?


  ~\\ {\bf Schwerere Aufgaben:}\\
 
 \item \label{Algorithmusaufgabe} Formuliere den Algorithmus zur Umwandlung von
 Entscheidungsbäumen in Entscheidungstabellen (siehe Abschnitt \ref{BaumTabelle}) so um, dass er auch
 für nicht binäre Entscheidungsbäume geeignet ist.
 
 \item Beweise (bzw. Erläutere), dass die Kombinationen von Zufallsereignissen
 in den durch den Algorithmus zur Umwandlung von Entscheidungsbäumen 
(siehe Abschnitt \ref{BaumTabelle}) generierten Tabellen immer noch
wechselseitig ausschließend und zugleich erschöpfend (d.h. eins der Ereignisse tritt auf jeden Fall ein) sind.
 
%  \item {\em Aufgabe zur Mengenlehre}: Unter einer {\em abzählbar unendlich}
%  großen Menge versteht man eine Menge, die sich bijektiv ("`ein-eindeutig"')
%  auf die Menge der natürlichen Zahlen abbilden lässt. Das
%  {\em kartesische Produkt} zweier Mengen A und B ist definiert als die Menge
%  aller Tupel $(a, b)$ mit $a \in A$ und $b \in B$. {\em Zeige: Das kartesische
%  Produkt zweier abzählbar unendlicher Mengen ist wieder abzählbar unendlich.} 
\end{enumerate}

% \subsection{Zusatzaufgaben}
% 
% \begin{enumerate}
%  \item Eine Ölfirma erwägt an einer bestimmten Stelle in der Nordsee nach Öl zu
%  bohren. Es ist nicht absolut sicher, ob sich an dem entsprechenden Ort
%  tatsächlich Öl befindet. Um dies mit Sicherheit festzustellen, kann die Firma
%  eine Probebohrung durchführen lassen. Der Bau einer Bohrinsel
%  kostet € 1.000.000. Liefert die Bohrinsel tatsächlich Öl, so gewinnt die
%  Ölfirma abzüglich der Betriebskosten € 10.000.000. Die Durchführung einer
%  Probebohrung kostet € 250.000. {\em Zeichne den Entscheidungsbaum und die
%  Entscheidungstabelle} (Zur Diskussion: Sollte die Firma in jedem Fall eine
%  Probebohrung durchführen?)
% \end{enumerate}

\newpage