\documentclass[10pt, a4paper, german]{article}
\usepackage[utf8x]{inputenc}
\usepackage{ucs} % unicode
%\usepackage[T1]{fontenc}
%\usepackage{t1enc}
%\usepackage{type1cm}
\usepackage[german]{babel}
 
\usepackage{eurosym} 
\usepackage{amsmath, amssymb}
\usepackage{graphicx}
\usepackage{natbib}
\usepackage{rotating}

\numberwithin {equation}{section}

\sloppy

\begin{document}

\begin{center}
{\large Universität Bayreuth: Philosophy \& Economics, SoSe 2009}
\end{center}
\vspace{0.4em}
\begin{center}
{\huge Klausur: Grundlagen des Entscheidens I}
\end{center}
\vspace{0.0em}
\begin{center}
Datum: 28. Juli 2009\\
Dozent: Eckhart Arnold
\end{center}


\section{Aufgabe: Bayes und Entscheidungsbäume}

Eine Ölfirma möchte in der Nordsee eine Bohrinsel (Kosten: 10 Mio Euro)
errichten, um Öl zu fördern. 
Die Wahrscheinlichkeit, innerhalb dieses Gebiets auf Öl zu stoßen, beträgt 60\%.
Im Erfolgsfall rechnet die Firma mit Einnahmen von 50 Mio Euro. 

Ein Geologe bietet der Firma für 1 Mio Euro für die Durchführung einer Expertise
an. Ist tatsächlich Öl vorhanden, so wird dies durch die Expertise mit
95\%-iger Sicherheit festgestellt werden. Ist kein Öl vorhanden, so wird dies
durch die Expertise mit 85\%-iger Sicherheit festgestellt.

\begin{enumerate}
  \item Angenommen, die Expertise wird durchgeführt und fällt positiv aus. Wie
  groß ist die Wahrscheinlichkeit, dass tatsächlich Öl vorhanden ist?
  \item Angenommen, die Expertise fällt negativ aus. Wie groß ist die
  Wahrscheinlichkeit, dass trotzdem Öl vorhanden ist?
%   \item Wie groß ist die Wahrscheinlichkeit, dass die Expertise positiv
%   ausfällt?
  \item Stellen Sie den Entscheidungsbaum auf und beantworten Sie die Frage, ob
  die Firma die Expertise durchführen sollte oder nicht.
%   \item Angesichts der Tatsache, dass die Entscheidung für den Bau einer
%   Ölplattform bei einem positiven Erwartungswert von
%   $0,6 \cdot 40 Mio + 0.4 \cdot -10 Mio = 20 Mio Euro$ auch ohne Expertise
%   kaum zweifelhaft sein dürfte, argumentiert jemand, dass die Durchführung einer Expertise 
%   überflüssig ist. Stimmt das Argument? Wenn ja, warum? Wenn nein, warum nicht?
\end{enumerate}



\section{Aufgabe: Wahlverfahren}

Bei der Abstimmung über drei Kandidaten $A, B, C$ für ein bestimmtes Amt stehen
folgende drei Abstimmungsverfahren zur Auswahl: a) {\em Stimme für Einen}:
Jeder Wähler schreibt seinen bevorzugten Kandidaten auf einen
Zettel; der Kandidat mit den meisten Stimmen gewinnt. b) {\em Stimme für Zwei}:
Jeder Wähler schreibt seine zwei bevorzugten Kandidaten auf einen Zettel; der
mit den meisten Stimmen bzw. Nennungen gewinnt. c) {\em Borda-Zählung}: Jeder
vergibt Punkte für die Kandidaten, und zwar 2 Punkte für den am meisten
bevorzugten Kandidaten, 1 Punkt für den mittleren und 0 Punkte für den am
wenigsten geschätzten Kandidaten; der Kandidat mit den meisten Punkten gewinnt.

\begin{enumerate}
  \item Finden Sie ein Präferenzprofil, bei dem nach dem Abstimmungsverfahren
  a) {\em Stimme für Einen} ein anderer Kandidat gewinnt als nach dem
  Abstimmungsverfahren b) {\em Stimme für Zwei}.
  \item Finden Sie ein Präferenzprofil, bei dem nach jedem der drei
  Abstimmungsverfahren ein anderer Kandidat gewinnt.
\end{enumerate}

Es sollen immer genau drei Kandidaten vorkommen. Die Anzahl der Wähler kann
frei gewählt werden.


\section{Aufgabe Lotterien}

Sei ${\cal G}$ eine Menge von Grundgütern auf der eine Menge ${\cal L}$ von
Lotterien der Form $L(a, x, y)$ definiert ist, wobei $0 \leq a \leq 1$ und $x$
und $y$ jeweils entweder Gründgüter oder wiederum Lotterien sind. Sei weiterhin
$B \in {\cal G}$ ein bestes Gut aus der Menge der Grundgüter, d.h. es gelte
für jedes Grundgut $x \in {\cal G}$, dass $B \succeq x$. Gezeigt werden soll
nun, dass auch für jede Lotterie $L \in {\cal L}$ gilt, dass $B \succeq L$.

Es darf nicht das Erwartungsnutzentheorem vorausgesetzt werden.
sondern nur die grundlegenden Bedingungen und Korrolarian für
Lotterien (Ordnung und Kontinuität von Lotterien, Bedingung der
höheren Gewinne und besseren Chancen, Reduzierbarkeit und
Subsitutionsgesetz).

Zur Erinnerung: Die {\em Bedingung der höheren Gewinne} besagt: Für
beliebige Lotterien oder Grundgüter $x,y,z$ und beliebige
Wahrscheinlichkeiten $a$ gilt sowohl:
$x \succ y$ {\em genau} dann wenn $L(a,x,z) \succ L(a,y,z)$,
als auch:
$x \succ y$ {\em genau} dann wenn $L(a,z,x) \succ L(a,z,y)$


\begin{enumerate}
  \item Zeige: Es existiert kein Grundgut $x$, für das $L(a,x,B) \succ L(a,B,B)$
  oder $L(a,B,x) \succ L(a,B,B)$ gilt.
  \item Zeige: Es existieren keine zwei Grundgüter $x,y$, für die gilt $L(a,x,y)
  \succ L(a,B,B)$ \\

  Im Folgenden sei der {\em Grad einer Lotterie} so definiert: Alle
  Grundgüter haben den Grad 0. Der Grad der Lotterie $L(a, L_1, L_2)$
  ist $n+1$, wenn $n$ der Grad derjenigen Lotterie ist, die von $L_1$
  und $L_2$ den höheren Grad hat. Umgangssprachlich gibt der Grad also
  die Verschachtelungstiefe einer Lotterie an. (Es gibt keine unendlich
  tief verschachtelten Lotterien.)
  
  \item Zeige: Angenommen, es gäbe keine Lotterie $L^*$ mit einem Grad kleiner
  $n$, die gegenüber $B$ vorgezogen wird, dann existiert auch keine Lotterie
  $L^*$ mit einem Grad kleiner $n$, für die $L(a,L^*,B) \succ L(a,B,B)$ oder 
  $L(a,B,L^*) \succ L(a,B,B)$ gilt.

  \item Zeige: Unter derselben Annahme existieren auch keine zwei Lotterien
  $L^*_1, L^*_2$ mit Graden kleiner $n$, für die gilt $L(a, L^*_1, L^*_2) \succ
  L(a,B,B)$

\item Zeige, dass es -- {\em unabhängig} vom Grad der Lotterie --
  keine Lotterie gibt, die gegenüber $B$ vorgezogen wird.
  
%  \item Erkläre: Der 3. und 4. Schritt allein hätten nicht ausgereicht, dies zu
%  zeigen.
  
\end{enumerate}


\vspace{1cm}
Viel Glück!


% \section{Aufgabe zu Bayes}
% 
% Polizeipsychologen haben einen neuartigen Test entwickelt, mit dem sie mit
% Hilfe eines Fragebogens und eines Lügendetektors einen Terroristen 90\%-iger
% Sicherheit als solchen erkennen können. Leider liefert der Test aber auch bei
% 2\% aller Nicht-Terrroristen das (falsche) Ergebnis, sie wären Terroristen.
% Aufgrund der im ``Krieg gegen den Terror'' gewonnen Erfahrungen weiß man
% weiterhin, dass sich unter 200.000 Bürgern 10 Terroristen befinden.
% 
% Bei einer Routinekontrolle unterzieht die Polizei zwei
% Personen, Bürger A und Bürger B, die in keiner Beziehung zu einander stehen,
% dem Terrorismustest. Bei beiden Bürgern liefert der Test ein positives Ergebnis.
% 
% \begin{enumerate}
%   \item Wie groß ist die Wahrscheinlichkeit, dass Bürger A tatsächlich ein
%   Terrorist ist?
%   \item Wie groß ist die Wahrscheinlichkeit, dass {\em wenigstens einer} der
%   beiden Bürger ein Terrorist ist?
% \end{enumerate}



\end{document}