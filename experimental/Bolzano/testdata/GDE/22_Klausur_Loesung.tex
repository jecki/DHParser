\subsection{Die Lösung}

\subsubsection*{Aufgabe: Entscheidungen unter Unwissenheit}

Bedauernstabelle:

\begin{center}
\begin{tabular}{c|c|c|c|c|}
\multicolumn{1}{c}{} & \multicolumn{1}{c}{$S_1$}
& \multicolumn{1}{c}{$S_2$} & \multicolumn{1}{c}{$S_3$}
& \multicolumn{1}{c}{$S_4$}
\\ \cline{2-5}
$A_1$ & {\bf {\em 197}} &   93 &          0 & 46 \\ \cline{2-5} 
$A_2$ &   0             &    0 &  {\em 497} &  0 \\ \cline{2-5}
$A_3$ &  50             &   40 &  {\em 498} & 25 \\ \cline{2-5}
\end{tabular}
\end{center}

Lösung: {\bf $A_1$} sollte gewählt werden, da bei $A_1$ der maximale
Gewinn, der entegehen könnte, mit 197 kleiner ist als bei $A_2$ mit
497 und $A_3$ mit 498.


\subsubsection*{Entscheidungsbäume}

\begin{enumerate}
\item Für den Erwartungswert der "`Handlung B"' gilt: $EW = 0.5 \cdot
  200 + 0.5 \cdot 800 = 500$ €.  Da die "`Handlung A"' nur 400 €
  liefert, würde eine rational handelnde Person die "`Handlung B"'
  wählen.

\item Es kann davon ausgegangen werden, dass die Personen von den
  beiden Handlungen des rechten Entscheidungsknotens die bessere
  wählt. Damit hat "`Ereignis 1"', wenn es eintritt, einen Wert von
  500 € (siehe die erste Aufgabe). Der Erwartungswert der
  "`Alternative 1"' des linken Entscheidungsknotens berechnet sich
  dann wie gehabt: $EW = 0.2 \cdot 1000 + 0.8 \cdot 500 = 600$ €

\item Um diese Frage zu beantworten, muss nur noch der Erwartungswert
  von "`Alternative 2"' berechnet werden: $EW = 0.3 \cdot 600 + 0.5
  \cdot 400 + 0.2 \cdot 1000 = 580$ €. Da die "`Alternative 1"' einen
  höheren Erwartungswert hat, sollte "`Alternative 1"' gewählt werden.
\end{enumerate}


\subsubsection*{Nash-Gleichgewichte}

\begin{enumerate}
\item Die beiden reinen Nash-Gleichgewichte sind ($Z_1,S_1$) und
  ($Z_2,S_2$). Weder der Zeilen- noch der Spaltenspieler kann sich im
  Gleichgewicht durch einen Wechsel seiner Strategie noch verbessern,
  wenn der andere Spieler seine Strategie beibehält.

\item Ansatz: Ein gemischtes Gleichgewicht kann nur dann vorkommen,
  wenn der jeweils andere Spieler bezüglich der gemischten
  Gleichgewichtsstrategie seines Gegenüber indifferent zwischen seinen
  reinen Strategien ist. Sei $a$ die Wahrscheinlichkeit, mit der der
  Zeilenspieler die erste seiner beiden Strategien $Z_1$ spielt. Dann
  errechnet sich die Auszahlung, die der Spaltenspieler erhält, wenn
  er die Strategie $S_1$ spielt nach:
\[ V(S_1) = a \cdot 1 + (1-a)\cdot 2 \]
  Und die Auszahlung, die er erhält, wenn er $S_2$ spielt, ist:
\[ V(S_2) = a \cdot 0 + (1-a)\cdot 4 \]
  Durch Gleichsetzen erhält man:
\begin{eqnarray}
a \cdot 1 + (1-a)\cdot 2 & = & (1-a)\cdot 4 \\
                  -a + 2 & = & 4 - 4a \\
                      3a & = & 2 \\
                       a & = & \frac{2}{3}
\end{eqnarray}
Im gemischten Gleichgewicht wird der Zeilenspieler also mit 2/3
Wahrscheinlichkeit $Z_1$ spielen und mit 1/3 Wahrscheinlichkeit
$Z_2$. Wegen der Symmetrie des Spiels spielt der Spaltenspieler mit
genau denselben Wahrscheinlichkeiten, nämlich mit 2/3
Wahrscheinlichkeit $S_1$ und mit 1/3 Wahrscheinlichkeit $S_2$.

\end{enumerate}

\subsubsection*{Aufgabe: Bayes'scher Lehrsatz}

Sei $p$ das Ereignis, dass die Probegrabung erfolgreich ausfällt und
$g$ das Ereignis, dass Gold vorhanden ist. Berechnet werden soll die
Wahrscheinlichkeit, dass Gold vorhanden ist, wenn die Probegrabung
negativ ausfällt, d.h. $P(g|\neg p)$. Nach dem Bayes'schen Lehrsatz
gilt:
\[ P(g|\neg p) = \frac{P(\neg p|g)P(g)}{P(\neg p|g)P(g) + P(\neg
  p|\neg g)P(\neg g)} \] 
Aus der Aufgabenstellung geht unmittelbar nur
hervor, dass $P(g)=0.3$, $P(p|g)=0.95$ und $P(p|\neg g)=0.1$. Alle
anderen benötigten Werte muss man aus diesen gegebenen Werten
berechnen, also:
\[ P(\neg g) = 1 - P(g) = 1 - 0,3 = 0,7 \]
\[ P(\neg p|g) = 1 - P (p|g) = 1- 0,95 = 0,05 \]
\[ P(\neg p|\neg g) = 1 - P(p|\neg g) = 1 - 0,1 = 0,9 \]
Durch Einsetzen erhalten wir:
\[ P(g|\neg p) = \frac{0,05 \cdot 0,3}{0,05 \cdot 0,3 + 0,9 \cdot 0,7}
= 0,023256 \] 
Die Lösung lautet also, dass nur noch mit ca. 2,3\%
Wahrscheinlichkeit davon ausgegangen werden kann, dass Gold vorhanden
ist, wenn die Probegrabung negativ ausfällt.

\subsubsection*{Aufgabe: Beweise}

\begin{enumerate}

\item $L(a,x,y) \equiv L(b,y,x)$ wenn $b = 1-a$. Begründung: Wenn
  $b=1-a$, dann kann man in beiden Lotterien mit genau denselben
  Gewinnchancen dieselben Gewinne bekommen. Damit sind die Lotterien
  aber identisch. 

  \begin{small}
  {\em Anmerkung}: Man kann in diesem Fall schon deshalb nicht mit dem
  Erwartungsnutzen argumentieren, weil damit höchstens die Indifferenz 
  zwischen beiden Lotterien gezeigt werden kann, aber noch nicht ihre Identität.
  (Wenn der Erwartungsnutzen von einem Apfel für eine bestimmte Person
  derselbe ist wie der von einer Birne, dann ist die Person zwischen Apfel und
  Birne indifferent, aber deshalb ist ein Apfel noch lange keine Birne!)
  \end{small}

\item Nach dem ersten Teil der Aufgabe ist die Lotterie $L(a, z, x)$ identisch
mit der Lotterie $L(1-a, x, z)$ und die Lottiere $L(a, z, y)$ identisch mit der
      Lotterie $L(1-a, y, z)$.

      Nun gilt aber: Für jedes $a$ mit $0 \leq a \leq 1$ liegt der Wert
      $1-a$ wieder in dem Intervall von 0 bis 1. Dann gilt aber nach der Bedingung der
      höheren Gewinne auf der ersten Stelle (Voraussetzung):
      \[x \succ y \Leftrightarrow L(1-a, x, z) \succ L(1-a, y, z)\]
      Aufgrund der oben festgestellten Identität gilt aber ebenfalls:
      \[L(1-a, x, z) \succ L(1-a, y, z) \Leftrightarrow
        L(a, z, x) \succ L(a, z, y)\]
      Damit gilt insgesamt:
      \[x \succ y \Leftrightarrow L(a, z, x) \succ L(a, z, y) \]      
      q.e.d.
      
      \begin{small}
      {\em Anmerkung}: Wichtig ist, dass der Beweis so geführt wird, dass klar
      ist, dass die Formel am Ende auch tatsächlich für alle $a$ gilt!
      \end{small}
\end{enumerate}
