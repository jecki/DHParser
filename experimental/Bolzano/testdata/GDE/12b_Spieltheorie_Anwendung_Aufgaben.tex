\subsection{Aufgaben}

\begin{enumerate}
  \item Stellen Sie das Vertrauensspiel als Tabelle dar.
  
  \item Zeige: Im 2-Personen Hirschjagdspiel gibt es kein gemischtes
  Gleichgewicht:

\begin{center}
\begin{tabular}{c|c|c|}
\multicolumn{1}{c}{} & \multicolumn{1}{c}{Hirsch} & \multicolumn{1}{c}{Hase} 
\\ \cline{2-3} 
Hirsch               & 5, 5                    & 0,2  \\ \cline{2-3}
Hase                 & 2,0                     & 2,2 \\ \cline{2-3}
\end{tabular}
\end{center}
  
  \item Zeige, dass im 2-Personen Spiel mit zwei Handlungsoptionen gilt:
  \label{gemischteStrategienAufgabe}
  \begin{enumerate} 
    \item Die beste
    Antwort auf eine reine Strategie ist immer eine reine Strategie, sofern
    zwischen den möglichen Antworten in reinen Strategien nicht Indifferenz herrscht.
    \item Sei $(Z; S)$ ein Gleichgewicht der Strategien $Z$ und $S$, und sei
    $Z$ eine gemischte Strategie, dann muss auch $S$ eine gemischte Strategie sein, es sei denn
    Spieler 1 () ist indifferent zwischen seinen möglichen reinen
    Antwortstrategien.
    \item Sei $(Z; S)$ ein Gleichgewicht und $Z$ eine gemischte Strategie, aber
     $S$ eine reine Strategie, dann ist auch $(x; S)$ ein Gleichgewicht für
     jede beliebige reine oder gemischte Strategie $x$.
    \item Gib ein Beispiel in Form einer Spielmatrix für den vorhergehenden
    Fall an.
  \end{enumerate}

  \item \label{chickenGameAufgabe} Berechne das gemischte Gleichgewicht im
  Angsthasenspiel:
  
  \begin{center}
\begin{tabular}{c|c|c|}
\multicolumn{1}{c}{} & \multicolumn{1}{c}{Ausweichen} &
                               \multicolumn{1}{c}{Gas geben } \\ \cline{2-3} 
Ausweichen               & 0, 0           & -5,5      
\\ \cline{2-3} 
Gas geben                & 5,-5           & -100,-100
\\ \cline{2-3}
\end{tabular}
\end{center}  

  Zusatzfrage: Wie wirkt es sich auf die Gleichgewichte aus, wenn man das
  Angsthasenspiel folgendermaßen abändert?
  
  \begin{center}
\begin{tabular}{c|c|c|}
\multicolumn{1}{c}{} & \multicolumn{1}{c}{Ausweichen} &
                               \multicolumn{1}{c}{Gas geben } \\ \cline{2-3} 
Ausweichen               & 0, 0           & -5,5      
\\ \cline{2-3} 
Gas geben                & 5,-5           & $-\infty$,$-\infty$
\\ \cline{2-3}
\end{tabular}
\end{center}   

\item Zeige: Im wiederholten Gefangenendilemma mit den Parametern T,R,P,S =
5,3,1,0 beträgt die zu erwartende Auszahlung von {\em TitForTat} gegen die
Strategie {\em Random} 2.25.

\item Welche Strategie ist im wiederholten Gefangenendilemma die beste Antwort
auf {\em Random}?

\item Gib zwei Strategien $A$ und $B$ an, für die gilt:
  \begin{enumerate}
     \item Die direkte Begegnung von $A$ und $B$ geht immer zugunsten von $B$
     aus, d.h. $V(B/A) > V(A/B)$
     \item $B$ kann trotzdem nicht in eine Population von $A$ eindringen.
  \end{enumerate}

\item Zeige: Die Strategie {\em Tit For Two Tats} (Bestrafe erst bei zwei
Defektionen) ist nicht kollektiv stabil. Es genügt dafür eine Strategie
anzugeben, die in eine Population von {\em Tit For Two Tat}-Spielern eindringen
kann.

\item Zeige: Die Strategie {\em Grim} (siehe Seite \pageref{Strategien}) ist
kollektiv stabil aber nicht evolutionär stabil.

\item In welchem Verhältnis stehen die Begriffe der {\em kollektiven
Stabilität} und der {\em evolutionären Stabilität} zu dem des
Nash-Gleichgewichts?

\item Mit welcher Wahrscheinlichkeit muss der Verkäufer mindestens ehrlich
sein, damit sich das Geschäft für den Käufer in dem folgenden Vertrauensspiel
lohnt?

\setlength{\unitlength}{1cm}
\begin{picture}(10,8)(-1,0)
\put(2,7){\makebox(6,1){Käufer}}
\put(5,7){\line(-1,-1){5}}
\put(5,7){\line(1,-1){2}}
\put(4,4){\makebox(6,1){Verkäufer}}
\put(7,4){\line(-1,-1){2}}
\put(7,4){\line(1,-1){2}}

\put(2,5.5){\makebox(1,1){{\small kaufe nicht}}}
\put(6.5,5.5){\makebox(1,1){{\small kaufe}}}

\put(4.5,2.5){\makebox(1,1){{\small verschicke}}}
\put(9.0,2.5){\makebox(1,1){{\small schicke nicht}}}

\put(-0.5,1){\makebox(1,1){35, 35}}
\put(4.5,1){\makebox(1,1){50, 50}}
\put(8.5,1){\makebox(1,1){0, 70}}
\end{picture}
\begin{center} {\small Quelle: Bolton, Katok, Ockenfels
\cite[]{bolton-katok-ockenfels:2004}} \end{center}

\end{enumerate}