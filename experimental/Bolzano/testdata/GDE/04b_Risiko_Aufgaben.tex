\subsection{Aufgaben}
\begin{enumerate}
  
  \item Zeige durch ein Beispiel, dass der berechnete Erwartungsnutzen
  sich bei gleichbleibenden Präferenzen ändern kann, wenn man bloß von einem
  {\em ordinalen Nutzen} ausgeht. M.a.W.: Um den Erwartungsnutzen sinnvoll
  einsetzen zu können, müssen wir immer das vergleichsweise stärkere aber
  empirisch schwerer zu rechtfertigende Konzept des {\em kardinalen Nutzens}
  voraussetzen.
  
  \item Stelle das folgende Entscheidungsproblem aus der Vorlesung (Seite
  \pageref{RisikoBeispiel1}) als Entscheidungsbaum dar und löse den
  Entscheidungsbaum schrittweise auf.

\begin{center}
\begin{tabular}{c|r|r|r|r}
\multicolumn{1}{c}{}  & \multicolumn{1}{c}{$S_1$ ($p=0.3$)}  &
\multicolumn{1}{c}{$S_2$ ($p=0.2$)} & \multicolumn{1}{c}{$S_3$ ($p=0.5$)} \\
\cline{2-4} $A_1$ & -100.000 € & -50.000 & € 60.000 & $EU = -10.000$ €\\ 
\cline{2-4} $A_2$ & 0 €        & -80.000 & € 0      & $EU = -16.000$ €\\ 
\cline{2-4}
\end{tabular}
\begin{small}
\begin{tabular}{llp{10cm}}
& & \\
$A_1$ & & Investiere in die rasche Entwicklung eines Kleinstlaptops.\\
$A_2$ & & Investiere nicht in die Entwicklung eines Kleinstlaptops.\\
& &\\
$S_1$ & & Kleinstlaptops bleiben auf dem Markt erfolglos.\\
$S_2$ & & Kleinstlaptops sind erfolgreich, aber die Konkurrenz
                   ist ebenfalls frühzeitig auf dem Markt präsent.\\
$S_3$ & & Kleinstlaptops sind erfolgreich, aber die Entwicklung 
                   der Konkurrenz verzögert sich. \\
\end{tabular}
\end{small}
\end{center}
  
 \item {\em Stelle das folgende Entscheidungsproblem als Entscheidungsbaum} dar:
  Eine Ärztin steht vor der Frage, ob sie die Infektion eines Patienten mit einem
Desinfektionsmittel oder mit einem Antibiotikum behandeln soll. Das Antibiotikum
schlägt bei 80\% der Patienten gut an, in welchem Fall die Heilungschance bei
70\% liegt. Bei den restlichen Patienten liegt die Heilungschance mit demselben
mittel jedoch nur bei 40\%. Das Desinfektionsmittel hat dagegen bei allen
Patienten eine Heilungschance von 50\% Da die Mittel miteinander unverträglich
sind, besteht nicht die Möglichkeit beide Mittel zu verabreichen.

\begin{center}
\begin{scriptsize}
\begin{tabular}{c|c|c|c|c|}
\multicolumn{1}{c}{} & \multicolumn{2}{c}{A. schlägt an (80\%)} 
                     & \multicolumn{2}{c}{schlägt nicht an (20\%)} \\
\multicolumn{1}{c}{} & \multicolumn{1}{c}{Heilung (70\%)}
                     & \multicolumn{1}{c}{$\neg$Heilung (30\%)}
                     & \multicolumn{1}{c}{Heilung (40\%)}
                     & \multicolumn{1}{c}{$\neg$Heilung (60\%)}
                     \\ \cline{2-5}                       
Antibiotikum  & gesund (56\%) & krank (24\%) & gesund (8\%) & krank (12\%) 
\\ \cline{2-5} 
Des.-Mittel   & gesund (40\%) & krank (40\%) & gesund (10\%) & krank (10\%) 
\\ \cline{2-5}
\end{tabular}
\end{scriptsize}
\end{center}
  
\item Wie kann man die Wahl eines Gesellschaftsmodells hinter einem Rawlsschen
{\em Schleier des Nichtwissens} als Entscheidungstablle darstellen? Und als
Entscheidungsbaum?  

\end{enumerate}

