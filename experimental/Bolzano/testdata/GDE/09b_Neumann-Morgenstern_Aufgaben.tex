\subsection{Aufgaben}
\begin{enumerate}


\item Eine Ölfirma erwägt, an einer
bestimmten Stelle in der Nordsee nach Öl zu bohren. Leider ist es keineswegs
sicher, ob an der entsprechenden Stelle tatsächlich Ölvorkommen vorhanden sind.
Das ist um so bedauerlicher als der Bau einer Ölplattform € 1.500.000 kostet,
eine Investition, die verloren wäre, sollte dort tatsächlich kein Öl zu finden
sein. Andererseits würde die Ölplattform € 30.000.000 
einbringen, wenn Öl vorhanden ist. Anhand
der geologischen Daten können die Fachleute der Ölfirma immerhin abschätzen, 
dass sich in dem fraglichen Gebiet mit 45\%-iger Wahrscheinlichkeit Ölvorkommen
befinden.

Um eine genauere Abschätzung zu erhalten, könnte die Firma ein 
Expertenteam damit beauftragen, eine Probebohrung durchzuführen. Eine
Probebohrung schlägt noch einmal mit € 400.000 zu Buche.
Leider bieten auch derartige Expertisen keine absolute Sicherheit. Es ist
bekannt, dass in 88\% der Fälle vorhande Ölvorkommen durch die Expertise
erkannt werden. Aber auch wenn kein Öl
vorhanden ist, liefert eine Expertise in 3\% der Fälle das falsche
Ergebnis, es wäre Öl zu finden.

{\em Aufgabe}:
\begin{enumerate}
\item Bestimme (mit Hilfe der Bayes'schen Formel) die bedingten
Wahrscheinlichkeiten, mit denen Öl vorhanden ist bzw. nicht vorhanden ist, 
wenn die Expertise positiv bzw. negativ ausfällt.
\item Stelle den Entscheidungsbaum für das beschriebene
Entscheidungsproblem auf. (Beachten Sie dabei an welcher Stelle welche bedingte
Wahrscheinlichkeiten eingetragen werden müssen.)
\item Löse den Entscheidungsbaum soweit auf, dass man eine Emp\-fehl\-ung geben
kann, ob es sich für die Firma lohnt, eine Expertise in Auftrag zu geben.
\end{enumerate}

\item $L'([0.5, 0.25, 0.25], A, B, C)$ sei eine Lotterie mit {\em drei} Preisen
A, B, C, die jeweils mit den Wahrscheinlichkeiten $0.5, 0.25, 0.25$ gezogen
werden. Zeige, dass man diese Lotterie aus Lotterien mit ausschließlich {\em
zwei} Preisen zusammensetzen kann \cite[S. 91]{resnik:1987}. 

\item Erkläre: Wenn eine Nutzenfunktion die Erwartungsnutzeneigenschaft 
$u(L(a,x,y)) = au(x) + (1-a)u(y)$ hat, dann bedeutet dies, dass sie dem
Nutzen einer Menge von unsicheren Ereignissen $E_1,\ldots , E_n$ mit
Wahrscheinlichkeiten $p_1,\ldots, p_n$ den Erwartungsnutzen
\[ EU = p_1u(E_1) + \ldots + p_nu(E_n) \]
zuordnet. (Sinn der Aufgabe: Damit wird gezeigt, dass sich die für den
Zwei-Güter-Fall definierte Erwartungsnutzeneigenschaft leicht auf den
$n$-Güter-Fall übertragen lässt.)

\item Zeige, dass bei einer zusammengesetzen Lotterie: \[L(a, L(b,x,y),
L(c,x,y))\] die Wahrscheinlichkeit dafür, dass der Gewinn $x$ gezogen wird:
$ab + (1-a)c$ ist, und die Wahrscheinlichkeit, dass $y$ gezogen wird:
$1 - (ab + (1-a)c)$ beträgt. (Damit wird gezeigt, dass die
Reduzierbarkeitsbedingung (Seite \pageref{Reduzierbarkeit}) im Einklang mit der
Wahrscheinlichkeitsrechnung steht.)

% \item Sei $u$ eine beliebige Nutzenfunktion, die einer
% vollständigen Menge von Lotterien, die die
% Neumann-Morgensternschen Bedingungen (auf Seite
% \pageref{LotterienBedingungen}) erfüllt, Nutzenwerte zuweist und für die das
% Erwartungsnutzenprinzip $u(L(a,x,y)) = au(x) + (1-a)u(y)$ gilt. Zeige, dass für
% jede Zahl $k$ auf der $u$-Skala eine Lotterie existiert mit $u(x)=k$. 
% \cite[S. 98]{resnik:1987}

\item Warum ist bei der Bedingung der höheren Gewinne (Seite
\pageref{BedHoehereGewinne}) sowie bei den entsprechenden Corrolarien die
Einschränkung $a > 0$ notwendig? Mit anderen Worten: Wieso wäre die Bedingung
für $a = 0$ unplausibel?\\
{\em Zusatzfrage}: Warum gilt beim Substitutionsgesetz (Seite
\pageref{Substitutionsgesetz}) keine entsprechende Einschränkung mehr?

\item Das St. Petersburg-Spiel wird
folgendermaßen gespielt: Es wird eine Münze geworfen. Zeigt sie Kopf, dann erhält der Spieler 2 € und das Spiel ist beendet.
Andernfalls wird sie ein weiteres Mal geworfen. Zeigt sie diesmal Kopf, so erhält der
Spieler 4 €. Wenn nicht wird die Münze ein weiteres Mal geworfen und bei Kopf 8
€ ausgezahlt usw. 
\begin{enumerate}
  \item Wie groß ist der Erwartungswert des Spiels, wenn das Spiel maximal 2
  Runden gespielt wird?
  \item Wie groß ist der Erwartungswert, wenn das Spiel maximal $n$ Runden
  gespielt wird?
  \item Wie groß ist der Erwartungswert des Spiels, wenn es mit un\-beschränkter
  Rundenzahl gespielt wird?
\end{enumerate}
\cite[S. 88]{resnik:1987}

%\item Zusatzaufgabe: Führe einen ausführlichen Beweis für sämtliche Corrolarien
%auf Seite \pageref{Corrolarien}.

~\\{\bf schwierigere Aufgabe}\\

\item Zeige: Jede Lotterie mit $n$ Preisen ($n > 2$) lässt sich aus Lotterien
mit zwei Preisen zusammensetzen.

\end{enumerate}

