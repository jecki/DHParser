%\clearpage\linenumbers%
% \PdUch{Vorrede}
\def\dieserteilseiten{I}%
\def\dieserteil{I}%
\RWch*{Vorrede.[uns
\RWSeitenwohne{III}\\
{\normalsize \protect\vorredekennzeichnung}}%
\PdUtoc{chapter}{Vorrede. \protect\vorredekennzeichnung}%
\markboth{\kopfzeilenfmt{Religionswissenschaft\enspace$\cdot$\enspace Vorrede \protect\vorredekennzeichnung}}{\kopfzeilenfmt{Religionswissenschaft\enspace$\cdot$\enspace Vorrede \protect\vorredekennzeichnung}}%
\linenumbers\noindent Der Verfasser dieses Lehrbuches, ein katholischer Geistlicher, vor beiläufig fünfzehn Jahren noch als Religionslehrer an der Universität seiner Vaterstadt angestellt, wegen seiner ausgebreiteten und tiefen Kenntniß der Wissenschaft eben so sehr als wegen seines vortrefflichen Charakters und wegen seiner nicht zu ermüdenden Thätigkeit, das Beste der Menschheit durch Aufklärung der Jugend zu fördern, hochgeachtet, ja verehrt von Allen, die ihn kennen, hatte schon während seiner Studienjahre Gründe gefunden, die seiner Ueberzeugung von der Wahrheit und Göttlichkeit unseres heiligen Glaubens eine unumstößliche Festigkeit gaben, und weder in den Schulen vorgetragen wurden, noch in den wissenschaftlichen Lehrbüchern aus einander gesetzt waren. Weil er nun hoffte, daß er im geistlichen Stande mehr als in jedem anderen Aufforderung und Gelegenheit finden würde, diese Gründe vielseitig zu prüfen, sich von ihrer Haltbarkeit zu überzeugen, und sie, wenn er ihre Richtigkeit und Sicherheit würde hinlänglich erprobt haben, zu verbreiten und zur allgemeinen Anerkennung zu bringen: so begann er, nach zurückgelegten philosophischen Lehrcursen, die theologischen Studien, und bewarb sich noch vor Beendigung dieser Studien, als ein junger Mann von \Abweichung{nicht}{nich}{} ganz vier und zwanzig Jahren, um die damals zu errichtende Lehrkanzel der Religionswissenschaft an der Hochschule seiner Vaterstadt, die er auch erhielt. Auf diesem Posten entwickelte er allmählig seine Ansichten über die katholische Religion und insbesondere über die Gründe ihrer Göttlichkeit vor seinen zahlreichen Schülern; und je mehr das wissenschaftliche Gebäude, das er errichtete, der Vollendung näher kam, um so mehr gewann es die Aufmerksamkeit und den Beifall nicht nur seiner Schüler, sondern beinahe aller Mitglieder der gelehrten, ja überhaupt der gebildeten Stände. Die angesehensten und aufgeklärtesten Personen bemühten sich, Abschriften seiner Vorträge zu erhalten, und wer nicht bloß herausgerissene Bruchstücke, sondern das Ganze las und gehörig auffaßte, der fühlte und gestand, daß seine religiöse Ueberzeugung an Klarheit und Festigkeit, daß er Liebe, Achtung und Vertrauen zu seinem Glauben gewonnen habe. Dieß war besonders in den~\RWSeitenw{IV}\ letztern Jahren seines öffentlichen Lebens der Fall; denn der allverehrte Lehrer ließ sich durch den Beifall, den seine Vorträge bereits gefunden \Abweichung{hatten}{hatte}{}, so wenig abhalten, an ihrer Verbesserung und Vervollkommnung zu arbeiten, daß er vielmehr, ohne in dem \RWbet{Wesen} seiner Ansichten eine Aenderung vorzunehmen, beinahe niemals den Lehrstuhl bestieg, ohne zu dem, was er im vorhergegangenen Jahre vorgetragen hatte, eine Zugabe oder Abänderung mitzubringen, die seinem Vortrage mehr Licht und Klarheit, mehr Eindringlichkeit und Ueberzeugungskraft gab, irgend einen möglichen Einwurf oder Zweifel beseitigte, oder die Brauchbarkeit und gehörige Anwendung einer Lehre besser darthat. So kam es, daß seine Vorlesungen immer segensreicher wirkten. Sie haben den Geist seiner Schüler, mehr als es sich sagen läßt, erhellt, geübt, geschärft, das Herz gebildet und veredelt, den Willen gebessert und gestärkt, das Leben und Wirken geordnet und nach jenem Ziele hingerichtet, dem wir mit jedem Tage näher und näher kommen sollen; sie haben dem Glauben, der die Tugend nährt, im Leiden tröstet und zum Heile führt, einen festen Grund gelegt, den Verstand mit jenen scheinbaren Widersprüchen ausgesöhnt, die der Religion so viele Feinde und Gegner erwecken; haben gezeigt, wie man die Lehren der Offenbarung gebrauchen müsse, wenn sie Demjenigen, von dem sie uns gegeben wurde, an Weisheit, Heiligkeit und Seligkeit ähnlicher und immer ähnlicher machen sollen; haben Wahrheiten, die nicht ohne großen Nachtheil für die Menschheit verkannt und verborgen bleiben können, und die gleichwohl nur selten richtig erkannt und angewendet werden, in ein helles Licht gesetzt, und \Abweichung{nicht}{nich}{} bloß im Verstande zur Anerkennung gebracht, sondern auch das Herz und den Willen Vieler so gänzlich für dieselben gewonnen, daß sie nun in den öffentlichen, einflußreichen Aemtern, die sie bekleiden, durch Einsicht, Rechtschaffenheit und Berufstreue als weise Freunde der Menschheit, als eifrige und muthige Verfechter ihres Wohles und ihrer Rechte sich erweisen, und mit Freunde unverholen gestehen, daß das Gute, welches in ihnen sich findet und durch sie in der Welt zu Stande gebracht wird, größtentheils ihrem Lehrer zuzuschreiben sey.\par
Wir -- die das Glück hatten, den Herrn Verfasser dieses Lehrbuches zum Lehrer zu haben -- wir können, was wir von diesem Lehrbuche so eben erzählt und gerühmt haben, mit um so größerer Zuversicht bezeugen, da wir es nicht nur selbst erlebt und an uns erfahren haben, sondern auch gewiß seyn dürfen, daß uns, was wir gesagt, alle diejenigen bestätigen werden, die ein gleiches Glück mit uns genossen haben; und ihre Zahl ist wahrlich nicht gering. Auch sind, seit wir ferne von unserem ehemaligen, verehrten Lehrer leben, mehrere Jahre verflossen, und während der Zeit ist Vieles über die Religion gedacht~\RWSeitenw{V}\ und geschrieben worden; dennoch, wenn wir die Ergebnisse der theologischen Forschungen, selbst der neuesten, beachten, scheinen uns die Begriffe, Ansichten und Grundsätze, die in diesem Lehrbuche vorgetragen werden, noch immer \RWbet{neu} und der Prüfung und öffentlichen Besprechung gar sehr werth zu seyn. Die wichtige Lehre \RWbet{von der Möglichkeit und den Kennzeichen der göttlichen Offenbarung} ist unseres Wissens nirgends so behandelt worden, daß sie geeignet wäre, den Streit zwischen dem Rationalismus und Supernaturalismus, der in Deutschland sich erhoben hat, und noch immer fortgeführet wird, beizulegen; und es scheint uns, daß die Ansichten unseres theuern Lehrers die Vernunft nicht nur von der Möglichkeit göttlicher Belehrungen überzeugen, sondern sie zugleich in den Stand setzen könnten, mit Sicherheit zu entscheiden, welche unter den verschiedenen Religionen, die auf Erden geglaubt und bekannt werden, sich in Wahrheit rühmen könne, göttlichen Ursprunges zu seyn. Eben so wenig ist der Beweis, \RWbet{daß die katholische Religion jene Zeichen aufzuweisen habe, an denen wir mit Sicherheit abnehmen können, daß Gott sie geglaubt und befolgt wissen will}, von irgend einem Lehrer unserer Kirche so geführt worden, daß nicht gar manche Frage unbeantwortet, gar mancher Zweifel übrig geblieben wäre; und es scheint uns, daß unser verehrter Lehrer diesen Beweis so dargestellt habe, daß, wenn man erst einige Mißverständnisse und Bedenklichkeiten, die sich wahrscheinlich erheben werden, wird beseitiget, und seine eigentliche Ansicht recht klar wird aus einander gesetzt haben, nichts mehr zu wünschen übrig bleiben wird. \RWbet{Der Sinn und die Bedeutung von gar mancher hochwichtigen Lehre unserer heiligen Religion} ist noch nirgends so erklärt worden, daß es leicht oder auch nur möglich wäre, zu zeigen, sie stehe mit keiner der Wahrheiten, die völlig ausgemacht und sicher sind, im Widerspruche; und es scheint uns, in diesem Lehrbuche sey dieß theils wirklich schon geschehen, theils sey wenigstens genug gesagt, um solche Erlärungen finden, und sie als übereinstimmend mit den Lehren der Kirche darstellen zu können. Um nicht weitschweifig zu werden, übergehen wir vieles, vieles Andere, das in diesem Lehrbuche geleistet ist, und selbst in den gefeiertesten Werken der theologischen Literatur, die dagegen Anderes vortrefflich behandeln, entweder gänzlich vermißt wird, oder wenigstens nicht so dargestellt ist, daß es geeignet wäre, allgemeine Anerkennung zu finden; wir übergehen dieß, weil schon die erwähnten drei Vorzüge dieses Lehrbuches hinreichen, die Drucklegung desselben zu rechtfertigen.\par
Daß aber wir, bloße \RWbet{Schüler} des Verfassers, uns zur Bekanntmachung seiner Vorlesungshefte entschließen, und sie nicht dem Herrn~\RWSeitenw{VI}\ Verfasser selbst überlassen, bedarf einer Rechtfertigung. Die Wahrheit hat noch jederzeit Gegner, und ihre Vertheidiger haben noch immer um so heftigere Feinde gefunden, je nachdrücklicher und schlagender ihre Vertheidigung war. Auch gegen unseren verehrungswürdigen Lehrer erhoben sich schon in der ersten Zeit nach Antritt seines Lehramtes Feinde, denen es, nach wiederholt zurückgewiesenen Angriffen, endlich gelang, höhern Orts Verdacht gegen ihn zu erwecken und ihn in eine Lage zu versetzen, daß er es jetzt für eine Verletzung seiner Unterthanspflichten halten würde, diese Vorträge durch die Presse bekannt zu machen. Deßhalb haben wir, denen solche Rücksichten die Hände nicht binden, uns entschlossen, in seinem Namen zu thun, woran er gehindert ist. Wir glauben auch, auf seinen Dank dafür rechnen zu können.\par
Daß er die Verbreitung seiner Ansichten wünsche, läßt sich nicht nur daraus schließen, weil er auf ihre Darstellung so viele Zeit, Mühe und Sorgfalt verwendet, und sie durch so viele Jahre öffentlich vorgetragen hat; sondern wir erinnern uns auch recht gut, und viele seiner Schüler werden es eben so wenig vergessen haben, daß er den Wunsch, es möchten seine Grundsätze allgemein bekannt und angenommen werden, zu wiederholten Malen öffentlich aussprach. Wenn wir hiernach seiner Zustimmung zur Drucklegung seiner Vorträge gewiß seyn möchten, so entsteht doch wieder daraus eine nicht zu übersehende Bedenklichkeit, daß unser verehrter Lehrer nicht selten die Absicht geäußert hat, daß er, wenn ihm Gott das Leben so lange erhalten sollte, seine Vorlesungen viel umständlicher und erschöpfender bearbeiten würde, so zwar, daß die hier abgedruckten drei Theile -- dieß sind nun seine eigenen Worte: -- wenigstens in einigen Partieen für nicht viel mehr, als für einen \RWbet{bloßen Entwurf} zu erachten wären. Doch ist, wie wir zuverlässig wissen, mit dieser so umständlichen und erschöpfenden Bearbeitung bis jetzt nicht einmal noch der Anfang gemacht worden; indem sich unser theuerer Lehrer mit anderen Disciplinen, namentlich mit der Logik und Mathematik ausschließlich beschäftigen soll. Auch ist die Gesundheit unseres verehrungswürdigen Lehrers, leider! schon seit seiner Kindheit viel zu schwach, sein kostbares Leben hängt an viel zu zarten Fäden, als daß man mit einiger Sicherheit hoffen dürfte, er werde mit dieser Arbeit fertig werden können. Und wenn wir auch so glücklich seyn sollten, daß er uns durch die mütterlich liebevolle Pflege, der er jetzt, wie man versichert, genießet, lange genug erhalten würde, um sein Vorhaben ausführen zu können; so wird nicht nur das Studium seiner Vorlesungshefte eine nützliche, bei \Abweichung{Manchem}{Manchen}{} sogar nothwendige Vorbereitung seyn zum Verständniß seiner vollständig ausgearbeiteten Religionslehre; sondern die vorläufige Bekanntmachung der Vorlesungshefte wird ihm auch~\RWSeitenw{VII}\ bei jener vollständigen Bearbeitung selbst sehr zu Statten kommen. Es läßt sich nämlich leicht voraussehen, daß seine Ansichten, die dem, was man bisher über diesen Gegenstand denket und meint, so ganz fremd, und dem in der Wissenschaft herrschenden Geist oder Ungeist der Zeit so ganz entgegen sind, von gar mancher Seite werden angegriffen werden. Solche Angriffe aber können ihn auf Manches, das er vielleicht übersehen, mit zu wenigen und gerade für seine Gegner unzureichenden Beweisgründen ausgestattet, nicht deutlich genug erklärt, nicht mit hinlänglicher Umständlichkeit gegen Mißverständnisse sicher gestellt hat, aufmerksam machen; können vielleicht sogar Veranlassung werden, daß er in einem \RWbet{Nachtrage} zum Theil wenigstens jene umständliche Bearbeitung liefere, zu der es ohne das Erscheinen seiner Vorlesungen gar nicht gekommen wäre. Auf jeden Fall sind diese Vorlesungen bereits in so vielen Abschriften, die alljährlich noch vervielfältiget werden, vorhanden, sind bereits so sehr ein Eigenthum des Publicums, daß der Herr Verfasser eine Drucklegung derselben gewiß schon seit Jahren vorhergesehen, vielleicht sogar erwartet und gewünscht hat.\par
Da wir nun bei dieser Herausgabe seiner Vorlesungen durchaus keinen Widerspruch von Seite unseres Lehrers zu fürchten haben: so erübriget nur noch, daß wir ihm und dem Publicum über die Art und Weise Rechenschaft geben, die bei der Sammlung seiner Vorträge beobachtet wurde. Unser verehrter Lehrer hat, wie wir schon vorhin bemerkten, an der Verbesserung und Vervollkommnung seiner Vorträge bis an's Ende seines Wirkens unausgesetzt gearbeitet. Wir durften uns also nicht erlauben, jene Bearbeitung, die wir in unserer Studienzeit uns verschafften, obgleich sie damals die beste war, in die Druckerei zu schicken; sondern wir konnten nur dann glauben, daß wir die Religionswissenschaft unseres Lehrers der Welt in jener vollkommensten Gestalt übergeben, die sie von seiner Hand bisher erhalten hat, wenn wir auch die letzten Verbesserungen, die er machte, in unsere Ausgabe aufnehmen. Das ist denn auch wirklich geschehen. Wir waren so glücklich, eine Abschrift seiner letzten Bearbeitung, und selbst jener Bogen zu erhalten, von denen er nicht eher als an dem nämlichen Tage, da ihm das Absetzungsdecret mitgetheilt wurde, die bessernde Hand zurückgezogen hat; und diese ganz letzte Bearbeitung ist es, die wir hiemit dem Publicum vorlegen. In dieser Bearbeitung aber fehlte Mehreres, das er, wie unsere Explicationshefte uns lehrten, in früheren Jahren vorgetragen hatte; und wir erlaubten uns, Einiges davon, dessen Unterdrückung uns besonders leid gethan hätte, in diesen Abdruck wieder aufzunehmen; und hieraus mag sich der Leser die Ungleichförmigkeiten erklären, die er in manchen Paragraphen vielleicht bemerken wird. Sonst aber findet man,~\RWSeitenw{VIII}\ außer einigen unbedeutenden Anmerkungen unter dem Texte, auch gar nichts in diesem Abdrucke, das nicht der Verfasser selbst seinen Schülern \RWbet{schriftlich} übergeben hätte; keine einzige jener Bemerkungen, die er beim \Abweichung{Vortrage}{Vortragen}{} bloß \RWbet{mündlich} machte, keine einzige Stelle aus seinen Erbauungsreden, deren uns einige hundert zu Gebote standen, und in denen er nicht selten seine Ansichten umständlicher und eindringlicher aus einander setzte, als es in einem wissenschaftlichen, in gemessener Stundenzahl zu beendigenden Unterricht geschehen kann; durchaus nichts findet der Leser in diesem Abdrucke, als nur dasjenige, was der Verfasser selbst in seine Schulvorträge aufgenommen hat. Aenderungen der Sprache aber und des Ortes, auch Auslassungen, die wir jedoch ehrlich anzeigen wollen, haben wir uns allerdings in einigen Stellen erlaubt. Zuvörderst haben wir die angeführten Stellen des neuen Testamentes, die der Herr Verfasser nach seinem eigenen wiederholten Geständniß, nur deßwegen unübersetzt ließ, weil es ihm an Zeit gebrach, eine Uebersetzung derselben nach seinem Sinne zu machen, zu größerer Bequemlichkeit mancher Leser deutsch gegeben. Ein Gleiches that der Eine und der Andere von uns mit mancher Stelle aus lateinischen Schriftstellern. Die \RWbet{Einschaltung über die kantische Philosophie und über die neueste Art zu philosophiren in Deutschland}, die, leider! noch immer die \RWbet{neueste} ist, trug unser verehrungswürdiger Lehrer darum erst nach dem Hauptstück, welches von den \RWbet{Erkenntnißquellen} des Katholicismus handelt, vor, weil diese Einschaltung eine gewisse philosophische Ausbildung seiner Schüler voraussetzt, um verstanden zu werden. Wir haben uns erlaubt, sie an jenen Ort zu verlegen, wohin sie uns zu gehören schien. Hinter jedem einzelnen Hauptstück endlich, ja wohl auch hinter einzelnen Abtheilungen der Hauptstücke fanden wir Bemerkungen, die \RWbet{Literatur} des eben behandelten Gegenstandes betreffend. Sie schienen uns nur skizzirt, nur für die Schüler gemacht und der öffentlichen Mittheilung nicht werth zu seyn. Wir haben sie ganz weggelassen. Das ist aber auch Alles, was wir uns erlauben zu dürfen glaubten.\\[\baselineskip]
Geschrieben im Juli 1834.\editorischeanmerkung{In \Alabel\ folgt auf den Seiten \RWS{IX--XX} das Inhaltsverzeichnis von Bd.\,I, das in dieser Edition zugunsten des Gesamtinhaltsverzeichnisses (S.\,\pageref{IhvzAnfang}--\pageref{IhvzEnde}) ausgelassen ist.}\par
\endinput
