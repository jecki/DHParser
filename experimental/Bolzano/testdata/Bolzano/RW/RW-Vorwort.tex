%
% \PdUch{Vorwort}
Das vorliegende Werk ist zunächst für junge Männer bestimmt, die auf Hochschulen wissenschaftlichen Studien obliegen und zum Eintritte in die höheren oder gelehrten Stände sich vorbereiten; es kann und soll aber Allen, die eine wissenschaftliche Bildung genossen haben und sich mit den Lehren des katholischen Christenthums so vertraut machen wollen, wie es ihrer Bildungsstufe angemessen ist, ein brauchbares Handbuch seyn. Da Bücher dieser Art unstreitig ein sehr dringendes Bedürfniß sind und die deutsche katholische Literatur dasselbe noch lange nicht hinreichend befriedigt hat; so glaubt man durch Herausgabe des gegenwärtigen allen gebildeten Katholiken einen wesentlichen Dienst zu erweisen. Denn erwägt man die zahlreichen Mängel, an welchen der in den niederen Schulen zu ertheilende Religionsunterricht immer noch leidet; bedenkt man, daß sehr viele Jünglinge, während sie in so manchen Zweigen des höheren Wissens die besten Fortschritte machen, nur in Beziehung auf Religion bei den unvollständigen, unfruchtbaren und oberflächlichen Begriffen stehen bleiben, die ihnen im Knabenalter hierüber beigebracht worden sind; läßt man nicht unbeachtet, daß jeder Gebildete theils durch eigenes Nachdenken, theils durch Lesung religionswidriger Bücher oder Umgang mit Ungläubigen auf eine Menge Zweifel geführt wird, die seinen Glauben erschüttern können und müssen, wofern sie nicht auf eine befriedigende Weise gelöst werden: so darf man einerseits sich nicht wundern, daß das katholische Christenthum unter den höheren und gelehrten Ständen in unsern Tagen noch immer nicht so viel Vertrauen und Achtung gefunden hat, als ihm mit Recht gebühret; anderseits muß man eingestehen, daß, um diesem großen Uebelstande abzuhelfen, nichts erwünschlicher sey, als eine wissenschaftliche Darstellung sämmtlicher, in diesem Religionssysteme enthaltenen Lehren.\par
Eine solche Darstellung liegt uns hier vor, und wenn wir nicht irren, so hat sie, verglichen mit den trefflichen Werken ähnlicher Art, die uns Beda Maier, Ildephons Schwarz, Michael Sailer, Jakob Frint und Andere geliefert haben, sehr wesentliche Vorzüge. Sie dürfte dieselbe sowohl durch Vollständigkeit des Inhaltes, wie auch durch Gründlichkeit und wissenschaftliche Schärfe in Aufstellung der Begriffe und Beweise weit übertreffen. Schon die Pflichten, die den Menschen in Beziehung auf Religion obliegen, und die Fehler, die man hierin zu begehen pflegt, finden wir in diesem Werke mit einer mehr als gewöhnlichen Ausführlichkeit behandelt. Einen besonderen Fleiß aber hat der Verfasser auf jene Abschnitte seiner Darstellung verwendet, die in neuerer Zeit die philosophische Einleitung zur christlichen Dogmatik genannt worden sind, und die Beweise für die Nothwendigkeit und Möglichkeit einer göttlichen Offenbarung enthalten. Die Art, wie er diese Beweise führt, ist größtentheils originell und scheint ganz geeignet, jeden Denkenden zu befriedigen. Von gleicher Sorgfalt zeuget die Abhandlung über die Aechtheit und Glaubwürdigkeit der Bücher des Neuen Bundes, nicht minder jene über die einzelnen Wunder und Weissagungen, die zur Bestätigung der göttlichen Offenbarung dienen. Kommt auch in diesen Abschnitten weniger Neues vor, so wird gewiß die schöne und lichtvolle Anordnung des vorhandenen Stoffes und die scharfsinnige Abfertigung der gewöhnlichen Einwürfe Kenner sehr angenehm überraschen. Die Erläuterung der dogmatischen, moralischen und asketischen Lehren, die zum eigentlichen Inhalt des katholischen Christenthums gehören, zeichnet sich vorzüglich dadurch aus, daß sie bei jeder einzelnen Lehre den Begriff derselben, der kirchlichen Rechtgläubigkeit gemäß, genau und bestimmt angibt, sodann den historischen Beweis dafür aus Bibel und Tradition anführt, ferner ihre Vernunftmäßigkeit und zuletzt den sittlichen Nutzen, den sie, recht gebraucht, stiften könnte oder auch wirklich gestiftet hat, gründlich auseinander setzt. Die letzten zwei Punkte sind unsers Wissens noch von Niemandem bei allen einzelnen Lehren so ausführlich und sorgfältig erörtert worden. Sehen wir aber von dem Inhalte des Werkes ab, und fassen bloß die Art der Darstellung, deren sich der Verfasser bedient, in's Auge; so müssen wir ihm ebenfalls den Vorzug vor vielen Andern zugestehen. Er ist, was in der Schriftstellerwelt jetzt so selten geworden ist, durchaus bescheiden und vermeidet ganz und gar den absprechenden, hochfahrenden Ton, den in unserer Zeit so viele Philosophen und Theologen angenommen haben. Ueberall beurtheilt er die Meinungen, welche den seinigen entgegenstehen, mit vieler Schonung und stellt seine Behauptungen nie als apodiktisch gewiß, sondern bloß als wahrscheinlich hin, ganz in dem Geiste der Demuth, durch den besonders katholische Forscher sich auszeichnen sollen. Auch vermeidet er aufs Sorgfältigste die so beliebt gewordenen unbestimmten bildlichen Ausdrücke und jene hohlen, gelehrt klingenden Phrasen, denen keine klaren Begriffe zu Grunde liegen; er redet durchaus eine deutliche, verständliche Sprache, gibt scharf bestimmte Begriffe, kurze und schlagende Beweise. Demzufolge scheint uns dieses Werk mehr als jedes andere geeignet, wissenschaftlich Gebildete, denen die Religion nicht ganz gleichgültig ist, von der Wahrheit und Vortrefflichkeit des katholischen Lehrbegriffes zu überzeugen.
Allein dieß ist nicht der einzige Grund, der uns bewogen hat, dasselbe der Oeffentlichkeit zu übergeben. Wir finden darin auch viele Ansichten, die neu und für die Wissenschaft von hoher Bedeutung sind. Namentlich scheinen uns die Begriffserklärungen, die der Verfasser von Religion, göttlicher Offenbarung, Wunder und Weissagung gegeben hat, wie auch seine Theorie von den Kennzeichen einer göttlichen Offenbarung überaus wichtig, vorzüglich in Beziehung auf den Zustand, in welchem die wissenschaftliche Theologie sich gegenwärtig befindet. Wir glauben uns nicht zu täuschen, wenn wir behaupten, daß durch die Ansichten des Verfassers mehrere der schwierigsten Streitpunkte, über welche sich die Verfechter des Supernaturalismus und Rationalismus entzweiet haben, vollkommen gelöset werden. Gleiche Wichtigkeit für die Wissenschaft dürfte die Art haben, wie der Verfasser die Lehre vom obersten Grundsatz oder den Erkenntnißquellen des Katholizismus auffaßt und darstellt. Sie verdient nach unserm Dafürhalten die Aufmerksamkeit Aller, welche sich den wahren Geist des Katholizismus recht klar machen wollen, weil einerseits hier nicht, wie es bei älteren und neueren Dogmatikern so häufig zu finden ist, ein idealisirtes, sondern jenes Glaubensprinzip zu Grunde gelegt ist, welches die unparteiische Geschichte als das richtige anerkennt, und weil andrerseit eben durch dieses Prinzip die Möglichkeit gegeben ist, die biblische Exegese und die Kirchengeschichte fortan nicht bloß nach praktischen Zwecken, sondern ächt wissenschaftlich zu behandeln. Nicht minder beachtenswert scheinen uns endlich jene Stellen zu seyn, in denen der Verfasser entweder seine eigenen philosophischen Meinungen gelegentlich ausspricht oder die gangbaren philosophischen Systeme beurtheilt; wir hoffen, daß Freunde wissenschaftlicher Forschung auch an diesen Orten manche VIII Knoten glücklich gelöset finden werden.\par
Hiermit glauben wir uns hinlänglich gerechtfertiget zu haben, daß wir dem gelehrten Publiko ein Werk übergeben, auf dessen Herausgabe der Verfasser aus besonderen, bloß für ihn verbindlichen Gründen Verzicht geleistet, das aber als Handschrift schon in den Händen vieler Tausende sich befindet und bereits großen Segen verbreitet hat. Es bleibt uns nur noch der Wunsch übrig, es möchten sich recht viele stimmfähige Richter über den Inhalt desselben aussprechen, und dadurch eben so, wie der würdige Verfasser es anstrebt, zur Fortbildung der Wissenschaft mitwirken.\\[\baselineskip]
Geschrieben im Juni 1834.\par
Die Herausgeber.

\endinput