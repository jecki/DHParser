\RWteil{II}{Zweiter Haupttheil.}{Von den Wundern, die zur Bestätigung des katholischen Christenthums dienen.}{\RWSeitenwohne{1}}
\clearpage\linenumbers%
\RWch{Nöthige Vorausschickungen.}

\RWpar{1}{Wie man die Untersuchung der verschiedenen positiven Religionen auf Erden am füglichsten einrichten könne?}
\begin{aufza}
\item Wenn wir durch alles Bisherige überzeugt worden sind, daß eine göttliche Offenbarung etwas für uns sehr Erwünschliches wäre; und wenn wir auch keine Ursache haben, sie für unmöglich zu halten; auch bereits wissen, an welchen Merkmalen sie zu erkennen seyn müßte: so können wir uns unter keinem Vorwande mehr von der Verbindlichkeit lossagen, nun auch in Untersuchung zu nehmen, \RWbet{ob eine solche göttliche Offenbarung in der That bestehe?}
\item Wir finden aber alsbald, daß es nicht bloß einen einzigen, sondern sehr \RWbet{viele} Religionsbegriffe gibt, die alle aussagen, daß sie von Gott \RWbet{geoffenbart} wären. Da stoßen wir erstlich schon auf mehre Lehrbegriffe, welche sich \RWbet{christliche}, bald mit diesem, bald jenem Beisatze, bald auch ganz ohne Beisatz nennen, und von Denjenigen, die sie uns vortragen, durchaus für wahre göttliche Offenbarungen erklärt werden. Nebst diesen christlichen Religionen rühmen sich auch noch die alte \RWbet{mosaische}, so wie die jüngere \RWbet{rabbinisch-jüdische} Religion, die \RWbet{jüdische Kabbala}, die \RWbet{indische}, \RWbet{persische}, \RWbet{chinesische}, \RWbet{muhamedanische} und mehre andere nicht nur lebende, sondern auch bereits ausgestorbene Religionen rühmen und rühmten sich göttlich geoffenbart zu seyn.
\item Bei diesen Umständen erhebt sich die Frage, \RWbet{auf welche Weise wir die Untersuchung dieser verschiedenen Religionsbegriffe einleiten müssen, um ohne der Sicherheit Abbruch zu thun, doch so viel möglich ist, an Zeit und Mühe zu ersparen}?~\RWSeitenw{4}
\item Da unter mehren Religionen, die ihre Entstehung insgesammt gewissen außerordentlichen Ereignissen verdanken, nach Th.\,I.\ \RWparnr{145}\ \no\,4 immer nur diejenige als Gottes wahre an uns ergangene Offenbarung angesehen werden darf, an deren Lehrbegriffe wir die größte sittliche Zuträglichkeit für uns entdecken; so könnten wir zuvörderst schon dadurch etwas an Zeit und Mühe ersparen, daß wir die Prüfung der mehren Religionen, die alle sich für geoffenbart ausgeben, ohne daß sich der Ungrund dieses Vorgebens gleich auf den ersten Blick erkennen läßt, nicht mit der Untersuchung der \RWbet{Wunder}, die sie zu ihrer Bestätigung aufweisen, sondern mit Untersuchung ihres \RWbet{Lehrbegriffes} anfangen. Denn schon im Voraus läßt sich erwarten, daß wir der Religionen, die ihre Entstehung, Erhaltung und Ausbreitung einem mehr oder weniger ungewöhnlichen Zusammenflusse günstiger Umstände zuschreiben können, und die wir somit, wenn ihre Lehre darnach ist, schon als bestätiget durch Wunder ansehen könnten, weit mehre antreffen werden als solche, an deren Lehrbegriffe wir nichts ausstellen können und vermissen. Dieß wieder aus einem doppelten Grunde:
\begin{aufzb}
\item einmal, weil es der Güte Gottes gemäß ist, daß er, um der Empfänglichkeit und dem Bedürfnisse verschiedener Völker und Zeitalter desto genauer zu entsprechen, nicht Allen \RWbet{Ein} und \RWbet{Dasselbe}, sondern Verschiedenen \RWbet{Verschiedenes} (wenn auch nicht eben Widersprechendes) offenbare.
\item Dann ist auch nichts begreiflicher, als daß aus \RWbet{Einer} und ebenderselben ursprünglich \RWbet{einfachen} Religion im Verlaufe der Zeit durch die versuchte Ausbildung derselben \RWbet{mehre} bald mehr, bald weniger von derselben abweichende Lehrbegriffe hervorgehen, die alle sich auf dieselben Wunder, durch welche ihre \RWbet{Mutterreligion} in die Welt eingeführt wurde, berufen können, wenn ihre Lehre nur sittliche Zuträglichkeit hat.
\end{aufzb}
\item Noch beträchtlicher würde sich unsere Arbeit abkürzen, wenn eine Religion, und zwar gerade diejenige, auf die wir glücklicher Weise zuerst unser Augenmerk richten, in ihrem Lehrbegriffe einen so hohen Grad der Vollkommenheit hätte, daß die Ansichten,~\RWSeitenw{5}\ welche sie uns über jeden Gegenstand beibringt, nicht nur \RWbet{zuträglicher} sind, als alles dasjenige, was uns die übrigen auf Erden befindlichen Religionen über denselben Gegenstand sagen, sondern auch \RWbet{zuträglicher}, als eine jede andere Ansicht, die wir nur auszudenken vermögen. In diesem Falle würde es nämlich nicht nöthig seyn, die Lehrbegriffe der übrigen Religionen alle der Reihe nach durchzugehen; sondern es würde höchstens, um sicher zu seyn, daß wir nicht eine Ansicht für die wohlthätigste aus allen erdenklichen halten, bloß weil uns noch Eine mögliche andere nicht einfällt, erforderlich seyn, unsere Blicke \RWbet{zuweilen} auf die noch übrigen Religionen zu richten.
\item Um uns nun diesen Vortheil, wo möglich, zuzuwenden, handeln wir wohl vernünftig, wenn wir die sämtlichen Religionen, welche sich für geoffenbart ausgeben, nicht in was immer für einer beliebigen Ordnung der Prüfung unterziehen, sondern den Anfang mit einer solchen machen, von der es schon nach der oberflächlichen Kenntniß, die wir im Voraus haben, am Wahrscheinlichsten ist, daß sie eine wahre göttliche Offenbarung seyn könnte.
\end{aufza}

\RWpar{2}{Die christlichen Religionen, und unter ihnen vornehmlich die katholische, verdienen es, daß wir unser Augenmerk auf sie zuerst richten}
\begin{aufza}
\item Es gibt in der That mehre und sehr vernünftige Gründe, die uns bestimmen können, bei unserer Untersuchung der verschiedenen Religionen auf Erden unser Augenmerk \RWbet{zuerst} auf eine der \RWbet{christlichen}, namentlich auf die \RWbet{katholisch-christliche} Religion zu richten. 
\item Statt mancher andern Gründe will ich hier nur diesen einzigen nennen: Unter allen auf Erden befindlichen \RWbet{Gesellschaftsreligionen}, welche uns näher bekannt sind, und ihrer höheren Vollkommenheit wegen unsere Aufmerksamkeit verdienen, ist die katholische die \RWbet{einzige}, welche den Satz aufstellt, daß \RWbet{nur dasjenige}, was~\RWSeitenw{6}\ ihre \RWbet{sämmtlichen Glieder gemeinschaftlich} bekennen, also gerade dasjenige, was nach Th.\,I.\ \RWparnr{22}\ \no\,4.\ einzig zu ihrem Inhalte gerechnet werden kann, mit der Zuversicht einer wahren göttlichen Offenbarung angenommen werden könne und solle. Durch diese Erklärung gibt die katholische Religion sich selbst für eine wahre göttliche Offenbarung aus, was wir von einer jeden Religion, die dieses wirklich ist, erwarten können. Da nun die übrigen Gesellschaftsreligionen dieß insgesammt unterlassen, da selbst die vorzüglichsten, die es noch gibt, nämlich die akatholisch-christlichen nicht auf den \RWbet{allgemeinen} unter ihren Bekennern herrschenden \RWbet{Glauben}, sondern auf irgend ein \RWbet{Buch}, als auf die \RWbet{einzig sichere} Quelle, aus der wir den Inhalt der wahren göttlichen Offenbarung kennen zu lernen vermögten, hinweisen: so erklären sie sich durch dieses Verfahren selbst alle für mangelhaft; und es ist also wohl vernünftig, daß wir unser Augenmerk vor allen andern Volksreligionen zuerst auf die katholische richten.
\item Zunächst nach dieser aber werden allerdings auch die andern christlichen Lehrbegriffe, sowohl diejenigen, welche von ganzen Völkern, als auch solche, die nur von einzelnen Gelehrten angenommen werden, unsere Aufmerksamkeit verdienen; wenn aus keinem andern Grunde, schon deßhalb, weil ihre Lehren in einer so nahen Verwandtschaft mit der katholischen stehen.
\end{aufza}

\RWpar{3}{Inhalt und Abtheilung dieses Haupttheiles}
\begin{aufza}
\item Nach dem so eben Gesagten wäre es wohl am zweckmäßigsten, daß wir uns gegenwärtig sofort zur Untersuchung der katholischen Religion, und zwar zur \RWbet{Prüfung ihres Lehrbegriffes} wenden; und wenn es sich zeigte, daß er uns über jeden Gegenstand Ansichten aufstellt, die Alles übertreffen, was uns in andern Religionen gesagt wird, ja was wir uns nur zu erdenken vermögen: so würden wir dann bloß nöthig haben zu untersuchen, ob dieser Religion auch das zweite Zeichen einer wahren göttlichen Offenbarung, nämlich die \RWbet{Bestätigung durch Wunder} zukomme. Allein schon Th.\,I.\ \RWparnr{8} bemerkte ich, daß mich gewisse Rücksichten bestimmen, die Ordnung umzu\RWSeitenw{7}kehren, und daß ich somit den Beweis, daß die katholische Religion das äußere Kennzeichen einer göttlichen Offenbarung habe, oder daß sie ihre Entstehung, Erhaltung und Ausbreitung einem Zusammenflusse der außerordentlichsten Begebenheiten verdanke, \RWbet{zuerst} vortragen, und mit der Prüfung des Lehrbegriffes dann den Beschluß machen werde. Damit es aber Niemand auch nur im Anfange für ein Zeichen der Parteilichkeit halte, daß ich gerade die katholische Religion vor allen andern geprüft wissen wolle; wie auch, damit man jene außerordentlichen Begebenheiten, die zur Entstehung, Erhaltung und Ausbreitung dieses Lehrbegriffes mitgewirkt haben, nöthigen Falles schon jetzt als Wunder, die seine Göttlichkeit beweisen, ansehen dürfe: wird es zweckmäßig seyn, in einem eigenen Hauptstücke zu zeigen, \RWbet{was für ein günstiges Vorurtheil für die Vortrefflichkeit des katholischen Lehrbegriffes bloß aus Betrachtung gewisser äußerer Gründe hervorgehe}, ohne daß man sich noch in eine Betrachtung der einzelnen Lehren selbst einzulassen braucht.
\item Da ferner die Wunder, die zur Bestätigung des katholischen Christenthumes dienen, auf gewissen zum Theile sehr alten Nachrichten beruhen, und da man in neuerer Zeit über die Natur der historischen Erkenntniß, besonders über die Möglichkeit einer Beglaubigung der Wunder durch ein historisches Zeugniß manche sehr unrichtige Begriffe verbreitet hat: so wird es nöthig seyn, auch über diesen Gegenstand, \RWbet{über die Natur des historischen Glaubens besonders in Hinsicht auf Wunder}, Ein und das Andere vorauszuschicken.
\item Ein sehr beträchtlicher Theil der Wunder, die zur Bestätigung des katholischen Christenthumes dienen, wird in den sogenannten \RWbet{Büchern des neuen Bundes} erzählt, und über ihren eigentlichen Hergang kann nur in sofern etwas Genaueres entschieden werden, als man die \RWbet{Glaubwürdigkeit} dieser Bücher einräumt. Es wird also zweckmäßig seyn, die \RWbet{Gründe} mitzutheilen, aus welchen die hohe Glaubenswürdigkeit dieser Bücher, namentlich der historischen erhellet.~\RWSeitenw{8}
\item Nach dieser dreifachen Vorausschickung wird es erst möglich seyn, die außerordentlichen Begebenheiten, denen der katholische Lehrbegriff seine Entstehung, Erhaltung und Ausbreitung verdankt, auf eine solche Art vorzuführen, daß ihre Zuverlässigkeit Jedem einleuchten wird.
\end{aufza}\par
Auch dieser Haupttheil wird also in folgende \RWbet{vier} Unterabtheilungen oder \RWbet{Hauptstücke} zerfallen:
\begin{aufzb}
\item[\RWbet{Erstes Hauptstück.}] Aeußerer Beweis für die sittliche Zuträglichkeit des katholischen Lehrbegriffes.\par
\item[\RWbet{Zweites Hauptstück.}] Ueber die Natur der historischen Erkenntniß, besonders in Hinsicht auf Wunder.\par
\item[\RWbet{Drittes Hauptstück.}] Aechtheit, Unverfälschtheit und Glaubwürdigkeit der historischen Bücher des neuen Bundes.\par
\item[\RWbet{Viertes Hauptstück.}] Einzelne Wunder, die zur Bestätigung des katholischen Lehrbegriffes dienen.~\RWSeitenw{9}
\end{aufzb}

\RWch{Erstes Hauptstück.\\ Aeußerer Beweis für die sittliche Zuträglichkeit des katholischen Lehrbegriffes.}
\RWpar{4}{Inhalt und Plan dieses Hauptstückes}
\begin{aufza}
\item Ohne die Lehre einer Religion selbst noch zu kennen, kann man zuweilen bloß aus Betrachtung derer, die ihr entweder zugethan oder abgeneigt sind, also aus einem bloß \RWbet{äußeren Grunde} mit einer bald größeren, bald geringeren Zuversicht beurtheilen, ob diese Lehre der Tugend und Glückseligkeit der Menschen zuträglich sey. Wir wollen also jetzt eben untersuchen, was sich schon bloß auf diese Art für oder wider die sittliche Zuträglichkeit des katholischen Lehrbegriffes ausmachen lasse.
\item Bei einigem Nachdenken wird Jeder gewahr werden, daß wir uns dieser Art, über die Wahrheit oder Falschheit eines uns vorliegenden Satzes zu urtheilen, in unzähligen Fällen, und oft mit sehr gutem Erfolge bedienen. So wohnten wir vielleicht schon oft als bloße Zuhörer einem Streite bei, welchen ein Paar Gelehrte über einen Gegenstand, vor dem wir eigentlich gar nichts verstanden, mit einander führten, und ohne die Gültigkeit der Gründe, welche der Eine oder der Andere vorbrachte, nur im Geringsten würdigen zu können, waren wir doch im Stande, mit ziemlicher Zuversicht zu errathen, welcher von Beiden Recht haben möge, bloß dadurch, daß wir auf ihr Betragen bei dem Streite, und auf einige andere äußere Umstände merkten, \zB\ wenn wir sahen, daß sich der Eine immer ruhig verhalte, während der Andere öfters in Hitze geräth, \usw~\RWSeitenw{10}
\item Immerhin mag nun das Urtheil, das wir aus solchen Gründen fällen, verglichen mit einem Urtheile, das aus Betrachtung der \RWbet{inneren Gründe} hervorgeht, ein bloßes \RWbet{Vorurtheil} heißen, es ist ein Vorurtheil in der guten Bedeutung des Wortes (ein vorgefaßtes, nicht eben voreiliges Urtheil, \RWlat{judicium praeconceptum non praecipitatum}).
\item Sehr billig also, daß wir von dieser eigenen Art, die Sätze zu beurtheilen, auch bei der Religion Gebrauch zu machen versuchen. Und wirklich wird sich, wie gesagt, zeigen, daß bloß auf diese Art schon ein sehr \RWbet{günstiges Vorurtheil} für die Vortrefflichkeit der Lehre des Christenthumes entstehe, ein Vorurtheil, welches man allenfalls den \RWbet{Auctoritätsbeweis} für die Vortrefflichkeit dieses Lehrbegriffes nennen könnte.
\item Das Christenthum hat aber Beydes, \RWbet{Freunde} sowohl als \RWbet{Feinde} in der Welt gefunden. Aus der Betrachtung der Freunde, die es gefunden hat, werden wir \RWbet{zwei zu Gunsten des Christenthums lautende Behauptungen} ableiten. Diesen stehet scheinbar Manches entgegen, was größtentheils von der Betrachtung der \RWbet{Feinde} des Christenthums entlehnt ist: wir werden es also unter der Form von \RWbet{Einwürfen} anführen und erwidern.
\end{aufza}

\RWpar{5}{Erste Behauptung}
\begin{aufza}
\item Wie die Geschichte lehrt, \RWbet{hat sich bei einer jeden seit der Entstehung des Christenthumes bis auf den heutigen Tag versuchten Verkündigung desselben, unter Menschen, die es bisher noch nicht gekannt hatten, immer die merkwürdige Erscheinung eingefunden, daß man sogleich bereitwillig gewesen, den alten angebornen Glauben gegen die neue christliche Lehre zu vertauschen}.
\item Aus dieser Thatsache leite ich den Schluß ab, \RWbet{daß die Lehre des Christenthumes vor allen nicht-christlichen Religionen einen gewissen Vorzug besitzen müsse, der selbst dem ungebildetsten Verstande einleuchtet}.~\RWSeitenw{11}\ 
\item Dieß meine erste Behauptung, die, wie man sieht, aus zwei Bestandtheilen, einem \RWbet{historischen} Satze, und einer aus diesem gezogenen \RWbet{Folgerung} bestehet.
\end{aufza}

\RWpar{6}{Beweis des ersten oder historischen Theils dieser Behauptung}
\begin{aufza}
\item Am allerauffallendsten ist die günstige und schnelle Aufnahme, welche das Christenthum gleich in den \RWbet{ersten Zeiten seiner Erscheinung} auf Erden gefunden hatte. Denn völlig so, wie es der Stifter desselben geweissaget hatte, noch vor Jerusalems Zerstörung (also in einem Zeitraume von weniger als 40 Jahren) hatte es sich durch das gesammte römische Reich verbreitet. Nicht nur im ganzen \RWbet{jüdischen Lande} (Judäa und Samaria), sondern auch in \RWbet{Kleinasien, Griechenland, Macedonien, Italien} und \RWbet{Afrika} gab es so viele Christen, daß schon in allen größeren Städten dieser Länder ansehnliche Gemeinden errichtet werden konnten, \zB\ zu Korinth, Thessalonich, in Galatien, zu Alexandrien \usw\ Selbst nach den unverdächtigen Berichten, welche uns \RWbet{heidnische} Schriftsteller, \zB\ \RWbet{Tacitus, Plinius} (der jüngere), \RWbet{Suetonius} \uA , hierüber geben, standen zu Anfange des zweiten Jahrhundertes die Tempel der Götzen beinahe leer, die Schlachtthiere fanden schon keine Käufer mehr, die Götzenpriester mußten von ihren Gewerben abstehen, \usw
\item Doch auch noch in den \RWbet{folgenden Jahrhunderten} fand das Christenthum in allen denjenigen Ländern, in welchen es geprediget wurde, einen sehr leichten Eingang, und eine sehr bereitwillige Aufnahme. Die \RWbet{Europäer} nämlich, besonders die Spanier, die Portugiesen und Franzosen, ließen das Christenthum durch eigends ausgesandte Missionäre in allen fünf Welttheilen verkünden; und mit so geringer Geschicklichkeit sich diese Prediger auch oft benahmen: doch fanden sich überall gleich nach den ersten Vorträgen viele tausend Menschen bereit, ihren bisherigen Glauben mit dieser neuen christlichen Lehre zu vertauschen. Zum Beweise lese man nur die Geschichtsbücher, die uns die erste Aufnahme~\RWSeitenw{12}\ des Christenthumes in Indien, Japan, China, Abyssinien, Amerika, Australien u.\,a.\,O.\ erzählen.
\end{aufza}

\RWpar{7}{Beweis des zweiten Theils jener Behauptung}
Es fragt sich nun, ob man den Grund dieser so großen Bereitwilligkeit aller Menschen, um der Religion Jesu willen ihre bisherigen Religionen zu verlassen, mit Recht in einem gewissen von ihnen bemerkten \RWbet{Vorzuge des christlichen Lehrbegriffes vor ihrem eigenen} setze? Und dieses wird, däucht mir, offenbar, wenn sich bei einer näheren Betrachtung zeigt, daß sich kein anderer Erklärungsgrund dieser so weit verbreiteten Erscheinung auffinden lasse.
\begin{aufza}
\item Zuvörderst leuchtet ein, daß eine bloße \RWbet{Liebe zu Neuerungen} diese Wirkung sicherlich nicht hervorgebracht habe. Neuerungssucht pflegt die Menschen höchstens zu solchen Umänderungen zu bewegen, die ihren Unternehmern nicht viele Mühe machen, bei denen sie sich wohl gar gewisse Vortheile zum wenigsten \RWbet{versprechen}; \zB\ in Kleidungen, Geräthschaften \udgl\  In \RWbet{Meinungen} dagegen, weil ihre Ablegung immer viel Mühe verursacht, bleiben die Menschen, wie die Erfahrung lehrt, gerne beim Alten, bei den Begriffen ihrer Kindheit stehen. Am allerwenigsten mögen sie \RWbet{religiöse} Meinungen aus bloßer Neuerungssucht verändern. Denn solche haften insgemein zu tief, haben den größten Einfluß auf ihr gesammtes Thun und Lassen, auf alle ihre Hoffnungen und Besorgnisse, stehen bei allen Völkern in dem Ansehen der höchsten Heiligkeit und Unverletzlichkeit, sind bei den meisten überdieß mit einer Menge politischer Gebräuche, ja mit der ganzen Verfassung des Staates auf's Innigste verbunden. -- Die Juden endlich, so wie ihr Charakter zu den Zeiten Jesu beschaffen war, ingleichen die selbstgenügsamen Chinesen in unserer neueren Zeit, waren und sind von Seite ihrer nur allzu hartnäckigen Anhänglichkeit an das Alte sattsam bekannt; und dennoch nahmen auch selbst aus diesen beiden Völkern so viele Tausende das Christenthum an, als es bei ihnen geprediget wurde.
\item Noch weniger kann der \RWbet{Hang zur Sinnlichkeit und zu einem ungebundenen Leben} als die wahre~\RWSeitenw{13}\ Ursache dieser Erscheinung angesehen werden. Bekanntlich ist das Christenthum eben nicht die Religion, welche der Sinnlichkeit der Menschen schmeichelt; es fordert schlechterdings und strenger, als eine jede andere Volksreligion es thut, die vollkommenste Unterordnung aller Neigungen und Begierden unter den Willen Gottes oder das Sittengesetz. Und so unvollkommen zuweilen auch die Begriffe seyn mochten, welche der Heide bei seinem Uebertritte zum Christenthume von diesem letztern hatte; so konnte er doch auf keine Weise hoffen, daß er hier seinen Leidenschaften ungehinderter werde fröhnen dürfen, als es bei seinem bisherigen Glauben geschah. Oder sollen wir sagen, daß ihn die \RWbet{mißverstandene} Lehre von der \RWbet{Vergebung der Sünden durch die Verdienste Jesu} so etwas habe erwarten lassen? Aber auch diese Bedenklichkeit fällt weg, wenn wir erwägen, daß es in keiner heidnischen Religion an vorgeblichen Versöhnungsmitteln, bestehend in Opfern, Bädern \udgl , die zur Beruhigung für den Leichtsinnigen hinreichen konnten, gefehlt habe. Ueberhaupt widerspricht dem argen Verdachte, als ob nur Hang zur Zügellosigkeit so viele tausend Menschen (denn von Einzelnen wollen wir es nicht in Abrede stellen) zur Annahme des Christenthumes bewogen hätte, die Erfahrung selbst zu offenbar, als daß man demselben Raum geben könnte. Denn im Ganzen genommen zeigten sich ja bei allen neubekehrten Völkern ein Geist der Buße und nicht der Zügellosigkeit, Besserung der Sitten und nicht vermehrte Ausschweifungen und neue Laster. Nicht schlimmer sondern \RWbet{besser} wurden die Völker durch ihren Uebertritt zum Christenthume, wenn nicht für immer, doch gewiß im Anfange.
\item Auch nicht der \RWbet{Wunsch, seine äußere Lage zu verbessern, Furcht vor Verfolgungen} oder \RWbet{die eitle Hoffnung, sich in die Gunst der Großen einzuschmeicheln, Ruhm-} oder \RWbet{Ehrbegierde} sind die Beweggründe gewesen, welche den mächtigen Anhang, den sich das Christenthum so schnell und allgemein erwarb, erklären; obgleich es wahr ist, daß solche Triebfedern demselben manchen, ja, wenn man will, viele Bekenner gewannen. Wahr ist es nämlich allerdings, daß
\begin{aufzb}
\item gleich in den ersten christlichen Gemeinden, besonders in jener zu Jerusalem, die Armen eine sehr reichliche Unter\RWSeitenw{14}stützung fanden; und so konnte es sich immerhin fügen, daß einige durch Armuth gedrückte Menschen, ohne noch eben den innern Vorzug des Christenthumes ganz deutlich einzusehen, bloß zur Verbesserung ihrer äußeren Lage, dasselbe annahmen. Allein die \RWbet{Armen} hätten solch eine Unterstützung nicht finden können, wenn nicht erst viele \RWbet{Reiche} zu diesem Christenthume übergetreten wären. Was hatte nun diese vermocht? Und ist es nicht bekannt, daß eben diese Juden, die man hier in den Verdacht bringen will, daß sie das Christenthum aus bloßem Eigennutze angenomen haben, vor und nach Jesu Zeiten bereit gewesen waren, eher die bittersten Leiden und Martern auszustehen, als sich zu ihres Glaubens Abläugnung zu entschließen? Wahr ist es ferner, daß
\item besonders nach Konstantin's Zeiten den Sclaven, welche zum Christenthume übertraten, die Freiheit ertheilt worden sey, und dieses mag wohl Manchen jener Unglücklichen mehr als Vernunftgründe zur Annahme des Christenthumes bewogen haben. Auf gleiche Weise kann auch
\item als man Gewalt zu brauchen anfing, die Furcht vor Verfolgungen oder die Hoffnung, sich bei den christlich gewordenen Kaisern einzuschmeicheln, Tausende bestimmt haben, das Bad der Taufe zu verlangen. Allein es traten doch auch so viele Freigeborne zum Christenthume über, und diese Religion war schon sehr allgemein verbreitet, bevor es noch christliche Kaiser gab, welchen es einfiel, Belohnung oder Strafe mit der Annahme oder Verwerfung des Christenthums zu verbinden.
\item Wenn endlich aufgeklärte Europäer bei einem rohen Volke zugleich mit dem Christenthume auch allerlei neue Kenntnisse verbreiteten, oder wenn sonst auf irgend eine Weise die Partei des Christenthumes sich ein überwiegendes Ansehen erworben hatte: so sah man von nun an eine Art von Ehre darin, zu dieser Partei zu gehören (beiläufig eben so, wie gegenwärtig Viele sich eine Ehre daraus machen, zu der Partei der Freigeister zu gehören); und nun mochten sich Manche aus bloßer Eitelkeit, ohne durch Gründe gehörig überzeugt zu seyn, zum Christenthume bekennen. Aber auch diese Triebfeder kann nicht so allgemein gewirket haben, wie denn auch heut zu Tage die Zunft der Freigeister nie sehr beträchtlich anwächst. Wenn man dagegen bedenkt, daß gerade dort, wo sich das Christenthum am Schnellsten ausgebreitet hatte, der Name~\RWSeitenw{15}\ eines Christen der \RWbet{öffentlichen Verachtung} Preis gegeben war; wenn man erwäget, daß bei Weitem die meisten Christen durch ihren Uebertritt zum Christenthume \RWbet{in jeder bloß irdischen Rücksicht eher verloren, als gewannen}, daß sie ihrer öffentlichen Aemter und Würden entsetzt, ihres Vermögens beraubt, verspottet, verachtet, auf alle erdenkliche Weise verfolgt und mißhandelt worden seyen: so kann man nicht ferner zweifeln, es sey nichts Anderes, als die Ueberzeugung gewesen, daß sie an dem Christenthume eine wahre göttliche Offenbarung hätten, durch deren Nichtachtung sie sich Gottes höchstes Mißfallen zuziehen würden, was Millionen bestimmte, sich zu demselben zu bekennen.
\end{aufzb}
\item Doch diese Ueberzeugung ward vielleicht nicht sowohl durch die einleuchtende Vortrefflichkeit des christlichen \RWbet{Lehrbegriffes}, als vielmehr nur durch die \RWbet{größere Menge und Glaubwürdigkeit der Wunder} bewirket, die sich einst zur Bestätigung desselben zugetragen hatten? Auch dieß nicht, sage ich.\par
Denn \RWbet{erstlich} sind \RWbet{Wunder allein}, wie wir wissen, auch in den Augen der gemeinen Menschenmenge kein hinreichender Beweis des göttlichen Willens, eine gewisse Religion als geoffenbart anzunehmen, wenn nicht auch ihre Lehre selbst einen \RWbet{höheren Grad sittlicher Vortrefflichkeit} hat, als jene, die man um ihrentwillen verlassen soll.\par
\RWbet{Zweitens} ist es auch hinlänglich bekannt, daß die große Menge der Menschen in Wundererzählungen, die sie von ihren Kindesjahren an gehört hat, und die im ganzen Lande geglaubt werden, \RWbet{nicht das geringste Mißtrauen} setze. Möchten also die Wundererzählungen, welche die übrigen Religionen aufzuweisen haben, in den Augen des Kritikers auch noch so wenig Glaubwürdigkeit besitzen: die große Menge zweifelte nie an denselben, sie fühlte nie das Bedürfniß, strenger erwiesene Wunder zu finden. Die Erzählung der evangelischen Wunder, die ihren Ohren neu klang, mußte der großen und ungebildeten Menge der Menschen in fernen Ländern noch eher verdächtig vorkommen, als daß sie eingesehen hätte, wie viel gewichtiger die Beweisgründe wären, die sich für ihre Wahrheit anführen lassen.~\RWSeitenw{16} Und so sehen wir denn, daß nur die größere sittliche Zuträglichkeit der christlichen Lehren, eine Zuträglichkeit, die so beschaffen ist, daß sie nicht nur von Gelehrten, sondern selbst von der großen Menge der Menschen begriffen werden kann, genügend erkläre, warum die Menschen auf dem ganzen Erdenrunde das Christenthum nur kennen zu lernen brauchen, um es bereitwillig mit ihren angebornen Religion zu vertauschen. Die Lehren des Christenthums besitzen also vor allen übrigen Religionen auf Erden einen Vorzug solcher Art, der selbst dem ungebildetsten Verstande einleuchtet.
\end{aufza}

\RWpar{8}{Einwürfe und Widerlegung}

\begin{center}\RWbet{Einwürfe.}\end{center}

\begin{aufza}
\item So glücklich auch das Geschäft der Bekehrung zum Christenthume im Anfange fortschritt, so gerieth es doch beinahe allenthalben bald wieder in Stocken, und wurde rückgängig. So viele Nationen gibt es, welche das Christenthum, nachdem es kaum Eingang bei ihnen gefunden hatte, plötzlich auf eine solche Art wieder von sich wiesen, daß es beinahe den Anschein erhält, als hätten sie es nur so lange geachtet, als sie die Lehren desselben noch nicht genauer kannten; bei einer näheren Bekanntschaft aber hätte sich's bald gezeigt, daß es die menschenfreundliche und weltbeglückende Religion nicht sey, für die es sich ausgibt. Solch einen unglücklichen Ausgang nahm das Bekehrungsgeschäft in Japan, in China, in Abyssinien und an vielen andern Orten.
\item Ja was noch mehr ist, gibt es nicht eine Nation, die seit Jahrhunderten mitten unter uns Christen wandelt, volle Gelegenheit hat, das Christenthum, das von ihr ausgegangen ist, kennen zu lernen, die erwünschteste Erleichterung ihres harten Schicksals dabei fände, wenn sie es annähme: und die es gleichwohl (bis auf wenige, meistens sehr zweideutige Ueberläufer) standhaft von sich weiset?
\item Gibt es nicht endlich auch ganze Völkerschaften, bei welchen das Christenthum schon durch Jahrhunderte wirklich geherrschet hatte, und die dann plötzlich, der reinen, recht\RWSeitenw{17}gläubig genannten Lehre müde, zu einem neuen Bekenntnisse übertraten? Von Zeit zu Zeit rissen sich solche bald mehr, bald weniger beträchtliche Theile von dem Körper der katholischen Kirche los, weil, wie sie wenigstens behaupteten, bei dieser die Wahrheit in Irrthum übergegangen wäre. Es scheint also doch nicht, daß die Lehre der katholisch-christlichen Kirche einen so einleuchtenden Vorzug vor allen übrigen behaupte.
\end{aufza}

\begin{center}\RWbet{Beantwortung dieser Einwürfe.}\end{center}

\begin{aufza}
\item Die Ursache jenes so plötzlichen Stockens, in welches die anfangs allenthalben so schnellen Fortschritte des christlichen Predigtamtes in heidnischen Ländern geriethen, ist wahrlich nicht die, weil eine nähere Bekanntschaft mit dem Christenthume Mängel an seiner Lehre entdecken ließ, die man im Anfange übersehen hatte; sondern die Schuld liegt lediglich an dem \RWbet{abscheulichen Benehmen der Europäer} in jenen Ländern; einem Benehmen, welches, obgleich von Christen herrührend, den Grundsätzen des Christenthums doch eben so sehr, wie aller gesunden Sittenlehre, zuwiderlief. Nicht an der Lehre des Christenthumes, sondern an dem dieser Lehre geradezu widersprechenden Lebenswandel, den nicht sowohl die Missionäre selbst, aber doch die mit ihnen zugleich dahin gekommenen Personen führten, fanden die Neubekehrten einen Stein des Anstoßes. Ein schnöder Eigennutz, eine bis an Muth grenzende Begier nach Gold und Edelsteinen, der häßlichste Neid, schamlose Ausbrüche der wildesten Leidenschaften, der Wollust, der Völlerei, des Trunkes, der Treulosigkeit, diese und andere dergleichen Laster, welche die Europäer erst nach und nach, aber immer deutlicher an sich wahrnehmen ließen, mußten sie nothwendiger Weise verächtlich und hassenswerth machen; und war es dann ein Wunder, wenn diese Verachtung und dieser Haß zuletzt auch auf die Religion, welche von ihnen geprediget ward, überging? -- Und wenn es sich am Ende offenbarte, daß die \RWbet{europäischen Fürsten} bei ihrer Verkündigung des Christenthumes in diesen Ländern zuletzt keine andere Absicht hatten, als die bekehrten Völker zu unterjochen und auszusaugen: was war natürlicher, als daß die letzteren endlich argwöhnisch gegen alle Europäer wurden, ja ihnen hie und da sogar den Eingang in ihre Länder verboten?~\RWSeitenw{18}
\item Daß Israels zerstreute Nachkommenschaft von ihrer unvolkommenen Kinderreligion noch immer nicht übertreten will zu den vollendeten Offenbarungen Gottes an die Menschheit, daran ist nichts, als [das] \RWbet{Verfahren der Christen selbst} Schuld. Seit wenigstens anderthalb tausend Jahren wird dieses unglückliche Volk von uns auf das Unverantwortlichste behandelt und mißhandelt. Zu allen höheren Aemtern und Würden im Staate ward ihm (mit Ausnahme nur dieser neuesten Zeit) der Zutritt abgeschnitten; nicht einmal zum Landbau oder zu bürgerlichen Gewerben ward es ohne die größten Beschwerlichkeiten zugelassen; die einzige Lebensart, die man ihm übrig ließ, war und ist gegenwärtig noch der Handel, eine Beschäftigungsweise, welche den sittlichen Charakter der Menschen insgemein verdirbt, sie meistens eigennützig, geldgeizig und betrügerisch macht. Dieß sittliche Verderben konnte bei dem Volke der Juden um desto weniger ausbleiben, da jene Abgaben, die man von ihnen forderte, in allen Jahrhunderten so übermäßig groß und auf ehrliche Weise wirklich nicht zu erschwingen waren. Zu der Bedrückung fügte man endlich noch Spott und Verachtung hinzu; und selbst in unseren Tagen erlauben sich nicht nur gemeine Christen, sondern auch Männer, die gebildet heißen wollen, jeden Israeliten, wenn er nur eben nicht reich ist, mit sichtbarer Geringschätzigkeit zu behandeln. Ist es ein Wunder, wenn sich hiedurch alle Ehrbegierde bei dem ärmeren Theile verlor, bei Einigen sogar eine Art Niederträchtigkeit erzeugte? Ich schweige von den ganz unmenschlichen Verfolgungen, die das verhaßte Volk in Zeitpuncten, wo irgend ein außerordentlicher Zufall die Rache der Christen von Neuem aufgereizt hatte, erfuhr. So kam es denn, daß die Nation der Juden immer zu wenig Bildung und Aufklärung hatte, als daß sie die Vorurtheile ihrer Vorfahren ablegen, und von der Wahrheit des Christenthumes sich hätte überzeugen können; immer zu sehr mit drückenden Nahrungssorgen geplagt war, als daß sie den Geist zu Untersuchungen über die Religion hätte erheben sollen; daß sie stets einen zu tief gewurzelten Haß gegen die Christen empfand, als daß sie den Glauben derselben jemals mit Unparteilichkeit hätte beurtheilen mögen. Daß sie aber bei allem dem im Ganzen noch immer viel zu gewissenhaft ist, als daß sie aus~\RWSeitenw{19}\ bloßem Eigennutz, nur um ihre äußere Lage zu verbessern, das Christenthum annehmen wollte, das müssen wir ihr unter diesen Umständen noch zum Lobe nachsagen.
\item Was endlich die Erscheinung betrifft, daß sich von Zeit zu Zeit von dem großen Körper der katholischen Kirche einzelne bald mehr, bald minder beträchtliche Theile gewaltsam losrissen, und einer eigenen und neuen Lehre folgten, auch diese Erscheinung verliert alles Bedenkliche, sobald man die näheren Verhältnisse, und die veranlassenden Ursachen derselben kennen lernt. Die meisten Ketzereien der \RWbet{acht früheren Jahrhunderte} entstanden aus \RWbet{Irreleitung des Volkes durch einige einzelne Lehrer}, die von der rechtgläubigen Meinung bald darum abgewichen waren, weil sie aus Vorliebe für ein gewisses philosophisches System Behauptungen wagten, welche der Sittlichkeit nicht zuträglich waren, oder weil sie gewisse Schriftstellen unrichtig auslegten, oder weil sie aus Eitelkeit etwas Neues aufbringen wollten, \udgl\  So war die Ketzerei der \RWbet{Doketen}, die Christo einen nur scheinbaren Leib beilegten, nur daraus entstanden, weil sie nach ihrer Philosophie die Materie als etwas wesentlich Böses ansahen, und darum nicht begreifen konnten, wie Christus sich mit ihr anders als nur scheinbarer Weise hätte verbinden können. Auch an der Ketzerei der \RWbet{Arianer} und \RWbet{Semiarianer} scheinen die Philosophie und mißverstandene Stellen der h.\,Schrift den größten Antheil gehabt zu haben. Auf mißverstandene Schriftstellen gründeten sich größtentheils auch die Irrthümer der \RWbet{Chiliasten}, der \RWbet{Nazarener}, der \RWbet{Montanisten} \umA\ Die Trennungen dagegen, die in den späteren Zeiten seit dem \RWbet{zwölften Jahrhunderte} Statt fanden (\zB\ durch die \RWbet{Waldenser, Wiklefiten, Hussiten, Lutheraner, Calviner} \uA ), wurden wie früher schon jene der Albigenser, Katharer \udgl\  durch die unausstehlichen und immer noch unausstehlicher werdenden \RWbet{Mißbräuche} veranlaßt, die in der Kirche Gottes, besonders am römischen Hofe, eingerissen waren. Der Anblick dieser Mißbräuche hatte schon alle Gemüther erbittert, und jeder Rechtschaffene wünschte nichts sehnlicher als eine Abstellung derselben. Wie es aber der gewöhnliche Fehler der Menschen ist, daß sie in ihrer Heftigkeit leicht das rechte Maß verfehlen, und aus dem einen Aeußersten in das andere verfallen, so geschah es auch hier bei vielen Christen. \RWbet{Mit dem Mißbrauche verwarfen sie}~\RWSeitenw{20}\ \RWbet{zugleich den guten und nützlichen Gebrauch, mit dem Aberglauben zugleich die wahre Lehre}. Nur daher kam es, daß man die Lehren von der Unfehlbarkeit der Kirche in ihren allgemeinen Entscheidungen, von einem Oberhaupte der ganzen Christenheit zur Erhaltung der Einheit, von einem Mittelzustande zwischen ewiger Seligkeit und Verdammniß, von der Gemeinschaft geistiger Güter, von der Verehrung der Heiligen und Engel, die Beichtanstalt, das Sacrament der Ehe, und noch so manche andere Lehren und Einrichtungen verwarf, deren Vortrefflichkeit man nie verkannt haben würde, wenn nicht gerade damals der Abscheu vor ihrem Mißbrauche die Gemüther allzusehr eingenommen und verblendet hätte. Erwägen wir noch, daß Fürsten bei diesem Abfalle von der katholischen Kirche gewünschte Sicherung und Erweiterung ihrer landesherrlichen Rechte, Vermehrung ihrer Einkünfte und andere dergleichen Vortheile gewannen; daß der geistliche Stand Befreiung von gewissen, ihm längst schon lästig gewordenen Einschränkungen, namentlich vom Cölibatgebote erhielt; daß selbst die Laien nun ein viel ungebundeneres Leben zu führen ermächtiget wurden; daß sie nicht mehr zur Beichte, zum Fasten, und mehren anderen religiösen Uebungen angehalten wurden \umA : so werden wir uns nicht im Geringsten wundern, warum die neue Lehre so sehr um sich gegriffen habe; wie denn selbst Luther so aufrichtig war, zu gestehen, daß er, wenn er sein Augenmerk auf die Beweggründe richte, um derentwillen sein Anhang so schnell sich vermehre, unmöglich sich darüber freuen könne. Ueberhaupt verdient hier angemerkt zu werden, daß fast noch Niemand es gewagt habe, der Lehre der katholischen Kirche den Vorwurf zu machen, daß sie schon an sich selbst der Sittlichkeit der Menschen nachtheilig wäre; sondern nur Zweierlei hat man derselben vorgeworfen: \RWbet{sie wäre in dem Sinne, in welchem die unterste Classe des Volkes sie wirklich aufgefaßt hat, verderblich; und sie wäre auch in dem Sinne, in welchem die Gelehrten sie nehmen, mit der Vernunft im Widerspruche}. Das \RWbet{Erste} ist nun kein Vorwurf, der die \RWbet{Lehre} an sich, sondern höchstens einer, der unsere \RWbet{Lehrer} betrifft. Nur diese sind zu beschuldigen, wenn sich nachweisen läßt, daß~\RWSeitenw{21}\ viele Lehren der Kirche von einem großen Theile des Volkes unrichtig aufgefaßt werden. Das \RWbet{Zweite} ist ein Vorwurf, der schon deßhalb von keiner Wichtigkeit ist, weil eine Lehre, welche sittliche Zuträglichkeit hat, nicht zu verwerfen ist, selbst wenn wir finden, daß sie in ihrem \RWbet{buchstäblichen Sinne} auf einen Widerspruch führt. Denn hieraus folgt nur, daß sie den \RWbet{bildlichen}, \dh\ denjenigen Lehren beizuzählen sey, die uns den Gegenstand nicht, wie er an sich ist, darstellen, sondern, wie seine Vorstellung für uns am Zuträglichsten ist. Hiezu kommt, daß wir in einem jeden Falle, wo wir einer Partei, welche doch auch verständige und scharfsinnige Männer in ihrer Mitte zählt, von einer anderen den Vorwurf machen hören, daß sie etwas der Vernunft Widersprechendes glaube, im Voraus annehmen können, daß diese Beschuldigung ungerecht seyn dürfte, und daß dasjenige, was jene wirklich lehren, auf eine Art aufgefaßt werden könne, daß es der Vernunft nicht widerspricht.
\end{aufza}

\RWpar{9}{Zweite Behauptung}
Die Geschichte der Menschheit erweiset so deutlich, als eine Sache von dieser Art durch Geschichte nur immer erwiesen werden kann, das doppelte Factum,
\begin{aufzb}
\item \RWbet{daß unter allen auf Erden befindlichen Religionen, welche sich für geoffenbart ausgeben, keine einzige sey, die eine so große Anzahl wohl unterrichteter Personen aufweisen kann, die ihre Anhänglichkeit an sie auf eine so unzweideutige Art zu erkennen gegeben haben, als es im Christenthume geschah und noch fortwährend geschieht}.
\item \RWbet{Daß keine einzige dieser Religionen auch Bekenner aufzuweisen habe, die in der Tugend und Glückseligkeit so fortgeschritten sind und so laut eingestanden haben, daß sie diese Fortschritte nur ihrem Glauben verdanken, als dieses abermals im Christenthume der Fall ist}.~\RWSeitenw{22}
\end{aufzb}
Aus dieser doppelten Erfahrung ziehe ich die doppelte Schlußfolgerung:
\begin{aufzb}
\item \RWbet{daß keine andere Religion auf Erden die Prüfung, auch selbst des scharfsinnigsten Verstandes, so gut, als das Christenthum, bestehe};
\item \RWbet{daß keine andere Religion auf Erden der Tugend und Glückseligkeit der Menschen so zuträglich sey, als die christliche, namentlich die katholisch-christliche}.
\end{aufzb}

\RWpar{10}{Beweis des ersten oder historischen Theils dieser Behauptung}
\begin{aufza}
\item Daß unter allen positiven Religionen keine einzige eine so große Anzahl wohlunterrichteter Personen aufweisen könne, die ihre Anhänglichkeit an sie auf eine so unzweideutige Art zu erkennen gegeben hätten, als dieß im Christenthume geschah und noch geschieht, ist wohl leicht darzuthun.
\begin{aufzb}
\item Unter den Fortschritten, welche das menschliche Geschlecht in den drei Hinsichten der \RWbet{Weisheit}, der \RWbet{Tugend} und der \RWbet{Glückseligkeit} von einem Jahrhunderte zum anderen macht, läßt sich der Fortschritt in der Weisheit am allerwenigsten verkennen. Nicht bloß in allerlei Künsten und Wissenschaften (die eigentlich gar nicht in das Gebiet der Weisheit gehören), sondern auch in den Begriffen über Tugend und Glückseligkeit, \dh\ in demjenigen, was eigentlich zur Weisheit gehört, übertrifft unsere neuere Zeit (die Zeit nach Christo meine ich) die älteren offenbar. Auch daß insonderheit diejenigen Völker der neueren Zeit, bei welchen das Christenthum herrscht, (die europäischen) aufgeklärter und weiser sind, als die übrigen, auch dieses können wir ohne Besorgniß, daß uns bei diesem Urtheile etwa die Eigenliebe täusche, behaupten; denn diesen Vorzug gestehen uns ja die Bewohner der übrigen Welttheile selbst zu. Nun mag es sich mit der Entstehungsursache dieser größeren Aufklärung verhalten, wie es will, es mag die letztere sich zum Theile selbst als~\RWSeitenw{23}\ eine wohlthätige Wirkung des Christenthumes ansehen lassen oder nicht: so bleibt doch so viel wahr, \RWbet{das Christenthum sey die Religion weit aufgeklärterer, weiserer Zeiten und Völker, als jede andere Religion auf Erden}. Außer Zweifel ist auch, daß in der neueren Zeit, und besonders in christlichen Ländern, der Volksunterricht, namentlich in der Religion, bei Weitem besser bestellt sey, als in älterer christlicher Zeit, oder auch jetzt noch bei den nicht christlichen Völkern. Wurden doch Volksschulen fast allenthalben erst durch das Christenthum eingeführt! Kein Zweifel also, \RWbet{daß sich unter den Christen eine viel größere Anzahl von wohl unterrichteten Menschen finde, als irgendwo anders}.
\item Und wie ganz unzweideutig hat dieser wohlunterrichtete Theil der Christen, haben selbst unsere größten Gelehrten ihre Anhänglichkeit an diese Religion an den Tag gelegt! Auch die nicht christlichen Völker haben ihre Gelehrten; aber sind diese auch Anhänger der Religion des Volkes? Haben sie nicht oft laut genug erklärt, daß sie für ihre Person einer ganz anderen Meinung, als der des Volkes wären? Und wie oft, wenn sie dieß unterließen, mögen sie, nicht aus Ueberzeugung, sondern aus bloßer Schonung der Schwachen, oder aus Menschenfurcht, oder aus eigennütziger Absicht so gehandelt haben! Ganz anders ist es im Christenthume. Hier sind es die Gelehrten, die diese Religion fast überal nicht nur die ersten angenommen, sondern sie auch den Anderen geprediget, sie mündlich und schriftlich verbreitet und vertheidiget haben. Schriften zu Tausenden werden noch heut zu Tage von den Gelehrten alljährlich verfaßt, um die gute Sache des Christenthumes in ein helleres Licht zu setzen; man prediget und schreibt, beweiset und vertheidiget mit einer Wärme des Herzens, mit einem Eifer und Fleiße, der jeden, auch den geringsten Zweifel darüber, ob es aus inniger Ueberzeugung geschehe, vernichtet! Einer so allgemeinen Anhänglichkeit, auch von Seite der Weisen im Volke, hat sich nebst der mosaischen sonst keine andere Religion auf Erden zu erfreuen.~\RWSeitenw{24}
\end{aufzb}
\item Daß keine andere Religion Bekenner aufzuweisen habe, die in der Tugend und Glückseligkeit so viele Fortschritte gemacht, und so laut eingeständen, daß sie diese Fortschritte nur ihrem Glauben verdanken, als dieß im Christenthume der Fall ist, erweise ich so:
\item Daß keine andere Religion Bekenner aufzuweisen habe, die in der Tugend und Glückseligkeit so viele Fortschritte gemacht, und so laut eingeständen, daß sie diese Fortschritte nur ihrem Glauben verdanken, als dieß im Christenthume der Fall ist, erweise ich so:
\begin{aufzb}
\item Wenn man den sittlichen Charakter christlicher und anderer Völkerschaften ohne Parteilichkeit vergleicht, so fällt, im Ganzen genommen, diese Vergleichung doch deutlich genug zum Vortheile der Ersteren aus. Nur muß man, um dieß anschaulicher zu finden, Völker vergleichen, bei welchen die übrigen Umstände, besonders der Grad der Cultur, nicht allzu ungleich ist. Mehr Kraft der Selbstbeherrschung, weniger Grausamkeit, mehr Anerkennung des hohen Werthes der menschlichen Natur auch in der Person des Verachtetsten und Geringsten, mehr Sinn für Wohlthätigkeit, mehr Pflege der Armen, Kranken und Krüppelhaften, mehr Rücksichtsnahme auf das zukünftige Leben, mehr Sorge für die Ausbildung des Geistes und für den Unterricht auch selbst der untersten Classe des Volkes ist doch gewiß in dem Christenthume zu finden. Daher ist denn auch die Menge der Unglücklichen, welche ganz ohne Hülfe verschmachteten, die Menge derjenigen, welche alles Unterrichtes und aller menschlichen Bildung entbehrten, die Menge des Elendes unter den Christen geringer, als anderwärts.
\item Und wenn man die einzelnen Helden der Tugend, welche das Christenthum erzeugt hat, und jene, welche die übrigen Religionen aufzuweisen haben, mit einander vergleicht: so sieht man abermals, daß diese von jenen weit übertroffen werden. Man vergleiche nur \zB\ einen Paulus und einen Sokrates; gute Fürsten unter den Christen, und unter den Heiden; Feldherren, edle Frauen christlichen und heidnischen Glaubens, \usw
\item Endlich ist allgemein bekannt, daß alle guten Menschen unter uns Christen von jeher eingestanden, und noch jetzt eingestehen, daß es ihr Glaube sey, dem alles Gute, das sich an ihnen befindet, als Wirkung zugeschrieben werden müsse. Gestehen dieß wohl auch die Bekenner der übrigen Religionen, oder können sie das gestehen? Hätte ein~\RWSeitenw{25}\ Heide sagen können, daß es sein Glaube an seine Gottheiten, an seinen Jupiter, an seine Venus \usw\ sey, der ihn so tugendhaft gemacht habe?
\end{aufzb}
\end{aufza}

\RWpar{11}{Beweis des zweiten Theils, oder der Folgerung}
\begin{aufza}
\item Aus dieser Erfahrung habe ich zuerst gefolgert, \RWbet{daß keine andere Religion auf Erden die Prüfung, auch selbst des scharfsinnigsten Verstandes, so gut, als das Christenthum, bestehe}.
\begin{aufzb}
\item Schon die Bemerkung, daß das Christenthum die Religion der aufgeklärtesten Völker des Erdbodens sey, gereichet demselben zu einer gewiß nicht geringen Empfehlung; denn sie erzeugt die Vermuthung, daß die Lehre des Christenthumes vielleicht selbst mehr oder weniger zur Entstehung dieser höheren Aufklärung beigetragen habe. Aber wenn dieses auch nicht seyn sollte: so muß man doch vermuthen, daß eine Religion, die sich bei so aufgeklärten Völkern in Ehren erhält, in ihrem Lehrbegriffe nicht ungereimt seyn könne.
\item Wenn aber nicht nur die große Menge, sondern auch selbst die Gelehrten derselben von ganzem Herzen zugethan sind: so folgt hieraus offenbar, daß ihre Lehre nichts der Vernunft Widersprechendes enthalten könne; sondern die Prüfung, auch selbst des scharfsinnigsten Verstandes aushalten müsse.
\item Und finden wir eine so treue Anhänglichkeit der Gelehrten bei keiner anderen Religion: so schließen wir wohl mit Recht, daß nur die christliche diese Prüfung aushalten müsse.
\end{aufzb}
\item Ich habe ferner behauptet, \RWbet{daß keine andere Religion der Tugend und Glückseligkeit der Menschen so zuträglich seyn müsse, als die christliche}, namentlich die \RWbet{katholisch-christliche}. So nämlich können wir mit allem Rechte schließen, wenn wir bemerken, daß bei den christlichen Völkern mehr Sittlichkeit herrsche, als unter anderen Religionsverwandten, daß es der Menschen, die sich durch Tugend auszeichnen, hier mehrere gebe, daß sie auf~\RWSeitenw{26}\ einer höheren Stufe der Vollkommenheit stehen, sich dabei glücklich fühlen, und selbst gestehen, daß sie Dieß alles nur ihrer Religion verdanken.
\end{aufza}

\RWpar{12}{Einwürfe und ihre Widerlegung}

\begin{center}\RWbet{Einwürfe.}\end{center}

Wenn auch Einiges von dem so eben Gesagten auf das Christenthum überhaupt anwendbar wäre, so gilt es doch nicht von dem katholischen Christenthume.
\begin{aufza}
\item Gerade in jenen christlichen Ländern, welche sich von der katholischen Kirche getrennt haben, ist die Gelehrsamkeit und wahre Aufklärung zu einem weit höheren Grade gestiegen, als in den katholischen. Das widerlegt nicht nur die Behauptung, daß die katholische Religion der Glaube der aufgeklärtesten Völker des Erdbodens sey, sondern beweiset sogar, daß sie der Aufklärung hinderlich falle. So viele Gelehrte unter den Protestanten, die es doch aufrichtig mit der Wahrheit meinten, konnten sich gleichwohl nie von der Richtigkeit des katholischen Lehrsystemes überzeugen. Und schämen sich denn nicht selbst unter den Katholiken die Aufgeklärteren der Lehrsätze ihrer Kirche? suchen sie nicht ihnen durch allerlei gezwungene Auslegungen einen vernünftigeren Sinn zu unterlegen? und nähern sie sich auf diese Weise nicht je länger, je mehr den Grundsätzen des Protestantismus? Ja, was noch mehr sagen will, haben nicht Mehrere aus ihnen sowohl, als aus den Protestanten, die christliche Religion zuletzt ganz abgeschworen? Wie merkwürdig ist es ferner, daß weder \RWbet{Spinoza} noch \RWbet{Mendelssohn}, diese zwei eben so weisen als tugendhaften Männer, geborne Israeliten, welche das Christenthum gewiß ohne alles Vorurtheil geprüft, sich von der Wahrheit desselben nie überzeugen konnten! Müssen wir also nicht vermuthen, daß alle diejenigen, die als Vertheidiger dieser Religion auftraten, entweder nicht aufrichtig ihre Ueberzeugung ausgesprochen, oder daß sie gar nicht fähig und befugt waren, über den Werth derselben zu entscheiden?
\item Und eben so wenig, wie die Lehre des Katholicismus die Prüfung des aufgeklärten Verstandes aushält, bewei\RWSeitenw{27}set sie sich auch als ein wirksames Mittel zur Beförderung der Tugend und Glückseligkeit unter den Menschen. Die rohen und unschuldigen Völkerschaften, welche den christlichen Glauben angenommen haben, sind durch ihn wahrlich nicht besser und glücklicher, sondern nur lasterhaft und unglücklich geworden. Ausschweifungen, Sünden und Laster, von welchen sie vorhin nichts gewußt, lernten sie durch das Christenthum kennen. Die Intoleranz, dieß Ungeheuer, diese Pest der Welt, ist (wie \RWbet{Voltaire} anmerkt) eine Tochter des Christenthumes; und die Heiligen der katholischen Kirche, was waren sie, näher betrachtet, Anderes, als fromme Müßiggänger und Schwärmer? Wenn ihre eingebildeten Tugenden, jene ewige Enthaltsamkeit, jene Selbstpeinigungen \usw\ allgemein angenommen werden sollten: müßte das ganze menschliche Geschlecht zu Grunde gehen! -- Die akatholischen Parteien, weil sie vom Christenthume nur wenig mehr, als die Lehrsätze der natürlichen Religion beibehalten haben, zeichnen sich eben deßhalb auch vor den Katholiken in den verschiedenen Tugenden der Geselligkeit, in ihrer Betriebsamkeit, und eben deßhalb auch in ihrem Wohlstande, in dem Genusse des häuslichen Glückes \usw\ sehr vortheilhaft aus. Können wir hieraus nicht entnehmen, wie noch viel mehr das menschliche Geschlecht dadurch gewinnen würde, wenn das Christenthum einmal ganz ausgerottet wäre?
\end{aufza}

\begin{center}\RWbet{Erwiederung.}\end{center}

Ich gebe zu, daß Mehreres von demjenigen, was \RWparnr{10} gesagt worden ist, nicht nur von der katholischen, sondern auch von den übrigen christlichen Religionen, ja Einiges sogar von einer oder der anderen aus diesen Letzteren in einem noch höheren Grade, als von der katholischen gelte; allein ich läugne, daß hieraus irgend etwas dem Katholicismus wesentlich Nachtheiliges gefolgert werden könne.
\begin{aufza}
\item So gebe ich denn
\begin{aufzb}
\item zu, daß in den akatholischen Ländern Europa's gerade gegenwärtig etwas mehr Aufklärung und Gelehrsamkeit herrsche, als in den katholischen; ich will auch annehmen, daß sich sogar mehr Tugend und Glückseligkeit in jenen Ländern finde; doch folgt hieraus noch nicht, daß der~\RWSeitenw{28}\ Lehrbegriff der katholischen Kirche der Tugend und Glückseligkeit der Menschen \RWbet{minder zuträglich} sey, als irgend einer der übrigen christlichen Religionen; denn der religiöse Lehrbegriff ist es gar nicht allein, sondern es sind noch viele andere Umstände, welche den Grad der Tugend und Glückseligkeit bei einem Volke bestimmen. Und in diesen anderen Umständen könnte wohl der Grund jenes Vorzuges liegen. Vornehmlich muß ich hier erinnern, daß in den katholischen Ländern eine beträchtliche Menge von Vorurtheilen herrsche, die zu dem Inhalte unserer Religion auf keine Weise gezählt werden dürfen, weil sie, obwohl sehr ausgebreitet, doch immer nicht allgemein sind, und nicht ohne Widerspruch von Seite der Gebildeten im Volke angenommen werden. Diese in unserer Religion bloß als zufällig zu betrachtenden Vorurtheile nun sind es, welche das Fortschreiten in der Vollkommenheit mächtiger hindern, als alle Irrthümer, die wir den akatholischen Religionen darum als wesentlich vorwerfen müssen, weil sie bei ihnen mit völliger Allgemeinheit, und mit Zustimmung auch der Gelehrten selbst, behauptet werden. Daß aber in den katholischen Ländern mehr Vorurtheile herrschen, als in den akatholischen, diese Erscheinung selbst darf Niemand Wunder nehmen; denn weil die Lehre der katholischen Kirche ausführlicher, als die der übrigen Parteien ist: so bietet sie auch mehr Veranlassungen zum Aberglauben für den ungebildeten Theil des Volkes dar; Veranlassungen, welche die Protestanten dadurch entfernt, daß sie die Lehren selbst, die diese Anlässe darbieten können, verwarfen. Dieß Mittel hilft nun zwar für die Gegenwart; wenn aber das Volk einst eine höhere Stufe der Bildung erstiegen hat, oder für denjenigen Theil desselben, der sich bereits auf dieser Stufe befindet, ist es kein Vortheil, sondern ein wahrer Verlust zu nennen, daß man bloß eines möglichen Mißbrauches wegen auch jeden guten Gebrauch der Sache aufgegeben. Und so kann man denn sagen, daß der katholische Lehrbegriff, wenn gleich für ungebildete Menschen durch einen möglichen Mißbrauch in einzelnen Stücken gefährlich, für den Gebildeten gewiß die vollkommnere Religion sey. Daher~\RWSeitenw{29}\ kommt es denn auch, daß man die protestantischen Gottesgelehrten, wenn sie den Katholicismus bestreiten wollen, durchaus nur reden hört, nicht von den Lehren, wie sie die Kirche vorträgt, sondern wie sie das Volk sich vorstellt; nicht von demjenigen Gebrauche derselben, der an sich möglich wäre, sondern von jenem Mißbrauche, der in der Wirklichkeit Statt findet. Ein Benehmen, an welchem derjenige, der einem solchen Streite als bloßer Zuschauer beiwohnet, ohne die inneren Gründe für oder wider die verhandelten Lehren zu kennen, schon allein schließen kann, daß die katholische Lehre an sich selbst untadelhaft seyn müsse.
\item Der Umstand, daß so viele akatholische Gelehrte sich von der Wahrheit unseres Glaubens nie überzeugen konnten, ist so befremdend nicht. Der äußere Anschein nämlich ist freilich wider den Katholicismus: \RWbet{die Vorsteher unserer Kirche dulden der Mißbräuche und Vorurtheile noch immer so viele} (ja leider!), und die Protestanten machen sich vollends die übertriebensten Vorstellungen von diesen Mißbräuchen und Vorurtheilen; sie wähnen, bei uns sey es noch immer, wie es im fünfzehnten Jahrhunderte, und selbst damals \RWbet{nicht} war; sie wähnen, wir beten die Heiligen an, erwarten den Sündenerlaß ohne alle Besserung, glauben, den Himmel durch fremde Verdienste erkaufen zu können \usw\ Bei diesen Vorstellungen fällt es dem Wackersten der Protestanten nicht einmal ein, den wahren Glauben bei uns zu suchen. Hiezu gesellt sich die stete Besorgniß der Protestanten, was sich in unserer Kirche etwa gebessert hat, dürfte sich wieder verschlimmern, sobald sich die Macht des Papstthums und der Geistlichkeit durch ihren Uebertritt vergrößert haben würde. Endlich mag wohl auch die Scheu vor lästigen Verbindlichkeiten, die Furcht vor zeitlichem Verluste oder vor Spott und Verachtung der besseren Erkenntniß der Wahrheit bei vielen Tausenden gar sehr im Wege stehen. Und in dieser Hinsicht kann man in Wahrheit sagen, \RWbet{der Protestantismus übe, ohne alle Anwendung äußerer Mittel, bloß durch die innere Beschaffenheit seiner Lehren}~\RWSeitenw{30}\ \RWbet{eine fortwährende Proselytenmacherei in Hinsicht auf uns Katholiken aus}. Merkwürdig ist es daher, und ein Beweis der inneren Güte des Katholicismus, daß trotz dieser Umstände gleichwohl von Zeit zu Zeit sehr aufgeklärte und rechtschaffen denkende Protestanten zu dem katholischen Systeme übertraten; \RWbet{Graf Friedrich Leopold von Stolberg} ist nicht der Einzige, der uns in dieser Hinsicht zur Ehre gereicht.
\item Wahr ist es, leider! daß sich ein großer Theil unserer jetzt lebenden katholischen Gelehrten seines eigenen Glaubens schämt, und ihn auf alle nur mögliche Weise der protestantischen, oder wohl gar der bloß natürlichen Religion näher zu bringen sucht. Dieses ist die Wirkung einer theils schwachsinnigen, theils eitlen Nachahmungssucht. Im protestantischen Deutschlande wird ungleich mehr, als in katholischen Ländern geschrieben. Die Folge ist, daß unsere Jugend, die sich den Wissenschaften widmet, sich größtentheils nur aus Werken unterrichtet, welche von Protestanten verfaßt sind; wodurch sie denn frühzeitig schon eine sehr hohe Meinung von der Gelehrsamkeit und Geistesüberlegenheit eben dieser Protestanten annimmt, und darum auch geneigt wird, Alles, was diese sagen, vernünftiger zu finden. Wer vollends als Schriftsteller auftritt, und eine günstige Beurtheilung in den gelehrten Zeitschriften, die größtentheils nur von Protestanten besorgt werden, zu erlangen wünscht, wie fühlt er sich versucht, selbst wenn er nicht daran glaubt, Gesinnungen zu äußern, die eine gewisse Hinneigung zum Protestantismus verrathen, da er auf diese Art am Sichersten erwarten kann, beifällig angezeigt, und mit Lobpreisungen überschüttet zu werden!
\item Wenn aber einige katholische Gelehrte dem katholischen Glauben ganz abgeschworen, und ihn mit dem Protestantismus vertauschet haben: so ist das meistentheils aus sehr weltlichen Absichten geschehen, oder es sind auf jeden Fall Männer, von welchen die Protestanten selbst eingestehen müssen, daß sie auf ihren Uebertritt nicht eben stolz seyn können, weil ihnen offenbar ein nüchternes Urtheil ermangelt.~\RWSeitenw{31}
\item Was aber diejenigen Gelehrten von katholischer sowohl, als protestantischer Seite betrifft, welche das ganze Christenthum verworfen haben: so sind es (mit seltener Ausnahme) Männer, die einen lasterhaften Lebenswandel führten, und die mit der Verwerfung des Christenthumes sich eines sehr lästigen Zaumes entlediget hatten, oder die sonst durch eine ungemessene Begierde, sich auszuzeichnen, oder durch Stolz, Rechthaberei, Gewinnsucht oder andere dergleichen Leidenschaften zu diesem Schritte verleitet wurden. So war es der Fall bei dem berüchtigten Patriarchen des Unglaubens \RWbet{Voltaire} (dessen eigene Schriften die Schlechtigkeit seines Charakters hinlänglich beurkunden), bei \RWbet{la Mettrie, Marquis d'Argens, Morgan} (der das ausschweifendste Leben geführt), bei \RWbet{Peter Bayle, Karl Bonnet, Diderot} (der seine eigene Tochter nach seinen sauberen Grundsätzen nicht erzogen wissen wollte), bei \RWbet{Thomas Hobbes} (der sich vor dem Tode so fürchtete, daß Niemand in seiner Gegenwart vom Tode sprechen durfte), bei \RWbet{Helvetius, Marmontel} (die ihre Irrthümer hintenher selbst widerriefen), \RWbet{Tindal, Vanini} \uA , deren Lebensgeschichte den besten Aufschluß zu ihrem Unglauben, und eine schon für sich hinreichende Widerlegung ihrer Schriften darbeut. -- Wie wenig diese Personen von dem, was sie behaupteten, oft selbst überzeugt waren, beweisen die ewigen Widersprüche, in die sie sich verstrickten, die unendlichen Verdrehungen, welche sie sich erlaubten, die schnellen Bekehrungen und Widerrufe, zu welchen sie bei einer herannahenden Todesgefahr, oder bei einem plötzlichen Unfalle ihre Zuflucht nahmen, \usw\ Bei jenen wenigen Gelehrten, welche das Christenthum weder aus bösem Gewissen, noch aus bloßer Eitelkeit verwarfen, sind gewisse Irrthümer, auf die sie durch ihre zu weit getriebenen Nachforschungen auf dem Gebiete der speculativen Philosophie geriethen, und denen sie unglücklicher Weise mehr, als den Aussprüchen eines gesunden Menschenverstandes, vertrauten, die Ursache ihres Abfalles geworden. Dieses zeigt sich, weil ihre Gründe bloß speculativer Art waren, \zB\ weil eine Offenbarung überhaupt unmöglich, oder wenigstens der~\RWSeitenw{32}\ Gottheit unanständig wäre, oder weil keine Wunder möglich seyen, oder weil es keine übervernünftige Wahrheiten geben könne, \udgl\ 
\item Freilich ist nicht zu läugnen, daß an dem Eifer, den ein oder der andere Vertheidiger des Christenthumes für diese Religion an den Tag gelegt hat, gewisse Vorurtheile der Jugend, die Liebe zur Bequemlichkeit und andere dergleichen unreine Triebfedern, einen sehr großen Antheil gehabt haben mögen; aber wer könnte so ungerecht seyn, zu behaupten, dieß sey der Fall bei Allen gewesen? Einmal die ersten gelehrten Vertheidiger des Christenthumes waren geborne Juden und Heiden gewesen; sie also hatte kein Vorurtheil der Jugend geblendet; sie zogen auch nicht den geringsten Vortheil davon, daß sie zu dieser Religion übertraten, und dadurch, daß sie die Vertheidigung derselben übernahmen, setzten sie sich den härtesten Verfolgungen aus. Hieher gehören der heil.\ \RWbet{Paulus}, der ein gelehrter Rabbine war, und dessen Uebertritt zum Christenthume der Engländer \RWbet{Lyttelton} so wichtig gefunden, daß er in seiner Betrachtung einen eigenen Beweis für die Wahrheit des Christenthumes gefunden; der \RWbet{heil.\ Lucas}, der ein griechischer Arzt gewesen war; \RWbet{Justinus der Märtyrer}, der, ein heidnischer Weltweise, erst im 30sten Jahre seines Alters zum Christenthume übertrat, und nur darum übertrat, weil ihn, wie er selbst gesteht, Alles, was er in den verschiedenen anderen Religionen, mitunter selbst in den berühmtesten Mysterien und in den Schulen der Weltweisen fand, nicht befriedigen konnte; \RWbet{Athenagoras, Tatian}, \uA , die alle früher Heiden gewesen, dann als Vertheidiger des Christenthumes durch Schriften auftraten.
\item Sollte das Ansehen so Vieler nicht einen \RWbet{Spinoza} oder \RWbet{Mendelssohn} aufwiegen können, gesetzt, daß wir auch gar nicht anzugeben wüßten, was diese beiden Männer von der Annahme des Christenthumes abgehalten habe?
\end{aufzb}
\item Auch wenn es wahr wäre,
\begin{aufzb}
\item daß einige rohe und unschuldige Völkerschaften, welchen die Europäer das Christenthum gebracht, in der Folge~\RWSeitenw{33}\ nur lasterhafter und unglücklicher geworden wären, als sie es ehedem waren: so wäre hieran doch nicht die Bekanntschaft mit dem Christenthume, sondern der Umgang mit so verdorbenen Menschen Schuld, als es diejenigen Europäer fast durchgängig waren, die sich aus bloßer Bereicherungssucht, oder weil sie in ihrem eigenen Lande kein Glück mehr zu machen hatten, zu jenen Völkern begaben.
\item Die \RWbet{Intoleranz} ist keineswegs ein Fehler, dessen sich nur die Christen schuldig gemacht. Auch die \RWbet{Aegyptier} bekriegten sich unaufhörlich unter einander, weil ein Theil des Volkes dieses, ein anderer Theil jenes Thier für heilig hielt, während der andere dasselbe auszurotten suchte. \RWbet{Diodor} erzählt uns als Augenzeuge, daß ein Römer, der aus Versehen eine Katze in Aegypten umgebracht hatte, von dem wüthenden Pöbel sogleich getödtet worden sey \udgl\  Die \RWbet{Perser} zerschlugen alle Statuen, welche sie in den Tempeln der Aegyptier, oder der Griechen fanden. Bei den \RWbet{Atheniensern} mußte ein jeder Bürger einen Eid schwören, daß er die Landesreligion bekennen, und gegen Jedermann vertheidigen wolle, und \RWbet{Protagoras} ward des Landes verwiesen, weil er das Daseyn der Götter bezweifelt hatte. Der edle \RWbet{Anaxagoras} wurde zum Tode verurtheilt, weil er die Sonne, welche die Athenienser als eine Gottheit verehrten und anbeteten, einen Feuerball nannte. Nicht besser erging es bekannter Maßen dem weisen \RWbet{Sokrates}, dessen Schüler \RWbet{Plato} gleichwohl in seinem besten Staate noch das Gesetz gegeben wissen wollte, daß Gottesläugner zuvörderst zwar \RWbet{belehrt}; dann aber \RWbet{bestraft}, selbst \RWbet{hingerichtet} werden sollten. Bekannt sind auch die vielen heiligen Kriege, welche die Griechen oft unter einander führten. Und bei den Römern lautete eines der Tafelgesetze: \RWlat{Deos peregrinos ne colunto; separatim nemo habessit Deos, neque novos; sed nec advenas, nisi publice adscitos, privatim colunto}, welches \RWbet{Cicero} für Eines der weisesten Gesetze erklärt. \RWbet{Mäcenas} rieth dem Kaiser Augustus: Hasse alle diejenigen, die Neuerungen in der Religion machen, und strafe~\RWSeitenw{34}\ sie. \RWbet{Tiberius} vertrieb Alle, die dem ägyptischen, jüdischen oder überhaupt einem fremden Gottesdienste ergeben waren; 400 dieser Unglücklichen wurden nach Sardinien geschickt, gegen die Räuber zu fechten, und \RWlat{si ob gravitatem coeli interirent, vile damnum}, sagt \RWbet{Tacitus} (\RWlat{Annal.\ 2.\ 85.})\RWlit{}{Tacitus1}. Selbst der gütige Kaiser \RWbet{Trajan} befahl dem \RWbet{Plinius}, die Christen hinrichten zu lassen. (\RWlat{Plin.\ ep.\ lib.\,10.\ ep.\,102.})\RWlit{}{Plinius1} Waren die Christen vielleicht zuweilen noch heftiger in der Verfechtung ihrer religiösen Meinungen: so kam es nur daher, weil sie den Werth und die Wichtigkeit derselben lebhafter fühlten. Das Christenthum selbst aber hat gewiß nicht den mindesten Antheil an dieser Intoleranz; denn es verbietet sie ausdrücklich.
\item Es ist die schändlichste Verläumdung, von den \RWbet{Heiligen der katholischen Kirche im Allgemeinen} zu sagen, daß sie nur fromme Müßiggänger und Schwärmer gewesen. Die edelsten, die vollkommensten Menschen, welche die Erde gesehen, sind in dem Verzeichnisse dieser Heiligen zu finden. Daß aber Einige, ja, wenn man will, selbst \RWbet{Viele} es in gewissen Stücken übertrieben, daß sie \zB\ einen zu hohen Werth auf gewisse Abtödtungen des Leibes, auf Beten, Fasten \udgl\  legten, ist allerdings wahr; aber diese Erscheinung ist aus dem Zeitalter, in welchem sie lebten, leicht zu erklären. Weil nämlich andere Menschen, die Heiden, oder auch selbst die große Menge der Christen, einen zu hohen Werth auf sinnliche Vergnügungen legten: so verfielen Jene auf das entgegengesetzte Aeußerste; und die göttliche Vorsehung scheint diese Verirrung der Letzteren nicht ohne Absicht zugelassen zu haben. Denn eben nur dadurch, daß ihr frommer Eifer so weit ging, erregten sie Aufmerksamkeit in einem Zeitalter, das eine sich allenthalben nur in den Grenzen der Mäßigung haltende Tugend gar nicht beachtet und gewürdiget hätte. Wenn man aber glaubt, den Wandel dieser Heiligen schon deßhalb tadeln zu dürfen, weil, wenn ein Jeder so leben wollte, wie sie, die Menschheit aussterben müßte: so irrt man sich; denn der bloße Umstand, daß eine gewisse Art zu handeln, nicht allge\RWSeitenw{35}mein nachgeahmt werden dürfte, ist noch kein hinreichender Grund, sie zu verwerfen; oder sind die so mannigfaltigen Gewerbe und Lebensarten unter den Menschen nicht sehr nützlich, obgleich es gewiß ist, daß die menschliche Gesellschaft nicht bestehen könnte, wenn alle sich nur einer und derselben Beschäftigung widmen wollten?
\item Woher es komme, daß sich bei einigen akatholischen Parteien mehr Tugend und Glückseligkeit vorfinde, ist schon erklärt worden. Man denke, um nur ein einziges Beispiel zu geben, an die katholische Lehre von Buße. Gewiß sind die Forderungen, die unsere Kirche hier aufstellt, strenger und heilsamer, als was der protestantische Glaube verlangt; und wenn sie von Allen gehörig gekannt und beobachtet würden: dann müßte unter uns Katholiken ohne Zweifel, schon um dieser einzigen Lehre willen, mehr Tugend anzutreffen seyn, als unter den Protestanten.\RWfootnote{Ein sehr aufgeklärter Protestant sprach der katholischen Religion, als er gelegenheitlich einen Unterricht über die Bußanstalt mit angehört hatte, das wohl verdiente Lob: Wer diese Lehre gehört, und nie unter Katholiken gelebt hätte, der müßte glauben, sie seyen lauter Heilige.} Allein aus Mangel an gehörigem Unterrichte geschah es, daß sich bei Vielen unter uns der äußerst verderbliche Gedanke festgesetzt hat, daß schon durch das \RWbet{bloße Beichten} die Vergebung der Sünden erwirkt werden könnte; und bei solchen nun freilich schadet die Beichtanstalt mehr, als sie nützt. Aber sollen wir sie um dessenwillen abschaffen, oder vielmehr nur für die Verbesserung unseres Volksunterrichtes sorgen? Und eben so ist es mit allen übrigen Lehren, welche die Protestanten verwarfen.
\item Ist es aber schon falsch, zu sagen, daß die Protestanten durch die Verwerfung dieser Lehren gewannen, da sie höchstens nur durch die Entfernung gewisser, mit diesen Lehren nur \RWbet{zufällig} verbundener Vorurtheile gewannen: so ist es die frecheste Gotteslästerung, zu sagen, es wäre ein Gewinn für die Menschheit, wenn man das ganze Christenthum vertilgte. Wäre der Glaube des Christenthumes in irgend einer der Gestalten, welche er~\RWSeitenw{36}\ im Verlaufe der Zeit angenommen hat, verderblich: wie könnten die besten und edelsten Mitglieder aller christlichen Confessionen mit Einem Munde bekennen, daß sie nur ihrem Glauben an das Christenthum das Gute, das sich an ihnen befindet, zu verdanken hätten? Kann sich der Mensch so irren in der Beurtheilung dessen, was auf sein eigenes Herz wohlthätig oder verderblich einwirke?~\RWSeitenw{37}
\end{aufzb}
\end{aufza}



\RWch{Zweites Hauptstück.\\ Ueber die Natur der historischen Erkenntniß, besonders in Hinsicht auf Wunder.}
\RWpar{13}{Inhalt und Zweck dieses Hauptstückes}
\begin{aufza}
\item Wer Alles, was in dem nächstvorhergehenden Hauptstücke zur Empfehlung der Lehre des Katholicismus gesagt worden ist, gehörig überlegt hat; wird es kaum unbillig finden, daß ich die Prüfung der verschiedenen Religionen gerade mit der katholisch-christlichen anfangen will.
\item Allein, wie ich bereits gesagt, so ist es meine Absicht, zuerst zu untersuchen, ob diese Religion dasjenige Kennzeichen einer göttlichen Offenbarung habe, welches in der Beglaubigung durch Wunder besteht.
\item Nun trifft es sich aber, daß gerade die Wunder, welche für die Bestätigung des Christenthums und auch in anderer Rücksicht die allerwichtigsten sind, sich gleich bei der Entstehung dieser Religion, \dh\ vor einem Zeitraume von etwa achtzehnhundert Jahren zugetragen haben. Daher kann denn das wirkliche Geschehenseyn derselben nicht anders, als durch geschichtliche Untersuchungen, und zwar solche, die in ein hohes Alterthum zurückgehen, dargethan werden.
\item In neuerer Zeit hat man aber auf verschiedene Art gesucht, den historischen Glauben, besonders in Hinsicht auf Wunder, wankend zu machen, und behauptet, daß Erzählungen von Wundern, vornehmlich solchen, die sich vor vielen Jahrhunderten ereignet haben, nie strenge erweislich wären. Dergleichen Behauptungen haben \zB\ \RWbet{Joh.~Grayg, Dav.~Hume, Bolingbroke, J.~J.~Rousseau, C.~F.~Bahrdt, Imm.~Kant} \umA\  vorgetragen.~\RWSeitenw{38}
\item Um nun ihren Einwürfen zu begegnen, wird es nothwendig seyn, Einiges über die Natur der historischen Erkenntnißart, und über den Grad ihrer Gewißheit, besonders in Hinsicht auf Wunder, vorauszuschicken. Dieses soll denn in dem gegenwärtigen Hauptstücke geschehen.
\end{aufza}

\RWpar{14}{Was ein historisches Urtheil in den verschiedenen Bedeutungen dieses Wortes sey?}
Man nimmt den Ausdruck: \RWbet{historische Urtheile} oder \RWbet{historische Erkenntnisse}, in sehr verschiedener, bald engerer, bald weiterer Bedeutung.
\begin{aufza}
\item In der \RWbet{weitesten} wird dieser Ausdruck genommen, wenn man die Summe aller menschlichen Erkenntnisse in zwei Classen: \RWbet{philosophische} und \RWbet{historische} abtheilt. In diesem Gegensatze nämlich versteht man unter den philosophischen Urtheilen bloß solche, die man sonst auch Urtheile \RWlat{a priori} nennt, \dh\ Urtheile, die aus bloßen Begriffen bestehen. Zu den \RWbet{historischen} Urtheilen zählt man denn also hier alle Urtheile, die nebst Begriffen auch \RWbet{Anschauungen} enthalten. Diese sind wieder von doppelter Art:
\begin{aufzb}
\item \RWbet{unmittelbare Wahrnehmungsurtheile}, die eine bloße Wahrnehmung aussagen, ohne über die Ursache derselben etwas entscheiden zu wollen; und
\item \RWbet{Erfahrungsurtheile}, worunter man hier alle nicht apriorischen Urtheile verstehet, die aus unmittelbaren Wahrnehmungsurtheilen auf irgend eine Art abgeleitet worden sind. Diese Erfahrungsurtheile sind abermals von sehr verschiedener Gattung. Sie können (um hier nur diesen einen Unterschied zu berühren) bald den Begriff der Gegenwart, bald den der Vergangenheit, bald jenen der Zukunft enthalten; \dh\ bald einen gegenwärtigen Zustand gewisser Dinge, bald einen vergangenen, bald einen künftigen aussagen. Z.\,B.\ das, was ich hier sehe, ist Cajus; derjenige, der diese Stadt erbauen ließ, war Alexander; Morgen wird es regnen.
\end{aufzb}
\item In einer \RWbet{engeren} Bedeutung versteht man unter den historischen Urtheilen \RWbet{bloß solche Erfahrungsurtheile, die den Begriff einer Vergangenheit}~\RWSeitenw{39}\ \RWbet{enthalten}, oder den Zustand gewisser Dinge in der Vergangenheit erzählen, \zB\ vor 1800 Jahren ereignete sich zur Zeit des Vollmondes eine Sonnenfinsterniß in Judäa.
\item In einer \RWbet{noch engeren}, und zwar der \RWbet{gewöhnlichsten} Bedeutung versteht man unter historischen Urtheilen nur solche Urtheile über die Vergangenheit, welche zugleich \RWbet{die Schicksale des menschlichen Geschlechtes} betreffen, und also in die Geschichte der Menschheit gehören.
\item Sieht man auf den Erkenntnißgrund, aus welchem die historischen Urtheile der letzten Art abgeleitet sind: so findet sich noch ein merkwürdiger Unterschied zwischen ihnen. Die Erkenntniß einiger nämlich gründet sich auf die \RWbet{Aussage} oder das \RWbet{Zeugniß von Menschen}; während dieß bei anderen nicht der Fall ist. Jene verdienen eine besondere Aufmerksamkeit in der Geschichte, und ihre Wahrheit muß auf eine eigene Art geprüft werden. Man könnte sie deßhalb historische Urtheile im \RWbet{engsten} Sinne nennen. So ist \zB\ das Urtheil, daß Kaiser Constantin einmal ein Kreuz am Himmel mit der Inschrift: \anf{Durch dieses wirst du siegen}, gesehen habe, ein historisches Urtheil in diesem engsten Sinne; denn es kann nicht anders, als auf das Zeugniß gewisser Geschichtschreiber (namentlich des \RWbet{Eusebius} und \RWbet{Lactantius}) angenommen werden. Dagegen, wenn Jemand die Gegend von Palmyra bereiset, und aus dem Anblicke der vielen Ruinen, die er hier antrifft, den Schluß zieht, daß vor Jahrhunderten hier eine blühende Stadt gestanden, daß die Bewohner derselben vielen richtigen Kunstsinn gehabt, \usw ; oder wenn Jemand auf einem Felde eine Menge von Menschenknochen nebst Schwertern und Spießen findet, und daraus schließt, hier müsse ehemals eine Schlacht geliefert worden seyn: so gründen sich diese Urtheile \RWbet{auf kein Zeugniß} eines Andern, und wären sonach von den historischen Urtheilen der ersten Art zu unterscheiden.
\end{aufza}

\RWpar{15}{Ueber den Begriff der Wahrscheinlichkeit und die verschiedenen Arten derselben}
\begin{aufza}
\item Alle historischen Urtheile, die es in dieses Wortes engerer Bedeutung sind, haben, wie dieß bereits bemerkt wurde,~\RWSeitenw{40}\ ihrer Natur nach \RWbet{bloße Wahrscheinlichkeit}. Daher dürfte es nicht am unrechten Orte seyn, hier diesen wichtigen Begriff mit seinen Unterarten genauer zu bestimmen; auch einige der merkwürdigsten Lehrsätze über die Wahrscheinlichkeit, besonders solche, deren Kenntniß zu einer gründlichen Widerlegung gewisser Einwürfe nothwendig ist, in Kürze mitzutheilen.
\item Ich werde mich aber hier in keine eigentliche \RWbet{Erklärung} oder Zerlegung des Begriffes der Wahrscheinlichkeit in seine einfachen Bestandtheile einlassen; sondern mich mit einer bloßen \RWbet{Verständigung} über ihn begnügen.
\item Wenn wir von irgend einem Satze keinen hinlänglichen Grund haben, weder ihn zu bejahen, noch auch ihn zu verneinen, \dh\ wenn das, was darin ausgesagt wird, problematische Möglichkeit für uns hat: so können wir, nach Beschaffenheit der Umstände, mit einer bald größeren, bald geringeren Wahrscheinlichkeit, bald das Urtheil, daß der Satz wahr, bald jenes, daß er falsch sey, fällen.
\item Wenn Jemand nicht recht begriffe, was hier das Wort \RWbet{Wahrscheinlichkeit} bedeutet: so würde ich etwa auf folgende Art versuchen, durch den Gebrauch einiger gleichbedeutender Ausdrücke, und durch die Zusammensetzung der Rede verständlich zu machen, welchen Begriff es bezeichne. Der Grad von Zuversicht, mit welchem wir etwas erwarten, oder vermuthen, oder annehmen können, ist der Grad der Wahrscheinlichkeit, welchen es hat. Wenn der \RWbet{vollständige Grund}, aus welchem die Wahrheit eines Satzes gefolgert werden könnte, \dh\ die Summe aller Prämissen, welche zu seiner Ableitung nothwendig sind, nicht da ist: so kann doch \RWbet{ein Theil dieses Grundes}, \dh\ es kann doch eine, oder es können etliche dieser Prämissen da seyn. Wir pflegen dergleichen Prämissen, die für diesen Satz sprechenden, Gründe in dieses Wortes \RWbet{vielfacher} Bedeutung zu nennen. Gründe in der vielfachen Bedeutung sind also eigentlich bloße \RWbet{Theilgründe}, die \RWbet{alle zusammen} genommen erst den \RWbet{vollständigen Grund} einer Wahrheit bilden. Gründe in dieser Bedeutung kann es zu gleicher Zeit sowohl für, als auch wider die Wahrheit eines und desselben Satzes geben; denn allerdings kann es neben mehreren Prämissen, aus wel\RWSeitenw{41}chen, wenn erst noch einige andere hinzukämen, ein gewisser Satz sich ableiten ließe, auch wieder andere Prämissen geben, aus welchen sich, wenn erst noch einige andere hinzukämen, das gerade Gegentheil des Satzes ableiten ließe. Die Gründe nun, die sowohl für als wider die Wahrheit eines Satzes bekannt sind, zusammengenommen, und gegen einander abgewogen, bestimmen die \RWbet{Wahrscheinlichkeit} desselben.
\item Die Wahrscheinlichkeit eines Urtheiles ist eine derjenigen Beschaffenheiten, die einen \RWbet{Grad} oder eine \RWbet{Größe} haben; und zwar wird diese Größe durch einen \RWbet{Bruch} gemessen. Um nämlich zu bestimmen, mit welchem Grade der Wahrscheinlichkeit angenommen werden kann, daß unter mehreren nicht erweislich falschen, \dh\ problematisch möglichen Antworten auf eine Frage die Antwort $A$ die richtige sey, müssen wir zählen, wie viele nicht erweislich falsche Fälle $\alpha$, $\beta$, $\gamma$, $\delta$, \usf\ es in Betreff der Antwort auf die Frage gibt, die alle, weil ein gleicher Theilgrund für sie spricht, auch eine gleiche Wahrscheinlichkeit haben. Die Summe dieser Fälle gibt uns den \RWbet{Nenner} des Bruches. Dann müssen wir zählen, wie viele dieser Fälle es gibt, bei deren Annahme die Antwort $A$ zum Vorscheine kommt. Die Summe dieser Fälle gibt uns den \RWbet{Zähler} des Bruches. Wenn wir \zB\ wüßten, daß sich in einem Behältnisse 40 \RWbet{Kugeln} befinden, deren 30 von \RWbet{weißer}, die übrigen 10 von \RWbet{schwarzer} Farbe sind; und wenn wir den Grad der Wahrscheinlichkeit bestimmen sollten, mit dem sich annehmen läßt, daß Jemand, der auf's Geradewohl in den Kasten greift, und Eine der Kugeln herausnimmt, eine \RWbet{weiße} herausziehen werde: so wäre die Anzahl der Fälle, welche hier \RWbet{überhaupt} problematisch möglich sind, 40; denn es sind 40 Kugeln in dem Kasten, und wir wissen von keiner, daß sie nicht könne herausgezogen werden; ja wir haben eben so viel Grund von der Einen, wie von der anderen, anzunehmen, daß sie herauskommen werde. Diese 40 Fälle haben somit alle eine gleiche Wahrscheinlichkeit, und folglich kann 40 zum \RWbet{Nenner} des Bruches angenommen werden. Unter diesen 40 Fällen sind aber nur 30, in welchen der Erfolg, dessen Wahrscheinlichkeit wir hier berechnen sollen, zu Stande kommt, nämlich, daß eine weiße Kugel herausgezogen wird; denn es~\RWSeitenw{42}\ gibt nur 30 weiße Kugeln. Also ist 30 der \RWbet{Zähler} des Bruches; und folglich der Grad der Wahrscheinlichkeit unseres Satzes $\frac{30}{40} = \frac{3}{4}$.
\item Je größer die Anzahl der Fälle ist, bei welchen der Erfolg, dessen Wahrscheinlichkeit berechnet werden soll, zum Vorschein kommt, während die Anzahl der Fälle überhaupt dieselbe bleibt, desto größer die Wahrscheinlichkeit, und der Bruch, der sie mißt, nähert sich um so mehr der \RWbet{Einheit}. Setze man \zB , es wären der weißen Kugeln 39: so wäre der Grad der Wahrscheinlichkeit, daß man eine weiße herausziehen werde $= \frac{39}{40}$.
\item Setzen wir, daß alle Kugeln weiß sind; so ist nicht mehr bloße Wahrscheinlichkeit, sondern \RWbet{Gewißheit} vorhanden, daß eine weiße herausgezogen wird. Die Rechnung aber würde den Grad der Wahrscheinlichkeit $\frac{40}{40} = 1$ geben. Also ist 1 das Maß der völligen Gewißheit, und jede Wahrscheinlichkeit, die noch nicht völlige Gewißheit hat, wird durch einen Bruch gemessen, der kleiner als 1 ($< 1$) ist.
\item Wenn der Grad der Wahrscheinlichkeit einer Behauptung eben so groß ist, als der Grad der Wahrscheinlichkeit ihres \RWbet{contradictorischen} Gegentheils, \dh\ wenn es eben so wahrscheinlich ist, daß etwas sey, als, daß es nicht sey: so nennt man diesen besonderen Grad der Wahrscheinlichkeit die \RWbet{Zweifelhaftigkeit}. Die Rechnung gibt diesen Grad $= \frac{1}{2}$; denn er tritt ein, wenn es eben so viele Fälle von gleicher Wahrscheinlichkeit gibt, in welchen der zu berechnende Erfolg zur Wirklichkeit gelangt, als solche, in denen er nicht zur Wirklichkeit gelangt. Z.\,B.\ Wenn es in jenem Behältnisse 20 weiße und 20 schwarze Kugeln gibt, so ist der Grad der Wahrscheinlichkeit, daß eine \RWbet{weiße} Kugel herauskommen werde, gewiß eben so groß, als der Grad der Wahrscheinlichkeit, daß \RWbet{keine weiße} herauskommen werde; \dh\ der Satz ist zweifelhaft. Und die Rechnung gibt diesen Grad der Wahrscheinlichkeit nach der Regel, die vorhin aufgestellt worden, $= \frac{20}{40} = \frac{1}{2}$.
\item Im \RWbet{gemeinen Leben} pflegt man das Wort: \RWbet{Wahrscheinlich}, in einer etwas engeren Bedeutung zu nehmen, indem man nur diejenigen Sachen \RWbet{wahrscheinliche}~\RWSeitenw{43}\ nennt, deren Wahrscheinlichkeit größer als $\frac{1}{2}$ ($> \frac{1}{2}$) ist, oder die mehr Gründe \RWbet{für}, als \RWbet{wider} sich haben. Sachen, deren Wahrscheinlichkeit kleiner, als $\frac{1}{2}$ ($< \frac{1}{2}$) ist, oder die mehr Gründe \RWbet{wider}, als \RWbet{für} sich haben, nennt man dann \RWbet{unwahrscheinlich}. In der \RWbet{Wissenschaft} aber ist es gewöhnlich, jeder, auch noch so unwahrscheinlichen Behauptung einen gewissen, wenn auch sehr kleinen, Grad von Wahrscheinlichkeit beizulegen; ohngefähr eben so, wie in der Mechanik jeder auch noch so langsamen Bewegung, eine gewisse, nämlich sehr kleine, Geschwindigkeit beigelegt wird; \zB\ dem Zeiger an der Uhr.
\item Die Wahrscheinlichkeit einer Behauptung wird um so kleiner, oder die Unwahrscheinlichkeit, wie man zu sagen pflegt, um so größer, je kleiner der Bruch wird, \dh\ je größer die Anzahl der überhaupt möglichen Fälle von einer gleichen Wahrscheinlichkeit gegen die Anzahl der Fälle wird, in welchen allein der zu berechnende Erfolg zum Vorschein kommt. Und wenn das Verhältniß der ersteren Zahl zur letzteren unberechenbar groß ist, \dh\ wenn die Menge der überhaupt möglichen Fälle von einer gleichen Wahrscheinlichkeit, die Anzahl der Fälle, in welchen der zu berechnende Erfolg zum Vorschein kommt, so oft übertrifft, daß wir es gar nicht zu berechnen vermögen: so pflegt man zu sagen, daß dieser Erfolg eine \RWbet{unendlich} (eigentlich nur eine \RWbet{unberechenbar) kleine Wahrscheinlichkeit}, oder auch wohl eine \RWbet{unendlich große Unwahrscheinlichkeit} habe.
\item Eine unendlich kleine Wahrscheinlichkeit pflegt in dem menschlichen Gemüthe eben dieselbe Wirkung hervorzubringen, wie eine völlige Unmöglichkeit, \dh\ wir geben die Erwartung eines Erfolges, der eine unendlich kleine Wahrscheinlichkeit hat, ganz auf, und betragen uns so, als ob er völlig \RWbet{unmöglich} wäre. Im gemeinen Leben nennt man ihn wohl auch so; in der Wissenschaft ertheilt man ihm den Namen eines \RWbet{moralisch unmöglichen} Falles. Das Beiwort: \RWbet{Moralisch}, hat er erhalten, weil es in den meisten Fällen wirklich \RWbet{Pflicht} ist, bei einer so geringen Wahrscheinlichkeit sich wie bei einer völligen oder absoluten Unmöglichkeit zu verhalten. Aus gleichem Grunde nennt man die contradictorisch~\RWSeitenw{44}\ entgegengesetzte Behauptung \RWbet{moralisch gewiß} oder \RWbet{moralisch nothwendig}. So hat es \zB\ eine unendlich kleine Wahrscheinlichkeit, daß Jemand, der aus einem Kasten voll griechischer Lettern auf's Geradewohl einen Buchstaben nach dem andern herauszieht, gerade auf solche treffen werde, welche zusammengestellt einen vernünftigen Sinn, \zB\ den ersten Vers der Iliade geben werden. Man nennte dieß also \RWbet{moralisch unmöglich}, und erwartet es gar nicht.
\item Allein obgleich die unendlich kleine Wahrscheinlichkeit in den meisten Fällen wie eine völlige Unmöglichkeit betrachtet werden darf und soll: so ist sie darum doch nicht ganz einerlei mit ihr, und in gewissen seltenen Fällen von ihr wohl zu unterscheiden. So kann, was sehr zu bemerken ist, die unendlich kleine Wahrscheinlichkeit oder die moralische Unmöglichkeit \RWbet{verschiedene Grade} haben, was bei der absoluten Unmöglichkeit bekanntlich nicht ist. Von zwei Behauptungen, deren jede für sich eine unendlich kleine Wahrscheinlichkeit hat, kann gleichwohl die Eine vielmal, ja unberechenbar vielmal, oder wie man zu sagen pflegt, unendlichmal unwahrscheinlicher seyn, als die andere. So ist es \zB\ unzähligemal unwahrscheinlicher, daß derjenige, der griechische Lettern aus einem Kasten auf die vorhin beschriebene Art herauszieht, die ganze Iliade, als daß er nur den ersten Vers zu Stande bringen werde. Sollten sich nun die Umstände gerade so fügen, daß nichts anders übrig bleibt, als zwischen zwei oder mehreren Annahmen, deren jede für sich eine unendlich große Unwahrscheinlichkeit hat, eine zu wählen: so müssen wir uns begreiflicher Weise an diejenige halten, deren Unwahrscheinlichkeit noch die geringste ist. Und wenn die übrigen Annahmen alle eine Unwahrscheinlichkeit haben, welche die der Einen unendliche Mal übertrifft; so wird die Wahrscheinlichkeit, mit der wir uns an diese Eine Annahme halten können, unter diesen Umständen \RWbet{moralische Gewißheit} werden.
\item Je nachdem wir unsere Aufmerksamkeit bald auf diese, bald auf jene Umstände richten, die für das Eintreten eines gewissen Erfolges, oder für die Wahrheit eines vorliegenden Satzes sprechen, kann er bald diesen, bald jenen Grad~\RWSeitenw{45}\ der Wahrscheinlichkeit erhalten. Diese nur aus Berücksichtigung gewisser Umstände allein hervorgehende Wahrscheinlichkeit eines Satzes heißt seine \RWbet{relative} oder \RWbet{beziehungsweise} Wahrscheinlichkeit, zum Unterschied von derjenigen, die er erhält, wenn wir auf \RWbet{alle} uns bekannten Umstände oder Gründe merken, die seine \RWbet{absolute Wahrscheinlichkeit} genannt wird. So kann die Nachricht, daß ein gewisser Mensch sich selbst entleibet habe, eine sehr geringe Wahrscheinlichkeit erhalten, wenn wir auf die allen Menschen natürliche Liebe zum Leben hinsehen; dagegen sehr viele Wahrscheinlichkeit gewinnen, wenn wir erwägen, daß er ein lasterhafter Mensch gewesen, für den die Strafe seiner Verbrechen so eben eintrat. Wieder einen andern Grad von Wahrscheinlichkeit kann dieser Selbstmord aus dem Umstande erhalten, daß jener Mensch sich gestern eine Pistole gekauft habe, \usw
\item Eine besondere Art der relativen Wahrscheinlichkeit eines Erfolges, nämlich diejenige, die aus der Betrachtung eines \RWbet{Zeugen} entsteht, pflegt man die \RWbet{äußere} zu nennen; und im Gegentheile diejenige Wahrscheinlichkeit desselben, welche man mit Berücksichtigung \RWbet{aller andern Umstände}, die nur \RWbet{nicht Zeugnisse} sind, erhält, nennt man die \RWbet{innere} Wahrscheinlichkeit. So ist \zB\ die Wahrscheinlichkeit, welche der Selbstmord des Menschen bloß aus dem Umstande erhält, daß Jemand erzählt, er habe die Leiche desselben gesehen, die äußere Wahrscheinlichkeit dieses Ereignisses; während alle übrigen, vorhin erwähnten Umstände zu seiner inneren Wahrscheinlichkeit gehören.
\item Wer einige Kenntnisse in der Buchstabenrechnung hat, wird auch noch folgende mathematische Sätze leicht zu verstehen vermögen, die ich nur darum hier beifügen will, weil sie zur gründlichen Widerlegung jener Einwürfe dienen, die selbst von \RWbet{Mathematikern}, \zB\ von Joh.~\RWbet{Grayg}, gegen die Möglichkeit der historischen Beglaubigung eines Wunders mit einem Anscheine von Gelehrsamkeit vorgebracht worden sind.
\begin{aufzb}
\item Wenn der Grad der Wahrscheinlichkeit der Behauptung des wirklichen Eintreffens eines Erfolges $= x$ ist: so ist der Grad der Wahrscheinlichkeit des contradictorischen~\RWSeitenw{46}\ Gegentheils, oder der Behauptung, daß sich der Erfolg nicht zutragen werde, $= 1 - x$. Denn sey die Anzahl der Fälle, die nicht erweislich unmöglich sind, $= m$, und es mag jeder derselben eine dem andern gleiche Wahrscheinlichkeit haben; unter diesen Fällen mögen sich $n$ Fälle (wo also $n < m$ seyn muß) befinden, bei deren Eintritt der zu berechnende Erfolg zum Vorschein kommt: so ist der Grad seiner Wahrscheinlichkeit $x = \frac{n}{m}$. Unter eben diesen Umständen ist nun die Anzahl der Fälle, in welcher dieser Erfolg nicht zu Stande kommt, $= m - n$, folglich der Grad der Wahrscheinlichkeit, daß er sich nicht ereignen werde, $= \frac{m - n}{m} = \frac{m}{m} - \frac{n}{m} = 1 - \frac{n}{m} = 1 - x$. 
\item Die Wahrscheinlichkeit eines Erfolges $M$, der nur bewirkt wird durch die Vereinigung zweier Umstände $A$ und $B$, deren jeder nur wahrscheinlich ist, gleicht dem Producte aus den Wahrscheinlichkeiten dieser beiden; \dh\ wenn die Wahrscheinlichkeit des Satzes, daß sich der Umstand $A$ einfinden werde, $= x$, die Wahrscheinlichkeit des Satzes aber, daß sich der Umstand $B$ einfinden werde, $= y$ ist: so ist die Wahrscheinlichkeit des Satzes, daß der Erfolg $M$, der nur durch die Vereinigung von $A$ und $B$ zugleich erzeugt wird, zum Vorschein kommen werde, $= xy$. Denn es sey $x = \frac{n}{m}$ und $y = \frac{q}{p}$: so kann man eben darum auch annehmen $x = \frac{nq}{mq}$ und $y = \frac{mq}{mp}$. Die letztere Gleichung aber zeigt uns, daß unter $mp$ Fällen, die überhaupt problematisch möglich sind, und als gleich wahrscheinlich angenommen werden, nur $mq$ Fälle seyen, in welchen der Umstand $B$ eintritt; und die erste $x = \frac{nq}{mq}$, daß unter diesen $mq$ Fällen, in welchen $B$ eintritt, nur $nq$ Fälle sind, in welchen auch noch der Umstand $A$ eintritt. Also kommen auf $mp$ Fälle, die man überhaupt als problematisch möglich und gleich wahrscheinlich annimmt, nur $nq$ Fälle, in welchen $A$ und $B$ zu\RWSeitenw{47}gleich eintreten. Also ist der Grad der Wahrscheinlichkeit, daß $A$ und $B$ zugleich eintreten, \dh\ der Grad der Wahrscheinlichkeit des Erfolges $M$ \Ahat{\ensuremath{= \frac{nq}{mp} = \frac{n}{m} \times \frac {q}{p} = x \cdot y}}{\ensuremath{= \frac{mp}{nq} = \frac{m}{n} \times \frac {p}{q} = x \cdot y}}. 
\item Wenn also ein Erfolg $M$ mehr als zwei Umstände, wenn er \zB\ drei Umstände $A$, $B$, $C$, braucht, deren Wahrscheinlichkeitsgrade $x$, $y$, $z$ sind: so ist seine Wahrscheinlichkeit $= xyz$. Denn die Wahrscheinlichkeit, daß $A$ und $B$ zugleich eintreten werden, ist (\RWlat{litt. b.}) $= xy$; und die Wahrscheinlichkeit, daß sich mit $A$ und $B$ noch $C$ verbinden werde, ist $= (x \cdot y) \times z = x \cdot y \cdot z$.
\item Wenn die Wahrscheinlichkeit eines Erfolges $M$ in einer gewissen Rücksicht $A$ (\dh\ wegen des Vorhandenseyns des Umstandes $A$) $= x$; in einer andern Rücksicht $B$ (\dh\ wegen des Vorhandenseyns des Umstandes $B$) $= y$ ist: so ist der Grad der Wahrscheinlichkeit des Erfolges $M$ aus der Vereinigung von beiden Rücksichten (\dh\ weil beide Umstände zugleich vorhanden sind) oder die \RWbet{absolute} Wahrscheinlichkeit, die aus Vereinigung jener zwei relativen hervorgehet $= \frac{xy}{xy + (1-x)(1-y)}$. Denn weil wir die beiden Umstände $A$ und $B$ vereinigt antreffen; so kann nur Eines von Beidem der Fall seyn: entweder sie kündigen beide den Eintritt des Erfolges $M$ an, oder sie täuschen uns beide. Weil nun die Wahrscheinlichkeit des Erfolges $M$ bloß aus dem Umstande $A$, $= x = \frac{n}{m}$ (\RWlat{litt. a}), und bloß aus dem Umstande $B$, $= y = \frac{q}{p}$ ist: so gibt es unter $mp$ Fällen von einer gleichen Wahrscheinlichkeit nur $nq$ Fälle, in welchen die Umstände $A$ und $B$ vereinigt anzutreffen sind, um den Erfolg $M$ zu verkünden. Weil ferner die Wahrscheinlichkeit, daß der Erfolg $M$ \RWbet{nicht eintreten} werde, bloß aus dem Umstande $A$, (nach \RWlat{litt.~a}) $= 1 - x = \frac{m-n}{m}$, und bloß aus dem Umstande $B$, $= 1 - y = \frac{p-q}{p}$~\RWSeitenw{48}\ ist: so gibt es unter den vorhin erwähnten $mp$ Fällen von einer gleichen Wahrscheinlichkeit, nur $(m - n) \times (p - q)$ Fälle, in welchen sich die Umstände $A$ und $B$ vereinigen, ohne den Erfolg $M$ zu verkünden. Es ist daher der Grad der Wahrscheinlichkeit, daß diese Umstände vereinigt sind, nicht um uns zu täuschen, sondern um den Erfolg $M$ anzukündigen $= \frac{nq}{nq+(m-n)\times (p-q)} =  \frac{xy}{xy+(1-x) \times (1-y)}$.
\item Hieraus erhellet von selbst, wie die Wahrscheinlichkeit einer Sache zu berechnen sey, für die mehr als zwei Umstände sprechen, \zB\ die drei Umstände $A$, $B$, $C$, welche derselben jeder für sich die Wahrscheinlichkeiten $x$, $y$, $z$ ertheilen. Diese Wahrscheinlichkeit ist nämlich $= \frac{xyz}{xyz + (1 - x) (1 - y) (1 - z)}$; \usw
\item Wenn der Grad der Wahrscheinlichkeit, die von dem Umstande $A$ für sich allein hervorgeht $= \frac{1}{2}$  ist, \dh\ wenn dieser Umstand für sich allein das Eintreten des Erfolges $M$ bloß zweifelhaft machet: so ist die Wahrscheinlichkeit, die er aus der Vereinigung des Umstandes $A$ mit $B$ erhält, $=  \frac{\frac{1}{2} \cdot y}{\frac{1}{2} y + \frac{1}{2} (1-y)}= y$; also eben so groß, als wenn der Umstand $A$ gar nicht vorhanden wäre. Wenn die Wahrscheinlichkeit, die der Umstand $A$ für sich allein gewähret, sogar noch kleiner als $\frac{1}{2}$ ist; \dh\ wenn es aus diesem Umstande für sich allein sogar wahrscheinlicher ist, daß der Erfolg $M$ nicht eintreten, als daß er eintreten werde: so ist die Wahrscheinlichkeit, die aus Vereinigung des Umstandes $A$ mit $B$ entstehet, sogar noch $< y$.
\item Wenn die Wahrscheinlichkeit eines Erfolges $M$, $= x$, und eines andern $N$, der jenem widerstreitet, $= y$ ist: so ist der Grad der Wahrscheinlichkeit, mit dem wir annehmen können, daß der Erfolg $M$ eher, als der Erfolg $N$, Statt finden werde, $= \frac{x}{x+y}$. Denn es sey $x = \frac{n}{m}$ und $y = \frac{p}{q}$: so kann man eben darum auch $x = \frac{nq}{mq}$  und $y = \frac{mp}{mq}$~\RWSeitenw{49}\ annehmen. Es gibt daher unter $mq$ Fällen, die problematisch möglich und von gleicher Wahrscheinlichkeit sind, nur $nq$ Fälle, in denen $M$, und nur $mp$ Fälle, in denen $N$ zum Vorschein kommt; der Fälle also, in welchen Einer von beiden Erfolgen $M$ oder $N$ Statt findet, gibt es $nq + mp$, und unter diesen $nq + mp$ Fällen sind nur $nq$ Fälle, in welchen $M$ Statt findet. Also die Wahrscheinlichkeit, mit der man annehmen kann, daß $M$ Statt finden werde, eher als $N$, $= \frac{nq}{mq+mp} =  \frac{\frac{n}{m}}{\frac{n}{m}+\frac{p}{q}} = \frac{x}{x+y}$.
\end{aufzb}
\end{aufza}

\RWpar{16}{Eintheilung der Zeugen in mittelbare und unmittelbare}
Die meisten Ereignisse, die ich in der Folge als Beweise zur Bestätigung der christlichen Offenbarung benützen will, gehören zu den historischen Ereignissen, in der oben aufgestellten engsten Bedeutung des Wortes, \dh\ sie sind auf \RWbet{Zeugnisse} gegründet. Ich muß also erst den \RWbet{Grad der Glaubwürdigkeit} eines Zeugen, \dh\ den \RWbet{Grad der Wahrscheinlichkeit}, welchen die Annahme hat, daß ein bestimmter Zeuge die Wahrheit rede, genauer erörtern. Zuvörderst muß ich aber den Unterschied festsetzen, der zwischen \RWbet{unmittelbaren} und \RWbet{mittelbaren} Zeugen Statt findet.
\begin{aufza}
\item Ein \RWbet{unmittelbarer Zeuge} heißt mir derjenige, der die Begebenheiten, welche er uns berichtet, selbst wahrgenommen hat, \dh\ der jene sinnlichen Wahrnehmungen, aus welchen die Begebenheit ohne Zuziehung eines anderen Zeugnisses gefolgert werden kann, selbst gemacht hat. Er heißt auch \RWbet{Augenzeuge} oder die \RWbet{erste Quelle}, oder auch vorzugsweise die Quelle. So heißt mir \zB\ Jemand, der die Gegend von Palmyra bereiset hat, ein unmittelbarer Zeuge von dem blühenden Zustande, der einst in dieser Gegend geherrscht; denn er kann, obgleich er diesen Zustand nicht mit Augen gesehen hat, ihn doch mit aller Sicherheit folgern, ohne hiezu erst eines anderen Zeugnisses zu bedürfen. In einer \RWbet{engern} aber und in der \RWbet{strengsten} Bedeutung heißt nur derjenige ein unmittelbarer Zeuge eines Ereignisses, der~\RWSeitenw{50}\ Alles dasjenige wahrnahm, was sich an einem Ereignisse von solcher Art von Menschen wahrnehmen läßt, um es mit größter Sicherheit zu folgern. In dieser Bedeutung wäre nun jener Reisende noch kein unmittelbarer Zeuge von dem blühenden Zustande in Palmyra zu nennen, sondern das wäre nur ein Mensch, der damals lebte, als Palmyra im Flor stand, und sich dort aufhielt.
\item Dagegen derjenige, der uns eine Begebenheit berichtet, die er nicht selbst wahrgenommen, sondern nur auf eines Anderen Zeugniß erkannt hat, heißt nur ein \RWbet{mittelbarer Zeuge} in Hinsicht auf diese Begebenheit. Man pflegt ihn auch \RWbet{Ohrenzeuge}, und nach Umständen die \RWbet{zweite}, die \RWbet{dritte Quelle} zu nennen. In einer weiteren Bedeutung könnte schon derjenige ein mittelbarer Zeuge heißen, der die Begebenheit, wenn auch nicht aus eines Anderen Zeugnisse, doch aus Wahrnehmungen schloß, die nicht die sichersten sind, aus welchen sie von Menschen gefolgert werden kann.
\item Wenn Jemand bloß bezeugt, daß er eine gewisse Begebenheit gehört oder gelesen habe, kurz, daß sie ihm bezeugt worden sey, ohne sich selbst für ihre Wahrheit zu verbürgen: so sollte er eigentlich gar kein Zeuge derselben, sondern ein unmittelbarer Zeuge ihrer mündlichen oder schriftlichen Erzählung oder Beschreibung heißen; indeß pflegt man ihn dennoch \RWbet{uneigentlicher} Weise einen Zeugen, und dieß zwar einen \RWbet{mittelbaren} Zeugen von der Begebenheit selbst zu nennen.
\end{aufza}

\RWpar{17}{Erfordernisse zur Glaubwürdigkeit eines unmittelbaren Zeugen}
Im letzten Hauptstücke des ersten Haupttheiles wurde von den Erfordernissen zu einer \RWbet{Zeugenschaft} überhaupt gehandelt, ohne zu bestimmen, ob der Zeuge ein \RWbet{Mensch}, oder irgend ein \RWbet{höheres} Wesen \zB\ \RWbet{Gott} selbst sey. Jetzt wollen wir diese Erfordernisse für den Fall, wo der Zeuge ein \RWbet{Mensch} ist, etwas genauer bestimmen, und zwar erstens für den Fall, wo er ein \RWbet{unmittelbarer} Zeuge seyn soll.~\RWSeitenw{51}\par

\vabst \textbf{A.}~Von Seite des \RWbet{Zeugen} wird erfordert:
\begin{aufza}
\item Eine gewisse \RWbet{Kenntniß der Begebenheit}, die er uns berichten soll; folglich
\begin{aufzb}
\item \RWbet{Gegenwart an dem Orte und zu der Zeit}, wo es dem Zeugen möglich war, die sinnlichen Erscheinungen wahrzunehmen, aus welchen die Begebenheit geschlossen werden sollte.
\item \RWbet{Gesunde Sinneswerkzeuge}; weil man nur mit diesen die sinnlichen Erscheinungen richtig beobachten und auffassen kann.
\item \RWbet{Aufmerksamkeit auf diese Erscheinungen}; denn wenn man zerstreut ist, so faßt man auch mit den besten Sinneswerkzeugen entweder gar nicht, oder unrichtig auf.
\item So viel \RWbet{Verstand} und \RWbet{Ueberlegung}, als nöthig ist, um aus dem, was in die Sinne fällt, richtig zu folgern, was eigentlich vorgegangen sey.
\item Wenn die Begebenheit zu ihrer richtigen Beobachtung gewisser \RWbet{künstlicher Werkzeuge}, \zB\ Sehröhre, Maßstäbe \udgl\  bedarf: so müssen auch diese ihm zu Gebote gestanden seyn, und er mußte die \RWbet{Fertigkeit} und \RWbet{Geschicklichkeit} besessen haben, sich derselben gehörig zu bedienen.
\item So viel \RWbet{Gedächtniß}, als nöthig ist, die Sache zu behalten, bis zu der Zeit, wo er sie mündlich oder schriftlich mittheilen soll.
\end{aufzb}
\item Die \RWbet{Fähigkeit, entsprechende Vorstellungen in uns hervorzubringen}. Also wenn dieses durch Sprache oder Schrift geschehen soll,
\begin{aufzb}
\item bestimmte \RWbet{Ausdrücke} in der Sprache, für jene sinnlichen Erscheinungen, die der Zeuge wahrgenommen hat;
\item \RWbet{Kenntniß dieser Ausdrücke}, um sie gebrauchen zu können, \dh\ der Zeuge muß unter den gebrauchten Worten nicht etwas Anderes verstehen, als wir.
\end{aufzb}
\item Der \RWbet{Wille, diese Vorstellungen wirklich in uns hervorzubringen}. Dieser Wille mag übrigens aus was immer für Gründen entstehen, aus pflichtschuldigem Eifer für die Verbreitung nützlicher Wahrheit, auch Wahrheitsliebe genannt, oder aus Hoffnung einer Belohnung, oder aus Furcht~\RWSeitenw{52}\ vor Strafe, oder aus Eigennutz, weil es der eigene Vortheil fordert, die Wahrheit zu erzählen \usw\ Unrichtig ist es also, wenn man Wahrheitsliebe zu einem nothwendigen Erfordernisse bei einem glaubwürdigen Zeugen macht; \RWbet{nicht Wahrheitsliebe}, sondern \RWbet{Wahrhaftigkeit} wird erfordert.
\end{aufza}\par

\vabst \textbf{B.}~Von Seite \RWbet{Unser}, die wir das Zeugniß annehmen sollen, ist nöthig, daß wir uns von dem Vorhandenseyn jener drei Stücke hinlänglich überzeugen; also
\begin{aufzb}
\item \RWbet{Ueberzeugung von der Sachkenntniß des Zeugen}. Wir müssen uns überzeugen, daß der Zeuge zu einer Zeit und an einem Orte gelebt, wo er die Begebenheit hat wahrnehmen können; \usw
\item \RWbet{Ueberzeugung von seiner Fähigkeit, sich mitzutheilen}. Also \zB\ wenn es ein Schriftsteller ist, der in einer schon ausgestorbenen Sprache erzählt, müssen wir mit gutem Grunde glauben können, daß er diese Sprache hinlänglich verstehe, \udgl
\item \RWbet{Ueberzeugung von seiner Wahrhaftigkeit}. Wir müssen einsehen, daß wirklich ein Grund vorhanden gewesen sey, der ihn bestimmt hat, die Wahrheit zu erzählen.
\end{aufzb}
\begin{RWanm}
Wenn die drei letztgenannten Ueberzeugungen keine völlige Gewißheit haben: so sind wir auch von der Wahrheit des erhaltenen Zeugnisses mit keiner völligen Gewißheit überzeugt. Der Zeuge, sagen wir, ist nicht völlig glaubwürdig. Der eigenthümliche Grad seiner Glaubwürdigkeit aber, \dh\ der Grad der Wahrscheinlichkeit, mit der wir uns auf seine Aussage verlassen können, wird durch das Product aus den drei Graden der Wahrscheinlichkeit, die jene drei Ueberzeugungen haben, bestimmt. Wäre \zB\ der Grad der Wahrscheinlichkeit, daß ein Zeuge Sachkenntniß habe, $= \frac{9}{10}$ (wir wüßten, daß \zB\ von zehn Menschen neun im Stande sind, eine Begebenheit an diesem Orte gehörig aufzufassen, \usw ); der Grad der Wahrscheinlichkeit, daß er die Fähigkeit der Gedankenmittheilung habe, $= \frac{4}{5}$ (wir wüßten \zB , daß er sich in fünf Fällen vier Mal richtig erkläre); endlich der Grund der Wahrscheinlichkeit, daß er Wahrhaftigkeit habe, $ = \frac{2}{3}$ (wir wüßten \zB , daß unter solchen Umständen, wie die seinigen sind, von drei Menschen zwei die Wahrheit reden): so wäre der Grad der Wahrscheinlichkeit, mit dem wir uns auf seine Aussage verlassen können $= \frac{9}{10} \cdot \frac{4}{5} \cdot \frac{2}{3} = \frac{12}{25}$.~\RWSeitenw{53}
\end{RWanm}

\RWpar{18}{Erfordernisse zur Glaubwürdigkeit eines mittelbaren Zeugen im eigentlichen sowohl, als uneigentlichen Sinne des Wortes}
\textbf{I.}~Die Erfordernisse zur Glaubwürdigkeit eines \RWbet{mittelbaren Zeugen im eigentlichen Sinne} sind nur in einigen Stücken, nähmlich in jenen, welche die Sachkenntniß betreffen, von den Erfordernissen eines unmittelbaren Zeugen unterschieden.\par

\vabst \textbf{A.}~Von Seite des \RWbet{Zeugen} wird nämlich jedesmal erfordert:
\begin{aufza}
\item Eine richtige \RWbet{Sachkenntniß} von der Begebenheit, welche er uns berichten soll; diese verlangt nun hier wieder drei Stücke:
\begin{aufzb}
\item Es müssen \RWbet{unmittelbare Zeugen} wirklich vorhanden seyn.
\item Er muß diese Zeugen \RWbet{richtig verstanden}, und ihre Aussagen sich \RWbet{gemerkt} haben, wozu wieder Kenntniß ihrer Sprache und Gedächtniß gehört.
\item Sie müssen ihm die Wahrheit berichtet haben.
\end{aufzb}
\item Nebst der Sachkenntniß muß auch der mittelbare Zeuge die Fähigkeit, entsprechende Vorstellungen in uns hervorzubringen und
\item den Willen dazu besitzen, wie der unmittelbare.
\end{aufza}\par

\vabst \textbf{B.}~Von Seite \RWbet{Unser}, die wir durch seine Aussage überzeugt werden sollen, wird Ueberzeugung von dem wirklichen Vorhandenseyn jener drei Stücke erfordert. Also
\begin{aufza}
\item Ueberzeugung von seiner Sachkenntniß. Diese erfordert nun wieder:
\begin{aufzb}
\item wir müssen uns überzeugen, daß Zeugen wirklich vorhanden gewesen sind. Von diesem Umstande ist er ein unmittelbarer Zeuge; die Untersuchung geschieht also hier, wie im vorigen Paragraph gelehrt wurde.
\item Wir müssen überzeugt seyn, daß er die Zeugen richtig verstanden, und ihre Aussagen sich gemerkt habe. Dieses erfordert abermals, daß wir ihm Kenntniß der Sprache,~\RWSeitenw{54}\ in welcher die Zeugen geredet oder geschrieben, und hinlängliches Gedächtniß zutrauen.
\item Wir müssen uns überzeugen, daß die Zeugen wirklich die Wahrheit geredet haben. Sollen wir dieses um seinetwillen glauben (und dieses muß geschehen, wenn er ein mittelbarer Zeuge im eigentlichen Sinne für uns werden soll): so wird hiezu neuerdings erfordert:
\end{aufzb}
\begin{aufzc}
\item von \RWbet{seiner} Seite
\begin{aufzb}[a.]
\item die Fähigkeit, Zeugen zu prüfen;
\item die Anwendung dieser Fähigkeit auf den vorhandenen Fall;
\item ein wahrhaftes Geständniß des Ergebnisses seiner Prüfung.
\end{aufzb}
\item Von \RWbet{unserer} Seite die Ueberzeugung von dem Vorhandenseyn dieser drei Stücke.
\end{aufzc}
\item Auch von der Fähigkeit, sich verständlich mitzutheilen, und
\item von dem Willen, es zu thun, müssen wir bei dem mittelbaren Zeugen eben so, wie bei dem unmittelbaren, überzeugt seyn.
\end{aufza}\par

\vabst II.~Zur Glaubwürdigkeit eines \RWbet{mittelbaren Zeugen}, der es in \RWbet{uneigentlicher} Bedeutung ist, oder zur Glaubwürdigkeit einer Begebenheit, welche uns Jemand bloß gehört oder gelesen zu haben bezeugt, ohne sie selbst verbürgen zu wollen, wird Zweierlei erfordert:
\begin{aufza}
\item Wir müssen uns überzeugen, daß der Mann diese Begebenheit \RWbet{wirklich gehört oder gelesen habe}. Von diesem Umstande ist er ein unmittelbarer Zeuge, die Untersuchung geschieht also nach dem vorhergehenden Paragraph.
\item Wir müssen uns überzeugen, \RWbet{daß jene Zeugen, die er gehört oder gelesen, die Wahrheit ausgesagt haben}. Da wir dieses nicht, wie im vorhergehenden Falle, auf sein Wort annehmen sollen, indem er es uns selbst nicht verbürgen will: so erübriget nichts Anderes, als daß wir unabhängig von ihm die Eigenschaften dieser Zeugen, so viel wir sie kennen, untersuchen. Sie können unmittelbare oder mittelbare Zeugen seyn; im ersten Falle hat man nach dem vorhergehenden, im zweiten nach dem ersten Theile des gegenwärtigen Paragraphen zu verfahren.~\RWSeitenw{55}
\end{aufza}

\RWpar{19}{Bestimmung der absoluten Wahrscheinlichkeit einer Begebenheit, für die man eine Zeugenaussage hat}
\begin{aufza}
\item Jede Begebenheit kann an und für sich betrachtet, \dh\ abgesehen davon, daß sie von einem Zeugen berichtet wird, schon einen gewissen Grad (innerer) Wahrscheinlichkeit oder Unwahrscheinlichkeit haben, der mit dem Grade der Wahrscheinlichkeit des Zeugen verbunden die \RWbet{absolute} Wahrscheinlichkeit derselben darstellt.
\item Bezeichnen wir nämlich die innere Wahrscheinlichkeit der Begebenheit durch $x$, die äußere, die ihr bloß durch das Zeugniß erwuchs, durch $y$:
so ist (\RWparnr{15}\ Lehrs.\,d.) die absolute Wahrscheinlichkeit derselben $= \frac{xy}{xy+(1-x)(1-y)}$.
\item Da nun der Werth des Bruches $\frac{xy}{xy+(1-x)(1-y)}$ der Einheit so nahe kommen kann, als man will, wenn nur Eine der Größen $x$ oder $y$ der Einheit so nahe kommt, als man will, die andere sey auch wie immer beschaffen: so ergibt sich, daß eine Begebenheit glaubwürdig bleiben kann, auch wenn der Zeuge für sie noch so wenig Glaubwürdigkeit hat, falls nur die innere Wahrscheinlichkeit derselben groß genug ist; ingleichen, daß die Begebenheit glaubwürdig werden kann, auch wenn ihre innere Wahrscheinlichkeit noch so gering ist, falls nur die Glaubwürdigkeit des Zeugen groß genug ist.
\item Wenn die Glaubwürdigkeit des Zeugen $= \frac{1}{2}$ ist; so wird die Glaubwürdigkeit des Ereignisses durch seine Zeugenschaft nicht vergrößert; und wenn seine Glaubwürdigkeit $< \frac{1}{2}$ ist: so wird die Glaubwürdigkeit des Ereignisses durch den Beitritt seines Zeugnisses sogar vermindert.
\item Dieß klingt befremdend, und ist von mehreren Gelehrten wirklich benützt worden, um den historischen Glauben, besonders in Hinsicht auf Wunder, wankend zu machen. Allein man muß sich erinnern, daß die Wahrscheinlichkeit eines Zeugen nur dann $=$ oder gar $< \frac{1}{2}$ angesetzt werden dürfe, wenn der Umstand, daß er das \Ahat{Ereigniß}{Zeugniß} berichtet, einen eben so starken, oder wohl gar noch stärkeren Grund dafür abgibt,~\RWSeitenw{56}\ daß sich das Ereigniß nicht zugetragen habe, als dafür, daß es sich zugetragen habe. Und dieß wird sicher nur äußerst selten der Fall seyn. Denn selbst ein Zeuge, der wenig, oder gar keine Liebe zur Wahrheit hat, begründet bloß dadurch, daß er uns etwas erzählt, nicht sogleich die Vermuthung, daß es sich nicht werde zugetragen haben; zumal wenn das Ereigniß, das er erzählt, innere Unwahrscheinlichkeit hat. Denn je größer diese ist, um so größer ist auch die Unwahrscheinlichkeit, daß er gerade auf den Gedanken, dieß zu erzählen, verfallen seyn sollte, wenn es sich nicht in der That ergeben hätte.
\end{aufza}

\RWpar{20}{Aufzählung einiger besonders merkwürdiger Fälle, in welchen ein Zeuge keinen Glauben verdient}
\begin{aufza}
\item Ein \RWbet{unmittelbarer} Zeuge verdient keinen Glauben:
\begin{aufzb}
\item wenn seine Aussage ihm selbst Vortheile bringt, und seine Wahrheitsliebe nicht aus anderen Proben als eine sehr unbestechliche bekannt ist; \zB\ wenn Jemand von einer bei einem Fürsten gehabten Audienz Dinge, die für ihn selbst sehr rühmlich sind, erzählet. Denn weil die Aussage ihm selbst Vortheile bringt: so ist es sehr leicht möglich, daß, wenn auch die Begebenheit ganz anders war, sie doch von ihm so dargestellt wird.
\item Wenn er aus andern Aussagen bereits als ein sehr ungetreuer und lügenhafter Erzähler bekannt ist. Denn ein solcher pflegt dann ohne besondere Vortheile, bloß aus Gewohnheit zu lügen.
\item Wenn seine Erzählung auffallend ängstlich, verschroben oder gekünstelt ist, da er doch sonst in einem weit natürlicheren Tone zu erzählen pflegt. Dieser gekünstelte, ängstliche Ton wird nun ein sehr starker (innerer) Wahrscheinlichkeitsgrund dafür, daß seine ganze Erzählung erdichtet sey, weil man, um Wahrheit zu reden, keiner Kunst bedarf.
\end{aufzb}
\item Ein \RWbet{mittelbarer} Zeuge verdienet keinen Glauben:
\begin{aufzb}
\item Wenn erweislich ist, daß die Quelle, aus welcher er seine Nachrichten geschöpft hat, lügenhaft sey. Denn er hat nur in sofern Sachkenntniß, als die Quelle, aus wel\RWSeitenw{57}cher er seine Nachrichten schöpfte, Wahrheit erzählt. Wenn also erwiesen ist, daß diese gelogen hat: so hat er keine Sachkenntniß, und verdient daher keinen Glauben.
\begin{RWanm}
Etwas zweideutig ist die Regel; der mittelbare Zeuge verdiene nie mehr Glauben, als seine Quelle verdient. Wahr ist es, daß er nie mehr Glauben verdient, als seine Quelle in aller Rücksicht, unter Anderem auch aus dem Grunde verdient, weil er ihr sein Zutrauen schenkte; falsch aber, daß er nie mehr Glauben verdient, als jene Quelle an und für sich verdient, auch abgesehen von dem Umstande, daß er ihr traute. Denn wenn der mittelbare Zeuge ein Mann von Einsichten ist: so können wir eben wegen des Umstandes, daß er dieser Quelle traute, auf ihre Glaubwürdigkeit schließen.
\end{RWanm}
\item Wenn er leichtgläubig ist. Dann steht zu befürchten, daß er auch manchem falschen Berichte geglaubt, und sich also keine richtige Sachkenntniß erworben habe.
\item Wenn es erweislich ist, daß er gar keine Quelle vor sich gehabt habe; \zB\ wenn sich die Begebenheit Jahrhunderte vor ihm zugetragen haben soll, und keine früheren Schriftsteller von derselben Meldung machen, \usw
\end{aufzb}
\end{aufza}

\RWpar{21}{Ueber Zeugenmehrheit}
\begin{aufza}
\item Es ist nicht schlechterdings nothwendig, daß man der Zeugen mehrere für eine Begebenheit aufzuweisen habe, um sich von ihrem Geschehenseyn hinlänglich zu versichern. Die oben \RWparnr{18}\ und 19 aufgestellten Erfordernisse zur Glaubwürdigkeit eines unmittelbaren sowohl als mittelbaren Zeugen können auch bei einem einzigen Zeugen in einem hinlänglich hohen Grade vorhanden seyn. Selbst wenn die Begebenheit eine (wie man sagt) unendlich große innere Unwahrscheinlichkeit oder (was eben so viel heißt) eine unendlich kleine innere Wahrscheinlichkeit ($= \frac{n}{\infty}$) hat, kann sie durch die Aussage eines einzigen Zeugen moralische Gewißheit erhalten, wie wir dieß \RWparnr{19}\ bereits gesehen haben.
\item Aber freilich ist es doch immer besser, wenn wir der Zeugen mehrere, als wenn wir nur Einen haben. Denn~\RWSeitenw{58}\ je mehrere Zeugen wir haben, um desto unwahrscheinlicher ist es, daß sie sich Alle entweder irren, oder uns betrügen wollen.
\begin{RWanm} 
Eine ähnliche Unrichtigkeit, wie in der Regel, die ich im vorigen Paragraph erwähnte, liegt auch in folgender: Mehrere Zeugen, welche aus einer Quelle schöpften, gelten immer nur für Einen.
\end{RWanm}
\item In dem besonderen Falle, wenn mehrere von einander ganz unabhängige Zeugen für eine Begebenheit sich vereinigen, wird man die Glaubwürdigkeit eines Jeden einzelnen aus ihnen selten oder nie $=$ oder $< \frac{1}{2}$ ansetzen können, wenn das Ereigniß ein ungewöhnliches ist. Denn setzen wir, daß Jemand aus einer Million Kugeln, die mit den Nummern $1$ bis $1.000.000$ bezeichnet sind, Eine hervorgezogen habe, und daß zwei Zeugen (deren der Eine nichts von der Angabe des Anderen weiß) der herausgezogenen Kugel die Nummer $275$ beilegen: wird es nicht schon durch diese Uebereinstimmung äußerst wahrscheinlich, daß Beide die Wahrheit berichten, weil es äußerst unwahrscheinlich ist, daß sie sonst Beide auf dieselbe Zahl verfallen wären?
\end{aufza}


\RWpar{22}{Ueber Zeugenwiderspruch}
Ich theile hier zuerst einige Regeln mit, wornach zu beurtheilen ist, ob \RWbet{Zeugenwiderspruch} wirklich vorhanden sey oder nicht; dann einige Regeln, wie man sich im Falle eines solchen Widerspruches zu verhalten habe.\par

\vabst \textbf{A.}~Man glaubt oft einen Widerspruch zwischen den Aussagen mehrerer Zeugen zu finden, wo doch in Wahrheit keiner vorhanden ist, und man sieht oft keinen dort, wo wirklich einer ist.
\begin{aufza}
\item Wenn die Aussagen zweier oder mehrerer Zeugen über eine und dieselbe Begebenheit verschiedene Umstände erwähnen: so ist demungeachtet kein Widerspruch unter ihnen, wofern nur
\begin{aufzb}
\item keiner die Umstände, welche der Andere anführt, ausdrücklich läugnet;
\item diese verschiedenen Umstände gar wohl neben einander bestehen können; auch endlich~\RWSeitenw{59}
\item begreiflich ist, warum der Eine gerade nur diese, der Andere nur jene Umstände anmerkt. Das Letztere kann nun wieder
\begin{aufzc}
\item aus den \RWbet{verschiedenen Standpuncten} begreiflich werden, aus welchen die Zeugen die Begebenheit (in wiefern sie unmittelbare Zeugen derselben sind) beobachtet haben. Je nachdem man nämlich eine Begebenheit bald aus diesem, bald aus jenem Standpuncte beobachtet, fallen bald diese, bald jene Umstände in's Auge;
\item aus der \RWbet{Verschiedenheit ihrer Charaktere}, oder der im Augenblicke der Beobachtung bei ihnen herrschenden \RWbet{Gemüthsstimmung}. Nach Verschiedenheit dieser beiden Stücke wirkt bald das Eine, bald das Andere stärker auf uns, prägt sich unserem Gedächtnisse tiefer ein, \usw ;
\item aus dem \RWbet{verschiedenen Zwecke}, zu welchem sie die Begebenheiten gerade jetzt erzählen. Auch derselbe Mensch pflegt andere Umstände herauszuheben, je nachdem er eine Begebenheit bald zu diesem, bald jenem Zwecke erzählt.
\end{aufzc}
\end{aufzb}
\item Dagegen ist ein wirklicher Widerspruch ohne Zweifel vorhanden, wenn
\begin{aufzb}
\item der eine Umstände erwähnt, die sich mit den Umständen, die der Andere anführt, nicht vereinbaren lassen; wenn ferner
\item der Eine zwar nicht ausdrücklich läugnet, was der Andere aussagt, aber doch davon schweigt, ob er gleich Ursache gehabt hätte, es zu erwähnen, wenn es sich wirklich zugetragen hätte. Denn dieses Stillschweigen ist dann ein Beweis, daß sich die Sache nicht zugetragen habe. Man pflegt es daher oft mit dem Namen eines \RWbet{stillschweigenden Zeugnisses} zu belegen. Z.\,B.\ Wenn wir Jemand allerlei Vorwürfe machen, und er vertheidiget sich gar nicht.
\end{aufzb}
\end{aufza}\par

\vabst \textbf{B.}~Wenn nun erwiesen ist, daß zwei oder mehrere Zeugen sich in der That widersprechen: so ist einleuchtender Weise demjenigen der Vorzug einzuräumen, bei welchem der Grad der Wahrscheinlichkeit, der aus Vereinigung seiner relativen Glaubwürdigkeit mit der inneren Wahrscheinlichkeit seiner~\RWSeitenw{60}\ Erzählung entsteht, am allergrößten ist. Z.\,B.\ wenn $A$ erzählt, daß ein gewisser Mann über einen schmalen Steg gehend unversehens in's Wasser gefallen sey; $B$ aber, daß er sich absichtlich hineingestürzt habe; und die Glaubwürdigkeit von $A$, $= \frac{3}{4}$, die innere Glaubwürdigkeit seiner Erzählung $= \frac{1}{2}$; die Glaubwürdigkeit des $B$ dagegen $= \frac{4}{5}$, die innere seiner Erzählung $\frac{1}{100}$ ist; so wird (nach \RWparnr{19}) die Wahrscheinlichkeit des ersten Ereignisses $= \frac{3}{4}$ die des zweiten $= \frac{4}{103}$ seyn. Also ist das Erste viel glaubwürdiger, als das Letztere, und der Grad der Wahrscheinlichkeit, mit dem wir annehmen können, daß sich das Erste und nicht das Letzte zugetragen habe, ist (nach \RWparnr{15}\ Lehrs.\,g.) $= \frac{309}{325}$.\par

\vabst \textbf{C.}~Aus diesen allgemeinen Regeln lassen sich leicht nachstehende besondere Regeln, die man gewöhnlich aufzustellen pflegt, herleiten.
\begin{aufzb}
\item Der unmittelbare Zeuge hat den Vorzug vor dem mittelbaren, wenn sonst alle übrigen Umstände gleich sind (\dh\ wenn die innere Wahrscheinlichkeit der Begebenheit, die Beide erzählen, gleich ist, und von den Zeugen entweder gar keine Umstände, oder nur gleiche, \zB\ eine gleiche Vermuthung für ihre Redlichkeit, bekannt sind). Denn bei dem mittelbaren Zeugen ist dann die Vermuthung für seine Sachkenntniß geringer, weil bei demselben mehrere Umstände sich vereinigen müssen, wenn er Sachkenntniß haben soll, als bei dem unmittelbaren.
\item Der Gelehrte vor dem Ungelehrten, wenn die Begebenheit zu ihrer richtigen Beobachtung einer eigenen Geschicklichkeit bedurfte.
\item Der Zeuge, dessen Rechtschaffenheit bereits erprobt ist, vor dem weniger geprüften.
\item Der Zeuge, der von seiner Aussage Nachtheile hat, vor dem, der Nutzen aus seiner Aussage zieht.
\item Die Mehreren gehen den Wenigeren vor, wenn ihre Glaubwürdigkeit sowohl, als auch die innere Wahrscheinlichkeit der erzählten Begebenheit gleich ist. Doch gilt dieß eigentlich nur, wenn die Glaubwürdigkeit jedes
einzelnen Zeugen für sich $> \frac{1}{2}$ ist, wie dieß fast insgemein zu seyn pflegt.~\RWSeitenw{61}
\begin{RWanm} Wenn Zeugen sich nicht widersprechen, sondern nur verschieden sind (\textbf{A.}\,1.): so wird die Glaubwürdigkeit ihrer Aussage durch einen solchen Umstand oft noch erhöhet. Denn wir können hieraus meistentheils schließen, daß sie sich nicht verabredet haben; und wenn sich Zeugen in gewissen Stücken widersprechen, in anderen übereinstimmen: so werden die letzteren Stücke durch diesen Umstand gleichfalls um so glaubwürdiger; doch eigentlich nur dann, wenn jener Widerspruch beweiset, daß sie entgegengesetzte Vortheile hatten, und also geneigt gewesen wären, einander auch in demjenigen, worin sie übereinstimmen, zu widersprechen, wenn es möglich gewesen wäre.
\end{RWanm}
\end{aufzb}

\RWpar{23}{Begriff der Aechtheit und Unverfälschtheit eines Buches}
\begin{aufza}
\item Die meisten Zeugnisse, deren man sich in der Geschichte bedient, insonderheit diejenigen, aus welchen wir die für das Christenthum gewirkten Wunder genauer kennen lernen sollen, sind in schriftlichen Aufsätzen enthalten. Nun gibt es zwar allerdings Fälle, in welchen ein schriftlicher Aufsatz Glauben verdienen soll, auch wenn man nicht bestimmt weiß, wer ihn verfaßt habe; nämlich um derjenigen Menschen willen, die diesen Aufsatz für glaubwürdig erklärten; aber wenn man den Verfasser eines Aufsatzes kennt, und die zu einem glaubwürdigen Zeugen erforderlichen Eigenschaften an ihm findet: so kann man dem Aufsatze dann um so völliger trauen, nicht mehr um Anderer, sondern um des Verfassers selbst willen.
\item  Begreiflicher Weise aber kann man nicht immer ohne vorläufige Prüfung annehmen, daß ein gewisser Aufsatz wirklich von dem Mann herrühre, der als Verfasser darin angegeben wird. Denn es gibt mancherlei Gründe, die einen Betrüger bestimmen können, seiner eigenen Arbeit den Namen eines Anderen vorzusetzen; und auch der Zufall kann zuweilen machen, daß man sich in dem Verfasser einer namenlosen Schrift irrt, und ihr zuletzt den Namen des bloß vermeinten Verfassers vorsetzt.
\item  Wenn nun ein Buch wirklich von demjenigen verfaßt ist, der in demselben (in der Ueberschrift, Vorrede, Unterschrift,~\RWSeitenw{62}\ \udgl ) als der Verfasser angegeben wird: so nennt man es \RWbet{ächt} oder \RWbet{authentisch}. Die Aechtheit eines Buches ist also die Wahrheit der in dem Buche enthaltenen Aussage über seinen Verfasser. -- Das Gegentheil der Aechtheit ist die \RWbet{Unächtheit} oder die \RWbet{Unterschobenheit}. Ein Buch ist unterschoben, wenn es nicht wirklich von dem verfaßt ist, den es doch als Verfasser angibt.
\begin{RWanm}
Auf \RWbet{anonyme Schriften}, \dh\ auf Schriften, die ihren Verfasser nicht nennen, läßt sich daher der Begriff der Aechtheit nicht anwenden. Ist aber der Verfasser, obwohl nicht seinem \RWbet{Namen} nach, doch nach einigen \RWbet{anderen Umständen} bestimmt, \zB\ nach seiner Religion, nach seinem Aufenthaltsorte \udgl : so kann man auch bei solchen Büchern von einer Aechtheit oder Unächtheit derselben sprechen. Diese bezieht sich nämlich nur auf die angegebenen Umstände.
\end{RWanm}
\item  \RWbet{Unverfälscht} heißt ein Buch, wenn es in allen seinen Theilen ächt ist, \dh\ wenn es nicht nur im Ganzen, sondern auch in seinen \RWbet{einzelnen Theilen} von dem angeblichen Verfasser herrührt, und keine spätere Abänderung erfahren hat.
\end{aufza}

\RWpar{24}{Wie man die Aechtheit eines Buches erkenne?}
Man erkennt, daß ein Buch ächt sey, wenn man die \RWbet{Annahme der Unterschiebung} aus was immer für Gründen \RWbet{höchst unwahrscheinlich}, oder sogar \RWbet{unmöglich} findet. Wenn aber die Annahme der Unterschiebung eines Buches nichts Unmögliches enthalten soll: so werden hiezu folgende drei Stücke erfordert:
\begin{aufza}
\item  Wenn das Buch wirklich nicht von dem Manne, der auf dem Titel genannt ist, herrührt: so muß es irgend ein Anderer geschrieben haben. Man muß also zeigen, \RWbet{daß es einem Anderen nicht physisch, und nicht psychologisch unmöglich gewesen sey, dieses Werk so herzustellen}, wie es vorhanden ist, \zB\ die schon aus anderen Werken bekannte Schreibart jenes Mannes so nachzuahmen, in seine Lage sich so gut hineinzudenken, \usw
\item  Es muß begreiflich gemacht werden können, \RWbet{wie alle Menschen}, die dieses Buch für ächt halten, \RWbet{in diesen Irrthum versetzt worden sind}.~\RWSeitenw{63}
\item  Wenn es vielleicht noch \RWbet{andere Bücher} gibt, die bei der Voraussetzung der Unächtheit dieses Einen gleichfalls für unächt erklärt werden müßten: so muß man \RWbet{die Möglichkeit der beiden vorigen Stücke auch noch in Ansehung jener Bücher erweisen}. Der Fall, von dem ich hier rede, tritt aber ein, wenn wir ein Buch für unterschoben erklären, auf das sich viele andere berufen, daraus citiren \usw ; und wenn wir behaupten, daß jenes erstere zu einer Zeit geschrieben worden sey, die später ist, als die angebliche Verfassungszeit dieser übrigen Bücher. Wie \zB\ wenn Jemand mit einem gewissen Italiener bei Joh.~\RWbet{Clericus} behaupten wollte, daß die Bücher des neuen Bundes erst im fünften Jahrhunderte geschrieben worden seyen: so müßte er auch alle Schriften der Kirchenväter aus den vier ersten Jahrhunderten für unterschoben erklären.
\end{aufza}

\RWpar{25}{Etwas umständlichere Beschreibung der Art, wie man bei Untersuchung der Aechtheit eines Buches vorzugehen habe}
Aus dem Vorhergehenden ergibt sich, daß wir beiläufig auf folgende Art bei dieser Untersuchung am Zweckmäßigsten vorgehen werden.
\begin{aufza}
\item  Vor allem Anderen werden wir zu untersuchen haben, ob das Buch mit \RWbet{allen bekannten Eigenschaften und Verhältnissen des Mannes, der als Verfasser angegeben wird, übereinstimme}, \dh\ ob es nichts enthalte, was mit jenen Eigenschaften im Widerspruche steht, was jener Mann unmöglich hätte schreiben können. Die Uebereinstimmung aller inneren Eigenschaften eines Buches mit den bekannten Eigenschaften seines angeblichen Verfassers nennt man das \RWbet{innere Merkmal der Aechtheit}. Hieher gehört nun:
\begin{aufzb}
\item  Wenn man das \RWbet{Zeitalter} des angeblichen Verfassers kennt; ob keine Worte, Erfindungen, Meinungen, Sitten, Gebräuche vorkommen, die eines erweislich späteren Ursprunges sind.~\RWSeitenw{64}
\item  Wenn man die \RWbet{Landsmannschaft} desselben kennt; ob nichts erwähnt wird, was in diesem Lande unbekannt war, \udgl\ 
\item Wenn man die \RWbet{Grundsätze} und den \RWbet{Charakter} des Mannes kennt; ob nichts denselben Widersprechendes vorkommt.
\item  Wenn man aus Schriften, welche erwiesener Maßen von ihm herrühren, seine \RWbet{Schreibart} \udgl\  kennt; ob hier nicht eine ganz andere Schreibart herrsche, \usw
\end{aufzb}
\item  Fehlt einem Buche das innere Merkmal der Aechtheit: so ist schon erwiesen, daß es ein unterschobenes Werk sey. Ist aber das innere Merkmal vorhanden: so kann man demungeachtet doch nicht sofort auf seine Aechtheit schließen, sondern man muß erst weiter untersuchen, \RWbet{ob es einem Betrüger möglich gewesen wäre, seiner Arbeit dieß innere Merkmal der Aechtheit zu ertheilen}, und zwar:
\begin{aufzb}
\item  Ob es ihm \RWbet{physisch möglich} gewesen wäre, \dh\ ob es die Kräfte seines Geistes nicht überstiegen hätte. Hiebei ist zu erwägen:
\begin{aufzc}
\item Wie entfernt derjenige, den wir als wahren Verfasser annehmen wollen, von der Zeit und dem Orte des angeblichen Verfassers gelebt haben soll. Denn soll er \zB\ viel später, in einem sehr entfernten Lande, gelebt haben: so muß es ihm um so schwerer geworden seyn, sich in das Zeitalter und in die Lage des angeblichen Verfassers hineinzudenken;
\item ob das Buch unverkennbar Spuren eines gelehrten oder ungelehrten Verfassers verrathe;
\item ob es von einem großen oder geringen Umfange sey. Ein sehr kleiner Aufsatz ist leicht so abzufassen, daß er das innere Merkmal der Aechtheit habe;
\item ob es viele oder wenige Gelegenheiten, wo ein Betrüger sich verrathen könnte, darbietet; \zB\ wenn sehr viele ganz individuelle Züge von Personen \udgl\  darin vorkommen.
\end{aufzc}
\item  Ob es ihm \RWbet{psychologisch möglich} gewesen wäre, \dh\ ob sich irgend eine dem menschlichen Herzen gewöhn\RWSeitenw{65}liche Triebfeder gedenken lasse, die ihn zu diesem Betruge vermochte. Es fragt sich also:
\begin{aufzc}
\item ob das Buch nicht aus einem falschen Eifer für die gute Sache, aus frommer Betrügerei hätte unterschoben werden können? oder
\item aus irgend einem eigennützigen Beweggrunde? \usw
\end{aufzc}
\end{aufzb}
\item  Wenn wir bei allen diesen Erörterungen auf keine Unmöglichkeit stoßen: so müssen wir \RWbet{drittens} untersuchen, ob es auch möglich gewesen wäre, alle diejenigen, welche das Buch für ächt gehalten, in diesen Irrthum zu versetzen. Hiebei kommt zu betrachten:
\begin{aufzb}
\item  ob deren Viele oder Wenige sind;
\item  ob einige darunter sind, die dem Zeitalter des angeblichen Verfassers sehr nahe gelebt, ihn vielleicht gar persönlich, oder doch seine Grundsätze, Verhältnisse \usw\ genau gekannt haben;
\item  ob sie wichtige Beweggründe zu einer strengen Prüfung der Aechtheit des Buches gehabt \usw
\end{aufzb}
\item  Wenn endlich durch die Annahme der Unterschiebung eines Buches zugleich auch mehrere andere Bücher für unächt erklärt werden: so müßten wir (wie schon gesagt) die Möglichkeit der Unterschiebung auch bei diesen Büchern zeigen.
\item  Stoßen wir nun bei diesen Untersuchungen irgendwo auf eine Unmöglichkeit, die bei Voraussetzung der Unterschiebung angenommen werden müßte: so folgt im Gegentheile hieraus die Aechtheit des Buches. Entdecken wir aber keine dergleichen Unmöglichkeit, so ist die Aechtheit unerweislich.
\item  Da Unverfälschtheit nichts Anderes, als Aechtheit aller einzelnen Theile ist: so wird sie nach eben den Regeln, wie diese untersucht. Besonders hat man hier darauf zu sehen:
\begin{aufzb}
\item  ob es von dem Buche vielerlei \RWbet{Handschriften} gibt;
\item  ob diese Handschriften \RWbet{nahe an das Zeitalter} der angeblichen Entstehung des Buches hinreichen;
\item  ob sie sich in den Händen \RWbet{verschiedener Parteien} befinden, und gleichwohl übereinstimmend lauten;
\item  ob das Buch frühzeitig auch \RWbet{übersetzt} worden sey;
\item  ob sich nicht häufige \RWbet{Anführungen} (Citaten) aus diesem Buche in anderen alten Büchern finden, \usw ~\RWSeitenw{66} 
\end{aufzb}
\end{aufza}


\RWpar{26}{Es ist möglich, sich von der Aechtheit oder Unächtheit gewisser Bücher befriedigend zu überzeugen}
Durch Anwendung der eben vorgetragenen Regeln, welche in einer eigenen Wissenschaft, die \RWbet{höhere Kritik} genannt, umständlicher entwickelt werden, ist es in der That möglich, über die Aechtheit oder Unächtheit gewisser Bücher, und eben so auch über ihre Unverfälschtheit mit einer hinlänglichen Sicherheit zu entscheiden. Daß dieses wirklich sey, beweiset die Erfahrung, daß die Kritiker in jenen Urtheilen, welche sie über die Aechtheit oder Unächtheit gewisser Bücher von einem größeren Umfange fällen, beinahe durchgängig übereinstimmen. So erklären \zB\ alle Kritiker diese und jene Schriften des \RWbet{Plato, Xenophon, Cicero, Horaz,} \usw\ für ächt, dagegen andere \zB\ das \RWlat{Evangelium infantiae Jesu}, die \RWlat{constitutiones apostolicas}, die \RWlat{decretales Isidori peccatoris}\RWlit{}{DecretalesIsidori1}, für unterschoben. Diese so große Uebereinstimmung kann nicht zufällig seyn, sondern muß irgend einen in der Natur der Sache liegenden Grund haben. Parteilichkeit, die alle Kritiker zu einer und derselben Unwahrheit vereinigt hätte, läßt sich bei Büchern gewisser Art nicht denken. Der Grund kann also nur darin liegen, daß jene Gelehrten wirklich im Stande sind, nach gehöriger Untersuchung ein richtiges Urtheil über die Aechtheit oder Unächtheit eines Buches zu fällen. Diese Erklärung ist um so zuverlässiger, da eine fernere Erfahrung lehrt, daß jene wenigen Fälle, wo sich die Kritiker in ihren Urtheilen gar nicht vereinigen können, fast insgemein nur bei Schriften Statt haben, die von geringem Umfange sind, oder bei deren Beurtheilung sich die Leidenschaft in's Spiel mengt.

\RWpar{27}{Auch Wunder können historisch beglaubiget werden}
\begin{aufza}
\item  Die Frage, \RWbet{ob auch Wunder historisch beglaubiget werden können}, \dh\ ob ein bloß \RWbet{menschliches} Zeugniß unter gewissen Umständen verlässig genug werden könne, um uns selbst zu dem Glauben, daß sich ein \RWbet{Wunder} zugetragen habe, zu bestimmen, ist für die Sache der~\RWSeitenw{67}\ göttlichen Offenbarung von einer solchen Wichtigkeit, daß wir zuvörderst nachsehen wollen, ob, und auf welche Art schon der \RWbet{gemeine Menschenverstand} hierüber entscheide. Die Geschichte der verschiedenen Religionen auf Erden zeigt nun, daß man zu allen Zeiten und bei allen Völkern an mancherlei Wunder geglaubt habe, und zwar nicht immer nur an solche, welche man selbst gesehen haben wollte, sondern mitunter auch an Wunder, die nur von Andern waren bezeuget worden. Dieß hätte man nicht thun können, wenn man nicht allenthalben vorausgesetzt hätte, daß auch das bloße Zeugniß eines andern Menschen zuweilen verlässig genug werden könne, um selbst ein Wunder glaubwürdig zu machen. Wir haben also über die obige Frage ein Urtheil, in welchem \RWbet{alle Menschen} auf dem ganzen Erdenrunde übereinstimmen. Es fragt sich nur noch, ob dieses Urtheil auch alle diejenigen Beschaffenheiten habe, welche (1.\,Hptthl.\ \RWparnr{14}\ 7.) einem Urtheile des gemeinen Menschenverstandes zukommen müssen, wenn es unfehlbar seyn soll.
\begin{aufzb}
\item  Ob das \RWbet{bestimmte} Wunder, das Dieser oder Jener erzählt, glaubwürdig sey, das allerdings ist eine Frage, zu deren gehöriger Beurtheilung Erfahrungen und Untersuchungen nothwendig sind, welche nicht jedem Menschen zu Gebote stehen. Allein die Frage, ob es nur \RWbet{überhaupt} möglich sey, daß uns ein Wunder durch die Aussage eines Zeugen beglaubigt werden könne, fordert zu ihrer Entscheidung nichts als Vernunft und Erfahrungen von einer solchen Art, wie sie ein jeder Mensch zu machen Gelegenheit hat.
\item  Es ist dieß ferner eine Frage, die nicht bloß für einige, sondern für alle Menschen von Wichtigkeit ist.
\item  Und da man sie allgemein \RWbet{bejahend} entschied, und zu Folge dieser bejahenden Entscheidung an das Geschehenseyn gewisser Wunder glaubte, und somit auch die Lehren, zu deren Bestätigung sie sich sollten zugetragen haben, als göttliche Offenbarungen annahm: so läßt sich nicht sagen, daß die Art, wie der gemeine Menschenverstand jene Frage entschied, den menschlichen Neigungen schmeichle. Aus Sinnlichkeit würde der Mensch vielmehr wünschen, die Frage \RWbet{verneinen} zu können, weil er sich so mit~\RWSeitenw{68}\ einem Male von aller Verbindlichkeit zur Annahme einer göttlichen Offenbarung lossagen würde. Es kann sonach nichts Anderes, als die einleuchtende Macht der Wahrheit selbst seyn, was alle Menschen bestimmte, zuzugestehen, daß Wunder allerdings auch durch bloße Zeugnisse glaubwürdig dargethan werden können. Eine Philosophie also, die dieß nicht zugeben will, macht sich im höchsten Grade \RWbet{verdächtig;} zumal da die Erfahrung zeigt, daß selbst jene wenigen Gelehrten, die keine Wunder auf fremdes Zeugniß annehmen wollten, doch manche andere Begebenheiten, die eine eben so große, wenn nicht noch größere innere Unwahrscheinlichkeit haben, als Wunder, auf bloßes Zeugniß glauben, \zB\ daß von Zeit zu Zeit Steine vom Himmel herabgefallen seyen, \udgl\  
\end{aufzb}
\item  Nachdem wir nun wissen, daß jene Frage bejahend zu entscheiden sey: so lasset uns auch noch den \RWbet{eigentlichen Grund, warum} dieß sey, erfahren. Es fragt sich also, wie man nach den nothwendigen Gesetzen des Verstandes veranlaßt werden könne, das Geschehenseyn eines Wunders, das die Geschichte meldet, anzunehmen.
\begin{aufzb}
\item  Alle \RWbet{historischen} Urtheile gründen sich auf gewisse, so eben gegenwärtige \RWbet{Erscheinungen} oder \RWbet{Wahrnehmungen}, zu deren Erklärung wir einen gewissen Gegenstand oder ein gewisses Ereigniß in der Vergangenheit annehmen. Setze man nun den Fall, der allerdings möglich ist, daß man eine gewisse Empfindung x habe, deren Vorhandenseyn sich schlechterdings nicht aus lauter gewöhnlichen Ereignissen, als seiner Ursache, ableiten läßt; setze man, irgendwo müsse hier eine ungewöhnliche Begebenheit als Ursache der gegenwärtigen Erscheinung angenommen werden; diese ungewöhnliche Begebenheit mag nun das Wunder \RWbet{seyn}, von dem man uns wirklich erzählt, oder wir mögen annehmen, daß die Erzählung \RWbet{lüge}, in welchem Falle es sich wieder nicht, ohne ein ganz \RWbet{ungewöhnliches} Ereigniß vorauszusetzen, erklären lassen soll, wie diese Lüge aufgekommen sey. Ein jedes ungewöhnliche Ereigniß, bloß als ein solches betrachtet, ist \RWbet{unwahrscheinlich}, und kann es sogar in einem~\RWSeitenw{69}\  \RWbet{unendlichen} Grade seyn, wenn eine unendliche, oder doch unzählbare Menge sonst gemachter Beobachtungen den entgegengesetzten Fall erwarten läßt. Von dieser Art ist \zB\ die Auferstehung eines Todten; denn weil man eine unzählige Menge von Beobachtungen gemacht hat, daß Todte, \zB\ am vierten Tage nach ihrem Absterben, nicht wieder zu sich kommen: so ist die Erwartung, daß dieses in einem gewissen Falle gleichwohl geschehen werde, an und für sich betrachtet, unendlich unwahrscheinlich. Allein auch das unendlich Unwahrscheinliche hat seine verschiedenen Grade; das Eine kann zwei, drei, ja unendliche Male unwahrscheinlicher seyn, als das Andere. Daher läßt sich denn gar wohl der Fall gedenken, daß unter allen ungewöhnlichen Ereignissen, deren Eines man als Ursache unserer gegenwärtigen Empfindung annehmen muß, gerade jenes \RWbet{Wunder}, von welchem die Geschichte meldet, noch den \RWbet{geringsten} Grad der Unwahrscheinlichkeit habe, daß jede andere Voraussetzung noch unendlich unwahrscheinlicher wäre. In diesem Falle also wäre es \RWbet{moralisch gewiß}, daß sich das Wunder zugetragen habe.
\item  Und dieser Fall kann sich um desto leichter ereignen, da eigentliche Wunder bei Weitem \RWbet{nicht so unwahrscheinlich sind, als andere ungewöhnliche Ereignisse}, die nicht zugleich auch Wunder, \dh\ nicht Zeichen einer an uns ergangenen göttlichen Offenbarung sind. Denn eine göttliche Offenbarung ist, wie wir wissen, nicht bloß möglich; sondern selbst wahrscheinlich. Da aber Wunder zu ihrer Bestätigung \RWbet{nothwendig} sind: so ist es gleichfalls wahrscheinlich, daß sich irgendwo Wunder ereignet haben. Treffen wir also einen religiösen Lehrbegriff an, der einen hohen Grad sittlicher Zuträglichkeit für uns hat: so wird es überaus wahrscheinlich, daß dieser eine wahre göttliche Offenbarung seyn, und somit auch durch eigene Wunder von Gott bestätiget seyn dürfte. Dieser Umstand vermindert also gar sehr die Unwahrscheinlichkeit, welche die Wunder in einer anderen Rücksicht, nämlich als ungewöhnliche Ereignisse, haben.
\item  In wirklicher Rechnung ließe sich dieses so nachweisen. Gesetzt, es wäre die innere Unwahrscheinlichkeit eines~\RWSeitenw{70}\ bestimmten Wunders $= \frac{n}{\infty}$; die Unwahrscheinlichkeit jeder anderen Voraussetzung aber, welche man machen muß, wenn die Geschichte lügt, wäre noch viel größer, \zB\ $= \frac{n}{m \infty}$, oder wohl gar $\frac{n}{\infty^2}$. In diesem Falle wäre denn also die Wahrscheinlichkeit, mit der wir annehmen müssen, daß sich das Wunder zugetragen habe nach der Formel $\frac{x}{x+y} =  \frac{\frac{n}{\infty}}{\frac{n}{\infty}+\frac{n}{m \infty}} =  \frac{m}{m+1}$, welches, wenn $m$ sehr groß ist, der Einheit so nahe kommen kann, als man nur immer will.
\end{aufzb}
\end{aufza}

\RWpar{28}{Prüfung des Gegengrundes, und wie viel man ihm einräumen müsse}

\begin{center}\RWbet{Einwurf.}\end{center}

Die Glaubwürdigkeit eines Zeugen kann nie so groß seyn, daß sie die Unwahrscheinlichkeit des Wunders aufwiege; denn die Annahme, daß ein Zeuge lüge, kann nie eine unendlich große Unwahrscheinlichkeit haben, wie sie das Wunder hat.\par

\begin{center}\RWbet{Antwort.}\end{center}

Ich gebe das zu, wenn sich das Zutrauen, das wir zu der Aussage eines Zeugen fassen, auf seine \RWbet{Wahrheitsliebe} stützen soll; denn unser Zutrauen zu der Wahrheitsliebe eines Menschen dürfte freilich nie einen unendlich hohen Grad der Wahrscheinlichkeit haben; denn von der Wahrheitsliebe eines Menschen können wir doch nicht anders, als durch eine \RWbet{endliche} (fast immer \RWbet{sehr kleine}) Menge von Beobachtungen überzeugt worden seyn; der moralische Charakter eines Menschen ist auch Veränderungen unterworfen, und derjenige, der bisher immer als ein redlicher Mann gehandelt, kann es doch in der Folge zu seyn aufhören. Wenn also das Geschehenseyn eines Wunders bloß auf die Wahrheitsliebe eines Zeugen angenommen werden sollte, dann wäre freilich die Voraussetzung, daß der Zeuge lüge, \RWbet{minder}~\RWSeitenw{71}\ \RWbet{unwahrscheinlich}, als jene des Wunders. Allein sehr unrichtiger Weise stellt man die Sache so vor, als ob sich das Zutrauen zu der Aussage eines Zeugen immer nur auf seine Wahrheitsliebe gründe; es kann sich auch auf den Umstand gründen, daß es dem Zeugen in seinen Verhältnissen \RWbet{psychologisch unmöglich} gewesen sey, zu lügen. Dieses ist nämlich der Fall, wenn weder die Vernunft, noch der Glückseligkeitstrieb ihm einen Beweggrund zur Lüge geben konnten; und dann kann allerdings die Annahme der Lüge eine weit unwahrscheinlichere Voraussetzung seyn, als die Annahme des Wunders selbst.

\RWpar{29}{Einige Erfordernisse zur historischen Glaubwürdigkeit eines Wunders}
Aus dem bisher Gesagten läßt sich die Richtigkeit folgender Sätze, die man gewöhnlich aufzustellen pflegt, erweisen:
\begin{aufza}
\item  Ein Wunder, welches glaubwürdig seyn soll, darf nicht bloß auf die vorausgesetzte Wahrheitsliebe des Zeugen angenommen werden.
\item  Die Erzählung eines Wunders darf nicht auf einer bloßen Volkssage von unbekanntem Ursprunge beruhen.
\item  Es muß nicht in Geheim oder an einem verborgenen Orte gewirkt worden seyn; \usw
\end{aufza}

\RWpar{30}{Ob historische Urtheile, besonders solche, wie sie der Glaube an eine Offenbarung erfordert, eben den Grad der Gewißheit, wie Urtheile a priori, ersteigen können?}
\begin{aufza}
\item  Man hört so häufig der \RWbet{Religion} (einer \RWbet{geoffenbarten} nämlich) den Vorwurf machen, daß sie sich auf \RWbet{Geschichte}, und zwar auf \RWbet{Wundererzählungen} stütze. Aus diesem Grunde, sagt man, sey bei ihr höchstens nur \RWbet{Wahrscheinlichkeit} zu erreichen, nie eine völlige \RWbet{Gewißheit}, wie sie in apriorischen Wissenschaften, namentlich in der Mathematik sich finde, wo Alles mit apodiktischer Strenge erwiesen werden kann.~\RWSeitenw{72}
\item  Es wird nicht überflüßig seyn, diesen Vorwurf näher zu betrachten. Denn wenn es sich wirklich so verhielte: so wäre es doch in der That traurig, daß wir uns gerade bei unseren \RWbet{wichtigsten Erkenntnissen} mit einem geringeren Grade der Gewißheit behelfen müßten, als es derjenige ist, dessen wir uns in anderen Erkenntnissen erfreuen.
\item  Aber so ist es zum Glücke nicht. Denn ob es gleich wahr ist, daß alle historische Urtheile ihrer Natur nach nur Wahrscheinlichkeitsurtheile sind: so können, wie wir schon oben sahen, und wie es Jedem auch sein eigenes Gefühl beweiset, auch dergleichen Urtheile einen so hohen Grad von Wahrscheinlichkeit ersteigen, daß wir uns bei demselben völlig beruhigen können. Dagegen muß man sich erinnern, daß auch \RWbet{alle apriorischen} Wahrheiten, wenigstens alle, welche auf einer \RWbet{längeren Reihe von Schlüssen} beruhen, nur mit Wahrscheinlichkeit von uns erkannt werden, und folglich, als Erkenntnisse betrachtet, nicht das Geringste vor den historischen Wahrheiten voraus haben. Bei allen Wahrheiten nämlich, welche durch eine längere Reihe von Schlüssen hergeleitet werden müssen, sie mögen empirische oder reine Begriffswahrheiten seyn, bleibt es immer möglich, daß wir uns in der \RWbet{Herleitung} irren; wie denn dieß in der That nur allzu häufig geschieht. Die Annahme nun, daß wir uns \RWbet{nicht geirrt} haben, hat immer nur einen mehr oder weniger hohen Grad von \RWbet{Wahrscheinlichkeit}, der von folgenden Umständen abhängt:
\begin{aufzb}
\item  Von der längeren oder kürzeren Reihe von Schlüssen, vermittelst welcher die Wahrheit abgeleitet wird.
\item  Wie oft wir diese Schlußreihe durchgegangen sind.
\item  Wie viele andere Menschen diese Schlüsse geprüft, und mit uns richtig gefunden.
\item  Wenn sich die Wahrheit durch Erfahrung und Versuche bestätigen läßt, ob und wie viele dergleichen Erfahrungen und Versuche wir bereits angestellt haben; \usw
\end{aufzb}
\item  Wenn nun die Mathematik nicht zwar in allen, aber doch in sehr vielen ihrer Lehren einen überaus hohen Grad der Gewißheit gewähret; so kommt dieß lediglich daher, weil~\RWSeitenw{73}
\begin{aufzb}
\item  die Schlußreihe, die zur Herleitung vieler ihrer Sätze hinreicht, nicht eben sehr lang ist; weil wir
\item  diese Schlüsse vielfältig durchgegangen sind, und
\item  nicht wir allein, sondern auch Tausende mit uns, und alle dasselbe Resultat gefunden haben, und zu demselben Schlußsatze gekommen sind,\RWfootnote{%
Dieß ist um so leichter, weil die Mathematiker eine \RWbet{sehr kurze Sprache} erfunden und eingeführt haben. Hätte die Philosophie diesen wichtigen Vortheil: so würde unter den Philosophen wohl eben so viel Einigkeit herrschen, als unter den Mathematikern herrschet.}
ein Umstand, der zum Theile auch wieder daher rührt, weil keine dem menschlichen Herzen gewöhnliche Triebfeder da ist, die sich der Anerkennung der mathematischen Urtheile widersetzte; weil endlich
\item  auch viele dieser Sätze durch mancherlei Erfahrungen und Versuche geprüft werden können, und wirklich geprüft worden sind.
\end{aufzb}
\item  Ist diese Ansicht von dem \RWbet{Grunde} der Gewißheit, die wir in unsern Urtheilen haben, richtig: so ist leicht zu erachten, daß man nicht eben Ursache habe, zwischen \RWbet{historischen} und \RWbet{apriorischen} Erkenntnissen einen solchen Gegensatz zu machen, und zu behaupten, daß jene diesen in Hinsicht auf Gewißheit jederzeit nachstehen müßten; denn was uns in einigen unserer Erkenntnisse \RWlat{a priori} so zuversichtlich macht, ist ja doch abermals nur \RWbet{Erfahrung}. Oder ist's nicht Erfahrung, wenn wir uns darauf berufen, daß wir eine gewisse Reihe von Schlüssen schon mehrmals durchgegangen wären? daß sie auch Andere geprüft, und gleich uns richtig befunden hätten? \usw
\item  Hieraus ergibt sich ferner, daß auch \RWbet{historische} Erkenntnisse unter gewissen Umständen einen Grad der Gewißheit ersteigen können, der selbst einem mathematischen Wissen nicht nachsteht. Daß es ein Troja einst gegeben habe, ist eine bloß historische Behauptung, die gleichwohl jeder Vernünftige ganz so gewiß finden wird, als etwa den Pythagoräischen Lehrsatz vom Quadrate der Hypothenuse.
\item  Aber wird auch der Glaube an eine Offenbarung, zumal an eine solche, die vor Jahrtausenden schon gegeben~\RWSeitenw{74}\ worden ist, und deren Wunder sich auf uralte Zeugnisse stützen, einen solchen Grad der Zuverlässigkeit ersteigen können?
\item  \RWbet{Einen eben so hohen}, antworte ich, \RWbet{wo nicht einen noch höheren}. Denn was ist dazu nöthig, damit der Glaube an eine solche Offenbarung nach sehr vernünftigen Gründen in uns entstehen könne? Nichts Anderes, als: daß wir uns
\begin{aufzb}
\item  eine hinlänglich sichere Kenntniß von dem \RWbet{Inhalte} dieser angeblichen Offenbarung verschaffen; dann uns
\item  überzeugen, daß diese Lehren das Merkmal \RWbet{sittlicher Zuträglichkeit} für uns besitzen; und daß wir es
\item  gewissen \RWbet{außerordentlichen Begebenheiten} verdanken, mit ihnen bekannt geworden zu seyn.
\end{aufzb}
\item  Was nun zuerst den \RWbet{Inhalt} der Offenbarung anlangt: so ist es nicht zu läugnen, daß eine sichere Erkenntniß desselben mit bedeutenden Schwierigkeiten verbunden seyn dürfte, wenn wir auf irgend ein altes, in einer ausgestorbenen Sprache verfaßtes \RWbet{Buch} verwiesen würden, aus dem wir entziffern sollten, was Gott geoffenbart habe. Aber könnte die Weisheit Gottes nicht eine andere Einrichtung treffen? Wie, wenn es hieße, daß der Inhalt der göttlichen Offenbarung nicht eben in einem Buche niedergelegt sey, sondern in dem \RWbet{Gesammtglauben} eines gewissen Volkes noch immer unverfälscht fortlebe? Dann wäre es wahrlich nichts Schweres, den Inhalt der göttlichen Offenbarung mit aller Sicherheit kennen zu lernen. Denn was ein ganzes Volk \RWbet{glaube}, oder vielmehr, wozu es sich einstimmig bekenne; das läßt sich wohl sehr leicht und sicher erfahren; das ist nicht schwieriger zu erfahren, als was alle Mathematiker in Betreff dieses oder jenes mathematischen Gegenstandes lehren. Auf jeden Fall dürften wir aber nicht vergessen, daß wir, so ferne wir nur unser Möglichstes thun, um zur Erkenntniß des vollständigsten Inhaltes der göttlichen Offenbarung zu gelangen, schon ganz beruhiget seyn können, daß uns Gott nicht zu unserem Nachtheile hier werde irren lassen.
\item  Was nun das Zweite oder die \RWbet{Prüfung der sittlichen Zuträglichkeit} der vorgefundenen Lehren betrifft: so ist wohl nichts leichter, als zu beurtheilen, ob~\RWSeitenw{75}\ gewisse Lehren, wenn wir sie gläubig annehmen könnten, wohlthätig oder nachtheilig auf unsere Tugend und Glückseligkeit einwirken würden. Bloß durch die \RWbet{mehrmalige Vorstellung} dieser Lehren werden wir inne, ob sie uns zuträglich oder nachtheilig seyn werden, auch wenn wir nicht im Stande sind, die Art ihres Einflusses in Worten anzugeben. Und wenn wir es erst versuchen, nach Lehren, die sich uns als wohlthätig darstellen, einige Zeit zu leben: so überzeugt uns die Erfahrung selbst mit jedem Tage mehr, was wir noch in der Zukunft von ihnen zu erwarten haben. Uebrigens gilt die vorhin gemachte Bemerkung auch hier wieder.
\item  Das Dritte endlich, oder die \RWbet{Untersuchung der Wunder}, die zur Bestätigung dieser Lehren dienen sollen, kann durch den Umstand, daß diese Wunder schon vor Jahrtausenden sich zugetragen haben, nur in sofern erschwert werden, als wir neugierig genug sind, den eigentlichen \RWbet{Hergang} derselben kennen zu lernen. Dieses wird aber zu unserer Beruhigung, daß wir hier eine wahre göttliche Offenbarung vor uns haben, gar nicht erfordert; sondern zu diesem Zwecke genügt es vollkommen, wenn wir nur die \RWbet{Erfahrung} machen, \RWbet{daß es auf keine Art uns gelingen wolle, die vor uns liegenden Wundererzählungen zu erklären, ohne irgend eine außerordentliche Begebenheit, von welcher Beschaffenheit sie auch immer seyn mag, vorauszusetzen}. Diese Erfahrung nun wird durch das Alter der zu prüfenden Erzählung weder erschweret noch erleichtert; sie fordert nichts, als einige Mühe, und in Betreff der Frage, ob wir denn diese Erfahrung wirklich gemacht oder nicht gemacht haben, ist gar kein Irrthum möglich.
\item  Und so ersehen wir denn, daß der Glaube an eine göttliche Offenbarung das Eigene hat, daß er, obschon auf historischen Sätzen von ungewisser Art beruhend, doch diese Ungewißheit nicht mit ihnen theilet, weil es sich hier durchgängig nicht darum handelt, wie die Sache an sich sey, sondern nur darum, wie sie uns erscheinet. \RWbet{Was sich uns nach gehöriger Prüfung als göttliche Offenbarung darstellt, ist es auch in der Wahrheit}.~\RWSeitenw{76}
\end{aufza}

\RWch{Drittes Hauptstück.\\ Aechtheit, Unverfälschtheit und Glaubwürdigkeit der Bücher des neuen Bundes.}
\RWpar{31}{Inhalt und Zweck dieses Hauptstückes}
\begin{aufza}
\item Unter dem Namen der Bücher des neuen Bundes (\RWgriech{Kain`h Diaj'hkh}) versteht man folgende 27 Aufsätze in griechischer Sprache, die unter den Christen allgemein bekannt sind.
\begin{aufzb}
\item \RWbet{Vier Evangelien} (\dh\ Lebensgeschichten Jesu), deren Ueberschriften sind: Das Evangelium nach \RWbet{Matthäus}\bindex{Mt}, nach \RWbet{Markus}\bindex{Mk}, nach \RWbet{Lukas}\bindex{Lk}, nach \RWbet{Johannes}\bindex{Joh}.
\item \RWbet{Ein} Buch mit der Ueberschrift: die Thaten der heiligen Apostel, von \RWbet{Lukas}\bindex{Apg} geschrieben.
\item \RWbet{Vierzehn Briefe Pauli}, als: 1.~an die \RWbet{Römer}\bindex{Röm}; 2.~und 3.~an die \RWbet{Korinther}\bindex{1\,Kor}\bindex{2\,Kor}; 4.~an die \RWbet{Galater}\bindex{Gal}; 5.~an die \RWbet{Epheser}\bindex{Eph}; 6.~an die \RWbet{Philippenser}\bindex{Phil}; 7.~an die \RWbet{Kolosser}\bindex{Kol}; 8.~und 9.~an die \RWbet{Thessalonicenser}\bindex{1\,Thess}\bindex{2\,Thess}; 10.~und 11.~an \RWbet{Timotheus}\bindex{1\,Tim}\bindex{2\,Tim}; 12.~an \RWbet{Titus}\bindex{Tit}; 13.~an \RWbet{Philemon}\bindex{Plmn}; 14.~an die \RWbet{Hebräer}\bindex{Hebr}.
\item \RWbet{Sieben} sogenannte \RWbet{katholische Briefe}, als: Ein Brief \RWbet{Jacobi}\bindex{Jak}; zwei Briefe \RWbet{Petri}\bindex{1\,Petr}\bindex{2\,Petr}; drei Briefe \RWbet{Johannis}\bindex{1\,Joh}\bindex{2\,Joh}\bindex{3\,Joh}, und Ein Brief \RWbet{Judä}\bindex{Jud}.
\item Noch ein Buch mit der Ueberschrift: Die \RWbet{Offenbarung Johannis des Gottesgelehrten}\bindex{Offb} (\RWgriech[to~u Jeologo~u]{to~u Jeol'ogou}).
\end{aufzb}
\item Aus diesen Schriften, besonders aus den \RWbet{fünf} ersteren, die rein historischen Inhaltes sind, kann man die vollständigsten Nachrichten, wie es mit der \RWbet{Entstehung} des~\RWSeitenw{77}\ Christenthums hergegangen, und durch welche Wunder dasselbe gleich Anfangs \RWbet{bestätiget} worden sey, schöpfen. Um uns nun auf die Nachrichten, welche uns diese Bücher ertheilen, verlassen zu können, will ich erst ihre \RWbet{historische Glaubwürdigkeit} beweisen. Zwar können wir uns, wie ich dieß in der Folge zu zeigen gedenke, auch ohne diese Bücher aus \RWbet{anderen Quellen} von der Wahrheit überzeugen, daß zur Bestätigung des Christenthums außerordentliche Begebenheiten gewirkt worden seyen; aber wenn wir eine \RWbet{ausführliche Beschreibung} von der Beschaffenheit dieser Ereignisse verlangen: dann müssen wir allerdings unsere Zuflucht zu diesen Büchern nehmen.
\item Doch die Glaubwürdigkeit dieser Bücher ist nach einer ganz anderen Hinsicht von großer Wichtigkeit für uns. In den vier Evangelien finden wir eine sehr ausführliche Beschreibung der \RWbet{Gesinnungen, Thaten} und \RWbet{Schicksale Jesu}, \dh\ des Vollkommensten aus allen Sterblichen. Auch schon aus diesem Grunde also, nämlich um zu erfahren, wie der Vollkommenste aus allen Sterblichen gelebt, um uns nach seinem Beispiele selbst bilden zu können, muß es uns wichtig seyn, den Grad der Glaubwürdigkeit, den jene Bücher haben, kennen zu lernen.
\item Zu diesem doppelten Zwecke wird denn in diesem Hauptstücke die \RWbet{Glaubwürdigkeit} der Bücher des neuen Bundes, vornehmlich der fünf historischen in Kürze untersucht.
\item Da aber die Glaubwürdigkeit dieser Bücher nicht füglich erwiesen werden könnte, wenn nicht erst ihre \RWbet{Aechtheit} und \RWbet{Unverfälschtheit} außer Zweifel gesetzt ist, so wird auch diese dargethan werden.
\end{aufza}

\RWpar{32}{Die Aechtheit und Unverfälschtheit der Bücher des neuen Bundes aus dem einstimmigen Urtheile der Kritiker gefolgert}
Ich habe bereits (\RWparnr{26}) gezeigt, daß Kritiker, besonders bei Schriften von einem größeren Umfange, allerdings im Stande sind, über ihre Aechtheit und Unverfälschtheit ein~\RWSeitenw{78}\ verlässiges Urtheil zu fällen. Die heiligen Bücher des neuen Bundes nun werden von allen Kritikern (bis auf einige sehr unbedeutende Ausnahmen) für ächt und unverfälscht erklärt. Dieß übereinstimmende Urtheil kann um so weniger ein \RWbet{zufälliger Irrthum} seyn, da man kein anderes Buch mit einer solchen Sorgfalt untersucht hat, als eben diese Bücher. Auch nicht \RWbet{Parteilichkeit} kann die Ursache seyn; denn unter diesen Kritikern gibt es ja mehrere, die sich gewiß durch keine eigennützige Rücksicht, durch Menschenfurcht \udgl\  hätten abhalten lassen, die Wahrheit freimüthig zu gestehen; ja es gibt Einige, die ihren Vortheil dabei gefunden haben würden, wenn sie die Unterschiebung oder Verfälschtheit der heiligen Schrift hätten beweisen können.

\RWpar{33}{Angebliche Verfasser dieser Bücher, und welche Beschaffenheiten wir von denselben kennen}
Doch nicht auf das Zeugniß der Kritiker allein wollen wir hier die Aechtheit und Unverfälschtheit der Bücher des neuen Bundes annehmen; sondern zu desto vollkommnerer Beruhigung laßt uns die \RWbet{Gründe}, auf welche sich diese Gelehrte bei ihrem Urtheile stützen, selbst kennen lernen. Vernehmen wir also zuerst, \RWbet{welche Personen} man uns als die Verfasser dieser Bücher angibt. Nach der \RWbet{in diesen Büchern selbst} enthaltenen Angabe sollen die Verfasser derselben
\begin{aufza}
\item \RWbet{acht verschiedene Personen} seyn, Namens: \RWbet{Matthäus, Markus, Lukas, Johannes, Paulus, Jakobus, Petrus} und \RWbet{Judas Thaddäus}. Diese Männer sollen sämmtlich
\item im \RWbet{ersten Jahrhunderte} der christlichen Zeitrechnung gelebt haben; ferner
\item bis auf Lukas \RWbet{geborne Juden}, und
\item größtentheils \RWbet{Augenzeugen} derjenigen Begebenheiten gewesen seyn, welche sie uns erzählen. \RWbet{Matthäus} nämlich, \RWbet{Johannes, Jakobus, Petrus} und \RWbet{Judas} sollen aus der Zahl jener zwölf sogenannten \RWbet{Apostel} oder beständigen Begleiter gewesen seyn, die unser Herr gleich im~\RWSeitenw{79}\ Anfange seines Lehramtes sich in der Absicht beigesellet hat, damit sie \RWbet{Zeugen} seiner Reden und Thaten werden möchten. \RWbet{Markus} dagegen soll unter Anleitung Petri sein Evangelium geschrieben haben; \RWbet{Lukas} ein Begleiter Pauli auf seinen Reisen, und also in denjenigen Begebenheiten, welche er in der Apostelgeschichte erzählt, größtentheils Augenzeuge gewesen seyn. Sie sollen
\item mit Ausnahme Pauli und Lucä \RWbet{ungelehrte Leute} gewesen seyn. \RWbet{Matthäus} ein Zolleinnehmer, \RWbet{Johannes} und \RWbet{Petrus} Fischer, \RWbet{Lukas} dagegen soll ein Arzt und \RWbet{Paulus} ein in der damaligen Gelehrsamkeit der Rabbinen wohl unterrichteter \RWbet{Jude}, aus Tharsus in Cilicien gebürtig, ein Schüler \RWbet{Gamaliel's} gewesen seyn, der aus einem hitzigen Verfolger des Christenthums plötzlich in einen eifrigen Vertheidiger desselben umgeschaffen wurde.
\end{aufza}

\RWpar{34}{Die Bücher des neuen Bundes haben das innere Merkmal der Aechtheit}
Mit diesen so eben aufgezählten Beschaffenheiten ihrer angeblichen Verfasser stimmt die innere Beschaffenheit der Bücher des neuen Bundes vollkommen überein.
\begin{aufza}
\item Es sollen \RWbet{acht verschiedene Personen} gewesen seyn. Wirklich herrscht in diesen Büchern nicht durchaus einerlei Styl und Charakter. Diejenigen, welche angeblicher Maßen von einerlei Verfasser herrühren sollen, \zB\ das \RWbet{Evangelium Lucä} und die \RWbet{Apostelgeschichte}, beweisen auch einerlei Schreibart; diejenigen aber, die von verschiedenen Verfassern herrühren sollen, unterscheiden sich wirklich so, daß ein aufmerksamer Leser, auch ohne es zu wissen, wem diese Bücher zugeschrieben werden, das Urtheil fällen würde: sie seyen von \RWbet{mehreren}, etwa von \RWbet{acht} verschiedenen Personen aufgesetzt. Im Evangelio \RWbet{Matthäi} herrscht ein ungezwungener, höchst einfacher Erzählungston, ohne alle eingestreute Bemerkungen, es sey denn die einzige, daß auf diese Weise in Erfüllung gegangen sey, was in den Büchern des alten Bundes an diesem und jenem Orte geweissaget worden ist. Die Sprache selbst ist voll Hebraismen. Bei \RWbet{Markus}~\RWSeitenw{80}\ findet man dergleichen Hinweisungen auf das alte Testament weit seltener, die Schreibart aber ist noch härter; und der Verfasser kennt beinahe keine anderen Uebergänge von einer Erzählung zur anderen, als ein: \RWgriech{Ka`i} oder \RWgriech{ka`i e>uj'ews}, so, daß er fast eine jede Periode mit diesen Worten anfängt. Hebräische Gebräuche und Worte erklärt er \zB\ \RWgriech{korb~an, <'o >estin d~wron}\editorischeanmerkung{So in Mk 7,\,11. \Alabel\ schreibt \RWgriech{korb~an, <'o >estu d~wron}.} (vergl.\ mit \RWbibel{Mk}{Mark.}{7}{2}\ \RWbibel{Mt}{Matth.}{5}{1}); braucht auch zuweilen lateinische Worte, \zB\ \RWgriech{kentur'iwn}, \uam\ Schon etwas gelehrter und zierlicher ist das \RWbet{dritte Evangelium} geschrieben. Die Sprache ist hier von Hebraismen größtentheils gereinigt und nähert sich der attischen Mundart; die Erzählungsart ist gebildet, die Auswahl der Worte gelehrter und bestimmter, die Krankheiten werden mit eben denselben \RWbet{technischen} Namen bezeichnet, die sie auch bei Hippokrates führen, \zB\ \RWbibel{Lk}{Luk.}{14}{2} \RWgriech{<udrwpik`os} \ua\ Dieselbe Schreibart herrscht auch in der \RWbet{Apostelgeschichte}. So wird \zB\ \RWbibel{Apg}{Apostelg.}{12}{23} die Krankheit des Königs Herodes mit ihrem technischen Namen ein Würmerfraß genannt, \RWgriech{ka`i gen'omenos skwlhk'obrwtos >ex'eyuxen}. Das Evangelium \RWbet{Johannis} zeichnet sich, so wie auch die \RWbet{Briefe} dieses Namens, vor allen übrigen Schriften des neuen Bundes durch den überall sichtbaren Hang zu eben so sanften als tiefen Rührungen aus. \RWbet{Pauli} Briefe verrathen Einen und denselben talentvollen Verfasser; überall herrscht derselbe Scharfsinn in den Beweisen, dieselbe Fülle der Ideen, dasselbe Feuer, \usw\ Nur in dem Briefe an die \RWbet{Hebräer} möchte man eine etwas reinere Schreibart, als in den übrigen, bemerken wollen.
\item Sie sollen im \RWbet{ersten Jahrhunderte} gelebt haben. -- Nicht das Geringste kommt in diesen weitläufigen Büchern vor, welches ein späteres Zeitalter verriethe. So zahlreich auch die historischen, geographischen, politischen Nachrichten und Beziehungen sind, welche in diesen Büchern auf jeder Seite vorkommen; so stimmen doch alle ganz mit dem ersten christlichen Jahrhunderte überein, und schildern es so, wie wir es auch aus anderen Schriftstellern kennen. Dergleichen sind \RWbet{Suetonius, Tacitus} \ua\ römische Geschichtschreiber jener Zeit; vor Allen aber der jüdische Geschichtschreiber \RWbet{Flavius Josephus}. Die Beschreibungen, welche~\RWSeitenw{81}\ uns diese Schriftsteller von Palästina, von den damaligen Einrichtungen, Gebräuchen und Sitten daselbst, von Kleinasien, und allen anderen Schauplätzen der biblischen Begebenheiten machen, die Ereignisse, die sie in diese Zeit setzen \usw\  Alles stimmt mit den Büchern des neuen Bundes zusammen. Viele dieser Bestimmungen betreffen oft sehr geringfügige Umstände.
\begin{RWanm} 
Aus einigen Stellen hat man gleichwohl einen späteren Ursprung der Evangelien vermuthen wollen.
\begin{aufzb}
\item Bei \RWbibel{Mt}{Matth.}{23}{35}\ sagt Jesus: \erganf{Alles Blut, das auf der Erde unschuldig vergossen worden ist, wird über euch kommen, anzufangen vom Blute Abels des Gerechten bis zu dem Blute Zachariä, des Sohnes Barachiä, den ihr zwischen dem Altare und Tempel erschlagen habt.} Dieser \RWbet{Zacharias} ist aber, nach Flavius Josephus, erst während der Belagerung von Jerusalem in einem Aufruhre umgebracht worden. Wie konnte hier also seiner bereits erwähnt werden? -- Nach \RWbet{Hieronymus} Berichte stand in dem \RWlat{Evangelio Nazaraeorum}, aus welchem das Evangelium Matthäi wahrscheinlicher Weise geschöpft ist, nicht Barachias, sondern Jojada, und Zacharias der Sohn Jojadä ist (\RWbibel{2\,Chr}{2\,Chron.}{24}{24}) allerdings zwischen dem Altar und dem Tempel gesteiniget worden. Dagegen ist der Zacharias bei Josephus ein Sohn Baruchs.
\item Bei \RWbibel{Mt}{Matth.}{18}{17} heißt es: \erganf{Wer die Kirche nicht hört}, \usw\ Eine Kirche aber gab es ja damals noch nicht. -- Allerdings; aber unter dem Worte \RWgriech{>ekklhs'ia} versteht hier Jesus bloß die \RWbet{Versammlung} der Gläubigen.
\end{aufzb}
\end{RWanm}
\item \RWbet{Bis auf Lukas geborne Juden}. -- Auch dieses offenbart sich in ihren Schriften. Zwar würde Mancher vielleicht erwarten, daß geborne Juden eher hebräisch oder syrochaldäisch als griechisch schreiben sollten; aber griechische Sprache war in jenem Zeitalter die ausgebreitetste, diejenige, in der man nothwendig schreiben mußte, wenn man allenthalben gelesen werden wollte. Zu Korinth, Thessalonich, Koloß, in Galatien verstand man schwerlich eine andere Sprache, als griechisch; die dahin geschriebenen Briefe also mußten nothwendig griechisch geschrieben seyn; aber auch zu Rom und selbst in Palästina verstand man das Griechische sehr wohl; daher denn auch andere jüdische Schriftsteller dieser~\RWSeitenw{82}\ Zeit, \zB\ \RWbet{Flavius Josephus, Philo} nicht etwa in hebräischer, sondern in griechischer Sprache schrieben. Inzwischen ist das Griechische des neuen Testamentes (mit Ausschluß der beiden Bücher Lucä) nichts weniger, als ein reines Griechische; sondern voll Hebraismen, und völlig so, wie es ein Jude schreiben konnte, der diese Sprache nur aus dem Umgange und vornehmlich aus der Lesung der siebenzig Dolmetscher erlernet hatte, \zB\ \RWgriech[>abb~a <o pat`hr]{>abb'a <o pat'hr}, statt \RWgriech{>~w p'ater}; das häufige \RWgriech{ka`i}, \RWgriech{ka`i >ido`u}, \uam\ Es geht dieß so weit, daß man sehr viele Stellen nicht einmal gehörig verstehen kann, wenn man sie nicht erst in das Hebräische, oder vielmehr Syrochaldäische übersetzt, \zB\ \RWgriech{b'iblos gen'esews >Ihso~u Qristo~u}\editorischeanmerkung{\RWbibel{Mt}{Mt}{1}{1}} -- \RWgriech{>enebr'imhto}\editorischeanmerkung{\RWbibel{Mk}{Mk}{14}{5} (\RWgriech{>enebrim~wnto}) oder \RWbibel{Mt}{Mt}{9}{30} (\RWgriech{>enebrim'hjh}).}. Nebstdem verrathen die Verfasser dieser Bücher auch ganz den jüdischen Geschmack, so viel er uns nur immer aus den Büchern des alten Bundes, dem Talmud, und anderen Schriften der Orientalen, bekannt ist. Hieher gehören ihre beständigen Anwendungen biblischer Stellen des alten Bundes auf gegenwärtige Begebenheiten, die vielen Parabeln, die jüdischen Sprichwörter, die allegorischen Beweise in den Briefen \RWbet{Pauli} \uam\ In den beiden Schriften des \RWbet{Lukas} dagegen, der ein griechischer Arzt gewesen seyn soll, gibt es bei Weitem nicht so häufige Hebraismen, noch weniger Anspielungen auf das alte Testament; auch bestimmt er die Zeit der Begebenheiten, so oft es angeht, nach der profanen Zeitrechnung \usw
\item \RWbet{Größtentheils Augenzeugen}. -- Auch dieß bestätiget sich. Männer, die etwas erzählen, welches sie selbst gesehen, und zu Personen sprechen, welchen die Hauptsache bereits gleichfalls bekannt ist, pflegen genaue Zeit- und Ortsbestimmungen, Beweise für ihre Glaubwürdigkeit \udgl\  wegzulassen, dagegen pflegen sie hie und da gewisse, ganz individuelle Umstände zu bemerken. Dieß Alles findet man nun auch bei den Verfassern der Bücher des neuen Bundes. Sie tragen (mit Ausschluß Lucä, der kein Augenzeuge war) ihre Erzählungen ohne genaue Zeit- und Ortsbestimmung vor. \RWbet{Damals}, sagen sie nur; sie geben nicht einmal das Geburtsjahr Jesu und das erste Jahr seines öffentlichen Lehramtes an. Nur Lukas (\RWbibel{Lk}{Luk.}{3}{1}) thut dieses: \erganf{Im fünfzehnten~\RWSeitenw{83}\ Regierungsjahre des Kaisers Tiberius, da Pontius Pilatus Statthalter in Judäa, Herodes Tetrarch in Galiläa, Philipp, dessen Bruder, Tetrarch in der Landschaft Ituräa und Trachonitis, und Lysanias Tetrarch in Abilene war, unter den Oberpriestern Annas und Kaiphas} \usw\ Keiner aus ihnen nennt sich in der Schrift selbst als ihren Verfasser; sondern nur in der Ueberschrift, die wahrscheinlich von Anderen herrührt, kommt ihr Name vor; Keiner sucht seine Glaubwürdigkeit zu beweisen, sondern setzt dieses Alles bei seinen Lesern schon als bekannt voraus. Nur \RWbet{Lukas} sagt im Eingange seines Evangeliums: \erganf{Nachdem schon Viele es unternommen, eine Erzählung der Begebenheiten, die unter uns sich zugetragen haben, aufzusetzen, wie sie uns Jene überliefert, die vom Anfange an Augenzeugen und Diener des Wortes gewesen sind: so habe auch ich für gut befunden, dir, bester \RWbet{Theophilus}! dieselben nach der Ordnung zu beschreiben, nachdem ich Allem von seinem Ursprunge an genau nachgeforscht habe}, \usw\ Wunder erzählen sie oft so bestimmt, wie nur ein Augenzeuge vermag, \zB\ die Auferweckung des \RWbet{Lazarus}, die Speisung der großen Menschenmenge mit wenigen Broden und Fischen, die Erscheinung Jesu nach seiner Auferstehung \udgl\ 
\item \RWbet{Bis auf Paulus und Lukas ungelehrte Leute}. -- Dieß zeigt sich in ihrer Sprache, in ihrer einfachen Erzählungsart. Nirgends verrathen die beiden Evangelien \RWbet{Matthäi} und \RWbet{Marci} irgend eine gelehrte Kenntniß ihrer Verfasser, deren ganze Weisheit nur Jesus Christus ist. In den beiden Büchern \RWbet{Lucä} bemerkt man schon einen durch Lesung anderer Schriftsteller gebildeten Verfasser, er beobachtet schon mehr die Regeln der Kunst. Die Briefe \RWbet{Pauli} aber verrathen offenbar einen Verfasser von vieler Gelehrsamkeit und großem Scharfsinne, eine vertraute Bekanntschaft mit den Büchern des alten Bundes, und den verschiedenen rabbinischen Erklärungsarten derselben, sogar Bekanntschaft mit mancherlei Schriften heidnischer Gelehrten, namentlich Dichter. Auch verräth sich deutlich genug der lebhafte Geist des Verfassers. Man begreift sehr wohl, wie ein Mann, der jetzt so feurig für das Christenthum spricht, einst ein Verfolger desselben seyn konnte.~\RWSeitenw{84}
\end{aufza}

\RWpar{35}{Es war keinem Betrüger möglich, den Büchern des neuen Bundes das innere Merkmal der Aechtheit zu ertheilen}
Die Bücher des neuen Bundes besitzen also nach Ausweis des vorhergehenden Paragraphen das sogenannte innere Merkmal der Aechtheit; sie \RWbet{können} daher von den acht Männern, welchen sie zugeschrieben werden, verfaßt seyn; \dh\ es ist dieß \RWbet{problematisch möglich}. Daß es aber auch \RWbet{wirklich} so sey, wird erst erwiesen seyn, wenn wir noch darthun, daß es einem Betrüger \RWbet{nicht möglich} gewesen wäre, seiner eigenen Arbeit dieß innere Merkmal der Aechtheit \RWbet{mitzutheilen}, und noch weniger, so viele Leute, welche die Aechtheit dieser Bücher von jeher geglaubt, \RWbet{in diesen Glauben zu versetzen}. Das Erstere, daß es keinem anderen Menschen, auch keiner Gesellschaft mehrerer anderer Menschen, möglich gewesen wäre, diese Bücher so zu schreiben, wie wir sie vorfinden; das läßt sich freilich mit keiner völligen Strenge, und bis auf den Umstand, der ihre Namen betrifft, erweisen. Was aber die übrigen, uns angegebenen Umstände anlangt: so läßt sich ihre Richtigkeit mit einer bald größeren, bald geringeren Wahrscheinlichkeit darthun; indem sich darthun läßt, daß die Bücher des neuen Bundes weder von einem Einzigen, noch von einer ganzen Gesellschaft von Menschen geschrieben seyn können, welche nicht alle (\RWparnr{33}) angegebenen Beschaffenheiten (etwa mit Ausschluß der Namen) an sich gehabt hätten.
\begin{aufza}
\item Unmöglich können die 27 Bücher des neuen Bundes alle von \RWbet{einem einzigen Manne} herrühren. Viel zu groß ist der Unterschied des Styls, welcher in diesen Büchern herrscht, als daß es einem und demselben Manne möglich gewesen seyn sollte, diese Verschiedenheiten des Styls anzunehmen, und durch ganze Bücher von solchem Umfange, wie es \zB\ die Evangelien, oder die 14 Briefe \RWbet{Pauli} sind, so genau beizubehalten, daß er auch nicht ein einziges Mal aus der einen Schreibart in die andere wieder zurück gefallen wäre.
\item Eben so können auch füglich \RWbet{nicht mehr als acht} Verfasser gewesen seyn, und die Bücher, die Einem und demselben Verfasser zugeschrieben werden, müssen auch wirklich~\RWSeitenw{85}\ von Einem verfaßt seyn; weil mehrere Menschen nie einen so durchaus gleichen Styl zu haben pflegen.
\item Die Bücher des neuen Bundes können auch nicht von Männern herrühren, welche in einem \RWbet{späteren Jahrhunderte} gelebt. Denn solche hätten sich unmöglich ganz in die Verhältnisse des ersten Jahrhunderts hineindenken, alle die zahlreichen historischen, geographischen und politischen Beziehungen so glücklich treffen können; sie würden sich irgendwo durch einen Anachronismus verstoßen haben; denn die Erfahrung lehrt, daß dieß den schlauesten Betrügern bei der Unterschiebung viel kleinerer Schriften, als es ein Evangelium ist, begegnet sey.
\item Die Bücher des neuen Bundes müssen (etwa mit Ausnahme des Evangelii Lucä\bindex{Lk} und der Apostelgeschichte\bindex{Apg}) höchst wahrscheinlicher Weise von \RWbet{gebornen Juden} geschrieben seyn. Denn schwerlich hätte ein Römer oder Grieche diese so hebraisirende Sprache sich aneignen können.
\item Und eben so auch von \RWbet{Augenzeugen}. Ein Mann, der nicht selbst Augenzeuge war, hätte gewisse Begebenheiten, welche in diesen Büchern erzählt werden, fast unmöglich so nach dem Leben darstellen können.
\item \RWbet{Theils von Gelehrten, theils von Ungelehrten}. Die Evangelien Matthäi\bindex{Mt} und Marci\bindex{Mk} kann nur ein Ungelehrter geschrieben haben; denn einem Gelehrten wäre es schwerlich gelungen, überall so einfach und kunstlos zu erzählen. Die Schriften \RWbet{Pauli} und \RWbet{Lucä}\bindex{Lk}\bindex{Apg} muß nothwendig ein gelehrter Mann abgefaßt haben; denn sie tragen unverkennbare Spuren einer Gelehrsamkeit an sich.
\end{aufza}
\begin{RWanm} 
Die Unmöglichkeit, von der ich hier überall sprach, war eine \RWbet{physische}. Man könnte sagen, ob sich nicht auch selbst eine \RWbet{psychologische} Unmöglichkeit der Unterschiebung dieser Bücher nachweisen ließe. Dieß würde, dächte ich, schwieriger seyn, da die Geschichte uns wirklich lehrt, daß es doch mehrere unterschobene Evangelien, welche den unsrigen nicht eben sehr unähnlich sind, gegeben habe. Ich übergehe also die bloßen Wahrscheinlichkeitsgründe, die man hier allenfalls aus der \RWbet{besonderen Beschaffenheit} unserer Evangelien herleiten könnte, um nicht zu weitläufig zu werden.~\RWSeitenw{86} 
\end{RWanm}

\RWpar{36}{Folgerungen für die Aechtheit der Bücher des neuen Bundes aus den Schriftstellern des fünften und vierten Jahrhunderts}
Selbst wenn wir annehmen wollten, daß es einem Betrüger oder auch einer Gesellschaft mehrerer möglich gewesen wäre, die Bücher des neuen Bundes zu schreiben, und ihnen das innere Merkmal der Aechtheit mitzutheilen: so bliebe doch noch \RWbet{zweitens} zu erklären, \RWbet{wie es möglich gewesen wäre, so viele Leute, welche die Aechtheit dieser Bücher von jeher geglaubt, in diesen Irrthum zu versetzen}. Auch dieses läßt sich auf keine Weise erklären, und dadurch wird denn die Aechtheit jener Bücher neuerdings, nämlich aus einem sogenannten \RWbet{äußeren Grunde}, oder aus \RWbet{Zeugnissen}, bewiesen. Ich will hier zuerst mit den Schriftstellern des \RWbet{fünften} und \RWbet{vierten} christlichen Jahrhunderts den Anfang machen.
\begin{aufza}
\item Es gibt ächte Schriften der Kirchenväter aus dem fünften und vierten Jahrhunderte. Wir besitzen eine so große Menge von Schriften, welche angeblicher Maßen von christlichen Schriftstellern aus dem fünften und vierten Jahrhunderte geschrieben seyn sollen, und von so großem Umfange sind, daß es eine offenbare Ungereimtheit wäre, diese Bücher alle für unterschoben zu erklären. -- Die Schriften des heil.\ \RWbet{Augustinus, Hieronymus, Chrysostomus, Gregor des Großen, Leo des Großen} \uA\ betragen oft mehrere Folianten. Der größte Theil dieser Schriften hat alle inneren Merkmale der Aechtheit. Wenn wir nun auch eine ganze Gesellschaft von Menschen, die sich zur Ausarbeitung dieser Bücher vereiniget hätte, annehmen wollten: so hätte gleichwohl ein Einziger seyn müssen, der den ganzen ungeheuern Plan zu dieser Unterschiebung entworfen, Alles leiten, und eine Uebersicht vom Ganzen sich hätte aneignen müssen, um jeden möglichen Widerspruch, der den Betrug verriethe, zu vermeiden. So etwas übersteigt alle bekannten Kräfte des menschlichen Geistes, ist durchaus unmöglich.
\item Es wird nicht überflüssig seyn, einige dieser Schriftsteller namentlich anzuführen:~\RWSeitenw{87}
\begin{aufzb}
\item \RWbet{Aurelius Augustinus} zu Tagaste in Numidien von reichen Eltern im Jahre 354 geboren, verfiel in seiner Jugend in den Manichäismus, und ergab sich allen Ausschweifungen. Weil er jedoch sehr ungewöhnliche Geistesanlagen hatte: so brachte er es, trotz seinem unordentlichen Leben, in einigen Wissenschaften, besonders in der Redekunst, sehr weit. Diese lehrte er an verschiedenen Orten, \zB\ in Rom; kam endlich nach Mayland, wo er in seinem 30sten Jahre von dem Bischofe \RWbet{Ambrosius} bekehrt und getauft wurde. Von nun an führte er einen sehr tugendhaften Wandel, kehrte nach Afrika wieder zurück, wurde erst Aeltester (Presbyter), endlich selbst Bischof zu Hippo; und vertheidigte die katholische Lehre gegen die Ketzereien der Manichäer, Donatisten, Pelagianer, \uA ; schrieb eine große Anzahl von Homilien, Sermonen, Briefen und viele andere Werke, in welchen durchgängig viel Witz und Scharfsinn und eine brennende Liebe für alles Gute herrscht.
\item \RWbet{Hieronymus} aus Stridon in Dalmatien gebürtig, Einer der gelehrtesten und fleißigsten Männer seines Jahrhunderts, erhielt in Rom seine erste Bildung, reiste dann nach Frankreich, Griechenland und Palästina, fast einzig nur in der Absicht, um sich recht gründliche Kenntnisse von der Sache des Christenthums zu verschaffen, übersetzte die Bücher des alten Bundes aus der hebräischen Sprache in die lateinische, und verbesserte die schon vorhandene lateinische Uebersetzung der Bücher des neuen Bundes (dieß auf Befehl des Papstes \RWbet{Damasus}), schrieb Commentarien beinahe über alle Theile der heil.\ Schrift, mehrere Werke über die Ketzer, sehr viele Briefe, ein Verzeichniß der Kirchenschriftsteller, \umA\  Von ihm sagt \RWbet{Augustinus}, er habe beinahe Alles gelesen, was bis zu seiner Zeit von Christen geschrieben worden sey. Sein Leben beschloß er in einem hohen Alter zu Bethlehem im Jahre 420.
\item \RWbet{Johannes Chrysostomus} von Antiochien, erst daselbst Presbyter, dann wider seinen Willen auf den bischöflichen Sitz von Konstantinopel erhoben, zog er sich durch seine freimüthigen Predigten den Haß der Kaiserin~\RWSeitenw{88}\ \RWbet{Eudoxia} und die größten Verfolgungen, selbst eine zweimalige Landesverweisung zu, schrieb in griechischer Sprache sehr viele Homilien, ein schönes Werk über das Priesterthum, Eines über die Vorsehung Gottes \uA\ und starb im Jahre 407.
\end{aufzb}
\item Aus diesen Schriftstellern ersehen wir offenbar, daß im fünften und vierten Jahrhunderte die christliche Kirche schon in allen drei Welttheilen verbreitet war, und besonders aus den Nachrichten des \RWbet{Hieronymus}, daß es bereits eine sehr große Menge von Schriftstellern, die über die christliche Religion und über die Bibel geschrieben hatten, gegeben habe. Die Bücher des neuen Bundes müssen also schon deßhalb eine geraume Zeit \RWbet{vor dem vierten} Jahrhunderte da gewesen seyn. Ja, da Einige dieser Männer nebst vielen anderen Kenntnissen auch insbesondere \RWbet{kritische Gelehrsamkeit} besaßen, \zB\ \RWbet{Hieronymus}; und um so viel mehrere Mittel in den Händen hatten, sich von der Aechtheit der heiligen Bücher zu überzeugen, je näher sie dem Zeitalter derselben lebten: so hätten sie die Unterschiebung derselben sicher entdeckt, und dann auch uns nicht verschwiegen. \RWbet{Hieronymus} \zB\ gesteht uns seine Zweifel an der Aechtheit einiger dieser Bücher, wie der zwei letzten Bücher der Makkabäer, unverholen; und \RWbet{Chrysostomus} glaubte, daß manche Christen seiner Zeit aus der Offenbarung Johannis zu viel machten. \RWbet{Gregor von Nazianz} gehet noch weiter, und schließt dieses Buch aus seinem Verzeichnisse der heil.\ Bücher ganz aus; \RWbet{Amphilochus} erklärt es nebst dem zweiten Briefe Petri, und dem zweiten und dritten Briefe Johannis für ungewiß. Hieraus ersieht man, daß diese Männer nichts weniger als leichtgläubig oder zurückhaltend waren; und so würden sie also die Aechtheit auch noch mehrerer anderer Bücher des neuen Bundes bezweifelt, und ihre Zweifel uns mitgetheilt haben, wenn sich die Aechtheit dieser Bücher bezweifeln ließe.
\end{aufza}

\RWpar{37}{Zwei wichtige Zeugen für die Aechtheit der Bücher des neuen Bundes aus dem dritten Jahrhunderte}
Unter den Schriftstellern aus dem \RWbet{dritten} Jahrhunderte sind \RWbet{Eusebius} und \RWbet{Origenes} für unser gegenwär\RWSeitenw{89}tiges Vorhaben deßhalb von größter Wichtigkeit, weil beide eben so gelehrte als unparteiische Untersuchungen über die Aechtheit der einzelnen Theile der Bibel angestellt, und die Ergebnisse ihrer Forschung uns hinterlassen haben.
\begin{aufza}
\item \RWbet{Eusebius Pamphili}, Bischof von Cäsarea in Palästina, schrieb eine Kirchengeschichte in 10 Büchern, ein Werk \RWlat{de praeparatione evangelica}, ein anderes \RWlat{de demonstratione evangelica}, ein Chronikon, das Leben Konstantins \uma\ , und starb 326. Dieser Mann hatte alle Documente des christlichen Alterthums durchgelesen, um zu erfahren, welche Schriften man seit dem Anfange des Christenthumes für ächte Werke der Apostel angenommen habe. Was er uns also (\RWlat{histor.\ 1.\ 3.}) mittheilet, ist nicht etwa seine Privatmeinung, sondern die Meinung der Kirche, und eben deßhalb ist uns dieß Zeugniß so wichtig; denn es ersetzt uns gewisser Maßen den Verlust so vieler früherer Zeugen. Er bringt nun alle Schriften, die unter dem Namen apostolischer herumgetragen wurden, in drei Classen:
\begin{aufzb}
\item \RWgriech{<Omologo'umena}, \dh\ allgemein als ächt anerkannte; dahin er die vier Evangelien, die Apostelgeschichte, die dreizehn Briefe \RWbet{Pauli} (nämlich mit Ausnahme des Briefes an die Hebräer) den ersten Brief \RWbet{Petri} und den ersten Brief \RWbet{Johannis} zählet.
\item \RWgriech{>Antileg'omena}, \dh\ Bücher, deren Aechtheit von Einigen bezweifelt, aber von Mehreren doch angenommen wurde. Dahin verlegt er den Brief an die Hebräer, den Brief \RWbet{Jakobi}, den zweiten \RWbet{Petri}, den zweiten und dritten \RWbet{Johannis}, den Brief \RWbet{Judä}, und die Offenbarung \RWbet{Johannis}.
\item \RWgriech{>'Atopa ka`i dusseb~h}, \dh\ unvernünftige und gottlose, oder entschieden unterschobene Schriften. Unter dieser Classe führt er uns auf: ein Evangelium \RWbet{Petri}, ein Evangelium des \RWbet{Thomas, Matthias}, gewisse Werke des \RWbet{Andreas, Johannis} \ua
\end{aufzb}
\item \RWbet{Origenes} zu Alexandria geboren, ein Mann von den außerordentlichsten Geisteskräften und von dem vortrefflichsten moralischen Charakter, von überaus vieler Frei\-mü\-thig\-\RWSeitenw{90}keit, und einem ganz unermüdeten Fleiße, der ihm den Beinamen der Diamantene verschaffte. Er hörte den berühmten Philosophen \RWbet{Ammonius Saccas}, dann den Alexandrinischen Katecheten \RWbet{Clemens}, und ward in seinem 18ten Jahre selbst als Lehrer bei der Alexandrinischen Schule angestellt. Bei diesem Amte erwarb er sich durch seine Gelehrsamkeit eine solche Hochachtung, daß ihm selbst heidnische Gelehrte häufig ihre Arbeiten widmeten, und zur Beurtheilung zuschickten. Da sein Vater \RWbet{Leontius} des christlichen Glaubens wegen unter dem Kaiser \RWbet{Severus} hingerichtet und seine Güter eingezogen wurden: so sah sich \RWbet{Origenes} genöthiget, für seine Schriften sich bezahlen zu lassen; begnügte sich aber mit vier Obolis des Tages, um davon sich und seine Mutter zu ernähren. Die Stelle \RWbibel{Mt}{Matth.}{19}{12} verleitete ihn, sich selbst zu entmannen. Als er darauf nach Antiochien berufen ward, um eine daselbst entstandene religiöse Streitigkeit beizulegen, ließ er sich von dem Bischofe zu Jerusalem zum Aeltesten einweihen. Darüber aufgebracht und seinem großen Ruhme neidisch, machte sein Bischof \RWbet{Demetrius} seine Entmannung bekannt, erklärte sie für ein Verbrechen, nahm ihm sein Lehramt, und entsetzte ihn sogar der Priesterwürde. \RWbet{Origenes} zog nun an mancherlei Orten umher, und starb 254. Seine Schriften (deren er gegen 6000 verfaßt haben soll) gaben nach seinem Tode zu vielen Streitigkeiten und Ketzereien Anlaß. Er hatte die Aechtheit jedes einzelnen Buches der heil.\ Schrift untersucht, und nennt uns (bei \RWbet{Eusebius}) diejenigen, die er mit ungezweifelter Gewißheit für ächt erkannte, und die man auch allgemein dafür annahm; dann aber auch diejenigen, an deren Aechtheit man hie und da gezweifelt. Von den vier Evangelien sagt er, sie seyen die einzigen, die in der ganzen Kirche Gottes ohne Widerspruch für ächt angenommen werden. Zweifelhaft findet er den Brief an die Hebräer, den zweiten \RWbet{Petri}, den zweiten und dritten \RWbet{Johannis}.
\end{aufza}

\RWpar{38}{Einige Zeugen für die Aechtheit der Bücher des neuen Bundes aus dem zweiten Jahrhunderte}
Obgleich wir uns auf das Zeugniß eines \RWbet{Eusebius} und \RWbet{Origenes} bereits verlassen könnten: will ich doch auch~\RWSeitenw{91}\ aus dem \RWbet{zweiten} und \RWbet{ersten} Jahrhunderte noch einige Zeugen nur darum nahmhaft machen, weil die Männer, die wir bei dieser Gelegenheit kennen lernen, uns noch in anderer Rücksicht wichtig seyn müssen.
\begin{aufza}
\item \RWbet{Irenäus} aus Smyrna gebürtig, ein Zögling des heil.\ Polykarpus, der selbst ein unmittelbarer Schüler des heil.\ \RWbet{Johannes} war, lebte als Bischof zu Lyon in Frankreich, und starb um das Jahr 205 als Märtyrer. In seinem, nur in einer lateinischen Uebersetzung, auf uns gekommenen Werke: Gegen die Ketzereien (\RWlat{lib.\,5.}) berichtet er uns: \RWbet{Matthäus} habe sein Evangelium für die Hebräer in ihrer Sprache herausgegeben, während \RWbet{Petrus} und \RWbet{Paulus} die Kirche zu Rom gestiftet. Nachher habe \RWbet{Markus}, ein Schüler und Dolmetscher \RWbet{Petri}, was dieser vorgetragen, schriftlich abgefaßt. \RWbet{Lukas} aber, ein Gefährte \RWbet{Pauli}, habe das Evangelium, das dieser verkündiget, in ein Buch zusammengetragen. Endlich habe auch \RWbet{Johannes}, der Jünger des Herrn, der bei dem letzten Abendmahle auf seinem Schooße lag, ein Evangelium herausgegeben, als er sich zu Ephesus in Asien aufhielt. Nebstdem kommen in diesem Werke so viele und so lange Stellen aus allen vier Evangelien, und aus den übrigen Büchern des neuen Bundes vor, daß gar kein Zweifel ist, \RWbet{Irenäus} habe, wenn er von diesen Büchern spricht, wirklich diejenigen gemeint, die wir noch heut zu Tage unter diesem Namen haben. Er nennte sie \RWbet{göttliche} Schriften. Wir sehen aus ihm, daß sie zu seiner Zeit nicht nur in den öffentlichen Versammlungen der Christen vorgelesen, sondern auch in den Häusern reicherer Privatpersonen zu finden waren. Er warnt auch vor fehlerhaften Abschriften, und räth, daß man seine Abschrift mit den bei den Lehrern der Kirche aufbewahrten Exemplaren vergleiche.
\item \RWbet{Tit.\ Flav.\ Clemens von Alexandria}, von heidnischen Eltern zu Athen geboren, seiner Philosophie nach ein Eklektiker und Schüler des \RWbet{Pantänus}, ging erst im männlichen Alter zum Christenthume über, ward hierauf Lehrer, dann Aeltester zu Alexandrien, und um sich eine recht gründliche Kenntniß vom Christenthume zu verschaffen, hatte er Griechenland, Aegypten, Italien und Asien durchreiset, er~\RWSeitenw{92}\ schrieb in griechischer Sprache \RWlat{Stromatum lib.\,8.,\RWlit{}{Clemens2} Cohortatio ad gentes, Paedagogus}, \umA\  und starb 204. In seinen Schriften werden gelegenheitlich fast alle Bücher des neuen Bundes, und sehr lange Stellen aus ihnen angeführt. Freilich citirt er auch manche apokryphische Schriften, doch nur des alten Bundes, und wohl wissend, daß es Apokryphen (\dh\ unterschobene Schriften) seyen.
\item \RWbet{Q.~S.~Fl.~Tertullian} von heidnischen Eltern zu Karthago geboren, hatte sich zuerst auf Rechtsgelehrsamkeit verlegt, ward dann ein Christ, und zu Karthago Presbyter. Er ist der älteste unter den lateinischen Kirchenvätern, und schrieb \RWlat{adversus Praxeam}\RWlit{}{Tertullian3}, \RWlat{de praescriptione}\RWlit{}{Tertullian2a}, \umA\  In seinem Alter neigte er sich zum Montanismus, und starb 203. In der Schrift gegen den \RWbet{Marcio} beweiset er die Aechtheit der Evangelien aus den ununterbrochenen Zeugnissen für sie von der Apostel Zeiten an; und in dem Buche \RWlat{de praescriptione}\RWlit{}{Tertullian2a} beruft er sich zu eben diesem Zwecke auf die von den Aposteln selbst gestifteten Gemeinden, bei welchen deren eigene authentische Schriften (\dh\ wohl nicht die Originale, die Autographen, sondern nur glaubwürdige Abschriften) vorgelesen werden. -- Uebrigens kommen in seinen Schriften eine Menge namentlicher Anführungen aus den Büchern des neuen Bundes vor. Auch sehen wir aus seinen Schriften, daß die Christen ihre Schriften gar nicht geheim hielten; sondern daß diese sich auch in den Händen ihrer Feinde befanden.
\item \RWbet{Justin der Märtyrer} oder der \RWbet{Philosoph}, von heidnischen Eltern in Palästina geboren, und mit den Werken der griechischen Philosophen und Dichter sehr wohl vertraut, ging erst in seinem dreißigsten Jahre zum Christenthume über, wobei man anmerkt, daß er noch als Christ den Philosophenmantel beibehalten habe. Er schrieb sodann (in griechischer Sprache) zwei Apologien des Christenthums; deren Eine er dem Kaiser \RWbet{Antonin} dem Frommen, die andere seinem Nachfolger überreichte, \RWlat{dialogum cum Tryphone}\RWlit{}{Justinus5}, \RWgriech{l'ogon parainetik`on pr`os <'Ellhnas}, \umA , und starb als Märtyrer im Jahre 165. In seinen Schriften führt er sehr viele Stellen an, die wir in unseren Evange\RWSeitenw{93}lien wörtlich wieder finden; er nennt zwar nicht die Namen \RWbet{Matthäus, Markus, Lukas, Johannes}; aber er sagt doch, daß diese Stellen aus gewissen \RWgriech{>apomnhmone'umasi Ihso~u} (Denkwürdigkeiten Jesu) wären, und einmal heißt es: \RWgriech{<oi >ap'ostoloi >en to~is genom'enois <up' a>ut~wn >apomnhmone'umasin, <`a kale~itai E>uagg'elia} (\RWlat{Apolog.\ 1.\ \RWparnr{66}}).\RWlit{}{Justinus1}%Justin: Apologia prima, § 66 = MPG VI, 429.
\end{aufza}

\RWpar{39}{Zeugen für die Aechtheit der Bücher des neuen Bundes aus dem ersten Jahrhunderte}
Da die Bücher des neuen Bundes angeblicher Maßen erst in der letzten Hälfte des ersten Jahrhunderts geschrieben wurden: so darf es Niemand wundern, wenn wir nur wenig Zeugnisse für ihr Daseyn aus diesem Jahrhunderte anführen können. Es gab der Schriftsteller in diesem Zeitalter noch nicht so viele, und was von ihnen sich bis auf uns erhalten hat, sind nur sehr wenige Bruchstücke. Die Bücher des neuen Bundes konnten bereits eine geraume Zeit vorhanden seyn, bevor sie allgemein bekannt geworden waren. Endlich fühlte man auch in jener Zeit noch nicht das Bedürfniß, seine Behauptungen aus diesen Schriften zu beweisen, weil die mündliche Ueberlieferung noch in ihrer ganzen Stärke war. Gleichwohl lassen sich auch selbst aus diesem Jahrhunderte einige Zeugen von Wichtigkeit anführen.
\begin{aufza}
\item \RWbet{Papias}, Bischof zu Hieropolis aus dem ersten Jahrhunderte, soll, wie uns \RWbet{Eusebius} berichtet, in seinem Buche: \RWgriech{l'ogwn kuri'akwn >ex'hghsis} (Auslegung der Reden des Herrn) verschiedene Nachrichten über die Lehren und Thaten Jesu aus dem Munde solcher Personen gesammelt haben, die Jesum oder seine Apostel unmittelbar gekannt. Dieses Buch selbst ist verloren gegangen. In den Auszügen aber, die uns \RWbet{Eusebius} daraus vorlegt, heißt es, daß \RWbet{Markus} sein Evangelium aus dem Vortrage \RWbet{Petri}, obgleich nicht nach chronologischer Ordnung, sondern so, wie er sich eben erinnerte, niedergeschrieben; \RWbet{Matthäus} sein Evangelium hebräisch abgefaßt habe. Auch sagt uns \RWbet{Eusebius}, daß sich \RWbet{Papias} einiger Stellen aus dem ersten Briefe \RWbet{Johannis} und aus dem ersten \RWbet{Petri} bediene.~\RWSeitenw{94}
\item \RWbet{Polykarpus}, ein Schüler des heil.\ \RWbet{Johannes} und Bischof zu Smyrna, der unter dem Kaiser \RWbet{Marc. Aurel. Antoninus} verbrannt wurde, erwähnt in einem einzigen, für ächt gehaltenen Briefe an die Philipper, den wir von ihm noch übrig haben, ausdrücklich des gleichnamigen Briefes \RWbet{Pauli}; auch führt er mehrere Stellen, die sich in anderen Büchern des neuen Bundes, als: \RWbibel[1.\,Kor.]{1\,Kor}{}{}{}, \RWbibel[Ephes.]{Eph}{}{}{}, \RWbibel[1.]{1\,Thess}{}{}{} und \RWbibel[2.\,Thessal.]{2\,Thess}{}{}{}, \RWbibel[Matth.]{Mt}{}{}{}, \RWbibel[Luk.]{Lk}{}{}{}\ finden, wörtlich und mit der Bemerkung an, daß so \RWlat{in sacris litteris, in scripturis} geschrieben stände.
\item \RWbet{Ignatius}, Bischof zu Antiochien (der dritte nach dem heil.\ Petrus daselbst), der auf Befehl des Kaisers \RWbet{Trajan} im Jahre 106 oder 116 zu Rom den Löwen vorgeworfen worden ist, hinterließ uns sieben ächte Briefe. In Einem derselben, der an die Epheser gerichtet ist, erwähnt er des gleichnamigen Briefes \RWbet{Pauli} ausdrücklich, führt auch Stellen aus noch anderen Briefen an, ohne jedoch sie ausdrücklich zu nennen, eben so aus gewissen Denkwürdigkeiten Jesu. Aus einer Stelle seines Briefes an die Philadelpher möchte man wohl gar schließen, daß es zu seiner Zeit schon eine Sammlung der Bücher des neuen Bundes in zwei Abtheilungen, deren eine, das Evangelium, die andere, die Apostel überschrieben war, gegeben habe. Die Stelle lautet:
\RWgriech{Prosfug`wn t~w| e>uaggel'iw| <ws sark`i >Ihso~u, ka`i to~is >apost'olois <ws pr`esbuter'iw| t~hs >ekklhs'ias, ka`i to`us prof'htas d`e >agap~wmen, di`a t`o ka`i a>uto`us e>is t`o e>uagg'elion kathggelk'enai ka`i e>is a>ut`on >elp'izein, ka`i a>ut`on >anam'enein.}\RWfootnote{%
	Halten wir uns an \RWbet{das Evangelium}, als an den Leib Jesu, und an \RWbet{die Apostel}, als an das Priesterthum der Kirche; aber auch \RWbet{die Propheten} wollen wir lieben, und auch sie mit dem Evangelium verkündigen, auf das wir hoffen, nach dem wir uns sehnen.}\RWlit{331}{Ignatius3}

\item \RWbet{Clemens von Rom}, Bischof daselbst, ein Schüler und Gehülfe \RWbet{Pauli}, der gleichfalls als Märtyrer unter dem Kaiser \RWbet{Trajan} im Jahre 98 starb, hat uns einen Brief an die Korinther\RWlit{}{ClemensvonRom1} hinterlassen, der ohne Zweifel ächt ist, und in der ersten Kirche in einem solchen Ansehen stand, daß~\RWSeitenw{95} \ man ihn, wie \RWbet{Hieronymus} berichtet, in vielen Gemeinden, gleich den apostolischen Briefen, öffentlich vorzulesen pflegte. In diesem Briefe erwähnt er nun eines Briefes \RWbet{Pauli} an die \RWbet{Korinther} namentlich, und aus den Stellen, welche er anführt, ersieht man, daß er den \RWbet{ersten} meine. Auch kommen noch sonst manche Stellen vor, die mit den Büchern des neuen Bundes sehr übereinstimmen.
\end{aufza}

\RWpar{40}{Rechtgläubige, Ketzer und Heiden erkennen die Aechtheit der Bücher des neuen Bundes}
\begin{aufza}
\item Wie man aus den bisher erwähnten Zeugen ersieht: so wurde die Aechtheit der Bücher des neuen Bundes, bis auf einige sehr kleine Theile derselben, von allen rechtgläubigen Christen, von unsern jetzigen Zeiten rückwärts, bis zu ihrer Entstehung im ersten Jahrhunderte hin, durchgängig angenommen.
\item Allein man muß auch wissen, daß es in der christlichen Kirche sehr frühzeitig, nämlich schon seit dem ersten Jahrhunderte, eine nicht unbeträchtliche Menge von sogenannten \RWbet{Ketzern} oder \RWbet{Häretikern} gegeben habe, \dh\ von Menschen, die, ob sie gleich die meisten Lehrsätze des Christenthums annahmen, in einigen einzelnen Stücken dennoch von dem gemeinen Glauben der Kirche abwichen, und ihre eigenthümliche Meinung mit großer Heftigkeit vertheidigten. So gab es \zB\ schon zu der Apostel Zeiten die sogenannten \RWbet{Gnostiker}, die \RWbet{Nikolaiten}; später die \RWbet{Nazaräer} \uA\ Diese verschiedenen Ketzer, ob sie gleich mit den Orthodoxen in der heftigsten Feindschaft lebten, nahmen nichts desto weniger die Bücher des neuen Bundes bis auf das Eine oder das andere Buch für ächt an.
\item Sogar die Feinde des Christenthums, die \RWbet{Heiden}, welche die Wahrheit dieser Religion zu bestreiten suchten, läugneten doch die Aechtheit der Bücher des neuen Bundes nicht, obgleich es darunter auch manche sehr gelehrte Männer, \zB\ \RWbet{Celsus, Porphyrius} \uA\ gegeben.~\RWSeitenw{96}
\end{aufza}

\RWpar{41}{Die Zeugen für die Aechtheit der Bücher des neuen Bundes verdienen allen Glauben}
Nachdem ich nun die Zeugen, die für die Aechtheit der Bücher des neuen Bundes sprechen, kennen gelehrt habe: so will ich beweisen, daß es nicht möglich gewesen wäre, alle diese Personen entweder zu hintergehen, oder sie wenigstens zu einer absichtlich falschen Aussage zu verleiten, \dh\ mit anderen Worten, daß wir aus ihrem Zeugnisse mit aller Sicherheit auf die Aechtheit der Bücher des neuen Bundes schließen können, oder \RWbet{daß diese Zeugen glaubwürdig sind}.
\begin{aufza}
\item Es hat den bisher angeführten Zeugen nicht an der nöthigen \RWbet{Sachkenntniß} gefehlt.
\begin{aufzb}
\item Mehrere der bisher angeführten Zeugen sind Männer gewesen, welche in aller Gelehrsamkeit, besonders in der Kritik, sehr wohl bewandert waren, \zB\ \RWbet{Origenes, Hieronymus, Porphyrius}. Sie lebten auch dem Zeitalter der Entstehung dieser Bücher so nahe, daß sie den Betrug, wenn einer vorhanden gewesen wäre, viel leichter, als wir, hätten entdecken können. Einige sogar waren unmittelbare Schüler der Apostel, \zB\ \RWbet{Polykarpus}; kannten also ihre Schreibart, ihre Grundsätze, \usw
\item Dabei hatten sie auch auf den Gegenstand alle \RWbet{Aufmerksamkeit} verwendet; denn es mußte ihnen doch sehr viel daran liegen, ob diese Bücher, auf welche sich der Glaube an das Christenthum, oder doch wenigstens jede genauere Kenntniß von Jesu Lebenswandel, stützt, ächt oder unächt, glaub- oder unglaubwürdig seyen. Von Mehreren wissen wir auch bestimmt, daß sie sehr viele Mühe sich gaben, weite Reisen unternahmen, bei den von den Aposteln selbst gestifteten Gemeinden sich erkundigten, \usw ; Alles nur, um die Aechtheit der Bücher des neuen Bundes zu erforschen; \zB\ \RWbet{Papias, Clemens von Alexandrien, Origenes, Eusebius, Hieronymus}.~\RWSeitenw{97}
\item Endlich beweiset es auch der \RWbet{Erfolg selbst}, daß diese Männer ächte Schriften von unächten zu unterscheiden wußten. Denn schon im zweiten Jahrhunderte gab es, nebst jenen vier von ihnen und von uns gemeinschaftlich als ächt erkannten Evangelien, noch mehrere andere, welche sie aber einstimmig als unächt verwarfen. Und gerade unter diesen verworfenen Evangelien und Schriften befinden sich solche, die von leichtgläubigen Leuten um so begieriger wären angenommen worden, je mehr sie die Einbildungskraft durch ihre sonderbaren Wundergeschichten unterhalten, \zB\ das Evangelium der Kindheit Jesu. -- Einzelne Kirchen zweifelten lange an dem kanonischen Ansehen (oder der Aechtheit) dieser und jener apostolischen Schrift, daher es eben kam, daß man erst in dem dritten Jahrhunderte (seit dem Concilium zu Laodicäa 320) einen gleichlautenden Kanon in allen Kirchen erhielt; nicht als wäre man von dieser Zeit an leichtgläubiger geworden, sondern weil man erst seitdem bei der erleichterten Gemeinschaft der christlichen Kirchen unter einander, Erörterungen, die früher nicht möglich waren, anstellen konnte, und wirklich auch angestellt hatte.
\end{aufzb}
\item Diese Zeugen haben ihre Meinung über die Aechtheit der Bücher des neuen Bundes auf eine sehr \RWbet{bestimmte} und \RWbet{unzweideutige} Art an den Tag gelegt. Nicht nur, daß sie diesen Büchern ausdrücklich den Namen \RWbet{apostolischer}, auch wohl gar \RWbet{göttlicher} Bücher ertheilen; sondern sie führen uns auch so viele Stellen aus ihnen an, daß wir unmöglich zweifeln können, sie meinten eben diejenigen, die wir noch heut zu Tage haben.
\item \RWbet{Sie sagen aus, was sie für Wahrheit hielten}. Denn sie hatten keine Beweggründe, die Aechtheit der Bücher des neuen Bundes zu behaupten, wenn sie dieselben nicht in der That für ächt anerkannt hätten.
\begin{aufzb}
\item Was einmal die \RWbet{Rechtgläubigen} betrifft: so ist es durchaus ungedenkbar, daß eine so große Anzahl Personen, größtentheils von der geprüftesten Rechtschaffenheit, sich zu dem schändlichen Betruge verabredet haben sollte, die Bibel für ächt auszugeben, wenn sie doch ihre Unter\RWSeitenw{98}schobenheit anerkannt hätte. Die ersten Christen, besonders die Geistlichen, erlitten von Seite der heidnischen Obrigkeit so viele Verfolgungen um dieser Bücher willen, man forderte ihre Auslieferung, und strafte diejenigen, die sie zurückbehielten. Und dennoch gab es, wie die Geschichte uns meldet, nur äußerst Wenige, die sich zu dieser, dem Christenthume so nachtheiligen, Auslieferung entschloßen. Wenn sie die Bibel für ein unterschobenes Werk gehalten haben würden, was hätte sie zu einem solchen Eifer für ihre Aufbewahrung bestimmen können?
\item Die \RWbet{Ketzer} hatten einen so regen Haß gegen die Rechtgläubigen, daß sie ihnen den Vorwurf, sie hätten gewisse heilige Bücher unterschoben, gewiß nicht erspart haben würden, wenn irgend eine Möglichkeit dazu gewesen wäre; zumal da es diese Bücher waren, aus welchen man sie meistens so siegreich widerlegte.
\item Und noch mehr Ursache hiezu hätten die Feinde des Christenthums, die \RWbet{Heiden}, gehabt; denn ihnen wäre es dann sehr erleichtert worden, die Wahrheit des Christenthums zu widerlegen. \RWbet{Porphyrius}, der leicht der Gelehrteste aus allen Gegnern des Christenthums war, nahm zur Bestreitung desselben an, daß gewisse prophetische Bücher des alten Bundes erst nach den Evangelien geschrieben wären. Welch eine unwahrscheinliche Behauptung! Und gleichwohl dünkte sie ihm noch leichter zu verfechten, als die Unterschiebung der Bücher des neuen Bundes.
\end{aufzb}
\end{aufza}

\RWpar{42}{Die Zeugen wider die Aechtheit der Bücher des neuen Bundes verdienen keinen Glauben}
Es wurde schon einige Male bemerkt, daß es auch einige Personen gab, die an der Aechtheit der Bücher des neuen Bundes, wenigstens an einem Theile derselben, gezweifelt; es ist also zu zeigen, daß diese Zweifel unser Vertrauen zu der Aechtheit jener Bücher nicht vermindern können.
\begin{aufza}
\item Selbst unter den \RWbet{Rechtgläubigen} gab es Einige, welche die Aechtheit gewisser Bücher des neuen Bundes, nämlich des Briefes an die Hebräer, des zweiten Briefes \RWbet{Petri},~\RWSeitenw{99}\ des zweiten und dritten \RWbet{Johannis} und der Offenbarung bezweifelten. Dieß kam, weil man in den ersten zwei Jahrhunderten, wo das Christenthum noch keine herrschende Religion war, in der Mittheilung sicherer Nachrichten sehr gehemmt war. Es beweiset uns die Sorgfalt, mit welcher man hier zu Werke ging, und befestiget unseren Glauben an die Aechtheit derjenigen Bücher, an welchen man nie gezweifelt.
\item Eben so wenig kann es uns beunruhigen, daß gewisse Bücher des neuen Bundes von einzelnen \RWbet{Ketzerparteien} nicht angenommen worden seyen. Denn dieses geschah entweder,
\begin{aufzb}
\item weil diese Parteien schon eher entstanden und von der rechtgläubigen Kirche sich losgerissen hatten, ehe diese Bücher noch geschrieben, oder in ihre Gegenden gebracht worden waren; oder
\item weil sie in diesen Büchern gewisse Lehrsätze fanden, die ihnen nicht anständig waren. So verwarfen im ersten Jahrhunderte die sogenannten \RWbet{Aloger} (in der Gegend von Thyatira in Kleinasien) das Evangelium und die Offenbarung \RWbet{Johannis}, und schrieben sie dem \RWbet{Kerinthus} zu, weil sie vermeinten, daß in diesen Schriften die Lehre vom tausendjährigen Reiche (oder der Chiliasmus) vorgetragen werde. So verwarfen die \RWbet{Ebioniten} und \RWbet{Nazaräer} die Apostelgeschichte und \RWbet{Pauli} Briefe, weil in denselben auf die Aufhebung des mosaischen Gesetzes gedrungen wird, \usw
\item Einige dieser Ketzerparteien waren höchst unwissend, und nahmen statt jener ächten Bücher so läppisch abgefaßte an, daß sie um deßwillen schon keine Aufmerksamkeit verdienen.
\item Nur einen einzigen Ketzer gibt es, der die Aechtheit sämmtlicher Evangelien geläugnet hat, \RWbet{Faustus}, ein Manichäer, aus Afrika, der zu Anfang des fünften Jahrhunderts gelebt. Seine Gründe, die uns der heil.\ \RWbet{Augustinus} in seiner Widerlegung (gegen den \RWbet{Faustus}) aufbewahrt hat, sind von gar keinem Gewichte. Ohne Kritik, ohne Geschichtskunde, ja selbst ohne die nöthigen Sprachkenntnisse (er kannte bloß lateinisch) behauptete er,~\RWSeitenw{100}\ die Evangelien wären erst spät nach den Aposteln von einigen unbekannten Personen aufgesetzt worden, welche sich klüglich das Ansehen gegeben hätten, als ob sie die Lebensgeschichte Jesu nach gewissen, schon vorhanden gewesenen Evangelien eines \RWbet{Matthäus} \usw\ geschrieben. Dieß wollte er aus den gewählten Ueberschriften: \RWbet{nach Matthäus} \usw\ beweisen, wie auch noch daraus, weil im Evangelio \RWbet{Matthäi} von \RWbet{Matthäus} selbst in der \RWbet{dritten Person} geredet wird; endlich weil diese Bücher voll \RWbet{Widersprüche} und \RWbet{Ungereimtheiten} wären. -- Hierauf ist nun zu erwiedern, daß jene Ueberschriften im Griechischen eigentlich so lauten: \RWgriech[E>uagg'elion kat`a Matja~ion]{E>uagg'elion kat`a Matja'ion} \usw , welches die Uebersetzung \RWbet{durch Matthäus} oder \RWbet{vom Matthäus} recht wohl verträgt. Auch ist es bekannt, daß diese Ueberschriften nicht von den Evangelisten selbst herrühren. Daß Geschichtschreiber aus Bescheidenheit in der \RWbet{dritten Person} von sich erzählen, ist gar nichts Ungewöhnliches. So that es \RWbet{Julius Cäsar} in seinen Commentarien, so der König \RWbet{Friedrich}, \uA\ -- Ob aber \RWbet{Widersprüche} und \RWbet{Ungereimtheiten} in den Evangelien vorkommen oder nicht, das können wir heut zu Tage noch eben so gut, als \RWbet{Faustus} damals, beurtheilen.
\end{aufzb}
\end{aufza}

\RWpar{43}{Unverfälschtheit der Bücher des neuen Bundes}
Haben wir uns durch das Bisherige überzeugt, daß die Bücher des neuen Bundes wirklich ächt sind, \dh\ daß sie wirklich schon in dem ersten Jahrhunderte von gebornen Juden, die größtentheils Augenzeugen waren, geschrieben worden sind, \usw : so entsteht nun die Frage, ob wir diese Bücher auch noch ganz \RWbet{unverändert so, wie sie aus den Händen jener Männer hervorgegangen sind}, besitzen, \dh\ ob sie auch \RWbet{unverfälscht} auf uns gekommen sind? -- Kleine, unwesentliche Veränderungen, vornehmlich solche, die durch die unvorsätzliche Unachtsamkeit der Abschreiber entstanden sind (\RWlat{quos aut incuria fundit, aut humana parum vitavit inertia}) können wir keineswegs läugnen. Daß aber~\RWSeitenw{101}\ diese Schriften unverfälscht in dem Sinne des Wortes sind, daß keine \RWbet{wesentlichen Zusätze}, oder \RWbet{Abänderungen} in ihnen vorgenommen würden, das läßt sich aus folgenden Gründen beweisen.
\begin{aufza}
\item Fragen wir die \RWbet{Geschichte}: so gibt sie uns nicht die geringste Nachricht von einer Verfälschung; sondern meldet uns vielmehr, daß die Christen zu aller Zeit, vornehmlich aber in den ersten Jahrhunderten über die Unverfälschtheit ihrer heiligen Schriften mit der größten Sorgfalt gewacht hätten. Wagten es ja einige Ketzer, wie uns \zB\ der heil.\ \RWbet{Augustinus} von Marcion erzählt, eine Stelle der heiligen Schrift in einigen Exemplaren zu verfälschen: so wurde ihr Betrug sogleich entdeckt, und durch das Ansehen der ältesten Handschriften zurückgewiesen.
\item[\RWbet{Einwurf.}] Die Geschichte meldet doch allerdings von einer Verfälschung der Evangelien, welche unter dem Kaiser \RWbet{Anastasius} im sechsten Jahrhunderte ausgeführt worden ist. Denn wie uns \RWbet{Victor}, Bischof zu Turnon in Afrika, in seiner Chronik erzählt, so ließ dieser Kaiser die Evangelien zu Konstantinopel verbessern, weil sie von ungelehrten Leuten wären geschrieben worden. -- (\RWlat{Collins} in seinem \RWlat{discourse of freethinking.})\RWlit{}{Collins1}
\item[\RWbet{Antwort.}] Diese Erzählung des \RWbet{Victor} ist durchaus unglaubwürdig; denn alle übrigen Schriftsteller schweigen davon gänzlich, da sie doch ohne Zweifel wider den ohnehin verhaßten Kaiser laute Klagen erhoben hätten. Gesetzt aber, es wäre so: doch hätten die Veränderungen, die \RWbet{Anastasius} vornehmen ließ, nach der ausdrücklichen Erzählung \RWbet{Victor's}, nur die Handschriften der konstantinopolitanischen Kirche betroffen; unmöglich hätte er seine Veränderungen auch in der occidentalischen Kirche, über welche er nichts zu gebieten hatte, anbringen können. Wir finden aber die Handschriften beider Kirchen übereinstimmend; und so widerlegt sich das Mährchen von selbst.
\item Die völligste Sicherheit, daß die Bücher des neuen Bundes niemals verfälscht worden seyen, erhalten wir jedoch erst dann, wenn wir die \RWbet{einzelnen Zeiträume}, in welchen dieß hätte geschehen müssen, betrachtet, und bei näherer~\RWSeitenw{102}\ Untersuchung gefunden haben, daß es in keinem derselben möglich gewesen sey.
\end{aufza}
\begin{aufza}[1)]
\item Zu der \RWbet{Apostel Lebzeiten} war eine Verfälschung schlechterdings unmöglich. -- Wir finden, daß die Apostel des Herrn mit der größten Aufmerksamkeit darüber gewacht, daß Niemand eine Lehre, die sie nicht wirklich billigen konnten, unter ihrem Namen verbreite. So warnt \zB\ der heil.\ \RWbet{Paulus} vor Briefen, die man ihm unterschieben könnte, und ordnet an, daß man nur solche Briefe als wirklich von ihm herkommend ansehe, die sich mit einem von seiner eigenen Hand geschriebenen Gruße bezeichnet finden. Wie also hätte es irgend ein Betrüger noch zu den Lebzeiten der Apostel selbst wagen sollen, ihre Schriften zu verfälschen? oder wie wäre es, wenn er es gewagt hätte, nicht gleich entdeckt worden?
\item \RWbet{Bald nach der Apostel Tode} war gleichfalls keine Verfälschung möglich. -- \RWbet{Johannes} lebte bis an das Ende des ersten Jahrhunderts. Um diese Zeit war nun die christliche Religion schon durch alle Theile der damals bekannten Welt verbreitet; es gab schon blühende Gemeinden zu Jerusalem, Antiochien, Alexandrien, in Kleinasien, in Griechenland, zu Rom, vielleicht selbst schon in Indien. In jeder dieser Kirchen hatte man Abschriften von den Büchern des neuen Bundes, und hie und da waren auch wohl die Originale (die Autographa) selbst vorhanden. Reichere Privatpersonen besaßen überdieß noch ihre eigenen Abschriften. Wem wäre es nun, bei dieser Ausbreitung der heil.\ Schrift, möglich gewesen, eine Verfälschung in alle Exemplare hineinzubringen? Wie läßt es sich denken, daß alle Kirchenvorsteher und alle Privatbesitzer zu einer Verfälschung (die doch mit ihrem Vorwissen geschehen müßte) sich verstanden hätten? Da es ferner gebräuchlich war, in jeder gottesdienstlichen Versammlung bald diesen, bald jenen beträchtlichen Abschnitt dieser Bücher vorzulesen: so war der Inhalt derselben jedem einzelnen Christen bekannt. Hätte es also nicht die Gemeinde selbst bemerken müssen, daß man ihr jetzt etwas Anderes lese, als man ihr ehedem vorlas? Ja, was noch mehr ist, seit ihrem ersten Entstehen hatte die Kirche stets mit Ketzern zu streiten: würden wohl diese zu einer Verfälschung der Rechtgläubigen~\RWSeitenw{103}\ geschwiegen haben? würden sie bereit gewesen seyn, die veränderten Lesarten auch in ihre Handschriften aufzunehmen? Und doch stimmen die Handschriften, die sich bei den verschiedenen Ketzerparteien befinden, mit denen der rechtgläubigen Kirche ganz überein.
\item \RWbet{Mit jedem folgenden Jahrhunderte wuchs die Unmöglichkeit einer Verfälschung}. Die Kirche wurde immer ausgebreiteter, die Anzahl der Abschriften immer größer. Seit dem zweiten Jahrhunderte kamen auch \RWbet{Uebersetzungen} der Bücher des neuen Bundes in mehrere andere Sprachen hinzu. Die \RWbet{syrische} Uebersetzung, diejenige nämlich, die man, weil sie ganz wörtlich ist, \RWbet{Peschito} nennt, ingleichen mehrere \RWbet{lateinische}, deren üblichste \RWbet{Itala} hieß, waren sehr frühzeitig, wo nicht im ersten, gewiß im zweiten Jahrhunderte schon vorhanden. (Der Itala bediente sich bereits Tertullian.) \RWbet{Eusebius} und \RWbet{Chrysostomus} schreiben, daß die Schriften des neuen Bundes zu ihrer Zeit in allen Sprachen, selbst bei barbarischen Nationen, \zB\ den Aegyptern, Indiern, Persern und Aethiopiern seyen gelesen worden. Eben so wurden auch je länger je mehrere \RWbet{Bücher über die Bibel} geschrieben, welche Citaten aus ihr, Commentarien und Auslegungen derselben enthielten. Die Citaten der Kirchenväter bis zu dem fünften Jahrhunderte sind so zahlreich, daß man beinahe die ganze heil.\ Schrift, wenn sie verloren ginge, aus ihnen allein wieder herstellen könnte. Da nun alle diese Uebersetzungen, Citaten \usw\ übereinstimmen: so hätte derjenige, der die Bibel späterhin hätte verfälschen wollen, alle diese Schriften zugleich mit verfälschen müssen, was eine offenbare Unmöglichkeit ist.
\item Was endlich \RWbet{die Zeiten nach dem fünften und sechsten Jahrhunderte} anlangt, so haben wir sogar noch \RWbet{Handschriften}, welche in diesen Zeiten geschrieben, und mit unserem gegenwärtigen Texte ganz übereinstimmig sind. Wir haben bereits über 250 verglichene Handschriften von den Büchern des neuen Bundes, die aus verschiedenen Zeiten und Ländern herrühren, die in Bibliotheken, ja oft selbst unter der Erde verborgen gelegen, von denen Niemand etwas gewußt, und die nur erst jetzo der Zufall uns entdecken ließ;~\RWSeitenw{104}\ und wir finden sie alle in Uebereinstimmung mit einander. Die ältesten derselben sind aus dem siebenten und sechsten oder wohl gar aus dem fünften Jahrhunderte \zB\ der \RWlat{Codex Vaticanus} im Vatikan zu Rom, den Montfaucon in das fünfte oder sechste Jahrhundert setzt, Andere aber noch älter machen. Der \RWlat{Codex Alexandrinus}\RWlit{}{CodexAlexandrinus1}, den der Patriarch von Konstantinopel \RWbet{Cyrillus Lukaris} dem Könige \RWbet{Karl~I.} von England zum Geschenke brachte, und der von der heiligen Thekla im Jahre 328 geschrieben seyn soll, von \RWbet{Wettstein} aber in das fünfte Jahrhundert gesetzt wird. Der \RWlat{Codex rescriptus Ephremi Syri} in der Bibliothek zu Paris, ein Pergament, auf welchem einst das griechische neue Testament gestanden, das man jedoch ausgelöscht, und statt dessen einige Schriften des \RWbet{Ephrem Syrus} darauf geschrieben. Doch schimmern noch allenthalben die älteren Schriftzüge des neuen Testamentes hervor, \usw\ In diesen Handschriften liegt nun ein sichtbarer Beweis, daß unser biblische Text wenigstens seit dem sechsten Jahrhunderte keine Veränderung erlitten hat.
\begin{RWanm}So ist denn also gar keiner Widerlegung werth, die Behauptung des Engländers \RWbet{Chubb} (den Einige zu einem Lichtzieher, Andere zu einem Handschuhmacher, noch Andere zu einem Strumpfwirker machen), daß die Bücher des neuen Bundes in den finsteren Zeiten des Papstthumes, (etwa zwischen dem zehnten und vierzehnten Jahrhunderte) von den Geistlichen zu ihrem Vortheile umgearbeitet worden wären. Größere Stücke, ganze Geschichten hätte man schon aus dem Grunde nicht verfälschen können, weil man zugleich die ungeheuere Menge der Kirchenschriftsteller, in welchen diese Begebenheiten citirt werden, hätte verfälschen und überarbeiten müssen. Aber auch jede kleinere Verfälschung war zu jener Zeit nicht möglich; indem der Papst aller Exemplare der Bibel in allen Welttheilen, auch derjenigen, die in den Händen der Ketzer waren, aller Uebersetzungen \usw\ hätte habhaft werden müssen, um seine Verfälschung überall gleichförmig anzubringen. Endlich wenn man in jenem finsteren Zeitalter eine Verfälschung der Bibel versucht hätte: so würde man den Versuch gewiß an solchen Stellen vorgenommen haben, die bei der damaligen Denkart am meisten anstößig geschienen. Wir würden also nicht Stellen in der heiligen Schrift lesen, wie \RWbibel{1\,Tim}{1.\,Tim.}{4}{1--8}, \RWbibel{Mt}{Matth.}{10}{25--27}\ \RWbibel{Gal}{Galat.}{2}{11--15}, \umA~\RWSeitenw{105}
\end{RWanm}
\end{aufza}\par
\RWbet{Einwurf}. Wie kann man doch von einer Unverfälschtheit der Bücher des neuen Bundes und von einer Uebereinstimmung aller Handschriften reden, da \RWbet{Millius} allein in den von ihm verglichenen Handschriften über 30.000 abweichende Lesearten (Varianten) aufgefunden! Im \RWlat{Codice Alexandrino}\RWlit{}{CodexAlexandrinus1} hat der mittelste Querstrich des ersten E in dem Worte \RWlat{EYCEBEIAC} dadurch, daß er in dem \RWlat{O} des Wortes \RWlat{OC} auf der anderen Seite des Pergamentes in der Stelle \RWbibel{1\,Tim}{1\,Tim.}{3}{16} durchschimmerte, aus \RWlat{OS}, \RWgriech{J}\RWlat{S}, \di\ \RWgriech{Je'os} und so aus Jesu einen Gott gemacht. (Siehe \RWbet{Wettstein} \RWlat{Prolegomena in N.\,T.})\RWlit{}{Wettstein1}\par
\RWbet{Antwort.} Ueber die Menge dieser \RWbet{Varianten} muß man sich nicht wundern. Wenn man von irgend einem andern Schriftsteller, \zB\ \RWbet{Cicero}, eben so viele Handschriften besäße, und verglichen hätte: so würde man eben so viele Varianten aufgefunden haben. Die meisten dieser Varianten betreffen bloße Schreibfehler; oder unwesentliche grammatikalische Veränderungen, \zB\ \RWgriech{<'otan sp'arh} oder \RWgriech{<'otan f'uh}; \RWgriech{<ellhvist`as} oder \RWgriech{<'ellhnas}; \RWgriech{>apekr'ijh l'egwn} oder \RWgriech{l'egei}. Einige wenige verändern zwar den Sinn, aber doch nur in Nebendingen, \zB\ \RWgriech{Gadarhn~wn} oder \RWgriech{Gergeshn~wn}; \RWgriech{bhjabar`a}  oder \RWgriech{bhjan'ia} \udgl\  Wo sie auch einen wichtigeren Umstand betreffen, \zB\ \RWgriech[ka`i >ed'akrise]{ka`i >ed'akrusen} \RWbibel{Joh}{Joh.}{11}{35}, oder \RWgriech{o>ud`e <o u<i`os} \RWbibel{Mk}{Mark.}{13}{32}, da kann man nach den Regeln der Kritik bald entscheiden, welche Leseart den Vorzug verdiene. -- Keine dieser Varianten läßt uns über Glaubenslehren oder wichtige Schicksale oder Thaten Jesu in Zweifel. Gesetzt, es hätte seine Richtigkeit, daß \RWbibel{1\,Tim}{1.\,Tim.}{3}{16} \RWgriech{<'os} statt \RWgriech{Je'os} zu lesen sey: die Gottheit Jesu wird nicht aus dieser Stelle allein, sondern aus unzähligen anderen bewiesen.


\RWpar{44}{Historische Glaubwürdigkeit der Bücher des neuen Bundes}
Haben wir uns durch das Bisherige überzeugt, daß die Bücher des neuen Bundes in der That ächt und unverfälscht sind: so wird es nicht schwer seyn, auch ihre \RWbet{Glaubwürdigkeit} zu erkennen. Indem ich mich aber zu dem Beweise~\RWSeitenw{106}\ dieser jetzt herbeilassen will: muß ich erst näher bestimmen, auf welche Gegenstände ich die Glaubwürdigkeit dieser Bücher hier beziehe. Wir wollen aus den Schriften des neuen Bundes, besonders aus den vier Evangelien und der Apostelgeschichte, die wichtigsten Ereignisse, welche sich bei der Entstehung und Ausbreitung des Christenthumes zugetragen haben, und den Charakter Jesu, seine Gesinnungen, Thaten und Schicksale kennen lernen. Auf diese Stücke allein beziehe ich die hier zu erweisende Glaubwürdigkeit jener Schriften, welche ich eben deßhalb die bloß \RWbet{historische} Glaubwürdigkeit derselben nennen will. Ich verstehe sonach unter der historischen Glaubwürdigkeit der Bücher des neuen Bundes nichts Mehreres, als daß nur diejenigen in diesen Büchern uns erzählten Begebenheiten, welche die Entstehung und Ausbreitung des Christenthums und den Charakter Jesu betreffen, sich wirklich zugetragen haben. Wenn uns \zB\ in diesen Büchern erzählt wird, daß Jesus Dieß oder Jenes gesprochen, gethan, oder erlitten habe: so soll es auch \RWbet{wahr} seyn, daß er dieses gesprochen, gethan oder erlitten hat. -- Ganz etwas Anderes wäre die \RWbet{dogmatische} Glaubwürdigkeit, unter welcher man verstände, daß die \RWbet{religiösen Lehren}, welche in diesen Büchern aufgestellt werden, wahre und richtige Lehren seyen. Auch eine dogmatische Glaubwürdigkeit kommt diesen Büchern zu; aber zu ihrem Beweise wird eine ganz andere Vorbereitung erfordert. Jetzt ist es bloß meine Absicht, die eben erklärte historische Glaubwürdigkeit dieser Bücher darzuthun. Und dieses leiste ich, indem ich zeige, daß ihre Verfasser \RWbet{alle diejenigen Beschaffenheiten hatten, welche zu einem glaubwürdigen Zeugen erfordert werden}.

\RWpar{45}{I.~Die Schriftsteller des neuen Bundes hatten allerdings Kenntniß von den Begebenheiten, die wir von ihnen lernen wollen}
Das erste Erforderniß zu einem glaubwürdigen Zeugen ist, daß er selbst \RWbet{Kenntniß} von jenen Wahrheiten habe, welche wir von ihm lernen wollen. Aus den Büchern des neuen Bundes wollen wir, wie gesagt, die wichtigsten Ereig\RWSeitenw{107}nisse, die bei der Entstehung und Ausbreitung des Christenthumes Statt fanden, und den Charakter Jesu, seine Gesinnungen, Thaten und Schicksale erfahren. Ueber diese Gegenstände mußten nun die Verfasser jener Bücher allerdings eine hinreichende Kenntniß haben; denn sie hatten volle \RWbet{Gelegenheit}, sich diese Kenntniß zu verschaffen, und sie hatten auch allen \RWbet{Antrieb} dazu.
\begin{aufza}
\item Sie konnten sich von den wichtigsten Ereignissen, die bei der Entstehung und ersten Ausbreitung des Christenthumes Statt fanden, ingleichen von dem Charakter Jesu, von seinen Gesinnungen, Thaten und Schicksalen vollständig unterrichten; denn sie lebten im ersten Jahrhunderte, also zu eben der Zeit, in welcher sich dieß Alles zutrug. \RWbet{Matthäus} und \RWbet{Johannes} lebten überdieß in eben dem Lande, und waren aus der Zahl jener zwölf innigsten Anhänger Jesu, die er sich gleich beim Antritte seines öffentlichen Lehramtes und in der Absicht beigesellt hatte, damit sie seine ununterbrochenen Begleiter und Beobachter, und auch einst seine Zeugen werden möchten. \RWbet{Markus} soll, wie uns erzählt wird, sein Evangelium nur unter Anleitung \RWbet{Petri}, der gleichfalls Einer von jenen Zwölfen war, geschrieben haben. \RWbet{Lukas} endlich war als Begleiter \RWbet{Pauli} größtentheils Augenzeuge derjenigen Begebenheiten, welche er uns in der Apostelgeschichte erzählt. Sein Evangelium aber setzte er, nach seiner eigenen Erklärung, lediglich nur aus solchen Nachrichten zusammen, die er von Augenzeugen hatte. Zum Wenigsten muß uns ein Jeder zugeben, daß die Verfasser der Evangelien im ersten Jahrhunderte gelebt, und wo nicht geborne Juden und Augenzeugen gewesen, doch ihre Evangelien aus gewissen, schon früher vorhandenen Aufsätzen, welche von Juden herrührten, zusammengesetzt haben. In jedem dieser Fälle aber \RWbet{konnten} sie die hier erforderliche Sachkenntniß im vollsten Maße erlangen.
\item Sie hatten auch alle \RWbet{Ursache}, sich diese Kenntniß zu verschaffen; denn die Begebenheiten, um die es sich hier handelt, waren doch viel zu wichtig, als daß sich irgend Jemand, der die Gelegenheit, sie zu erfahren, hatte, nicht schon aus bloßer Neugierde zur Aufmerksamkeit bewogen gefunden hätte. Besonders aber Personen, welche eine Geschichte der\RWSeitenw{108}selben zu schreiben unternahmen, konnten auf keinen Fall die ihnen hier zu Gebote stehende Gelegenheit unbenützt lassen. Uebrigens gehet es auch aus der Beschaffenheit ihrer Schriften selbst deutlich genug hervor, daß sie sich jene Kenntniß erworben; denn ihre Erzählungen stimmen so gut überein, wie es nie hätte geschehen können, wenn ihre Verfasser, ohne eine Kenntniß von den Begebenheiten zu haben, Jeder aus seinem eigenen Kopfe gedichtet hätte.
\end{aufza}

\RWpar{46}{Auflösung einiger Einwürfe gegen diese Behauptung}
\RWbet{1.~Einwurf.} Die Verfasser der Bücher des neuen Bundes sind allem Anscheine nach \RWbet{äußerst leichtgläubige} Leute gewesen, die Alles annahmen, was ihnen von irgend einem, auch noch so unglaubwürdigen Munde berichtet wurde; sie sind überdieß auch noch \RWbet{Schwärmer} gewesen, welche bei ihrer erhitzten Einbildungskraft Manches zu sehen und zu hören glaubten, was gar nicht wirklich vorging. Besonders \RWbet{Johannes} und \RWbet{Paulus} tragen die sichtbarsten Spuren der Schwärmerei an sich.\par
\RWbet{Antwort.} In diesem Einwurfe gibt man die Wahrheitsliebe der heiligen Schriftsteller zu, und stellt sie nicht als Betrüger, sondern als Selbstbetrogene vor. Hiegegen erinnere ich nun:
\begin{aufza}
\item So groß man sich auch den Grad der \RWbet{Leichtgläubigkeit} oder der \RWbet{Schwärmerei} bei diesen Personen denken möchte: so wird doch diese Annahme allein noch immer nicht hinreichen, um die Entstehung aller der Wundergeschichten, die sie erzählen, auf bloß gewöhnlichem Wege zu erklären. Denn unter dieser Voraussetzung ist \zB\ der Evangelist \RWbet{Johannes} ein wirklicher Augenzeuge; wie aber könnte sich auch der leichtgläubigste und der schwärmerischeste Mensch einbilden zu sehen, daß ein Todter wieder auferstehe, ein Blinder das Gesicht erhalte \usw, wenn nichts von allem dem geschieht? So ferne es sich also bloß darum handelt, zu untersuchen, ob sich gewisse Wunder zur Bestätigung des Christenthums zugetragen haben: so wird durch die Annahme, daß die Ge\RWSeitenw{109}schichtschreiber des neuen Bundes leichtgläubig oder Schwärmer gewesen sind, nichts widerlegt.
\item Doch diese Schriftsteller verdienen in der That nichts weniger, als den Vorwurf der Leichtgläubigkeit, oder auch den der Schwärmerei; denn\par

\vabst \textbf{A.}~sie haben keineswegs die \RWbet{Kennzeichen leichtgläubiger Menschen} an sich.
\begin{aufzb}
\item Leichtgläubige Menschen lassen es sich nicht einmal einfallen, ihre Leser von der Glaubwürdigkeit der Quellen, aus welchen sie schöpfen, zu unterrichten. \RWbet{Lukas} thut aber dieses gleich in der ersten Zeile seines Evangeliums; \RWbet{Johannes} sagt auch einmal, daß er das selbst gesehen habe, was er hier berichtet.
\item Leichtgläubige Menschen erzählen eine Menge von Dingen, die sie vom bloßen Hörensagen haben; unsere Evangelisten dagegen übergehen in ihren Erzählungen sichtbar so manchen Umstand bloß darum stillschweigend, weil sie hierüber aus keiner anderen Quelle, als aus bloßen Volkssagen hätten berichten können. Z.\,B.\ Wer jene Magier gewesen, wie viele es waren, welche Namen sie gehabt, die ganze Jugendgeschichte Jesu bis auf einige Vorfälle in Jerusalem. Ueberhaupt ist fast Alles, was sie uns erzählen, von der Art, daß wir sehr leicht begreifen, wie sie eine ganz sichere Nachricht davon erhalten konnten.
\end{aufzb}\par

\vabst \textbf{B.}~Eben so wenig haben sie die \RWbet{Kennzeichen der Schwärmerei} an sich.
\begin{aufzb}
\item Schwärmer verrathen sich insgemein durch einen affectvollen, dunklen, unordentlichen, sich oft widersprechenden Vortrag. Die Bücher des neuen Bundes sind in einer sehr leidenschaftslosen, ruhigen Sprache geschrieben. Und wenn gleich \RWbet{Paulus} in seinen Briefen viel Lebhaftigkeit des Geistes verräth: so reißt ihn seine Hitze doch nie zu Widersprüchen oder nur Uebertreibungen fort.
\item Schwärmer haben sehr überspannte Begriffe von der Vollkommenheit des Menschen, verbieten \zB\ allen Genuß des sinnlichen Vergnügens, die Ehe \udgl\  -- Die Verfasser des neuen Bundes, namentlich der heilige \RWbet{PauIus}, übertreibet es in keiner ihrer Forderungen, sie prei\RWSeitenw{110}sen den ehelosen Stand, aber empfehlen ihn nur mit vieler Behutsamkeit.
\item Schwärmer, besonders welche Offenbarungen Gottes empfangen zu haben glauben, halten sich eben deßhalb für ausgezeichnete Lieblinge Gottes, und verachten andere Menschen um sich her. Die Verfasser des neuen Bundes sind gar nicht für sich selbst eingenommen, sie gestehen uns ihre eigenen Fehler, sie halten sich für nichts Höheres, als für bloße Diener der Gläubigen \usw
\item Schwärmer pflegen die älteren Offenbarungen gering zu achten, und ihre neuen anzupreisen. Die Schriftsteller des neuen Bundes nennen die Bücher des alten Bundes Gottes Wort, das tauglich sey zu aller Belehrung und allem Unterrichte, empfehlen die Lesung derselben, \usw
\item Schwärmer pflegen ihren Feinden einen wüthenden Haß zu bezeigen, und diesen zugleich durch den Grund, daß ihre Feinde eigentlich Feinde Gottes wären, zu rechtfertigen. Nichts von dem Allen thun die heil.\ Schriftsteller, sie lehren, daß man seine Feinde lieben solle, verordnen Fürbitten für sie, \zB\ namentlich für den Kaiser \usw
\item Schwärmer pflegen Verfolgungen und den Tod nicht sowohl dann erst, wenn sie müssen, geduldig zu ertragen, sondern vielmehr mit Hitze aufzusuchen und zu beschleunigen. Die Verfasser der Bücher des neuen Bundes suchen den Verfolgungen, so lange es ohne Verletzung höherer Pflichten geschehen kann, auszuweichen; kann es aber nicht mehr geschehen: dann wissen sie auch standhaft zu sterben. So läßt sich \RWbet{Paulus} zu Damaskus in einem Korbe über die Stadtmauern herunter, um den Personen, die seinem Leben auf eine sogar nicht rechtliche Weise nachstellen, zu entfliehen; und derselbe \RWbet{Paulus} will zu Philippi bei offenen Thüren nicht aus dem Kerker sich entfernen, sondern er wartet, bis ihn die Obrigkeit selbst für unschuldig erkläre, weil er in rechtlicher Form war eingebracht worden.
\end{aufzb}
\end{aufza}\par
\RWbet{2.~Einwurf.} Aber wie konnten sich die Schriftsteller des neuen Bundes, selbst wenn sie Alle unmittelbare Begleiter Jesu gewesen wären, die verschiedenen Reden, die er bei~\RWSeitenw{111}\ dieser und jener Gelegenheit hielt, merken, und zwar so lange merken, bis sie 10, 12 und mehrere Jahre nach seiner Himmelfahrt, sie schriftlich aufsetzten? Daß sie dieselben, so wie er sie hielt, ihm nachgeschrieben, und sich gleichsam gewisse Tagebücher gehalten hätten, ist wohl nicht anzunehmen.\par
\RWbet{Antwort.} Wir behaupten nicht, daß die Geschichtschreiber des neuen Bundes die Reden Jesu alle von Wort zu Wort so, wie er sie gehalten, liefern; sondern nur, daß sie uns den Hauptinhalt derselben wieder geben, daß sie uns den Sinn, und hie und da auch wohl das Wort (in griechischer Uebersetzung) hinterlassen haben. Dieses nun läßt sich sehr wohl als möglich denken, auch wenn keiner von diesen Schriftstellern ein unmittelbarer Begleiter Jesu gewesen wäre. Denn die Apostel, die bei den Reden Jesu zugegen waren, sind ohne Zweifel junge Männer gewesen, die zwar gar keine gelehrten Kenntnisse, aber doch gute Fähigkeiten, und ein nicht untreues Gedächtniß hatten. Da nun die Reden Jesu niemals sehr lange währten, niemals abstract und schwer zu fassen waren; da er denselben durch eingestreute Erzählungen und Gleichnisse, ohne Zweifel auch durch einen sehr guten und lebhaften Vortrag, überaus viel Annehmlichkeit und eindringende Kraft zu geben wußte: so prägten sie sich dem Gemüthe seiner Zuhörer sehr tief und unvergeßlich ein. Die Apostel besprachen sich über diese Reden gewiß oft unter einander; was also der Eine vielleicht vergessen hatte, wußte der Andere hinzuzufügen. Und je weniger eigene Kenntnisse sie zu Jesu mitgebracht hatten; je mehr es von ihnen galt, daß ihre ganze Weisheit nur Jesus sey: um desto weniger war zu befürchten, daß sie etwas Fremdartiges zusetzen würden. (So finden wir auch noch heut zu Tage, daß ganz gemeine Leute Reden, die ihnen gefallen, wohl Jahre lang und beinahe wörtlich behalten.) Endlich von jenem Tage an, da die Apostel nach Ausgießung des heil.\ Geistes die Predigt des Evangeliums anfingen, hatten sie stets Veranlassung, die Reden ihres Meisters zu wiederholen. Aus ihrem Unterrichte nun konnten diejenigen, welche die Bücher des neuen Bundes verfaßten, die hier vorkommenden Reden Jesu recht wohl erlernt haben. Wie wenig willkürlich sie hiebei verfuhren, ersehen wir aus folgenden Umständen:~\RWSeitenw{112} 
\begin{aufzb}
\item Aus der beinahe wörtlichen Uebereinstimmung, welche so viele Reden Jesu bei den vier heiligen Evangelisten haben.
\item Aus dem gleichen Charakter, den diese Reden auch bei denjenigen Evangelisten tragen, die noch am Meisten von einander abzuweichen pflegen, und in der That auch den verschiedensten Charakter hatten. So hat \zB\ die Rede Jesu bei \RWbibel{Mt}{Matthäus}{9}{25--30}\ ganz das Gepräge der Reden, die bei \RWbet{Johannes}\bindex{Joh} vorkommen.
\item Endlich auch aus dem Umstande, daß sie sehr oft nur anmerken, Jesus habe bei dieser Gelegenheit eine Rede gehalten, ohne die Rede selbst anzuführen; \zB\ \RWbibel{Mt}{Matth.}{4}{23}\ \RWbibel{Mt}{Matth.}{5}{35}\ \RWbibel{Mt}{}{13}{54}\ \RWbibel{Lk}{Luk.}{4}{21}\ \uam
\end{aufzb}\par
\RWbet{3.~Einwurf.} Aber die Schriftsteller des neuen Bundes erzählen uns zuweilen Dinge, die sie unmöglich haben in Erfahrung bringen können; \zB\ was für ein Gebet Jesus in Gethsemane gesprochen habe, während sie Alle schliefen! Wie es mit Jesu Auferstehung zugegangen sey, bei der doch Keiner von ihnen zugegen war! Was in der Rathssitzung, die jenes Blindgebornen wegen gehalten wurde, gesprochen worden sey! Was \RWbet{Pilatus} im Verhöre mit Jesu vorgenommen habe! \usw\par
\RWbet{Antwort.} Von jenem Gebete, und was noch sonst bei jener Gelegenheit vorfiel, konnte sie Jesus nach seiner Auferstehung selbst belehrt haben. Was im Verhöre mit Jesu und bei jener Rathssitzung gesprochen worden sey, konnten sie von Rathsgliedern haben, welche zum Christenthume übertraten; und eben so die wenigen Umstände, die sie bei Jesu Auferstehung erzählen, von jenen Wächtern \usw


\RWpar{47}{II.~Die Schriftsteller des neuen Bundes erzählen uns verständlich und in der ernsten Absicht, daß wir ihnen glauben sollen}
Ein zweites Erforderniß zu einem glaubwürdigen Zeugnisse ist, daß es \RWbet{verständlich}, und in der \RWbet{ernsten Absicht, daß wir es glauben sollen}, abgelegt sey. Beides ist nun bei unsern heil.\ Geschichtschreibern der Fall. Sie schreiben~\RWSeitenw{113}
\begin{aufza}
\item \RWbet{verständlich}. Wer nur die Evangelien gelesen hat, der wird gestehen müssen, daß ihre Verfasser bei aller Kürze doch größtentheils so bestimmt und lichtvoll erzählen, daß man nur selten einen Umstand, der zum gehörigen Verständnisse ihrer Erzählung nöthig wäre, vermisset.
\item[\RWbet{Einwurf.}] Die Erzählung der Evangelisten muß doch nicht überall so deutlich seyn, da der berüchtigte Engländer \RWbet{Woolston} vermeinen konnte, daß uns diese Schriftsteller dort, wo sie gewisse Wunderbegebenheiten erzählen, keine Ereignisse, sondern nur \RWbet{Allegorien} hätten mittheilen wollen; wie er denn auch behauptet, daß viele Kirchenväter diese Vorstellung gleicher Weise gehabt, weil auch sie diese Erzählungen allegorisch auslegten.
\item[\RWbet{Antwort.}] Kein Vernünftiger kann im Ernste glauben, daß die Wundererzählungen in den Büchern des neuen Bundes nach der Absicht ihrer Verfasser nur allegorisch verstanden werden sollten; denn als bloße Allegorien hätten die meisten in diesen Erzählungen vorkommenden Umstände, die bei der Darstellung eines wirklichen Ereignisses überaus zweckmäßig angemerkt sind, gar keinen vernünftigen Sinn, und manche andere Begebenheiten, die keine Wunder sind, hätten dann unmöglich Statt finden können. Wenn \zB\ die Heilung jenes Gichtbrüchigen (\RWbibel{Mt}{Matth.}{9}{1\,ff}) nach \RWbet{Woolston} die Besserung eines Wollüstlings bedeuten sollte: was für einen Sinn hätten die Worte: \erganf{Damit ihr sehet, daß der Sohn des Menschen Macht habe, Sünden zu vergeben, so stehe auf, und wandle}? Und wenn die Heilung jenes Blindgebornen (\RWbibel[Joh.\,9.]{Joh}{Joh.}{9}{}) die Aufklärung eines Unwissenden anzeigt: wie konnte man Jesum beschuldigen, daß er den Sabbath durch jene Heilung geschändet? wie hätte man in einer Rathssitzung untersuchen können, durch welche Mittel er ihn geheilet, \dh\ aufgekläret habe? \usw\ Doch \RWbet{Woolston} selbst glaubte dieß nicht im Ernste, sondern gab nur so vor, um hiedurch, wo möglich, die Wunder des neuen Bundes in den Augen Einiger zweifelhaft zu machen. Die Kirchenväter dagegen allegorisirten zwar oft über die Wundererzählungen des neuen Bundes; \zB\ der heil.\ \RWbet{Augustin} vergleicht die \RWbet{10 Aussätzigen}, die Jesus geheilt hat, mit Menschen, die~\RWSeitenw{114}\ verschiedenen Irrthümern anhingen; aber nie thun sie dieß in der Meinung, als ob diese Erzählungen selbst nur Allegorien wären; sondern weil die Methode des Allegorisirens sehr beliebt war, und sie dabei Gelegenheit fanden, eine Menge der erbaulichsten Bemerkungen beizubringen. Daher legen sie denn auch Begebenheiten, die keine Wunder sind, so aus. Diese Absicht des Allegorisirens gibt unter Anderen der heil.\ \RWbet{Gregor} in seiner \RWlat{Hom.~13.\ in Evang.\ Lucae 1, 22.} sehr deutlich an, wenn er sagt: \erganf{Das Lesestück des heiligen Evangeliums liegt offen vor euch, geliebteste Brüder! und wird euch vorgelesen. Damit aber nicht etwa selbst der klare Sinn desselben Einigen zu hoch scheine: so will ich es in Kürze auf eine Art durchgehen, daß die Auslegung desselben denen, welche sie noch nicht verstehen, bekannt werde, und denen, welche sie bereits kennen, nicht beschwerlich falle.} -- Daß sie hierbei nichts weniger im Sinne gehabt, als die historische Wahrheit der evangelischen Erzählung zu bezweifeln, sagen sie ausdrücklich. So schreibt derselbe heil.\ \RWbet{Gregor} (\RWlat{hom.\ in Luc.\ 7.}): \erganf{So wie derjenige, der ein schön geschriebenes Buch sieht, und nicht lesen kann, zwar die Hand des Schreibers lobt und die Schönheit der Schriftzüge bewundert, aber nicht weiß, was diese Schriftzüge wollen, was sie bedeuten, und nur mit dem Munde lobt, was sein Verstand nicht begreift; während ein Anderer, der nicht nur sehen kann, was alle Menschen können, sondern auch lesen, was derjenige nicht vermag, der es nicht gelernt hat, sowohl das Kunstwerk lobt, als auch den Inhalt (des Buches) begreift: so haben diejenigen, welche die Wunderwerke Christi sahen, ohne zu verstehen, was sie sollen und was sie den Verständigen lehren, diese Wunderwerke nur bewundert, weil sie geschahen; während Andere, was geschah, bewundert, und was sie verstanden, befolgt haben.} -- Und (\RWlat{Hom.\ in Luc.\ 18.}): \erganf{Die Wunder unseres Herrn und Erlösers sind so zu nehmen, geliebte Brüder! daß man glaube, sie seyen wirklich geschehen, daß sie uns aber auch durch ihre Bedeutung etwas lehren. Seine Werke zeigen uns nämlich etwas Anderes durch die Macht (mit der sie gewirkt sind), und wieder etwas Anderes lehren sie uns durch das Geheimniß (das sie umhüllen).}~\RWSeitenw{115}
\item \RWbet{In ernster Absicht.} Schon die Wichtigkeit des Gegenstandes verstattet nicht den Gedanken, daß die Schriftsteller des neuen Bundes nur aus einem bloßen Scherze hätten schreiben sollen. Sie betheuern es aber auch selbst, daß sie in der allerernstesten Absicht, um bei uns Glauben zu finden, geschrieben hätten. So sagt es \zB\ \RWbet{Lukas} im Eingange seines Evangeliums. Und \RWbet{Paulus} (1. Kor. 15, 17.) schreibt sogar: \erganf{Ist Christus nicht auferstanden: so ist unser Glaube eitel, und wir Christen sind die elendesten aller Menschen.}
\end{aufza}

\RWpar{48}{III.~Die Geschichtschreiber des neuen Bundes erzählen auch in der That die Wahrheit. Plan des Beweises dieser Behauptung}
Die Behauptung, daß die Geschichtschreiber des neuen Bundes wirklich die Wahrheit erzählen, ist aus den (\RWparnr{31}) angedeuteten Gründen von einer so hohen Wichtigkeit, daß ich bei diesem Beweise schon etwas umständlicher verweilen darf. Ich will sie denn aus folgenden Umständen darthun:
\begin{aufza}
\item Aus jenem \RWbet{Beifalle}, welchen die Schriften des neuen Bundes vorzugsweise vor andern Geschichtsbüchern über denselben Gegenstand erhielten;
\item aus der \RWbet{inneren Vortrefflichkeit}, welche die religiöse Lehre des neuen Bundes, und die in diesen Büchern erzählte Geschichte Jesu besitzet;
\item aus der \RWbet{Vergleichung der Erzählungen}, die wir in den verschiedenen Büchern dieser Sammlung \RWbet{über Ein und dasselbe Ereigniß} antreffen;
\item aus dem \RWbet{Mangel eines hinlänglichen Beweggrundes zur Lüge} bei den Verfassern dieser Bücher;
\item aus den \RWbet{Beweisen der Aufrichtigkeit und Wahrheitsliebe}, die wir von den Verfassern derselben haben;
\item aus der \RWbet{Vergleichung ihrer Erzählungen mit den Erzählungen anderer Geschichtsschreiber}.
\end{aufza}

\begin{RWanm} 
Einzelne Facta, die uns in diesen Büchern berichtet werden, \zB\ der unschuldige Lebenswandel Jesu, oder die außerordentlichen Thaten desselben, oder sein Kreuzestod, seine Aufer\RWSeitenw{116}stehung \udgl\  haben nebst den so eben aufgezählten noch manche eigene Gründe für sich, \zB\ die Bestätigung durch diesen oder jenen heidnischen Schriftsteller, oder die völlige Unmöglichkeit der Erdichtung einer Begebenheit von dieser Art \udgl\  -- Solche Gründe, die nur die Glaubwürdigkeit eines einzelnen in diesen Büchern erzählten Factums betreffen, ohne die der übrigen bedeutend zu erhöhen, führe ich hierorts nicht an; wohl aber werden einige derselben in dem nächst folgenden Hauptstücke, wo von den einzelnen Begebenheiten gesprochen werden soll, die zur Bestätigung des Christenthumes dienen, noch angedeutet werden.
\end{RWanm}

\RWpar{49}{A.~Aus jenem Beifalle, welchen die Bücher des neuen Bundes vorzugsweise vor anderen historischen Büchern über denselben Gegenstand erhielten}
\begin{aufza}
\item Die Begebenheiten, welche uns in den Büchern des neuen Bundes erzählt werden, und die wir auch vornehmlich aus ihnen kennen lernen wollen, nämlich die Thaten und Schicksale Jesu, und die Ereignisse, die bei der Entstehung und ersten Ausbreitung des Christenthumes Statt fanden, sind, zum Wenigsten theilweise, auch von verschiedenen \RWbet{anderen Geschichtsschreibern} bearbeitet worden. Einige dieser Geschichtsschreiber sind, wie es scheint, noch früher, als die heil.\ Evangelisten, andere mit ihnen gleichzeitig, die meisten aber erst später aufgetreten.
\item Der berühmte Kritiker \RWbet{Fabricius} zählt (in seinem \RWlat{Codex apocryphus N.\,T.}) gegen 50 Titel von Evangelien auf, die es, nebst jenen vieren, welche wir in der Sammlung der Bücher des neuen Bundes haben, einstens gegeben haben soll; nebst diesen Evangelien, die man im Gegensatze von unseren vier Evangelien \RWbet{apokryphische} nennt, gab oder gibt es noch mancherlei anders betitelte Bücher, \zB\ Apostelgeschichten \udgl , welche zum Theil denselben Gegenstand, der in den Büchern des neuen Bundes vorkommt, behandeln. Doch ist es beinahe gewiß, daß mehrere von jenen Titeln, die uns \RWbet{Fabricius} anführt, Schriften, die \RWbet{nie vorhanden waren}, bezeichnen, \zB\ die Titel: \RWlat{Evangelium Pauli, Andreae, Barnabae} \udgl\  Auch ist erwiesen, daß einige dieser Titel nicht mehrere, son\RWSeitenw{117}dern ein und dasselbe Buch bezeichnen, \zB\ die Benennungen: \RWbet{Evangelium der zwölf Apostel}, der \RWbet{Hebräer}, der \RWbet{Nazaräer}, bezeichnen unfehlbar ein und dasselbe Werk; eben so die beiden Namen: Evangelium der \RWbet{Aegyptier} und der \RWbet{Enkratiten}, \usw
\item Daß es aber nebst unsern vier kanonischen Evangelien schon zu der Zeit ihrer Abfassung einige andere Lebensbeschreibungen Jesu gegeben habe, beweiset selbst die Aeußerung \RWbibel{Lk}{Luk.}{1}{1}\ -- Zu dieser Classe scheinen die beiden Evangelien der \RWbet{Hebräer} und der \RWbet{Aegyptier} zu gehören. Diese beiden werden schon von den ältesten Kirchenschriftstellern gebraucht; ja es scheint sogar, daß jene Schriftsteller, welche uns Aussprüche Jesu anführen, ohne den Namen eines Evangelisten dabei zu nennen, sie aus Einem von diesen Evangelien entlehnt, und diese sonach früher als die kanonischen benützt wurden. Kaum aber waren die letzteren erschienen und bekannter geworden: so verließ man jene, und hielt sich ausschließlich an diese. So unterscheidet \zB\ schon \RWbet{Clemens von Alexandrien} die kanonischen Evangelien auf eine sehr ehrenvolle Art von den ägyptischen.
\item Alle andern apokryphischen Evangelien, die es nebst jenen zweien gab, oder die wir noch jetzt haben, sind eines jüngeren Ursprungs, und hatten sich auch eines viel geringeren Anhangs zu freuen, als jene beiden. Eins der ältesten aus ihnen scheint das \RWbet{Evangelium Petri} gewesen zu seyn, welches der Bischof \RWbet{Serapion zu Antiochia} gegen das Ende des zweiten Jahrhundertes prüfte, und, weil es verschiedene Thorheiten enthielt, verwarf. Eben so gibt es auch zwei \RWbet{Evangelien der Kindheit Jesu}, ein Evangelium \RWbet{von der Geburt Marien's}, ein \RWbet{Protoevangelium Jakobi}, ein \RWbet{Evangelium Nikodemi}, \uam ; die aber sämmtlich nur von einigen abergläubigen Leuten in Ehren gehalten, von dem vernünftigen Theile der Christen nie angenommen wurden. Alle tragen die offenbarsten Spuren eines späteren Zeitalters an sich. -- Der heil.\ \RWbet{Irenäus} \zB\ redet von \RWbet{vier} falschen Evangelien, die er nebst unsern vier kanonischen gekannt hatte, aber sämmtlich verwirft. \RWbet{Origenes} kennt deren \RWbet{fünf} oder \RWbet{sechs} (\RWlat{homil.\ in Luc.\ 1.}),~\RWSeitenw{118}\ schreibt ihre Abfassung Ketzern zu, und spricht von ihnen mit Verachtung.
\item Noch weniger Aufmerksamkeit fanden einige andere Aufsätze, die angeblicher Maßen von Heiden herrühren sollen, aber offenbar nur unterschoben sind; \zB\ die \RWlat{Acta Pilati}, Berichte, die dieser Landpfleger an den Kaiser \RWbet{Tiberius} in Betreff Jesu erstattet haben soll; der \RWbet{Brief des Lentulus}, enthaltend die kurze Beschreibung der äußeren Gestalt Jesu und einige seiner Charakterzüge, \udgl\ 
\item Endlich versuchten auch \RWbet{Feinde} des Christenthums einige Mahle, das Leben Jesu zu beschreiben. Hieher gehören jene zwei Lebensbeschreibungen Jesu (\RWbet{Toldot Jeschu}), welche das \RWbet{israelitische} Volk in hebräischer Sprache besitzt; ingleichen die Aeußerungen über Jesum, die sich im \RWbet{Talmud} befinden; \uam\ Alle diese Aufsätze, die erst in späteren Zeiten, etwa im zweiten bis zum vierten Jahrhunderte, geschrieben worden sind, fanden nirgends, als nur bei dem Pöbel derjenigen Partei, bei der sie zum Vorschein kamen, einigen Beifall.
\item Vergleichen wir nun mit diesem nur auf so wenige Menschen und auf so kurze Zeit beschränkten Beifalle, den die so eben erwähnten historischen Aufsätze fanden, den \RWbet{allgemeinen} und schon durch \RWbet{achtzehn Jahrhunderte} fortwährenden Beifall, dessen sich unsere vier heil.\ Evangelien erfreuen; erwägen wir, daß dieser letztere gleich bei Erscheinung derselben, also im ersten Jahrhunderte, \dh\ zu einer Zeit eintrat, wo es vergleichungsweise sehr leicht war, sich über den Grund oder Ungrund der in ihnen erzählten Begebenheiten zu unterrichten; erwägen wir, daß man diese Aufsätze in die Sammlung der \RWbet{heiligen Bücher} der Christen aufnahm, nicht sowohl darum, weil sie die Namen \RWbet{Matthäus, Johannes,} \usw\ an ihrer Stirne trugen; als vielmehr, weil man die in denselben vorgetragene Geschichte Jesu und religiöse Lehre ganz demjenigen gemäß fand, was bisher allgemein geglaubt worden war, und als erwiesen und vernünftig galt; erwägen wir endlich, daß selbst die \RWbet{Feinde} des Christenthums von diesen Evangelien meistens mit Achtung gesprochen, oder es wenigstens nie gewagt, sie der Erdichtung~\RWSeitenw{119}\ und Lüge zu beschuldigen:\RWfootnote{%
	So entlehnt \zB\ \RWbet{Chalcidius}, ein heidnischer Philosoph (\RWlat{Commernt.\ in Timaeum}) aus dem Evangelio Matthäi die Nachricht von dem Sterne, der den Weisen erschienen ist, und nennt bei dieser Gelegenheit dieß Evangelium eine heilige und ehrwürdige Geschichte.}
so können wir nicht umhin, aus Allem diesen zu schließen, daß diese Schriften in der That glaubwürdige Geschichtsbücher seyn müssen.
\item Besonders jene Ereignisse, welche von einer großen Wichtigkeit waren, und die in diesen Büchern als etwas \RWbet{öffentlich Vorgefallenes} erzählt werden, können auf keine Weise erdichtet seyn. Denn weil der Schauplatz dieser Ereignisse \RWbet{Palästina}, zum Theile sogar \RWbet{Jerusalem} selbst war, und weil so viele tausend Juden, mitunter auch sehr viele Heiden (heidnische Proselyten) aus allen Gegenden des römischen Reiches alljährlich zum Passahfeste nach Jerusalem zogen: so hatten diese die beste Gelegenheit, sich nach dem wirklichen Geschehenseyn dieser Ereignisse an Ort und Stelle zu erkundigen. Hätten sich also die Evangelisten erlaubt, dergleichen als öffentlich vorgefallene Ereignisse willkürlich zu erdichten: so wären ihre Schriften gewiß nie mit dem Beifalle aufgenommen worden, den sie doch wirklich erhielten.
\end{aufza}
\begin{RWanm} 
Daher sehen wir denn, daß selbst die apokryphischen Evangelien, so viele sehr unglaubwürdige Mährchen sie auch erzählen, gleichwohl in den Begebenheiten, die sie als öffentlich darstellen, von den kanonischen nicht abweichen, es seyen denn diejenigen, welche aus einem viel späteren Zeitalter herrühren.
\end{RWanm}

\RWpar{50}{B.~Aus der inneren Vortrefflichkeit, welche die religiöse Lehre des neuen Bundes, und die in diesen Büchern erzählte Geschichte Jesu hat}
\begin{aufza}
\item Die \RWbet{religiöse Lehre}, die in den Büchern des neuen Bundes enthalten ist, übersteigt an innerer Vortrefflichkeit Alles, was auch die größten Weisen vor jener Zeit, jüdische sowohl als heidnische, geliefert haben. Einzelne schöne Lehren trifft man freilich auch bei den heidnischen Schriftstellern vor Jesu Zeiten an; aber wie selten werden sie folgerecht durchgeführt! welch eine Menge nutzloser Untersuchungen~\RWSeitenw{120}\ muß man erst durchlesen, bis man auf eine fruchtbare Bemerkung kommt! und wie viele schädliche Irrthümer sind diesen Wahrheiten beigemischt! In den Büchern des neuen Bundes dagegen kommt auch nicht eine einzige unnütze Speculation, und wie viel weniger eine der Tugend und Glückseligkeit der Menschen nachtheilige Behauptung vor. Hier ist Alles, was immer gesagt wird, von einer praktisch-wohlthätigen Tendenz, hier werden uns Ansichten über das wahre Wesen der Tugend und Glückseligkeit, über die einzelnen Pflichten und Obliegenheiten des Menschen, über die Eigenschaften Gottes und unsere Verhältnisse zu ihm, über das andere Leben, mitgetheilt, wie sie sonst nirgends, selbst nicht in den Büchern des alten Bundes, anzutreffen sind.
\item Läßt es sich nun wohl denken, daß so ungelehrte Leute, als die Verfasser der meisten historischen Bücher des neuen Bundes waren, auf so erhabene und den im Judenlande damals allgemein herrschenden Begriffen so sehr entgegengesetzte Ansichten von selbst gekommen wären? oder, wenn sie darauf gekommen, daß sie dieselben nicht lieber in ihrem eigenen, als im Namen eines Andern vorgetragen hätten? So undenkbar dieß ist: so nothwendig nehmen wir an, daß kurz vor Erscheinung dieser Bücher irgend ein mit ganz besonderen Geisteskräften ausgerüsteter Mann, ein Mann, beiläufig wie uns die Evangelien \RWbet{Jesum} selbst schildern, in Palästina aufgetreten sey, und jene besseren Begriffe der Welt bekannt gemacht habe, und daß somit die Hauptsache der evangelischen Geschichte wahr sey.
\item Betrachten wir ferner die ganze \RWbet{Geschichte Jesu}, wie sie erst aus \RWbet{Vereinigung jener vier evangelischen Berichte} sich ergibt: so finden wir 
\begin{aufzb}
\item in dem Charakter dieses Jesu das \RWbet{Ideal menschlicher Vollkommenheit} auf eine so unvergleichliche Weise verwirklicht, und
\item in den Wunderwerken, die er durch Gottes Macht verrichtet, und in den Schicksalen, die er erfährt, ein so \RWbet{vollendetes Gepräge der Gotteswürdigkeit}, daß wir nicht zweifeln können, diese Geschichte sey keine Erdichtung.~\RWSeitenw{121}
\end{aufzb}
\item Der \RWbet{Charakter Jesu}, wie er uns in den vier Evangelien erscheint, übertrifft bei Weitem alle auch bloß erdichtete Beschreibungen eines Ideals menschlicher Vollkommenheit, welche die größten Weltweisen des Alterthums zur Uebung ihres sittlichen Gefühles versuchten. Man vergleiche \zB\ mit unserem evangelischen \RWbet{Jesu} den \RWbet{Justum}, den uns \RWbet{Seneca} (\RWlat{epist.\ 115.})\RWlit{}{Seneca4a} schildert, in dessen Schilderung er doch sichtbar einige Züge von \RWbet{Jesu} erborgt; oder die Heroen, die uns die heidnischen Dichter in ihren Tragödien, oder in ihren Heldengedichten, oder in ihren Mythen dargestellt haben. Nach der Vermuthung einiger neuerer Gelehrten soll der bekannte griechische \RWbet{Mythus} vom \RWbet{Herkules} das Ideal menschlicher Vollkommenheit darstellen; und wie tief unter \RWbet{Jesu} steht dieser \RWbet{Herkules}! -- An unserem \RWbet{Jesu} läßt sich durchaus nicht eine einzige, der menschlichen Vollkommenheit Abbruch thuende Schwäche nachweisen; und von der anderen Seite zeigt sich auch nicht die geringste Affectation von einer Stärke, oder sonst einer Eigenschaft, die sich wohl etwa für Wesen höherer Art, aber nicht für Menschen schickte. Eine solche Eigenschaft wäre \zB\ jene Erhabenheit über alle Rührungen und Gemüthsbewegungen (\RWgriech{>ap'ajeia ka`i >atarax'ia}), welche die stoischen Weltweisen so häufig affectirten, die aber der Weise von Nazareth nicht kennt, \udgl\  Wer einige unparteiische Lobsprüche auf den Charakter \RWbet{Jesu} zu lesen wünscht, sehe \RWbet{Voltaire's} \RWlat{Traité sur la tolérance, ch.\,14;} oder \RWbet{Rousseau's} \RWlat{Émile t.\,3.\ p.\,165.\RWlit{}{Rousseau1} Lettres écrites de la Montagne P.\,1.\ p.\,21.\ 71.\ 117.,}\RWlit{}{Rousseau2} oder \RWbet{Helvetius} \RWlat{de l'homme T.\,1.\ p.\,335.\ 556.}, oder \RWbet{Wieland's} Agathodämon B.\,6.\RWlit{}{Wieland2} \uAm\
\item Da es nun 
\begin{aufzb}
\item viel leichter ist, das wahre Wesen menschlicher Vollkommenheit in einigen allgemeinen Sätzen auszusprechen, als es zu schildern in einzelnen Beispielen, und zu beschreiben, wie sich ein Mann, der dieses Ideal in sich verwirklichet hätte, in jeder Lage des Lebens benehmen müßte; weil zu dem Letzteren erfordert wird, daß man sich eine erschöpfende Kenntniß von den in einer jeden Lage zu berücksichtigenden Umständen erworben, und aus dem allge\RWSeitenw{122}meinen Begriffe der Vollkommenheit gehörig abgeleitet habe, welche Verfahrungsart für diese Umstände die allerzweckmäßigste sey: so läßt sich durchaus nicht erwarten, daß die Schriftsteller des neuen Bundes dieß in so vielerlei Verhältnissen, in welche sie \RWbet{Jesum} gerathen lassen, immer so glücklich getroffen haben würden, wenn er sich nicht in Wirklichkeit so, wie sie ihn schildern, dargestellt hätte. Zumal da,
\item wie gesagt, keine heidnischen Gelehrten, die doch so viel mehr Bildung und Belesenheit hatten, etwas so Vortreffliches zu leisten vermochten.
\item Sollte es gleichwohl möglich seyn, daß dieser Charakter \RWbet{Jesu} bloße Erdichtung wäre, so hätten die Personen, die ihn ersonnen, die richtigste und vollständigste Kenntniß vom wahren Wesen der menschlichen Vollkommenheit gehabt. Bei dieser vollständigen Kenntniß ist es nun wieder nicht zu begreifen, wie sie sich hätten entschließen können, die Welt durch eine solche Erdichtung des Lebens \RWbet{Jesu} zu täuschen.
\item So weise Männer, als die Erfinder dieser Geschichte, wenn sie Erdichtung ist, gewesen seyn mußten, würden ja eben um ihrer Weisheit willen auch so viel Klugheit gehabt haben, um einzusehen, daß es kein schickliches Mittel zur Ausführung ihres Zweckes (nämlich zur Einführung einer besseren Religion) sey, eine so umständliche Geschichte zu erdichten. Denn wie leicht hätte nicht eine oder die andere ihrer Erdichtungen aufgedeckt werden können! Sie hätten sich wenigstens vor so detaillirten Erzählungen, als wir gerade hier durchgehends antreffen, in Acht genommen; sie würden auch nicht den größten Theil der Geschichte von Personen (Markus, Lukas), die keine unmittelbare Zeugen seyn sollen, erzählen lassen, \usw
\item Wofern es endlich wahr ist, was einige Feinde des Christenthumes behaupten, daß die evangelischen Erzähler selbst nicht immer den rechten Sinn gewisser Aeußerungen \RWbet{Jesu} verstanden haben: so ist es um so gewisser, daß diese Aeußerungen nicht von ihnen ausgesonnen sind, und~\RWSeitenw{123}\ daß die hohe Vortrefflichkeit, die der Charakter \RWbet{Jesu} selbst in ihrer Schilderung behält, nicht Dichtung, sondern Thatsache sey!
\end{aufzb}
\item Ein Gleiches gilt auch von den \RWbet{Wunderwerken Jesu}, und von den \RWbet{Schicksalen}, die er erlebt hat. Denn außerdem, daß diese Thaten und Schicksale im genauesten Zusammenhange mit dem Charakter \RWbet{Jesu} stehen, so daß, wenn jene Erdichtung wären, auch dieser nothwendig erdichtet seyn müßte: so liefert die Gotteswürdigkeit, die diese Thaten und Schicksale haben, noch einen neuen Beweis, daß sie nicht erdichtet seyn können. Denn man vergleiche nur die erdichteten Wundererzählungen, die man bei anderen Schriftstellern derselben, oder auch einer anderen, es sey nun früherer oder späterer Zeit, antrifft, mit unseren evangelischen Wundern, und man wird einen sehr großen Unterschied gewahren. In den erdichteten Wundern ist immer so vieles Ueberflüßige, so Vieles, das nur geeignet ist, die Einbildungskraft zu ergötzen, so Vieles, das einem Taschenspieler eher, als einem göttlichen Gesandten ziemt, so vieles Läppische und Abgeschmackte. Nichts von dem Allen läßt sich den Wundern des neuen Bundes vorwerfen. Diese sind durchaus wohlthätig, und ihr Zweck, zur Beglaubigung der Lehre \RWbet{Jesu} zu dienen, ist gar nicht zu verkennen. Hier kommen sogar keine Umstände vor, die nur die Einbildungskraft unterhalten (Jesus berührt nur den Kranken, und er wird schon gesund). Sie werden von dem, der sie verrichtet, nie pomphaft angekündiget, und noch viel weniger geht etwas Unanständiges bei ihnen vor. Je genauer man sie nach allen Umständen betrachtet und zergliedert, um desto völliger überzeugt man sich, daß sie ganz gotteswürdig seyen. Und eben dieß sind auch die Schicksale, die \RWbet{Jesus} von seiner Geburt an bis zu seiner Aufnahme in den Himmel erfährt. Sie fügen sich alle so ganz entsprechend für den Zweck, um ihn zum Weisesten und zum Besten aller Sterblichen heranzubilden, und dann Gelegenheit ihm zu verschaffen, daß er der Welt Beweise von seiner Weisheit und Tugend gebe, um die Aufmerksamkeit auf seine Lehre zu richten, um alle Schriftstellen des alten Bundes, die man auf den Messias deutete, an ihm in~\RWSeitenw{124}\ Erfüllung zu bringen, um ihn als einen göttlichen Gesandten auszuzeichnen; sie haben mit einem Worte ganz das Gepräge der göttlichen Weisheit an sich.
\item Ist es nun glaublich, daß solche Wunderthaten und solche Schicksale von Menschen ausgesonnen seyen? Zumal, da ihre Zweckmäßigkeit zuweilen erst dann zum Vorscheine kommt, wenn wir die Sache so auffassen, wie sie, nicht aus der Erzählung eines einzelnen Evangelisten, sondern aus der Vergleichung Aller hervorgeht?
\end{aufza}
\begin{RWanm} 
\RWbet{Wizmann} (Geschichte Jesu nach Matthäus als Selbstbeweis ihrer Zuverlässigkeit betrachtet. Leipzig, 1787)\RWlit{}{Wizenmann1} war meines Wissens der Erste, der einen freilich nur unvollkommenen Versuch anstellte, die Wahrheit der evangelischen Geschichte aus ihrer inneren Vortrefflichkeit zu beweisen. Etwas zu übereilt, wir mir däucht, erklärte \RWbet{Eichhorn} (in der orientalischen Bibliothek) die Sache für etwas Unmögliches, obgleich ich gern gestehe, daß sich auf diese Art schwerlich die Wahrheit eines jeden einzelnen Umstandes in der evangelischen Geschichte erweisen lasse, wie auch, daß man bei vielen Versuchen sich leicht selbst täuschen könne.
\end{RWanm}

\RWpar{51}{C.~Aus der Vergleichung der Erzählungen, die wir in den verschiedenen Büchern dieser Sammlung über Ein und dasselbe Ereigniß antreffen}
\begin{aufza}
\item In den historischen Büchern des neuen Bundes, besonders in den drei Evangelien \RWbet{Matthäi, Marci} und \RWbet{Lucä} (welche man eben deßhalb auch die \RWbet{harmonischen} zu nennen pflegt) gibt es
\begin{aufzb}
\item sehr viele Erzählungen, die beinahe wörtlich mit einander übereinstimmen; \zB\ die Geschichte der Taufe \RWbet{Jesu} durch \RWbet{Johannes} bei \RWbet{Matthäus} (\Ahat{\RWbibel{Mt}{}{3}{13--17}}{4,13--17.}), \RWbet{Markus} (\RWbibel{Mk}{}{1}{9--11}) und \RWbet{Lukas} (\RWbibel{Lk}{}{3}{21--22}); die Verklärung \RWbet{Christi} bei \RWbet{Matthäus} (\RWbibel{Mt}{}{17}{1--9}) und \RWbet{Markus} (\RWbibel{Mk}{}{9}{2--10}); das Leiden Jesu in Gethsemane, \RWbet{Matthäus} (\RWbibel{Mt}{}{26}{36}) und \RWbet{Markus} (\RWbibel{Mk}{}{14}{32}); \uam
\item Dagegen gibt es auch wieder bei jedem Evangelisten Stellen, in welchen er von der Erzählung der Anderen auf eine solche Art abweicht, daß diese Abweichung den~\RWSeitenw{125}\ größten Anschein eines Widerspruches hat, und oft auch ein wirklicher ist, obgleich immer nur einen an sich sehr gleichgültigen Umstand betreffend. So lautet \zB\ die Geschlechtstafel \RWbet{Jesu} ganz anders bei \RWbet{Matthäus} (\RWbibel{Mt}{}{1}{1\,ff}) als sie bei \RWbet{Lukas} (\RWbibel{Lk}{}{3}{23\,ff}) lautet; so wird die Versuchungsgeschichte \RWbet{Jesu} in einer anderen Ordnung bei \RWbet{Matthäus} (\RWbibel{Mt}{}{4}{1--11}), in einer anderen bei \RWbet{Lukas} (\RWbibel{Lk}{}{4}{1--13}) erzählt. Sehr merkliche Verschiedenheiten kommen auch in der sogenannten Bergpredigt bei \RWbet{Matthäus} (\RWbibel{Mt}{}{5}{1\,ff}) und bei \RWbet{Lukas} (\RWbibel{Lk}{}{6}{12\,ff}) vor. In der Geschichte des römischen Hauptmannes wird bei \RWbet{Matthäus} (\RWbibel{Mt}{}{8}{5}) erzählt, daß er in eigener Person erschienen ist, bei \RWbet{Lukas} (\RWbibel{Lk}{}{7}{3}) kommen nur Abgeordnete vor. Nach \RWbet{Matthäus} (\Ahat{\RWbibel{Mt}{}{8}{28}}{3,28.}) heilt Jesus in der Gegend der Gergesener zwei Besessene, bei \RWbet{Markus} (\RWbibel{Mk}{}{5}{1}) und \RWbet{Lukas} (\RWbibel{Lk}{}{8}{26}) heißt diese Gegend Gerasa, und es ist nur von Einem Besessenen die Rede. Bei \RWbet{Matthäus} (\RWbibel{Mt}{}{17}{1}) wird die Zeit, die zwischen gewissen Reden Jesu und der Verklärung auf dem Berge verstrich, auf sechs, bei \RWbet{Lukas} (\RWbibel{Lk}{}{9}{28}) auf acht Tage angesetzt. Nach \RWbet{Matthäus} (\RWbibel{Mt}{}{20}{21}) ist es die Mutter der beiden Söhne des \RWbet{Zebedäus}, nach \RWbet{Markus} (\Ahat{\RWbibel{Mk}{}{10}{35}}{10,15.}) sind es die Söhne selbst, die unserem Herrn die ehrgeizige Bitte vorgetragen, sie in dem neuen Reiche zu seiner Rechten und Linken sitzen zu lassen. Bei \RWbet{Matthäus} (\RWbibel{Mt}{}{26}{7}) gießet \RWbet{Maria} die köstliche Salbe über das Haupt \RWbet{Jesu} aus, bei \RWbet{Johannes} (\RWbibel{Joh}{}{12}{3}) salbet sie seine Füße. Bei \RWbet{Matthäus} (\RWbibel{Mt}{}{20}{29}) heilt \RWbet{Jesus} zwei Blinde, als er von Jericho weggeht; bei \RWbet{Markus} (\RWbibel{Mk}{}{10}{46}) ist es nur Einer; bei \RWbet{Lukas} (\RWbibel{Lk}{}{18}{35}) geschieht es, als \RWbet{Jesus} sich der Stadt Jericho nähert, \uam
\end{aufzb}
\item Diese Abweichungen können uns zwar in Betreff derjenigen Stücke, bei welchen sie eben vorhanden sind, in einiger Ungewißheit darüber lassen, wie es sich eigentlich begeben habe; unser Vertrauen aber zu denjenigen Erzählungen, in welchen die heiligen Evangelisten mit einander einstimmen, muß hiedurch nur um so größer werden. Denn die Erscheinung, daß sie hier einstimmen, läßt sich auf keine andere Weise~\RWSeitenw{126}\ erklären, als durch die Annahme, daß sich die Sache in der That so verhalten habe, wie sie dieselbe erzählen. Oder wie anders wollte man diese Uebereinstimmung erklären?
\begin{aufzb}
\item Aus einem \RWbet{bloßen Zufalle?} -- Ein Zufall kann wohl machen, daß zwei oder mehrere Lügner in einigen wenigen, aber gewiß nicht in so vielen Umständen übereinstimmend erzählen, als es in unseren vier Evangelien der Fall ist.
\item Aus einer \RWbet{Verabredung?} -- Allein wenn sich die Evangelisten verabredet hätten: so würde sich ihre Verabredung gewiß auf alle Theile ihrer Geschichte erstreckt haben. Sie würden nicht in einigen (und zwar den meisten) Theilen ihrer Erzählung sich bis auf die Worte, deren sie sich bedienen wollen, verabredet, in einigen anderen Theilen wieder so Vieles unbestimmt gelassen haben. Die wörtliche Uebereinstimmung, die wir in so vielen Theilen der evangelischen Geschichte antreffen, beweiset deutlich, daß der eine Evangelist die Erzählung des Anderen (etwa \RWbet{Lukas} jene des \RWbet{Matthäus}, \RWbet{Markus} jene des \RWbet{Matthäus} und \RWbet{Lukas} zugleich) oder daß Alle irgend einige noch ältere Aufsätze vor sich gehabt haben. Hätten sie also die Absicht gehabt, eine recht übereinstimmende Geschichte zu hinterlassen, gleichviel, ob sie auch wahr sey oder nicht: so würde nicht ein Jeder von dem Aufsatze, den er vor sich liegen hatte, in so manchen Stücken abgewichen seyn, und eine andere Darstellung geliefert haben.
\item Aus einem \RWbet{gemeinschaftlichen Vortheile?} -- Es können freilich zuweilen Menschen in einer gewissen Aussage, ob sie gleich lügenhaft ist, übereinstimmen, weil es vielleicht einem Jeden aus ihnen Vortheile bringt, die Sache gerade so, und nicht anders darzustellen. Allein in den vier Evangelien gibt es so viele Erzählungen, die übereinstimmend lauten, und wieder so viele andere, in denen Abweichungen herrschen, welche doch Alle ein und dasselbe Interesse haben: so daß sich durchaus nicht absehen läßt, was für ein Vortheil die Verfasser bestimmt haben sollte, in jenen übereinzustimmen, in diesen abzuweichen. Man denke nur an die oben angeführten Beispiele zurück, um hievon überzeugt zu werden.~\RWSeitenw{127}
\item Es bleibt also keine andere Erklärung übrig, als daß diese vier Männer Wahrheit gesucht, und daß sie in ihren Erzählungen dort mit einander übereinstimmten, wo sie die vorliegende Quelle glaubwürdig fanden, dort von einander abwichen, wo sie aus anderen Quellen (mündlichen Berichten, eigener Erinnerung \udgl ) einen Irrthum zu entdecken glaubten. Sie haben also mit prüfendem Geiste geschrieben, und da sie dem Zeitalter der Begebenheiten so nahe lebten, zum Theile selbst Augenzeugen waren: so können sie nicht hintergangen worden seyn; so ist dasjenige, was sie mit gemeinschaftlicher Uebereinstimmung erzählen, die sicherste Wahrheit.
\end{aufzb}
\end{aufza}

\RWpar{52}{D.~Aus dem Mangel jedes Beweggrundes zur Lüge bei den Verfassern dieser Bücher}
So eben habe ich gezeigt, daß die heiligen Schriftsteller bei der Darstellung ihrer Geschichte wirklich nach Wahrheit geforscht haben. Jetzt werde ich aber dieß aus einem anderen Grunde beweisen, nämlich daraus, weil sich \RWbet{gar kein Beweggrund} entdecken läßt, der sie zu einer Lüge hätte verleiten können. Dieß zeige ich, indem ich alle Beweggründe, auf die man hier etwa eine Vermuthung haben könnte, der Reihe nach prüfe.
\begin{aufza}
\item Derjenige Beweggrund, auf den man am Ehesten noch verfallen könnte, ist \RWbet{die Absicht, die Sache des Christenthums in eine um desto bessere Aufnahme zu bringen}. Man könnte argwöhnen, daß die Schriftsteller des neuen Bundes verschiedene Wundergeschichten nur darum ausgedacht, um die Sache \RWbet{Jesu} desto empfehlender zu machen. Allein
\begin{aufzb}
\item wie wir noch künftig sehen werden, so läßt sich, auch ohne die Wahrhaftigkeit des evangelischen Berichtes vorauszusetzen, darthun, daß sich mit der Person unseres Herrn mindestens einige Wunder unläugbar zugetragen haben. Wozu also wäre es nöthig gewesen, zu diesen noch mehrere andere zu dichten? Hätten wohl so verständige Männer, als es die Verfasser unserer vier Evan\RWSeitenw{128}gelien ohne Zweifel waren, nicht einsehen sollen, daß die gute Sache des Christenthums bei jeder Erdichtung von dieser Art gefährde? nicht einsehen sollen, daß ein erdichtetes Wunder sehr leicht verrathen werden könne, und dann auch die wirklichen ungewiß machen würde?
\item Hiezu kommt noch, daß die Schriftsteller des neuen Bundes von ihrer zur Ehre des Christenthumes unternommenen Arbeit nichts als Verfolgungen erfuhren, also unmöglich aus einem anderen, als einem sittlichen Beweggrunde dazu bestimmt werden konnten. Aus einem sittlichen Beweggrunde aber kann man sich nur in dem einzigen Falle zu einer Lüge oder Erdichtung entschließen, wenn man des Irrthums lebt, daß es erlaubt sey, zu einem guten Zwecke zu lügen. Allein die heiligen Schriftsteller geben uns die unzweideutigsten Beweise, daß sie einem solchen Irrthume keineswegs zugethan waren. So schreibt \zB\ der heil.\ \RWbet{Petrus} im ersten seiner Briefe (\RWbibel{1\,Petr}{}{2}{1}): \erganf{Leget ab alle Bosheit, Arglist und Heuchelei, allen Neid und alle Verleumdung, und strebet, als neugeborne Kinder, nach der unverfälschten Milch des Wortes.} Und im zweiten Briefe \RWbet{Petri} (\RWbibel{2\,Petr}{}{1}{12}) heißt es: \erganf{Daher werde ich nie unterlassen, euch daran zu erinnern, wiewohl ihr es schon wisset, und fest seyet im Besitze der Wahrheit; denn ich halte es für Pflicht, euch durch Ermahnungen zu ermuntern, so lange ich diese Hütte bewohne, da ich weiß, daß ich diese Hütte bald verlassen muß, wie unser Herr Jesus Christus mir geoffenbaret hat. (Vermuthlich eine Beziehung auf \RWbibel{Joh}{Joh.}{21}{18\,ff}) Ich will also unablässig dafür sorgen, daß ihr euch nach meinem Hinscheiden daran erinnern könnet, \RWbet{daß wir gewiß nicht schlau ersonnenen Mährchen folgten, als wir euch die Macht und Erscheinung unsers Herrn Jesu Christi bekannt machten}; denn wir waren Augenzeugen seiner Herrlichkeit, als er von Gott dem Vater Ehre und Preis empfing, bei jener herrlichen Verklärung, da über ihn die Stimme erscholl: Dieser ist mein geliebter Sohn, an dem ich Wohlgefallen habe, und hörten diese Stimme vom Himmel erschallen, indem wir mit ihm auf dem heiligen Berge waren.} Der heil.\ \RWbet{Paulus} schreibt~\RWSeitenw{129}\ von sich selbst (\RWbibel{2\,Kor}{2.\,Kor.}{4}{1}): \erganf{Da ich dieß apostolische Lehramt überkommen habe: so will ich mit der Gnade, die mir ward, nicht ermüden. Aber \RWbet{ich verschmähe schändliche Heimlichkeit und Taschenspielerkunst; und nie verfälsche ich Gottes Wort; sondern durch offene Darlegung der Wahrheit stelle ich mich jeder Prüfung der Menschen dar vor Gott}.} Im Briefe an die Epheser (\RWbibel{Eph}{}{4}{25}) schreibt er: \erganf{Leget die \RWbet{Lügen} ab, und Jeder rede mit seinem Nächsten die \RWbet{Wahrheit}; denn wir sind Glieder unter einander.} In seinem Briefe an \RWbet{Titus} (\RWbibel{Tit}{}{1}{13}) spricht er von der Lügenhaftigkeit der Kretenser, und trägt dem \RWbet{Titus} auf: \erganf{Weise sie deßwegen nur nachdrücklich zurecht, damit sie im Glauben zu gesunden Begriffen gelangen, und nicht jüdischen Mährchen und Menschensatzungen anhangen, die von der Wahrheit abweichen.} -- Männer, die so geschrieben haben, konnten unmöglich glauben, daß es erlaubt wäre, zur besseren Aufnahme des Christenthums Wundererzählungen zu erdichten. Von ihnen werden auch \RWbet{Markus} und \RWbet{Lukas} dieselben Gesinnungen angenommen haben.
\end{aufzb}
\item Hiedurch ist nun größtentheils auch schon folgender zweite Verdacht widerlegt, daß die Verfasser unserer vier Evangelien vielleicht eine und die andere \RWbet{besondere Lieblingsneigung} gehabt, die zu verbreiten sie gewisse Reden und Thaten \RWbet{Jesu} erdichtet hätten? In solcher Absicht sind allerdings einige Pseudoevangelien, und andere unterschobene und lügenhafte Schriften abgefaßt worden. Aber man sieht es diesen auch insgesammt an, daß und zu welchem Zwecke sie erdichtet worden sind. Nicht also ist es mit unseren vier Evangelien der Fall.
\begin{aufzb}
\item Der heilige \RWbet{Matthäus} sucht zwar bei jeder Gelegenheit zu beweisen, hier wäre eine Weissagung des alten Bundes erfüllt worden; aber dieses Bestreben konnte ihn wohl nicht zur Erdichtung jener Wunder, die er erzählt, verleiten; denn er erzählt ja fast dieselben, die auch die übrigen Evangelisten melden; auch befindet sich in den Büchern des alten Bundes nicht eine einzige Stelle, die~\RWSeitenw{130}\ auf ein specielles Wunder des Messias gedeutet werden müßte; und \RWbet{Matthäus} führt nirgends, wenn er dergleichen Wunder erzählt, Stellen des alten Bundes, die so erfüllt wären, an.
\item \RWbet{Johannes} sucht bei jeder Gelegenheit die göttliche Würde \RWbet{Jesu} recht in das Licht zu stellen; aber auch dieser Zweck hatte ihn zu keiner Lüge verleiten können. Wie wir aus den übrigen Evangelisten und aus den Briefen \RWbet{Pauli} ersehen: so muß sich \RWbet{Jesus} über seine göttliche Würde deutlich genug erklärt haben, und es war also nicht nöthig, um diese Würde in ein helles Licht zu setzen, unserm Herrn Reden, welche er nie gehalten hatte, in den Mund zu legen. Noch weniger war es zu diesem Zwecke nöthig, besondere Wunder zu erdichten. Denn aus dergleichen Wundern, so groß sie immer gewesen seyn möchten, ließ sich die Gottheit Jesu ohnehin nie beweisen.
\end{aufzb}
\item Jemand möchte vielleicht auf den Gedanken kommen, zu sagen, daß die Verfasser der Evangelien, weil sie zu Gunsten des Christenthumes schrieben, \RWbet{dafür bezahlt, oder auf irgend eine andere Art} (etwa durch Ehrenämter) \RWbet{belohnt worden wären.} -- Gesetzt, es hätte wirklich ein Christ jener Zeit aus einem so eigennützigen Grunde den Entschluß gefaßt, ein Evangelium \RWbet{Jesu} zu schreiben: auch da noch ließe sich nicht vermuthen, daß er Erdichtungen in seine Geschichte eingewebt haben würde. Bevor man dem Verfasser den versprochenen Preis zuerkannt hätte: würde sein Aufsatz von den Aposteln und andern christlichen Lehrern geprüft worden seyn. Daß aber diese keinen Beweggrund gehabt, eine Geschichte zu billigen, die lügenhaft ist, habe ich eben gezeigt.
\item Endlich könnte man sagen, daß die \RWbet{Liebe zum Wunderbaren}, die schon so viele erdichtete Wunder in den apokryphischen Evangelien und anderwärts zum Vorscheine gebracht hat, auch die Verfasser unserer vier heil.\ Evangelien zu mancher Vergrößerung und Erdichtung verleitet habe. Eine genauere Betrachtung aber zeigt, daß die Verfasser unserer vier Evangelien den Fehler der Wundersucht gar nicht an sich haben. Die Wunder, welche sie erzählen, haben immer einen~\RWSeitenw{131}\ vernünftigen Zweck, und sind unter den Umständen, unter denen sie sich, ihrer Erzählung nach, zugetragen haben, beinahe nothwendig gewesen. Wofern es also wahr ist, daß \RWbet{Jesus} überhaupt einige Wunder gewirkt (wie wir das in der Folge sehr deutlich sehen werden): so ist es höchst wahrscheinlich, daß er gerade diejenigen gewirkt habe, welche uns in den Evangelien erzählt werden.
Wären unsere Evangelisten wundersüchtige Leute gewesen: so würden sie jene abgeschmackten Wundererzählungen, welche wir in den apokryphischen Evangelien antreffen (deren einige man schon gewiß auch zu ihrer Zeit herumtrug), nicht weggelassen, sondern begierig aufgenommen haben.
\item Und so gibt es denn schlechterdings keinen Beweggrund, der die Verfasser der Bücher des neuen Bundes zu Lügen verleiten konnte. Fühlt der Mensch aber keine Versuchung zur Lüge, hat er keinen Beweggrund dazu: so ist es ihm auch eben deßhalb \RWbet{psychologisch unmöglich}, daß er sich eine Lüge zu Schulden kommen lasse. Die heil.\ Schriftsteller haben also geschrieben, was sie für Wahrheit hielten. Sie hatten aber auch hinlängliche Kenntniß davon, was sich in jenen Tagen zutrug; also ist das, was sie berichten, Wahrheit.
\end{aufza}

\RWpar{53}{E.~Aus den Beweisen der Aufrichtigkeit und Wahrheitsliebe, die wir von diesen Schriftstellern haben}
Alles, was ich bisher (\RWparnr{48--52}) über die historische Glaubwürdigkeit der Bücher des neuen Bundes behauptet habe, erwies ich, ohne mich auch nur ein einziges Mal auf den rechtschaffenen Charakter, und auf die Wahrheitsliebe derjenigen zu berufen, die hier als Zeugen auftraten. Diese Behutsamkeit war nöthig, wenn das bisher Erwiesene einen so hohen Grad von Gewißheit erreichen sollte, als eben nöthig ist, um einen Glauben an Wunder darauf zu gründen. Indeß ist es doch gewiß, daß die Geschichtschreiber des neuen Bundes auch Männer von dem \RWbet{rechtschaffensten Charakter} waren, und aus der \RWbet{reinsten Liebe zur Wahrheit} schrieben. Es verlohnt sich also der Mühe, auch dieses zu zeigen, weil es das Zutrauen, das wir zu diesen Schrift\RWSeitenw{132}stellern hegen, doch immer vergrößern muß, wenn wir erfahren, daß sie auch diese Eigenschaft gehabt. Ueberdieß gibt es auch andere Ereignisse in ihren Schriften, die keine Wunder sind, die wir daher, wenn wir uns erst von der Wahrheitsliebe der heil.\ Schriftsteller gehörig überzeugt haben, allerdings auch auf ihr bloßes Wort werden annehmen können.
\begin{aufza}
\item Die Schriftsteller des neuen Bundes sind, so weit die Geschichte sie kennt, Männer von dem \RWbet{vortrefflichsten Charakter} gewesen. \RWbet{Johannes, Petrus} (der einen so großen Einfluß auf die Entstehung und den Inhalt des Evangeliums \RWbet{Marci} gehabt) und \RWbet{Paulus} (der uns zugleich für den Charakter \RWbet{Lucä}, seines Freundes und Geschichtschreibers, bürgt) sind Männer gewesen, von welchen die Geschichte nichts als ruhmwürdige Thaten kennt. Der Eifer, den sie für die Ausbreitung des Christenthumes und für die Beförderung der Tugend und Glückseligkeit unter den Menschen an den Tag gelegt haben, läßt keinen Zweifel übrig, daß sie den edelsten Männern, die je gelebt, beigezählt werden müssen. Die Schriften selbst, die wir noch von ihnen übrig haben, sind ein getreuer Spiegel ihres vortrefflichen Charakters. Oder wer müßte nicht mit Achtung und Liebe erfüllt werden gegen einen Mann, der von sich schreiben konnte, was wir \zB\ \RWbibel{2\,Kor}{2.\,Kor.}{10}{1--12}\ oder \RWbibel{Eph}{Ephes.}{1}{1--26}\ \uaO\ lesen! -- Männer von so vielen Tugenden haben gewiß auch die Tugend der Aufrichtigkeit und Wahrheitsliebe gehabt; und wenn sie sich entschloßen, eine Geschichte der Stiftung und ersten Ausbreitung des Christenthumes zu schreiben: so haben sie sich bei diesem Geschäfte sicher von keinen andern Rücksichten, als von der reinsten Liebe zur Wahrheit leiten lassen.
\item Doch dieses beweiset auch ihre \RWbet{Arbeit selbst}; denn in den Büchern des neuen Bundes kommen die unverkennbarsten Spuren der Aufrichtigkeit und Wahrheitsliebe vor. Wir können wohl schließen, daß Jemand etwas aus reiner Wahrheitsliebe erzähle, wenn sich aus keinem andern Grunde erklären läßt, warum er dieses erzähle, es wäre denn nur darum, weil er es für seine Pflicht gehalten, die Sache, wie sie ist, zu schildern. Dieser Fall ist aber offenbar dann vorhanden, wenn Jemand Dinge erzählt, die er aus anderen Gründ\RWSeitenw{133}en, \zB\ aus Selbstliebe, Eitelkeit, Menschenfurcht, Vorliebe für Andere \udgl , lieber hätte verschweigen müssen. Und dieses findet sich bei den Verfassern der Bücher des neuen Bundes häufig. Sie erzählen
\begin{aufzb}
\item so Manches, wovon sie \RWbet{Menschenfurcht} hätte abhalten sollen. Schon der Entschluß, eine Geschichte \RWbet{Jesu} zu schreiben, welche ihn, den die Obrigkeit für einen Missethäter erklärt und hingerichtet hatte, als einen göttlichen Gesandten, ja als den längst erwarteten Messias darstellt; welche alle Erwartung der Nation von einer irdischen Herrschaft mit Einem Male vernichtet; welche ihre Gebrechen und Laster und das abscheuliche Betragen des hohen Rathes vor aller Welt aufdeckt: schon der Entschluß, eine solche Geschichte zu \RWbet{schreiben}, und unter \RWbet{eigenem Namen} herauszugeben, setzte in jener Zeit eine mehr als gewöhnliche Freimüthigkeit voraus. Verfolgte man alle Christen: so hatten diejenigen, die zur Verbreitung des Christenthumes durch Schriften beigetragen, gewiß eine um desto grausamere Verfolgung zu befürchten. Lesen wir aber erst, \RWbet{wie}  die Verfasser schrieben: so werden wir über den Muth, den sie hiebei an den Tag legten, erstaunen. Mit aller Vollständigkeit erzählt uns \RWbet{Matthäus} die bitteren Strafreden \RWbet{Jesu} gegen die Zunft der Pharisäer und Schriftgelehrten; mit aller Genauigkeit wagt er es, das widerrechtliche Betragen des hohen Rathes, die Begebenheit mit den Wächtern, die man bestach, und andere dergleichen Ereignisse, acht Jahre nach dem Tode \RWbet{Jesu} in eben demselben Lande, wo alles dieß geschehen war, niederzuschreiben. -- Man wende nicht ein, zu dieser Kühnheit habe die Evangelisten nicht sowohl Liebe zur Wahrheit, als vielmehr leidenschaftlicher Haß gegen die Feinde und Mörder ihres Herrn verleitet. -- Hätte sie Leidenschaft beherrscht: so hätten sie sich gewiß ein und den andern Ausbruch ihres Hasses erlaubt: so würden sie bald da, bald dort in Uebertreibung verfallen, so würden sie nicht auch selbst ihren Feinden Gerechtigkeit widerfahren lassen, und das Gute, das sich an ihnen befindet, anerkennen. Man lese \zB , was sie von des \RWbet{Pilatus} Bestreben, \RWbet{Jesum} zu retten,~\RWSeitenw{134}\ erzählen, oder von \RWbet{Nicodemus} oder \RWbet{Gamaliel}, die doch auch Mitglieder des hohen Rathes waren; oder die Bitte \RWbet{Jesu} am Kreuze: \erganf{Vater! vergib ihnen; denn sie wissen nicht, was sie thun!} \udgl\ 
\item Sie erzählen so Manches, wovon sie eine gewisse \RWbet{parteiliche Vorliebe für ihren Meister}, und die Besorgniß, sein Ansehen hiedurch in den Augen Einiger zu schwächen, hätte abhalten sollen. -- So erzählen sie uns, daß \RWbet{Jesus} bei Mehreren seiner eigenen Anverwandten (bei seinen Brüdern und Schwestern, wie sie sich ausdrücken) keinen Glauben gefunden habe. Sie gebrauchen den Ausdruck, daß er zu Nazareth wegen des Unglaubens der Einwohner keine Wunder habe wirken können. Sie lassen uns vermuthen, daß \RWbet{Johannes} der Täufer am Ende selbst gezweifelt habe, ob \RWbet{Jesus} der wirkliche Messias sey, oder nicht. Sie machen uns bekannt, daß unser Herr von seinen Feinden ein Fresser, Weinsäufer, Zöllner -- und Sünderfreund (\RWbibel{Mt}{Matth.}{11}{19}) genannt worden sey, daß man ihm Wahnsinn vorgerückt (ihn für besessen erklärt), ihn des Volksaufruhrs beschuldiget habe \usw\ Sie beschreiben die große Angst, die er vor seinem Tode zu Gethsemane empfunden. Sie verhehlen uns nicht, daß Herodes ihn nur verspottet habe. Sie melden, daß er am Kreuze ausgerufen habe: \erganf{Mein Gott! mein Gott! warum hast du mich verlassen!} \RWbet{Lukas} bemerkt, daß Einige an der Wirklichkeit der Auferstehung \RWbet{Jesu} sogar noch damals gezweifelt hätten, als er bereits den Berg zu seiner Himmelfahrt bestiegen. -- Ohne Zweifel mußten die Evangelisten sehr wohl einsehen, daß solche Erzählungen von Manchen gemißdeutet werden, und verschiedene Bedenklichkeiten gegen die Sache des Christenthumes anregen dürften; gleichwohl ließen sie sich nicht abhalten, dieß Alles, so wie es war, zu erzählen, sicher nur in der Voraussetzung, auf welche sich eben die Pflicht der Wahrheitsliebe gründet, daß es, im Ganzen genommen, doch immer mehr Vortheil habe, wenn man die Wahrheit unentstellt läßt. Hieher gehört auch, daß sich diese Schriftsteller so Manches nicht erlaubt haben, was sie aus Liebe für \RWbet{Jesum} sich wohl versucht fühlen möch\RWSeitenw{135}ten, \zB\ daß sie die schmerzlichsten Leiden und die empörendsten Mißhandlungen ihres Herrn erzählen, ohne irgend ein auch noch so gewöhnliches Mittel der Redekunst zu benützen, dessen sich andere Schriftsteller in solchen Fällen bedienen, um ihre Leser zu rühren; daß sie nicht einmal sich einen Ausruf erlauben, oder den Leser nur auf die Größe der ihrem Herrn erwiesenen Unbilden aufmerksam machen, und seinem Urtheile so gleichsam vorgreifen wollen; daß sie eben so wenig versuchen, die Wahrheit, die in den Reden \RWbet{Jesu} liegt, die Vortrefflichkeit und das Bewunderungswürdige seines Benehmens eigens in's Licht zu setzen; \usw\
\item Sie erzählen endlich so Manches, wovon die \RWbet{Eitelkeit}, und eine \RWbet{allen Menschen natürliche Selbstliebe} sie hätte zurückhalten können. -- Sie haben der Nachwelt entdeckt ihre geringe Herkunft; ihre früheren so niedrigen Gewerbsarten; die eigennützigen Absichten, die sie im Anfange bestimmten, ihrem Herrn nachzufolgen; ihre unedlen Rangstreitigkeiten, durch die sie dem Herrn, selbst noch den letzten Abend, den er mit ihnen verlebte, verbittert hatten; ihre Vorurtheile; ihre Langsamkeit im Begreifen gewisser Wahrheiten; ihre schimpfliche Furchtsamkeit bei allem Großsprechen, das ihr vorhergegangen war, \usw\ Sie gestehen es, daß aus ihrer eigenen Mitte Einer den Herrn dreimal verläugnet, und mit einem Eidschwur betheuert habe, daß er ihn nicht kenne; daß sie bei seiner Gefangennehmung Alle die Flucht ergriffen hätten; daß ein Anderer aus ihrer Mitte selbst sein Verräther gewesen sey, und dieß zwar um 30 Silberlinge; \usw\ Sie erzählen, daß ihnen der Versuch, einen Besessenen zu heilen, einmal mißlungen wäre; daß sie rachsüchtig genug gewesen wären, ein Feuer vom Himmel herab über eine Stadt zu verlangen, die ihren Meister nicht aufnehmen wollte. Sie bekennen, daß man ihr ungewöhnliches Betragen am Feste der Pfingsten mit dem Betragen eines Trunkenen verglichen; daß es so manche Mißhelligkeiten unter ihnen gegeben, daß (\RWbibel{Gal}{Gal.}{2}{11}) \RWbet{Paulus} zu Antiochien dem \RWbet{Petrus} in's Angesicht widersprochen; daß \RWbet{Paulus} und \RWbet{Barnabas} sich veruneini\RWSeitenw{136}get hätten (\RWbibel{Apg}{Apostelg.}{15}{36}); \usw\ Hieher gehört auch noch, daß sie so Manches nicht sagen, was sie aus Eitelkeit zu sagen versucht seyn konnten. So hätten sie ja \zB\ überhaupt mehr von ihrer eigenen Person, von ihren Wunderthaten, und von den Leiden, die sie des Christenthums wegen bestanden, der Nachwelt mittheilen können; hätten den Vorzug, welchen sie als Apostel vor allen übrigen Christen gehabt, heller in's Licht setzen mögen. Statt dessen führen sie uns das Gebot \RWbet{Jesu} an, daß unter den Christen durchaus keine Rangordnung Statt finden solle, und daß derjenige, der sich der Größte zu seyn dünket, sich zu betragen habe, als wäre er der Diener Aller. (\RWbibel{Mt}{Matth.}{20}{25\,ff}, \RWbibel{Mk}{Mark.}{10}{35\,ff}, \RWbibel{Lk}{Luk.}{22}{24\,ff}, \RWbibel{Joh}{Joh.}{13}{12\,ff})
\end{aufzb}
\item Wenn nun Geschichtschreiber, die uns von ihrer Liebe zur Wahrheit und von ihrer Aufrichtigkeit so vielfältige und unzweideutige Proben geliefert, es nicht verdienen sollten, daß wir sie hochschätzen, und ihren Erzählungen unser ganzes Zutrauen schenken: wem sollten wir dann noch vertrauen?
\end{aufza}

\RWpar{54}{F.~Aus der Vergleichung ihrer Erzählungen mit den Erzählungen anderer Geschichtschreiber}
Noch einen eigenen Beweis für die historische Glaubwürdigkeit der Bücher des neuen Bundes erhalten wir, wenn wir den Inhalt dieser Bücher mit dem vergleichen, was wir bei \RWbet{anderen Schriftstellern} über denselben Gegenstand finden. Dieß soll der letzte Punct seyn, den ich hier noch berühre. Wir finden nämlich
\begin{aufzb}
\item bei einer großen Menge anderer Schriftsteller, die doch mit unseren biblischen in keiner Verbindung gestanden, mehrere Erzählungen, durch welche die Erzählungen der Bibel bestätiget werden. Und zwar geschieht dieß auf eine \RWbet{doppelte} Weise
\begin{aufzc}
\item entweder dadurch, daß die Erzählung des Profanscribenten mit der des heiligen Geschichtschreibers \RWbet{gleichlautend} ist, daß beide ausdrücklich dasselbe aussagen; oder~\RWSeitenw{137}
\item dadurch, daß der Profanscribent eine Begebenheit erzählt, durch welche die in der Bibel erzählte einen \RWbet{höheren Grad innerer Wahrscheinlichkeit} erhält. Wir finden ferner, daß
\end{aufzc}
\item andere Geschichtschreiber, mitunter auch geachtete, wenn sie Begebenheiten erzählen, die auch in unserer Bibel vorkommen, in ihre Erzählungen so manche Umstände aufnehmen, die \RWbet{äußerst unwahrscheinlich} sind, die aber von unseren biblischen Erzählern lieber ganz weggelassen wurden. Dieses beweiset uns denn, daß die letzteren ungleich vorsichtiger, als jene, zu Werke gegangen sind, und folglich auch mehr Glauben, als sie verdienen.
\end{aufzb}
Ich werde nur einige Beispiele, die Alles dieß beweisen, anführen; in Betreff eines Mehreren aber verweise ich auf \RWbet{Lardner's} \RWlat{Credibility of the Gospel History. Lond.\ 1722. 2.\ Vol. in 8.}\RWlit{}{Lardner1b}
\begin{aufza}
\item \RWbet{Andere Schriftsteller erzählen dasselbe, was die Verfasser der Bücher des neuen Bundes erzählen.}
\begin{aufzb}
\item In den Büchern des neuen Bundes wird uns erzählt, daß die Juden um die Zeit \RWbet{Jesu} die Ankunft ihres Messias erwartet hätten. Daher, daß der hohe Rath \RWbet{Johannes} den Täufer befragen läßt, ob etwa er selbst der Messias sey, \udgl\  -- Dasselbe bestätigen nun auch auswärtige Geschichtschreiber, \zB\ \RWbet{Tacitus} (\RWlat{histor.\ lib.\,5. cap.\,13.}): \erganf{Mehrere waren der Ueberzeugung, es stehe in den alten Schriften der Priester, daß zu derselben Zeit der Orient die Oberhand gewinnen, daß sie aus Judäa ausziehen und die Welt erobern würden.} Und \RWbet{Suetonius} (\RWlat{Vespas.\ c.\,4.}): \erganf{Im ganzen Orient hatte sich die alte Meinung bleibend festgesetzt, es sey im Rathe des Schicksals beschlossen, daß sie zu jener Zeit aus Judäa ziehen und die Welt unterjochen würden.} Beide behaupten, daß diese Meinung die Juden zur Empörung gegen die Römer verleitet habe. Dasselbe versichert auch der jüdische Geschichtschreiber \RWbet{Flavius Josephus}.
\item Bei \RWbet{Lukas} (\RWbibel{Lk}{}{2}{1}) wird uns von einer Aufschreibung (Conscription oder Schätzung) erzählt, die auf Befehl des~\RWSeitenw{138}\ Kaisers \RWbet{Augustus} zu eben der Zeit, als \RWbet{Kyrenius} (Quirinus) Statthalter von Syrien war, geschehen sey. Dieser Beschreibung gedenkt auch der Kaiser \RWbet{Julian}.
\item In der \RWbet{Apostelgeschichte} (\RWbibel{Apg}{}{5}{34}) wird einer zweiten Aufschreibung gedacht, die zu der Zeit geschehen sey, als \RWbet{Judas} der Galiläer einen Aufruhr unter den Juden erregte. Von dieser und von dem Aufstande des \RWbet{Judas} erzählt ein Mehreres \RWbet{Flavius Josephus} (\RWlat{Antiquit.\ lib.\,18.\ c.\,1.}).\RWlit{}{Josephus1}
\item Des Bethlehemitischen Kindermordes, den uns \RWbet{Matthäus} (\RWbibel{Mt}{}{2}{1\,ff}) erzählt, gedenkt auch der heidnische Philosoph \RWbet{Celsus} (\RWlat{Orig.\ contra Celsum lib.\,1.\ 58.}),\RWlit{}{Origenes1} ingleichen \RWbet{Makrobius}, ein heidnischer Schriftsteller aus dem vierten Jahrhunderte, der von dem Kaiser \RWbet{Augustus} erzählt: \erganf{Als er gehört hatte, mit den nicht über zwei Jahre alten Knaben, die \RWbet{Herodes}, der König der Juden, in Syrien ermorden ließ, sey auch der Sohn desselben ermordet worden (das war eine Irrung), sagte er: Es ist besser, das Schwein des \RWbet{Herodes} zu seyn, als der Sohn} (\RWlat{Saturn.\ 1.\ 2.\ c.\,24.}).
\item Den Aufenthalt \RWbet{Jesu} in Aegypten, dessen \RWbet{Matthäus} (\RWbibel{Mt}{}{2}{14}) erwähnt, gibt auch \RWbet{Celsus} zu, und nimmt hievon Gelegenheit, zu behaupten, daß Jesus dort magische Künste erlernet habe.
\item Die Bücher des neuen Bundes schildern uns \RWbet{Johannes} den Täufer als einen frommen Mann, der seine Zeitgenossen zur Sinnesänderung aufgefordert, und Jene, die Besserung gelobten, getauft habe; der aber von dem Tetrarchen \RWbet{Herodes Antipas} gefangen genommen, und unschuldig hingerichtet worden sey. Dasselbe erzählt nun auch \RWbet{Flavius Josephus}. Erst berichtet er, daß \RWbet{Herodes} seines Bruders \RWbet{Philippus} Gemahlin geheirathet habe (bekanntlich geben dieß die Evangelisten als die Veranlassung des Todes \RWbet{Johannis} an) und erzählt gleich darauf, \RWbet{daß die Juden die Niederlage, welche Herodes erlitten, für eine göttliche Strafe erklärt hätten, weil er Johannes den Täufer getödtet}; und hierauf schreibt er: \erganf{\RWbet{Herodes} tödtete diesen vortrefflichen Mann, der die Juden zur Ausübung~\RWSeitenw{139}\ der Tugend und zur Gerechtigkeit gegen einander ermahnte (gerade so charakterisirt das Evangelium die Predigten \RWbet{Johannis}), und, wenn sie zur Gottesfurcht sich bekehrten, zum Empfang der Taufe aufforderte.}
\item In der \RWbet{Apostelgeschichte} (\RWbibel{Apg}{}{18}{1}) wird gelegenheitlich erwähnt, daß Kaiser \RWbet{Claudius} die Juden aus Rom vertrieben habe. Dasselbe lesen wir auch bei \RWbet{Suetonius} im Leben des \RWbet{Claudius}: \erganf{Er vertrieb die Juden aus Rom, weil sie, aufgereizt vom \RWbet{Chrestus} (\RWlat{impulsore Chresto}), fortwährend Unruhe erregten (\RWlat{perpetuo tumultuantes}).}
\item In der \RWbet{Apostelgeschichte} (\RWbibel{Apg}{}{11}{28}) wird einer großen Theuerung erwähnt, die unter eben dieses Kaisers Regierung in ganz Judäa gewesen seyn soll. Von dieser redet auch \RWbet{Flavius Josephus}, und sagt, daß eine große Menge von Menschen vor Hunger gestorben sey. (\RWlat{Antiquit.\ 20.\ c.\,2.\ \RWparnr{6}})\RWlit{}{Josephus1}
\item In der \RWbet{Apostelgeschichte} (\RWbibel{Apg}{}{21}{38}) fragt der römische \RWbet{Chiliarch} den Apostel \RWbet{Paulus}: \erganf{Bist du nicht etwa der Aegyptier, der vor Kurzem einen Aufruhr stiftete, und 4000 Meuchelmörder in der Wüste zusammenrottete?} Diese Geschichte erzählt uns \RWbet{Flavius Josephus} (\RWlat{de bello judaico lib.\,2.\ c.\,13.}\RWlit{}{Josephus4} und \RWlat{Antiquit.\ lib.\,20.\ c.\,8.})\RWlit{}{Josephus1}: \erganf{Unter dem Landpfleger \RWbet{Felix} (also zu eben der Zeit, von welcher \RWbet{Lukas} redet) gab sich ein Aegyptier für einen Propheten aus, und zog in der Wüste eine Menge von 30.000 Menschen an sich (\RWbet{Lukas} erwähnt nur 4000; diese mochten vielleicht in einer besonderen Abtheilung gestanden, und die gefährlichsten gewesen seyn), ging auf Jerusalem zu, mit dem Versprechen, daß die Mauern der Stadt auf seinen Befehl sogleich einstürzen würden; ward aber vom Statthalter \RWbet{Felix} geschlagen, und entwich mit einigen Wenigen.}
\end{aufzb}
\item \RWbet{Andere Schriftsteller erzählen uns Dinge, wodurch die innere Glaubwürdigkeit dessen, was die biblischen erzählen, erhöht wird.}
\begin{aufzb}
\item Man hat den Umstand, daß \RWbet{Flavius Josephus}, der Geschichtschreiber der Juden, von dem zu Bethlehem~\RWSeitenw{140}\ verübten Kindermorde schweigt, als einen Einwurf gegen die Wahrheit dieser Geschichte betrachtet. Gleichwohl beschreibt uns eben dieser \RWbet{Josephus} den Charakter des Königs \RWbet{Herodes} (des Großen oder des Askaloniten) dergestalt, und erzählt uns von ihm solche Grausamkeiten, daß jene evangelische Erzählung einen sehr hohen Grad innerer Glaubwürdigkeit erhält. Dieser König hatte sich den Weg zum Throne durch die Ermordung der zwei noch übrigen Prinzen vom makkabäischen Geschlechte, \RWbet{Antigonus} und \RWbet{Hyrkanus} (deren Letzterer sein naher Anverwandter war, und ihm selbst das Leben gerettet hatte), gebahnet; ermordete den Bruder seiner Gemahlin, diese selbst, ihre Mutter, ja drei seiner eigenen Söhne. In seiner letzten Krankheit ließ er die Vornehmsten der Nation nach Jericho (wo er darnieder lag) bringen, sie in den Hippodromus einsperren, und ertheilte sterbend den Befehl: \erganf{Sobald der Athem aus mir fährt, lasset sie Alle ermorden, damit ganz Judäa genöthiget werde, bei meinem Tode zu trauern.} (\RWlat{Antiquit.\ lib.\,17.\ c.\,6.\ \RWparnr{5}\RWlit{}{Josephus1}\ de bello judaico lib.\,1.\ c.\,33.\ \RWparnr{6}})\RWlit{}{Josephus4} -- Daß nun \RWbet{Josephus} des bethlehemitischen Kindermordes nicht gedenkt, geschieht vermuthlich nur darum, weil er das ganze Ereigniß (bei welchem höchstens einige 20 Kinder um das Leben gekommen seyn dürften) nicht für so wichtig hielt, um neben den übrigen Grausamkeiten des Königs Erwähnung zu verdienen.
\item Bei \RWbet{Lukas} (\RWbibel{Lk}{}{3}{14}) wird erzählt, daß unter anderm Volke, das zu \RWbet{Johannes} dem Täufer in die Wüste gekommen, auch Soldaten gewesen. Man möchte fragen, wie diese dahin gekommen, da sonst in Palästina keine Soldaten zu seyn pflegten, außer einer kleinen Besatzung in der Antoniaburg zu Jerusalem? -- Aus \RWbet{Flavius Josephus} (\RWlat{Ant.\ lib.\,18.\ c.\,15.}\RWlit{}{Josephus1}) ersehen wir, daß gerade um die Zeit, da \RWbet{Johannes} predigte, \RWbet{Herodes}, der Tetrarch von Galiläa, in einem Kriege mit \RWbet{Aretas}, dem Könige des peträischen Arabiens, begriffen gewesen sey, und deßhalb ein Heer dahin abgeschickt habe, welches dann nothwendig durch die Wüste ziehen mußte. Dieß also waren die Soldaten, die auf ihrem Durch\RWSeitenw{141}marsche bei \RWbet{Johannes} anhielten. Daher ihr Name \RWgriech{strateu'omenoi} (im Dienst begriffene Soldaten) und nicht \RWgriech{str'attai}, wie eine Besatzung heißt. Auf die Umstände solcher Leute schickte sich auch ganz die Ermahnung, die ihnen \RWbet{Johannes} mitgab.
\item Nach der \RWbet{Apostelgeschichte} (\RWbibel{Apg}{}{24}{24--26}) spricht \RWbet{Paulus} vor dem Landpfleger \RWbet{Felix} und seiner Gemahlin \RWbet{Drusilla} von den Tugenden der Gerechtigkeit und Keuschheit und von dem künftigen Gerichte. \RWbet{Felix} erschrickt über diese Rede, und heißt den Apostel aufhören. Gleichwohl, erzählt \RWbet{Lukas} weiter, ließ er ihn noch manchmal zu sich kommen, und besprach sich mit ihm, weil er hoffte, daß \RWbet{Paulus} sich mit Geld bei ihm loskaufen werde. -- Wie viel innere Wahrscheinlichkeit erhält nicht diese Erzählung durch die Beschreibung, welche uns \RWbet{Tacitus} und \RWbet{Josephus} von diesem Statthalter machen. Nach ihrem Berichte ist \RWbet{Felix} wegen der Räubereien, die er in Judäa verübte, und wegen der schändlichen Handlung mit \RWbet{Drusilla}, die er ihrem Gemahle \RWbet{Azizus}, dem Könige der Edessener abwendig gemacht, berüchtigt (\RWlat{Tacitus hist.\ lib.\,5.\ c.\,10.,\RWlit{}{Tacitus2}} \RWbet{Josephus} \RWlat{Antiquit.\ lib.\,20.\ c.\,7.\ \RWparnr{12}}\RWlit{}{Josephus1}). Daraus begreift sich, warum \RWbet{Paulus} bei dieser Gelegenheit nicht von den wesentlichen Grundsätzen des Christenthumes, sondern von jenen beiden Tugenden, und von dem künftigen Gerichte gesprochen habe, \usw\
\item Ganz anders schildert uns \RWbet{Lukas} den römischen Statthalter in Achaja, \RWbet{Gallion} (\RWbibel{Apg}{Apostelg.}{18}{14--16}). Er läßt ihn den Juden, die \RWbet{Paulum} bei ihm verklagen, erwiedern: \erganf{Beträfe es ein Verbrechen, so würde die Vernunft fordern, euch zu unterstützen. Da aber der Streit, wie ich höre, bloße Worte und Meinungen, und euer Gesetz betrifft: so ist dieß eure eigene Sache. In solchen Dingen kann ich nicht Richter seyn!} -- Wie genau stimmt dieses nicht mit der nämlichen Schilderung überein, die uns auswärtige Schriftsteller, \zB\ \RWbet{Seneca} (\RWlat{in quaest.\ lib.\,4.}: \erganf{Niemand ist auch nur Einem so hold, als dieser es Allen ist}), \RWbet{Tacitus} (\RWlat{Annal.\ lib.\,15.}\RWlit{}{Tacitus1}) \uA\ von diesem \RWbet{Gallion} machen!~\RWSeitenw{142}
\end{aufzb}
\item \RWbet{Andere Geschichtschreiber nehmen in ihre Erzählungen Umstände auf, welche sehr unwahrscheinlich sind, und von unsern heil.\ Schriftstellern lieber ganz weggelassen wurden.}
\begin{aufzb}
\item Im Evangelio der \RWbet{Nazaräer} wird bei der Taufe \RWbet{Jesu} erzählt, der heil.\ Geist habe gesprochen: \erganf{Du mein Sohn! in allen Propheten erwartete ich Dich, damit Du kämest, und ich auf Dir ruhete. Denn Du bist meine Ruhe, mein erstgeborner Sohn, und König auf immer.} -- Die Verfasser unserer vier Evangelien ließen diese Worte weg, gewiß nur, weil sie ihnen nicht genug innere Glaubwürdigkeit hatten.
\item Das Evangelium der \RWbet{Aegyptier} legt die Aussprüche \RWbet{Jesu} alle nach essenischen Grundsätzen aus. So heißt es \zB , daß \RWbet{Jesus} einst gesagt haben soll: \erganf{Ich bin gekommen, die Werke des Weibes aufzuheben; des Weibes, nämlich der Begierde; ihre Werke, nämlich die Zeugung und den Tod.} -- Wie unwahrscheinlich, daß \RWbet{Jesus}, wenn er ja jene ersteren Worte gesprochen, sie dergestalt ausgelegt oder verstanden haben sollte!
\item Mehrere apokryphische Evangelien (\zB\ das \RWbet{Evangelium Petri}) sind nach den Grundsätzen der Doketen abgefaßt, \dh\ sie behaupten, daß \RWbet{Jesus} nur einen scheinbaren Leib gehabt. -- Wie ungereimt!
\item Die \RWbet{Evangelien der Kindheit Jesu} enthalten eine Menge abgeschmackter Wundergeschichten; das ältere schildert überdieß \RWbet{Jesum} auf eine Art, daß, wenn er wirklich so gehandelt hätte, er ein rachsüchtiges und grausames Kind gewesen seyn müßte. -- Beweiset es nicht eine geläuterte Urtheilskraft von Seite unserer vier Evangelisten, daß sie keine dieser Erzählungen, deren mehrere schon zu ihrer Zeit vorhanden seyn mochten, in ihre Geschichte aufgenommen haben?
\item \RWbet{Flavius Josephus} gilt als ein glaubwürdiger Geschichtschreiber; denn er war ein Mann, der eine gelehrte Bildung genossen, die Schriften der Heiden gelesen, und von den Kaisern \RWbet{Vespasianus, Titus} und \RWbet{Domitianus} in Ehren gehalten wurde. Gleichwohl kann fol\RWSeitenw{143}gendes Beispiel beweisen, daß unser heil.\ \RWbet{Lukas} ein weit besserer Geschichtschreiber sey, als jener. Beide erzählen uns nämlich den Tod des Königs \RWbet{Herodes Agrippa} im Ganzen sehr übereinstimmend; in einzelnen Stücken weichen sie aber von einander auf eine solche Weise ab, daß die Erzählungsart \RWbet{Lucä} fast immer mehr Lob verdient. \RWbet{Lucas} erzählt (\RWbibel{Apg}{Apostelg.}{12}{20--23}), daß dieser stolze König die Abgesandten der Tyrier und Sidonier, die ihn um Frieden baten, auf einen gewissen Tag beschieden habe. An diesem zog er sein königliches Kleid an, bestieg den Thron, und hielt eine Rede an sie. Nach ihrer Beendigung rief das versammelte Volk: Das ist die Stimme eines Gottes, und nicht eines Menschen! Plötzlich aber schlug ihn der Engel des Herrn, weil er nicht Gott die Ehre gegeben; er bekam den Würmerfraß und starb. -- \RWbet{Josephus} erzählt diese Geschichte viel weitläufiger; und doch vergißt er zu erwähnen, daß die Gesandten von Tyrus und Sidon gekommen waren, welches begreiflicher macht, warum sich der König in einer solchen Pracht habe darstellen wollen. Dagegen berichtet er uns, daß an jenem Tage ein Fest gewesen, das man dem römischen Kaiser zu Ehren angestellt hatte, wodurch es begreiflich wird, wie sich das Volk, im Rausche der Freude dieses Tages, aufgelegt fühlen konnte, \RWbet{Herodes} für einen Gott zu erklären. Ferner beschreibt er das Kleid des Königs als ein mit Silber durchwirktes, das bei den Strahlen der eben aufgehenden Sonne einen blendenden Glanz von sich geworfen, und alle Zuschauer in Erstaunen gesetzt hätte. Den Zuruf des Volkes, oder vielmehr der Schmeichler, verlängert \RWbet{Josephus} auf eine unwahrscheinliche Weise: Sey uns gnädig! bisher haben wir dich wie einen Menschen verehrt; jetzt sehen wir, daß du erhaben bist über die menschliche Natur! Auch bestimmt er die wahre Ursache des Todes dieses Königs nicht so gerade, wie der Arzt \RWbet{Lukas}, der sie einen Würmerfraß nennt, während \RWbet{Josephus} bloß sagt, der König habe einen heftigen Schmerz um das Herz und in den Eingeweiden verspürt, und sey am fünften Tage gestorben. Das Merkwürdigste ist aber, daß~\RWSeitenw{144}\ \RWbet{Josephus} das abergläubige Mährchen einmischt, der König habe über seinem Kopfe eine Eule sitzend auf einem Stricke gesehen, und diese sogleich für einen Unglücksboten erkannt. Auch \RWbet{Lukas} dürfte wohl von dieser Eule gehört haben; aber dieser Umstand schien ihm nicht glaubwürdig. Wenn wir statt dessen bei ihm den Ausdruck antreffen: \erganf{der Engel des Herrn schlug ihn}, so wollte er dadurch gewiß nichts Anderes sagen, als daß der Schmerz, den der König plötzlich in seinen Eingeweiden fühlte, eine Schickung Gottes gewesen sey. Diese Vergleichung der beiden Geschichtschreiber bestätiget also nicht nur auf das Vollkommenste die Wahrheit dessen, was uns der heilige Evangelist berichtet, sondern sie zeigt auch, daß wir an ihm einen weit zuverlässigeren Erzähler haben.~\RWSeitenw{145}
\end{aufzb}
\end{aufza}

\clearpage

\RWch[Viertes Hauptstück.\\ Einzelne Wunder, die zur Bestätigung des katholischen Christenthums dienen.]{Viertes Hauptstück.\\ Einzelne Wunder, die zur Bestätigung des katholischen Christenthums dienen.}

\RWpar{55}{Inhalt und Zweck dieses Hauptstückes}
\begin{aufza}
\item Es ist jetzt unser Vorhaben, darzuthun, daß die katholische Religion für Jeden, der an der Lehre derselben das Merkmal sittlicher Zuträglichkeit entdeckt, auch eine \RWbet{durch Wunder oder Zeichen bestätigte, also geoffenbarte} Religion sey.
\item Nun sind zwar nach den Untersuchungen, die wir im vorhergehenden Hauptstücke angestellt haben, alle Ereignisse, die in den Büchern des neuen Bundes erzählt werden, hinlänglich glaubwürdig; nichts destoweniger haben die Gegner des Christenthums gegen einzelne dieser Erzählungen so viele, und auch zum Theile so \RWbet{scheinbare Einwürfe} ausgedacht, daß selbst ein Wohlunterrichteter, wenn er sie alle hört, in seiner Ueberzeugung schwankend gemacht werden kann. Dieß geschieht um so leichter, und endigt um so verderblicher, wenn sich noch Sinnlichkeit und Leidenschaft einmischt, und wir also dadurch, daß wir die Richtigkeit gewisser biblischer Erzählungen bezweifeln oder läugnen, der Verbindlichkeit, das Christenthum als eine wahre göttliche Offenbarung anzusehen und zu befolgen, enthoben zu seyn glauben.
\item Aus diesem Grunde däucht es mir nun nicht wohlgethan, wenn man den Glauben an die Göttlichkeit des Christenthums auf die erwiesene Richtigkeit der neutestamentlichen Erzählungen auf eine solche Art bauet, daß jener umgestoßen ist, sobald uns irgend eine der letzteren zweifelhaft wird.~\RWSeitenw{146}
\item Dieß werde ich also hier zu vermeiden suchen; und der Beweise mehrere liefern, die, ohne die Glaubwürdigkeit der evangelischen Geschichte zu bedürfen, bis zur vollkommensten Beruhigung darthun, daß sich gewisse Wunder zur Bestätigung des Christenthums zugetragen haben.
\item Diese Beweise werden uns aber gar keine nähere Kenntniß davon gewähren, worin diese wundervollen Ereignisse bestanden seyen. Es ist nichts Sträfliches an dem Wunsche, auch dieses zu erfahren, wenn anders derselbe immer in den Schranken der Mäßigung bleibt, und nicht etwa ausartet in den ungebührlichen Vorsatz, daß wir das Christenthum nicht eher als eine von Gott uns gegebene Offenbarung erkennen wollen, als bis wir auf das Bestimmteste wissen, nicht nur, daß es durch Wunder bestätiget sey, sondern auch, welches der eigentliche Hergang bei diesen Wundern gewesen.
\item Um diesem Wunsche zu willfahren, werde ich also versuchen, von den merkwürdigsten einzelnen Wundern, die zur Bestätigung des Christenthums dienen, einen Begriff zu geben, wobei ich mich aber meistens genöthiget sehen werde, mich auf die Glaubwürdigkeit dessen, was uns die Bücher des neuen Bundes, oder auch andere Schriften, erzählen, mehr oder weniger zu verlassen.
\end{aufza}

\RWabs{Erste Abtheilung.}{Allgemeine Beweise für das Vorhandenseyn des äußern Merkmals einer Offenbarung an dem Christenthume.}

\RWpar{56}{I.~Das Daseyn der Bibel, ein zur Bestätigung des Christenthums dienendes Wunder.}
Das erste Wunder, das schon für sich allein hinreicht, den göttlichen Ursprung des Christenthums zu erweisen, vorausgesetzt, daß man an seiner Lehre das Merkmal der sittlichen Zuträglichkeit gefunden habe, ist das \RWbet{Vorhandenseyn der Bibel.}~\RWSeitenw{147}
\begin{aufza}
\item Bei allen besseren Religionen auf Erden fanden wir gewisse heilige Bücher, deren sich ihre Bekenner bald zur Bestimmung des Inhaltes ihres Glaubens, bald zum Beweise seiner Göttlichkeit, bald sonst zu ihrer Belehrung und Erbauung bedienten. Auch wir Christen haben dergleichen Bücher, und zwar ist es die Sammlung der sogenannten \RWbet{Bücher des alten und neuen Bundes}, welche wir vorzugsweise \RWbet{die Bücher (Bibel)} oder die \RWbet{heiligen Bücher}, oder die \RWbet{heilige Schrift} nennen. Bei einer näheren Betrachtung zeigt es sich, daß diese heiligen Bücher der Christen an innerer Vortrefflichkeit, an wahrer Brauchbarkeit für den Zweck der Belehrung und Erbauung nicht nur die heiligen Bücher aller anderen Religionen weit hinter sich zurücklassen, sondern auch etwas an sich so Wunderbares sind, daß wir ihr Daseyn in der That als ein zur Bestätigung des Christenthums dienendes Wunder ansehen dürfen.
\begin{aufzb}
\item Zuvörderst gibt es schon kaum irgend ein Verhältniß des Lebens, und irgend eine Pflicht, worüber wir uns nicht aus diesen Büchern belehren könnten. Je fleißiger wir sie lesen, um desto mehr Stoff zur sittlichen Belehrung entdecken wir in jeder Zeile derselben, um desto mehr werden wir inne, daß sie in dieser Rücksicht nicht zu erschöpfen sind, und daß kein anderes Buch auf Erden mit ihnen verglichen werden kann. Und was wohl zu bemerken ist,
\item die Lehren, die sie uns ertheilen, werden hier nicht in \RWbet{abgezogenen Begriffen}, wo sie nur für die Wenigsten verständlich und eindringlich wären, sondern fast durchgängig \RWbet{veranschaulicht durch geschichtliche Begebenheiten} und so ergreifend vorgetragen, daß wir der Kraft ihrer Darstellung kaum widerstehen können. In diesen Büchern wird uns
\item in der Person \RWbet{Jesu} das \RWbet{Ideal menschlicher Vollkommenheit} verwirklicht dargestellt, wie sie es sonst nirgends anzutreffen ist. Diese Bibel gibt uns
\item durch die Begebenheiten, die sie erzählt, und durch die Behauptungen, welche sie aufstellt, Stoff und Aufforderung zu den lehrreichsten Untersuchungen in allen Fächern des menschlichen Wissens, zu Untersuchungen, die in das~\RWSeitenw{148}\ Gebiet der Philosophie, der Mathematik, der Physik, der Psychologie, der Naturgeschichte, der Geschichte der Menschheit, der Sprachkunde, der Arzneiwissenschaft, der Astronomie gehören.
\item Diese Bibel enthält auch Aufsätze, die in \RWbet{ästhetischer Hinsicht} zu den gelungensten Werken gehören, und die hiedurch ungemein viel zur Ausbildung unseres Geschmackes und mittelbar auch zur Beförderung unserer Sittlichkeit beitragen können.
\item Selbst was in diesen heiligen Büchern Einigen anstößig werden könnte, wenn es in seinem buchstäblichen Sinne genommen wird, ist einer geistigen (mystischen) Auslegung fähig, bei der es nicht nur aufhört, gefährlich zu seyn, sondern selbst wohlthätig auf uns einwirkt.
\item Die Erzählungen dieser Bücher, an deren Wahrheit dem Christenthume etwas gelegen seyn kann, haben (wie in dem vorhergehenden Hauptstücke sattsam gezeigt worden ist) so viele Gründe der Glaubwürdigkeit für sich, als man nur billiger Weise verlangen kann, und bei keinem andern religiösen Geschichtsbuche antrifft.
\item Fast alle Lehren, welche das Christenthum nach irgend einem der verschiedenen Bekenntnisse (Confessionen), selbst in der katholischen Kirche (die doch die meisten Lehren hat) aufstellt, lassen sich aus gewissen, in diesen Büchern befindlichen Stellen bald näherer, bald entfernterer Weise herleiten und rechtfertigen; zumal da in eben diesen Büchern erzählt wird, daß \RWbet{Jesus Christus} auch seinen künftigen Bekennern den Beistand des göttlichen Geistes, der sie in aller Wahrheit leiten soll, bis an das Ende der Zeit verheißen habe.
\end{aufzb}
\item Wer alle diese Vorzüge, welche den heil.\ Büchern der Christen vor den heil.\ Büchern anderer Religionen zukommen, in Erwähnung ziehet; wer noch dazu bedenkt, von der Einen Seite, wie viele überaus zufällige Umstände sich vereinigen mußten, um Büchern von solchem Inhalte und von solcher Brauchbarkeit das Daseyn zu geben, und von der andern Seite, wie der katholische Lehrbegriff (nebst jedem andern christlichen) diesen Büchern wo nicht seine Entstehung,~\RWSeitenw{149}\ doch seine Ausbildung, Erhaltung und Ausbreitung großen Theils verdanke: der wird gestehen müssen, daß das Vorhandenseyn der Bibel ein unverkennbares Zeichen des göttlichen Willens enthalte, daß wir den Lehren des Christenthums unser Vertrauen schenken sollen, so ferne wir anders finden, daß sie uns sittlich zuträglich sind.
\end{aufza}

\RWpar{57}{II.~Die Predigt der Apostel ein Beweis, daß die christliche Religion durch Wunder bestätiget sey}
\begin{aufza}
\item Das Factum, von dem ich in diesem Beweise ausgehe, ist kürzlich folgendes. In der ersten Hälfte des ersten christlichen Jahrhundertes traten ohngefähr zwölf Personen, die sich Apostel Jesu nannten, als Lehrer einer neuen Religion auf, und zogen, ausgehend aus Palästina, durch alle Länder des römischen Reiches, ja selbst auch anderwärts, umher, erzählten überall von einem gewissen \RWbet{Jesu} von Nazareth, den Gott mit Weisheit und Wunderkraft ausgerüstet hätte, von dessen Großthaten sie und Tausende mit ihnen Augenzeugen gewesen wären, und der, nachdem er gekreuziget worden, glorreich wieder vom Tode auferstanden wäre, und sie beauftragt hätte, die Verkündiger seiner Lehre zu werden. Diese Personen wurden von Juden sowohl als Heiden auf's Grausamste verfolgt, obgleich man sie niemals eines bestimmten Verbrechens oder Betruges überwiesen, ja auch nur beschuldiget hatte; sondern nur das Bestreben, eine neue Religion auf Erden einzuführen, war es, was man als ein Verbrechen an ihnen bestrafte. Dieses Schicksal der Verfolgung hatten sie Alle vorausgesagt, und zwar als eine Weissagung, die sie von ihrem Meister selbst empfangen hatten.
\item Dieß Factum ist beinahe unläugbar. Nicht nur besitzt es die größte innere Wahrscheinlichkeit, indem sich die Entstehung und Ausbreitung der christlichen Religion ohne gewisse erste Verkündiger derselben gar nicht denken läßt, und nichts begreiflicher ist, als daß diese Widerstand fanden, \usw ; sondern dieß Factum wird auch von allen christlichen sowohl, als andern Geschichtschreibern bestätiget. Die kurzen Aeußerungen, die sich bei \RWbet{Josephus, Tacitus, Plinius, Sue}\RWSeitenw{150}\RWbet{tonius} \uA\ finden, sind schon allein hinreichend, um Alles zu beweisen. Daß aber diese Apostel ihr Schicksal der Verfolgung als eine Weissagung ihres Meisters selbst angegeben haben; beweisen die Briefe \RWbet{Pauli} (\zB\ \RWbibel{2\,Tim}{2\,Tim.}{3}{12}) und die Evangelien (\zB\ \RWbibel{Mt}{Matth.}{10}{16}), auf welche wir uns in diesem Stücke immerhin berufen können.
\item Aus diesem Facto nun folgere ich:
\begin{aufzb}
\item daß die Apostel für ihre eigene Person fest überzeugt seyn mußten, \RWbet{Jesus} sey in der That ein göttlicher Gesandter, und die Wunder, welche sie von ihm gesehen zu haben bezeugten, hätten sich auch wirklich zugetragen. Denn im entgegengesetzten Falle hätten sie sich auf keine Weise entschließen können, Verkündiger des Evangeliums zu werden, oder sie hätten in diesem Entschlusse wenigstens nicht so standhaft ausharren können, weil sich kein hinlänglich starker Beweggrund hiezu erdenken läßt. Welch ein Beweggrund hätte hier abwalten können?
\begin{aufzc}
\item Die Aussicht auf \RWbet{irdische Vortheile}, auf Reichthum oder Wohlergehen schon aus dem Grunde nicht, weil sie vernünftiger Weise gar nicht erwarten konnten, dergleichen Vortheile zu finden, auch sich selbst Lügen gestraft haben würden, wenn sie, trotz der angeblichen Weissagung vom Gegentheile, einst zu dem ungestörten Besitze eines irdischen Wohlstandes gelanget wären. Hätten sie auch nur den leisesten Wunsch und die geringste Hoffnung, irdisches Glück zu erreichen, in ihrem Herzen genährt: so hätten sie zwar vielleicht vorgeben dürfen, daß sie dieß nicht erwarten, aber sie hätten sich sorgfältig hüten müssen, dieses als eine ganz ausgemachte Sache, als eine Weissagung ihres Herrn selbst vorzutragen; denn eben dadurch schnitten sie sich den Weg zu diesem irdischen Glücke selbst ab.
\item Oder soll etwa die Begierde, \RWbet{sich einen unsterblichen Namen zu machen}, diesen Entschluß bei den Aposteln hervorbracht haben? Die Ruhmsucht hat freilich schon manchen Menschen zu großen Aufopferungen vermocht; allein daß diese Begierde sich so vieler Per\RWSeitenw{151}sonen auf einmal bemächtiget haben sollte, und dieß zwar solcher Personen, die kurz vorher bei einem niederen Gewerbe ruhig und zufrieden gelebt; daß sich überdieß gar keine Spur von dieser Triebfeder in ihrer Behandlung der geworbenen Anhänger, in ihrem wechselseitigen Betragen gegen einander hätte verrathen sollen; daß man für einen Ruhm von der Art, wie die Apostel ihn allein erwarten konnten, für einen Ruhm, von dem man in der Gegenwart gar nichts genoß, den man nur in der fernsten Zukunft und mit Ungewißheit zu erlangen hoffen konnte, Leiden von einer solchen Größe und Dauer sollten ertragen haben: das ist beispiellos, und läßt sich nach vernünftigen Gründen keineswegs erwarten, zumal da wir nicht sehen, daß die Apostel nur das Geringste gethan, um diesen Nachruhm sich zu sichern, da sie kaum dafür gesorgt, daß wir nur ihre Namen und die Länder, in welchen ein Jeder geprediget, noch wissen.
\item Wollte man endlich behaupten, daß diese Männer vielleicht aus reiner \RWbet{uneigennütziger Liebe zur Menschheit} den Entschluß gefaßt hätten, eine verbesserte Religion auf Erden einzuführen, und daß sie als Mittel zu diesem Endzwecke die Geschichte \RWbet{Jesu} und seiner Wunder erdichtet hätten: so antworte ich, daß dieses Mittel erstens nicht zweckmäßig gewesen wäre, weil ihre Täuschung sehr leicht hätte entdeckt werden können, wenn \RWbet{Jesus} nicht wirklich gelebt, und jene außerordentlichen Thaten nicht verrichtet hätte. Zweitens ist es auch nicht zu gedenken, daß sich so viele Personen zu einer Unternehmung, von welcher nach aller menschlichen Einsicht nicht zu erwarten stand, daß sie trotz allen den Opfern, die man ihr brächte, gelingen werde, hätten vereinigen und in diesem Bestreben, ohne daß auch nur ein Einziger abfiel, bis an ihr Ende hätten ausharren sollen, wenn sich nicht irgend ein Ereigniß zugetragen, welches sie als eine ihnen von Gott gegebene Bürgschaft betrachteten, daß sie ihr Vorhaben zu Stande bringen würden; wenn sie in diesem Ereignisse nicht einen ausdrücklichen Auf\RWSeitenw{152}trag Gottes, Hand an dieß Werk zu legen, erkannt, und durch die Nichtbeachtung desselben sich zu versündigen befürchtet, bei der Befolgung aber noch in der Ewigkeit eine Vergeltung zu finden gehofft hatten. Dieses voraussetzen heißt aber nichts Anderes, als voraussetzen, daß die Apostel die Sache des Christenthums für eine wahre göttliche Offenbarung hielten. Und also muß man jedenfalls annehmen, daß die Apostel, wenigstens für ihre eigene Person, fest überzeugt gewesen seyn mußten, \RWbet{Jesus} sey wirklich ein Gesandter Gottes, und die Wunder, welche sie von ihm erzählen, hätten sich wirklich zugetragen.
\end{aufzc}
\item Diese Ueberzeugung aber hätte bei ihnen unmöglich eintreten können, wenn \RWbet{Jesus} nicht in Wahrheit viele sehr außerordentliche Thaten verrichtet hätte. Um die Nothwendigkeit dieser Folgerung desto deutlicher einzusehen, erwäge man,
\begin{aufzc}
\item daß die Apostel gewiß keine blöden und schwachsinnigen Leute gewesen, welche man etwa so leicht hätte betrügen können. Die Schriften, die uns Einige derselben, \RWbet{Matthäus, Lukas, Paulus} \uA\ hinterlassen haben, oder falls man an der Aechtheit dieser Schriften noch zweifeln sollte, die Thaten selbst, welche sie ausgeführt, sind uns ein redender Beweis vom Gegentheile. Blöden, schwachsinnigen Menschen wird es wohl nicht gelingen, an allen Orten, dahin sie kommen, mit einem so glücklichen Erfolge zu predigen, und so viele Anhänger, nicht nur unter den Ungebildeten, sondern auch unter den Gebildeten und Gelehrten, anzuwerben.
\item Daß sie auch nicht etwa schon im Voraus eingenommen für die Sache \RWbet{Jesu} waren. -- Dieß läßt sich wenigstens von Einigen (\zB\ von Paulus) unwidersprechlich darthun. Denn dieser war ja zuerst ein heftiger Feind und Verfolger des Christenthums; gewiß also mußte er unwiderstehliche Beweise finden, bevor er sich überzeugte, daß die Partei der Christen Recht habe, und daß das mosaische Gesetz, dem er als Pha\RWSeitenw{153}risäer einen so hohen Werth beilegte, durch \RWbet{Jesum} abgeschafft sey. Allein auch bei den Uebrigen, die in den ersten Tagen, als \RWbet{Jesus} sie zu Aposteln berief, eine gewisse Vorliebe für ihn gefaßt haben mochten, mußte sich diese allmählich wieder verlieren, und der strengste Prüfungsgeist an ihre Stelle treten, wenn es sich zeigte, daß er kein irdisches Reich zu stiften gedenke, und daß sie zum Eingange in sein Reich auf keinen Fall anders, als durch Leiden und durch den Tod selbst gelangen könnten.
\item Daß diese Apostel endlich auch immer sehr nahe um \RWbet{Jesum} gewesen. -- So hatte \zB\ der Eine derselben die Kasse in Verwahrung. Sicher hätte es ihnen also, wenn ein Betrug gespielt worden wäre, Ein oder das andere Mal gelingen müssen, etwas davon zu bemerken.
\item Der Umstand, daß sie keine Gelehrte, keine Naturforscher, \udgl\ waren, kann uns ihr Zeugniß nicht verwerflich machen. Denn wenn sie nur nicht schwachsinnige Menschen gewesen, und nicht im Voraus für die Sache \RWbet{Jesu} eingenommen waren, so werden sie gewiß die außerordentlichen Thaten, die er verrichtete, nicht etwa als Wunder angenommen haben, als bis sie sahen, daß diese Thaten auch von den Gelehrten ihrer Zeit und ihres Landes nicht erklärt werden können. Ist aber dieses der Fall gewesen, hat \RWbet{Jesus} Thaten gewirkt, deren Hervorbringung sich selbst die Gelehrten seiner Zeit, welche der Haß gegen ihn doppelt scharfsichtig gemacht hatte, nicht zu erklären wußten: so ist der Umstand, daß Gott nur ihn allein zu dem Besitze solcher Kenntnisse und Kräfte gelangen ließ, als er zu diesen Thaten brauchte, gewiß so außerordentlich, daß Niemand über die Absicht Gottes hiebei im Zweifel bleiben kann. Diese war offenbar keine andere, als daß alle Menschen, welche sich von der Vortrefflichkeit seiner Lehre überzeugt hatten, aus diesen Großthaten den Schluß ziehen möchten, er sey ein göttlicher Gesandte.~\RWSeitenw{154}
\end{aufzc}
\end{aufzb}
\end{aufza}

\RWpar{58}{III.~Der Glaube der ersten Christen, ein Beweis gewisser Wunder, die zur Bestätigung des Christenthumes Statt fanden}
Die Predigt der Apostel, von der ich so eben sprach, fand auch an allen Orten Glauben, und dieser Glaube der ersten Christen, den sie mit ihrem Blute versiegelten, ist ein neuer Beweis, daß sich gewisse außerordentliche Begebenheiten zur Bestätigung des Christenthumes müssen zugetragen haben. -- Besonders merkwürdig sind in dieser Hinsicht folgende Personen:
\begin{aufza}
\item \RWbet{Paulus}, den wir jedoch schon vorhin zu den Aposteln gezählt.
\item \RWbet{Lukas}, der als ein griechischer Arzt hinlängliche Kenntnisse hatte, um zu beurtheilen, ob die Krankenheilungen \RWbet{Jesu} etwas für seine Zeit Außerordentliches waren. Er überzeugte sich hievon so sehr, daß er selbst als Geschichtschreiber \RWbet{Jesu} in der bestimmten Absicht auftrat, um aus den Thaten desselben darzuthun, daß er ein göttlicher Gesandte gewesen sey.
\item Nach der Erzählung der Apostelgeschichte (\RWbibel{Apg}{}{13}{7--12}), die uns in diesem Stücke unmöglich täuschen kann, nahm auch der römische Proconsul der Insel Cypern \RWbet{Sergius Paulus} das Christenthum an:
\item Nach derselben Apostelgeschichte (\RWbibel{Apg}{}{17}{10--14}) befanden sich zu \RWbet{Beröa mehrere Juden}, die, nachdem sie sich erst von der Richtigkeit dessen, was \RWbet{Paulus} gelehrt, durch Vergleichung mit den Aussprüchen des alten Testamentes überzeugt hatten, das Christenthum annahmen. Dieß waren also Männer von Kenntniß und prüfendem Geiste. Eben daselbst wird erzählt, daß auch \RWbet{mehrere heidnische Personen} vornehmen Standes, Männer und Frauen, das Christenthum angenommen hätten.
\item Auch war (\RWbibel{Apg}{Apostelg.}{17}{34}) unter denjenigen, die \RWbet{Paulus} zu Athen bekehrt hatte, \RWbet{Dionysius}, ein Mitglied des \RWbet{Areopagus}, nebst anderen angesehenen Personen.
\item Im Briefe an die Philipper (\RWbibel{Phil}{}{4}{22}) bestellt \RWbet{Paulus}, der sich damals zu Rom befand, einen Gruß von den~\RWSeitenw{155}\ \RWbet{kaiserlichen Hofleuten} (\RWgriech[>asp'azontai <um~as p'antes o<i <ag'ioi, m'alista d`e o<i >ek t~hs Ka~isaros o>ikias]{>asp'azontai <um~as p'antes o<i <'agioi, m'alista d`e o<i >ek t~hs Ka'isaros o>ik'ias}). Also gab es schon selbst unter dem Hofpersonale Christen.
\item Einen ähnlichen Gruß bestellt der Apostel in seinem an die Römer geschriebenen Briefe (\RWbibel{Röm}{}{16}{23}) von \RWbet{Erastus}, den er Rentmeister oder Kämmerer der Stadt (Korinth) nennt.
\item Einige von den \RWbet{Asiarchen} (\di\ Vorstehern der öffentlichen Schauspiele) zu Ephesus lassen (\RWbibel{Apg}{Apostelg.}{19}{31}) den Apostel \RWbet{Paulus} aus Liebe für ihn bitten, sich nicht auf den Schauplatz zu wagen, wo eben ein Auflauf entstanden war. Sie waren also wahrscheinlich Christen oder doch Freunde des Christenthums.
\item Sollte Jemand vielleicht glauben, daß mehrere aus den genannten Personen die christliche Religion nur um der inneren Vortrefflichkeit ihrer Lehre willen angenommen hätten, ohne die Wunder derselben genau geprüft zu haben: so will ich nun Männer anführen, welche als Lehrer und Vertheidiger des Christenthums sogar in Schriften auftraten, von denen es also hinlänglich bekannt ist, daß sie den Beweis der Wahrheit des Christenthums nicht bloß auf die innere Vortrefflichkeit seiner Lehre, sondern auch auf seine Wunder gegründet. Hieher gehören, \RWbet{Aristo}, ein geborner Jude aus Pella in Palästina, der älteste christliche Apologet, dessen Schrift (ein Gespräch zwischen einem Juden und Christen) jedoch verloren gegangen ist. \RWbet{Quadratus}, Bischof zu Athen, welcher zuerst eine (gleichfalls verloren gegangene) Schutzschrift für die Christen (im Jahr 131.) ausarbeitete, und sie dem Kaiser \RWbet{Hadrian} überreichte. \RWbet{Aristides}, ein atheniensischer Philosoph, der auch nach seinem Uebertritt zum Christenthume die vorige Lebensart und Kleidung beibehielt, und eine (gleichfalls verloren gegangene) Vertheidigungsschrift der Christen abfaßte. \RWbet{Hegesippus}, der erste christliche Geschichtschreiber, dessen Werk aber gleichfalls verloren gegangen. \RWbet{Justin der Märtyrer}, ein heidnischer Philosoph, der erst im 30sten Jahre seines Alters zum Christenthume übertrat, und dann zwei Apologien für dasselbe schrieb, deren Eine er dem Kaiser \RWbet{Antoninus} dem Frommen, die zweite dem Kaiser~\RWSeitenw{156}\ \RWbet{Markus Aurelius} überreichte. \RWbet{Melito}, Bischof von Sardes in Lydien, der mehrere Schriften für das Christenthum geschrieben. Ferner gehören hieher: \RWbet{Miltiades, Claudius Apollinaris, Irenäus, Athenagoras, Theophilus} von Antiochien, \RWbet{Tatian} (der in Assyrien geboren, erst Philosophie und Rhetorik, dann die Mysterien mehrerer Völker studirte, endlich das Christenthum kennen lernte, und allem Bisherigen vorzog), \RWbet{Hermias, Hippolytus Portuensis, Ammonius, Pantänus, T.~Fl.~Clemens} von Alexandrien, \RWbet{Origenes, Tertullian, Dionysius} (Bischof von Korinth), \RWbet{Markus Minutius Felix, Thascius Cäcilius Cyprianus} (ein Afrikaner, im Anfange des dritten Jahrhunderts, ein geborner Heide, der sich auf die Beredtsamkeit verlegte, in früheren Jahren etwas unordentlich gewesen, und erst gegen sein vierzigstes Lebensjahr das Christenthum annahm, dann Bischof zu Karthago wurde, vertheidigte die christliche Religion in verschiedenen Schriften, und wurde endlich enthauptet); \RWbet{Arnobius} (gleichfalls ein Afrikaner, Anfangs ein Gegner, dann ein Vertheidiger des Christenthums), \umA\par 
Alle diese Männer waren als geborne Juden oder Heiden erst Feinde des Christenthums, dann traten sie als Vertheidiger desselben auf. Sie lebten im zweiten oder dritten Jahrhunderte, und waren also der Entstehung der christlichen Religion nahe genug, um sich von der Wirklichkeit der erzählten Wunderbegebenheiten zu überzeugen. Ihr Bekenntniß zum Christenthume brachte ihnen nicht den geringsten Vortheil, sondern zog ihnen vielmehr allerlei Verfolgungen, sogar den Tod zu. Sie mußten also die christliche Religion nicht nur für besser, als ihre vorige (jüdische oder heidnische) ansehen, sondern mit fester Ueberzeugung für jene einzige halten, die wahrhaft selig macht.
\item Wenn Jemand zweifeln wollte, ob es auch seine Richtigkeit habe, was uns die Martyrologen und andere christliche Schriftsteller, \zB\ \RWbet{Tertullian, Eusebius} \umA\  von jener Wuth erzählen, mit der die ersten Christen von Juden und Heiden verfolgt wurden, und von der Standhaftigkeit, mit der sie diese Verfolgungen aushielten: so kann~\RWSeitenw{157}\ uns hievon das ganz unverdächtige Zeugniß heidnischer Schriftsteller überweisen. \RWbet{Tacitus} erzählt uns auf folgende Art die Veranlassung, welche den Kaiser \RWbet{Nero} bestimmte, die Christen zu verfolgen (\RWlat{Annal.\ lib.\,15.\ num.\,44.}\RWlit{}{Tacitus1}): \erganf{Durch kein menschliches Mittel, weder durch die Bestechungen des Kaisers, noch durch Sühnopfer, den Göttern dargebracht, konnte der Schande, daß man den Brand (der Stadt Rom) für befohlen (vom Kaiser) hielt, vorgebeugt werden. Um also dem Gerede Einhalt zu thun, \RWbet{unterschob Nero} als Thäter Menschen, die von dem Pöbel ihrer Verbrechen wegen gehaßt und \RWbet{Christen} genannt wurden, und belegte sie mit den \RWbet{ausgesuchtesten Strafen}. Der Urheber jenes Namens war \RWbet{Christus}, der unter dem Kaiser \RWbet{Tiberius} von dem Procurator \RWbet{Pontius Pilatus} hingerichtet wurde. Ihr verdammlicher Aberglaube für einige Zeit unterdrückt, erhob sich wieder, und nicht nur in \RWbet{Judäa}, wo dieses Uebel seinen Ursprung nahm, sondern auch in Rom, wo alles Schlechte und Schändliche von allen Seiten zusammen fließt und gehegt wird. Es wurden also zuerst Einige, welche bekannten (nämlich daß sie Christen seyen\RWfootnote{%
Worte des Tacitus.}), und auf ihre Angabe ward sodann eine \RWbet{ungeheuere Menge} ergriffen, und nicht wegen des Verbrechens der Brandstiftung, sondern \RWbet{vom Haß gegen das Geschlecht der Menschen} verurtheilt. \RWbet{In Thierhäute gehüllt und von Hunden zerfleischt, oder an's Kreuz geschlagen, oder} (in brennbare Stoffe gebunden und) \RWbet{angezündet, und zur Nachtzeit gleich Lichtern benützt, dienten sie den Vorübergehenden zum Schauspiel}. Hiezu gab \RWbet{Nero} seine Gärten her, und mischte sich, wie im Circensischen Spiele, als Fuhrmann verkleidet oder auf einem Wagen sitzend, unter den Pöbel. Dieß aber erregte, obschon sie (als Christen) schuldig waren, und diese unerhörte Strafe verdienten, Mitleid und Erbarmung.} -- \RWbet{Plinius der Jüngere} schreibt an den Kaiser \RWbet{Trajan} (\RWlat{lib.\,10.\ epist.\,101.}\RWlit{}{Plinius1}), daß sich in dem Gebiete seiner Statthalterschaft (Bithynien) sehr viele Christen befänden, und erzählt von Vielen, die wieder abgefallen wären, und ausgesagt hätten: \erganf{daß sie gewohnt wären,~\RWSeitenw{158}\ an bestimmten Tagen vor Anbruch des Morgens zusammen zu kommen, \RWbet{Christo gleich einem Gotte} gemeinschaftlich Lieder zu singen, und sich durch ein Gelübde nicht zu einem Verbrechen, sondern \RWbet{zur Unterlassung des Diebstahles, des Raubes, des Ehebruches, der Treulosigkeit} und \RWbet{zur Rückstellung des anvertrauten Gutes} zu verpflichten, hierauf aus einander zu gehen, sich aber wieder zu versammeln zu einem unschuldigen Mahle, das sie unter einander halten; und auch dieß hätten sie auf mein Edict unterlassen, in welchem ich, nach Deinem Befehle Zusammenkünfte verbot. Daher (heißt es in dem Briefe weiter) hielt ich es für um so nöthiger, von zwei Mägden, die man Dienerinnen (Diakonissinnen) nannte, selbst \RWbet{durch Qualen}, herauszubringen, was an der Sache Wahres sey; aber ich fand nichts, als einen verderblichen Aberglauben ohne Maß und Ziel. Ich trage Dir also, was ich weiß, zur Berathung vor; und die Sache scheint mir der Berathung werth, \RWbet{besonders wegen der Menge derer, die in Gefahr stehen, angesteckt zu werden; denn aus jedem Alter, aus jedem Stande und aus beiden Geschlechtern werden Viele verführt, und Viele werden noch verführet werden}. Schon hat sich dieser Aberglaube, gleich einer Seuche, nicht nur in den \RWbet{Städten}, sondern auch in den \RWbet{Dörfern} und einsamen \RWbet{Landhäusern} verbreitet; und es scheint, daß man sie noch hemmen und ausrotten könne und solle. Nur zu gewiß ist es, \RWbet{daß die Tempel schon beinahe leer stehen, daß der heilige Dienst lange unterbleibe, und daß nur sehr selten noch Opferthiere gebracht werden, die beinahe Niemand mehr kaufen mag}.}
\item Freilich hat man den \RWbet{Einwurf} gemacht, die christliche Religion sey nicht die einzige, die ihre Märtyrer aufzuweisen habe, auch unter allen übrigen Religionen auf Erden habe es Menschen gegeben, welche bereit waren, die empfindlichsten Verfolgungen und selbst den Tod für ihren Glauben auszustehen. -- Diese oder jene müssen sich also geirrt haben; und warum sollten wir nur der Ueberzeugung, welche die christlichen Märtyrer hatten, mehr als den Ueberzeugungen~\RWSeitenw{159}\ anderer Religionsverwandten trauen? -- Hierauf \RWbet{erwiedere} ich aber, es sey für's Erste falsch, was man in diesem Einwurfe annimmt, daß unter mehreren Menschen, deren jeder eine andere Religion für eine göttliche Offenbarung hält, nothwendig Alle bis auf Einen irren müßten. Es können Mehrere aus ihnen, ja Alle Recht haben, weil Gott verschiedenen Menschen verschiedene (wenn gleich nicht eben einander widersprechende) Offenbarungen mittheilen kann. Zweitens der Umstand, daß eine Religion Märtyrer aufzuweisen hat, beweiset freilich noch nicht, daß diese Religion auch für uns eine göttliche Offenbarung sey; beweiset nicht einmal in jedem Falle, daß sich bei ihrer Entstehung und Ausbreitung außerordentliche Begebenheiten zugetragen haben. Wenn nämlich jene Märtyrer Menschen ohne allen Prüfungsgeist waren, und in der Religion, für deren Göttlichkeit sie sterben, bereits geboren wurden; dann läßt sich aus ihrem Martyrthume allerdings nicht viel schließen; denn es ist da begreiflich, wie sie ihre Religion mit Ueberzeugung festhalten konnten, ohne sie je geprüft zu haben. -- Ganz anders verhält es sich aber mit jenen ersten Christen, auf deren Zeugniß wir hier unsern Beweis stützen, daß sich zur Bestätigung des Christenthums gewisse Wunder zugetragen haben. Diese christlichen Märtyrer starben für eine Religion, welche sie keineswegs von ihren Vorfahren schon überkommen hatten, für die kein Vorurtheil des Alterthums sprach, die sich durch nichts, als durch die Vortrefflichkeit ihrer Lehre und die Glaubwürdigkeit ihrer Wunder empfehlen konnte. Hier können wir also mit Recht voraussetzen, daß sie nicht ohne Prüfung geglaubt haben werden. Und da sie in einem Zeitalter lebten, wo es ein Leichtes war, den Grund oder Ungrund der von \RWbet{Jesu} gewirkten und von den Aposteln erzählten Wunder zu erkennen, \dh\ auszumitteln, ob sich hier etwas (wenn gleich vielleicht Natürliches, doch gewiß) Ungewöhnliches zugetragen habe, oder nicht; da Mehrere aus ihnen für das Bekenntniß, daß sich dergleichen ungewöhnliche Dinge ergeben hätten, selbst in den Tod gingen, da sie behaupteten, diese Wunder nicht bloß vom Hörensagen zu kennen, sondern mit ihren eigenen Augen gesehen zu haben; da sich in ihrer Mitte so viele sachkundige, gelehrte Personen befanden: so können wir in~\RWSeitenw{160}\ der That auf keine Weise zweifeln, daß sich hier manches Außerordentliche müsse ergeben haben. Und so betrachtet hat keine andere Religion auf Erden ein ähnliches Zeugniß für sich, wie dieses der ersten Märtyrer des Christenthums, wodurch wir aber, wie gesagt, nicht läugnen wollen, daß nicht auch einige andere Religionen ihre Wunder aufzuweisen haben.
\end{aufza}

\RWpar{59}{IV.~Das Betragen der Feinde des Christenthums, ein Beweis gewisser Wunder, die zur Bestätigung desselben Statt gefunden haben}
Die christliche Religion fand bei sehr vielem Beifalle auch ihre bedeutenden \RWbet{Feinde} und \RWbet{Gegner}; sie wurde von jüdischen und heidnischen \RWbet{Obrigkeiten}, von jüdischen und heidnischen, mitunter auch von christlich-gebornen \RWbet{Gelehrten} angefeindet. Aber das Betragen, das diese Gegner des Christenthums beobachteten, ist von einer solchen Art, daß sich gerade aus ihm einer der stärksten Beweise herleiten läßt, daß diese Religion nicht ohne gewisse, ganz außerordentliche Begebenheiten entstanden, und verbreitet worden sey.
\begin{aufza}
\item Merkwürdig ist in dieser Rücksicht zuerst schon das Betragen \RWbet{Judä}, der aus einem Apostel ein Verräther ward. Wenn die bekannte Geschichte, die uns das Evangelium von dem Verrathe \RWbet{Jesu} erzählt, nicht wirklich wahr wäre: so hätte sie unmöglich dort erzählt werden können. Denn die Erzähler berufen sich ja auf ein Denkmal dieser Begebenheit, welches noch zu der Zeit, da sie schrieben, bestanden haben soll (nämlich auf den Namen Hakeldama). Es ist also wahr, daß Einer von den zwölf Jüngern \RWbet{Jesu}, und zwar derjenige, dem eben die Kasse der Gesellschaft (das Werkzeug, ohne dessen Mitwirkung kein Betrug hätte ausgeführt werden können) anvertraut war, aus schnöder Gewinnsucht sein Verräther geworden; es ist wahr, daß er bei diesem Verrathe nicht das geringste Verbrechen von seinem Meister aufzudecken gewußt, sondern nichts Anderes, als seinen Aufenthalt angezeigt habe; denn im entgegengesetzten Falle würde man, statt sich mit anderen erkauften Zeugen, die noch dazu nichts Wichtiges und Uebereinstimmendes zu sagen wußten, schlecht genug~\RWSeitenw{161}\ zu behelfen, das Zeugniß dieses Apostels vor Gericht aufgeführet haben. Es ist endlich wahr, daß dieser Unglückliche bald nach vollbrachter Schandthat, als er den traurigen Ausgang derselben zu ahnen anfing, von Reue ergriffen das Geld zurückgestellt, seinen Meister für unschuldig erklärt, und, als er sah, daß er auch hiedurch ihn nicht mehr retten könne, sich aus Verzweiflung selbst entleibet habe. Würde dieß wohl geschehen seyn, wenn \RWbet{Judas} nicht auf das Innigste von der Unschuld \RWbet{Jesu} und von der Wirklichkeit seiner Wunder überzeugt gewesen wäre?


\begin{RWanm} Ein Ungenannter hat in der Schrift: \RWlat{Observations on the Conduct and Character of Judas Ischariot} (2.~Edit.\ Edinb.\ 1751, p.\,43.)\RWlit{}{Anonym1} diesen Beweis umständlicher, aber mit etwas Uebertreibung und Declamation auseinandergesetzt. \end{RWanm}

\item Eben so merkwürdig ist auch das Benehmen des \RWbet{hohen Rathes} in der Sache \RWbet{Jesu}. Ohne auch eben die Glaubwürdigkeit der heil.\ Evangelien vorauszusetzen, kann man doch als gewiß annehmen, daß der hohe Rath auf die Hinrichtung \RWbet{Jesu} unter keinem anderen Vorwande gedrungen habe, als unter dem, daß er Gott gelästert, und das Volk aufgewiegelt habe. Aus diesem Betragen nun folgt
\begin{aufzb}
\item daß jene außerordentlichen Thaten, die \RWbet{Jesus} gewirkt (jene Krankenheilungen \usw ), über jeden Verdacht eines Betruges erhaben seyn mußten, weil man sonst nicht ermangelt hätte, ihn lieber des Betruges oder der Zauberei \udgl\ zu beschuldigen;
\item daß auch der Vorwand einer Volksaufwiegelung nur ein erdichteter gewesen seyn müsse, weil sonst gewiß nicht die jüdische, sondern die heidnische Obrigkeit auf seine Bestrafung gedrungen hätte, wie denn auch die Jünger, als seine Gehülfen bei diesem Aufruhre, dann keineswegs hätten verschont bleiben können.
\end{aufzb}
\item Eine eigene Betrachtung verdient das Benehmen der \RWbet{heidnischen Obrigkeiten}.
\begin{aufzb}
\item Der Kaiser \RWbet{Nero} war der Erste, der ein Verfolgungsedict wider die Christen ausgab (die Veranlassung hörten wir oben). Dieses Edict ward von den folgenden Kaisern bis auf \RWbet{Constantin den Großen} \Ahat{nie}{wie} förmlich~\RWSeitenw{162}\ widerrufen, obgleich verschiedentlich abgeändert; daher es drei Jahrhunderte hindurch jedem grausamen oder habsüchtigen Statthalter frei stand, eine Verfolgung der Christen anzufangen. Niemals beschuldigte man sie hiebei eines Betruges, sondern bloß dieses, daß sie dem Christenthume anhingen, und also die Götzen der Heiden nicht verehren wollten, rechnete man ihnen zu einem todeswürdigen Verbrechen an. Dabei ließ man sich nie in eine eigentliche Widerlegung ihrer Religion ein, \dh\ mit einem Worte, man betrug sich ganz so, wie sich zu allen Zeiten Richter betragen, wenn sie eine unparteiliche Untersuchung scheuen, weil sie befürchten, daß sie nicht zu ihrem Vortheile ausfallen würde.
\item Hiezu kommt noch, daß mehrere römische Kaiser und obrigkeitliche Personen dem Christenthume eine gewisse Achtung zu zollen, sich nicht erwehren konnten. Daß \RWbet{Plinius} den Christen nicht abgeneigt gewesen, war aus der oben angeführten Stelle seines Briefes deutlich genug zu ersehen. Aus der Antwort des Kaisers \RWbet{Trajan} ersieht man, daß dieser die Christen gleichfalls nicht als Verbrecher behandelt wissen wollte. Der Kaiser \RWbet{Antonin der Fromme} aber verordnete sogar, wenn Jemand einen Christen als solchen anklage, so soll nicht der Beklagte, sondern der Ankläger bestrafet werden. \RWbet{Alexander Severus} bezeugte sich als einen so großen Freund des Christenthums, daß er unter Anderm wollte, man sollte die Worte \RWbet{Jesu}: \anf{Was ihr nicht wollet} \usw\ an alle öffentlichen Gebäude schreiben. Auch soll er nach \RWbet{Lampidii} Zeugniß neben den Göttern der Römer \RWbet{Christum} verehret haben. Sein Nachfolger \RWbet{Philipp} war den Christen so günstig, daß mehrere Geschichtschreiber ihn schon für einen wirklichen Christen erklären.
\end{aufzb}
\item Betrachten wir nun noch das Benehmen der \RWbet{Gelehrten}, und zwar zuerst das Benehmen einiger \RWbet{jüdischer Schriftsteller}. Die Juden hatten gerade zur Zeit der Entstehung des Christenthums ein Paar sehr achtungswürdige Schriftsteller, \RWbet{Philo} und \RWbet{Josephus}. Ein dritter \RWbet{Justus von Tiberias} ist minder merkwürdig.~\RWSeitenw{163}
\begin{aufzb}
\item \RWbet{Philo}, der Aeltere von jenen Beiden, mit dem Zunamen \RWbet{Judäus}, war ein völliger Zeitgenosse \RWbet{Jesu}. Er lebte zu Alexandrien (also in eben dem Lande, in welchem \RWbet{Jesus} nach der Beschuldigung Einiger die Zauberkunst erlernt haben sollte) und schrieb in jüdischer Sprache: \RWlat{adversus Flaccum\RWlit{}{Philon1}, de legatione ad Cajum\RWlit{}{Philon2}}, und mystische Commentarien über das alte Testament. In diesen Schriften hatte er freilich keine Veranlassung, des Christenthums zu erwähnen; es fragt sich aber, warum er bei seiner Gelehrsamkeit nicht als Widerleger desselben auftrat, wodurch er seiner Religion und seinen Landsleuten, die in so großer Menge zu dieser Religion übertraten, einen sehr zeitgemäßen und wichtigen Dienst geleistet haben würde, wenn anders die Sache des Christenthums ein Betrug war. Wir müssen also vermuthen, er habe sich außer Stande gefühlt, einen Betrug in dieser Religion nachzuweisen.
\item \RWbet{Flavius Josephus} schrieb: \RWlat{de bello judaico\RWlit{}{Josephus4}, contra Apionem\RWlit{}{Josephus3}}, \RWgriech{>Arqaiolog'ian} (eine Geschichte des jüdischen Volkes von seinem Ursprunge) \RWlat{de vita sua\RWlit{}{Josephus2}, de Maccabaeis\RWlit{}{Josephus6}}. In diesen Schriften kommen, wie wir schon oben sahen, verschiedene Stellen vor, welche die Wahrheit unserer evangelischen Geschichte auf eine recht auffallende Art bestätigen; aber nur eine einzige (\RWlat{Antiquit.\ lib.\,18.\ c.\,3.\ \RWparnr{3}}\RWlit{}{Josephus1}), in der er von der Person \RWbet{Jesu Christi} etwas Ausführliches erwähnet. Sie lautet: \anf{Um diese Zeit (im Vorhergehenden war von der Tyrannei des Statthalters \RWbet{Pilatus} die Rede) lebte \RWbet{Jesus}, ein weiser Mann, darf man ihn anders einen Menschen nennen; denn er verrichtete die außerordentlichsten Thaten, und war ein Lehrer derer, welche die Wahrheit hören. Er zog sehr Viele, Juden sowohl als Heiden, an sich, und war der Messias. Obgleich \RWbet{Pilatus} auf Anklage unserer Vornehmen ihn zum Kreuze verurtheilte: so hörte doch die Gesellschaft derjenigen, welche ihn vorher geliebt, nicht auf; denn er zeigte sich ihnen am dritten Tage lebendig; ganz so, wie die Propheten dieß und mehrere andere wundervolle Dinge von ihm vorhergesagt hatten. Noch bis jetzt (etwa im Jahre~\RWSeitenw{164}\ 90, als \RWbet{Josephus} seine Archäologie schrieb) hat die Gesellschaft der von ihm genannten Christen nicht aufgehört.} -- Wenn diese Stelle ächt und unverfälscht ist: so ist sie offenbar ein sehr entscheidendes Zeugniß für die Wirklichkeit der Wunder \RWbet{Jesu}. Allein die Aechtheit und Unverfälschtheit derselben unterliegt großen Zweifeln, besonders darum, weil die älteren Vertheidiger des Christenthums ihrer nie erwähnen. Indessen findet sich noch eine zweite Stelle (\RWlat{Antiquit.\ l.\,20.\ c.\,9.\ \RWparnr{1}}\RWlit{}{Josephus1}), in welcher der Person unseres Herrn, obwohl nur im Vorübergange, erwähnt wird: Die Zeit, als kein Landpfleger in Judäa war, als \RWbet{Festus} gestorben, und sein Nachfolger \RWbet{Albinus} noch nicht gekommen war, benützte der damalige hohe Priester \RWbet{Ananus} der Jüngere, um ein Gericht zu versammeln, vor welches er den \RWbet{Bruder} jenes \RWbet{Jesu}, der der Messias genannt wird, Namens \RWbet{Jakobus}, nebst einigen Anderen vorführte. Er klagte sie als Uebertreter des Gesetzes an, und ließ sie steinigen. Die Billigsten der Juden aber verklagten ihn deßhalb, und er verlor die Hohepriesterwürde. Diese Stelle beweist:
\begin{aufzc}
\item daß \RWbet{Josephus} irgendwo ausführlicher von der Person \RWbet{Jesu} gesprochen haben muß, weil er seiner hier auf eine Art erwähnt, als setzte er voraus, daß seine Leser ihn schon kennen;
\item daß der sogenannte Bruder \RWbet{Jesu, Jakobus} (den uns die christliche Kirchengeschichte als Bischof von Jerusalem beschreibt) ein sehr rechtschaffener Mann gewesen seyn müsse, dem der hohe Priester kein anderes Verbrechen, als die Abweichung vom mosaischen Gesetze (\dh\ das Christenthum) vorzuwerfen wußte. Wenn nun der Bruder \RWbet{Jesu}, sein Anhänger, ein so rechtschaffener Mann gewesen, kann wohl \RWbet{Jesus} selbst ein Betrüger gewesen seyn?
\item daß die Sache des Christenthums bei dem vernünftigen Theile der Nation von Jahr zu Jahr in ein größeres Ansehen gekommen sey. Folgt nun hieraus nicht, daß die Thaten \RWbet{Jesu} wirklich sehr außerordentlich gewesen~\RWSeitenw{165}\ seyn mußten; weil man im widrigen Falle von ihrer Bewunderung schon lange hätte zurückgekommen seyn müssen?
\end{aufzc}
\item Die späteren jüdischen Schriftsteller, insonderheit die Verfasser des \RWbet{Talmud} (im 2ten, 4ten und 6ten Jahrhunderte) ingleichen die Verfasser der sogenannten \RWbet{Lebensgeschichten Jesu} (Toldot Jeschu) geben die Wunder \RWbet{Jesu} durchgängig zu, gestehen, daß er Aussätzige gereinigt, Todte auferweckt habe, \udgl ; nur behaupten sie, bald daß er dieß durch den Namen \RWbet{Jehova} (dessen wahre Aussprache er durch einen Zufall erfahren), bald durch die Hülfe böser Geister, bald wieder durch gewisse in Aegypten erlernte Zauberkünste ausgeführt habe.
\end{aufzb}
\item Nicht minder merkwürdig ist das Benehmen der \RWbet{heidnischen Schriftsteller}.
\begin{aufzb}
\item \RWbet{Phlegon}, ein Freigelassener des Kaisers \RWbet{Hadrian} (im Jahr 138) gesteht in seinen \RWgriech{>Olumpioniko~is}, wovon jedoch nur einige Fragmente übrig sind, unserm Herrn \RWbet{Jesu} die Vorherwissenheit künftiger Dinge zu, und bezeugt (bei Origenes), daß die Weissagung \RWbet{Jesu} (wir wissen aber nicht, von welcher die Rede sey) genau erfüllt worden ist.
\item \RWbet{Lucian}, aus Samosata in Syrien gebürtig, Landpfleger (Präses) von Aegypten, dieser berüchtigte Spötter (\RWlat{hominumque deumque irrisor}\RWlit{}{Lukianos6}) der \RWlat{dialogos deorum\RWlit{}{Lukianos1}}, \RWlat{de dea Syria\RWlit{}{Lukianos2}}, \RWlat{de morte Peregrini\RWlit{}{Lukianos4}}, \RWlat{vitam Alexandri\RWlit{}{Lukianos3}}, \RWlat{de vera historia\RWlit{}{Lukianos5}}, und andere Schriften verfaßt, und im Jahre 112 von Hunden zerrissen worden seyn soll, stellt die Christen als Menschen dar, deren Herr in Palästina gekreuziget wurde, die durch die zuversichtliche Erwartung eines ewig seligen Lebens nach dem Tode allen Reizungen der Welt, und allen Martern Trotz böten, bei ihrer Ehrlichkeit von Anderen zwar oft betrogen würden, aber den Zauberern gleichwohl sehr gefährlich wären. So erzählt er in \RWlat{vita Alexandri}, daß dieser Betrüger Epikuräer und Christen von seinen Versammlungen jederzeit abgehalten habe: \RWgriech{>'exw Qristiano'us >'exw >Epikoureio'us}. Hätte \RWbet{Lucian} ir\RWSeitenw{166}gend etwas wider den Charakter \RWbet{Jesu} oder wider die Wirklichkeit seiner Wunder zu sagen gewußt: so würde er sicher nicht ermangelt haben, es in seinen Schriften anzubringen.
\item \RWbet{Celsus}, ein Epikuräischer, oder, wie es wahrscheinlicher ist, ein Neuplatonischer Philosoph, der im zweiten Jahrhunderte gelebt, Syrien und Palästina durchreiset, die Bücher des alten und neuen Bundes gelesen, und also in Ansehung des Christenthums von sich sagen konnte: Ich weiß alles, schrieb ein Werk gegen die Christen (\RWgriech[l'ogos >alhj~hs]{l'ogos >alhj'hs}), das zwar verloren gegangen, dessen Inhalt jedoch \RWbet{Origenes} in seiner Widerlegung (\RWlat{contra Celsum lib.\,8.}\RWlit{}{Origenes1}) vollständig aufbewahrt hat, indem er die Einwürfe seines Gegners beinahe wörtlich anführt. Aus diesen Anführungen sehen wir nun, daß \RWbet{Celsus} ein sehr unredlicher Gegner des Christenthums gewesen, der diese Religion keineswegs mit vernünftigen Gründen, sondern mit Spöttereien, Beschimpfungen, \udgl\  angegriffen. Er beweist nirgends die Falschheit der evangelischen Geschichte, sondern behauptet nur, daß diese und jene Begebenheit an sich selbst ungereimt wäre, setzet voraus, daß \RWbet{Jesus} seine Wunder durch eine in Aegypten erlernte Zauberkunst gewirkt habe, und läugnet seine Auferstehung bloß aus dem Grunde, weil \RWbet{Jesus} sich nicht öffentlich dargestellt habe.
\item \RWbet{Porphyrius}, zu Tyrus im Jahr 233 geboren, genoß in seiner frühesten Jugend den Unterricht des \RWbet{Origenes}, dann mehrerer heidnischer Lehrer, vorzüglich des \RWbet{Plotinus} zu Rom. Aus Schwermuth hätte er sich beinahe selbst entleibet. Um sich also zu zerstreuen, reisete er nach Sicilien, wo er seine gegen das Christenthum gerichtete Schriften (15 Bücher gegen die Christen) in griechischer Sprache herausgab. Sie sind verloren gegangen, allein \RWbet{Eusebius, Hieronymus, Augustinus}, \uA , die ihn widerlegten, haben uns seine Gründe aufbewahrt; und wir ersehen hieraus, daß dieser gelehrte Mann, der die Bücher des alten und neuen Bundes genau kannte, nirgends die Wahrheit derselben angefochten habe. Er behauptet nur, daß die Stammtafeln bei \RWbet{Matthäus}~\RWSeitenw{167}\ und \RWbet{Lukas} im Widerspruche mit einander stünden; sodann, daß \RWbet{Daniel's} Weissagung vom Messias von Christen unterschoben sey, eine Behauptung, die schon durch den Umstand, daß sich die Weissagung auch in den Handschriften der Juden vorfindet, alle Wahrscheinlichkeit verliert, und nur beweiset, daß \RWbet{Jesus} die in den Büchern des alten Bundes vorkommenden Kennzeichen des Messias in seiner Person genau vereiniget haben müsse, weil sich \RWbet{Porphyrius} nicht anders zu helfen gewußt, als daß er diese Weissagungen für unterschoben erklärte. Uebrigens gestand er ausdrücklich, daß \RWbet{Jesus} ein frommer Mann gewesen, der zur Belohnung seiner Frömmigkeit in den Himmel sey aufgenommen worden. Aus dieser Aeußerung ersieht man, daß \RWbet{Porphyrius} im Grunde ein Christ gewesen, und nur, weil ihm die Sitten der Christen zu verdorben schienen, nicht förmlich übertreten sey.
\item \RWbet{Hierokles}, Statthalter zu Bithynien, dann zu Alexandrien, schrieb im Jahre 303: (Freundschaftliche Ermahnungen an die Christen\RWlit{}{Herokles}) nachdem er die Christen vorher eben nicht freundschaftlich behandelt hatte. Auch seine Schrift ist bis auf ein kleines Fragment bei \RWbet{Eusebius} untergegangen. Doch können wir aus der Widerlegung eben dieses \RWbet{Eusebius} und des \RWbet{Lactantius} zur Genüge schließen, welche Gründe er gegen das Christenthum vorgebracht habe. Auch er beschimpfte, wie \RWbet{Celsus}, den Charakter \RWbet{Jesu}, behauptete, daß die Bücher des neuen Bundes eine Menge Widersprüche enthielten, und daß \RWbet{Apollonius} von Tyana ein viel größerer Wunderthäter, als \RWbet{Jesus}, gewesen wäre. Auch er gab also zu, daß \RWbet{Jesus} Wunder gewirkt, und wollte sie nur durch die noch größeren des \RWbet{Apollonius} ihrer beweisenden Kraft berauben.
\item \RWbet{Julian der Abtrünnige}, jener römische Kaiser im vierten Jahrhunderte, der in der christlichen Religion erzogen, sie später mit dem Götzendienste vertauschte, schrieb drei Bücher wider das Christenthum, die aber (wie alle bisher erwähnten Schriften der heidnischen Philosophen auf Befehl der späteren christlichen Kaiser) vertilgt wurden, daher wir über ihren Inhalt nur zum Theile~\RWSeitenw{168}\ aus des alexandrinischen Bischofs \RWbet{Cyrillus} Widerlegung (in 10 Büchern) urtheilen können. \RWbet{Julian} gab, wie wir hieraus entnehmen, zu, daß \RWbet{Jesus} Lahme, Blinde, Besessene geheilt; dieß seyen aber, meint dieser Kaiser, keine so großen Werke, sondern nur Beweise ärztlicher Erfahrung, zumal da sie nur an gemeinen Leuten verrichtet worden wären. Hätte \RWbet{Julian} historische Gründe gegen die Richtigkeit der Wunder \RWbet{Jesu} beigebracht: so würde \RWbet{Cyrillus} sie gewiß mit aller Ehrlichkeit angeführt haben, da er der Einwürfe manche anführt, deren Widerlegung ihm Mühe genug gekostet.
\end{aufzb}
\end{aufza}

\RWpar{60}{V.~Die Urtheile der Gelehrten unserer Zeit beweisen abermals, daß sich gewisse Wunder zur Bestätigung des Christenthums zugetragen}
Wenn Jemand glauben sollte, daß die Entdeckungen, die man in unserer neuesten Zeit in so verschiedenen Fächern des menschlichen Wissens, in der Geschichte, in der Naturwissenschaft, in der Arzneikunde \usw\ gemacht hat, uns vielleicht Aufschlüsse geben, die uns berechtigen, die Wahrhaftigkeit der evangelischen Geschichte, oder das Wunderbare der hier erzählten Ereignisse zu läugnen: so würden die Urtheile, welche die \RWbet{Gelehrten unserer Zeit} über diesen Gegenstand gefällt, ihn ganz beruhigen müssen. In unserer neuesten Zeit ward die Geschichte der Entstehung und Ausbreitung des Christenthums nicht einmal, sondern unzählige Male der strengsten Prüfung unterworfen; häufig mit der entschiedensten Neigung, Alles, was einem Wunder ähnlich wäre, und diese Religion als eine göttliche beurkunden würde, im Voraus zu bezweifeln und abzuläugnen; und das Ergebniß dieser Prüfungen war, wie folgt.
\begin{aufza}
\item Der größere Theil der Gelehrten, und zwar gerade die Classe derjenigen, die ihre Prüfung mit der gewissenhaftesten Unbefangenheit unternommen zu haben scheinen, die auch durch andere Untersuchungen sich als die bescheidensten Forscher und als die gründlichsten Beurtheiler des Alterthums bewiesen haben, entschieden für das Christenthum; entschieden,~\RWSeitenw{169}\ daß bei der Entstehung und Ausbreitung desselben allerdings viele und unläugbare Wunder (nämlich gerade diejenigen, die uns die Bücher des neuen Bundes erzählen) Statt gefunden hätten. Zu dieser Classe gehören \zB\ \RWbet{Hugo Grotius, Leibnitz, Lardner, G.\,E.~Lessing, G.~Leß, J.\,M.~Schröckh, \Ahat{J.\,J.~Heß}{J.\,F.~Heß}, Joh.~v.~Müller, Eichhorn, Plank, u.\,v.\,A.}
\item Nur ein sehr kleiner Theil war es, der wider das Christenthum entschied, \dh\ der läugnete, daß bei Entstehung und Ausbreitung des Christenthums Wunder von der Art Statt gefunden hätten, wie sie die Bücher des neuen Bundes erzählen. Bei einer näheren Betrachtung zeigt sich jedoch, daß auch schon dasjenige, was diese Gelehrten zugeben, vollkommen hinreichend sey, die Wahrheit des Christenthums, als einer göttlichen Offenbarung zu beurkunden. Auch diese Gelehrten behaupten nämlich nichts Anderes, als daß es mit dieser oder jener einzelnen Begebenheit, die uns die Bücher des neuen Bundes erzählen, \zB\ mit der Auferstehung \RWbet{Jesu}, nicht seine Richtigkeit habe; sie behaupten ferner, daß bei der Entstehung und Ausbreitung des Christenthums nirgends Wunder in der Bedeutung, die man bisher in den Schulen als die allein gültige ansieht, nämlich nicht \RWbet{übernatürliche} und \RWbet{unmittelbare Wirkungen Gottes} Statt gefunden hätten. Aber sie sehen sich alle genöthiget, zuzugestehen, daß sich zu Gunsten des Christenthums sehr viele ungewöhnliche Ereignisse, die auch ganz anders hätten erfolgen können, ergeben haben. Dieses ist aber, nach unserer Theorie von den Kennzeichen einer Offenbarung, schon völlig genug, um zu schließen, es sey der Wille Gottes, daß wir die Lehre des Christenthums annehmen, \dh\ daß wir das Christenthum als seine Offenbarung erkennen.
\end{aufza}

\RWpar{61}{Schlußfolgerung aus dem Bisherigen}
So ist denn die Sache des Christenthums durch einen Zeitraum von achtzehn Jahrhunderten von Feinden und Freunden geprüft worden, und nicht nur jene, die \RWbet{für}, sondern auch jene, die \RWbet{gegen} dasselbe entschieden, behaupteten (die Letz\RWSeitenw{170}teren ohne es zu wissen und zu wollen) Dinge, aus welchen folgt, daß diese Religion das äußere Kennzeichen einer göttlichen Offenbarung, nämlich die \RWbet{Bestätigung durch Wunder}, habe. Wer sich daher nicht weiser dünken will, als alle diese Personen, wird zugeben müssen, daß sie in einem Stücke, darin sie \RWbet{Alle übereinstimmen}, sich nicht geirrt haben. Und wer behaupten wollte, sie hätten sich dennoch geirrt, der würde eben hiedurch ein Ereigniß annehmen, welches so ungewöhnlich ist, und zugleich so sehr zum Vortheile des Christenthums dient, daß man es abermals als ein die Göttlichkeit dieser Religion erweisendes Wunder ansehen, und somit behaupten müßte, Gott habe diesen so allgemeinen Irrthum nur darum zugelassen, weil er gewollt, daß wir die christliche Religion als seine Offenbarung annehmen sollen.\par
Auf jeden Fall ist es also \RWbet{nicht bloß wahrscheinlich}, sondern \RWbet{moralisch gewiß}, daß unser Christenthum das äußere Kennzeichen einer göttlichen Offenbarung habe.


\RWabs{Zweite Abtheilung.}{Einzelne Wunder, die zur Bestätigung des Christenthums dienen.}

\RWpar{62}{I.~Eine das Christenthum betreffende Weissagung}
Das Bisherige wird hoffentlich hingereicht haben, um einen Jeden, der es gehörig erwägt, bis zur vollkommensten Beruhigung zu überzeugen, daß dem Christenthume das äußere Merkmal einer göttlichen Offenbarung, die Bestätigung durch Wunder zukomme. Damit wir uns aber auch einen Begriff machen können von der Beschaffenheit dieser Wunder: so wollen wir jetzt einige der merkwürdigsten, welche uns die Geschichte aufbewahrt hat, \RWbet{im Einzelnen} betrachten. Obwohl nun jedes der in den Büchern des neuen Bundes erzählten Wunder schon hinlänglich glaubwürdig ist, wie aus den Untersuchungen des vorigen Hauptstückes erhellet: so haben einige~\RWSeitenw{171}\ derselben doch einen noch höheren Grad der Verlässigkeit, als andere; denn der Grad der Glaubwürdigkeit menschlicher Zeugnisse kann in's Unendliche wachsen, und es gibt keine sittliche Gewißheit, die nicht noch größer werden könnte. Ich werde also solche zur Bestätigung des Christenthums dienende Wunder berühren, die meiner Ansicht nach unter sehr vielen glaubwürdigen die allerglaubwürdigsten sind. Hieher rechne ich nun zuvörderst folgende, \RWbet{von unserm Herrn Jesu selbst ausgesprochene Weissagung, die schnelle Verbreitung, und die nachherige stete Fortdauer des Christenthums betreffend}.
\begin{aufza}
\item In den Büchern des neuen Bundes wird nämlich an mehreren Orten erzählt, daß der Herr \RWbet{Jesus} mit aller Bestimmtheit vorhergesagt habe, die religiöse Gesellschaft, die er jetzt stifte, werde mit seinem Tode nicht nur nicht wieder untergehen, sondern sich vielmehr überaus schnell (innerhalb eines Menschenalters) durch das ganze römische Reich verbreiten, ja einstens alle Menschen in ihren Schooß aufnehmen, und bis an das Ende unseres Geschlechtes bestehen. Gehet, und lehret alle Völker \usw\ Und wisset, ich bin bei euch alle Tage bis an das Ende der Zeit (\RWbibel{Mt}{Matth.}{28}{19}). Er verspricht also, daß es den Aposteln (und ihren Nachfolgern) gelingen werde, das Christenthum bei allen Völkern einzuführen. Ich habe auch noch andere Schafe, die nicht von dieser Heerde sind; auch diese muß ich herbeiführen; auch sie werden meiner Stimme folgen, und es wird Eine Heerde nur seyn und ein Hirt (\RWbibel{Joh}{Joh.}{10}{16}). Ich sage dir, du bist \RWbet{Petrus} (ein Fels), und auf diesen Felsen will ich meine Kirche bauen, und die Pforten der Hölle sollen sie nicht überwältigen (\RWbibel{Mt}{Matth.}{16}{18}). Also wird die Kirche immer fortdauern. Merkwürdig ist in dieser Hinsicht auch schon die Einsetzung des heiligen Abendmahles zu seinem Andenken \RWbet{für alle künftige Zeit}. Denn setzt diese Anordnung von \RWbet{Jesu} Seite nicht die gewisseste Ueberzeugung voraus, daß seine Religion einst einen glänzenden Sieg über das Juden- und Heidenthum davon tragen werde?
\item Wie vollkommen diese Vorhersagung \RWbet{Jesu} erfüllt worden sey, wie schnell sich das Christenthum verbreitet habe,~\RWSeitenw{172}\ daß es noch vor dem Ableben der Apostel durch das ganze römische Reich und selbst noch anderwärts verkündigt worden sey, überall Anhänger gefunden habe, und daß es nun bereits achtzehn Jahrhunderte bestehe, und über alle Verfolgungen der Heiden, der Juden und der Freigeister den herrlichsten Sieg davon getragen habe, ist eine allbekannte Sache.
\item Daß aber in jener Vorhersagung, verbunden mit dieser Erfüllung, etwas Wunderbares liege, wird Jedermann zugestehen müssen, der nur das Folgende erwägt. Weder die schnelle Verbreitung des Christenthums, noch seiner Fortdauer durch einen Zeitraum von bereits achtzehnhundert Jahren konnte irgend Jemand zu jener Zeit, da unsere Evangelien niedergeschrieben worden sind, durch seine bloße Vernunft als etwas völlig Gewisses vorhersehen. Denn wie leicht hätte es der jüdischen Obrigkeit nicht gelingen können, das aufkeimende Christenthum gleich bei seiner Entstehung zu ersticken? Es wäre nur nöthig gewesen, nach der Hinrichtung \RWbet{Jesu}, auch die kleine Schaar seiner Jünger aus dem Wege zu räumen: und das Christenthum wäre vertilgt gewesen. Die Apostel hätten nur weniger standhaft seyn dürfen, und durch die Bedrohungen des hohen Rathes sich sollen abschrecken lassen: so wäre es um die Verbreitung dieser Religion geschehen. Und wie leicht hätte dieß nicht auch in den späteren Zeiten bei den oftmaligen Verfolgungen durch die römischen Kaiser geschehen können? oder durch jene Völkerwanderung? oder durch die so furchtbare Gewalt der Sarazenen, wenn sie nicht \RWbet{Karl Martell} zurückgeschlagen hätte? \usw\ -- Wir können hier offenbar nur Eines von Beiden annehmen; die Verfasser der Evangelien haben die schnelle Verbreitung und den Sieg des Christenthums über alle feindlichen Mächte entweder mit Sicherheit vorhergewußt, oder auf's Gerathewohl behauptet.
\begin{aufzb}
\item Haben sie Alles nur auf's Gerathewohl behauptet: ist es nicht merkwürdig, daß Gott ihre Vorhersagungen in Erfüllung gebracht hat? gibt er uns hiedurch nicht zu erkennen, daß er die Religion, die diese Männer geprediget, begünstigt sehen wolle?
\item Doch in der That läßt sich nicht einmal annehmen, daß diese Männer nur auf's Gerathewohl gesprochen hätten; sie~\RWSeitenw{173}\ mußten vielmehr fest überzeugt seyn, daß sie ihr Leben nicht umsonst aufopfern, wenn sie für die Verbreitung des Christenthums sterben. Wie ist nun diese Ueberzeugung in ihnen entstanden? Was hat sie vermocht, dieß nicht nur zu glauben, sondern es selbst für eine von ihrem Meister empfangene Weissagung auszugeben, wenn nicht er selbst so gesprochen, und durch Thaten, die Niemand zu wirken vermag, erwiesen, daß Alles eintreffen müsse, was er vorhersage?
\end{aufzb}
\end{aufza}

\RWpar{63}{II.~Weissagungen Jesu, das Volk der Juden betreffend}
\begin{aufza}
\item Die drei Evangelisten \RWbet{Matthäus, Markus} und \RWbet{Lukas} erzählen uns, daß \RWbet{Jesus} bei Gelegenheit, als er seinen letzten Einzug in Jerusalem gehalten, den baldigen Untergang dieser Stadt beweinet habe; daß er hierauf, als die vier Jünger \RWbet{Petrus, Jakobus, Johannes} und \RWbet{Andreas} ihn über die näheren Umstände dieses Unterganges befragt, eine ausführliche Belehrung hierüber mitgetheilt habe, welche bei \RWbet{Matthäus} (\RWbibel{Mt}{}{24}{1--51}), \RWbet{Markus} (\RWbibel{Mk}{}{13}{1--37}), \RWbet{Lukas} (\RWbibel{Lk}{}{21}{5--36}) verzeichnet ist. Die vornehmsten Umstände in dieser Weissagung sind:
\begin{aufzb}
\item Vor dem Untergange des israelitischen Staates werden noch viele Lügenpropheten und falsche Messiasse auftreten, welche das Volk durch Wunder täuschen, aber sein Elend nur vergrößern werden: \erganf{Es werden falsche Christus und falsche Propheten aufstehen, und große Zeichen und Wunder thun, also, daß auch die Auserwählten, wenn es möglich wäre, in Irrthum geführt würden. Sehet, ich habe es euch vorhergesagt; darum, wenn sie zu euch sagen werden: Sehet, er ist in der Wüste, so gehet nicht hinaus; sehet, er ist in dem Innersten des Hauses, so glaubet es nicht} (\RWbibel{Mt}{Matth.}{24}{24}).
\item Auch mehrere Kriege und Kriegsgerüchte, Verfolgungen der Christen, Hungersnoth, Pest, Erdbeben werden vorhergehen. \erganf{Ihr werdet auch von Schlachten und Kriegsgerüchten hören. Habet Acht, daß ihr euch nicht schrecken lasset; denn es muß dieß Alles geschehen. Aber~\RWSeitenw{174}\ dieß ist noch nicht das Ende; denn es wird ein Volk wider das andere, ein Königreich wider das andere aufstehen, und es werden Pest, Hunger und Erdbeben an verschiedenen Orten seyn. Dieß Alles aber ist nur ein Anfang des Elendes. Alsdann werden sie euch den Peinigern überliefern, euch tödten}, \usw\ (\RWbibel{Mt}{Matth.}{24}{6}).
\item Wie ein Adler über seine Beute, so wird das Kriegsheer der Feinde über Jerusalem herfallen, und diese Stadt mit einem Walle umzingeln. Wer sich um diese Zeit nicht durch die schnellste Flucht entfernt, für den wird keine Rettung mehr seyn. \erganf{Wo ein Aas ist, da sammeln sich die Adler. Bald nach der Trübsal jener Tage wird die Sonne verfinstert werden, der Mond keinen Schein mehr geben, und die Sterne werden vom Himmel fallen} (\RWbibel{Mt}{Matth.}{24}{28}). \erganf{Wenn ihr nun sehen werdet, daß jener Gräuel der Verwüstung, wovon der Prophet \RWbet{Daniel} geweissaget hat, in dem heiligen Orte sey (Wer dieses lieset, der verstehe es wohl!): dann sollen die, welche in dem Juden-Lande sind, auf die Berge fliehen, und wer auf dem Dache ist, der steige nicht herab, etwas aus seinem Hause zu holen, und wer auf dem Felde ist, der kehre nicht zurück, seinen Rock zu nehmen. Wehe aber den Schwangern und Säugenden zu jener Zeit! Bittet, daß eure Flucht nicht im Winter, oder am Sabbate geschehe} (\RWbibel{Mt}{Matth.}{24}{15}).
\item Das Elend der Belagerten zu Jerusalem wird eine furchtbare Höhe ersteigen, dergestalt, daß die Geschichte nichts Gräßlicheres kennt. \erganf{Es wird alsdann ein so großes Elend seyn, dergleichen vom Anfange der Welt bis auf diese Zeit nicht gewesen ist, und hinfort auch nicht mehr seyn wird. Und wenn dieselben Tage nicht wären abgekürzt worden: so würde kein Mensch davon kommen} (\RWbibel{Mt}{Matth.}{24}{21}).
\item Der Tempel wird dann zerstört werden, und kein Stein desselben auf dem andern bleiben. \erganf{Sehet ihr dieses Alles (nämlich das herrliche Tempelgebäude)? Wahrlich, ich versichere euch, kein Stein wird unverrückt auf dem andern bleiben!} (\RWbibel{Mt}{Matth.}{24}{2}).~\RWSeitenw{175}
\item Und dieses Alles wird noch in Erfüllung gehen innerhalb eines Menschenalters; diese hier Stehenden werden es erleben. \erganf{Wahrlich, ich sage es euch, dieß Geschlecht wird nicht vergehen, bis dieses Alles geschieht} (\RWbibel{Mt}{Matth.}{24}{34}\ \RWbibel{Mk}{Mark.}{13}{30}\ \RWbibel{Lk}{Luk.}{21}{32}).
\end{aufzb}
\item Daß dieses Alles auf das Genaueste in Erfüllung gegangen sey, ersehen wir aus der Erzählung mehrerer Profanschriftsteller, besonders des \RWbet{Flavius Josephus} (\RWlat{de bello judaico}\RWlit{}{Josephus4}).
\begin{aufzb}
\item Daß sich \RWbet{vor} Christi Zeiten irgend Jemand für den Messias ausgegeben hätte, davon weiß die Geschichte nichts; allein \RWbet{nach} Christo sehen wir eine Menge Betrüger auftreten, welche sich theils für Propheten, theils für Messiasse ausgeben, Wunder vorspiegeln, das Volk aufzuwiegeln versuchen, und zuletzt nur Ursache seines Unglückes werden. \RWbet{Josephus} gibt die Erscheinung dieser Lügenpropheten als eine Hauptursache jener Verblendung an, welche die Juden bethörte, sich gegen die Römer zu empören. Gleich im Jahre 45 verleitete ein gewisser Betrüger \RWbet{Theudas} eine große Menge Volkes, ihm bis an den Jordan zu folgen, wo er sie trockenen Fußes durchzuführen versprach. Um das Jahr 55 gab es eine Menge solcher Betrüger, welche das Volk bewogen, ihnen aus der Stadt in eine Wüste zu folgen. (Vgl.\ \RWbibel{Mt}{Matth.}{24}{25}) -- Sogar als der Tempel schon brannte, folgten 6000 bethörte Menschen einem falschen Propheten, und bestiegen einen Gang bei dem Tempel, wo der Betrüger ihnen wunderbare Rettung versprach, aber mit ihnen zugleich verbrannte. (Vgl.\ \RWbibel{Mt}{Matth.}{24}{26})
\item Dem letzten Kriege, welcher sich mit Jerusalems Zerstörung endigte, gingen erst mehrere andere Kriege und blutige Auftritte vorher. Durch die Erpressungen und Grausamkeiten der römischen Statthalter \RWbet{Albinus} und \RWbet{Florus} wurde die jüdische Nation zu einem Aufstande beinahe gezwungen. Zu Cäsarea baute ein Grieche hart an der Synagoge ein Haus hin, so daß der Zugang zu jener beinahe versperrt war. Man bot dem \RWbet{Florus} Geld, um diesen Bau einzustellen; er versprach es, ent\RWSeitenw{176}fernte sich aber aus der Stadt. Am folgenden Tage, der ein Sabbat war, opferte ein Heide vor dem Eingange der Synagoge Vögel. Darüber kam es zu unruhigen Auftritten; man schickte Abgeordnete an \RWbet{Florus}, welcher sie aber in das Gefängniß warf. Darüber wurde man auch zu Jerusalem unruhig. \RWbet{Florus}, der einen Aufruhr wünschte, schickte nun nach Jerusalem, und begehrte 17 Talente aus dem Tempel für den Kaiser. Durch dieses Benehmen entstand ein Aufruhr, der über 3000 Juden das Leben kostete. Doch König \RWbet{Agrippa} hatte die Juden größtentheils wieder besänftigt, als ein unbehutsames Wort den Unwillen der Menge dergestalt erweckte, daß man mit Steinen nach ihm warf. Es bildeten sich nun gewisse Rotten, die sich selbst Eiferer oder Zeloten nannten, haufenweise im Lande herumzogen, Städte und Dörfer der Syrer plünderten, \usw\ -- Daß auch die Christen verfolgt wurden, ist bekannt. Aber auch Hungersnoth, auch Pest und Erdbeben fanden sich ein, wie \RWbet{Jesus} vorausgesagt hatte. Unter dem Kaiser \RWbet{Claudius} gab es, wie uns Profanscribenten melden, vier Hungersnöthen, zwei zu Rom, Eine in Palästina (wovon auch \RWbibel{Apg}{Apostelg.}{11}{28}\ gesprochen wird) und Eine in Griechenland. So gab es auch eine Pest in Italien im Jahr 66. -- Auch der Erdbeben werden mehrere von den Geschichtschreibern gedacht; \zB\ eines in Campanien im Jahr 63, eines zu Jerusalem, als die Idumäer eben ihr Lager bei der Stadtmauer aufschlugen, \usw
\item Der Ausdruck \RWbet{Jesu: Wo ein Aas ist, versammeln sich die Adler}, paßt ganz vortrefflich auf die Römer, welche bekanntlich einen goldenen Adler zu ihrem Heereszeichen hatten. Nachdem die Juden die römische Besatzung im königlichen Schlosse zu Jerusalem ermordet hatten, brach der römische Feldherr \RWbet{Cestius} mit einem Heere wider sie auf, rückte ungehindert bis Jerusalem vor, hatte bereits zwei Stadtviertel erobert, und schon das königliche Schloß belagert, als \RWbet{Florus} ihn durch einige bestochene Officiere bereden ließ, die Belagerung wieder aufzuheben, weil er den Krieg verlängert zu sehen wünschte. Die Bessergesinnten hatten dem \RWbet{Cestius} bereits die Thore~\RWSeitenw{177}\ öffnen wollen; er aber, ihnen nicht trauend, zog wider Aller Erwartung zurück. Jetzt bekamen die Zeloten neuen Muth, verfolgten das römische Heer, und erlegten nahe an 6000 Mann. Dieß war der günstige Zeitpunct, in dessen Benützung die Christen sich, nach dem Rathe \RWbet{Jesu}, von Jerusalem flüchten konnten, und wirklich nach Pella flüchteten. Die Juden bereiteten sich mittlerweile zu einem förmlichen Kriege; \RWbet{Josephus} brachte ein Heer von 10,000 Mann zusammen; doch Kaiser \RWbet{Nero} schickte den Feldherrn \RWbet{Vespasian} wider sie ab; er siegte, und nahm den \RWbet{Josephus} gefangen. Als \RWbet{Vespasianus} hierauf, da \RWbet{Nero} sich entleibet hatte, von seinen Soldaten zum Kaiser ausgerufen wurde, überließ er die Vollendung des Krieges seinem Sohne \RWbet{Titus}. Dieser menschenfreundliche Fürst bot, nach der Erzählung \RWbet{Josephi} (der Talmud schildert die Sache anders), Alles Erdenkliche auf, die Juden durch gütliche Vorstellungen zum Gehorsam zu bringen, um ihre Stadt, und ihren prächtigen Tempel zu retten. Er sandte ihren eigenen Landsmann \RWbet{Josephus} an sie, um sie zur Uebergabe zu bereden; aber vergeblich. Endlich beschloß er, die Stadt mit einer Mauer zu umringen, und so die Juden auszuhungern.
\item Da es gerade um die Osterzeit war, so befand sich damals eine sehr große Menge Menschen zu Jerusalem; die beabsichtigte Hungersnoth mußte also nur zu bald eintreten, und stieg wirklich zu einer furchtbaren Höhe. Die ungenießbarsten Gegenstände (Leder \udgl ) wurden von den Unglücklichen verschluckt; es starben der Menschen so viele, daß man sie gar nicht mehr beerdigen konnte. Daher gesellte sich auch noch die Pest zu diesem Elende, welches (nach dem eigenen Ausdrucke \RWbet{Josephi}) nie eine andere Stadt in einem solchen Maße erfahren. In einer Nacht wurden gegen 2000 Juden von den römischen Soldaten ermordet, und ihre Eingeweide durchwühlt, in der Hoffnung, verschlucktes Gold darin zu finden! Ein vornehmes Frauenzimmer, Namens \RWbet{Maria}, verzehrte aus Hunger ihr eigenes Kind, \usw\ Als \RWbet{Titus} von diesem Gräuel hörte, schwor er, das Andenken desselben unter dem Schutte dieser Stadt zu begraben.~\RWSeitenw{178}
\item Als nun die Stadt bereits mit Sturm eingenommen war, wollte der menschenfreundliche Feldherr doch noch des Tempels geschont wissen; aber ein römischer Soldat warf einen Brand hinein, wodurch er in Feuer gerieth. So schnell nun auch \RWbet{Titus} herbeieilte, um zu löschen; so war es doch schon zu spät; das herrliche Gebäude ging in Rauch auf. Jerusalem ward so zerstört, daß, wie \RWbet{Josephus} sich ausdrückt, kaum mehr ein Merkmal ihrer ehemaligen Bewohnung übrig blieb!
\item Dieses geschah den 8ten September des Jahres 70 nach der gemeinen christlichen Zeitrechnung, also ohngefähr 37 Jahre nach dem Tode Jesu.
\end{aufzb}
\item Die drei Evangelien \RWbet{Matthäi, Marci} und \RWbet{Lucä} sind allen Berichten zufolge noch \RWbet{vor dem Jahre 70} (das Evangelium \RWbet{Matthäi} etwa im Jahre 40) geschrieben, also gewiß noch vor der Erfüllung jener Weissagungen. Auch schon der Inhalt jener Weissagungen selbst beweiset, daß sie nicht etwa erst nach der Zerstörung Jerusalems ausgedacht worden sind; denn in diesem Falle würden sie sicher noch weit bestimmter lauten, und es würden dann nicht so manche andere Dinge, sogar die Wiederkunft \RWbet{Jesu} zu einem allgemeinen Weltgerichte mit dem Untergange des jüdischen Staates vermengt worden seyn. \RWbet{Johannes}, der Einzige, der sein Evangelium erst \RWbet{nach} Zerstörung Jerusalems schrieb, ist auch der Einzige, der diese Weissagung mit Stillschweigen übergehet. Setzen wir aber voraus, daß jene Reden \RWbet{Jesu} wirklich schon früher niedergeschrieben waren; so ist das Wunderbare in dieser Sache durchaus nicht zu verkennen. Denn unmöglich hatte ein Mensch dieß Alles durch bloße Vernunft voraus wissen können. Oder von wie viel Zufällen hing es nicht ab! wie leicht und überaus leicht hätte die Sache nicht einen ganz anderen Verlauf nehmen können! Hätten \RWbet{Albinus} und \RWbet{Florus} die Juden nicht so gereizt: sie würden sich vielleicht gar nicht empört haben. Hätte \RWbet{Cestius} schon dem Kriege ein Ende gemacht: so wären mehrere Theile der Weissagung unerfüllt geblieben. Wäre \RWbet{Titus} um ein Paar Stunden eher gekommen: so war der Tempel gerettet. -- Wer hat auch selbst nur dieß Eine mit~\RWSeitenw{179}\ so viel Bestimmtheit vorauswissen können, daß künftig so manche falsche Messiasse auftreten würden, da früher noch keine erschienen? daß sie bald in der Wüste, bald in dem Tempel sich aufhalten würden? \usw\
\item Doch der Mund \RWbet{Jesu} sprach noch eine sehr merkwürdige Weissagung in Betreff des Volkes Israel aus. In einem Zustande der Unterdrückung soll dieses unglückliche Volk so lange fortdauern, bis alle übrigen Völker zum Christenthume übergehen, worauf es dann gleichfalls übertreten werde. Bei seinem letzten Tempelbesuche sprach \RWbet{Jesus} (\RWbibel{Mt}{Matth.}{23}{38}): \anf{Sieh, deßhalb soll euch eure Wohnung öde gelassen werden.} Ferner sagt \RWbet{Jesus} von den Juden (\RWbibel{Lk}{Luk.}{21}{24}): \erganf{Theils werden sie durch das Schlachtschwert fallen, theils als Gefangene unter allerlei Völker fortgeschleppt werden, -- bis auch die Zeit der Heiden wird zu Ende seyn.} Die Juden werden also zerstreut unter der Botmäßigkeit der Völker bleiben, bis daß die Zeit der Heiden ein Ende nehmen wird. Was dieser letzte Ausdruck bedeute, ersehen wir noch deutlicher aus \RWbet{Paulus} (\RWbibel{Röm}{Röm.}{11}{25}): \erganf{Ich will euch, was sonst verborgen ist, nicht verhehlen, daß nämlich die Verblendung eines Theils der Israeliten so lange dauern wird, bis die Gesammtzahl der Heiden (zum Christenthume) wird übertreten seyn.} -- Bis also alle Heiden zum Christenthume übertreten seyn werden, dann wird auch noch der Ueberrest der Juden bekehrt werden.
\item Bekanntlich sind die Juden unter allen Völkern zerstreut. In \RWbet{Asien}, namentlich in \RWbet{Persien} und \RWbet{Sina}, sind sie am Zahlreichsten vorhanden, dann in \RWbet{Arabien, Ostindien} und in der \RWbet{asiatischen Türkei}; sehr zahlreich sind sie auch in \RWbet{Afrika}, besonders in \RWbet{Aegypten}, in \RWbet{Aethiopien, Abyssinien}, und in der \RWbet{Tartarei}. In Europa sind sie am Häufigsten in dem Gebiete der \RWbet{Türken}, in den vereinigten \RWbet{Niederlanden} und in \RWbet{Polen}; aber es ist kein Land, wo nicht Einige wohneten. -- Selbst in \RWbet{Amerika}, \zB\ in \RWbet{Pensylvanien} gibt es schon Juden. Wenn man Arabien ausnimmt, in welchem Lande die Juden während des sechsten Jahrhunderts sich in einem ziemlich blühenden Zustande befanden, auch ihre eigenen Fürsten und~\RWSeitenw{180}\ etliche angesehene Akademien hatten: so lebten sie sonst überall in einem Zustande der Unterjochung, und wurden von Heiden, Türken und Christen immer sehr grausam behandelt.
\item Es ist einleuchtend, daß Niemand durch seine bloße Vernunft diese Schicksale der Juden habe voraussehen können. Wie leicht hätte es bei den vielen Verfolgungen, welche die Juden erfuhren, \zB\ schon unter den heidnischen Kaisern \RWbet{Vespasian, Domitian, Trajan, Hadrian} \uA\ geschehen können, daß sie ganz wären ausgerottet worden! Wie viele andere alte Nationen sind nicht bis auf die letzte Spur verschwunden! Wie leicht hätten die wenigen Juden, welche aus diesen Verfolgungen noch ihr Leben retteten, nicht zu dem Christenthume übertreten können, in welchem Falle jene Vorhersagung, daß dieß erst nach dem Ende der heidnischen Zeiten geschehen soll, unerfüllt geblieben wäre! -- Wenn man dieß Alles bedenkt: so wird man in dieser Vorhersagung, die sich nun schon durch achtzehn Jahrhunderte bewährt hat, eine Weissagung erkennen, und die vor unserm Auge umherwandelnde Nation der Juden mit einem noch größeren Rechte, als es der heil.\ \RWbet{Augustinus} schon im fünften Jahrhunderte that, ein \RWlat{argumentum ambulatorium veritatis} nennen.
\end{aufza}

\RWpar{64}{III.~Merkwürdiges Ereigniß bei der versuchten Wiedererbauung des jüdischen Tempels}
\begin{aufza}
\item Im Jahre 363 ertheilte der abtrünnige Kaiser \RWbet{Julian} den Juden die Erlaubniß, den Tempel neuerdings aufzubauen, und ihren vormaligen Gottesdienst wieder herzustellen. Der Kaiser that dieß den Christen zum Trotz, zumal da Einige aus ihnen, \zB\ \RWbet{Origenes} gelehrt hatten, daß, der Weissagung \RWbet{Jesu} zu Folge, die Juden nimmermehr ihr Land wieder bekommen, und daß der Tempel nie wieder aufgebaut werden sollte. Allein dieß boshafte Vorhaben wurde, nachdem man die noch vorhandenen Ruinen des vorigen Tempels weggeräumt hatte, auf eine sehr merkwürdige Art vereitelt. Denn nicht nur so viele \RWbet{christliche Schriftsteller} jener Zeit, \zB\ \RWbet{Gregor von Nazianz, Ambrosius, Chrysostomus, Sokrates, Sozomenus} \uA ; sondern selbst~\RWSeitenw{181}\ \RWbet{heidnische} und \RWbet{jüdische Geschichtschreiber}, insonderheit \RWbet{Ammianus Marcellinus}, des Kaisers \RWbet{Julian's} Freund und Lebensbeschreiber, ja dieser \RWbet{Kaiser selbst} erzählen uns, daß der angefangene Bau durch heftige Erdbeben, und durch den Ausbruch eines gewissen unterirdischen Feuers (feuerige Kugeln sagen sie) verhindert worden sey, ja daß hiebei viele Arbeiter um's Leben gekommen wären.
\item Nach \RWbet{Michaelis} Erklärung dürfte dieß unterirdische Feuer eine Art brennbarer Luft gewesen seyn, welche beim Aufgraben aus den in den Bergen Sion und Moria befindlichen unterirdischen Gängen und Gewölben hervorbrach, und sich entzündete. Aber auch sonach gehört das Ereigniß gewiß zu den seltenen Erscheinungen; und daß es sich gerade jetzt zutrug, daß der Kaiser sich dadurch genöthiget sah, jenen den Christen zum Trotz unternommenen Bau wieder aufzugeben, daß er durch seine Unternehmung nur eben dazu behülflich seyn mußte, daß die Weissagung Jesu: \RWbet{Kein Stein soll auf dem andern bleiben!} nun erst \RWbet{buchstäblich} erfüllt ward: dieß Alles zusammengenommen läßt uns den Finger Gottes in diesem Ereignisse nicht verkennen.
\end{aufza}

\RWpar{65}{IV.~Einige Wunder, welche uns die Apostelgeschichte erzählet}
Wenn wir erwägen, daß der Verfasser der Apostelgeschichte eine gelehrte Bildung genossen, und vornehmlich auf Heilkunde sich eigens verlegt habe; daß ihm in der Erzählung, die er in diesem Buche liefert, durch keine von einer anderen Hand bereits entworfene Darstellung eine Art von Zwang aufgelegt worden sey, die Sache nur eben aus diesem und keinem anderen Gesichtspuncte zu betrachten; daß er bei einem guten Theile der hier erzählten Ereignisse sogar als Augenzeuge zugegen gewesen; daß endlich der Schauplatz der uns von ihm benachrichtigten Wunder nicht irgend ein einzelner Ort im wundersüchtigen Judäa war, sondern daß diese Wunder sich in den verschiedensten Ländern und Städten zugetragen haben: so werden wir sehr geneigt, seiner Geschichte ein ganz besonderes Zutrauen zu schenken. Es mö\RWSeitenw{182}gen nun also die Wunder, welche uns die Apostelgeschichte erzählt, hier erst den Weg zu jenen des Evangeliums bahnen.
\begin{aufza}
\item In der Apostelgeschichte nun wird (\RWbibel{Apg}{}{2}{1\,ff}) erzählt, an welchem Tage die Bekenner des Christenthums zum ersten Male als eine eigene, für sich bestehende Gesellschaft auftraten, und zugleich die öffentliche Verkündigung des Evangeliums begannen. Durch ein überaus schickliches Wunder ward dieses wichtige Ereigniß eingeleitet. Am ersten Pfingsttage nämlich waren die Apostel und die übrigen Jünger des Herrn (ohngefähr 120 Personen) in einem und eben demselben Gebäude, (wahrscheinlich in einer der mehreren Tempelabtheilungen) versammelt, als plötzlich ein Brausen, wie das eines nahenden Sturmwindes, entstand, und das ganze Haus, darin sie versammelt waren, erfüllte (da entstand plötzlich ein Sausen, welches gleich einem gewaltigen Winde vom Himmel kam, und das ganze Haus erfüllte). Auch zeigte sich eine glänzende Lufterscheinung, vergleichbar mit Feuerzungen, die über den Häuptern der einzelnen schwebten; (es erschienen ihnen feurige Zungen, die sich theilten, und auf einen Jeden unter ihnen setzten). Durch diese Ereignisse wurde eine beträchtliche Menge von Menschen, Juden und Proselyten, welche des Festes wegen so eben aus allen Gegenden des römischen Reiches in Jerusalems Mauern versammelt waren, herbeigezogen. Die Jünger des Herrn aber wurden durch dieses Alles in der Art ermuthiget und begeistert, daß sie sofort begannen, Lieder zur Ehre Gottes in den verschiedensten Sprachen zu singen. (Und sie wurden Alle mit dem heiligen Geiste erfüllet, und fingen an, unterschiedliche Sprachen zu reden, wie ihnen der heil.\ Geist zu reden eingab.) Die Zuhörer staunten, daß sie die Großthaten Gottes verkündigen hören, ein Jeder in seiner eigenen Landessprache (da ein Jeder seine Sprache reden hörte). Einige erklärten spottweise (\RWgriech{[dia]qleu'azontes}), sie müßten trunken seyn. Da stand nun \RWbet{Petrus} auf, und hielt eine Rede, in der er die Wahrheit von \RWbet{Jesu} Auferstehung und seiner göttlichen Sendung mit einem solchen Nachdrucke vortrug, daß an demselben Tage noch gegen 3000 Menschen das Christenthum annahmen.
\item[]Man hat erinnert, daß jenes Reden in fremden Sprachen ganz gegen die Absicht des heil.\ Geschichtschreibers so~\RWSeitenw{183}\ ausgelegt werde, als ob den ersten Christen die Fähigkeit beigewohnt hätte, in einer jeden beliebigen Sprache zu reden, da doch selbst \RWbet{Petrus} sich genöthiget gesehen, einen eigenen Dolmetscher in der Person des \RWbet{Markus} zu halten. Das Ungewöhnliche ihrer Unternehmung am ersten Pfingsttage sey bloß darin bestanden, daß sie gewagt, sich \RWbet{fremder}, \dh\ \RWbet{barbarischer}, und von dem öffentlichen Gottesdienste bisher ausgeschlossener Sprachen zu bedienen, um das Lob Gottes darin zu verkündigen; und gewundert habe man sich bloß darüber, wie diese Galiläer (Leute, die für sehr unwissend galten) es doch erlernt hätten, der Eine in dieser, der Andere in jener Sprache einen erklecklichen Vortrag zu halten. Gesetzt dieß wäre so: dann wäre die Gabe der Sprachen freilich kein Wunder in dieses Wortes eigentlichem Sinne zu nennen; aber die außerordentlichen Naturerscheinungen, welche an jenem Tage so ganz zur rechten Zeit und an dem rechten Orte eintraten, und einen so erwünschten Eindruck auf Beide, auf die versammelte Christenmenge sowohl, als auch auf die herbeigekommenen Zuhörer machte, diese sind doch ein unläugbares Wunder, das zur Verbreitung des Christenthumes diente.
\item In derselben Apostelgeschichte wird (\RWbibel{Apg}{}{3}{1\,ff}) erzählt, daß \RWbet{Petrus} und \RWbet{Johannes} in den Tempel gingen, als eben ein \RWbet{Lahmgeborner}, der an der schönen Pforte täglich zu sitzen pflegte, um die Vorübergehenden um ein Almosen anzusprechen, vorübergetragen wurde. Er sprach auch die Apostel an; da erwiederte \RWbet{Petrus}: \erganf{Gold und Silber habe ich nicht; was ich aber habe, das will ich dir geben. Im Namen \RWbet{Jesu Christi} von Nazareth stehe auf und wandle! So sprechend faßte er ihn bei der rechten Hand, half ihm, und alsbald hatten die Füße des Lahmen eine gehörige Festigkeit erhalten, und er sprang auf, er konnte stehen, und auf- und abgehen; ging in den Tempel, sprang freudig umher, und lobete Gott.}
\item[]Wer könnte hier einen Betrug argwöhnen? wer müßte nicht vielmehr die Würde, mit der sich der Apostel bei seiner ersten That schon benimmt, bewundern? Und da weiter erzählt wird, daß beide Apostel dieser That wegen vor das~\RWSeitenw{184}\ Synedrium gezogen wurden, so fällt auch jeder Verdacht einer Erdichtung weg.
\item Von dem Zauberer (Magier) \RWbet{Simon} erzählt \RWbet{Lukas} (\RWbibel{Apg}{Apostelg.}{8}{5}), daß er erstaunet, und zum Christenthume übergetreten sey, als er die Wunder sah, welche \RWbet{Philippus} (der Diakon vermuthlich) zu Samaria wirkte. Als in der Folge einige Apostel (Petrus und Johannes) selbst nach Samaria kamen, und den bereits getauften Christen durch die Auflegung ihrer Hände den heil.\ Geist mittheilten, bot ihnen Simon Geld dafür an, daß sie auch ihm die Macht, durch Händeauflegung Andern die Gaben des heiligen Geistes zu verschaffen, mittheilen möchten. Sie aber verwiesen ihm diese niedrige Handlung auf's Schärfste. -- Liegt nun in diesem Benehmen \RWbet{Simon's} des Zauberers nicht ein ganz unwidersprechlicher Beweis, daß die Apostel und ersten Christen überhaupt Thaten von einer ganz anderen Art, ungleich bewunderungswürdigere Thaten nämlich, ausgeführt haben mußten, als alle Kunststücke der Zauberer waren?
\item Auch die \RWbet{Bekehrungsgeschichte Pauli} finden wir in der Apostelgeschichte (\RWbibel{Apg}{}{9}{1\,ff}). Dieser damals noch junge Mann, der zuerst \RWbet{Saulus} hieß, aus Tarsus in Cilicien gebürtig, ein Schüler des weisen \RWbet{Gamaliel's}, hatte die Christen Anfangs mit vieler Heftigkeit verfolgt. Bei der Steinigung \RWbet{Stephan's} (\RWbibel{Apg}{Apostelg.}{7}{58}) hatte er die Kleider derjenigen, welche den ersten Blutzeugen des Christenthums steinigten, bewacht. Jetzt befand er sich eben auf einer Reise nach \RWbet{Damaskus}, um die daselbst befindlichen Christen gefänglich nach Jerusalem zu bringen. \erganf{Als er der Stadt schon nahe war, umstrahlte ihn plötzlich ein Licht vom Himmel; und als er darüber zur Erde stürzte, vernahm er eine Stimme, die zu ihm sprach (\RWbibel{Apg}{Apostelg.}{22}{6}\ wird noch hinzugesetzt: in hebräischer Sprache): \RWbet{Saul, Saul!} warum verfolgst du mich? Er erwiederte: Herr! wer bist du? Die Stimme sprach: Ich bin \RWbet{Jesus}, den zu verfolgest; allein es wird dir schwer werden, wider die Stachel auszuschlagen. Da erwiederte \RWbet{Saulus} mit Zittern: Herr! was verlangst du, daß ich thun soll? Und der Herr sprach: Stehe auf, und gehe in die Stadt, dort wird dir gesagt~\RWSeitenw{185}\ werden, was du thun sollst. Die Reisegefährten standen bestürzt da; denn sie hörten zwar den Schall, sahen aber Niemanden (\RWbibel{Apg}{Apostelg.}{22}{9}\ heißt es: sie hätten das Licht zwar gesehen, aber die Stimme nicht verstanden). \RWbet{Saulus} stand nun von der Erde auf, konnte aber, als er die Augen aufschlug, nichts sehen. Sie führten ihn an der Hand nach Damaskus, wo er drei Tage lang blind blieb, und weder Speise noch Nahrung zu sich nahm. In einer Erscheinung kam es ihm vor, als ob ein Mann, Namens \RWbet{Ananias}, durch Auflegung seiner Hände ihm das Gesicht wieder gebe, und so geschah es wirklich. Es fiel wie Schuppen von seinen Augen}; \usw\ Eben dieses Ereignisses gedenket auch \RWbet{Paulus} selbst (\RWbibel{1\,Kor}{1\,Kor.}{15}{8}) mit den Worten: \erganf{Zuletzt unter Allen ist er (nämlich Jesus) auch mir erschienen, den er hiedurch, gleich einer unzeitigen Geburt, gewaltsam zu einem besseren Leben erweckte. Denn ich bin der Geringste unter den Aposteln, ja nicht einmal werth, ein Apostel zu heißen, da ich die Kirche Gottes selbst verfolget habe.} (Eben so \RWbibel{Gal}{Gal.}{1}{11\,ff}) Da also \RWbet{Paulus} in seinen eigenen Schriften selbst zu verstehen gibt, daß etwas Außerordentliches seine Bekehrung bewirkt habe, und daß ihm eigentlich \RWbet{Jesus} erschienen sey: so müssen wir wohl annehmen, daß wirklich etwas von der Art vorgefallen sey, wie uns in den so eben angeführten Stellen der Apostelgeschichte erzählt wird. Diese Erscheinung kann aber keine durch Menschen veranstaltete Täuschung gewesen seyn. \RWbet{Paulus} war nicht so thöricht, daß er so leicht sich hätte täuschen lassen. Das Geringste also, was man hier annehmen muß, ist eine ungewöhnliche Naturerscheinung mit Blitz und Donner verbunden. Daß sich nun eine solche Erscheinung gerade in dem Augenblicke einstellte, und in dem Gemüthe \RWbet{Pauli} gerade diese so wohlthätige Veränderung bewirkte, dieß Alles deutet auf eine absichtliche Verfügung Gottes; und die Begebenheit ist ein Wunder.
\item Nach der Apostelgeschichte (\RWbibel{Apg}{}{9}{32\,ff}) kommt \RWbet{Petrus} auf seinen Reisen auch nach \RWbet{Lydda} (einem Flecken in Judäa), und findet hier \erganf{einen Mann, Namens \RWbet{Aeneas}, der bereits acht Jahre an der Gicht darniedergelegen. Und Petrus sprach zu ihm: Aeneas! Jesus Christus gibt dir die Gesundheit.~\RWSeitenw{186} Stehe auf, und mache dir selbst dein Bett zurecht. Sogleich stand er auf (heißt es weiter), und alle Einwohner von Lydda und Saron (welches unweit von Lydda liegt) sahen ihn und bekehrten sich zu dem Herrn.}
\item Von hier wird \RWbet{Petrus} (\RWbibel{Apg}{Apostelg.}{9}{36\,ff}) nach \RWbet{Joppe} gebeten, wo er eine so eben gestorbene Christin mit Namen \RWbet{Tabitha} oder \RWbet{Dorkas}, die eine sehr wohlthätige Person gewesen, vom Tode auferweckt.
\item Nachdem \RWbet{Paulus} aus Ikonien (einer Stadt in Kleinasien), wo er viel Wunder gewirket (\RWbibel{Apg}{Apostelg.}{14}{3}), vertrieben war, begab er sich nach \RWbet{Lystra} (einer Stadt in Lykaonien). Dort \erganf{war ein Mann, der, unvermögend seine Füße zu gebrauchen, noch nie hatte gehen können; denn er war von Geburt aus lahm. Er horchte auf die Rede Pauli; dieser blickte ihn an, und da er erkannte, daß er den Glauben habe, es könne ihm geholfen werden, rief er laut: Stelle dich aufrecht auf deine Füße! Er sprang auf, und ging umher.}
\item Nach der Apostelgeschichte (\RWbibel{Apg}{}{19}{1\,ff}) hatte \RWbet{Paulus} in der Stadt \RWbet{Ephesus} der Wunder besonders viele gewirkt. Die Epheser werden hievon so gerührt, daß Viele aus ihnen, die sich bisher mit Zaubereien abgegeben hatten, nun ihre Zauberbücher alle herbeibringen, und verbrennen. Ein Ereigniß, welches die schon (Nr.\,3.) gemachte Bemerkung noch mehr bestätiget.
\item Als \RWbet{Paulus} zu \RWbet{Troas} (an der Küste von Kleinasien) sich aufhielt, und, weil er Tages darauf schon abzureisen gesonnen war, seinen Unterricht bis Mitternacht fortsetzte, versank ein Jüngling, Namens \RWbet{Eutyches}, der sich an ein Fenster gesetzt hatte, in Schlaf, und stürzte so drei Stockwerke hoch hinunter. Paulus eilt sogleich hinab, und bringt den Jüngling wieder zum Leben. (\RWbibel{Apg}{Apostelg.}{20}{1\,ff}) Man sage immerhin, der Jüngling sey (wie sein Name bedeutet) so glücklich in seinem Falle gewesen, daß er sich wesentlich nicht verletzte, sondern nur etwas betäubt vom Schrecken war: gerade darin, daß er so glücklich fiel, liegt das Wunder; denn wäre es dem Apostel nicht gelungen, ihn wie\RWSeitenw{187}der zum Leben zu bringen: so hätten gewiß sehr viele Menschen einen Anstoß daran genommen.
\item Als \RWbet{Paulus} vom römischen Landpfleger \RWbet{Festus} in Cäsarea gefänglich nach Rom geschickt wurde, um sich dort vor dem Kaiser zu verantworten, kam man zu Schiffe bis nach \RWbet{Schönhafen} auf der Insel Kreta. Weil die Fahrt schon bis dahin gefährlich gewesen, und, da die Wintermonate hereinbrachen, noch weit gefährlicher zu werden drohte: so rieth \RWbet{Paulus} dem Schiffshauptmann, sie möchten hier überwintern. Denn, sagte er, ich sehe voraus, daß die weitere Fahrt sehr gefährlich nicht allein für die Ladung und das Schiff, sondern auch für uns Alle ablaufen werde. -- Allein man folgte seinem Rathe nicht, weil dieser Hafen zum Ueberwintern unbequem war, und wollte, wo möglich, nur noch bis \RWbet{Phönice} (einem andern Seehafen auf derselben Insel) kommen. Als man ausfuhr, hatte man günstigen Wind (gelinden Südwind); aber nicht lange, so erhob sich ein stürmischer Nordostwind, der das Schiff mit sich fortriß. Um es vor dem Zertrümmern zu bewahren, umwand man dasselbe mit Ketten und Tauwerken, kappte den Mast, und überließ es den Wellen. Am folgenden Tage sah man sich, weil der Sturm noch immer nicht nachließ, genöthiget, einen Theil der Ladung über Bord zu werfen; am dritten Tage mußte man dieß auch mit den Schiffsgeräthschaften thun, und gab schon alle Hoffnung zur Rettung auf, so zwar, daß man in der Bestürzung nicht einmal Speise und Trank zu sich nehmen wollte. Da trat nun Paulus auf, ermahnte die Verzagten, Muth zu fassen, weil zwar das Schiff zu Grunde gehen, aber kein Einziger aus der Gesellschaft das Leben verlieren würde, wie er aus der Erscheinung eines Engels wisse. In der Nacht glaubte man endlich, daß sich Land nähere. Unter dem Vorwande, einen Anker auszuwerfen, setzten einige Matrosen das Boot aus, um darauf zu entfliehen. Paulus bemerkte dieß, und zeigte es dem Hauptmanne an, mit dem Beisatze, daß, wenn diese nicht im Schiffe bleiben, sie nicht gerettet werden könnten. Sogleich hieb man die Seile des Bootes entzwei, und ließ es fahren. Hierauf ermahnte \erganf{Paulus} Alle, Speise zu sich zu nehmen, und guten Muthes zu seyn, weil kein Einziger das Leben verlieren würde. Als es~\RWSeitenw{188}\ Tag geworden war, sah man, daß wirklich Land da sey, aber ein unbekanntes. Indem man zu landen versuchte, stieß man an eine Sandbank, auf welcher das Schiff scheiterte. Nun blieb nichts Anderes übrig, als sich durch Schwimmen an das Land zu retten. Da verfielen die Soldaten auf den grausamen Gedanken, die Gefangenen, die man im Schiffe hatte, zu ermorden, damit sie beim Schwimmen nicht etwa entfliehen möchten. Doch der Hauptmann verhinderte es, und so kamen denn wirklich von 276 Seelen Alle glücklich an's Land, an die Insel Malta (Melita). Die Bewohner nahmen sie freundschaftlich auf, und zündeten ein Feuer an, damit man sich trockne. Als nun auch Paulus einen Haufen dürrer Reiser zusammenrafft, um sie dem Feuer zuzuwerfen, springt eine Schlange hervor, und umschlingt seinen Arm. Doch er schleudert das Thier in das Feuer, und erfährt keinen Nachtheil. Darüber erstaunen die Inselbewohner so sehr, daß sie den Mann für ein höheres Wesen erklären. Der Befehlshaber der Insel nimmt ihn nun auf sein Landgut, und Paulus macht seinen an einer Ruhr darniederliegenden Vater durch bloße Auflegung seiner Hände gesund, worauf er dann auch eine Menge anderer Kranken, die man herbeibringt heilet. (\RWbibel[Apostelg.\ 27.\ u.\ 28.\ Kap.]{Apg}{}{27--28}{}) Bei diesem Ereignisse ist der Erzähler selbst Augenzeuge gewesen; es sind Begebenheiten, die sich vor vielen Zuschauern zutrugen, und von so großer Wichtigkeit für so viele Andere waren, daß eine Erdichtung derselben unmöglich ist; es sind im Grunde auch Ereignisse, die um so leichter geglaubt werden können, je weniger etwas uns durchaus Unbegreifliches in ihnen vorkommt. Und bei allem dem ist doch das Wunderbare in diesen Begebenheiten durchaus nicht zu verkennen. Denn sage man immerhin, ein bloßer Zufall habe hier jedesmal geholfen: daß sich dergleichen Zufälle \erganf{jedesmal} einfanden, das eben ist das Werk Gottes, das unverkennbare Wunder.
\item[\RWbet{Einwurf.}] Nicht alle in der Apostelgeschichte erzählten Wunder haben das Gepräge der Gotteswürdigkeit. Ist es \zB\ nicht läppisch und unanständig, wenn Petri \RWbet{Schatten} und Pauli \RWbet{Schweißtücher} Kranke gesund machen sollen? Ist es nicht unschicklich, wenn \RWbet{Paulus} über den Jüngling~\RWSeitenw{189}\ von \RWbet{Troas}, den er zum Leben bringen will, erst sich \RWbet{hinlegen} muß? Und geziemt es sich, daß \RWbet{Petrus} und \RWbet{Paulus} mehr als einmal den Händen ihrer Obrigkeit durch eine \RWbet{angeblich wunderbare} Befreiung \RWbet{entfliehen}? (\RWbibel{Apg}{Apstg.}{9}{23}\ \RWbibel{Apg}{}{12}{3}\ \RWbibel{Apg}{}{16}{25}) -- War das, was Sokrates in einem ähnlichen Falle that, nicht ungleich edler?
\item[\RWbet{Antwort.}] In der Beurtheilung, ob etwas anständig oder unanständig sey, kann man sich sehr leicht übereilen, wenn man die Sitten und den Geschmack der Menschen, vor deren Augen es geschieht, nicht kennet; denn eben diese in verschiedenen Ländern und zu verschiedenen Zeiten so sehr verschiedenen Dinge bestimmen, ob etwas anständig oder unanständig sey. Wenn die Begriffe des Volkes damals nichts Unanständiges darin fanden, daß man durch seinen bloßen Schatten, oder durch gewisse Kleidungsstücke heile; wenn sie von einem Wunderthäter vielmehr erwarteten, daß er auch dieses vermöge; wenn sie in dieser Zuversicht bereits Versuche anstellten, Kranke herbeitrugen \usw : dann war es nothwendig, daß ihre Erwartung nicht getäuscht werde, oder die Apostel des Herrn hätten jenes so volle Vertrauen beim Volke, das sie bisher besaßen, und das zur weiteren Verbreitung des Christenthums so nothwendig war, unwiederbringlich verloren. -- Mag immer die Art, wie Paulus den Jüngling von Troas erweckte, in unsern Augen für minder anständig gehalten werden: genug, wenn sie zu jener Zeit in Jedermanns Augen den höchsten Grad von Schicklichkeit hatte, und vielleicht mit zu den natürlichen Mitteln gehörte, durch welche die unterdrückte Lebenskraft in dem Jünglinge wieder erweckt werden mußte. \RWbet{Paulus} lehnte sich gar nicht gegen die rechtmäßige Gewalt seiner Obrigkeit auf, wenn er (\RWbibel{Apg}{Apostelg.}{9}{23}) nur dadurch aus Damaskus entkam, da er in einem Korbe über die Stadtmauer sich herabließ; denn noch hatte man ihm gar nicht den Willen, ihn gefänglich einzuziehen, von Seite der \RWbet{Obrigkeit} andeuten lassen; nur von \RWbet{Privatleuten} erfuhr er, daß man die Stadtthore bewache, um ihn einzuziehen. Da stand es ihm also, nach allen Grundsätzen einer gesunden Rechtslehre, frei, zu entfliehen. -- Der Apostel \RWbet{Petrus} ward (\RWbibel{Apg}{Apostelg.}{12}{3\,ff}), ohne im An\RWSeitenw{190}fange selbst zu wissen, ob er wache oder träume, von seinen Banden befreit und aus dem Kerker geführt; erst als er nahe vor seinem Hause war, erkannte er, daß ihn ein Wunder gerettet habe; und da er leicht hoffen konnte, daß auch die Obrigkeit dieß werde einsehen müssen: so war es keineswegs seine Pflicht, wieder zurückzukehren. -- Die Erzählung von der Befreiung Pauli (\RWbibel{Apg}{Apostelg.}{16}{25}) braucht man nur ganz zu lesen, um zu erkennen, daß sie vielmehr das Gegentheil von dem beweise, was unsere Gegner wollen. Auf eine ganz außerordentliche Art, welche sich weder Paulus, noch der Kerkermeister zu erklären wußten, wurden die Thüren des Gefängnisses geöffnet, und die Bande des Apostels und seines Gefährten \RWbet{Silas} gelöset. Gleichwohl entfernt sich Paulus noch nicht. Der Kerkermeister, als er die Thüren geöffnet sah, glaubt, auch die Gefangenen seyen entronnen, und will sich schon entleiben; da ruft ihm Paulus zu, er möge sich doch kein Leid anthun, weil Alle hier wären. Nun führt der gerührte Kerkermeister Paulum in seine Wohnung, läßt sich von ihm unterrichten, und noch in dieser Nacht taufen. Bei Tages Anbruch lassen die Vorsteher der Stadt (Philippi) dem Kerkermeister sagen, er habe Paulum loszulassen, (vermuthlich mochten sie von seiner wunderbaren Befreiung schon gehört haben.) Paulus erwiedert: Oeffentlich ließen sie uns, da wir doch römische Bürger sind, schlagen, und in's Gefängniß werfen, öffentlich mögen sie uns auch Gerechtigkeit widerfahren lassen. -- \erganf{Da kamen die Richter, baten um Vergebung, und ersuchten nur, daß er die Stadt verlassen möchte.}
\end{aufza}

\RWpar{66}{V.~Wunder Jesu}

\begin{center}\RWbet{A.~Allgemeine Berichte von den wunderthätigen Krankenheilungen Jesu.}\end{center}

Wenn es erwiesen ist, daß die Apostel Jesu verschiedene Wunder gewirkt: so steht im Voraus zu erwarten, daß auch \RWbet{Jesus selbst}, als eben diejenige Person, von welcher die Apostel die neue Lehre, welche sie verkündigten, empfangen hatten, durch mancherlei Wunder die Wahrheit seines Unterrichtes bewiesen haben werde. Und dieß erzählen uns die Evangelien wirklich.~\RWSeitenw{191}
\begin{aufza}
\item Bei Matth.\ (\RWbibel{Mt}{}{4}{23}) heißt es: Jesus \erganf{reisete in ganz Galiläa umher, lehrte in den Synagogen, predigte das Evangelium vom Reiche und heilte allerlei Krankheit und Gebrechen unter dem Volke. Sein Ruf verbreitete sich in ganz Syrien. Man brachte ihm Alle, die unwohl, nämlich mit verschiedenen Krankheiten und Qualen behaftet, vom Teufel besessen, mondsüchtig und gichtisch waren; und er heilte sie.} Dasselbe wird (\RWbibel{Mt}{}{8}{16}\ \RWbibel{Mt}{}{9}{35}\ und \RWbibel{Mt}{}{11}{5}, wo Jesus vor den Abgeordneten Johannis des Täufers Wunder wirkt, Tauben das Gehör, Blinden das Gesicht, Lahmen den Gebrauch ihrer Gliedmaßen gibt) wiederholt. Auch in der Gegend des Sees Genesareth wirkt Jesus dergleichen Wunder (\RWbibel{Mt}{}{14}{34}): \erganf{Die Leute dieser Gegend kannten ihn und schickten ihn in die ganze Umgegend; und man brachte alle Kranke zu ihm, und bat ihn, daß sie nur den Saum seines Kleides berühren durften; und so viele ihn berührten, wurden gesund.} In eben dieser Gegend heilt er (\RWbibel{Mt}{}{15}{29}) Blinde, Lahme, Stumme, \usw , die man aus den benachbarten Gegenden herbeigebracht hatte. -- Dieselbe Nachricht geben uns auch die übrigen Evangelisten, \RWbet{Mark.} (\RWbibel{Mk}{}{1}{39}\ \RWbibel{Mk}{}{6}{53}), \RWbet{Luk.} (\RWbibel{Lk}{}{4}{10}), \RWbet{Joh.} (\RWbibel{Joh}{}{2}{23})
\item Ohne die größte Frechheit hätten die Evangelisten, besonders der Verfasser jenes ältesten hebräischen Evangeliums, welches entweder der Grundtext Matthäi war, oder woraus Matthäus doch geschöpft hat, solche Erzählungen nicht wagen können, wenn von dem Allen nichts geschehen wäre. Denn Jeder hätte sie dann der Lüge strafen, Jeder sie nur befragen dürfen, wo denn die Menschen wären, die Jesus geheilt? wer davon je gehört, daß ein so großer Wunderthäter vor wenigen Jahren gelebt habe? Auf alle diese Fragen hätten sie beschämt verstummen müssen, und die Evangelien hätten nie eine so gute Aufnahme, das Christenthum nie so viele Anhänger gefunden. Wenn aber Jesus in der That so viele Kranke geheilt: so liegt am Tage, daß hier Wunder Statt gefunden haben. Denn unmöglich kann man
\begin{aufzb}
\item dem Gedanken Raum geben, \RWbet{daß diese Krankenheilungen alle nur ein Betrug gewesen wären}, den Jesus in Verbindung mit Mehreren ausgeführt~\RWSeitenw{192}\ habe. Zu diesem Ende müßte man annehmen, daß alle jene Menschen, deren wunderbare Heilung uns in den Evangelien erzählt wird, entweder nicht wirklich krank gewesen, oder nicht in der That geheilt worden seyen. Im ersten Falle ist gar nicht abzusehen, warum die Pharisäer und andere Feinde unseres Herrn nicht einige wirklich kranke Personen zu ihm gebracht, und den verhaßten Wunderthäter aufgefordert hätten, daß er an \RWbet{diesen} seine Wunderkraft beweise. Hätte er ihnen die Bitte verweigert: so würden sie nicht ermangelt haben, seine Krankenheilungen für das, was sie wirklich gewesen wären, für bloßen Betrug zu erklären. Im zweiten Falle, wenn jene Kranken nur scheinbar geheilt worden wären, würde der Scharfsinn der Feinde Jesu dieß wohl bemerkt haben. In beiden Fällen endlich hätte sein Betrug sehr viele Mitwisser gebraucht, und wäre eben deßhalb, wie dieses allezeit zu geschehen pflegt, durch den Einen oder den Anderen derselben aus Ungeschicklichkeit, Bosheit oder Reue verrathen und aufgedeckt worden.
\item Eben so wenig läßt sich die Heilung aller dieser Kranken durch die Annahme, \RWbet{daß Jesus sich vielleicht gewisser Arzneimittel bedient habe}, ohne ein Wunder erklären. Denn wäre es auch, daß er bei seinen Krankenheilungen gewisse Arzneimittel gebrauchte: so wäre doch immer \RWbet{das} außerordentlich und wunderbar gewesen, wie gerade er zu so hohen Einsichten in der Arzneikunde, dergleichen Niemand aus seinen Zeitgenossen hatte, gelanget sey; zumal da manche dieser Krankheiten, \zB\ Blindheit, Taubheit, Lahmheit der Glieder selbst unsere jetzige Arzneikunde noch nicht zu heilen weiß. In der That ersehen wir aber aus den Evangelien, daß Jesus nie Arzneimittel angewandt habe; denn nicht nur erwähnen die Evangelisten niemals etwas hievon; sondern sie erzählen uns vielmehr das Gegentheil; erzählen, daß er die Kranken durch sein bloßes Wort, oder durch eine bloße Berührung \udgl\  geheilt. Auch würden die Feinde Jesu, falls er sich einiger Arzneimittel bedient hätte, diesen, und nicht dem Beistande~\RWSeitenw{193}\ Baalzebubs den glücklichen Erfolg seiner Versuche zugeschrieben haben, (\RWbibel{Mt}{Matth.}{12}{20}\ \RWbibel{Lk}{Luk.}{11}{14})
\item[\RWbet{Einwurf.}] Aber woher so viele Kranke, so viele Lahme, Blinde, Stumme, die Jesus aller Orten antrifft, heilt, und wieder von Neuem antrifft, wenn er nach einigen Monaten an denselben Ort zurückkehrt?
\item[\RWbet{Antwort.}] Die evangelische Erzählung nöthiget uns gar nicht, anzunehmen, daß es der Lahmen, Blinden, Stummen \usw , die Jesus an allen Orten traf, ganze Schaaren gegeben habe. Genug, wenn es nur Einige waren, und Einige konnten allerdings zusammenkommen; da (wie uns der evangelische Text ausdrücklich sagt und wie es auch an sich wahrscheinlich ist) dergleichen Unglückliche nicht erst abwarteten, bis Jesus in ihre Heimath kommen würde, sondern den großen Wunderthäter selbst aufsuchten. Die Kranken, die er an gewissen Orten antraf, waren also nicht alle von diesem Orte, sondern aus allen umliegenden Ortschaften, oft auch aus sehr entlegenen Gegenden zusammengebracht, und eben darum konnte er auch an eben demselben Orte bei einem zweiten Besuche abermals mehrere dergleichen Unglückliche finden, ohne daß man Ursache hätte, zu argwöhnen, es wären dieß vielleicht die nämlichen gewesen, die er bei seinem ersten Besuche schon geheilt, aber nicht recht geheilt hätte, daher sie in ihre vorige Krankheit wieder zurückgefallen wären. In Dänemark zählt man unter 2000 bis 2500 Menschen einen Taubstummen; Blinde, Lahme, Wahnsinnige \udgl\  sind gewiß noch weit häufiger, als Taubstumme. Galiläa war sehr volkreich; die Erziehung war im ganzen Judenlande sehr schlecht; die Arzneiwissenschaft noch äußerst unvollkommen. Es mußte also allerdings eine bedeutende Menge krüppelhafter Menschen im Lande geben.
\end{aufzb}
\end{aufza}\par

\begin{center}\RWbet{B.~Einige einzelne Wunder Jesu.}\end{center}

Die Evangelisten begnügen sich nicht bloß damit, im Allgemeinen zu sagen, daß Jesus der Wunder viele gewirkt; sondern sie erzählen uns einige einzelne dieser Wunder mit der erwünschtesten Umständlichkeit. Ich will denn auch hier einige anführen.~\RWSeitenw{194}
\begin{aufza}
\item \RWbet{Heilung des Knechtes eines römischen Hauptmannes} (\Ahat{\RWbibel{Mt}{Matth.}{8}{5\,ff}}{8,15\,ff.}\ \RWbibel{Lk}{Luk.}{7}{1\,ff}): Der Knecht (oder Sklave) eines römischen Hauptmannes zu Kapharnaum war tödtlich krank. Der menschenfreundliche Hauptmann hielt dieses Knechtes Leben nicht für zu gering, um nicht Alles, was in seinen Kräften stand, aufzubieten, ihn, wo möglich, zu retten. Er verfügte sich also zu Jesu, oder vielmehr (wie Lukas uns dieß umständlicher berichtet) er bat die Aeltesten (die Vorsteher) der Juden zu Kapharnaum, daß sie in seinem Namen zu Jesu gehen, und ihn bereden möchten, seinen Knecht zu heilen. Diese gaben dem Hauptmanne das beste Zeugniß, und Jesus machte sich gleich auf den Weg zu ihm. Als nun der Hauptmann erfuhr, daß Jesus selbst sich in sein Haus bemühen wolle, schickte er seine Freunde an ihn ab, die ihm ausrichten mußten: Herr! bemühe dich nicht: denn ich bin nicht würdig, daß du unter mein Dach kommest. Deßwegen habe ich mich selbst nicht erkühnet, zu dir zu kommen. Sag es nur mit einem Worte: so wird mein Knecht gesund werden. Jesus bewunderte dieß Zutrauen, pries es vor dem umstehenden Volke mit den Worten: In Wahrheit sage ich euch, in Israel (\di\ bei den Juden) selbst habe ich noch kein so großes Vertrauen gefunden. Dem Hauptmann aber ließ er entbiethen: Es geschehe dir, wie du geglaubt hast. Und der Knecht ward gesund von dieser Stunde. --\par
Mag immerhin die Genesung des Knechtes nur durch Naturkräfte erfolgt seyn, wie denn nicht selten eine Krankheit, die schon mit dem Tode gedrohet, sich plötzlich wieder bessert; mag immerhin auch die Gemüthstimmung, durch welche Jesus sich bewogen fühlte, diesen Erfolg der Genesung mit voller Zuversicht vorherzusagen, auf bloß natürliche Art entstanden seyn: darin, daß dieses Beide so zusammentraf, verräth sich deutlich die Absicht der Alles leitenden Vorsehung Gottes, ihren Gesandten zu verherrlichen. Die Krankheit des Knechtes hätte von einer anderen Art seyn können, oder die Abgeordneten des Hauptmannes hätten seine Bitte dem Herrn um einige Stunden später vortragen dürfen, und das ganze Ereigniß hätte aufgehört, zur Verherrlichung Jesu zu dienen.~\RWSeitenw{195}
\item \RWbet{Heilung der Schwiegermutter Petri} (\RWbibel{Mt}{Matth.}{8}{14}\ \RWbibel{Mk}{Mark.}{1}{29}\ \RWbibel{Lk}{Luk.}{4}{38}) Dieses Wunder ging den Apostel Petrus besonders nahe an; er mußte sich also von der Richtigkeit der Sache vollkommen überzeugt haben. Bekanntlich aber ist das Evangelium Marci unter seiner Leitung geschrieben; wir haben hier also eigentlich Petri eigenen Bericht hierüber. Uebrigens gilt die Bemerkung, die ich Nr.\,1.\ angebracht habe, auch hier.
\item \RWbet{Stillung eines Meersturmes} (\RWbibel{Mt}{Matth.}{8}{23}\ \RWbibel{Mk}{Mark.}{4}{35}\ \RWbibel{Lk}{Luk.}{8}{22}) Jesus befand sich mit seinen Jüngern auf einem Schiffe auf dem See Genesareth, als eben ein sehr gewaltiger Sturm entstand. (\erganf{Es entstand ein so großes Ungewitter auf dem Meere, daß die Wellen auch über das Schifflein hingingen.} Matth.) Jesus aber schlief ruhig im Vordertheile des Schiffes. \erganf{Die Jünger traten zu ihm, weckten ihn, und riefen: Herr, hilf uns, wir gehen unter! Jesus antwortete: Warum seyd ihr so furchtsam, ihr Kleingläubigen? Dann stand er auf, und gebot den Winden und dem See, und es ward eine große Stille. Mit Erstaunen sprachen die Leute (aus Markus sehen wir, daß noch mehrere Schiffe auf dem See waren): Welch ein Mann ist das! Selbst Winde und Meere gehorchen ihm!} --\par
Wie ungereimt die Behauptung sey, Jesus habe die Stillung dieses Sturmes durch Oel bewirkt, das er in großer Menge über die Meeresfläche habe ausgießen lassen, leuchtet von selbst ein. Woher dieß viele Oel? warum sagen die Evangelisten nichts davon? Noch ungereimter ist die Zumuthung, die der Verfasser des Horus an seine Leser macht, ihm auf sein Wort zu glauben, daß einem jeden Menschen die Macht zustehe, durch seinen bloßen Willen, wenn er recht fest ist, der ganzen Natur zu gebieten, und somit ähnliche Dinge, wie sie von Jesu hier erzählt werden, hervorzubringen. Andere sagen, der Sturm habe aufgehört, weil um diese Zeit gerade die natürliche Ursache, die ihn hervorgebracht und erhalten hatte, aufgehört habe, und er hätte jetzt aufgehört, auch wenn ihm Jesus nicht geboten hätte. Sey es; aber eben darin, daß von Gottes allwaltender Vorsehung alle Umstände so geleitet wurden, daß der Sturm gerade damals~\RWSeitenw{196}\ aufhören mußte, als der Herr Jesus das Machtwort aussprach, wie auch, daß er sich aufgelegt fühlte, dieß Machtwort auszusprechen, welches voraussetzt, daß er die baldige Beendigung des Sturmes mit aller Zuversicht erwartete, entweder im Vertrauen auf den allmächtigen Beistand Gottes, oder aus gewissen natürlichen Anzeichen: eben darin liegt ein deutlicher Beweis, daß Gott bei diesem ungewöhnlichen Ereignisse die Absicht der Verherrlichung seines Sohnes hatte, \dh\ daß es ein Wunder sey.
\item \RWbet{Heilung eines Gichtbrüchigen.} (\RWbibel{Mt}{Matth.}{9}{1\,ff}) Als Jesus zu Kapharnaum in einem Hause (vermuthlich des Apostels Petrus) Unterricht ertheilte, hatte sich eine große Menge Menschen, mitunter auch Pharisäer und Schriftgelehrte aus Jerusalem eingefunden, so daß ein Gichtbrüchiger, an allen Gliedern Gelähmter, den man auf einem Bette herbeigetragen hatte, nicht durch den gewöhnlichen Eingang in das Innere des Hauses gebracht werden konnte; daher man denn auf den Gedanken verfiel, ihn über die Treppe, die auf das flache Dach des Hauses von Außen führte, hinaufzutragen, und dann nach Wegbrechung eines Theiles der Brustwehre von oben in das Innere des Hauses herabzulassen. Als Jesus diese Probe des Vertrauens erblickte, sprach er zu dem Gichtbrüchigen: Fasse Muth, mein Sohn! deine Sünden sollen dir vergeben seyn. (Er mochte sich nämlich diese Krankheit durch einen ausschweifenden Lebenswandel zugezogen haben.) An diesen Worten Jesu nahmen die Schriftgelehrten Aergerniß, weil nur Gott allein Sünden vergeben könne. Jesus bemerkte dieß und sprach: Was denket ihr Arges in euren Herzen? Was ist wohl leichter zu sagen: Dir sind deine Sünden vergeben? oder zu sagen: Stehe auf, und wandle? Damit ihr aber sehet, daß der Sohn des Menschen die Macht habe, auch Sünden zu vergeben: so stehe auf, und trage selbst dein Bett von hinnen. Und es erfolgte.\par
Dieß Wunder also wirkte der Herr in Gegenwart seiner erbitterten Feinde; wie sicher mußte er nicht seines Erfolges seyn? Wollte man etwa sagen, der Kranke habe die Kräfte, die er jetzt äußerte, schon früher gehabt, ohne sich ihrer bewußt zu werden, weil er aus Muthlosigkeit keinen~\RWSeitenw{197}\ Versuch gewagt: so ist eben dieses schon ein seltenes Ereigniß, welches die Vorsehung Gottes sichtbar nur zur Verherrlichung Jesu herbeigeführt hat.
\item \RWbet{Erweckung der Tochter des Jairus.} (\RWbibel{Mt}{Matth.}{9}{18}\ \RWbibel{Mk}{Mark.}{5}{21}\ \RWbibel{Lk}{Luk.}{8}{40}) Der oberste Vorsteher der Synagoge zu Kapharnaum Jairus kam zu dem Herrn mit der Bitte, seine schon in den letzten Zügen liegende Tochter durch Auflegung der Hände gesund zu machen. Jesus begab sich alsogleich mit ihm, und ward von einer großen Menge Volkes begleitet. Sie waren noch nicht bis zum Hause gelangt, als sie die Botschaft erhielten, das Mädchen sey bereits verschieden. Fürchte dich nicht, sprach Jesus zu dem verzagenden Vater, behalte nur dein Vertrauen. Als man in das Haus eintrat, hörte man schon das Weinen und Klagen derjenigen, die mit der Zurüstung zum Begräbnisse beschäftiget waren. Jesus wies alle diese Leute hinaus; versicherte, daß das Mädchen nicht todt sey, sondern nur schlafe, trat zu ihr hin, und richtete sie, indem er ihre Hand ergriff, von ihrem Lager auf. Alsbald war sie gesund, und konnte Speise genießen.
\end{aufza}\par
Allerdings ist es nichts Unerhörtes, daß Jemand, der schon für todt gehalten wird, aus seiner Ohnmacht wieder zurückkehre; aber daß dieß gerade bei dem Mädchen geschah, das Jesus auferwecken wollte, beweiset deutlich die Absicht Gottes, ihn zu verherrlichen.
\begin{aufza}\setcounter{enumi}{5}
\item \RWbet{Auferweckung des Jünglings von Naim.} (\RWbibel{Lk}{Luk.}{7}{11\,ff}) Als Jesus einst in Begleitung seiner Jünger und einer großen Menge Volkes auf die Stadt Naim in Galiläa zuging, trug man eben die Leiche eines Jünglings hinaus, welcher der einzige Sohn einer Wittwe war. Jesus wird durch diesen Anblick gerührt; er läßt die Träger des Sarges halten, und übergibt den wieder lebenden Jüngling der Mutter. Alle preisen Gott, und der Ruf dieser That verbreitet sich durch ganz Judäa und in der umliegenden Gegend.
\end{aufza}\par
Hier gelten dieselben Bemerkungen wie vorher.
\begin{aufza}\setcounter{enumi}{6}
\item \RWbet{Heilung der ausgetrockneten Hand in der Synagoge.} (\Ahat{\RWbibel{Mt}{Matth.}{12}{9}}{12,7.}\ \RWbibel{Mk}{Mark.}{3}{1}\ \RWbibel{Lk}{Luk.}{6}{6}) In~\RWSeitenw{198}\ einer Synagoge in Galiläa (ob zu Kapharnaum, ist unbekannt), dahin sich Jesus an einem Sabbathe begeben, befand sich ein Mann, dessen rechter Arm ganz ausgetrocknet war. Die Pharisäer lauerten schon darauf, ob Jesus ihn am Sabbathe heilen werde. Jesus befahl dem Manne, sich in die Mitte des Saales zu stellen; dann fragte er die Pharisäer: Welches von Beiden geziemt sich wohl eher am Sabbathe, Gutes oder Böses zu thun? retten, oder zu Grunde gehen lassen? Man schwieg. Wo ist Einer aus euch, fuhr Jesus weiter fort, der, wenn sein Schaf an einem Sabbathe in die Grube fiele, es nicht alsbald retten wollte? Ist nun ein Mensch nicht ungleich mehr werth, als ein Schaf? Ist es denn also ein Zweifel, es sey dem Menschen erlaubt, wohlzuthun auch am Sabbathe? Strecke deine Hand aus! Er streckte sie aus, und sie ward gesund, wie die andere. Die Pharisäer faßten von diesem Tage an den Entschluß, Jesum aus dem Wege zu räumen.\par
Hier war doch keine Täuschung möglich. Die Veränderung, welche durch das Wort Jesu in der Hand erfolgte, muß sehr auffallend gewesen seyn; sonst hätten die Pharisäer nicht ermangelt, die Wirklichkeit des Wunders in Zweifel zu ziehen.
\item \RWbet{Wunderthätige Speisung einer sehr großen Volksmenge.} (\RWbibel{Mt}{Matth.}{14}{14}\ \RWbibel{Mk}{Mark.}{6}{30}\ \RWbibel{Lk}{Luk.}{9}{10}\ \RWbibel{Joh}{Joh.}{6}{1}) Als Jesus hörte, daß Herodes Johannes den Täufer habe umbringen lassen, begab er sich (vermuthlich um dem hierüber mißvergnügten Volke die Gelegenheit zum Aufstande zu benehmen) jenseits des Sees Genesareth in eine Einöde. Das Volk aber, das seinen Aufenthalt erfuhr, suchte ihn auch selbst hier auf. Hierüber ward Jesus gerührt, heilte die Kranken, die sich in ihrer Mitte befanden, und unterrichtete sie. Als es Abend wurde, ermahnten ihn die Jünger, daß er das Volk entlassen möge, damit es die benachbarten Ortschaften beziehen, und sich daselbst Lebensmittel anschaffen könnte; denn ihr Mundvorrath sey bereits aufgezehrt. Da sagte Jesus: Es ist nicht nöthig, daß sie weiter ziehen, sondern gebt ihnen selbst Speise. Wie können wir das? war die Antwort, da wir nicht mehr als fünf Brode und zwei~\RWSeitenw{199}\ Fische haben; was ist das für so Viele? Bringet nur diese her, erwiedert Jesus, und befiehlt dem Volke, daß es sich lagern sollte. Darauf nahm er die fünf Brode und die zwei Fische, blickte gen Himmel, segnete sie, brach sie, reichte die Theile den Jüngern, und diese trugen sie dem Volke zu. Es aßen Alle, heißt es, und wurden gesättiget; und nach beendigter Mahlzeit, als Jesus die Ueberbleibsel zu sammeln befahl, füllte man damit zwölf Körbe. Es waren aber der Personen, welche gegessen hatten, gegen 5000 an der Zahl, Weiber und Kinder nicht mitgerechnet. Das über dieß Wunder erstaunte Volk wollte ihn sofort zum Könige machen; er aber befahl seinen Jüngern, augenblicklich auf das entgegengesetzte Ufer zu fahren, selbst aber bestieg er einen nahe gelegenen Berg, um zu beten. Auf eine ähnliche Art speisete er ein zweites Mal 4000 Menschen, und es wurden sieben Körbe Ueberbleibsel gesammelt.\par
Auf einen Betrug, vermöge dessen Jesus der Brode viel mehrere in irgend einem Hinterhalte verborgen gehabt haben sollte, ist hier auf keinen Fall zu denken. Wie hätte ein solcher Betrug unbemerkt bleiben können? Könnte man sich aber vorstellen, daß der kleine Vorrath, den Jesus hergab, die Sättigung des Volkes nicht unmittelbarer, sondern mittelbarer Weise bewirkt habe, nämlich nur dadurch, daß sein Beispiel der Freigebigkeit für alle diejenigen, welche noch einigen Mundvorrath hatten, eine Art von Aufforderung war, ihm nachzuahmen, und denen, die nichts hatten, etwas davon zukommen zu lassen: auch so noch würde die glückliche Ausführung dieser That und ihr Erfolg beweisen, welch eine außerordentliche Ueberlegenheit des Geistes über andere Menschen unser Herr Jesus gehabt, und wie es so völlig in seiner Macht gestanden, die Herzen der Menschen zu lenken, wohin er nur wollte. Ist nicht auch dieses ein Wunder?
\item \RWbet{Heilung des Blinden bei Jericho.} (\RWbibel{Mt}{Matth.}{20}{29}\ \RWbibel{Mk}{Mark.}{10}{46}\ \RWbibel{Lk}{Luk.}{18}{35\,ff}) Als Jesus aus der Stadt Jericho ging (oder wie Lukas erzählt, als er in diese Stadt hineinging), traf er zwei Blinde an, die an der Straße saßen, und die Vorübergehenden um ein Almosen ansprachen. (Markus und Lukas erwähnen nur eines einzigen.) Als diese~\RWSeitenw{200}\ hörten, daß Jesus vorübergehe, riefen sie mit lauter Stimme: Erbarme dich unser, du Sohn Davids! Er fragte nach ihrem Begehren, und als sie erklärten, daß sie nichts Anderes, als die Wiedererlangung des Augenlichts wünschten, so machte sein Wort sie sehend.\par
Der Umstand, daß die drei Evangelisten in einigen Stücken ihrer Erzählung von einander abweichen, überzeugt uns nur um so völliger von der Wahrheit der Hauptsache. Denn offenbar kann diese ganze Erzählung keine aus bloßer Verabredung entstandene Erdichtung seyn, weil wir dann annehmen müßten, daß sich die Evangelisten bis auf die Worte, welche sie den Blinden und Jesum sprechen lassen, verabredet hätten; da diese bei allen Dreien ganz übereinstimmen. Wären sie aber in ihrer Verabredung einmal so weit gegangen: so hätten sie sich gewiß auch über die anderen Umstände verabredet, in welchen sie von einander abweichen; nämlich über die Anzahl der Blinden, und ob sie beim Eintritte oder beim Ausgange aus Jericho geheilt worden seyen. 
\item \RWbet{Heilung des acht und dreißig jährigen Kranken am Schwemmteiche.} (\RWbibel{Joh}{Joh.}{5}{1\,ff}) Es war ein Fest der Juden, weßwegen Jesus nach Jerusalem reisete. Bei dem Schafthore zu Jerusalem ist ein Badeteich (Bethesda) mit fünf bedeckten Gängen. In diesen lag eine große Menge Kranke, Blinde, Lahme und Abgezehrte, welche die Aufwallung des Wassers abwarteten; denn zu gewissen Zeiten stieg ein Engel in den Teich hinab, und machte das Wasser aufwallen. Wer dann nach der Aufwallung zuerst in's Wasser kam, der wurde gesund, welche Krankheit er auch haben mochte. Da war nun ein Mensch, der schon acht und dreißig Jahre krank war. Als ihn Jesus da liegen sah, und wußte, daß er schon lange krank sey, sprach er zu ihm: Möchtest du gerne gesund werden? Der Kranke antwortete: Herr! ich habe Niemand, der mich bei Aufwallung des Wassers in den Teich brächte; ehe ich aber selbst dahin komme, ist schon ein Anderer vor mir hineingestiegen. Jesus sagte ihm: Stehe auf, nimm dein Bett und gehe. Der Mensch wurde auf der Stelle gesund, nahm sein Bett und ging. Es war aber Sabbath an diesem Tage. Daher sprachen die Juden zu~\RWSeitenw{201}\ dem Gesundgewordenen: Es ist Sabbath; du darfst dein Bett nicht tragen. Er aber antwortete ihnen: Der mich gesund gemacht, befahl mir: Nimm dein Bett und gehe. Sie fragten, wer der Mensch wäre, der dieses gesagt; allein der Geheilte wußte es nicht; denn Jesus hatte sich entfernt, weil viele Leute da waren. Später traf ihn Jesus im Tempel, und sprach zu ihm: Siehe, du bist gesund geworden; sündige in Zukunft nicht wieder, damit dir nicht Schlimmeres widerfahre. Nun ging der Mensch hin, und gab den Juden die Nachricht, daß es Jesus sey, der ihn gesund gemacht; deßwegen verfolgten die Juden Jesum, weil er dieß am Sabbathe gethan, und trachteten ihm nach dem Leben. 
\end{aufza}\par
Was hier von einer übernatürlichen Heilkraft jenes Gesundheitsbades und von dem Engel erzählt wird, dürfte wohl starken Zweifeln unterliegen. Vielleicht daß dieses Bad nichts Anderes, als eine von jenen mineralischen Quellen gewesen, die zu gewissen Jahreszeiten in eine eigene Art von Gährung (nämlich durch Zufluß einer anderen Quelle) gerathen, und dann besondere Heilkräfte äußern. Der Wahn des Volkes hielt nun dafür, daß diese Quelle für alle Krankheiten tauge, und daß ein Engel die nicht zu erklärende Bewegung, die man im Wasser bemerkte, hervorbringe. Wenn Jemand durch den Gebrauch des Bades nicht genaß, schob man die Schuld darauf, daß er nicht zeitlich genug hinabgestiegen sey. -- Der berühmte Arzt \RWbet{Boerhave} stellte die scharfsinnige Vermuthung auf, daß in dieses Bad vielleicht das frische Blut der geschlachteten Opferthiere geleitet worden sey, welches dem Wasser um die Zeit solcher Festtage, wo viel geopfert wurde, allerdings eine gewisse Heilkraft hätte mittheilen können. Das Wort Engel (meint er) bedeute hier vielleicht nur einen Abgeordneten (Diener, Boten), den man abschickte, um das Wasser mit einem Spate umzurühren. Es mag sich hiemit so oder anders verhalten haben: so ist doch das sehr merkwürdig, daß der Evangelist gar nicht abgeneigt ist, die Heilung der Kranken in diesem Bade zuzugestehen. Dieses beweiset uns nämlich die Unparteilichkeit, mit welcher er erzählt. Er konnte voraussehen, daß die wunderbare Heilkraft, die er dem Bade zu Bethesda zuschreibt, das Ansehen~\RWSeitenw{202}\ der wunderbaren Heilungen Jesu in etwas herabsetzen werde; und dennoch bestritt er sie nicht. Die Handlung Jesu selbst, die hier erzählt wird, ist ohne Widerspruch ein Wunder. Wie konnte ein Betrug Statt finden, da dieser Kranke so Vielen bekannt war? Merkwürdig ist auch, daß Jesus kein Bedenken trug, durch den Befehl, den er dem Genesenen gab, sein Bette sogleich (an einem Sabbathe) nach Hause zu tragen, den Unwillen der Pharisäer nur noch mehr aufzureizen, und sie auf diese Art dahin zu vermögen, daß sie nichts unversucht lassen, was zur Herabwürdigung des Wunders dienen konnte! Wie hätte er sich gehütet, dieses zu thun, wäre ein Betrug dabei im Spiele gewesen? Die Pharisäer sind auch so wenig im Stande, die Sache in den Verdacht eines solchen Betruges zu bringen, daß sie in ihrer Verzweiflung lieber den Entschluß fassen, Jesum je eher, je lieber aus dem Wege zu räumen, damit er der Wunder nicht immer mehrere wirke.
\begin{aufza}\setcounter{enumi}{10}
\item \RWbet{Heilung eines Blindgebornen.} (\RWbibel{Joh}{Joh.}{9}{1\,ff}) Zu Jerusalem sah Jesus im Vorübergehen einen Blindgebornen. Seine Jünger, welche diese Blindheit für eine Strafe Gottes hielten, fragten: Lehrer! wer hat es verschuldet, daß er blind geboren wurde, er selbst oder seine Eltern? Jesus antwortete: Weder er hat es verschuldet, noch seine Eltern; sondern die Macht Gottes soll an ihm offenbar werden. Hierauf spie er auf die Erde, machte aus dem Speichel (und Staub) einen Teig, strich ihn auf die Augen des Blinden, und sprach zu ihm: Gehe hin, und wasche dich in dem Teiche Siloa. Er ging hin, wusch sich und kam sehend zurück. Einige, die ihn zuvor als Bettler gesehen hatten, erkannten ihn sogleich wieder; Andere dagegen meinten, er sey ihm bloß ähnlich; er selbst aber erklärte: Ich bin es wirklich. Sie fragten ihn nun: Wie bist du denn sehend geworden? Er antwortete: Ein Mann, der Jesus heißt, machte einen Teig, bestrich damit meine Augen, und sagte: Gehe hin zum Teiche Siloa, und wasche dich. Ich ging, wusch mich, und wurde sehend. Man fragte ihn, wo Jesus sey; er aber wußte es nicht. Nun führten sie den Blindgebornen vor die Pharisäer; denn es war am Sabbathe, als Jesus den Teig gemacht und ihm das Gesicht gegeben hatte. Die Pharisäer~\RWSeitenw{203}\ fragten nun auch, wie er sehend geworden; und er antwortete: Einen Teig legte er mir auf die Augen, ich wusch mich, und wurde sehend. Hierauf sagten Einige von den Pharisäern: Dieser Mensch (Jesus) ist nicht von Gott, weil er den Sabbath nicht hält; Andere aber bemerkten: Wie kann ein lasterhafter Mensch solche Wunder wirken? Sie waren daher nicht einig mit einander, und fragten den Blindgewesenen: Was sagst denn du von ihm, der dir die Augen geöffnet hat? Dieser antwortete: Er ist ein Prophet. Nun wollten es die Juden gar nicht glauben, daß er blind gewesen, und sehend geworden sey, bis sie die Eltern desselben vorgefordert hatten. Diese wurden nun befragt: Ist dieß euer Sohn, von dem ihr saget, daß er blind geboren worden? Wie kommt es, daß er jetzt sehen kann? Die Eltern antworteten: Daß er unser Sohn ist, und daß er blind geboren worden, das wissen wir; wie es aber kommt, daß er jetzt sehen kann, und wer ihm die Augen geöffnet, das wissen wir nicht. Er ist alt genug; fragt ihn selbst; er kann für sich selbst sprechen. So sagten die Eltern aus Furcht vor den Juden; denn diese hatten beschlossen, Jeden aus der Synagoge zu stoßen, der Jesum für den Messias erkennen würde. Man ließ nun den Blindgewesenen wieder kommen, und forderte ihn auf: Gib Gott die Ehre! Wir wissen, daß jener Mensch ein Sünder sey. Ob er ein Sünder sey, erwiederte der Geheilte, das weiß ich nicht; aber Eines weiß ich: daß ich blind war, und nun sehe. Man fragte ihn abermals: Was hat er mit dir vorgenommen? Wie hat er dir die Augen geöffnet? Er antwortete ihnen: Ich habe es euch schon gesagt, und ihr habt es auch gehört. Warum wollet ihr es noch einmal hören? Wollet ihr seine Jünger werden? Da verfluchten sie ihn, und sprachen: Du magst sein Jünger seyn; wir sind Mosis Jünger. Daß mit Moses Gott geredet habe, wissen wir; wer aber diesen Menschen gesendet, das wissen wir nicht. Hierauf antwortete der Geheilte: Das ist sonderbar, daß ihr nicht wisset, wer ihn gesendet habe, da er mich doch sehend gemacht hat, und da es gewiß ist, daß Gott nicht die Sünder, sondern nur den erhöret, der ihn verehrt und seinen Willen thut. Seit Menschengedenken ist es unerhört, daß Jemand einen Blindgebornen~\RWSeitenw{204}\ sehend gemacht hätte; und wenn dieser nicht von Gott wäre: so könnte auch er dergleichen nicht thun. Du bist ganz und gar in Sünden \RWbet{geboren}, und willst uns belehren!? antworteten die Pharisäer, und stießen ihn hinaus.\par
Vertrauen zu dieser Erzählung, die aus der Feder eines \RWbet{Johannes} geflossen, muß uns schon das beibringen, daß gleich im Anfange gestanden wird, die Apostel selbst hätten das Vorurtheil gehegt, daß dieser Blinde entweder in eigener Person, oder daß seine Eltern gesündiget haben müßten. An eine Augenoperation, dergleichen wir heut zu Tage oft mit dem glücklichsten Erfolge vornehmen sehen, kann man schon darum nicht denken, weil bei der scharfen Untersuchung, welche die Feinde Jesu hierüber anstellten, gewiß etwas entdeckt worden wäre. Auch ist bekannt, daß Blindgeborne, denen der Staar glücklich gestochen wird, erst mehrere Wochen bedürfen, um mit der \RWbet{Fähigkeit} zu sehen, auch die \RWbet{Fertigkeit} des Sehens zu erlangen. Jesus verrichtete hier also ein \RWbet{doppeltes} Wunder.
\item \RWbet{Erweckung des Lazarus.} (\RWbibel{Joh}{Joh.}{11}{1\,ff}) Lazarus aus Bethanien (einem Flecken bei Jerusalem) war krank, und seine Schwestern Maria und Martha schickten zu Jesus, der sich jenseits des Jordans, in der Gegend, wo ihn Johannes getauft hatte, aufhielt, und ließen ihm sagen: Den du lieb hast, der liegt krank. Jesus schickte den Boten mit der Versicherung zurück, daß die Krankheit nicht zum Tode, sondern zur Ehre Gottes sey; und erst nach zwei Tagen, nachdem er den Jüngern bekannt gemacht hatte, daß Lazarus gestorben, und daß er ihn erwecken wolle, ging er nach Bethanien, wo sich viele Juden aus Jerusalem eingefunden hatten, um die Schwestern zu trösten. Martha hatte seine Ankunft vernommen, lief ihm entgegen und sprach: Wärest du doch hier gewesen, Herr! so wäre mein Bruder nicht gestorben; aber ich weiß, daß du auch jetzt noch Gott bitten, und bei ihm Erhörung finden kannst. Jesus gab ihr zur Antwort: Dein Bruder wird wieder auferstehen! Das weiß ich wohl, entgegnete Martha, daß er wieder auferstehen werde, nämlich bei der allgemeinen Auferstehung am letzten Tage. Jesus aber sagte: Ich bin die Auferstehung und das Leben.~\RWSeitenw{205}\ Wer an mich glaubt, der wird leben, wenn er auch gestorben ist; und wer da lebt, und an mich glaubt, der wird nicht sterben in Ewigkeit. Glaubst du das? Martha antwortete: Ja, Herr! ich glaube, daß du Christus, der Sohn des lebendigen Gottes bist, der auf die Welt kommen sollte. Hierauf rief sie heimlich (sie wünschte, daß die Juden aus Jerusalem nicht mitgingen) ihre Schwester; und als Maria zu Jesu kam, fiel sie vor ihm nieder und sprach: Wärest du hier gewesen, Herr! so wäre mein Bruder nicht gestorben. Mittlerweile waren auch die Freunde und Bekannten des Verstorbenen hinzugekommen; und als Jesus den allgemeinen Schmerz sah, wurde er tief gerührt und sprach: Wo habt ihr ihn hingelegt? Sie sagten: Herr! komm, und siehe! Jesus ging und weinte. Da sprachen die Juden zu einander: Sehet, wie lieb er ihn gehabt hat! Konnte denn Er, der den Blinden sehend machte, nicht machen, daß dieser nicht gestorben wäre? Man war nun zur Gruft gekommen, einer Höhle, die mit einem Stein verschlossen war; und Jesus, neuerdings erschüttert, sagte: Nehmet den Stein hinweg! Martha erinnerte ihn: Herr! er riecht schon; denn er liegt bereits vier Tage. Jesus antwortete: Habe ich dir nicht gesagt, du werdest die Herrlichkeit Gottes sehen, wenn du glauben wirst? Man nahm also den Stein hinweg. Jesus erhob seine Augen zum Himmel und sprach: Vater! ich danke dir, daß du mich erhöret hast. Ich weiß zwar wohl, daß du mich immer erhörest; aber des Volkes wegen sage ich es, damit man glaube, daß ich von dir gesendet bin. Hierauf rief er mit lauter Stimme: Lazarus! komm heraus! Und der Verstorbene kam heraus, Hände und Füße mit Binden umwunden, und das Gesicht mit einem Tuche umhüllt. Jesus sprach: Löset ihm die Binden ab, und lasset ihn geh'n. -- Viele Juden, die gesehen hatten, was Jesus gethan, glaubten an ihn; die Oberpriester aber und die Pharisäer, als sie die Nachricht von diesem Wunder erhielten, versammelten sich, und beschloßen, Jesum nach geendigtem Osterfeste gefänglich einzuziehen.\par
Je öfter man diese Geschichte mit prüfendem Geiste liest, um desto mehrere Beweise ihrer Glaubwürdigkeit entdeckt man in ihr selbst. Alles ist hier so nach dem Leben~\RWSeitenw{206}\ gezeichnet, der Charakter des Herrn so erhaben, und doch auch so menschlich schön, daß es das größte Kunststück gewesen wäre, diese Geschichte zu erdichten. Und von der anderen Seite findet man gleichwohl in eben dieser Geschichte auch so viel Kunstlosigkeit in der Erzählungsart, hie und da selbst so manche Dunkelheit, die der Erzähler nimmermehr stehen gelassen, wenn er die ganze Sache erdichtet hätte. Der Umstand \zB : Er riecht schon, wird etwas zu früh erzählt; er hätte später besser benützt werden können. Bemerkungswerth ist auch, daß Martha bei aller Versicherung, die ihr gegeben wird, doch nicht eher an die Möglichkeit des Wunders glaubt, als bis sie die Wirklichkeit sieht. Mag also immerhin die Frage, warum die drei anderen Evangelisten von diesem Wunder schweigen, nicht so leicht zu beantworten seyn: doch werden wir dieser Erzählung dennoch um ihrer inneren Glaubwürdigkeit wegen vertrauen dürfen, auch wenn es nicht so viele äußere Gründe gäbe, die für die Wahrhaftigkeit der erzählten Begebenheit sprechen. -- Gibt man uns aber nur \RWbet{zwei} Umstände zu, welche in dieser Erzählung als öffentlich bekannte angegeben werden; so ist das Wunder schon entschieden. Die Umstände, die ich hier meine, sind:
\begin{aufzb}
\item daß \RWbet{mehrere Juden}, welche als Zuschauer bei dieser Begebenheit zugegen waren, durch sie \RWbet{gläubig geworden} seyen; und
\item daß ein \RWbet{Leichengeruch} zu verspüren gewesen.
\end{aufzb}\par
Aus jedem dieser Umstände für sich allein läßt sich beweisen, daß hier ein Wunder müsse obgewaltet haben.
\begin{aufzb}
\item Mehrere Augenzeugen wurden gläubig. Sie wären gewiß nicht gläubig geworden, wenn sie in dieser Begebenheit nicht ein wahres Wunder erkannt hätten. Da aber diese Personen sich in dem Hause der Martha bereits einige Tage lang aufgehalten hatten: so ist nicht zu begreifen, wie ihnen ein Betrug, wenn einer hier gespielt worden wäre, unbemerkt bleiben konnte. Ueberhaupt läßt sich hier nur dreierlei annehmen:
\begin{aufzc}
\item entweder die ganze Sache sey ein mit der Familie des Lazarus \RWbet{verabredeter Betrug} gewesen, so daß sich Lazarus bloß zum Scheine begraben ließ; oder~\RWSeitenw{207}
\item er wurde \RWbet{in einer Ohnmacht fälschlich für todt gehalten}, aus Irrthum begraben, und erholte sich im Grabe wieder; oder
\item er war \RWbet{wirklich todt} gewesen.
\item[Gegen] die Annahme des Ersten (\RWgriech{a}) streitet nicht nur der Charakter Jesu und die Wohlhabenheit, in welcher sich die Familie des Lazarus befand, welche sie hinlänglich abhalten konnte, einen auf jeden Fall so gefährlichen Betrug zu wagen, sondern auch noch der Umstand, daß die Leichen nach jüdischer Sitte von fremden Personen bestattet wurden. Auch diese hätten also Mitwisser des Betruges werden müssen; und so wäre er denn eben darum, weil er so viele Mitwisser hatte, auch wohl verrathen worden.
\item[Gegen] die zweite Annahme (\RWgriech{b}) streitet der Umstand, daß Lazarus eben durch jene Gebräuche, die man bei der Bestattung vornahm, wieder zu sich gebracht worden wäre, falls es nur eine Ohnmacht gewesen, in welche er verfallen. Und wenn dieß nicht geschehen wäre, dann hätte die Einbalsamirung, die Einhüllung des Hauptes und die Versperrung in eine dumpfige Grabeshöhle sein Wiedererwachen eher verhindern, als befördern sollen. Auch bliebe hier noch immer das wunderbar, daß Jesus es voraus gewußt habe, daß Lazarus nur ohnmächtig sey, und wieder erwachen werde. Wollte man aber läugnen, daß Jesus dieses bestimmt voraus gewußt habe: so sage ich: auch schon der bloße Umstand, daß Gott diesen Liebling Jesu, dessen Wiederauflebung er wünschte, in der That wieder aufleben ließ, ist ein Beweis des höchsten göttlichen Wohlgefallens an Jesu, und als ein Wunder zu betrachten.
\item[Ist Lazarus] wirklich todt gewesen (\RWgriech{g}): so wird Niemand in Abrede stellen, daß sein Wiederaufleben ein wahres Wunder gewesen.
\end{aufzc}
\item Für dieses spricht auch der \RWbet{Leichengeruch}, der nur bei \RWbet{wirklich Todten} sich einstellt. Aber man möchte (mit dem Verfasser des Horus) vielleicht einwenden, daß Martha nur vorgegeben habe, Leichengeruch zu verspüren.~\RWSeitenw{208}\ Wenn aber nicht wirklich Leichengeruch zu verspüren war: so hätte sich Martha wohl hüten sollen, durch eine solche Aeußerung die umstehenden Juden zu erinnern, daß bei einem viertägigen Todten allerdings Leichengeruch zu verspüren seyn müsse, denn eben jetzt hätten sie, aufmerksam gemacht auf dieß mangelnde Kennzeichen des Todes, den Betrug entdecken müssen.
\end{aufzb}
\end{aufza}

\RWpar{67}{VI.~Die Auferstehung Jesu}
Die Begebenheit der \RWbet{Auferstehung Jesu} läßt sich aus zweierlei sehr wohl zu unterscheidenden Gesichtspuncten betrachten:
\begin{aufzb}
\item aus dem Gesichtspuncte eines \RWbet{Wunders, das zur Bestätigung des Christenthumes als einer göttlichen Offenbarung dienet}.
\item aus dem Gesichtspuncte eines Ereignisses, \RWbet{das die Unsterblichkeit unserer Seele factisch erweisen soll,} \dh\ das uns ein Beispiel geben soll von Einem aus unseren Brüdern, der auch nach seinem Tode noch mit Bewußtseyn fortgewirkt, und also auch fortgelebt hat. --
\end{aufzb}
Für unseren gegenwärtigen Zweck brauchten wir eigentlich diese Begebenheit nur aus dem \RWbet{ersten} Gesichtspuncte zu betrachten; da jedoch auch der zweite von einer sehr großen Wichtigkeit ist; so will ich bei dieser Gelegenheit auch ihm einige Aufmerksamkeit schenken.\par

\vabst\textbf{A.}~Ich betrachte denn also zuerst die Auferstehung Jesu \RWbet{als ein die Wahrheit des Christenthums bestätigendes Wunder}. In dieser Hinsicht nimmt dieß Ereigniß eine der vornehmsten Stellen ein; und zwar aus \RWbet{doppeltem} Grunde: \RWbet{einmal}, weil sich von keiner Begebenheit, die in den Büchern des neuen Bundes erzählt wird, so leicht und so unwidersprechlich erweisen läßt, daß sie ein Wunder sey, als eben von dieser; \RWbet{sodann} weil in den Evangelien erzählt wird, daß Jesus Christus dieses Wunder vorausgesagt, und gleichsam als den letzten von seiner göttlichen Sendung noch zu gebenden Hauptbeweis auf das Ausdrücklichste ver\RWSeitenw{209}sprochen. Es kamen nämlich (\RWbibel{Mt}{Matth.}{12}{38}\ \Ahat{\RWbibel{Mk}{Mark.}{8}{11}}{8,10.}\ \RWbibel{Lk}{Luk.}{11}{29}) einst Pharisäer zu Jesu, ein Zeichen vom Himmel zu fordern. Dieses verweigert er ihnen mit dem (bei Matth. und Lukas angeführten) Beisatze, daß nur noch Ein Zeichen, das des Propheten Jonas ihnen gegeben werden sollte. Dieß erklärt er (bei Matth.) noch näher dahin: \erganf{Wie Jonas drei Tage und drei Nächte in dem Bauche des Seeungeheuers war: so wird auch der Sohn des Menschen drei Tage und drei Nächte im Schooße der Erde seyn.} Auch soll Jesus (\RWbibel{Joh}{Joh.}{2}{19}) bei einer ähnlichen Veranlassung gesprochen haben: \erganf{Zerstöret diesen Tempel: und in drei Tagen will ich ihn wieder herstellen.} Die Juden stellten sich zwar, als ob sie diese Worte Jesu von jenem steinernen Tempel verständen; in der Folge aber bei seiner Kreuzigung zeigte der bittere Spott (\RWbibel{Mt}{Matth.}{27}{40}, auch \RWbibel{Mt}{}{26}{61}), daß sie den richtigen Sinn derselben gar wohl gefaßt hatten. Noch deutlicher hat Jesus sich gegen die Jünger selbst (\RWbibel{Mt}{Matth.}{17}{22}) hierüber ausgesprochen: \erganf{Der Menschensohn wird in die Hände der Menschen geliefert werden, und sie werden ihn tödten; am dritten Tage aber wird er wieder auferstehen.} (So auch bei \RWbibel{Mt}{Matth.}{20}{17}\ \uaO ) Daß aber alle diese Vorhersagungen Jesu nicht erst hinterher von den Jüngern erdichtet worden seyen, (wie der Verfasser der Wolfenbüttlischen Fragmente behauptet) erhellet daraus, daß man, wenn Jesus nie von einer Auferstehung gesprochen hätte, nie auf den Einfall, eine Wache zu seinem Grabe zu setzen, gekommen wäre. Um dieser so deutlichen Vorhersagungen willen wird es nun nothwendig, zu zeigen, daß sich nach Jesu Kreuzigung wirklich etwas von der Art zugetragen habe, wie er mit solcher Gewißheit versprochen. Denn im entgegengesetzten Falle könnte man sagen, daß ein Mann, der sich in einem so wichtigen Stücke seiner Hoffnungen getäuscht, und zu Schanden geworden sey, kein Liebling Gottes, kein göttlicher Gesandte seyn könne. So hatte schon Paulus geschlossen, der (\RWbibel{1\,Kor}{1\,Kor.}{15}{14}) schreibt: \erganf{Ist Christus nicht auferstanden: so ist unsere Lehre falsch, und euer Glaube ohne Grund.} Hoffentlich werden aber nachstehende \RWbet{drei} Beweise für Jeden, der es aufrichtig mit der Wahrheit meint, zur Ueberzeugung von diesem Wunder hinreichen.~\RWSeitenw{210}\par

\vabst \textbf{I.}~\RWbet{Beweis dieses Wunders aus dem allgemeinen Glauben, den es gleich anfänglich in Palästina gefunden.}\par
Es ist unläugbar, daß die christliche Religion nach Jesu Kreuzigung durch ganz Palästina (\dh\ durch ganz Judäa, Samaria und Galiläa) ausgebreitet worden sey, und aller Orten sehr viele Anhänger gefunden habe. Unläugbar ist es auch ferner, daß die damaligen Verkündiger des Christenthums überall, wo sie nur hinkamen, die Behauptung aufstellten, daß Jesus nach seiner Kreuzigung wieder auferstanden und erschienen sey; so daß man eben darum die Predigt des Christenthums auch das \RWbet{Evangelium vom Auferstandenen} nannte. Jeder, der in die Gemeinschaft der Christen wollte aufgenommen werden, mußte sich zu dem Glauben an die Auferstehung Jesu bekennen, und eben dieß Ereigniß war, der Geschichte zu Folge, der stärkste Beweggrund, der die Meisten zur Annahme der christlichen Religion bestimmte. Wäre nun Jesus nicht wirklich erstanden, und hätte die Vorsehung nicht Alles so geleitet, daß sich ein Jeder von dem Geschehenseyn dieses Wunders mit hinlänglicher Sicherheit überzeugen konnte: so hätte dasselbe auch nimmermehr einen so allgemeinen Glauben, also die christliche Religion nie so viele Anhänger gefunden. Sey nun der eigentliche Hergang der Begebenheit gewesen, welcher er wolle: so ist doch das gewiß, daß sich die mannigfaltigsten Umstände vereinigen mußten, um die Meinung, daß Jesus auferstanden sey, so vielen Tausenden glaubwürdig zu machen. Daß sich nun diese Umstände gerade so, und nicht anders gefügt, darin liegt unläugbar der Beweis des göttlichen Wohlgefallens an Jesu und seiner Lehre; ihre Vereinigung ist als ein unläugbares Wunder zur Bestätigung des Christenthums zu betrachten. Besonders merkwürdig muß uns in dieser Rücksicht seyn
\begin{aufza}
\item der Glaube, den \RWbet{die Apostel selbst} an den Tag gelegt. Diese versicherten nämlich, daß sich ihr Herr nach seiner Kreuzigung ihnen zu wiederholten Malen lebendig dargestellt habe, von ihnen gesehen, gehört, sogar betastet worden sey, \udgl\  (\RWbibel{1\,Joh}{1\,Joh.}{1}{1--3}\ \RWbibel{1\,Kor}{1\,Kor.}{15}{4--8}). Aus Gründen, die wir schon oben angegeben, konnten die Apostel~\RWSeitenw{211}\ dieses unmöglich vorgeben, wenn sie nicht selbst daran glaubten.
\item Die Aeußerung \RWbet{Pauli} (\RWbibel{1\,Kor}{1\,Kor.}{15}{6}), daß Jesus nicht nur ihm selbst (auf jener Reise nach Damaskus) sondern auch mehr als \RWbet{fünf hundert Menschen auf einmal} erschienen sey, von denen Mehrere zur Zeit der Abfassung dieses Briefes noch gelebt haben sollen.
\item Der Glaube mehrerer \RWbet{Mitglieder des hohen Rathes}, die durch die Annahme des Christenthumes so Vieles zu verlieren hatten; die alle Umstände, wie es mit der Entstehung jener Nachricht vom Auferstandenen hergegangen sey, so leicht erfahren konnten, \usw\
\end{aufza}\par

\vabst \textbf{II.}~\RWbet{Beweis dieses Wunders aus dem Betragen des hohen Rathes gegen die Jünger Jesu.}\par
Aus der Apostelgeschichte des heiligen Lukas ist zu ersehen, daß unmittelbar nach der Kreuzigung Jesu die noch sehr kleine und furchtsame Gemeine, die er zurückgelassen hatte, höchstens aus hundert und zwanzig Personen bestehend, von Seite des hohen Rathes mit einer besonderen Schonung und Zurückhaltung behandelt worden sey. Man fordert sie nicht nur nicht vor Gericht, sondern man duldet sogar, daß sie Zusammenkünfte halten; und selbst, als sie am nächsten Pfingstfeste die Lehre vom Auferstandenen öffentlich vortragen, und drei tausend Anhänger gewinnen, schweigt man dazu. Erst als des folgenden Tages durch Petrus und Johannes sogar ein Wunder geschieht, zieht man diese Beiden gefänglich ein, findet sich aber des anderen Tages schon wieder bewogen, sie zu entlassen, bloß mit dem beigefügten Verbote, daß sie in Zukunft nicht mehr von Jesu reden sollten. Ein zweites Mal erlaubt man sich, alle Apostel gefänglich einzuziehen; entläßt sie aber nach einer kleinen Züchtigung, mit demselben Verbote, wie damals. Erst nach geraumer Zeit hierauf veranlaßt der Diakon \RWbet{Stephanus} eine Verfolgung, wobei er selbst gesteiniget, und mehrere andere Christen aus der Stadt vertrieben werden; an die Apostel aber wagt man sich nicht; sie bleiben. In der Folge läßt zwar Herodes Agrippa den \RWbet{Jakobus} (den \RWbet{Größeren}, den Bruder Johannis) enthaupten (\RWbibel{Apg}{Apostelg.}{12}{2}); und später verurtheilt der hohe~\RWSeitenw{212}\ Priester Ananus auch \RWbet{Jakobus den Kleineren} (oder den sogenannten \RWbet{Bruder Jesu}) zum Tode, wie uns Flavius Josephus erzählt (nach Eusebius \RWlat{hist.\ lib.\,2.\ cap.\,1 et 23.} soll er vom Tempel herabgestürzt worden seyn); doch thut dieß Ananus nur unter dem eigenen Vorwande, daß Jakobus von dem mosaischen Gesetze abgewichen wäre; und gleichwohl war der größte Theil des Volkes so unzufrieden über diese Hinrichtung des Mannes, daß Jener um deßwillen seine hohe Priesterwürde verliert. Auch wer die Glaubwürdigkeit der Apostelgeschichte nicht durchgängig annehmen wollte, könnte die Richtigkeit der meisten hier angegebenen Umstände schon deßhalb nicht bezweifeln, weil sie auch von anderen Schriftstellern, \zB\ von Flavius Josephus, gleichmäßig berührt werden; er müßte wenigstens zugestehen, daß der hohe Rath mit den Aposteln auf jeden Fall sehr schonend umgegangen sey, weil es ihnen sonst auf keine Weise möglich geworden wäre, das Christenthum in allen Städten des jüdischen Landes zu predigen und überall Gemeinden zu errichten. -- Es fragt sich nun, aus welchem Grunde man diese Schonung beobachtet habe? warum man nicht nach der Kreuzigung Jesu auch seine Apostel gefänglich eingezogen, und als Gehülfen eines angeblichen Volksaufwieglers zugleich mit ihm hingerichtet habe? warum man insbesondere selbst bei denjenigen dieser Personen, die man zum Tode verurtheilte, nicht den so nahe gelegenen Rechtsgrund, \anf{weil sie durch eine erdichtete Erzählung von der Auferstehung Jesu die Menge irre geführet, und sie zu abgöttischer Verehrung eines Mannes verleitet hätte, der als erwiesener Verbrecher am Kreuze starb}; sondern statt dessen einen viel unwichtigeren Vorwand gebrauchte, weil sie sich Abweichungen von dem Gesetze Mosis hätten zu Schulden kommen lassen? Hatte der hohe Rath Muth genug gehabt, Jesum, der in den Augen des Volkes doch für einen so großen Propheten und Wunderthäter gegolten hatte, der vor acht Tagen noch unter lautem Hosiannazurufe in die Stadt eingezogen war, an das Kreuz zu schlagen: was wären ihm seine zwölf unberühmten Anhänger gewesen? warum verschonte er gleichwohl diese? -- Nothwendig muß man, um diese Frage sich beantworten zu können, voraussetzen, daß sich nach Jesu Kreuzigung etwas ganz~\RWSeitenw{213}\ Außerordentliches zugetragen habe, etwas von der Art, wodurch der hohe Rath selbst in Bestürzung gerieth, und sich bewogen fühlte, von nun an glimpflicher mit den Anhängern Jesu umzugehen. Und was kann dieß Anderes gewesen seyn, als das Gerücht eines neuen Wunders, das sich zu Gunsten Jesu zutrug, als das Gerücht, daß er am dritten Tage nach seiner Hinrichtung wieder aufgelebt sey, und sich lebendig dargestellt habe? Allein wenn dieses Gerücht eine bloße grundlose Sage gewesen wäre, wie viele Mittel wären dem hohen Rathe nicht zu Gebote gestanden, diesem Gerüchte wenigstens bei jedem Vernünftigen alle Glaubwürdigkeit zu benehmen? Aber das unterließ man nicht nur; sondern statt dessen traten, wie wir schon oben bemerkt, selbst Mitglieder des hohen Rathes zum Christenthume über. Ist also nicht gerade dieses Betragen des hohen Rathes der sicherste Beweis, daß hier ein Wunder müsse obgewaltet haben? --\par

\vabst \textbf{III.}~\RWbet{Beweis dieses Wunders aus einigen Umständen, die in den Evangelien erzählt werden, und unmöglich erdichtet seyn können.}\par
Die zwei bisherigen Beweise sind, wie man sieht, ganz unabhängig von der Glaubwürdigkeit der evangelischen Berichte. Nehmen wir aber die \RWbet{Evangelien} zu Hülfe: so erhalten wir, auch wenn wir uns vor der Hand nur auf dasjenige allein verlassen wollen, was darin unmöglich eine Erdichtung seyn kann, noch einen neuen Beweis für die Wirklichkeit des Auferstehungswunders, der auf folgenden Sätzen beruhet:
\begin{aufza}
\item \RWbet{Der Leib des Gekreuzigten wurde vom Kreuze abgenommen, als aller Anschein eines schon wirklich eingetretenen Todes da war.}\par
Wenn Jemand auch in Zweifel setzen wollte, was uns das Evangelium Johannis (\RWbibel{Joh}{}{19}{31\,ff}) von jenem Hauptmanne erzählt, der die Aufsicht über die Kreuzigung Jesu hatte, daß nämlich dieser unserem Herrn die Seite mit einem Speere geöffnet habe, und daß hierauf Blut und Wasser hervorgeflossen sey: so ist doch das schon genug, daß es so viele Feinde gegeben, die den Tod Jesu wünschten, die also gewiß nicht werden zugelassen haben, daß sein Leib eher vom~\RWSeitenw{214}\ Kreuze abgenommen werde, als bis sie vernünftiger Weise nicht ferner zweifeln konnten, daß er schon wirklich todt sey.
\item \RWbet{Das steinerne Grab, worein man den Leichnam Jesu gelegt hatte, wurde mit einer Wache besetzt.}\par
Der Evangelist Matthäus erzählt uns (\Ahat{\RWbibel{Mt}{}{27}{57\,ff}}{27,55\,ff.}), daß Joseph von Arimathäa den Leichnam Jesu, den er mit Pilati Erlaubniß vom Kreuze abgenommen, in ein neues in Felsen gehauenes Grab gelegt habe; daß sich des folgenden Tages der hohe Rath bei Pilatus mit der Bitte eingestellt, daß man das Grab bewachen lassen möchte; daß dieses bewilligt worden sey, und daß man sonach das Grab zuerst versiegelt, dann eine Wache (\RWgriech{koustwd'ia}, wahrscheinlich von vier Mann) beigesetzt habe. Als nun durch das bekannte Erdbeben und die darauf erfolgte Auferstehung Jesu diese Wächter vom Grabe verscheucht worden (erzählt \RWbibel{Mt}{Matthäus}{28}{4\,ff}\ weiter), habe der hohe Rath sie bestochen, damit sie aussagen möchten, die Jünger Jesu hätten den Leichnam, während sie schliefen, gestohlen. -- Und diese Sage, heißt es zuletzt, hat sich bis auf den heutigen Tag erhalten. Diese Erzählung Matthäi kann um so weniger eine Erdichtung seyn, da allen Zeugnissen zu Folge gerade dieses Evangelium das älteste, und in Palästina selbst geschrieben worden ist, und der Verfasser sich hier auf eine Sage beruft, welche zu seiner Zeit allgemein herrschend gewesen seyn soll. Diese Sage also muß damals wirklich bestanden haben: oder man hätte den Evangelisten verspottet, daß er sich selbst und seinen Mitaposteln ein Verbrechen aufbürde, welches man ihnen noch niemals Schuld gegeben habe. War aber nie eine Wache beim Grabe Jesu: so ist einleuchtend, daß diese Sage nie hätte entstehen können, indem sonst wieder der hohe Rath sehr lächerlich gehandelt hätte, wenn er, statt zu gestehen, daß er das Grab unbewacht gelassen habe, die unwahrscheinliche Sage verbreitet hätte, daß die Jünger den Leichnam gestohlen hätten, während die Wache schlief.

\begin{RWanm}
Gleichwohl bemühte sich der Verfasser der Wolfenbüttlischen Fragmente, die Erzählung Matthäi von der Grabwache durch folgende Gründe verdächtig zu machen:~\RWSeitenw{215}
\begin{aufzb}
\item Es ist sehr unwahrscheinlich, daß sich der hohe Rath an jene von Jesu einst versprochene Auferstehung, um derentwillen er (nach der Erzählung Matthäi) eine Wache verlangt, erinnert haben sollte, da die Jünger selbst an keine Auferstehung dachten.
\item Eben so unwahrscheinlich ist es, daß der hohe Rath die Feier des Sabbaths (und zwar des heiligsten im ganzen Jahre) durch den Gang zu Pilatus und dann zum Grabe verletzt, und durch Berührung dieses Grabes sich sollte verunreiniget haben.
\item Noch unbegreiflicher, daß, wenn dieß geschehen wäre, wenn sich der hohe Rath \RWlat{in corpore} erst zu Pilatus, dann zum Grabe verfügt hätte, die Jünger Jesu nichts davon erfahren hätten, wie sie denn nichts davon wissen mußten, weil sie (nach der Erzählung der übrigen Evangelisten) am Sonntage frühe zum Grabe hinausgehen, um Jesum einzubalsamiren.
\item Klüglich ändert dabei Matthäus den Zweck dieses Ausganges der Weiber dahin ab, daß sie das Grab hätten besehen wollen.
\item Wenn das Grab Jesu wirklich bewacht worden wäre; so hätten die Apostel auf ein Verhör dieser Wache dringen, und sich ein Certificat hierüber von der Obrigkeit ausbitten sollen.
\item Endlich muß man bei dieser Annahme den ganzen hohen Rath zu einer Gesellschaft von Betrügern machen, die von der Wirklichkeit der Auferstehung Jesu vollkommen überwiesen, sich dennoch nicht zu seiner Lehre bekennen.
\end{aufzb}
Auf diese Gründe läßt sich erwidern:
\begin{aufzb}
\item Es ist nicht unerklärlich, wienach sich die Feinde des Herrn in jener Unruhe, in welche ein begangenes Verbrechen allemal versetzt, der versprochenen Auferstehung Jesu erinnern konnten, an welche zu denken die Jünger durch ihre zu große Betrübniß verhindert wurden. Und wie nun böse Menschen von Anderen immer Böses erwarten: so war es sehr natürlich, daß der hohe Rath einen Diebstahl von Seite der Jünger befürchtete, und deßhalb eine Wache verlangte.
\item Die Worte, deren sich Matthäus bedient (die Oberpriester und Pharisäer kamen bei Pilatus zusammen), zwingen uns gar nicht, vorauszusetzen, daß sich der hohe Rath \RWlat{in corpore} zu Pilatus, noch weniger aber zum Grabe verfügt habe. An den letzteren Ort wird man gewiß nur einige Abgeordnete geschickt haben; und diese konnten sich die kleine Unbequemlichkeit, welche mit der Verunreinigung durch die Berührung eines~\RWSeitenw{216}\ Grabes verbunden war, sehr leicht gefallen lassen, da es sich hier um die Erreichung eines so wichtigen Zweckes handelte.
\item Hieraus ist auch schon begreiflich, wie die Besetzung des Grabes mit einer Wache der Gemeinde Jesu habe verborgen bleiben können; besonders da sich diese aus Furcht vor den Juden jene Tage hindurch größtentheils zu Hause hielt. Endlich mußte auch dem hohen Rathe selbst daran gelegen seyn, die Wache so heimlich als möglich abzuschicken; denn von der Verheimlichung dieser Anstalt ließ sich der Vortheil erwarten, daß man die Jünger auf dem Versuche eines Diebstahls vielleicht ertappen würde, was dem hohen Rathe äußerst erwünscht seyn mußte.
\item Matthäus ist gerade derjenige Evangelist, der, wenn die Grabwache eine Erdichtung wäre, die Lüge am wenigsten aus Allen wagen konnte; denn er schrieb ja zu Palästina, er schrieb wenige Jahre nach Jesu Tode, und er beruft sich noch überdieß auf eine herrschende Sage. Doch diese Sage scheint eben diese Ursache gewesen zu seyn, die ihn bestimmte, des Daseyns der Wache zu erwähnen, von welcher die übrigen Evangelisten nur deßhalb schweigen, ~weil in die Länder, für welche sie ihre Evangelien schrieben, die Sage von der Wache gar nicht gekommen war. Der Umstand aber, daß er die frommen Frauen des Sonntags früh zum Grabe gehen läßt, dasselbe zu \RWbet{beschauen}, ist wohl nur zufällig, zumal da er zuerst sein Evangelium schrieb, und die Uebrigen, hätten sie täuschen wollen, leicht mit ihm einstimmen konnten.
\item Nimmermehr würde der politische Pilatus etlicher Fischergesellen wegen einen hohen Rath so sehr beleidigt, und ihnen ein Zeugniß ausgestellt haben, welches am Ende auch \RWbet{wider ihn selbst} ausgesagt hätte. Sie wären also nur mit Hohn zurückgewiesen worden, wenn sie um ein solches Zeugniß angesucht hätten.
\item Mehrere Mitglieder des hohen Rathes waren mit dem Verfahren des Hohenpriesters und seines Anhanges gewiß sehr unzufrieden (\RWbibel{Joh}{Joh.}{9}{16}); Einige gingen sogar zum Christenthume über. Daß aber die Mehrzahl nicht Muth genug hatte, um einen Mann, in dessen Hinrichtung man kurz zuvor eingewilliget, jetzo auf einmal für einen Wunderthäter und für einen göttlichen Gesandten zu erklären, das ist nichts Unbegreifliches. Weit unbegreiflicher ist wohl dasjenige, was der Fragmentist, um die Ehre des hohen Rathes zu retten, annimmt und annehmen muß, daß nämlich die Apostel und alle ersten~\RWSeitenw{217}\ Christen, welche den Auferstandenen gesehen, gesprochen, betastet zu haben aussagten, schamlose Lügner gewesen wären.
\end{aufzb}
\end{RWanm}

\item \RWbet{Als man die Wache zum Grabe setzte, befand sich der Leichnam noch darin.}\par
Ohne Zweifel wird der hohe Rath, wenn er die Vorsicht traf, das Grab bewachen zu lassen, sich erst versichert haben, ob das auch da sey, was er bewachen lasse. Gesetzt aber, dieß wäre nicht geschehen, oder es wäre nur möglich gewesen, in der Folge vorzugeben, daß es durch ein Versehen unterblieben wäre: so würde der hohe Rath, statt des so unwahrscheinlichen Mährchens von der schlafenden Wache, lieber die Sage ausgesprengt haben, die Jünger hätten den Leichnam entwendet, bevor man noch die Wache zum Grabe gesetzt.
\item \RWbet{Am dritten Tage war der Leichnam nicht mehr im Grabe vorhanden.} \par
Wäre er noch vorhanden gewesen: so hätte der hohe Rath gewiß nicht ermangelt, ihn nun herausnehmen zu lassen, und durch eine öffentliche Vorzeigung desselben das Vorgeben der Jünger von Jesu Auferstehung und sein eigenes Versprechen hierüber zu Schanden zu machen.

\begin{RWanm}
Im Grunde hätte zwar Jesus erstehen, und den Jüngern erscheinen können, ohne daß jener irdische Leib, den er als Mensch getragen, dazu ganz nothwendig gewesen wäre; allein in den Augen des Volkes hätte dann die Glaubwürdigkeit der Erzählung der Jünger ungemein viel verloren; so wie im Gegentheile jetzt das wunderbare Verschwinden seines Leichnames, davon sich Jeder überzeugen konnte, auch schon die Wahrheit dessen verbürgte, was die Apostel und die übrigen Christen von den Erscheinungen, deren sie Jesus gewürdiget hatte, erzählten. -- Hiezu kommt noch, daß der Leichnam Jesu auch schon deßhalb entfernt werden mußte, damit der Pöbel ihn nicht zur größten Unehre für Jesum selbst mißhandle.
\end{RWanm}

\item \RWbet{Dieses Verschwinden des Leichnams aus dem bewachten Grabe läßt sich ohne Wunder nicht erklären.}\par
Um sicher zu seyn, daß wir hier keine der etwa möglichen Erklärungsarten übergehen, will ich eine \RWbet{dichotomi}\RWSeitenw{218}\RWbet{sche} Eintheilung derselben vornehmen. Der Leib Jesu kam aus dem Grabe weg. Diese Veränderung seines Ortes bedurfte einer bewegenden Kraft; und diese kann entweder eine \RWbet{menschliche} oder \RWbet{nicht menschliche} gewesen seyn. War es die Kraft eines Menschen; so war es entweder \RWbet{Jesu eigene Kraft}, oder die Kraft \RWbet{gewisser anderer Menschen}. In diesem zweiten Falle kann man wieder nur Eines von Beiden annehmen: Die Menschen, die den Leib Jesu aus dem Grabe entfernten, thaten es entweder \RWbet{mit Wissen und Willen des römischen Landpflegers}, oder \RWbet{nicht}. Mit seinem Wissen und Willen etwa \RWbet{im Einverständnisse auch mit den Aposteln}, oder \RWbet{nicht}. Geschah es dagegen ohne Pilati Vorwissen, so geschah es entweder durch die \RWbet{Apostel selbst}, oder durch eine \RWbet{andere Person}, \zB\ Judas. Soll aber die Kraft, die den Leib Jesu entfernte, gar \RWbet{keine menschliche} gewesen seyn; so kann man, wenn man nicht seine Zuflucht etwa zu einem Engel nehmen, und also sogleich ein offenbares Wunder voraussetzen will, wohl auf nichts Anderes verfallen, als auf das \RWbet{Erdbeben}, dessen Matthäus erwähnt; dieß, muß man sagen, habe das Grab gespalten, und den Leib Jesu verschlungen.\par
Ich will nun jede dieser Hypothesen im Einzelnen prüfen.
\begin{aufzb}
\item Jesus ist durch seine eigene Kraft aus dem Grabe hervorgegangen.\par
Hier kann man wieder zweierlei annehmen; entweder daß er vorher \RWbet{am Kreuze gestorben}, oder \RWbet{nicht}.
\begin{aufzc}
\item Ist er vorher am Kreuze gestorben, am dritten Tage aber wieder lebendig geworden: so ist dieß offenbar ein Wunder; und zwar gerade dasjenige, was uns die Evangelien hier erzählen.
\item Die zweite Annahme ist, daß Jesus am Kreuze nicht wirklich gestorben, sondern in einer Art von Ohnmacht, oder in einer Verstellung, die man für wirklichen Tod hielt, vom Kreuze abgenommen worden sey, sich hierauf im Grabe, etwa durch Einwirkung jener Gewürze, die man beilegte, allmählich erholt habe, \usw\ Diese Erklärungsart der Auferstehung Jesu haben in der~\RWSeitenw{219}\ That verschiedene Gelehrte angenommen, \zB\ \RWbet{Bahrdt} in seiner Ausführung des Zweckes \RWbet{Jesu}\RWlit{}{Bahrdt1}; der Verfasser des \RWbet{Horus}; \RWbet{Wieland} im Agathodämon\RWlit{}{Wieland2}, \umA\  Ich sage aber, daß auch, wenn sich die Sache auf diese Art verhielt, sie als ein Wunder zu betrachten wäre; denn auch auf diese Art hätte sie nicht zu Stande kommen können, wenn sich nicht eine Menge sehr zufälliger Umstände gerade so, wie man sie nöthig hatte, fügten. Daß sich nun diese Umstände gerade so, und nicht anders gefügt, würde auf jeden Fall beweisen, daß Gott die Absicht gehabt, \RWbet{Jesum} als seinen Gesandten, und \RWbet{Jesu} Lehre als eine Offenbarung darzustellen. Erwäge man nur, welch ein besonderes Glück das hätte seyn müssen, daß \RWbet{Jesus} am Kreuze nicht gestorben; daß er in jener Ohnmacht oder Verstellung für wirklich todt gehalten wurde; daß diese Ohnmacht bei der Herabnahme vom Kreuze nicht gehoben, oder die Verstellung, von Niemand wahrgenommen wurde; daß er das glückliche Eintreffen aller dieser Umstände mit solcher Bestimmtheit vorausgesagt habe, \usw\
\end{aufzc}
\item Der Leichnam \RWbet{Jesu} ist auf Veranlassung \RWbet{Pilati} im Einverständnisse mit den Jüngern entwendet worden. -- \RWbet{Pilatus}, hat man gesagt, wollte sich an dem hohen Rathe, der ihm die Einwilligung zur Kreuzigung \RWbet{Jesu} beinahe abgezwungen hatte, rächen; und da die Wache unter seinem Befehle stand, so war es ihm ein Leichtes, den Leichnam aus dem Grabe entfernen zu lassen, wovon er dann die Jünger verständigte, und sie ermunterte, vorzugeben, daß \RWbet{Jesus} auferstanden und ihnen erschienen sey. \par
Ich erinnere hiegegen:
\begin{aufzc}
\item Daß erstlich \RWbet{Pilatus} selbst zu diesem Betruge keinen hinlänglichen Beweggrund hatte. Denn wenn durch diese Täuschung dem hohen Rathe ein Verdruß verursacht werden sollte: so hätte sie eben so leicht für ihn selbst äußerst gefährlich ausfallen, und dem Volke eine Veranlassung zum Aufruhr geben können. Eben so wenig konnte er vorher wissen, wie die Jünger seine Aufforderung aufnehmen würden, \usw~\RWSeitenw{220}
\item Noch weniger aber hatten die Jünger selbst einen hinlänglichen Beweggrund, zu diesem Betruge mitzuwirken. Sie waren zu furchtsam; sie hatten durch die Predigt des Christenthums nichts zu gewinnen; sie waren bereits, wie wir aus ihrer eigenen Erzählung sehen, entschlossen, zu ihrem verlassenen Gewerbe wieder zurückzukehren.
\item Auch wäre ein Betrug, welcher der Mitwisser so viele gehabt, über kurz oder lang an's Tageslicht gekommen. Und wenn dieß nicht geschehen wäre: so läge in der Vereinigung so vieler Umstände, daß sich \RWbet{Pilatus} selbst zu diesem Betruge entschlossen; daß ihm die Jünger hierin zu Willen gewesen; daß keiner von so vielen Mitwissern jemals etwas von diesem Betruge weder aus Unachtsamkeit, noch Bosheit, noch Wahrheitsliebe \udgl\  verrathen, in der Vereinigung von allen diesen so seltenen Umständen, sage ich, läge abermals ein nicht zu verkennendes Wunder.
\end{aufzc}
\item \RWbet{Pilatus} hat den Leichnam ohne Vorwissen der Jünger entfernt. -- In dieser Hypothese kann man die Jünger
\begin{aufzc}
\item einmal als bloß Betrogene ansehen, und es fragt sich nun, auf welche Weise es dem Statthalter möglich geworden sey, die Jünger glauben zu machen, daß sie den Auferstandenen bald hier, bald dort sehen, hören, betasten, \usw ? Unmöglich kann man, was uns die heil.\ Evangelisten von den Erscheinungen \RWbet{Jesu} an so verschiedenen Orten, zu so verschiedener Zeit, vor so vielen Zuschauern, auf so verschiedene Weise erzählen, für eine bloße Wirkung ihrer erhitzten Einbildungskraft erklären, ohne eine so ungewöhnliche Wirkung dieser Einbildungskraft vorauszusetzen, daß eben diese schon ein neues Wunder wäre, besonders wenn man erwägt, wie ungläubig sich die Jünger, vornehmlich \RWbet{Thomas}, bei dieser Gelegenheit bezeugten.
\item Oder sollen wir vielleicht annehmen, daß die Jünger nur zum Theile (etwa nur in Betreff der Auferstehung Jesu) Betrogene gewesen; das Uebrige aber, die Erscheinungen, selbst hinzugedichtet hätten? -- Hiebei~\RWSeitenw{221}\ bleibt unerklärlich, was sie zu dieser Dichtung bewogen? durch welche Mittel sie auch so viele andere Christen dergestalt täuschen konnten, daß \RWbet{Paulus} an eine ganze Gemeinde (\RWbibel{1\,Kor}{1\,Kor.}{15}{6}) zu schreiben wagte, \RWbet{Jesus} sey von mehr als fünfhundert Menschen auf einmal gesehen worden, deren Mehrere noch am Leben waren; und daß kein Einziger in der Gemeinde, in der er doch so viele Feinde hatte, ihm widersprochen habe? \usw\
\end{aufzc}
\item Die Jünger selbst haben den Leichnam ohne Mitwissenschaft des Landpflegers entfernt.\par
Wie? durch Bestechung der Wache? durch Raub? durch einen heimlichen Diebstahl?
\begin{aufzc}
\item Durch Bestechung? -- Wie Vieles konnten sie der Wache bieten?
\item Durch Raub? -- Hätten sie sich in ein Gefecht mit der Wache eingelassen: so hätte der hohe Rath sie auch als Räuber und Staatsverbrecher, die sich in jener Wache an der Obrigkeit selbst vergriffen hatten, gefänglich eingezogen, und bestraft. Dann wäre nimmermehr das Mährchen von den schlafenden Wächtern entstanden.
\item Durch heimlichen Diebstahl? -- Wie etwa durch einen verborgenen Gang, der in das Innere des Grabes führte? Aber ein solcher Gang hätte nicht in der Geschwindigkeit und ohne vieles Geräusch ausgehauen werden können. Und wenn man sagen wollte, daß er vielleicht schon früher da gewesen: wie war es möglich, die Oeffnung desselben in einem steinernen Grabe so genau zu versperren, daß keine Fuge, ja nicht die mindeste Spur zu bemerken war? -- Oder behauptet man, was schon der hohe Rath behauptet hatte, die Jünger wären von Außen eingedrungen, während die Wache schlief? -- Aber wie unwahrscheinlich, daß vier römische Soldaten, die auf einen so wichtigen Posten gestellt sind, sich alle zugleich sorglosem Schlafe überlassen? wie unwahrscheinlich, daß sie, selbst wenn sie dieß zu thun Willens wären, nicht~\RWSeitenw{222}\ wenigstens die leichte Vorsicht getroffen hatten, sich so zu lagern, daß sie durch Wegwälzung des Steines, und durch das Geräusch, das hiebei unvermeidlich war, geweckt würden? Warum hat endlich in allen diesen Fällen der hohe Rath die Jünger nicht in's Verhör gezogen, und wenn der geringste Verdacht auf ihnen lag, sie auf das Schärfste bestraft? -- Muß man, wenn man sich vorstellen will, daß sich die Sache wirklich auf diese Art zugetragen habe, nicht abermal eine Menge so ungewöhnlich günstiger Umstände voraussetzen, daß die Vereinigung derselben hinlänglich beweisen würde, es sey die Absicht Gottes gewesen, daß wir die Lehre \RWbet{Jesu} als eine göttliche Offenbarung annehmen sollen?
\end{aufzc}
\item Irgend ein anderer Mensch, etwa \RWbet{Judas Ischariotes}, oder \RWbet{Joseph} von Arimathäa, hat den Leichnam entfremdet?\par
Hier gelten, was die Schwierigkeit der Ausführung betrifft, die nämlichen Gründe, wie vorhin. Und nebstdem fragt es sich (wie bei c), wodurch die ersten Christen auf die Einbildung gebracht werden konnten, den Auferstandenen gesehen zu haben? \usw
\item Das Erdbeben hat den Leichnam \RWbet{Jesu} versenkt. \par
Wenn jenes Erdbeben, dessen \RWbet{Matthäus} gedenkt, das Grab gespalten, und den Leichnam so verschlungen hätte: so hätten Spuren dieser Spaltung zu sehen seyn müssen; und der hohe Rath hätte dann nicht ermangelt, das Verschwinden des Leichnames auf diese Art zu erklären, besonders da man jener Verschüttung noch eine eigene, der Ehre \RWbet{Jesu} äußerst nachtheilige Deutung (die Erde hat ihn verschlungen!) ertheilen konnte. Hat aber das Erdbeben seinen Dienst auf eine solche Weise verrichtet, daß auch nicht einmal ein Riß zu bemerken war: so leistete es durch seine Erscheinung zu so gelegener Zeit, und auf so zweckmäßige Art dem Ansehen \RWbet{Jesu} einen so wichtigen Vorschub, daß wir abermals nicht umhin können, in dieser Verfügung Gottes ein eigentliches Wunder zur Verherrlichung des Christenthumes anzuerkennen. Und so mögen wir uns~\RWSeitenw{223}\ also den eigentlichen Hergang des Ereignisses, das die in unseren Evangelien befindliche Geschichte der Auferstehung \RWbet{Jesu} veranlaßt hat, vorstellen, wie wir wollen, wir sind in jedem Falle gezwungen, etwas ganz Ungewöhnliches dabei vorauszusetzen, dieses Ereigniß ist ein Wunder.
\end{aufzb}
\end{aufza}

\vabst\textbf{B.}~\RWbet{Die Auferstehung Jesu als ein Ereigniß, das die Unsterblichkeit unserer Seele factisch erweist.}
\begin{aufza}
\item Wenn die Begebenheit der Auferstehung \RWbet{Jesu} zu einem factischen Beweise für die Unsterblichkeit unserer Seele benützt werden soll: so muß sie uns in der Person \RWbet{Jesu} das Beispiel eines Menschen darstellen, der nach seinem Tode noch \RWbet{mit Bewußtseyn} fortgewirkt, und zwar in einem Zustande \RWbet{höherer Vollkommenheit}, als es diejenige ist, deren wir Lebenden genießen, fortgewirkt habe. Zu diesem Zwecke muß also dargethan werden:
\begin{aufzb}
\item daß \RWbet{Jesus wirklich gestorben} sey, und
\item daß nach seinem Tode Erscheinungen eintraten, die sich am Füglichsten erklären lassen, wenn wir voraussetzen, daß sie \RWbet{von Jesu hervorgebracht worden} seyen vermittelst jener höheren Kräfte, die er nach seinem Tode erhielt.
\end{aufzb}
\item Wer nun nicht zweifelt, daß die Verfasser der Evangelien das, was sie von \RWbet{Jesu} Kreuzigung und Auferstehung schreiben, nach ihrem besten Wissen erzählen, der wird sich von der Richtigkeit der eben genannten zwei Voraussetzungen hinlänglich überzeugen können. Daß aber die heiligen Evangelisten vornehmlich hier mit aller nur möglichen Aufrichtigkeit erzählen, dafür lassen sich nebst jenen allgemeinen Gründen, die wir schon oben kennen lernten, noch folgende zwei besondere anführen:
\begin{aufzb}
\item daß unter ihnen in diesem Stücke offenbar \RWbet{nicht die geringste Verabredung} Statt fand, weil sie in ihren Erzählungen hier wirklich so stark von einander abweichen, daß es sehr schwer hält, sie ganz zu vereinigen. Je schwerer dieß hält, desto schwerer ist es, daß sie nicht in~\RWSeitenw{224}\ der Absicht, uns zu betrügen, geschrieben haben; sondern daß sie ein Jeder die Sache nach seinem besten Wissen, \dh\ so, wie er dieselbe durch Nachfrage und Nachforschungen erkannt zu haben glaubte, darstellen wollten.
\item Die hohe \RWbet{Gotteswürdigkeit} aller der Umstände, mit welchen diese Begebenheit, ihrer Erzählung zu Folge, verbunden war. Denn wenn wir alle Umstände, von welchen die Auferstehung und die darauf folgenden Erscheinungen \RWbet{Jesu}, nach evangelischem Berichte, begleitet waren, zusammennehmen, und ihre Schicklichkeit prüfen: so finden wir diese so groß, daß wir kaum glauben können, es wäre einem menschlichen Verstande möglich, eine so vortreffliche Geschichte zu erdichten; um so weniger können wir annehmen, daß die Verfasser der Evangelien dieses vermochten; und vollends dann nicht, wenn wir bemerken, daß die Begebenheit der Auferstehung die jetzt gerühmte hohe Vollkommenheit erst dann erreiche, wenn wir voraussetzen dürfen, daß Alles sich so zugetragen habe, wie es nicht aus dem Berichte eines \RWbet{einzelnen}, sondern nur aus der \RWbet{Vereinigung der Erzählungen aller Evangelien} hervorgeht; indem ein jeder Erzähler für sich allein noch eine Lücke zurückläßt, die erst durch das gehoben wird, was uns der andere erzählt. Da nun die Abweichungen, welche gerade in dieser Geschichte so häufig Statt finden, beweisen, daß die Verfasser sich nicht verabredet haben: so kann jene hohe Zweckmäßigkeit, die das Ereigniß erst aus der Vereinigung ihrer vier fragmentarischen Erzählungen erhält, unmöglich das Werk der Erzähler seyn, sondern sie liegt in dem Ereignisse selbst. Die Auseinandersetzung dieses Beweises muß ich ihrer zu großen Weitläufigkeit wegen hier übergehen; und nur als ein einziges Beispiel will ich der weisen Stufenfolge erwähnen, die \RWbet{Jesus} bei seinen Erscheinungen beobachtete, welche erst aus der Vereinigung aller vier Evangelien ersichtlich wird.
\end{aufzb}
\item Setzen wir aber die Aufrichtigkeit der evangelischen Erzählungen voraus: so können wir uns überzeugen,
\begin{aufzb}
\item daß \RWbet{Jesus wirklich gestorben} sey.~\RWSeitenw{225}\par
Gegen die Wirklichkeit des Todes \RWbet{Jesu} hat man zwar schon öfters eingewendet, daß sich die Frühzeitigkeit dieses Todes nicht wohl begreifen lasse, indem man weiß, daß andere Gekreuzigte öfters mehrere Tage am Kreuze gelebt; nach evangelischem Berichte aber soll \RWbet{Jesus} nur einige Stunden am Kreuze gehangen seyn, als er schon seinen Geist aufgab. Dieß wird um desto unbegreiflicher, wenn man mit dem Verfasser des Horus annimmt, daß \RWbet{Jesus} nicht so, wie die gemeine Vorstellung ist, mit Nägeln an's Kreuz geschlagen, sondern so, wie es insgemein üblich gewesen, bloß angebunden worden sey.\par
Hierauf antworte ich nun:
\begin{aufzc}
\item Mag es immerhin seyn, daß man Missethäter oft bloß mit Stricken an's Kreuz gebunden: so war das Anschlagen mit Nägeln doch auch nicht ungewöhnlich. Daß aber bei \RWbet{Jesu} wirklich das Letztere geschehen sey, setzt nicht nur \RWbet{Johannes} ausdrücklich voraus, wenn er den Apostel \RWbet{Thomas} sagen läßt (\RWbibel{Joh}{}{20}{25}): \erganf{Wenn ich nicht in seinen Händen die \RWbet{Nägelmahle} sehe, und nicht meine Finger in diese Nägelmahle lege: so werde ich es nicht glauben (daß Jesus lebe)}; sondern das setzen auch stillschweigend \RWbet{Lukas} und \RWbet{Paulus} voraus; Jener, wenn er den Auferstandenen die Jünger auf seine Hände und Füße aufmerksam machen läßt (\erganf{er wies ihnen die Hände und die Füße} \RWbibel{Lk}{}{24}{40}), damit sie daran erkennen, daß er wirklich der Gekreuzigte sey; dieser, indem er von den Wundenmahlen \RWbet{Jesu} in vielfacher Zahl spricht (\RWbibel{Gal}{Galat.}{6}{17}: \erganf{Ich trage die \RWbet{Wundenmahle Jesu} an meinem Leibe}).
\item Daß aber Gekreuzigte selbst in dem Falle, wenn sie mit Nägeln angeschlagen waren, länger am Kreuze zu leben pflegten, als es bei \RWbet{Jesu} soll geschehen seyn, beweiset freilich schon das Beispiel der beiden Schächer, wenn sie auf eine gleiche Art mit ihm gekreuziget waren, weil sie noch lebten, als man an \RWbet{Jesu} bereits kein Zeichen des Lebens vorfand.
Aber wir haben auch eigene Gründe, die uns die Möglichkeit des frühe\RWSeitenw{226}ren Hinscheidens \RWbet{Jesu} begreiflich machen; denn allerdings mußten die Lebenskräfte \RWbet{Jesu} durch Alles dasjenige, was seither mit ihm vorgegangen war, überaus erschöpft werden. Schon mehrere Tage hindurch konnte er seinem Körper nicht einmal zur Nachtzeit die nöthige Ruhe gewähren; sondern wenn er den ganzen Tag im Tempel gelehrt, mußte er beim Einbruche der Nacht aus Jerusalems Mauern flüchten, um sich in irgend einer abgelegenen Ortschaft vor den Nachstellungen seiner Feinde zu bergen. Die Nacht vor seiner Kreuzigung brachte er ganz schlaflos und in der größten Anstrengung des Geistes, in ungemeiner Unruhe und Angst (selbst bis zum Ausbruche eines blutig rothen Schweißes) zu; dann ward er von einem Richter zum anderen fortgeschleppt, und von den Soldaten mißhandelt, geschlagen, gegeißelt. Daher bewies sich denn auch die Schwäche seines Leibes nicht erst am Kreuze durch einen früheren Tod; sondern schon, als er sein Kreuz auf Golgathas Höhe hinan schleppen sollte, erlag er unter der Bürde (was von den Schächern nicht angemerkt wird). So war denn also nichts Anderes zu erwarten, als daß \RWbet{Jesus} früher verscheiden würde, als jene beiden Missethäter, die man vielleicht kurz vor ihrer Hinrichtung noch mit reichlicher Nahrung gepfleget hatte.
\item Doch dieses Alles soll nur die \RWbet{Möglichkeit} des frühzeitigen Todes \RWbet{Jesu}, noch nicht seine \RWbet{Wirklichkeit} beweisen. Für diese letztere bürgt uns (nebst dem schon oben Gesagten) vornehmlich jener Seitenstich, von welchem \RWbet{Johannes} (\RWbibel{Joh}{}{19}{31--35}) und zwar mit einer ausdrücklich beigefügten Betheuerung erzählet. Schon aus der Absicht dieser Handlung, zu prüfen, ob \RWbet{Jesus} todt sey, und im Falle er dieß nicht wäre, ihn jetzt zu tödten; dann aus dem Erfolge, daß so häufiges Blut und Wasser hervorgeflossen, können wir schließen, daß jener Stich das Herz getroffen habe. Und in diesem Falle war er offenbar tödtlich. Wäre also \RWbet{Jesus} bisher noch nicht todt gewesen: so hätte er jetzt wenigstens seinen Geist aufgeben müssen. In der That~\RWSeitenw{227}\ aber muß er schon vorher todt gewesen seyn; denn weder Ohnmacht noch Verstellung hätten verhindern können, daß dieser Stich ihm nicht einige Lebenszeichen abgenöthigt hätte, wovon doch nichts erzählt wird. Es heißt ferner, daß aus der Wunde \RWbet{Blut und Wasser} (\RWgriech{a<~ima ka`i <'udwr}) hervorgeflossen seyen; dieß Wasser mochte nun eine im Herzbeutel angesammelte Feuchtigkeit, oder der Eine von den beiden Bestandtheilen des Blutes (Serum und Lymphe) seyn, in welche sich das bereits geronnene Blut aufgelöset hatte: so ist es in jedem Falle gewiß, daß \RWbet{Jesus} nach einer solchen Verwundung sterben, oder schon todt gewesen seyn mußte. Denn nach einer Verwundung des Herzbeutels kann man das Leben nicht mehr fortsetzen, und das Geronnenseyn des Blutes ist das sicherste Kennzeichen des Todes, das die Arzneikunde noch heut zu Tage kennt. Wie gotteswürdig also, daß die Vorsehung, um uns von der Gewißheit des Todes \RWbet{Jesu} zu überzeugen, an seinem Leichname gerade diejenige Probe vornehmen ließ, die wir noch heut zu Tage als die gewisseste erkennen!
\end{aufzc}
\item \RWbet{Daß er gleichwohl noch immer fortgelebt, und zwar in einem noch weit vollkommeneren Zustande fortgelebt habe, als dieser irdische ist.}
\begin{aufzc}
\item Wie uns die Schriftsteller des neuen Bundes erzählen, so \RWbet{zeigte sich Jesus vierzig Tage hindurch} an den verschiedensten Orten des Landes, bald zu Jerusalem, bald auf dem Wege nach Emmaus, bald in Galiläa, bald an dem See zu Genesareth (oder Tiberias), bald auch auf einem Berge; zeigte sich jetzt einem Einzelnen, jetzt Mehreren zugleich, einmal sogar fünf Hunderten.
\item Er hatte die Macht, \RWbet{seine Gestalt zu ändern}. Er zeigt sich zuweilen in der Gestalt eines Fremdlings; wie seinen nach Emmaus wandernden Jüngern (\RWbibel{Lk}{Luk.}{24}{15}); oder \RWbet{Marien}, die ihn für den Gärtner hielt (\RWbibel{Joh}{Joh.}{20}{15}); und gehet alsbald aus dieser unbekannten Gestalt in die bekannte ihres ehemaligen~\RWSeitenw{228}\ Lehrers über, wie in den zwei eben angeführten Fällen. Zuweilen wieder erscheint er in einer Gestalt, die seine Jünger erschrecken und glauben macht, sie sähen nicht ihn selbst, sondern nur einen Geist vor sich; wie den Jüngern zu Jerusalem (\RWbibel{Lk}{Luk.}{27}{37}), wo er sie erst dadurch, daß er sich ihnen zu betasten gibt, und vor ihren Augen Speise zu sich nimmt, zu überzeugen sucht, daß er kein bloßer Geist sey.
\item Er hatte die Macht selbst \RWbet{durch verschlossene Thüren zu dringen}, wie dieß \RWbet{Johannes} (\RWbibel{Joh}{}{20}{19} und \RWbibel[26.]{Joh}{}{20}{26}) zweimal bemerkt. Eben so kann er sich den Blicken der Anwesenden plötzlich entziehen, ohne erst weggegangen zu seyn; wie dieß \zB\ in Emmaus (\RWbibel{Lk}{Luk.}{24}{31}) geschieht.
\item Ja es scheint sogar, als ob er \RWbet{zu gleicher Zeit an mehreren Orten erscheinen}, oder doch große Entfernungen in einer ungewöhnlich kurzen Zeit zurücklegen könne. Denn während er die nach Emmaus wandernden Jünger auf ihrem Wege begleitet, war er auch zu Jerusalem dem \RWbet{Simon Petrus} erschienen (\RWbibel{Lk}{Luk.}{24}{34}).
\item Er kann sich \RWbet{mit seinem Leibe gen Himmel erheben}, bis eine Wolke ihn den Blicken der Nachstaunenden entzieht (\RWbibel{Apg}{Apostelg.}{1}{2.\,3.\,9} vgl.\ mit \RWbibel{Lk}{Luk.}{24}{51}\ \RWbibel{Mk}{Mark.}{16}{19}\ \RWbibel{1\,Tim}{1\,Tim.}{3}{16}\ \uma ).
\item Endlich nachdem er auch gen Himmel aufgestiegen, \RWbet{erscheint} er nach mehreren Jahren wieder unter Begleitung von Blitz und Donner, und dieß zwar einem Verfolger der Christen, \RWbet{Saulus}; erscheint ihm so, daß nur dieser allein, nicht aber die Umstehenden seine Gestalt sehen, und seine Stimme hören (\RWbibel{Apg}{Apstg.}{9}{4\,ff}).
\end{aufzc}
\end{aufzb}
\item Es fragt sich nun, \RWbet{wie dieses Alles sich am Natürlichsten erklären lasse?} Und darauf antworte ich: Nur durch die Annahme, daß wir Menschen auch nach dem Tode fortdauern, und zwar die Besseren aus uns in einem Zustande erhöhter Vollkommenheit, durch die es uns unter gewissen Umständen allerdings möglich wird, den noch auf Erden Lebenden zu erscheinen, \dh\ solche Eindrücke auf~\RWSeitenw{229}\ ihre Sinne zu machen, daß sie uns vor sich zu sehen, zu hören, zu betasten glauben, \usw\ Daß diese Annahme hinreichend sey, um alle oben erzählten Erscheinungen \RWbet{Jesu} zu erklären, leuchtet von selbst ein. Daß sie aber auch wahrscheinlicher als jede andere Annahme sey, behaupte ich aus folgenden Gründen:
\begin{aufzb}
\item Einmal, daß unsere Seele auch nach dem Tode fortdauere, und zwar bei allen Guten in einem Zustande höherer Vollkommenheit fortdauere, das ist aus Gründen der Vernunft sowohl, als auch der Offenbarung außer Zweifel. Daß aber dieser Zustand höherer Vollkommenheit unter Anderem auch darin bestehe, \RWbet{daß wir unter gewissen Umständen auf unsere Mitmenschen auf Erden verschiedentlich einwirken können}, ist wenigstens überaus wahrscheinlich; unter Anderem auch schon darum, weil aus einer solchen Einrichtung so manches Gute hervorgehen kann.
\item Da man in neuerer Zeit die Entdeckung gemacht hat, daß es in der That gewisse, uns selbst auf Erden schon zu Gebote stehende Kräfte gebe, wodurch wir auf das Gemüth unserer Mitmenschen einwirken können, ohne der sonst gewöhnlichen Werkzeuge zu bedürfen: warum sollte es höheren Wesen, dergleichen wir nach dem Tode werden, nicht möglich seyn, gleichfalls gewisse Vorstellungen in uns hervorzubringen, ohne daß eben dieselben Gegenstände, durch welche diese Vorstellungen geweckt werden, vorhanden seyn mußten?
\item Aus dieser Annahme läßt sich am Besten erklären, warum unser Herr bei seinen Erscheinungen offenbar ein gewisses Gesetz der Stufenfolge beobachtet, und das Gemüth derjenigen, denen er sich darstellen wollte, immer auf irgend eine Art erst dazu vorbereitet habe (s.\ \RWbibel{Joh}{Joh.}{20}{15}\ \RWbibel{Lk}{Luk.}{24}{15}\ \Ahat{\RWbibel{Joh}{Joh.}{21}{6\,ff}}{21,26\,ff.}\ \RWbibel{Apg}{Apostelg.}{9}{4\,ff}\ \udgl ).
\item Jede andere Erklärung däucht mir gewaltsamer zu seyn. Wenn man \zB\ annehmen wollte, daß \RWbet{Jesus} diese Erscheinungen \RWbet{als ein noch wirklich lebender Mensch hervorgebracht habe}: kann man da wohl erklären, wie es geschehen sey, daß er am Kreuze nicht~\RWSeitenw{230}\ gestorben; daß er aus dem versperrten Grabe lebend entkommen; von seinen am Kreuze empfangenen Wunden in so kurzer Zeit glücklich geheilt worden; verschlossene Thüren durchdrungen; plötzlich gekommen, plötzlich verschwunden; an so entlegenen Orten fast zu derselben Stunde erschienen; endlich gen Himmel aufgestiegen sey? Hätte er sich auf diese Weise nicht wenigstens einiger absichtlicher Täuschungen bedienen müssen? und vertragen sich diese mit der Vortrefflichkeit seines Charakters, wie wir denselben aus anderen Umständen kennen? -- Und noch viel weniger läßt sich gedenken, daß irgend ein anderer Mensch an \RWbet{Jesu} Statt den Betrug ausgeführet habe, daß er entweder die Person \RWbet{Jesu} nachgeahmt, oder gewisse optische Täuschungen angewandt habe; \udgl\ 
\item Wahr ist es, daß auch bei unserer Annahme einige Schwierigkeiten sich finden. Man könnte fragen, wie \RWbet{Jesus} bei einem nicht mehr sterblichen, sondern \RWbet{verklärten Leibe} noch Speise zu sich nehmen mochte? man dürfte fragen, ob denn die Seligen auch Fleisch und Bein haben, \udgl ? Hierauf entgegne ich aber, der verklärte Heiland habe alles dasjenige gethan, was immer nöthig war, um seine Jünger nach dem Begriffe, welchen ihr Zeitalter hatte, auf das Vollkommenste davon zu überzeugen, daß er noch lebe, und nicht ein bloßes Gespenst, sondern daß er selbst es sey, was sie vor sich sehen, \usw\
\end{aufzb}
\end{aufza}

\ctaddtocontents{ptoc}{\protect\vspace{2ex}}
\RWwiepar{VII.}{Messianische Weissagungen.}
\RWpar{68}{Begriff und Beweiskraft messianischer Weissagungen im Allgemeinen}
Eine eigene Classe von Wundern, die zur Bestätigung des Christenthums, besonders für Gelehrte, sehr geschickt sind, bilden die messianischen Weissagungen. Unter diesen verstehe ich aber gewisse (vornehmlich in den heiligen Büchern des alten Bundes enthaltene) schriftliche Aufsätze, Erzählungen oder Verfügungen, die einer solchen Auslegung fähig sind,~\RWSeitenw{231}\ daß eine auffallende Aehnlichkeit zwischen dem, was sie nach dieser Auslegung bedeuten, und den bekannten Schicksalen \RWbet{Jesu} von Nazareth zum Vorscheine kommt; eine Aehnlichkeit der Art, daß man sich des Gedankens nicht erwehren kann, die Vorsehung habe nur darum von Einer Seite veranstaltet, daß diese Aufsätze geschrieben worden, und von der anderen Seite, daß sie an \RWbet{Jesu} in Erfüllung gegangen sind, um uns den Letzteren als den von aller Welt erwarteten Messias darzustellen. Um die beweisende Kraft dieser Weissagungen zu fühlen, braucht man nichts Anderes vorauszusetzen, als:
\begin{aufza}
\item Daß die sogenannten Bücher des alten Bundes bereits \RWbet{vor Jesu Zeiten} vorhanden gewesen, und jene Stellen, die wir als messianische Weissagungen aus ihnen anführen, \RWbet{wirklich enthalten} haben. -- Diese Voraussetzung kann man nun darum mit aller Sicherheit thun, weil nicht zu gedenken ist, daß es den Christen hätte gelingen sollen, in den heil.\ Kanon der Juden, den sie in so vielen Abschriften und mit einer solchen Sorgfalt verwahrten, und darin jede Zeile, ja jeder Buchstabe abgezählt war, Stellen und um so weniger ganze Bücher zu Gunsten des Christenthums einzuschieben.
\item Daß jene Schicksale und Thaten Jesu, die wir mit gewissen Stellen des alten Bundes vergleichen, ihre \RWbet{historische Richtigkeit} haben. -- Von diesem Umstande überzeugt dasjenige, was oben von der Glaubwürdigkeit der evangelischen Geschichte in ihrer Hauptsache, und in den Nebenumständen beigebracht worden ist, zu geschweigen, daß einige dieser Schicksale \RWbet{Jesu} selbst aus Profanscribenten gewiß und unläugbar sind.
\item Die Uebereinstimmung zwischen jenen Stellen des alten Bundes und diesen Schicksalen oder Thaten des Herrn muß so \RWbet{auffallend} seyn, daß der Gedanke sich \RWbet{von selbst aufdringt}, Gottes allwaltende Vorsehung habe eine solche Uebereinstimmung nur in der Absicht veranstaltet, damit der Letztere als ein wahrer göttlicher Gesandte, und überdieß ein solcher, der vor Jahrhunderten schon verheißen war, erscheine. -- Wenn diese Absicht Gottes auch nicht bei einer jeden~\RWSeitenw{232}\ einzelnen Stelle für sich allein unläugbar wäre: so wird sie doch für Jeden, der diese Stellen alle in Vereinigung betrachtet, unverkennbar; und man erstaunet in der That, in den Büchern des alten Bundes eine Art Evangelium -- eine Geschichte der Schicksale und Thaten \RWbet{Jesu} -- vor \RWbet{Jesu} aufgesetzt zu finden!
\end{aufza}
\begin{RWanm}
Nicht nothwendig dagegen ist die Voraussetzung, daß die Verfasser der heil.\ Bücher des alten Bundes bei Niederschreibung jener Stellen, die wir auf \RWbet{Jesum} deuten, \RWbet{immer auch selbst an ihn gedacht}, und mit Bestimmtheit vorausgewußt haben, daß sie jetzt eine Weissagung schreiben, und \RWbet{wie} eigentlich dieselbe an \RWbet{Jesu} einst in Erfüllung gehen werde. Nein, jene Stellen bleiben, auch wenn sich hie und da gar nicht erweisen ließe, daß die Verfasser derselben an den Messias gedacht, immer noch messianische Weissagungen. In einem solchen Falle wäre nur um so bewunderungswürdiger die Weisheit der göttlichen Vorsehung, die sich des Menschen auch als eines \RWbet{blinden} Werkzeuges zur Ausführung ihrer Pläne zu bedienen weiß. Uebrigens halte ich diese Bemerkung deßhalb für wichtig, weil es sehr viele Stellen des alten Bundes gibt, die eine auffallende Aehnlichkeit mit den Schicksalen \RWbet{Jesu} haben, von welchen sich gleichwohl schwer darthun ließe, daß ihre Verfasser wirklich an \RWbet{Jesum} gedacht. Aus diesem Grunde wurden dergleichen Stellen von neueren Theologen häufig aus der Reihe der messianischen Weissagung ausgestrichen; Andere behielten sie zwar noch bei, konnten sie aber nur mit vieler Mühe und nicht bis zur hinlänglichen Befriedigung für den Zweifler rechtfertigen. Nach dieser Ansicht aber ist es nicht nöthig, sich in diese so äußerst schwierige Untersuchung einzulassen; und jene Weissagungen können gleichwohl getrost als solche beibehalten werden. Eine ähnliche Meinung äußerte auch schon \RWbet{Peutinger} (Religion, Offenbarung und Kirche. Salzburg, 1795. S.\,372.).
\end{RWanm}

\RWpar{69}{Aufzählung der merkwürdigsten messianischen Weissagungen, und zwar \newline A.~in den Büchern Mosis}%NEUE ZEILE VOR A.?
Der Raum verstattet nicht, eine vollständige Uebersicht \RWbet{aller} Weissagungen zu geben, die in den Büchern des alten~\RWSeitenw{233}\ Bundes, um so weniger auch derjenigen, die \RWbet{anderwärts} in Hinsicht auf \RWbet{Jesum} vorkommen. Ich hebe hier also nur die \RWbet{merkwürdigsten} aus, und ordne sie nach den Büchern, in welchen sie erscheinen. In Rücksicht dessen, was bei Profanscribenten vorkommt, ist die allgemeine Erwartung der Völker, daß um jene Zeit, in welcher \RWbet{Jesus} geboren ward, ein mächtiger König und Weltbeglücker auftreten werde, am Merkwürdigsten (s.~\RWlat{Sueton.\ in Vesp.\ c.\,4. Tacit.\ histor.\ l.~V.\ c.\,13.}\RWlit{}{Tacitus2}). Auch in der Gegend, woher die Magier kamen (\RWbibel{Mt}{Matth.}{2}{1}), muß man diese Erwartung genähret haben. Und in den verloren gegangenen \RWbet{sibyllinischen Büchern} (diejenigen, die wir jetzt haben, sind von Christenhand unterschoben), mag nach \RWbet{Josephus, Justinus, Theophilus Antiochenus} und \RWbet{Clemens Alexandrinus} zu urtheilen, etwas dergleichen gestanden haben. Was aber die Bücher des alten Bundes, und unter diesen zuerst die \RWbet{Bücher Mosis} betrifft: so hat man von jeher schon
\begin{aufza}
\item die Stelle \RWbibel{Gen}{1\,Mos.}{3}{15}\ bemerkenswerth gefunden. Hier nämlich spricht Gott zu einer Schlange, welche die Eva verführte: \erganf{Ich will Feindschaft setzen zwischen dir und dem Weibe, zwischen deinem und ihrem Samen. Er wird dir einst den Kopf zertreten, während du seiner Ferse mit List nachstellen wirst.} Unter dieser Schlange kann man begreiflicher Weise den Verführer zum Bösen, und also den Urheber alles menschlichen Elendes, den Teufel, verstehen, unter dem Samen des Weibes aber \RWbet{Jesum}, der die Macht des Teufels besiegt hat. -- Selbst der berühmte Rabbi \RWbet{Maimonides} im zwölften Jahrhunderte gestand, daß sich ein tiefes Geheimniß in dieser wundersamen Stelle ausspreche.
\item Noch merkwürdiger ist aber dasjenige, was Gott dem \RWbet{Abraham} (\RWbibel{Gen}{1\,Mos.}{12}{2}) gesagt hat: \erganf{Ich will ein großes Volk von dir entstehen lassen, \textsymmdots\ ja durch dich sollen mir einst alle Geschlechter der Erde beglücket werden.} Dasselbe wird später noch (\Ahat{\RWbibel{Gen}{}{18}{18}}{18,17.}\ und \RWbibel{Gen}{}{22}{18}) wiederholt: \erganf{Durch Einen deiner Nachkommen (im Hebräischen heißt es: in deinem Samen) sollen einst alle Völker der Erde beglücket werden, und dieses, weil du mir gehorchet hast.} Diese Weissagung ist nun durch \RWbet{Jesum} buchstäblich in Erfüllung~\RWSeitenw{234}\ gegangen; denn bekanntlich stammte er als ein Jude von \RWbet{Abraham} ab; und ist der Beglücker der ganzen Menschheit.
\item Diese dem \RWbet{Abraham} geschehene Verheißung wiederholte Gott auch dem \RWbet{Isaak} (\RWbibel{Gen}{1\,Mos.}{26}{4}) und dem \RWbet{Jakob} (\RWbibel{Gen}{1\,Mos.}{28}{14}). Der sterbende \RWbet{Jakob} aber übertrug diesen Segen auf seinen viertgebornen Sohn \RWbet{Juda} (\RWbibel{Gen}{1\,Mos.}{49}{8}): \erganf{Dich, \RWbet{Juda}! werden deine Brüder preisen, verehren werden dich die Söhne deines Vaters. Ein junger Löwe ist \RWbet{Juda} -- -- -- nie wird es an Königen aus \RWbet{Juda} fehlen, nie an Gesetzgebern von ihm, bis jener Friedensbringer (Schilo) erscheinet, welchem die Völker gehorchen werden. Sein Füllen wird er an einen Weinstock binden, und sein Gewand im Blute der Trauben waschen.} -- In dieser Stelle, wenn wir sie anders so recht übersetzt haben, wird aus dem Stamme \RWbet{Juda} ein Mann versprochen, der alle Völker friedlich beherrschen soll, welches an \RWbet{Jesu} im buchstäblichsten Sinne wahr geworden, oder noch wahr werden wird. Auch wird versprochen, daß dieser Mann nicht eher erscheinen werde, als bis alle Könige und Gesetzgeber aus \RWbet{Juda} aufhören würden. Dieses geschah nun wirklich um die Zeit \RWbet{Christi.} Bis dahin nämlich waren die Beherrscher der Juden immer aus dem Stamme \RWbet{Juda}; denn \RWbet{David} war aus dem Stamme \RWbet{Juda}; seine Nachkommen aber herrschten bis zur babylonischen Gefangenschaft immer wenigstens über einen Theil des Landes; und \RWbet{Zorobabel}, der die aus der babylonischen Gefangenschaft zurückgekehrten Juden (nicht zwar als König, aber doch als Oberhaupt) beherrschte, war aus dem Stamme \RWbet{Juda} und aus \RWbet{David's} Geschlechte. -- Zur Zeit der Makkabäer oder der Hohenpriester endlich herrschte eigentlich der hohe Rath, dessen vornehmste Mitglieder wieder aus dem Stamme \RWbet{Juda} waren.

\begin{RWanm} 
Doch dürfen wir uns nicht verhehlen, daß diese Stelle auch eine andere Uebersetzung leide: nie wird es an Königen aus \RWbet{Juda} fehlen, nie an Gesetzgebern aus ihm, so lang er Kinder haben wird. -- Aber bleibt nicht immer schon dieses merkwürdig, daß so viele Rabbinen die Stelle wirklich nur von dem Messias auslegen?
\end{RWanm}


\item Von dem Genusse des \RWbet{Osterlammes} wird (\RWbibel{Num}{4\,Mos.}{9}{12}) der Befehl ertheilt, demselben \RWbet{kein Bein} zu bre\RWSeitenw{235}chen. -- Dieser Befehl, der so nachdrücklich eingeschärft wurde, erhielt eine merkwürdige Deutung, als man den Herrn vom Kreuze abnehmen wollte. Denn damals wurden jenen beiden Schächern zu seiner Rechten und Linken die Beine gebrochen; ihm aber, weil man ihn bereits todt gefunden, wurde kein Bein gebrochen.
\end{aufza}

\RWpar{70}{B.~In den Psalmen}
\begin{aufza}
\item \RWbibel{Ps}{Ps.}{2}{}:
\end{aufza}\par
\erganf{Was toben die Heiden, was wähnen\\
Die Völker für Tand? \RWbet{Es erheben} \\
\RWbet{Ihr Haupt die Herrscher der Erde,}\par
\RWbet{Es gehen die Fürsten zu Rath}\\
Vereiniget wider Jehova,\\
\RWbet{Und wider Jehova's Gesalbten:}\\
Laßt ihre Band' uns zerreißen,\par
Vom Nacken uns schleudern ihr Joch!\\
Doch lacht nur; der thronet im Himmel,\\
Mit Spott auf sie schauet Jehova.\\
Einst spricht er im Zorne mit ihnen,\par
Vernichtet im Grimme sie einst.\\
Denn \RWbet{König gesalbt bin von Ihm ich}\\
\RWbet{Auf Sion dem heiligen Berge!}\\
\RWbet{Und künd' euch Jehova's Gebot an.}\par
Er sprach zu mir: \RWbet{Du bist mein Sohn,}\\
\RWbet{Heut zeugt' ich dich! Heisch', und du erbest}\\
\RWbet{Die Völker, Dir eign' ich den Erdkreis.}\\
Sie bändige mit eiserner Keule\par
Zerschmett're wie ird'nes Gefäß sie. --\\
So werdet, o Herrscher! denn weise;\\
Laßt, Erdenrichter! euch warnen;\\
Verehret Jehova mit Schauer,\par
Erfreut euch mit Zittern, \RWbet{und küßt}\RWfootnote{d.\,i.\ huldiget ihm.}\\
\RWbet{Den Sohn}, damit er nicht zürne,\\
Ihr nicht auf dem Wege verderbet.\\
Sein Zorn wird plötzlich entflammen:\par
\RWbet{O selig, die fliehen zu ihm!}}~\RWSeitenw{236}\par

Mag es immer seyn, was einige Schriftausleger behaupten, daß der Verfasser dieses Psalmes (David oder wer es sonst war) mit keiner Sylbe an den Messias gedacht habe, daß die Ausdrücke: Du bist mein Sohn \usw , nicht ungewöhnliche Orientalismen seyen; gewiß ist doch, daß dieser Psalm in einem weit höheren Sinne, als auf irgend einen irdischen König, auf \RWbet{Jesum} passe. Es ist also eine absichtliche Veranstaltung Gottes, daß jener Dichter sich gerade solcher Ausdrücke bediente, welche auf \RWbet{Jesum} einst so gut angewendet werden könnten.

\begin{aufza}\setcounter{enumi}{1}
\item \Ahat{\RWbibel{Ps}{Ps.}{16}{5}}{16,8.}:\par
\erganf{Du Gott bist Erbtheil mir und Becher!\par
Du hast mir ausgewählt mein Loos!\par
Es fiel mir zu der Erde schönstes;\par
Mir ward ein glänzend Eigenthum.\par
Drum preis' ich Gott, der mir gerathen;\par
Auch Nachts wallt meine Brust ihm zu.\par
Gott hab' ich alle Zeit vor Augen;\par
Er steht mir bei; ich wanke nicht!\par
Es hüpft mein Herz, mein Geist frohlocket;\par
Mein Fleisch auch ruhet hoffnungsvoll.\par
\RWbet{Denn nicht verlässest Du im Grabe,}\par
\RWbet{Nicht lassest Deinen Heiligen Du}\par
\RWbet{Verwesung schau'n; Du zeigst die Wege}\par
\RWbet{Zum Leben mir.} O! Ueberfluß\par
Der Freuden ist vor Deinem Antlitz,\par
Zu Deiner Rechten ew'ge Lust!}\par
\end{aufza}\par

Möchte es wahr seyn, daß der Verfasser dieses Psalmes nur die allen Tugendhaften gemeinschaftliche Hoffnung einer Auferstehung, dichterisch habe ausdrücken wollen: so ist doch gewiß, daß an \RWbet{Jesu} so wörtlich, wie sonst an Niemand in Erfüllung gegangen sey, was er hier sagte.

\begin{aufza}\setcounter{enumi}{2}
\item \RWbibel{Ps}{Ps.}{22}{}:\par
\erganf{\RWbet{Mein Gott! mein Gott!} \RWbet{ach, warum hast du mich verlassen?}\par
\begin{aufzb}\item[]
Ach warum tönt mein klagendes Geschrei\\
So von meinem Erretter entfernt?~\RWSeitenw{237}
\end{aufzb}\par
Mein Gott! des Tages fleh' ich; aber Du nicht einmal\par
\begin{aufzb}\item[]
Antwortest mir; des Nachts fleh' ich zu Dir;\par
Aber nie wird mein Jammer gestillt.\par
\end{aufzb}\par
Doch, Heiligster! thronst Du in Israels Gesängen.\par
\begin{aufzb}\item[]
Auf Dich vertrauten uns're Väter stets;\par
Und die Trauenden rettetest Du!\par
\end{aufzb}\par
Sie schrieen hinauf zu Dir, und fanden bei Dir Hülfe,\par
\begin{aufzb}\item[]
Sie hofften unerschüttert stets zu Dir,\par
Und nicht wurden zu Schanden sie je.\par
\end{aufzb}\par
\RWbet{Ich aber bin ein Wurm, nicht mehr ein Mensch zu nennen!}\par
\begin{aufzb}\item[]
\RWbet{Der Leute Spott, des Volk's Verachtetster.}\par
\RWbet{Alle höhnen mich, die mich nur seh'n!}\par
\end{aufzb}\par
\RWbet{Sie ziehen krumm die Lippen, schütteln mit dem Haupte:}\par
\begin{aufzb}\item[]
\RWbet{Auf Gott verließ er sich? Der helf' ihm nun,}\par
\RWbet{Der befrei' ihn, wenn er ihn liebt!}\par
\end{aufzb}\par
Du hast entzogen mich dem Leibe meiner Mutter,\par
\begin{aufzb}\item[]
An ihren Brüsten warst Du mein Vertrau'n,\par
Ihrem Schooße entfiel ich auf Dich!\par
\end{aufzb}\par
Du warst mein Gott vom ersten meiner Lebenstage;\par
\begin{aufzb}\item[]
Sey mir auch jetzt nicht fern; denn nah' ist Angst:\par
Und kein Einziger ist, der mir hilft!\par
\end{aufzb}\par
\RWbet{Ich bin umringt von großen Farren, eng umschlossen}\par
\begin{aufzb}\item[]
\RWbet{Von Basans Rindern; Löwen dräuen mir,}\par
\RWbet{Brüllen, öffnen den Rachen nach mir!}\par
\end{aufzb}\par
Wie Wasser bin ich ausgeschüttet; \RWbet{aufgelöset}\par
\begin{aufzb}\item[]
\RWbet{Ist alles mein Gebein!} Mein Herz zerschmilzt,\par
Mir im Innersten schmilzt es wie Wachs!\par
\end{aufzb}\par
Gebrannten Scherben gleich vertrocknen meine Kräfte,\par
\begin{aufzb}\item[]
\RWbet{Am Gaumen klebet meine Zunge mir:}\par
Ach! du legst in den Staub mich des Tod's!\par
\end{aufzb}\par
\RWbet{Denn mich umgibt ein Schwarm von mörderischen Hunden;}\par
\begin{aufzb}\item[]
\RWbet{Der Frevler Rotte lagert sich um mich;}\par
\RWbet{Händ' und Füße durchgraben sie mir!}~\RWSeitenw{238}\par
\end{aufzb}\par
\RWbet{O! zählen könnte man all mein Gebein!} Sie sehen\par
\begin{aufzb}\item[]
Mich so, und sehen ihre Lust an mir!\par
\RWbet{Meine Kleider vertheilen sie sich.}\par
\end{aufzb}\par
\RWbet{Und werfen Loos um mein Gewand!} Allein, Jehova!\par
\begin{aufzb}\item[]
Sey jetzt nicht ferne! Meine Stärke Du!\par
Komm und eile zu Hülfe mir jetzt!\par
\end{aufzb}\par
Entreiß' den Schwertern meine Seele! frechen Händen\par
\begin{aufzb}\item[]
Sie! meine Einzige! vom Schlund des Leuen\par
Rette mich, und den Hörnern des Stier's!\par
\end{aufzb}\par
Dann will ich Deinen Preis vor meinen Brüdern singen,\par
\begin{aufzb}\item[]
Will rühmen Dich in den Versammlungen:\par
Lobt ihn, die ihr Jehova verehrt!\par
\end{aufzb}\par
Ihr, alle Söhne Jakobs! ehret ihn, verehret\par
\begin{aufzb}\item[]
Den Herrn, ihr alle Kinder Israels!\par
Denn er sieht nicht verschmähend herab,\par
\end{aufzb}\par
Nicht ekelnd sieht er auf des Hülfelosen Leiden,\par
\begin{aufzb}\item[]
Verhüllet nicht vor ihm sein Angesicht;\par
Nein, er höret sein flehend Geschrei!\par
\end{aufzb}\par
So preis' ich Dich vor der Versammlung großer Schaaren,\par
\begin{aufzb}\item[]
So zahl' ich freudig der Gelübde Schuld\par
Dir vor Deinen Verehrern, o Gott!\par
\end{aufzb}\par
Dann essen satt sich die Bedrängten, dann lobsinget\par
\begin{aufzb}\item[]
Dem Höchsten, wer ihn sucht; auf ewig, dann\par
Werden eure Herzen erquickt.\par
\end{aufzb}\par
Dann denket an Jehova jeder Erdbewohner,\par
\begin{aufzb}\item[]
Bekehret sich zu ihm; vom Heidenvolk\par
Beten alle Geschlechter ihn an}; \usw\par
\end{aufzb}
\end{aufza}

Man sagt, daß \RWbet{David} diesen Psalm vielleicht nur auf sich selbst gefertigt habe, als er sich in irgend einer Drangsal befand. Mag seyn; so ist es doch Gottes Veranstaltung, daß \RWbet{David} in diesem Gemälde seiner Leiden so viele Züge anbrachte, die so genau an \RWbet{Jesu} in Erfüllung gingen.

\begin{aufzb}
\item Die Anfangsworte sind eben dieselben, die \RWbet{Jesus} am Kreuze ausrief.
\item Die Spottreden der Feinde wörtlich dieselben, welche die Schriftgelehrten und Pharisäer gebrauchten.
\item Auch \RWbet{Jesus} dürstete am Kreuze, und~\RWSeitenw{239}
\item seine Gebeine (die Ribben) waren so auseinander gezerrt, daß man sie -- zählen konnte.
\item Man theilte sich unter seine Kleider, und warf das Loos um seinen Ueberrock.
\item Hände und Füße wurden nicht \RWbet{David}, wohl aber ihm durchbohrt.
\item Seine wunderbare Errettung diente zu Gottes Verherrlichung, wie sonst noch keines anderen Menschen Rettung \usw\
\end{aufzb}

\begin{aufza}\setcounter{enumi}{3}
\item Im \Ahat{\RWbibel{Ps}{Ps.}{41}{8\,ff}}{41,10\,ff.}\ sagt \RWbet{David} im buchstäblichen Sinne wohl nur von sich selbst: \erganf{Mir fluchen sie: Weil er nun liegt; so komme er nicht mehr auf! -- und gegen mich erhebt er selbst, der mein Brod aß, dem ich vertraute, mein Freund, die Ferse.} -- Nach \Ahat{2\,Sam.}{2\,Kön.}\ (\RWbibel{2\,Sam}{}{17}{1\,ff}) gab \RWbet{Ahitophel, David's} Feldherr und Freund, seinem Sohne \RWbet{Absolon} den Rath, den König in der Nacht zu überfallen und zu tödten. Als sein Vorschlag vereitelt wurde, erhing er sich. -- Wie viele Aehnlichkeiten in diesem ganzen Ereignisse mit \RWbet{Judas}, dem Verräther! Soll man nicht sagen dürfen, daß jener ein Vorbild von diesem gewesen?
\item \RWbibel{Ps}{Ps.}{55}{13}: \erganf{Schändete mich ein Feind, ich duldet' es; trotzte ein Hasser mir nur: vor ihm verbärg' ich mich. Aber du, geehrt wie ich, mein Rath und mein Vertrauter, mit dem ich Speise oft genoß, mit dem ich von singenden Chören umgeben oft wallete zu Gottes Haus!} -- -- \par
Wer sieht nicht, daß diese Worte an \RWbet{Jesu} und seinem Verräther Judas genau in Erfüllung gegangen sind?
\item \RWbibel{Ps}{Ps.}{69}{}: \erganf{Gott! errette mich Du! mir dringt die Fluth an das Leben; ich sink' in tiefen Schlamm hinab; kein Grund ist hier. -- -- Als Haare meines Haupts sind mehr, die mich ohne Ursache hassen; die Widersacher sind zu mächtig, die ungereizt mir dräu'n! Was ich nicht geraubt, soll ich erstatten. -- Mache doch nicht zu Schanden an mir, die Dir vertrauen! denn ich dulde Beschimpfung um Deinetwillen! -- Fremd selbst meinen Brüdern bin ich! -- Mich verzehrt der Eifer für Dein Haus, und über mich fällt die Lästerung Deiner Entehrer. -- Sie reichen statt Speise Galle mir, und laben durch Essig mich in meinem Durste. Doch~\RWSeitenw{240}\ ihre Tafel soll ihnen ein Netz zur Wiedervergeltung werden, öde sey einst ihr Schloß! verlassen von Allen und einsam seyen ihre Hütten! Mich aber setzet hoch empor mein Retter; ich preis' ihn dann mit Dankgesängen, die besser ihm gefallen, als Stiere mit gespaltener Klau' und hohem Horn.}
\end{aufza}\par
Dieser Psalm wird dem Könige \RWbet{David} zugeschrieben. Gesetzt nun, er habe nur seine eigenen Leiden darin beschrieben; so ließ doch die göttliche Vorsehung nicht ohne Absicht es geschehen, daß er sich solcher Ausdrücke bediente, welche in einem noch höheren Sinne, als auf ihn selbst, auf \RWbet{Jesum} passen. \RWbet{Jesus} ward
\begin{aufzb}
\item in der That ohne alle Ursache gehaßt; litt
\item in ganz ausnehmendem Sinne für Andere, und
\item um Gottes willen; ward
\item selbst von Brüdern und Schwestern verkannt;
\item der Eifer für das Haus Gottes verzehrte ihn; er wurde
\item mit Galle und Essig getränkt, welches dem \RWbet{David} nie wiederfuhr;
\item seine Errettung veranlaßte Gefühle des Dankes und Verehrungsweisen Gottes, die ihm weit besser gefallen, als alle Schlachtopfer des alten Bundes; und
\item die Strafen, welche die Juden um der Verwerfung \RWbet{Jesu} willen erfuhren, werden hier sehr buchstäblich vorausgesagt. Um das Osterlamm zu essen, versammelten sie sich zu Jerusalem, als \RWbet{Titus} sie einschloß, ihre Tafel also ward ihnen ein Netz der Wiedervergeltung! Verödet wurde dann ihr Tempel, \usw\
\end{aufzb}

\begin{aufza}\setcounter{enumi}{6}
\item \RWbibel{Ps}{Ps.}{72}{}:\par
\erganf{Verleih', o Gott! dem Könige dein Gericht,\par
\begin{aufzb}\item[]
Dem Königssohne Deine Gerechtigkeit,\par
Daß Recht er spreche Deinem Volke,\par
Richte nach Billigkeit Deine Armen.\par
\end{aufzb}\par
Die Berge müssen Frieden dem Volke, Heil\par
\begin{aufzb}\item[]
Die Hügel bringen durch die Gerechtigkeit!\par
Er müsse Recht dem armen Volke\par
Schaffen, der Dürftigen Kindern helfen,~\RWSeitenw{241}\par
\end{aufzb}\par
Den Unterdrücker malmen zu Staub: dann ehrt\par
\begin{aufzb}\item[]
Man Dich, so lange Sonne noch währt und Mond,\par
Von Kind zu Kindeskind! -- Wie Regen\par
Sanft auf gemähete Flur, wie Tropfen,\par
\end{aufzb}\par
Die Erde fruchtend, lasse herab er sich:\par
\begin{aufzb}\item[]
In seinen Tagen blüht dann Gerechtigkeit,\par
Es blühet Glück in vollem Segen,\par
Bis an dem Himmel der Mond erblasset.\par
\end{aufzb}\par
Er herrscht von Meer zu Meer, vom Gestad' Euphrat's\par
\begin{aufzb}\item[]
Bis an der Erde Gränzen. Ihm beugen sich\par
Die Wüstenwohner; seine Feinde\par
Lecken den Staub; die Beherrscher Tarsis\par
\end{aufzb}\par
Und ferner Inseln führen ihm Gaben zu,\par
\begin{aufzb}\item[]
Ihm bringen Saba's Fürsten und Meroes\par
Geschenke; es fallen alle\par
Könige nieder vor ihm; ihm dienen\par
\end{aufzb}\par
Die Völker alle. -- Denn wenn der Arme fleht,\par
\begin{aufzb}\item[]
Wenn Hülfe fehlt dem Elenden, rettet er.\par
Er schont des Dürft'gen und Geringen,\par
Schützet das Leben der Unterdrückten;\par
\end{aufzb}\par
Entreißt sie Trug und Frevel, und achtet hoch\par
\begin{aufzb}\item[]
Ihr Blut. Wer lebet, bringt ihm aus Saba Gold,\par
Und betet an.} -- --
\end{aufzb}
\end{aufza}\par

Sey es, daß dieser Psalm ursprünglich nur ein \RWbet{Glückwunsch an Salomo}, etwa bei seiner Thronbesteigung, gewesen: genug, daß Mehreres in einem viel höheren Sinne als selbst auf \RWbet{Salomo}, auf \RWbet{Jesum} passet.
\begin{aufzb}
\item \RWbet{Jesus} ist vorzugsweise vor Allen der König, der durch Gerechtigkeit herrschet, und
\item dessen Herrschaft währet, so lange der Mond am Himmel stehet; der
\item das Blut der Elenden hoch achtet; vor dem
\item auch Könige niederfallen; dem man
\item aus Saba Gold zum Geschenk brachte, als er ein Kind noch war.~\RWSeitenw{242}
\end{aufzb}

\begin{aufza}\setcounter{enumi}{7}
\item \RWbibel{Ps}{Ps.}{89}{20\,ff}:\par
\erganf{Einst im Gesichte schloßest die Zukunft auf\par
\begin{aufzb}\item[]
Du, Deinem Frommen, sagtest: Zu helfen geb'\par
Ich einem Helden Macht, erhebe\par
Einen Erwählten aus meinem Volke;\par
\end{aufzb}\par
Erkiese \RWbet{David} meinen Verehrer mir,\par
\begin{aufzb}\item[]
Mit meinem heil'gen Oele zu salben ihn.\par
Stets soll ihn meine Hand erhalten,\par
Stärken mein Arm, nie ein Feind bezwingen,\par
\end{aufzb}\par
Ein Ungerechter niemals bewältigen. -- \par
\begin{aufzb}\item[]
Ich stelle seine Hand an die Meere hin,\par
An ferne Ströme seine Rechte.\par
Du, o mein Vater! mein Gott! so ruf' er\par
\end{aufzb}\par
Zu mir, Du, meine sichere Felsenburg!\par
\begin{aufzb}\item[]
Denn mir zum Erstgebornen ernenn' ich ihn,\par
Zum Höchsten unter Erdenherrschern;\par
Ewig bewahr' ich ihm meine Liebe.}\par
\end{aufzb}
\end{aufza}\par

Wenn der Verfasser dieses Psalmes (der in der babylonischen Gefangenschaft gelebt haben mag) auch nur von \RWbet{David} gesprochen: so passen doch seine Ausdrücke in einem viel höheren Sinne auf \RWbet{Jesum}:
\begin{aufzb}
\item dieser verdiente, wie Niemand sonst, zu heißen der \RWbet{Sohn} und \RWbet{Erstgeborne} Gottes;
\item sein Reich erstrecket sich über den ganzen Erdball; und
\item wird ewig fortdauern.
\end{aufzb}

\begin{aufza}\setcounter{enumi}{8}
\item \RWbibel{Ps}{Ps.}{110}{1\,ff}:\par
\erganf{Es sprach zu meinem Herrn der Ew'ge:\par
\begin{aufzb}\item[]
Zu meiner Rechten setze Dich,\par
\end{aufzb}\par
Bis ich Dir deine Feinde lege\par
\begin{aufzb}\item[]
Zum Schemel deiner Füße hin!\par
\end{aufzb}\par
Der Ew'ge strecket aus von Sion\par
\begin{aufzb}\item[]
Das Zepter Deiner Majestät:\par
\end{aufzb}\par
Sey Herrscher über deine Feinde,\par
\begin{aufzb}\item[]
In ihrer Mitte Herrscher Du!\par
\end{aufzb}\par
Dein Volk ergießt an Deinem Siegestage\par
\begin{aufzb}\item[]
Freiwillig sich im heil'gen Schmuck.~\RWSeitenw{243}\par
\end{aufzb}\par
Und wie der Thau der Morgenröthe\par
\begin{aufzb}\item[]
Ist Deiner Unterthanen Zahl.\par
\end{aufzb}\par
Der Ew'ge hat es Dir geschworen,\par
\begin{aufzb}\item[]
Und nie wird ihn sein Schwur gereu'n:\par
\end{aufzb}\par
\RWbet{Auf ewig bist der Gottheit Priester}\par
\begin{aufzb}\item[]
\RWbet{Nach Melchisedek's Weise Du!}\par
\end{aufzb}\par
Der Ewige zu Deiner Rechten,\par
\begin{aufzb}\item[]
Er hat am Tage seines Grimms\par
\end{aufzb}\par
Der Erde Könige geschlagen}; \usw 
\end{aufza}\par

Vor der Zerstörung Jerusalems deuteten die Juden selbst diesen Psalm auf den Messias. Und möchte er ursprünglich zu einem anderen Zwecke geschrieben worden seyn: so paßt er doch auf keinen anderen Menschen so vollkommen, als auf \RWbet{Jesum}. Denn
\begin{aufzb}
\item nur \RWbet{Jesus} verdiente es, daß ihm selbst \RWbet{David} (den man insgemein schon vor \RWbet{Jesu} Zeiten als den Verfasser dieses Psalmes ansah) den Namen eines \RWbet{Herrn} und \RWbet{seines Herrn} ertheile. Seinem Sohne \RWbet{Salomo} konnte er diesen Namen schicklicher Weise nicht geben. (Nach einer gewissen Leseart bezeichnet das Wort, welches wir Herr übersetzen, sogar \RWbet{Gott.})
\item Von \RWbet{Jesu} konnte man im eigentlichsten Sinne sagen, daß Gott ihn zu seiner Rechten erhoben habe.
\item \RWbet{Jesus} war in der That ein \RWbet{Priester nach Melchisedek's Weise}. Die Aehnlichkeiten zwischen \RWbet{Melchisedek} und \RWbet{Jesu} sind kürzlich folgende:
\begin{aufzc}
\item \RWbet{Melchisedek} war ein Verehrer des einzigen wahren Gottes im Gegensatze der Götzendiener zu seiner Zeit,
\item er war ein König und zugleich ein Priester;
\item er opferte Brod und Wein;
\item der heil.\ \RWbet{Paulus} bemerkt noch diese Aehnlichkeit: so wie die Herkunft \RWbet{Jesu} ein Geheimniß ist, so meldet die Schrift des alten Bundes auch nichts von dieses \RWbet{Melchisedek's} Abstammung.
\end{aufzc}
\item Nach einer gewissen Leseart läßt sich der dritte Vers auch so übersetzen: \anf{denn ehe noch ward die Morgenröthe, gebar ich Dich aus meinem Schooß,} \dh\ Du bist von Ewigkeit her aus meinem Wesen gezeugt.~\RWSeitenw{244}
\end{aufzb}



\RWpar{71}{C.~In dem Propheten Isaias}
Der Prophet \RWbet{Isaias} enthält die meisten und wichtigsten Weissagungen auf den Messias, so daß man sein Buch in der That nicht mit Unrecht ein \RWbet{Protevangelium} genannt hat.
\begin{aufza}
\item \RWbibel{Jes}{\RWbet{Isaias}}{2}{2\,ff}: \erganf{In späten Tagen wird der Berg, auf dem \RWbet{Jehova}'s Haus steht, befestiget seyn auf anderer Berge Rücken, und alle Völker werden dann zu ihm strömen. Laßt uns hinaufgehen, werden sie sprechen, zum Berge \RWbet{Jehova's}, zum Hause des Gottes \RWbet{Jacob's}, daß er uns lehre seine Wege. Denn von Sion wird das Gesetz ausgehen, und von Jerusalem das Wort des Herrn. An jenem Tage werden die Menschen ihre Götzen von Gold und Silber, die sie zum Anbeten sich verfertiget hatten, den Maulwürfen und den Feldmäusen hinwerfen,} \usw\par
Wie dieses Alles durch \RWbet{Jesum} erfüllt worden sey, ist bekannt.
\item \RWbibel{Jes}{\RWbet{Isaias}}{7}{10\,ff}: \erganf{\RWbet{Jehova} sprach weiter zu \RWbet{Achaz}: Fordere dir ein Zeichen, es sey auf Erden, oder am Himmel. \RWbet{Achaz} erwiederte: Ich will Gott nicht versuchen. Da sprach der Prophet: So höret denn, ihr vom Hause \RWbet{David's}! Ist's euch zu wenig, Menschen zu ermüden, wollt ihr auch Gott ermüden? Der Herr selbst wird euch ein Zeichen geben: Sieh! eine Jungfrau wird empfangen, und einen Sohn gebären, deß Name wird seyn \RWbet{Emmanuel}.}\par
\RWbet{Isaias} mag sich bei diesen Worten gedacht haben, was er will: so ist diese Stelle immer als eine eigentliche Weissagung auf \RWbet{Jesum} anzusehen. Denn die Worte: \erganf{Sieh! eine Jungfrau} \usw\ sind an \RWbet{Jesu} in einem ganz ausnehmenden Sinne in Erfüllung gegangen. Er wurde wirklich von einer Jungfrau empfangen, und \RWbet{hieß} zwar nicht, \RWbet{war} aber, was das Wort \RWbet{Emmanuel} (Gott mit uns) bedeutet. Hiezu kommt noch, daß diese Stelle schon zu den Zeiten \RWbet{Jesu} auf den Messias gedeutet wurde, wie wir daraus ersehen können, weil nicht nur \RWbet{Matthäus} sie auf \RWbet{Jesum} anwendet, sondern weil auch mehrere alte Rabbinen (\zB\ Jehuda Hakkadosch) zu Folge dieser Stelle behaupten,~\RWSeitenw{245}\ daß der Messias von einer Jungfrau müsse geboren werden. Dieses erhellet auch aus \RWbet{Johannes} (\RWbibel{Joh}{}{7}{27}), wo die Juden vom Messias sagen, daß man die Herkunft desselben nicht wissen werde. Daß also \RWbet{Isaias} jene so merkwürdigen Worte hinschrieb, daß sie auch diese Auslegung schon vor den Zeiten \RWbet{Jesu} erhielten, dieß Alles ist eine unverkennbare Leitung Gottes, welche den Zweck hat, \RWbet{Jesum} als einen göttlichen Gesandten auszuzeichnen.
\item \RWbibel{Jes}{\RWbet{Iesaias}}{9}{1\,ff}: \erganf{Dem Volke, das im Finstern wandelt, geht auf ein großes Licht, die Sonne geht den Bewohnern des Schattenlandes auf \textsymmdots\ Denn ein Kind ist uns geboren, ein Sohn geschenkt, auf dessen Schultern ruhet Herrscherwürde. Sein Name heißt der Wunderbare, der Rath (des Herrn), der starke Gott, der Vater der Ewigkeiten, der Fürst des Friedens; und seiner Herrschaft wird kein Ende seyn auf \RWbet{David's} Thron, den er befestigen wird durch Recht und durch Gerechtigkeit von nun an bis in Ewigkeit. Der Eifer \RWbet{Jehova}'s ist's, der dieß Alles vollzieht.}\par
Diese Worte des Propheten bezogen sich ursprünglich vielleicht nur auf den wackeren König \RWbet{Hiskias}, der \RWbet{Achaz} Sohn und Nachfolger war. -- Aber in welch einer höheren Bedeutung sind sie an \RWbet{Jesu} nicht in Erfüllung gegangen!
\item \RWbibel{Jes}{\RWbet{Isaias}}{11}{1\,ff}: \erganf{Dem abgehauenen Stamme \RWbet{Isaias} wird einst ein Reis entsprießen, aufgrünen wird ein Zweig aus seinen Wurzeln. Auf diesem wird der Geist \RWbet{Jehova}'s ruhen, der Geist der Weisheit und der Einsicht, der Geist des Rathes und der Stärke, der Geist der Kenntniß und der Gottesfurcht. Die Furcht \RWbet{Jehova}'s wird er athmen. Nicht richten wird er nach bloßem Augenschein, kein Urtheil nach Gerüchten fällen. Er wird die Armen richten nach Gerechtigkeit, mit Billigkeit den Unterdrückten im Lande das Urtheil sprechen. Tyrannen wird er schlagen mit dem Stabe seines Mundes, mit seiner Lippe Hauch die Frevler tödten. Gerechtigkeit wird seyn der Gürtel seiner Lenden, und Treue seiner Hüften Gurt. Dann wird der Wolf beim Lamme weilen, der Parder bei dem Böckchen liegen; und Kalb und junge Löwen und Mastvieh werden bei einander seyn; ein Knabe wird sie leiten. -- Der Säugling wird~\RWSeitenw{246}\ am Loch der Natter spielen, und der Entwöhnte (von der Mutterbrust) seine Hand in Basiliskenhöhlen stecken. Unschädlich werden alle Thiere auf meinem heiligen Berge seyn; denn voll der Erkenntniß Gottes wird dann der Erdball seyn, als wenn ihn Meeresfluthen deckten. An jenem Tage werden sich zum Sprossen \RWbet{Isai}, der als ein Heereszeichen da steht, die Heiden wenden, hochgeehrt wird seine Ruhestätte (man könnte auch übersetzen Grabstätte) seyn.}\par
Der Verfasser habe vom Könige \RWbet{Hiskias} gesprochen; aber seine Worte passen weit mehr auf \RWbet{Jesum Christum}. Denn daß auch \RWbet{Jesus} aus dem Stamme \RWbet{David'}s sey, beweisen die Geschlechtsregister bei \RWbet{Matthäus} und \RWbet{Lukas}, und er nur entsproßte diesem Stamme, als er schon einem ganz abgehauenen glich, und er nur führte ein Zeitalter herbei, das mit so prächtigen Farben als das Zeitalter der Weisheit und der Tugend geschildert zu werden verdient.
\item \RWbibel{Jes}{\RWbet{Isaias}}{35}{1\,ff}: \erganf{Es freue sich die Wüste und das dürre Land, es jauchze das Gefild' und blühe auf wie eine Rose. -- Sie sollen sehen \RWbet{Jehova}'s Majestät und unsers Gottes Herrlichkeit. Die matten Hände stärkt er, macht straff die Knie, die schon wanken wollen. Saget den Muthlosen, sie seyen getrost und ohne Furcht! denn sehet, Gott selbst wird kommen und euch retten! Dann öffnen sich der Blinden Augen, dann thut sich auf der Tauben Ohr. Dann springt der Lahme wie ein Hirsch, und Jubel singt des Stummen Zunge.}\par
Diese Stelle mag nach der Absicht ihres Verfassers nichts als eine poetische Schilderung des glücklichen Zeitalters in der letzten Hälfte der Regierung des Königs \RWbet{Hiskias} gewesen seyn: aber wie wörtlich ging sie an \RWbet{Jesu} in Erfüllung! Er nur wirkte die Wunder, die hier beschrieben werden.
\item \RWbibel{Jes}{\RWbet{Isaias}}{40}{2\,ff}: \erganf{Richtet Jerusalem auf, und rufet ihr zu, daß ihr Frohndienst vollendet und ihre Schuld versöhnet sey; daß sie empfangen habe aus der Hand \RWbet{Jehova}'s zwiefaches Maß für alle ihre Sünden. Eine Stimme ruft in der Wüste: Bereitet die Wege des Herrn, ebnet in der Einöde eine Bahn für unsern Gott. Füllt alle Thäler aus, tragt jeden Berg und Hügel ab; was krumm ist, werde~\RWSeitenw{247}\ gerade, und ebener Weg, was höckericht ist. Denn \RWbet{Jehova}'s Majestät wird sich zeigen, und alle Sterblichen werden mit einander sehen, daß der Mund \RWbet{Jehova}'s geredet habe. Sehet! mächtig kommt der Herr, mit ihm ist sein Lohn, und vor ihm sein Vergelter. Er weidet seine Heerde wie ein Hirt, er faßt die Lämmer in den Arm, trägt sie am Busen, und leitet sanft die Säugenden.}\par
Eine Weissagung auf den Vorläufer \RWbet{Jesu, Johannes} den Täufer, und eine treffende Schilderung der Sanftmuth \RWbet{Jesu}, die in der gleich folgenden Stelle noch schöner ausgemahlt wird.
\item \RWbibel{Jes}{\RWbet{Isaias}}{42}{1\,ff}: \erganf{Sieh meinen Diener, dem ich die Hand reiche, den Auserwählten, der mir so wohlgefällt. Ich senke meinen Geist auf ihn herab, daß er verkündige das Recht den Völkern. Er wird nicht schreien und nicht rufen, nicht hören lassen auf den Straßen seine Stimme. Er wird nicht brechen das zerknickte Rohr, und den glimmenden Docht nicht verlöschen.}
\item \RWbibel{Jes}{\RWbet{Isaias}}{52}{13\,ff}: \erganf{Sehet, glücklich führt er es aus, mein Knecht; berühmt und groß und sehr erhaben wird er seyn. Wie Viele staunen über ihn, weil sehr entstellt sein Aussehen, sein Antlitz mehr als irgend eines Menschensohnes. Er aber wird viele Völker reinigen (mit Gott versöhnen), es werden Könige vor ihm verschließen ihren Mund, weil sie gesehen, was ihnen nicht verkündiget war, gehört, was sie nie vernommen. Wer glaubet denn, was wir zu sagen haben? \RWbet{Jehova}'s Arm, wem ist er offenbar? Sehet, wie ein Sprosse wächst er auf, wie einer Wurzel Zweig aus dürrem Boden. Er hat nicht Schönheit, hat nicht Würde; sonst achteten wir auf ihn; kein Ansehen, daß er uns gefiele. Verachtet und der Männer Letzter! Ein Mann der Schmerzen, und vertraut mit Leiden. Allein nur unsere Leiden sind es, die er auf sich ladet. Wir halten ihn gestraft von Gott, er aber ist durchbohrt um unserer Sünden willen, zerschlagen wegen unserer Missethat. Die Strafe ruht zu unserem Wohl auf ihm, durch seine Wunden wurden wir geheilt. Wir Alle irreten wie Schafe. \RWbet{Jehova} aber warf auf ihn die Sünden von uns Allen. Man forderte die Schuld, und er hat~\RWSeitenw{248}\ sich erniedrigt und nicht geöffnet seinen Mund, dem Lamme gleich, das man zur Schlachtbank führet. Und wie das Schaf verstummt vor seinem Scheerer, so hat er nicht geöffnet seinen Mund. Man riß ihn fort aus dem Gericht. Doch wer beschreibt uns seine Zeitgenossen? Denn er ward abgeschnitten aus dem Lande der Lebenden, getödtet für die Sünden seines Volkes. Bei Missethätern war ihm sein Grab bestimmt; allein bei einem Reichen ward ihm eine Gruft; weil er kein Unrecht hatte ausgeübt, und kein Betrug aus seinem Munde gekommen. Nachdem er sein Leben zum Sühnopfer dargebracht hat, wird er viel Kinder sehen, fortsetzen seine Tage, und was \RWbet{Jehova} will, gelingt durch ihn. Nach saurer Arbeit wird er sich erfreuen. Durch seine Erkenntniß wird er Viele rechtfertigen, er, mein gerechter Knecht, der ihrer Sünden Schuld auf sich nahm. D'rum theilt er Viele ihm zur Beute zu, und gibt ihm Mächtige zum Raube, weil er sein Leben weihete dem Tode, und Missethätern zugezählt wurde, weil er die Sünden Vieler trug, und betete für die Verbrecher.}\par
Wäre es auch wirklich, was einige andere Schriftausleger behaupten, daß der Prophet \RWbet{Isaias} in dieser Stelle nicht an den Messias gedacht, sondern den Tod des Königs \RWbet{Usias} oder sonst eines Patrioten beweine, der in der babylonischen Gefangenschaft vielleicht für Recht und Wahrheit hingemordet ward: eine Weissagung ist diese Stelle unläugbar; denn sie enthält so viele und treffende Züge von \RWbet{Jesu}, daß man die Absicht Gottes, \RWbet{Jesum} hier schildern zu lassen, unmöglich verkennen kann.
\begin{aufzb}
\item \RWbet{Jesus} litt in der That, und zwar 
\item den Tod.
\item Er ward geschlagen,
\item durchbohrt,
\item den Missethätern beigezählt,
\item sollte dieselbe Grabstätte mit ihnen finden,
\item war aber zufälliger Weise bei einem Reichen begraben;
\item er litt geduldig wie ein Lamm, öffnete seinen Mund vor seinen Richtern nicht,
\item betete sterbend noch für seine Kreuziger,
\stepcounter{enumii}\item durch seine Leiden und durch seinen Tod wurden wir Alle von unseren Sünden befreit.~\RWSeitenw{249}
\end{aufzb}
\end{aufza}


\RWpar{72}{D.~In den übrigen Propheten}
\begin{aufza}
\item \RWbibel{Jer}{\RWbet{Jerem.}}{32}{1}: \erganf{Wehe euch Hirten! die ihr die Heerde eurer Weide zu Grunde richtet und zerstreuet! spricht \RWbet{Jehova}. Ich will euch strafen, spricht \RWbet{Jehova}; aber die Ueberbleibsel meiner Heerde will ich aus allen Ländern sammeln. Siehe, es kommt die Zeit, spricht \RWbet{Jehova}, da ich von \RWbet{David}'s Stamme einen ächten Sprößling auferwachsen lasse, der glücklich herrschen, und Recht und Gerechtigkeit auf Erden handhaben wird. In seinen Tagen wird \RWbet{Juda} gerettet werden, und Israel sicher wohnen. Und dieses ist der Name, den man ihm geben wird: \RWbet{Jehova}, unsere Gerechtigkeit.}
\item[]Selbst die jüdischen Schriftausleger wissen diese Stelle nicht anders als auf den Messias zu deuten.
\item \RWbibel{Ez}{\RWbet{Ezechiel}}{9}{4}: \erganf{\RWbet{Jehova} sprach zu ihm (einem in Leinwand gekleideten Manne, den Ezechiel im Traume sieht): Gehe durch die Stadt, und zeichne ein Thau (\RWhebr{t}) auf die Stirne der Männer, welche seufzen und wehklagen über die Gräuel, die zu Jerusalem begangen werden. Zu jenen Anderen aber (sechs mit Keulen bewaffneten Männern) sprach er: Geht durch die Stadt ihm nach, und tödtet. Nicht schonen soll euer Auge, kein Mitleid sollt ihr fühlen, vertilgen sollt ihr Greise und Jünglinge, Kinder und Weiber, aber rührt Niemand an von den Bezeichneten.}
\item[]Das Thau hatte in althebräischer, phönizischer und samaritanischer Schrift die Gestalt eines Kreuzes, welches bei den Aegyptern und Tyriern ein Sinnbild des ewigen Lebens gewesen seyn soll. Diese Handlung war also eine sinnvolle Andeutung jener Erlösung, die allen Menschen einst durch das Kreuz (nämlich dasjenige, an welchem der Gottmensch starb) zu Theil werden sollte.
\item \RWbibel{Ez}{\RWbet{Ezechiel}}{34}{23}: \erganf{Ich will über sie nur einen einzigen Hirten setzen, welcher sie weiden soll, meinen Diener \RWbet{David}. Ich \RWbet{Jehova} will ihr Gott, und mein Diener \RWbet{David} soll ihr Fürst seyn. Einen Bund des Friedens will ich für sie schließen, und die reißenden Thiere aus den Ländern wegschaffen, damit sie in Wüsten und Wäldern sicher schlafen können. Sie sollen nicht mehr den Völkern zur Beute werden}, \usw~\RWSeitenw{250}
\item[]Im buchstäblichen Verstande ist alles dieß nie in Erfüllung gegangen; aber in einem höheren geistlichen Verstande durch \RWbet{Jesum} zum Theile schon wirklich, zum Theile soll es noch werden.
\item Der Prophet \RWbet{Daniel} verheißt (\RWbibel{Dan}{}{2}{44}) ein Reich, das in Ewigkeit nicht zernichtet werden, auf kein anderes Volk übergehen, alle anderen Reiche zermalmen, selbst aber ewig fortdauern wird. Es wird durch keines Menschen Zuthun entstehen, sondern so wie ein Stein, der sich von selbst von einem Felsen losreißt. -- Wie passend auf das Christenthum! -- Auch sieht (\Ahat{\RWbibel{Dan}{}{7}{13}}{3,7.}) eben dieser Prophet in einem nächtlichen Gesichte, wie Jemand in Gestalt eines Menschensohnes auf den Wolken des Himmels kommt, und hinschwebt bis zu dem betagten Greise, der in schneeweißem Gewand auf einem Feuerthrone sitzt. Diesem wird Herrschaft, Ehre und Reichthum gegeben, daß alle Völker und Nationen ihm dienen. Seine Herrschaft ist eine ewige Herrschaft, die kein Ende nehmen wird. Wer sieht hier nicht eine Weissagung auf Jesum? -- Allein am Merkwürdigsten ist die Stelle \RWbibel{Dan}{Daniel}{9}{24}: Auf ein sehr dringendes Gebet, das \RWbet{Daniel} für die Rückkehr seines Volkes aus der Gefangenschaft und für die Erfüllung der Weissagungen \RWbet{Jeremiä} hierüber zu Gott gesandt hatte, erscheint ihm der Engel \RWbet{Gabriel} und spricht: \erganf{Nicht volle siebenzig Wochen sollen über dein Volk und über die heilige Stadt verfließen; so wird dem Abfalle gewehrt, der Sünde ein Ende gemacht, die Missethat versöhnt, die ewige Gerechtigkeit herbeigeführt, Gesicht und Weissagung (des Jeremias) erfüllt, und der Heilige aller Heiligen gesalbt (oder in sein Heiligthum eingeführt) seyn. So wisse nun, und merke wohl auf: Vom Ausgange des Befehls, Jerusalem wieder aufzubauen, bis zum Messias dem Sieger (König) vergehen sieben Wochen und zwei und sechzig Wochen. Inzwischen werden die Straßen und Mauern wieder aufgebaut, doch in bedrängten Zeiten. Nach zwei und sechzig Wochen aber wird der Messias ermordet werden, doch nicht um seinetwillen. Die Stadt und das Heiligthum aber wird ein siegreiches Volk, welches herbeikommen wird, zerstören. Einer Wasserfluth wird sein Hereinbrechen gleichen, und das Ende des Krieges, dessen Tage abgekürzt werden sollen, wird die Verwüstung seyn. In einer Woche wird der Bund mit Vielen bestätiget werden, und in der Mitte der~\RWSeitenw{251}\ Woche werden alle Schlacht- und Speiseopfer ihr Ende nehmen. Im Tempel wird der Gräuel der Verwüstung seyn, und ewig wird Verwüstung und Verheerung über dem heiligen Orte ruhen.}
\item[]In dieser hier gegebenen Uebersetzung, die dem hebräischen Texte, wie er bei Juden und Christen sich findet, völlig gemäß ist, enthält diese Stelle die unverkennbarsten Hindeutungen auf Jesum. -- Unter einer Woche sind nicht Wochen von Tagen, sondern \RWbet{von Jahren} zu verstehen, indem das hebräische Wort \RWhebr{+sAbU/a`} (Schabua) beiläufig eben so, wie das griechische oder lateinische \RWlat{hebdomas} bloß eine Anzahl von sieben bedeutet. So wie nun Griechen und Lateiner zuweilen von \RWlat{hebdomadibus annorum} sprechen (\zB\ \RWlat{Aristoteles Polit.\ 7, 16.\RWlit{}{Aristoteles} Gellius noct.\ Att.\ 3, 10.\RWlit{}{Gellius} Varro}, der sein Buch \RWlat{hebdomades}\RWlit{}{Varro1} in dieser Bedeutung betitelte): so geschieht dieß auch bei den Hebräern selbst in der Prosa, \zB\ \RWbibel{Lev}{3\,Mos.}{25}{8}\ -- Daß nun Daniel hier nicht von Tagwochen, sondern von Jahrwochen rede, erhellet schon daraus, weil ein Zeitraum von siebenzig Tagwochen offenbar viel zu kurz ist, als daß nur möglicher Weise in ihm Alles, was Daniel hier verspricht, hätte vollzogen werden können. Auch setzt Daniel in der Folge (\RWbibel{Dan}{}{10}{2}), wo er von Tagwochen redet, das Wort Tage (\RWhebr{yAmiym}, Jamim) ausdrücklich bei. -- Diese siebenzig Wochen betragen also einen Zeitraum von \RWbet{vierhundert neunzig Jahren}, welche Daniel in drei Epochen theilt; in eine Epoche von sieben Jahrwochen, oder vierzig neun Jahren, innerhalb deren Jerusalem wieder aufgebaut und befestiget werden soll; in eine Epoche von sechzig zwei Jahrwochen, oder vier hundert dreißig vier Jahren, nach deren Verlauf der Messias auftreten soll; und in eine Epoche von Einer Jahrwoche, oder sieben Jahren, in deren Mitte der Messias sterben, und durch seinen Tod die Opfer aufheben (oder ungültig machen) soll. Hierauf soll ein fremdes Volk auftreten, und Jerusalem verwüsten. Dieses Alles ist nun in Erfüllung gegangen. Die siebenzig Wochen sollen von der Zeit an gerechnet werden, da der Befehl, Jerusalem wieder aufzubauen, gegeben seyn würde. Nun kennen wir zwar drei solcher Befehle; aber erst der dritte, den König \RWbet{Artaxerxes Longimanus} im zwanzigsten Jahre seiner Regie\RWSeitenw{252}rung ertheilte, scheint derjenige zu seyn, den man hier annehmen muß, weil nur dieser eigentlich in Ausführung gebracht werden konnte; und es noch zweifelhaft ist, ob der erste (der Befehl des Cyrus) auch eine Erlaubniß, die Stadt und Festung zu bauen, enthalten hatte. Von diesem zwanzigsten Regierungsjahre des Artaxerxes aber bis zu dem fünfzehnten Regierungsjahre des Kaisers \RWbet{Tiberius}, oder dem Antritte des öffentlichen Lehramtes Jesu sind (wenn man jenen Befehl um das Jahr 3550 a.~m.~c.\ und das Geburtsjahr Jesu Christi 4004 setzt) wirklich ungefähr 7 + 62 Jahrwochen = 483 Jahre verflossen. Auch daß die Stadt unter bedrängten Umständen aufgebaut worden sey, hat seine Richtigkeit; ob man aber gerade sieben Jahrwochen, \di\ 49~J.\ dazu gebraucht habe, wissen wir freilich nicht. Bekanntlich lehrte aber Jesus 3--4~J., und so ist also buchstäblich wahr, daß der Messias in der Mitte der siebenzigsten Jahrwoche getödtet worden sey; \usw
\item[]Allerdings ist nicht zu läugnen, daß diese merkwürdige Bibelstelle auch viele andere Auslegungen zuläßt, besonders wenn man sich erlaubt, im Texte hie und da eine kleine Abänderung vorzunehmen; aber gesetzt auch, der Sinn, den wir ihr oben ertheilten, wäre nicht ihr ursprünglicher: so wäre doch schon genug, daß sie \RWbet{auch so} sich auslegen läßt, ja selbst von den Juden so ausgelegt wurde. Schon das reicht vollkommen hin, um dieser Stelle den Rang einer Weissagung zu sichern und zu beweisen, daß Gott Jesum als den verheißenen Messias, den Untergang des israelitischen Staates aber als eine nur wegen der Verwerfung des Messias an diesem Volke vollzogene Strafe habe darstellen wollen. Nicht ohne Grund fanden daher die Talmudisten diese Stelle so gefährlich für ihren Glauben, daß sie den Juden die Nachrechnung dieser siebenzig Wochen unter den schärfesten Drohungen verboten.
\item \RWbibel{Mi}{\RWbet{Michäas}}{5}{2}: \erganf{Du, Bethlehem Ephrata! mit Unrecht heißest du die kleinste unter den Tausenden Juda's; denn aus dir wird entspringen, der mein Volk Israel regieren wird. Sein Ursprung ist von Anbeginn, von den Tagen der Ewigkeit her.}
\end{aufza}\par
Die letzten Worte können nach hebräischem Sprachgebrauche auch die Bedeutung haben: er stammt von uraltem~\RWSeitenw{253}\ Geschlechte. Die ganze Stelle mag von Zorobabel handeln. Einleuchtend aber ist es, daß sie auf Jesum sich noch buchstäblicher anwenden läßt.
\begin{aufza}\setcounter{enumi}{5}
\item \RWbibel{Sach}{\RWbet{Zacharias}}{9}{9}: \erganf{Freue dich, Jerusalem! frohlocke, Tochter Sion! denn siehe, dein König kommt zu dir, ein Tugendheld, ein siegreicher Fürst! doch demuthsvoll auf einem Lastthiere sitzend.}
\end{aufza}\par
Die wörtliche Erfüllung bei dem Einzuge Jesu zu Jerusalem ist bekannt.
\begin{aufza}\setcounter{enumi}{6}
\item \RWbibel{Sach}{\RWbet{Zacharias}}{11}{12}: \erganf{Hierauf sprach ich zu ihnen: Gebt mir doch, wenn es euch beliebt, meinen Sold; wo nicht, so unterlasset es. Da wogen sie mir einen Sold von dreißig Silberlingen. Jehova aber sprach zu mir: Gib ihn dem Töpfer, den schnöden Preis, dessen sie mich werthgeschätzt haben. Da nahm ich die dreißig Silberlinge, und legte sie nieder im Hause Gottes für den Töpfer. Hierauf zerbrach ich auch jenen anderen Stab, um anzudeuten, daß alle Verbindung zwischen Juda und Israel nun aufgehoben sey.}
\end{aufza}\par
Wer sieht hier nicht den Verrath Judä, bis auf die kleinsten ihn begleitenden Umstände vorhergesagt? Deutlicher hätte die Andeutung wahrlich nicht geschehen dürfen; sollte sie in Erfüllung gebracht werden können, ohne die Freiheit der Menschen zu verletzen. Matthäus, der die Geschichte jenes Verrathes (\RWbibel{Mt}{}{27}{3\,ff}) erzählt, merkt an, daß der gekaufte Acker noch heut zu Tage den Namen \RWbet{Blutacker} trage. Nothwendig also muß zu der Zeit, als dieses Evangelium geschrieben wurde, ein so benannter Acker zu dem erwähnten Gebrauche für das Begräbniß der Fremden wirklich vorhanden gewesen seyn; ein sicherer Beweis, daß die ganze Geschichte keine Erdichtung sey.
\begin{aufza}\setcounter{enumi}{7}
\item \RWbibel{Mal}{\RWbet{Malachias}}{3}{1}: \erganf{Seht, meinen Gesandten sende ich vor ihm her, auf daß er ihm den Weg bereite. Hierauf wird alsbald er selbst, der Herr, den ihr schon lange erwartet, der neue Bundesstifter, in seinen Tempel eingehen.}
\end{aufza}\par
Jener Gesandte war \RWbet{Johannes}, jener neue Bundesstifter \RWbet{Jesus}; und daß der Tempel, der sonst das Eigenthum Jehova's hieß, sein Tempel genannt wird, deutet auf Jesu göttliche Würde.~\RWSeitenw{254}


\RWpar{73}{Schlußfolgerung aus diesen Weissagungen}
Nachdem wir auf diese Art die messianischen Weissagungen des alten Bundes zwar nicht alle, aber doch die vorzüglichsten kennen gelernt haben, muß jeder Unparteiliche fühlen, daß ihr Vorhandenseyn sich nicht anders, als durch ein Wunder, \dh\ durch eine solche Veranstaltung Gottes erklären lasse, welche den Zweck hatte, Jesum als einen göttlichen Gesandten auszuzeichnen. Denn
\begin{aufza}
\item gäbe es nur eine geringe Anzahl dieser Weissagungen: so könnte man das Zusammentreffen zwischen Weissagung und Erfüllung für einen bloßen Zufall erklären. Allein da ihrer so viele sind, wer könnte auch jetzt noch diese Erklärung für hinreichend halten?
\item Eben so wenig läßt sich gedenken, daß es dem Herrn, der diese Weissagungen kannte, und sich absichtlich so betrug, daß sie an ihm in Erfüllung gehen möchten, ohne einen besonderen Beistand Gottes möglich gewesen wäre, sie alle in Erfüllung zu bringen. Denn könnte man dieß auch von einigen (\zB\ von seinem Einzuge in Jerusalem) sagen: so gibt es eine große Menge anderer, die in Erfüllung zu bringen von ihm selbst gar nicht abhing, \zB\ der Verkauf um dreißig Silberlinge, und was mit diesem Gelde noch weiter geschah, die Geburt zu Bethlehem, aus dem Stamme David, von einer Jungfrau, seine Wunder, seine Auferstehung, \usw\ Und so können wir denn nicht umhin, in dem Vorhandenseyn aller dieser Weissagungen, wenn wir sie als ein Ganzes betrachten, die Absicht Gottes anzuerkennen, daß er Jesum von Nazareth als seinen Gesandten habe auszeichnen wollen.
\end{aufza}

\ctaddtocontents{ptoc}{\protect\vspace{2ex}}

\RWpar{74}{Die bisher angeführten Wunder dienen zur wirklichen Bestätigung des katholischen Christenthums als einer göttlichen Offenbarung}
Vorausgesetzt, was ich eigentlich erst in dem folgenden dritten Haupttheile umständlich darthun will, daß die Lehre des katholischen Christenthums sittliche Zuträglichkeit für uns in einem Grade besitzt, wie sonst kein anderes Religionssystem, namentlich auch das evangelische, das reformirte, das~\RWSeitenw{255}\ Socinianische \uam\ : so sage ich, wir konnten mit allem Rechte behaupten, daß die bisher erzählten Wunder und Weissagungen zur Bestätigung des katholischen Lehrbegriffes als einer göttlichen Offenbarung von Gott gewirkt worden seyen.\par
Denn diese Wunder und Weissagungen stehen
\begin{aufza}
\item nicht nur mit der katholischen, sondern mit jeder christlichen Religion in einer so engen Verbindung, daß man eigentlich von jeder rühmen kann, sie habe das Merkmal der Wunder, und könnte mithin für eine wahre göttliche Offenbarung gelten, wenn ihre Lehre den höchsten Grad sittlicher Zuträglichkeit besäße. Offenbar nämlich verdanken alle christliche Religionen ihre Entstehung sowohl, als auch ihre Ausbreitung (den Glauben, den sie bei so viel Tausenden gefunden haben) den vorhin aufgezählten Wundern und Weissagungen. Denn sicher würde weder der katholische, noch der lutherische, noch irgend ein anderer der jetzt herrschenden christlichen Lehrbegriffe jemals zu Stande gekommen seyn, und so viel Anhänger gefunden haben, wenn sich nicht einerseits jene außerordentlichen Ereignisse (Wunder und Weissagungen) begeben hätten, aus denen die Menschen bis auf den heutigen Tag geschlossen, daß Jesus Christus ein wahrer göttlicher Gesandte sey; und wenn man nicht andererseits die mehr oder weniger begründete Meinung gehabt hätte, daß jene Lehrsysteme (das katholische oder lutherische oder sonst ein anderes) ganz mit dem übereinstimmen, was Jesus selbst und seine Apostel einst gelehrt hatten. Jeder Christ ist es gewiß nur darum, weil er Jesum Christum für einen (durch jene Wunder) beglaubigten göttlichen Gesandten hält, und er bekennt sich nur darum zu diesem oder jenem Lehrbegriffe, weil er sich vorstellt, daß dieser Lehrbegriff entweder ganz einerlei mit der Lehre Jesu, oder doch in dieser gegründet sey. Nun ist es aber eben der Glaube, den das Christenthum von jeher gefunden, durch welchen es sich erhalten und ausgebreitet hat, und auch uns jetzt Lebenden zu Ohren gekommen ist. Jene außerordentlichen Ereignisse also, die uns die Schrift erzählt, stehen mit dem Lehrbegriffe jeder christlichen Kirche gerade in jener engsten Verbindung, in welcher Wunder mit einer Lehre, die sie beglaubigen sollen, nur immer stehen können, nämlich in der, wo sie zu ihrer~\RWSeitenw{256}\ Entstehung und zu ihrem Daseyn selbst mitgewirkt haben (1.\,Hptthl.\ \RWparnr{142}). Mit allem Rechte wird also Jeder, der an Einem von diesen Lehrbegriffen, \zB\ an dem katholischen, das Merkmal der höchsten sittlichen Zuträglichkeit für sich gefunden hat, jene außerordentlichen Ereignisse als Zeichen ansehen können, durch welche ihm Gott bedeutet, daß er diesen Lehrbegriff als seine Offenbarung annehmen soll. Denn nur, wenn er dieses thut, begreift er den Zweck und Nutzen dieser Ereignisse, so wie der ganzen Leitung der Vorsehung, durch die es geschah, daß er mit diesem religiösen Lehrbegriffe bekannt wurde, und sich von seiner Vortrefflichkeit überzeugte, nämlich um an ihn glauben zu können. Im Gegentheile aber, wenn er dieß unterläßt, sind diese Ereignisse, und diese Leitung der Vorsehung ohne irgend einen für ihn bemerkbaren Nutzen.
\item Wenn gleich die biblischen Wunder mit einem jeden christlichen Lehrbegriffe in einer hinlänglich engen Verbindung stehen, um ihn (unter gewissen Umständen) für eine göttliche Offenbarung geltend zu machen: so läßt sich doch von der katholischen Lehre behaupten, daß sie mit ihr in einer \RWbet{noch viel genaueren Verbindung} stehe; oder (was einerlei ist) der katholische Lehrbegriff kann sich vorzugsweise vor jedem anderen christlichen Lehrbegriffe diese Wunder aneignen; und dieß zwar aus folgenden Gründen:
\begin{aufzb}
\item erstlich, weil in den Reden Jesu, die uns das Evangelium berichtet, dessen Erzählung einen so hohen Grad der Glaubwürdigkeit hat, von einer Kirche gesprochen wird, die in alle Zeiten bestehen, und sich des Beistandes des Geistes der Wahrheit erfreuen werde, wie dieses von der katholischen bald mit Mehrerem dargethan werden soll.
\item Zweitens ist es bekannt, daß alle akatholischen Religionen, namentlich die lutherische, die reformirte, die socinianische, nicht so fast \RWbet{andere}, als nur \RWbet{wenigere Lehren} in ihren Lehrbegriff aufnehmen, als die katholische. Da aber auch diese Lehren, um welche der katholische Lehrbegriff reicher als andere ist, sittliche Brauchbarkeit haben; so ist kein Grund vorhanden, aus dem wir schließen dürften, daß diese Lehren dem Geiste Jesu zuwider seyen, und daß sich seine Wunder nicht auch auf sie beziehen. Sollten doch gleichwohl diese Lehren falsch seyn:~\RWSeitenw{257}\ so hätte Gott durch irgend ein anderes Wunder anzeigen müssen, daß wir diese Lehren nicht anzunehmen hätten. Dieses ist aber bekanntlich nie geschehen, die Akatholiken haben sich nie (oder auf eine Art, die gar keine Aufmerksamkeit verdient) auf solche Wunder berufen.
\end{aufzb}
\end{aufza}


\RWpar{75}{Auflösung einiger Einwürfe}
\begin{aufza}
\item Die irrigen Begriffe, die man bisher von Wundern überhaupt gehabt, und die hieraus entspringende Abneigung so mancher Gelehrten vor allen Wundererzählungen, vermochte Einige derselben, \zB\ \RWbet{Rousseau} und \RWbet{Bahrdt}, zu der Behauptung, daß Jesus selbst niemals gewollt habe, daß man ihm um der Wunder willen, die er gewirkt hatte, glaube.
\end{aufza}\par
Für diese Behauptung führten sie Folgendes an:
\begin{aufzb}
\item Den scharfen Verweis, den Jesus (\RWbibel{Mt}{Matth.}{16}{1\,ff}) den Pharisäern gibt, als sie ein Zeichen von ihm verlangen;
\item den Umstand, daß Jesus (\RWbibel{Lk}{Luk.}{23}{8}) vor dem Könige Herodes, der sich doch schon so lange gesehnt, von ihm ein Wunder zu sehen, keines gewirkt habe; daß ferner
\item Jesus (\RWbibel{Mt}{Matth.}{27}{41\,ff}) hangend am Kreuze nicht von demselben herabgestiegen, ob er gleich durch dieß Wunder die Juden alle gläubig gemacht haben würde; daß endlich
\item Jesus (\RWbibel{Joh}{Joh.}{4}{48}) einen allgemeinen Tadel über die Wundersucht überhaupt ausspricht: \anf{Wenn ihr nicht Zeichen und Wunder sehet, so glaubet ihr nicht.}
\end{aufzb}
Ich antworte hierauf:
\begin{aufzb}
\item Die Pharisäer verdienten einen Verweis, weil sie mit den bereits gewirkten Wundern noch nicht zufrieden, nutzlose Zeichen \RWbet{am Himmel} forderten.
\item Herodes verdiente nicht, daß Jesus seine Wunderkraft vor ihm beweise. Er wollte Zeichen bloß zur Befriedigung seiner Neugierde, wie die Kunststücke eines Taschenspielers, sehen.
\item das Herabsteigen Jesu vom Kreuze wäre verschiedenen höheren Zwecken, um derentwillen sein Tod nothwendig war, zuwider gewesen. Ueberdieß that Jesus in der Folge viel mehr, als man jetzt forderte, er ging lebendig aus seinem Grabe hervor.~\RWSeitenw{258}
\item Nicht unbedingt tadelt Jesus diejenigen, welche nicht eher glauben, als bis sie Zeichen sehen; sondern nur seine Zeitgenossen, die dieser Wunder nie genug hatten. Uebrigens sagte er selbst (\RWbibel{Joh}{Joh.}{15}{24}): \erganf{Wenn ich nicht Werke gethan hätte unter ihnen, wie sie kein Anderer gethan, so hätten sie keine Sünde (deßwegen, weil sie mich verwerfen).} Man sehe auch \RWbibel{Mt}{Matth.}{11}{1}\ \RWbibel{Mt}{}{9}{6}\ \RWbibel{Joh}{Joh.}{11}{42}\ \RWbibel{Joh}{}{12}{27}\ \uma\  Stellen, wo Jesus die Absicht, warum er Wunder wirkte, deutlich genug zu erkennen gegeben.
\end{aufzb}
\begin{aufza}\setcounter{enumi}{1}
\item Nach unserer Theorie (1.\,Hptthl.\ 4.\,Hptst.) werden zwei Merkmale zu einem Wunder erfordert:
\begin{aufzb}
\item Daß es ein ungewöhnliches Ereigniß sey;
\item daß sich kein Zweck und Nutzen, zu dem es dienen soll, angeben lasse, wenn es nicht der ist, daß es uns zur Bestätigung einer gewissen sittlich zuträglichen Lehre dienet.
\end{aufzb}
\end{aufza}
Vielleicht daß Jemand eben aus dieser Theorie eine Art von Einwurf gegen die Beweiskraft der evangelischen Wunder herleiten möchte. Die meisten Wunder Jesu sind nämlich Krankenheilungen, also Ereignisse gewesen, die schon an sich wohlthätig waren, und folglich ließe sich von ihnen nicht behaupten, daß sie, wofern sie uns nicht die göttliche Gesandtschaft Jesu darthun sollten, zwecklos und unnütz seyn müßten.\par
Hierauf erwiedere ich nun:\par
Nicht nur der Umstand, daß Kranke genesen sind, sondern auch der Umstand, daß es \RWbet{gerade zu der Zeit}, als Jesus lebte, und \RWbet{auf sein Wort} geschah, ist etwas Ungewöhnliches. Auch dieses Ungewöhnliche fordert eine Erklärung, zu welchem Zwecke und Nutzen es Gott veranlasset habe. Und diese Erklärung läßt sich nur dadurch geben, daß wir annehmen, Jesus habe hiedurch als ein Gesandter Gottes beglaubigt werden sollen. Hätten jene Krankenheilungen nicht diese Wirkung gehabt: so würden sie uns zwar nicht als an sich nutzlos erscheinen, wohl aber würden wir uns von dem Umstande, warum sie gerade auf das Wort Jesu erfolgten, keinen vernünftigen Zweck und Nutzen angeben können.~\RWSeitenw{259}

\RWabs{Anhang.}{Ueber die Wundererzählungen im Heidenthume.}
\RWpar{76}{Inhalt und Zweck dieses Anhanges}
\begin{aufza}
\item Den Vertheidigern des Christenthumes, welche die Wahrheit desselben bisher nur immer aus seinen Wundern zu beweisen suchten, hat man sehr oft entgegnet, daß auch manche \RWbet{andere Religionen ihre Wunder aufzuweisen hätten}. Man berief sich hiebei auf das \RWbet{Orakel zu Delphi}, auf die \RWbet{sibyllinischen Weissagungen}, auf den \RWbet{Genius des Sokrates}, auf die \RWbet{Wunder des Apollonius von Tyana}, des \RWbet{Kaisers Vespasian}, des \RWbet{Kaisers Hadrian, \uA } -- (So thaten es Hierokles, Tindal, Morgan, Hume, \uA )
\item Nach jenen Grundsätzen, welche ich oben aufgestellt, können diese Wunder, gesetzt, sie wären auch noch so historisch gewiß, die Wahrheit des Christenthumes als einer göttlichen Offenbarung für uns nicht anfechten, wenn nicht erst dargethan wird, daß auch die \RWbet{Lehre}, deren Wahrheit durch jene heidnischen Wunder beglaubigt werden sollte, einen höheren Grad von sittlicher Vollkommenheit, als unsere christkatholische, besitze. Dieses hat aber nicht nur noch Niemand gewagt, sondern es läßt sich auch auf keine Art erweisen; ja die meisten jener als außerordentlich geschilderten Begebenheiten stehen nicht einmal mit einer bestimmten Lehre in Verbindung.
\item Nichts desto weniger gibt es noch eine andere Rücksicht, in welcher die Betrachtung dieser Begebenheiten auch~\RWSeitenw{260}\ für uns wichtig seyn kann. In neueren Zeiten nämlich führte man jene Wundergeschichten nicht sowohl darum an, um zu beweisen, daß auch heidnische Religionen ihre Wunder hätten, sondern man setzte es vielmehr von einer Seite als etwas allerdings Entschiedenes voraus, daß diese Wunder \RWbet{bloße Erdichtungen} wären, von einer anderen Seite erhob man aber gleichwohl die Beweise, die sie für ihre Wirklichkeit aufzuweisen hätten, so sehr man nur konnte, verglich sie mit den Beweisen, welche die evangelische Geschichte für sich hat, und stellte sich an, als ob man die ersteren am Ende wohl noch überzeugender, als die letzteren fände. So wie nun gleichwohl die Wunder der Heiden bloße Erdichtungen sind, so, schloß man, müßten es auch die evangelischen seyn. Um nun die Unredlichkeit, mit welcher die Gegner des Christenthums in diesem Stücke verfuhren, deutlicher einzusehen, wollen wir jene Wunder der Heiden etwas umständlicher betrachten.
\end{aufza}


\RWpar{77}{Die Weissagungen des delphischen Orakels}
\begin{aufza}
\item Der Tempel des Jupiter, in der ägyptischen Landschaft Thebais, in welchem der Gott durch den Mund seiner Priester Weissagungen aussprach, erweckte auch bei den Griechen die Lust, Orakel zu haben. Das erste stiftete eine Priesterin aus dem Tempel zu Theben in Pelasgia, hierauf entstanden mehrere, worunter die zu Delphi und Dodonä die vornehmsten waren. Die Entstehung des Orakels zu Delphi (in der Landschaft Phokis) wird von Herodot und Diodor folgender Maßen erzählt. In der Gegend des Berges Parnassus weidete einst ein Hirte, Namens Koretas, seine Ziegen. Als diese zu einer gewissen Stelle gekommen waren, verfielen sie in sonderbare Zuckungen, und gaben ungewöhnliche Töne von sich. Der Hirte sah nach, was wohl die Ursache seyn möchte; und entdeckte ein Loch in der Erde mit einer engen Oeffnung. Als er sein Angesicht dieser genähert hatte, fühlte er sich wie betäubt, und redete gleich einem Irrenden. Dieß war genug, um auf den Gedanken zu verfallen, daß wohl Apollo diesen Ort bewohne, und sich den Menschen offenbare. Man erbaute ihm also hier einen Tem\RWSeitenw{261}pel, setzte über die Oeffnung der Höhle eine Art Dreifuß, und wenn man Offenbarungen erhalten wollte, ward eine Jungfrau, welche den Namen der Wahrsagenden (Pythia) erhielt, auf diesen Dreifuß gesetzt, und von den Priestern festgehalten. Alsbald gerieth sie in eine Art von Wuth, schäumte, stieß einige unzusammenhängende Worte aus, die wenig sagen wollten; aus welchen aber die Priester die Aussprüche des Gottes bildeten, die sie denjenigen, welche um Rath gefragt hatten, schriftlich übergaben. -- Weil nun Niemand fragen durfte, ohne Geschenke zu bringen, so wurden in dem Tempel zu Delphi nach und nach ungeheure Schätze zusammengebracht. Die Antworten des Gottes aber waren fast durchgängig so dunkel und vieldeutig, daß man sie nach der Hand, wie der Erfolg auch immer ausfallen mochte, ihm anpassen konnte. Die Griechen selbst gaben aus diesem Grunde ihrem Apollo den Namen \RWgriech{Lox'ias} (\RWgriech{par`a t`o lox`hn >'eqein t`hn >'ian, <o >est`i fwn'hn}). So gab er \zB\ dem Könige Pyrrhus von Epirus, als er sich anfragte, ob er die Römer angreifen sollte, die Antwort: \RWlat{Ajo, te, Aeacida, Romanos vincere posse}; welches bekanntlich sowohl heißen konnte, daß der König die Römer, als auch, daß die Römer den König überwältigen werden. Der König Krösus von Lydien erhielt auf die Frage, ob sein Reich dauern werde, die Antwort: Wenn ein Maulesel bei den Medern König seyn wird, dann fliehe. Krösus verstand dieß so, daß sein Reich immer fortdauern würde; als es aber durch Cyrus zerstört ward, erklärte die Pythia, Cyrus sey dieser Maulesel, weil er eine medische Mutter und einen persischen Vater habe. -- Eben diesem Krösus ward auf die Frage, ob er die Perser (den Cyrus) bekriegen solle, zur Antwort gegeben; wenn er dieß thue, so werde er ein großes Reich zerstören. Krösus meinte, es sey das Persische zu verstehen; in der Folge aber zeigte sich's, daß es sein eigenes wäre, \usw\ Eines der merkwürdigsten dieser Orakel ist folgendes. Der König Krösus schickte, um das Orakel auf die Probe zu stellen, Gesandte nach Delphi, welche den Gott befragen mußten, was für eine Handlung der König jetzt eben vornehme. An diesem Tage aber nahm er eine gar seltsame Handlung vor; er kochte nämlich eine Schild\RWSeitenw{262}kröte mit Lammfleisch in einem kupfernen Kessel. Das Orakel soll es, wie uns Herodot erzählt, errathen haben. Uebrigens erzählet eben dieser Herodot, und auch so manche andere Schriftsteller, das Orakel habe sich häufig bestechen lassen; und dieß war überhaupt selbst bei dem Volke so bekannt, daß Demosthenes das delphische Orakel in seinen öffentlichen Reden beschuldigen durfte, es philippisire (\RWlat{Cicero de divinitate 2.~57.}\RWlit{}{Cicero5}). Auch wissen wir, daß man an allen Orten, wo es Orakel gab, eigene Kundschafter unterhielt, welche Kenntnisse über diejenigen, die sich befragten, einsammeln mußten.
\item Aus allem diesem geht nun hinlänglich hervor, daß das Orakel zu Delphi auf nichts als Täuschung und Betrug gegründet war. Jene Betäubung nämlich, in welche alle diejenigen verfielen, die sich der Oeffnung der unterirdischen Höhle genähert hatten, war die natürliche Wirkung irgend einer uneinathembaren Luftart, die sich aus dieser Höhle entwickelte (\zB\ kohlensaures Gas \udgl ). Der Glaube also, daß diese Betäubung \RWbet{Begeisterung} sey, war ein Irrthum. In der Folge mochte man diesen Irrthum wohl eingesehen haben, aber man unterhielt ihn absichtlich, weil man in ihm ein Mittel sah, sich zu bereichern, und weil auch mehrere Obrigkeiten sich der Aussprüche des bestochenen Orakels bedienten, um ihre Unterthanen leichter nach ihren Absichten zu lenken. Und so gesellte sich denn zum Irrthume auch noch absichtlicher Betrug. So hatte schon Jupiter Ammon die Gefälligkeit, Alexandern für seinen Sohn zu erklären, wofür ihn dieser auch reichlich beschenkte. Und Herodot gibt mehrere Beispiele von theils zweideutigen, theils durch Bestechung erwirkten Aussprüchen des Gottes an. Was aber die Handlung des Krösus betrifft, so dürfte wohl hier eine Verabredung zwischen den Priestern und den Hofleuten des Königs obgewaltet haben; wenigstens hätte dieß Wunder gar keinen nützlichen Zweck gehabt. Vielleicht ist es auch ganz erdichtet; denn Herodot selbst gesteht, daß man nicht Alles, was er erzählt, unbedingt glauben müsse, indem er Vieles anführe, was bloße Volkssage ist, obgleich er es selbst nicht glaube.~\RWSeitenw{263}
\end{aufza}

\RWpar{78}{Die Weissagungen der sibyllinischen Bücher}
\begin{aufza}
\item Sibyllen (\dh\ Wahrsagerinnen) gab es überhaupt mehrere im Alterthume. Besonders berühmt aber war die \RWbet{Erythräische} (in Jonien), welche den Griechen, als sie nach Ilium gingen, weissagte, und die \RWbet{Cumanische} (in Campanien), welche die sibyllinischen Bücher der Römer verfaßt haben soll. Nach der Erzählung Virgil's (\RWlat{Aeneid.}\ 3, 441--461, und 6, 25\,ff.) lebte die Cumanische Sibylle, die eigentlich \RWbet{Amalthea} hieß, in einer Höhle bei Cuma, als eine Priesterin Apollo's, zur Zeit des Aeneas. Sie schrieb ihre Weissagungen auf Baumblätter, welche sie über einander legte, und in ihrer Höhle aufbewahrte. Wenn aber ein Wind bei geöffneten Thüren hineinblies, und diese Blätter durch einander wehete, brachte sie Amalthea nicht wieder in Ordnung. Gleichwohl sollen auf diese Art neun Bücher voll Weissagungen über die Schicksale des römischen Staates zusammen gekommen seyn, und wurden von einer anderen Sibylle zur Zeit des Tarquinius Priskus nach Rom gebracht und ihm angeboten. Er fand den Preis von drei hundert Goldstücken zu hoch; daher das Weib drei dieser Bücher in seiner Gegenwart verbrannte und ihm die übrigen nun um denselben Preis anbot. Als er noch weniger bereit war, diese zu kaufen, verbrannte sie abermals drei; -- und der verwunderte König kaufte nun die drei letzten um eben den Preis, um den er vorhin alle erhalten haben würde. Diese \RWlat{libri Sybillini}\RWlit{}{SibyllinischeBuecher}, auch \RWlat{fatidici}, wurden in einer steinernen Urne im Capitol unter der Aufsicht erst zweier, dann zehen, dann fünfzehen Männer verwahrt, und bei wichtigen Staatsangelegenheiten zu Rathe gezogen. Als im J.~671 ab \RWlat{u.~c.} das Capitol unter Sylla abbrannte, verbrannten auch diese Bücher. Sie wieder zu ersetzen, wurden drei Abgeordnete an den König Attalus von Pergamus geschickt, um sibyllinische Weissagungen zu sammeln. Augustus ließ aus dem mitgebrachten Vorrathe eine Auswahl treffen, und die übrigen verbrennen. Die ausgewählten wurden in zwei goldenen Kästchen im Tempel Apollo's verwahrt, und wie die ersteren gebraucht. -- Allein auch diese sibyllinischen Bücher sind heut~\RWSeitenw{264}\ zu Tage nicht mehr vorhanden, sondern die \RWlat{oracula Sibyllina}\RWlit{}{SibyllinischeSprueche}, welche wir gegenwärtig haben, sind von irgend einem christlichen Schriftsteller, (etwa im dritten Jahrhunderte) unterschoben.
\item Was diesen sibyllinischen Büchern der Römer allen Glauben benimmt, ist der Umstand, daß man Niemand eine Einsicht in diese Bücher gestattete, als nur bestimmten Staatspersonen. Wäre dasjenige, was man in ihnen zu lesen vorgab, wirklich darin gestanden, so hätte man gewiß Jedem die Einsicht erlaubt. Aus der Beschaffenheit der Aussprüche, die man in diesen Büchern zu finden vorgab, ersieht man deutlich, daß man sie bloß zu dem politischen Zwecke gebrauchte, die große Volksmenge im Unglücke aufzurichten, und sie dahin zu stimmen, daß sie dasjenige, was man einmal zu thun beschlossen hatte, desto geneigter vollziehe. Wenn \zB\ eine Hungersnoth, eine Pest, ein Steinregen, ein Erdbeben \udgl\  eintrat: so schlug man die sibyllinischen Bücher auf, und verordnete \RWlat{lectisternia} (Göttermahle) und andere dergleichen Andachten, die den gesunkenen Muth des Volkes wieder aufrichteten. Als Cäsar den zu Rom so verhaßten Titel eines Königs gern annehmen wollte, fand man in den sibyllinischen Büchern, daß die Parther nicht anders, als von einem Könige überwunden werden könnten \udgl\  Die Geschichte von dem Weibe, das dem Tarquinius Priskus erschienen seyn soll, dürfte ein bloßes Vorgeben dieses Königes seyn.
\end{aufza}


\RWpar{79}{Der Genius des Sokrates}
\begin{aufza}
\item Plato, Xenophon und Plutarch erzählen uns, \RWbet{Sokrates} habe sich gegen seine Schüler öfters geäußert, daß ein gewisses höheres Wesen ihn leite, das ihm zwar eben nicht seine Lehren eingegeben hätte, wohl aber zuweilen von einer Handlung abhalte, ohne ihn doch zu etwas anderem Bestimmten anzutreiben.\RWfootnote{%
	\RWgriech{Je~i'on ti ka`i daim'onion}; -- \RWgriech{fwn`h t'is, <`h >ae`i >apotr'epei me to'utou, <`o >`an m'ellw pr'attein, protr'epei d`e o>'upote.}\RWlit{}{}{Platon, Apologia 31d.}}
Einst ging er mit einigen seiner jungen Freunde spazieren; plötzlich wandelt ihn eine gewisse~\RWSeitenw{265}\ Aengstlichkeit an, er erkennt den Wink seines Genius, und kehrt um. Etliche nasenweise Jünglinge aber gehen dennoch fort, und werden von einer Herde Schweine besudelt. Sokrates warnt den Glaukus, auf den Nemeischen Spielen nicht zu erscheinen, dieser geht doch, und findet Ursache, es zu bereuen. Auf der Flucht nach der unglücklichen Schlacht bei Delium ermahnt Sokrates seine Gefährten, als sie zu einem Scheidewege kommen, einerlei Weg mit ihm einzuschlagen; Einige thun es nicht, und fallen unter die Reiterei der Feinde. Bei einem Gastmahle, wo Timarchus zugegen war, wollte sich dieser zweimal entfernen; Sokrates bedeutete ihm zu bleiben, das dritte Mal schlich er sich, ohne daß Sokrates es merkte, weg, und beging einen Mord. Auch Krito, der Freund des Sokrates, ging seiner Warnung zuwider spazieren, und siehe, er wurde von dem Aste eines Baumes am Auge beschädiget. Endlich weissagte auch Sokrates den unglücklichen Ausgang der Unternehmungen gegen Sicilien, Jonien und Ephesus. Noch zu bemerken ist, daß Sokrates von diesem Genius nie vor dem Volke geredet, sondern nur vor seinen Richtern sich auf ihn berufen, um zu zeigen, daß ihn die Gottheit selbst zu den Atheniensern gesandt habe. Auch ist zu wissen, daß Plato sagt, Sokrates habe seinen Genius nur gehört; Apulejus, er sey ihm auch manchmal erschienen, Maximus Tyrius, er sey auch Anderen, wenn es nöthig gewesen, beigestanden; Plutarch sogar, daß schon dem Vater des Sokrates das Orakel gesagt: Laß dir nicht beikommen, deinen Sohn zu irgend etwas zu zwingen, sondern nach seinem freien Willen laß ihn handeln; denn er hat in sich selbst einen Rathgeber und Führer, der besser ist, als tausend Lehrmeister. Da nun Plato und Xenophon Sokrates unmittelbare Schüler waren, Plutarch einer der glaubwürdigsten Geschichtschreiber ist: so hat man noch selbst in neuerer Zeit behauptet, der Genius des Sokrates habe stärkere Beweise für sich, als selbst die Wunder Jesu. -- (Deutsches Museum.\ 1777.)\RWlit{}{DeutschesMuseum1}
\item Plato und Xenophon verdienen allerdings Glauben, und wenn sie sagen, daß Sokrates von einem gewissen Genius gesprochen \usw , so muß es auch wahr seyn, daß Sokrates etwas von dieser Art erzählet habe. Auch der Charakter~\RWSeitenw{266}\ des Sokrates verdient Zutrauen, und es ist nicht vorauszusetzen, daß er betrügerischer Weise vorgegeben hätte, einen gewissen Genius zu besitzen, wenn er nicht selbst geglaubt hätte, daß es also sey. -- Wie aber Sokrates zu diesem Glauben gekommen, und wie er sich eingebildet habe, diese und jene Eingebung des Genius zu vernehmen, das Alles läßt sich erklären, ohne ein eigentliches Wunder, \dh\ ein solches Ereigniß anzunehmen, wodurch der Wille Gottes, den Sokrates als einen göttlichen Gesandten darzustellen, bewiesen würde. In jenem Zeitalter nämlich glaubte man insgemein an Geistererscheinungen, Ahnungen, Träume \udgl\  Auch Sokrates also konnte gewisse Gefühle, die ihn zuweilen anwandelten, gewisse Einfälle, die ihm mit einer besonderen Lebhaftigkeit vorschwebten \udgl\  für die Eingebung eines Geistes halten, ob sie gleich nur das Werk seiner zufälligen Ideenverbindung, oder ein Auspruch seines gesunden Menschenverstandes waren. Auf diese Art lassen sich wenigstens alle vorhin angeführten Eingebungen seines Genius sehr leicht erklären. In den ersten drei Beispielen und in dem fünften that wohl der Zufall Alles. Die vierte und sechste Eingebung betrifft eine Sache, die Sokrates durch bloße Gründe der Vernunft mit hinlänglicher Wahrscheinlichkeit vorhersehen konnte. Auf jeden Fall ist schon der Umstand, daß Sokrates selbst gesagt, sein Genius zeige ihm nicht, was er zu lehren habe, ein hinreichender Grund, warum wir in diesem Genius kein Zeichen des göttlichen Willens, daß wir die Lehren des Sokrates annehmen sollen, erkennen dürfen.
\end{aufza}

\RWpar{80}{Wunder des Apollonius von Tyana}
\begin{aufza}
\item Apollonius von Tyana (in Kappadocien) soll ein pythagoräischer Philosoph gewesen seyn, der zu den Zeiten des Kaisers Nero gelebt. Ein gewisser Damis von Ninive, sein Schüler, soll sein Leben geschrieben, und ein ganz Unbekannter soll diese Handschrift der Kaiserin Julia, Gemahlin des Kaisers Severus im dritten Jahrhunderte, übergeben haben. Aus dieser Handschrift, welche die Kaiserin nicht unterhaltend genug fand, lieferte durch Ueberarbeitung der~\RWSeitenw{267}\ Philosoph Philostratus dasjenige Leben des Apollonius in griechischer Sprache, welches wir jetzt noch besitzen. Zufolge dieses Buches nun reisete Apollonius umher, die Sitten der Menschen zu bessern, wirkte gar mancherlei Wunder, heilte die Kranken, vertrieb böse Geister \udgl\  Zu Ephesus, wohin er von Smyrna durch eine Luftfahrt in wenigen Augenblicken gekommen, vertrieb er die Pest bloß dadurch, daß er einen alten Bettler steinigen ließ, bei dessen Tode es sich aus seinen blitzenden Augen hinlänglich zeigte, daß er ein böser Geist sey; wie man denn auch einige Zeit nachher statt des vermeintlichen Bettlers unter den Steinen einen todten Hund gefunden. Derselbe Apollonius erweckte zu Rom ein Mädchen, welches man zu Grabe trug, indem er ihr etwas in's Ohr raunte. Auch heilte er einen Jüngling, der von einem wüthenden Hunde war gebissen worden, und wußte, daß in dem Hunde die Seele des Telephus (eines Sohnes des Herkules) stecke; heilte endlich auch diesen Hund, indem er ihn durch einen Fluß führte. Als einmal eine Sonnenfinsterniß eintrat, und man zugleich einen Donnerschlag hörte, sprach er, daß etwas geschehen, oder vielmehr nicht geschehen werde; und nach drei Tagen erfuhr man, der Blitz habe den Becher getroffen, den Nero in der Hand hielt, ohne ihn selbst zu tödten. Als Apollonius einmal in Fesseln lag, konnte er die Beine nach Belieben den Fesseln entziehen, und sie wieder hineinbringen; und als der Richter die wider ihn eingebrachte Klageschrift vorlesen wollte, fand er sie unbeschrieben. Da ihn Domitian gerichtlich verhören wollte, verschwand er, und ward zu Puteoli sichtbar. Den Tod dieses Kaisers kündigte er an eben dem Tage zu Ephesus an, da er zu Rom erfolgte. Endlich traf er auf seinen Reisen auch Zwergvölker, redende Bäume, tanzende Tische und Teller, Fässer mit Wind und Wetter gefüllt \udgl\  an, verstand die Sprache der Thiere, und unterhielt sich mit ihnen, und fuhr zuletzt gen Himmel. --
\item Philostratus überarbeitete die Geschichte des Damis, um sie unterhaltender zu machen; wie kann also seine Erzählung historische Glaubwürdigkeit haben? Damis selbst ist der Geschichte ganz unbekannt; wir wissen also auch nicht, ob er die Eigenschaften eines glaubwürdigen Zeugen gehabt. --~\RWSeitenw{268}\ Lucian, der noch vor Philostratus im zweiten Jahrhunderte gelebt, erwähnt des Apollonius im Leben des Alexanders, welcher des Ersteren Schüler gewesen, als eines Betrügers, und zwar auf eine solche Art, daß zu ersehen ist, man habe ihn allgemein dafür gehalten. Du siehest wohl, schreibt er von Alexander, aus welcher Sippschaft der Mann ist, den ich dir vorführe. -- Uebrigens verdienen diese Erzählungen auch schon darum keinen Glauben, weil die hier aufgestellten Wunder durchaus nichts Gotteswürdiges enthalten, manche sogar sehr abgeschmackt und thöricht sind., \zB\ die Pestvertreibung durch jene Steinigung, die Seelenwanderung in einen Hund, das Einflüstern in das Ohr einer Todten, die Zwergvölker \usw\ Die ganze Geschichte ist allem Anscheine nach nichts Anderes, als eine übelgerathene Nachäffung der Wunder Jesu, geschrieben, um das sinkende Ansehen des Heidenthumes wo möglich noch auf eine Zeit zu unterstützen.
\end{aufza}

\begin{RWanm} 
\RWbet{Wieland} hat in seinem \RWbet{Agathodämon}\RWlit{}{Wieland2} die Geschichte dieses Apollonius in einer neuen dichterischen Einkleidung vorgetragen, in der sie dem Christenthume leicht gefährlicher werden kann, als in der Darstellung des Philostratus. Nach Wieland's Schilderung ist Apollonius ein Mann, der von dem edlen Wunsche beseelt ist, die Menschen aufzuklären und zu bessern. Er sieht, daß er zu diesem Zwecke nicht gelangen könne, wenn er sich seinen Zeitgenossen nicht als einen außerordentlichen Mann darstellt. Er versucht es also, und es gelingt ihm nach Wunsch. Damis, sein Schüler und Lebensbeschreiber, ist ein orientalischer Schwärmer, der alle auffallenden Handlungen seines Meisters in Wunder verwandelt. Wenn Apollonius \zB\ von raschen Pferden geführt sehr schnell von Smyrna nach Ephesus kommt, erzählt uns Damis schon, er sey durch die Lüfte dahin versetzt worden, wenn Apollonius eine scheintodte Braut, deren Bräutigam sein Schüler ist, deren hysterische Zufälle er genau kennt, in einem verdunkelten Saale durch unvermerktes Eingießen einer flüchtigreizenden Essenz wieder zu sich bringt, macht Damis eine eigentliche Todtenerweckung daraus, \usw\ Was nun der wundersüchtige Damis für Apollonius war, das sollen (nach einer Behauptung, die Wieland dem Apollonius selbst in den Mund legt) die Apostel für unseren Herrn Jesus gewesen seyn; nur mit dem Unterschiede, daß Jesus selbst geglaubt habe, er sey ein Wunderthäter. -- Der unverkennbare Zweck dieser Wielandischen~\RWSeitenw{269}\ Arbeit ist, uns zu verstehen zu geben, wie beiläufig es mit der Entstehung der Evangelien hergegangen sey. Wir haben aber schon oben gesehen, daß die heil.\ Evangelisten nicht getäuscht worden sind, auch nicht getäuscht werden konnten. Die Mühe, die es der glücklichen Erfindungsgabe eines Wielands selbst kostete, uns auch nur einiger Maßen begreiflich zu machen, wie nicht alle, sondern nur einige der in den Büchern des neuen Bundes enthaltenen Wundererzählungen zum Vorschein kommen konnten, ohne daß etwas Außerordentliches sich wirklich zugetragen hatte, schon diese Mühe beweist die Unausführbarkeit der ganzen Unternehmung. Hiezu kommt noch das Geständniß, das Wieland dem Agathodämon (\di\ Apollonius) in den Mund legt, Jesus sey das, wofür er (Apollonius) sich bloß fälschlich ausgegeben, ein Gesandter Gottes, in der That gewesen, und bei dem Ereigniß seiner Auferstehung habe Gott sichtbar mitgewirkt.
\end{RWanm}


\RWpar{81}{Wunder des Kaisers Vespasian}
\begin{aufza}
\item Die beiden Quellen dieser Wundererzählung lauten: 
\begin{aufzb}
\item \RWbet{Tacitus} (\RWlat{hist.\ lib.\,4, 81.}\RWlit{}{Tacitus2}): \erganf{Als Vespasian zu Alexandrien aufgehalten wurde, ereigneten sich viele Wunder, durch die eine gewisse Gunst und Zuneigung der Götter gegen ihn sich kund gab. Ein gemeiner Mensch aus Alexandrien, \RWbet{der als blind bekannt war}, warf sich ihm zu Füßen, und flehte, auf Antrieb des Gottes Serapis, den das abergläubische Volk vor anderen Göttern verehrt, seufzend um Hülfe wider seine Blindheit, bittend, er möchte ihm doch auf die Wangen und in die Augen speien. Ein Anderer \RWbet{mit gelähmtem Arme}, von demselben Gotte bewogen, bat den Kaiser, er möchte ihm mit der Fußsohle einen Tritt versetzen. Vespasian lachte anfangs verächtlich; als sie nicht aufhörten zu bitten, fürchtete er, für eitel gehalten zu werden; dann flößte ihm ihr dringendes Flehen und \RWbet{das Zureden der Schmeichler} einige Hoffnung ein; und endlich \RWbet{ließ er Aerzte entscheiden}, ob eine solche Blindheit und Lähmung durch menschliche Hülfe heilbar wäre. Die Aerzte redeten verschiedentlich hin und her: Jenem sey die Sehkraft nicht ganz versiegt, sie würde wiederkehren,~\RWSeitenw{270}\ wenn die Hindernisse beseitiget wären; diesem könnten die gelähmten Glieder allerdings hergestellt werden, wenn eine heilsame Kraft angewendet würde; ihre Herstellung liege vielleicht den Göttern am Herzen, und der Kaiser könnte wohl zu ihrem Werkzeuge erwählet seyn; auf jeden Fall bliebe der Ruhm, wenn die Heilung gelänge, dem Kaiser; der Spott aber, wenn sie mißrathen sollte, fiele auf die Elenden zurück. Nun glaubte Vespasian, seinem Glücke müsse Alles gelingen, und nichts sey mehr unglaublich, flößte der umstehenden Menge mit seinem heitern Blick Vertrauen ein, that, was man verlangte, und sogleich konnte der Lahme die Hand gehörig brauchen, und der Blinde erhielt das Gesicht. Diejenigen, welche zugegen waren, erzählen Beides noch jetzt, obschon die Lüge keinen Nutzen mehr hätte.}
\item \RWbet{Suetonius} (\RWlat{in Vespas.~7.}): \erganf{Noch fehlte ihm, der so unvermuthet Kaiser geworden und in seiner Würde noch neu war, Ansehen und Majestät; und auch diese ward ihm. Als er eben zu Gericht saß, kamen Zweie, von denen der Eine \RWbet{blind}, der Andere am \RWbet{Fuße lahm} war, zugleich vor ihn, um Heilung bittend, weil ihnen Serapis im Traume verheißen, er (Vespasian) würde die Augen sogleich herstellen, wenn er sie nur anspeien, und dem Fuße die Kraft wieder geben, wenn er ihn nur mit seiner Ferse berühren wollte. Da kaum zu vermuthen stand, daß dieser Erfolg eintreten könnte: so wollte er nicht einmal den Versuch wagen; doch \RWbet{auf Zureden seiner Freunde} versuchte er endlich öffentlich vor der Versammlung Beides, und es gelang.}

\end{aufzb}
\item Es ist kaum zu begreifen, wie man eine solche Geschichte mit den evangelischen Wundererzählungen vergleichen kann.
\begin{aufzb}
\item Tacitus und Suetonius sind freilich glaubwürdige Geschichtschreiber, aber sie waren doch bei dieser Begebenheit keine Augenzeugen.
\item Sie geben uns selbst zu verstehen, daß die Sache nichts Anderes als ein Betrug gewesen, welcher den Zweck hatte, dem neuerwählten Kaiser mehr Ansehen zu verschaffen.~\RWSeitenw{271}
\item Sie weichen in einem sehr merkwürdigen Umstande ab, bei Tacitus war der Eine am \RWbet{Arme} gelähmt, bei Sueton schwach an den \RWbet{Füßen}.
\item Es wurden die Aerzte zu Rathe gezogen, welche die Heilung nicht für unmöglich erklärten.
\item Der Kaiser wußte es selbst nicht, daß jene Wunderkraft ihm beiwohnen solle \udgl\ 
\item Der Umstand, dessen Tacitus erwähnt: \erganf{Diejenigen, welche zugegen waren, erzählen} \usw , läßt sich wohl leicht erklären. Viele mochten von dem Betruge nichts wissen; Andere, die darum wußten, hatten doch keinen Vortheil davon, ihn jetzt zu offenbaren.
\end{aufzb}
\end{aufza}


\RWpar{82}{Wunder des Kaisers Hadrian}
\begin{aufza}
\item \RWlat{Aelius Spartian.\ in Adv.\ c.\,25.\RWlit{}{Aelius}} erzählt, daß der kranke und schon des Lebens überdrüßige Kaiser Hadrian zu wiederholten Malen sich habe entleiben wollen, wovor ihn sein angenommener Sohn, der edle Antonin, mit Mühe nur zurückgehalten hätte. Endlich ergab sich folgendes: Zu jener Zeit kam ein Weib, welches sagte, sie sey im Traume beauftragt worden, dem Hadrian zu bedeuten, daß er sich nicht umbringen solle, und daß er völlig genesen würde; weil sie aber dieses nicht gethan: so sey sie erblindet. Sie hätte jedoch wieder die Weisung erhalten, daß sie dieß thun, und Hadrian's Füße küssen solle, und daß sie das Gesicht erhalten würde, wenn sie es thäte. -- Als sie diesen Traum erfüllte, ward sie sehend, nachdem sie sich die Augen mit dem Wasser gewaschen hatte, das in dem Tempel war, aus welchem sie kam. Auch ein blindgeborner Mensch aus Panonien erschien bei dem fieberkranken Hadrian; und als er ihn berührte, erhielt er das Gesicht, und Hadrian ward vom Fieber geheilt. Marius Maximus aber bemerkt, dieß Alles sey durch Verstellung (\RWlat{per simulationem}) so veranstaltet worden.
\item Der letzte Zusatz macht es völlig deutlich, was Jedem schon bei Durchlesung der Geschichte einfallen mußte, daß Alles eine sehr leichte Veranstaltung des guten Antonin's~\RWSeitenw{272}\ war, um den Kaiser von seinen schwermüthigen Gedanken abzuziehen, und neue Lebenslust in ihm zu wecken.
\end{aufza}


\RWpar{83}{Schlußfolgerung}
Wie unbedeutend ist also Alles, was man den Wundern des Christenthumes entgegenstellt, um ihre Glaubwürdigkeit zu vermindern! Statt daß diese Wunder durch eine solche Vergleichung verlieren sollten, gewinnen sie vielmehr; und so wird denn wohl ein jeder Unparteiliche, der alle Gründe, die ich bisher für ihre Wahrheit angeführt habe, einer aufmerksamen Betrachtung würdiget, mit Ueberzeugung eingestehen müssen, daß das katholische Christenthum das Eine und erste Kennzeichen einer wahren göttlichen Offenbarung, nämlich die \RWbet{Bestätigung durch Wunder} wirklich besitze. Laßt uns nun sehen, ob es auch eben so das zweite habe.

\endinput