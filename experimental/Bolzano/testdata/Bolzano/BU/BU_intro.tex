% intro.tex aus dem Projekt PdU
%\RWch{Einleitung}\markboth{\kopfzeilenfmt{Einleitung}}{\kopfzeilenfmt{Christian Tapp}}
\vspace{-1.5\baselineskip}{
\let\footnote\footnoteC
\PdUsec{Der Ausgangstext}
Die \bet{Beurtheilende Übersicht} [=BÜ] -- kurz für: \bet{Bolzano's Wissenschaftslehre und Religionswissenschaft in einer beurtheilenden Uebersicht} -- erschien 1841 bei Seidel in Sulzbach. Im Kern bietet sie eine kurze Zusammenfassung der wichtigsten Gedankengänge aus den beiden Hauptwerken Bernard Bolzanos (1781--1848), nämlich der \bet{Religionswissenschaft} [=RW] von 1834 und der \bet{Wissenschaftslehre} [=WL] von 1837. Ergänzt wird diese Zusammenfassung durch Erläuterungen und Erklärungen, die augenscheinlich Missverständnisse in der (spärlichen) Rezeption dieser beiden Werke ausräumen sollen.

\PdUsec{Die Autorschaft}
Die BÜ ist durchgehend aus der Perspektive eines von Bolzano verschiedenen Autors geschrieben, der Bolzanos Werk \anf{von außen} beschreibt und diskutiert. Dennoch ist allgemeiner Forschungskonsens, dass das Werk bis auf die Einleitung von Bolzano selbst verfasst wurde, während die Einleitung von Bolzanos Schüler und Freund Michael Josef Fesl (1788--1864) verfasst wurde.\par
Bolzanos Autorschaft bezeugt \zB\ \priholang\ (1788--1859), wenn er im Nekrolog Bolzanos [1850] schreibt:
\zit[BGA 4,1/3, S. 216--217]{Für den Zweck der Erläuterung und Empfehlung seiner philosophischen und religiösen Ansichten schrieb B.[olzano] das Büchlein: B.[olzano]'s Wissenschaftslehre und Religionswissenschaft in einer beurtheilenden Uebersicht (Sulzbach 1841). -- Schon aus den Titeln ist zu entnehmen, daß diese Schriften nicht von ihm, sondern von seinen Freunden sind herausgegeben worden und dieß zwar im Auslande mit Umgehung der österr. Censur, die eigens angewiesen war, von ihm nichts durchgehen zu lassen.}
Ähnlich schreibt Fesl (1788--1864) im Jahre 1849 an die Kaiserliche Akademie der Wissenschaften in Wien:
\zit[BGA 4,1/3, S. 190]{Was er auf diesem Gebiete [=dem der Philosophie, C.\,T.] Neues oder Vollkommneres geleistet zu haben glaubte, darüber berichtet er selbst in seiner \anf{Uebersicht der Wissenschaftslehre und Religionswissenschaft} (1841) ebenso aufrichtig als bescheiden.}\par
Die Darstellungen \priho s und Fesls werden dadurch untermauert, dass mehrere Manuskripte des ersten Teils der BÜ von Bolzanos Hand erhalten sind. Sie liegen im Prager Bolzano-Nachlass.\footnote{%
	Signaturen DVb1--DVb3 lt. BGA E 2/3, S. 36.}
Weitere Hinweise auf die Autorschaft finden sich in Zeithammers Biographie (BGA 4,2, S. 155--156), sowie in Bolzanos Briefen an \priho, BGA 3,3/1--3, S. 216, 217, 233.\par
Die Autorschaft Bolzanos hat mithin als gesichert gelten.


\PdUsec{Die Textgestaltung}
Die vorliegende Ausgabe gibt den Text der gedruckten Ausgabe von 1841 wieder. Der Rohtext wurde aus Scans dieser Ausgabe mit Unterstützung des Programms \anf{Transkribus} gewonnen.\par
Bei der Abwägung zwischen originalgetreuer Wiedergabe und guter Lesbarkeit wurden folgende Entscheidungen getroffen:
\begin{aufza}
\item Die für den deutschen Text verwendete Frakturschrift wird durch eine heute übliche Antiqua ersetzt.
\item Es wird grundsätzlich \bet{buchstabengetreu} ediert, \dh
	\begin{aufzb}
	\item Wenn das Original den Buchstaben \anf{ä} hat, wird auch ein \anf{ä} wiedergegeben, auch wenn als Umlautzeichen damals noch häufig ein kleines e über den Vokal geschrieben wurde (\anf{\ensuremath{\mathrm{\overset{\raisebox{-2pt}{\scalebox{0.4}{e}}}{a}}}}). Dies wird also als eine andere Schreibung des Buchstabens \anf{ä} angesehen und mithin als \anf{ä} wiedergegeben. 
	\item Wenn das Original im Fall von Diphthongen, bei denen der erste Buchstabe ein Umlaut ist (v.a. \anf{äu} bzw. \anf{Äu}), das Umlautzeichen auf den zweiten Vokal verschoben hat, wird das beibehalten. Statt \anf{Äußerung} wird also, wie im Originaltext, \anf{Aüßerung} gedruckt, denn die Buchstaben waren dort eben \anf{A} und \anf{ü} und nicht \anf{Ä} und \anf{u}. -- Diese Satzbesonderheit ist für heutige Augen ungewöhnlich. Sie war zur damaligen Zeit aber durchaus regelmäßig anzutreffen und ist auch in der BÜ 1841 durchgängig zu finden. Sie stört den Lesefluss kaum. Ihre Beibehaltung vermittelt  -- gemeinsam mit der vollständig beibehaltenen Orthographie -- das \anf{Flair} eines Drucks aus der ersten Hälfte des 19.~Jahrhunderts. 
	\end{aufzb}
\item Von der buchstabengetreuen Wiedergabe wird nur abgewichen, wenn es sich offensichtlich um echte Druckfehler handelt. Der häufigste Druckfehler in der BÜ ist die Verwechslung von \anf{\textswab{n}} und \anf{\textswab{u}} -- was wohl durch ihre annähernde Rotationssymmetrie erklärlich ist.\footnote{%
	Zum Beispiel: \anf{Verneinnng} statt \anf{Verneinung} (BÜ 65, Z.\,4); \anf{Ansdrücke} statt \anf{Ausdrücke} (BÜ 136, Z.\,38); \anf{deu} statt \anf{den} (BÜ 151, Z.\,29). -- Bei der Zeilenzählung wird stets die Seitennummer im Kolumnentitel als Zeile 1 gezählt.}
\item Entsprechend werden auch weitere Druckfehler korrigiert, sofern es sich offensichtlich um solche handelt, wie etwa bei dem einfachen Anführungszeichen statt eines dopppelten am Zeilenanfang (BÜ 96, Z.\,10), bei der Graphie \fbox{,\grq ,} statt \fbox{,\grqq }, wo vermutlich die dritte Drucktype um 180° gedreht war, (BÜ 100, Z.\,4) oder \anf{und} statt \anf{uns} (BÜ 127, Z.\,23).
\item Was die Setzung von Anführungszeichen angeht, lässt sich Folgendes beobachten.
\begin{aufzb}
\item Grundsätzlich, aber nicht immer, setzt Bolzano\footnote{%
	Der Einfachheit halber wird hier \anf{Bolzano} genannt, ohne damit sagen zu wollen, dass diese Eigentümlichkeiten auf ihn persönlich zurückgehen, und nicht etwa auf die Herausgeber, den Verlag oder den Drucker.}
wörtliche Zitate in Anführungszeichen. 
\item In der BÜ folgt er außerdem der damaligen Gepflogenheit, bei in Anführungszeichen gesetzten Zitaten, die sich über mehrere Zeilen erstrecken, am Beginn jeder neuen Zeile noch einmal ein öffnendes Anführungszeichen zu setzen.
\item Die heute geläufige Praxis, mittels Anführungszeichen zwischen der Erwähnung eines Ausdrucks und seiner Verwendung (d.h. der Erwähnung des durch den Ausdruck Bezeichneten) zu unterscheiden, hat Bolzano nicht geteilt. Wenn er herausstreichen will, dass es gerade um das Wort als Wort geht, so verwendet er eher den \gesperrt{Gesperrtdruck}, den er ganz allgemein als Hervorhebungmarkierung einsetzt. 
\item In dieser Ausgabe werden deutsche Anführungszeichen (\danf{\ }) verwendet, wenn in der Vorlage Anführungszeichen gesetzt sind (dies sind i.d.R. auch deutsche Anführungszeichen). Falls Anführungszeichen hingegen aus Gründen der Verständlichkeit hinzugefügt werden, werden dazu wie dieselben inversen französischen Anführungszeichen (\anf{\ }) verwendet, wie sie in den Textbeiträgen des Herausgebers als Anführungszeichen benutzt werden.
\item Wenn Bolzano, wie etwa in BÜ 92 ein Zitat-im-Zitat in eigene Anführungszeichen setzt, verwendet er dazu auch die doppelten deutschen Anführungszeichen, sodass die Graphie am Zeilenanfang entsprechend \fbox{\glqq\glqq} lautet. Wir ersetzen, heutigen Gepflogenheiten entsprechend, bei Zitaten-im-Zitat die doppelten durch einfache deutsche Anführungszeichen (\deinfanf{\ }).
\end{aufzb}
\item Ergänzungen des Herausgebers werden durch eckige Klammern (\erg{\ }) markiert.
\item Zur Orientierung in diesem Band:
\begin{aufzb}
\item In den Kolumnentiteln werden links das Bezugswerk und die Bandnummer genannt (z. B. \anf{Zu WL III} bedeutet: zum dritten Band der Wissenschaftslehre). 
\item Die Wissenschaftslehre weist Paragraphenzählungen auf, die über die Bandgrenzen fortlaufen. Außerdem bespricht Bolzano sie chronologisch. Daher können Paragraphenangaben im rechten Kolumnentitel hier die Orientierung erleichtern. 
\item Die Religionswissenschaft hat hingegen in jedem Haupttheil neue Paragraphenzählungen und Bolzano bespricht sie nicht chronologisch, sondern mit veränderter Reihenfolge zwischen manchen Hauptstücken. Daher werden hier im rechten Kolumnentitel die Nummer und der Paragraphenbereich des betreffenden Hauptstücks angeführt. Zusätzlich werden die Angabe des \RWHSfmt{Hauptstücks} und seines \RWHSfmt{Titels} im Text unterstrichen.
\item Bolzano verwendet zwar häufig Anführungszeichen, wenn er aus seinen eigenen Werken zitiert, dies bedeutet aber nicht unbedingt, dass der Text wortgetreu wiedergegeben wird. Zitate, bei denen nur zusammengefasst oder stilistisch verändert wurde, werden so belassen. Sind hingegen größere, u.\,U. inhaltlich relevante Änderungen gegenüber dem zitierten Text festzustellen, werden die Änderungen editorisch kenntlich gemacht.
\end{aufzb}
\end{aufza}
}
\endinput