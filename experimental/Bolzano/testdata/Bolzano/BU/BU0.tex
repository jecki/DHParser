\global\zeilennummerntrue\clearpage\ifnackt\else\linenumbers\fi%
%\PdUsec{[BÜ 3--15]}
\thispagestyle{ctplain}\noindent\seitenwohne{3}Wer seinen Ansichten, sey es auf dem Wege der mündlichen Unterweisung oder durch Schrift, Eingang bei Andern und allmäliche Verbreitung zu verschaffen wünschet, muß seine Behauptungen nicht nur auf Vordersätze stützen, die Allen einleuchten, sondern auch dafür sorgen, daß es von seinen Zuhörern oder Lesern wahrgenommen werde, sein Schlußsatz stütze sich in der That nur auf diese und sonst keine andern Voraussetzungen. Dieß aber wird bloß dadurch, daß wir in unsern Beweisführungen keiner unnöthigen Sätze erwähnend, nur das allein zur Sprache bringen, was als ein wirklicher Vordersatz zu unserm Schlußsatze führt, noch eben nicht immer erreicht; besonders nicht bei Lesern oder Hörern, welche im Denken noch ungeübt sind, oder -- was schlimmer ist -- des aufrichtigen Willens, zur Erkenntniß der Wahrheit und zu einer festen Ueberzeugung von ihr zu gelangen, ermangeln. \par
Eigentlich wäre nöthig, daß wir bei einer jeden Gelegenheit sagten, und nicht bloß sagten, sondern durch eine darüber eigens angestellte Betrachtung bei jedem unserer Beweise anschaulich machten, daß und wie nach außer den wenigen von uns so eben angezogenen Sätzen nichts Anderes als wahr vorausgesetzt zu werden brauche von Allem, was wir in unserm vorhergehenden Unterrichte behauptet hatten. Mit einer so großen Weitschweifigkeit können wir aber höchstens zu Werke gehen in einer Unterweisung, welche wir mündlich ertheilen: in Schriften müssen wir uns aus mehr als Einem Grunde kürzer fassen; und schon um deßwillen bleibt es in Büchern ungleich mehr als beim mündlichen Vortrage der \seitenw{4} eigenen Denkfertigkeit, ja dem bloßen guten Willen der Leser anheimgestellt, ob und in welchem Grade sie auch bei den triftigsten Beweisen, welche wir ihnen vorlegen, von der Wahrheit unserer Lehren Ueberzeugung gewinnen. Daher ist es besonders bei Werken eines größeren Umfangs und in systematischer Form eine nur zu gewöhnliche Erscheinung, daß Leser, denen nicht Alles, was wir darin zu behaupten wagen, vollkommen klar und bis zur unwidersprechlichen Gewißheit einleuchtend geworden war, der Meinung leben, sie dürften auch den Satz, dessen Beweis wir uns zur Hauptaufgabe gemacht, der gleichsam den Schluß des Ganzen bildet, nicht als befriedigend erwiesen ansehen; und daß sie Solches wähnen, selbst wenn sie bei einiger Aufmerksamkeit wohl hätten abnehmen mögen, daß alle die Sätze, an deren Beweise sie noch etwas vermißt hatten, zu jenem Hauptsatze eben nicht in dem Verhältnisse unentbehrlicher Vordersätze gestanden. Wer sieht nicht, wie sehr dieser einzige Umstand die Verbreitung neuer Wahrheiten erschweret und verzögert, wenn sie auf keine andere Weise als durch ein Buch mitgetheilt werden dürfen!\par
Und doch gibt es der Umstände, die hier hemmend entgegentreten, noch manche andere. Wer weiß es nicht, wie viele Schwierigkeiten es hat, bei der ungeheueren Menge von Schriften, die jede Messe zu Tage gefördert werden, der eigenen, wenn auch noch so gediegenen Arbeit eine nur mäßige Anzahl aufmerksamer Leser zu verschaffen, will man sich keiner Mittel, die der Bescheidene verschmäht, bedienen? Welch eine leichte Sache doch ist's für die Herausgeber gelehrter Zeitschriften und deren Mitarbeiter, auch das gehaltreichste Werk, wenn sie es wollen, der Aufmerksamkeit des Publicums zu entrücken? Wen trifft es nicht, besonders wenn er nicht einmal sich zu nennen braucht, die gelehrte Welt zu versichern, daß ein in Rede stehendes Product nichts Neues enthalte oder daß all das Neue desselben keine Beachtung verdiene?\par
Und wird denn einer solchen Versicherung nicht um so lieber geglaubt, wenn man sie mit der schmeichelhaften Erklärung verbindet, daß die Leser sich lange schon zu einer weit höheren Stufe der Wissenschaft, als der Verfasser jenes Buches, emporgeschwungen haben? -- Ferner, wie schwer gehen doch wir \seitenw{5} Alle -- auch die Gelehrten, wenn ich nicht irre -- daran, wenn wir uns einmal an gewisse Ansichten gewöhnt, uns wieder loszureißen von ihnen, und den Werth einer erst später uns bekannt gewordenen gerecht zu würdigen? Ansichten aber, die nur ein Buch vorträgt, werden sie uns nicht insgemein erst bekannt, nachdem wir schon allerlei ihnen Entgegenstehendes eingelernt haben? Wer begreift endlich nicht, daß bei dem mündlichen Unterrichte gar manche in der Seele des Zuhörers sich erhebende Zweifel und Einwürfe gleich auf der Stelle sich beseitigen lassen, die der Verfasser eines Buches selbst wenn er sie vorhersah, nur der Weitläufigkeit wegen nicht anführen und beantworten durfte, was denn zur Folge hat, daß bei dem bloßen Lesen so oft die Ueberzeugung ausbleibt, die wohl zu Stande gekommen wäre, wenn wir denselben Mann über seinen Gegenstand mündlich hätten vernehmen können! Wer nun dieß Alles, und noch Manches, das wir hier unberührt lassen, erwäget, dem wird es klar werden, mit welchen überwiegenden Schwierigkeiten derjenige zu kämpfen habe, der einer neuen Lehre Gattung und Anerkennung zu erringen strebt, wenn ihm kein anderes Mittel, um sich Gehör zu verschaffen, als das des stummen Buchstabens, zu Gebote stehet, und alle mündliche Mittheilung, gleichviel aus welchem Grunde, versagt 
ist.\par
In dieser Lage, oder vielmehr noch in einer beträchtlich schlimmeren, befindet sich Bolzano, so ferne er den Wunsch hegt, seinen in die Gebiete der Philosophie und Theologie einschlagenden eigenthümlichen Ansichten auch nur so viel Aufmerksamkeit zu verschaffen, als zu einer endlichen Entscheidung über ihren Werth oder Unwerth erforderlich wäre. Nicht nur ihm selbst, sondern auch allen seinen ehemaligen Schülern und Anhängern ist -- wir wissen nicht recht, warum? -- bereits seit einem Zeitraume von vollen zwanzig Jahren, nicht nur ein jeder mündliche Vortrag seiner Ansichten untersagt; sondern sogar durch Schriften können nur ein und der andere seiner im Auslande lebenden Freunde bloß dadurch etwas thun, daß sie sich nothge\seitenw{6}gedrungen des so verdächtigen Mittels der Anonymität bedienen. Nur schüchtern wagten sie es im J.~1827 seine \danf{Athanasia oder Gründe für die Unsterblichkeit der Seele;} in einem langen Zeitraume darauf, 1834 das \danf{Lehrbuch der Religionswissenschaft,} und endlich im Spätherbst des J.~1837 die \danf{Wissenschaftslehre} -- bei den zwei ersteren Werken nicht einmal den Namen des Verfassers angebend -- durch die Bereitwilligkeit der J. E. v. Seidelschen Buchhandlung in Sulzbach an's Licht zu stellen. Welch eine Aufnahme nun diese Werke, in denen ein eigenthümliches philosophisch-theologisches System in seinen Grundzügen dargelegt ist, bis jetzt erfahren haben, das wird mit Mehrerem erzählt in dem \danf{Anhange zur zweiten Auflage der Athanasia, enthaltend eine kritische Übersicht der Literatur über Unsterblichkeit seit 1827} (Sulzbach, 1838), und in der Schrift: \danf{Bolzano und seine Gegner, Ein Beitrag zur neuesten Literaturgeschichte.} (Sulzbach, 1839.) Wer diese kleinen Schriften, die erste nöthigenfalls nur von S.\,99 bis 114 durchlesen will, der wird auf's Klärste dargethan finden, daß bis jetzt keine einzige der vielen eigenthümlichen Lehren, welche B.\ in den genannten drei Werken aufgestellt hat, auch nur von Einem Gelehrten auf eine wissenschaftliche Weise geprüft, und ein Versuch zu ihrer Widerlegung gemacht worden sey. Nun wissen wir zwar recht wohl, es könne auch ephemere Erscheinungen von einer so gänzlichen Werthlosigkeit geben, daß es zu ihrer verdienten Abfertigung wirklich nicht nöthig ist, alle Sonderbarkeiten derselben im Einzelnen herauszuheben, und sich in eine umständliche Widerlegung derselben einzulassen; weil ja das Publicum einmal auch wohl einem Berichterstatter, der sich bei hundert andern Gelegenheiten als einen gewissenhaften Mann bereits erwiesen hat, auf sein Wort glauben darf, daß sich in dieser Schrift auch nicht Ein Körnchen brauchbarer Wahrheit befinde. Aber ist etwa das der Fall bei den Werken B.'s? Wo hätte denn irgend ein achtbarer \Hgkorr{Gelehrte}{Gelehrter} den Muth gehabt zu sagen, die eigenthümlichen philosophischen, oder auch nur die theologischen Begriffe, \seitenw{7}  welche B. in seinen Schriften entwickelt, verdienten es gar nicht, näher geprüfet zu werden? Nein, wir gehen Alles durch, was für oder wider diese Schriften seit ihrer Erscheinung bis auf den heutigen Tag vorgebracht worden ist; wir  überschauen alle günstige und widrige Schicksale, die sie bis jetzt erlebten: und wir behaupten kühn: einen viel höheren  Werth, als der Verfasser selbst ihnen je wünschte beigelegt zu sehen, könnten die Schriften B.'s besitzen; das große Wort,  welches das Räthsel der Gegenwart löst, könnten sie darbieten: und dennoch hätten sie auf keine andere, namentlich  bessere Weise, als es bisher geschah, empfangen werden müssen. Schon daß B.'s Begriffe zu ihrer Verbreitung des Mittels aller mündlichen Gedankenmittheilung entbehren müssen;  daß es keine Lehrstühle gibt, von welchen herab sie einer aus  allen Gegenden Deutschlands herbeiströmenden Jugend vorgetragen werden, verstattet, wie wir gesehen, kein schnelles  Vordringen derselben. Damit vereiniget sich aber noch eine  Menge anderer mißlicher Umstände. B.'s Hauptwerk, die  \danf{Wissenschaftslehre,} welche ihn eigentlich erst darüber zu rechtfertigen vermag, warum er in seiner \danf{Athanasia}  sowohl als in der \danf{Religionswissenschaft} eine von  den jetzt herrschenden so durchaus abweichende Weise zu philosophiren befolgt hat, erschien nicht vor, sondern nach diesen  Schriften, und liegt dem Publico erst zwei Jahre und einige  Monate vor; ein Zeitraum, innerhalb dessen es einem Gelehrten, der sich nicht eben nach Beiseitelegung aller anderer Arbeiten mit diesem Werke allein beschäftigen konnte, noch gar  nicht möglich war, alle in den vier starken Bänden enthaltene  neue Behauptungen, alle denselben beigegebene Gründe, und  Alles, was zur Widerlegung der entgegenstehenden Ansichten  Anderer vorgebracht wird, kennen zu lernen. Daher denn  auch, daß noch kein Mann vom Fache, kein Einziger derjenigen, die in der Philosophie selbst einen Namen sich erworben haben, bis jetzt noch seine belehrende Stimme für oder wider  das Buch zu vernehmen gegeben. Jene Wenigen, die sich  bisher in einem wegwerfenden Tone über dasselbe geäußert,  sorgten schon durch die Eile, mit der sie ihr Urtheil abgaben,  mehr noch aber durch ihre schlecht verhüllte Leidenschaftlichkeit   \seitenw{8} dafür, jeden Vernünftigen über den Werth, welchen er ihrem anonymen Ausspruche beilegen dürfe, zu enttäuschen.\BUfootnote{%
	Sie traten, wie gesagt, Alle mit verhängtem Visiere auf, die tadelnd auftraten; doch hatte einer unter ihnen, der Vf.\ der Recension in Dr.~Gersdorfs Repert. der deutschen Literatur (Leipzig, 1838. Bd.\,15. Heft 6.) hinterher den Muth, sich in demselben Repertorio (1839. Bd.\,20. Heft 4.) zu nennen, als er bemühet war, folgenden drei Schriften auf einmal: der neuen Auflage der Athanasia, dem dazu gehörigen Anhange und der Schrift: Bolzano und seine Gegner, ihre verdiente Abfertigung zu ertheilen. Wir wissen also nunmehr, daß der Vf.\ jener Recension -- wer denn? -- der jugendliche Hr. Dr. und Prof. Friedrich Carl Biedermann in Leipzig ist, der um die nämliche Zeit, da er an jener Anzeige arbeiten mochte, die philosophische Literatur mit einer neuen \danf{Fundamentalphilosophie} (Leipzig b. Reichenbach, 1838) bereicherte. Obgleich wir nun dem jungen Manne für die besondere Beruhigung, welche er uns durch die Angabe seines Namens gewährte, auf das Verbindlichste danken; auch in seine Versicherung, daß er die Wissenschaftslehre vom Anfang bis zum Ende zu lesen den Zwang sich angethan habe, kein ferneres Mißtrauen setzen; da wirklich nicht Jeder, der liest, auch versteht, was er liest: so können ihm doch weder der Vf.\ der Schrift: Bolzano und s. Gegner, noch Schreiber dieses die Gegengefälligkeit erweisen, auch ihre Namen zu nennen, und dieß zwar aus einem Grunde, den wir schon oben angegeben haben. Indessen kommt eine Zeit -- Hr. Biedermann kann sie erleben -- wo auch unsere Namen bekannt werden dürften; und hoffentlich wird er dann, nicht -- wie er spottend frägt -- vor denselben verstummen, aber wie älter, so bescheidener geworden, sich seines ganzen gegenwärtigen Benehmens, besonders aber der Bolzano ertheilten Zurechtweisung schämen; schämen, nicht eben nur, weil es eine an sich ganz irrige Zurechtweisung ist, (denn daß \BUSeitenwmarkierung\ man selber irrt, indem man Andere zurecht zu weisen glaubt, kann dem Besonnensten begegnen); wohl aber, weil er sie ertheilt hat in einem Tone, welchen Bolzano, als er so jung noch war, wie Hr. Biedermann jetzt, nicht einmal gegen seine eigenen Schüler sich erlaubte. \par
Bis jetzt kennt unser rüstige Gelehrte -- laut Ausweis seiner \danf{Fundamentalphilosophie} -- für Lehren, die nicht die seinigen sind, noch keine andern Beinamen als etwa folgende: ein vages, niedriges, leeres, nichtssagendes, triviales, flaches und mehr als flaches, absurdes, täppisches Gerede, Verkehrtheit, Einbildung, Gedankenlosigkeit, ungeheuere \DruckVariante{Täuschung}{Taüschung}, metaphysische Schwindelei \usw\ Auch glaubt er S.\,295 \danf{ohne Anmaßung das Anathema aussprechen zu dürfen über zwei alterschwache Disciplinen, die sich zwischen stumpfer Anpreisung und hohnlächelnder Verachtung in einem sehr zweideutigen Zustande schon lange nur kümmerlich noch fortschleppen, und ihr Bestehen einzig nur übelverstandener Pietät oder der Scheu vor dem Besseren und der dumpfen Trägheit des Schlendrians zu verdanken haben.} Diese zwei Disciplinen sind \danf{die Logik, als eine Wissenschaft der abstrakten Denkbestimmungen, und die Metaphysik, als eine Wissenschaft der abstrakten Begriffe in ihrer Anwendung, auf die verschiedenen Gebiete des Seyns, wie man sich diese aus jenem eigenthümlichen Gesichtspunkte dachte.} Zur Rechtfertigung dieses Anathems macht er uns S.\,296 \danf{aufmerksam auf die vielen Ungereimtheiten in den Functionen der Urtheile, ganz vorzüglich aber der Schlüsse, welche wahre Zaubersprüche der absoluten Philosophie und dieser unentbehrlich, der echten Erfahrung dagegen völlig unbrauchbar, ja unverständlich sind.} -- Muß sich Bolzano nicht Glück wünschen, daß seine Schriften bei einem Gelehrten von solcher Denkungsart keinen Beifall gefunden haben?}
\seitenw{9}
Ein noch ungünstigerer Umstand war es, daß der vielschreibende Hr. Prof. Krug den wunderlichen Einfall gehabt, in zweien seiner Schriften -- in dem Henotikon nämlich, und in dem gegen Bolzanos Lehrbuch der Religionswissenschaft eigends gerichteten Antidoton (beide in \seitenw{10} Leipzig bei Kollmann, 1836) der gelehrten Welt das Mährchen aufzubinden, daß es bei diesem Lehrbuche auf nichts Geringeres abgesehen sey, als alle Protestanten sammt und sonders in den Schooß der allein seligmachenden katholischen Kirche zurückzuführen, und dem heil. Vater in Rom zu Füßen zu legen! So ungereimt nun auch ein solches Mährchen ist, so hat es doch -- wie natürlich -- auf die Versicherung eines so angesehenen Gelehrten bei Tausenden Glauben gefunden, und hiedurch im Voraus die Stimmung, mit der man B.'s harmlose Begriffe nicht nur in diesem Buche, sondern auch in der Wissenschaftslehre unter den Protestanten aufnahm, verdorben. Es ist auf solche Art wirklich dahin gekommen, daß Protestanten B.\ nicht beipflichten können, weil sie in ihm einen auf Proselytenmacherei ausgehenden Katholiken wittern; Katholiken aber -- nämlich die große Anzahl der servil denkenden, besonders im geistlichen Stande -- ihm widersprechen müssen, weil seine Schriften im Indice stehen, und weil er selbst nicht in Abrede stellt, daß er auch manche \BUlat{sententias piarum aurium offensivas} lehre. \par
Doch dieses Letztere erinnert uns schon an gewisse Fehler, die sich B.\ selbst vorwerfen mag. Nicht dieses meinen wir, daß er auch einige anstößig klingende Sätze in Schutz nehme: sondern ganz andere Dinge sind es, die wir ihm hier zur Last legen müssen. Schon daß die \danf{Wissenschaftslehre} so spät herausgegeben wurde, ist seine eigene Schuld. Denn schon um das Jahr 1822 war sie so sorgfältig ausgearbeitet, daß sie füglich hätte an's Licht treten können; und schwerlich würden die Censurgesetze seines Landes, besonders damals, ihn daran gehindert haben; warum verschob er also diese Herausgabe dergestalt von einem Jahre zum andern, daß das Werk sicherlich auch noch 1837 nicht an's Licht getreten wäre, wenn man nicht ohne sein Wissen und Wollen endlich den Druck veranlasset hätte? Würden nicht die Athanasia sowohl als die Religionswissenschaft eine ganz andre Aufnahme gefunden haben, wenn die Wissenschaftslehre zu ihrem Verständnisse und zu ihrer Würdigung vorbereitet hätte? und wäre nicht selbst diese weit \seitenw{11} unbefangener betrachtet worden, wenn man in ihrem Verfasser nicht einen Vertheidiger des Katholicismus, oder (wie schon erwähnt) wohl gar den Krugschen Proselytenmacher gesehen? -- Ein noch ärgerer Fehler war und ist gegenwärtig noch B.'s gänzliche Vernachlässigung alles desjenigen, was zwar den inneren Werth einer Sache nicht im Geringsten erhöht, doch zu ihrer Empfehlung und Beförderung gar Vieles beitragen kann. Seit 1805 auf einer philosophischen Katheder stehend und Schriftsteller in mehr als Einem Fache,\BUfootnote{%
	Nebst mehreren in das Gebiet der elementaren sowohl als höheren Mathematik einschlagenden Schriften versuchte B.\ sich auch im asketischen Fache und gab 1813 einen Band seiner \danf{Erbauungsreden an die studirende Jugend} heraus, wovon 1839 eine zweite Auflage erschienen ist, auch eine Fortsetzung angekündiget wurde.} 
wie manche Gelegenheiten und Anlässe hatte er nicht, Bekanntschaften und Verbindungen mit Gelehrten des In- und Auslandes anzuknüpfen, und bald mit diesem, bald jenem in einen schriftlichen Verkehr zu treten. Man weiß, wie sehr dergleichen Verhältnisse zu statten kommen, wenn es sich um die Verbreitung neuer Ansichten handelt. Allein wer alles dieß unbenützt ließ, wer nie von irgend einer seiner Schriften ein Exemplar an einen auswärtigen Gelehrten abgesendet, wer außer seinen eigenen Schülern beinahe mit Niemand verkehrte, seit seiner Absetzung aber bis auf den heutigen Tag in einer solchen Zurückgezogenheit lebt, daß man ihn fast gar nicht zu sehen bekömmt: das war und ist Bolzano. -- Selbst seine Schreibart legt der Verbreitung seiner Begriffe Hindernisse. Es ist zwar, wie wir glauben, eine verständliche Weise, sich auszudrücken, die er sich angeeignet hat: aber wie einförmig und trocken, wie ohne allen gelehrten Flitterstaat, wie sieht sich der Leser so ganz vergeblich um nach allen den beliebten Redensarten, ohne die unsre neueste Zeit einem Buche einmal keinen Geschmack abzugewinnen vermag; wie wird nicht Alles so klar und natürlich hingestellt, daß Jeder meint, er hätte das auch schon gewußt! Das hat \seitenw{12} dem Manne offenbar unendlich viel geschadet, und mehr als Einen Recensenten veranlaßt, zu sagen, daß er nichts von der neuesten Philosophie wisse, weil er die Sprache derselben nicht spricht. Wäre es nun nicht besser gewesen, in allen diesen Dingen ein wenig mehr Rücksicht auf die menschliche Schwachheit zu nehmen? \par
Was insbesondere das Lehrbuch der Religionswissenschaft belangt, an diesem müssen wir neben gar vielen die Form betreffenden Mängeln vornehmlich Einen sehr wichtigen Fehler rügen, den auch B.\ selbst, wie wir wissen, sich öfters vorgerückt hat. Ohne den Umfang des Buches zu vergrößern, hätte doch allenthalben mit einem einzigen Worte bezeichnet werden können, ob das Behauptete nothwendig sey oder nicht, um zu dem Schlußsatze des Ganzen zu gelangen. Durch eine solche Einrichtung wäre freilich noch nicht jede falsche Auffassung des innern Zusammenhangs zwischen den Lehren des Buches unmöglich geworden, aber man hätte gleichwohl gar manche Irrung in dieser Hinsicht, besonders bei Lesern, welche nicht eben böswillig sind, verhindert. \par
Doch schlimmer als Alles, was wir bisher erwähnten, ist die große Schwierigkeit, welche die eigenthümliche Beschaffenheit der Lehren B.'s selbst verursacht. Wenn wir mit Einem Worte bezeichnen sollen, worin das Eigene von der ganzen Denkweise des Mannes liege, so müssen wir es -- ein Streben nach Deutlichkeit nennen. Von seiner Jugend an war er darauf bedacht, allen seinen Begriffen einen möglichst hohen Grad der Deutlichkeit zu verschaffen, sich's zum Bewußtseyn zu bringen, aus welchen andern einfachern jeder zusammengesetzte Begriff bestehe, oder (was eben so viel heißt), wie er zu erklären sey. Es ist wohl nicht zu wundern, wenn eine durch so viele Jahre fortgesetzte Bemühung auch einigen Erfolg gehabt, und wenn somit gegenwärtig in B.'s Begriffen etwas mehr Deutlichkeit herrscht, als bei denjenigen gewöhnlich angetroffen wird, die nie beflissen waren, sich ihre Gedanken klar zu machen. Was man befremdend finden könnte, ist, daß uns B.\ versichert, nur durch dieß Streben nach Verdeutlichung allein zu all den eigenthüm\seitenw{13}lichen Lehren, denen man in seinen Schriften, den philosophischen und theologischen nicht nur, sondern auch den mathematischen begegnet, gelanget zu seyn. Jedenfalls ist unläugbar, daß seine Schriften durchgängig eine sehr leichte Verständlichkeit haben; daß es so ganz und gar keiner gelehrten Vorkenntnisse, am allerwenigsten einer \DruckVariante{vorläufigen}{vorlaüfigen} Bekanntschaft mit allen früheren Systemen der Philosophie bedarf, um zu verstehen, was er irgendwo sagt, und zu verstehen, aus welchen Gründen er es sagt, und mehr oder weniger sich auch überzeugt zu fühlen, daß er die Wahrheit sagt. Das Einzige, was dieser Schriftsteller von seinen Lesern verlangt, ist, daß sie nicht durchaus ungeschickt seyen zu jedem längeren Nachdenken über Gegenstände, die sich nicht mit den Händen greifen lassen, und daß sie Geduld genug haben, um nach der Ordnung zu lesen, und nichts zu überspringen, es sey denn, was er selbst als Untersuchungen, die man auch übergehen könne, bezeichnet. Das ist nun wohl eine Eigenheit, die man nicht eben nothwendig für einen Fehler erklären müßte, die Vielen so gar willkommen seyn, und als ein Zeichen der Wahrheit erscheinen könnte. Das möchte vielleicht auch bei manchem unserer Nachbarvölker der Fall seyn; allein in unserm gelehrten Deutschlande hat es mindestens in der Jetztzeit eine ganz andere Bewandtniß. Nie war man in Deutschland weniger aufgelegt, ja -- um es offen zu sagen -- nie weniger geschickt, sich in genaue Bestimmungen seiner Begriffe einzulassen. Kaum noch dem Mathematiker will man es heut zu Tage vergönnen, bei seinen scharfbestimmten Begriffen stehen zu bleiben; auch ihn möchte man, wo möglich, zu einem bloßen Spiele mit schwankenden Worten und Bildern verlocken. Nichts bedünkt unsern \danf{modernen Philosophen} kleinlicher, nichts widert sie mehr an, als ein Verweilen bei Begriffen, die dem gemeinsten Manne bekannt sind. Obwohl ein Verweilen dabei in der Absicht, um allerlei den gemeinen Menschenverstand in Staunen und Verwirrung setzende Aussprüche über sie zu thun, möchte noch ihrem Geschmacke entsprechen: aber ein Verweilen in der Art, wie B., der sich nur zum Bewußtseyn bringen will, was man bei diesen Begriffen sich eigentlich denke, und der nur Dinge vorbringt, wozu ein \seitenw{14} Jeder sagen muß: So ist es! -- das ist ganz unter der Würde unserer Philosophen, die eine jede Wahrheit, welche so leichten Preises zu gewinnen ist, verschmähen. Unsere deutschen Gelehrten haben nun einmal ein Jeder so unsäglich viel Zeit und Kraft daran gewendet, sich mit fast allen philosophischen Systemen, die es noch je gegeben hat, vertraut zu machen; sie haben sich die abstrusen und sich selbst aufhebenden Systeme, die ein und derselbe Philosoph in kurz aufeinander folgenden Zeiträumen zu Tage gefördert hatte, anzueignen gesucht, so gut es nur immer gehen wollte; es ist natürlich, daß sie dieß Alles nicht umsonst gethan haben wollen, und nun von jedem Andern verlangen, daß er dieselbe Mühe sich gebe, wie sie. Ihr Kopf ist durch die Aneignung so vieler widersinniger Lehren ihnen so schwindlich geworden, daß sie Wahres und Falsches so wenig als Oben und Unten zu unterscheiden vermögen. Daher daß man sie sagen hört, ein jedes dieser Systeme sey wahr in seiner Art, widerlege aber gleichwohl daß nächstvorhergehende und werde durch das nächstfolgende abermal selbst widerlegt. Wie nun von solchen Gelehrten eine gerechte Würdigung eines Systemes erwarten, das alle ihre Anstrengungen für entbehrlich erklärt, und nachweist, daß und wienach in den gepriesensten Systemen schon in den ersten Sätzen gefehlt sey? --\par
Das Übel wäre indessen noch einigermaßen erträglich, wenn B.\ nur einen einzelnen Begriff von Wichtigkeit, etwa den von Wahrheiten an sich oder den der Erfahrung zergliedert, und sich in dieser Zergliederung begnüget hätte, die nächsten Bestandtheile desselben anzugeben; auch nicht weiter fortgeschritten, als bis zu den ersten wichtigen Folgerungen, die sich aus dieser Verdeutlichung eines unsrer Begriffe ergeben; wenn er auf solche Art seine Wissenschaftslehre, statt sie bis zu der Stärke von anderthalbhundert Bogen anwachsen zu lassen, der gelehrten Welt vorgelegt hätte, als sie etwa den zehnten Theil dieses Umfanges hatte. Wir könnten dann gewiß viel eher hoffen, daß man mit ihrem Inhalte sich bekannt machen werde. Jetzt aber, da man ein Werk von solcher Ausdehnung vor sich sieht, und wenn man anfängt, \seitenw{15} darin zu lesen, inne wird, daß man die eine abweichende Ansicht noch nicht zur Hälfte sich geläufig gemacht hat, als der Vf.\ schon mit einer zweiten auftritt, muß da nicht auch der rüstigste Leser ermüden, und sich versucht fühlen, zu überschlagen, oder mit flüchtiger Eile zu lesen, um nur an's Ende zu gelangen? Thut er dieß aber, dann ist auch nichts natürlicher, als daß er bald mißverstehe, bald doch nicht überzeuget werde, die Schuld aber jedesmal nicht in sich selbst, sondern im Buche finde. Es ist auch nichts natürlicher, als daß die vielen Angriffe, die der Verf. -- zwar immer in der bescheidensten Weise -- auf die Lehren Anderer sich erlaubt, den Leser nur beleidigen, ja ihn zuletzt sogar auf den Verdacht bringen, daß sich B.\ aus bloßer Neuerungssucht über so viele Gegenstände anders, als es bisher gewöhnlich war, entscheide! -- \par 
Wie ist nun da zu helfen? Ein Mittel, wodurch alle bisher genannten Hindernisse beseitiget würden, scheint uns nicht im Bereiche der Möglichkeit zu liegen; aber etwas, wodurch doch eine oder die andere dieser Schwierigkeiten vermindert würde, wüßten wir anzugeben: B.\ hat zu viel auf einmal dargeboten; läßt dieser Fehler sich nicht durch eine Art von Auszug verbessern? Im Grunde sind es aber nur die zwei größeren Werke: die Wissenschaftslehre und die Religionswissenschaft, welche die Geduld ihrer Leser auf eine etwas härtere Probe stellen; denn über die Athanasia hat unseres Wissens noch Niemand geklagt, sie habe ihn ermüdet. So däucht es uns denn, durch eine kurze, mehr an die Sache als an das Wort sich haltende Übersicht nur der vorzüglichsten in jenen beiden Werken enthaltenen Lehren, wobei wir den Leser zugleich aufmerksam darauf machten, was eine jede aus dem Vorhergehenden voraussetzt, wäre in sofern viel gewonnen, in wiefern es nun einem Jeden, der nur den Aufwand etlicher Stunden nicht scheut, möglich gemacht wäre, sich eine Kenntniß von B.'s eigenthümlichsten Ansichten zu verschaffen. Und hiebei würde überdieß noch der Vortheil erreicht, den falschen und entstellenden Berichten, die Andere über ihn dem Publico erstattet, mindestens in der Art entgegen zu wirken, daß es nunmehr für Jeden, der enttäuschet \seitenw{16} werden will, ein Leichtes wäre, die Wahrheit zu erfahren. Wer endlich, sey es durch diese Schrift selbst, oder auf sonst eine andere Weise dazu veranlaßt, das philosophisch-theologische System B.'s aus seinen eigenen Schriften vollständig kennen zu lernen wünschet, dem wird die hier gelieferte Übersicht, wie wir uns schmeicheln, die Auffassung sowohl als auch die richtige Beurtheilung erleichtern. \seitenw{17}\par
\endinput